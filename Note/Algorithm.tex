\section{Algorithms and their performance}
\label{sec:algorithm}

Waveform analysis is to obtain $t_i$ and $q_i$ estimators $\hat{t}_i$ and $\hat{q}_i$ from waveform $w(t)$, where the output indices $i$ are from 1 to $\hat{N}_\mathrm{PE}$ and $\hat{N}_\mathrm{PE}$ is an estimator of $N_\mathrm{PE}$ in eq.~\eqref{eq:lc-sample}. Figure~\ref{fig:pile} illustrates the input waveform $w(t)$ and the outputs charge $\bm{\hat{t}}, \hat{\bm{q}}$ obtained from $w(t)$, where boldface $\hat{\bm{t}}$ denotes the vector $\hat{t}_i$. 

$\hat{N}_\mathrm{PE}$ fails to estimate $N_\mathrm{PE}$ due to the fluctuation of $q_i$ and the ambiguity of $\hat{t}_i$. For example, 1, 2 and even 3~PEs can generate the same charge as $1.6$ units.  A single PE charged $1$ might be misinterpreted as 2~PEs at consecutive $\hat{t}_i$ and $\hat{t}_{i+1}$ with $\hat{q}_i=\hat{q}_{i+1}=0.5$.

\subsection{Evaluation criteria}
\label{sec:criteria}
Subject to such ambiguity of $t_i/q_i$, we introduce a set of evaluation criteria to assess the algorithms' performance.

\subsubsection{Kullback-Leibler divergence}
\label{sec:pseudo}

We construct a light curve estimator $\hat{\phi}(t)$ from $\bm{\hat{t}}$, $\bm{\hat{q}}$ and $\hat{N}_\mathrm{PE}$,
\begin{equation}
  \label{eq:lc}
  \hat{\phi}(t) = \sum_{i=1}^{\hat{N}_\mathrm{PE}} \hat{q}_i\delta(t-\hat{t}_i),
\end{equation}
which resembles eq.~\eqref{eq:lc-sample}.

Basu et al.'s \textit{density power divergence}~\cite{basu_robust_1998} contains the classical Kullback-Leibler~(KL) divergence~\cite{kullback_information_1951} as a special case.  Non-normalized KL divergence is defined accordingly if we do not normalize $\hat{\phi}(t)$ and $\mu \phi(t-t_{0})$ to 1 when considering their divergence in eq.~\eqref{eq:kl},
\begin{equation}
  \begin{aligned}
    D_\mathrm{KL}\left[\hat{\phi}(t) \parallel \mu\phi(t-t_0)\right] & =\int \left[\hat{\phi}(t) \log\frac{\hat{\phi}(t)}{\mu\phi(t-t_0)} + \mu\phi(t-t_0) - \hat{\phi}(t) \right]\mathrm{d}t \\
    & = - \int \hat{\phi}(t) \log\phi(t-t_0)\mathrm{d}t - \log(\mu)\int\hat{\phi}(t)\mathrm{d}t + \mu + \int \left[\hat{\phi}(t) \log\hat{\phi}(t) - \hat{\phi}(t) \right]\mathrm{d}t \\
    & = - \sum_{i=1}^{\hat{N}_\mathrm{PE}}\left[\int \hat{q}_i\delta(t-\hat{t_i}) \log\phi(t-t_0)\mathrm{d}t - \log(\mu)\int\hat{q}_i\delta(t-\hat{t_i})\mathrm{d}t\right] + \mu +  C \\
    & = -\log \left\{\prod_{i=1}^{\hat{N}_\mathrm{PE}} \left[\phi(\hat{t}_i-t_0)\right]^{\hat{q}_i}\right\} - \log(\mu)\sum_{i=1}^{\hat{N}_\mathrm{PE}} \hat{q}_i + \mu + C
  \label{eq:kl}
  \end{aligned}
\end{equation}
where $C$ is a constant regarding $t_0$ and $\mu$.  Define the time KL estimator as
\begin{equation}
  \begin{aligned}
  \label{eq:pseudo}
  \hat{t}_\mathrm{KL} &= \arg\underset{t_0}{\min}~D_\mathrm{KL}\left[\hat{\phi}(t) \parallel \mu\phi(t-t_0)\right] \\
  &= \arg\underset{t_0}{\max} \prod_{i=1}^{\hat{N}_\mathrm{PE}} \left[\phi(\hat{t}_i-t_0)\right]^{\hat{q}_i},
  \end{aligned}
\end{equation}
which reduces to an MLE like eq.~\eqref{eq:2} if $\hat{q}_i\equiv 1$.  $\hat{t}_\mathrm{KL}$ estimates $t_0$ when $t_i, q_i, N_\mathrm{PE}$ are all uncertain.
Denoting $\Delta t_0$ as $\hat{t}_\mathrm{KL} - t_0$, the standard deviation $\sigma_\mathrm{KL}$ of $\Delta t_0$ on a batch of waveforms is the resolution of an algorithm.

The intensity KL estimator is,
\begin{equation}
  \label{eq:pseudo-mu}
  \hat{\mu}_\mathrm{KL} = \arg\underset{\mu}{\min}~D_\mathrm{KL}\left[\hat{\phi}(t) \parallel \mu\phi(t-t_0)\right] = \sum_{i=1}^{\hat{N}_\mathrm{PE}} \hat{q}_i.
\end{equation}


\subsubsection{Residual sum of squares}
\label{sec:rss}

Following eqs.~\eqref{eq:1} and~\eqref{eq:lc}, we construct an estimator of a waveform,
\begin{equation}
  \label{eq:w-hat}
  \hat{w}(t) = \sum_{i=1}^{\hat{N}_\mathrm{PE}}\hat{q}_i V_\mathrm{PE}(t-\hat{t}_i) = \hat{\phi}(t) \otimes V_\mathrm{PE}(t).
\end{equation}

For a noise-free evaluation of $\hat{w}(t)$, residual sum of squares~(RSS) is a $\ell_2$-distance of it to $\tilde{w}(t)$,
\begin{equation}
  \label{eq:rss}
  \mathrm{RSS} \coloneqq\int\left[\hat{w}(t) - \tilde{w}(t)\right]^2\mathrm{d}t.
\end{equation}
We choose $\tilde{w}(t)$ for evaluating algorithms because otherwise with the raw waveform $w(t)$ RSS will be dominated by the white noise term $\epsilon(t)$.

Figure~\ref{fig:l2} demonstrates that if two functions do not overlap, their $\mathrm{RSS}$ remain constant regardless of relative positions.  The delta functions in the sampled light curves $\hat{\phi}(t)$ and $\tilde{\phi}(t)$ hardly overlap, rendering $\mathrm{RSS}$ useless.  Furthermore, RSS cannot compare a discrete function with a continuous one.  We shall only consider the $\mathrm{RSS}$ of waveforms.

\begin{figure}[H]
  \centering
  \resizebox{0.6\textwidth}{!}{%% Creator: Matplotlib, PGF backend
%%
%% To include the figure in your LaTeX document, write
%%   \input{<filename>.pgf}
%%
%% Make sure the required packages are loaded in your preamble
%%   \usepackage{pgf}
%%
%% Also ensure that all the required font packages are loaded; for instance,
%% the lmodern package is sometimes necessary when using math font.
%%   \usepackage{lmodern}
%%
%% Figures using additional raster images can only be included by \input if
%% they are in the same directory as the main LaTeX file. For loading figures
%% from other directories you can use the `import` package
%%   \usepackage{import}
%%
%% and then include the figures with
%%   \import{<path to file>}{<filename>.pgf}
%%
%% Matplotlib used the following preamble
%%   \usepackage[detect-all,locale=DE]{siunitx}
%%
\begingroup%
\makeatletter%
\begin{pgfpicture}%
\pgfpathrectangle{\pgfpointorigin}{\pgfqpoint{8.000000in}{4.000000in}}%
\pgfusepath{use as bounding box, clip}%
\begin{pgfscope}%
\pgfsetbuttcap%
\pgfsetmiterjoin%
\definecolor{currentfill}{rgb}{1.000000,1.000000,1.000000}%
\pgfsetfillcolor{currentfill}%
\pgfsetlinewidth{0.000000pt}%
\definecolor{currentstroke}{rgb}{1.000000,1.000000,1.000000}%
\pgfsetstrokecolor{currentstroke}%
\pgfsetdash{}{0pt}%
\pgfpathmoveto{\pgfqpoint{0.000000in}{0.000000in}}%
\pgfpathlineto{\pgfqpoint{8.000000in}{0.000000in}}%
\pgfpathlineto{\pgfqpoint{8.000000in}{4.000000in}}%
\pgfpathlineto{\pgfqpoint{0.000000in}{4.000000in}}%
\pgfpathlineto{\pgfqpoint{0.000000in}{0.000000in}}%
\pgfpathclose%
\pgfusepath{fill}%
\end{pgfscope}%
\begin{pgfscope}%
\pgfsetbuttcap%
\pgfsetmiterjoin%
\definecolor{currentfill}{rgb}{1.000000,1.000000,1.000000}%
\pgfsetfillcolor{currentfill}%
\pgfsetlinewidth{0.000000pt}%
\definecolor{currentstroke}{rgb}{0.000000,0.000000,0.000000}%
\pgfsetstrokecolor{currentstroke}%
\pgfsetstrokeopacity{0.000000}%
\pgfsetdash{}{0pt}%
\pgfpathmoveto{\pgfqpoint{0.400000in}{0.400000in}}%
\pgfpathlineto{\pgfqpoint{7.760000in}{0.400000in}}%
\pgfpathlineto{\pgfqpoint{7.760000in}{3.880000in}}%
\pgfpathlineto{\pgfqpoint{0.400000in}{3.880000in}}%
\pgfpathlineto{\pgfqpoint{0.400000in}{0.400000in}}%
\pgfpathclose%
\pgfusepath{fill}%
\end{pgfscope}%
\begin{pgfscope}%
\pgfpathrectangle{\pgfqpoint{0.400000in}{0.400000in}}{\pgfqpoint{7.360000in}{3.480000in}}%
\pgfusepath{clip}%
\pgfsetbuttcap%
\pgfsetroundjoin%
\definecolor{currentfill}{rgb}{1.000000,0.549020,0.000000}%
\pgfsetfillcolor{currentfill}%
\pgfsetfillopacity{0.500000}%
\pgfsetlinewidth{1.003750pt}%
\definecolor{currentstroke}{rgb}{1.000000,0.549020,0.000000}%
\pgfsetstrokecolor{currentstroke}%
\pgfsetstrokeopacity{0.500000}%
\pgfsetdash{}{0pt}%
\pgfsys@defobject{currentmarker}{\pgfqpoint{0.400000in}{0.397613in}}{\pgfqpoint{7.760000in}{3.714172in}}{%
\pgfpathmoveto{\pgfqpoint{0.400000in}{0.397613in}}%
\pgfpathlineto{\pgfqpoint{0.400000in}{0.397613in}}%
\pgfpathlineto{\pgfqpoint{0.436800in}{0.397613in}}%
\pgfpathlineto{\pgfqpoint{0.473600in}{0.397613in}}%
\pgfpathlineto{\pgfqpoint{0.510400in}{0.397613in}}%
\pgfpathlineto{\pgfqpoint{0.547200in}{0.397613in}}%
\pgfpathlineto{\pgfqpoint{0.584000in}{0.397613in}}%
\pgfpathlineto{\pgfqpoint{0.620800in}{0.397613in}}%
\pgfpathlineto{\pgfqpoint{0.657600in}{0.397613in}}%
\pgfpathlineto{\pgfqpoint{0.694400in}{0.397613in}}%
\pgfpathlineto{\pgfqpoint{0.731200in}{0.397613in}}%
\pgfpathlineto{\pgfqpoint{0.768000in}{0.397613in}}%
\pgfpathlineto{\pgfqpoint{0.804800in}{0.397613in}}%
\pgfpathlineto{\pgfqpoint{0.841600in}{0.397613in}}%
\pgfpathlineto{\pgfqpoint{0.878400in}{0.397613in}}%
\pgfpathlineto{\pgfqpoint{0.915200in}{0.397613in}}%
\pgfpathlineto{\pgfqpoint{0.952000in}{0.397613in}}%
\pgfpathlineto{\pgfqpoint{0.988800in}{0.397613in}}%
\pgfpathlineto{\pgfqpoint{1.025600in}{0.397613in}}%
\pgfpathlineto{\pgfqpoint{1.062400in}{0.397613in}}%
\pgfpathlineto{\pgfqpoint{1.099200in}{0.397613in}}%
\pgfpathlineto{\pgfqpoint{1.136000in}{0.397613in}}%
\pgfpathlineto{\pgfqpoint{1.172800in}{0.397613in}}%
\pgfpathlineto{\pgfqpoint{1.209600in}{0.397613in}}%
\pgfpathlineto{\pgfqpoint{1.246400in}{0.397613in}}%
\pgfpathlineto{\pgfqpoint{1.283200in}{0.397613in}}%
\pgfpathlineto{\pgfqpoint{1.320000in}{0.397613in}}%
\pgfpathlineto{\pgfqpoint{1.356800in}{0.397613in}}%
\pgfpathlineto{\pgfqpoint{1.393600in}{0.397613in}}%
\pgfpathlineto{\pgfqpoint{1.430400in}{0.397613in}}%
\pgfpathlineto{\pgfqpoint{1.467200in}{0.397613in}}%
\pgfpathlineto{\pgfqpoint{1.504000in}{0.397613in}}%
\pgfpathlineto{\pgfqpoint{1.540800in}{0.397613in}}%
\pgfpathlineto{\pgfqpoint{1.577600in}{0.397613in}}%
\pgfpathlineto{\pgfqpoint{1.614400in}{0.397613in}}%
\pgfpathlineto{\pgfqpoint{1.651200in}{0.397613in}}%
\pgfpathlineto{\pgfqpoint{1.688000in}{0.397613in}}%
\pgfpathlineto{\pgfqpoint{1.724800in}{0.397613in}}%
\pgfpathlineto{\pgfqpoint{1.761600in}{0.397613in}}%
\pgfpathlineto{\pgfqpoint{1.798400in}{0.397613in}}%
\pgfpathlineto{\pgfqpoint{1.835200in}{0.397613in}}%
\pgfpathlineto{\pgfqpoint{1.872000in}{0.397613in}}%
\pgfpathlineto{\pgfqpoint{1.908800in}{0.397613in}}%
\pgfpathlineto{\pgfqpoint{1.945600in}{0.397613in}}%
\pgfpathlineto{\pgfqpoint{1.982400in}{0.397613in}}%
\pgfpathlineto{\pgfqpoint{2.019200in}{0.397613in}}%
\pgfpathlineto{\pgfqpoint{2.056000in}{0.397613in}}%
\pgfpathlineto{\pgfqpoint{2.092800in}{0.397613in}}%
\pgfpathlineto{\pgfqpoint{2.129600in}{0.397613in}}%
\pgfpathlineto{\pgfqpoint{2.166400in}{0.397613in}}%
\pgfpathlineto{\pgfqpoint{2.203200in}{0.397613in}}%
\pgfpathlineto{\pgfqpoint{2.240000in}{0.397613in}}%
\pgfpathlineto{\pgfqpoint{2.276800in}{0.397613in}}%
\pgfpathlineto{\pgfqpoint{2.313600in}{0.397613in}}%
\pgfpathlineto{\pgfqpoint{2.350400in}{0.397613in}}%
\pgfpathlineto{\pgfqpoint{2.387200in}{0.397613in}}%
\pgfpathlineto{\pgfqpoint{2.424000in}{0.397613in}}%
\pgfpathlineto{\pgfqpoint{2.460800in}{0.397613in}}%
\pgfpathlineto{\pgfqpoint{2.497600in}{0.397613in}}%
\pgfpathlineto{\pgfqpoint{2.534400in}{0.397613in}}%
\pgfpathlineto{\pgfqpoint{2.571200in}{0.397613in}}%
\pgfpathlineto{\pgfqpoint{2.608000in}{0.397613in}}%
\pgfpathlineto{\pgfqpoint{2.644800in}{0.397613in}}%
\pgfpathlineto{\pgfqpoint{2.681600in}{0.397613in}}%
\pgfpathlineto{\pgfqpoint{2.718400in}{0.397613in}}%
\pgfpathlineto{\pgfqpoint{2.755200in}{0.397613in}}%
\pgfpathlineto{\pgfqpoint{2.792000in}{0.397613in}}%
\pgfpathlineto{\pgfqpoint{2.828800in}{0.397613in}}%
\pgfpathlineto{\pgfqpoint{2.865600in}{0.397613in}}%
\pgfpathlineto{\pgfqpoint{2.902400in}{0.397613in}}%
\pgfpathlineto{\pgfqpoint{2.939200in}{0.397613in}}%
\pgfpathlineto{\pgfqpoint{2.976000in}{0.397613in}}%
\pgfpathlineto{\pgfqpoint{3.012800in}{0.397613in}}%
\pgfpathlineto{\pgfqpoint{3.049600in}{0.397613in}}%
\pgfpathlineto{\pgfqpoint{3.086400in}{0.397613in}}%
\pgfpathlineto{\pgfqpoint{3.123200in}{0.397613in}}%
\pgfpathlineto{\pgfqpoint{3.160000in}{0.397613in}}%
\pgfpathlineto{\pgfqpoint{3.196800in}{0.397613in}}%
\pgfpathlineto{\pgfqpoint{3.233600in}{0.397613in}}%
\pgfpathlineto{\pgfqpoint{3.270400in}{0.397613in}}%
\pgfpathlineto{\pgfqpoint{3.307200in}{0.397613in}}%
\pgfpathlineto{\pgfqpoint{3.344000in}{0.397613in}}%
\pgfpathlineto{\pgfqpoint{3.380800in}{0.397613in}}%
\pgfpathlineto{\pgfqpoint{3.417600in}{0.397613in}}%
\pgfpathlineto{\pgfqpoint{3.454400in}{0.397613in}}%
\pgfpathlineto{\pgfqpoint{3.491200in}{0.397613in}}%
\pgfpathlineto{\pgfqpoint{3.528000in}{0.397613in}}%
\pgfpathlineto{\pgfqpoint{3.564800in}{0.397613in}}%
\pgfpathlineto{\pgfqpoint{3.601600in}{0.397613in}}%
\pgfpathlineto{\pgfqpoint{3.638400in}{0.397613in}}%
\pgfpathlineto{\pgfqpoint{3.675200in}{0.397613in}}%
\pgfpathlineto{\pgfqpoint{3.712000in}{0.397613in}}%
\pgfpathlineto{\pgfqpoint{3.748800in}{0.397613in}}%
\pgfpathlineto{\pgfqpoint{3.785600in}{0.397613in}}%
\pgfpathlineto{\pgfqpoint{3.822400in}{0.397613in}}%
\pgfpathlineto{\pgfqpoint{3.859200in}{0.397613in}}%
\pgfpathlineto{\pgfqpoint{3.896000in}{0.397613in}}%
\pgfpathlineto{\pgfqpoint{3.932800in}{0.397613in}}%
\pgfpathlineto{\pgfqpoint{3.969600in}{0.397613in}}%
\pgfpathlineto{\pgfqpoint{4.006400in}{0.397613in}}%
\pgfpathlineto{\pgfqpoint{4.043200in}{0.397613in}}%
\pgfpathlineto{\pgfqpoint{4.080000in}{0.397613in}}%
\pgfpathlineto{\pgfqpoint{4.116800in}{0.397613in}}%
\pgfpathlineto{\pgfqpoint{4.153600in}{0.397613in}}%
\pgfpathlineto{\pgfqpoint{4.190400in}{0.397613in}}%
\pgfpathlineto{\pgfqpoint{4.227200in}{0.397613in}}%
\pgfpathlineto{\pgfqpoint{4.264000in}{0.397613in}}%
\pgfpathlineto{\pgfqpoint{4.300800in}{0.397613in}}%
\pgfpathlineto{\pgfqpoint{4.337600in}{0.397613in}}%
\pgfpathlineto{\pgfqpoint{4.374400in}{0.397613in}}%
\pgfpathlineto{\pgfqpoint{4.411200in}{0.397613in}}%
\pgfpathlineto{\pgfqpoint{4.448000in}{0.397613in}}%
\pgfpathlineto{\pgfqpoint{4.484800in}{0.397613in}}%
\pgfpathlineto{\pgfqpoint{4.521600in}{0.397613in}}%
\pgfpathlineto{\pgfqpoint{4.558400in}{0.397613in}}%
\pgfpathlineto{\pgfqpoint{4.595200in}{0.397613in}}%
\pgfpathlineto{\pgfqpoint{4.632000in}{0.397613in}}%
\pgfpathlineto{\pgfqpoint{4.668800in}{0.397613in}}%
\pgfpathlineto{\pgfqpoint{4.705600in}{0.397613in}}%
\pgfpathlineto{\pgfqpoint{4.742400in}{0.397613in}}%
\pgfpathlineto{\pgfqpoint{4.779200in}{0.397613in}}%
\pgfpathlineto{\pgfqpoint{4.816000in}{0.397613in}}%
\pgfpathlineto{\pgfqpoint{4.852800in}{0.397613in}}%
\pgfpathlineto{\pgfqpoint{4.889600in}{0.397613in}}%
\pgfpathlineto{\pgfqpoint{4.926400in}{0.397613in}}%
\pgfpathlineto{\pgfqpoint{4.963200in}{0.397613in}}%
\pgfpathlineto{\pgfqpoint{5.000000in}{0.397613in}}%
\pgfpathlineto{\pgfqpoint{5.036800in}{0.397613in}}%
\pgfpathlineto{\pgfqpoint{5.073600in}{0.397613in}}%
\pgfpathlineto{\pgfqpoint{5.110400in}{0.397613in}}%
\pgfpathlineto{\pgfqpoint{5.147200in}{0.397613in}}%
\pgfpathlineto{\pgfqpoint{5.184000in}{0.397613in}}%
\pgfpathlineto{\pgfqpoint{5.220800in}{0.397613in}}%
\pgfpathlineto{\pgfqpoint{5.257600in}{0.397613in}}%
\pgfpathlineto{\pgfqpoint{5.294400in}{0.397613in}}%
\pgfpathlineto{\pgfqpoint{5.331200in}{0.397613in}}%
\pgfpathlineto{\pgfqpoint{5.368000in}{0.397613in}}%
\pgfpathlineto{\pgfqpoint{5.404800in}{0.397613in}}%
\pgfpathlineto{\pgfqpoint{5.441600in}{0.397613in}}%
\pgfpathlineto{\pgfqpoint{5.478400in}{0.397613in}}%
\pgfpathlineto{\pgfqpoint{5.515200in}{0.397613in}}%
\pgfpathlineto{\pgfqpoint{5.552000in}{0.397613in}}%
\pgfpathlineto{\pgfqpoint{5.588800in}{0.397613in}}%
\pgfpathlineto{\pgfqpoint{5.625600in}{0.397613in}}%
\pgfpathlineto{\pgfqpoint{5.662400in}{0.397613in}}%
\pgfpathlineto{\pgfqpoint{5.699200in}{0.397613in}}%
\pgfpathlineto{\pgfqpoint{5.736000in}{0.397613in}}%
\pgfpathlineto{\pgfqpoint{5.772800in}{0.397613in}}%
\pgfpathlineto{\pgfqpoint{5.809600in}{0.397613in}}%
\pgfpathlineto{\pgfqpoint{5.846400in}{0.397613in}}%
\pgfpathlineto{\pgfqpoint{5.883200in}{0.397613in}}%
\pgfpathlineto{\pgfqpoint{5.920000in}{0.397613in}}%
\pgfpathlineto{\pgfqpoint{5.956800in}{0.397613in}}%
\pgfpathlineto{\pgfqpoint{5.993600in}{0.397613in}}%
\pgfpathlineto{\pgfqpoint{6.030400in}{0.397613in}}%
\pgfpathlineto{\pgfqpoint{6.067200in}{0.397613in}}%
\pgfpathlineto{\pgfqpoint{6.104000in}{0.397613in}}%
\pgfpathlineto{\pgfqpoint{6.140800in}{0.397613in}}%
\pgfpathlineto{\pgfqpoint{6.177600in}{0.397613in}}%
\pgfpathlineto{\pgfqpoint{6.214400in}{0.397613in}}%
\pgfpathlineto{\pgfqpoint{6.251200in}{0.397613in}}%
\pgfpathlineto{\pgfqpoint{6.288000in}{0.397613in}}%
\pgfpathlineto{\pgfqpoint{6.324800in}{0.397613in}}%
\pgfpathlineto{\pgfqpoint{6.361600in}{0.397613in}}%
\pgfpathlineto{\pgfqpoint{6.398400in}{0.397613in}}%
\pgfpathlineto{\pgfqpoint{6.435200in}{0.397613in}}%
\pgfpathlineto{\pgfqpoint{6.472000in}{0.397613in}}%
\pgfpathlineto{\pgfqpoint{6.508800in}{0.397613in}}%
\pgfpathlineto{\pgfqpoint{6.545600in}{0.397613in}}%
\pgfpathlineto{\pgfqpoint{6.582400in}{0.397613in}}%
\pgfpathlineto{\pgfqpoint{6.619200in}{0.397613in}}%
\pgfpathlineto{\pgfqpoint{6.656000in}{0.397613in}}%
\pgfpathlineto{\pgfqpoint{6.692800in}{0.397613in}}%
\pgfpathlineto{\pgfqpoint{6.729600in}{0.397613in}}%
\pgfpathlineto{\pgfqpoint{6.766400in}{0.397613in}}%
\pgfpathlineto{\pgfqpoint{6.803200in}{0.397613in}}%
\pgfpathlineto{\pgfqpoint{6.840000in}{0.397613in}}%
\pgfpathlineto{\pgfqpoint{6.876800in}{0.397613in}}%
\pgfpathlineto{\pgfqpoint{6.913600in}{0.397613in}}%
\pgfpathlineto{\pgfqpoint{6.950400in}{0.397613in}}%
\pgfpathlineto{\pgfqpoint{6.987200in}{0.397613in}}%
\pgfpathlineto{\pgfqpoint{7.024000in}{0.397613in}}%
\pgfpathlineto{\pgfqpoint{7.060800in}{0.397613in}}%
\pgfpathlineto{\pgfqpoint{7.097600in}{0.397613in}}%
\pgfpathlineto{\pgfqpoint{7.134400in}{0.397613in}}%
\pgfpathlineto{\pgfqpoint{7.171200in}{0.397613in}}%
\pgfpathlineto{\pgfqpoint{7.208000in}{0.397613in}}%
\pgfpathlineto{\pgfqpoint{7.244800in}{0.397613in}}%
\pgfpathlineto{\pgfqpoint{7.281600in}{0.397613in}}%
\pgfpathlineto{\pgfqpoint{7.318400in}{0.397613in}}%
\pgfpathlineto{\pgfqpoint{7.355200in}{0.397613in}}%
\pgfpathlineto{\pgfqpoint{7.392000in}{0.397613in}}%
\pgfpathlineto{\pgfqpoint{7.428800in}{0.397613in}}%
\pgfpathlineto{\pgfqpoint{7.465600in}{0.397613in}}%
\pgfpathlineto{\pgfqpoint{7.502400in}{0.397613in}}%
\pgfpathlineto{\pgfqpoint{7.539200in}{0.397613in}}%
\pgfpathlineto{\pgfqpoint{7.576000in}{0.397613in}}%
\pgfpathlineto{\pgfqpoint{7.612800in}{0.397613in}}%
\pgfpathlineto{\pgfqpoint{7.649600in}{0.397613in}}%
\pgfpathlineto{\pgfqpoint{7.686400in}{0.397613in}}%
\pgfpathlineto{\pgfqpoint{7.723200in}{0.397613in}}%
\pgfpathlineto{\pgfqpoint{7.760000in}{0.397613in}}%
\pgfpathlineto{\pgfqpoint{7.760000in}{0.397613in}}%
\pgfpathlineto{\pgfqpoint{7.760000in}{0.397613in}}%
\pgfpathlineto{\pgfqpoint{7.723200in}{0.397613in}}%
\pgfpathlineto{\pgfqpoint{7.686400in}{0.397613in}}%
\pgfpathlineto{\pgfqpoint{7.649600in}{0.397613in}}%
\pgfpathlineto{\pgfqpoint{7.612800in}{0.397613in}}%
\pgfpathlineto{\pgfqpoint{7.576000in}{0.397613in}}%
\pgfpathlineto{\pgfqpoint{7.539200in}{0.397613in}}%
\pgfpathlineto{\pgfqpoint{7.502400in}{0.397613in}}%
\pgfpathlineto{\pgfqpoint{7.465600in}{0.397613in}}%
\pgfpathlineto{\pgfqpoint{7.428800in}{0.397613in}}%
\pgfpathlineto{\pgfqpoint{7.392000in}{0.397613in}}%
\pgfpathlineto{\pgfqpoint{7.355200in}{0.397613in}}%
\pgfpathlineto{\pgfqpoint{7.318400in}{0.397613in}}%
\pgfpathlineto{\pgfqpoint{7.281600in}{0.397613in}}%
\pgfpathlineto{\pgfqpoint{7.244800in}{0.397613in}}%
\pgfpathlineto{\pgfqpoint{7.208000in}{0.397613in}}%
\pgfpathlineto{\pgfqpoint{7.171200in}{0.397613in}}%
\pgfpathlineto{\pgfqpoint{7.134400in}{0.397613in}}%
\pgfpathlineto{\pgfqpoint{7.097600in}{0.397613in}}%
\pgfpathlineto{\pgfqpoint{7.060800in}{0.397613in}}%
\pgfpathlineto{\pgfqpoint{7.024000in}{0.397613in}}%
\pgfpathlineto{\pgfqpoint{6.987200in}{0.397613in}}%
\pgfpathlineto{\pgfqpoint{6.950400in}{0.397613in}}%
\pgfpathlineto{\pgfqpoint{6.913600in}{0.397613in}}%
\pgfpathlineto{\pgfqpoint{6.876800in}{0.397613in}}%
\pgfpathlineto{\pgfqpoint{6.840000in}{0.397613in}}%
\pgfpathlineto{\pgfqpoint{6.803200in}{0.397613in}}%
\pgfpathlineto{\pgfqpoint{6.766400in}{0.397613in}}%
\pgfpathlineto{\pgfqpoint{6.729600in}{0.397613in}}%
\pgfpathlineto{\pgfqpoint{6.692800in}{0.397613in}}%
\pgfpathlineto{\pgfqpoint{6.656000in}{0.397613in}}%
\pgfpathlineto{\pgfqpoint{6.619200in}{0.397613in}}%
\pgfpathlineto{\pgfqpoint{6.582400in}{0.397613in}}%
\pgfpathlineto{\pgfqpoint{6.545600in}{0.397613in}}%
\pgfpathlineto{\pgfqpoint{6.508800in}{0.397613in}}%
\pgfpathlineto{\pgfqpoint{6.472000in}{0.397613in}}%
\pgfpathlineto{\pgfqpoint{6.435200in}{0.397613in}}%
\pgfpathlineto{\pgfqpoint{6.398400in}{0.397613in}}%
\pgfpathlineto{\pgfqpoint{6.361600in}{0.397613in}}%
\pgfpathlineto{\pgfqpoint{6.324800in}{0.397613in}}%
\pgfpathlineto{\pgfqpoint{6.288000in}{0.397613in}}%
\pgfpathlineto{\pgfqpoint{6.251200in}{0.397613in}}%
\pgfpathlineto{\pgfqpoint{6.214400in}{0.397614in}}%
\pgfpathlineto{\pgfqpoint{6.177600in}{0.397614in}}%
\pgfpathlineto{\pgfqpoint{6.140800in}{0.397614in}}%
\pgfpathlineto{\pgfqpoint{6.104000in}{0.397614in}}%
\pgfpathlineto{\pgfqpoint{6.067200in}{0.397614in}}%
\pgfpathlineto{\pgfqpoint{6.030400in}{0.397614in}}%
\pgfpathlineto{\pgfqpoint{5.993600in}{0.397615in}}%
\pgfpathlineto{\pgfqpoint{5.956800in}{0.397615in}}%
\pgfpathlineto{\pgfqpoint{5.920000in}{0.397615in}}%
\pgfpathlineto{\pgfqpoint{5.883200in}{0.397615in}}%
\pgfpathlineto{\pgfqpoint{5.846400in}{0.397616in}}%
\pgfpathlineto{\pgfqpoint{5.809600in}{0.397616in}}%
\pgfpathlineto{\pgfqpoint{5.772800in}{0.397617in}}%
\pgfpathlineto{\pgfqpoint{5.736000in}{0.397617in}}%
\pgfpathlineto{\pgfqpoint{5.699200in}{0.397618in}}%
\pgfpathlineto{\pgfqpoint{5.662400in}{0.397618in}}%
\pgfpathlineto{\pgfqpoint{5.625600in}{0.397619in}}%
\pgfpathlineto{\pgfqpoint{5.588800in}{0.397620in}}%
\pgfpathlineto{\pgfqpoint{5.552000in}{0.397621in}}%
\pgfpathlineto{\pgfqpoint{5.515200in}{0.397622in}}%
\pgfpathlineto{\pgfqpoint{5.478400in}{0.397623in}}%
\pgfpathlineto{\pgfqpoint{5.441600in}{0.397624in}}%
\pgfpathlineto{\pgfqpoint{5.404800in}{0.397626in}}%
\pgfpathlineto{\pgfqpoint{5.368000in}{0.397628in}}%
\pgfpathlineto{\pgfqpoint{5.331200in}{0.397629in}}%
\pgfpathlineto{\pgfqpoint{5.294400in}{0.397632in}}%
\pgfpathlineto{\pgfqpoint{5.257600in}{0.397634in}}%
\pgfpathlineto{\pgfqpoint{5.220800in}{0.397637in}}%
\pgfpathlineto{\pgfqpoint{5.184000in}{0.397640in}}%
\pgfpathlineto{\pgfqpoint{5.147200in}{0.397644in}}%
\pgfpathlineto{\pgfqpoint{5.110400in}{0.397648in}}%
\pgfpathlineto{\pgfqpoint{5.073600in}{0.397653in}}%
\pgfpathlineto{\pgfqpoint{5.036800in}{0.397658in}}%
\pgfpathlineto{\pgfqpoint{5.000000in}{0.397664in}}%
\pgfpathlineto{\pgfqpoint{4.963200in}{0.397671in}}%
\pgfpathlineto{\pgfqpoint{4.926400in}{0.397679in}}%
\pgfpathlineto{\pgfqpoint{4.889600in}{0.397688in}}%
\pgfpathlineto{\pgfqpoint{4.852800in}{0.397698in}}%
\pgfpathlineto{\pgfqpoint{4.816000in}{0.397710in}}%
\pgfpathlineto{\pgfqpoint{4.779200in}{0.397723in}}%
\pgfpathlineto{\pgfqpoint{4.742400in}{0.397738in}}%
\pgfpathlineto{\pgfqpoint{4.705600in}{0.397755in}}%
\pgfpathlineto{\pgfqpoint{4.668800in}{0.397774in}}%
\pgfpathlineto{\pgfqpoint{4.632000in}{0.397796in}}%
\pgfpathlineto{\pgfqpoint{4.595200in}{0.397821in}}%
\pgfpathlineto{\pgfqpoint{4.558400in}{0.397850in}}%
\pgfpathlineto{\pgfqpoint{4.521600in}{0.397883in}}%
\pgfpathlineto{\pgfqpoint{4.484800in}{0.397920in}}%
\pgfpathlineto{\pgfqpoint{4.448000in}{0.397963in}}%
\pgfpathlineto{\pgfqpoint{4.411200in}{0.398011in}}%
\pgfpathlineto{\pgfqpoint{4.374400in}{0.398066in}}%
\pgfpathlineto{\pgfqpoint{4.337600in}{0.398130in}}%
\pgfpathlineto{\pgfqpoint{4.300800in}{0.398202in}}%
\pgfpathlineto{\pgfqpoint{4.264000in}{0.398284in}}%
\pgfpathlineto{\pgfqpoint{4.227200in}{0.398378in}}%
\pgfpathlineto{\pgfqpoint{4.190400in}{0.398485in}}%
\pgfpathlineto{\pgfqpoint{4.153600in}{0.398608in}}%
\pgfpathlineto{\pgfqpoint{4.116800in}{0.398748in}}%
\pgfpathlineto{\pgfqpoint{4.080000in}{0.398908in}}%
\pgfpathlineto{\pgfqpoint{4.043200in}{0.399091in}}%
\pgfpathlineto{\pgfqpoint{4.006400in}{0.399301in}}%
\pgfpathlineto{\pgfqpoint{3.969600in}{0.399540in}}%
\pgfpathlineto{\pgfqpoint{3.932800in}{0.399814in}}%
\pgfpathlineto{\pgfqpoint{3.896000in}{0.400127in}}%
\pgfpathlineto{\pgfqpoint{3.859200in}{0.400485in}}%
\pgfpathlineto{\pgfqpoint{3.822400in}{0.400894in}}%
\pgfpathlineto{\pgfqpoint{3.785600in}{0.401362in}}%
\pgfpathlineto{\pgfqpoint{3.748800in}{0.401898in}}%
\pgfpathlineto{\pgfqpoint{3.712000in}{0.402511in}}%
\pgfpathlineto{\pgfqpoint{3.675200in}{0.403212in}}%
\pgfpathlineto{\pgfqpoint{3.638400in}{0.404014in}}%
\pgfpathlineto{\pgfqpoint{3.601600in}{0.404932in}}%
\pgfpathlineto{\pgfqpoint{3.564800in}{0.405982in}}%
\pgfpathlineto{\pgfqpoint{3.528000in}{0.407184in}}%
\pgfpathlineto{\pgfqpoint{3.491200in}{0.408558in}}%
\pgfpathlineto{\pgfqpoint{3.454400in}{0.410129in}}%
\pgfpathlineto{\pgfqpoint{3.417600in}{0.411926in}}%
\pgfpathlineto{\pgfqpoint{3.380800in}{0.413981in}}%
\pgfpathlineto{\pgfqpoint{3.344000in}{0.416331in}}%
\pgfpathlineto{\pgfqpoint{3.307200in}{0.419016in}}%
\pgfpathlineto{\pgfqpoint{3.270400in}{0.422085in}}%
\pgfpathlineto{\pgfqpoint{3.233600in}{0.425590in}}%
\pgfpathlineto{\pgfqpoint{3.196800in}{0.429593in}}%
\pgfpathlineto{\pgfqpoint{3.160000in}{0.434163in}}%
\pgfpathlineto{\pgfqpoint{3.123200in}{0.439377in}}%
\pgfpathlineto{\pgfqpoint{3.086400in}{0.445325in}}%
\pgfpathlineto{\pgfqpoint{3.049600in}{0.452106in}}%
\pgfpathlineto{\pgfqpoint{3.012800in}{0.459831in}}%
\pgfpathlineto{\pgfqpoint{2.976000in}{0.468627in}}%
\pgfpathlineto{\pgfqpoint{2.939200in}{0.478636in}}%
\pgfpathlineto{\pgfqpoint{2.902400in}{0.490017in}}%
\pgfpathlineto{\pgfqpoint{2.865600in}{0.502945in}}%
\pgfpathlineto{\pgfqpoint{2.828800in}{0.517620in}}%
\pgfpathlineto{\pgfqpoint{2.792000in}{0.534261in}}%
\pgfpathlineto{\pgfqpoint{2.755200in}{0.553109in}}%
\pgfpathlineto{\pgfqpoint{2.718400in}{0.574434in}}%
\pgfpathlineto{\pgfqpoint{2.681600in}{0.598529in}}%
\pgfpathlineto{\pgfqpoint{2.644800in}{0.625717in}}%
\pgfpathlineto{\pgfqpoint{2.608000in}{0.656347in}}%
\pgfpathlineto{\pgfqpoint{2.571200in}{0.690798in}}%
\pgfpathlineto{\pgfqpoint{2.534400in}{0.729477in}}%
\pgfpathlineto{\pgfqpoint{2.497600in}{0.772817in}}%
\pgfpathlineto{\pgfqpoint{2.460800in}{0.821276in}}%
\pgfpathlineto{\pgfqpoint{2.424000in}{0.875330in}}%
\pgfpathlineto{\pgfqpoint{2.387200in}{0.935471in}}%
\pgfpathlineto{\pgfqpoint{2.350400in}{1.002197in}}%
\pgfpathlineto{\pgfqpoint{2.313600in}{1.076003in}}%
\pgfpathlineto{\pgfqpoint{2.276800in}{1.157363in}}%
\pgfpathlineto{\pgfqpoint{2.240000in}{1.246719in}}%
\pgfpathlineto{\pgfqpoint{2.203200in}{1.344459in}}%
\pgfpathlineto{\pgfqpoint{2.166400in}{1.450887in}}%
\pgfpathlineto{\pgfqpoint{2.129600in}{1.566198in}}%
\pgfpathlineto{\pgfqpoint{2.092800in}{1.690443in}}%
\pgfpathlineto{\pgfqpoint{2.056000in}{1.823487in}}%
\pgfpathlineto{\pgfqpoint{2.019200in}{1.964966in}}%
\pgfpathlineto{\pgfqpoint{1.982400in}{2.114237in}}%
\pgfpathlineto{\pgfqpoint{1.945600in}{2.270329in}}%
\pgfpathlineto{\pgfqpoint{1.908800in}{2.431893in}}%
\pgfpathlineto{\pgfqpoint{1.872000in}{2.597144in}}%
\pgfpathlineto{\pgfqpoint{1.835200in}{2.763823in}}%
\pgfpathlineto{\pgfqpoint{1.798400in}{2.929159in}}%
\pgfpathlineto{\pgfqpoint{1.761600in}{3.089851in}}%
\pgfpathlineto{\pgfqpoint{1.724800in}{3.242069in}}%
\pgfpathlineto{\pgfqpoint{1.688000in}{3.381490in}}%
\pgfpathlineto{\pgfqpoint{1.651200in}{3.503370in}}%
\pgfpathlineto{\pgfqpoint{1.614400in}{3.602663in}}%
\pgfpathlineto{\pgfqpoint{1.577600in}{3.674201in}}%
\pgfpathlineto{\pgfqpoint{1.540800in}{3.712922in}}%
\pgfpathlineto{\pgfqpoint{1.504000in}{3.714172in}}%
\pgfpathlineto{\pgfqpoint{1.467200in}{3.674050in}}%
\pgfpathlineto{\pgfqpoint{1.430400in}{3.589806in}}%
\pgfpathlineto{\pgfqpoint{1.393600in}{3.460244in}}%
\pgfpathlineto{\pgfqpoint{1.356800in}{3.286117in}}%
\pgfpathlineto{\pgfqpoint{1.320000in}{3.070442in}}%
\pgfpathlineto{\pgfqpoint{1.283200in}{2.818697in}}%
\pgfpathlineto{\pgfqpoint{1.246400in}{2.538815in}}%
\pgfpathlineto{\pgfqpoint{1.209600in}{2.240936in}}%
\pgfpathlineto{\pgfqpoint{1.172800in}{1.936858in}}%
\pgfpathlineto{\pgfqpoint{1.136000in}{1.639189in}}%
\pgfpathlineto{\pgfqpoint{1.099200in}{1.360254in}}%
\pgfpathlineto{\pgfqpoint{1.062400in}{1.110852in}}%
\pgfpathlineto{\pgfqpoint{1.025600in}{0.899055in}}%
\pgfpathlineto{\pgfqpoint{0.988800in}{0.729278in}}%
\pgfpathlineto{\pgfqpoint{0.952000in}{0.601833in}}%
\pgfpathlineto{\pgfqpoint{0.915200in}{0.513153in}}%
\pgfpathlineto{\pgfqpoint{0.878400in}{0.456693in}}%
\pgfpathlineto{\pgfqpoint{0.841600in}{0.424346in}}%
\pgfpathlineto{\pgfqpoint{0.804800in}{0.408026in}}%
\pgfpathlineto{\pgfqpoint{0.768000in}{0.400979in}}%
\pgfpathlineto{\pgfqpoint{0.731200in}{0.398472in}}%
\pgfpathlineto{\pgfqpoint{0.694400in}{0.397774in}}%
\pgfpathlineto{\pgfqpoint{0.657600in}{0.397633in}}%
\pgfpathlineto{\pgfqpoint{0.620800in}{0.397614in}}%
\pgfpathlineto{\pgfqpoint{0.584000in}{0.397613in}}%
\pgfpathlineto{\pgfqpoint{0.547200in}{0.397613in}}%
\pgfpathlineto{\pgfqpoint{0.510400in}{0.397613in}}%
\pgfpathlineto{\pgfqpoint{0.473600in}{0.397613in}}%
\pgfpathlineto{\pgfqpoint{0.436800in}{0.397613in}}%
\pgfpathlineto{\pgfqpoint{0.400000in}{0.397613in}}%
\pgfpathlineto{\pgfqpoint{0.400000in}{0.397613in}}%
\pgfpathclose%
\pgfusepath{stroke,fill}%
}%
\begin{pgfscope}%
\pgfsys@transformshift{0.000000in}{0.000000in}%
\pgfsys@useobject{currentmarker}{}%
\end{pgfscope}%
\end{pgfscope}%
\begin{pgfscope}%
\pgfpathrectangle{\pgfqpoint{0.400000in}{0.400000in}}{\pgfqpoint{7.360000in}{3.480000in}}%
\pgfusepath{clip}%
\pgfsetbuttcap%
\pgfsetroundjoin%
\definecolor{currentfill}{rgb}{0.000000,0.000000,0.545098}%
\pgfsetfillcolor{currentfill}%
\pgfsetfillopacity{0.500000}%
\pgfsetlinewidth{1.003750pt}%
\definecolor{currentstroke}{rgb}{0.000000,0.000000,0.545098}%
\pgfsetstrokecolor{currentstroke}%
\pgfsetstrokeopacity{0.500000}%
\pgfsetdash{}{0pt}%
\pgfsys@defobject{currentmarker}{\pgfqpoint{0.400000in}{0.397613in}}{\pgfqpoint{7.760000in}{3.714172in}}{%
\pgfpathmoveto{\pgfqpoint{0.400000in}{0.397613in}}%
\pgfpathlineto{\pgfqpoint{0.400000in}{0.397613in}}%
\pgfpathlineto{\pgfqpoint{0.436800in}{0.397613in}}%
\pgfpathlineto{\pgfqpoint{0.473600in}{0.397613in}}%
\pgfpathlineto{\pgfqpoint{0.510400in}{0.397613in}}%
\pgfpathlineto{\pgfqpoint{0.547200in}{0.397613in}}%
\pgfpathlineto{\pgfqpoint{0.584000in}{0.397613in}}%
\pgfpathlineto{\pgfqpoint{0.620800in}{0.397613in}}%
\pgfpathlineto{\pgfqpoint{0.657600in}{0.397613in}}%
\pgfpathlineto{\pgfqpoint{0.694400in}{0.397613in}}%
\pgfpathlineto{\pgfqpoint{0.731200in}{0.397613in}}%
\pgfpathlineto{\pgfqpoint{0.768000in}{0.397613in}}%
\pgfpathlineto{\pgfqpoint{0.804800in}{0.397613in}}%
\pgfpathlineto{\pgfqpoint{0.841600in}{0.397613in}}%
\pgfpathlineto{\pgfqpoint{0.878400in}{0.397613in}}%
\pgfpathlineto{\pgfqpoint{0.915200in}{0.397613in}}%
\pgfpathlineto{\pgfqpoint{0.952000in}{0.397613in}}%
\pgfpathlineto{\pgfqpoint{0.988800in}{0.397613in}}%
\pgfpathlineto{\pgfqpoint{1.025600in}{0.397613in}}%
\pgfpathlineto{\pgfqpoint{1.062400in}{0.397613in}}%
\pgfpathlineto{\pgfqpoint{1.099200in}{0.397613in}}%
\pgfpathlineto{\pgfqpoint{1.136000in}{0.397613in}}%
\pgfpathlineto{\pgfqpoint{1.172800in}{0.397613in}}%
\pgfpathlineto{\pgfqpoint{1.209600in}{0.397613in}}%
\pgfpathlineto{\pgfqpoint{1.246400in}{0.397613in}}%
\pgfpathlineto{\pgfqpoint{1.283200in}{0.397613in}}%
\pgfpathlineto{\pgfqpoint{1.320000in}{0.397613in}}%
\pgfpathlineto{\pgfqpoint{1.356800in}{0.397613in}}%
\pgfpathlineto{\pgfqpoint{1.393600in}{0.397613in}}%
\pgfpathlineto{\pgfqpoint{1.430400in}{0.397613in}}%
\pgfpathlineto{\pgfqpoint{1.467200in}{0.397613in}}%
\pgfpathlineto{\pgfqpoint{1.504000in}{0.397613in}}%
\pgfpathlineto{\pgfqpoint{1.540800in}{0.397613in}}%
\pgfpathlineto{\pgfqpoint{1.577600in}{0.397613in}}%
\pgfpathlineto{\pgfqpoint{1.614400in}{0.397613in}}%
\pgfpathlineto{\pgfqpoint{1.651200in}{0.397613in}}%
\pgfpathlineto{\pgfqpoint{1.688000in}{0.397613in}}%
\pgfpathlineto{\pgfqpoint{1.724800in}{0.397613in}}%
\pgfpathlineto{\pgfqpoint{1.761600in}{0.397613in}}%
\pgfpathlineto{\pgfqpoint{1.798400in}{0.397613in}}%
\pgfpathlineto{\pgfqpoint{1.835200in}{0.397613in}}%
\pgfpathlineto{\pgfqpoint{1.872000in}{0.397613in}}%
\pgfpathlineto{\pgfqpoint{1.908800in}{0.397613in}}%
\pgfpathlineto{\pgfqpoint{1.945600in}{0.397613in}}%
\pgfpathlineto{\pgfqpoint{1.982400in}{0.397613in}}%
\pgfpathlineto{\pgfqpoint{2.019200in}{0.397613in}}%
\pgfpathlineto{\pgfqpoint{2.056000in}{0.397613in}}%
\pgfpathlineto{\pgfqpoint{2.092800in}{0.397613in}}%
\pgfpathlineto{\pgfqpoint{2.129600in}{0.397613in}}%
\pgfpathlineto{\pgfqpoint{2.166400in}{0.397613in}}%
\pgfpathlineto{\pgfqpoint{2.203200in}{0.397613in}}%
\pgfpathlineto{\pgfqpoint{2.240000in}{0.397613in}}%
\pgfpathlineto{\pgfqpoint{2.276800in}{0.397613in}}%
\pgfpathlineto{\pgfqpoint{2.313600in}{0.397613in}}%
\pgfpathlineto{\pgfqpoint{2.350400in}{0.397613in}}%
\pgfpathlineto{\pgfqpoint{2.387200in}{0.397613in}}%
\pgfpathlineto{\pgfqpoint{2.424000in}{0.397613in}}%
\pgfpathlineto{\pgfqpoint{2.460800in}{0.397613in}}%
\pgfpathlineto{\pgfqpoint{2.497600in}{0.397613in}}%
\pgfpathlineto{\pgfqpoint{2.534400in}{0.397613in}}%
\pgfpathlineto{\pgfqpoint{2.571200in}{0.397613in}}%
\pgfpathlineto{\pgfqpoint{2.608000in}{0.397613in}}%
\pgfpathlineto{\pgfqpoint{2.644800in}{0.397613in}}%
\pgfpathlineto{\pgfqpoint{2.681600in}{0.397613in}}%
\pgfpathlineto{\pgfqpoint{2.718400in}{0.397613in}}%
\pgfpathlineto{\pgfqpoint{2.755200in}{0.397613in}}%
\pgfpathlineto{\pgfqpoint{2.792000in}{0.397613in}}%
\pgfpathlineto{\pgfqpoint{2.828800in}{0.397613in}}%
\pgfpathlineto{\pgfqpoint{2.865600in}{0.397613in}}%
\pgfpathlineto{\pgfqpoint{2.902400in}{0.397613in}}%
\pgfpathlineto{\pgfqpoint{2.939200in}{0.397613in}}%
\pgfpathlineto{\pgfqpoint{2.976000in}{0.397613in}}%
\pgfpathlineto{\pgfqpoint{3.012800in}{0.397613in}}%
\pgfpathlineto{\pgfqpoint{3.049600in}{0.397613in}}%
\pgfpathlineto{\pgfqpoint{3.086400in}{0.397613in}}%
\pgfpathlineto{\pgfqpoint{3.123200in}{0.397613in}}%
\pgfpathlineto{\pgfqpoint{3.160000in}{0.397613in}}%
\pgfpathlineto{\pgfqpoint{3.196800in}{0.397613in}}%
\pgfpathlineto{\pgfqpoint{3.233600in}{0.397613in}}%
\pgfpathlineto{\pgfqpoint{3.270400in}{0.397613in}}%
\pgfpathlineto{\pgfqpoint{3.307200in}{0.397613in}}%
\pgfpathlineto{\pgfqpoint{3.344000in}{0.397613in}}%
\pgfpathlineto{\pgfqpoint{3.380800in}{0.397613in}}%
\pgfpathlineto{\pgfqpoint{3.417600in}{0.397613in}}%
\pgfpathlineto{\pgfqpoint{3.454400in}{0.397613in}}%
\pgfpathlineto{\pgfqpoint{3.491200in}{0.397613in}}%
\pgfpathlineto{\pgfqpoint{3.528000in}{0.397613in}}%
\pgfpathlineto{\pgfqpoint{3.564800in}{0.397613in}}%
\pgfpathlineto{\pgfqpoint{3.601600in}{0.397613in}}%
\pgfpathlineto{\pgfqpoint{3.638400in}{0.397613in}}%
\pgfpathlineto{\pgfqpoint{3.675200in}{0.397613in}}%
\pgfpathlineto{\pgfqpoint{3.712000in}{0.397613in}}%
\pgfpathlineto{\pgfqpoint{3.748800in}{0.397613in}}%
\pgfpathlineto{\pgfqpoint{3.785600in}{0.397613in}}%
\pgfpathlineto{\pgfqpoint{3.822400in}{0.397613in}}%
\pgfpathlineto{\pgfqpoint{3.859200in}{0.397613in}}%
\pgfpathlineto{\pgfqpoint{3.896000in}{0.397613in}}%
\pgfpathlineto{\pgfqpoint{3.932800in}{0.397613in}}%
\pgfpathlineto{\pgfqpoint{3.969600in}{0.397613in}}%
\pgfpathlineto{\pgfqpoint{4.006400in}{0.397613in}}%
\pgfpathlineto{\pgfqpoint{4.043200in}{0.397613in}}%
\pgfpathlineto{\pgfqpoint{4.080000in}{0.397613in}}%
\pgfpathlineto{\pgfqpoint{4.116800in}{0.397613in}}%
\pgfpathlineto{\pgfqpoint{4.153600in}{0.397613in}}%
\pgfpathlineto{\pgfqpoint{4.190400in}{0.397613in}}%
\pgfpathlineto{\pgfqpoint{4.227200in}{0.397613in}}%
\pgfpathlineto{\pgfqpoint{4.264000in}{0.397613in}}%
\pgfpathlineto{\pgfqpoint{4.300800in}{0.397613in}}%
\pgfpathlineto{\pgfqpoint{4.337600in}{0.397613in}}%
\pgfpathlineto{\pgfqpoint{4.374400in}{0.397613in}}%
\pgfpathlineto{\pgfqpoint{4.411200in}{0.397613in}}%
\pgfpathlineto{\pgfqpoint{4.448000in}{0.397613in}}%
\pgfpathlineto{\pgfqpoint{4.484800in}{0.397613in}}%
\pgfpathlineto{\pgfqpoint{4.521600in}{0.397613in}}%
\pgfpathlineto{\pgfqpoint{4.558400in}{0.397613in}}%
\pgfpathlineto{\pgfqpoint{4.595200in}{0.397613in}}%
\pgfpathlineto{\pgfqpoint{4.632000in}{0.397613in}}%
\pgfpathlineto{\pgfqpoint{4.668800in}{0.397613in}}%
\pgfpathlineto{\pgfqpoint{4.705600in}{0.397613in}}%
\pgfpathlineto{\pgfqpoint{4.742400in}{0.397613in}}%
\pgfpathlineto{\pgfqpoint{4.779200in}{0.397613in}}%
\pgfpathlineto{\pgfqpoint{4.816000in}{0.397613in}}%
\pgfpathlineto{\pgfqpoint{4.852800in}{0.397613in}}%
\pgfpathlineto{\pgfqpoint{4.889600in}{0.397613in}}%
\pgfpathlineto{\pgfqpoint{4.926400in}{0.397613in}}%
\pgfpathlineto{\pgfqpoint{4.963200in}{0.397613in}}%
\pgfpathlineto{\pgfqpoint{5.000000in}{0.397613in}}%
\pgfpathlineto{\pgfqpoint{5.036800in}{0.397613in}}%
\pgfpathlineto{\pgfqpoint{5.073600in}{0.397613in}}%
\pgfpathlineto{\pgfqpoint{5.110400in}{0.397613in}}%
\pgfpathlineto{\pgfqpoint{5.147200in}{0.397613in}}%
\pgfpathlineto{\pgfqpoint{5.184000in}{0.397613in}}%
\pgfpathlineto{\pgfqpoint{5.220800in}{0.397613in}}%
\pgfpathlineto{\pgfqpoint{5.257600in}{0.397613in}}%
\pgfpathlineto{\pgfqpoint{5.294400in}{0.397613in}}%
\pgfpathlineto{\pgfqpoint{5.331200in}{0.397613in}}%
\pgfpathlineto{\pgfqpoint{5.368000in}{0.397613in}}%
\pgfpathlineto{\pgfqpoint{5.404800in}{0.397613in}}%
\pgfpathlineto{\pgfqpoint{5.441600in}{0.397613in}}%
\pgfpathlineto{\pgfqpoint{5.478400in}{0.397613in}}%
\pgfpathlineto{\pgfqpoint{5.515200in}{0.397613in}}%
\pgfpathlineto{\pgfqpoint{5.552000in}{0.397613in}}%
\pgfpathlineto{\pgfqpoint{5.588800in}{0.397613in}}%
\pgfpathlineto{\pgfqpoint{5.625600in}{0.397613in}}%
\pgfpathlineto{\pgfqpoint{5.662400in}{0.397613in}}%
\pgfpathlineto{\pgfqpoint{5.699200in}{0.397613in}}%
\pgfpathlineto{\pgfqpoint{5.736000in}{0.397613in}}%
\pgfpathlineto{\pgfqpoint{5.772800in}{0.397613in}}%
\pgfpathlineto{\pgfqpoint{5.809600in}{0.397613in}}%
\pgfpathlineto{\pgfqpoint{5.846400in}{0.397613in}}%
\pgfpathlineto{\pgfqpoint{5.883200in}{0.397613in}}%
\pgfpathlineto{\pgfqpoint{5.920000in}{0.397613in}}%
\pgfpathlineto{\pgfqpoint{5.956800in}{0.397613in}}%
\pgfpathlineto{\pgfqpoint{5.993600in}{0.397613in}}%
\pgfpathlineto{\pgfqpoint{6.030400in}{0.397613in}}%
\pgfpathlineto{\pgfqpoint{6.067200in}{0.397613in}}%
\pgfpathlineto{\pgfqpoint{6.104000in}{0.397613in}}%
\pgfpathlineto{\pgfqpoint{6.140800in}{0.397613in}}%
\pgfpathlineto{\pgfqpoint{6.177600in}{0.397613in}}%
\pgfpathlineto{\pgfqpoint{6.214400in}{0.397613in}}%
\pgfpathlineto{\pgfqpoint{6.251200in}{0.397613in}}%
\pgfpathlineto{\pgfqpoint{6.288000in}{0.397613in}}%
\pgfpathlineto{\pgfqpoint{6.324800in}{0.397613in}}%
\pgfpathlineto{\pgfqpoint{6.361600in}{0.397613in}}%
\pgfpathlineto{\pgfqpoint{6.398400in}{0.397613in}}%
\pgfpathlineto{\pgfqpoint{6.435200in}{0.397613in}}%
\pgfpathlineto{\pgfqpoint{6.472000in}{0.397613in}}%
\pgfpathlineto{\pgfqpoint{6.508800in}{0.397613in}}%
\pgfpathlineto{\pgfqpoint{6.545600in}{0.397613in}}%
\pgfpathlineto{\pgfqpoint{6.582400in}{0.397613in}}%
\pgfpathlineto{\pgfqpoint{6.619200in}{0.397613in}}%
\pgfpathlineto{\pgfqpoint{6.656000in}{0.397613in}}%
\pgfpathlineto{\pgfqpoint{6.692800in}{0.397613in}}%
\pgfpathlineto{\pgfqpoint{6.729600in}{0.397613in}}%
\pgfpathlineto{\pgfqpoint{6.766400in}{0.397613in}}%
\pgfpathlineto{\pgfqpoint{6.803200in}{0.397613in}}%
\pgfpathlineto{\pgfqpoint{6.840000in}{0.397613in}}%
\pgfpathlineto{\pgfqpoint{6.876800in}{0.397613in}}%
\pgfpathlineto{\pgfqpoint{6.913600in}{0.397613in}}%
\pgfpathlineto{\pgfqpoint{6.950400in}{0.397613in}}%
\pgfpathlineto{\pgfqpoint{6.987200in}{0.397613in}}%
\pgfpathlineto{\pgfqpoint{7.024000in}{0.397613in}}%
\pgfpathlineto{\pgfqpoint{7.060800in}{0.397613in}}%
\pgfpathlineto{\pgfqpoint{7.097600in}{0.397613in}}%
\pgfpathlineto{\pgfqpoint{7.134400in}{0.397613in}}%
\pgfpathlineto{\pgfqpoint{7.171200in}{0.397613in}}%
\pgfpathlineto{\pgfqpoint{7.208000in}{0.397613in}}%
\pgfpathlineto{\pgfqpoint{7.244800in}{0.397613in}}%
\pgfpathlineto{\pgfqpoint{7.281600in}{0.397613in}}%
\pgfpathlineto{\pgfqpoint{7.318400in}{0.397613in}}%
\pgfpathlineto{\pgfqpoint{7.355200in}{0.397613in}}%
\pgfpathlineto{\pgfqpoint{7.392000in}{0.397613in}}%
\pgfpathlineto{\pgfqpoint{7.428800in}{0.397613in}}%
\pgfpathlineto{\pgfqpoint{7.465600in}{0.397613in}}%
\pgfpathlineto{\pgfqpoint{7.502400in}{0.397613in}}%
\pgfpathlineto{\pgfqpoint{7.539200in}{0.397613in}}%
\pgfpathlineto{\pgfqpoint{7.576000in}{0.397613in}}%
\pgfpathlineto{\pgfqpoint{7.612800in}{0.397613in}}%
\pgfpathlineto{\pgfqpoint{7.649600in}{0.397613in}}%
\pgfpathlineto{\pgfqpoint{7.686400in}{0.397613in}}%
\pgfpathlineto{\pgfqpoint{7.723200in}{0.397613in}}%
\pgfpathlineto{\pgfqpoint{7.760000in}{0.397613in}}%
\pgfpathlineto{\pgfqpoint{7.760000in}{0.398908in}}%
\pgfpathlineto{\pgfqpoint{7.760000in}{0.398908in}}%
\pgfpathlineto{\pgfqpoint{7.723200in}{0.399091in}}%
\pgfpathlineto{\pgfqpoint{7.686400in}{0.399301in}}%
\pgfpathlineto{\pgfqpoint{7.649600in}{0.399540in}}%
\pgfpathlineto{\pgfqpoint{7.612800in}{0.399814in}}%
\pgfpathlineto{\pgfqpoint{7.576000in}{0.400127in}}%
\pgfpathlineto{\pgfqpoint{7.539200in}{0.400485in}}%
\pgfpathlineto{\pgfqpoint{7.502400in}{0.400894in}}%
\pgfpathlineto{\pgfqpoint{7.465600in}{0.401362in}}%
\pgfpathlineto{\pgfqpoint{7.428800in}{0.401898in}}%
\pgfpathlineto{\pgfqpoint{7.392000in}{0.402511in}}%
\pgfpathlineto{\pgfqpoint{7.355200in}{0.403212in}}%
\pgfpathlineto{\pgfqpoint{7.318400in}{0.404014in}}%
\pgfpathlineto{\pgfqpoint{7.281600in}{0.404932in}}%
\pgfpathlineto{\pgfqpoint{7.244800in}{0.405982in}}%
\pgfpathlineto{\pgfqpoint{7.208000in}{0.407184in}}%
\pgfpathlineto{\pgfqpoint{7.171200in}{0.408558in}}%
\pgfpathlineto{\pgfqpoint{7.134400in}{0.410129in}}%
\pgfpathlineto{\pgfqpoint{7.097600in}{0.411926in}}%
\pgfpathlineto{\pgfqpoint{7.060800in}{0.413981in}}%
\pgfpathlineto{\pgfqpoint{7.024000in}{0.416331in}}%
\pgfpathlineto{\pgfqpoint{6.987200in}{0.419016in}}%
\pgfpathlineto{\pgfqpoint{6.950400in}{0.422085in}}%
\pgfpathlineto{\pgfqpoint{6.913600in}{0.425590in}}%
\pgfpathlineto{\pgfqpoint{6.876800in}{0.429593in}}%
\pgfpathlineto{\pgfqpoint{6.840000in}{0.434163in}}%
\pgfpathlineto{\pgfqpoint{6.803200in}{0.439377in}}%
\pgfpathlineto{\pgfqpoint{6.766400in}{0.445325in}}%
\pgfpathlineto{\pgfqpoint{6.729600in}{0.452106in}}%
\pgfpathlineto{\pgfqpoint{6.692800in}{0.459831in}}%
\pgfpathlineto{\pgfqpoint{6.656000in}{0.468627in}}%
\pgfpathlineto{\pgfqpoint{6.619200in}{0.478636in}}%
\pgfpathlineto{\pgfqpoint{6.582400in}{0.490017in}}%
\pgfpathlineto{\pgfqpoint{6.545600in}{0.502945in}}%
\pgfpathlineto{\pgfqpoint{6.508800in}{0.517620in}}%
\pgfpathlineto{\pgfqpoint{6.472000in}{0.534261in}}%
\pgfpathlineto{\pgfqpoint{6.435200in}{0.553109in}}%
\pgfpathlineto{\pgfqpoint{6.398400in}{0.574434in}}%
\pgfpathlineto{\pgfqpoint{6.361600in}{0.598529in}}%
\pgfpathlineto{\pgfqpoint{6.324800in}{0.625717in}}%
\pgfpathlineto{\pgfqpoint{6.288000in}{0.656347in}}%
\pgfpathlineto{\pgfqpoint{6.251200in}{0.690798in}}%
\pgfpathlineto{\pgfqpoint{6.214400in}{0.729477in}}%
\pgfpathlineto{\pgfqpoint{6.177600in}{0.772817in}}%
\pgfpathlineto{\pgfqpoint{6.140800in}{0.821276in}}%
\pgfpathlineto{\pgfqpoint{6.104000in}{0.875330in}}%
\pgfpathlineto{\pgfqpoint{6.067200in}{0.935471in}}%
\pgfpathlineto{\pgfqpoint{6.030400in}{1.002197in}}%
\pgfpathlineto{\pgfqpoint{5.993600in}{1.076003in}}%
\pgfpathlineto{\pgfqpoint{5.956800in}{1.157363in}}%
\pgfpathlineto{\pgfqpoint{5.920000in}{1.246719in}}%
\pgfpathlineto{\pgfqpoint{5.883200in}{1.344459in}}%
\pgfpathlineto{\pgfqpoint{5.846400in}{1.450887in}}%
\pgfpathlineto{\pgfqpoint{5.809600in}{1.566198in}}%
\pgfpathlineto{\pgfqpoint{5.772800in}{1.690443in}}%
\pgfpathlineto{\pgfqpoint{5.736000in}{1.823487in}}%
\pgfpathlineto{\pgfqpoint{5.699200in}{1.964966in}}%
\pgfpathlineto{\pgfqpoint{5.662400in}{2.114237in}}%
\pgfpathlineto{\pgfqpoint{5.625600in}{2.270329in}}%
\pgfpathlineto{\pgfqpoint{5.588800in}{2.431893in}}%
\pgfpathlineto{\pgfqpoint{5.552000in}{2.597144in}}%
\pgfpathlineto{\pgfqpoint{5.515200in}{2.763823in}}%
\pgfpathlineto{\pgfqpoint{5.478400in}{2.929159in}}%
\pgfpathlineto{\pgfqpoint{5.441600in}{3.089851in}}%
\pgfpathlineto{\pgfqpoint{5.404800in}{3.242069in}}%
\pgfpathlineto{\pgfqpoint{5.368000in}{3.381490in}}%
\pgfpathlineto{\pgfqpoint{5.331200in}{3.503370in}}%
\pgfpathlineto{\pgfqpoint{5.294400in}{3.602663in}}%
\pgfpathlineto{\pgfqpoint{5.257600in}{3.674201in}}%
\pgfpathlineto{\pgfqpoint{5.220800in}{3.712922in}}%
\pgfpathlineto{\pgfqpoint{5.184000in}{3.714172in}}%
\pgfpathlineto{\pgfqpoint{5.147200in}{3.674050in}}%
\pgfpathlineto{\pgfqpoint{5.110400in}{3.589806in}}%
\pgfpathlineto{\pgfqpoint{5.073600in}{3.460244in}}%
\pgfpathlineto{\pgfqpoint{5.036800in}{3.286117in}}%
\pgfpathlineto{\pgfqpoint{5.000000in}{3.070442in}}%
\pgfpathlineto{\pgfqpoint{4.963200in}{2.818697in}}%
\pgfpathlineto{\pgfqpoint{4.926400in}{2.538815in}}%
\pgfpathlineto{\pgfqpoint{4.889600in}{2.240936in}}%
\pgfpathlineto{\pgfqpoint{4.852800in}{1.936858in}}%
\pgfpathlineto{\pgfqpoint{4.816000in}{1.639189in}}%
\pgfpathlineto{\pgfqpoint{4.779200in}{1.360254in}}%
\pgfpathlineto{\pgfqpoint{4.742400in}{1.110852in}}%
\pgfpathlineto{\pgfqpoint{4.705600in}{0.899055in}}%
\pgfpathlineto{\pgfqpoint{4.668800in}{0.729278in}}%
\pgfpathlineto{\pgfqpoint{4.632000in}{0.601833in}}%
\pgfpathlineto{\pgfqpoint{4.595200in}{0.513153in}}%
\pgfpathlineto{\pgfqpoint{4.558400in}{0.456693in}}%
\pgfpathlineto{\pgfqpoint{4.521600in}{0.424346in}}%
\pgfpathlineto{\pgfqpoint{4.484800in}{0.408026in}}%
\pgfpathlineto{\pgfqpoint{4.448000in}{0.400979in}}%
\pgfpathlineto{\pgfqpoint{4.411200in}{0.398472in}}%
\pgfpathlineto{\pgfqpoint{4.374400in}{0.397774in}}%
\pgfpathlineto{\pgfqpoint{4.337600in}{0.397633in}}%
\pgfpathlineto{\pgfqpoint{4.300800in}{0.397614in}}%
\pgfpathlineto{\pgfqpoint{4.264000in}{0.397613in}}%
\pgfpathlineto{\pgfqpoint{4.227200in}{0.397613in}}%
\pgfpathlineto{\pgfqpoint{4.190400in}{0.397613in}}%
\pgfpathlineto{\pgfqpoint{4.153600in}{0.397613in}}%
\pgfpathlineto{\pgfqpoint{4.116800in}{0.397613in}}%
\pgfpathlineto{\pgfqpoint{4.080000in}{0.397613in}}%
\pgfpathlineto{\pgfqpoint{4.043200in}{0.397613in}}%
\pgfpathlineto{\pgfqpoint{4.006400in}{0.397613in}}%
\pgfpathlineto{\pgfqpoint{3.969600in}{0.397613in}}%
\pgfpathlineto{\pgfqpoint{3.932800in}{0.397613in}}%
\pgfpathlineto{\pgfqpoint{3.896000in}{0.397613in}}%
\pgfpathlineto{\pgfqpoint{3.859200in}{0.397613in}}%
\pgfpathlineto{\pgfqpoint{3.822400in}{0.397613in}}%
\pgfpathlineto{\pgfqpoint{3.785600in}{0.397613in}}%
\pgfpathlineto{\pgfqpoint{3.748800in}{0.397613in}}%
\pgfpathlineto{\pgfqpoint{3.712000in}{0.397613in}}%
\pgfpathlineto{\pgfqpoint{3.675200in}{0.397613in}}%
\pgfpathlineto{\pgfqpoint{3.638400in}{0.397613in}}%
\pgfpathlineto{\pgfqpoint{3.601600in}{0.397613in}}%
\pgfpathlineto{\pgfqpoint{3.564800in}{0.397613in}}%
\pgfpathlineto{\pgfqpoint{3.528000in}{0.397613in}}%
\pgfpathlineto{\pgfqpoint{3.491200in}{0.397613in}}%
\pgfpathlineto{\pgfqpoint{3.454400in}{0.397613in}}%
\pgfpathlineto{\pgfqpoint{3.417600in}{0.397613in}}%
\pgfpathlineto{\pgfqpoint{3.380800in}{0.397613in}}%
\pgfpathlineto{\pgfqpoint{3.344000in}{0.397613in}}%
\pgfpathlineto{\pgfqpoint{3.307200in}{0.397613in}}%
\pgfpathlineto{\pgfqpoint{3.270400in}{0.397613in}}%
\pgfpathlineto{\pgfqpoint{3.233600in}{0.397613in}}%
\pgfpathlineto{\pgfqpoint{3.196800in}{0.397613in}}%
\pgfpathlineto{\pgfqpoint{3.160000in}{0.397613in}}%
\pgfpathlineto{\pgfqpoint{3.123200in}{0.397613in}}%
\pgfpathlineto{\pgfqpoint{3.086400in}{0.397613in}}%
\pgfpathlineto{\pgfqpoint{3.049600in}{0.397613in}}%
\pgfpathlineto{\pgfqpoint{3.012800in}{0.397613in}}%
\pgfpathlineto{\pgfqpoint{2.976000in}{0.397613in}}%
\pgfpathlineto{\pgfqpoint{2.939200in}{0.397613in}}%
\pgfpathlineto{\pgfqpoint{2.902400in}{0.397613in}}%
\pgfpathlineto{\pgfqpoint{2.865600in}{0.397613in}}%
\pgfpathlineto{\pgfqpoint{2.828800in}{0.397613in}}%
\pgfpathlineto{\pgfqpoint{2.792000in}{0.397613in}}%
\pgfpathlineto{\pgfqpoint{2.755200in}{0.397613in}}%
\pgfpathlineto{\pgfqpoint{2.718400in}{0.397613in}}%
\pgfpathlineto{\pgfqpoint{2.681600in}{0.397613in}}%
\pgfpathlineto{\pgfqpoint{2.644800in}{0.397613in}}%
\pgfpathlineto{\pgfqpoint{2.608000in}{0.397613in}}%
\pgfpathlineto{\pgfqpoint{2.571200in}{0.397613in}}%
\pgfpathlineto{\pgfqpoint{2.534400in}{0.397613in}}%
\pgfpathlineto{\pgfqpoint{2.497600in}{0.397613in}}%
\pgfpathlineto{\pgfqpoint{2.460800in}{0.397613in}}%
\pgfpathlineto{\pgfqpoint{2.424000in}{0.397613in}}%
\pgfpathlineto{\pgfqpoint{2.387200in}{0.397613in}}%
\pgfpathlineto{\pgfqpoint{2.350400in}{0.397613in}}%
\pgfpathlineto{\pgfqpoint{2.313600in}{0.397613in}}%
\pgfpathlineto{\pgfqpoint{2.276800in}{0.397613in}}%
\pgfpathlineto{\pgfqpoint{2.240000in}{0.397613in}}%
\pgfpathlineto{\pgfqpoint{2.203200in}{0.397613in}}%
\pgfpathlineto{\pgfqpoint{2.166400in}{0.397613in}}%
\pgfpathlineto{\pgfqpoint{2.129600in}{0.397613in}}%
\pgfpathlineto{\pgfqpoint{2.092800in}{0.397613in}}%
\pgfpathlineto{\pgfqpoint{2.056000in}{0.397613in}}%
\pgfpathlineto{\pgfqpoint{2.019200in}{0.397613in}}%
\pgfpathlineto{\pgfqpoint{1.982400in}{0.397613in}}%
\pgfpathlineto{\pgfqpoint{1.945600in}{0.397613in}}%
\pgfpathlineto{\pgfqpoint{1.908800in}{0.397613in}}%
\pgfpathlineto{\pgfqpoint{1.872000in}{0.397613in}}%
\pgfpathlineto{\pgfqpoint{1.835200in}{0.397613in}}%
\pgfpathlineto{\pgfqpoint{1.798400in}{0.397613in}}%
\pgfpathlineto{\pgfqpoint{1.761600in}{0.397613in}}%
\pgfpathlineto{\pgfqpoint{1.724800in}{0.397613in}}%
\pgfpathlineto{\pgfqpoint{1.688000in}{0.397613in}}%
\pgfpathlineto{\pgfqpoint{1.651200in}{0.397613in}}%
\pgfpathlineto{\pgfqpoint{1.614400in}{0.397613in}}%
\pgfpathlineto{\pgfqpoint{1.577600in}{0.397613in}}%
\pgfpathlineto{\pgfqpoint{1.540800in}{0.397613in}}%
\pgfpathlineto{\pgfqpoint{1.504000in}{0.397613in}}%
\pgfpathlineto{\pgfqpoint{1.467200in}{0.397613in}}%
\pgfpathlineto{\pgfqpoint{1.430400in}{0.397613in}}%
\pgfpathlineto{\pgfqpoint{1.393600in}{0.397613in}}%
\pgfpathlineto{\pgfqpoint{1.356800in}{0.397613in}}%
\pgfpathlineto{\pgfqpoint{1.320000in}{0.397613in}}%
\pgfpathlineto{\pgfqpoint{1.283200in}{0.397613in}}%
\pgfpathlineto{\pgfqpoint{1.246400in}{0.397613in}}%
\pgfpathlineto{\pgfqpoint{1.209600in}{0.397613in}}%
\pgfpathlineto{\pgfqpoint{1.172800in}{0.397613in}}%
\pgfpathlineto{\pgfqpoint{1.136000in}{0.397613in}}%
\pgfpathlineto{\pgfqpoint{1.099200in}{0.397613in}}%
\pgfpathlineto{\pgfqpoint{1.062400in}{0.397613in}}%
\pgfpathlineto{\pgfqpoint{1.025600in}{0.397613in}}%
\pgfpathlineto{\pgfqpoint{0.988800in}{0.397613in}}%
\pgfpathlineto{\pgfqpoint{0.952000in}{0.397613in}}%
\pgfpathlineto{\pgfqpoint{0.915200in}{0.397613in}}%
\pgfpathlineto{\pgfqpoint{0.878400in}{0.397613in}}%
\pgfpathlineto{\pgfqpoint{0.841600in}{0.397613in}}%
\pgfpathlineto{\pgfqpoint{0.804800in}{0.397613in}}%
\pgfpathlineto{\pgfqpoint{0.768000in}{0.397613in}}%
\pgfpathlineto{\pgfqpoint{0.731200in}{0.397613in}}%
\pgfpathlineto{\pgfqpoint{0.694400in}{0.397613in}}%
\pgfpathlineto{\pgfqpoint{0.657600in}{0.397613in}}%
\pgfpathlineto{\pgfqpoint{0.620800in}{0.397613in}}%
\pgfpathlineto{\pgfqpoint{0.584000in}{0.397613in}}%
\pgfpathlineto{\pgfqpoint{0.547200in}{0.397613in}}%
\pgfpathlineto{\pgfqpoint{0.510400in}{0.397613in}}%
\pgfpathlineto{\pgfqpoint{0.473600in}{0.397613in}}%
\pgfpathlineto{\pgfqpoint{0.436800in}{0.397613in}}%
\pgfpathlineto{\pgfqpoint{0.400000in}{0.397613in}}%
\pgfpathlineto{\pgfqpoint{0.400000in}{0.397613in}}%
\pgfpathclose%
\pgfusepath{stroke,fill}%
}%
\begin{pgfscope}%
\pgfsys@transformshift{0.000000in}{0.000000in}%
\pgfsys@useobject{currentmarker}{}%
\end{pgfscope}%
\end{pgfscope}%
\begin{pgfscope}%
\definecolor{textcolor}{rgb}{0.000000,0.000000,0.000000}%
\pgfsetstrokecolor{textcolor}%
\pgfsetfillcolor{textcolor}%
\pgftext[x=4.080000in,y=0.342057in,,top]{\color{textcolor}\sffamily\fontsize{20.000000}{24.000000}\selectfont \(\displaystyle \mathrm{Time}\)}%
\end{pgfscope}%
\begin{pgfscope}%
\definecolor{textcolor}{rgb}{0.000000,0.000000,0.000000}%
\pgfsetstrokecolor{textcolor}%
\pgfsetfillcolor{textcolor}%
\pgftext[x=0.344444in,y=2.140000in,,bottom,rotate=90.000000]{\color{textcolor}\sffamily\fontsize{20.000000}{24.000000}\selectfont \(\displaystyle \mathrm{Voltage}\)}%
\end{pgfscope}%
\begin{pgfscope}%
\pgfsetbuttcap%
\pgfsetmiterjoin%
\definecolor{currentfill}{rgb}{0.000000,0.000000,0.000000}%
\pgfsetfillcolor{currentfill}%
\pgfsetlinewidth{1.003750pt}%
\definecolor{currentstroke}{rgb}{0.000000,0.000000,0.000000}%
\pgfsetstrokecolor{currentstroke}%
\pgfsetdash{}{0pt}%
\pgfsys@defobject{currentmarker}{\pgfqpoint{-0.027778in}{-0.027778in}}{\pgfqpoint{0.027778in}{0.027778in}}{%
\pgfpathmoveto{\pgfqpoint{0.027778in}{-0.000000in}}%
\pgfpathlineto{\pgfqpoint{-0.027778in}{0.027778in}}%
\pgfpathlineto{\pgfqpoint{-0.027778in}{-0.027778in}}%
\pgfpathlineto{\pgfqpoint{0.027778in}{-0.000000in}}%
\pgfpathclose%
\pgfusepath{stroke,fill}%
}%
\begin{pgfscope}%
\pgfsys@transformshift{7.760000in}{0.397613in}%
\pgfsys@useobject{currentmarker}{}%
\end{pgfscope}%
\end{pgfscope}%
\begin{pgfscope}%
\pgfsetbuttcap%
\pgfsetmiterjoin%
\definecolor{currentfill}{rgb}{0.000000,0.000000,0.000000}%
\pgfsetfillcolor{currentfill}%
\pgfsetlinewidth{1.003750pt}%
\definecolor{currentstroke}{rgb}{0.000000,0.000000,0.000000}%
\pgfsetstrokecolor{currentstroke}%
\pgfsetdash{}{0pt}%
\pgfsys@defobject{currentmarker}{\pgfqpoint{-0.027778in}{-0.027778in}}{\pgfqpoint{0.027778in}{0.027778in}}{%
\pgfpathmoveto{\pgfqpoint{0.000000in}{0.027778in}}%
\pgfpathlineto{\pgfqpoint{-0.027778in}{-0.027778in}}%
\pgfpathlineto{\pgfqpoint{0.027778in}{-0.027778in}}%
\pgfpathlineto{\pgfqpoint{0.000000in}{0.027778in}}%
\pgfpathclose%
\pgfusepath{stroke,fill}%
}%
\begin{pgfscope}%
\pgfsys@transformshift{0.400000in}{3.880000in}%
\pgfsys@useobject{currentmarker}{}%
\end{pgfscope}%
\end{pgfscope}%
\begin{pgfscope}%
\pgfpathrectangle{\pgfqpoint{0.400000in}{0.400000in}}{\pgfqpoint{7.360000in}{3.480000in}}%
\pgfusepath{clip}%
\pgfsetrectcap%
\pgfsetroundjoin%
\pgfsetlinewidth{2.007500pt}%
\definecolor{currentstroke}{rgb}{1.000000,0.549020,0.000000}%
\pgfsetstrokecolor{currentstroke}%
\pgfsetdash{}{0pt}%
\pgfpathmoveto{\pgfqpoint{0.400000in}{0.397613in}}%
\pgfpathlineto{\pgfqpoint{0.731200in}{0.398472in}}%
\pgfpathlineto{\pgfqpoint{0.768000in}{0.400979in}}%
\pgfpathlineto{\pgfqpoint{0.804800in}{0.408026in}}%
\pgfpathlineto{\pgfqpoint{0.841600in}{0.424346in}}%
\pgfpathlineto{\pgfqpoint{0.878400in}{0.456693in}}%
\pgfpathlineto{\pgfqpoint{0.915200in}{0.513153in}}%
\pgfpathlineto{\pgfqpoint{0.952000in}{0.601833in}}%
\pgfpathlineto{\pgfqpoint{0.988800in}{0.729278in}}%
\pgfpathlineto{\pgfqpoint{1.025600in}{0.899055in}}%
\pgfpathlineto{\pgfqpoint{1.062400in}{1.110852in}}%
\pgfpathlineto{\pgfqpoint{1.099200in}{1.360254in}}%
\pgfpathlineto{\pgfqpoint{1.136000in}{1.639189in}}%
\pgfpathlineto{\pgfqpoint{1.246400in}{2.538815in}}%
\pgfpathlineto{\pgfqpoint{1.283200in}{2.818697in}}%
\pgfpathlineto{\pgfqpoint{1.320000in}{3.070442in}}%
\pgfpathlineto{\pgfqpoint{1.356800in}{3.286117in}}%
\pgfpathlineto{\pgfqpoint{1.393600in}{3.460244in}}%
\pgfpathlineto{\pgfqpoint{1.430400in}{3.589806in}}%
\pgfpathlineto{\pgfqpoint{1.467200in}{3.674050in}}%
\pgfpathlineto{\pgfqpoint{1.504000in}{3.714172in}}%
\pgfpathlineto{\pgfqpoint{1.540800in}{3.712922in}}%
\pgfpathlineto{\pgfqpoint{1.577600in}{3.674201in}}%
\pgfpathlineto{\pgfqpoint{1.614400in}{3.602663in}}%
\pgfpathlineto{\pgfqpoint{1.651200in}{3.503370in}}%
\pgfpathlineto{\pgfqpoint{1.688000in}{3.381490in}}%
\pgfpathlineto{\pgfqpoint{1.724800in}{3.242069in}}%
\pgfpathlineto{\pgfqpoint{1.761600in}{3.089851in}}%
\pgfpathlineto{\pgfqpoint{1.835200in}{2.763823in}}%
\pgfpathlineto{\pgfqpoint{1.908800in}{2.431893in}}%
\pgfpathlineto{\pgfqpoint{1.945600in}{2.270329in}}%
\pgfpathlineto{\pgfqpoint{1.982400in}{2.114237in}}%
\pgfpathlineto{\pgfqpoint{2.019200in}{1.964966in}}%
\pgfpathlineto{\pgfqpoint{2.056000in}{1.823487in}}%
\pgfpathlineto{\pgfqpoint{2.092800in}{1.690443in}}%
\pgfpathlineto{\pgfqpoint{2.129600in}{1.566198in}}%
\pgfpathlineto{\pgfqpoint{2.166400in}{1.450887in}}%
\pgfpathlineto{\pgfqpoint{2.203200in}{1.344459in}}%
\pgfpathlineto{\pgfqpoint{2.240000in}{1.246719in}}%
\pgfpathlineto{\pgfqpoint{2.276800in}{1.157363in}}%
\pgfpathlineto{\pgfqpoint{2.313600in}{1.076003in}}%
\pgfpathlineto{\pgfqpoint{2.350400in}{1.002197in}}%
\pgfpathlineto{\pgfqpoint{2.387200in}{0.935471in}}%
\pgfpathlineto{\pgfqpoint{2.424000in}{0.875330in}}%
\pgfpathlineto{\pgfqpoint{2.460800in}{0.821276in}}%
\pgfpathlineto{\pgfqpoint{2.497600in}{0.772817in}}%
\pgfpathlineto{\pgfqpoint{2.534400in}{0.729477in}}%
\pgfpathlineto{\pgfqpoint{2.571200in}{0.690798in}}%
\pgfpathlineto{\pgfqpoint{2.608000in}{0.656347in}}%
\pgfpathlineto{\pgfqpoint{2.644800in}{0.625717in}}%
\pgfpathlineto{\pgfqpoint{2.681600in}{0.598529in}}%
\pgfpathlineto{\pgfqpoint{2.718400in}{0.574434in}}%
\pgfpathlineto{\pgfqpoint{2.755200in}{0.553109in}}%
\pgfpathlineto{\pgfqpoint{2.792000in}{0.534261in}}%
\pgfpathlineto{\pgfqpoint{2.828800in}{0.517620in}}%
\pgfpathlineto{\pgfqpoint{2.865600in}{0.502945in}}%
\pgfpathlineto{\pgfqpoint{2.902400in}{0.490017in}}%
\pgfpathlineto{\pgfqpoint{2.939200in}{0.478636in}}%
\pgfpathlineto{\pgfqpoint{2.976000in}{0.468627in}}%
\pgfpathlineto{\pgfqpoint{3.049600in}{0.452106in}}%
\pgfpathlineto{\pgfqpoint{3.123200in}{0.439377in}}%
\pgfpathlineto{\pgfqpoint{3.196800in}{0.429593in}}%
\pgfpathlineto{\pgfqpoint{3.270400in}{0.422085in}}%
\pgfpathlineto{\pgfqpoint{3.380800in}{0.413981in}}%
\pgfpathlineto{\pgfqpoint{3.491200in}{0.408558in}}%
\pgfpathlineto{\pgfqpoint{3.638400in}{0.404014in}}%
\pgfpathlineto{\pgfqpoint{3.822400in}{0.400894in}}%
\pgfpathlineto{\pgfqpoint{4.080000in}{0.398908in}}%
\pgfpathlineto{\pgfqpoint{4.558400in}{0.397850in}}%
\pgfpathlineto{\pgfqpoint{6.251200in}{0.397613in}}%
\pgfpathlineto{\pgfqpoint{7.760000in}{0.397613in}}%
\pgfpathlineto{\pgfqpoint{7.760000in}{0.397613in}}%
\pgfusepath{stroke}%
\end{pgfscope}%
\begin{pgfscope}%
\pgfpathrectangle{\pgfqpoint{0.400000in}{0.400000in}}{\pgfqpoint{7.360000in}{3.480000in}}%
\pgfusepath{clip}%
\pgfsetrectcap%
\pgfsetroundjoin%
\pgfsetlinewidth{2.007500pt}%
\definecolor{currentstroke}{rgb}{0.000000,0.000000,0.545098}%
\pgfsetstrokecolor{currentstroke}%
\pgfsetdash{}{0pt}%
\pgfpathmoveto{\pgfqpoint{0.400000in}{0.397613in}}%
\pgfpathlineto{\pgfqpoint{4.411200in}{0.398472in}}%
\pgfpathlineto{\pgfqpoint{4.448000in}{0.400979in}}%
\pgfpathlineto{\pgfqpoint{4.484800in}{0.408026in}}%
\pgfpathlineto{\pgfqpoint{4.521600in}{0.424346in}}%
\pgfpathlineto{\pgfqpoint{4.558400in}{0.456693in}}%
\pgfpathlineto{\pgfqpoint{4.595200in}{0.513153in}}%
\pgfpathlineto{\pgfqpoint{4.632000in}{0.601833in}}%
\pgfpathlineto{\pgfqpoint{4.668800in}{0.729278in}}%
\pgfpathlineto{\pgfqpoint{4.705600in}{0.899055in}}%
\pgfpathlineto{\pgfqpoint{4.742400in}{1.110852in}}%
\pgfpathlineto{\pgfqpoint{4.779200in}{1.360254in}}%
\pgfpathlineto{\pgfqpoint{4.816000in}{1.639189in}}%
\pgfpathlineto{\pgfqpoint{4.926400in}{2.538815in}}%
\pgfpathlineto{\pgfqpoint{4.963200in}{2.818697in}}%
\pgfpathlineto{\pgfqpoint{5.000000in}{3.070442in}}%
\pgfpathlineto{\pgfqpoint{5.036800in}{3.286117in}}%
\pgfpathlineto{\pgfqpoint{5.073600in}{3.460244in}}%
\pgfpathlineto{\pgfqpoint{5.110400in}{3.589806in}}%
\pgfpathlineto{\pgfqpoint{5.147200in}{3.674050in}}%
\pgfpathlineto{\pgfqpoint{5.184000in}{3.714172in}}%
\pgfpathlineto{\pgfqpoint{5.220800in}{3.712922in}}%
\pgfpathlineto{\pgfqpoint{5.257600in}{3.674201in}}%
\pgfpathlineto{\pgfqpoint{5.294400in}{3.602663in}}%
\pgfpathlineto{\pgfqpoint{5.331200in}{3.503370in}}%
\pgfpathlineto{\pgfqpoint{5.368000in}{3.381490in}}%
\pgfpathlineto{\pgfqpoint{5.404800in}{3.242069in}}%
\pgfpathlineto{\pgfqpoint{5.441600in}{3.089851in}}%
\pgfpathlineto{\pgfqpoint{5.515200in}{2.763823in}}%
\pgfpathlineto{\pgfqpoint{5.588800in}{2.431893in}}%
\pgfpathlineto{\pgfqpoint{5.625600in}{2.270329in}}%
\pgfpathlineto{\pgfqpoint{5.662400in}{2.114237in}}%
\pgfpathlineto{\pgfqpoint{5.699200in}{1.964966in}}%
\pgfpathlineto{\pgfqpoint{5.736000in}{1.823487in}}%
\pgfpathlineto{\pgfqpoint{5.772800in}{1.690443in}}%
\pgfpathlineto{\pgfqpoint{5.809600in}{1.566198in}}%
\pgfpathlineto{\pgfqpoint{5.846400in}{1.450887in}}%
\pgfpathlineto{\pgfqpoint{5.883200in}{1.344459in}}%
\pgfpathlineto{\pgfqpoint{5.920000in}{1.246719in}}%
\pgfpathlineto{\pgfqpoint{5.956800in}{1.157363in}}%
\pgfpathlineto{\pgfqpoint{5.993600in}{1.076003in}}%
\pgfpathlineto{\pgfqpoint{6.030400in}{1.002197in}}%
\pgfpathlineto{\pgfqpoint{6.067200in}{0.935471in}}%
\pgfpathlineto{\pgfqpoint{6.104000in}{0.875330in}}%
\pgfpathlineto{\pgfqpoint{6.140800in}{0.821276in}}%
\pgfpathlineto{\pgfqpoint{6.177600in}{0.772817in}}%
\pgfpathlineto{\pgfqpoint{6.214400in}{0.729477in}}%
\pgfpathlineto{\pgfqpoint{6.251200in}{0.690798in}}%
\pgfpathlineto{\pgfqpoint{6.288000in}{0.656347in}}%
\pgfpathlineto{\pgfqpoint{6.324800in}{0.625717in}}%
\pgfpathlineto{\pgfqpoint{6.361600in}{0.598529in}}%
\pgfpathlineto{\pgfqpoint{6.398400in}{0.574434in}}%
\pgfpathlineto{\pgfqpoint{6.435200in}{0.553109in}}%
\pgfpathlineto{\pgfqpoint{6.472000in}{0.534261in}}%
\pgfpathlineto{\pgfqpoint{6.508800in}{0.517620in}}%
\pgfpathlineto{\pgfqpoint{6.545600in}{0.502945in}}%
\pgfpathlineto{\pgfqpoint{6.582400in}{0.490017in}}%
\pgfpathlineto{\pgfqpoint{6.619200in}{0.478636in}}%
\pgfpathlineto{\pgfqpoint{6.656000in}{0.468627in}}%
\pgfpathlineto{\pgfqpoint{6.729600in}{0.452106in}}%
\pgfpathlineto{\pgfqpoint{6.803200in}{0.439377in}}%
\pgfpathlineto{\pgfqpoint{6.876800in}{0.429593in}}%
\pgfpathlineto{\pgfqpoint{6.950400in}{0.422085in}}%
\pgfpathlineto{\pgfqpoint{7.060800in}{0.413981in}}%
\pgfpathlineto{\pgfqpoint{7.171200in}{0.408558in}}%
\pgfpathlineto{\pgfqpoint{7.318400in}{0.404014in}}%
\pgfpathlineto{\pgfqpoint{7.502400in}{0.400894in}}%
\pgfpathlineto{\pgfqpoint{7.760000in}{0.398908in}}%
\pgfpathlineto{\pgfqpoint{7.760000in}{0.398908in}}%
\pgfusepath{stroke}%
\end{pgfscope}%
\begin{pgfscope}%
\pgfsetrectcap%
\pgfsetmiterjoin%
\pgfsetlinewidth{0.803000pt}%
\definecolor{currentstroke}{rgb}{0.000000,0.000000,0.000000}%
\pgfsetstrokecolor{currentstroke}%
\pgfsetdash{}{0pt}%
\pgfpathmoveto{\pgfqpoint{0.400000in}{0.400000in}}%
\pgfpathlineto{\pgfqpoint{0.400000in}{3.880000in}}%
\pgfusepath{stroke}%
\end{pgfscope}%
\begin{pgfscope}%
\pgfsetrectcap%
\pgfsetmiterjoin%
\pgfsetlinewidth{0.803000pt}%
\definecolor{currentstroke}{rgb}{0.000000,0.000000,0.000000}%
\pgfsetstrokecolor{currentstroke}%
\pgfsetdash{}{0pt}%
\pgfpathmoveto{\pgfqpoint{0.400000in}{0.397613in}}%
\pgfpathlineto{\pgfqpoint{7.760000in}{0.397613in}}%
\pgfusepath{stroke}%
\end{pgfscope}%
\begin{pgfscope}%
\pgfsetroundcap%
\pgfsetroundjoin%
\pgfsetlinewidth{1.003750pt}%
\definecolor{currentstroke}{rgb}{0.000000,0.000000,0.000000}%
\pgfsetstrokecolor{currentstroke}%
\pgfsetdash{}{0pt}%
\pgfpathmoveto{\pgfqpoint{4.649991in}{2.785012in}}%
\pgfpathquadraticcurveto{\pgfqpoint{3.466657in}{2.785012in}}{\pgfqpoint{2.283323in}{2.785012in}}%
\pgfusepath{stroke}%
\end{pgfscope}%
\begin{pgfscope}%
\pgfsetroundcap%
\pgfsetroundjoin%
\pgfsetlinewidth{1.003750pt}%
\definecolor{currentstroke}{rgb}{0.000000,0.000000,0.000000}%
\pgfsetstrokecolor{currentstroke}%
\pgfsetdash{}{0pt}%
\pgfpathmoveto{\pgfqpoint{4.538880in}{2.840567in}}%
\pgfpathlineto{\pgfqpoint{4.649991in}{2.785012in}}%
\pgfpathlineto{\pgfqpoint{4.538880in}{2.729456in}}%
\pgfusepath{stroke}%
\end{pgfscope}%
\begin{pgfscope}%
\pgfsetroundcap%
\pgfsetroundjoin%
\pgfsetlinewidth{1.003750pt}%
\definecolor{currentstroke}{rgb}{0.000000,0.000000,0.000000}%
\pgfsetstrokecolor{currentstroke}%
\pgfsetdash{}{0pt}%
\pgfpathmoveto{\pgfqpoint{2.394434in}{2.729456in}}%
\pgfpathlineto{\pgfqpoint{2.283323in}{2.785012in}}%
\pgfpathlineto{\pgfqpoint{2.394434in}{2.840567in}}%
\pgfusepath{stroke}%
\end{pgfscope}%
\begin{pgfscope}%
\definecolor{textcolor}{rgb}{0.000000,0.000000,0.000000}%
\pgfsetstrokecolor{textcolor}%
\pgfsetfillcolor{textcolor}%
\pgftext[x=3.098667in,y=3.023751in,left,base]{\color{textcolor}\sffamily\fontsize{20.000000}{24.000000}\selectfont \(\displaystyle \sim D_\mathrm{w}\)}%
\end{pgfscope}%
\begin{pgfscope}%
\definecolor{textcolor}{rgb}{0.000000,0.000000,0.000000}%
\pgfsetstrokecolor{textcolor}%
\pgfsetfillcolor{textcolor}%
\pgftext[x=1.258667in,y=1.113832in,left,base]{\color{textcolor}\sffamily\fontsize{20.000000}{24.000000}\selectfont \(\displaystyle \sim \mathrm{RSS}\)}%
\end{pgfscope}%
\begin{pgfscope}%
\definecolor{textcolor}{rgb}{0.000000,0.000000,0.000000}%
\pgfsetstrokecolor{textcolor}%
\pgfsetfillcolor{textcolor}%
\pgftext[x=4.938667in,y=1.113832in,left,base]{\color{textcolor}\sffamily\fontsize{20.000000}{24.000000}\selectfont \(\displaystyle \sim \mathrm{RSS}\)}%
\end{pgfscope}%
\end{pgfpicture}%
\makeatother%
\endgroup%
}
  \caption{\label{fig:l2} The $\mathrm{RSS}$ of red and blue curves is a function of the two shaded regions. It is a constant when the curves shift horizontally when they do not overlap.  In contrast, the Wasserstein distance $D_\mathrm{w}$ of the two curves is associated with their separation.  It complements $\mathrm{RSS}$ and offers a time-sensitive metric suitable for the sparse PE space.}
\end{figure}

\subsubsection{Wasserstein distance}
\label{sec:W-dist}

Wasserstein distance $D_\mathrm{w}$ is a metric between two distributions, either of which can be discrete or continuous. It can capture the difference between a waveform analysis result $\hat{\phi}$ and the sampled light curve $\tilde{\phi}(t)$ in eq.~\eqref{eq:lc-sample}.
\begin{equation}
  D_\mathrm{w}\left[\hat{\phi}_*, \tilde{\phi}_*\right] = \inf_{\gamma \in \Gamma} \left[\int \left\vert t_1 - t_2 \right\vert \gamma(t_1, t_2)\mathrm{d}t_1\mathrm{d}t_2\right],
\end{equation}
where $*$ denotes the normalized light curves and $\Gamma$ is the collection of joint distributions with marginals $\hat{\phi}_*(t)$ and $\tilde{\phi}_*(t)$,
\begin{equation*}
  \label{eq:joint}
  \Gamma = \left\{\gamma(t_1, t_2) ~\middle\vert~ \int\gamma(t_1,t_2)\mathrm{d}t_1 = \tilde{\phi}_*(t_2) , \int\gamma(t_1,t_2)\mathrm{d}t_2 = \hat{\phi}_*(t_1) \right\}.
\end{equation*}
It is also known as the \textit{earth mover's distance}~\cite{levina_earth_2001}, encoding the minimum cost to transport mass from one distribution to another in figure~\ref{fig:l2}.

Alternatively, we can calculate $D_\mathrm{w}$ from cumulative distribution functions (CDF). Let $\hat\Phi(t)$ and $\tilde\Phi(t)$ denote the CDF of $\hat{\phi}_*(t)$ and $\tilde{\phi}_*(t)$, respectively. Then $D_\mathrm{w}$ is equivalent to the $\ell_1$-metric between the two CDFs,
\begin{equation}
    D_\mathrm{w}\left[\hat{\phi}_*, \tilde{\phi}_*\right] = \int\left|\hat{\Phi}(t) - \tilde{\Phi}(t)\right| \mathrm{d}t.
    \label{eq:numerical}
\end{equation}

In the following, we assess the performance of waveform analysis algorithms ranging from heuristics, deconvolution, neural network to regression by the criteria discussed in this section.

\subsection{Heuristic methods}
By directly extracting the patterns in the waveforms, \textit{heuristics} refer to the methods making minimal assumptions of the instrumental and statistical features.  Straightforward to implement and widely deployed in neutrino and dark matter experiments~\cite{students22}, they are poorly documented in the literature.  In this section, we try to formulate the heuristics actually have been used in the experiments so as to make an objective comparison with more advanced techniques.

\subsubsection{Waveform shifting}
\label{sec:shifting}
Some experiments use waveforms as direct input of analysis. Proton decay search at KamLAND~\cite{kamland_collaboration_search_2015} sums up all the PMT waveforms after shifting by time-of-flight for each event candidate.  The total waveform shape is used for a $\chi^2$-based particle identification (PID). The Double Chooz experiment also superposes waveforms to extract PID information by Fourier transformation~\cite{chooz_2018}. Samani~et~al.\cite{samani_pulse_2020} extracts pulse width from a raw waveform and use it as a PID discriminator.  Such techniques are extensions of pulse shape discrimination~(PSD) to large neutrino and dark matter experiments.  In the view of this study, extended PSD uses shifted waveform to approximate PE hit pattern, thus named \textit{waveform shifting}.

As illustrated in figure~\ref{fig:shifting}, we firstly select all the $t_i$'s where the waveform $w(t_i)$ exceeds a threshold $V_\mathrm{th}$ to suppress noise, shift them by a constant $\Delta t$. For a SER pulse $V_\mathrm{PE}(t)$ whose truth PE time is $t=0$, $\Delta t$ should minimize the Wasserstein distance $D_\mathrm{W}$. Thus,
\begin{equation}
    \Delta t \equiv \arg\underset{\Delta t'}{\min} D_\mathrm{w}\left[ V_\mathrm{PE*}(t), \delta(t-\Delta t') \right] \implies \int_{0}^{\Delta t} V_\mathrm{PE}(t) \mathrm{d}t = \frac{1}{2} \int_{0}^{\infty} V_\mathrm{PE}(t) \mathrm{d}t.
  \label{eq:waveform-shift-dt}
\end{equation}
The PE times are inferred as $\hat{t}_i = t_i - \Delta t$.  Corresponding $w(t_i)$'s are scaled by $\alpha$ to minimize RSS:
\begin{equation}
  \hat{\alpha} = \arg\underset{\alpha}{\min}~\mathrm{RSS}\left[ \alpha \sum_iw(t_i) \otimes V_\mathrm{PE}(t-\hat{t}_i), w(t) \right] .
  \label{eq:alpha}
\end{equation}
The charges are determined as $\hat{q}_i = \hat{\alpha} w(t_i)$.  Notice the difference from eq.~\eqref{eq:rss}: $\tilde{w}(t)$ unknown in data analysis, we replace it with $w(t)$.

Since the whole over-threshold waveform sample points are treated as PEs, one PE can be split into many. Thus, the obtained $\hat{q}_i$ are smaller than truth PE charges. The waveform shifting model formulated above captures the logic behind waveform superposition methods.  The underlying assumption to treat a waveform as PEs is simply not true and time precision suffers.  It works only if the width of $V_\mathrm{PE}$ is negligible for the purpose, sometimes when classifying events.

\begin{figure}[H]
  \begin{subfigure}{.5\textwidth}
    \centering
    \resizebox{\textwidth}{!}{%% Creator: Matplotlib, PGF backend
%%
%% To include the figure in your LaTeX document, write
%%   \input{<filename>.pgf}
%%
%% Make sure the required packages are loaded in your preamble
%%   \usepackage{pgf}
%%
%% Also ensure that all the required font packages are loaded; for instance,
%% the lmodern package is sometimes necessary when using math font.
%%   \usepackage{lmodern}
%%
%% Figures using additional raster images can only be included by \input if
%% they are in the same directory as the main LaTeX file. For loading figures
%% from other directories you can use the `import` package
%%   \usepackage{import}
%%
%% and then include the figures with
%%   \import{<path to file>}{<filename>.pgf}
%%
%% Matplotlib used the following preamble
%%   \usepackage[detect-all,locale=DE]{siunitx}
%%
\begingroup%
\makeatletter%
\begin{pgfpicture}%
\pgfpathrectangle{\pgfpointorigin}{\pgfqpoint{8.000000in}{6.000000in}}%
\pgfusepath{use as bounding box, clip}%
\begin{pgfscope}%
\pgfsetbuttcap%
\pgfsetmiterjoin%
\definecolor{currentfill}{rgb}{1.000000,1.000000,1.000000}%
\pgfsetfillcolor{currentfill}%
\pgfsetlinewidth{0.000000pt}%
\definecolor{currentstroke}{rgb}{1.000000,1.000000,1.000000}%
\pgfsetstrokecolor{currentstroke}%
\pgfsetdash{}{0pt}%
\pgfpathmoveto{\pgfqpoint{0.000000in}{0.000000in}}%
\pgfpathlineto{\pgfqpoint{8.000000in}{0.000000in}}%
\pgfpathlineto{\pgfqpoint{8.000000in}{6.000000in}}%
\pgfpathlineto{\pgfqpoint{0.000000in}{6.000000in}}%
\pgfpathlineto{\pgfqpoint{0.000000in}{0.000000in}}%
\pgfpathclose%
\pgfusepath{fill}%
\end{pgfscope}%
\begin{pgfscope}%
\pgfsetbuttcap%
\pgfsetmiterjoin%
\definecolor{currentfill}{rgb}{1.000000,1.000000,1.000000}%
\pgfsetfillcolor{currentfill}%
\pgfsetlinewidth{0.000000pt}%
\definecolor{currentstroke}{rgb}{0.000000,0.000000,0.000000}%
\pgfsetstrokecolor{currentstroke}%
\pgfsetstrokeopacity{0.000000}%
\pgfsetdash{}{0pt}%
\pgfpathmoveto{\pgfqpoint{1.000000in}{0.720000in}}%
\pgfpathlineto{\pgfqpoint{7.200000in}{0.720000in}}%
\pgfpathlineto{\pgfqpoint{7.200000in}{5.340000in}}%
\pgfpathlineto{\pgfqpoint{1.000000in}{5.340000in}}%
\pgfpathlineto{\pgfqpoint{1.000000in}{0.720000in}}%
\pgfpathclose%
\pgfusepath{fill}%
\end{pgfscope}%
\begin{pgfscope}%
\pgfsetbuttcap%
\pgfsetroundjoin%
\definecolor{currentfill}{rgb}{0.000000,0.000000,0.000000}%
\pgfsetfillcolor{currentfill}%
\pgfsetlinewidth{0.803000pt}%
\definecolor{currentstroke}{rgb}{0.000000,0.000000,0.000000}%
\pgfsetstrokecolor{currentstroke}%
\pgfsetdash{}{0pt}%
\pgfsys@defobject{currentmarker}{\pgfqpoint{0.000000in}{-0.048611in}}{\pgfqpoint{0.000000in}{0.000000in}}{%
\pgfpathmoveto{\pgfqpoint{0.000000in}{0.000000in}}%
\pgfpathlineto{\pgfqpoint{0.000000in}{-0.048611in}}%
\pgfusepath{stroke,fill}%
}%
\begin{pgfscope}%
\pgfsys@transformshift{1.310000in}{0.720000in}%
\pgfsys@useobject{currentmarker}{}%
\end{pgfscope}%
\end{pgfscope}%
\begin{pgfscope}%
\definecolor{textcolor}{rgb}{0.000000,0.000000,0.000000}%
\pgfsetstrokecolor{textcolor}%
\pgfsetfillcolor{textcolor}%
\pgftext[x=1.310000in,y=0.622778in,,top]{\color{textcolor}\sffamily\fontsize{20.000000}{24.000000}\selectfont \(\displaystyle {450}\)}%
\end{pgfscope}%
\begin{pgfscope}%
\pgfsetbuttcap%
\pgfsetroundjoin%
\definecolor{currentfill}{rgb}{0.000000,0.000000,0.000000}%
\pgfsetfillcolor{currentfill}%
\pgfsetlinewidth{0.803000pt}%
\definecolor{currentstroke}{rgb}{0.000000,0.000000,0.000000}%
\pgfsetstrokecolor{currentstroke}%
\pgfsetdash{}{0pt}%
\pgfsys@defobject{currentmarker}{\pgfqpoint{0.000000in}{-0.048611in}}{\pgfqpoint{0.000000in}{0.000000in}}{%
\pgfpathmoveto{\pgfqpoint{0.000000in}{0.000000in}}%
\pgfpathlineto{\pgfqpoint{0.000000in}{-0.048611in}}%
\pgfusepath{stroke,fill}%
}%
\begin{pgfscope}%
\pgfsys@transformshift{2.860000in}{0.720000in}%
\pgfsys@useobject{currentmarker}{}%
\end{pgfscope}%
\end{pgfscope}%
\begin{pgfscope}%
\definecolor{textcolor}{rgb}{0.000000,0.000000,0.000000}%
\pgfsetstrokecolor{textcolor}%
\pgfsetfillcolor{textcolor}%
\pgftext[x=2.860000in,y=0.622778in,,top]{\color{textcolor}\sffamily\fontsize{20.000000}{24.000000}\selectfont \(\displaystyle {500}\)}%
\end{pgfscope}%
\begin{pgfscope}%
\pgfsetbuttcap%
\pgfsetroundjoin%
\definecolor{currentfill}{rgb}{0.000000,0.000000,0.000000}%
\pgfsetfillcolor{currentfill}%
\pgfsetlinewidth{0.803000pt}%
\definecolor{currentstroke}{rgb}{0.000000,0.000000,0.000000}%
\pgfsetstrokecolor{currentstroke}%
\pgfsetdash{}{0pt}%
\pgfsys@defobject{currentmarker}{\pgfqpoint{0.000000in}{-0.048611in}}{\pgfqpoint{0.000000in}{0.000000in}}{%
\pgfpathmoveto{\pgfqpoint{0.000000in}{0.000000in}}%
\pgfpathlineto{\pgfqpoint{0.000000in}{-0.048611in}}%
\pgfusepath{stroke,fill}%
}%
\begin{pgfscope}%
\pgfsys@transformshift{4.410000in}{0.720000in}%
\pgfsys@useobject{currentmarker}{}%
\end{pgfscope}%
\end{pgfscope}%
\begin{pgfscope}%
\definecolor{textcolor}{rgb}{0.000000,0.000000,0.000000}%
\pgfsetstrokecolor{textcolor}%
\pgfsetfillcolor{textcolor}%
\pgftext[x=4.410000in,y=0.622778in,,top]{\color{textcolor}\sffamily\fontsize{20.000000}{24.000000}\selectfont \(\displaystyle {550}\)}%
\end{pgfscope}%
\begin{pgfscope}%
\pgfsetbuttcap%
\pgfsetroundjoin%
\definecolor{currentfill}{rgb}{0.000000,0.000000,0.000000}%
\pgfsetfillcolor{currentfill}%
\pgfsetlinewidth{0.803000pt}%
\definecolor{currentstroke}{rgb}{0.000000,0.000000,0.000000}%
\pgfsetstrokecolor{currentstroke}%
\pgfsetdash{}{0pt}%
\pgfsys@defobject{currentmarker}{\pgfqpoint{0.000000in}{-0.048611in}}{\pgfqpoint{0.000000in}{0.000000in}}{%
\pgfpathmoveto{\pgfqpoint{0.000000in}{0.000000in}}%
\pgfpathlineto{\pgfqpoint{0.000000in}{-0.048611in}}%
\pgfusepath{stroke,fill}%
}%
\begin{pgfscope}%
\pgfsys@transformshift{5.960000in}{0.720000in}%
\pgfsys@useobject{currentmarker}{}%
\end{pgfscope}%
\end{pgfscope}%
\begin{pgfscope}%
\definecolor{textcolor}{rgb}{0.000000,0.000000,0.000000}%
\pgfsetstrokecolor{textcolor}%
\pgfsetfillcolor{textcolor}%
\pgftext[x=5.960000in,y=0.622778in,,top]{\color{textcolor}\sffamily\fontsize{20.000000}{24.000000}\selectfont \(\displaystyle {600}\)}%
\end{pgfscope}%
\begin{pgfscope}%
\definecolor{textcolor}{rgb}{0.000000,0.000000,0.000000}%
\pgfsetstrokecolor{textcolor}%
\pgfsetfillcolor{textcolor}%
\pgftext[x=4.100000in,y=0.311155in,,top]{\color{textcolor}\sffamily\fontsize{20.000000}{24.000000}\selectfont \(\displaystyle \mathrm{t}/\si{ns}\)}%
\end{pgfscope}%
\begin{pgfscope}%
\pgfsetbuttcap%
\pgfsetroundjoin%
\definecolor{currentfill}{rgb}{0.000000,0.000000,0.000000}%
\pgfsetfillcolor{currentfill}%
\pgfsetlinewidth{0.803000pt}%
\definecolor{currentstroke}{rgb}{0.000000,0.000000,0.000000}%
\pgfsetstrokecolor{currentstroke}%
\pgfsetdash{}{0pt}%
\pgfsys@defobject{currentmarker}{\pgfqpoint{-0.048611in}{0.000000in}}{\pgfqpoint{-0.000000in}{0.000000in}}{%
\pgfpathmoveto{\pgfqpoint{-0.000000in}{0.000000in}}%
\pgfpathlineto{\pgfqpoint{-0.048611in}{0.000000in}}%
\pgfusepath{stroke,fill}%
}%
\begin{pgfscope}%
\pgfsys@transformshift{1.000000in}{1.045390in}%
\pgfsys@useobject{currentmarker}{}%
\end{pgfscope}%
\end{pgfscope}%
\begin{pgfscope}%
\definecolor{textcolor}{rgb}{0.000000,0.000000,0.000000}%
\pgfsetstrokecolor{textcolor}%
\pgfsetfillcolor{textcolor}%
\pgftext[x=0.770670in, y=0.945371in, left, base]{\color{textcolor}\sffamily\fontsize{20.000000}{24.000000}\selectfont \(\displaystyle {0}\)}%
\end{pgfscope}%
\begin{pgfscope}%
\pgfsetbuttcap%
\pgfsetroundjoin%
\definecolor{currentfill}{rgb}{0.000000,0.000000,0.000000}%
\pgfsetfillcolor{currentfill}%
\pgfsetlinewidth{0.803000pt}%
\definecolor{currentstroke}{rgb}{0.000000,0.000000,0.000000}%
\pgfsetstrokecolor{currentstroke}%
\pgfsetdash{}{0pt}%
\pgfsys@defobject{currentmarker}{\pgfqpoint{-0.048611in}{0.000000in}}{\pgfqpoint{-0.000000in}{0.000000in}}{%
\pgfpathmoveto{\pgfqpoint{-0.000000in}{0.000000in}}%
\pgfpathlineto{\pgfqpoint{-0.048611in}{0.000000in}}%
\pgfusepath{stroke,fill}%
}%
\begin{pgfscope}%
\pgfsys@transformshift{1.000000in}{1.794259in}%
\pgfsys@useobject{currentmarker}{}%
\end{pgfscope}%
\end{pgfscope}%
\begin{pgfscope}%
\definecolor{textcolor}{rgb}{0.000000,0.000000,0.000000}%
\pgfsetstrokecolor{textcolor}%
\pgfsetfillcolor{textcolor}%
\pgftext[x=0.638563in, y=1.694240in, left, base]{\color{textcolor}\sffamily\fontsize{20.000000}{24.000000}\selectfont \(\displaystyle {10}\)}%
\end{pgfscope}%
\begin{pgfscope}%
\pgfsetbuttcap%
\pgfsetroundjoin%
\definecolor{currentfill}{rgb}{0.000000,0.000000,0.000000}%
\pgfsetfillcolor{currentfill}%
\pgfsetlinewidth{0.803000pt}%
\definecolor{currentstroke}{rgb}{0.000000,0.000000,0.000000}%
\pgfsetstrokecolor{currentstroke}%
\pgfsetdash{}{0pt}%
\pgfsys@defobject{currentmarker}{\pgfqpoint{-0.048611in}{0.000000in}}{\pgfqpoint{-0.000000in}{0.000000in}}{%
\pgfpathmoveto{\pgfqpoint{-0.000000in}{0.000000in}}%
\pgfpathlineto{\pgfqpoint{-0.048611in}{0.000000in}}%
\pgfusepath{stroke,fill}%
}%
\begin{pgfscope}%
\pgfsys@transformshift{1.000000in}{2.543128in}%
\pgfsys@useobject{currentmarker}{}%
\end{pgfscope}%
\end{pgfscope}%
\begin{pgfscope}%
\definecolor{textcolor}{rgb}{0.000000,0.000000,0.000000}%
\pgfsetstrokecolor{textcolor}%
\pgfsetfillcolor{textcolor}%
\pgftext[x=0.638563in, y=2.443109in, left, base]{\color{textcolor}\sffamily\fontsize{20.000000}{24.000000}\selectfont \(\displaystyle {20}\)}%
\end{pgfscope}%
\begin{pgfscope}%
\pgfsetbuttcap%
\pgfsetroundjoin%
\definecolor{currentfill}{rgb}{0.000000,0.000000,0.000000}%
\pgfsetfillcolor{currentfill}%
\pgfsetlinewidth{0.803000pt}%
\definecolor{currentstroke}{rgb}{0.000000,0.000000,0.000000}%
\pgfsetstrokecolor{currentstroke}%
\pgfsetdash{}{0pt}%
\pgfsys@defobject{currentmarker}{\pgfqpoint{-0.048611in}{0.000000in}}{\pgfqpoint{-0.000000in}{0.000000in}}{%
\pgfpathmoveto{\pgfqpoint{-0.000000in}{0.000000in}}%
\pgfpathlineto{\pgfqpoint{-0.048611in}{0.000000in}}%
\pgfusepath{stroke,fill}%
}%
\begin{pgfscope}%
\pgfsys@transformshift{1.000000in}{3.291997in}%
\pgfsys@useobject{currentmarker}{}%
\end{pgfscope}%
\end{pgfscope}%
\begin{pgfscope}%
\definecolor{textcolor}{rgb}{0.000000,0.000000,0.000000}%
\pgfsetstrokecolor{textcolor}%
\pgfsetfillcolor{textcolor}%
\pgftext[x=0.638563in, y=3.191977in, left, base]{\color{textcolor}\sffamily\fontsize{20.000000}{24.000000}\selectfont \(\displaystyle {30}\)}%
\end{pgfscope}%
\begin{pgfscope}%
\pgfsetbuttcap%
\pgfsetroundjoin%
\definecolor{currentfill}{rgb}{0.000000,0.000000,0.000000}%
\pgfsetfillcolor{currentfill}%
\pgfsetlinewidth{0.803000pt}%
\definecolor{currentstroke}{rgb}{0.000000,0.000000,0.000000}%
\pgfsetstrokecolor{currentstroke}%
\pgfsetdash{}{0pt}%
\pgfsys@defobject{currentmarker}{\pgfqpoint{-0.048611in}{0.000000in}}{\pgfqpoint{-0.000000in}{0.000000in}}{%
\pgfpathmoveto{\pgfqpoint{-0.000000in}{0.000000in}}%
\pgfpathlineto{\pgfqpoint{-0.048611in}{0.000000in}}%
\pgfusepath{stroke,fill}%
}%
\begin{pgfscope}%
\pgfsys@transformshift{1.000000in}{4.040866in}%
\pgfsys@useobject{currentmarker}{}%
\end{pgfscope}%
\end{pgfscope}%
\begin{pgfscope}%
\definecolor{textcolor}{rgb}{0.000000,0.000000,0.000000}%
\pgfsetstrokecolor{textcolor}%
\pgfsetfillcolor{textcolor}%
\pgftext[x=0.638563in, y=3.940846in, left, base]{\color{textcolor}\sffamily\fontsize{20.000000}{24.000000}\selectfont \(\displaystyle {40}\)}%
\end{pgfscope}%
\begin{pgfscope}%
\pgfsetbuttcap%
\pgfsetroundjoin%
\definecolor{currentfill}{rgb}{0.000000,0.000000,0.000000}%
\pgfsetfillcolor{currentfill}%
\pgfsetlinewidth{0.803000pt}%
\definecolor{currentstroke}{rgb}{0.000000,0.000000,0.000000}%
\pgfsetstrokecolor{currentstroke}%
\pgfsetdash{}{0pt}%
\pgfsys@defobject{currentmarker}{\pgfqpoint{-0.048611in}{0.000000in}}{\pgfqpoint{-0.000000in}{0.000000in}}{%
\pgfpathmoveto{\pgfqpoint{-0.000000in}{0.000000in}}%
\pgfpathlineto{\pgfqpoint{-0.048611in}{0.000000in}}%
\pgfusepath{stroke,fill}%
}%
\begin{pgfscope}%
\pgfsys@transformshift{1.000000in}{4.789734in}%
\pgfsys@useobject{currentmarker}{}%
\end{pgfscope}%
\end{pgfscope}%
\begin{pgfscope}%
\definecolor{textcolor}{rgb}{0.000000,0.000000,0.000000}%
\pgfsetstrokecolor{textcolor}%
\pgfsetfillcolor{textcolor}%
\pgftext[x=0.638563in, y=4.689715in, left, base]{\color{textcolor}\sffamily\fontsize{20.000000}{24.000000}\selectfont \(\displaystyle {50}\)}%
\end{pgfscope}%
\begin{pgfscope}%
\definecolor{textcolor}{rgb}{0.000000,0.000000,0.000000}%
\pgfsetstrokecolor{textcolor}%
\pgfsetfillcolor{textcolor}%
\pgftext[x=0.583007in,y=3.030000in,,bottom,rotate=90.000000]{\color{textcolor}\sffamily\fontsize{20.000000}{24.000000}\selectfont \(\displaystyle \mathrm{Voltage}/\si{mV}\)}%
\end{pgfscope}%
\begin{pgfscope}%
\pgfpathrectangle{\pgfqpoint{1.000000in}{0.720000in}}{\pgfqpoint{6.200000in}{4.620000in}}%
\pgfusepath{clip}%
\pgfsetrectcap%
\pgfsetroundjoin%
\pgfsetlinewidth{2.007500pt}%
\definecolor{currentstroke}{rgb}{0.121569,0.466667,0.705882}%
\pgfsetstrokecolor{currentstroke}%
\pgfsetdash{}{0pt}%
\pgfpathmoveto{\pgfqpoint{0.990000in}{1.092170in}}%
\pgfpathlineto{\pgfqpoint{1.000000in}{1.121606in}}%
\pgfpathlineto{\pgfqpoint{1.031000in}{1.015436in}}%
\pgfpathlineto{\pgfqpoint{1.062000in}{1.077915in}}%
\pgfpathlineto{\pgfqpoint{1.093000in}{1.067536in}}%
\pgfpathlineto{\pgfqpoint{1.124000in}{1.055793in}}%
\pgfpathlineto{\pgfqpoint{1.155000in}{1.129978in}}%
\pgfpathlineto{\pgfqpoint{1.186000in}{1.064674in}}%
\pgfpathlineto{\pgfqpoint{1.217000in}{1.012487in}}%
\pgfpathlineto{\pgfqpoint{1.248000in}{1.120117in}}%
\pgfpathlineto{\pgfqpoint{1.279000in}{1.065396in}}%
\pgfpathlineto{\pgfqpoint{1.310000in}{1.001021in}}%
\pgfpathlineto{\pgfqpoint{1.341000in}{1.082596in}}%
\pgfpathlineto{\pgfqpoint{1.372000in}{1.032627in}}%
\pgfpathlineto{\pgfqpoint{1.403000in}{1.210201in}}%
\pgfpathlineto{\pgfqpoint{1.434000in}{0.997577in}}%
\pgfpathlineto{\pgfqpoint{1.465000in}{0.950261in}}%
\pgfpathlineto{\pgfqpoint{1.496000in}{1.014751in}}%
\pgfpathlineto{\pgfqpoint{1.527000in}{0.971743in}}%
\pgfpathlineto{\pgfqpoint{1.558000in}{0.951087in}}%
\pgfpathlineto{\pgfqpoint{1.589000in}{1.038590in}}%
\pgfpathlineto{\pgfqpoint{1.620000in}{1.106645in}}%
\pgfpathlineto{\pgfqpoint{1.651000in}{1.088575in}}%
\pgfpathlineto{\pgfqpoint{1.682000in}{1.175808in}}%
\pgfpathlineto{\pgfqpoint{1.713000in}{1.109736in}}%
\pgfpathlineto{\pgfqpoint{1.744000in}{1.128948in}}%
\pgfpathlineto{\pgfqpoint{1.775000in}{1.140197in}}%
\pgfpathlineto{\pgfqpoint{1.806000in}{1.114718in}}%
\pgfpathlineto{\pgfqpoint{1.837000in}{1.030771in}}%
\pgfpathlineto{\pgfqpoint{1.868000in}{0.954116in}}%
\pgfpathlineto{\pgfqpoint{1.899000in}{1.060560in}}%
\pgfpathlineto{\pgfqpoint{1.930000in}{1.022669in}}%
\pgfpathlineto{\pgfqpoint{1.961000in}{1.111181in}}%
\pgfpathlineto{\pgfqpoint{1.992000in}{1.106732in}}%
\pgfpathlineto{\pgfqpoint{2.023000in}{1.275737in}}%
\pgfpathlineto{\pgfqpoint{2.054000in}{1.556350in}}%
\pgfpathlineto{\pgfqpoint{2.085000in}{1.880958in}}%
\pgfpathlineto{\pgfqpoint{2.116000in}{2.447028in}}%
\pgfpathlineto{\pgfqpoint{2.147000in}{2.590124in}}%
\pgfpathlineto{\pgfqpoint{2.178000in}{2.910341in}}%
\pgfpathlineto{\pgfqpoint{2.209000in}{3.011606in}}%
\pgfpathlineto{\pgfqpoint{2.240000in}{2.824858in}}%
\pgfpathlineto{\pgfqpoint{2.271000in}{2.811544in}}%
\pgfpathlineto{\pgfqpoint{2.302000in}{2.748957in}}%
\pgfpathlineto{\pgfqpoint{2.333000in}{2.358261in}}%
\pgfpathlineto{\pgfqpoint{2.364000in}{2.223198in}}%
\pgfpathlineto{\pgfqpoint{2.395000in}{2.214500in}}%
\pgfpathlineto{\pgfqpoint{2.426000in}{1.986489in}}%
\pgfpathlineto{\pgfqpoint{2.457000in}{1.688838in}}%
\pgfpathlineto{\pgfqpoint{2.488000in}{1.618070in}}%
\pgfpathlineto{\pgfqpoint{2.519000in}{1.553943in}}%
\pgfpathlineto{\pgfqpoint{2.550000in}{1.519361in}}%
\pgfpathlineto{\pgfqpoint{2.581000in}{1.690486in}}%
\pgfpathlineto{\pgfqpoint{2.612000in}{2.141940in}}%
\pgfpathlineto{\pgfqpoint{2.643000in}{2.331229in}}%
\pgfpathlineto{\pgfqpoint{2.674000in}{2.641464in}}%
\pgfpathlineto{\pgfqpoint{2.705000in}{2.711884in}}%
\pgfpathlineto{\pgfqpoint{2.736000in}{2.622407in}}%
\pgfpathlineto{\pgfqpoint{2.767000in}{2.494285in}}%
\pgfpathlineto{\pgfqpoint{2.798000in}{2.382759in}}%
\pgfpathlineto{\pgfqpoint{2.829000in}{2.252616in}}%
\pgfpathlineto{\pgfqpoint{2.860000in}{2.160485in}}%
\pgfpathlineto{\pgfqpoint{2.891000in}{1.940107in}}%
\pgfpathlineto{\pgfqpoint{2.922000in}{1.976649in}}%
\pgfpathlineto{\pgfqpoint{2.953000in}{1.819075in}}%
\pgfpathlineto{\pgfqpoint{2.984000in}{2.228177in}}%
\pgfpathlineto{\pgfqpoint{3.015000in}{2.233304in}}%
\pgfpathlineto{\pgfqpoint{3.046000in}{2.486134in}}%
\pgfpathlineto{\pgfqpoint{3.077000in}{2.385422in}}%
\pgfpathlineto{\pgfqpoint{3.108000in}{2.402818in}}%
\pgfpathlineto{\pgfqpoint{3.139000in}{2.340686in}}%
\pgfpathlineto{\pgfqpoint{3.170000in}{2.134828in}}%
\pgfpathlineto{\pgfqpoint{3.201000in}{2.046176in}}%
\pgfpathlineto{\pgfqpoint{3.232000in}{1.833094in}}%
\pgfpathlineto{\pgfqpoint{3.263000in}{1.720769in}}%
\pgfpathlineto{\pgfqpoint{3.294000in}{1.598220in}}%
\pgfpathlineto{\pgfqpoint{3.325000in}{1.427239in}}%
\pgfpathlineto{\pgfqpoint{3.356000in}{1.522510in}}%
\pgfpathlineto{\pgfqpoint{3.387000in}{1.533980in}}%
\pgfpathlineto{\pgfqpoint{3.418000in}{1.355109in}}%
\pgfpathlineto{\pgfqpoint{3.449000in}{1.197318in}}%
\pgfpathlineto{\pgfqpoint{3.480000in}{1.160982in}}%
\pgfpathlineto{\pgfqpoint{3.511000in}{1.327406in}}%
\pgfpathlineto{\pgfqpoint{3.542000in}{1.147175in}}%
\pgfpathlineto{\pgfqpoint{3.573000in}{1.129832in}}%
\pgfpathlineto{\pgfqpoint{3.604000in}{1.228973in}}%
\pgfpathlineto{\pgfqpoint{3.635000in}{1.097201in}}%
\pgfpathlineto{\pgfqpoint{3.666000in}{1.063108in}}%
\pgfpathlineto{\pgfqpoint{3.697000in}{1.116233in}}%
\pgfpathlineto{\pgfqpoint{3.728000in}{1.058005in}}%
\pgfpathlineto{\pgfqpoint{3.759000in}{1.013322in}}%
\pgfpathlineto{\pgfqpoint{3.790000in}{1.220508in}}%
\pgfpathlineto{\pgfqpoint{3.821000in}{1.297051in}}%
\pgfpathlineto{\pgfqpoint{3.852000in}{1.669948in}}%
\pgfpathlineto{\pgfqpoint{3.883000in}{1.842410in}}%
\pgfpathlineto{\pgfqpoint{3.914000in}{2.204255in}}%
\pgfpathlineto{\pgfqpoint{3.945000in}{2.172447in}}%
\pgfpathlineto{\pgfqpoint{3.976000in}{2.399722in}}%
\pgfpathlineto{\pgfqpoint{4.007000in}{2.298353in}}%
\pgfpathlineto{\pgfqpoint{4.038000in}{2.021979in}}%
\pgfpathlineto{\pgfqpoint{4.069000in}{2.021015in}}%
\pgfpathlineto{\pgfqpoint{4.100000in}{1.922787in}}%
\pgfpathlineto{\pgfqpoint{4.131000in}{1.755248in}}%
\pgfpathlineto{\pgfqpoint{4.162000in}{1.615101in}}%
\pgfpathlineto{\pgfqpoint{4.193000in}{1.566692in}}%
\pgfpathlineto{\pgfqpoint{4.224000in}{1.601906in}}%
\pgfpathlineto{\pgfqpoint{4.255000in}{1.551051in}}%
\pgfpathlineto{\pgfqpoint{4.286000in}{1.311597in}}%
\pgfpathlineto{\pgfqpoint{4.317000in}{1.287303in}}%
\pgfpathlineto{\pgfqpoint{4.348000in}{1.242991in}}%
\pgfpathlineto{\pgfqpoint{4.379000in}{1.179426in}}%
\pgfpathlineto{\pgfqpoint{4.410000in}{1.090708in}}%
\pgfpathlineto{\pgfqpoint{4.441000in}{1.227153in}}%
\pgfpathlineto{\pgfqpoint{4.472000in}{1.144632in}}%
\pgfpathlineto{\pgfqpoint{4.534000in}{1.046422in}}%
\pgfpathlineto{\pgfqpoint{4.565000in}{0.994289in}}%
\pgfpathlineto{\pgfqpoint{4.596000in}{1.165280in}}%
\pgfpathlineto{\pgfqpoint{4.627000in}{1.176658in}}%
\pgfpathlineto{\pgfqpoint{4.658000in}{1.037088in}}%
\pgfpathlineto{\pgfqpoint{4.689000in}{1.134012in}}%
\pgfpathlineto{\pgfqpoint{4.720000in}{0.927239in}}%
\pgfpathlineto{\pgfqpoint{4.751000in}{1.040304in}}%
\pgfpathlineto{\pgfqpoint{4.782000in}{1.058466in}}%
\pgfpathlineto{\pgfqpoint{4.813000in}{0.961150in}}%
\pgfpathlineto{\pgfqpoint{4.844000in}{1.007046in}}%
\pgfpathlineto{\pgfqpoint{4.875000in}{0.829124in}}%
\pgfpathlineto{\pgfqpoint{4.906000in}{0.962842in}}%
\pgfpathlineto{\pgfqpoint{4.937000in}{0.983034in}}%
\pgfpathlineto{\pgfqpoint{4.968000in}{0.905135in}}%
\pgfpathlineto{\pgfqpoint{4.999000in}{1.180389in}}%
\pgfpathlineto{\pgfqpoint{5.030000in}{1.089369in}}%
\pgfpathlineto{\pgfqpoint{5.061000in}{1.041003in}}%
\pgfpathlineto{\pgfqpoint{5.092000in}{1.103827in}}%
\pgfpathlineto{\pgfqpoint{5.123000in}{0.931944in}}%
\pgfpathlineto{\pgfqpoint{5.154000in}{0.999898in}}%
\pgfpathlineto{\pgfqpoint{5.185000in}{0.960275in}}%
\pgfpathlineto{\pgfqpoint{5.216000in}{0.901928in}}%
\pgfpathlineto{\pgfqpoint{5.247000in}{1.038061in}}%
\pgfpathlineto{\pgfqpoint{5.278000in}{1.055742in}}%
\pgfpathlineto{\pgfqpoint{5.309000in}{1.113835in}}%
\pgfpathlineto{\pgfqpoint{5.371000in}{1.028582in}}%
\pgfpathlineto{\pgfqpoint{5.402000in}{1.086474in}}%
\pgfpathlineto{\pgfqpoint{5.433000in}{1.105038in}}%
\pgfpathlineto{\pgfqpoint{5.464000in}{0.997720in}}%
\pgfpathlineto{\pgfqpoint{5.495000in}{0.974884in}}%
\pgfpathlineto{\pgfqpoint{5.557000in}{1.023677in}}%
\pgfpathlineto{\pgfqpoint{5.588000in}{0.968956in}}%
\pgfpathlineto{\pgfqpoint{5.619000in}{0.893204in}}%
\pgfpathlineto{\pgfqpoint{5.650000in}{1.004677in}}%
\pgfpathlineto{\pgfqpoint{5.681000in}{1.007862in}}%
\pgfpathlineto{\pgfqpoint{5.712000in}{1.026634in}}%
\pgfpathlineto{\pgfqpoint{5.743000in}{1.060375in}}%
\pgfpathlineto{\pgfqpoint{5.774000in}{1.156555in}}%
\pgfpathlineto{\pgfqpoint{5.805000in}{1.037340in}}%
\pgfpathlineto{\pgfqpoint{5.836000in}{0.987742in}}%
\pgfpathlineto{\pgfqpoint{5.867000in}{1.009246in}}%
\pgfpathlineto{\pgfqpoint{5.898000in}{1.133432in}}%
\pgfpathlineto{\pgfqpoint{5.929000in}{1.125456in}}%
\pgfpathlineto{\pgfqpoint{5.960000in}{1.002879in}}%
\pgfpathlineto{\pgfqpoint{5.991000in}{0.977965in}}%
\pgfpathlineto{\pgfqpoint{6.022000in}{1.021643in}}%
\pgfpathlineto{\pgfqpoint{6.053000in}{0.961899in}}%
\pgfpathlineto{\pgfqpoint{6.084000in}{0.980833in}}%
\pgfpathlineto{\pgfqpoint{6.115000in}{1.030184in}}%
\pgfpathlineto{\pgfqpoint{6.146000in}{1.058298in}}%
\pgfpathlineto{\pgfqpoint{6.177000in}{1.016886in}}%
\pgfpathlineto{\pgfqpoint{6.208000in}{1.090078in}}%
\pgfpathlineto{\pgfqpoint{6.239000in}{0.975049in}}%
\pgfpathlineto{\pgfqpoint{6.270000in}{1.110899in}}%
\pgfpathlineto{\pgfqpoint{6.301000in}{0.944705in}}%
\pgfpathlineto{\pgfqpoint{6.332000in}{0.992998in}}%
\pgfpathlineto{\pgfqpoint{6.363000in}{1.008079in}}%
\pgfpathlineto{\pgfqpoint{6.394000in}{1.003228in}}%
\pgfpathlineto{\pgfqpoint{6.425000in}{1.161825in}}%
\pgfpathlineto{\pgfqpoint{6.456000in}{1.057196in}}%
\pgfpathlineto{\pgfqpoint{6.487000in}{1.121648in}}%
\pgfpathlineto{\pgfqpoint{6.518000in}{1.084176in}}%
\pgfpathlineto{\pgfqpoint{6.549000in}{1.090138in}}%
\pgfpathlineto{\pgfqpoint{6.580000in}{1.053182in}}%
\pgfpathlineto{\pgfqpoint{6.611000in}{0.981170in}}%
\pgfpathlineto{\pgfqpoint{6.642000in}{1.092496in}}%
\pgfpathlineto{\pgfqpoint{6.673000in}{1.017208in}}%
\pgfpathlineto{\pgfqpoint{6.704000in}{0.918950in}}%
\pgfpathlineto{\pgfqpoint{6.735000in}{1.043125in}}%
\pgfpathlineto{\pgfqpoint{6.766000in}{1.115649in}}%
\pgfpathlineto{\pgfqpoint{6.797000in}{1.092398in}}%
\pgfpathlineto{\pgfqpoint{6.828000in}{1.084850in}}%
\pgfpathlineto{\pgfqpoint{6.859000in}{0.911509in}}%
\pgfpathlineto{\pgfqpoint{6.890000in}{1.053703in}}%
\pgfpathlineto{\pgfqpoint{6.921000in}{0.865708in}}%
\pgfpathlineto{\pgfqpoint{6.952000in}{1.053683in}}%
\pgfpathlineto{\pgfqpoint{6.983000in}{0.952696in}}%
\pgfpathlineto{\pgfqpoint{7.014000in}{1.037220in}}%
\pgfpathlineto{\pgfqpoint{7.045000in}{1.058746in}}%
\pgfpathlineto{\pgfqpoint{7.076000in}{0.989765in}}%
\pgfpathlineto{\pgfqpoint{7.107000in}{0.980386in}}%
\pgfpathlineto{\pgfqpoint{7.138000in}{1.094728in}}%
\pgfpathlineto{\pgfqpoint{7.169000in}{1.182178in}}%
\pgfpathlineto{\pgfqpoint{7.200000in}{1.093863in}}%
\pgfpathlineto{\pgfqpoint{7.210000in}{1.042047in}}%
\pgfpathlineto{\pgfqpoint{7.210000in}{1.042047in}}%
\pgfusepath{stroke}%
\end{pgfscope}%
\begin{pgfscope}%
\pgfpathrectangle{\pgfqpoint{1.000000in}{0.720000in}}{\pgfqpoint{6.200000in}{4.620000in}}%
\pgfusepath{clip}%
\pgfsetbuttcap%
\pgfsetroundjoin%
\pgfsetlinewidth{2.007500pt}%
\definecolor{currentstroke}{rgb}{0.000000,0.500000,0.000000}%
\pgfsetstrokecolor{currentstroke}%
\pgfsetdash{}{0pt}%
\pgfpathmoveto{\pgfqpoint{0.990000in}{1.419825in}}%
\pgfpathlineto{\pgfqpoint{7.210000in}{1.419825in}}%
\pgfusepath{stroke}%
\end{pgfscope}%
\begin{pgfscope}%
\pgfsetrectcap%
\pgfsetmiterjoin%
\pgfsetlinewidth{0.803000pt}%
\definecolor{currentstroke}{rgb}{0.000000,0.000000,0.000000}%
\pgfsetstrokecolor{currentstroke}%
\pgfsetdash{}{0pt}%
\pgfpathmoveto{\pgfqpoint{1.000000in}{0.720000in}}%
\pgfpathlineto{\pgfqpoint{1.000000in}{5.340000in}}%
\pgfusepath{stroke}%
\end{pgfscope}%
\begin{pgfscope}%
\pgfsetrectcap%
\pgfsetmiterjoin%
\pgfsetlinewidth{0.803000pt}%
\definecolor{currentstroke}{rgb}{0.000000,0.000000,0.000000}%
\pgfsetstrokecolor{currentstroke}%
\pgfsetdash{}{0pt}%
\pgfpathmoveto{\pgfqpoint{7.200000in}{0.720000in}}%
\pgfpathlineto{\pgfqpoint{7.200000in}{5.340000in}}%
\pgfusepath{stroke}%
\end{pgfscope}%
\begin{pgfscope}%
\pgfsetrectcap%
\pgfsetmiterjoin%
\pgfsetlinewidth{0.803000pt}%
\definecolor{currentstroke}{rgb}{0.000000,0.000000,0.000000}%
\pgfsetstrokecolor{currentstroke}%
\pgfsetdash{}{0pt}%
\pgfpathmoveto{\pgfqpoint{1.000000in}{0.720000in}}%
\pgfpathlineto{\pgfqpoint{7.200000in}{0.720000in}}%
\pgfusepath{stroke}%
\end{pgfscope}%
\begin{pgfscope}%
\pgfsetrectcap%
\pgfsetmiterjoin%
\pgfsetlinewidth{0.803000pt}%
\definecolor{currentstroke}{rgb}{0.000000,0.000000,0.000000}%
\pgfsetstrokecolor{currentstroke}%
\pgfsetdash{}{0pt}%
\pgfpathmoveto{\pgfqpoint{1.000000in}{5.340000in}}%
\pgfpathlineto{\pgfqpoint{7.200000in}{5.340000in}}%
\pgfusepath{stroke}%
\end{pgfscope}%
\begin{pgfscope}%
\pgfsetbuttcap%
\pgfsetroundjoin%
\definecolor{currentfill}{rgb}{0.000000,0.000000,0.000000}%
\pgfsetfillcolor{currentfill}%
\pgfsetlinewidth{0.803000pt}%
\definecolor{currentstroke}{rgb}{0.000000,0.000000,0.000000}%
\pgfsetstrokecolor{currentstroke}%
\pgfsetdash{}{0pt}%
\pgfsys@defobject{currentmarker}{\pgfqpoint{0.000000in}{0.000000in}}{\pgfqpoint{0.048611in}{0.000000in}}{%
\pgfpathmoveto{\pgfqpoint{0.000000in}{0.000000in}}%
\pgfpathlineto{\pgfqpoint{0.048611in}{0.000000in}}%
\pgfusepath{stroke,fill}%
}%
\begin{pgfscope}%
\pgfsys@transformshift{7.200000in}{1.045390in}%
\pgfsys@useobject{currentmarker}{}%
\end{pgfscope}%
\end{pgfscope}%
\begin{pgfscope}%
\definecolor{textcolor}{rgb}{0.000000,0.000000,0.000000}%
\pgfsetstrokecolor{textcolor}%
\pgfsetfillcolor{textcolor}%
\pgftext[x=7.297222in, y=0.945371in, left, base]{\color{textcolor}\sffamily\fontsize{20.000000}{24.000000}\selectfont 0.0}%
\end{pgfscope}%
\begin{pgfscope}%
\pgfsetbuttcap%
\pgfsetroundjoin%
\definecolor{currentfill}{rgb}{0.000000,0.000000,0.000000}%
\pgfsetfillcolor{currentfill}%
\pgfsetlinewidth{0.803000pt}%
\definecolor{currentstroke}{rgb}{0.000000,0.000000,0.000000}%
\pgfsetstrokecolor{currentstroke}%
\pgfsetdash{}{0pt}%
\pgfsys@defobject{currentmarker}{\pgfqpoint{0.000000in}{0.000000in}}{\pgfqpoint{0.048611in}{0.000000in}}{%
\pgfpathmoveto{\pgfqpoint{0.000000in}{0.000000in}}%
\pgfpathlineto{\pgfqpoint{0.048611in}{0.000000in}}%
\pgfusepath{stroke,fill}%
}%
\begin{pgfscope}%
\pgfsys@transformshift{7.200000in}{1.676950in}%
\pgfsys@useobject{currentmarker}{}%
\end{pgfscope}%
\end{pgfscope}%
\begin{pgfscope}%
\definecolor{textcolor}{rgb}{0.000000,0.000000,0.000000}%
\pgfsetstrokecolor{textcolor}%
\pgfsetfillcolor{textcolor}%
\pgftext[x=7.297222in, y=1.576931in, left, base]{\color{textcolor}\sffamily\fontsize{20.000000}{24.000000}\selectfont 0.2}%
\end{pgfscope}%
\begin{pgfscope}%
\pgfsetbuttcap%
\pgfsetroundjoin%
\definecolor{currentfill}{rgb}{0.000000,0.000000,0.000000}%
\pgfsetfillcolor{currentfill}%
\pgfsetlinewidth{0.803000pt}%
\definecolor{currentstroke}{rgb}{0.000000,0.000000,0.000000}%
\pgfsetstrokecolor{currentstroke}%
\pgfsetdash{}{0pt}%
\pgfsys@defobject{currentmarker}{\pgfqpoint{0.000000in}{0.000000in}}{\pgfqpoint{0.048611in}{0.000000in}}{%
\pgfpathmoveto{\pgfqpoint{0.000000in}{0.000000in}}%
\pgfpathlineto{\pgfqpoint{0.048611in}{0.000000in}}%
\pgfusepath{stroke,fill}%
}%
\begin{pgfscope}%
\pgfsys@transformshift{7.200000in}{2.308511in}%
\pgfsys@useobject{currentmarker}{}%
\end{pgfscope}%
\end{pgfscope}%
\begin{pgfscope}%
\definecolor{textcolor}{rgb}{0.000000,0.000000,0.000000}%
\pgfsetstrokecolor{textcolor}%
\pgfsetfillcolor{textcolor}%
\pgftext[x=7.297222in, y=2.208492in, left, base]{\color{textcolor}\sffamily\fontsize{20.000000}{24.000000}\selectfont 0.5}%
\end{pgfscope}%
\begin{pgfscope}%
\pgfsetbuttcap%
\pgfsetroundjoin%
\definecolor{currentfill}{rgb}{0.000000,0.000000,0.000000}%
\pgfsetfillcolor{currentfill}%
\pgfsetlinewidth{0.803000pt}%
\definecolor{currentstroke}{rgb}{0.000000,0.000000,0.000000}%
\pgfsetstrokecolor{currentstroke}%
\pgfsetdash{}{0pt}%
\pgfsys@defobject{currentmarker}{\pgfqpoint{0.000000in}{0.000000in}}{\pgfqpoint{0.048611in}{0.000000in}}{%
\pgfpathmoveto{\pgfqpoint{0.000000in}{0.000000in}}%
\pgfpathlineto{\pgfqpoint{0.048611in}{0.000000in}}%
\pgfusepath{stroke,fill}%
}%
\begin{pgfscope}%
\pgfsys@transformshift{7.200000in}{2.940071in}%
\pgfsys@useobject{currentmarker}{}%
\end{pgfscope}%
\end{pgfscope}%
\begin{pgfscope}%
\definecolor{textcolor}{rgb}{0.000000,0.000000,0.000000}%
\pgfsetstrokecolor{textcolor}%
\pgfsetfillcolor{textcolor}%
\pgftext[x=7.297222in, y=2.840052in, left, base]{\color{textcolor}\sffamily\fontsize{20.000000}{24.000000}\selectfont 0.8}%
\end{pgfscope}%
\begin{pgfscope}%
\pgfsetbuttcap%
\pgfsetroundjoin%
\definecolor{currentfill}{rgb}{0.000000,0.000000,0.000000}%
\pgfsetfillcolor{currentfill}%
\pgfsetlinewidth{0.803000pt}%
\definecolor{currentstroke}{rgb}{0.000000,0.000000,0.000000}%
\pgfsetstrokecolor{currentstroke}%
\pgfsetdash{}{0pt}%
\pgfsys@defobject{currentmarker}{\pgfqpoint{0.000000in}{0.000000in}}{\pgfqpoint{0.048611in}{0.000000in}}{%
\pgfpathmoveto{\pgfqpoint{0.000000in}{0.000000in}}%
\pgfpathlineto{\pgfqpoint{0.048611in}{0.000000in}}%
\pgfusepath{stroke,fill}%
}%
\begin{pgfscope}%
\pgfsys@transformshift{7.200000in}{3.571631in}%
\pgfsys@useobject{currentmarker}{}%
\end{pgfscope}%
\end{pgfscope}%
\begin{pgfscope}%
\definecolor{textcolor}{rgb}{0.000000,0.000000,0.000000}%
\pgfsetstrokecolor{textcolor}%
\pgfsetfillcolor{textcolor}%
\pgftext[x=7.297222in, y=3.471612in, left, base]{\color{textcolor}\sffamily\fontsize{20.000000}{24.000000}\selectfont 1.0}%
\end{pgfscope}%
\begin{pgfscope}%
\pgfsetbuttcap%
\pgfsetroundjoin%
\definecolor{currentfill}{rgb}{0.000000,0.000000,0.000000}%
\pgfsetfillcolor{currentfill}%
\pgfsetlinewidth{0.803000pt}%
\definecolor{currentstroke}{rgb}{0.000000,0.000000,0.000000}%
\pgfsetstrokecolor{currentstroke}%
\pgfsetdash{}{0pt}%
\pgfsys@defobject{currentmarker}{\pgfqpoint{0.000000in}{0.000000in}}{\pgfqpoint{0.048611in}{0.000000in}}{%
\pgfpathmoveto{\pgfqpoint{0.000000in}{0.000000in}}%
\pgfpathlineto{\pgfqpoint{0.048611in}{0.000000in}}%
\pgfusepath{stroke,fill}%
}%
\begin{pgfscope}%
\pgfsys@transformshift{7.200000in}{4.203192in}%
\pgfsys@useobject{currentmarker}{}%
\end{pgfscope}%
\end{pgfscope}%
\begin{pgfscope}%
\definecolor{textcolor}{rgb}{0.000000,0.000000,0.000000}%
\pgfsetstrokecolor{textcolor}%
\pgfsetfillcolor{textcolor}%
\pgftext[x=7.297222in, y=4.103172in, left, base]{\color{textcolor}\sffamily\fontsize{20.000000}{24.000000}\selectfont 1.2}%
\end{pgfscope}%
\begin{pgfscope}%
\pgfsetbuttcap%
\pgfsetroundjoin%
\definecolor{currentfill}{rgb}{0.000000,0.000000,0.000000}%
\pgfsetfillcolor{currentfill}%
\pgfsetlinewidth{0.803000pt}%
\definecolor{currentstroke}{rgb}{0.000000,0.000000,0.000000}%
\pgfsetstrokecolor{currentstroke}%
\pgfsetdash{}{0pt}%
\pgfsys@defobject{currentmarker}{\pgfqpoint{0.000000in}{0.000000in}}{\pgfqpoint{0.048611in}{0.000000in}}{%
\pgfpathmoveto{\pgfqpoint{0.000000in}{0.000000in}}%
\pgfpathlineto{\pgfqpoint{0.048611in}{0.000000in}}%
\pgfusepath{stroke,fill}%
}%
\begin{pgfscope}%
\pgfsys@transformshift{7.200000in}{4.834752in}%
\pgfsys@useobject{currentmarker}{}%
\end{pgfscope}%
\end{pgfscope}%
\begin{pgfscope}%
\definecolor{textcolor}{rgb}{0.000000,0.000000,0.000000}%
\pgfsetstrokecolor{textcolor}%
\pgfsetfillcolor{textcolor}%
\pgftext[x=7.297222in, y=4.734733in, left, base]{\color{textcolor}\sffamily\fontsize{20.000000}{24.000000}\selectfont 1.5}%
\end{pgfscope}%
\begin{pgfscope}%
\definecolor{textcolor}{rgb}{0.000000,0.000000,0.000000}%
\pgfsetstrokecolor{textcolor}%
\pgfsetfillcolor{textcolor}%
\pgftext[x=7.698906in,y=3.030000in,,top,rotate=90.000000]{\color{textcolor}\sffamily\fontsize{20.000000}{24.000000}\selectfont \(\displaystyle \mathrm{Charge}\)}%
\end{pgfscope}%
\begin{pgfscope}%
\pgfpathrectangle{\pgfqpoint{1.000000in}{0.720000in}}{\pgfqpoint{6.200000in}{4.620000in}}%
\pgfusepath{clip}%
\pgfsetbuttcap%
\pgfsetroundjoin%
\pgfsetlinewidth{0.501875pt}%
\definecolor{currentstroke}{rgb}{1.000000,0.000000,0.000000}%
\pgfsetstrokecolor{currentstroke}%
\pgfsetdash{}{0pt}%
\pgfpathmoveto{\pgfqpoint{1.744000in}{1.045390in}}%
\pgfpathlineto{\pgfqpoint{1.744000in}{1.164852in}}%
\pgfusepath{stroke}%
\end{pgfscope}%
\begin{pgfscope}%
\pgfpathrectangle{\pgfqpoint{1.000000in}{0.720000in}}{\pgfqpoint{6.200000in}{4.620000in}}%
\pgfusepath{clip}%
\pgfsetbuttcap%
\pgfsetroundjoin%
\pgfsetlinewidth{0.501875pt}%
\definecolor{currentstroke}{rgb}{1.000000,0.000000,0.000000}%
\pgfsetstrokecolor{currentstroke}%
\pgfsetdash{}{0pt}%
\pgfpathmoveto{\pgfqpoint{1.775000in}{1.045390in}}%
\pgfpathlineto{\pgfqpoint{1.775000in}{1.240745in}}%
\pgfusepath{stroke}%
\end{pgfscope}%
\begin{pgfscope}%
\pgfpathrectangle{\pgfqpoint{1.000000in}{0.720000in}}{\pgfqpoint{6.200000in}{4.620000in}}%
\pgfusepath{clip}%
\pgfsetbuttcap%
\pgfsetroundjoin%
\pgfsetlinewidth{0.501875pt}%
\definecolor{currentstroke}{rgb}{1.000000,0.000000,0.000000}%
\pgfsetstrokecolor{currentstroke}%
\pgfsetdash{}{0pt}%
\pgfpathmoveto{\pgfqpoint{1.806000in}{1.045390in}}%
\pgfpathlineto{\pgfqpoint{1.806000in}{1.373091in}}%
\pgfusepath{stroke}%
\end{pgfscope}%
\begin{pgfscope}%
\pgfpathrectangle{\pgfqpoint{1.000000in}{0.720000in}}{\pgfqpoint{6.200000in}{4.620000in}}%
\pgfusepath{clip}%
\pgfsetbuttcap%
\pgfsetroundjoin%
\pgfsetlinewidth{0.501875pt}%
\definecolor{currentstroke}{rgb}{1.000000,0.000000,0.000000}%
\pgfsetstrokecolor{currentstroke}%
\pgfsetdash{}{0pt}%
\pgfpathmoveto{\pgfqpoint{1.837000in}{1.045390in}}%
\pgfpathlineto{\pgfqpoint{1.837000in}{1.406547in}}%
\pgfusepath{stroke}%
\end{pgfscope}%
\begin{pgfscope}%
\pgfpathrectangle{\pgfqpoint{1.000000in}{0.720000in}}{\pgfqpoint{6.200000in}{4.620000in}}%
\pgfusepath{clip}%
\pgfsetbuttcap%
\pgfsetroundjoin%
\pgfsetlinewidth{0.501875pt}%
\definecolor{currentstroke}{rgb}{1.000000,0.000000,0.000000}%
\pgfsetstrokecolor{currentstroke}%
\pgfsetdash{}{0pt}%
\pgfpathmoveto{\pgfqpoint{1.868000in}{1.045390in}}%
\pgfpathlineto{\pgfqpoint{1.868000in}{1.481413in}}%
\pgfusepath{stroke}%
\end{pgfscope}%
\begin{pgfscope}%
\pgfpathrectangle{\pgfqpoint{1.000000in}{0.720000in}}{\pgfqpoint{6.200000in}{4.620000in}}%
\pgfusepath{clip}%
\pgfsetbuttcap%
\pgfsetroundjoin%
\pgfsetlinewidth{0.501875pt}%
\definecolor{currentstroke}{rgb}{1.000000,0.000000,0.000000}%
\pgfsetstrokecolor{currentstroke}%
\pgfsetdash{}{0pt}%
\pgfpathmoveto{\pgfqpoint{1.899000in}{1.045390in}}%
\pgfpathlineto{\pgfqpoint{1.899000in}{1.505089in}}%
\pgfusepath{stroke}%
\end{pgfscope}%
\begin{pgfscope}%
\pgfpathrectangle{\pgfqpoint{1.000000in}{0.720000in}}{\pgfqpoint{6.200000in}{4.620000in}}%
\pgfusepath{clip}%
\pgfsetbuttcap%
\pgfsetroundjoin%
\pgfsetlinewidth{0.501875pt}%
\definecolor{currentstroke}{rgb}{1.000000,0.000000,0.000000}%
\pgfsetstrokecolor{currentstroke}%
\pgfsetdash{}{0pt}%
\pgfpathmoveto{\pgfqpoint{1.930000in}{1.045390in}}%
\pgfpathlineto{\pgfqpoint{1.930000in}{1.461428in}}%
\pgfusepath{stroke}%
\end{pgfscope}%
\begin{pgfscope}%
\pgfpathrectangle{\pgfqpoint{1.000000in}{0.720000in}}{\pgfqpoint{6.200000in}{4.620000in}}%
\pgfusepath{clip}%
\pgfsetbuttcap%
\pgfsetroundjoin%
\pgfsetlinewidth{0.501875pt}%
\definecolor{currentstroke}{rgb}{1.000000,0.000000,0.000000}%
\pgfsetstrokecolor{currentstroke}%
\pgfsetdash{}{0pt}%
\pgfpathmoveto{\pgfqpoint{1.961000in}{1.045390in}}%
\pgfpathlineto{\pgfqpoint{1.961000in}{1.458315in}}%
\pgfusepath{stroke}%
\end{pgfscope}%
\begin{pgfscope}%
\pgfpathrectangle{\pgfqpoint{1.000000in}{0.720000in}}{\pgfqpoint{6.200000in}{4.620000in}}%
\pgfusepath{clip}%
\pgfsetbuttcap%
\pgfsetroundjoin%
\pgfsetlinewidth{0.501875pt}%
\definecolor{currentstroke}{rgb}{1.000000,0.000000,0.000000}%
\pgfsetstrokecolor{currentstroke}%
\pgfsetdash{}{0pt}%
\pgfpathmoveto{\pgfqpoint{1.992000in}{1.045390in}}%
\pgfpathlineto{\pgfqpoint{1.992000in}{1.443682in}}%
\pgfusepath{stroke}%
\end{pgfscope}%
\begin{pgfscope}%
\pgfpathrectangle{\pgfqpoint{1.000000in}{0.720000in}}{\pgfqpoint{6.200000in}{4.620000in}}%
\pgfusepath{clip}%
\pgfsetbuttcap%
\pgfsetroundjoin%
\pgfsetlinewidth{0.501875pt}%
\definecolor{currentstroke}{rgb}{1.000000,0.000000,0.000000}%
\pgfsetstrokecolor{currentstroke}%
\pgfsetdash{}{0pt}%
\pgfpathmoveto{\pgfqpoint{2.023000in}{1.045390in}}%
\pgfpathlineto{\pgfqpoint{2.023000in}{1.352338in}}%
\pgfusepath{stroke}%
\end{pgfscope}%
\begin{pgfscope}%
\pgfpathrectangle{\pgfqpoint{1.000000in}{0.720000in}}{\pgfqpoint{6.200000in}{4.620000in}}%
\pgfusepath{clip}%
\pgfsetbuttcap%
\pgfsetroundjoin%
\pgfsetlinewidth{0.501875pt}%
\definecolor{currentstroke}{rgb}{1.000000,0.000000,0.000000}%
\pgfsetstrokecolor{currentstroke}%
\pgfsetdash{}{0pt}%
\pgfpathmoveto{\pgfqpoint{2.054000in}{1.045390in}}%
\pgfpathlineto{\pgfqpoint{2.054000in}{1.320760in}}%
\pgfusepath{stroke}%
\end{pgfscope}%
\begin{pgfscope}%
\pgfpathrectangle{\pgfqpoint{1.000000in}{0.720000in}}{\pgfqpoint{6.200000in}{4.620000in}}%
\pgfusepath{clip}%
\pgfsetbuttcap%
\pgfsetroundjoin%
\pgfsetlinewidth{0.501875pt}%
\definecolor{currentstroke}{rgb}{1.000000,0.000000,0.000000}%
\pgfsetstrokecolor{currentstroke}%
\pgfsetdash{}{0pt}%
\pgfpathmoveto{\pgfqpoint{2.085000in}{1.045390in}}%
\pgfpathlineto{\pgfqpoint{2.085000in}{1.318727in}}%
\pgfusepath{stroke}%
\end{pgfscope}%
\begin{pgfscope}%
\pgfpathrectangle{\pgfqpoint{1.000000in}{0.720000in}}{\pgfqpoint{6.200000in}{4.620000in}}%
\pgfusepath{clip}%
\pgfsetbuttcap%
\pgfsetroundjoin%
\pgfsetlinewidth{0.501875pt}%
\definecolor{currentstroke}{rgb}{1.000000,0.000000,0.000000}%
\pgfsetstrokecolor{currentstroke}%
\pgfsetdash{}{0pt}%
\pgfpathmoveto{\pgfqpoint{2.116000in}{1.045390in}}%
\pgfpathlineto{\pgfqpoint{2.116000in}{1.265418in}}%
\pgfusepath{stroke}%
\end{pgfscope}%
\begin{pgfscope}%
\pgfpathrectangle{\pgfqpoint{1.000000in}{0.720000in}}{\pgfqpoint{6.200000in}{4.620000in}}%
\pgfusepath{clip}%
\pgfsetbuttcap%
\pgfsetroundjoin%
\pgfsetlinewidth{0.501875pt}%
\definecolor{currentstroke}{rgb}{1.000000,0.000000,0.000000}%
\pgfsetstrokecolor{currentstroke}%
\pgfsetdash{}{0pt}%
\pgfpathmoveto{\pgfqpoint{2.147000in}{1.045390in}}%
\pgfpathlineto{\pgfqpoint{2.147000in}{1.195828in}}%
\pgfusepath{stroke}%
\end{pgfscope}%
\begin{pgfscope}%
\pgfpathrectangle{\pgfqpoint{1.000000in}{0.720000in}}{\pgfqpoint{6.200000in}{4.620000in}}%
\pgfusepath{clip}%
\pgfsetbuttcap%
\pgfsetroundjoin%
\pgfsetlinewidth{0.501875pt}%
\definecolor{currentstroke}{rgb}{1.000000,0.000000,0.000000}%
\pgfsetstrokecolor{currentstroke}%
\pgfsetdash{}{0pt}%
\pgfpathmoveto{\pgfqpoint{2.178000in}{1.045390in}}%
\pgfpathlineto{\pgfqpoint{2.178000in}{1.179282in}}%
\pgfusepath{stroke}%
\end{pgfscope}%
\begin{pgfscope}%
\pgfpathrectangle{\pgfqpoint{1.000000in}{0.720000in}}{\pgfqpoint{6.200000in}{4.620000in}}%
\pgfusepath{clip}%
\pgfsetbuttcap%
\pgfsetroundjoin%
\pgfsetlinewidth{0.501875pt}%
\definecolor{currentstroke}{rgb}{1.000000,0.000000,0.000000}%
\pgfsetstrokecolor{currentstroke}%
\pgfsetdash{}{0pt}%
\pgfpathmoveto{\pgfqpoint{2.209000in}{1.045390in}}%
\pgfpathlineto{\pgfqpoint{2.209000in}{1.164289in}}%
\pgfusepath{stroke}%
\end{pgfscope}%
\begin{pgfscope}%
\pgfpathrectangle{\pgfqpoint{1.000000in}{0.720000in}}{\pgfqpoint{6.200000in}{4.620000in}}%
\pgfusepath{clip}%
\pgfsetbuttcap%
\pgfsetroundjoin%
\pgfsetlinewidth{0.501875pt}%
\definecolor{currentstroke}{rgb}{1.000000,0.000000,0.000000}%
\pgfsetstrokecolor{currentstroke}%
\pgfsetdash{}{0pt}%
\pgfpathmoveto{\pgfqpoint{2.240000in}{1.045390in}}%
\pgfpathlineto{\pgfqpoint{2.240000in}{1.156204in}}%
\pgfusepath{stroke}%
\end{pgfscope}%
\begin{pgfscope}%
\pgfpathrectangle{\pgfqpoint{1.000000in}{0.720000in}}{\pgfqpoint{6.200000in}{4.620000in}}%
\pgfusepath{clip}%
\pgfsetbuttcap%
\pgfsetroundjoin%
\pgfsetlinewidth{0.501875pt}%
\definecolor{currentstroke}{rgb}{1.000000,0.000000,0.000000}%
\pgfsetstrokecolor{currentstroke}%
\pgfsetdash{}{0pt}%
\pgfpathmoveto{\pgfqpoint{2.271000in}{1.045390in}}%
\pgfpathlineto{\pgfqpoint{2.271000in}{1.196213in}}%
\pgfusepath{stroke}%
\end{pgfscope}%
\begin{pgfscope}%
\pgfpathrectangle{\pgfqpoint{1.000000in}{0.720000in}}{\pgfqpoint{6.200000in}{4.620000in}}%
\pgfusepath{clip}%
\pgfsetbuttcap%
\pgfsetroundjoin%
\pgfsetlinewidth{0.501875pt}%
\definecolor{currentstroke}{rgb}{1.000000,0.000000,0.000000}%
\pgfsetstrokecolor{currentstroke}%
\pgfsetdash{}{0pt}%
\pgfpathmoveto{\pgfqpoint{2.302000in}{1.045390in}}%
\pgfpathlineto{\pgfqpoint{2.302000in}{1.301762in}}%
\pgfusepath{stroke}%
\end{pgfscope}%
\begin{pgfscope}%
\pgfpathrectangle{\pgfqpoint{1.000000in}{0.720000in}}{\pgfqpoint{6.200000in}{4.620000in}}%
\pgfusepath{clip}%
\pgfsetbuttcap%
\pgfsetroundjoin%
\pgfsetlinewidth{0.501875pt}%
\definecolor{currentstroke}{rgb}{1.000000,0.000000,0.000000}%
\pgfsetstrokecolor{currentstroke}%
\pgfsetdash{}{0pt}%
\pgfpathmoveto{\pgfqpoint{2.333000in}{1.045390in}}%
\pgfpathlineto{\pgfqpoint{2.333000in}{1.346018in}}%
\pgfusepath{stroke}%
\end{pgfscope}%
\begin{pgfscope}%
\pgfpathrectangle{\pgfqpoint{1.000000in}{0.720000in}}{\pgfqpoint{6.200000in}{4.620000in}}%
\pgfusepath{clip}%
\pgfsetbuttcap%
\pgfsetroundjoin%
\pgfsetlinewidth{0.501875pt}%
\definecolor{currentstroke}{rgb}{1.000000,0.000000,0.000000}%
\pgfsetstrokecolor{currentstroke}%
\pgfsetdash{}{0pt}%
\pgfpathmoveto{\pgfqpoint{2.364000in}{1.045390in}}%
\pgfpathlineto{\pgfqpoint{2.364000in}{1.418550in}}%
\pgfusepath{stroke}%
\end{pgfscope}%
\begin{pgfscope}%
\pgfpathrectangle{\pgfqpoint{1.000000in}{0.720000in}}{\pgfqpoint{6.200000in}{4.620000in}}%
\pgfusepath{clip}%
\pgfsetbuttcap%
\pgfsetroundjoin%
\pgfsetlinewidth{0.501875pt}%
\definecolor{currentstroke}{rgb}{1.000000,0.000000,0.000000}%
\pgfsetstrokecolor{currentstroke}%
\pgfsetdash{}{0pt}%
\pgfpathmoveto{\pgfqpoint{2.395000in}{1.045390in}}%
\pgfpathlineto{\pgfqpoint{2.395000in}{1.435015in}}%
\pgfusepath{stroke}%
\end{pgfscope}%
\begin{pgfscope}%
\pgfpathrectangle{\pgfqpoint{1.000000in}{0.720000in}}{\pgfqpoint{6.200000in}{4.620000in}}%
\pgfusepath{clip}%
\pgfsetbuttcap%
\pgfsetroundjoin%
\pgfsetlinewidth{0.501875pt}%
\definecolor{currentstroke}{rgb}{1.000000,0.000000,0.000000}%
\pgfsetstrokecolor{currentstroke}%
\pgfsetdash{}{0pt}%
\pgfpathmoveto{\pgfqpoint{2.426000in}{1.045390in}}%
\pgfpathlineto{\pgfqpoint{2.426000in}{1.414095in}}%
\pgfusepath{stroke}%
\end{pgfscope}%
\begin{pgfscope}%
\pgfpathrectangle{\pgfqpoint{1.000000in}{0.720000in}}{\pgfqpoint{6.200000in}{4.620000in}}%
\pgfusepath{clip}%
\pgfsetbuttcap%
\pgfsetroundjoin%
\pgfsetlinewidth{0.501875pt}%
\definecolor{currentstroke}{rgb}{1.000000,0.000000,0.000000}%
\pgfsetstrokecolor{currentstroke}%
\pgfsetdash{}{0pt}%
\pgfpathmoveto{\pgfqpoint{2.457000in}{1.045390in}}%
\pgfpathlineto{\pgfqpoint{2.457000in}{1.384140in}}%
\pgfusepath{stroke}%
\end{pgfscope}%
\begin{pgfscope}%
\pgfpathrectangle{\pgfqpoint{1.000000in}{0.720000in}}{\pgfqpoint{6.200000in}{4.620000in}}%
\pgfusepath{clip}%
\pgfsetbuttcap%
\pgfsetroundjoin%
\pgfsetlinewidth{0.501875pt}%
\definecolor{currentstroke}{rgb}{1.000000,0.000000,0.000000}%
\pgfsetstrokecolor{currentstroke}%
\pgfsetdash{}{0pt}%
\pgfpathmoveto{\pgfqpoint{2.488000in}{1.045390in}}%
\pgfpathlineto{\pgfqpoint{2.488000in}{1.358065in}}%
\pgfusepath{stroke}%
\end{pgfscope}%
\begin{pgfscope}%
\pgfpathrectangle{\pgfqpoint{1.000000in}{0.720000in}}{\pgfqpoint{6.200000in}{4.620000in}}%
\pgfusepath{clip}%
\pgfsetbuttcap%
\pgfsetroundjoin%
\pgfsetlinewidth{0.501875pt}%
\definecolor{currentstroke}{rgb}{1.000000,0.000000,0.000000}%
\pgfsetstrokecolor{currentstroke}%
\pgfsetdash{}{0pt}%
\pgfpathmoveto{\pgfqpoint{2.519000in}{1.045390in}}%
\pgfpathlineto{\pgfqpoint{2.519000in}{1.327638in}}%
\pgfusepath{stroke}%
\end{pgfscope}%
\begin{pgfscope}%
\pgfpathrectangle{\pgfqpoint{1.000000in}{0.720000in}}{\pgfqpoint{6.200000in}{4.620000in}}%
\pgfusepath{clip}%
\pgfsetbuttcap%
\pgfsetroundjoin%
\pgfsetlinewidth{0.501875pt}%
\definecolor{currentstroke}{rgb}{1.000000,0.000000,0.000000}%
\pgfsetstrokecolor{currentstroke}%
\pgfsetdash{}{0pt}%
\pgfpathmoveto{\pgfqpoint{2.550000in}{1.045390in}}%
\pgfpathlineto{\pgfqpoint{2.550000in}{1.306098in}}%
\pgfusepath{stroke}%
\end{pgfscope}%
\begin{pgfscope}%
\pgfpathrectangle{\pgfqpoint{1.000000in}{0.720000in}}{\pgfqpoint{6.200000in}{4.620000in}}%
\pgfusepath{clip}%
\pgfsetbuttcap%
\pgfsetroundjoin%
\pgfsetlinewidth{0.501875pt}%
\definecolor{currentstroke}{rgb}{1.000000,0.000000,0.000000}%
\pgfsetstrokecolor{currentstroke}%
\pgfsetdash{}{0pt}%
\pgfpathmoveto{\pgfqpoint{2.581000in}{1.045390in}}%
\pgfpathlineto{\pgfqpoint{2.581000in}{1.254574in}}%
\pgfusepath{stroke}%
\end{pgfscope}%
\begin{pgfscope}%
\pgfpathrectangle{\pgfqpoint{1.000000in}{0.720000in}}{\pgfqpoint{6.200000in}{4.620000in}}%
\pgfusepath{clip}%
\pgfsetbuttcap%
\pgfsetroundjoin%
\pgfsetlinewidth{0.501875pt}%
\definecolor{currentstroke}{rgb}{1.000000,0.000000,0.000000}%
\pgfsetstrokecolor{currentstroke}%
\pgfsetdash{}{0pt}%
\pgfpathmoveto{\pgfqpoint{2.612000in}{1.045390in}}%
\pgfpathlineto{\pgfqpoint{2.612000in}{1.263117in}}%
\pgfusepath{stroke}%
\end{pgfscope}%
\begin{pgfscope}%
\pgfpathrectangle{\pgfqpoint{1.000000in}{0.720000in}}{\pgfqpoint{6.200000in}{4.620000in}}%
\pgfusepath{clip}%
\pgfsetbuttcap%
\pgfsetroundjoin%
\pgfsetlinewidth{0.501875pt}%
\definecolor{currentstroke}{rgb}{1.000000,0.000000,0.000000}%
\pgfsetstrokecolor{currentstroke}%
\pgfsetdash{}{0pt}%
\pgfpathmoveto{\pgfqpoint{2.643000in}{1.045390in}}%
\pgfpathlineto{\pgfqpoint{2.643000in}{1.226277in}}%
\pgfusepath{stroke}%
\end{pgfscope}%
\begin{pgfscope}%
\pgfpathrectangle{\pgfqpoint{1.000000in}{0.720000in}}{\pgfqpoint{6.200000in}{4.620000in}}%
\pgfusepath{clip}%
\pgfsetbuttcap%
\pgfsetroundjoin%
\pgfsetlinewidth{0.501875pt}%
\definecolor{currentstroke}{rgb}{1.000000,0.000000,0.000000}%
\pgfsetstrokecolor{currentstroke}%
\pgfsetdash{}{0pt}%
\pgfpathmoveto{\pgfqpoint{2.674000in}{1.045390in}}%
\pgfpathlineto{\pgfqpoint{2.674000in}{1.321924in}}%
\pgfusepath{stroke}%
\end{pgfscope}%
\begin{pgfscope}%
\pgfpathrectangle{\pgfqpoint{1.000000in}{0.720000in}}{\pgfqpoint{6.200000in}{4.620000in}}%
\pgfusepath{clip}%
\pgfsetbuttcap%
\pgfsetroundjoin%
\pgfsetlinewidth{0.501875pt}%
\definecolor{currentstroke}{rgb}{1.000000,0.000000,0.000000}%
\pgfsetstrokecolor{currentstroke}%
\pgfsetdash{}{0pt}%
\pgfpathmoveto{\pgfqpoint{2.705000in}{1.045390in}}%
\pgfpathlineto{\pgfqpoint{2.705000in}{1.323123in}}%
\pgfusepath{stroke}%
\end{pgfscope}%
\begin{pgfscope}%
\pgfpathrectangle{\pgfqpoint{1.000000in}{0.720000in}}{\pgfqpoint{6.200000in}{4.620000in}}%
\pgfusepath{clip}%
\pgfsetbuttcap%
\pgfsetroundjoin%
\pgfsetlinewidth{0.501875pt}%
\definecolor{currentstroke}{rgb}{1.000000,0.000000,0.000000}%
\pgfsetstrokecolor{currentstroke}%
\pgfsetdash{}{0pt}%
\pgfpathmoveto{\pgfqpoint{2.736000in}{1.045390in}}%
\pgfpathlineto{\pgfqpoint{2.736000in}{1.382234in}}%
\pgfusepath{stroke}%
\end{pgfscope}%
\begin{pgfscope}%
\pgfpathrectangle{\pgfqpoint{1.000000in}{0.720000in}}{\pgfqpoint{6.200000in}{4.620000in}}%
\pgfusepath{clip}%
\pgfsetbuttcap%
\pgfsetroundjoin%
\pgfsetlinewidth{0.501875pt}%
\definecolor{currentstroke}{rgb}{1.000000,0.000000,0.000000}%
\pgfsetstrokecolor{currentstroke}%
\pgfsetdash{}{0pt}%
\pgfpathmoveto{\pgfqpoint{2.767000in}{1.045390in}}%
\pgfpathlineto{\pgfqpoint{2.767000in}{1.358688in}}%
\pgfusepath{stroke}%
\end{pgfscope}%
\begin{pgfscope}%
\pgfpathrectangle{\pgfqpoint{1.000000in}{0.720000in}}{\pgfqpoint{6.200000in}{4.620000in}}%
\pgfusepath{clip}%
\pgfsetbuttcap%
\pgfsetroundjoin%
\pgfsetlinewidth{0.501875pt}%
\definecolor{currentstroke}{rgb}{1.000000,0.000000,0.000000}%
\pgfsetstrokecolor{currentstroke}%
\pgfsetdash{}{0pt}%
\pgfpathmoveto{\pgfqpoint{2.798000in}{1.045390in}}%
\pgfpathlineto{\pgfqpoint{2.798000in}{1.362755in}}%
\pgfusepath{stroke}%
\end{pgfscope}%
\begin{pgfscope}%
\pgfpathrectangle{\pgfqpoint{1.000000in}{0.720000in}}{\pgfqpoint{6.200000in}{4.620000in}}%
\pgfusepath{clip}%
\pgfsetbuttcap%
\pgfsetroundjoin%
\pgfsetlinewidth{0.501875pt}%
\definecolor{currentstroke}{rgb}{1.000000,0.000000,0.000000}%
\pgfsetstrokecolor{currentstroke}%
\pgfsetdash{}{0pt}%
\pgfpathmoveto{\pgfqpoint{2.829000in}{1.045390in}}%
\pgfpathlineto{\pgfqpoint{2.829000in}{1.348229in}}%
\pgfusepath{stroke}%
\end{pgfscope}%
\begin{pgfscope}%
\pgfpathrectangle{\pgfqpoint{1.000000in}{0.720000in}}{\pgfqpoint{6.200000in}{4.620000in}}%
\pgfusepath{clip}%
\pgfsetbuttcap%
\pgfsetroundjoin%
\pgfsetlinewidth{0.501875pt}%
\definecolor{currentstroke}{rgb}{1.000000,0.000000,0.000000}%
\pgfsetstrokecolor{currentstroke}%
\pgfsetdash{}{0pt}%
\pgfpathmoveto{\pgfqpoint{2.860000in}{1.045390in}}%
\pgfpathlineto{\pgfqpoint{2.860000in}{1.300099in}}%
\pgfusepath{stroke}%
\end{pgfscope}%
\begin{pgfscope}%
\pgfpathrectangle{\pgfqpoint{1.000000in}{0.720000in}}{\pgfqpoint{6.200000in}{4.620000in}}%
\pgfusepath{clip}%
\pgfsetbuttcap%
\pgfsetroundjoin%
\pgfsetlinewidth{0.501875pt}%
\definecolor{currentstroke}{rgb}{1.000000,0.000000,0.000000}%
\pgfsetstrokecolor{currentstroke}%
\pgfsetdash{}{0pt}%
\pgfpathmoveto{\pgfqpoint{2.891000in}{1.045390in}}%
\pgfpathlineto{\pgfqpoint{2.891000in}{1.279373in}}%
\pgfusepath{stroke}%
\end{pgfscope}%
\begin{pgfscope}%
\pgfpathrectangle{\pgfqpoint{1.000000in}{0.720000in}}{\pgfqpoint{6.200000in}{4.620000in}}%
\pgfusepath{clip}%
\pgfsetbuttcap%
\pgfsetroundjoin%
\pgfsetlinewidth{0.501875pt}%
\definecolor{currentstroke}{rgb}{1.000000,0.000000,0.000000}%
\pgfsetstrokecolor{currentstroke}%
\pgfsetdash{}{0pt}%
\pgfpathmoveto{\pgfqpoint{2.922000in}{1.045390in}}%
\pgfpathlineto{\pgfqpoint{2.922000in}{1.229554in}}%
\pgfusepath{stroke}%
\end{pgfscope}%
\begin{pgfscope}%
\pgfpathrectangle{\pgfqpoint{1.000000in}{0.720000in}}{\pgfqpoint{6.200000in}{4.620000in}}%
\pgfusepath{clip}%
\pgfsetbuttcap%
\pgfsetroundjoin%
\pgfsetlinewidth{0.501875pt}%
\definecolor{currentstroke}{rgb}{1.000000,0.000000,0.000000}%
\pgfsetstrokecolor{currentstroke}%
\pgfsetdash{}{0pt}%
\pgfpathmoveto{\pgfqpoint{2.953000in}{1.045390in}}%
\pgfpathlineto{\pgfqpoint{2.953000in}{1.203293in}}%
\pgfusepath{stroke}%
\end{pgfscope}%
\begin{pgfscope}%
\pgfpathrectangle{\pgfqpoint{1.000000in}{0.720000in}}{\pgfqpoint{6.200000in}{4.620000in}}%
\pgfusepath{clip}%
\pgfsetbuttcap%
\pgfsetroundjoin%
\pgfsetlinewidth{0.501875pt}%
\definecolor{currentstroke}{rgb}{1.000000,0.000000,0.000000}%
\pgfsetstrokecolor{currentstroke}%
\pgfsetdash{}{0pt}%
\pgfpathmoveto{\pgfqpoint{2.984000in}{1.045390in}}%
\pgfpathlineto{\pgfqpoint{2.984000in}{1.174641in}}%
\pgfusepath{stroke}%
\end{pgfscope}%
\begin{pgfscope}%
\pgfpathrectangle{\pgfqpoint{1.000000in}{0.720000in}}{\pgfqpoint{6.200000in}{4.620000in}}%
\pgfusepath{clip}%
\pgfsetbuttcap%
\pgfsetroundjoin%
\pgfsetlinewidth{0.501875pt}%
\definecolor{currentstroke}{rgb}{1.000000,0.000000,0.000000}%
\pgfsetstrokecolor{currentstroke}%
\pgfsetdash{}{0pt}%
\pgfpathmoveto{\pgfqpoint{3.015000in}{1.045390in}}%
\pgfpathlineto{\pgfqpoint{3.015000in}{1.134666in}}%
\pgfusepath{stroke}%
\end{pgfscope}%
\begin{pgfscope}%
\pgfpathrectangle{\pgfqpoint{1.000000in}{0.720000in}}{\pgfqpoint{6.200000in}{4.620000in}}%
\pgfusepath{clip}%
\pgfsetbuttcap%
\pgfsetroundjoin%
\pgfsetlinewidth{0.501875pt}%
\definecolor{currentstroke}{rgb}{1.000000,0.000000,0.000000}%
\pgfsetstrokecolor{currentstroke}%
\pgfsetdash{}{0pt}%
\pgfpathmoveto{\pgfqpoint{3.046000in}{1.045390in}}%
\pgfpathlineto{\pgfqpoint{3.046000in}{1.156940in}}%
\pgfusepath{stroke}%
\end{pgfscope}%
\begin{pgfscope}%
\pgfpathrectangle{\pgfqpoint{1.000000in}{0.720000in}}{\pgfqpoint{6.200000in}{4.620000in}}%
\pgfusepath{clip}%
\pgfsetbuttcap%
\pgfsetroundjoin%
\pgfsetlinewidth{0.501875pt}%
\definecolor{currentstroke}{rgb}{1.000000,0.000000,0.000000}%
\pgfsetstrokecolor{currentstroke}%
\pgfsetdash{}{0pt}%
\pgfpathmoveto{\pgfqpoint{3.077000in}{1.045390in}}%
\pgfpathlineto{\pgfqpoint{3.077000in}{1.159622in}}%
\pgfusepath{stroke}%
\end{pgfscope}%
\begin{pgfscope}%
\pgfpathrectangle{\pgfqpoint{1.000000in}{0.720000in}}{\pgfqpoint{6.200000in}{4.620000in}}%
\pgfusepath{clip}%
\pgfsetbuttcap%
\pgfsetroundjoin%
\pgfsetlinewidth{0.501875pt}%
\definecolor{currentstroke}{rgb}{1.000000,0.000000,0.000000}%
\pgfsetstrokecolor{currentstroke}%
\pgfsetdash{}{0pt}%
\pgfpathmoveto{\pgfqpoint{3.542000in}{1.045390in}}%
\pgfpathlineto{\pgfqpoint{3.542000in}{1.191411in}}%
\pgfusepath{stroke}%
\end{pgfscope}%
\begin{pgfscope}%
\pgfpathrectangle{\pgfqpoint{1.000000in}{0.720000in}}{\pgfqpoint{6.200000in}{4.620000in}}%
\pgfusepath{clip}%
\pgfsetbuttcap%
\pgfsetroundjoin%
\pgfsetlinewidth{0.501875pt}%
\definecolor{currentstroke}{rgb}{1.000000,0.000000,0.000000}%
\pgfsetstrokecolor{currentstroke}%
\pgfsetdash{}{0pt}%
\pgfpathmoveto{\pgfqpoint{3.573000in}{1.045390in}}%
\pgfpathlineto{\pgfqpoint{3.573000in}{1.231733in}}%
\pgfusepath{stroke}%
\end{pgfscope}%
\begin{pgfscope}%
\pgfpathrectangle{\pgfqpoint{1.000000in}{0.720000in}}{\pgfqpoint{6.200000in}{4.620000in}}%
\pgfusepath{clip}%
\pgfsetbuttcap%
\pgfsetroundjoin%
\pgfsetlinewidth{0.501875pt}%
\definecolor{currentstroke}{rgb}{1.000000,0.000000,0.000000}%
\pgfsetstrokecolor{currentstroke}%
\pgfsetdash{}{0pt}%
\pgfpathmoveto{\pgfqpoint{3.604000in}{1.045390in}}%
\pgfpathlineto{\pgfqpoint{3.604000in}{1.316331in}}%
\pgfusepath{stroke}%
\end{pgfscope}%
\begin{pgfscope}%
\pgfpathrectangle{\pgfqpoint{1.000000in}{0.720000in}}{\pgfqpoint{6.200000in}{4.620000in}}%
\pgfusepath{clip}%
\pgfsetbuttcap%
\pgfsetroundjoin%
\pgfsetlinewidth{0.501875pt}%
\definecolor{currentstroke}{rgb}{1.000000,0.000000,0.000000}%
\pgfsetstrokecolor{currentstroke}%
\pgfsetdash{}{0pt}%
\pgfpathmoveto{\pgfqpoint{3.635000in}{1.045390in}}%
\pgfpathlineto{\pgfqpoint{3.635000in}{1.308895in}}%
\pgfusepath{stroke}%
\end{pgfscope}%
\begin{pgfscope}%
\pgfpathrectangle{\pgfqpoint{1.000000in}{0.720000in}}{\pgfqpoint{6.200000in}{4.620000in}}%
\pgfusepath{clip}%
\pgfsetbuttcap%
\pgfsetroundjoin%
\pgfsetlinewidth{0.501875pt}%
\definecolor{currentstroke}{rgb}{1.000000,0.000000,0.000000}%
\pgfsetstrokecolor{currentstroke}%
\pgfsetdash{}{0pt}%
\pgfpathmoveto{\pgfqpoint{3.666000in}{1.045390in}}%
\pgfpathlineto{\pgfqpoint{3.666000in}{1.362031in}}%
\pgfusepath{stroke}%
\end{pgfscope}%
\begin{pgfscope}%
\pgfpathrectangle{\pgfqpoint{1.000000in}{0.720000in}}{\pgfqpoint{6.200000in}{4.620000in}}%
\pgfusepath{clip}%
\pgfsetbuttcap%
\pgfsetroundjoin%
\pgfsetlinewidth{0.501875pt}%
\definecolor{currentstroke}{rgb}{1.000000,0.000000,0.000000}%
\pgfsetstrokecolor{currentstroke}%
\pgfsetdash{}{0pt}%
\pgfpathmoveto{\pgfqpoint{3.697000in}{1.045390in}}%
\pgfpathlineto{\pgfqpoint{3.697000in}{1.338331in}}%
\pgfusepath{stroke}%
\end{pgfscope}%
\begin{pgfscope}%
\pgfpathrectangle{\pgfqpoint{1.000000in}{0.720000in}}{\pgfqpoint{6.200000in}{4.620000in}}%
\pgfusepath{clip}%
\pgfsetbuttcap%
\pgfsetroundjoin%
\pgfsetlinewidth{0.501875pt}%
\definecolor{currentstroke}{rgb}{1.000000,0.000000,0.000000}%
\pgfsetstrokecolor{currentstroke}%
\pgfsetdash{}{0pt}%
\pgfpathmoveto{\pgfqpoint{3.728000in}{1.045390in}}%
\pgfpathlineto{\pgfqpoint{3.728000in}{1.273715in}}%
\pgfusepath{stroke}%
\end{pgfscope}%
\begin{pgfscope}%
\pgfpathrectangle{\pgfqpoint{1.000000in}{0.720000in}}{\pgfqpoint{6.200000in}{4.620000in}}%
\pgfusepath{clip}%
\pgfsetbuttcap%
\pgfsetroundjoin%
\pgfsetlinewidth{0.501875pt}%
\definecolor{currentstroke}{rgb}{1.000000,0.000000,0.000000}%
\pgfsetstrokecolor{currentstroke}%
\pgfsetdash{}{0pt}%
\pgfpathmoveto{\pgfqpoint{3.759000in}{1.045390in}}%
\pgfpathlineto{\pgfqpoint{3.759000in}{1.273490in}}%
\pgfusepath{stroke}%
\end{pgfscope}%
\begin{pgfscope}%
\pgfpathrectangle{\pgfqpoint{1.000000in}{0.720000in}}{\pgfqpoint{6.200000in}{4.620000in}}%
\pgfusepath{clip}%
\pgfsetbuttcap%
\pgfsetroundjoin%
\pgfsetlinewidth{0.501875pt}%
\definecolor{currentstroke}{rgb}{1.000000,0.000000,0.000000}%
\pgfsetstrokecolor{currentstroke}%
\pgfsetdash{}{0pt}%
\pgfpathmoveto{\pgfqpoint{3.790000in}{1.045390in}}%
\pgfpathlineto{\pgfqpoint{3.790000in}{1.250525in}}%
\pgfusepath{stroke}%
\end{pgfscope}%
\begin{pgfscope}%
\pgfpathrectangle{\pgfqpoint{1.000000in}{0.720000in}}{\pgfqpoint{6.200000in}{4.620000in}}%
\pgfusepath{clip}%
\pgfsetbuttcap%
\pgfsetroundjoin%
\pgfsetlinewidth{0.501875pt}%
\definecolor{currentstroke}{rgb}{1.000000,0.000000,0.000000}%
\pgfsetstrokecolor{currentstroke}%
\pgfsetdash{}{0pt}%
\pgfpathmoveto{\pgfqpoint{3.821000in}{1.045390in}}%
\pgfpathlineto{\pgfqpoint{3.821000in}{1.211354in}}%
\pgfusepath{stroke}%
\end{pgfscope}%
\begin{pgfscope}%
\pgfpathrectangle{\pgfqpoint{1.000000in}{0.720000in}}{\pgfqpoint{6.200000in}{4.620000in}}%
\pgfusepath{clip}%
\pgfsetbuttcap%
\pgfsetroundjoin%
\pgfsetlinewidth{0.501875pt}%
\definecolor{currentstroke}{rgb}{1.000000,0.000000,0.000000}%
\pgfsetstrokecolor{currentstroke}%
\pgfsetdash{}{0pt}%
\pgfpathmoveto{\pgfqpoint{3.852000in}{1.045390in}}%
\pgfpathlineto{\pgfqpoint{3.852000in}{1.178588in}}%
\pgfusepath{stroke}%
\end{pgfscope}%
\begin{pgfscope}%
\pgfpathrectangle{\pgfqpoint{1.000000in}{0.720000in}}{\pgfqpoint{6.200000in}{4.620000in}}%
\pgfusepath{clip}%
\pgfsetbuttcap%
\pgfsetroundjoin%
\pgfsetlinewidth{0.501875pt}%
\definecolor{currentstroke}{rgb}{1.000000,0.000000,0.000000}%
\pgfsetstrokecolor{currentstroke}%
\pgfsetdash{}{0pt}%
\pgfpathmoveto{\pgfqpoint{3.883000in}{1.045390in}}%
\pgfpathlineto{\pgfqpoint{3.883000in}{1.167270in}}%
\pgfusepath{stroke}%
\end{pgfscope}%
\begin{pgfscope}%
\pgfpathrectangle{\pgfqpoint{1.000000in}{0.720000in}}{\pgfqpoint{6.200000in}{4.620000in}}%
\pgfusepath{clip}%
\pgfsetbuttcap%
\pgfsetroundjoin%
\pgfsetlinewidth{0.501875pt}%
\definecolor{currentstroke}{rgb}{1.000000,0.000000,0.000000}%
\pgfsetstrokecolor{currentstroke}%
\pgfsetdash{}{0pt}%
\pgfpathmoveto{\pgfqpoint{3.914000in}{1.045390in}}%
\pgfpathlineto{\pgfqpoint{3.914000in}{1.175503in}}%
\pgfusepath{stroke}%
\end{pgfscope}%
\begin{pgfscope}%
\pgfpathrectangle{\pgfqpoint{1.000000in}{0.720000in}}{\pgfqpoint{6.200000in}{4.620000in}}%
\pgfusepath{clip}%
\pgfsetbuttcap%
\pgfsetroundjoin%
\pgfsetlinewidth{0.501875pt}%
\definecolor{currentstroke}{rgb}{1.000000,0.000000,0.000000}%
\pgfsetstrokecolor{currentstroke}%
\pgfsetdash{}{0pt}%
\pgfpathmoveto{\pgfqpoint{3.945000in}{1.045390in}}%
\pgfpathlineto{\pgfqpoint{3.945000in}{1.163613in}}%
\pgfusepath{stroke}%
\end{pgfscope}%
\begin{pgfscope}%
\pgfsetrectcap%
\pgfsetmiterjoin%
\pgfsetlinewidth{0.803000pt}%
\definecolor{currentstroke}{rgb}{0.000000,0.000000,0.000000}%
\pgfsetstrokecolor{currentstroke}%
\pgfsetdash{}{0pt}%
\pgfpathmoveto{\pgfqpoint{1.000000in}{0.720000in}}%
\pgfpathlineto{\pgfqpoint{1.000000in}{5.340000in}}%
\pgfusepath{stroke}%
\end{pgfscope}%
\begin{pgfscope}%
\pgfsetrectcap%
\pgfsetmiterjoin%
\pgfsetlinewidth{0.803000pt}%
\definecolor{currentstroke}{rgb}{0.000000,0.000000,0.000000}%
\pgfsetstrokecolor{currentstroke}%
\pgfsetdash{}{0pt}%
\pgfpathmoveto{\pgfqpoint{7.200000in}{0.720000in}}%
\pgfpathlineto{\pgfqpoint{7.200000in}{5.340000in}}%
\pgfusepath{stroke}%
\end{pgfscope}%
\begin{pgfscope}%
\pgfsetrectcap%
\pgfsetmiterjoin%
\pgfsetlinewidth{0.803000pt}%
\definecolor{currentstroke}{rgb}{0.000000,0.000000,0.000000}%
\pgfsetstrokecolor{currentstroke}%
\pgfsetdash{}{0pt}%
\pgfpathmoveto{\pgfqpoint{1.000000in}{0.720000in}}%
\pgfpathlineto{\pgfqpoint{7.200000in}{0.720000in}}%
\pgfusepath{stroke}%
\end{pgfscope}%
\begin{pgfscope}%
\pgfsetrectcap%
\pgfsetmiterjoin%
\pgfsetlinewidth{0.803000pt}%
\definecolor{currentstroke}{rgb}{0.000000,0.000000,0.000000}%
\pgfsetstrokecolor{currentstroke}%
\pgfsetdash{}{0pt}%
\pgfpathmoveto{\pgfqpoint{1.000000in}{5.340000in}}%
\pgfpathlineto{\pgfqpoint{7.200000in}{5.340000in}}%
\pgfusepath{stroke}%
\end{pgfscope}%
\begin{pgfscope}%
\pgfsetroundcap%
\pgfsetroundjoin%
\definecolor{currentfill}{rgb}{0.000000,0.000000,0.000000}%
\pgfsetfillcolor{currentfill}%
\pgfsetlinewidth{1.003750pt}%
\definecolor{currentstroke}{rgb}{0.000000,0.000000,0.000000}%
\pgfsetstrokecolor{currentstroke}%
\pgfsetdash{}{0pt}%
\pgfpathmoveto{\pgfqpoint{4.911261in}{1.537170in}}%
\pgfpathquadraticcurveto{\pgfqpoint{3.916069in}{1.537170in}}{\pgfqpoint{2.920878in}{1.537170in}}%
\pgfpathlineto{\pgfqpoint{2.920878in}{1.523281in}}%
\pgfpathquadraticcurveto{\pgfqpoint{2.837557in}{1.537170in}}{\pgfqpoint{2.754236in}{1.551059in}}%
\pgfpathquadraticcurveto{\pgfqpoint{2.837557in}{1.564948in}}{\pgfqpoint{2.920878in}{1.578837in}}%
\pgfpathlineto{\pgfqpoint{2.920878in}{1.564948in}}%
\pgfpathquadraticcurveto{\pgfqpoint{3.916069in}{1.564948in}}{\pgfqpoint{4.911261in}{1.564948in}}%
\pgfpathlineto{\pgfqpoint{4.911261in}{1.537170in}}%
\pgfpathlineto{\pgfqpoint{4.911261in}{1.537170in}}%
\pgfpathclose%
\pgfusepath{stroke,fill}%
\end{pgfscope}%
\begin{pgfscope}%
\pgfsetbuttcap%
\pgfsetmiterjoin%
\definecolor{currentfill}{rgb}{1.000000,1.000000,1.000000}%
\pgfsetfillcolor{currentfill}%
\pgfsetfillopacity{0.800000}%
\pgfsetlinewidth{1.003750pt}%
\definecolor{currentstroke}{rgb}{0.800000,0.800000,0.800000}%
\pgfsetstrokecolor{currentstroke}%
\pgfsetstrokeopacity{0.800000}%
\pgfsetdash{}{0pt}%
\pgfpathmoveto{\pgfqpoint{4.976872in}{3.932908in}}%
\pgfpathlineto{\pgfqpoint{7.005556in}{3.932908in}}%
\pgfpathquadraticcurveto{\pgfqpoint{7.061111in}{3.932908in}}{\pgfqpoint{7.061111in}{3.988464in}}%
\pgfpathlineto{\pgfqpoint{7.061111in}{5.145556in}}%
\pgfpathquadraticcurveto{\pgfqpoint{7.061111in}{5.201111in}}{\pgfqpoint{7.005556in}{5.201111in}}%
\pgfpathlineto{\pgfqpoint{4.976872in}{5.201111in}}%
\pgfpathquadraticcurveto{\pgfqpoint{4.921317in}{5.201111in}}{\pgfqpoint{4.921317in}{5.145556in}}%
\pgfpathlineto{\pgfqpoint{4.921317in}{3.988464in}}%
\pgfpathquadraticcurveto{\pgfqpoint{4.921317in}{3.932908in}}{\pgfqpoint{4.976872in}{3.932908in}}%
\pgfpathlineto{\pgfqpoint{4.976872in}{3.932908in}}%
\pgfpathclose%
\pgfusepath{stroke,fill}%
\end{pgfscope}%
\begin{pgfscope}%
\pgfsetrectcap%
\pgfsetroundjoin%
\pgfsetlinewidth{2.007500pt}%
\definecolor{currentstroke}{rgb}{0.121569,0.466667,0.705882}%
\pgfsetstrokecolor{currentstroke}%
\pgfsetdash{}{0pt}%
\pgfpathmoveto{\pgfqpoint{5.032428in}{4.987184in}}%
\pgfpathlineto{\pgfqpoint{5.310206in}{4.987184in}}%
\pgfpathlineto{\pgfqpoint{5.587983in}{4.987184in}}%
\pgfusepath{stroke}%
\end{pgfscope}%
\begin{pgfscope}%
\definecolor{textcolor}{rgb}{0.000000,0.000000,0.000000}%
\pgfsetstrokecolor{textcolor}%
\pgfsetfillcolor{textcolor}%
\pgftext[x=5.810206in,y=4.889962in,left,base]{\color{textcolor}\sffamily\fontsize{20.000000}{24.000000}\selectfont Waveform}%
\end{pgfscope}%
\begin{pgfscope}%
\pgfsetbuttcap%
\pgfsetroundjoin%
\pgfsetlinewidth{2.007500pt}%
\definecolor{currentstroke}{rgb}{0.000000,0.500000,0.000000}%
\pgfsetstrokecolor{currentstroke}%
\pgfsetdash{}{0pt}%
\pgfpathmoveto{\pgfqpoint{5.032428in}{4.592227in}}%
\pgfpathlineto{\pgfqpoint{5.587983in}{4.592227in}}%
\pgfusepath{stroke}%
\end{pgfscope}%
\begin{pgfscope}%
\definecolor{textcolor}{rgb}{0.000000,0.000000,0.000000}%
\pgfsetstrokecolor{textcolor}%
\pgfsetfillcolor{textcolor}%
\pgftext[x=5.810206in,y=4.495005in,left,base]{\color{textcolor}\sffamily\fontsize{20.000000}{24.000000}\selectfont Threshold}%
\end{pgfscope}%
\begin{pgfscope}%
\pgfsetbuttcap%
\pgfsetroundjoin%
\pgfsetlinewidth{0.501875pt}%
\definecolor{currentstroke}{rgb}{1.000000,0.000000,0.000000}%
\pgfsetstrokecolor{currentstroke}%
\pgfsetdash{}{0pt}%
\pgfpathmoveto{\pgfqpoint{5.032428in}{4.197271in}}%
\pgfpathlineto{\pgfqpoint{5.587983in}{4.197271in}}%
\pgfusepath{stroke}%
\end{pgfscope}%
\begin{pgfscope}%
\definecolor{textcolor}{rgb}{0.000000,0.000000,0.000000}%
\pgfsetstrokecolor{textcolor}%
\pgfsetfillcolor{textcolor}%
\pgftext[x=5.810206in,y=4.100048in,left,base]{\color{textcolor}\sffamily\fontsize{20.000000}{24.000000}\selectfont Charge}%
\end{pgfscope}%
\end{pgfpicture}%
\makeatother%
\endgroup%
}
    \caption{\label{fig:shifting} A waveform shifting example gives \\ $\Delta t_0=\SI{-0.69}{ns}$, $\mathrm{RSS}=\SI{369.6}{mV^2}$, $D_\mathrm{w}=\SI{3.11}{ns}$.}
  \end{subfigure}
  \begin{subfigure}{.5\textwidth}
    \centering
    \resizebox{\textwidth}{!}{%% Creator: Matplotlib, PGF backend
%%
%% To include the figure in your LaTeX document, write
%%   \input{<filename>.pgf}
%%
%% Make sure the required packages are loaded in your preamble
%%   \usepackage{pgf}
%%
%% Also ensure that all the required font packages are loaded; for instance,
%% the lmodern package is sometimes necessary when using math font.
%%   \usepackage{lmodern}
%%
%% Figures using additional raster images can only be included by \input if
%% they are in the same directory as the main LaTeX file. For loading figures
%% from other directories you can use the `import` package
%%   \usepackage{import}
%%
%% and then include the figures with
%%   \import{<path to file>}{<filename>.pgf}
%%
%% Matplotlib used the following preamble
%%   \usepackage[detect-all,locale=DE]{siunitx}
%%
\begingroup%
\makeatletter%
\begin{pgfpicture}%
\pgfpathrectangle{\pgfpointorigin}{\pgfqpoint{8.000000in}{6.000000in}}%
\pgfusepath{use as bounding box, clip}%
\begin{pgfscope}%
\pgfsetbuttcap%
\pgfsetmiterjoin%
\definecolor{currentfill}{rgb}{1.000000,1.000000,1.000000}%
\pgfsetfillcolor{currentfill}%
\pgfsetlinewidth{0.000000pt}%
\definecolor{currentstroke}{rgb}{1.000000,1.000000,1.000000}%
\pgfsetstrokecolor{currentstroke}%
\pgfsetdash{}{0pt}%
\pgfpathmoveto{\pgfqpoint{0.000000in}{0.000000in}}%
\pgfpathlineto{\pgfqpoint{8.000000in}{0.000000in}}%
\pgfpathlineto{\pgfqpoint{8.000000in}{6.000000in}}%
\pgfpathlineto{\pgfqpoint{0.000000in}{6.000000in}}%
\pgfpathlineto{\pgfqpoint{0.000000in}{0.000000in}}%
\pgfpathclose%
\pgfusepath{fill}%
\end{pgfscope}%
\begin{pgfscope}%
\pgfsetbuttcap%
\pgfsetmiterjoin%
\definecolor{currentfill}{rgb}{1.000000,1.000000,1.000000}%
\pgfsetfillcolor{currentfill}%
\pgfsetlinewidth{0.000000pt}%
\definecolor{currentstroke}{rgb}{0.000000,0.000000,0.000000}%
\pgfsetstrokecolor{currentstroke}%
\pgfsetstrokeopacity{0.000000}%
\pgfsetdash{}{0pt}%
\pgfpathmoveto{\pgfqpoint{1.000000in}{0.720000in}}%
\pgfpathlineto{\pgfqpoint{7.200000in}{0.720000in}}%
\pgfpathlineto{\pgfqpoint{7.200000in}{5.340000in}}%
\pgfpathlineto{\pgfqpoint{1.000000in}{5.340000in}}%
\pgfpathlineto{\pgfqpoint{1.000000in}{0.720000in}}%
\pgfpathclose%
\pgfusepath{fill}%
\end{pgfscope}%
\begin{pgfscope}%
\pgfsetbuttcap%
\pgfsetroundjoin%
\definecolor{currentfill}{rgb}{0.000000,0.000000,0.000000}%
\pgfsetfillcolor{currentfill}%
\pgfsetlinewidth{0.803000pt}%
\definecolor{currentstroke}{rgb}{0.000000,0.000000,0.000000}%
\pgfsetstrokecolor{currentstroke}%
\pgfsetdash{}{0pt}%
\pgfsys@defobject{currentmarker}{\pgfqpoint{0.000000in}{-0.048611in}}{\pgfqpoint{0.000000in}{0.000000in}}{%
\pgfpathmoveto{\pgfqpoint{0.000000in}{0.000000in}}%
\pgfpathlineto{\pgfqpoint{0.000000in}{-0.048611in}}%
\pgfusepath{stroke,fill}%
}%
\begin{pgfscope}%
\pgfsys@transformshift{1.310000in}{0.720000in}%
\pgfsys@useobject{currentmarker}{}%
\end{pgfscope}%
\end{pgfscope}%
\begin{pgfscope}%
\definecolor{textcolor}{rgb}{0.000000,0.000000,0.000000}%
\pgfsetstrokecolor{textcolor}%
\pgfsetfillcolor{textcolor}%
\pgftext[x=1.310000in,y=0.622778in,,top]{\color{textcolor}\sffamily\fontsize{20.000000}{24.000000}\selectfont \(\displaystyle {450}\)}%
\end{pgfscope}%
\begin{pgfscope}%
\pgfsetbuttcap%
\pgfsetroundjoin%
\definecolor{currentfill}{rgb}{0.000000,0.000000,0.000000}%
\pgfsetfillcolor{currentfill}%
\pgfsetlinewidth{0.803000pt}%
\definecolor{currentstroke}{rgb}{0.000000,0.000000,0.000000}%
\pgfsetstrokecolor{currentstroke}%
\pgfsetdash{}{0pt}%
\pgfsys@defobject{currentmarker}{\pgfqpoint{0.000000in}{-0.048611in}}{\pgfqpoint{0.000000in}{0.000000in}}{%
\pgfpathmoveto{\pgfqpoint{0.000000in}{0.000000in}}%
\pgfpathlineto{\pgfqpoint{0.000000in}{-0.048611in}}%
\pgfusepath{stroke,fill}%
}%
\begin{pgfscope}%
\pgfsys@transformshift{2.860000in}{0.720000in}%
\pgfsys@useobject{currentmarker}{}%
\end{pgfscope}%
\end{pgfscope}%
\begin{pgfscope}%
\definecolor{textcolor}{rgb}{0.000000,0.000000,0.000000}%
\pgfsetstrokecolor{textcolor}%
\pgfsetfillcolor{textcolor}%
\pgftext[x=2.860000in,y=0.622778in,,top]{\color{textcolor}\sffamily\fontsize{20.000000}{24.000000}\selectfont \(\displaystyle {500}\)}%
\end{pgfscope}%
\begin{pgfscope}%
\pgfsetbuttcap%
\pgfsetroundjoin%
\definecolor{currentfill}{rgb}{0.000000,0.000000,0.000000}%
\pgfsetfillcolor{currentfill}%
\pgfsetlinewidth{0.803000pt}%
\definecolor{currentstroke}{rgb}{0.000000,0.000000,0.000000}%
\pgfsetstrokecolor{currentstroke}%
\pgfsetdash{}{0pt}%
\pgfsys@defobject{currentmarker}{\pgfqpoint{0.000000in}{-0.048611in}}{\pgfqpoint{0.000000in}{0.000000in}}{%
\pgfpathmoveto{\pgfqpoint{0.000000in}{0.000000in}}%
\pgfpathlineto{\pgfqpoint{0.000000in}{-0.048611in}}%
\pgfusepath{stroke,fill}%
}%
\begin{pgfscope}%
\pgfsys@transformshift{4.410000in}{0.720000in}%
\pgfsys@useobject{currentmarker}{}%
\end{pgfscope}%
\end{pgfscope}%
\begin{pgfscope}%
\definecolor{textcolor}{rgb}{0.000000,0.000000,0.000000}%
\pgfsetstrokecolor{textcolor}%
\pgfsetfillcolor{textcolor}%
\pgftext[x=4.410000in,y=0.622778in,,top]{\color{textcolor}\sffamily\fontsize{20.000000}{24.000000}\selectfont \(\displaystyle {550}\)}%
\end{pgfscope}%
\begin{pgfscope}%
\pgfsetbuttcap%
\pgfsetroundjoin%
\definecolor{currentfill}{rgb}{0.000000,0.000000,0.000000}%
\pgfsetfillcolor{currentfill}%
\pgfsetlinewidth{0.803000pt}%
\definecolor{currentstroke}{rgb}{0.000000,0.000000,0.000000}%
\pgfsetstrokecolor{currentstroke}%
\pgfsetdash{}{0pt}%
\pgfsys@defobject{currentmarker}{\pgfqpoint{0.000000in}{-0.048611in}}{\pgfqpoint{0.000000in}{0.000000in}}{%
\pgfpathmoveto{\pgfqpoint{0.000000in}{0.000000in}}%
\pgfpathlineto{\pgfqpoint{0.000000in}{-0.048611in}}%
\pgfusepath{stroke,fill}%
}%
\begin{pgfscope}%
\pgfsys@transformshift{5.960000in}{0.720000in}%
\pgfsys@useobject{currentmarker}{}%
\end{pgfscope}%
\end{pgfscope}%
\begin{pgfscope}%
\definecolor{textcolor}{rgb}{0.000000,0.000000,0.000000}%
\pgfsetstrokecolor{textcolor}%
\pgfsetfillcolor{textcolor}%
\pgftext[x=5.960000in,y=0.622778in,,top]{\color{textcolor}\sffamily\fontsize{20.000000}{24.000000}\selectfont \(\displaystyle {600}\)}%
\end{pgfscope}%
\begin{pgfscope}%
\definecolor{textcolor}{rgb}{0.000000,0.000000,0.000000}%
\pgfsetstrokecolor{textcolor}%
\pgfsetfillcolor{textcolor}%
\pgftext[x=4.100000in,y=0.311155in,,top]{\color{textcolor}\sffamily\fontsize{20.000000}{24.000000}\selectfont \(\displaystyle \mathrm{t}/\si{ns}\)}%
\end{pgfscope}%
\begin{pgfscope}%
\pgfsetbuttcap%
\pgfsetroundjoin%
\definecolor{currentfill}{rgb}{0.000000,0.000000,0.000000}%
\pgfsetfillcolor{currentfill}%
\pgfsetlinewidth{0.803000pt}%
\definecolor{currentstroke}{rgb}{0.000000,0.000000,0.000000}%
\pgfsetstrokecolor{currentstroke}%
\pgfsetdash{}{0pt}%
\pgfsys@defobject{currentmarker}{\pgfqpoint{-0.048611in}{0.000000in}}{\pgfqpoint{-0.000000in}{0.000000in}}{%
\pgfpathmoveto{\pgfqpoint{-0.000000in}{0.000000in}}%
\pgfpathlineto{\pgfqpoint{-0.048611in}{0.000000in}}%
\pgfusepath{stroke,fill}%
}%
\begin{pgfscope}%
\pgfsys@transformshift{1.000000in}{1.137431in}%
\pgfsys@useobject{currentmarker}{}%
\end{pgfscope}%
\end{pgfscope}%
\begin{pgfscope}%
\definecolor{textcolor}{rgb}{0.000000,0.000000,0.000000}%
\pgfsetstrokecolor{textcolor}%
\pgfsetfillcolor{textcolor}%
\pgftext[x=0.770670in, y=1.037412in, left, base]{\color{textcolor}\sffamily\fontsize{20.000000}{24.000000}\selectfont \(\displaystyle {0}\)}%
\end{pgfscope}%
\begin{pgfscope}%
\pgfsetbuttcap%
\pgfsetroundjoin%
\definecolor{currentfill}{rgb}{0.000000,0.000000,0.000000}%
\pgfsetfillcolor{currentfill}%
\pgfsetlinewidth{0.803000pt}%
\definecolor{currentstroke}{rgb}{0.000000,0.000000,0.000000}%
\pgfsetstrokecolor{currentstroke}%
\pgfsetdash{}{0pt}%
\pgfsys@defobject{currentmarker}{\pgfqpoint{-0.048611in}{0.000000in}}{\pgfqpoint{-0.000000in}{0.000000in}}{%
\pgfpathmoveto{\pgfqpoint{-0.000000in}{0.000000in}}%
\pgfpathlineto{\pgfqpoint{-0.048611in}{0.000000in}}%
\pgfusepath{stroke,fill}%
}%
\begin{pgfscope}%
\pgfsys@transformshift{1.000000in}{2.098126in}%
\pgfsys@useobject{currentmarker}{}%
\end{pgfscope}%
\end{pgfscope}%
\begin{pgfscope}%
\definecolor{textcolor}{rgb}{0.000000,0.000000,0.000000}%
\pgfsetstrokecolor{textcolor}%
\pgfsetfillcolor{textcolor}%
\pgftext[x=0.638563in, y=1.998107in, left, base]{\color{textcolor}\sffamily\fontsize{20.000000}{24.000000}\selectfont \(\displaystyle {10}\)}%
\end{pgfscope}%
\begin{pgfscope}%
\pgfsetbuttcap%
\pgfsetroundjoin%
\definecolor{currentfill}{rgb}{0.000000,0.000000,0.000000}%
\pgfsetfillcolor{currentfill}%
\pgfsetlinewidth{0.803000pt}%
\definecolor{currentstroke}{rgb}{0.000000,0.000000,0.000000}%
\pgfsetstrokecolor{currentstroke}%
\pgfsetdash{}{0pt}%
\pgfsys@defobject{currentmarker}{\pgfqpoint{-0.048611in}{0.000000in}}{\pgfqpoint{-0.000000in}{0.000000in}}{%
\pgfpathmoveto{\pgfqpoint{-0.000000in}{0.000000in}}%
\pgfpathlineto{\pgfqpoint{-0.048611in}{0.000000in}}%
\pgfusepath{stroke,fill}%
}%
\begin{pgfscope}%
\pgfsys@transformshift{1.000000in}{3.058822in}%
\pgfsys@useobject{currentmarker}{}%
\end{pgfscope}%
\end{pgfscope}%
\begin{pgfscope}%
\definecolor{textcolor}{rgb}{0.000000,0.000000,0.000000}%
\pgfsetstrokecolor{textcolor}%
\pgfsetfillcolor{textcolor}%
\pgftext[x=0.638563in, y=2.958802in, left, base]{\color{textcolor}\sffamily\fontsize{20.000000}{24.000000}\selectfont \(\displaystyle {20}\)}%
\end{pgfscope}%
\begin{pgfscope}%
\pgfsetbuttcap%
\pgfsetroundjoin%
\definecolor{currentfill}{rgb}{0.000000,0.000000,0.000000}%
\pgfsetfillcolor{currentfill}%
\pgfsetlinewidth{0.803000pt}%
\definecolor{currentstroke}{rgb}{0.000000,0.000000,0.000000}%
\pgfsetstrokecolor{currentstroke}%
\pgfsetdash{}{0pt}%
\pgfsys@defobject{currentmarker}{\pgfqpoint{-0.048611in}{0.000000in}}{\pgfqpoint{-0.000000in}{0.000000in}}{%
\pgfpathmoveto{\pgfqpoint{-0.000000in}{0.000000in}}%
\pgfpathlineto{\pgfqpoint{-0.048611in}{0.000000in}}%
\pgfusepath{stroke,fill}%
}%
\begin{pgfscope}%
\pgfsys@transformshift{1.000000in}{4.019517in}%
\pgfsys@useobject{currentmarker}{}%
\end{pgfscope}%
\end{pgfscope}%
\begin{pgfscope}%
\definecolor{textcolor}{rgb}{0.000000,0.000000,0.000000}%
\pgfsetstrokecolor{textcolor}%
\pgfsetfillcolor{textcolor}%
\pgftext[x=0.638563in, y=3.919498in, left, base]{\color{textcolor}\sffamily\fontsize{20.000000}{24.000000}\selectfont \(\displaystyle {30}\)}%
\end{pgfscope}%
\begin{pgfscope}%
\pgfsetbuttcap%
\pgfsetroundjoin%
\definecolor{currentfill}{rgb}{0.000000,0.000000,0.000000}%
\pgfsetfillcolor{currentfill}%
\pgfsetlinewidth{0.803000pt}%
\definecolor{currentstroke}{rgb}{0.000000,0.000000,0.000000}%
\pgfsetstrokecolor{currentstroke}%
\pgfsetdash{}{0pt}%
\pgfsys@defobject{currentmarker}{\pgfqpoint{-0.048611in}{0.000000in}}{\pgfqpoint{-0.000000in}{0.000000in}}{%
\pgfpathmoveto{\pgfqpoint{-0.000000in}{0.000000in}}%
\pgfpathlineto{\pgfqpoint{-0.048611in}{0.000000in}}%
\pgfusepath{stroke,fill}%
}%
\begin{pgfscope}%
\pgfsys@transformshift{1.000000in}{4.980213in}%
\pgfsys@useobject{currentmarker}{}%
\end{pgfscope}%
\end{pgfscope}%
\begin{pgfscope}%
\definecolor{textcolor}{rgb}{0.000000,0.000000,0.000000}%
\pgfsetstrokecolor{textcolor}%
\pgfsetfillcolor{textcolor}%
\pgftext[x=0.638563in, y=4.880193in, left, base]{\color{textcolor}\sffamily\fontsize{20.000000}{24.000000}\selectfont \(\displaystyle {40}\)}%
\end{pgfscope}%
\begin{pgfscope}%
\definecolor{textcolor}{rgb}{0.000000,0.000000,0.000000}%
\pgfsetstrokecolor{textcolor}%
\pgfsetfillcolor{textcolor}%
\pgftext[x=0.583007in,y=3.030000in,,bottom,rotate=90.000000]{\color{textcolor}\sffamily\fontsize{20.000000}{24.000000}\selectfont \(\displaystyle \mathrm{Voltage}/\si{mV}\)}%
\end{pgfscope}%
\begin{pgfscope}%
\pgfpathrectangle{\pgfqpoint{1.000000in}{0.720000in}}{\pgfqpoint{6.200000in}{4.620000in}}%
\pgfusepath{clip}%
\pgfsetrectcap%
\pgfsetroundjoin%
\pgfsetlinewidth{2.007500pt}%
\definecolor{currentstroke}{rgb}{0.121569,0.466667,0.705882}%
\pgfsetstrokecolor{currentstroke}%
\pgfsetdash{}{0pt}%
\pgfpathmoveto{\pgfqpoint{0.990000in}{1.197442in}}%
\pgfpathlineto{\pgfqpoint{1.000000in}{1.235205in}}%
\pgfpathlineto{\pgfqpoint{1.031000in}{1.099004in}}%
\pgfpathlineto{\pgfqpoint{1.062000in}{1.179155in}}%
\pgfpathlineto{\pgfqpoint{1.093000in}{1.165841in}}%
\pgfpathlineto{\pgfqpoint{1.124000in}{1.150777in}}%
\pgfpathlineto{\pgfqpoint{1.155000in}{1.245945in}}%
\pgfpathlineto{\pgfqpoint{1.186000in}{1.162170in}}%
\pgfpathlineto{\pgfqpoint{1.217000in}{1.095220in}}%
\pgfpathlineto{\pgfqpoint{1.248000in}{1.233295in}}%
\pgfpathlineto{\pgfqpoint{1.279000in}{1.163096in}}%
\pgfpathlineto{\pgfqpoint{1.310000in}{1.080511in}}%
\pgfpathlineto{\pgfqpoint{1.341000in}{1.185161in}}%
\pgfpathlineto{\pgfqpoint{1.372000in}{1.121058in}}%
\pgfpathlineto{\pgfqpoint{1.403000in}{1.348860in}}%
\pgfpathlineto{\pgfqpoint{1.434000in}{1.076093in}}%
\pgfpathlineto{\pgfqpoint{1.465000in}{1.015394in}}%
\pgfpathlineto{\pgfqpoint{1.496000in}{1.098124in}}%
\pgfpathlineto{\pgfqpoint{1.527000in}{1.042952in}}%
\pgfpathlineto{\pgfqpoint{1.558000in}{1.016453in}}%
\pgfpathlineto{\pgfqpoint{1.589000in}{1.128707in}}%
\pgfpathlineto{\pgfqpoint{1.620000in}{1.216013in}}%
\pgfpathlineto{\pgfqpoint{1.651000in}{1.192830in}}%
\pgfpathlineto{\pgfqpoint{1.682000in}{1.304739in}}%
\pgfpathlineto{\pgfqpoint{1.713000in}{1.219978in}}%
\pgfpathlineto{\pgfqpoint{1.744000in}{1.244623in}}%
\pgfpathlineto{\pgfqpoint{1.775000in}{1.259055in}}%
\pgfpathlineto{\pgfqpoint{1.806000in}{1.226369in}}%
\pgfpathlineto{\pgfqpoint{1.837000in}{1.118677in}}%
\pgfpathlineto{\pgfqpoint{1.868000in}{1.020339in}}%
\pgfpathlineto{\pgfqpoint{1.899000in}{1.156892in}}%
\pgfpathlineto{\pgfqpoint{1.930000in}{1.108282in}}%
\pgfpathlineto{\pgfqpoint{1.961000in}{1.221832in}}%
\pgfpathlineto{\pgfqpoint{1.992000in}{1.216124in}}%
\pgfpathlineto{\pgfqpoint{2.023000in}{1.432934in}}%
\pgfpathlineto{\pgfqpoint{2.054000in}{1.792922in}}%
\pgfpathlineto{\pgfqpoint{2.085000in}{2.209349in}}%
\pgfpathlineto{\pgfqpoint{2.116000in}{2.935539in}}%
\pgfpathlineto{\pgfqpoint{2.147000in}{3.119111in}}%
\pgfpathlineto{\pgfqpoint{2.178000in}{3.529906in}}%
\pgfpathlineto{\pgfqpoint{2.209000in}{3.659815in}}%
\pgfpathlineto{\pgfqpoint{2.240000in}{3.420242in}}%
\pgfpathlineto{\pgfqpoint{2.271000in}{3.403162in}}%
\pgfpathlineto{\pgfqpoint{2.302000in}{3.322872in}}%
\pgfpathlineto{\pgfqpoint{2.333000in}{2.821663in}}%
\pgfpathlineto{\pgfqpoint{2.364000in}{2.648395in}}%
\pgfpathlineto{\pgfqpoint{2.395000in}{2.637238in}}%
\pgfpathlineto{\pgfqpoint{2.426000in}{2.344731in}}%
\pgfpathlineto{\pgfqpoint{2.457000in}{1.962886in}}%
\pgfpathlineto{\pgfqpoint{2.488000in}{1.872100in}}%
\pgfpathlineto{\pgfqpoint{2.519000in}{1.789834in}}%
\pgfpathlineto{\pgfqpoint{2.550000in}{1.745470in}}%
\pgfpathlineto{\pgfqpoint{2.581000in}{1.965000in}}%
\pgfpathlineto{\pgfqpoint{2.612000in}{2.544152in}}%
\pgfpathlineto{\pgfqpoint{2.643000in}{2.786985in}}%
\pgfpathlineto{\pgfqpoint{2.674000in}{3.184973in}}%
\pgfpathlineto{\pgfqpoint{2.705000in}{3.275313in}}%
\pgfpathlineto{\pgfqpoint{2.736000in}{3.160526in}}%
\pgfpathlineto{\pgfqpoint{2.767000in}{2.996163in}}%
\pgfpathlineto{\pgfqpoint{2.798000in}{2.853090in}}%
\pgfpathlineto{\pgfqpoint{2.829000in}{2.686134in}}%
\pgfpathlineto{\pgfqpoint{2.860000in}{2.567944in}}%
\pgfpathlineto{\pgfqpoint{2.891000in}{2.285229in}}%
\pgfpathlineto{\pgfqpoint{2.922000in}{2.332108in}}%
\pgfpathlineto{\pgfqpoint{2.953000in}{2.129962in}}%
\pgfpathlineto{\pgfqpoint{2.984000in}{2.654784in}}%
\pgfpathlineto{\pgfqpoint{3.015000in}{2.661361in}}%
\pgfpathlineto{\pgfqpoint{3.046000in}{2.985706in}}%
\pgfpathlineto{\pgfqpoint{3.077000in}{2.856507in}}%
\pgfpathlineto{\pgfqpoint{3.108000in}{2.878823in}}%
\pgfpathlineto{\pgfqpoint{3.139000in}{2.799117in}}%
\pgfpathlineto{\pgfqpoint{3.170000in}{2.535029in}}%
\pgfpathlineto{\pgfqpoint{3.201000in}{2.421300in}}%
\pgfpathlineto{\pgfqpoint{3.232000in}{2.147946in}}%
\pgfpathlineto{\pgfqpoint{3.263000in}{2.003849in}}%
\pgfpathlineto{\pgfqpoint{3.294000in}{1.846635in}}%
\pgfpathlineto{\pgfqpoint{3.325000in}{1.627290in}}%
\pgfpathlineto{\pgfqpoint{3.356000in}{1.749510in}}%
\pgfpathlineto{\pgfqpoint{3.387000in}{1.764224in}}%
\pgfpathlineto{\pgfqpoint{3.418000in}{1.534757in}}%
\pgfpathlineto{\pgfqpoint{3.449000in}{1.332333in}}%
\pgfpathlineto{\pgfqpoint{3.480000in}{1.285719in}}%
\pgfpathlineto{\pgfqpoint{3.511000in}{1.499218in}}%
\pgfpathlineto{\pgfqpoint{3.542000in}{1.268007in}}%
\pgfpathlineto{\pgfqpoint{3.573000in}{1.245758in}}%
\pgfpathlineto{\pgfqpoint{3.604000in}{1.372943in}}%
\pgfpathlineto{\pgfqpoint{3.635000in}{1.203897in}}%
\pgfpathlineto{\pgfqpoint{3.666000in}{1.160160in}}%
\pgfpathlineto{\pgfqpoint{3.697000in}{1.228312in}}%
\pgfpathlineto{\pgfqpoint{3.728000in}{1.153614in}}%
\pgfpathlineto{\pgfqpoint{3.759000in}{1.096291in}}%
\pgfpathlineto{\pgfqpoint{3.790000in}{1.362082in}}%
\pgfpathlineto{\pgfqpoint{3.821000in}{1.460276in}}%
\pgfpathlineto{\pgfqpoint{3.852000in}{1.938652in}}%
\pgfpathlineto{\pgfqpoint{3.883000in}{2.159897in}}%
\pgfpathlineto{\pgfqpoint{3.914000in}{2.624094in}}%
\pgfpathlineto{\pgfqpoint{3.945000in}{2.583290in}}%
\pgfpathlineto{\pgfqpoint{3.976000in}{2.874852in}}%
\pgfpathlineto{\pgfqpoint{4.007000in}{2.744809in}}%
\pgfpathlineto{\pgfqpoint{4.038000in}{2.390259in}}%
\pgfpathlineto{\pgfqpoint{4.069000in}{2.389022in}}%
\pgfpathlineto{\pgfqpoint{4.100000in}{2.263010in}}%
\pgfpathlineto{\pgfqpoint{4.131000in}{2.048081in}}%
\pgfpathlineto{\pgfqpoint{4.162000in}{1.868292in}}%
\pgfpathlineto{\pgfqpoint{4.193000in}{1.806188in}}%
\pgfpathlineto{\pgfqpoint{4.224000in}{1.851364in}}%
\pgfpathlineto{\pgfqpoint{4.255000in}{1.786123in}}%
\pgfpathlineto{\pgfqpoint{4.286000in}{1.478938in}}%
\pgfpathlineto{\pgfqpoint{4.317000in}{1.447772in}}%
\pgfpathlineto{\pgfqpoint{4.348000in}{1.390926in}}%
\pgfpathlineto{\pgfqpoint{4.379000in}{1.309380in}}%
\pgfpathlineto{\pgfqpoint{4.410000in}{1.195567in}}%
\pgfpathlineto{\pgfqpoint{4.441000in}{1.370608in}}%
\pgfpathlineto{\pgfqpoint{4.472000in}{1.264745in}}%
\pgfpathlineto{\pgfqpoint{4.534000in}{1.138754in}}%
\pgfpathlineto{\pgfqpoint{4.565000in}{1.071875in}}%
\pgfpathlineto{\pgfqpoint{4.596000in}{1.291233in}}%
\pgfpathlineto{\pgfqpoint{4.627000in}{1.305829in}}%
\pgfpathlineto{\pgfqpoint{4.658000in}{1.126780in}}%
\pgfpathlineto{\pgfqpoint{4.689000in}{1.251121in}}%
\pgfpathlineto{\pgfqpoint{4.720000in}{0.985859in}}%
\pgfpathlineto{\pgfqpoint{4.751000in}{1.130905in}}%
\pgfpathlineto{\pgfqpoint{4.782000in}{1.154206in}}%
\pgfpathlineto{\pgfqpoint{4.813000in}{1.029362in}}%
\pgfpathlineto{\pgfqpoint{4.844000in}{1.088241in}}%
\pgfpathlineto{\pgfqpoint{4.875000in}{0.859991in}}%
\pgfpathlineto{\pgfqpoint{4.906000in}{1.031532in}}%
\pgfpathlineto{\pgfqpoint{4.937000in}{1.057437in}}%
\pgfpathlineto{\pgfqpoint{4.968000in}{0.957503in}}%
\pgfpathlineto{\pgfqpoint{4.999000in}{1.310615in}}%
\pgfpathlineto{\pgfqpoint{5.030000in}{1.193850in}}%
\pgfpathlineto{\pgfqpoint{5.061000in}{1.131802in}}%
\pgfpathlineto{\pgfqpoint{5.092000in}{1.212397in}}%
\pgfpathlineto{\pgfqpoint{5.123000in}{0.991895in}}%
\pgfpathlineto{\pgfqpoint{5.154000in}{1.079071in}}%
\pgfpathlineto{\pgfqpoint{5.185000in}{1.028240in}}%
\pgfpathlineto{\pgfqpoint{5.216000in}{0.953388in}}%
\pgfpathlineto{\pgfqpoint{5.247000in}{1.128029in}}%
\pgfpathlineto{\pgfqpoint{5.278000in}{1.150710in}}%
\pgfpathlineto{\pgfqpoint{5.309000in}{1.225236in}}%
\pgfpathlineto{\pgfqpoint{5.371000in}{1.115869in}}%
\pgfpathlineto{\pgfqpoint{5.402000in}{1.190136in}}%
\pgfpathlineto{\pgfqpoint{5.433000in}{1.213951in}}%
\pgfpathlineto{\pgfqpoint{5.464000in}{1.076276in}}%
\pgfpathlineto{\pgfqpoint{5.495000in}{1.046981in}}%
\pgfpathlineto{\pgfqpoint{5.526000in}{1.077426in}}%
\pgfpathlineto{\pgfqpoint{5.557000in}{1.109575in}}%
\pgfpathlineto{\pgfqpoint{5.588000in}{1.039376in}}%
\pgfpathlineto{\pgfqpoint{5.619000in}{0.942197in}}%
\pgfpathlineto{\pgfqpoint{5.650000in}{1.085201in}}%
\pgfpathlineto{\pgfqpoint{5.681000in}{1.089287in}}%
\pgfpathlineto{\pgfqpoint{5.712000in}{1.113369in}}%
\pgfpathlineto{\pgfqpoint{5.743000in}{1.156655in}}%
\pgfpathlineto{\pgfqpoint{5.774000in}{1.280040in}}%
\pgfpathlineto{\pgfqpoint{5.805000in}{1.127104in}}%
\pgfpathlineto{\pgfqpoint{5.836000in}{1.063476in}}%
\pgfpathlineto{\pgfqpoint{5.867000in}{1.091063in}}%
\pgfpathlineto{\pgfqpoint{5.898000in}{1.250376in}}%
\pgfpathlineto{\pgfqpoint{5.929000in}{1.240144in}}%
\pgfpathlineto{\pgfqpoint{5.960000in}{1.082895in}}%
\pgfpathlineto{\pgfqpoint{5.991000in}{1.050934in}}%
\pgfpathlineto{\pgfqpoint{6.022000in}{1.106966in}}%
\pgfpathlineto{\pgfqpoint{6.053000in}{1.030323in}}%
\pgfpathlineto{\pgfqpoint{6.084000in}{1.054613in}}%
\pgfpathlineto{\pgfqpoint{6.115000in}{1.117924in}}%
\pgfpathlineto{\pgfqpoint{6.146000in}{1.153989in}}%
\pgfpathlineto{\pgfqpoint{6.177000in}{1.100864in}}%
\pgfpathlineto{\pgfqpoint{6.208000in}{1.194759in}}%
\pgfpathlineto{\pgfqpoint{6.239000in}{1.047192in}}%
\pgfpathlineto{\pgfqpoint{6.270000in}{1.221469in}}%
\pgfpathlineto{\pgfqpoint{6.301000in}{1.008265in}}%
\pgfpathlineto{\pgfqpoint{6.332000in}{1.070219in}}%
\pgfpathlineto{\pgfqpoint{6.363000in}{1.089565in}}%
\pgfpathlineto{\pgfqpoint{6.394000in}{1.083343in}}%
\pgfpathlineto{\pgfqpoint{6.425000in}{1.286801in}}%
\pgfpathlineto{\pgfqpoint{6.456000in}{1.152576in}}%
\pgfpathlineto{\pgfqpoint{6.487000in}{1.235259in}}%
\pgfpathlineto{\pgfqpoint{6.518000in}{1.187188in}}%
\pgfpathlineto{\pgfqpoint{6.549000in}{1.194836in}}%
\pgfpathlineto{\pgfqpoint{6.580000in}{1.147427in}}%
\pgfpathlineto{\pgfqpoint{6.611000in}{1.055045in}}%
\pgfpathlineto{\pgfqpoint{6.642000in}{1.197860in}}%
\pgfpathlineto{\pgfqpoint{6.673000in}{1.101277in}}%
\pgfpathlineto{\pgfqpoint{6.704000in}{0.975225in}}%
\pgfpathlineto{\pgfqpoint{6.735000in}{1.134524in}}%
\pgfpathlineto{\pgfqpoint{6.766000in}{1.227563in}}%
\pgfpathlineto{\pgfqpoint{6.797000in}{1.197735in}}%
\pgfpathlineto{\pgfqpoint{6.828000in}{1.188052in}}%
\pgfpathlineto{\pgfqpoint{6.859000in}{0.965679in}}%
\pgfpathlineto{\pgfqpoint{6.890000in}{1.148094in}}%
\pgfpathlineto{\pgfqpoint{6.921000in}{0.906923in}}%
\pgfpathlineto{\pgfqpoint{6.952000in}{1.148070in}}%
\pgfpathlineto{\pgfqpoint{6.983000in}{1.018517in}}%
\pgfpathlineto{\pgfqpoint{7.014000in}{1.126950in}}%
\pgfpathlineto{\pgfqpoint{7.045000in}{1.154565in}}%
\pgfpathlineto{\pgfqpoint{7.076000in}{1.066071in}}%
\pgfpathlineto{\pgfqpoint{7.107000in}{1.054039in}}%
\pgfpathlineto{\pgfqpoint{7.138000in}{1.200725in}}%
\pgfpathlineto{\pgfqpoint{7.169000in}{1.312911in}}%
\pgfpathlineto{\pgfqpoint{7.200000in}{1.199615in}}%
\pgfpathlineto{\pgfqpoint{7.210000in}{1.133142in}}%
\pgfpathlineto{\pgfqpoint{7.210000in}{1.133142in}}%
\pgfusepath{stroke}%
\end{pgfscope}%
\begin{pgfscope}%
\pgfpathrectangle{\pgfqpoint{1.000000in}{0.720000in}}{\pgfqpoint{6.200000in}{4.620000in}}%
\pgfusepath{clip}%
\pgfsetbuttcap%
\pgfsetroundjoin%
\pgfsetlinewidth{2.007500pt}%
\definecolor{currentstroke}{rgb}{0.000000,0.500000,0.000000}%
\pgfsetstrokecolor{currentstroke}%
\pgfsetdash{}{0pt}%
\pgfpathmoveto{\pgfqpoint{0.990000in}{1.617779in}}%
\pgfpathlineto{\pgfqpoint{7.210000in}{1.617779in}}%
\pgfusepath{stroke}%
\end{pgfscope}%
\begin{pgfscope}%
\pgfsetrectcap%
\pgfsetmiterjoin%
\pgfsetlinewidth{0.803000pt}%
\definecolor{currentstroke}{rgb}{0.000000,0.000000,0.000000}%
\pgfsetstrokecolor{currentstroke}%
\pgfsetdash{}{0pt}%
\pgfpathmoveto{\pgfqpoint{1.000000in}{0.720000in}}%
\pgfpathlineto{\pgfqpoint{1.000000in}{5.340000in}}%
\pgfusepath{stroke}%
\end{pgfscope}%
\begin{pgfscope}%
\pgfsetrectcap%
\pgfsetmiterjoin%
\pgfsetlinewidth{0.803000pt}%
\definecolor{currentstroke}{rgb}{0.000000,0.000000,0.000000}%
\pgfsetstrokecolor{currentstroke}%
\pgfsetdash{}{0pt}%
\pgfpathmoveto{\pgfqpoint{7.200000in}{0.720000in}}%
\pgfpathlineto{\pgfqpoint{7.200000in}{5.340000in}}%
\pgfusepath{stroke}%
\end{pgfscope}%
\begin{pgfscope}%
\pgfsetrectcap%
\pgfsetmiterjoin%
\pgfsetlinewidth{0.803000pt}%
\definecolor{currentstroke}{rgb}{0.000000,0.000000,0.000000}%
\pgfsetstrokecolor{currentstroke}%
\pgfsetdash{}{0pt}%
\pgfpathmoveto{\pgfqpoint{1.000000in}{0.720000in}}%
\pgfpathlineto{\pgfqpoint{7.200000in}{0.720000in}}%
\pgfusepath{stroke}%
\end{pgfscope}%
\begin{pgfscope}%
\pgfsetrectcap%
\pgfsetmiterjoin%
\pgfsetlinewidth{0.803000pt}%
\definecolor{currentstroke}{rgb}{0.000000,0.000000,0.000000}%
\pgfsetstrokecolor{currentstroke}%
\pgfsetdash{}{0pt}%
\pgfpathmoveto{\pgfqpoint{1.000000in}{5.340000in}}%
\pgfpathlineto{\pgfqpoint{7.200000in}{5.340000in}}%
\pgfusepath{stroke}%
\end{pgfscope}%
\begin{pgfscope}%
\pgfsetroundcap%
\pgfsetroundjoin%
\definecolor{currentfill}{rgb}{0.000000,0.000000,0.000000}%
\pgfsetfillcolor{currentfill}%
\pgfsetlinewidth{1.003750pt}%
\definecolor{currentstroke}{rgb}{0.000000,0.000000,0.000000}%
\pgfsetstrokecolor{currentstroke}%
\pgfsetdash{}{0pt}%
\pgfpathmoveto{\pgfqpoint{2.243472in}{4.851698in}}%
\pgfpathquadraticcurveto{\pgfqpoint{2.243472in}{4.464286in}}{\pgfqpoint{2.243472in}{4.076874in}}%
\pgfpathlineto{\pgfqpoint{2.253889in}{4.076874in}}%
\pgfpathquadraticcurveto{\pgfqpoint{2.246944in}{3.993536in}}{\pgfqpoint{2.240000in}{3.910197in}}%
\pgfpathquadraticcurveto{\pgfqpoint{2.233056in}{3.993536in}}{\pgfqpoint{2.226111in}{4.076874in}}%
\pgfpathlineto{\pgfqpoint{2.236528in}{4.076874in}}%
\pgfpathquadraticcurveto{\pgfqpoint{2.236528in}{4.464286in}}{\pgfqpoint{2.236528in}{4.851698in}}%
\pgfpathlineto{\pgfqpoint{2.243472in}{4.851698in}}%
\pgfpathlineto{\pgfqpoint{2.243472in}{4.851698in}}%
\pgfpathclose%
\pgfusepath{stroke,fill}%
\end{pgfscope}%
\begin{pgfscope}%
\pgfsetroundcap%
\pgfsetroundjoin%
\definecolor{currentfill}{rgb}{0.000000,0.000000,0.000000}%
\pgfsetfillcolor{currentfill}%
\pgfsetlinewidth{1.003750pt}%
\definecolor{currentstroke}{rgb}{0.000000,0.000000,0.000000}%
\pgfsetstrokecolor{currentstroke}%
\pgfsetdash{}{0pt}%
\pgfpathmoveto{\pgfqpoint{2.770472in}{4.427619in}}%
\pgfpathquadraticcurveto{\pgfqpoint{2.770472in}{4.040207in}}{\pgfqpoint{2.770472in}{3.652794in}}%
\pgfpathlineto{\pgfqpoint{2.780889in}{3.652794in}}%
\pgfpathquadraticcurveto{\pgfqpoint{2.773944in}{3.569456in}}{\pgfqpoint{2.767000in}{3.486118in}}%
\pgfpathquadraticcurveto{\pgfqpoint{2.760056in}{3.569456in}}{\pgfqpoint{2.753111in}{3.652794in}}%
\pgfpathlineto{\pgfqpoint{2.763528in}{3.652794in}}%
\pgfpathquadraticcurveto{\pgfqpoint{2.763528in}{4.040207in}}{\pgfqpoint{2.763528in}{4.427619in}}%
\pgfpathlineto{\pgfqpoint{2.770472in}{4.427619in}}%
\pgfpathlineto{\pgfqpoint{2.770472in}{4.427619in}}%
\pgfpathclose%
\pgfusepath{stroke,fill}%
\end{pgfscope}%
\begin{pgfscope}%
\pgfsetroundcap%
\pgfsetroundjoin%
\definecolor{currentfill}{rgb}{0.000000,0.000000,0.000000}%
\pgfsetfillcolor{currentfill}%
\pgfsetlinewidth{1.003750pt}%
\definecolor{currentstroke}{rgb}{0.000000,0.000000,0.000000}%
\pgfsetstrokecolor{currentstroke}%
\pgfsetdash{}{0pt}%
\pgfpathmoveto{\pgfqpoint{3.111472in}{4.310279in}}%
\pgfpathquadraticcurveto{\pgfqpoint{3.111472in}{3.922867in}}{\pgfqpoint{3.111472in}{3.535455in}}%
\pgfpathlineto{\pgfqpoint{3.121889in}{3.535455in}}%
\pgfpathquadraticcurveto{\pgfqpoint{3.114944in}{3.452117in}}{\pgfqpoint{3.108000in}{3.368778in}}%
\pgfpathquadraticcurveto{\pgfqpoint{3.101056in}{3.452117in}}{\pgfqpoint{3.094111in}{3.535455in}}%
\pgfpathlineto{\pgfqpoint{3.104528in}{3.535455in}}%
\pgfpathquadraticcurveto{\pgfqpoint{3.104528in}{3.922867in}}{\pgfqpoint{3.104528in}{4.310279in}}%
\pgfpathlineto{\pgfqpoint{3.111472in}{4.310279in}}%
\pgfpathlineto{\pgfqpoint{3.111472in}{4.310279in}}%
\pgfpathclose%
\pgfusepath{stroke,fill}%
\end{pgfscope}%
\begin{pgfscope}%
\pgfsetroundcap%
\pgfsetroundjoin%
\definecolor{currentfill}{rgb}{0.000000,0.000000,0.000000}%
\pgfsetfillcolor{currentfill}%
\pgfsetlinewidth{1.003750pt}%
\definecolor{currentstroke}{rgb}{0.000000,0.000000,0.000000}%
\pgfsetstrokecolor{currentstroke}%
\pgfsetdash{}{0pt}%
\pgfpathmoveto{\pgfqpoint{4.010472in}{4.176266in}}%
\pgfpathquadraticcurveto{\pgfqpoint{4.010472in}{3.788853in}}{\pgfqpoint{4.010472in}{3.401441in}}%
\pgfpathlineto{\pgfqpoint{4.020889in}{3.401441in}}%
\pgfpathquadraticcurveto{\pgfqpoint{4.013944in}{3.318103in}}{\pgfqpoint{4.007000in}{3.234765in}}%
\pgfpathquadraticcurveto{\pgfqpoint{4.000056in}{3.318103in}}{\pgfqpoint{3.993111in}{3.401441in}}%
\pgfpathlineto{\pgfqpoint{4.003528in}{3.401441in}}%
\pgfpathquadraticcurveto{\pgfqpoint{4.003528in}{3.788853in}}{\pgfqpoint{4.003528in}{4.176266in}}%
\pgfpathlineto{\pgfqpoint{4.010472in}{4.176266in}}%
\pgfpathlineto{\pgfqpoint{4.010472in}{4.176266in}}%
\pgfpathclose%
\pgfusepath{stroke,fill}%
\end{pgfscope}%
\begin{pgfscope}%
\pgfsetbuttcap%
\pgfsetroundjoin%
\definecolor{currentfill}{rgb}{0.000000,0.000000,0.000000}%
\pgfsetfillcolor{currentfill}%
\pgfsetlinewidth{0.803000pt}%
\definecolor{currentstroke}{rgb}{0.000000,0.000000,0.000000}%
\pgfsetstrokecolor{currentstroke}%
\pgfsetdash{}{0pt}%
\pgfsys@defobject{currentmarker}{\pgfqpoint{0.000000in}{0.000000in}}{\pgfqpoint{0.048611in}{0.000000in}}{%
\pgfpathmoveto{\pgfqpoint{0.000000in}{0.000000in}}%
\pgfpathlineto{\pgfqpoint{0.048611in}{0.000000in}}%
\pgfusepath{stroke,fill}%
}%
\begin{pgfscope}%
\pgfsys@transformshift{7.200000in}{1.137431in}%
\pgfsys@useobject{currentmarker}{}%
\end{pgfscope}%
\end{pgfscope}%
\begin{pgfscope}%
\definecolor{textcolor}{rgb}{0.000000,0.000000,0.000000}%
\pgfsetstrokecolor{textcolor}%
\pgfsetfillcolor{textcolor}%
\pgftext[x=7.297222in, y=1.037412in, left, base]{\color{textcolor}\sffamily\fontsize{20.000000}{24.000000}\selectfont 0.0}%
\end{pgfscope}%
\begin{pgfscope}%
\pgfsetbuttcap%
\pgfsetroundjoin%
\definecolor{currentfill}{rgb}{0.000000,0.000000,0.000000}%
\pgfsetfillcolor{currentfill}%
\pgfsetlinewidth{0.803000pt}%
\definecolor{currentstroke}{rgb}{0.000000,0.000000,0.000000}%
\pgfsetstrokecolor{currentstroke}%
\pgfsetdash{}{0pt}%
\pgfsys@defobject{currentmarker}{\pgfqpoint{0.000000in}{0.000000in}}{\pgfqpoint{0.048611in}{0.000000in}}{%
\pgfpathmoveto{\pgfqpoint{0.000000in}{0.000000in}}%
\pgfpathlineto{\pgfqpoint{0.048611in}{0.000000in}}%
\pgfusepath{stroke,fill}%
}%
\begin{pgfscope}%
\pgfsys@transformshift{7.200000in}{1.755456in}%
\pgfsys@useobject{currentmarker}{}%
\end{pgfscope}%
\end{pgfscope}%
\begin{pgfscope}%
\definecolor{textcolor}{rgb}{0.000000,0.000000,0.000000}%
\pgfsetstrokecolor{textcolor}%
\pgfsetfillcolor{textcolor}%
\pgftext[x=7.297222in, y=1.655436in, left, base]{\color{textcolor}\sffamily\fontsize{20.000000}{24.000000}\selectfont 0.2}%
\end{pgfscope}%
\begin{pgfscope}%
\pgfsetbuttcap%
\pgfsetroundjoin%
\definecolor{currentfill}{rgb}{0.000000,0.000000,0.000000}%
\pgfsetfillcolor{currentfill}%
\pgfsetlinewidth{0.803000pt}%
\definecolor{currentstroke}{rgb}{0.000000,0.000000,0.000000}%
\pgfsetstrokecolor{currentstroke}%
\pgfsetdash{}{0pt}%
\pgfsys@defobject{currentmarker}{\pgfqpoint{0.000000in}{0.000000in}}{\pgfqpoint{0.048611in}{0.000000in}}{%
\pgfpathmoveto{\pgfqpoint{0.000000in}{0.000000in}}%
\pgfpathlineto{\pgfqpoint{0.048611in}{0.000000in}}%
\pgfusepath{stroke,fill}%
}%
\begin{pgfscope}%
\pgfsys@transformshift{7.200000in}{2.373481in}%
\pgfsys@useobject{currentmarker}{}%
\end{pgfscope}%
\end{pgfscope}%
\begin{pgfscope}%
\definecolor{textcolor}{rgb}{0.000000,0.000000,0.000000}%
\pgfsetstrokecolor{textcolor}%
\pgfsetfillcolor{textcolor}%
\pgftext[x=7.297222in, y=2.273461in, left, base]{\color{textcolor}\sffamily\fontsize{20.000000}{24.000000}\selectfont 0.5}%
\end{pgfscope}%
\begin{pgfscope}%
\pgfsetbuttcap%
\pgfsetroundjoin%
\definecolor{currentfill}{rgb}{0.000000,0.000000,0.000000}%
\pgfsetfillcolor{currentfill}%
\pgfsetlinewidth{0.803000pt}%
\definecolor{currentstroke}{rgb}{0.000000,0.000000,0.000000}%
\pgfsetstrokecolor{currentstroke}%
\pgfsetdash{}{0pt}%
\pgfsys@defobject{currentmarker}{\pgfqpoint{0.000000in}{0.000000in}}{\pgfqpoint{0.048611in}{0.000000in}}{%
\pgfpathmoveto{\pgfqpoint{0.000000in}{0.000000in}}%
\pgfpathlineto{\pgfqpoint{0.048611in}{0.000000in}}%
\pgfusepath{stroke,fill}%
}%
\begin{pgfscope}%
\pgfsys@transformshift{7.200000in}{2.991505in}%
\pgfsys@useobject{currentmarker}{}%
\end{pgfscope}%
\end{pgfscope}%
\begin{pgfscope}%
\definecolor{textcolor}{rgb}{0.000000,0.000000,0.000000}%
\pgfsetstrokecolor{textcolor}%
\pgfsetfillcolor{textcolor}%
\pgftext[x=7.297222in, y=2.891486in, left, base]{\color{textcolor}\sffamily\fontsize{20.000000}{24.000000}\selectfont 0.8}%
\end{pgfscope}%
\begin{pgfscope}%
\pgfsetbuttcap%
\pgfsetroundjoin%
\definecolor{currentfill}{rgb}{0.000000,0.000000,0.000000}%
\pgfsetfillcolor{currentfill}%
\pgfsetlinewidth{0.803000pt}%
\definecolor{currentstroke}{rgb}{0.000000,0.000000,0.000000}%
\pgfsetstrokecolor{currentstroke}%
\pgfsetdash{}{0pt}%
\pgfsys@defobject{currentmarker}{\pgfqpoint{0.000000in}{0.000000in}}{\pgfqpoint{0.048611in}{0.000000in}}{%
\pgfpathmoveto{\pgfqpoint{0.000000in}{0.000000in}}%
\pgfpathlineto{\pgfqpoint{0.048611in}{0.000000in}}%
\pgfusepath{stroke,fill}%
}%
\begin{pgfscope}%
\pgfsys@transformshift{7.200000in}{3.609530in}%
\pgfsys@useobject{currentmarker}{}%
\end{pgfscope}%
\end{pgfscope}%
\begin{pgfscope}%
\definecolor{textcolor}{rgb}{0.000000,0.000000,0.000000}%
\pgfsetstrokecolor{textcolor}%
\pgfsetfillcolor{textcolor}%
\pgftext[x=7.297222in, y=3.509511in, left, base]{\color{textcolor}\sffamily\fontsize{20.000000}{24.000000}\selectfont 1.0}%
\end{pgfscope}%
\begin{pgfscope}%
\pgfsetbuttcap%
\pgfsetroundjoin%
\definecolor{currentfill}{rgb}{0.000000,0.000000,0.000000}%
\pgfsetfillcolor{currentfill}%
\pgfsetlinewidth{0.803000pt}%
\definecolor{currentstroke}{rgb}{0.000000,0.000000,0.000000}%
\pgfsetstrokecolor{currentstroke}%
\pgfsetdash{}{0pt}%
\pgfsys@defobject{currentmarker}{\pgfqpoint{0.000000in}{0.000000in}}{\pgfqpoint{0.048611in}{0.000000in}}{%
\pgfpathmoveto{\pgfqpoint{0.000000in}{0.000000in}}%
\pgfpathlineto{\pgfqpoint{0.048611in}{0.000000in}}%
\pgfusepath{stroke,fill}%
}%
\begin{pgfscope}%
\pgfsys@transformshift{7.200000in}{4.227555in}%
\pgfsys@useobject{currentmarker}{}%
\end{pgfscope}%
\end{pgfscope}%
\begin{pgfscope}%
\definecolor{textcolor}{rgb}{0.000000,0.000000,0.000000}%
\pgfsetstrokecolor{textcolor}%
\pgfsetfillcolor{textcolor}%
\pgftext[x=7.297222in, y=4.127536in, left, base]{\color{textcolor}\sffamily\fontsize{20.000000}{24.000000}\selectfont 1.2}%
\end{pgfscope}%
\begin{pgfscope}%
\pgfsetbuttcap%
\pgfsetroundjoin%
\definecolor{currentfill}{rgb}{0.000000,0.000000,0.000000}%
\pgfsetfillcolor{currentfill}%
\pgfsetlinewidth{0.803000pt}%
\definecolor{currentstroke}{rgb}{0.000000,0.000000,0.000000}%
\pgfsetstrokecolor{currentstroke}%
\pgfsetdash{}{0pt}%
\pgfsys@defobject{currentmarker}{\pgfqpoint{0.000000in}{0.000000in}}{\pgfqpoint{0.048611in}{0.000000in}}{%
\pgfpathmoveto{\pgfqpoint{0.000000in}{0.000000in}}%
\pgfpathlineto{\pgfqpoint{0.048611in}{0.000000in}}%
\pgfusepath{stroke,fill}%
}%
\begin{pgfscope}%
\pgfsys@transformshift{7.200000in}{4.845580in}%
\pgfsys@useobject{currentmarker}{}%
\end{pgfscope}%
\end{pgfscope}%
\begin{pgfscope}%
\definecolor{textcolor}{rgb}{0.000000,0.000000,0.000000}%
\pgfsetstrokecolor{textcolor}%
\pgfsetfillcolor{textcolor}%
\pgftext[x=7.297222in, y=4.745561in, left, base]{\color{textcolor}\sffamily\fontsize{20.000000}{24.000000}\selectfont 1.5}%
\end{pgfscope}%
\begin{pgfscope}%
\definecolor{textcolor}{rgb}{0.000000,0.000000,0.000000}%
\pgfsetstrokecolor{textcolor}%
\pgfsetfillcolor{textcolor}%
\pgftext[x=7.698906in,y=3.030000in,,top,rotate=90.000000]{\color{textcolor}\sffamily\fontsize{20.000000}{24.000000}\selectfont \(\displaystyle \mathrm{Charge}\)}%
\end{pgfscope}%
\begin{pgfscope}%
\pgfpathrectangle{\pgfqpoint{1.000000in}{0.720000in}}{\pgfqpoint{6.200000in}{4.620000in}}%
\pgfusepath{clip}%
\pgfsetbuttcap%
\pgfsetroundjoin%
\pgfsetlinewidth{1.505625pt}%
\definecolor{currentstroke}{rgb}{1.000000,0.000000,0.000000}%
\pgfsetstrokecolor{currentstroke}%
\pgfsetdash{}{0pt}%
\pgfpathmoveto{\pgfqpoint{1.992000in}{1.137431in}}%
\pgfpathlineto{\pgfqpoint{1.992000in}{5.113449in}}%
\pgfusepath{stroke}%
\end{pgfscope}%
\begin{pgfscope}%
\pgfpathrectangle{\pgfqpoint{1.000000in}{0.720000in}}{\pgfqpoint{6.200000in}{4.620000in}}%
\pgfusepath{clip}%
\pgfsetbuttcap%
\pgfsetroundjoin%
\pgfsetlinewidth{1.505625pt}%
\definecolor{currentstroke}{rgb}{1.000000,0.000000,0.000000}%
\pgfsetstrokecolor{currentstroke}%
\pgfsetdash{}{0pt}%
\pgfpathmoveto{\pgfqpoint{2.519000in}{1.137431in}}%
\pgfpathlineto{\pgfqpoint{2.519000in}{4.374821in}}%
\pgfusepath{stroke}%
\end{pgfscope}%
\begin{pgfscope}%
\pgfpathrectangle{\pgfqpoint{1.000000in}{0.720000in}}{\pgfqpoint{6.200000in}{4.620000in}}%
\pgfusepath{clip}%
\pgfsetbuttcap%
\pgfsetroundjoin%
\pgfsetlinewidth{1.505625pt}%
\definecolor{currentstroke}{rgb}{1.000000,0.000000,0.000000}%
\pgfsetstrokecolor{currentstroke}%
\pgfsetdash{}{0pt}%
\pgfpathmoveto{\pgfqpoint{2.860000in}{1.137431in}}%
\pgfpathlineto{\pgfqpoint{2.860000in}{4.170448in}}%
\pgfusepath{stroke}%
\end{pgfscope}%
\begin{pgfscope}%
\pgfpathrectangle{\pgfqpoint{1.000000in}{0.720000in}}{\pgfqpoint{6.200000in}{4.620000in}}%
\pgfusepath{clip}%
\pgfsetbuttcap%
\pgfsetroundjoin%
\pgfsetlinewidth{1.505625pt}%
\definecolor{currentstroke}{rgb}{1.000000,0.000000,0.000000}%
\pgfsetstrokecolor{currentstroke}%
\pgfsetdash{}{0pt}%
\pgfpathmoveto{\pgfqpoint{3.759000in}{1.137431in}}%
\pgfpathlineto{\pgfqpoint{3.759000in}{3.937034in}}%
\pgfusepath{stroke}%
\end{pgfscope}%
\begin{pgfscope}%
\pgfsetrectcap%
\pgfsetmiterjoin%
\pgfsetlinewidth{0.803000pt}%
\definecolor{currentstroke}{rgb}{0.000000,0.000000,0.000000}%
\pgfsetstrokecolor{currentstroke}%
\pgfsetdash{}{0pt}%
\pgfpathmoveto{\pgfqpoint{1.000000in}{0.720000in}}%
\pgfpathlineto{\pgfqpoint{1.000000in}{5.340000in}}%
\pgfusepath{stroke}%
\end{pgfscope}%
\begin{pgfscope}%
\pgfsetrectcap%
\pgfsetmiterjoin%
\pgfsetlinewidth{0.803000pt}%
\definecolor{currentstroke}{rgb}{0.000000,0.000000,0.000000}%
\pgfsetstrokecolor{currentstroke}%
\pgfsetdash{}{0pt}%
\pgfpathmoveto{\pgfqpoint{7.200000in}{0.720000in}}%
\pgfpathlineto{\pgfqpoint{7.200000in}{5.340000in}}%
\pgfusepath{stroke}%
\end{pgfscope}%
\begin{pgfscope}%
\pgfsetrectcap%
\pgfsetmiterjoin%
\pgfsetlinewidth{0.803000pt}%
\definecolor{currentstroke}{rgb}{0.000000,0.000000,0.000000}%
\pgfsetstrokecolor{currentstroke}%
\pgfsetdash{}{0pt}%
\pgfpathmoveto{\pgfqpoint{1.000000in}{0.720000in}}%
\pgfpathlineto{\pgfqpoint{7.200000in}{0.720000in}}%
\pgfusepath{stroke}%
\end{pgfscope}%
\begin{pgfscope}%
\pgfsetrectcap%
\pgfsetmiterjoin%
\pgfsetlinewidth{0.803000pt}%
\definecolor{currentstroke}{rgb}{0.000000,0.000000,0.000000}%
\pgfsetstrokecolor{currentstroke}%
\pgfsetdash{}{0pt}%
\pgfpathmoveto{\pgfqpoint{1.000000in}{5.340000in}}%
\pgfpathlineto{\pgfqpoint{7.200000in}{5.340000in}}%
\pgfusepath{stroke}%
\end{pgfscope}%
\begin{pgfscope}%
\pgfsetbuttcap%
\pgfsetmiterjoin%
\definecolor{currentfill}{rgb}{1.000000,1.000000,1.000000}%
\pgfsetfillcolor{currentfill}%
\pgfsetfillopacity{0.800000}%
\pgfsetlinewidth{1.003750pt}%
\definecolor{currentstroke}{rgb}{0.800000,0.800000,0.800000}%
\pgfsetstrokecolor{currentstroke}%
\pgfsetstrokeopacity{0.800000}%
\pgfsetdash{}{0pt}%
\pgfpathmoveto{\pgfqpoint{4.976872in}{3.932908in}}%
\pgfpathlineto{\pgfqpoint{7.005556in}{3.932908in}}%
\pgfpathquadraticcurveto{\pgfqpoint{7.061111in}{3.932908in}}{\pgfqpoint{7.061111in}{3.988464in}}%
\pgfpathlineto{\pgfqpoint{7.061111in}{5.145556in}}%
\pgfpathquadraticcurveto{\pgfqpoint{7.061111in}{5.201111in}}{\pgfqpoint{7.005556in}{5.201111in}}%
\pgfpathlineto{\pgfqpoint{4.976872in}{5.201111in}}%
\pgfpathquadraticcurveto{\pgfqpoint{4.921317in}{5.201111in}}{\pgfqpoint{4.921317in}{5.145556in}}%
\pgfpathlineto{\pgfqpoint{4.921317in}{3.988464in}}%
\pgfpathquadraticcurveto{\pgfqpoint{4.921317in}{3.932908in}}{\pgfqpoint{4.976872in}{3.932908in}}%
\pgfpathlineto{\pgfqpoint{4.976872in}{3.932908in}}%
\pgfpathclose%
\pgfusepath{stroke,fill}%
\end{pgfscope}%
\begin{pgfscope}%
\pgfsetrectcap%
\pgfsetroundjoin%
\pgfsetlinewidth{2.007500pt}%
\definecolor{currentstroke}{rgb}{0.121569,0.466667,0.705882}%
\pgfsetstrokecolor{currentstroke}%
\pgfsetdash{}{0pt}%
\pgfpathmoveto{\pgfqpoint{5.032428in}{4.987184in}}%
\pgfpathlineto{\pgfqpoint{5.310206in}{4.987184in}}%
\pgfpathlineto{\pgfqpoint{5.587983in}{4.987184in}}%
\pgfusepath{stroke}%
\end{pgfscope}%
\begin{pgfscope}%
\definecolor{textcolor}{rgb}{0.000000,0.000000,0.000000}%
\pgfsetstrokecolor{textcolor}%
\pgfsetfillcolor{textcolor}%
\pgftext[x=5.810206in,y=4.889962in,left,base]{\color{textcolor}\sffamily\fontsize{20.000000}{24.000000}\selectfont Waveform}%
\end{pgfscope}%
\begin{pgfscope}%
\pgfsetbuttcap%
\pgfsetroundjoin%
\pgfsetlinewidth{2.007500pt}%
\definecolor{currentstroke}{rgb}{0.000000,0.500000,0.000000}%
\pgfsetstrokecolor{currentstroke}%
\pgfsetdash{}{0pt}%
\pgfpathmoveto{\pgfqpoint{5.032428in}{4.592227in}}%
\pgfpathlineto{\pgfqpoint{5.587983in}{4.592227in}}%
\pgfusepath{stroke}%
\end{pgfscope}%
\begin{pgfscope}%
\definecolor{textcolor}{rgb}{0.000000,0.000000,0.000000}%
\pgfsetstrokecolor{textcolor}%
\pgfsetfillcolor{textcolor}%
\pgftext[x=5.810206in,y=4.495005in,left,base]{\color{textcolor}\sffamily\fontsize{20.000000}{24.000000}\selectfont Threshold}%
\end{pgfscope}%
\begin{pgfscope}%
\pgfsetbuttcap%
\pgfsetroundjoin%
\pgfsetlinewidth{1.505625pt}%
\definecolor{currentstroke}{rgb}{1.000000,0.000000,0.000000}%
\pgfsetstrokecolor{currentstroke}%
\pgfsetdash{}{0pt}%
\pgfpathmoveto{\pgfqpoint{5.032428in}{4.197271in}}%
\pgfpathlineto{\pgfqpoint{5.587983in}{4.197271in}}%
\pgfusepath{stroke}%
\end{pgfscope}%
\begin{pgfscope}%
\definecolor{textcolor}{rgb}{0.000000,0.000000,0.000000}%
\pgfsetstrokecolor{textcolor}%
\pgfsetfillcolor{textcolor}%
\pgftext[x=5.810206in,y=4.100048in,left,base]{\color{textcolor}\sffamily\fontsize{20.000000}{24.000000}\selectfont Charge}%
\end{pgfscope}%
\end{pgfpicture}%
\makeatother%
\endgroup%
}
    \caption{\label{fig:peak} A peak finding example gives \\ $\Delta t_0=\SI{-1.93}{ns}$, $\mathrm{RSS}=\SI{266.9}{mV^2}$, $D_\mathrm{w}=\SI{2.40}{ns}$.}
  \end{subfigure}
  \caption{\label{fig:method}Demonstrations of heuristic methods on a waveform sampled from $\mu=4$, $\tau_\ell=\SI{20}{ns}$, $\sigma_\ell=\SI{5}{ns}$ light curve conditions.  Peak finding in~\subref{fig:peak} handles charges more realistically than waveform shifting in~\subref{fig:shifting}, giving better numbers by the $\mathrm{RSS}$ and $D_\mathrm{w}$ criteria in section \ref{sec:criteria}. $\Delta t_0$ is the bias of the $\hat{t}_\mathrm{KL}$ estimator.}
\end{figure}

\subsubsection{Peak finding}
\label{sec:findpeak}

The peak of $V_\mathrm{PE}$ is a distinct feature in waveforms, making \textit{Peak finding} more effective than waveform shifting.  We smooth a waveform by a low-pass Savitzky-Golay filter~\cite{savitzky_smoothing_1964} and find all the peaks at $t_i$'s.  The following resembles waveform shifting: apply a constant shift $\Delta t \equiv \arg\underset{t}{\max} V_\mathrm{PE}(t)$ to get $\hat{t}_i = t_i - \Delta t$, and calculate a scaling factor $\alpha$ to get $\hat{q_i}=\hat{\alpha} w(t_i)$ in the same way as eq.~\eqref{eq:alpha}.  As shown in figure~\ref{fig:peak}, peak finding outputs charges close to 1 and works well for lower PE counts.  But when PEs pile up closely, peaks overlap intensively, making this method unreliable.  Peak finding is usually too trivial to be documented but found almost everywhere~\cite{students22}.

\subsection{Deconvolution}
\label{sec:deconv}
Deconvolution is motivated by viewing the waveform as a convolution of sparse spike train $\tilde{\phi}$ and $V_\mathrm{PE}$ in eq.~\eqref{eq:1}.  Huang et al.~\cite{huang_flash_2018} from DayaBay and Grassi et al.~\cite{grassi_charge_2018} introduced deconvolution-based waveform analysis in charge reconstruction and linearity studies.  Zhang et al.~\cite{zhang_comparison_2019} then applied it to the JUNO prototype.  Deconvolution methods are better than heuristic ones by using the full shape of $V_\mathrm{PE}(t)$, thus can accommodate overshoots and pile-ups.  Noise and Nyquist limit make deconvolution sensitive to fluctuations in real-world applications.  A carefully selected low-pass filter mitigates the difficulty but might introduce Gibbs ringing artifacts in the smoothed waveforms and the deconvoluted results. Despite such drawbacks, deconvolution algorithms are fast and useful to give initial crude solutions for the more advanced algorithms.  Deployed in running experiments, they are discussed in this section to make an objective evaluation. 

\subsubsection{Fourier deconvolution}
\label{sec:fourier}
The deconvolution relation is evident in the Fourier transform $\mathcal{F}$ to eq.~\eqref{eq:1},
\begin{equation}
  \label{eq:fourier}
  \mathcal{F}[w]  = \mathcal{F}[\tilde{\phi}]\mathcal{F}[V_\mathrm{PE}] + \mathcal{F}[\epsilon]
  \implies \mathcal{F}[\tilde{\phi}]  = \frac{\mathcal{F}[w]}{\mathcal{F}[V_\mathrm{PE}]} - \frac{\mathcal{F}[\epsilon]}{\mathcal{F}[V_\mathrm{PE}]}.
\end{equation}
By low-pass filtering the waveform $w(t)$ to get $\tilde{w}(t)$, we suppress the noise term $\epsilon$.  In the inverse Fourier transform $\hat{\phi}_1(t) = \mathcal{F}^{-1}\left[\frac{\mathcal{F}[\tilde{w}]}{\mathcal{F}[V_\mathrm{PE}]}\right](t)$, remaining noise and limited bandwidth lead to smaller and even negative $\hat{q}_i$.  We apply a $q_\mathrm{th}$ threshold regularizer to cut off the unphysical parts of $\hat{\phi}_1(t)$,
\begin{equation}
  \label{eq:fdconv2}
    \hat{\phi}(t) = \hat{\alpha}\underbrace{\hat{\phi}_1(t) I\left(\hat{\phi}_1(t) - q_\mathrm{th}\right)}_{\text{over-threshold part of} \hat{\phi}_1(t)}  
\end{equation}
where $I(x)$ is the indicator function, and $\hat{\alpha}$ is the scaling factor to minimize $\mathrm{RSS}$ like in eq.~\eqref{eq:alpha},
\begin{equation*}
  \begin{aligned}
  \label{eq:id}
  I(x) = \left\{
    \begin{array}{ll}
      1 & \mbox{, if $x\ge0$}, \\
      0 & \mbox{, otherwise}
    \end{array}
    \right.
    \quad~~~
    \hat{\alpha} = \arg \underset{\alpha}{\min}\mathrm{RSS}\left[\alpha \hat{\phi} \otimes V_\mathrm{PE}, w\right]. \\
  \end{aligned}
\end{equation*}

Figure~\ref{fig:fd} illustrates that Fourier deconvolution outperforms heuristic methods, but still with a lot of small-charged PEs.

\begin{figure}[H]
  \begin{subfigure}{0.5\textwidth}
    \centering
    \resizebox{\textwidth}{!}{%% Creator: Matplotlib, PGF backend
%%
%% To include the figure in your LaTeX document, write
%%   \input{<filename>.pgf}
%%
%% Make sure the required packages are loaded in your preamble
%%   \usepackage{pgf}
%%
%% Also ensure that all the required font packages are loaded; for instance,
%% the lmodern package is sometimes necessary when using math font.
%%   \usepackage{lmodern}
%%
%% Figures using additional raster images can only be included by \input if
%% they are in the same directory as the main LaTeX file. For loading figures
%% from other directories you can use the `import` package
%%   \usepackage{import}
%%
%% and then include the figures with
%%   \import{<path to file>}{<filename>.pgf}
%%
%% Matplotlib used the following preamble
%%   \usepackage[detect-all,locale=DE]{siunitx}
%%
\begingroup%
\makeatletter%
\begin{pgfpicture}%
\pgfpathrectangle{\pgfpointorigin}{\pgfqpoint{8.000000in}{6.000000in}}%
\pgfusepath{use as bounding box, clip}%
\begin{pgfscope}%
\pgfsetbuttcap%
\pgfsetmiterjoin%
\definecolor{currentfill}{rgb}{1.000000,1.000000,1.000000}%
\pgfsetfillcolor{currentfill}%
\pgfsetlinewidth{0.000000pt}%
\definecolor{currentstroke}{rgb}{1.000000,1.000000,1.000000}%
\pgfsetstrokecolor{currentstroke}%
\pgfsetdash{}{0pt}%
\pgfpathmoveto{\pgfqpoint{0.000000in}{0.000000in}}%
\pgfpathlineto{\pgfqpoint{8.000000in}{0.000000in}}%
\pgfpathlineto{\pgfqpoint{8.000000in}{6.000000in}}%
\pgfpathlineto{\pgfqpoint{0.000000in}{6.000000in}}%
\pgfpathlineto{\pgfqpoint{0.000000in}{0.000000in}}%
\pgfpathclose%
\pgfusepath{fill}%
\end{pgfscope}%
\begin{pgfscope}%
\pgfsetbuttcap%
\pgfsetmiterjoin%
\definecolor{currentfill}{rgb}{1.000000,1.000000,1.000000}%
\pgfsetfillcolor{currentfill}%
\pgfsetlinewidth{0.000000pt}%
\definecolor{currentstroke}{rgb}{0.000000,0.000000,0.000000}%
\pgfsetstrokecolor{currentstroke}%
\pgfsetstrokeopacity{0.000000}%
\pgfsetdash{}{0pt}%
\pgfpathmoveto{\pgfqpoint{1.000000in}{0.720000in}}%
\pgfpathlineto{\pgfqpoint{7.200000in}{0.720000in}}%
\pgfpathlineto{\pgfqpoint{7.200000in}{5.340000in}}%
\pgfpathlineto{\pgfqpoint{1.000000in}{5.340000in}}%
\pgfpathlineto{\pgfqpoint{1.000000in}{0.720000in}}%
\pgfpathclose%
\pgfusepath{fill}%
\end{pgfscope}%
\begin{pgfscope}%
\pgfsetbuttcap%
\pgfsetroundjoin%
\definecolor{currentfill}{rgb}{0.000000,0.000000,0.000000}%
\pgfsetfillcolor{currentfill}%
\pgfsetlinewidth{0.803000pt}%
\definecolor{currentstroke}{rgb}{0.000000,0.000000,0.000000}%
\pgfsetstrokecolor{currentstroke}%
\pgfsetdash{}{0pt}%
\pgfsys@defobject{currentmarker}{\pgfqpoint{0.000000in}{-0.048611in}}{\pgfqpoint{0.000000in}{0.000000in}}{%
\pgfpathmoveto{\pgfqpoint{0.000000in}{0.000000in}}%
\pgfpathlineto{\pgfqpoint{0.000000in}{-0.048611in}}%
\pgfusepath{stroke,fill}%
}%
\begin{pgfscope}%
\pgfsys@transformshift{1.310000in}{0.720000in}%
\pgfsys@useobject{currentmarker}{}%
\end{pgfscope}%
\end{pgfscope}%
\begin{pgfscope}%
\definecolor{textcolor}{rgb}{0.000000,0.000000,0.000000}%
\pgfsetstrokecolor{textcolor}%
\pgfsetfillcolor{textcolor}%
\pgftext[x=1.310000in,y=0.622778in,,top]{\color{textcolor}\sffamily\fontsize{20.000000}{24.000000}\selectfont \(\displaystyle {450}\)}%
\end{pgfscope}%
\begin{pgfscope}%
\pgfsetbuttcap%
\pgfsetroundjoin%
\definecolor{currentfill}{rgb}{0.000000,0.000000,0.000000}%
\pgfsetfillcolor{currentfill}%
\pgfsetlinewidth{0.803000pt}%
\definecolor{currentstroke}{rgb}{0.000000,0.000000,0.000000}%
\pgfsetstrokecolor{currentstroke}%
\pgfsetdash{}{0pt}%
\pgfsys@defobject{currentmarker}{\pgfqpoint{0.000000in}{-0.048611in}}{\pgfqpoint{0.000000in}{0.000000in}}{%
\pgfpathmoveto{\pgfqpoint{0.000000in}{0.000000in}}%
\pgfpathlineto{\pgfqpoint{0.000000in}{-0.048611in}}%
\pgfusepath{stroke,fill}%
}%
\begin{pgfscope}%
\pgfsys@transformshift{2.860000in}{0.720000in}%
\pgfsys@useobject{currentmarker}{}%
\end{pgfscope}%
\end{pgfscope}%
\begin{pgfscope}%
\definecolor{textcolor}{rgb}{0.000000,0.000000,0.000000}%
\pgfsetstrokecolor{textcolor}%
\pgfsetfillcolor{textcolor}%
\pgftext[x=2.860000in,y=0.622778in,,top]{\color{textcolor}\sffamily\fontsize{20.000000}{24.000000}\selectfont \(\displaystyle {500}\)}%
\end{pgfscope}%
\begin{pgfscope}%
\pgfsetbuttcap%
\pgfsetroundjoin%
\definecolor{currentfill}{rgb}{0.000000,0.000000,0.000000}%
\pgfsetfillcolor{currentfill}%
\pgfsetlinewidth{0.803000pt}%
\definecolor{currentstroke}{rgb}{0.000000,0.000000,0.000000}%
\pgfsetstrokecolor{currentstroke}%
\pgfsetdash{}{0pt}%
\pgfsys@defobject{currentmarker}{\pgfqpoint{0.000000in}{-0.048611in}}{\pgfqpoint{0.000000in}{0.000000in}}{%
\pgfpathmoveto{\pgfqpoint{0.000000in}{0.000000in}}%
\pgfpathlineto{\pgfqpoint{0.000000in}{-0.048611in}}%
\pgfusepath{stroke,fill}%
}%
\begin{pgfscope}%
\pgfsys@transformshift{4.410000in}{0.720000in}%
\pgfsys@useobject{currentmarker}{}%
\end{pgfscope}%
\end{pgfscope}%
\begin{pgfscope}%
\definecolor{textcolor}{rgb}{0.000000,0.000000,0.000000}%
\pgfsetstrokecolor{textcolor}%
\pgfsetfillcolor{textcolor}%
\pgftext[x=4.410000in,y=0.622778in,,top]{\color{textcolor}\sffamily\fontsize{20.000000}{24.000000}\selectfont \(\displaystyle {550}\)}%
\end{pgfscope}%
\begin{pgfscope}%
\pgfsetbuttcap%
\pgfsetroundjoin%
\definecolor{currentfill}{rgb}{0.000000,0.000000,0.000000}%
\pgfsetfillcolor{currentfill}%
\pgfsetlinewidth{0.803000pt}%
\definecolor{currentstroke}{rgb}{0.000000,0.000000,0.000000}%
\pgfsetstrokecolor{currentstroke}%
\pgfsetdash{}{0pt}%
\pgfsys@defobject{currentmarker}{\pgfqpoint{0.000000in}{-0.048611in}}{\pgfqpoint{0.000000in}{0.000000in}}{%
\pgfpathmoveto{\pgfqpoint{0.000000in}{0.000000in}}%
\pgfpathlineto{\pgfqpoint{0.000000in}{-0.048611in}}%
\pgfusepath{stroke,fill}%
}%
\begin{pgfscope}%
\pgfsys@transformshift{5.960000in}{0.720000in}%
\pgfsys@useobject{currentmarker}{}%
\end{pgfscope}%
\end{pgfscope}%
\begin{pgfscope}%
\definecolor{textcolor}{rgb}{0.000000,0.000000,0.000000}%
\pgfsetstrokecolor{textcolor}%
\pgfsetfillcolor{textcolor}%
\pgftext[x=5.960000in,y=0.622778in,,top]{\color{textcolor}\sffamily\fontsize{20.000000}{24.000000}\selectfont \(\displaystyle {600}\)}%
\end{pgfscope}%
\begin{pgfscope}%
\definecolor{textcolor}{rgb}{0.000000,0.000000,0.000000}%
\pgfsetstrokecolor{textcolor}%
\pgfsetfillcolor{textcolor}%
\pgftext[x=4.100000in,y=0.311155in,,top]{\color{textcolor}\sffamily\fontsize{20.000000}{24.000000}\selectfont \(\displaystyle \mathrm{t}/\si{ns}\)}%
\end{pgfscope}%
\begin{pgfscope}%
\pgfsetbuttcap%
\pgfsetroundjoin%
\definecolor{currentfill}{rgb}{0.000000,0.000000,0.000000}%
\pgfsetfillcolor{currentfill}%
\pgfsetlinewidth{0.803000pt}%
\definecolor{currentstroke}{rgb}{0.000000,0.000000,0.000000}%
\pgfsetstrokecolor{currentstroke}%
\pgfsetdash{}{0pt}%
\pgfsys@defobject{currentmarker}{\pgfqpoint{-0.048611in}{0.000000in}}{\pgfqpoint{-0.000000in}{0.000000in}}{%
\pgfpathmoveto{\pgfqpoint{-0.000000in}{0.000000in}}%
\pgfpathlineto{\pgfqpoint{-0.048611in}{0.000000in}}%
\pgfusepath{stroke,fill}%
}%
\begin{pgfscope}%
\pgfsys@transformshift{1.000000in}{1.145834in}%
\pgfsys@useobject{currentmarker}{}%
\end{pgfscope}%
\end{pgfscope}%
\begin{pgfscope}%
\definecolor{textcolor}{rgb}{0.000000,0.000000,0.000000}%
\pgfsetstrokecolor{textcolor}%
\pgfsetfillcolor{textcolor}%
\pgftext[x=0.770670in, y=1.045815in, left, base]{\color{textcolor}\sffamily\fontsize{20.000000}{24.000000}\selectfont \(\displaystyle {0}\)}%
\end{pgfscope}%
\begin{pgfscope}%
\pgfsetbuttcap%
\pgfsetroundjoin%
\definecolor{currentfill}{rgb}{0.000000,0.000000,0.000000}%
\pgfsetfillcolor{currentfill}%
\pgfsetlinewidth{0.803000pt}%
\definecolor{currentstroke}{rgb}{0.000000,0.000000,0.000000}%
\pgfsetstrokecolor{currentstroke}%
\pgfsetdash{}{0pt}%
\pgfsys@defobject{currentmarker}{\pgfqpoint{-0.048611in}{0.000000in}}{\pgfqpoint{-0.000000in}{0.000000in}}{%
\pgfpathmoveto{\pgfqpoint{-0.000000in}{0.000000in}}%
\pgfpathlineto{\pgfqpoint{-0.048611in}{0.000000in}}%
\pgfusepath{stroke,fill}%
}%
\begin{pgfscope}%
\pgfsys@transformshift{1.000000in}{1.760228in}%
\pgfsys@useobject{currentmarker}{}%
\end{pgfscope}%
\end{pgfscope}%
\begin{pgfscope}%
\definecolor{textcolor}{rgb}{0.000000,0.000000,0.000000}%
\pgfsetstrokecolor{textcolor}%
\pgfsetfillcolor{textcolor}%
\pgftext[x=0.770670in, y=1.660209in, left, base]{\color{textcolor}\sffamily\fontsize{20.000000}{24.000000}\selectfont \(\displaystyle {5}\)}%
\end{pgfscope}%
\begin{pgfscope}%
\pgfsetbuttcap%
\pgfsetroundjoin%
\definecolor{currentfill}{rgb}{0.000000,0.000000,0.000000}%
\pgfsetfillcolor{currentfill}%
\pgfsetlinewidth{0.803000pt}%
\definecolor{currentstroke}{rgb}{0.000000,0.000000,0.000000}%
\pgfsetstrokecolor{currentstroke}%
\pgfsetdash{}{0pt}%
\pgfsys@defobject{currentmarker}{\pgfqpoint{-0.048611in}{0.000000in}}{\pgfqpoint{-0.000000in}{0.000000in}}{%
\pgfpathmoveto{\pgfqpoint{-0.000000in}{0.000000in}}%
\pgfpathlineto{\pgfqpoint{-0.048611in}{0.000000in}}%
\pgfusepath{stroke,fill}%
}%
\begin{pgfscope}%
\pgfsys@transformshift{1.000000in}{2.374622in}%
\pgfsys@useobject{currentmarker}{}%
\end{pgfscope}%
\end{pgfscope}%
\begin{pgfscope}%
\definecolor{textcolor}{rgb}{0.000000,0.000000,0.000000}%
\pgfsetstrokecolor{textcolor}%
\pgfsetfillcolor{textcolor}%
\pgftext[x=0.638563in, y=2.274603in, left, base]{\color{textcolor}\sffamily\fontsize{20.000000}{24.000000}\selectfont \(\displaystyle {10}\)}%
\end{pgfscope}%
\begin{pgfscope}%
\pgfsetbuttcap%
\pgfsetroundjoin%
\definecolor{currentfill}{rgb}{0.000000,0.000000,0.000000}%
\pgfsetfillcolor{currentfill}%
\pgfsetlinewidth{0.803000pt}%
\definecolor{currentstroke}{rgb}{0.000000,0.000000,0.000000}%
\pgfsetstrokecolor{currentstroke}%
\pgfsetdash{}{0pt}%
\pgfsys@defobject{currentmarker}{\pgfqpoint{-0.048611in}{0.000000in}}{\pgfqpoint{-0.000000in}{0.000000in}}{%
\pgfpathmoveto{\pgfqpoint{-0.000000in}{0.000000in}}%
\pgfpathlineto{\pgfqpoint{-0.048611in}{0.000000in}}%
\pgfusepath{stroke,fill}%
}%
\begin{pgfscope}%
\pgfsys@transformshift{1.000000in}{2.989016in}%
\pgfsys@useobject{currentmarker}{}%
\end{pgfscope}%
\end{pgfscope}%
\begin{pgfscope}%
\definecolor{textcolor}{rgb}{0.000000,0.000000,0.000000}%
\pgfsetstrokecolor{textcolor}%
\pgfsetfillcolor{textcolor}%
\pgftext[x=0.638563in, y=2.888996in, left, base]{\color{textcolor}\sffamily\fontsize{20.000000}{24.000000}\selectfont \(\displaystyle {15}\)}%
\end{pgfscope}%
\begin{pgfscope}%
\pgfsetbuttcap%
\pgfsetroundjoin%
\definecolor{currentfill}{rgb}{0.000000,0.000000,0.000000}%
\pgfsetfillcolor{currentfill}%
\pgfsetlinewidth{0.803000pt}%
\definecolor{currentstroke}{rgb}{0.000000,0.000000,0.000000}%
\pgfsetstrokecolor{currentstroke}%
\pgfsetdash{}{0pt}%
\pgfsys@defobject{currentmarker}{\pgfqpoint{-0.048611in}{0.000000in}}{\pgfqpoint{-0.000000in}{0.000000in}}{%
\pgfpathmoveto{\pgfqpoint{-0.000000in}{0.000000in}}%
\pgfpathlineto{\pgfqpoint{-0.048611in}{0.000000in}}%
\pgfusepath{stroke,fill}%
}%
\begin{pgfscope}%
\pgfsys@transformshift{1.000000in}{3.603409in}%
\pgfsys@useobject{currentmarker}{}%
\end{pgfscope}%
\end{pgfscope}%
\begin{pgfscope}%
\definecolor{textcolor}{rgb}{0.000000,0.000000,0.000000}%
\pgfsetstrokecolor{textcolor}%
\pgfsetfillcolor{textcolor}%
\pgftext[x=0.638563in, y=3.503390in, left, base]{\color{textcolor}\sffamily\fontsize{20.000000}{24.000000}\selectfont \(\displaystyle {20}\)}%
\end{pgfscope}%
\begin{pgfscope}%
\pgfsetbuttcap%
\pgfsetroundjoin%
\definecolor{currentfill}{rgb}{0.000000,0.000000,0.000000}%
\pgfsetfillcolor{currentfill}%
\pgfsetlinewidth{0.803000pt}%
\definecolor{currentstroke}{rgb}{0.000000,0.000000,0.000000}%
\pgfsetstrokecolor{currentstroke}%
\pgfsetdash{}{0pt}%
\pgfsys@defobject{currentmarker}{\pgfqpoint{-0.048611in}{0.000000in}}{\pgfqpoint{-0.000000in}{0.000000in}}{%
\pgfpathmoveto{\pgfqpoint{-0.000000in}{0.000000in}}%
\pgfpathlineto{\pgfqpoint{-0.048611in}{0.000000in}}%
\pgfusepath{stroke,fill}%
}%
\begin{pgfscope}%
\pgfsys@transformshift{1.000000in}{4.217803in}%
\pgfsys@useobject{currentmarker}{}%
\end{pgfscope}%
\end{pgfscope}%
\begin{pgfscope}%
\definecolor{textcolor}{rgb}{0.000000,0.000000,0.000000}%
\pgfsetstrokecolor{textcolor}%
\pgfsetfillcolor{textcolor}%
\pgftext[x=0.638563in, y=4.117784in, left, base]{\color{textcolor}\sffamily\fontsize{20.000000}{24.000000}\selectfont \(\displaystyle {25}\)}%
\end{pgfscope}%
\begin{pgfscope}%
\pgfsetbuttcap%
\pgfsetroundjoin%
\definecolor{currentfill}{rgb}{0.000000,0.000000,0.000000}%
\pgfsetfillcolor{currentfill}%
\pgfsetlinewidth{0.803000pt}%
\definecolor{currentstroke}{rgb}{0.000000,0.000000,0.000000}%
\pgfsetstrokecolor{currentstroke}%
\pgfsetdash{}{0pt}%
\pgfsys@defobject{currentmarker}{\pgfqpoint{-0.048611in}{0.000000in}}{\pgfqpoint{-0.000000in}{0.000000in}}{%
\pgfpathmoveto{\pgfqpoint{-0.000000in}{0.000000in}}%
\pgfpathlineto{\pgfqpoint{-0.048611in}{0.000000in}}%
\pgfusepath{stroke,fill}%
}%
\begin{pgfscope}%
\pgfsys@transformshift{1.000000in}{4.832197in}%
\pgfsys@useobject{currentmarker}{}%
\end{pgfscope}%
\end{pgfscope}%
\begin{pgfscope}%
\definecolor{textcolor}{rgb}{0.000000,0.000000,0.000000}%
\pgfsetstrokecolor{textcolor}%
\pgfsetfillcolor{textcolor}%
\pgftext[x=0.638563in, y=4.732177in, left, base]{\color{textcolor}\sffamily\fontsize{20.000000}{24.000000}\selectfont \(\displaystyle {30}\)}%
\end{pgfscope}%
\begin{pgfscope}%
\definecolor{textcolor}{rgb}{0.000000,0.000000,0.000000}%
\pgfsetstrokecolor{textcolor}%
\pgfsetfillcolor{textcolor}%
\pgftext[x=0.583007in,y=3.030000in,,bottom,rotate=90.000000]{\color{textcolor}\sffamily\fontsize{20.000000}{24.000000}\selectfont \(\displaystyle \mathrm{Voltage}/\si{mV}\)}%
\end{pgfscope}%
\begin{pgfscope}%
\pgfpathrectangle{\pgfqpoint{1.000000in}{0.720000in}}{\pgfqpoint{6.200000in}{4.620000in}}%
\pgfusepath{clip}%
\pgfsetrectcap%
\pgfsetroundjoin%
\pgfsetlinewidth{2.007500pt}%
\definecolor{currentstroke}{rgb}{0.121569,0.466667,0.705882}%
\pgfsetstrokecolor{currentstroke}%
\pgfsetdash{}{0pt}%
\pgfpathmoveto{\pgfqpoint{0.990000in}{1.222593in}}%
\pgfpathlineto{\pgfqpoint{1.000000in}{1.270894in}}%
\pgfpathlineto{\pgfqpoint{1.031000in}{1.096684in}}%
\pgfpathlineto{\pgfqpoint{1.062000in}{1.199203in}}%
\pgfpathlineto{\pgfqpoint{1.093000in}{1.182173in}}%
\pgfpathlineto{\pgfqpoint{1.124000in}{1.162905in}}%
\pgfpathlineto{\pgfqpoint{1.155000in}{1.284631in}}%
\pgfpathlineto{\pgfqpoint{1.186000in}{1.177477in}}%
\pgfpathlineto{\pgfqpoint{1.217000in}{1.091844in}}%
\pgfpathlineto{\pgfqpoint{1.248000in}{1.268450in}}%
\pgfpathlineto{\pgfqpoint{1.279000in}{1.178661in}}%
\pgfpathlineto{\pgfqpoint{1.310000in}{1.073030in}}%
\pgfpathlineto{\pgfqpoint{1.341000in}{1.206884in}}%
\pgfpathlineto{\pgfqpoint{1.372000in}{1.124892in}}%
\pgfpathlineto{\pgfqpoint{1.403000in}{1.416265in}}%
\pgfpathlineto{\pgfqpoint{1.434000in}{1.067380in}}%
\pgfpathlineto{\pgfqpoint{1.465000in}{0.989741in}}%
\pgfpathlineto{\pgfqpoint{1.496000in}{1.095559in}}%
\pgfpathlineto{\pgfqpoint{1.527000in}{1.024990in}}%
\pgfpathlineto{\pgfqpoint{1.558000in}{0.991097in}}%
\pgfpathlineto{\pgfqpoint{1.589000in}{1.134677in}}%
\pgfpathlineto{\pgfqpoint{1.620000in}{1.246345in}}%
\pgfpathlineto{\pgfqpoint{1.651000in}{1.216694in}}%
\pgfpathlineto{\pgfqpoint{1.682000in}{1.359832in}}%
\pgfpathlineto{\pgfqpoint{1.713000in}{1.251417in}}%
\pgfpathlineto{\pgfqpoint{1.744000in}{1.282940in}}%
\pgfpathlineto{\pgfqpoint{1.775000in}{1.301399in}}%
\pgfpathlineto{\pgfqpoint{1.806000in}{1.259591in}}%
\pgfpathlineto{\pgfqpoint{1.837000in}{1.121847in}}%
\pgfpathlineto{\pgfqpoint{1.868000in}{0.996067in}}%
\pgfpathlineto{\pgfqpoint{1.899000in}{1.170727in}}%
\pgfpathlineto{\pgfqpoint{1.930000in}{1.108552in}}%
\pgfpathlineto{\pgfqpoint{1.961000in}{1.253788in}}%
\pgfpathlineto{\pgfqpoint{1.992000in}{1.246488in}}%
\pgfpathlineto{\pgfqpoint{2.023000in}{1.523801in}}%
\pgfpathlineto{\pgfqpoint{2.054000in}{1.984247in}}%
\pgfpathlineto{\pgfqpoint{2.085000in}{2.516883in}}%
\pgfpathlineto{\pgfqpoint{2.116000in}{3.445723in}}%
\pgfpathlineto{\pgfqpoint{2.147000in}{3.680523in}}%
\pgfpathlineto{\pgfqpoint{2.178000in}{4.205954in}}%
\pgfpathlineto{\pgfqpoint{2.209000in}{4.372116in}}%
\pgfpathlineto{\pgfqpoint{2.240000in}{4.065688in}}%
\pgfpathlineto{\pgfqpoint{2.271000in}{4.043842in}}%
\pgfpathlineto{\pgfqpoint{2.302000in}{3.941145in}}%
\pgfpathlineto{\pgfqpoint{2.333000in}{3.300069in}}%
\pgfpathlineto{\pgfqpoint{2.364000in}{3.078449in}}%
\pgfpathlineto{\pgfqpoint{2.395000in}{3.064178in}}%
\pgfpathlineto{\pgfqpoint{2.426000in}{2.690044in}}%
\pgfpathlineto{\pgfqpoint{2.457000in}{2.201641in}}%
\pgfpathlineto{\pgfqpoint{2.488000in}{2.085521in}}%
\pgfpathlineto{\pgfqpoint{2.519000in}{1.980297in}}%
\pgfpathlineto{\pgfqpoint{2.550000in}{1.923553in}}%
\pgfpathlineto{\pgfqpoint{2.581000in}{2.204345in}}%
\pgfpathlineto{\pgfqpoint{2.612000in}{2.945116in}}%
\pgfpathlineto{\pgfqpoint{2.643000in}{3.255714in}}%
\pgfpathlineto{\pgfqpoint{2.674000in}{3.764764in}}%
\pgfpathlineto{\pgfqpoint{2.705000in}{3.880315in}}%
\pgfpathlineto{\pgfqpoint{2.736000in}{3.733495in}}%
\pgfpathlineto{\pgfqpoint{2.767000in}{3.523265in}}%
\pgfpathlineto{\pgfqpoint{2.798000in}{3.340267in}}%
\pgfpathlineto{\pgfqpoint{2.829000in}{3.126720in}}%
\pgfpathlineto{\pgfqpoint{2.860000in}{2.975547in}}%
\pgfpathlineto{\pgfqpoint{2.891000in}{2.613938in}}%
\pgfpathlineto{\pgfqpoint{2.922000in}{2.673898in}}%
\pgfpathlineto{\pgfqpoint{2.953000in}{2.415341in}}%
\pgfpathlineto{\pgfqpoint{2.984000in}{3.086620in}}%
\pgfpathlineto{\pgfqpoint{3.015000in}{3.095033in}}%
\pgfpathlineto{\pgfqpoint{3.046000in}{3.509890in}}%
\pgfpathlineto{\pgfqpoint{3.077000in}{3.344637in}}%
\pgfpathlineto{\pgfqpoint{3.108000in}{3.373180in}}%
\pgfpathlineto{\pgfqpoint{3.139000in}{3.271231in}}%
\pgfpathlineto{\pgfqpoint{3.170000in}{2.933447in}}%
\pgfpathlineto{\pgfqpoint{3.201000in}{2.787981in}}%
\pgfpathlineto{\pgfqpoint{3.232000in}{2.438344in}}%
\pgfpathlineto{\pgfqpoint{3.263000in}{2.254036in}}%
\pgfpathlineto{\pgfqpoint{3.294000in}{2.052950in}}%
\pgfpathlineto{\pgfqpoint{3.325000in}{1.772394in}}%
\pgfpathlineto{\pgfqpoint{3.356000in}{1.928720in}}%
\pgfpathlineto{\pgfqpoint{3.387000in}{1.947541in}}%
\pgfpathlineto{\pgfqpoint{3.418000in}{1.654039in}}%
\pgfpathlineto{\pgfqpoint{3.449000in}{1.395126in}}%
\pgfpathlineto{\pgfqpoint{3.480000in}{1.335504in}}%
\pgfpathlineto{\pgfqpoint{3.511000in}{1.608582in}}%
\pgfpathlineto{\pgfqpoint{3.542000in}{1.312849in}}%
\pgfpathlineto{\pgfqpoint{3.573000in}{1.284391in}}%
\pgfpathlineto{\pgfqpoint{3.604000in}{1.447068in}}%
\pgfpathlineto{\pgfqpoint{3.635000in}{1.230849in}}%
\pgfpathlineto{\pgfqpoint{3.666000in}{1.174907in}}%
\pgfpathlineto{\pgfqpoint{3.697000in}{1.262077in}}%
\pgfpathlineto{\pgfqpoint{3.728000in}{1.166534in}}%
\pgfpathlineto{\pgfqpoint{3.759000in}{1.093215in}}%
\pgfpathlineto{\pgfqpoint{3.790000in}{1.433177in}}%
\pgfpathlineto{\pgfqpoint{3.821000in}{1.558773in}}%
\pgfpathlineto{\pgfqpoint{3.852000in}{2.170644in}}%
\pgfpathlineto{\pgfqpoint{3.883000in}{2.453631in}}%
\pgfpathlineto{\pgfqpoint{3.914000in}{3.047367in}}%
\pgfpathlineto{\pgfqpoint{3.945000in}{2.995175in}}%
\pgfpathlineto{\pgfqpoint{3.976000in}{3.368100in}}%
\pgfpathlineto{\pgfqpoint{4.007000in}{3.201769in}}%
\pgfpathlineto{\pgfqpoint{4.038000in}{2.748277in}}%
\pgfpathlineto{\pgfqpoint{4.069000in}{2.746696in}}%
\pgfpathlineto{\pgfqpoint{4.100000in}{2.585518in}}%
\pgfpathlineto{\pgfqpoint{4.131000in}{2.310610in}}%
\pgfpathlineto{\pgfqpoint{4.162000in}{2.080649in}}%
\pgfpathlineto{\pgfqpoint{4.193000in}{2.001216in}}%
\pgfpathlineto{\pgfqpoint{4.224000in}{2.058998in}}%
\pgfpathlineto{\pgfqpoint{4.255000in}{1.975551in}}%
\pgfpathlineto{\pgfqpoint{4.286000in}{1.582643in}}%
\pgfpathlineto{\pgfqpoint{4.317000in}{1.542779in}}%
\pgfpathlineto{\pgfqpoint{4.348000in}{1.470070in}}%
\pgfpathlineto{\pgfqpoint{4.379000in}{1.365768in}}%
\pgfpathlineto{\pgfqpoint{4.410000in}{1.220194in}}%
\pgfpathlineto{\pgfqpoint{4.441000in}{1.444082in}}%
\pgfpathlineto{\pgfqpoint{4.472000in}{1.308676in}}%
\pgfpathlineto{\pgfqpoint{4.534000in}{1.147527in}}%
\pgfpathlineto{\pgfqpoint{4.565000in}{1.061984in}}%
\pgfpathlineto{\pgfqpoint{4.596000in}{1.342557in}}%
\pgfpathlineto{\pgfqpoint{4.627000in}{1.361226in}}%
\pgfpathlineto{\pgfqpoint{4.658000in}{1.132211in}}%
\pgfpathlineto{\pgfqpoint{4.689000in}{1.291251in}}%
\pgfpathlineto{\pgfqpoint{4.720000in}{0.951966in}}%
\pgfpathlineto{\pgfqpoint{4.751000in}{1.137488in}}%
\pgfpathlineto{\pgfqpoint{4.782000in}{1.167291in}}%
\pgfpathlineto{\pgfqpoint{4.813000in}{1.007608in}}%
\pgfpathlineto{\pgfqpoint{4.844000in}{1.082917in}}%
\pgfpathlineto{\pgfqpoint{4.875000in}{0.790972in}}%
\pgfpathlineto{\pgfqpoint{4.906000in}{1.010384in}}%
\pgfpathlineto{\pgfqpoint{4.937000in}{1.043517in}}%
\pgfpathlineto{\pgfqpoint{4.968000in}{0.915696in}}%
\pgfpathlineto{\pgfqpoint{4.999000in}{1.367348in}}%
\pgfpathlineto{\pgfqpoint{5.030000in}{1.217998in}}%
\pgfpathlineto{\pgfqpoint{5.061000in}{1.138635in}}%
\pgfpathlineto{\pgfqpoint{5.092000in}{1.241721in}}%
\pgfpathlineto{\pgfqpoint{5.123000in}{0.959686in}}%
\pgfpathlineto{\pgfqpoint{5.154000in}{1.071188in}}%
\pgfpathlineto{\pgfqpoint{5.185000in}{1.006173in}}%
\pgfpathlineto{\pgfqpoint{5.216000in}{0.910433in}}%
\pgfpathlineto{\pgfqpoint{5.247000in}{1.133809in}}%
\pgfpathlineto{\pgfqpoint{5.278000in}{1.162820in}}%
\pgfpathlineto{\pgfqpoint{5.309000in}{1.258142in}}%
\pgfpathlineto{\pgfqpoint{5.371000in}{1.118255in}}%
\pgfpathlineto{\pgfqpoint{5.402000in}{1.213248in}}%
\pgfpathlineto{\pgfqpoint{5.433000in}{1.243709in}}%
\pgfpathlineto{\pgfqpoint{5.464000in}{1.067614in}}%
\pgfpathlineto{\pgfqpoint{5.495000in}{1.030143in}}%
\pgfpathlineto{\pgfqpoint{5.526000in}{1.069084in}}%
\pgfpathlineto{\pgfqpoint{5.557000in}{1.110206in}}%
\pgfpathlineto{\pgfqpoint{5.588000in}{1.020416in}}%
\pgfpathlineto{\pgfqpoint{5.619000in}{0.896119in}}%
\pgfpathlineto{\pgfqpoint{5.650000in}{1.079030in}}%
\pgfpathlineto{\pgfqpoint{5.681000in}{1.084256in}}%
\pgfpathlineto{\pgfqpoint{5.712000in}{1.115058in}}%
\pgfpathlineto{\pgfqpoint{5.743000in}{1.170423in}}%
\pgfpathlineto{\pgfqpoint{5.774000in}{1.328240in}}%
\pgfpathlineto{\pgfqpoint{5.805000in}{1.132625in}}%
\pgfpathlineto{\pgfqpoint{5.836000in}{1.051242in}}%
\pgfpathlineto{\pgfqpoint{5.867000in}{1.086527in}}%
\pgfpathlineto{\pgfqpoint{5.898000in}{1.290298in}}%
\pgfpathlineto{\pgfqpoint{5.929000in}{1.277211in}}%
\pgfpathlineto{\pgfqpoint{5.960000in}{1.076079in}}%
\pgfpathlineto{\pgfqpoint{5.991000in}{1.035199in}}%
\pgfpathlineto{\pgfqpoint{6.022000in}{1.106868in}}%
\pgfpathlineto{\pgfqpoint{6.053000in}{1.008838in}}%
\pgfpathlineto{\pgfqpoint{6.084000in}{1.039905in}}%
\pgfpathlineto{\pgfqpoint{6.115000in}{1.120884in}}%
\pgfpathlineto{\pgfqpoint{6.146000in}{1.167014in}}%
\pgfpathlineto{\pgfqpoint{6.177000in}{1.099064in}}%
\pgfpathlineto{\pgfqpoint{6.208000in}{1.219160in}}%
\pgfpathlineto{\pgfqpoint{6.239000in}{1.030414in}}%
\pgfpathlineto{\pgfqpoint{6.270000in}{1.253325in}}%
\pgfpathlineto{\pgfqpoint{6.301000in}{0.980623in}}%
\pgfpathlineto{\pgfqpoint{6.332000in}{1.059867in}}%
\pgfpathlineto{\pgfqpoint{6.363000in}{1.084611in}}%
\pgfpathlineto{\pgfqpoint{6.394000in}{1.076653in}}%
\pgfpathlineto{\pgfqpoint{6.425000in}{1.336888in}}%
\pgfpathlineto{\pgfqpoint{6.456000in}{1.165206in}}%
\pgfpathlineto{\pgfqpoint{6.487000in}{1.270963in}}%
\pgfpathlineto{\pgfqpoint{6.518000in}{1.209477in}}%
\pgfpathlineto{\pgfqpoint{6.549000in}{1.219259in}}%
\pgfpathlineto{\pgfqpoint{6.580000in}{1.158620in}}%
\pgfpathlineto{\pgfqpoint{6.611000in}{1.040458in}}%
\pgfpathlineto{\pgfqpoint{6.642000in}{1.223128in}}%
\pgfpathlineto{\pgfqpoint{6.673000in}{1.099591in}}%
\pgfpathlineto{\pgfqpoint{6.704000in}{0.938364in}}%
\pgfpathlineto{\pgfqpoint{6.735000in}{1.142117in}}%
\pgfpathlineto{\pgfqpoint{6.766000in}{1.261119in}}%
\pgfpathlineto{\pgfqpoint{6.797000in}{1.222967in}}%
\pgfpathlineto{\pgfqpoint{6.828000in}{1.210582in}}%
\pgfpathlineto{\pgfqpoint{6.859000in}{0.926154in}}%
\pgfpathlineto{\pgfqpoint{6.890000in}{1.159474in}}%
\pgfpathlineto{\pgfqpoint{6.921000in}{0.851001in}}%
\pgfpathlineto{\pgfqpoint{6.952000in}{1.159442in}}%
\pgfpathlineto{\pgfqpoint{6.983000in}{0.993737in}}%
\pgfpathlineto{\pgfqpoint{7.014000in}{1.132429in}}%
\pgfpathlineto{\pgfqpoint{7.045000in}{1.167750in}}%
\pgfpathlineto{\pgfqpoint{7.076000in}{1.054561in}}%
\pgfpathlineto{\pgfqpoint{7.107000in}{1.039172in}}%
\pgfpathlineto{\pgfqpoint{7.138000in}{1.226791in}}%
\pgfpathlineto{\pgfqpoint{7.169000in}{1.370284in}}%
\pgfpathlineto{\pgfqpoint{7.200000in}{1.225372in}}%
\pgfpathlineto{\pgfqpoint{7.210000in}{1.140349in}}%
\pgfpathlineto{\pgfqpoint{7.210000in}{1.140349in}}%
\pgfusepath{stroke}%
\end{pgfscope}%
\begin{pgfscope}%
\pgfpathrectangle{\pgfqpoint{1.000000in}{0.720000in}}{\pgfqpoint{6.200000in}{4.620000in}}%
\pgfusepath{clip}%
\pgfsetbuttcap%
\pgfsetroundjoin%
\pgfsetlinewidth{2.007500pt}%
\definecolor{currentstroke}{rgb}{0.000000,0.500000,0.000000}%
\pgfsetstrokecolor{currentstroke}%
\pgfsetdash{}{0pt}%
\pgfpathmoveto{\pgfqpoint{0.990000in}{1.760228in}}%
\pgfpathlineto{\pgfqpoint{7.210000in}{1.760228in}}%
\pgfusepath{stroke}%
\end{pgfscope}%
\begin{pgfscope}%
\pgfsetrectcap%
\pgfsetmiterjoin%
\pgfsetlinewidth{0.803000pt}%
\definecolor{currentstroke}{rgb}{0.000000,0.000000,0.000000}%
\pgfsetstrokecolor{currentstroke}%
\pgfsetdash{}{0pt}%
\pgfpathmoveto{\pgfqpoint{1.000000in}{0.720000in}}%
\pgfpathlineto{\pgfqpoint{1.000000in}{5.340000in}}%
\pgfusepath{stroke}%
\end{pgfscope}%
\begin{pgfscope}%
\pgfsetrectcap%
\pgfsetmiterjoin%
\pgfsetlinewidth{0.803000pt}%
\definecolor{currentstroke}{rgb}{0.000000,0.000000,0.000000}%
\pgfsetstrokecolor{currentstroke}%
\pgfsetdash{}{0pt}%
\pgfpathmoveto{\pgfqpoint{7.200000in}{0.720000in}}%
\pgfpathlineto{\pgfqpoint{7.200000in}{5.340000in}}%
\pgfusepath{stroke}%
\end{pgfscope}%
\begin{pgfscope}%
\pgfsetrectcap%
\pgfsetmiterjoin%
\pgfsetlinewidth{0.803000pt}%
\definecolor{currentstroke}{rgb}{0.000000,0.000000,0.000000}%
\pgfsetstrokecolor{currentstroke}%
\pgfsetdash{}{0pt}%
\pgfpathmoveto{\pgfqpoint{1.000000in}{0.720000in}}%
\pgfpathlineto{\pgfqpoint{7.200000in}{0.720000in}}%
\pgfusepath{stroke}%
\end{pgfscope}%
\begin{pgfscope}%
\pgfsetrectcap%
\pgfsetmiterjoin%
\pgfsetlinewidth{0.803000pt}%
\definecolor{currentstroke}{rgb}{0.000000,0.000000,0.000000}%
\pgfsetstrokecolor{currentstroke}%
\pgfsetdash{}{0pt}%
\pgfpathmoveto{\pgfqpoint{1.000000in}{5.340000in}}%
\pgfpathlineto{\pgfqpoint{7.200000in}{5.340000in}}%
\pgfusepath{stroke}%
\end{pgfscope}%
\begin{pgfscope}%
\pgfsetbuttcap%
\pgfsetroundjoin%
\definecolor{currentfill}{rgb}{0.000000,0.000000,0.000000}%
\pgfsetfillcolor{currentfill}%
\pgfsetlinewidth{0.803000pt}%
\definecolor{currentstroke}{rgb}{0.000000,0.000000,0.000000}%
\pgfsetstrokecolor{currentstroke}%
\pgfsetdash{}{0pt}%
\pgfsys@defobject{currentmarker}{\pgfqpoint{0.000000in}{0.000000in}}{\pgfqpoint{0.048611in}{0.000000in}}{%
\pgfpathmoveto{\pgfqpoint{0.000000in}{0.000000in}}%
\pgfpathlineto{\pgfqpoint{0.048611in}{0.000000in}}%
\pgfusepath{stroke,fill}%
}%
\begin{pgfscope}%
\pgfsys@transformshift{7.200000in}{1.145834in}%
\pgfsys@useobject{currentmarker}{}%
\end{pgfscope}%
\end{pgfscope}%
\begin{pgfscope}%
\definecolor{textcolor}{rgb}{0.000000,0.000000,0.000000}%
\pgfsetstrokecolor{textcolor}%
\pgfsetfillcolor{textcolor}%
\pgftext[x=7.297222in, y=1.045815in, left, base]{\color{textcolor}\sffamily\fontsize{20.000000}{24.000000}\selectfont 0.0}%
\end{pgfscope}%
\begin{pgfscope}%
\pgfsetbuttcap%
\pgfsetroundjoin%
\definecolor{currentfill}{rgb}{0.000000,0.000000,0.000000}%
\pgfsetfillcolor{currentfill}%
\pgfsetlinewidth{0.803000pt}%
\definecolor{currentstroke}{rgb}{0.000000,0.000000,0.000000}%
\pgfsetstrokecolor{currentstroke}%
\pgfsetdash{}{0pt}%
\pgfsys@defobject{currentmarker}{\pgfqpoint{0.000000in}{0.000000in}}{\pgfqpoint{0.048611in}{0.000000in}}{%
\pgfpathmoveto{\pgfqpoint{0.000000in}{0.000000in}}%
\pgfpathlineto{\pgfqpoint{0.048611in}{0.000000in}}%
\pgfusepath{stroke,fill}%
}%
\begin{pgfscope}%
\pgfsys@transformshift{7.200000in}{1.762623in}%
\pgfsys@useobject{currentmarker}{}%
\end{pgfscope}%
\end{pgfscope}%
\begin{pgfscope}%
\definecolor{textcolor}{rgb}{0.000000,0.000000,0.000000}%
\pgfsetstrokecolor{textcolor}%
\pgfsetfillcolor{textcolor}%
\pgftext[x=7.297222in, y=1.662604in, left, base]{\color{textcolor}\sffamily\fontsize{20.000000}{24.000000}\selectfont 0.2}%
\end{pgfscope}%
\begin{pgfscope}%
\pgfsetbuttcap%
\pgfsetroundjoin%
\definecolor{currentfill}{rgb}{0.000000,0.000000,0.000000}%
\pgfsetfillcolor{currentfill}%
\pgfsetlinewidth{0.803000pt}%
\definecolor{currentstroke}{rgb}{0.000000,0.000000,0.000000}%
\pgfsetstrokecolor{currentstroke}%
\pgfsetdash{}{0pt}%
\pgfsys@defobject{currentmarker}{\pgfqpoint{0.000000in}{0.000000in}}{\pgfqpoint{0.048611in}{0.000000in}}{%
\pgfpathmoveto{\pgfqpoint{0.000000in}{0.000000in}}%
\pgfpathlineto{\pgfqpoint{0.048611in}{0.000000in}}%
\pgfusepath{stroke,fill}%
}%
\begin{pgfscope}%
\pgfsys@transformshift{7.200000in}{2.379413in}%
\pgfsys@useobject{currentmarker}{}%
\end{pgfscope}%
\end{pgfscope}%
\begin{pgfscope}%
\definecolor{textcolor}{rgb}{0.000000,0.000000,0.000000}%
\pgfsetstrokecolor{textcolor}%
\pgfsetfillcolor{textcolor}%
\pgftext[x=7.297222in, y=2.279393in, left, base]{\color{textcolor}\sffamily\fontsize{20.000000}{24.000000}\selectfont 0.5}%
\end{pgfscope}%
\begin{pgfscope}%
\pgfsetbuttcap%
\pgfsetroundjoin%
\definecolor{currentfill}{rgb}{0.000000,0.000000,0.000000}%
\pgfsetfillcolor{currentfill}%
\pgfsetlinewidth{0.803000pt}%
\definecolor{currentstroke}{rgb}{0.000000,0.000000,0.000000}%
\pgfsetstrokecolor{currentstroke}%
\pgfsetdash{}{0pt}%
\pgfsys@defobject{currentmarker}{\pgfqpoint{0.000000in}{0.000000in}}{\pgfqpoint{0.048611in}{0.000000in}}{%
\pgfpathmoveto{\pgfqpoint{0.000000in}{0.000000in}}%
\pgfpathlineto{\pgfqpoint{0.048611in}{0.000000in}}%
\pgfusepath{stroke,fill}%
}%
\begin{pgfscope}%
\pgfsys@transformshift{7.200000in}{2.996202in}%
\pgfsys@useobject{currentmarker}{}%
\end{pgfscope}%
\end{pgfscope}%
\begin{pgfscope}%
\definecolor{textcolor}{rgb}{0.000000,0.000000,0.000000}%
\pgfsetstrokecolor{textcolor}%
\pgfsetfillcolor{textcolor}%
\pgftext[x=7.297222in, y=2.896182in, left, base]{\color{textcolor}\sffamily\fontsize{20.000000}{24.000000}\selectfont 0.8}%
\end{pgfscope}%
\begin{pgfscope}%
\pgfsetbuttcap%
\pgfsetroundjoin%
\definecolor{currentfill}{rgb}{0.000000,0.000000,0.000000}%
\pgfsetfillcolor{currentfill}%
\pgfsetlinewidth{0.803000pt}%
\definecolor{currentstroke}{rgb}{0.000000,0.000000,0.000000}%
\pgfsetstrokecolor{currentstroke}%
\pgfsetdash{}{0pt}%
\pgfsys@defobject{currentmarker}{\pgfqpoint{0.000000in}{0.000000in}}{\pgfqpoint{0.048611in}{0.000000in}}{%
\pgfpathmoveto{\pgfqpoint{0.000000in}{0.000000in}}%
\pgfpathlineto{\pgfqpoint{0.048611in}{0.000000in}}%
\pgfusepath{stroke,fill}%
}%
\begin{pgfscope}%
\pgfsys@transformshift{7.200000in}{3.612991in}%
\pgfsys@useobject{currentmarker}{}%
\end{pgfscope}%
\end{pgfscope}%
\begin{pgfscope}%
\definecolor{textcolor}{rgb}{0.000000,0.000000,0.000000}%
\pgfsetstrokecolor{textcolor}%
\pgfsetfillcolor{textcolor}%
\pgftext[x=7.297222in, y=3.512971in, left, base]{\color{textcolor}\sffamily\fontsize{20.000000}{24.000000}\selectfont 1.0}%
\end{pgfscope}%
\begin{pgfscope}%
\pgfsetbuttcap%
\pgfsetroundjoin%
\definecolor{currentfill}{rgb}{0.000000,0.000000,0.000000}%
\pgfsetfillcolor{currentfill}%
\pgfsetlinewidth{0.803000pt}%
\definecolor{currentstroke}{rgb}{0.000000,0.000000,0.000000}%
\pgfsetstrokecolor{currentstroke}%
\pgfsetdash{}{0pt}%
\pgfsys@defobject{currentmarker}{\pgfqpoint{0.000000in}{0.000000in}}{\pgfqpoint{0.048611in}{0.000000in}}{%
\pgfpathmoveto{\pgfqpoint{0.000000in}{0.000000in}}%
\pgfpathlineto{\pgfqpoint{0.048611in}{0.000000in}}%
\pgfusepath{stroke,fill}%
}%
\begin{pgfscope}%
\pgfsys@transformshift{7.200000in}{4.229780in}%
\pgfsys@useobject{currentmarker}{}%
\end{pgfscope}%
\end{pgfscope}%
\begin{pgfscope}%
\definecolor{textcolor}{rgb}{0.000000,0.000000,0.000000}%
\pgfsetstrokecolor{textcolor}%
\pgfsetfillcolor{textcolor}%
\pgftext[x=7.297222in, y=4.129760in, left, base]{\color{textcolor}\sffamily\fontsize{20.000000}{24.000000}\selectfont 1.2}%
\end{pgfscope}%
\begin{pgfscope}%
\pgfsetbuttcap%
\pgfsetroundjoin%
\definecolor{currentfill}{rgb}{0.000000,0.000000,0.000000}%
\pgfsetfillcolor{currentfill}%
\pgfsetlinewidth{0.803000pt}%
\definecolor{currentstroke}{rgb}{0.000000,0.000000,0.000000}%
\pgfsetstrokecolor{currentstroke}%
\pgfsetdash{}{0pt}%
\pgfsys@defobject{currentmarker}{\pgfqpoint{0.000000in}{0.000000in}}{\pgfqpoint{0.048611in}{0.000000in}}{%
\pgfpathmoveto{\pgfqpoint{0.000000in}{0.000000in}}%
\pgfpathlineto{\pgfqpoint{0.048611in}{0.000000in}}%
\pgfusepath{stroke,fill}%
}%
\begin{pgfscope}%
\pgfsys@transformshift{7.200000in}{4.846569in}%
\pgfsys@useobject{currentmarker}{}%
\end{pgfscope}%
\end{pgfscope}%
\begin{pgfscope}%
\definecolor{textcolor}{rgb}{0.000000,0.000000,0.000000}%
\pgfsetstrokecolor{textcolor}%
\pgfsetfillcolor{textcolor}%
\pgftext[x=7.297222in, y=4.746550in, left, base]{\color{textcolor}\sffamily\fontsize{20.000000}{24.000000}\selectfont 1.5}%
\end{pgfscope}%
\begin{pgfscope}%
\definecolor{textcolor}{rgb}{0.000000,0.000000,0.000000}%
\pgfsetstrokecolor{textcolor}%
\pgfsetfillcolor{textcolor}%
\pgftext[x=7.698906in,y=3.030000in,,top,rotate=90.000000]{\color{textcolor}\sffamily\fontsize{20.000000}{24.000000}\selectfont \(\displaystyle \mathrm{Charge}\)}%
\end{pgfscope}%
\begin{pgfscope}%
\pgfpathrectangle{\pgfqpoint{1.000000in}{0.720000in}}{\pgfqpoint{6.200000in}{4.620000in}}%
\pgfusepath{clip}%
\pgfsetbuttcap%
\pgfsetroundjoin%
\pgfsetlinewidth{0.501875pt}%
\definecolor{currentstroke}{rgb}{1.000000,0.000000,0.000000}%
\pgfsetstrokecolor{currentstroke}%
\pgfsetdash{}{0pt}%
\pgfpathmoveto{\pgfqpoint{1.837000in}{1.145834in}}%
\pgfpathlineto{\pgfqpoint{1.837000in}{1.478675in}}%
\pgfusepath{stroke}%
\end{pgfscope}%
\begin{pgfscope}%
\pgfpathrectangle{\pgfqpoint{1.000000in}{0.720000in}}{\pgfqpoint{6.200000in}{4.620000in}}%
\pgfusepath{clip}%
\pgfsetbuttcap%
\pgfsetroundjoin%
\pgfsetlinewidth{0.501875pt}%
\definecolor{currentstroke}{rgb}{1.000000,0.000000,0.000000}%
\pgfsetstrokecolor{currentstroke}%
\pgfsetdash{}{0pt}%
\pgfpathmoveto{\pgfqpoint{1.868000in}{1.145834in}}%
\pgfpathlineto{\pgfqpoint{1.868000in}{1.640648in}}%
\pgfusepath{stroke}%
\end{pgfscope}%
\begin{pgfscope}%
\pgfpathrectangle{\pgfqpoint{1.000000in}{0.720000in}}{\pgfqpoint{6.200000in}{4.620000in}}%
\pgfusepath{clip}%
\pgfsetbuttcap%
\pgfsetroundjoin%
\pgfsetlinewidth{0.501875pt}%
\definecolor{currentstroke}{rgb}{1.000000,0.000000,0.000000}%
\pgfsetstrokecolor{currentstroke}%
\pgfsetdash{}{0pt}%
\pgfpathmoveto{\pgfqpoint{1.899000in}{1.145834in}}%
\pgfpathlineto{\pgfqpoint{1.899000in}{1.770765in}}%
\pgfusepath{stroke}%
\end{pgfscope}%
\begin{pgfscope}%
\pgfpathrectangle{\pgfqpoint{1.000000in}{0.720000in}}{\pgfqpoint{6.200000in}{4.620000in}}%
\pgfusepath{clip}%
\pgfsetbuttcap%
\pgfsetroundjoin%
\pgfsetlinewidth{0.501875pt}%
\definecolor{currentstroke}{rgb}{1.000000,0.000000,0.000000}%
\pgfsetstrokecolor{currentstroke}%
\pgfsetdash{}{0pt}%
\pgfpathmoveto{\pgfqpoint{1.930000in}{1.145834in}}%
\pgfpathlineto{\pgfqpoint{1.930000in}{1.849318in}}%
\pgfusepath{stroke}%
\end{pgfscope}%
\begin{pgfscope}%
\pgfpathrectangle{\pgfqpoint{1.000000in}{0.720000in}}{\pgfqpoint{6.200000in}{4.620000in}}%
\pgfusepath{clip}%
\pgfsetbuttcap%
\pgfsetroundjoin%
\pgfsetlinewidth{0.501875pt}%
\definecolor{currentstroke}{rgb}{1.000000,0.000000,0.000000}%
\pgfsetstrokecolor{currentstroke}%
\pgfsetdash{}{0pt}%
\pgfpathmoveto{\pgfqpoint{1.961000in}{1.145834in}}%
\pgfpathlineto{\pgfqpoint{1.961000in}{1.866576in}}%
\pgfusepath{stroke}%
\end{pgfscope}%
\begin{pgfscope}%
\pgfpathrectangle{\pgfqpoint{1.000000in}{0.720000in}}{\pgfqpoint{6.200000in}{4.620000in}}%
\pgfusepath{clip}%
\pgfsetbuttcap%
\pgfsetroundjoin%
\pgfsetlinewidth{0.501875pt}%
\definecolor{currentstroke}{rgb}{1.000000,0.000000,0.000000}%
\pgfsetstrokecolor{currentstroke}%
\pgfsetdash{}{0pt}%
\pgfpathmoveto{\pgfqpoint{1.992000in}{1.145834in}}%
\pgfpathlineto{\pgfqpoint{1.992000in}{1.823220in}}%
\pgfusepath{stroke}%
\end{pgfscope}%
\begin{pgfscope}%
\pgfpathrectangle{\pgfqpoint{1.000000in}{0.720000in}}{\pgfqpoint{6.200000in}{4.620000in}}%
\pgfusepath{clip}%
\pgfsetbuttcap%
\pgfsetroundjoin%
\pgfsetlinewidth{0.501875pt}%
\definecolor{currentstroke}{rgb}{1.000000,0.000000,0.000000}%
\pgfsetstrokecolor{currentstroke}%
\pgfsetdash{}{0pt}%
\pgfpathmoveto{\pgfqpoint{2.023000in}{1.145834in}}%
\pgfpathlineto{\pgfqpoint{2.023000in}{1.728367in}}%
\pgfusepath{stroke}%
\end{pgfscope}%
\begin{pgfscope}%
\pgfpathrectangle{\pgfqpoint{1.000000in}{0.720000in}}{\pgfqpoint{6.200000in}{4.620000in}}%
\pgfusepath{clip}%
\pgfsetbuttcap%
\pgfsetroundjoin%
\pgfsetlinewidth{0.501875pt}%
\definecolor{currentstroke}{rgb}{1.000000,0.000000,0.000000}%
\pgfsetstrokecolor{currentstroke}%
\pgfsetdash{}{0pt}%
\pgfpathmoveto{\pgfqpoint{2.054000in}{1.145834in}}%
\pgfpathlineto{\pgfqpoint{2.054000in}{1.596364in}}%
\pgfusepath{stroke}%
\end{pgfscope}%
\begin{pgfscope}%
\pgfpathrectangle{\pgfqpoint{1.000000in}{0.720000in}}{\pgfqpoint{6.200000in}{4.620000in}}%
\pgfusepath{clip}%
\pgfsetbuttcap%
\pgfsetroundjoin%
\pgfsetlinewidth{0.501875pt}%
\definecolor{currentstroke}{rgb}{1.000000,0.000000,0.000000}%
\pgfsetstrokecolor{currentstroke}%
\pgfsetdash{}{0pt}%
\pgfpathmoveto{\pgfqpoint{2.085000in}{1.145834in}}%
\pgfpathlineto{\pgfqpoint{2.085000in}{1.443710in}}%
\pgfusepath{stroke}%
\end{pgfscope}%
\begin{pgfscope}%
\pgfpathrectangle{\pgfqpoint{1.000000in}{0.720000in}}{\pgfqpoint{6.200000in}{4.620000in}}%
\pgfusepath{clip}%
\pgfsetbuttcap%
\pgfsetroundjoin%
\pgfsetlinewidth{0.501875pt}%
\definecolor{currentstroke}{rgb}{1.000000,0.000000,0.000000}%
\pgfsetstrokecolor{currentstroke}%
\pgfsetdash{}{0pt}%
\pgfpathmoveto{\pgfqpoint{2.364000in}{1.145834in}}%
\pgfpathlineto{\pgfqpoint{2.364000in}{1.422827in}}%
\pgfusepath{stroke}%
\end{pgfscope}%
\begin{pgfscope}%
\pgfpathrectangle{\pgfqpoint{1.000000in}{0.720000in}}{\pgfqpoint{6.200000in}{4.620000in}}%
\pgfusepath{clip}%
\pgfsetbuttcap%
\pgfsetroundjoin%
\pgfsetlinewidth{0.501875pt}%
\definecolor{currentstroke}{rgb}{1.000000,0.000000,0.000000}%
\pgfsetstrokecolor{currentstroke}%
\pgfsetdash{}{0pt}%
\pgfpathmoveto{\pgfqpoint{2.395000in}{1.145834in}}%
\pgfpathlineto{\pgfqpoint{2.395000in}{1.601874in}}%
\pgfusepath{stroke}%
\end{pgfscope}%
\begin{pgfscope}%
\pgfpathrectangle{\pgfqpoint{1.000000in}{0.720000in}}{\pgfqpoint{6.200000in}{4.620000in}}%
\pgfusepath{clip}%
\pgfsetbuttcap%
\pgfsetroundjoin%
\pgfsetlinewidth{0.501875pt}%
\definecolor{currentstroke}{rgb}{1.000000,0.000000,0.000000}%
\pgfsetstrokecolor{currentstroke}%
\pgfsetdash{}{0pt}%
\pgfpathmoveto{\pgfqpoint{2.426000in}{1.145834in}}%
\pgfpathlineto{\pgfqpoint{2.426000in}{1.740223in}}%
\pgfusepath{stroke}%
\end{pgfscope}%
\begin{pgfscope}%
\pgfpathrectangle{\pgfqpoint{1.000000in}{0.720000in}}{\pgfqpoint{6.200000in}{4.620000in}}%
\pgfusepath{clip}%
\pgfsetbuttcap%
\pgfsetroundjoin%
\pgfsetlinewidth{0.501875pt}%
\definecolor{currentstroke}{rgb}{1.000000,0.000000,0.000000}%
\pgfsetstrokecolor{currentstroke}%
\pgfsetdash{}{0pt}%
\pgfpathmoveto{\pgfqpoint{2.457000in}{1.145834in}}%
\pgfpathlineto{\pgfqpoint{2.457000in}{1.807740in}}%
\pgfusepath{stroke}%
\end{pgfscope}%
\begin{pgfscope}%
\pgfpathrectangle{\pgfqpoint{1.000000in}{0.720000in}}{\pgfqpoint{6.200000in}{4.620000in}}%
\pgfusepath{clip}%
\pgfsetbuttcap%
\pgfsetroundjoin%
\pgfsetlinewidth{0.501875pt}%
\definecolor{currentstroke}{rgb}{1.000000,0.000000,0.000000}%
\pgfsetstrokecolor{currentstroke}%
\pgfsetdash{}{0pt}%
\pgfpathmoveto{\pgfqpoint{2.488000in}{1.145834in}}%
\pgfpathlineto{\pgfqpoint{2.488000in}{1.787992in}}%
\pgfusepath{stroke}%
\end{pgfscope}%
\begin{pgfscope}%
\pgfpathrectangle{\pgfqpoint{1.000000in}{0.720000in}}{\pgfqpoint{6.200000in}{4.620000in}}%
\pgfusepath{clip}%
\pgfsetbuttcap%
\pgfsetroundjoin%
\pgfsetlinewidth{0.501875pt}%
\definecolor{currentstroke}{rgb}{1.000000,0.000000,0.000000}%
\pgfsetstrokecolor{currentstroke}%
\pgfsetdash{}{0pt}%
\pgfpathmoveto{\pgfqpoint{2.519000in}{1.145834in}}%
\pgfpathlineto{\pgfqpoint{2.519000in}{1.683912in}}%
\pgfusepath{stroke}%
\end{pgfscope}%
\begin{pgfscope}%
\pgfpathrectangle{\pgfqpoint{1.000000in}{0.720000in}}{\pgfqpoint{6.200000in}{4.620000in}}%
\pgfusepath{clip}%
\pgfsetbuttcap%
\pgfsetroundjoin%
\pgfsetlinewidth{0.501875pt}%
\definecolor{currentstroke}{rgb}{1.000000,0.000000,0.000000}%
\pgfsetstrokecolor{currentstroke}%
\pgfsetdash{}{0pt}%
\pgfpathmoveto{\pgfqpoint{2.550000in}{1.145834in}}%
\pgfpathlineto{\pgfqpoint{2.550000in}{1.518684in}}%
\pgfusepath{stroke}%
\end{pgfscope}%
\begin{pgfscope}%
\pgfpathrectangle{\pgfqpoint{1.000000in}{0.720000in}}{\pgfqpoint{6.200000in}{4.620000in}}%
\pgfusepath{clip}%
\pgfsetbuttcap%
\pgfsetroundjoin%
\pgfsetlinewidth{0.501875pt}%
\definecolor{currentstroke}{rgb}{1.000000,0.000000,0.000000}%
\pgfsetstrokecolor{currentstroke}%
\pgfsetdash{}{0pt}%
\pgfpathmoveto{\pgfqpoint{2.767000in}{1.145834in}}%
\pgfpathlineto{\pgfqpoint{2.767000in}{1.404347in}}%
\pgfusepath{stroke}%
\end{pgfscope}%
\begin{pgfscope}%
\pgfpathrectangle{\pgfqpoint{1.000000in}{0.720000in}}{\pgfqpoint{6.200000in}{4.620000in}}%
\pgfusepath{clip}%
\pgfsetbuttcap%
\pgfsetroundjoin%
\pgfsetlinewidth{0.501875pt}%
\definecolor{currentstroke}{rgb}{1.000000,0.000000,0.000000}%
\pgfsetstrokecolor{currentstroke}%
\pgfsetdash{}{0pt}%
\pgfpathmoveto{\pgfqpoint{2.798000in}{1.145834in}}%
\pgfpathlineto{\pgfqpoint{2.798000in}{1.547646in}}%
\pgfusepath{stroke}%
\end{pgfscope}%
\begin{pgfscope}%
\pgfpathrectangle{\pgfqpoint{1.000000in}{0.720000in}}{\pgfqpoint{6.200000in}{4.620000in}}%
\pgfusepath{clip}%
\pgfsetbuttcap%
\pgfsetroundjoin%
\pgfsetlinewidth{0.501875pt}%
\definecolor{currentstroke}{rgb}{1.000000,0.000000,0.000000}%
\pgfsetstrokecolor{currentstroke}%
\pgfsetdash{}{0pt}%
\pgfpathmoveto{\pgfqpoint{2.829000in}{1.145834in}}%
\pgfpathlineto{\pgfqpoint{2.829000in}{1.637367in}}%
\pgfusepath{stroke}%
\end{pgfscope}%
\begin{pgfscope}%
\pgfpathrectangle{\pgfqpoint{1.000000in}{0.720000in}}{\pgfqpoint{6.200000in}{4.620000in}}%
\pgfusepath{clip}%
\pgfsetbuttcap%
\pgfsetroundjoin%
\pgfsetlinewidth{0.501875pt}%
\definecolor{currentstroke}{rgb}{1.000000,0.000000,0.000000}%
\pgfsetstrokecolor{currentstroke}%
\pgfsetdash{}{0pt}%
\pgfpathmoveto{\pgfqpoint{2.860000in}{1.145834in}}%
\pgfpathlineto{\pgfqpoint{2.860000in}{1.652141in}}%
\pgfusepath{stroke}%
\end{pgfscope}%
\begin{pgfscope}%
\pgfpathrectangle{\pgfqpoint{1.000000in}{0.720000in}}{\pgfqpoint{6.200000in}{4.620000in}}%
\pgfusepath{clip}%
\pgfsetbuttcap%
\pgfsetroundjoin%
\pgfsetlinewidth{0.501875pt}%
\definecolor{currentstroke}{rgb}{1.000000,0.000000,0.000000}%
\pgfsetstrokecolor{currentstroke}%
\pgfsetdash{}{0pt}%
\pgfpathmoveto{\pgfqpoint{2.891000in}{1.145834in}}%
\pgfpathlineto{\pgfqpoint{2.891000in}{1.591142in}}%
\pgfusepath{stroke}%
\end{pgfscope}%
\begin{pgfscope}%
\pgfpathrectangle{\pgfqpoint{1.000000in}{0.720000in}}{\pgfqpoint{6.200000in}{4.620000in}}%
\pgfusepath{clip}%
\pgfsetbuttcap%
\pgfsetroundjoin%
\pgfsetlinewidth{0.501875pt}%
\definecolor{currentstroke}{rgb}{1.000000,0.000000,0.000000}%
\pgfsetstrokecolor{currentstroke}%
\pgfsetdash{}{0pt}%
\pgfpathmoveto{\pgfqpoint{2.922000in}{1.145834in}}%
\pgfpathlineto{\pgfqpoint{2.922000in}{1.472719in}}%
\pgfusepath{stroke}%
\end{pgfscope}%
\begin{pgfscope}%
\pgfpathrectangle{\pgfqpoint{1.000000in}{0.720000in}}{\pgfqpoint{6.200000in}{4.620000in}}%
\pgfusepath{clip}%
\pgfsetbuttcap%
\pgfsetroundjoin%
\pgfsetlinewidth{0.501875pt}%
\definecolor{currentstroke}{rgb}{1.000000,0.000000,0.000000}%
\pgfsetstrokecolor{currentstroke}%
\pgfsetdash{}{0pt}%
\pgfpathmoveto{\pgfqpoint{3.635000in}{1.145834in}}%
\pgfpathlineto{\pgfqpoint{3.635000in}{1.441997in}}%
\pgfusepath{stroke}%
\end{pgfscope}%
\begin{pgfscope}%
\pgfpathrectangle{\pgfqpoint{1.000000in}{0.720000in}}{\pgfqpoint{6.200000in}{4.620000in}}%
\pgfusepath{clip}%
\pgfsetbuttcap%
\pgfsetroundjoin%
\pgfsetlinewidth{0.501875pt}%
\definecolor{currentstroke}{rgb}{1.000000,0.000000,0.000000}%
\pgfsetstrokecolor{currentstroke}%
\pgfsetdash{}{0pt}%
\pgfpathmoveto{\pgfqpoint{3.666000in}{1.145834in}}%
\pgfpathlineto{\pgfqpoint{3.666000in}{1.552701in}}%
\pgfusepath{stroke}%
\end{pgfscope}%
\begin{pgfscope}%
\pgfpathrectangle{\pgfqpoint{1.000000in}{0.720000in}}{\pgfqpoint{6.200000in}{4.620000in}}%
\pgfusepath{clip}%
\pgfsetbuttcap%
\pgfsetroundjoin%
\pgfsetlinewidth{0.501875pt}%
\definecolor{currentstroke}{rgb}{1.000000,0.000000,0.000000}%
\pgfsetstrokecolor{currentstroke}%
\pgfsetdash{}{0pt}%
\pgfpathmoveto{\pgfqpoint{3.697000in}{1.145834in}}%
\pgfpathlineto{\pgfqpoint{3.697000in}{1.632760in}}%
\pgfusepath{stroke}%
\end{pgfscope}%
\begin{pgfscope}%
\pgfpathrectangle{\pgfqpoint{1.000000in}{0.720000in}}{\pgfqpoint{6.200000in}{4.620000in}}%
\pgfusepath{clip}%
\pgfsetbuttcap%
\pgfsetroundjoin%
\pgfsetlinewidth{0.501875pt}%
\definecolor{currentstroke}{rgb}{1.000000,0.000000,0.000000}%
\pgfsetstrokecolor{currentstroke}%
\pgfsetdash{}{0pt}%
\pgfpathmoveto{\pgfqpoint{3.728000in}{1.145834in}}%
\pgfpathlineto{\pgfqpoint{3.728000in}{1.662152in}}%
\pgfusepath{stroke}%
\end{pgfscope}%
\begin{pgfscope}%
\pgfpathrectangle{\pgfqpoint{1.000000in}{0.720000in}}{\pgfqpoint{6.200000in}{4.620000in}}%
\pgfusepath{clip}%
\pgfsetbuttcap%
\pgfsetroundjoin%
\pgfsetlinewidth{0.501875pt}%
\definecolor{currentstroke}{rgb}{1.000000,0.000000,0.000000}%
\pgfsetstrokecolor{currentstroke}%
\pgfsetdash{}{0pt}%
\pgfpathmoveto{\pgfqpoint{3.759000in}{1.145834in}}%
\pgfpathlineto{\pgfqpoint{3.759000in}{1.630553in}}%
\pgfusepath{stroke}%
\end{pgfscope}%
\begin{pgfscope}%
\pgfpathrectangle{\pgfqpoint{1.000000in}{0.720000in}}{\pgfqpoint{6.200000in}{4.620000in}}%
\pgfusepath{clip}%
\pgfsetbuttcap%
\pgfsetroundjoin%
\pgfsetlinewidth{0.501875pt}%
\definecolor{currentstroke}{rgb}{1.000000,0.000000,0.000000}%
\pgfsetstrokecolor{currentstroke}%
\pgfsetdash{}{0pt}%
\pgfpathmoveto{\pgfqpoint{3.790000in}{1.145834in}}%
\pgfpathlineto{\pgfqpoint{3.790000in}{1.541354in}}%
\pgfusepath{stroke}%
\end{pgfscope}%
\begin{pgfscope}%
\pgfpathrectangle{\pgfqpoint{1.000000in}{0.720000in}}{\pgfqpoint{6.200000in}{4.620000in}}%
\pgfusepath{clip}%
\pgfsetbuttcap%
\pgfsetroundjoin%
\pgfsetlinewidth{0.501875pt}%
\definecolor{currentstroke}{rgb}{1.000000,0.000000,0.000000}%
\pgfsetstrokecolor{currentstroke}%
\pgfsetdash{}{0pt}%
\pgfpathmoveto{\pgfqpoint{3.821000in}{1.145834in}}%
\pgfpathlineto{\pgfqpoint{3.821000in}{1.411848in}}%
\pgfusepath{stroke}%
\end{pgfscope}%
\begin{pgfscope}%
\pgfsetrectcap%
\pgfsetmiterjoin%
\pgfsetlinewidth{0.803000pt}%
\definecolor{currentstroke}{rgb}{0.000000,0.000000,0.000000}%
\pgfsetstrokecolor{currentstroke}%
\pgfsetdash{}{0pt}%
\pgfpathmoveto{\pgfqpoint{1.000000in}{0.720000in}}%
\pgfpathlineto{\pgfqpoint{1.000000in}{5.340000in}}%
\pgfusepath{stroke}%
\end{pgfscope}%
\begin{pgfscope}%
\pgfsetrectcap%
\pgfsetmiterjoin%
\pgfsetlinewidth{0.803000pt}%
\definecolor{currentstroke}{rgb}{0.000000,0.000000,0.000000}%
\pgfsetstrokecolor{currentstroke}%
\pgfsetdash{}{0pt}%
\pgfpathmoveto{\pgfqpoint{7.200000in}{0.720000in}}%
\pgfpathlineto{\pgfqpoint{7.200000in}{5.340000in}}%
\pgfusepath{stroke}%
\end{pgfscope}%
\begin{pgfscope}%
\pgfsetrectcap%
\pgfsetmiterjoin%
\pgfsetlinewidth{0.803000pt}%
\definecolor{currentstroke}{rgb}{0.000000,0.000000,0.000000}%
\pgfsetstrokecolor{currentstroke}%
\pgfsetdash{}{0pt}%
\pgfpathmoveto{\pgfqpoint{1.000000in}{0.720000in}}%
\pgfpathlineto{\pgfqpoint{7.200000in}{0.720000in}}%
\pgfusepath{stroke}%
\end{pgfscope}%
\begin{pgfscope}%
\pgfsetrectcap%
\pgfsetmiterjoin%
\pgfsetlinewidth{0.803000pt}%
\definecolor{currentstroke}{rgb}{0.000000,0.000000,0.000000}%
\pgfsetstrokecolor{currentstroke}%
\pgfsetdash{}{0pt}%
\pgfpathmoveto{\pgfqpoint{1.000000in}{5.340000in}}%
\pgfpathlineto{\pgfqpoint{7.200000in}{5.340000in}}%
\pgfusepath{stroke}%
\end{pgfscope}%
\begin{pgfscope}%
\pgfsetbuttcap%
\pgfsetmiterjoin%
\definecolor{currentfill}{rgb}{1.000000,1.000000,1.000000}%
\pgfsetfillcolor{currentfill}%
\pgfsetfillopacity{0.800000}%
\pgfsetlinewidth{1.003750pt}%
\definecolor{currentstroke}{rgb}{0.800000,0.800000,0.800000}%
\pgfsetstrokecolor{currentstroke}%
\pgfsetstrokeopacity{0.800000}%
\pgfsetdash{}{0pt}%
\pgfpathmoveto{\pgfqpoint{4.976872in}{3.932908in}}%
\pgfpathlineto{\pgfqpoint{7.005556in}{3.932908in}}%
\pgfpathquadraticcurveto{\pgfqpoint{7.061111in}{3.932908in}}{\pgfqpoint{7.061111in}{3.988464in}}%
\pgfpathlineto{\pgfqpoint{7.061111in}{5.145556in}}%
\pgfpathquadraticcurveto{\pgfqpoint{7.061111in}{5.201111in}}{\pgfqpoint{7.005556in}{5.201111in}}%
\pgfpathlineto{\pgfqpoint{4.976872in}{5.201111in}}%
\pgfpathquadraticcurveto{\pgfqpoint{4.921317in}{5.201111in}}{\pgfqpoint{4.921317in}{5.145556in}}%
\pgfpathlineto{\pgfqpoint{4.921317in}{3.988464in}}%
\pgfpathquadraticcurveto{\pgfqpoint{4.921317in}{3.932908in}}{\pgfqpoint{4.976872in}{3.932908in}}%
\pgfpathlineto{\pgfqpoint{4.976872in}{3.932908in}}%
\pgfpathclose%
\pgfusepath{stroke,fill}%
\end{pgfscope}%
\begin{pgfscope}%
\pgfsetrectcap%
\pgfsetroundjoin%
\pgfsetlinewidth{2.007500pt}%
\definecolor{currentstroke}{rgb}{0.121569,0.466667,0.705882}%
\pgfsetstrokecolor{currentstroke}%
\pgfsetdash{}{0pt}%
\pgfpathmoveto{\pgfqpoint{5.032428in}{4.987184in}}%
\pgfpathlineto{\pgfqpoint{5.310206in}{4.987184in}}%
\pgfpathlineto{\pgfqpoint{5.587983in}{4.987184in}}%
\pgfusepath{stroke}%
\end{pgfscope}%
\begin{pgfscope}%
\definecolor{textcolor}{rgb}{0.000000,0.000000,0.000000}%
\pgfsetstrokecolor{textcolor}%
\pgfsetfillcolor{textcolor}%
\pgftext[x=5.810206in,y=4.889962in,left,base]{\color{textcolor}\sffamily\fontsize{20.000000}{24.000000}\selectfont Waveform}%
\end{pgfscope}%
\begin{pgfscope}%
\pgfsetbuttcap%
\pgfsetroundjoin%
\pgfsetlinewidth{2.007500pt}%
\definecolor{currentstroke}{rgb}{0.000000,0.500000,0.000000}%
\pgfsetstrokecolor{currentstroke}%
\pgfsetdash{}{0pt}%
\pgfpathmoveto{\pgfqpoint{5.032428in}{4.592227in}}%
\pgfpathlineto{\pgfqpoint{5.587983in}{4.592227in}}%
\pgfusepath{stroke}%
\end{pgfscope}%
\begin{pgfscope}%
\definecolor{textcolor}{rgb}{0.000000,0.000000,0.000000}%
\pgfsetstrokecolor{textcolor}%
\pgfsetfillcolor{textcolor}%
\pgftext[x=5.810206in,y=4.495005in,left,base]{\color{textcolor}\sffamily\fontsize{20.000000}{24.000000}\selectfont Threshold}%
\end{pgfscope}%
\begin{pgfscope}%
\pgfsetbuttcap%
\pgfsetroundjoin%
\pgfsetlinewidth{0.501875pt}%
\definecolor{currentstroke}{rgb}{1.000000,0.000000,0.000000}%
\pgfsetstrokecolor{currentstroke}%
\pgfsetdash{}{0pt}%
\pgfpathmoveto{\pgfqpoint{5.032428in}{4.197271in}}%
\pgfpathlineto{\pgfqpoint{5.587983in}{4.197271in}}%
\pgfusepath{stroke}%
\end{pgfscope}%
\begin{pgfscope}%
\definecolor{textcolor}{rgb}{0.000000,0.000000,0.000000}%
\pgfsetstrokecolor{textcolor}%
\pgfsetfillcolor{textcolor}%
\pgftext[x=5.810206in,y=4.100048in,left,base]{\color{textcolor}\sffamily\fontsize{20.000000}{24.000000}\selectfont Charge}%
\end{pgfscope}%
\end{pgfpicture}%
\makeatother%
\endgroup%
}
    \caption{\label{fig:fd} A Fourier deconvolution example: \\ $\Delta t_0=\SI{-1.16}{ns}$, $\mathrm{RSS}=\SI{124.7}{mV^2}$, $D_\mathrm{w}=\SI{2.03}{ns}$.}
  \end{subfigure}
  \begin{subfigure}{0.5\textwidth}
    \centering
    \resizebox{\textwidth}{!}{%% Creator: Matplotlib, PGF backend
%%
%% To include the figure in your LaTeX document, write
%%   \input{<filename>.pgf}
%%
%% Make sure the required packages are loaded in your preamble
%%   \usepackage{pgf}
%%
%% Also ensure that all the required font packages are loaded; for instance,
%% the lmodern package is sometimes necessary when using math font.
%%   \usepackage{lmodern}
%%
%% Figures using additional raster images can only be included by \input if
%% they are in the same directory as the main LaTeX file. For loading figures
%% from other directories you can use the `import` package
%%   \usepackage{import}
%%
%% and then include the figures with
%%   \import{<path to file>}{<filename>.pgf}
%%
%% Matplotlib used the following preamble
%%   \usepackage[detect-all,locale=DE]{siunitx}
%%
\begingroup%
\makeatletter%
\begin{pgfpicture}%
\pgfpathrectangle{\pgfpointorigin}{\pgfqpoint{8.000000in}{6.000000in}}%
\pgfusepath{use as bounding box, clip}%
\begin{pgfscope}%
\pgfsetbuttcap%
\pgfsetmiterjoin%
\definecolor{currentfill}{rgb}{1.000000,1.000000,1.000000}%
\pgfsetfillcolor{currentfill}%
\pgfsetlinewidth{0.000000pt}%
\definecolor{currentstroke}{rgb}{1.000000,1.000000,1.000000}%
\pgfsetstrokecolor{currentstroke}%
\pgfsetdash{}{0pt}%
\pgfpathmoveto{\pgfqpoint{0.000000in}{0.000000in}}%
\pgfpathlineto{\pgfqpoint{8.000000in}{0.000000in}}%
\pgfpathlineto{\pgfqpoint{8.000000in}{6.000000in}}%
\pgfpathlineto{\pgfqpoint{0.000000in}{6.000000in}}%
\pgfpathlineto{\pgfqpoint{0.000000in}{0.000000in}}%
\pgfpathclose%
\pgfusepath{fill}%
\end{pgfscope}%
\begin{pgfscope}%
\pgfsetbuttcap%
\pgfsetmiterjoin%
\definecolor{currentfill}{rgb}{1.000000,1.000000,1.000000}%
\pgfsetfillcolor{currentfill}%
\pgfsetlinewidth{0.000000pt}%
\definecolor{currentstroke}{rgb}{0.000000,0.000000,0.000000}%
\pgfsetstrokecolor{currentstroke}%
\pgfsetstrokeopacity{0.000000}%
\pgfsetdash{}{0pt}%
\pgfpathmoveto{\pgfqpoint{1.000000in}{0.720000in}}%
\pgfpathlineto{\pgfqpoint{7.200000in}{0.720000in}}%
\pgfpathlineto{\pgfqpoint{7.200000in}{5.340000in}}%
\pgfpathlineto{\pgfqpoint{1.000000in}{5.340000in}}%
\pgfpathlineto{\pgfqpoint{1.000000in}{0.720000in}}%
\pgfpathclose%
\pgfusepath{fill}%
\end{pgfscope}%
\begin{pgfscope}%
\pgfsetbuttcap%
\pgfsetroundjoin%
\definecolor{currentfill}{rgb}{0.000000,0.000000,0.000000}%
\pgfsetfillcolor{currentfill}%
\pgfsetlinewidth{0.803000pt}%
\definecolor{currentstroke}{rgb}{0.000000,0.000000,0.000000}%
\pgfsetstrokecolor{currentstroke}%
\pgfsetdash{}{0pt}%
\pgfsys@defobject{currentmarker}{\pgfqpoint{0.000000in}{-0.048611in}}{\pgfqpoint{0.000000in}{0.000000in}}{%
\pgfpathmoveto{\pgfqpoint{0.000000in}{0.000000in}}%
\pgfpathlineto{\pgfqpoint{0.000000in}{-0.048611in}}%
\pgfusepath{stroke,fill}%
}%
\begin{pgfscope}%
\pgfsys@transformshift{1.310000in}{0.720000in}%
\pgfsys@useobject{currentmarker}{}%
\end{pgfscope}%
\end{pgfscope}%
\begin{pgfscope}%
\definecolor{textcolor}{rgb}{0.000000,0.000000,0.000000}%
\pgfsetstrokecolor{textcolor}%
\pgfsetfillcolor{textcolor}%
\pgftext[x=1.310000in,y=0.622778in,,top]{\color{textcolor}\sffamily\fontsize{20.000000}{24.000000}\selectfont \(\displaystyle {450}\)}%
\end{pgfscope}%
\begin{pgfscope}%
\pgfsetbuttcap%
\pgfsetroundjoin%
\definecolor{currentfill}{rgb}{0.000000,0.000000,0.000000}%
\pgfsetfillcolor{currentfill}%
\pgfsetlinewidth{0.803000pt}%
\definecolor{currentstroke}{rgb}{0.000000,0.000000,0.000000}%
\pgfsetstrokecolor{currentstroke}%
\pgfsetdash{}{0pt}%
\pgfsys@defobject{currentmarker}{\pgfqpoint{0.000000in}{-0.048611in}}{\pgfqpoint{0.000000in}{0.000000in}}{%
\pgfpathmoveto{\pgfqpoint{0.000000in}{0.000000in}}%
\pgfpathlineto{\pgfqpoint{0.000000in}{-0.048611in}}%
\pgfusepath{stroke,fill}%
}%
\begin{pgfscope}%
\pgfsys@transformshift{2.860000in}{0.720000in}%
\pgfsys@useobject{currentmarker}{}%
\end{pgfscope}%
\end{pgfscope}%
\begin{pgfscope}%
\definecolor{textcolor}{rgb}{0.000000,0.000000,0.000000}%
\pgfsetstrokecolor{textcolor}%
\pgfsetfillcolor{textcolor}%
\pgftext[x=2.860000in,y=0.622778in,,top]{\color{textcolor}\sffamily\fontsize{20.000000}{24.000000}\selectfont \(\displaystyle {500}\)}%
\end{pgfscope}%
\begin{pgfscope}%
\pgfsetbuttcap%
\pgfsetroundjoin%
\definecolor{currentfill}{rgb}{0.000000,0.000000,0.000000}%
\pgfsetfillcolor{currentfill}%
\pgfsetlinewidth{0.803000pt}%
\definecolor{currentstroke}{rgb}{0.000000,0.000000,0.000000}%
\pgfsetstrokecolor{currentstroke}%
\pgfsetdash{}{0pt}%
\pgfsys@defobject{currentmarker}{\pgfqpoint{0.000000in}{-0.048611in}}{\pgfqpoint{0.000000in}{0.000000in}}{%
\pgfpathmoveto{\pgfqpoint{0.000000in}{0.000000in}}%
\pgfpathlineto{\pgfqpoint{0.000000in}{-0.048611in}}%
\pgfusepath{stroke,fill}%
}%
\begin{pgfscope}%
\pgfsys@transformshift{4.410000in}{0.720000in}%
\pgfsys@useobject{currentmarker}{}%
\end{pgfscope}%
\end{pgfscope}%
\begin{pgfscope}%
\definecolor{textcolor}{rgb}{0.000000,0.000000,0.000000}%
\pgfsetstrokecolor{textcolor}%
\pgfsetfillcolor{textcolor}%
\pgftext[x=4.410000in,y=0.622778in,,top]{\color{textcolor}\sffamily\fontsize{20.000000}{24.000000}\selectfont \(\displaystyle {550}\)}%
\end{pgfscope}%
\begin{pgfscope}%
\pgfsetbuttcap%
\pgfsetroundjoin%
\definecolor{currentfill}{rgb}{0.000000,0.000000,0.000000}%
\pgfsetfillcolor{currentfill}%
\pgfsetlinewidth{0.803000pt}%
\definecolor{currentstroke}{rgb}{0.000000,0.000000,0.000000}%
\pgfsetstrokecolor{currentstroke}%
\pgfsetdash{}{0pt}%
\pgfsys@defobject{currentmarker}{\pgfqpoint{0.000000in}{-0.048611in}}{\pgfqpoint{0.000000in}{0.000000in}}{%
\pgfpathmoveto{\pgfqpoint{0.000000in}{0.000000in}}%
\pgfpathlineto{\pgfqpoint{0.000000in}{-0.048611in}}%
\pgfusepath{stroke,fill}%
}%
\begin{pgfscope}%
\pgfsys@transformshift{5.960000in}{0.720000in}%
\pgfsys@useobject{currentmarker}{}%
\end{pgfscope}%
\end{pgfscope}%
\begin{pgfscope}%
\definecolor{textcolor}{rgb}{0.000000,0.000000,0.000000}%
\pgfsetstrokecolor{textcolor}%
\pgfsetfillcolor{textcolor}%
\pgftext[x=5.960000in,y=0.622778in,,top]{\color{textcolor}\sffamily\fontsize{20.000000}{24.000000}\selectfont \(\displaystyle {600}\)}%
\end{pgfscope}%
\begin{pgfscope}%
\definecolor{textcolor}{rgb}{0.000000,0.000000,0.000000}%
\pgfsetstrokecolor{textcolor}%
\pgfsetfillcolor{textcolor}%
\pgftext[x=4.100000in,y=0.311155in,,top]{\color{textcolor}\sffamily\fontsize{20.000000}{24.000000}\selectfont \(\displaystyle \mathrm{t}/\si{ns}\)}%
\end{pgfscope}%
\begin{pgfscope}%
\pgfsetbuttcap%
\pgfsetroundjoin%
\definecolor{currentfill}{rgb}{0.000000,0.000000,0.000000}%
\pgfsetfillcolor{currentfill}%
\pgfsetlinewidth{0.803000pt}%
\definecolor{currentstroke}{rgb}{0.000000,0.000000,0.000000}%
\pgfsetstrokecolor{currentstroke}%
\pgfsetdash{}{0pt}%
\pgfsys@defobject{currentmarker}{\pgfqpoint{-0.048611in}{0.000000in}}{\pgfqpoint{-0.000000in}{0.000000in}}{%
\pgfpathmoveto{\pgfqpoint{-0.000000in}{0.000000in}}%
\pgfpathlineto{\pgfqpoint{-0.048611in}{0.000000in}}%
\pgfusepath{stroke,fill}%
}%
\begin{pgfscope}%
\pgfsys@transformshift{1.000000in}{1.145834in}%
\pgfsys@useobject{currentmarker}{}%
\end{pgfscope}%
\end{pgfscope}%
\begin{pgfscope}%
\definecolor{textcolor}{rgb}{0.000000,0.000000,0.000000}%
\pgfsetstrokecolor{textcolor}%
\pgfsetfillcolor{textcolor}%
\pgftext[x=0.770670in, y=1.045815in, left, base]{\color{textcolor}\sffamily\fontsize{20.000000}{24.000000}\selectfont \(\displaystyle {0}\)}%
\end{pgfscope}%
\begin{pgfscope}%
\pgfsetbuttcap%
\pgfsetroundjoin%
\definecolor{currentfill}{rgb}{0.000000,0.000000,0.000000}%
\pgfsetfillcolor{currentfill}%
\pgfsetlinewidth{0.803000pt}%
\definecolor{currentstroke}{rgb}{0.000000,0.000000,0.000000}%
\pgfsetstrokecolor{currentstroke}%
\pgfsetdash{}{0pt}%
\pgfsys@defobject{currentmarker}{\pgfqpoint{-0.048611in}{0.000000in}}{\pgfqpoint{-0.000000in}{0.000000in}}{%
\pgfpathmoveto{\pgfqpoint{-0.000000in}{0.000000in}}%
\pgfpathlineto{\pgfqpoint{-0.048611in}{0.000000in}}%
\pgfusepath{stroke,fill}%
}%
\begin{pgfscope}%
\pgfsys@transformshift{1.000000in}{1.760228in}%
\pgfsys@useobject{currentmarker}{}%
\end{pgfscope}%
\end{pgfscope}%
\begin{pgfscope}%
\definecolor{textcolor}{rgb}{0.000000,0.000000,0.000000}%
\pgfsetstrokecolor{textcolor}%
\pgfsetfillcolor{textcolor}%
\pgftext[x=0.770670in, y=1.660209in, left, base]{\color{textcolor}\sffamily\fontsize{20.000000}{24.000000}\selectfont \(\displaystyle {5}\)}%
\end{pgfscope}%
\begin{pgfscope}%
\pgfsetbuttcap%
\pgfsetroundjoin%
\definecolor{currentfill}{rgb}{0.000000,0.000000,0.000000}%
\pgfsetfillcolor{currentfill}%
\pgfsetlinewidth{0.803000pt}%
\definecolor{currentstroke}{rgb}{0.000000,0.000000,0.000000}%
\pgfsetstrokecolor{currentstroke}%
\pgfsetdash{}{0pt}%
\pgfsys@defobject{currentmarker}{\pgfqpoint{-0.048611in}{0.000000in}}{\pgfqpoint{-0.000000in}{0.000000in}}{%
\pgfpathmoveto{\pgfqpoint{-0.000000in}{0.000000in}}%
\pgfpathlineto{\pgfqpoint{-0.048611in}{0.000000in}}%
\pgfusepath{stroke,fill}%
}%
\begin{pgfscope}%
\pgfsys@transformshift{1.000000in}{2.374622in}%
\pgfsys@useobject{currentmarker}{}%
\end{pgfscope}%
\end{pgfscope}%
\begin{pgfscope}%
\definecolor{textcolor}{rgb}{0.000000,0.000000,0.000000}%
\pgfsetstrokecolor{textcolor}%
\pgfsetfillcolor{textcolor}%
\pgftext[x=0.638563in, y=2.274603in, left, base]{\color{textcolor}\sffamily\fontsize{20.000000}{24.000000}\selectfont \(\displaystyle {10}\)}%
\end{pgfscope}%
\begin{pgfscope}%
\pgfsetbuttcap%
\pgfsetroundjoin%
\definecolor{currentfill}{rgb}{0.000000,0.000000,0.000000}%
\pgfsetfillcolor{currentfill}%
\pgfsetlinewidth{0.803000pt}%
\definecolor{currentstroke}{rgb}{0.000000,0.000000,0.000000}%
\pgfsetstrokecolor{currentstroke}%
\pgfsetdash{}{0pt}%
\pgfsys@defobject{currentmarker}{\pgfqpoint{-0.048611in}{0.000000in}}{\pgfqpoint{-0.000000in}{0.000000in}}{%
\pgfpathmoveto{\pgfqpoint{-0.000000in}{0.000000in}}%
\pgfpathlineto{\pgfqpoint{-0.048611in}{0.000000in}}%
\pgfusepath{stroke,fill}%
}%
\begin{pgfscope}%
\pgfsys@transformshift{1.000000in}{2.989016in}%
\pgfsys@useobject{currentmarker}{}%
\end{pgfscope}%
\end{pgfscope}%
\begin{pgfscope}%
\definecolor{textcolor}{rgb}{0.000000,0.000000,0.000000}%
\pgfsetstrokecolor{textcolor}%
\pgfsetfillcolor{textcolor}%
\pgftext[x=0.638563in, y=2.888996in, left, base]{\color{textcolor}\sffamily\fontsize{20.000000}{24.000000}\selectfont \(\displaystyle {15}\)}%
\end{pgfscope}%
\begin{pgfscope}%
\pgfsetbuttcap%
\pgfsetroundjoin%
\definecolor{currentfill}{rgb}{0.000000,0.000000,0.000000}%
\pgfsetfillcolor{currentfill}%
\pgfsetlinewidth{0.803000pt}%
\definecolor{currentstroke}{rgb}{0.000000,0.000000,0.000000}%
\pgfsetstrokecolor{currentstroke}%
\pgfsetdash{}{0pt}%
\pgfsys@defobject{currentmarker}{\pgfqpoint{-0.048611in}{0.000000in}}{\pgfqpoint{-0.000000in}{0.000000in}}{%
\pgfpathmoveto{\pgfqpoint{-0.000000in}{0.000000in}}%
\pgfpathlineto{\pgfqpoint{-0.048611in}{0.000000in}}%
\pgfusepath{stroke,fill}%
}%
\begin{pgfscope}%
\pgfsys@transformshift{1.000000in}{3.603409in}%
\pgfsys@useobject{currentmarker}{}%
\end{pgfscope}%
\end{pgfscope}%
\begin{pgfscope}%
\definecolor{textcolor}{rgb}{0.000000,0.000000,0.000000}%
\pgfsetstrokecolor{textcolor}%
\pgfsetfillcolor{textcolor}%
\pgftext[x=0.638563in, y=3.503390in, left, base]{\color{textcolor}\sffamily\fontsize{20.000000}{24.000000}\selectfont \(\displaystyle {20}\)}%
\end{pgfscope}%
\begin{pgfscope}%
\pgfsetbuttcap%
\pgfsetroundjoin%
\definecolor{currentfill}{rgb}{0.000000,0.000000,0.000000}%
\pgfsetfillcolor{currentfill}%
\pgfsetlinewidth{0.803000pt}%
\definecolor{currentstroke}{rgb}{0.000000,0.000000,0.000000}%
\pgfsetstrokecolor{currentstroke}%
\pgfsetdash{}{0pt}%
\pgfsys@defobject{currentmarker}{\pgfqpoint{-0.048611in}{0.000000in}}{\pgfqpoint{-0.000000in}{0.000000in}}{%
\pgfpathmoveto{\pgfqpoint{-0.000000in}{0.000000in}}%
\pgfpathlineto{\pgfqpoint{-0.048611in}{0.000000in}}%
\pgfusepath{stroke,fill}%
}%
\begin{pgfscope}%
\pgfsys@transformshift{1.000000in}{4.217803in}%
\pgfsys@useobject{currentmarker}{}%
\end{pgfscope}%
\end{pgfscope}%
\begin{pgfscope}%
\definecolor{textcolor}{rgb}{0.000000,0.000000,0.000000}%
\pgfsetstrokecolor{textcolor}%
\pgfsetfillcolor{textcolor}%
\pgftext[x=0.638563in, y=4.117784in, left, base]{\color{textcolor}\sffamily\fontsize{20.000000}{24.000000}\selectfont \(\displaystyle {25}\)}%
\end{pgfscope}%
\begin{pgfscope}%
\pgfsetbuttcap%
\pgfsetroundjoin%
\definecolor{currentfill}{rgb}{0.000000,0.000000,0.000000}%
\pgfsetfillcolor{currentfill}%
\pgfsetlinewidth{0.803000pt}%
\definecolor{currentstroke}{rgb}{0.000000,0.000000,0.000000}%
\pgfsetstrokecolor{currentstroke}%
\pgfsetdash{}{0pt}%
\pgfsys@defobject{currentmarker}{\pgfqpoint{-0.048611in}{0.000000in}}{\pgfqpoint{-0.000000in}{0.000000in}}{%
\pgfpathmoveto{\pgfqpoint{-0.000000in}{0.000000in}}%
\pgfpathlineto{\pgfqpoint{-0.048611in}{0.000000in}}%
\pgfusepath{stroke,fill}%
}%
\begin{pgfscope}%
\pgfsys@transformshift{1.000000in}{4.832197in}%
\pgfsys@useobject{currentmarker}{}%
\end{pgfscope}%
\end{pgfscope}%
\begin{pgfscope}%
\definecolor{textcolor}{rgb}{0.000000,0.000000,0.000000}%
\pgfsetstrokecolor{textcolor}%
\pgfsetfillcolor{textcolor}%
\pgftext[x=0.638563in, y=4.732177in, left, base]{\color{textcolor}\sffamily\fontsize{20.000000}{24.000000}\selectfont \(\displaystyle {30}\)}%
\end{pgfscope}%
\begin{pgfscope}%
\definecolor{textcolor}{rgb}{0.000000,0.000000,0.000000}%
\pgfsetstrokecolor{textcolor}%
\pgfsetfillcolor{textcolor}%
\pgftext[x=0.583007in,y=3.030000in,,bottom,rotate=90.000000]{\color{textcolor}\sffamily\fontsize{20.000000}{24.000000}\selectfont \(\displaystyle \mathrm{Voltage}/\si{mV}\)}%
\end{pgfscope}%
\begin{pgfscope}%
\pgfpathrectangle{\pgfqpoint{1.000000in}{0.720000in}}{\pgfqpoint{6.200000in}{4.620000in}}%
\pgfusepath{clip}%
\pgfsetrectcap%
\pgfsetroundjoin%
\pgfsetlinewidth{2.007500pt}%
\definecolor{currentstroke}{rgb}{0.121569,0.466667,0.705882}%
\pgfsetstrokecolor{currentstroke}%
\pgfsetdash{}{0pt}%
\pgfpathmoveto{\pgfqpoint{0.990000in}{1.222593in}}%
\pgfpathlineto{\pgfqpoint{1.000000in}{1.270894in}}%
\pgfpathlineto{\pgfqpoint{1.031000in}{1.096684in}}%
\pgfpathlineto{\pgfqpoint{1.062000in}{1.199203in}}%
\pgfpathlineto{\pgfqpoint{1.093000in}{1.182173in}}%
\pgfpathlineto{\pgfqpoint{1.124000in}{1.162905in}}%
\pgfpathlineto{\pgfqpoint{1.155000in}{1.284631in}}%
\pgfpathlineto{\pgfqpoint{1.186000in}{1.177477in}}%
\pgfpathlineto{\pgfqpoint{1.217000in}{1.091844in}}%
\pgfpathlineto{\pgfqpoint{1.248000in}{1.268450in}}%
\pgfpathlineto{\pgfqpoint{1.279000in}{1.178661in}}%
\pgfpathlineto{\pgfqpoint{1.310000in}{1.073030in}}%
\pgfpathlineto{\pgfqpoint{1.341000in}{1.206884in}}%
\pgfpathlineto{\pgfqpoint{1.372000in}{1.124892in}}%
\pgfpathlineto{\pgfqpoint{1.403000in}{1.416265in}}%
\pgfpathlineto{\pgfqpoint{1.434000in}{1.067380in}}%
\pgfpathlineto{\pgfqpoint{1.465000in}{0.989741in}}%
\pgfpathlineto{\pgfqpoint{1.496000in}{1.095559in}}%
\pgfpathlineto{\pgfqpoint{1.527000in}{1.024990in}}%
\pgfpathlineto{\pgfqpoint{1.558000in}{0.991097in}}%
\pgfpathlineto{\pgfqpoint{1.589000in}{1.134677in}}%
\pgfpathlineto{\pgfqpoint{1.620000in}{1.246345in}}%
\pgfpathlineto{\pgfqpoint{1.651000in}{1.216694in}}%
\pgfpathlineto{\pgfqpoint{1.682000in}{1.359832in}}%
\pgfpathlineto{\pgfqpoint{1.713000in}{1.251417in}}%
\pgfpathlineto{\pgfqpoint{1.744000in}{1.282940in}}%
\pgfpathlineto{\pgfqpoint{1.775000in}{1.301399in}}%
\pgfpathlineto{\pgfqpoint{1.806000in}{1.259591in}}%
\pgfpathlineto{\pgfqpoint{1.837000in}{1.121847in}}%
\pgfpathlineto{\pgfqpoint{1.868000in}{0.996067in}}%
\pgfpathlineto{\pgfqpoint{1.899000in}{1.170727in}}%
\pgfpathlineto{\pgfqpoint{1.930000in}{1.108552in}}%
\pgfpathlineto{\pgfqpoint{1.961000in}{1.253788in}}%
\pgfpathlineto{\pgfqpoint{1.992000in}{1.246488in}}%
\pgfpathlineto{\pgfqpoint{2.023000in}{1.523801in}}%
\pgfpathlineto{\pgfqpoint{2.054000in}{1.984247in}}%
\pgfpathlineto{\pgfqpoint{2.085000in}{2.516883in}}%
\pgfpathlineto{\pgfqpoint{2.116000in}{3.445723in}}%
\pgfpathlineto{\pgfqpoint{2.147000in}{3.680523in}}%
\pgfpathlineto{\pgfqpoint{2.178000in}{4.205954in}}%
\pgfpathlineto{\pgfqpoint{2.209000in}{4.372116in}}%
\pgfpathlineto{\pgfqpoint{2.240000in}{4.065688in}}%
\pgfpathlineto{\pgfqpoint{2.271000in}{4.043842in}}%
\pgfpathlineto{\pgfqpoint{2.302000in}{3.941145in}}%
\pgfpathlineto{\pgfqpoint{2.333000in}{3.300069in}}%
\pgfpathlineto{\pgfqpoint{2.364000in}{3.078449in}}%
\pgfpathlineto{\pgfqpoint{2.395000in}{3.064178in}}%
\pgfpathlineto{\pgfqpoint{2.426000in}{2.690044in}}%
\pgfpathlineto{\pgfqpoint{2.457000in}{2.201641in}}%
\pgfpathlineto{\pgfqpoint{2.488000in}{2.085521in}}%
\pgfpathlineto{\pgfqpoint{2.519000in}{1.980297in}}%
\pgfpathlineto{\pgfqpoint{2.550000in}{1.923553in}}%
\pgfpathlineto{\pgfqpoint{2.581000in}{2.204345in}}%
\pgfpathlineto{\pgfqpoint{2.612000in}{2.945116in}}%
\pgfpathlineto{\pgfqpoint{2.643000in}{3.255714in}}%
\pgfpathlineto{\pgfqpoint{2.674000in}{3.764764in}}%
\pgfpathlineto{\pgfqpoint{2.705000in}{3.880315in}}%
\pgfpathlineto{\pgfqpoint{2.736000in}{3.733495in}}%
\pgfpathlineto{\pgfqpoint{2.767000in}{3.523265in}}%
\pgfpathlineto{\pgfqpoint{2.798000in}{3.340267in}}%
\pgfpathlineto{\pgfqpoint{2.829000in}{3.126720in}}%
\pgfpathlineto{\pgfqpoint{2.860000in}{2.975547in}}%
\pgfpathlineto{\pgfqpoint{2.891000in}{2.613938in}}%
\pgfpathlineto{\pgfqpoint{2.922000in}{2.673898in}}%
\pgfpathlineto{\pgfqpoint{2.953000in}{2.415341in}}%
\pgfpathlineto{\pgfqpoint{2.984000in}{3.086620in}}%
\pgfpathlineto{\pgfqpoint{3.015000in}{3.095033in}}%
\pgfpathlineto{\pgfqpoint{3.046000in}{3.509890in}}%
\pgfpathlineto{\pgfqpoint{3.077000in}{3.344637in}}%
\pgfpathlineto{\pgfqpoint{3.108000in}{3.373180in}}%
\pgfpathlineto{\pgfqpoint{3.139000in}{3.271231in}}%
\pgfpathlineto{\pgfqpoint{3.170000in}{2.933447in}}%
\pgfpathlineto{\pgfqpoint{3.201000in}{2.787981in}}%
\pgfpathlineto{\pgfqpoint{3.232000in}{2.438344in}}%
\pgfpathlineto{\pgfqpoint{3.263000in}{2.254036in}}%
\pgfpathlineto{\pgfqpoint{3.294000in}{2.052950in}}%
\pgfpathlineto{\pgfqpoint{3.325000in}{1.772394in}}%
\pgfpathlineto{\pgfqpoint{3.356000in}{1.928720in}}%
\pgfpathlineto{\pgfqpoint{3.387000in}{1.947541in}}%
\pgfpathlineto{\pgfqpoint{3.418000in}{1.654039in}}%
\pgfpathlineto{\pgfqpoint{3.449000in}{1.395126in}}%
\pgfpathlineto{\pgfqpoint{3.480000in}{1.335504in}}%
\pgfpathlineto{\pgfqpoint{3.511000in}{1.608582in}}%
\pgfpathlineto{\pgfqpoint{3.542000in}{1.312849in}}%
\pgfpathlineto{\pgfqpoint{3.573000in}{1.284391in}}%
\pgfpathlineto{\pgfqpoint{3.604000in}{1.447068in}}%
\pgfpathlineto{\pgfqpoint{3.635000in}{1.230849in}}%
\pgfpathlineto{\pgfqpoint{3.666000in}{1.174907in}}%
\pgfpathlineto{\pgfqpoint{3.697000in}{1.262077in}}%
\pgfpathlineto{\pgfqpoint{3.728000in}{1.166534in}}%
\pgfpathlineto{\pgfqpoint{3.759000in}{1.093215in}}%
\pgfpathlineto{\pgfqpoint{3.790000in}{1.433177in}}%
\pgfpathlineto{\pgfqpoint{3.821000in}{1.558773in}}%
\pgfpathlineto{\pgfqpoint{3.852000in}{2.170644in}}%
\pgfpathlineto{\pgfqpoint{3.883000in}{2.453631in}}%
\pgfpathlineto{\pgfqpoint{3.914000in}{3.047367in}}%
\pgfpathlineto{\pgfqpoint{3.945000in}{2.995175in}}%
\pgfpathlineto{\pgfqpoint{3.976000in}{3.368100in}}%
\pgfpathlineto{\pgfqpoint{4.007000in}{3.201769in}}%
\pgfpathlineto{\pgfqpoint{4.038000in}{2.748277in}}%
\pgfpathlineto{\pgfqpoint{4.069000in}{2.746696in}}%
\pgfpathlineto{\pgfqpoint{4.100000in}{2.585518in}}%
\pgfpathlineto{\pgfqpoint{4.131000in}{2.310610in}}%
\pgfpathlineto{\pgfqpoint{4.162000in}{2.080649in}}%
\pgfpathlineto{\pgfqpoint{4.193000in}{2.001216in}}%
\pgfpathlineto{\pgfqpoint{4.224000in}{2.058998in}}%
\pgfpathlineto{\pgfqpoint{4.255000in}{1.975551in}}%
\pgfpathlineto{\pgfqpoint{4.286000in}{1.582643in}}%
\pgfpathlineto{\pgfqpoint{4.317000in}{1.542779in}}%
\pgfpathlineto{\pgfqpoint{4.348000in}{1.470070in}}%
\pgfpathlineto{\pgfqpoint{4.379000in}{1.365768in}}%
\pgfpathlineto{\pgfqpoint{4.410000in}{1.220194in}}%
\pgfpathlineto{\pgfqpoint{4.441000in}{1.444082in}}%
\pgfpathlineto{\pgfqpoint{4.472000in}{1.308676in}}%
\pgfpathlineto{\pgfqpoint{4.534000in}{1.147527in}}%
\pgfpathlineto{\pgfqpoint{4.565000in}{1.061984in}}%
\pgfpathlineto{\pgfqpoint{4.596000in}{1.342557in}}%
\pgfpathlineto{\pgfqpoint{4.627000in}{1.361226in}}%
\pgfpathlineto{\pgfqpoint{4.658000in}{1.132211in}}%
\pgfpathlineto{\pgfqpoint{4.689000in}{1.291251in}}%
\pgfpathlineto{\pgfqpoint{4.720000in}{0.951966in}}%
\pgfpathlineto{\pgfqpoint{4.751000in}{1.137488in}}%
\pgfpathlineto{\pgfqpoint{4.782000in}{1.167291in}}%
\pgfpathlineto{\pgfqpoint{4.813000in}{1.007608in}}%
\pgfpathlineto{\pgfqpoint{4.844000in}{1.082917in}}%
\pgfpathlineto{\pgfqpoint{4.875000in}{0.790972in}}%
\pgfpathlineto{\pgfqpoint{4.906000in}{1.010384in}}%
\pgfpathlineto{\pgfqpoint{4.937000in}{1.043517in}}%
\pgfpathlineto{\pgfqpoint{4.968000in}{0.915696in}}%
\pgfpathlineto{\pgfqpoint{4.999000in}{1.367348in}}%
\pgfpathlineto{\pgfqpoint{5.030000in}{1.217998in}}%
\pgfpathlineto{\pgfqpoint{5.061000in}{1.138635in}}%
\pgfpathlineto{\pgfqpoint{5.092000in}{1.241721in}}%
\pgfpathlineto{\pgfqpoint{5.123000in}{0.959686in}}%
\pgfpathlineto{\pgfqpoint{5.154000in}{1.071188in}}%
\pgfpathlineto{\pgfqpoint{5.185000in}{1.006173in}}%
\pgfpathlineto{\pgfqpoint{5.216000in}{0.910433in}}%
\pgfpathlineto{\pgfqpoint{5.247000in}{1.133809in}}%
\pgfpathlineto{\pgfqpoint{5.278000in}{1.162820in}}%
\pgfpathlineto{\pgfqpoint{5.309000in}{1.258142in}}%
\pgfpathlineto{\pgfqpoint{5.371000in}{1.118255in}}%
\pgfpathlineto{\pgfqpoint{5.402000in}{1.213248in}}%
\pgfpathlineto{\pgfqpoint{5.433000in}{1.243709in}}%
\pgfpathlineto{\pgfqpoint{5.464000in}{1.067614in}}%
\pgfpathlineto{\pgfqpoint{5.495000in}{1.030143in}}%
\pgfpathlineto{\pgfqpoint{5.526000in}{1.069084in}}%
\pgfpathlineto{\pgfqpoint{5.557000in}{1.110206in}}%
\pgfpathlineto{\pgfqpoint{5.588000in}{1.020416in}}%
\pgfpathlineto{\pgfqpoint{5.619000in}{0.896119in}}%
\pgfpathlineto{\pgfqpoint{5.650000in}{1.079030in}}%
\pgfpathlineto{\pgfqpoint{5.681000in}{1.084256in}}%
\pgfpathlineto{\pgfqpoint{5.712000in}{1.115058in}}%
\pgfpathlineto{\pgfqpoint{5.743000in}{1.170423in}}%
\pgfpathlineto{\pgfqpoint{5.774000in}{1.328240in}}%
\pgfpathlineto{\pgfqpoint{5.805000in}{1.132625in}}%
\pgfpathlineto{\pgfqpoint{5.836000in}{1.051242in}}%
\pgfpathlineto{\pgfqpoint{5.867000in}{1.086527in}}%
\pgfpathlineto{\pgfqpoint{5.898000in}{1.290298in}}%
\pgfpathlineto{\pgfqpoint{5.929000in}{1.277211in}}%
\pgfpathlineto{\pgfqpoint{5.960000in}{1.076079in}}%
\pgfpathlineto{\pgfqpoint{5.991000in}{1.035199in}}%
\pgfpathlineto{\pgfqpoint{6.022000in}{1.106868in}}%
\pgfpathlineto{\pgfqpoint{6.053000in}{1.008838in}}%
\pgfpathlineto{\pgfqpoint{6.084000in}{1.039905in}}%
\pgfpathlineto{\pgfqpoint{6.115000in}{1.120884in}}%
\pgfpathlineto{\pgfqpoint{6.146000in}{1.167014in}}%
\pgfpathlineto{\pgfqpoint{6.177000in}{1.099064in}}%
\pgfpathlineto{\pgfqpoint{6.208000in}{1.219160in}}%
\pgfpathlineto{\pgfqpoint{6.239000in}{1.030414in}}%
\pgfpathlineto{\pgfqpoint{6.270000in}{1.253325in}}%
\pgfpathlineto{\pgfqpoint{6.301000in}{0.980623in}}%
\pgfpathlineto{\pgfqpoint{6.332000in}{1.059867in}}%
\pgfpathlineto{\pgfqpoint{6.363000in}{1.084611in}}%
\pgfpathlineto{\pgfqpoint{6.394000in}{1.076653in}}%
\pgfpathlineto{\pgfqpoint{6.425000in}{1.336888in}}%
\pgfpathlineto{\pgfqpoint{6.456000in}{1.165206in}}%
\pgfpathlineto{\pgfqpoint{6.487000in}{1.270963in}}%
\pgfpathlineto{\pgfqpoint{6.518000in}{1.209477in}}%
\pgfpathlineto{\pgfqpoint{6.549000in}{1.219259in}}%
\pgfpathlineto{\pgfqpoint{6.580000in}{1.158620in}}%
\pgfpathlineto{\pgfqpoint{6.611000in}{1.040458in}}%
\pgfpathlineto{\pgfqpoint{6.642000in}{1.223128in}}%
\pgfpathlineto{\pgfqpoint{6.673000in}{1.099591in}}%
\pgfpathlineto{\pgfqpoint{6.704000in}{0.938364in}}%
\pgfpathlineto{\pgfqpoint{6.735000in}{1.142117in}}%
\pgfpathlineto{\pgfqpoint{6.766000in}{1.261119in}}%
\pgfpathlineto{\pgfqpoint{6.797000in}{1.222967in}}%
\pgfpathlineto{\pgfqpoint{6.828000in}{1.210582in}}%
\pgfpathlineto{\pgfqpoint{6.859000in}{0.926154in}}%
\pgfpathlineto{\pgfqpoint{6.890000in}{1.159474in}}%
\pgfpathlineto{\pgfqpoint{6.921000in}{0.851001in}}%
\pgfpathlineto{\pgfqpoint{6.952000in}{1.159442in}}%
\pgfpathlineto{\pgfqpoint{6.983000in}{0.993737in}}%
\pgfpathlineto{\pgfqpoint{7.014000in}{1.132429in}}%
\pgfpathlineto{\pgfqpoint{7.045000in}{1.167750in}}%
\pgfpathlineto{\pgfqpoint{7.076000in}{1.054561in}}%
\pgfpathlineto{\pgfqpoint{7.107000in}{1.039172in}}%
\pgfpathlineto{\pgfqpoint{7.138000in}{1.226791in}}%
\pgfpathlineto{\pgfqpoint{7.169000in}{1.370284in}}%
\pgfpathlineto{\pgfqpoint{7.200000in}{1.225372in}}%
\pgfpathlineto{\pgfqpoint{7.210000in}{1.140349in}}%
\pgfpathlineto{\pgfqpoint{7.210000in}{1.140349in}}%
\pgfusepath{stroke}%
\end{pgfscope}%
\begin{pgfscope}%
\pgfpathrectangle{\pgfqpoint{1.000000in}{0.720000in}}{\pgfqpoint{6.200000in}{4.620000in}}%
\pgfusepath{clip}%
\pgfsetbuttcap%
\pgfsetroundjoin%
\pgfsetlinewidth{2.007500pt}%
\definecolor{currentstroke}{rgb}{0.000000,0.500000,0.000000}%
\pgfsetstrokecolor{currentstroke}%
\pgfsetdash{}{0pt}%
\pgfpathmoveto{\pgfqpoint{0.990000in}{1.760228in}}%
\pgfpathlineto{\pgfqpoint{7.210000in}{1.760228in}}%
\pgfusepath{stroke}%
\end{pgfscope}%
\begin{pgfscope}%
\pgfsetrectcap%
\pgfsetmiterjoin%
\pgfsetlinewidth{0.803000pt}%
\definecolor{currentstroke}{rgb}{0.000000,0.000000,0.000000}%
\pgfsetstrokecolor{currentstroke}%
\pgfsetdash{}{0pt}%
\pgfpathmoveto{\pgfqpoint{1.000000in}{0.720000in}}%
\pgfpathlineto{\pgfqpoint{1.000000in}{5.340000in}}%
\pgfusepath{stroke}%
\end{pgfscope}%
\begin{pgfscope}%
\pgfsetrectcap%
\pgfsetmiterjoin%
\pgfsetlinewidth{0.803000pt}%
\definecolor{currentstroke}{rgb}{0.000000,0.000000,0.000000}%
\pgfsetstrokecolor{currentstroke}%
\pgfsetdash{}{0pt}%
\pgfpathmoveto{\pgfqpoint{7.200000in}{0.720000in}}%
\pgfpathlineto{\pgfqpoint{7.200000in}{5.340000in}}%
\pgfusepath{stroke}%
\end{pgfscope}%
\begin{pgfscope}%
\pgfsetrectcap%
\pgfsetmiterjoin%
\pgfsetlinewidth{0.803000pt}%
\definecolor{currentstroke}{rgb}{0.000000,0.000000,0.000000}%
\pgfsetstrokecolor{currentstroke}%
\pgfsetdash{}{0pt}%
\pgfpathmoveto{\pgfqpoint{1.000000in}{0.720000in}}%
\pgfpathlineto{\pgfqpoint{7.200000in}{0.720000in}}%
\pgfusepath{stroke}%
\end{pgfscope}%
\begin{pgfscope}%
\pgfsetrectcap%
\pgfsetmiterjoin%
\pgfsetlinewidth{0.803000pt}%
\definecolor{currentstroke}{rgb}{0.000000,0.000000,0.000000}%
\pgfsetstrokecolor{currentstroke}%
\pgfsetdash{}{0pt}%
\pgfpathmoveto{\pgfqpoint{1.000000in}{5.340000in}}%
\pgfpathlineto{\pgfqpoint{7.200000in}{5.340000in}}%
\pgfusepath{stroke}%
\end{pgfscope}%
\begin{pgfscope}%
\pgfsetbuttcap%
\pgfsetroundjoin%
\definecolor{currentfill}{rgb}{0.000000,0.000000,0.000000}%
\pgfsetfillcolor{currentfill}%
\pgfsetlinewidth{0.803000pt}%
\definecolor{currentstroke}{rgb}{0.000000,0.000000,0.000000}%
\pgfsetstrokecolor{currentstroke}%
\pgfsetdash{}{0pt}%
\pgfsys@defobject{currentmarker}{\pgfqpoint{0.000000in}{0.000000in}}{\pgfqpoint{0.048611in}{0.000000in}}{%
\pgfpathmoveto{\pgfqpoint{0.000000in}{0.000000in}}%
\pgfpathlineto{\pgfqpoint{0.048611in}{0.000000in}}%
\pgfusepath{stroke,fill}%
}%
\begin{pgfscope}%
\pgfsys@transformshift{7.200000in}{1.145834in}%
\pgfsys@useobject{currentmarker}{}%
\end{pgfscope}%
\end{pgfscope}%
\begin{pgfscope}%
\definecolor{textcolor}{rgb}{0.000000,0.000000,0.000000}%
\pgfsetstrokecolor{textcolor}%
\pgfsetfillcolor{textcolor}%
\pgftext[x=7.297222in, y=1.045815in, left, base]{\color{textcolor}\sffamily\fontsize{20.000000}{24.000000}\selectfont 0.0}%
\end{pgfscope}%
\begin{pgfscope}%
\pgfsetbuttcap%
\pgfsetroundjoin%
\definecolor{currentfill}{rgb}{0.000000,0.000000,0.000000}%
\pgfsetfillcolor{currentfill}%
\pgfsetlinewidth{0.803000pt}%
\definecolor{currentstroke}{rgb}{0.000000,0.000000,0.000000}%
\pgfsetstrokecolor{currentstroke}%
\pgfsetdash{}{0pt}%
\pgfsys@defobject{currentmarker}{\pgfqpoint{0.000000in}{0.000000in}}{\pgfqpoint{0.048611in}{0.000000in}}{%
\pgfpathmoveto{\pgfqpoint{0.000000in}{0.000000in}}%
\pgfpathlineto{\pgfqpoint{0.048611in}{0.000000in}}%
\pgfusepath{stroke,fill}%
}%
\begin{pgfscope}%
\pgfsys@transformshift{7.200000in}{1.762623in}%
\pgfsys@useobject{currentmarker}{}%
\end{pgfscope}%
\end{pgfscope}%
\begin{pgfscope}%
\definecolor{textcolor}{rgb}{0.000000,0.000000,0.000000}%
\pgfsetstrokecolor{textcolor}%
\pgfsetfillcolor{textcolor}%
\pgftext[x=7.297222in, y=1.662604in, left, base]{\color{textcolor}\sffamily\fontsize{20.000000}{24.000000}\selectfont 0.2}%
\end{pgfscope}%
\begin{pgfscope}%
\pgfsetbuttcap%
\pgfsetroundjoin%
\definecolor{currentfill}{rgb}{0.000000,0.000000,0.000000}%
\pgfsetfillcolor{currentfill}%
\pgfsetlinewidth{0.803000pt}%
\definecolor{currentstroke}{rgb}{0.000000,0.000000,0.000000}%
\pgfsetstrokecolor{currentstroke}%
\pgfsetdash{}{0pt}%
\pgfsys@defobject{currentmarker}{\pgfqpoint{0.000000in}{0.000000in}}{\pgfqpoint{0.048611in}{0.000000in}}{%
\pgfpathmoveto{\pgfqpoint{0.000000in}{0.000000in}}%
\pgfpathlineto{\pgfqpoint{0.048611in}{0.000000in}}%
\pgfusepath{stroke,fill}%
}%
\begin{pgfscope}%
\pgfsys@transformshift{7.200000in}{2.379413in}%
\pgfsys@useobject{currentmarker}{}%
\end{pgfscope}%
\end{pgfscope}%
\begin{pgfscope}%
\definecolor{textcolor}{rgb}{0.000000,0.000000,0.000000}%
\pgfsetstrokecolor{textcolor}%
\pgfsetfillcolor{textcolor}%
\pgftext[x=7.297222in, y=2.279393in, left, base]{\color{textcolor}\sffamily\fontsize{20.000000}{24.000000}\selectfont 0.5}%
\end{pgfscope}%
\begin{pgfscope}%
\pgfsetbuttcap%
\pgfsetroundjoin%
\definecolor{currentfill}{rgb}{0.000000,0.000000,0.000000}%
\pgfsetfillcolor{currentfill}%
\pgfsetlinewidth{0.803000pt}%
\definecolor{currentstroke}{rgb}{0.000000,0.000000,0.000000}%
\pgfsetstrokecolor{currentstroke}%
\pgfsetdash{}{0pt}%
\pgfsys@defobject{currentmarker}{\pgfqpoint{0.000000in}{0.000000in}}{\pgfqpoint{0.048611in}{0.000000in}}{%
\pgfpathmoveto{\pgfqpoint{0.000000in}{0.000000in}}%
\pgfpathlineto{\pgfqpoint{0.048611in}{0.000000in}}%
\pgfusepath{stroke,fill}%
}%
\begin{pgfscope}%
\pgfsys@transformshift{7.200000in}{2.996202in}%
\pgfsys@useobject{currentmarker}{}%
\end{pgfscope}%
\end{pgfscope}%
\begin{pgfscope}%
\definecolor{textcolor}{rgb}{0.000000,0.000000,0.000000}%
\pgfsetstrokecolor{textcolor}%
\pgfsetfillcolor{textcolor}%
\pgftext[x=7.297222in, y=2.896182in, left, base]{\color{textcolor}\sffamily\fontsize{20.000000}{24.000000}\selectfont 0.8}%
\end{pgfscope}%
\begin{pgfscope}%
\pgfsetbuttcap%
\pgfsetroundjoin%
\definecolor{currentfill}{rgb}{0.000000,0.000000,0.000000}%
\pgfsetfillcolor{currentfill}%
\pgfsetlinewidth{0.803000pt}%
\definecolor{currentstroke}{rgb}{0.000000,0.000000,0.000000}%
\pgfsetstrokecolor{currentstroke}%
\pgfsetdash{}{0pt}%
\pgfsys@defobject{currentmarker}{\pgfqpoint{0.000000in}{0.000000in}}{\pgfqpoint{0.048611in}{0.000000in}}{%
\pgfpathmoveto{\pgfqpoint{0.000000in}{0.000000in}}%
\pgfpathlineto{\pgfqpoint{0.048611in}{0.000000in}}%
\pgfusepath{stroke,fill}%
}%
\begin{pgfscope}%
\pgfsys@transformshift{7.200000in}{3.612991in}%
\pgfsys@useobject{currentmarker}{}%
\end{pgfscope}%
\end{pgfscope}%
\begin{pgfscope}%
\definecolor{textcolor}{rgb}{0.000000,0.000000,0.000000}%
\pgfsetstrokecolor{textcolor}%
\pgfsetfillcolor{textcolor}%
\pgftext[x=7.297222in, y=3.512971in, left, base]{\color{textcolor}\sffamily\fontsize{20.000000}{24.000000}\selectfont 1.0}%
\end{pgfscope}%
\begin{pgfscope}%
\pgfsetbuttcap%
\pgfsetroundjoin%
\definecolor{currentfill}{rgb}{0.000000,0.000000,0.000000}%
\pgfsetfillcolor{currentfill}%
\pgfsetlinewidth{0.803000pt}%
\definecolor{currentstroke}{rgb}{0.000000,0.000000,0.000000}%
\pgfsetstrokecolor{currentstroke}%
\pgfsetdash{}{0pt}%
\pgfsys@defobject{currentmarker}{\pgfqpoint{0.000000in}{0.000000in}}{\pgfqpoint{0.048611in}{0.000000in}}{%
\pgfpathmoveto{\pgfqpoint{0.000000in}{0.000000in}}%
\pgfpathlineto{\pgfqpoint{0.048611in}{0.000000in}}%
\pgfusepath{stroke,fill}%
}%
\begin{pgfscope}%
\pgfsys@transformshift{7.200000in}{4.229780in}%
\pgfsys@useobject{currentmarker}{}%
\end{pgfscope}%
\end{pgfscope}%
\begin{pgfscope}%
\definecolor{textcolor}{rgb}{0.000000,0.000000,0.000000}%
\pgfsetstrokecolor{textcolor}%
\pgfsetfillcolor{textcolor}%
\pgftext[x=7.297222in, y=4.129760in, left, base]{\color{textcolor}\sffamily\fontsize{20.000000}{24.000000}\selectfont 1.2}%
\end{pgfscope}%
\begin{pgfscope}%
\pgfsetbuttcap%
\pgfsetroundjoin%
\definecolor{currentfill}{rgb}{0.000000,0.000000,0.000000}%
\pgfsetfillcolor{currentfill}%
\pgfsetlinewidth{0.803000pt}%
\definecolor{currentstroke}{rgb}{0.000000,0.000000,0.000000}%
\pgfsetstrokecolor{currentstroke}%
\pgfsetdash{}{0pt}%
\pgfsys@defobject{currentmarker}{\pgfqpoint{0.000000in}{0.000000in}}{\pgfqpoint{0.048611in}{0.000000in}}{%
\pgfpathmoveto{\pgfqpoint{0.000000in}{0.000000in}}%
\pgfpathlineto{\pgfqpoint{0.048611in}{0.000000in}}%
\pgfusepath{stroke,fill}%
}%
\begin{pgfscope}%
\pgfsys@transformshift{7.200000in}{4.846569in}%
\pgfsys@useobject{currentmarker}{}%
\end{pgfscope}%
\end{pgfscope}%
\begin{pgfscope}%
\definecolor{textcolor}{rgb}{0.000000,0.000000,0.000000}%
\pgfsetstrokecolor{textcolor}%
\pgfsetfillcolor{textcolor}%
\pgftext[x=7.297222in, y=4.746550in, left, base]{\color{textcolor}\sffamily\fontsize{20.000000}{24.000000}\selectfont 1.5}%
\end{pgfscope}%
\begin{pgfscope}%
\definecolor{textcolor}{rgb}{0.000000,0.000000,0.000000}%
\pgfsetstrokecolor{textcolor}%
\pgfsetfillcolor{textcolor}%
\pgftext[x=7.698906in,y=3.030000in,,top,rotate=90.000000]{\color{textcolor}\sffamily\fontsize{20.000000}{24.000000}\selectfont \(\displaystyle \mathrm{Charge}\)}%
\end{pgfscope}%
\begin{pgfscope}%
\pgfpathrectangle{\pgfqpoint{1.000000in}{0.720000in}}{\pgfqpoint{6.200000in}{4.620000in}}%
\pgfusepath{clip}%
\pgfsetbuttcap%
\pgfsetroundjoin%
\pgfsetlinewidth{0.501875pt}%
\definecolor{currentstroke}{rgb}{1.000000,0.000000,0.000000}%
\pgfsetstrokecolor{currentstroke}%
\pgfsetdash{}{0pt}%
\pgfpathmoveto{\pgfqpoint{1.899000in}{1.145834in}}%
\pgfpathlineto{\pgfqpoint{1.899000in}{2.266516in}}%
\pgfusepath{stroke}%
\end{pgfscope}%
\begin{pgfscope}%
\pgfpathrectangle{\pgfqpoint{1.000000in}{0.720000in}}{\pgfqpoint{6.200000in}{4.620000in}}%
\pgfusepath{clip}%
\pgfsetbuttcap%
\pgfsetroundjoin%
\pgfsetlinewidth{0.501875pt}%
\definecolor{currentstroke}{rgb}{1.000000,0.000000,0.000000}%
\pgfsetstrokecolor{currentstroke}%
\pgfsetdash{}{0pt}%
\pgfpathmoveto{\pgfqpoint{1.961000in}{1.145834in}}%
\pgfpathlineto{\pgfqpoint{1.961000in}{4.051816in}}%
\pgfusepath{stroke}%
\end{pgfscope}%
\begin{pgfscope}%
\pgfpathrectangle{\pgfqpoint{1.000000in}{0.720000in}}{\pgfqpoint{6.200000in}{4.620000in}}%
\pgfusepath{clip}%
\pgfsetbuttcap%
\pgfsetroundjoin%
\pgfsetlinewidth{0.501875pt}%
\definecolor{currentstroke}{rgb}{1.000000,0.000000,0.000000}%
\pgfsetstrokecolor{currentstroke}%
\pgfsetdash{}{0pt}%
\pgfpathmoveto{\pgfqpoint{1.992000in}{1.145834in}}%
\pgfpathlineto{\pgfqpoint{1.992000in}{1.681152in}}%
\pgfusepath{stroke}%
\end{pgfscope}%
\begin{pgfscope}%
\pgfpathrectangle{\pgfqpoint{1.000000in}{0.720000in}}{\pgfqpoint{6.200000in}{4.620000in}}%
\pgfusepath{clip}%
\pgfsetbuttcap%
\pgfsetroundjoin%
\pgfsetlinewidth{0.501875pt}%
\definecolor{currentstroke}{rgb}{1.000000,0.000000,0.000000}%
\pgfsetstrokecolor{currentstroke}%
\pgfsetdash{}{0pt}%
\pgfpathmoveto{\pgfqpoint{2.457000in}{1.145834in}}%
\pgfpathlineto{\pgfqpoint{2.457000in}{2.562393in}}%
\pgfusepath{stroke}%
\end{pgfscope}%
\begin{pgfscope}%
\pgfpathrectangle{\pgfqpoint{1.000000in}{0.720000in}}{\pgfqpoint{6.200000in}{4.620000in}}%
\pgfusepath{clip}%
\pgfsetbuttcap%
\pgfsetroundjoin%
\pgfsetlinewidth{0.501875pt}%
\definecolor{currentstroke}{rgb}{1.000000,0.000000,0.000000}%
\pgfsetstrokecolor{currentstroke}%
\pgfsetdash{}{0pt}%
\pgfpathmoveto{\pgfqpoint{2.488000in}{1.145834in}}%
\pgfpathlineto{\pgfqpoint{2.488000in}{3.248536in}}%
\pgfusepath{stroke}%
\end{pgfscope}%
\begin{pgfscope}%
\pgfpathrectangle{\pgfqpoint{1.000000in}{0.720000in}}{\pgfqpoint{6.200000in}{4.620000in}}%
\pgfusepath{clip}%
\pgfsetbuttcap%
\pgfsetroundjoin%
\pgfsetlinewidth{0.501875pt}%
\definecolor{currentstroke}{rgb}{1.000000,0.000000,0.000000}%
\pgfsetstrokecolor{currentstroke}%
\pgfsetdash{}{0pt}%
\pgfpathmoveto{\pgfqpoint{2.829000in}{1.145834in}}%
\pgfpathlineto{\pgfqpoint{2.829000in}{2.394289in}}%
\pgfusepath{stroke}%
\end{pgfscope}%
\begin{pgfscope}%
\pgfpathrectangle{\pgfqpoint{1.000000in}{0.720000in}}{\pgfqpoint{6.200000in}{4.620000in}}%
\pgfusepath{clip}%
\pgfsetbuttcap%
\pgfsetroundjoin%
\pgfsetlinewidth{0.501875pt}%
\definecolor{currentstroke}{rgb}{1.000000,0.000000,0.000000}%
\pgfsetstrokecolor{currentstroke}%
\pgfsetdash{}{0pt}%
\pgfpathmoveto{\pgfqpoint{2.860000in}{1.145834in}}%
\pgfpathlineto{\pgfqpoint{2.860000in}{2.337619in}}%
\pgfusepath{stroke}%
\end{pgfscope}%
\begin{pgfscope}%
\pgfpathrectangle{\pgfqpoint{1.000000in}{0.720000in}}{\pgfqpoint{6.200000in}{4.620000in}}%
\pgfusepath{clip}%
\pgfsetbuttcap%
\pgfsetroundjoin%
\pgfsetlinewidth{0.501875pt}%
\definecolor{currentstroke}{rgb}{1.000000,0.000000,0.000000}%
\pgfsetstrokecolor{currentstroke}%
\pgfsetdash{}{0pt}%
\pgfpathmoveto{\pgfqpoint{3.697000in}{1.145834in}}%
\pgfpathlineto{\pgfqpoint{3.697000in}{2.265162in}}%
\pgfusepath{stroke}%
\end{pgfscope}%
\begin{pgfscope}%
\pgfpathrectangle{\pgfqpoint{1.000000in}{0.720000in}}{\pgfqpoint{6.200000in}{4.620000in}}%
\pgfusepath{clip}%
\pgfsetbuttcap%
\pgfsetroundjoin%
\pgfsetlinewidth{0.501875pt}%
\definecolor{currentstroke}{rgb}{1.000000,0.000000,0.000000}%
\pgfsetstrokecolor{currentstroke}%
\pgfsetdash{}{0pt}%
\pgfpathmoveto{\pgfqpoint{3.728000in}{1.145834in}}%
\pgfpathlineto{\pgfqpoint{3.728000in}{2.931684in}}%
\pgfusepath{stroke}%
\end{pgfscope}%
\begin{pgfscope}%
\pgfsetrectcap%
\pgfsetmiterjoin%
\pgfsetlinewidth{0.803000pt}%
\definecolor{currentstroke}{rgb}{0.000000,0.000000,0.000000}%
\pgfsetstrokecolor{currentstroke}%
\pgfsetdash{}{0pt}%
\pgfpathmoveto{\pgfqpoint{1.000000in}{0.720000in}}%
\pgfpathlineto{\pgfqpoint{1.000000in}{5.340000in}}%
\pgfusepath{stroke}%
\end{pgfscope}%
\begin{pgfscope}%
\pgfsetrectcap%
\pgfsetmiterjoin%
\pgfsetlinewidth{0.803000pt}%
\definecolor{currentstroke}{rgb}{0.000000,0.000000,0.000000}%
\pgfsetstrokecolor{currentstroke}%
\pgfsetdash{}{0pt}%
\pgfpathmoveto{\pgfqpoint{7.200000in}{0.720000in}}%
\pgfpathlineto{\pgfqpoint{7.200000in}{5.340000in}}%
\pgfusepath{stroke}%
\end{pgfscope}%
\begin{pgfscope}%
\pgfsetrectcap%
\pgfsetmiterjoin%
\pgfsetlinewidth{0.803000pt}%
\definecolor{currentstroke}{rgb}{0.000000,0.000000,0.000000}%
\pgfsetstrokecolor{currentstroke}%
\pgfsetdash{}{0pt}%
\pgfpathmoveto{\pgfqpoint{1.000000in}{0.720000in}}%
\pgfpathlineto{\pgfqpoint{7.200000in}{0.720000in}}%
\pgfusepath{stroke}%
\end{pgfscope}%
\begin{pgfscope}%
\pgfsetrectcap%
\pgfsetmiterjoin%
\pgfsetlinewidth{0.803000pt}%
\definecolor{currentstroke}{rgb}{0.000000,0.000000,0.000000}%
\pgfsetstrokecolor{currentstroke}%
\pgfsetdash{}{0pt}%
\pgfpathmoveto{\pgfqpoint{1.000000in}{5.340000in}}%
\pgfpathlineto{\pgfqpoint{7.200000in}{5.340000in}}%
\pgfusepath{stroke}%
\end{pgfscope}%
\begin{pgfscope}%
\pgfsetbuttcap%
\pgfsetmiterjoin%
\definecolor{currentfill}{rgb}{1.000000,1.000000,1.000000}%
\pgfsetfillcolor{currentfill}%
\pgfsetfillopacity{0.800000}%
\pgfsetlinewidth{1.003750pt}%
\definecolor{currentstroke}{rgb}{0.800000,0.800000,0.800000}%
\pgfsetstrokecolor{currentstroke}%
\pgfsetstrokeopacity{0.800000}%
\pgfsetdash{}{0pt}%
\pgfpathmoveto{\pgfqpoint{4.976872in}{3.932908in}}%
\pgfpathlineto{\pgfqpoint{7.005556in}{3.932908in}}%
\pgfpathquadraticcurveto{\pgfqpoint{7.061111in}{3.932908in}}{\pgfqpoint{7.061111in}{3.988464in}}%
\pgfpathlineto{\pgfqpoint{7.061111in}{5.145556in}}%
\pgfpathquadraticcurveto{\pgfqpoint{7.061111in}{5.201111in}}{\pgfqpoint{7.005556in}{5.201111in}}%
\pgfpathlineto{\pgfqpoint{4.976872in}{5.201111in}}%
\pgfpathquadraticcurveto{\pgfqpoint{4.921317in}{5.201111in}}{\pgfqpoint{4.921317in}{5.145556in}}%
\pgfpathlineto{\pgfqpoint{4.921317in}{3.988464in}}%
\pgfpathquadraticcurveto{\pgfqpoint{4.921317in}{3.932908in}}{\pgfqpoint{4.976872in}{3.932908in}}%
\pgfpathlineto{\pgfqpoint{4.976872in}{3.932908in}}%
\pgfpathclose%
\pgfusepath{stroke,fill}%
\end{pgfscope}%
\begin{pgfscope}%
\pgfsetrectcap%
\pgfsetroundjoin%
\pgfsetlinewidth{2.007500pt}%
\definecolor{currentstroke}{rgb}{0.121569,0.466667,0.705882}%
\pgfsetstrokecolor{currentstroke}%
\pgfsetdash{}{0pt}%
\pgfpathmoveto{\pgfqpoint{5.032428in}{4.987184in}}%
\pgfpathlineto{\pgfqpoint{5.310206in}{4.987184in}}%
\pgfpathlineto{\pgfqpoint{5.587983in}{4.987184in}}%
\pgfusepath{stroke}%
\end{pgfscope}%
\begin{pgfscope}%
\definecolor{textcolor}{rgb}{0.000000,0.000000,0.000000}%
\pgfsetstrokecolor{textcolor}%
\pgfsetfillcolor{textcolor}%
\pgftext[x=5.810206in,y=4.889962in,left,base]{\color{textcolor}\sffamily\fontsize{20.000000}{24.000000}\selectfont Waveform}%
\end{pgfscope}%
\begin{pgfscope}%
\pgfsetbuttcap%
\pgfsetroundjoin%
\pgfsetlinewidth{2.007500pt}%
\definecolor{currentstroke}{rgb}{0.000000,0.500000,0.000000}%
\pgfsetstrokecolor{currentstroke}%
\pgfsetdash{}{0pt}%
\pgfpathmoveto{\pgfqpoint{5.032428in}{4.592227in}}%
\pgfpathlineto{\pgfqpoint{5.587983in}{4.592227in}}%
\pgfusepath{stroke}%
\end{pgfscope}%
\begin{pgfscope}%
\definecolor{textcolor}{rgb}{0.000000,0.000000,0.000000}%
\pgfsetstrokecolor{textcolor}%
\pgfsetfillcolor{textcolor}%
\pgftext[x=5.810206in,y=4.495005in,left,base]{\color{textcolor}\sffamily\fontsize{20.000000}{24.000000}\selectfont Threshold}%
\end{pgfscope}%
\begin{pgfscope}%
\pgfsetbuttcap%
\pgfsetroundjoin%
\pgfsetlinewidth{0.501875pt}%
\definecolor{currentstroke}{rgb}{1.000000,0.000000,0.000000}%
\pgfsetstrokecolor{currentstroke}%
\pgfsetdash{}{0pt}%
\pgfpathmoveto{\pgfqpoint{5.032428in}{4.197271in}}%
\pgfpathlineto{\pgfqpoint{5.587983in}{4.197271in}}%
\pgfusepath{stroke}%
\end{pgfscope}%
\begin{pgfscope}%
\definecolor{textcolor}{rgb}{0.000000,0.000000,0.000000}%
\pgfsetstrokecolor{textcolor}%
\pgfsetfillcolor{textcolor}%
\pgftext[x=5.810206in,y=4.100048in,left,base]{\color{textcolor}\sffamily\fontsize{20.000000}{24.000000}\selectfont Charge}%
\end{pgfscope}%
\end{pgfpicture}%
\makeatother%
\endgroup%
}
    \caption{\label{fig:lucy} A Richardson-Lucy direct demodulation example:\\ $\Delta t_0=\SI{-0.75}{ns}$, $\mathrm{RSS}=\SI{70.3}{mV^2}$, $D_\mathrm{w}=\SI{1.10}{ns}$.}
  \end{subfigure}
  \caption{\label{fig:deconv}Demonstrations of deconvolution methods on a waveform sampled from the same setup as figure~\ref{fig:method}. Richardson-Lucy direct demodulation in~\subref{fig:lucy} imposes positive charges in iterations and obtains better results than Fourier deconvolution in~\subref{fig:fd}.}
\end{figure}

\subsubsection{Richardson-Lucy direct demodulation}
\label{sec:lucyddm}

\textit{Richardson-Lucy direct demodulation}~(LucyDDM)~\cite{lucy_iterative_1974} with a non-linear iteration to calculate deconvolution has a wide application in astronomy~\cite{li_richardson-lucy_2019} and image processing. We view $V_{\mathrm{PE}*}(t-s)$ as a conditional probability distribution $p(t|s)$ where $t$ denotes PMT amplified electron time, and $s$ represents the given PE time. By the Bayesian rule,
\begin{equation}
  \label{eq:lucy}
  \tilde{\phi}_*(s) V_{\mathrm{PE}*}(t-s) = \tilde{\phi}_*(s)p(t|s) = p(t,s) = \tilde{w}_*(t)p(s|t),
\end{equation}
where $p(t, s)$ is the joint distribution of amplified electron $t$ and PE time $s$, and $\tilde{w}$ is the smoothed $w$.  Cancel out the normalization factors,
\begin{equation}
  \label{eq:ptt}
  p(s|t) = \frac{\tilde{\phi}_*(s) V_{\mathrm{PE}*}(t-s)}{\tilde{w}_*(t)} = \frac{\tilde{\phi}(s) V_{\mathrm{PE}}(t-s)}{\int\tilde{\phi}(s') V_{\mathrm{PE}}(t-s')\mathrm{d}s'}.
\end{equation}
Then a recurrence relation $\phi_*$ is,
\begin{equation}
  \label{eq:iter}
  \begin{aligned}
    \tilde{\phi}_*(s) & = \int p(s|t) \tilde{w}_*(t)\mathrm{d}t = \int \frac{\tilde{\phi}(s) V_{\mathrm{PE}}(t-s)}{\int\tilde{\phi}(s') V_{\mathrm{PE}}(t-s')\mathrm{d}s'} \tilde{w}_*(t) \mathrm{d}t \\
    \implies \hat{\phi}^{n+1}(s) & = \int \frac{\hat{\phi}^n(s) V_{\mathrm{PE}*}(t-s)}{\int\hat{\phi}^n(s') V_{\mathrm{PE}}(t-s')\mathrm{d}s'} \tilde{w}(t) \mathrm{d}t,
  \end{aligned}
\end{equation}
where only $V_{\mathrm{PE}*}$ in the numerator is normalized, and superscript $n$ denotes the iteration step.
Like Fourier deconvolution in eq.~\eqref{eq:fdconv2}, we threshold and scale the converged $\hat{\phi}^\infty$ to get $\hat{\phi}$.  As shown in figure~\ref{fig:lucy}, the positive constraint of $\hat{\phi}$ makes LucyDDM more resilient to noise.

The remaining noise in the smoothed $\tilde{w}$ crucially influences deconvolution.  A probabilistic method will correctly model the noise term $\epsilon$, as we shall see in section \ref{sec:regression}.

\subsection{Convolutional neural network}
\label{sec:cnn}
Convolutional neural networks~(CNN) made breakthroughs in various fields like computer vision~\cite{he_deep_2016} and natural language processing~\cite{vaswani_attention_2017}.  As an efficient composition of weighted additions and non-linear functions, neural networks outperform many traditional algorithms.  The success of CNN induces many ongoing efforts to apply it to waveform analysis~\cite{students22}.  It is thus interesting and insightful to make a comparison of CNN with the remaining traditional methods.

We choose a shallow network structure of 5 layers to recognize patterns as shown in figure~\ref{fig:struct}, motivated by the pulse shape and universality of $V_\mathrm{PE}(t)$ for all the PEs.  The convolutional widths are selected considering the localized nature of $V_\mathrm{PE}(t)$.

The workflow of data processing consists of training and predicting. We find a mapping from waveform $w(t)$ to PE $\tilde{\phi}(t)$ with the backpropagation method and supervised learning. We formulate the training loss as optimizing $D_\mathrm{w}[\hat{\phi}, \tilde{\phi}] $, the Wasserstein distance between the truth $\tilde{\phi}$ and predicted $\hat{\phi}$. As discussed in section~\ref{sec:W-dist}, $D_w$ can handle the PE sparsity in training iterations. Figure~\ref{fig:loss} shows the convergence of Wasserstein distance during training.

The output of CNN is scaled by $\hat{\alpha}$ following eq.~\eqref{eq:fdconv2} to get $\hat{\phi}$.

\begin{figure}[H]
  \begin{subfigure}{.4\textwidth}
    \centering
    \begin{adjustbox}{width=0.5\textwidth}
      \input{model}
    \end{adjustbox}
    \caption{\label{fig:struct} Structure of the neural network.}
  \end{subfigure}
  \begin{subfigure}{.5\textwidth}
    \centering
    \resizebox{\textwidth}{!}{%% Creator: Matplotlib, PGF backend
%%
%% To include the figure in your LaTeX document, write
%%   \input{<filename>.pgf}
%%
%% Make sure the required packages are loaded in your preamble
%%   \usepackage{pgf}
%%
%% Also ensure that all the required font packages are loaded; for instance,
%% the lmodern package is sometimes necessary when using math font.
%%   \usepackage{lmodern}
%%
%% Figures using additional raster images can only be included by \input if
%% they are in the same directory as the main LaTeX file. For loading figures
%% from other directories you can use the `import` package
%%   \usepackage{import}
%%
%% and then include the figures with
%%   \import{<path to file>}{<filename>.pgf}
%%
%% Matplotlib used the following preamble
%%   \usepackage[detect-all,locale=DE]{siunitx}
%%
\begingroup%
\makeatletter%
\begin{pgfpicture}%
\pgfpathrectangle{\pgfpointorigin}{\pgfqpoint{8.000000in}{6.000000in}}%
\pgfusepath{use as bounding box, clip}%
\begin{pgfscope}%
\pgfsetbuttcap%
\pgfsetmiterjoin%
\definecolor{currentfill}{rgb}{1.000000,1.000000,1.000000}%
\pgfsetfillcolor{currentfill}%
\pgfsetlinewidth{0.000000pt}%
\definecolor{currentstroke}{rgb}{1.000000,1.000000,1.000000}%
\pgfsetstrokecolor{currentstroke}%
\pgfsetdash{}{0pt}%
\pgfpathmoveto{\pgfqpoint{0.000000in}{0.000000in}}%
\pgfpathlineto{\pgfqpoint{8.000000in}{0.000000in}}%
\pgfpathlineto{\pgfqpoint{8.000000in}{6.000000in}}%
\pgfpathlineto{\pgfqpoint{0.000000in}{6.000000in}}%
\pgfpathlineto{\pgfqpoint{0.000000in}{0.000000in}}%
\pgfpathclose%
\pgfusepath{fill}%
\end{pgfscope}%
\begin{pgfscope}%
\pgfsetbuttcap%
\pgfsetmiterjoin%
\definecolor{currentfill}{rgb}{1.000000,1.000000,1.000000}%
\pgfsetfillcolor{currentfill}%
\pgfsetlinewidth{0.000000pt}%
\definecolor{currentstroke}{rgb}{0.000000,0.000000,0.000000}%
\pgfsetstrokecolor{currentstroke}%
\pgfsetstrokeopacity{0.000000}%
\pgfsetdash{}{0pt}%
\pgfpathmoveto{\pgfqpoint{1.000000in}{0.720000in}}%
\pgfpathlineto{\pgfqpoint{7.200000in}{0.720000in}}%
\pgfpathlineto{\pgfqpoint{7.200000in}{5.340000in}}%
\pgfpathlineto{\pgfqpoint{1.000000in}{5.340000in}}%
\pgfpathlineto{\pgfqpoint{1.000000in}{0.720000in}}%
\pgfpathclose%
\pgfusepath{fill}%
\end{pgfscope}%
\begin{pgfscope}%
\pgfpathrectangle{\pgfqpoint{1.000000in}{0.720000in}}{\pgfqpoint{6.200000in}{4.620000in}}%
\pgfusepath{clip}%
\pgfsetrectcap%
\pgfsetroundjoin%
\pgfsetlinewidth{0.803000pt}%
\definecolor{currentstroke}{rgb}{0.690196,0.690196,0.690196}%
\pgfsetstrokecolor{currentstroke}%
\pgfsetdash{}{0pt}%
\pgfpathmoveto{\pgfqpoint{1.164394in}{0.720000in}}%
\pgfpathlineto{\pgfqpoint{1.164394in}{5.340000in}}%
\pgfusepath{stroke}%
\end{pgfscope}%
\begin{pgfscope}%
\pgfsetbuttcap%
\pgfsetroundjoin%
\definecolor{currentfill}{rgb}{0.000000,0.000000,0.000000}%
\pgfsetfillcolor{currentfill}%
\pgfsetlinewidth{0.803000pt}%
\definecolor{currentstroke}{rgb}{0.000000,0.000000,0.000000}%
\pgfsetstrokecolor{currentstroke}%
\pgfsetdash{}{0pt}%
\pgfsys@defobject{currentmarker}{\pgfqpoint{0.000000in}{-0.048611in}}{\pgfqpoint{0.000000in}{0.000000in}}{%
\pgfpathmoveto{\pgfqpoint{0.000000in}{0.000000in}}%
\pgfpathlineto{\pgfqpoint{0.000000in}{-0.048611in}}%
\pgfusepath{stroke,fill}%
}%
\begin{pgfscope}%
\pgfsys@transformshift{1.164394in}{0.720000in}%
\pgfsys@useobject{currentmarker}{}%
\end{pgfscope}%
\end{pgfscope}%
\begin{pgfscope}%
\definecolor{textcolor}{rgb}{0.000000,0.000000,0.000000}%
\pgfsetstrokecolor{textcolor}%
\pgfsetfillcolor{textcolor}%
\pgftext[x=1.164394in,y=0.622778in,,top]{\color{textcolor}\sffamily\fontsize{20.000000}{24.000000}\selectfont \(\displaystyle {0}\)}%
\end{pgfscope}%
\begin{pgfscope}%
\pgfpathrectangle{\pgfqpoint{1.000000in}{0.720000in}}{\pgfqpoint{6.200000in}{4.620000in}}%
\pgfusepath{clip}%
\pgfsetrectcap%
\pgfsetroundjoin%
\pgfsetlinewidth{0.803000pt}%
\definecolor{currentstroke}{rgb}{0.690196,0.690196,0.690196}%
\pgfsetstrokecolor{currentstroke}%
\pgfsetdash{}{0pt}%
\pgfpathmoveto{\pgfqpoint{2.338636in}{0.720000in}}%
\pgfpathlineto{\pgfqpoint{2.338636in}{5.340000in}}%
\pgfusepath{stroke}%
\end{pgfscope}%
\begin{pgfscope}%
\pgfsetbuttcap%
\pgfsetroundjoin%
\definecolor{currentfill}{rgb}{0.000000,0.000000,0.000000}%
\pgfsetfillcolor{currentfill}%
\pgfsetlinewidth{0.803000pt}%
\definecolor{currentstroke}{rgb}{0.000000,0.000000,0.000000}%
\pgfsetstrokecolor{currentstroke}%
\pgfsetdash{}{0pt}%
\pgfsys@defobject{currentmarker}{\pgfqpoint{0.000000in}{-0.048611in}}{\pgfqpoint{0.000000in}{0.000000in}}{%
\pgfpathmoveto{\pgfqpoint{0.000000in}{0.000000in}}%
\pgfpathlineto{\pgfqpoint{0.000000in}{-0.048611in}}%
\pgfusepath{stroke,fill}%
}%
\begin{pgfscope}%
\pgfsys@transformshift{2.338636in}{0.720000in}%
\pgfsys@useobject{currentmarker}{}%
\end{pgfscope}%
\end{pgfscope}%
\begin{pgfscope}%
\definecolor{textcolor}{rgb}{0.000000,0.000000,0.000000}%
\pgfsetstrokecolor{textcolor}%
\pgfsetfillcolor{textcolor}%
\pgftext[x=2.338636in,y=0.622778in,,top]{\color{textcolor}\sffamily\fontsize{20.000000}{24.000000}\selectfont \(\displaystyle {10}\)}%
\end{pgfscope}%
\begin{pgfscope}%
\pgfpathrectangle{\pgfqpoint{1.000000in}{0.720000in}}{\pgfqpoint{6.200000in}{4.620000in}}%
\pgfusepath{clip}%
\pgfsetrectcap%
\pgfsetroundjoin%
\pgfsetlinewidth{0.803000pt}%
\definecolor{currentstroke}{rgb}{0.690196,0.690196,0.690196}%
\pgfsetstrokecolor{currentstroke}%
\pgfsetdash{}{0pt}%
\pgfpathmoveto{\pgfqpoint{3.512879in}{0.720000in}}%
\pgfpathlineto{\pgfqpoint{3.512879in}{5.340000in}}%
\pgfusepath{stroke}%
\end{pgfscope}%
\begin{pgfscope}%
\pgfsetbuttcap%
\pgfsetroundjoin%
\definecolor{currentfill}{rgb}{0.000000,0.000000,0.000000}%
\pgfsetfillcolor{currentfill}%
\pgfsetlinewidth{0.803000pt}%
\definecolor{currentstroke}{rgb}{0.000000,0.000000,0.000000}%
\pgfsetstrokecolor{currentstroke}%
\pgfsetdash{}{0pt}%
\pgfsys@defobject{currentmarker}{\pgfqpoint{0.000000in}{-0.048611in}}{\pgfqpoint{0.000000in}{0.000000in}}{%
\pgfpathmoveto{\pgfqpoint{0.000000in}{0.000000in}}%
\pgfpathlineto{\pgfqpoint{0.000000in}{-0.048611in}}%
\pgfusepath{stroke,fill}%
}%
\begin{pgfscope}%
\pgfsys@transformshift{3.512879in}{0.720000in}%
\pgfsys@useobject{currentmarker}{}%
\end{pgfscope}%
\end{pgfscope}%
\begin{pgfscope}%
\definecolor{textcolor}{rgb}{0.000000,0.000000,0.000000}%
\pgfsetstrokecolor{textcolor}%
\pgfsetfillcolor{textcolor}%
\pgftext[x=3.512879in,y=0.622778in,,top]{\color{textcolor}\sffamily\fontsize{20.000000}{24.000000}\selectfont \(\displaystyle {20}\)}%
\end{pgfscope}%
\begin{pgfscope}%
\pgfpathrectangle{\pgfqpoint{1.000000in}{0.720000in}}{\pgfqpoint{6.200000in}{4.620000in}}%
\pgfusepath{clip}%
\pgfsetrectcap%
\pgfsetroundjoin%
\pgfsetlinewidth{0.803000pt}%
\definecolor{currentstroke}{rgb}{0.690196,0.690196,0.690196}%
\pgfsetstrokecolor{currentstroke}%
\pgfsetdash{}{0pt}%
\pgfpathmoveto{\pgfqpoint{4.687121in}{0.720000in}}%
\pgfpathlineto{\pgfqpoint{4.687121in}{5.340000in}}%
\pgfusepath{stroke}%
\end{pgfscope}%
\begin{pgfscope}%
\pgfsetbuttcap%
\pgfsetroundjoin%
\definecolor{currentfill}{rgb}{0.000000,0.000000,0.000000}%
\pgfsetfillcolor{currentfill}%
\pgfsetlinewidth{0.803000pt}%
\definecolor{currentstroke}{rgb}{0.000000,0.000000,0.000000}%
\pgfsetstrokecolor{currentstroke}%
\pgfsetdash{}{0pt}%
\pgfsys@defobject{currentmarker}{\pgfqpoint{0.000000in}{-0.048611in}}{\pgfqpoint{0.000000in}{0.000000in}}{%
\pgfpathmoveto{\pgfqpoint{0.000000in}{0.000000in}}%
\pgfpathlineto{\pgfqpoint{0.000000in}{-0.048611in}}%
\pgfusepath{stroke,fill}%
}%
\begin{pgfscope}%
\pgfsys@transformshift{4.687121in}{0.720000in}%
\pgfsys@useobject{currentmarker}{}%
\end{pgfscope}%
\end{pgfscope}%
\begin{pgfscope}%
\definecolor{textcolor}{rgb}{0.000000,0.000000,0.000000}%
\pgfsetstrokecolor{textcolor}%
\pgfsetfillcolor{textcolor}%
\pgftext[x=4.687121in,y=0.622778in,,top]{\color{textcolor}\sffamily\fontsize{20.000000}{24.000000}\selectfont \(\displaystyle {30}\)}%
\end{pgfscope}%
\begin{pgfscope}%
\pgfpathrectangle{\pgfqpoint{1.000000in}{0.720000in}}{\pgfqpoint{6.200000in}{4.620000in}}%
\pgfusepath{clip}%
\pgfsetrectcap%
\pgfsetroundjoin%
\pgfsetlinewidth{0.803000pt}%
\definecolor{currentstroke}{rgb}{0.690196,0.690196,0.690196}%
\pgfsetstrokecolor{currentstroke}%
\pgfsetdash{}{0pt}%
\pgfpathmoveto{\pgfqpoint{5.861364in}{0.720000in}}%
\pgfpathlineto{\pgfqpoint{5.861364in}{5.340000in}}%
\pgfusepath{stroke}%
\end{pgfscope}%
\begin{pgfscope}%
\pgfsetbuttcap%
\pgfsetroundjoin%
\definecolor{currentfill}{rgb}{0.000000,0.000000,0.000000}%
\pgfsetfillcolor{currentfill}%
\pgfsetlinewidth{0.803000pt}%
\definecolor{currentstroke}{rgb}{0.000000,0.000000,0.000000}%
\pgfsetstrokecolor{currentstroke}%
\pgfsetdash{}{0pt}%
\pgfsys@defobject{currentmarker}{\pgfqpoint{0.000000in}{-0.048611in}}{\pgfqpoint{0.000000in}{0.000000in}}{%
\pgfpathmoveto{\pgfqpoint{0.000000in}{0.000000in}}%
\pgfpathlineto{\pgfqpoint{0.000000in}{-0.048611in}}%
\pgfusepath{stroke,fill}%
}%
\begin{pgfscope}%
\pgfsys@transformshift{5.861364in}{0.720000in}%
\pgfsys@useobject{currentmarker}{}%
\end{pgfscope}%
\end{pgfscope}%
\begin{pgfscope}%
\definecolor{textcolor}{rgb}{0.000000,0.000000,0.000000}%
\pgfsetstrokecolor{textcolor}%
\pgfsetfillcolor{textcolor}%
\pgftext[x=5.861364in,y=0.622778in,,top]{\color{textcolor}\sffamily\fontsize{20.000000}{24.000000}\selectfont \(\displaystyle {40}\)}%
\end{pgfscope}%
\begin{pgfscope}%
\pgfpathrectangle{\pgfqpoint{1.000000in}{0.720000in}}{\pgfqpoint{6.200000in}{4.620000in}}%
\pgfusepath{clip}%
\pgfsetrectcap%
\pgfsetroundjoin%
\pgfsetlinewidth{0.803000pt}%
\definecolor{currentstroke}{rgb}{0.690196,0.690196,0.690196}%
\pgfsetstrokecolor{currentstroke}%
\pgfsetdash{}{0pt}%
\pgfpathmoveto{\pgfqpoint{7.035606in}{0.720000in}}%
\pgfpathlineto{\pgfqpoint{7.035606in}{5.340000in}}%
\pgfusepath{stroke}%
\end{pgfscope}%
\begin{pgfscope}%
\pgfsetbuttcap%
\pgfsetroundjoin%
\definecolor{currentfill}{rgb}{0.000000,0.000000,0.000000}%
\pgfsetfillcolor{currentfill}%
\pgfsetlinewidth{0.803000pt}%
\definecolor{currentstroke}{rgb}{0.000000,0.000000,0.000000}%
\pgfsetstrokecolor{currentstroke}%
\pgfsetdash{}{0pt}%
\pgfsys@defobject{currentmarker}{\pgfqpoint{0.000000in}{-0.048611in}}{\pgfqpoint{0.000000in}{0.000000in}}{%
\pgfpathmoveto{\pgfqpoint{0.000000in}{0.000000in}}%
\pgfpathlineto{\pgfqpoint{0.000000in}{-0.048611in}}%
\pgfusepath{stroke,fill}%
}%
\begin{pgfscope}%
\pgfsys@transformshift{7.035606in}{0.720000in}%
\pgfsys@useobject{currentmarker}{}%
\end{pgfscope}%
\end{pgfscope}%
\begin{pgfscope}%
\definecolor{textcolor}{rgb}{0.000000,0.000000,0.000000}%
\pgfsetstrokecolor{textcolor}%
\pgfsetfillcolor{textcolor}%
\pgftext[x=7.035606in,y=0.622778in,,top]{\color{textcolor}\sffamily\fontsize{20.000000}{24.000000}\selectfont \(\displaystyle {50}\)}%
\end{pgfscope}%
\begin{pgfscope}%
\definecolor{textcolor}{rgb}{0.000000,0.000000,0.000000}%
\pgfsetstrokecolor{textcolor}%
\pgfsetfillcolor{textcolor}%
\pgftext[x=4.100000in,y=0.311155in,,top]{\color{textcolor}\sffamily\fontsize{20.000000}{24.000000}\selectfont \(\displaystyle \mathrm{epoch}\)}%
\end{pgfscope}%
\begin{pgfscope}%
\pgfpathrectangle{\pgfqpoint{1.000000in}{0.720000in}}{\pgfqpoint{6.200000in}{4.620000in}}%
\pgfusepath{clip}%
\pgfsetrectcap%
\pgfsetroundjoin%
\pgfsetlinewidth{0.803000pt}%
\definecolor{currentstroke}{rgb}{0.690196,0.690196,0.690196}%
\pgfsetstrokecolor{currentstroke}%
\pgfsetdash{}{0pt}%
\pgfpathmoveto{\pgfqpoint{1.000000in}{0.989859in}}%
\pgfpathlineto{\pgfqpoint{7.200000in}{0.989859in}}%
\pgfusepath{stroke}%
\end{pgfscope}%
\begin{pgfscope}%
\pgfsetbuttcap%
\pgfsetroundjoin%
\definecolor{currentfill}{rgb}{0.000000,0.000000,0.000000}%
\pgfsetfillcolor{currentfill}%
\pgfsetlinewidth{0.803000pt}%
\definecolor{currentstroke}{rgb}{0.000000,0.000000,0.000000}%
\pgfsetstrokecolor{currentstroke}%
\pgfsetdash{}{0pt}%
\pgfsys@defobject{currentmarker}{\pgfqpoint{-0.048611in}{0.000000in}}{\pgfqpoint{-0.000000in}{0.000000in}}{%
\pgfpathmoveto{\pgfqpoint{-0.000000in}{0.000000in}}%
\pgfpathlineto{\pgfqpoint{-0.048611in}{0.000000in}}%
\pgfusepath{stroke,fill}%
}%
\begin{pgfscope}%
\pgfsys@transformshift{1.000000in}{0.989859in}%
\pgfsys@useobject{currentmarker}{}%
\end{pgfscope}%
\end{pgfscope}%
\begin{pgfscope}%
\definecolor{textcolor}{rgb}{0.000000,0.000000,0.000000}%
\pgfsetstrokecolor{textcolor}%
\pgfsetfillcolor{textcolor}%
\pgftext[x=0.428108in, y=0.889839in, left, base]{\color{textcolor}\sffamily\fontsize{20.000000}{24.000000}\selectfont \(\displaystyle {0.40}\)}%
\end{pgfscope}%
\begin{pgfscope}%
\pgfpathrectangle{\pgfqpoint{1.000000in}{0.720000in}}{\pgfqpoint{6.200000in}{4.620000in}}%
\pgfusepath{clip}%
\pgfsetrectcap%
\pgfsetroundjoin%
\pgfsetlinewidth{0.803000pt}%
\definecolor{currentstroke}{rgb}{0.690196,0.690196,0.690196}%
\pgfsetstrokecolor{currentstroke}%
\pgfsetdash{}{0pt}%
\pgfpathmoveto{\pgfqpoint{1.000000in}{1.857802in}}%
\pgfpathlineto{\pgfqpoint{7.200000in}{1.857802in}}%
\pgfusepath{stroke}%
\end{pgfscope}%
\begin{pgfscope}%
\pgfsetbuttcap%
\pgfsetroundjoin%
\definecolor{currentfill}{rgb}{0.000000,0.000000,0.000000}%
\pgfsetfillcolor{currentfill}%
\pgfsetlinewidth{0.803000pt}%
\definecolor{currentstroke}{rgb}{0.000000,0.000000,0.000000}%
\pgfsetstrokecolor{currentstroke}%
\pgfsetdash{}{0pt}%
\pgfsys@defobject{currentmarker}{\pgfqpoint{-0.048611in}{0.000000in}}{\pgfqpoint{-0.000000in}{0.000000in}}{%
\pgfpathmoveto{\pgfqpoint{-0.000000in}{0.000000in}}%
\pgfpathlineto{\pgfqpoint{-0.048611in}{0.000000in}}%
\pgfusepath{stroke,fill}%
}%
\begin{pgfscope}%
\pgfsys@transformshift{1.000000in}{1.857802in}%
\pgfsys@useobject{currentmarker}{}%
\end{pgfscope}%
\end{pgfscope}%
\begin{pgfscope}%
\definecolor{textcolor}{rgb}{0.000000,0.000000,0.000000}%
\pgfsetstrokecolor{textcolor}%
\pgfsetfillcolor{textcolor}%
\pgftext[x=0.428108in, y=1.757783in, left, base]{\color{textcolor}\sffamily\fontsize{20.000000}{24.000000}\selectfont \(\displaystyle {0.42}\)}%
\end{pgfscope}%
\begin{pgfscope}%
\pgfpathrectangle{\pgfqpoint{1.000000in}{0.720000in}}{\pgfqpoint{6.200000in}{4.620000in}}%
\pgfusepath{clip}%
\pgfsetrectcap%
\pgfsetroundjoin%
\pgfsetlinewidth{0.803000pt}%
\definecolor{currentstroke}{rgb}{0.690196,0.690196,0.690196}%
\pgfsetstrokecolor{currentstroke}%
\pgfsetdash{}{0pt}%
\pgfpathmoveto{\pgfqpoint{1.000000in}{2.725745in}}%
\pgfpathlineto{\pgfqpoint{7.200000in}{2.725745in}}%
\pgfusepath{stroke}%
\end{pgfscope}%
\begin{pgfscope}%
\pgfsetbuttcap%
\pgfsetroundjoin%
\definecolor{currentfill}{rgb}{0.000000,0.000000,0.000000}%
\pgfsetfillcolor{currentfill}%
\pgfsetlinewidth{0.803000pt}%
\definecolor{currentstroke}{rgb}{0.000000,0.000000,0.000000}%
\pgfsetstrokecolor{currentstroke}%
\pgfsetdash{}{0pt}%
\pgfsys@defobject{currentmarker}{\pgfqpoint{-0.048611in}{0.000000in}}{\pgfqpoint{-0.000000in}{0.000000in}}{%
\pgfpathmoveto{\pgfqpoint{-0.000000in}{0.000000in}}%
\pgfpathlineto{\pgfqpoint{-0.048611in}{0.000000in}}%
\pgfusepath{stroke,fill}%
}%
\begin{pgfscope}%
\pgfsys@transformshift{1.000000in}{2.725745in}%
\pgfsys@useobject{currentmarker}{}%
\end{pgfscope}%
\end{pgfscope}%
\begin{pgfscope}%
\definecolor{textcolor}{rgb}{0.000000,0.000000,0.000000}%
\pgfsetstrokecolor{textcolor}%
\pgfsetfillcolor{textcolor}%
\pgftext[x=0.428108in, y=2.625726in, left, base]{\color{textcolor}\sffamily\fontsize{20.000000}{24.000000}\selectfont \(\displaystyle {0.44}\)}%
\end{pgfscope}%
\begin{pgfscope}%
\pgfpathrectangle{\pgfqpoint{1.000000in}{0.720000in}}{\pgfqpoint{6.200000in}{4.620000in}}%
\pgfusepath{clip}%
\pgfsetrectcap%
\pgfsetroundjoin%
\pgfsetlinewidth{0.803000pt}%
\definecolor{currentstroke}{rgb}{0.690196,0.690196,0.690196}%
\pgfsetstrokecolor{currentstroke}%
\pgfsetdash{}{0pt}%
\pgfpathmoveto{\pgfqpoint{1.000000in}{3.593689in}}%
\pgfpathlineto{\pgfqpoint{7.200000in}{3.593689in}}%
\pgfusepath{stroke}%
\end{pgfscope}%
\begin{pgfscope}%
\pgfsetbuttcap%
\pgfsetroundjoin%
\definecolor{currentfill}{rgb}{0.000000,0.000000,0.000000}%
\pgfsetfillcolor{currentfill}%
\pgfsetlinewidth{0.803000pt}%
\definecolor{currentstroke}{rgb}{0.000000,0.000000,0.000000}%
\pgfsetstrokecolor{currentstroke}%
\pgfsetdash{}{0pt}%
\pgfsys@defobject{currentmarker}{\pgfqpoint{-0.048611in}{0.000000in}}{\pgfqpoint{-0.000000in}{0.000000in}}{%
\pgfpathmoveto{\pgfqpoint{-0.000000in}{0.000000in}}%
\pgfpathlineto{\pgfqpoint{-0.048611in}{0.000000in}}%
\pgfusepath{stroke,fill}%
}%
\begin{pgfscope}%
\pgfsys@transformshift{1.000000in}{3.593689in}%
\pgfsys@useobject{currentmarker}{}%
\end{pgfscope}%
\end{pgfscope}%
\begin{pgfscope}%
\definecolor{textcolor}{rgb}{0.000000,0.000000,0.000000}%
\pgfsetstrokecolor{textcolor}%
\pgfsetfillcolor{textcolor}%
\pgftext[x=0.428108in, y=3.493669in, left, base]{\color{textcolor}\sffamily\fontsize{20.000000}{24.000000}\selectfont \(\displaystyle {0.46}\)}%
\end{pgfscope}%
\begin{pgfscope}%
\pgfpathrectangle{\pgfqpoint{1.000000in}{0.720000in}}{\pgfqpoint{6.200000in}{4.620000in}}%
\pgfusepath{clip}%
\pgfsetrectcap%
\pgfsetroundjoin%
\pgfsetlinewidth{0.803000pt}%
\definecolor{currentstroke}{rgb}{0.690196,0.690196,0.690196}%
\pgfsetstrokecolor{currentstroke}%
\pgfsetdash{}{0pt}%
\pgfpathmoveto{\pgfqpoint{1.000000in}{4.461632in}}%
\pgfpathlineto{\pgfqpoint{7.200000in}{4.461632in}}%
\pgfusepath{stroke}%
\end{pgfscope}%
\begin{pgfscope}%
\pgfsetbuttcap%
\pgfsetroundjoin%
\definecolor{currentfill}{rgb}{0.000000,0.000000,0.000000}%
\pgfsetfillcolor{currentfill}%
\pgfsetlinewidth{0.803000pt}%
\definecolor{currentstroke}{rgb}{0.000000,0.000000,0.000000}%
\pgfsetstrokecolor{currentstroke}%
\pgfsetdash{}{0pt}%
\pgfsys@defobject{currentmarker}{\pgfqpoint{-0.048611in}{0.000000in}}{\pgfqpoint{-0.000000in}{0.000000in}}{%
\pgfpathmoveto{\pgfqpoint{-0.000000in}{0.000000in}}%
\pgfpathlineto{\pgfqpoint{-0.048611in}{0.000000in}}%
\pgfusepath{stroke,fill}%
}%
\begin{pgfscope}%
\pgfsys@transformshift{1.000000in}{4.461632in}%
\pgfsys@useobject{currentmarker}{}%
\end{pgfscope}%
\end{pgfscope}%
\begin{pgfscope}%
\definecolor{textcolor}{rgb}{0.000000,0.000000,0.000000}%
\pgfsetstrokecolor{textcolor}%
\pgfsetfillcolor{textcolor}%
\pgftext[x=0.428108in, y=4.361613in, left, base]{\color{textcolor}\sffamily\fontsize{20.000000}{24.000000}\selectfont \(\displaystyle {0.48}\)}%
\end{pgfscope}%
\begin{pgfscope}%
\pgfpathrectangle{\pgfqpoint{1.000000in}{0.720000in}}{\pgfqpoint{6.200000in}{4.620000in}}%
\pgfusepath{clip}%
\pgfsetrectcap%
\pgfsetroundjoin%
\pgfsetlinewidth{0.803000pt}%
\definecolor{currentstroke}{rgb}{0.690196,0.690196,0.690196}%
\pgfsetstrokecolor{currentstroke}%
\pgfsetdash{}{0pt}%
\pgfpathmoveto{\pgfqpoint{1.000000in}{5.329575in}}%
\pgfpathlineto{\pgfqpoint{7.200000in}{5.329575in}}%
\pgfusepath{stroke}%
\end{pgfscope}%
\begin{pgfscope}%
\pgfsetbuttcap%
\pgfsetroundjoin%
\definecolor{currentfill}{rgb}{0.000000,0.000000,0.000000}%
\pgfsetfillcolor{currentfill}%
\pgfsetlinewidth{0.803000pt}%
\definecolor{currentstroke}{rgb}{0.000000,0.000000,0.000000}%
\pgfsetstrokecolor{currentstroke}%
\pgfsetdash{}{0pt}%
\pgfsys@defobject{currentmarker}{\pgfqpoint{-0.048611in}{0.000000in}}{\pgfqpoint{-0.000000in}{0.000000in}}{%
\pgfpathmoveto{\pgfqpoint{-0.000000in}{0.000000in}}%
\pgfpathlineto{\pgfqpoint{-0.048611in}{0.000000in}}%
\pgfusepath{stroke,fill}%
}%
\begin{pgfscope}%
\pgfsys@transformshift{1.000000in}{5.329575in}%
\pgfsys@useobject{currentmarker}{}%
\end{pgfscope}%
\end{pgfscope}%
\begin{pgfscope}%
\definecolor{textcolor}{rgb}{0.000000,0.000000,0.000000}%
\pgfsetstrokecolor{textcolor}%
\pgfsetfillcolor{textcolor}%
\pgftext[x=0.428108in, y=5.229556in, left, base]{\color{textcolor}\sffamily\fontsize{20.000000}{24.000000}\selectfont \(\displaystyle {0.50}\)}%
\end{pgfscope}%
\begin{pgfscope}%
\definecolor{textcolor}{rgb}{0.000000,0.000000,0.000000}%
\pgfsetstrokecolor{textcolor}%
\pgfsetfillcolor{textcolor}%
\pgftext[x=0.372552in,y=3.030000in,,bottom,rotate=90.000000]{\color{textcolor}\sffamily\fontsize{20.000000}{24.000000}\selectfont \(\displaystyle \mathrm{Wasserstein\ Distance}/\si{ns}\)}%
\end{pgfscope}%
\begin{pgfscope}%
\pgfpathrectangle{\pgfqpoint{1.000000in}{0.720000in}}{\pgfqpoint{6.200000in}{4.620000in}}%
\pgfusepath{clip}%
\pgfsetrectcap%
\pgfsetroundjoin%
\pgfsetlinewidth{2.007500pt}%
\definecolor{currentstroke}{rgb}{1.000000,0.498039,0.054902}%
\pgfsetstrokecolor{currentstroke}%
\pgfsetdash{}{0pt}%
\pgfpathmoveto{\pgfqpoint{1.281818in}{5.130000in}}%
\pgfpathlineto{\pgfqpoint{1.399242in}{3.133873in}}%
\pgfpathlineto{\pgfqpoint{1.516667in}{2.499256in}}%
\pgfpathlineto{\pgfqpoint{1.634091in}{2.361655in}}%
\pgfpathlineto{\pgfqpoint{1.751515in}{1.927728in}}%
\pgfpathlineto{\pgfqpoint{1.868939in}{2.146409in}}%
\pgfpathlineto{\pgfqpoint{1.986364in}{1.888659in}}%
\pgfpathlineto{\pgfqpoint{2.103788in}{1.913137in}}%
\pgfpathlineto{\pgfqpoint{2.221212in}{1.661565in}}%
\pgfpathlineto{\pgfqpoint{2.338636in}{1.479917in}}%
\pgfpathlineto{\pgfqpoint{2.456061in}{1.683259in}}%
\pgfpathlineto{\pgfqpoint{2.573485in}{1.347986in}}%
\pgfpathlineto{\pgfqpoint{2.690909in}{1.395304in}}%
\pgfpathlineto{\pgfqpoint{2.808333in}{1.264047in}}%
\pgfpathlineto{\pgfqpoint{2.925758in}{1.185394in}}%
\pgfpathlineto{\pgfqpoint{3.043182in}{1.374431in}}%
\pgfpathlineto{\pgfqpoint{3.160606in}{1.242646in}}%
\pgfpathlineto{\pgfqpoint{3.278030in}{1.456932in}}%
\pgfpathlineto{\pgfqpoint{3.395455in}{1.885365in}}%
\pgfpathlineto{\pgfqpoint{3.512879in}{1.270353in}}%
\pgfpathlineto{\pgfqpoint{3.630303in}{1.160417in}}%
\pgfpathlineto{\pgfqpoint{3.747727in}{1.691847in}}%
\pgfpathlineto{\pgfqpoint{3.865152in}{1.367693in}}%
\pgfpathlineto{\pgfqpoint{3.982576in}{1.150072in}}%
\pgfpathlineto{\pgfqpoint{4.100000in}{1.065955in}}%
\pgfpathlineto{\pgfqpoint{4.217424in}{1.203217in}}%
\pgfpathlineto{\pgfqpoint{4.334848in}{1.150620in}}%
\pgfpathlineto{\pgfqpoint{4.452273in}{1.055445in}}%
\pgfpathlineto{\pgfqpoint{4.569697in}{1.006523in}}%
\pgfpathlineto{\pgfqpoint{4.687121in}{1.312413in}}%
\pgfpathlineto{\pgfqpoint{4.804545in}{1.129911in}}%
\pgfpathlineto{\pgfqpoint{4.921970in}{1.095274in}}%
\pgfpathlineto{\pgfqpoint{5.039394in}{1.269601in}}%
\pgfpathlineto{\pgfqpoint{5.156818in}{0.941450in}}%
\pgfpathlineto{\pgfqpoint{5.274242in}{0.930000in}}%
\pgfpathlineto{\pgfqpoint{5.391667in}{0.983079in}}%
\pgfpathlineto{\pgfqpoint{5.509091in}{1.035117in}}%
\pgfpathlineto{\pgfqpoint{5.626515in}{0.977242in}}%
\pgfpathlineto{\pgfqpoint{5.743939in}{0.955153in}}%
\pgfpathlineto{\pgfqpoint{5.861364in}{1.050171in}}%
\pgfpathlineto{\pgfqpoint{5.978788in}{1.011530in}}%
\pgfpathlineto{\pgfqpoint{6.096212in}{1.095229in}}%
\pgfpathlineto{\pgfqpoint{6.213636in}{0.984220in}}%
\pgfpathlineto{\pgfqpoint{6.331061in}{0.960965in}}%
\pgfpathlineto{\pgfqpoint{6.448485in}{1.108561in}}%
\pgfpathlineto{\pgfqpoint{6.565909in}{0.935303in}}%
\pgfpathlineto{\pgfqpoint{6.683333in}{1.060336in}}%
\pgfpathlineto{\pgfqpoint{6.800758in}{1.040871in}}%
\pgfpathlineto{\pgfqpoint{6.918182in}{1.035405in}}%
\pgfusepath{stroke}%
\end{pgfscope}%
\begin{pgfscope}%
\pgfsetrectcap%
\pgfsetmiterjoin%
\pgfsetlinewidth{0.803000pt}%
\definecolor{currentstroke}{rgb}{0.000000,0.000000,0.000000}%
\pgfsetstrokecolor{currentstroke}%
\pgfsetdash{}{0pt}%
\pgfpathmoveto{\pgfqpoint{1.000000in}{0.720000in}}%
\pgfpathlineto{\pgfqpoint{1.000000in}{5.340000in}}%
\pgfusepath{stroke}%
\end{pgfscope}%
\begin{pgfscope}%
\pgfsetrectcap%
\pgfsetmiterjoin%
\pgfsetlinewidth{0.803000pt}%
\definecolor{currentstroke}{rgb}{0.000000,0.000000,0.000000}%
\pgfsetstrokecolor{currentstroke}%
\pgfsetdash{}{0pt}%
\pgfpathmoveto{\pgfqpoint{7.200000in}{0.720000in}}%
\pgfpathlineto{\pgfqpoint{7.200000in}{5.340000in}}%
\pgfusepath{stroke}%
\end{pgfscope}%
\begin{pgfscope}%
\pgfsetrectcap%
\pgfsetmiterjoin%
\pgfsetlinewidth{0.803000pt}%
\definecolor{currentstroke}{rgb}{0.000000,0.000000,0.000000}%
\pgfsetstrokecolor{currentstroke}%
\pgfsetdash{}{0pt}%
\pgfpathmoveto{\pgfqpoint{1.000000in}{0.720000in}}%
\pgfpathlineto{\pgfqpoint{7.200000in}{0.720000in}}%
\pgfusepath{stroke}%
\end{pgfscope}%
\begin{pgfscope}%
\pgfsetrectcap%
\pgfsetmiterjoin%
\pgfsetlinewidth{0.803000pt}%
\definecolor{currentstroke}{rgb}{0.000000,0.000000,0.000000}%
\pgfsetstrokecolor{currentstroke}%
\pgfsetdash{}{0pt}%
\pgfpathmoveto{\pgfqpoint{1.000000in}{5.340000in}}%
\pgfpathlineto{\pgfqpoint{7.200000in}{5.340000in}}%
\pgfusepath{stroke}%
\end{pgfscope}%
\begin{pgfscope}%
\pgfsetbuttcap%
\pgfsetmiterjoin%
\definecolor{currentfill}{rgb}{1.000000,1.000000,1.000000}%
\pgfsetfillcolor{currentfill}%
\pgfsetfillopacity{0.800000}%
\pgfsetlinewidth{1.003750pt}%
\definecolor{currentstroke}{rgb}{0.800000,0.800000,0.800000}%
\pgfsetstrokecolor{currentstroke}%
\pgfsetstrokeopacity{0.800000}%
\pgfsetdash{}{0pt}%
\pgfpathmoveto{\pgfqpoint{5.730170in}{4.722821in}}%
\pgfpathlineto{\pgfqpoint{7.005556in}{4.722821in}}%
\pgfpathquadraticcurveto{\pgfqpoint{7.061111in}{4.722821in}}{\pgfqpoint{7.061111in}{4.778377in}}%
\pgfpathlineto{\pgfqpoint{7.061111in}{5.145556in}}%
\pgfpathquadraticcurveto{\pgfqpoint{7.061111in}{5.201111in}}{\pgfqpoint{7.005556in}{5.201111in}}%
\pgfpathlineto{\pgfqpoint{5.730170in}{5.201111in}}%
\pgfpathquadraticcurveto{\pgfqpoint{5.674615in}{5.201111in}}{\pgfqpoint{5.674615in}{5.145556in}}%
\pgfpathlineto{\pgfqpoint{5.674615in}{4.778377in}}%
\pgfpathquadraticcurveto{\pgfqpoint{5.674615in}{4.722821in}}{\pgfqpoint{5.730170in}{4.722821in}}%
\pgfpathlineto{\pgfqpoint{5.730170in}{4.722821in}}%
\pgfpathclose%
\pgfusepath{stroke,fill}%
\end{pgfscope}%
\begin{pgfscope}%
\pgfsetrectcap%
\pgfsetroundjoin%
\pgfsetlinewidth{2.007500pt}%
\definecolor{currentstroke}{rgb}{1.000000,0.498039,0.054902}%
\pgfsetstrokecolor{currentstroke}%
\pgfsetdash{}{0pt}%
\pgfpathmoveto{\pgfqpoint{5.785726in}{4.987184in}}%
\pgfpathlineto{\pgfqpoint{6.063504in}{4.987184in}}%
\pgfpathlineto{\pgfqpoint{6.341281in}{4.987184in}}%
\pgfusepath{stroke}%
\end{pgfscope}%
\begin{pgfscope}%
\definecolor{textcolor}{rgb}{0.000000,0.000000,0.000000}%
\pgfsetstrokecolor{textcolor}%
\pgfsetfillcolor{textcolor}%
\pgftext[x=6.563504in,y=4.889962in,left,base]{\color{textcolor}\sffamily\fontsize{20.000000}{24.000000}\selectfont \(\displaystyle D_\mathrm{w}\)}%
\end{pgfscope}%
\end{pgfpicture}%
\makeatother%
\endgroup%
}
    \caption{\label{fig:loss} Evolution of loss.}
  \end{subfigure}
  \caption{\label{fig:CNN} Training process of a CNN. A shallow network structure of 5 layers in~\subref{fig:struct} is trained to converge in Wasserstein distance as shown in~\subref{fig:loss}.  ``kernel=21'' stands for a 1 dimensional convolutional kernel of length 21. ``1029'' is the number of voltage samples in a waveform.  $1\times$ represents the number of channels in each layer.}
\end{figure}

\vspace{-0.5cm}
In figure~\ref{fig:cnn-npe}, $D_w$ is the smallest for one PE.  $D_w$ stops increasing with $N_\mathrm{PE}$ at about 6 PEs, where the PE times are the most challenging to extract.  When $N_\mathrm{PE}$ is more than 6, pile-ups tend to produce a continuous waveform and the average PE time accuracy stays flat. Thus waveform analysis is the most important in recovering time accuracy for PEs less than 10.

Such small $D_\mathrm{w}$ in figure~\ref{fig:cnn-npe} provides a precise matching of waveforms horizontally to guarantee effective $\hat{\alpha}$ scaling, explaining why $\mathrm{RSS}$ is also small in figure~\ref{fig:cnn}.

\begin{figure}[H]
  \begin{subfigure}{.5\textwidth}
    \centering
    \resizebox{\textwidth}{!}{%% Creator: Matplotlib, PGF backend
%%
%% To include the figure in your LaTeX document, write
%%   \input{<filename>.pgf}
%%
%% Make sure the required packages are loaded in your preamble
%%   \usepackage{pgf}
%%
%% Also ensure that all the required font packages are loaded; for instance,
%% the lmodern package is sometimes necessary when using math font.
%%   \usepackage{lmodern}
%%
%% Figures using additional raster images can only be included by \input if
%% they are in the same directory as the main LaTeX file. For loading figures
%% from other directories you can use the `import` package
%%   \usepackage{import}
%%
%% and then include the figures with
%%   \import{<path to file>}{<filename>.pgf}
%%
%% Matplotlib used the following preamble
%%   \usepackage[detect-all,locale=DE]{siunitx}
%%
\begingroup%
\makeatletter%
\begin{pgfpicture}%
\pgfpathrectangle{\pgfpointorigin}{\pgfqpoint{8.000000in}{6.000000in}}%
\pgfusepath{use as bounding box, clip}%
\begin{pgfscope}%
\pgfsetbuttcap%
\pgfsetmiterjoin%
\definecolor{currentfill}{rgb}{1.000000,1.000000,1.000000}%
\pgfsetfillcolor{currentfill}%
\pgfsetlinewidth{0.000000pt}%
\definecolor{currentstroke}{rgb}{1.000000,1.000000,1.000000}%
\pgfsetstrokecolor{currentstroke}%
\pgfsetdash{}{0pt}%
\pgfpathmoveto{\pgfqpoint{0.000000in}{0.000000in}}%
\pgfpathlineto{\pgfqpoint{8.000000in}{0.000000in}}%
\pgfpathlineto{\pgfqpoint{8.000000in}{6.000000in}}%
\pgfpathlineto{\pgfqpoint{0.000000in}{6.000000in}}%
\pgfpathlineto{\pgfqpoint{0.000000in}{0.000000in}}%
\pgfpathclose%
\pgfusepath{fill}%
\end{pgfscope}%
\begin{pgfscope}%
\pgfsetbuttcap%
\pgfsetmiterjoin%
\definecolor{currentfill}{rgb}{1.000000,1.000000,1.000000}%
\pgfsetfillcolor{currentfill}%
\pgfsetlinewidth{0.000000pt}%
\definecolor{currentstroke}{rgb}{0.000000,0.000000,0.000000}%
\pgfsetstrokecolor{currentstroke}%
\pgfsetstrokeopacity{0.000000}%
\pgfsetdash{}{0pt}%
\pgfpathmoveto{\pgfqpoint{1.000000in}{0.720000in}}%
\pgfpathlineto{\pgfqpoint{5.800000in}{0.720000in}}%
\pgfpathlineto{\pgfqpoint{5.800000in}{5.340000in}}%
\pgfpathlineto{\pgfqpoint{1.000000in}{5.340000in}}%
\pgfpathlineto{\pgfqpoint{1.000000in}{0.720000in}}%
\pgfpathclose%
\pgfusepath{fill}%
\end{pgfscope}%
\begin{pgfscope}%
\pgfpathrectangle{\pgfqpoint{1.000000in}{0.720000in}}{\pgfqpoint{4.800000in}{4.620000in}}%
\pgfusepath{clip}%
\pgfsetbuttcap%
\pgfsetroundjoin%
\definecolor{currentfill}{rgb}{0.121569,0.466667,0.705882}%
\pgfsetfillcolor{currentfill}%
\pgfsetfillopacity{0.100000}%
\pgfsetlinewidth{0.000000pt}%
\definecolor{currentstroke}{rgb}{0.000000,0.000000,0.000000}%
\pgfsetstrokecolor{currentstroke}%
\pgfsetdash{}{0pt}%
\pgfpathmoveto{\pgfqpoint{1.342857in}{2.719108in}}%
\pgfpathlineto{\pgfqpoint{1.342857in}{1.028409in}}%
\pgfpathlineto{\pgfqpoint{1.685714in}{1.850487in}}%
\pgfpathlineto{\pgfqpoint{2.028571in}{2.190866in}}%
\pgfpathlineto{\pgfqpoint{2.371429in}{2.432418in}}%
\pgfpathlineto{\pgfqpoint{2.714286in}{2.588121in}}%
\pgfpathlineto{\pgfqpoint{3.057143in}{2.653915in}}%
\pgfpathlineto{\pgfqpoint{3.400000in}{2.724629in}}%
\pgfpathlineto{\pgfqpoint{3.742857in}{2.787017in}}%
\pgfpathlineto{\pgfqpoint{4.085714in}{2.805802in}}%
\pgfpathlineto{\pgfqpoint{4.428571in}{2.904855in}}%
\pgfpathlineto{\pgfqpoint{4.771429in}{2.940827in}}%
\pgfpathlineto{\pgfqpoint{5.114286in}{3.158127in}}%
\pgfpathlineto{\pgfqpoint{5.457143in}{3.069866in}}%
\pgfpathlineto{\pgfqpoint{5.457143in}{3.069866in}}%
\pgfpathlineto{\pgfqpoint{5.457143in}{3.069866in}}%
\pgfpathlineto{\pgfqpoint{5.114286in}{3.429061in}}%
\pgfpathlineto{\pgfqpoint{4.771429in}{3.980115in}}%
\pgfpathlineto{\pgfqpoint{4.428571in}{3.900433in}}%
\pgfpathlineto{\pgfqpoint{4.085714in}{3.909542in}}%
\pgfpathlineto{\pgfqpoint{3.742857in}{4.039470in}}%
\pgfpathlineto{\pgfqpoint{3.400000in}{4.176844in}}%
\pgfpathlineto{\pgfqpoint{3.057143in}{4.119591in}}%
\pgfpathlineto{\pgfqpoint{2.714286in}{4.131161in}}%
\pgfpathlineto{\pgfqpoint{2.371429in}{4.132991in}}%
\pgfpathlineto{\pgfqpoint{2.028571in}{3.937394in}}%
\pgfpathlineto{\pgfqpoint{1.685714in}{3.617835in}}%
\pgfpathlineto{\pgfqpoint{1.342857in}{2.719108in}}%
\pgfpathlineto{\pgfqpoint{1.342857in}{2.719108in}}%
\pgfpathclose%
\pgfusepath{fill}%
\end{pgfscope}%
\begin{pgfscope}%
\pgfsetbuttcap%
\pgfsetroundjoin%
\definecolor{currentfill}{rgb}{0.000000,0.000000,0.000000}%
\pgfsetfillcolor{currentfill}%
\pgfsetlinewidth{0.803000pt}%
\definecolor{currentstroke}{rgb}{0.000000,0.000000,0.000000}%
\pgfsetstrokecolor{currentstroke}%
\pgfsetdash{}{0pt}%
\pgfsys@defobject{currentmarker}{\pgfqpoint{0.000000in}{-0.048611in}}{\pgfqpoint{0.000000in}{0.000000in}}{%
\pgfpathmoveto{\pgfqpoint{0.000000in}{0.000000in}}%
\pgfpathlineto{\pgfqpoint{0.000000in}{-0.048611in}}%
\pgfusepath{stroke,fill}%
}%
\begin{pgfscope}%
\pgfsys@transformshift{1.342857in}{0.720000in}%
\pgfsys@useobject{currentmarker}{}%
\end{pgfscope}%
\end{pgfscope}%
\begin{pgfscope}%
\definecolor{textcolor}{rgb}{0.000000,0.000000,0.000000}%
\pgfsetstrokecolor{textcolor}%
\pgfsetfillcolor{textcolor}%
\pgftext[x=1.342857in,y=0.622778in,,top]{\color{textcolor}\sffamily\fontsize{20.000000}{24.000000}\selectfont 1}%
\end{pgfscope}%
\begin{pgfscope}%
\pgfsetbuttcap%
\pgfsetroundjoin%
\definecolor{currentfill}{rgb}{0.000000,0.000000,0.000000}%
\pgfsetfillcolor{currentfill}%
\pgfsetlinewidth{0.803000pt}%
\definecolor{currentstroke}{rgb}{0.000000,0.000000,0.000000}%
\pgfsetstrokecolor{currentstroke}%
\pgfsetdash{}{0pt}%
\pgfsys@defobject{currentmarker}{\pgfqpoint{0.000000in}{-0.048611in}}{\pgfqpoint{0.000000in}{0.000000in}}{%
\pgfpathmoveto{\pgfqpoint{0.000000in}{0.000000in}}%
\pgfpathlineto{\pgfqpoint{0.000000in}{-0.048611in}}%
\pgfusepath{stroke,fill}%
}%
\begin{pgfscope}%
\pgfsys@transformshift{2.028571in}{0.720000in}%
\pgfsys@useobject{currentmarker}{}%
\end{pgfscope}%
\end{pgfscope}%
\begin{pgfscope}%
\definecolor{textcolor}{rgb}{0.000000,0.000000,0.000000}%
\pgfsetstrokecolor{textcolor}%
\pgfsetfillcolor{textcolor}%
\pgftext[x=2.028571in,y=0.622778in,,top]{\color{textcolor}\sffamily\fontsize{20.000000}{24.000000}\selectfont 3}%
\end{pgfscope}%
\begin{pgfscope}%
\pgfsetbuttcap%
\pgfsetroundjoin%
\definecolor{currentfill}{rgb}{0.000000,0.000000,0.000000}%
\pgfsetfillcolor{currentfill}%
\pgfsetlinewidth{0.803000pt}%
\definecolor{currentstroke}{rgb}{0.000000,0.000000,0.000000}%
\pgfsetstrokecolor{currentstroke}%
\pgfsetdash{}{0pt}%
\pgfsys@defobject{currentmarker}{\pgfqpoint{0.000000in}{-0.048611in}}{\pgfqpoint{0.000000in}{0.000000in}}{%
\pgfpathmoveto{\pgfqpoint{0.000000in}{0.000000in}}%
\pgfpathlineto{\pgfqpoint{0.000000in}{-0.048611in}}%
\pgfusepath{stroke,fill}%
}%
\begin{pgfscope}%
\pgfsys@transformshift{2.714286in}{0.720000in}%
\pgfsys@useobject{currentmarker}{}%
\end{pgfscope}%
\end{pgfscope}%
\begin{pgfscope}%
\definecolor{textcolor}{rgb}{0.000000,0.000000,0.000000}%
\pgfsetstrokecolor{textcolor}%
\pgfsetfillcolor{textcolor}%
\pgftext[x=2.714286in,y=0.622778in,,top]{\color{textcolor}\sffamily\fontsize{20.000000}{24.000000}\selectfont 5}%
\end{pgfscope}%
\begin{pgfscope}%
\pgfsetbuttcap%
\pgfsetroundjoin%
\definecolor{currentfill}{rgb}{0.000000,0.000000,0.000000}%
\pgfsetfillcolor{currentfill}%
\pgfsetlinewidth{0.803000pt}%
\definecolor{currentstroke}{rgb}{0.000000,0.000000,0.000000}%
\pgfsetstrokecolor{currentstroke}%
\pgfsetdash{}{0pt}%
\pgfsys@defobject{currentmarker}{\pgfqpoint{0.000000in}{-0.048611in}}{\pgfqpoint{0.000000in}{0.000000in}}{%
\pgfpathmoveto{\pgfqpoint{0.000000in}{0.000000in}}%
\pgfpathlineto{\pgfqpoint{0.000000in}{-0.048611in}}%
\pgfusepath{stroke,fill}%
}%
\begin{pgfscope}%
\pgfsys@transformshift{3.400000in}{0.720000in}%
\pgfsys@useobject{currentmarker}{}%
\end{pgfscope}%
\end{pgfscope}%
\begin{pgfscope}%
\definecolor{textcolor}{rgb}{0.000000,0.000000,0.000000}%
\pgfsetstrokecolor{textcolor}%
\pgfsetfillcolor{textcolor}%
\pgftext[x=3.400000in,y=0.622778in,,top]{\color{textcolor}\sffamily\fontsize{20.000000}{24.000000}\selectfont 7}%
\end{pgfscope}%
\begin{pgfscope}%
\pgfsetbuttcap%
\pgfsetroundjoin%
\definecolor{currentfill}{rgb}{0.000000,0.000000,0.000000}%
\pgfsetfillcolor{currentfill}%
\pgfsetlinewidth{0.803000pt}%
\definecolor{currentstroke}{rgb}{0.000000,0.000000,0.000000}%
\pgfsetstrokecolor{currentstroke}%
\pgfsetdash{}{0pt}%
\pgfsys@defobject{currentmarker}{\pgfqpoint{0.000000in}{-0.048611in}}{\pgfqpoint{0.000000in}{0.000000in}}{%
\pgfpathmoveto{\pgfqpoint{0.000000in}{0.000000in}}%
\pgfpathlineto{\pgfqpoint{0.000000in}{-0.048611in}}%
\pgfusepath{stroke,fill}%
}%
\begin{pgfscope}%
\pgfsys@transformshift{4.085714in}{0.720000in}%
\pgfsys@useobject{currentmarker}{}%
\end{pgfscope}%
\end{pgfscope}%
\begin{pgfscope}%
\definecolor{textcolor}{rgb}{0.000000,0.000000,0.000000}%
\pgfsetstrokecolor{textcolor}%
\pgfsetfillcolor{textcolor}%
\pgftext[x=4.085714in,y=0.622778in,,top]{\color{textcolor}\sffamily\fontsize{20.000000}{24.000000}\selectfont 9}%
\end{pgfscope}%
\begin{pgfscope}%
\pgfsetbuttcap%
\pgfsetroundjoin%
\definecolor{currentfill}{rgb}{0.000000,0.000000,0.000000}%
\pgfsetfillcolor{currentfill}%
\pgfsetlinewidth{0.803000pt}%
\definecolor{currentstroke}{rgb}{0.000000,0.000000,0.000000}%
\pgfsetstrokecolor{currentstroke}%
\pgfsetdash{}{0pt}%
\pgfsys@defobject{currentmarker}{\pgfqpoint{0.000000in}{-0.048611in}}{\pgfqpoint{0.000000in}{0.000000in}}{%
\pgfpathmoveto{\pgfqpoint{0.000000in}{0.000000in}}%
\pgfpathlineto{\pgfqpoint{0.000000in}{-0.048611in}}%
\pgfusepath{stroke,fill}%
}%
\begin{pgfscope}%
\pgfsys@transformshift{4.771429in}{0.720000in}%
\pgfsys@useobject{currentmarker}{}%
\end{pgfscope}%
\end{pgfscope}%
\begin{pgfscope}%
\definecolor{textcolor}{rgb}{0.000000,0.000000,0.000000}%
\pgfsetstrokecolor{textcolor}%
\pgfsetfillcolor{textcolor}%
\pgftext[x=4.771429in,y=0.622778in,,top]{\color{textcolor}\sffamily\fontsize{20.000000}{24.000000}\selectfont 11}%
\end{pgfscope}%
\begin{pgfscope}%
\pgfsetbuttcap%
\pgfsetroundjoin%
\definecolor{currentfill}{rgb}{0.000000,0.000000,0.000000}%
\pgfsetfillcolor{currentfill}%
\pgfsetlinewidth{0.803000pt}%
\definecolor{currentstroke}{rgb}{0.000000,0.000000,0.000000}%
\pgfsetstrokecolor{currentstroke}%
\pgfsetdash{}{0pt}%
\pgfsys@defobject{currentmarker}{\pgfqpoint{0.000000in}{-0.048611in}}{\pgfqpoint{0.000000in}{0.000000in}}{%
\pgfpathmoveto{\pgfqpoint{0.000000in}{0.000000in}}%
\pgfpathlineto{\pgfqpoint{0.000000in}{-0.048611in}}%
\pgfusepath{stroke,fill}%
}%
\begin{pgfscope}%
\pgfsys@transformshift{5.457143in}{0.720000in}%
\pgfsys@useobject{currentmarker}{}%
\end{pgfscope}%
\end{pgfscope}%
\begin{pgfscope}%
\definecolor{textcolor}{rgb}{0.000000,0.000000,0.000000}%
\pgfsetstrokecolor{textcolor}%
\pgfsetfillcolor{textcolor}%
\pgftext[x=5.457143in,y=0.622778in,,top]{\color{textcolor}\sffamily\fontsize{20.000000}{24.000000}\selectfont 13}%
\end{pgfscope}%
\begin{pgfscope}%
\definecolor{textcolor}{rgb}{0.000000,0.000000,0.000000}%
\pgfsetstrokecolor{textcolor}%
\pgfsetfillcolor{textcolor}%
\pgftext[x=3.400000in,y=0.311155in,,top]{\color{textcolor}\sffamily\fontsize{20.000000}{24.000000}\selectfont \(\displaystyle N_{\mathrm{PE}}\)}%
\end{pgfscope}%
\begin{pgfscope}%
\pgfsetbuttcap%
\pgfsetroundjoin%
\definecolor{currentfill}{rgb}{0.000000,0.000000,0.000000}%
\pgfsetfillcolor{currentfill}%
\pgfsetlinewidth{0.803000pt}%
\definecolor{currentstroke}{rgb}{0.000000,0.000000,0.000000}%
\pgfsetstrokecolor{currentstroke}%
\pgfsetdash{}{0pt}%
\pgfsys@defobject{currentmarker}{\pgfqpoint{-0.048611in}{0.000000in}}{\pgfqpoint{-0.000000in}{0.000000in}}{%
\pgfpathmoveto{\pgfqpoint{-0.000000in}{0.000000in}}%
\pgfpathlineto{\pgfqpoint{-0.048611in}{0.000000in}}%
\pgfusepath{stroke,fill}%
}%
\begin{pgfscope}%
\pgfsys@transformshift{1.000000in}{0.720000in}%
\pgfsys@useobject{currentmarker}{}%
\end{pgfscope}%
\end{pgfscope}%
\begin{pgfscope}%
\definecolor{textcolor}{rgb}{0.000000,0.000000,0.000000}%
\pgfsetstrokecolor{textcolor}%
\pgfsetfillcolor{textcolor}%
\pgftext[x=0.560215in, y=0.619981in, left, base]{\color{textcolor}\sffamily\fontsize{20.000000}{24.000000}\selectfont \(\displaystyle {0.0}\)}%
\end{pgfscope}%
\begin{pgfscope}%
\pgfsetbuttcap%
\pgfsetroundjoin%
\definecolor{currentfill}{rgb}{0.000000,0.000000,0.000000}%
\pgfsetfillcolor{currentfill}%
\pgfsetlinewidth{0.803000pt}%
\definecolor{currentstroke}{rgb}{0.000000,0.000000,0.000000}%
\pgfsetstrokecolor{currentstroke}%
\pgfsetdash{}{0pt}%
\pgfsys@defobject{currentmarker}{\pgfqpoint{-0.048611in}{0.000000in}}{\pgfqpoint{-0.000000in}{0.000000in}}{%
\pgfpathmoveto{\pgfqpoint{-0.000000in}{0.000000in}}%
\pgfpathlineto{\pgfqpoint{-0.048611in}{0.000000in}}%
\pgfusepath{stroke,fill}%
}%
\begin{pgfscope}%
\pgfsys@transformshift{1.000000in}{1.539365in}%
\pgfsys@useobject{currentmarker}{}%
\end{pgfscope}%
\end{pgfscope}%
\begin{pgfscope}%
\definecolor{textcolor}{rgb}{0.000000,0.000000,0.000000}%
\pgfsetstrokecolor{textcolor}%
\pgfsetfillcolor{textcolor}%
\pgftext[x=0.560215in, y=1.439346in, left, base]{\color{textcolor}\sffamily\fontsize{20.000000}{24.000000}\selectfont \(\displaystyle {0.2}\)}%
\end{pgfscope}%
\begin{pgfscope}%
\pgfsetbuttcap%
\pgfsetroundjoin%
\definecolor{currentfill}{rgb}{0.000000,0.000000,0.000000}%
\pgfsetfillcolor{currentfill}%
\pgfsetlinewidth{0.803000pt}%
\definecolor{currentstroke}{rgb}{0.000000,0.000000,0.000000}%
\pgfsetstrokecolor{currentstroke}%
\pgfsetdash{}{0pt}%
\pgfsys@defobject{currentmarker}{\pgfqpoint{-0.048611in}{0.000000in}}{\pgfqpoint{-0.000000in}{0.000000in}}{%
\pgfpathmoveto{\pgfqpoint{-0.000000in}{0.000000in}}%
\pgfpathlineto{\pgfqpoint{-0.048611in}{0.000000in}}%
\pgfusepath{stroke,fill}%
}%
\begin{pgfscope}%
\pgfsys@transformshift{1.000000in}{2.358731in}%
\pgfsys@useobject{currentmarker}{}%
\end{pgfscope}%
\end{pgfscope}%
\begin{pgfscope}%
\definecolor{textcolor}{rgb}{0.000000,0.000000,0.000000}%
\pgfsetstrokecolor{textcolor}%
\pgfsetfillcolor{textcolor}%
\pgftext[x=0.560215in, y=2.258712in, left, base]{\color{textcolor}\sffamily\fontsize{20.000000}{24.000000}\selectfont \(\displaystyle {0.4}\)}%
\end{pgfscope}%
\begin{pgfscope}%
\pgfsetbuttcap%
\pgfsetroundjoin%
\definecolor{currentfill}{rgb}{0.000000,0.000000,0.000000}%
\pgfsetfillcolor{currentfill}%
\pgfsetlinewidth{0.803000pt}%
\definecolor{currentstroke}{rgb}{0.000000,0.000000,0.000000}%
\pgfsetstrokecolor{currentstroke}%
\pgfsetdash{}{0pt}%
\pgfsys@defobject{currentmarker}{\pgfqpoint{-0.048611in}{0.000000in}}{\pgfqpoint{-0.000000in}{0.000000in}}{%
\pgfpathmoveto{\pgfqpoint{-0.000000in}{0.000000in}}%
\pgfpathlineto{\pgfqpoint{-0.048611in}{0.000000in}}%
\pgfusepath{stroke,fill}%
}%
\begin{pgfscope}%
\pgfsys@transformshift{1.000000in}{3.178096in}%
\pgfsys@useobject{currentmarker}{}%
\end{pgfscope}%
\end{pgfscope}%
\begin{pgfscope}%
\definecolor{textcolor}{rgb}{0.000000,0.000000,0.000000}%
\pgfsetstrokecolor{textcolor}%
\pgfsetfillcolor{textcolor}%
\pgftext[x=0.560215in, y=3.078077in, left, base]{\color{textcolor}\sffamily\fontsize{20.000000}{24.000000}\selectfont \(\displaystyle {0.6}\)}%
\end{pgfscope}%
\begin{pgfscope}%
\pgfsetbuttcap%
\pgfsetroundjoin%
\definecolor{currentfill}{rgb}{0.000000,0.000000,0.000000}%
\pgfsetfillcolor{currentfill}%
\pgfsetlinewidth{0.803000pt}%
\definecolor{currentstroke}{rgb}{0.000000,0.000000,0.000000}%
\pgfsetstrokecolor{currentstroke}%
\pgfsetdash{}{0pt}%
\pgfsys@defobject{currentmarker}{\pgfqpoint{-0.048611in}{0.000000in}}{\pgfqpoint{-0.000000in}{0.000000in}}{%
\pgfpathmoveto{\pgfqpoint{-0.000000in}{0.000000in}}%
\pgfpathlineto{\pgfqpoint{-0.048611in}{0.000000in}}%
\pgfusepath{stroke,fill}%
}%
\begin{pgfscope}%
\pgfsys@transformshift{1.000000in}{3.997462in}%
\pgfsys@useobject{currentmarker}{}%
\end{pgfscope}%
\end{pgfscope}%
\begin{pgfscope}%
\definecolor{textcolor}{rgb}{0.000000,0.000000,0.000000}%
\pgfsetstrokecolor{textcolor}%
\pgfsetfillcolor{textcolor}%
\pgftext[x=0.560215in, y=3.897443in, left, base]{\color{textcolor}\sffamily\fontsize{20.000000}{24.000000}\selectfont \(\displaystyle {0.8}\)}%
\end{pgfscope}%
\begin{pgfscope}%
\pgfsetbuttcap%
\pgfsetroundjoin%
\definecolor{currentfill}{rgb}{0.000000,0.000000,0.000000}%
\pgfsetfillcolor{currentfill}%
\pgfsetlinewidth{0.803000pt}%
\definecolor{currentstroke}{rgb}{0.000000,0.000000,0.000000}%
\pgfsetstrokecolor{currentstroke}%
\pgfsetdash{}{0pt}%
\pgfsys@defobject{currentmarker}{\pgfqpoint{-0.048611in}{0.000000in}}{\pgfqpoint{-0.000000in}{0.000000in}}{%
\pgfpathmoveto{\pgfqpoint{-0.000000in}{0.000000in}}%
\pgfpathlineto{\pgfqpoint{-0.048611in}{0.000000in}}%
\pgfusepath{stroke,fill}%
}%
\begin{pgfscope}%
\pgfsys@transformshift{1.000000in}{4.816827in}%
\pgfsys@useobject{currentmarker}{}%
\end{pgfscope}%
\end{pgfscope}%
\begin{pgfscope}%
\definecolor{textcolor}{rgb}{0.000000,0.000000,0.000000}%
\pgfsetstrokecolor{textcolor}%
\pgfsetfillcolor{textcolor}%
\pgftext[x=0.560215in, y=4.716808in, left, base]{\color{textcolor}\sffamily\fontsize{20.000000}{24.000000}\selectfont \(\displaystyle {1.0}\)}%
\end{pgfscope}%
\begin{pgfscope}%
\definecolor{textcolor}{rgb}{0.000000,0.000000,0.000000}%
\pgfsetstrokecolor{textcolor}%
\pgfsetfillcolor{textcolor}%
\pgftext[x=0.504660in,y=3.030000in,,bottom,rotate=90.000000]{\color{textcolor}\sffamily\fontsize{20.000000}{24.000000}\selectfont \(\displaystyle \mathrm{Wasserstein\ Distance}/\si{ns}\)}%
\end{pgfscope}%
\begin{pgfscope}%
\pgfpathrectangle{\pgfqpoint{1.000000in}{0.720000in}}{\pgfqpoint{4.800000in}{4.620000in}}%
\pgfusepath{clip}%
\pgfsetrectcap%
\pgfsetroundjoin%
\pgfsetlinewidth{2.007500pt}%
\definecolor{currentstroke}{rgb}{0.000000,0.000000,1.000000}%
\pgfsetstrokecolor{currentstroke}%
\pgfsetdash{}{0pt}%
\pgfpathmoveto{\pgfqpoint{1.342857in}{1.794427in}}%
\pgfpathlineto{\pgfqpoint{1.685714in}{2.571287in}}%
\pgfpathlineto{\pgfqpoint{2.028571in}{2.934216in}}%
\pgfpathlineto{\pgfqpoint{2.371429in}{3.113787in}}%
\pgfpathlineto{\pgfqpoint{2.714286in}{3.224888in}}%
\pgfpathlineto{\pgfqpoint{3.057143in}{3.309849in}}%
\pgfpathlineto{\pgfqpoint{3.400000in}{3.317158in}}%
\pgfpathlineto{\pgfqpoint{3.742857in}{3.366817in}}%
\pgfpathlineto{\pgfqpoint{4.085714in}{3.290004in}}%
\pgfpathlineto{\pgfqpoint{4.428571in}{3.308514in}}%
\pgfpathlineto{\pgfqpoint{4.771429in}{3.462108in}}%
\pgfpathlineto{\pgfqpoint{5.114286in}{3.326009in}}%
\pgfpathlineto{\pgfqpoint{5.457143in}{3.069866in}}%
\pgfusepath{stroke}%
\end{pgfscope}%
\begin{pgfscope}%
\pgfpathrectangle{\pgfqpoint{1.000000in}{0.720000in}}{\pgfqpoint{4.800000in}{4.620000in}}%
\pgfusepath{clip}%
\pgfsetbuttcap%
\pgfsetroundjoin%
\pgfsetlinewidth{1.003750pt}%
\definecolor{currentstroke}{rgb}{0.000000,0.000000,1.000000}%
\pgfsetstrokecolor{currentstroke}%
\pgfsetdash{}{0pt}%
\pgfpathmoveto{\pgfqpoint{1.342857in}{1.028409in}}%
\pgfpathlineto{\pgfqpoint{1.342857in}{2.719108in}}%
\pgfusepath{stroke}%
\end{pgfscope}%
\begin{pgfscope}%
\pgfpathrectangle{\pgfqpoint{1.000000in}{0.720000in}}{\pgfqpoint{4.800000in}{4.620000in}}%
\pgfusepath{clip}%
\pgfsetbuttcap%
\pgfsetroundjoin%
\pgfsetlinewidth{1.003750pt}%
\definecolor{currentstroke}{rgb}{0.000000,0.000000,1.000000}%
\pgfsetstrokecolor{currentstroke}%
\pgfsetdash{}{0pt}%
\pgfpathmoveto{\pgfqpoint{1.685714in}{1.850487in}}%
\pgfpathlineto{\pgfqpoint{1.685714in}{3.617835in}}%
\pgfusepath{stroke}%
\end{pgfscope}%
\begin{pgfscope}%
\pgfpathrectangle{\pgfqpoint{1.000000in}{0.720000in}}{\pgfqpoint{4.800000in}{4.620000in}}%
\pgfusepath{clip}%
\pgfsetbuttcap%
\pgfsetroundjoin%
\pgfsetlinewidth{1.003750pt}%
\definecolor{currentstroke}{rgb}{0.000000,0.000000,1.000000}%
\pgfsetstrokecolor{currentstroke}%
\pgfsetdash{}{0pt}%
\pgfpathmoveto{\pgfqpoint{2.028571in}{2.190866in}}%
\pgfpathlineto{\pgfqpoint{2.028571in}{3.937394in}}%
\pgfusepath{stroke}%
\end{pgfscope}%
\begin{pgfscope}%
\pgfpathrectangle{\pgfqpoint{1.000000in}{0.720000in}}{\pgfqpoint{4.800000in}{4.620000in}}%
\pgfusepath{clip}%
\pgfsetbuttcap%
\pgfsetroundjoin%
\pgfsetlinewidth{1.003750pt}%
\definecolor{currentstroke}{rgb}{0.000000,0.000000,1.000000}%
\pgfsetstrokecolor{currentstroke}%
\pgfsetdash{}{0pt}%
\pgfpathmoveto{\pgfqpoint{2.371429in}{2.432418in}}%
\pgfpathlineto{\pgfqpoint{2.371429in}{4.132991in}}%
\pgfusepath{stroke}%
\end{pgfscope}%
\begin{pgfscope}%
\pgfpathrectangle{\pgfqpoint{1.000000in}{0.720000in}}{\pgfqpoint{4.800000in}{4.620000in}}%
\pgfusepath{clip}%
\pgfsetbuttcap%
\pgfsetroundjoin%
\pgfsetlinewidth{1.003750pt}%
\definecolor{currentstroke}{rgb}{0.000000,0.000000,1.000000}%
\pgfsetstrokecolor{currentstroke}%
\pgfsetdash{}{0pt}%
\pgfpathmoveto{\pgfqpoint{2.714286in}{2.588121in}}%
\pgfpathlineto{\pgfqpoint{2.714286in}{4.131161in}}%
\pgfusepath{stroke}%
\end{pgfscope}%
\begin{pgfscope}%
\pgfpathrectangle{\pgfqpoint{1.000000in}{0.720000in}}{\pgfqpoint{4.800000in}{4.620000in}}%
\pgfusepath{clip}%
\pgfsetbuttcap%
\pgfsetroundjoin%
\pgfsetlinewidth{1.003750pt}%
\definecolor{currentstroke}{rgb}{0.000000,0.000000,1.000000}%
\pgfsetstrokecolor{currentstroke}%
\pgfsetdash{}{0pt}%
\pgfpathmoveto{\pgfqpoint{3.057143in}{2.653915in}}%
\pgfpathlineto{\pgfqpoint{3.057143in}{4.119591in}}%
\pgfusepath{stroke}%
\end{pgfscope}%
\begin{pgfscope}%
\pgfpathrectangle{\pgfqpoint{1.000000in}{0.720000in}}{\pgfqpoint{4.800000in}{4.620000in}}%
\pgfusepath{clip}%
\pgfsetbuttcap%
\pgfsetroundjoin%
\pgfsetlinewidth{1.003750pt}%
\definecolor{currentstroke}{rgb}{0.000000,0.000000,1.000000}%
\pgfsetstrokecolor{currentstroke}%
\pgfsetdash{}{0pt}%
\pgfpathmoveto{\pgfqpoint{3.400000in}{2.724629in}}%
\pgfpathlineto{\pgfqpoint{3.400000in}{4.176844in}}%
\pgfusepath{stroke}%
\end{pgfscope}%
\begin{pgfscope}%
\pgfpathrectangle{\pgfqpoint{1.000000in}{0.720000in}}{\pgfqpoint{4.800000in}{4.620000in}}%
\pgfusepath{clip}%
\pgfsetbuttcap%
\pgfsetroundjoin%
\pgfsetlinewidth{1.003750pt}%
\definecolor{currentstroke}{rgb}{0.000000,0.000000,1.000000}%
\pgfsetstrokecolor{currentstroke}%
\pgfsetdash{}{0pt}%
\pgfpathmoveto{\pgfqpoint{3.742857in}{2.787017in}}%
\pgfpathlineto{\pgfqpoint{3.742857in}{4.039470in}}%
\pgfusepath{stroke}%
\end{pgfscope}%
\begin{pgfscope}%
\pgfpathrectangle{\pgfqpoint{1.000000in}{0.720000in}}{\pgfqpoint{4.800000in}{4.620000in}}%
\pgfusepath{clip}%
\pgfsetbuttcap%
\pgfsetroundjoin%
\pgfsetlinewidth{1.003750pt}%
\definecolor{currentstroke}{rgb}{0.000000,0.000000,1.000000}%
\pgfsetstrokecolor{currentstroke}%
\pgfsetdash{}{0pt}%
\pgfpathmoveto{\pgfqpoint{4.085714in}{2.805802in}}%
\pgfpathlineto{\pgfqpoint{4.085714in}{3.909542in}}%
\pgfusepath{stroke}%
\end{pgfscope}%
\begin{pgfscope}%
\pgfpathrectangle{\pgfqpoint{1.000000in}{0.720000in}}{\pgfqpoint{4.800000in}{4.620000in}}%
\pgfusepath{clip}%
\pgfsetbuttcap%
\pgfsetroundjoin%
\pgfsetlinewidth{1.003750pt}%
\definecolor{currentstroke}{rgb}{0.000000,0.000000,1.000000}%
\pgfsetstrokecolor{currentstroke}%
\pgfsetdash{}{0pt}%
\pgfpathmoveto{\pgfqpoint{4.428571in}{2.904855in}}%
\pgfpathlineto{\pgfqpoint{4.428571in}{3.900433in}}%
\pgfusepath{stroke}%
\end{pgfscope}%
\begin{pgfscope}%
\pgfpathrectangle{\pgfqpoint{1.000000in}{0.720000in}}{\pgfqpoint{4.800000in}{4.620000in}}%
\pgfusepath{clip}%
\pgfsetbuttcap%
\pgfsetroundjoin%
\pgfsetlinewidth{1.003750pt}%
\definecolor{currentstroke}{rgb}{0.000000,0.000000,1.000000}%
\pgfsetstrokecolor{currentstroke}%
\pgfsetdash{}{0pt}%
\pgfpathmoveto{\pgfqpoint{4.771429in}{2.940827in}}%
\pgfpathlineto{\pgfqpoint{4.771429in}{3.980115in}}%
\pgfusepath{stroke}%
\end{pgfscope}%
\begin{pgfscope}%
\pgfpathrectangle{\pgfqpoint{1.000000in}{0.720000in}}{\pgfqpoint{4.800000in}{4.620000in}}%
\pgfusepath{clip}%
\pgfsetbuttcap%
\pgfsetroundjoin%
\pgfsetlinewidth{1.003750pt}%
\definecolor{currentstroke}{rgb}{0.000000,0.000000,1.000000}%
\pgfsetstrokecolor{currentstroke}%
\pgfsetdash{}{0pt}%
\pgfpathmoveto{\pgfqpoint{5.114286in}{3.158127in}}%
\pgfpathlineto{\pgfqpoint{5.114286in}{3.429061in}}%
\pgfusepath{stroke}%
\end{pgfscope}%
\begin{pgfscope}%
\pgfpathrectangle{\pgfqpoint{1.000000in}{0.720000in}}{\pgfqpoint{4.800000in}{4.620000in}}%
\pgfusepath{clip}%
\pgfsetbuttcap%
\pgfsetroundjoin%
\pgfsetlinewidth{1.003750pt}%
\definecolor{currentstroke}{rgb}{0.000000,0.000000,1.000000}%
\pgfsetstrokecolor{currentstroke}%
\pgfsetdash{}{0pt}%
\pgfpathmoveto{\pgfqpoint{5.457143in}{3.069866in}}%
\pgfpathlineto{\pgfqpoint{5.457143in}{3.069866in}}%
\pgfusepath{stroke}%
\end{pgfscope}%
\begin{pgfscope}%
\pgfpathrectangle{\pgfqpoint{1.000000in}{0.720000in}}{\pgfqpoint{4.800000in}{4.620000in}}%
\pgfusepath{clip}%
\pgfsetbuttcap%
\pgfsetroundjoin%
\definecolor{currentfill}{rgb}{0.000000,0.000000,1.000000}%
\pgfsetfillcolor{currentfill}%
\pgfsetlinewidth{1.003750pt}%
\definecolor{currentstroke}{rgb}{0.000000,0.000000,1.000000}%
\pgfsetstrokecolor{currentstroke}%
\pgfsetdash{}{0pt}%
\pgfsys@defobject{currentmarker}{\pgfqpoint{-0.041667in}{-0.000000in}}{\pgfqpoint{0.041667in}{0.000000in}}{%
\pgfpathmoveto{\pgfqpoint{0.041667in}{-0.000000in}}%
\pgfpathlineto{\pgfqpoint{-0.041667in}{0.000000in}}%
\pgfusepath{stroke,fill}%
}%
\begin{pgfscope}%
\pgfsys@transformshift{1.342857in}{1.028409in}%
\pgfsys@useobject{currentmarker}{}%
\end{pgfscope}%
\begin{pgfscope}%
\pgfsys@transformshift{1.685714in}{1.850487in}%
\pgfsys@useobject{currentmarker}{}%
\end{pgfscope}%
\begin{pgfscope}%
\pgfsys@transformshift{2.028571in}{2.190866in}%
\pgfsys@useobject{currentmarker}{}%
\end{pgfscope}%
\begin{pgfscope}%
\pgfsys@transformshift{2.371429in}{2.432418in}%
\pgfsys@useobject{currentmarker}{}%
\end{pgfscope}%
\begin{pgfscope}%
\pgfsys@transformshift{2.714286in}{2.588121in}%
\pgfsys@useobject{currentmarker}{}%
\end{pgfscope}%
\begin{pgfscope}%
\pgfsys@transformshift{3.057143in}{2.653915in}%
\pgfsys@useobject{currentmarker}{}%
\end{pgfscope}%
\begin{pgfscope}%
\pgfsys@transformshift{3.400000in}{2.724629in}%
\pgfsys@useobject{currentmarker}{}%
\end{pgfscope}%
\begin{pgfscope}%
\pgfsys@transformshift{3.742857in}{2.787017in}%
\pgfsys@useobject{currentmarker}{}%
\end{pgfscope}%
\begin{pgfscope}%
\pgfsys@transformshift{4.085714in}{2.805802in}%
\pgfsys@useobject{currentmarker}{}%
\end{pgfscope}%
\begin{pgfscope}%
\pgfsys@transformshift{4.428571in}{2.904855in}%
\pgfsys@useobject{currentmarker}{}%
\end{pgfscope}%
\begin{pgfscope}%
\pgfsys@transformshift{4.771429in}{2.940827in}%
\pgfsys@useobject{currentmarker}{}%
\end{pgfscope}%
\begin{pgfscope}%
\pgfsys@transformshift{5.114286in}{3.158127in}%
\pgfsys@useobject{currentmarker}{}%
\end{pgfscope}%
\begin{pgfscope}%
\pgfsys@transformshift{5.457143in}{3.069866in}%
\pgfsys@useobject{currentmarker}{}%
\end{pgfscope}%
\end{pgfscope}%
\begin{pgfscope}%
\pgfpathrectangle{\pgfqpoint{1.000000in}{0.720000in}}{\pgfqpoint{4.800000in}{4.620000in}}%
\pgfusepath{clip}%
\pgfsetbuttcap%
\pgfsetroundjoin%
\definecolor{currentfill}{rgb}{0.000000,0.000000,1.000000}%
\pgfsetfillcolor{currentfill}%
\pgfsetlinewidth{1.003750pt}%
\definecolor{currentstroke}{rgb}{0.000000,0.000000,1.000000}%
\pgfsetstrokecolor{currentstroke}%
\pgfsetdash{}{0pt}%
\pgfsys@defobject{currentmarker}{\pgfqpoint{-0.041667in}{-0.000000in}}{\pgfqpoint{0.041667in}{0.000000in}}{%
\pgfpathmoveto{\pgfqpoint{0.041667in}{-0.000000in}}%
\pgfpathlineto{\pgfqpoint{-0.041667in}{0.000000in}}%
\pgfusepath{stroke,fill}%
}%
\begin{pgfscope}%
\pgfsys@transformshift{1.342857in}{2.719108in}%
\pgfsys@useobject{currentmarker}{}%
\end{pgfscope}%
\begin{pgfscope}%
\pgfsys@transformshift{1.685714in}{3.617835in}%
\pgfsys@useobject{currentmarker}{}%
\end{pgfscope}%
\begin{pgfscope}%
\pgfsys@transformshift{2.028571in}{3.937394in}%
\pgfsys@useobject{currentmarker}{}%
\end{pgfscope}%
\begin{pgfscope}%
\pgfsys@transformshift{2.371429in}{4.132991in}%
\pgfsys@useobject{currentmarker}{}%
\end{pgfscope}%
\begin{pgfscope}%
\pgfsys@transformshift{2.714286in}{4.131161in}%
\pgfsys@useobject{currentmarker}{}%
\end{pgfscope}%
\begin{pgfscope}%
\pgfsys@transformshift{3.057143in}{4.119591in}%
\pgfsys@useobject{currentmarker}{}%
\end{pgfscope}%
\begin{pgfscope}%
\pgfsys@transformshift{3.400000in}{4.176844in}%
\pgfsys@useobject{currentmarker}{}%
\end{pgfscope}%
\begin{pgfscope}%
\pgfsys@transformshift{3.742857in}{4.039470in}%
\pgfsys@useobject{currentmarker}{}%
\end{pgfscope}%
\begin{pgfscope}%
\pgfsys@transformshift{4.085714in}{3.909542in}%
\pgfsys@useobject{currentmarker}{}%
\end{pgfscope}%
\begin{pgfscope}%
\pgfsys@transformshift{4.428571in}{3.900433in}%
\pgfsys@useobject{currentmarker}{}%
\end{pgfscope}%
\begin{pgfscope}%
\pgfsys@transformshift{4.771429in}{3.980115in}%
\pgfsys@useobject{currentmarker}{}%
\end{pgfscope}%
\begin{pgfscope}%
\pgfsys@transformshift{5.114286in}{3.429061in}%
\pgfsys@useobject{currentmarker}{}%
\end{pgfscope}%
\begin{pgfscope}%
\pgfsys@transformshift{5.457143in}{3.069866in}%
\pgfsys@useobject{currentmarker}{}%
\end{pgfscope}%
\end{pgfscope}%
\begin{pgfscope}%
\pgfpathrectangle{\pgfqpoint{1.000000in}{0.720000in}}{\pgfqpoint{4.800000in}{4.620000in}}%
\pgfusepath{clip}%
\pgfsetbuttcap%
\pgfsetroundjoin%
\definecolor{currentfill}{rgb}{0.000000,0.000000,1.000000}%
\pgfsetfillcolor{currentfill}%
\pgfsetlinewidth{1.003750pt}%
\definecolor{currentstroke}{rgb}{0.000000,0.000000,1.000000}%
\pgfsetstrokecolor{currentstroke}%
\pgfsetdash{}{0pt}%
\pgfsys@defobject{currentmarker}{\pgfqpoint{-0.027778in}{-0.027778in}}{\pgfqpoint{0.027778in}{0.027778in}}{%
\pgfpathmoveto{\pgfqpoint{0.000000in}{-0.027778in}}%
\pgfpathcurveto{\pgfqpoint{0.007367in}{-0.027778in}}{\pgfqpoint{0.014433in}{-0.024851in}}{\pgfqpoint{0.019642in}{-0.019642in}}%
\pgfpathcurveto{\pgfqpoint{0.024851in}{-0.014433in}}{\pgfqpoint{0.027778in}{-0.007367in}}{\pgfqpoint{0.027778in}{0.000000in}}%
\pgfpathcurveto{\pgfqpoint{0.027778in}{0.007367in}}{\pgfqpoint{0.024851in}{0.014433in}}{\pgfqpoint{0.019642in}{0.019642in}}%
\pgfpathcurveto{\pgfqpoint{0.014433in}{0.024851in}}{\pgfqpoint{0.007367in}{0.027778in}}{\pgfqpoint{0.000000in}{0.027778in}}%
\pgfpathcurveto{\pgfqpoint{-0.007367in}{0.027778in}}{\pgfqpoint{-0.014433in}{0.024851in}}{\pgfqpoint{-0.019642in}{0.019642in}}%
\pgfpathcurveto{\pgfqpoint{-0.024851in}{0.014433in}}{\pgfqpoint{-0.027778in}{0.007367in}}{\pgfqpoint{-0.027778in}{0.000000in}}%
\pgfpathcurveto{\pgfqpoint{-0.027778in}{-0.007367in}}{\pgfqpoint{-0.024851in}{-0.014433in}}{\pgfqpoint{-0.019642in}{-0.019642in}}%
\pgfpathcurveto{\pgfqpoint{-0.014433in}{-0.024851in}}{\pgfqpoint{-0.007367in}{-0.027778in}}{\pgfqpoint{0.000000in}{-0.027778in}}%
\pgfpathlineto{\pgfqpoint{0.000000in}{-0.027778in}}%
\pgfpathclose%
\pgfusepath{stroke,fill}%
}%
\begin{pgfscope}%
\pgfsys@transformshift{1.342857in}{1.794427in}%
\pgfsys@useobject{currentmarker}{}%
\end{pgfscope}%
\begin{pgfscope}%
\pgfsys@transformshift{1.685714in}{2.571287in}%
\pgfsys@useobject{currentmarker}{}%
\end{pgfscope}%
\begin{pgfscope}%
\pgfsys@transformshift{2.028571in}{2.934216in}%
\pgfsys@useobject{currentmarker}{}%
\end{pgfscope}%
\begin{pgfscope}%
\pgfsys@transformshift{2.371429in}{3.113787in}%
\pgfsys@useobject{currentmarker}{}%
\end{pgfscope}%
\begin{pgfscope}%
\pgfsys@transformshift{2.714286in}{3.224888in}%
\pgfsys@useobject{currentmarker}{}%
\end{pgfscope}%
\begin{pgfscope}%
\pgfsys@transformshift{3.057143in}{3.309849in}%
\pgfsys@useobject{currentmarker}{}%
\end{pgfscope}%
\begin{pgfscope}%
\pgfsys@transformshift{3.400000in}{3.317158in}%
\pgfsys@useobject{currentmarker}{}%
\end{pgfscope}%
\begin{pgfscope}%
\pgfsys@transformshift{3.742857in}{3.366817in}%
\pgfsys@useobject{currentmarker}{}%
\end{pgfscope}%
\begin{pgfscope}%
\pgfsys@transformshift{4.085714in}{3.290004in}%
\pgfsys@useobject{currentmarker}{}%
\end{pgfscope}%
\begin{pgfscope}%
\pgfsys@transformshift{4.428571in}{3.308514in}%
\pgfsys@useobject{currentmarker}{}%
\end{pgfscope}%
\begin{pgfscope}%
\pgfsys@transformshift{4.771429in}{3.462108in}%
\pgfsys@useobject{currentmarker}{}%
\end{pgfscope}%
\begin{pgfscope}%
\pgfsys@transformshift{5.114286in}{3.326009in}%
\pgfsys@useobject{currentmarker}{}%
\end{pgfscope}%
\begin{pgfscope}%
\pgfsys@transformshift{5.457143in}{3.069866in}%
\pgfsys@useobject{currentmarker}{}%
\end{pgfscope}%
\end{pgfscope}%
\begin{pgfscope}%
\pgfsetrectcap%
\pgfsetmiterjoin%
\pgfsetlinewidth{0.803000pt}%
\definecolor{currentstroke}{rgb}{0.000000,0.000000,0.000000}%
\pgfsetstrokecolor{currentstroke}%
\pgfsetdash{}{0pt}%
\pgfpathmoveto{\pgfqpoint{1.000000in}{0.720000in}}%
\pgfpathlineto{\pgfqpoint{1.000000in}{5.340000in}}%
\pgfusepath{stroke}%
\end{pgfscope}%
\begin{pgfscope}%
\pgfsetrectcap%
\pgfsetmiterjoin%
\pgfsetlinewidth{0.803000pt}%
\definecolor{currentstroke}{rgb}{0.000000,0.000000,0.000000}%
\pgfsetstrokecolor{currentstroke}%
\pgfsetdash{}{0pt}%
\pgfpathmoveto{\pgfqpoint{5.800000in}{0.720000in}}%
\pgfpathlineto{\pgfqpoint{5.800000in}{5.340000in}}%
\pgfusepath{stroke}%
\end{pgfscope}%
\begin{pgfscope}%
\pgfsetrectcap%
\pgfsetmiterjoin%
\pgfsetlinewidth{0.803000pt}%
\definecolor{currentstroke}{rgb}{0.000000,0.000000,0.000000}%
\pgfsetstrokecolor{currentstroke}%
\pgfsetdash{}{0pt}%
\pgfpathmoveto{\pgfqpoint{1.000000in}{0.720000in}}%
\pgfpathlineto{\pgfqpoint{5.800000in}{0.720000in}}%
\pgfusepath{stroke}%
\end{pgfscope}%
\begin{pgfscope}%
\pgfsetrectcap%
\pgfsetmiterjoin%
\pgfsetlinewidth{0.803000pt}%
\definecolor{currentstroke}{rgb}{0.000000,0.000000,0.000000}%
\pgfsetstrokecolor{currentstroke}%
\pgfsetdash{}{0pt}%
\pgfpathmoveto{\pgfqpoint{1.000000in}{5.340000in}}%
\pgfpathlineto{\pgfqpoint{5.800000in}{5.340000in}}%
\pgfusepath{stroke}%
\end{pgfscope}%
\begin{pgfscope}%
\pgfsetbuttcap%
\pgfsetmiterjoin%
\definecolor{currentfill}{rgb}{1.000000,1.000000,1.000000}%
\pgfsetfillcolor{currentfill}%
\pgfsetfillopacity{0.800000}%
\pgfsetlinewidth{1.003750pt}%
\definecolor{currentstroke}{rgb}{0.800000,0.800000,0.800000}%
\pgfsetstrokecolor{currentstroke}%
\pgfsetstrokeopacity{0.800000}%
\pgfsetdash{}{0pt}%
\pgfpathmoveto{\pgfqpoint{4.330170in}{4.722821in}}%
\pgfpathlineto{\pgfqpoint{5.605556in}{4.722821in}}%
\pgfpathquadraticcurveto{\pgfqpoint{5.661111in}{4.722821in}}{\pgfqpoint{5.661111in}{4.778377in}}%
\pgfpathlineto{\pgfqpoint{5.661111in}{5.145556in}}%
\pgfpathquadraticcurveto{\pgfqpoint{5.661111in}{5.201111in}}{\pgfqpoint{5.605556in}{5.201111in}}%
\pgfpathlineto{\pgfqpoint{4.330170in}{5.201111in}}%
\pgfpathquadraticcurveto{\pgfqpoint{4.274615in}{5.201111in}}{\pgfqpoint{4.274615in}{5.145556in}}%
\pgfpathlineto{\pgfqpoint{4.274615in}{4.778377in}}%
\pgfpathquadraticcurveto{\pgfqpoint{4.274615in}{4.722821in}}{\pgfqpoint{4.330170in}{4.722821in}}%
\pgfpathlineto{\pgfqpoint{4.330170in}{4.722821in}}%
\pgfpathclose%
\pgfusepath{stroke,fill}%
\end{pgfscope}%
\begin{pgfscope}%
\pgfsetbuttcap%
\pgfsetroundjoin%
\pgfsetlinewidth{1.003750pt}%
\definecolor{currentstroke}{rgb}{0.000000,0.000000,1.000000}%
\pgfsetstrokecolor{currentstroke}%
\pgfsetdash{}{0pt}%
\pgfpathmoveto{\pgfqpoint{4.663504in}{4.848295in}}%
\pgfpathlineto{\pgfqpoint{4.663504in}{5.126073in}}%
\pgfusepath{stroke}%
\end{pgfscope}%
\begin{pgfscope}%
\pgfsetbuttcap%
\pgfsetroundjoin%
\definecolor{currentfill}{rgb}{0.000000,0.000000,1.000000}%
\pgfsetfillcolor{currentfill}%
\pgfsetlinewidth{1.003750pt}%
\definecolor{currentstroke}{rgb}{0.000000,0.000000,1.000000}%
\pgfsetstrokecolor{currentstroke}%
\pgfsetdash{}{0pt}%
\pgfsys@defobject{currentmarker}{\pgfqpoint{-0.041667in}{-0.000000in}}{\pgfqpoint{0.041667in}{0.000000in}}{%
\pgfpathmoveto{\pgfqpoint{0.041667in}{-0.000000in}}%
\pgfpathlineto{\pgfqpoint{-0.041667in}{0.000000in}}%
\pgfusepath{stroke,fill}%
}%
\begin{pgfscope}%
\pgfsys@transformshift{4.663504in}{4.848295in}%
\pgfsys@useobject{currentmarker}{}%
\end{pgfscope}%
\end{pgfscope}%
\begin{pgfscope}%
\pgfsetbuttcap%
\pgfsetroundjoin%
\definecolor{currentfill}{rgb}{0.000000,0.000000,1.000000}%
\pgfsetfillcolor{currentfill}%
\pgfsetlinewidth{1.003750pt}%
\definecolor{currentstroke}{rgb}{0.000000,0.000000,1.000000}%
\pgfsetstrokecolor{currentstroke}%
\pgfsetdash{}{0pt}%
\pgfsys@defobject{currentmarker}{\pgfqpoint{-0.041667in}{-0.000000in}}{\pgfqpoint{0.041667in}{0.000000in}}{%
\pgfpathmoveto{\pgfqpoint{0.041667in}{-0.000000in}}%
\pgfpathlineto{\pgfqpoint{-0.041667in}{0.000000in}}%
\pgfusepath{stroke,fill}%
}%
\begin{pgfscope}%
\pgfsys@transformshift{4.663504in}{5.126073in}%
\pgfsys@useobject{currentmarker}{}%
\end{pgfscope}%
\end{pgfscope}%
\begin{pgfscope}%
\pgfsetbuttcap%
\pgfsetroundjoin%
\definecolor{currentfill}{rgb}{0.000000,0.000000,1.000000}%
\pgfsetfillcolor{currentfill}%
\pgfsetlinewidth{1.003750pt}%
\definecolor{currentstroke}{rgb}{0.000000,0.000000,1.000000}%
\pgfsetstrokecolor{currentstroke}%
\pgfsetdash{}{0pt}%
\pgfsys@defobject{currentmarker}{\pgfqpoint{-0.027778in}{-0.027778in}}{\pgfqpoint{0.027778in}{0.027778in}}{%
\pgfpathmoveto{\pgfqpoint{0.000000in}{-0.027778in}}%
\pgfpathcurveto{\pgfqpoint{0.007367in}{-0.027778in}}{\pgfqpoint{0.014433in}{-0.024851in}}{\pgfqpoint{0.019642in}{-0.019642in}}%
\pgfpathcurveto{\pgfqpoint{0.024851in}{-0.014433in}}{\pgfqpoint{0.027778in}{-0.007367in}}{\pgfqpoint{0.027778in}{0.000000in}}%
\pgfpathcurveto{\pgfqpoint{0.027778in}{0.007367in}}{\pgfqpoint{0.024851in}{0.014433in}}{\pgfqpoint{0.019642in}{0.019642in}}%
\pgfpathcurveto{\pgfqpoint{0.014433in}{0.024851in}}{\pgfqpoint{0.007367in}{0.027778in}}{\pgfqpoint{0.000000in}{0.027778in}}%
\pgfpathcurveto{\pgfqpoint{-0.007367in}{0.027778in}}{\pgfqpoint{-0.014433in}{0.024851in}}{\pgfqpoint{-0.019642in}{0.019642in}}%
\pgfpathcurveto{\pgfqpoint{-0.024851in}{0.014433in}}{\pgfqpoint{-0.027778in}{0.007367in}}{\pgfqpoint{-0.027778in}{0.000000in}}%
\pgfpathcurveto{\pgfqpoint{-0.027778in}{-0.007367in}}{\pgfqpoint{-0.024851in}{-0.014433in}}{\pgfqpoint{-0.019642in}{-0.019642in}}%
\pgfpathcurveto{\pgfqpoint{-0.014433in}{-0.024851in}}{\pgfqpoint{-0.007367in}{-0.027778in}}{\pgfqpoint{0.000000in}{-0.027778in}}%
\pgfpathlineto{\pgfqpoint{0.000000in}{-0.027778in}}%
\pgfpathclose%
\pgfusepath{stroke,fill}%
}%
\begin{pgfscope}%
\pgfsys@transformshift{4.663504in}{4.987184in}%
\pgfsys@useobject{currentmarker}{}%
\end{pgfscope}%
\end{pgfscope}%
\begin{pgfscope}%
\definecolor{textcolor}{rgb}{0.000000,0.000000,0.000000}%
\pgfsetstrokecolor{textcolor}%
\pgfsetfillcolor{textcolor}%
\pgftext[x=5.163504in,y=4.889962in,left,base]{\color{textcolor}\sffamily\fontsize{20.000000}{24.000000}\selectfont \(\displaystyle D_\mathrm{w}\)}%
\end{pgfscope}%
\begin{pgfscope}%
\pgfsetbuttcap%
\pgfsetmiterjoin%
\definecolor{currentfill}{rgb}{1.000000,1.000000,1.000000}%
\pgfsetfillcolor{currentfill}%
\pgfsetlinewidth{0.000000pt}%
\definecolor{currentstroke}{rgb}{0.000000,0.000000,0.000000}%
\pgfsetstrokecolor{currentstroke}%
\pgfsetstrokeopacity{0.000000}%
\pgfsetdash{}{0pt}%
\pgfpathmoveto{\pgfqpoint{5.800000in}{0.720000in}}%
\pgfpathlineto{\pgfqpoint{7.200000in}{0.720000in}}%
\pgfpathlineto{\pgfqpoint{7.200000in}{5.340000in}}%
\pgfpathlineto{\pgfqpoint{5.800000in}{5.340000in}}%
\pgfpathlineto{\pgfqpoint{5.800000in}{0.720000in}}%
\pgfpathclose%
\pgfusepath{fill}%
\end{pgfscope}%
\begin{pgfscope}%
\pgfpathrectangle{\pgfqpoint{5.800000in}{0.720000in}}{\pgfqpoint{1.400000in}{4.620000in}}%
\pgfusepath{clip}%
\pgfsetbuttcap%
\pgfsetmiterjoin%
\definecolor{currentfill}{rgb}{0.121569,0.466667,0.705882}%
\pgfsetfillcolor{currentfill}%
\pgfsetlinewidth{0.000000pt}%
\definecolor{currentstroke}{rgb}{0.000000,0.000000,0.000000}%
\pgfsetstrokecolor{currentstroke}%
\pgfsetstrokeopacity{0.000000}%
\pgfsetdash{}{0pt}%
\pgfpathmoveto{\pgfqpoint{5.800000in}{0.720000in}}%
\pgfpathlineto{\pgfqpoint{5.893179in}{0.720000in}}%
\pgfpathlineto{\pgfqpoint{5.893179in}{0.835500in}}%
\pgfpathlineto{\pgfqpoint{5.800000in}{0.835500in}}%
\pgfpathlineto{\pgfqpoint{5.800000in}{0.720000in}}%
\pgfpathclose%
\pgfusepath{fill}%
\end{pgfscope}%
\begin{pgfscope}%
\pgfpathrectangle{\pgfqpoint{5.800000in}{0.720000in}}{\pgfqpoint{1.400000in}{4.620000in}}%
\pgfusepath{clip}%
\pgfsetbuttcap%
\pgfsetmiterjoin%
\definecolor{currentfill}{rgb}{0.121569,0.466667,0.705882}%
\pgfsetfillcolor{currentfill}%
\pgfsetlinewidth{0.000000pt}%
\definecolor{currentstroke}{rgb}{0.000000,0.000000,0.000000}%
\pgfsetstrokecolor{currentstroke}%
\pgfsetstrokeopacity{0.000000}%
\pgfsetdash{}{0pt}%
\pgfpathmoveto{\pgfqpoint{5.800000in}{0.835500in}}%
\pgfpathlineto{\pgfqpoint{5.910515in}{0.835500in}}%
\pgfpathlineto{\pgfqpoint{5.910515in}{0.951000in}}%
\pgfpathlineto{\pgfqpoint{5.800000in}{0.951000in}}%
\pgfpathlineto{\pgfqpoint{5.800000in}{0.835500in}}%
\pgfpathclose%
\pgfusepath{fill}%
\end{pgfscope}%
\begin{pgfscope}%
\pgfpathrectangle{\pgfqpoint{5.800000in}{0.720000in}}{\pgfqpoint{1.400000in}{4.620000in}}%
\pgfusepath{clip}%
\pgfsetbuttcap%
\pgfsetmiterjoin%
\definecolor{currentfill}{rgb}{0.121569,0.466667,0.705882}%
\pgfsetfillcolor{currentfill}%
\pgfsetlinewidth{0.000000pt}%
\definecolor{currentstroke}{rgb}{0.000000,0.000000,0.000000}%
\pgfsetstrokecolor{currentstroke}%
\pgfsetstrokeopacity{0.000000}%
\pgfsetdash{}{0pt}%
\pgfpathmoveto{\pgfqpoint{5.800000in}{0.951000in}}%
\pgfpathlineto{\pgfqpoint{5.927851in}{0.951000in}}%
\pgfpathlineto{\pgfqpoint{5.927851in}{1.066500in}}%
\pgfpathlineto{\pgfqpoint{5.800000in}{1.066500in}}%
\pgfpathlineto{\pgfqpoint{5.800000in}{0.951000in}}%
\pgfpathclose%
\pgfusepath{fill}%
\end{pgfscope}%
\begin{pgfscope}%
\pgfpathrectangle{\pgfqpoint{5.800000in}{0.720000in}}{\pgfqpoint{1.400000in}{4.620000in}}%
\pgfusepath{clip}%
\pgfsetbuttcap%
\pgfsetmiterjoin%
\definecolor{currentfill}{rgb}{0.121569,0.466667,0.705882}%
\pgfsetfillcolor{currentfill}%
\pgfsetlinewidth{0.000000pt}%
\definecolor{currentstroke}{rgb}{0.000000,0.000000,0.000000}%
\pgfsetstrokecolor{currentstroke}%
\pgfsetstrokeopacity{0.000000}%
\pgfsetdash{}{0pt}%
\pgfpathmoveto{\pgfqpoint{5.800000in}{1.066500in}}%
\pgfpathlineto{\pgfqpoint{5.919183in}{1.066500in}}%
\pgfpathlineto{\pgfqpoint{5.919183in}{1.182000in}}%
\pgfpathlineto{\pgfqpoint{5.800000in}{1.182000in}}%
\pgfpathlineto{\pgfqpoint{5.800000in}{1.066500in}}%
\pgfpathclose%
\pgfusepath{fill}%
\end{pgfscope}%
\begin{pgfscope}%
\pgfpathrectangle{\pgfqpoint{5.800000in}{0.720000in}}{\pgfqpoint{1.400000in}{4.620000in}}%
\pgfusepath{clip}%
\pgfsetbuttcap%
\pgfsetmiterjoin%
\definecolor{currentfill}{rgb}{0.121569,0.466667,0.705882}%
\pgfsetfillcolor{currentfill}%
\pgfsetlinewidth{0.000000pt}%
\definecolor{currentstroke}{rgb}{0.000000,0.000000,0.000000}%
\pgfsetstrokecolor{currentstroke}%
\pgfsetstrokeopacity{0.000000}%
\pgfsetdash{}{0pt}%
\pgfpathmoveto{\pgfqpoint{5.800000in}{1.182000in}}%
\pgfpathlineto{\pgfqpoint{5.934352in}{1.182000in}}%
\pgfpathlineto{\pgfqpoint{5.934352in}{1.297500in}}%
\pgfpathlineto{\pgfqpoint{5.800000in}{1.297500in}}%
\pgfpathlineto{\pgfqpoint{5.800000in}{1.182000in}}%
\pgfpathclose%
\pgfusepath{fill}%
\end{pgfscope}%
\begin{pgfscope}%
\pgfpathrectangle{\pgfqpoint{5.800000in}{0.720000in}}{\pgfqpoint{1.400000in}{4.620000in}}%
\pgfusepath{clip}%
\pgfsetbuttcap%
\pgfsetmiterjoin%
\definecolor{currentfill}{rgb}{0.121569,0.466667,0.705882}%
\pgfsetfillcolor{currentfill}%
\pgfsetlinewidth{0.000000pt}%
\definecolor{currentstroke}{rgb}{0.000000,0.000000,0.000000}%
\pgfsetstrokecolor{currentstroke}%
\pgfsetstrokeopacity{0.000000}%
\pgfsetdash{}{0pt}%
\pgfpathmoveto{\pgfqpoint{5.800000in}{1.297500in}}%
\pgfpathlineto{\pgfqpoint{5.951688in}{1.297500in}}%
\pgfpathlineto{\pgfqpoint{5.951688in}{1.413000in}}%
\pgfpathlineto{\pgfqpoint{5.800000in}{1.413000in}}%
\pgfpathlineto{\pgfqpoint{5.800000in}{1.297500in}}%
\pgfpathclose%
\pgfusepath{fill}%
\end{pgfscope}%
\begin{pgfscope}%
\pgfpathrectangle{\pgfqpoint{5.800000in}{0.720000in}}{\pgfqpoint{1.400000in}{4.620000in}}%
\pgfusepath{clip}%
\pgfsetbuttcap%
\pgfsetmiterjoin%
\definecolor{currentfill}{rgb}{0.121569,0.466667,0.705882}%
\pgfsetfillcolor{currentfill}%
\pgfsetlinewidth{0.000000pt}%
\definecolor{currentstroke}{rgb}{0.000000,0.000000,0.000000}%
\pgfsetstrokecolor{currentstroke}%
\pgfsetstrokeopacity{0.000000}%
\pgfsetdash{}{0pt}%
\pgfpathmoveto{\pgfqpoint{5.800000in}{1.413000in}}%
\pgfpathlineto{\pgfqpoint{5.988526in}{1.413000in}}%
\pgfpathlineto{\pgfqpoint{5.988526in}{1.528500in}}%
\pgfpathlineto{\pgfqpoint{5.800000in}{1.528500in}}%
\pgfpathlineto{\pgfqpoint{5.800000in}{1.413000in}}%
\pgfpathclose%
\pgfusepath{fill}%
\end{pgfscope}%
\begin{pgfscope}%
\pgfpathrectangle{\pgfqpoint{5.800000in}{0.720000in}}{\pgfqpoint{1.400000in}{4.620000in}}%
\pgfusepath{clip}%
\pgfsetbuttcap%
\pgfsetmiterjoin%
\definecolor{currentfill}{rgb}{0.121569,0.466667,0.705882}%
\pgfsetfillcolor{currentfill}%
\pgfsetlinewidth{0.000000pt}%
\definecolor{currentstroke}{rgb}{0.000000,0.000000,0.000000}%
\pgfsetstrokecolor{currentstroke}%
\pgfsetstrokeopacity{0.000000}%
\pgfsetdash{}{0pt}%
\pgfpathmoveto{\pgfqpoint{5.800000in}{1.528500in}}%
\pgfpathlineto{\pgfqpoint{6.057869in}{1.528500in}}%
\pgfpathlineto{\pgfqpoint{6.057869in}{1.644000in}}%
\pgfpathlineto{\pgfqpoint{5.800000in}{1.644000in}}%
\pgfpathlineto{\pgfqpoint{5.800000in}{1.528500in}}%
\pgfpathclose%
\pgfusepath{fill}%
\end{pgfscope}%
\begin{pgfscope}%
\pgfpathrectangle{\pgfqpoint{5.800000in}{0.720000in}}{\pgfqpoint{1.400000in}{4.620000in}}%
\pgfusepath{clip}%
\pgfsetbuttcap%
\pgfsetmiterjoin%
\definecolor{currentfill}{rgb}{0.121569,0.466667,0.705882}%
\pgfsetfillcolor{currentfill}%
\pgfsetlinewidth{0.000000pt}%
\definecolor{currentstroke}{rgb}{0.000000,0.000000,0.000000}%
\pgfsetstrokecolor{currentstroke}%
\pgfsetstrokeopacity{0.000000}%
\pgfsetdash{}{0pt}%
\pgfpathmoveto{\pgfqpoint{5.800000in}{1.644000in}}%
\pgfpathlineto{\pgfqpoint{6.109876in}{1.644000in}}%
\pgfpathlineto{\pgfqpoint{6.109876in}{1.759500in}}%
\pgfpathlineto{\pgfqpoint{5.800000in}{1.759500in}}%
\pgfpathlineto{\pgfqpoint{5.800000in}{1.644000in}}%
\pgfpathclose%
\pgfusepath{fill}%
\end{pgfscope}%
\begin{pgfscope}%
\pgfpathrectangle{\pgfqpoint{5.800000in}{0.720000in}}{\pgfqpoint{1.400000in}{4.620000in}}%
\pgfusepath{clip}%
\pgfsetbuttcap%
\pgfsetmiterjoin%
\definecolor{currentfill}{rgb}{0.121569,0.466667,0.705882}%
\pgfsetfillcolor{currentfill}%
\pgfsetlinewidth{0.000000pt}%
\definecolor{currentstroke}{rgb}{0.000000,0.000000,0.000000}%
\pgfsetstrokecolor{currentstroke}%
\pgfsetstrokeopacity{0.000000}%
\pgfsetdash{}{0pt}%
\pgfpathmoveto{\pgfqpoint{5.800000in}{1.759500in}}%
\pgfpathlineto{\pgfqpoint{6.142380in}{1.759500in}}%
\pgfpathlineto{\pgfqpoint{6.142380in}{1.875000in}}%
\pgfpathlineto{\pgfqpoint{5.800000in}{1.875000in}}%
\pgfpathlineto{\pgfqpoint{5.800000in}{1.759500in}}%
\pgfpathclose%
\pgfusepath{fill}%
\end{pgfscope}%
\begin{pgfscope}%
\pgfpathrectangle{\pgfqpoint{5.800000in}{0.720000in}}{\pgfqpoint{1.400000in}{4.620000in}}%
\pgfusepath{clip}%
\pgfsetbuttcap%
\pgfsetmiterjoin%
\definecolor{currentfill}{rgb}{0.121569,0.466667,0.705882}%
\pgfsetfillcolor{currentfill}%
\pgfsetlinewidth{0.000000pt}%
\definecolor{currentstroke}{rgb}{0.000000,0.000000,0.000000}%
\pgfsetstrokecolor{currentstroke}%
\pgfsetstrokeopacity{0.000000}%
\pgfsetdash{}{0pt}%
\pgfpathmoveto{\pgfqpoint{5.800000in}{1.875000in}}%
\pgfpathlineto{\pgfqpoint{6.296235in}{1.875000in}}%
\pgfpathlineto{\pgfqpoint{6.296235in}{1.990500in}}%
\pgfpathlineto{\pgfqpoint{5.800000in}{1.990500in}}%
\pgfpathlineto{\pgfqpoint{5.800000in}{1.875000in}}%
\pgfpathclose%
\pgfusepath{fill}%
\end{pgfscope}%
\begin{pgfscope}%
\pgfpathrectangle{\pgfqpoint{5.800000in}{0.720000in}}{\pgfqpoint{1.400000in}{4.620000in}}%
\pgfusepath{clip}%
\pgfsetbuttcap%
\pgfsetmiterjoin%
\definecolor{currentfill}{rgb}{0.121569,0.466667,0.705882}%
\pgfsetfillcolor{currentfill}%
\pgfsetlinewidth{0.000000pt}%
\definecolor{currentstroke}{rgb}{0.000000,0.000000,0.000000}%
\pgfsetstrokecolor{currentstroke}%
\pgfsetstrokeopacity{0.000000}%
\pgfsetdash{}{0pt}%
\pgfpathmoveto{\pgfqpoint{5.800000in}{1.990500in}}%
\pgfpathlineto{\pgfqpoint{6.326572in}{1.990500in}}%
\pgfpathlineto{\pgfqpoint{6.326572in}{2.106000in}}%
\pgfpathlineto{\pgfqpoint{5.800000in}{2.106000in}}%
\pgfpathlineto{\pgfqpoint{5.800000in}{1.990500in}}%
\pgfpathclose%
\pgfusepath{fill}%
\end{pgfscope}%
\begin{pgfscope}%
\pgfpathrectangle{\pgfqpoint{5.800000in}{0.720000in}}{\pgfqpoint{1.400000in}{4.620000in}}%
\pgfusepath{clip}%
\pgfsetbuttcap%
\pgfsetmiterjoin%
\definecolor{currentfill}{rgb}{0.121569,0.466667,0.705882}%
\pgfsetfillcolor{currentfill}%
\pgfsetlinewidth{0.000000pt}%
\definecolor{currentstroke}{rgb}{0.000000,0.000000,0.000000}%
\pgfsetstrokecolor{currentstroke}%
\pgfsetstrokeopacity{0.000000}%
\pgfsetdash{}{0pt}%
\pgfpathmoveto{\pgfqpoint{5.800000in}{2.106000in}}%
\pgfpathlineto{\pgfqpoint{6.478260in}{2.106000in}}%
\pgfpathlineto{\pgfqpoint{6.478260in}{2.221500in}}%
\pgfpathlineto{\pgfqpoint{5.800000in}{2.221500in}}%
\pgfpathlineto{\pgfqpoint{5.800000in}{2.106000in}}%
\pgfpathclose%
\pgfusepath{fill}%
\end{pgfscope}%
\begin{pgfscope}%
\pgfpathrectangle{\pgfqpoint{5.800000in}{0.720000in}}{\pgfqpoint{1.400000in}{4.620000in}}%
\pgfusepath{clip}%
\pgfsetbuttcap%
\pgfsetmiterjoin%
\definecolor{currentfill}{rgb}{0.121569,0.466667,0.705882}%
\pgfsetfillcolor{currentfill}%
\pgfsetlinewidth{0.000000pt}%
\definecolor{currentstroke}{rgb}{0.000000,0.000000,0.000000}%
\pgfsetstrokecolor{currentstroke}%
\pgfsetstrokeopacity{0.000000}%
\pgfsetdash{}{0pt}%
\pgfpathmoveto{\pgfqpoint{5.800000in}{2.221500in}}%
\pgfpathlineto{\pgfqpoint{6.564939in}{2.221500in}}%
\pgfpathlineto{\pgfqpoint{6.564939in}{2.337000in}}%
\pgfpathlineto{\pgfqpoint{5.800000in}{2.337000in}}%
\pgfpathlineto{\pgfqpoint{5.800000in}{2.221500in}}%
\pgfpathclose%
\pgfusepath{fill}%
\end{pgfscope}%
\begin{pgfscope}%
\pgfpathrectangle{\pgfqpoint{5.800000in}{0.720000in}}{\pgfqpoint{1.400000in}{4.620000in}}%
\pgfusepath{clip}%
\pgfsetbuttcap%
\pgfsetmiterjoin%
\definecolor{currentfill}{rgb}{0.121569,0.466667,0.705882}%
\pgfsetfillcolor{currentfill}%
\pgfsetlinewidth{0.000000pt}%
\definecolor{currentstroke}{rgb}{0.000000,0.000000,0.000000}%
\pgfsetstrokecolor{currentstroke}%
\pgfsetstrokeopacity{0.000000}%
\pgfsetdash{}{0pt}%
\pgfpathmoveto{\pgfqpoint{5.800000in}{2.337000in}}%
\pgfpathlineto{\pgfqpoint{6.697123in}{2.337000in}}%
\pgfpathlineto{\pgfqpoint{6.697123in}{2.452500in}}%
\pgfpathlineto{\pgfqpoint{5.800000in}{2.452500in}}%
\pgfpathlineto{\pgfqpoint{5.800000in}{2.337000in}}%
\pgfpathclose%
\pgfusepath{fill}%
\end{pgfscope}%
\begin{pgfscope}%
\pgfpathrectangle{\pgfqpoint{5.800000in}{0.720000in}}{\pgfqpoint{1.400000in}{4.620000in}}%
\pgfusepath{clip}%
\pgfsetbuttcap%
\pgfsetmiterjoin%
\definecolor{currentfill}{rgb}{0.121569,0.466667,0.705882}%
\pgfsetfillcolor{currentfill}%
\pgfsetlinewidth{0.000000pt}%
\definecolor{currentstroke}{rgb}{0.000000,0.000000,0.000000}%
\pgfsetstrokecolor{currentstroke}%
\pgfsetstrokeopacity{0.000000}%
\pgfsetdash{}{0pt}%
\pgfpathmoveto{\pgfqpoint{5.800000in}{2.452500in}}%
\pgfpathlineto{\pgfqpoint{6.863980in}{2.452500in}}%
\pgfpathlineto{\pgfqpoint{6.863980in}{2.568000in}}%
\pgfpathlineto{\pgfqpoint{5.800000in}{2.568000in}}%
\pgfpathlineto{\pgfqpoint{5.800000in}{2.452500in}}%
\pgfpathclose%
\pgfusepath{fill}%
\end{pgfscope}%
\begin{pgfscope}%
\pgfpathrectangle{\pgfqpoint{5.800000in}{0.720000in}}{\pgfqpoint{1.400000in}{4.620000in}}%
\pgfusepath{clip}%
\pgfsetbuttcap%
\pgfsetmiterjoin%
\definecolor{currentfill}{rgb}{0.121569,0.466667,0.705882}%
\pgfsetfillcolor{currentfill}%
\pgfsetlinewidth{0.000000pt}%
\definecolor{currentstroke}{rgb}{0.000000,0.000000,0.000000}%
\pgfsetstrokecolor{currentstroke}%
\pgfsetstrokeopacity{0.000000}%
\pgfsetdash{}{0pt}%
\pgfpathmoveto{\pgfqpoint{5.800000in}{2.568000in}}%
\pgfpathlineto{\pgfqpoint{6.935489in}{2.568000in}}%
\pgfpathlineto{\pgfqpoint{6.935489in}{2.683500in}}%
\pgfpathlineto{\pgfqpoint{5.800000in}{2.683500in}}%
\pgfpathlineto{\pgfqpoint{5.800000in}{2.568000in}}%
\pgfpathclose%
\pgfusepath{fill}%
\end{pgfscope}%
\begin{pgfscope}%
\pgfpathrectangle{\pgfqpoint{5.800000in}{0.720000in}}{\pgfqpoint{1.400000in}{4.620000in}}%
\pgfusepath{clip}%
\pgfsetbuttcap%
\pgfsetmiterjoin%
\definecolor{currentfill}{rgb}{0.121569,0.466667,0.705882}%
\pgfsetfillcolor{currentfill}%
\pgfsetlinewidth{0.000000pt}%
\definecolor{currentstroke}{rgb}{0.000000,0.000000,0.000000}%
\pgfsetstrokecolor{currentstroke}%
\pgfsetstrokeopacity{0.000000}%
\pgfsetdash{}{0pt}%
\pgfpathmoveto{\pgfqpoint{5.800000in}{2.683500in}}%
\pgfpathlineto{\pgfqpoint{6.965827in}{2.683500in}}%
\pgfpathlineto{\pgfqpoint{6.965827in}{2.799000in}}%
\pgfpathlineto{\pgfqpoint{5.800000in}{2.799000in}}%
\pgfpathlineto{\pgfqpoint{5.800000in}{2.683500in}}%
\pgfpathclose%
\pgfusepath{fill}%
\end{pgfscope}%
\begin{pgfscope}%
\pgfpathrectangle{\pgfqpoint{5.800000in}{0.720000in}}{\pgfqpoint{1.400000in}{4.620000in}}%
\pgfusepath{clip}%
\pgfsetbuttcap%
\pgfsetmiterjoin%
\definecolor{currentfill}{rgb}{0.121569,0.466667,0.705882}%
\pgfsetfillcolor{currentfill}%
\pgfsetlinewidth{0.000000pt}%
\definecolor{currentstroke}{rgb}{0.000000,0.000000,0.000000}%
\pgfsetstrokecolor{currentstroke}%
\pgfsetstrokeopacity{0.000000}%
\pgfsetdash{}{0pt}%
\pgfpathmoveto{\pgfqpoint{5.800000in}{2.799000in}}%
\pgfpathlineto{\pgfqpoint{7.069841in}{2.799000in}}%
\pgfpathlineto{\pgfqpoint{7.069841in}{2.914500in}}%
\pgfpathlineto{\pgfqpoint{5.800000in}{2.914500in}}%
\pgfpathlineto{\pgfqpoint{5.800000in}{2.799000in}}%
\pgfpathclose%
\pgfusepath{fill}%
\end{pgfscope}%
\begin{pgfscope}%
\pgfpathrectangle{\pgfqpoint{5.800000in}{0.720000in}}{\pgfqpoint{1.400000in}{4.620000in}}%
\pgfusepath{clip}%
\pgfsetbuttcap%
\pgfsetmiterjoin%
\definecolor{currentfill}{rgb}{0.121569,0.466667,0.705882}%
\pgfsetfillcolor{currentfill}%
\pgfsetlinewidth{0.000000pt}%
\definecolor{currentstroke}{rgb}{0.000000,0.000000,0.000000}%
\pgfsetstrokecolor{currentstroke}%
\pgfsetstrokeopacity{0.000000}%
\pgfsetdash{}{0pt}%
\pgfpathmoveto{\pgfqpoint{5.800000in}{2.914500in}}%
\pgfpathlineto{\pgfqpoint{6.998331in}{2.914500in}}%
\pgfpathlineto{\pgfqpoint{6.998331in}{3.030000in}}%
\pgfpathlineto{\pgfqpoint{5.800000in}{3.030000in}}%
\pgfpathlineto{\pgfqpoint{5.800000in}{2.914500in}}%
\pgfpathclose%
\pgfusepath{fill}%
\end{pgfscope}%
\begin{pgfscope}%
\pgfpathrectangle{\pgfqpoint{5.800000in}{0.720000in}}{\pgfqpoint{1.400000in}{4.620000in}}%
\pgfusepath{clip}%
\pgfsetbuttcap%
\pgfsetmiterjoin%
\definecolor{currentfill}{rgb}{0.121569,0.466667,0.705882}%
\pgfsetfillcolor{currentfill}%
\pgfsetlinewidth{0.000000pt}%
\definecolor{currentstroke}{rgb}{0.000000,0.000000,0.000000}%
\pgfsetstrokecolor{currentstroke}%
\pgfsetstrokeopacity{0.000000}%
\pgfsetdash{}{0pt}%
\pgfpathmoveto{\pgfqpoint{5.800000in}{3.030000in}}%
\pgfpathlineto{\pgfqpoint{7.013500in}{3.030000in}}%
\pgfpathlineto{\pgfqpoint{7.013500in}{3.145500in}}%
\pgfpathlineto{\pgfqpoint{5.800000in}{3.145500in}}%
\pgfpathlineto{\pgfqpoint{5.800000in}{3.030000in}}%
\pgfpathclose%
\pgfusepath{fill}%
\end{pgfscope}%
\begin{pgfscope}%
\pgfpathrectangle{\pgfqpoint{5.800000in}{0.720000in}}{\pgfqpoint{1.400000in}{4.620000in}}%
\pgfusepath{clip}%
\pgfsetbuttcap%
\pgfsetmiterjoin%
\definecolor{currentfill}{rgb}{0.121569,0.466667,0.705882}%
\pgfsetfillcolor{currentfill}%
\pgfsetlinewidth{0.000000pt}%
\definecolor{currentstroke}{rgb}{0.000000,0.000000,0.000000}%
\pgfsetstrokecolor{currentstroke}%
\pgfsetstrokeopacity{0.000000}%
\pgfsetdash{}{0pt}%
\pgfpathmoveto{\pgfqpoint{5.800000in}{3.145500in}}%
\pgfpathlineto{\pgfqpoint{6.863980in}{3.145500in}}%
\pgfpathlineto{\pgfqpoint{6.863980in}{3.261000in}}%
\pgfpathlineto{\pgfqpoint{5.800000in}{3.261000in}}%
\pgfpathlineto{\pgfqpoint{5.800000in}{3.145500in}}%
\pgfpathclose%
\pgfusepath{fill}%
\end{pgfscope}%
\begin{pgfscope}%
\pgfpathrectangle{\pgfqpoint{5.800000in}{0.720000in}}{\pgfqpoint{1.400000in}{4.620000in}}%
\pgfusepath{clip}%
\pgfsetbuttcap%
\pgfsetmiterjoin%
\definecolor{currentfill}{rgb}{0.121569,0.466667,0.705882}%
\pgfsetfillcolor{currentfill}%
\pgfsetlinewidth{0.000000pt}%
\definecolor{currentstroke}{rgb}{0.000000,0.000000,0.000000}%
\pgfsetstrokecolor{currentstroke}%
\pgfsetstrokeopacity{0.000000}%
\pgfsetdash{}{0pt}%
\pgfpathmoveto{\pgfqpoint{5.800000in}{3.261000in}}%
\pgfpathlineto{\pgfqpoint{6.855312in}{3.261000in}}%
\pgfpathlineto{\pgfqpoint{6.855312in}{3.376500in}}%
\pgfpathlineto{\pgfqpoint{5.800000in}{3.376500in}}%
\pgfpathlineto{\pgfqpoint{5.800000in}{3.261000in}}%
\pgfpathclose%
\pgfusepath{fill}%
\end{pgfscope}%
\begin{pgfscope}%
\pgfpathrectangle{\pgfqpoint{5.800000in}{0.720000in}}{\pgfqpoint{1.400000in}{4.620000in}}%
\pgfusepath{clip}%
\pgfsetbuttcap%
\pgfsetmiterjoin%
\definecolor{currentfill}{rgb}{0.121569,0.466667,0.705882}%
\pgfsetfillcolor{currentfill}%
\pgfsetlinewidth{0.000000pt}%
\definecolor{currentstroke}{rgb}{0.000000,0.000000,0.000000}%
\pgfsetstrokecolor{currentstroke}%
\pgfsetstrokeopacity{0.000000}%
\pgfsetdash{}{0pt}%
\pgfpathmoveto{\pgfqpoint{5.800000in}{3.376500in}}%
\pgfpathlineto{\pgfqpoint{6.749131in}{3.376500in}}%
\pgfpathlineto{\pgfqpoint{6.749131in}{3.492000in}}%
\pgfpathlineto{\pgfqpoint{5.800000in}{3.492000in}}%
\pgfpathlineto{\pgfqpoint{5.800000in}{3.376500in}}%
\pgfpathclose%
\pgfusepath{fill}%
\end{pgfscope}%
\begin{pgfscope}%
\pgfpathrectangle{\pgfqpoint{5.800000in}{0.720000in}}{\pgfqpoint{1.400000in}{4.620000in}}%
\pgfusepath{clip}%
\pgfsetbuttcap%
\pgfsetmiterjoin%
\definecolor{currentfill}{rgb}{0.121569,0.466667,0.705882}%
\pgfsetfillcolor{currentfill}%
\pgfsetlinewidth{0.000000pt}%
\definecolor{currentstroke}{rgb}{0.000000,0.000000,0.000000}%
\pgfsetstrokecolor{currentstroke}%
\pgfsetstrokeopacity{0.000000}%
\pgfsetdash{}{0pt}%
\pgfpathmoveto{\pgfqpoint{5.800000in}{3.492000in}}%
\pgfpathlineto{\pgfqpoint{6.673287in}{3.492000in}}%
\pgfpathlineto{\pgfqpoint{6.673287in}{3.607500in}}%
\pgfpathlineto{\pgfqpoint{5.800000in}{3.607500in}}%
\pgfpathlineto{\pgfqpoint{5.800000in}{3.492000in}}%
\pgfpathclose%
\pgfusepath{fill}%
\end{pgfscope}%
\begin{pgfscope}%
\pgfpathrectangle{\pgfqpoint{5.800000in}{0.720000in}}{\pgfqpoint{1.400000in}{4.620000in}}%
\pgfusepath{clip}%
\pgfsetbuttcap%
\pgfsetmiterjoin%
\definecolor{currentfill}{rgb}{0.121569,0.466667,0.705882}%
\pgfsetfillcolor{currentfill}%
\pgfsetlinewidth{0.000000pt}%
\definecolor{currentstroke}{rgb}{0.000000,0.000000,0.000000}%
\pgfsetstrokecolor{currentstroke}%
\pgfsetstrokeopacity{0.000000}%
\pgfsetdash{}{0pt}%
\pgfpathmoveto{\pgfqpoint{5.800000in}{3.607500in}}%
\pgfpathlineto{\pgfqpoint{6.560605in}{3.607500in}}%
\pgfpathlineto{\pgfqpoint{6.560605in}{3.723000in}}%
\pgfpathlineto{\pgfqpoint{5.800000in}{3.723000in}}%
\pgfpathlineto{\pgfqpoint{5.800000in}{3.607500in}}%
\pgfpathclose%
\pgfusepath{fill}%
\end{pgfscope}%
\begin{pgfscope}%
\pgfpathrectangle{\pgfqpoint{5.800000in}{0.720000in}}{\pgfqpoint{1.400000in}{4.620000in}}%
\pgfusepath{clip}%
\pgfsetbuttcap%
\pgfsetmiterjoin%
\definecolor{currentfill}{rgb}{0.121569,0.466667,0.705882}%
\pgfsetfillcolor{currentfill}%
\pgfsetlinewidth{0.000000pt}%
\definecolor{currentstroke}{rgb}{0.000000,0.000000,0.000000}%
\pgfsetstrokecolor{currentstroke}%
\pgfsetstrokeopacity{0.000000}%
\pgfsetdash{}{0pt}%
\pgfpathmoveto{\pgfqpoint{5.800000in}{3.723000in}}%
\pgfpathlineto{\pgfqpoint{6.419752in}{3.723000in}}%
\pgfpathlineto{\pgfqpoint{6.419752in}{3.838500in}}%
\pgfpathlineto{\pgfqpoint{5.800000in}{3.838500in}}%
\pgfpathlineto{\pgfqpoint{5.800000in}{3.723000in}}%
\pgfpathclose%
\pgfusepath{fill}%
\end{pgfscope}%
\begin{pgfscope}%
\pgfpathrectangle{\pgfqpoint{5.800000in}{0.720000in}}{\pgfqpoint{1.400000in}{4.620000in}}%
\pgfusepath{clip}%
\pgfsetbuttcap%
\pgfsetmiterjoin%
\definecolor{currentfill}{rgb}{0.121569,0.466667,0.705882}%
\pgfsetfillcolor{currentfill}%
\pgfsetlinewidth{0.000000pt}%
\definecolor{currentstroke}{rgb}{0.000000,0.000000,0.000000}%
\pgfsetstrokecolor{currentstroke}%
\pgfsetstrokeopacity{0.000000}%
\pgfsetdash{}{0pt}%
\pgfpathmoveto{\pgfqpoint{5.800000in}{3.838500in}}%
\pgfpathlineto{\pgfqpoint{6.339574in}{3.838500in}}%
\pgfpathlineto{\pgfqpoint{6.339574in}{3.954000in}}%
\pgfpathlineto{\pgfqpoint{5.800000in}{3.954000in}}%
\pgfpathlineto{\pgfqpoint{5.800000in}{3.838500in}}%
\pgfpathclose%
\pgfusepath{fill}%
\end{pgfscope}%
\begin{pgfscope}%
\pgfpathrectangle{\pgfqpoint{5.800000in}{0.720000in}}{\pgfqpoint{1.400000in}{4.620000in}}%
\pgfusepath{clip}%
\pgfsetbuttcap%
\pgfsetmiterjoin%
\definecolor{currentfill}{rgb}{0.121569,0.466667,0.705882}%
\pgfsetfillcolor{currentfill}%
\pgfsetlinewidth{0.000000pt}%
\definecolor{currentstroke}{rgb}{0.000000,0.000000,0.000000}%
\pgfsetstrokecolor{currentstroke}%
\pgfsetstrokeopacity{0.000000}%
\pgfsetdash{}{0pt}%
\pgfpathmoveto{\pgfqpoint{5.800000in}{3.954000in}}%
\pgfpathlineto{\pgfqpoint{6.257230in}{3.954000in}}%
\pgfpathlineto{\pgfqpoint{6.257230in}{4.069500in}}%
\pgfpathlineto{\pgfqpoint{5.800000in}{4.069500in}}%
\pgfpathlineto{\pgfqpoint{5.800000in}{3.954000in}}%
\pgfpathclose%
\pgfusepath{fill}%
\end{pgfscope}%
\begin{pgfscope}%
\pgfpathrectangle{\pgfqpoint{5.800000in}{0.720000in}}{\pgfqpoint{1.400000in}{4.620000in}}%
\pgfusepath{clip}%
\pgfsetbuttcap%
\pgfsetmiterjoin%
\definecolor{currentfill}{rgb}{0.121569,0.466667,0.705882}%
\pgfsetfillcolor{currentfill}%
\pgfsetlinewidth{0.000000pt}%
\definecolor{currentstroke}{rgb}{0.000000,0.000000,0.000000}%
\pgfsetstrokecolor{currentstroke}%
\pgfsetstrokeopacity{0.000000}%
\pgfsetdash{}{0pt}%
\pgfpathmoveto{\pgfqpoint{5.800000in}{4.069500in}}%
\pgfpathlineto{\pgfqpoint{6.235560in}{4.069500in}}%
\pgfpathlineto{\pgfqpoint{6.235560in}{4.185000in}}%
\pgfpathlineto{\pgfqpoint{5.800000in}{4.185000in}}%
\pgfpathlineto{\pgfqpoint{5.800000in}{4.069500in}}%
\pgfpathclose%
\pgfusepath{fill}%
\end{pgfscope}%
\begin{pgfscope}%
\pgfpathrectangle{\pgfqpoint{5.800000in}{0.720000in}}{\pgfqpoint{1.400000in}{4.620000in}}%
\pgfusepath{clip}%
\pgfsetbuttcap%
\pgfsetmiterjoin%
\definecolor{currentfill}{rgb}{0.121569,0.466667,0.705882}%
\pgfsetfillcolor{currentfill}%
\pgfsetlinewidth{0.000000pt}%
\definecolor{currentstroke}{rgb}{0.000000,0.000000,0.000000}%
\pgfsetstrokecolor{currentstroke}%
\pgfsetstrokeopacity{0.000000}%
\pgfsetdash{}{0pt}%
\pgfpathmoveto{\pgfqpoint{5.800000in}{4.185000in}}%
\pgfpathlineto{\pgfqpoint{6.155382in}{4.185000in}}%
\pgfpathlineto{\pgfqpoint{6.155382in}{4.300500in}}%
\pgfpathlineto{\pgfqpoint{5.800000in}{4.300500in}}%
\pgfpathlineto{\pgfqpoint{5.800000in}{4.185000in}}%
\pgfpathclose%
\pgfusepath{fill}%
\end{pgfscope}%
\begin{pgfscope}%
\pgfpathrectangle{\pgfqpoint{5.800000in}{0.720000in}}{\pgfqpoint{1.400000in}{4.620000in}}%
\pgfusepath{clip}%
\pgfsetbuttcap%
\pgfsetmiterjoin%
\definecolor{currentfill}{rgb}{0.121569,0.466667,0.705882}%
\pgfsetfillcolor{currentfill}%
\pgfsetlinewidth{0.000000pt}%
\definecolor{currentstroke}{rgb}{0.000000,0.000000,0.000000}%
\pgfsetstrokecolor{currentstroke}%
\pgfsetstrokeopacity{0.000000}%
\pgfsetdash{}{0pt}%
\pgfpathmoveto{\pgfqpoint{5.800000in}{4.300500in}}%
\pgfpathlineto{\pgfqpoint{6.075205in}{4.300500in}}%
\pgfpathlineto{\pgfqpoint{6.075205in}{4.416000in}}%
\pgfpathlineto{\pgfqpoint{5.800000in}{4.416000in}}%
\pgfpathlineto{\pgfqpoint{5.800000in}{4.300500in}}%
\pgfpathclose%
\pgfusepath{fill}%
\end{pgfscope}%
\begin{pgfscope}%
\pgfpathrectangle{\pgfqpoint{5.800000in}{0.720000in}}{\pgfqpoint{1.400000in}{4.620000in}}%
\pgfusepath{clip}%
\pgfsetbuttcap%
\pgfsetmiterjoin%
\definecolor{currentfill}{rgb}{0.121569,0.466667,0.705882}%
\pgfsetfillcolor{currentfill}%
\pgfsetlinewidth{0.000000pt}%
\definecolor{currentstroke}{rgb}{0.000000,0.000000,0.000000}%
\pgfsetstrokecolor{currentstroke}%
\pgfsetstrokeopacity{0.000000}%
\pgfsetdash{}{0pt}%
\pgfpathmoveto{\pgfqpoint{5.800000in}{4.416000in}}%
\pgfpathlineto{\pgfqpoint{6.092540in}{4.416000in}}%
\pgfpathlineto{\pgfqpoint{6.092540in}{4.531500in}}%
\pgfpathlineto{\pgfqpoint{5.800000in}{4.531500in}}%
\pgfpathlineto{\pgfqpoint{5.800000in}{4.416000in}}%
\pgfpathclose%
\pgfusepath{fill}%
\end{pgfscope}%
\begin{pgfscope}%
\pgfpathrectangle{\pgfqpoint{5.800000in}{0.720000in}}{\pgfqpoint{1.400000in}{4.620000in}}%
\pgfusepath{clip}%
\pgfsetbuttcap%
\pgfsetmiterjoin%
\definecolor{currentfill}{rgb}{0.121569,0.466667,0.705882}%
\pgfsetfillcolor{currentfill}%
\pgfsetlinewidth{0.000000pt}%
\definecolor{currentstroke}{rgb}{0.000000,0.000000,0.000000}%
\pgfsetstrokecolor{currentstroke}%
\pgfsetstrokeopacity{0.000000}%
\pgfsetdash{}{0pt}%
\pgfpathmoveto{\pgfqpoint{5.800000in}{4.531500in}}%
\pgfpathlineto{\pgfqpoint{6.014529in}{4.531500in}}%
\pgfpathlineto{\pgfqpoint{6.014529in}{4.647000in}}%
\pgfpathlineto{\pgfqpoint{5.800000in}{4.647000in}}%
\pgfpathlineto{\pgfqpoint{5.800000in}{4.531500in}}%
\pgfpathclose%
\pgfusepath{fill}%
\end{pgfscope}%
\begin{pgfscope}%
\pgfpathrectangle{\pgfqpoint{5.800000in}{0.720000in}}{\pgfqpoint{1.400000in}{4.620000in}}%
\pgfusepath{clip}%
\pgfsetbuttcap%
\pgfsetmiterjoin%
\definecolor{currentfill}{rgb}{0.121569,0.466667,0.705882}%
\pgfsetfillcolor{currentfill}%
\pgfsetlinewidth{0.000000pt}%
\definecolor{currentstroke}{rgb}{0.000000,0.000000,0.000000}%
\pgfsetstrokecolor{currentstroke}%
\pgfsetstrokeopacity{0.000000}%
\pgfsetdash{}{0pt}%
\pgfpathmoveto{\pgfqpoint{5.800000in}{4.647000in}}%
\pgfpathlineto{\pgfqpoint{5.971190in}{4.647000in}}%
\pgfpathlineto{\pgfqpoint{5.971190in}{4.762500in}}%
\pgfpathlineto{\pgfqpoint{5.800000in}{4.762500in}}%
\pgfpathlineto{\pgfqpoint{5.800000in}{4.647000in}}%
\pgfpathclose%
\pgfusepath{fill}%
\end{pgfscope}%
\begin{pgfscope}%
\pgfpathrectangle{\pgfqpoint{5.800000in}{0.720000in}}{\pgfqpoint{1.400000in}{4.620000in}}%
\pgfusepath{clip}%
\pgfsetbuttcap%
\pgfsetmiterjoin%
\definecolor{currentfill}{rgb}{0.121569,0.466667,0.705882}%
\pgfsetfillcolor{currentfill}%
\pgfsetlinewidth{0.000000pt}%
\definecolor{currentstroke}{rgb}{0.000000,0.000000,0.000000}%
\pgfsetstrokecolor{currentstroke}%
\pgfsetstrokeopacity{0.000000}%
\pgfsetdash{}{0pt}%
\pgfpathmoveto{\pgfqpoint{5.800000in}{4.762500in}}%
\pgfpathlineto{\pgfqpoint{5.990693in}{4.762500in}}%
\pgfpathlineto{\pgfqpoint{5.990693in}{4.878000in}}%
\pgfpathlineto{\pgfqpoint{5.800000in}{4.878000in}}%
\pgfpathlineto{\pgfqpoint{5.800000in}{4.762500in}}%
\pgfpathclose%
\pgfusepath{fill}%
\end{pgfscope}%
\begin{pgfscope}%
\pgfpathrectangle{\pgfqpoint{5.800000in}{0.720000in}}{\pgfqpoint{1.400000in}{4.620000in}}%
\pgfusepath{clip}%
\pgfsetbuttcap%
\pgfsetmiterjoin%
\definecolor{currentfill}{rgb}{0.121569,0.466667,0.705882}%
\pgfsetfillcolor{currentfill}%
\pgfsetlinewidth{0.000000pt}%
\definecolor{currentstroke}{rgb}{0.000000,0.000000,0.000000}%
\pgfsetstrokecolor{currentstroke}%
\pgfsetstrokeopacity{0.000000}%
\pgfsetdash{}{0pt}%
\pgfpathmoveto{\pgfqpoint{5.800000in}{4.878000in}}%
\pgfpathlineto{\pgfqpoint{5.962522in}{4.878000in}}%
\pgfpathlineto{\pgfqpoint{5.962522in}{4.993500in}}%
\pgfpathlineto{\pgfqpoint{5.800000in}{4.993500in}}%
\pgfpathlineto{\pgfqpoint{5.800000in}{4.878000in}}%
\pgfpathclose%
\pgfusepath{fill}%
\end{pgfscope}%
\begin{pgfscope}%
\pgfpathrectangle{\pgfqpoint{5.800000in}{0.720000in}}{\pgfqpoint{1.400000in}{4.620000in}}%
\pgfusepath{clip}%
\pgfsetbuttcap%
\pgfsetmiterjoin%
\definecolor{currentfill}{rgb}{0.121569,0.466667,0.705882}%
\pgfsetfillcolor{currentfill}%
\pgfsetlinewidth{0.000000pt}%
\definecolor{currentstroke}{rgb}{0.000000,0.000000,0.000000}%
\pgfsetstrokecolor{currentstroke}%
\pgfsetstrokeopacity{0.000000}%
\pgfsetdash{}{0pt}%
\pgfpathmoveto{\pgfqpoint{5.800000in}{4.993500in}}%
\pgfpathlineto{\pgfqpoint{5.912682in}{4.993500in}}%
\pgfpathlineto{\pgfqpoint{5.912682in}{5.109000in}}%
\pgfpathlineto{\pgfqpoint{5.800000in}{5.109000in}}%
\pgfpathlineto{\pgfqpoint{5.800000in}{4.993500in}}%
\pgfpathclose%
\pgfusepath{fill}%
\end{pgfscope}%
\begin{pgfscope}%
\pgfpathrectangle{\pgfqpoint{5.800000in}{0.720000in}}{\pgfqpoint{1.400000in}{4.620000in}}%
\pgfusepath{clip}%
\pgfsetbuttcap%
\pgfsetmiterjoin%
\definecolor{currentfill}{rgb}{0.121569,0.466667,0.705882}%
\pgfsetfillcolor{currentfill}%
\pgfsetlinewidth{0.000000pt}%
\definecolor{currentstroke}{rgb}{0.000000,0.000000,0.000000}%
\pgfsetstrokecolor{currentstroke}%
\pgfsetstrokeopacity{0.000000}%
\pgfsetdash{}{0pt}%
\pgfpathmoveto{\pgfqpoint{5.800000in}{5.109000in}}%
\pgfpathlineto{\pgfqpoint{5.895346in}{5.109000in}}%
\pgfpathlineto{\pgfqpoint{5.895346in}{5.224500in}}%
\pgfpathlineto{\pgfqpoint{5.800000in}{5.224500in}}%
\pgfpathlineto{\pgfqpoint{5.800000in}{5.109000in}}%
\pgfpathclose%
\pgfusepath{fill}%
\end{pgfscope}%
\begin{pgfscope}%
\pgfpathrectangle{\pgfqpoint{5.800000in}{0.720000in}}{\pgfqpoint{1.400000in}{4.620000in}}%
\pgfusepath{clip}%
\pgfsetbuttcap%
\pgfsetmiterjoin%
\definecolor{currentfill}{rgb}{0.121569,0.466667,0.705882}%
\pgfsetfillcolor{currentfill}%
\pgfsetlinewidth{0.000000pt}%
\definecolor{currentstroke}{rgb}{0.000000,0.000000,0.000000}%
\pgfsetstrokecolor{currentstroke}%
\pgfsetstrokeopacity{0.000000}%
\pgfsetdash{}{0pt}%
\pgfpathmoveto{\pgfqpoint{5.800000in}{5.224500in}}%
\pgfpathlineto{\pgfqpoint{5.893179in}{5.224500in}}%
\pgfpathlineto{\pgfqpoint{5.893179in}{5.340000in}}%
\pgfpathlineto{\pgfqpoint{5.800000in}{5.340000in}}%
\pgfpathlineto{\pgfqpoint{5.800000in}{5.224500in}}%
\pgfpathclose%
\pgfusepath{fill}%
\end{pgfscope}%
\begin{pgfscope}%
\definecolor{textcolor}{rgb}{0.000000,0.000000,0.000000}%
\pgfsetstrokecolor{textcolor}%
\pgfsetfillcolor{textcolor}%
\pgftext[x=6.500000in,y=0.664444in,,top]{\color{textcolor}\sffamily\fontsize{20.000000}{24.000000}\selectfont \(\displaystyle \mathrm{arb.\ unit}\)}%
\end{pgfscope}%
\begin{pgfscope}%
\pgfsetrectcap%
\pgfsetmiterjoin%
\pgfsetlinewidth{0.803000pt}%
\definecolor{currentstroke}{rgb}{0.000000,0.000000,0.000000}%
\pgfsetstrokecolor{currentstroke}%
\pgfsetdash{}{0pt}%
\pgfpathmoveto{\pgfqpoint{5.800000in}{0.720000in}}%
\pgfpathlineto{\pgfqpoint{5.800000in}{5.340000in}}%
\pgfusepath{stroke}%
\end{pgfscope}%
\begin{pgfscope}%
\pgfsetrectcap%
\pgfsetmiterjoin%
\pgfsetlinewidth{0.803000pt}%
\definecolor{currentstroke}{rgb}{0.000000,0.000000,0.000000}%
\pgfsetstrokecolor{currentstroke}%
\pgfsetdash{}{0pt}%
\pgfpathmoveto{\pgfqpoint{7.200000in}{0.720000in}}%
\pgfpathlineto{\pgfqpoint{7.200000in}{5.340000in}}%
\pgfusepath{stroke}%
\end{pgfscope}%
\begin{pgfscope}%
\pgfsetrectcap%
\pgfsetmiterjoin%
\pgfsetlinewidth{0.803000pt}%
\definecolor{currentstroke}{rgb}{0.000000,0.000000,0.000000}%
\pgfsetstrokecolor{currentstroke}%
\pgfsetdash{}{0pt}%
\pgfpathmoveto{\pgfqpoint{5.800000in}{0.720000in}}%
\pgfpathlineto{\pgfqpoint{7.200000in}{0.720000in}}%
\pgfusepath{stroke}%
\end{pgfscope}%
\begin{pgfscope}%
\pgfsetrectcap%
\pgfsetmiterjoin%
\pgfsetlinewidth{0.803000pt}%
\definecolor{currentstroke}{rgb}{0.000000,0.000000,0.000000}%
\pgfsetstrokecolor{currentstroke}%
\pgfsetdash{}{0pt}%
\pgfpathmoveto{\pgfqpoint{5.800000in}{5.340000in}}%
\pgfpathlineto{\pgfqpoint{7.200000in}{5.340000in}}%
\pgfusepath{stroke}%
\end{pgfscope}%
\end{pgfpicture}%
\makeatother%
\endgroup%
}
    \caption{\label{fig:cnn-npe} $D_\mathrm{w}$ histogram and its distributions conditioned \\ on $N_{\mathrm{PE}}$. ``arbi. unit'' means arbitrary unit.}
  \end{subfigure}
  \begin{subfigure}{.5\textwidth}
    \centering
    \resizebox{\textwidth}{!}{%% Creator: Matplotlib, PGF backend
%%
%% To include the figure in your LaTeX document, write
%%   \input{<filename>.pgf}
%%
%% Make sure the required packages are loaded in your preamble
%%   \usepackage{pgf}
%%
%% Also ensure that all the required font packages are loaded; for instance,
%% the lmodern package is sometimes necessary when using math font.
%%   \usepackage{lmodern}
%%
%% Figures using additional raster images can only be included by \input if
%% they are in the same directory as the main LaTeX file. For loading figures
%% from other directories you can use the `import` package
%%   \usepackage{import}
%%
%% and then include the figures with
%%   \import{<path to file>}{<filename>.pgf}
%%
%% Matplotlib used the following preamble
%%   \usepackage[detect-all,locale=DE]{siunitx}
%%
\begingroup%
\makeatletter%
\begin{pgfpicture}%
\pgfpathrectangle{\pgfpointorigin}{\pgfqpoint{8.000000in}{6.000000in}}%
\pgfusepath{use as bounding box, clip}%
\begin{pgfscope}%
\pgfsetbuttcap%
\pgfsetmiterjoin%
\definecolor{currentfill}{rgb}{1.000000,1.000000,1.000000}%
\pgfsetfillcolor{currentfill}%
\pgfsetlinewidth{0.000000pt}%
\definecolor{currentstroke}{rgb}{1.000000,1.000000,1.000000}%
\pgfsetstrokecolor{currentstroke}%
\pgfsetdash{}{0pt}%
\pgfpathmoveto{\pgfqpoint{0.000000in}{0.000000in}}%
\pgfpathlineto{\pgfqpoint{8.000000in}{0.000000in}}%
\pgfpathlineto{\pgfqpoint{8.000000in}{6.000000in}}%
\pgfpathlineto{\pgfqpoint{0.000000in}{6.000000in}}%
\pgfpathlineto{\pgfqpoint{0.000000in}{0.000000in}}%
\pgfpathclose%
\pgfusepath{fill}%
\end{pgfscope}%
\begin{pgfscope}%
\pgfsetbuttcap%
\pgfsetmiterjoin%
\definecolor{currentfill}{rgb}{1.000000,1.000000,1.000000}%
\pgfsetfillcolor{currentfill}%
\pgfsetlinewidth{0.000000pt}%
\definecolor{currentstroke}{rgb}{0.000000,0.000000,0.000000}%
\pgfsetstrokecolor{currentstroke}%
\pgfsetstrokeopacity{0.000000}%
\pgfsetdash{}{0pt}%
\pgfpathmoveto{\pgfqpoint{1.000000in}{0.720000in}}%
\pgfpathlineto{\pgfqpoint{7.200000in}{0.720000in}}%
\pgfpathlineto{\pgfqpoint{7.200000in}{5.340000in}}%
\pgfpathlineto{\pgfqpoint{1.000000in}{5.340000in}}%
\pgfpathlineto{\pgfqpoint{1.000000in}{0.720000in}}%
\pgfpathclose%
\pgfusepath{fill}%
\end{pgfscope}%
\begin{pgfscope}%
\pgfsetbuttcap%
\pgfsetroundjoin%
\definecolor{currentfill}{rgb}{0.000000,0.000000,0.000000}%
\pgfsetfillcolor{currentfill}%
\pgfsetlinewidth{0.803000pt}%
\definecolor{currentstroke}{rgb}{0.000000,0.000000,0.000000}%
\pgfsetstrokecolor{currentstroke}%
\pgfsetdash{}{0pt}%
\pgfsys@defobject{currentmarker}{\pgfqpoint{0.000000in}{-0.048611in}}{\pgfqpoint{0.000000in}{0.000000in}}{%
\pgfpathmoveto{\pgfqpoint{0.000000in}{0.000000in}}%
\pgfpathlineto{\pgfqpoint{0.000000in}{-0.048611in}}%
\pgfusepath{stroke,fill}%
}%
\begin{pgfscope}%
\pgfsys@transformshift{1.310000in}{0.720000in}%
\pgfsys@useobject{currentmarker}{}%
\end{pgfscope}%
\end{pgfscope}%
\begin{pgfscope}%
\definecolor{textcolor}{rgb}{0.000000,0.000000,0.000000}%
\pgfsetstrokecolor{textcolor}%
\pgfsetfillcolor{textcolor}%
\pgftext[x=1.310000in,y=0.622778in,,top]{\color{textcolor}\sffamily\fontsize{20.000000}{24.000000}\selectfont \(\displaystyle {450}\)}%
\end{pgfscope}%
\begin{pgfscope}%
\pgfsetbuttcap%
\pgfsetroundjoin%
\definecolor{currentfill}{rgb}{0.000000,0.000000,0.000000}%
\pgfsetfillcolor{currentfill}%
\pgfsetlinewidth{0.803000pt}%
\definecolor{currentstroke}{rgb}{0.000000,0.000000,0.000000}%
\pgfsetstrokecolor{currentstroke}%
\pgfsetdash{}{0pt}%
\pgfsys@defobject{currentmarker}{\pgfqpoint{0.000000in}{-0.048611in}}{\pgfqpoint{0.000000in}{0.000000in}}{%
\pgfpathmoveto{\pgfqpoint{0.000000in}{0.000000in}}%
\pgfpathlineto{\pgfqpoint{0.000000in}{-0.048611in}}%
\pgfusepath{stroke,fill}%
}%
\begin{pgfscope}%
\pgfsys@transformshift{2.860000in}{0.720000in}%
\pgfsys@useobject{currentmarker}{}%
\end{pgfscope}%
\end{pgfscope}%
\begin{pgfscope}%
\definecolor{textcolor}{rgb}{0.000000,0.000000,0.000000}%
\pgfsetstrokecolor{textcolor}%
\pgfsetfillcolor{textcolor}%
\pgftext[x=2.860000in,y=0.622778in,,top]{\color{textcolor}\sffamily\fontsize{20.000000}{24.000000}\selectfont \(\displaystyle {500}\)}%
\end{pgfscope}%
\begin{pgfscope}%
\pgfsetbuttcap%
\pgfsetroundjoin%
\definecolor{currentfill}{rgb}{0.000000,0.000000,0.000000}%
\pgfsetfillcolor{currentfill}%
\pgfsetlinewidth{0.803000pt}%
\definecolor{currentstroke}{rgb}{0.000000,0.000000,0.000000}%
\pgfsetstrokecolor{currentstroke}%
\pgfsetdash{}{0pt}%
\pgfsys@defobject{currentmarker}{\pgfqpoint{0.000000in}{-0.048611in}}{\pgfqpoint{0.000000in}{0.000000in}}{%
\pgfpathmoveto{\pgfqpoint{0.000000in}{0.000000in}}%
\pgfpathlineto{\pgfqpoint{0.000000in}{-0.048611in}}%
\pgfusepath{stroke,fill}%
}%
\begin{pgfscope}%
\pgfsys@transformshift{4.410000in}{0.720000in}%
\pgfsys@useobject{currentmarker}{}%
\end{pgfscope}%
\end{pgfscope}%
\begin{pgfscope}%
\definecolor{textcolor}{rgb}{0.000000,0.000000,0.000000}%
\pgfsetstrokecolor{textcolor}%
\pgfsetfillcolor{textcolor}%
\pgftext[x=4.410000in,y=0.622778in,,top]{\color{textcolor}\sffamily\fontsize{20.000000}{24.000000}\selectfont \(\displaystyle {550}\)}%
\end{pgfscope}%
\begin{pgfscope}%
\pgfsetbuttcap%
\pgfsetroundjoin%
\definecolor{currentfill}{rgb}{0.000000,0.000000,0.000000}%
\pgfsetfillcolor{currentfill}%
\pgfsetlinewidth{0.803000pt}%
\definecolor{currentstroke}{rgb}{0.000000,0.000000,0.000000}%
\pgfsetstrokecolor{currentstroke}%
\pgfsetdash{}{0pt}%
\pgfsys@defobject{currentmarker}{\pgfqpoint{0.000000in}{-0.048611in}}{\pgfqpoint{0.000000in}{0.000000in}}{%
\pgfpathmoveto{\pgfqpoint{0.000000in}{0.000000in}}%
\pgfpathlineto{\pgfqpoint{0.000000in}{-0.048611in}}%
\pgfusepath{stroke,fill}%
}%
\begin{pgfscope}%
\pgfsys@transformshift{5.960000in}{0.720000in}%
\pgfsys@useobject{currentmarker}{}%
\end{pgfscope}%
\end{pgfscope}%
\begin{pgfscope}%
\definecolor{textcolor}{rgb}{0.000000,0.000000,0.000000}%
\pgfsetstrokecolor{textcolor}%
\pgfsetfillcolor{textcolor}%
\pgftext[x=5.960000in,y=0.622778in,,top]{\color{textcolor}\sffamily\fontsize{20.000000}{24.000000}\selectfont \(\displaystyle {600}\)}%
\end{pgfscope}%
\begin{pgfscope}%
\definecolor{textcolor}{rgb}{0.000000,0.000000,0.000000}%
\pgfsetstrokecolor{textcolor}%
\pgfsetfillcolor{textcolor}%
\pgftext[x=4.100000in,y=0.311155in,,top]{\color{textcolor}\sffamily\fontsize{20.000000}{24.000000}\selectfont \(\displaystyle \mathrm{t}/\si{ns}\)}%
\end{pgfscope}%
\begin{pgfscope}%
\pgfsetbuttcap%
\pgfsetroundjoin%
\definecolor{currentfill}{rgb}{0.000000,0.000000,0.000000}%
\pgfsetfillcolor{currentfill}%
\pgfsetlinewidth{0.803000pt}%
\definecolor{currentstroke}{rgb}{0.000000,0.000000,0.000000}%
\pgfsetstrokecolor{currentstroke}%
\pgfsetdash{}{0pt}%
\pgfsys@defobject{currentmarker}{\pgfqpoint{-0.048611in}{0.000000in}}{\pgfqpoint{-0.000000in}{0.000000in}}{%
\pgfpathmoveto{\pgfqpoint{-0.000000in}{0.000000in}}%
\pgfpathlineto{\pgfqpoint{-0.048611in}{0.000000in}}%
\pgfusepath{stroke,fill}%
}%
\begin{pgfscope}%
\pgfsys@transformshift{1.000000in}{1.109002in}%
\pgfsys@useobject{currentmarker}{}%
\end{pgfscope}%
\end{pgfscope}%
\begin{pgfscope}%
\definecolor{textcolor}{rgb}{0.000000,0.000000,0.000000}%
\pgfsetstrokecolor{textcolor}%
\pgfsetfillcolor{textcolor}%
\pgftext[x=0.770670in, y=1.008983in, left, base]{\color{textcolor}\sffamily\fontsize{20.000000}{24.000000}\selectfont \(\displaystyle {0}\)}%
\end{pgfscope}%
\begin{pgfscope}%
\pgfsetbuttcap%
\pgfsetroundjoin%
\definecolor{currentfill}{rgb}{0.000000,0.000000,0.000000}%
\pgfsetfillcolor{currentfill}%
\pgfsetlinewidth{0.803000pt}%
\definecolor{currentstroke}{rgb}{0.000000,0.000000,0.000000}%
\pgfsetstrokecolor{currentstroke}%
\pgfsetdash{}{0pt}%
\pgfsys@defobject{currentmarker}{\pgfqpoint{-0.048611in}{0.000000in}}{\pgfqpoint{-0.000000in}{0.000000in}}{%
\pgfpathmoveto{\pgfqpoint{-0.000000in}{0.000000in}}%
\pgfpathlineto{\pgfqpoint{-0.048611in}{0.000000in}}%
\pgfusepath{stroke,fill}%
}%
\begin{pgfscope}%
\pgfsys@transformshift{1.000000in}{2.004270in}%
\pgfsys@useobject{currentmarker}{}%
\end{pgfscope}%
\end{pgfscope}%
\begin{pgfscope}%
\definecolor{textcolor}{rgb}{0.000000,0.000000,0.000000}%
\pgfsetstrokecolor{textcolor}%
\pgfsetfillcolor{textcolor}%
\pgftext[x=0.638563in, y=1.904251in, left, base]{\color{textcolor}\sffamily\fontsize{20.000000}{24.000000}\selectfont \(\displaystyle {10}\)}%
\end{pgfscope}%
\begin{pgfscope}%
\pgfsetbuttcap%
\pgfsetroundjoin%
\definecolor{currentfill}{rgb}{0.000000,0.000000,0.000000}%
\pgfsetfillcolor{currentfill}%
\pgfsetlinewidth{0.803000pt}%
\definecolor{currentstroke}{rgb}{0.000000,0.000000,0.000000}%
\pgfsetstrokecolor{currentstroke}%
\pgfsetdash{}{0pt}%
\pgfsys@defobject{currentmarker}{\pgfqpoint{-0.048611in}{0.000000in}}{\pgfqpoint{-0.000000in}{0.000000in}}{%
\pgfpathmoveto{\pgfqpoint{-0.000000in}{0.000000in}}%
\pgfpathlineto{\pgfqpoint{-0.048611in}{0.000000in}}%
\pgfusepath{stroke,fill}%
}%
\begin{pgfscope}%
\pgfsys@transformshift{1.000000in}{2.899538in}%
\pgfsys@useobject{currentmarker}{}%
\end{pgfscope}%
\end{pgfscope}%
\begin{pgfscope}%
\definecolor{textcolor}{rgb}{0.000000,0.000000,0.000000}%
\pgfsetstrokecolor{textcolor}%
\pgfsetfillcolor{textcolor}%
\pgftext[x=0.638563in, y=2.799519in, left, base]{\color{textcolor}\sffamily\fontsize{20.000000}{24.000000}\selectfont \(\displaystyle {20}\)}%
\end{pgfscope}%
\begin{pgfscope}%
\pgfsetbuttcap%
\pgfsetroundjoin%
\definecolor{currentfill}{rgb}{0.000000,0.000000,0.000000}%
\pgfsetfillcolor{currentfill}%
\pgfsetlinewidth{0.803000pt}%
\definecolor{currentstroke}{rgb}{0.000000,0.000000,0.000000}%
\pgfsetstrokecolor{currentstroke}%
\pgfsetdash{}{0pt}%
\pgfsys@defobject{currentmarker}{\pgfqpoint{-0.048611in}{0.000000in}}{\pgfqpoint{-0.000000in}{0.000000in}}{%
\pgfpathmoveto{\pgfqpoint{-0.000000in}{0.000000in}}%
\pgfpathlineto{\pgfqpoint{-0.048611in}{0.000000in}}%
\pgfusepath{stroke,fill}%
}%
\begin{pgfscope}%
\pgfsys@transformshift{1.000000in}{3.794806in}%
\pgfsys@useobject{currentmarker}{}%
\end{pgfscope}%
\end{pgfscope}%
\begin{pgfscope}%
\definecolor{textcolor}{rgb}{0.000000,0.000000,0.000000}%
\pgfsetstrokecolor{textcolor}%
\pgfsetfillcolor{textcolor}%
\pgftext[x=0.638563in, y=3.694787in, left, base]{\color{textcolor}\sffamily\fontsize{20.000000}{24.000000}\selectfont \(\displaystyle {30}\)}%
\end{pgfscope}%
\begin{pgfscope}%
\pgfsetbuttcap%
\pgfsetroundjoin%
\definecolor{currentfill}{rgb}{0.000000,0.000000,0.000000}%
\pgfsetfillcolor{currentfill}%
\pgfsetlinewidth{0.803000pt}%
\definecolor{currentstroke}{rgb}{0.000000,0.000000,0.000000}%
\pgfsetstrokecolor{currentstroke}%
\pgfsetdash{}{0pt}%
\pgfsys@defobject{currentmarker}{\pgfqpoint{-0.048611in}{0.000000in}}{\pgfqpoint{-0.000000in}{0.000000in}}{%
\pgfpathmoveto{\pgfqpoint{-0.000000in}{0.000000in}}%
\pgfpathlineto{\pgfqpoint{-0.048611in}{0.000000in}}%
\pgfusepath{stroke,fill}%
}%
\begin{pgfscope}%
\pgfsys@transformshift{1.000000in}{4.690074in}%
\pgfsys@useobject{currentmarker}{}%
\end{pgfscope}%
\end{pgfscope}%
\begin{pgfscope}%
\definecolor{textcolor}{rgb}{0.000000,0.000000,0.000000}%
\pgfsetstrokecolor{textcolor}%
\pgfsetfillcolor{textcolor}%
\pgftext[x=0.638563in, y=4.590054in, left, base]{\color{textcolor}\sffamily\fontsize{20.000000}{24.000000}\selectfont \(\displaystyle {40}\)}%
\end{pgfscope}%
\begin{pgfscope}%
\definecolor{textcolor}{rgb}{0.000000,0.000000,0.000000}%
\pgfsetstrokecolor{textcolor}%
\pgfsetfillcolor{textcolor}%
\pgftext[x=0.583007in,y=3.030000in,,bottom,rotate=90.000000]{\color{textcolor}\sffamily\fontsize{20.000000}{24.000000}\selectfont \(\displaystyle \mathrm{Voltage}/\si{mV}\)}%
\end{pgfscope}%
\begin{pgfscope}%
\pgfpathrectangle{\pgfqpoint{1.000000in}{0.720000in}}{\pgfqpoint{6.200000in}{4.620000in}}%
\pgfusepath{clip}%
\pgfsetrectcap%
\pgfsetroundjoin%
\pgfsetlinewidth{2.007500pt}%
\definecolor{currentstroke}{rgb}{0.121569,0.466667,0.705882}%
\pgfsetstrokecolor{currentstroke}%
\pgfsetdash{}{0pt}%
\pgfpathmoveto{\pgfqpoint{0.990000in}{1.164926in}}%
\pgfpathlineto{\pgfqpoint{1.000000in}{1.200117in}}%
\pgfpathlineto{\pgfqpoint{1.031000in}{1.073192in}}%
\pgfpathlineto{\pgfqpoint{1.062000in}{1.147885in}}%
\pgfpathlineto{\pgfqpoint{1.093000in}{1.135477in}}%
\pgfpathlineto{\pgfqpoint{1.124000in}{1.121439in}}%
\pgfpathlineto{\pgfqpoint{1.155000in}{1.210126in}}%
\pgfpathlineto{\pgfqpoint{1.186000in}{1.132056in}}%
\pgfpathlineto{\pgfqpoint{1.217000in}{1.069666in}}%
\pgfpathlineto{\pgfqpoint{1.248000in}{1.198337in}}%
\pgfpathlineto{\pgfqpoint{1.279000in}{1.132919in}}%
\pgfpathlineto{\pgfqpoint{1.310000in}{1.055958in}}%
\pgfpathlineto{\pgfqpoint{1.341000in}{1.153482in}}%
\pgfpathlineto{\pgfqpoint{1.372000in}{1.093744in}}%
\pgfpathlineto{\pgfqpoint{1.403000in}{1.306032in}}%
\pgfpathlineto{\pgfqpoint{1.434000in}{1.051842in}}%
\pgfpathlineto{\pgfqpoint{1.465000in}{0.995276in}}%
\pgfpathlineto{\pgfqpoint{1.496000in}{1.072373in}}%
\pgfpathlineto{\pgfqpoint{1.527000in}{1.020957in}}%
\pgfpathlineto{\pgfqpoint{1.558000in}{0.996263in}}%
\pgfpathlineto{\pgfqpoint{1.589000in}{1.100873in}}%
\pgfpathlineto{\pgfqpoint{1.620000in}{1.182232in}}%
\pgfpathlineto{\pgfqpoint{1.651000in}{1.160629in}}%
\pgfpathlineto{\pgfqpoint{1.682000in}{1.264916in}}%
\pgfpathlineto{\pgfqpoint{1.713000in}{1.185927in}}%
\pgfpathlineto{\pgfqpoint{1.744000in}{1.208894in}}%
\pgfpathlineto{\pgfqpoint{1.775000in}{1.222343in}}%
\pgfpathlineto{\pgfqpoint{1.806000in}{1.191883in}}%
\pgfpathlineto{\pgfqpoint{1.837000in}{1.091525in}}%
\pgfpathlineto{\pgfqpoint{1.868000in}{0.999884in}}%
\pgfpathlineto{\pgfqpoint{1.899000in}{1.127138in}}%
\pgfpathlineto{\pgfqpoint{1.930000in}{1.081839in}}%
\pgfpathlineto{\pgfqpoint{1.961000in}{1.187655in}}%
\pgfpathlineto{\pgfqpoint{1.992000in}{1.182336in}}%
\pgfpathlineto{\pgfqpoint{2.023000in}{1.384380in}}%
\pgfpathlineto{\pgfqpoint{2.054000in}{1.719851in}}%
\pgfpathlineto{\pgfqpoint{2.085000in}{2.107918in}}%
\pgfpathlineto{\pgfqpoint{2.116000in}{2.784651in}}%
\pgfpathlineto{\pgfqpoint{2.147000in}{2.955721in}}%
\pgfpathlineto{\pgfqpoint{2.178000in}{3.338539in}}%
\pgfpathlineto{\pgfqpoint{2.209000in}{3.459601in}}%
\pgfpathlineto{\pgfqpoint{2.240000in}{3.236344in}}%
\pgfpathlineto{\pgfqpoint{2.271000in}{3.220427in}}%
\pgfpathlineto{\pgfqpoint{2.302000in}{3.145605in}}%
\pgfpathlineto{\pgfqpoint{2.333000in}{2.678530in}}%
\pgfpathlineto{\pgfqpoint{2.364000in}{2.517063in}}%
\pgfpathlineto{\pgfqpoint{2.395000in}{2.506665in}}%
\pgfpathlineto{\pgfqpoint{2.426000in}{2.234079in}}%
\pgfpathlineto{\pgfqpoint{2.457000in}{1.878240in}}%
\pgfpathlineto{\pgfqpoint{2.488000in}{1.793637in}}%
\pgfpathlineto{\pgfqpoint{2.519000in}{1.716973in}}%
\pgfpathlineto{\pgfqpoint{2.550000in}{1.675631in}}%
\pgfpathlineto{\pgfqpoint{2.581000in}{1.880210in}}%
\pgfpathlineto{\pgfqpoint{2.612000in}{2.419920in}}%
\pgfpathlineto{\pgfqpoint{2.643000in}{2.646214in}}%
\pgfpathlineto{\pgfqpoint{2.674000in}{3.017097in}}%
\pgfpathlineto{\pgfqpoint{2.705000in}{3.101285in}}%
\pgfpathlineto{\pgfqpoint{2.736000in}{2.994316in}}%
\pgfpathlineto{\pgfqpoint{2.767000in}{2.841146in}}%
\pgfpathlineto{\pgfqpoint{2.798000in}{2.707818in}}%
\pgfpathlineto{\pgfqpoint{2.829000in}{2.552232in}}%
\pgfpathlineto{\pgfqpoint{2.860000in}{2.442091in}}%
\pgfpathlineto{\pgfqpoint{2.891000in}{2.178630in}}%
\pgfpathlineto{\pgfqpoint{2.922000in}{2.222316in}}%
\pgfpathlineto{\pgfqpoint{2.953000in}{2.033937in}}%
\pgfpathlineto{\pgfqpoint{2.984000in}{2.523017in}}%
\pgfpathlineto{\pgfqpoint{3.015000in}{2.529146in}}%
\pgfpathlineto{\pgfqpoint{3.046000in}{2.831402in}}%
\pgfpathlineto{\pgfqpoint{3.077000in}{2.711002in}}%
\pgfpathlineto{\pgfqpoint{3.108000in}{2.731798in}}%
\pgfpathlineto{\pgfqpoint{3.139000in}{2.657520in}}%
\pgfpathlineto{\pgfqpoint{3.170000in}{2.411418in}}%
\pgfpathlineto{\pgfqpoint{3.201000in}{2.305435in}}%
\pgfpathlineto{\pgfqpoint{3.232000in}{2.050697in}}%
\pgfpathlineto{\pgfqpoint{3.263000in}{1.916413in}}%
\pgfpathlineto{\pgfqpoint{3.294000in}{1.769907in}}%
\pgfpathlineto{\pgfqpoint{3.325000in}{1.565500in}}%
\pgfpathlineto{\pgfqpoint{3.356000in}{1.679396in}}%
\pgfpathlineto{\pgfqpoint{3.387000in}{1.693108in}}%
\pgfpathlineto{\pgfqpoint{3.418000in}{1.479269in}}%
\pgfpathlineto{\pgfqpoint{3.449000in}{1.290630in}}%
\pgfpathlineto{\pgfqpoint{3.480000in}{1.247191in}}%
\pgfpathlineto{\pgfqpoint{3.511000in}{1.446150in}}%
\pgfpathlineto{\pgfqpoint{3.542000in}{1.230685in}}%
\pgfpathlineto{\pgfqpoint{3.573000in}{1.209951in}}%
\pgfpathlineto{\pgfqpoint{3.604000in}{1.328474in}}%
\pgfpathlineto{\pgfqpoint{3.635000in}{1.170942in}}%
\pgfpathlineto{\pgfqpoint{3.666000in}{1.130183in}}%
\pgfpathlineto{\pgfqpoint{3.697000in}{1.193694in}}%
\pgfpathlineto{\pgfqpoint{3.728000in}{1.124083in}}%
\pgfpathlineto{\pgfqpoint{3.759000in}{1.070664in}}%
\pgfpathlineto{\pgfqpoint{3.790000in}{1.318354in}}%
\pgfpathlineto{\pgfqpoint{3.821000in}{1.409860in}}%
\pgfpathlineto{\pgfqpoint{3.852000in}{1.855656in}}%
\pgfpathlineto{\pgfqpoint{3.883000in}{2.061834in}}%
\pgfpathlineto{\pgfqpoint{3.914000in}{2.494417in}}%
\pgfpathlineto{\pgfqpoint{3.945000in}{2.456392in}}%
\pgfpathlineto{\pgfqpoint{3.976000in}{2.728097in}}%
\pgfpathlineto{\pgfqpoint{4.007000in}{2.606911in}}%
\pgfpathlineto{\pgfqpoint{4.038000in}{2.276507in}}%
\pgfpathlineto{\pgfqpoint{4.069000in}{2.275355in}}%
\pgfpathlineto{\pgfqpoint{4.100000in}{2.157924in}}%
\pgfpathlineto{\pgfqpoint{4.131000in}{1.957633in}}%
\pgfpathlineto{\pgfqpoint{4.162000in}{1.790088in}}%
\pgfpathlineto{\pgfqpoint{4.193000in}{1.732214in}}%
\pgfpathlineto{\pgfqpoint{4.224000in}{1.774313in}}%
\pgfpathlineto{\pgfqpoint{4.255000in}{1.713516in}}%
\pgfpathlineto{\pgfqpoint{4.286000in}{1.427251in}}%
\pgfpathlineto{\pgfqpoint{4.317000in}{1.398207in}}%
\pgfpathlineto{\pgfqpoint{4.348000in}{1.345233in}}%
\pgfpathlineto{\pgfqpoint{4.379000in}{1.269241in}}%
\pgfpathlineto{\pgfqpoint{4.410000in}{1.163179in}}%
\pgfpathlineto{\pgfqpoint{4.441000in}{1.326299in}}%
\pgfpathlineto{\pgfqpoint{4.472000in}{1.227645in}}%
\pgfpathlineto{\pgfqpoint{4.534000in}{1.110235in}}%
\pgfpathlineto{\pgfqpoint{4.565000in}{1.047911in}}%
\pgfpathlineto{\pgfqpoint{4.596000in}{1.252330in}}%
\pgfpathlineto{\pgfqpoint{4.627000in}{1.265932in}}%
\pgfpathlineto{\pgfqpoint{4.658000in}{1.099076in}}%
\pgfpathlineto{\pgfqpoint{4.689000in}{1.214949in}}%
\pgfpathlineto{\pgfqpoint{4.720000in}{0.967753in}}%
\pgfpathlineto{\pgfqpoint{4.751000in}{1.102921in}}%
\pgfpathlineto{\pgfqpoint{4.782000in}{1.124635in}}%
\pgfpathlineto{\pgfqpoint{4.813000in}{1.008293in}}%
\pgfpathlineto{\pgfqpoint{4.844000in}{1.063162in}}%
\pgfpathlineto{\pgfqpoint{4.875000in}{0.850457in}}%
\pgfpathlineto{\pgfqpoint{4.906000in}{1.010315in}}%
\pgfpathlineto{\pgfqpoint{4.937000in}{1.034456in}}%
\pgfpathlineto{\pgfqpoint{4.968000in}{0.941328in}}%
\pgfpathlineto{\pgfqpoint{4.999000in}{1.270392in}}%
\pgfpathlineto{\pgfqpoint{5.030000in}{1.161578in}}%
\pgfpathlineto{\pgfqpoint{5.061000in}{1.103757in}}%
\pgfpathlineto{\pgfqpoint{5.092000in}{1.178863in}}%
\pgfpathlineto{\pgfqpoint{5.123000in}{0.973378in}}%
\pgfpathlineto{\pgfqpoint{5.154000in}{1.054616in}}%
\pgfpathlineto{\pgfqpoint{5.185000in}{1.007248in}}%
\pgfpathlineto{\pgfqpoint{5.216000in}{0.937494in}}%
\pgfpathlineto{\pgfqpoint{5.247000in}{1.100240in}}%
\pgfpathlineto{\pgfqpoint{5.278000in}{1.121377in}}%
\pgfpathlineto{\pgfqpoint{5.309000in}{1.190827in}}%
\pgfpathlineto{\pgfqpoint{5.371000in}{1.088908in}}%
\pgfpathlineto{\pgfqpoint{5.402000in}{1.158118in}}%
\pgfpathlineto{\pgfqpoint{5.433000in}{1.180311in}}%
\pgfpathlineto{\pgfqpoint{5.464000in}{1.052012in}}%
\pgfpathlineto{\pgfqpoint{5.495000in}{1.024712in}}%
\pgfpathlineto{\pgfqpoint{5.526000in}{1.053084in}}%
\pgfpathlineto{\pgfqpoint{5.557000in}{1.083044in}}%
\pgfpathlineto{\pgfqpoint{5.588000in}{1.017625in}}%
\pgfpathlineto{\pgfqpoint{5.619000in}{0.927064in}}%
\pgfpathlineto{\pgfqpoint{5.650000in}{1.060330in}}%
\pgfpathlineto{\pgfqpoint{5.681000in}{1.064137in}}%
\pgfpathlineto{\pgfqpoint{5.712000in}{1.086579in}}%
\pgfpathlineto{\pgfqpoint{5.743000in}{1.126917in}}%
\pgfpathlineto{\pgfqpoint{5.774000in}{1.241899in}}%
\pgfpathlineto{\pgfqpoint{5.805000in}{1.099378in}}%
\pgfpathlineto{\pgfqpoint{5.836000in}{1.040084in}}%
\pgfpathlineto{\pgfqpoint{5.867000in}{1.065792in}}%
\pgfpathlineto{\pgfqpoint{5.898000in}{1.214255in}}%
\pgfpathlineto{\pgfqpoint{5.929000in}{1.204720in}}%
\pgfpathlineto{\pgfqpoint{5.960000in}{1.058180in}}%
\pgfpathlineto{\pgfqpoint{5.991000in}{1.028396in}}%
\pgfpathlineto{\pgfqpoint{6.022000in}{1.080612in}}%
\pgfpathlineto{\pgfqpoint{6.053000in}{1.009189in}}%
\pgfpathlineto{\pgfqpoint{6.084000in}{1.031824in}}%
\pgfpathlineto{\pgfqpoint{6.115000in}{1.090823in}}%
\pgfpathlineto{\pgfqpoint{6.146000in}{1.124433in}}%
\pgfpathlineto{\pgfqpoint{6.177000in}{1.074926in}}%
\pgfpathlineto{\pgfqpoint{6.208000in}{1.162426in}}%
\pgfpathlineto{\pgfqpoint{6.239000in}{1.024909in}}%
\pgfpathlineto{\pgfqpoint{6.270000in}{1.187317in}}%
\pgfpathlineto{\pgfqpoint{6.301000in}{0.988633in}}%
\pgfpathlineto{\pgfqpoint{6.332000in}{1.046368in}}%
\pgfpathlineto{\pgfqpoint{6.363000in}{1.064396in}}%
\pgfpathlineto{\pgfqpoint{6.394000in}{1.058598in}}%
\pgfpathlineto{\pgfqpoint{6.425000in}{1.248199in}}%
\pgfpathlineto{\pgfqpoint{6.456000in}{1.123116in}}%
\pgfpathlineto{\pgfqpoint{6.487000in}{1.200168in}}%
\pgfpathlineto{\pgfqpoint{6.518000in}{1.155370in}}%
\pgfpathlineto{\pgfqpoint{6.549000in}{1.162498in}}%
\pgfpathlineto{\pgfqpoint{6.580000in}{1.118317in}}%
\pgfpathlineto{\pgfqpoint{6.611000in}{1.032227in}}%
\pgfpathlineto{\pgfqpoint{6.642000in}{1.165316in}}%
\pgfpathlineto{\pgfqpoint{6.673000in}{1.075310in}}%
\pgfpathlineto{\pgfqpoint{6.704000in}{0.957843in}}%
\pgfpathlineto{\pgfqpoint{6.735000in}{1.106293in}}%
\pgfpathlineto{\pgfqpoint{6.766000in}{1.192996in}}%
\pgfpathlineto{\pgfqpoint{6.797000in}{1.165199in}}%
\pgfpathlineto{\pgfqpoint{6.828000in}{1.156176in}}%
\pgfpathlineto{\pgfqpoint{6.859000in}{0.948947in}}%
\pgfpathlineto{\pgfqpoint{6.890000in}{1.118939in}}%
\pgfpathlineto{\pgfqpoint{6.921000in}{0.894193in}}%
\pgfpathlineto{\pgfqpoint{6.952000in}{1.118916in}}%
\pgfpathlineto{\pgfqpoint{6.983000in}{0.998187in}}%
\pgfpathlineto{\pgfqpoint{7.014000in}{1.099235in}}%
\pgfpathlineto{\pgfqpoint{7.045000in}{1.124969in}}%
\pgfpathlineto{\pgfqpoint{7.076000in}{1.042502in}}%
\pgfpathlineto{\pgfqpoint{7.107000in}{1.031290in}}%
\pgfpathlineto{\pgfqpoint{7.138000in}{1.167985in}}%
\pgfpathlineto{\pgfqpoint{7.169000in}{1.272531in}}%
\pgfpathlineto{\pgfqpoint{7.200000in}{1.166951in}}%
\pgfpathlineto{\pgfqpoint{7.210000in}{1.105005in}}%
\pgfpathlineto{\pgfqpoint{7.210000in}{1.105005in}}%
\pgfusepath{stroke}%
\end{pgfscope}%
\begin{pgfscope}%
\pgfsetrectcap%
\pgfsetmiterjoin%
\pgfsetlinewidth{0.803000pt}%
\definecolor{currentstroke}{rgb}{0.000000,0.000000,0.000000}%
\pgfsetstrokecolor{currentstroke}%
\pgfsetdash{}{0pt}%
\pgfpathmoveto{\pgfqpoint{1.000000in}{0.720000in}}%
\pgfpathlineto{\pgfqpoint{1.000000in}{5.340000in}}%
\pgfusepath{stroke}%
\end{pgfscope}%
\begin{pgfscope}%
\pgfsetrectcap%
\pgfsetmiterjoin%
\pgfsetlinewidth{0.803000pt}%
\definecolor{currentstroke}{rgb}{0.000000,0.000000,0.000000}%
\pgfsetstrokecolor{currentstroke}%
\pgfsetdash{}{0pt}%
\pgfpathmoveto{\pgfqpoint{7.200000in}{0.720000in}}%
\pgfpathlineto{\pgfqpoint{7.200000in}{5.340000in}}%
\pgfusepath{stroke}%
\end{pgfscope}%
\begin{pgfscope}%
\pgfsetrectcap%
\pgfsetmiterjoin%
\pgfsetlinewidth{0.803000pt}%
\definecolor{currentstroke}{rgb}{0.000000,0.000000,0.000000}%
\pgfsetstrokecolor{currentstroke}%
\pgfsetdash{}{0pt}%
\pgfpathmoveto{\pgfqpoint{1.000000in}{0.720000in}}%
\pgfpathlineto{\pgfqpoint{7.200000in}{0.720000in}}%
\pgfusepath{stroke}%
\end{pgfscope}%
\begin{pgfscope}%
\pgfsetrectcap%
\pgfsetmiterjoin%
\pgfsetlinewidth{0.803000pt}%
\definecolor{currentstroke}{rgb}{0.000000,0.000000,0.000000}%
\pgfsetstrokecolor{currentstroke}%
\pgfsetdash{}{0pt}%
\pgfpathmoveto{\pgfqpoint{1.000000in}{5.340000in}}%
\pgfpathlineto{\pgfqpoint{7.200000in}{5.340000in}}%
\pgfusepath{stroke}%
\end{pgfscope}%
\begin{pgfscope}%
\pgfsetbuttcap%
\pgfsetroundjoin%
\definecolor{currentfill}{rgb}{0.000000,0.000000,0.000000}%
\pgfsetfillcolor{currentfill}%
\pgfsetlinewidth{0.803000pt}%
\definecolor{currentstroke}{rgb}{0.000000,0.000000,0.000000}%
\pgfsetstrokecolor{currentstroke}%
\pgfsetdash{}{0pt}%
\pgfsys@defobject{currentmarker}{\pgfqpoint{0.000000in}{0.000000in}}{\pgfqpoint{0.048611in}{0.000000in}}{%
\pgfpathmoveto{\pgfqpoint{0.000000in}{0.000000in}}%
\pgfpathlineto{\pgfqpoint{0.048611in}{0.000000in}}%
\pgfusepath{stroke,fill}%
}%
\begin{pgfscope}%
\pgfsys@transformshift{7.200000in}{1.109002in}%
\pgfsys@useobject{currentmarker}{}%
\end{pgfscope}%
\end{pgfscope}%
\begin{pgfscope}%
\definecolor{textcolor}{rgb}{0.000000,0.000000,0.000000}%
\pgfsetstrokecolor{textcolor}%
\pgfsetfillcolor{textcolor}%
\pgftext[x=7.297222in, y=1.008983in, left, base]{\color{textcolor}\sffamily\fontsize{20.000000}{24.000000}\selectfont 0.0}%
\end{pgfscope}%
\begin{pgfscope}%
\pgfsetbuttcap%
\pgfsetroundjoin%
\definecolor{currentfill}{rgb}{0.000000,0.000000,0.000000}%
\pgfsetfillcolor{currentfill}%
\pgfsetlinewidth{0.803000pt}%
\definecolor{currentstroke}{rgb}{0.000000,0.000000,0.000000}%
\pgfsetstrokecolor{currentstroke}%
\pgfsetdash{}{0pt}%
\pgfsys@defobject{currentmarker}{\pgfqpoint{0.000000in}{0.000000in}}{\pgfqpoint{0.048611in}{0.000000in}}{%
\pgfpathmoveto{\pgfqpoint{0.000000in}{0.000000in}}%
\pgfpathlineto{\pgfqpoint{0.048611in}{0.000000in}}%
\pgfusepath{stroke,fill}%
}%
\begin{pgfscope}%
\pgfsys@transformshift{7.200000in}{1.731208in}%
\pgfsys@useobject{currentmarker}{}%
\end{pgfscope}%
\end{pgfscope}%
\begin{pgfscope}%
\definecolor{textcolor}{rgb}{0.000000,0.000000,0.000000}%
\pgfsetstrokecolor{textcolor}%
\pgfsetfillcolor{textcolor}%
\pgftext[x=7.297222in, y=1.631188in, left, base]{\color{textcolor}\sffamily\fontsize{20.000000}{24.000000}\selectfont 0.2}%
\end{pgfscope}%
\begin{pgfscope}%
\pgfsetbuttcap%
\pgfsetroundjoin%
\definecolor{currentfill}{rgb}{0.000000,0.000000,0.000000}%
\pgfsetfillcolor{currentfill}%
\pgfsetlinewidth{0.803000pt}%
\definecolor{currentstroke}{rgb}{0.000000,0.000000,0.000000}%
\pgfsetstrokecolor{currentstroke}%
\pgfsetdash{}{0pt}%
\pgfsys@defobject{currentmarker}{\pgfqpoint{0.000000in}{0.000000in}}{\pgfqpoint{0.048611in}{0.000000in}}{%
\pgfpathmoveto{\pgfqpoint{0.000000in}{0.000000in}}%
\pgfpathlineto{\pgfqpoint{0.048611in}{0.000000in}}%
\pgfusepath{stroke,fill}%
}%
\begin{pgfscope}%
\pgfsys@transformshift{7.200000in}{2.353413in}%
\pgfsys@useobject{currentmarker}{}%
\end{pgfscope}%
\end{pgfscope}%
\begin{pgfscope}%
\definecolor{textcolor}{rgb}{0.000000,0.000000,0.000000}%
\pgfsetstrokecolor{textcolor}%
\pgfsetfillcolor{textcolor}%
\pgftext[x=7.297222in, y=2.253394in, left, base]{\color{textcolor}\sffamily\fontsize{20.000000}{24.000000}\selectfont 0.5}%
\end{pgfscope}%
\begin{pgfscope}%
\pgfsetbuttcap%
\pgfsetroundjoin%
\definecolor{currentfill}{rgb}{0.000000,0.000000,0.000000}%
\pgfsetfillcolor{currentfill}%
\pgfsetlinewidth{0.803000pt}%
\definecolor{currentstroke}{rgb}{0.000000,0.000000,0.000000}%
\pgfsetstrokecolor{currentstroke}%
\pgfsetdash{}{0pt}%
\pgfsys@defobject{currentmarker}{\pgfqpoint{0.000000in}{0.000000in}}{\pgfqpoint{0.048611in}{0.000000in}}{%
\pgfpathmoveto{\pgfqpoint{0.000000in}{0.000000in}}%
\pgfpathlineto{\pgfqpoint{0.048611in}{0.000000in}}%
\pgfusepath{stroke,fill}%
}%
\begin{pgfscope}%
\pgfsys@transformshift{7.200000in}{2.975619in}%
\pgfsys@useobject{currentmarker}{}%
\end{pgfscope}%
\end{pgfscope}%
\begin{pgfscope}%
\definecolor{textcolor}{rgb}{0.000000,0.000000,0.000000}%
\pgfsetstrokecolor{textcolor}%
\pgfsetfillcolor{textcolor}%
\pgftext[x=7.297222in, y=2.875600in, left, base]{\color{textcolor}\sffamily\fontsize{20.000000}{24.000000}\selectfont 0.8}%
\end{pgfscope}%
\begin{pgfscope}%
\pgfsetbuttcap%
\pgfsetroundjoin%
\definecolor{currentfill}{rgb}{0.000000,0.000000,0.000000}%
\pgfsetfillcolor{currentfill}%
\pgfsetlinewidth{0.803000pt}%
\definecolor{currentstroke}{rgb}{0.000000,0.000000,0.000000}%
\pgfsetstrokecolor{currentstroke}%
\pgfsetdash{}{0pt}%
\pgfsys@defobject{currentmarker}{\pgfqpoint{0.000000in}{0.000000in}}{\pgfqpoint{0.048611in}{0.000000in}}{%
\pgfpathmoveto{\pgfqpoint{0.000000in}{0.000000in}}%
\pgfpathlineto{\pgfqpoint{0.048611in}{0.000000in}}%
\pgfusepath{stroke,fill}%
}%
\begin{pgfscope}%
\pgfsys@transformshift{7.200000in}{3.597824in}%
\pgfsys@useobject{currentmarker}{}%
\end{pgfscope}%
\end{pgfscope}%
\begin{pgfscope}%
\definecolor{textcolor}{rgb}{0.000000,0.000000,0.000000}%
\pgfsetstrokecolor{textcolor}%
\pgfsetfillcolor{textcolor}%
\pgftext[x=7.297222in, y=3.497805in, left, base]{\color{textcolor}\sffamily\fontsize{20.000000}{24.000000}\selectfont 1.0}%
\end{pgfscope}%
\begin{pgfscope}%
\pgfsetbuttcap%
\pgfsetroundjoin%
\definecolor{currentfill}{rgb}{0.000000,0.000000,0.000000}%
\pgfsetfillcolor{currentfill}%
\pgfsetlinewidth{0.803000pt}%
\definecolor{currentstroke}{rgb}{0.000000,0.000000,0.000000}%
\pgfsetstrokecolor{currentstroke}%
\pgfsetdash{}{0pt}%
\pgfsys@defobject{currentmarker}{\pgfqpoint{0.000000in}{0.000000in}}{\pgfqpoint{0.048611in}{0.000000in}}{%
\pgfpathmoveto{\pgfqpoint{0.000000in}{0.000000in}}%
\pgfpathlineto{\pgfqpoint{0.048611in}{0.000000in}}%
\pgfusepath{stroke,fill}%
}%
\begin{pgfscope}%
\pgfsys@transformshift{7.200000in}{4.220030in}%
\pgfsys@useobject{currentmarker}{}%
\end{pgfscope}%
\end{pgfscope}%
\begin{pgfscope}%
\definecolor{textcolor}{rgb}{0.000000,0.000000,0.000000}%
\pgfsetstrokecolor{textcolor}%
\pgfsetfillcolor{textcolor}%
\pgftext[x=7.297222in, y=4.120011in, left, base]{\color{textcolor}\sffamily\fontsize{20.000000}{24.000000}\selectfont 1.2}%
\end{pgfscope}%
\begin{pgfscope}%
\pgfsetbuttcap%
\pgfsetroundjoin%
\definecolor{currentfill}{rgb}{0.000000,0.000000,0.000000}%
\pgfsetfillcolor{currentfill}%
\pgfsetlinewidth{0.803000pt}%
\definecolor{currentstroke}{rgb}{0.000000,0.000000,0.000000}%
\pgfsetstrokecolor{currentstroke}%
\pgfsetdash{}{0pt}%
\pgfsys@defobject{currentmarker}{\pgfqpoint{0.000000in}{0.000000in}}{\pgfqpoint{0.048611in}{0.000000in}}{%
\pgfpathmoveto{\pgfqpoint{0.000000in}{0.000000in}}%
\pgfpathlineto{\pgfqpoint{0.048611in}{0.000000in}}%
\pgfusepath{stroke,fill}%
}%
\begin{pgfscope}%
\pgfsys@transformshift{7.200000in}{4.842236in}%
\pgfsys@useobject{currentmarker}{}%
\end{pgfscope}%
\end{pgfscope}%
\begin{pgfscope}%
\definecolor{textcolor}{rgb}{0.000000,0.000000,0.000000}%
\pgfsetstrokecolor{textcolor}%
\pgfsetfillcolor{textcolor}%
\pgftext[x=7.297222in, y=4.742216in, left, base]{\color{textcolor}\sffamily\fontsize{20.000000}{24.000000}\selectfont 1.5}%
\end{pgfscope}%
\begin{pgfscope}%
\definecolor{textcolor}{rgb}{0.000000,0.000000,0.000000}%
\pgfsetstrokecolor{textcolor}%
\pgfsetfillcolor{textcolor}%
\pgftext[x=7.698906in,y=3.030000in,,top,rotate=90.000000]{\color{textcolor}\sffamily\fontsize{20.000000}{24.000000}\selectfont \(\displaystyle \mathrm{Charge}\)}%
\end{pgfscope}%
\begin{pgfscope}%
\pgfpathrectangle{\pgfqpoint{1.000000in}{0.720000in}}{\pgfqpoint{6.200000in}{4.620000in}}%
\pgfusepath{clip}%
\pgfsetbuttcap%
\pgfsetroundjoin%
\pgfsetlinewidth{0.501875pt}%
\definecolor{currentstroke}{rgb}{1.000000,0.000000,0.000000}%
\pgfsetstrokecolor{currentstroke}%
\pgfsetdash{}{0pt}%
\pgfpathmoveto{\pgfqpoint{1.930000in}{1.109002in}}%
\pgfpathlineto{\pgfqpoint{1.930000in}{3.270061in}}%
\pgfusepath{stroke}%
\end{pgfscope}%
\begin{pgfscope}%
\pgfpathrectangle{\pgfqpoint{1.000000in}{0.720000in}}{\pgfqpoint{6.200000in}{4.620000in}}%
\pgfusepath{clip}%
\pgfsetbuttcap%
\pgfsetroundjoin%
\pgfsetlinewidth{0.501875pt}%
\definecolor{currentstroke}{rgb}{1.000000,0.000000,0.000000}%
\pgfsetstrokecolor{currentstroke}%
\pgfsetdash{}{0pt}%
\pgfpathmoveto{\pgfqpoint{1.961000in}{1.109002in}}%
\pgfpathlineto{\pgfqpoint{1.961000in}{1.752433in}}%
\pgfusepath{stroke}%
\end{pgfscope}%
\begin{pgfscope}%
\pgfpathrectangle{\pgfqpoint{1.000000in}{0.720000in}}{\pgfqpoint{6.200000in}{4.620000in}}%
\pgfusepath{clip}%
\pgfsetbuttcap%
\pgfsetroundjoin%
\pgfsetlinewidth{0.501875pt}%
\definecolor{currentstroke}{rgb}{1.000000,0.000000,0.000000}%
\pgfsetstrokecolor{currentstroke}%
\pgfsetdash{}{0pt}%
\pgfpathmoveto{\pgfqpoint{1.992000in}{1.109002in}}%
\pgfpathlineto{\pgfqpoint{1.992000in}{2.981556in}}%
\pgfusepath{stroke}%
\end{pgfscope}%
\begin{pgfscope}%
\pgfpathrectangle{\pgfqpoint{1.000000in}{0.720000in}}{\pgfqpoint{6.200000in}{4.620000in}}%
\pgfusepath{clip}%
\pgfsetbuttcap%
\pgfsetroundjoin%
\pgfsetlinewidth{0.501875pt}%
\definecolor{currentstroke}{rgb}{1.000000,0.000000,0.000000}%
\pgfsetstrokecolor{currentstroke}%
\pgfsetdash{}{0pt}%
\pgfpathmoveto{\pgfqpoint{2.457000in}{1.109002in}}%
\pgfpathlineto{\pgfqpoint{2.457000in}{2.945749in}}%
\pgfusepath{stroke}%
\end{pgfscope}%
\begin{pgfscope}%
\pgfpathrectangle{\pgfqpoint{1.000000in}{0.720000in}}{\pgfqpoint{6.200000in}{4.620000in}}%
\pgfusepath{clip}%
\pgfsetbuttcap%
\pgfsetroundjoin%
\pgfsetlinewidth{0.501875pt}%
\definecolor{currentstroke}{rgb}{1.000000,0.000000,0.000000}%
\pgfsetstrokecolor{currentstroke}%
\pgfsetdash{}{0pt}%
\pgfpathmoveto{\pgfqpoint{2.488000in}{1.109002in}}%
\pgfpathlineto{\pgfqpoint{2.488000in}{2.713041in}}%
\pgfusepath{stroke}%
\end{pgfscope}%
\begin{pgfscope}%
\pgfpathrectangle{\pgfqpoint{1.000000in}{0.720000in}}{\pgfqpoint{6.200000in}{4.620000in}}%
\pgfusepath{clip}%
\pgfsetbuttcap%
\pgfsetroundjoin%
\pgfsetlinewidth{0.501875pt}%
\definecolor{currentstroke}{rgb}{1.000000,0.000000,0.000000}%
\pgfsetstrokecolor{currentstroke}%
\pgfsetdash{}{0pt}%
\pgfpathmoveto{\pgfqpoint{2.829000in}{1.109002in}}%
\pgfpathlineto{\pgfqpoint{2.829000in}{2.333591in}}%
\pgfusepath{stroke}%
\end{pgfscope}%
\begin{pgfscope}%
\pgfpathrectangle{\pgfqpoint{1.000000in}{0.720000in}}{\pgfqpoint{6.200000in}{4.620000in}}%
\pgfusepath{clip}%
\pgfsetbuttcap%
\pgfsetroundjoin%
\pgfsetlinewidth{0.501875pt}%
\definecolor{currentstroke}{rgb}{1.000000,0.000000,0.000000}%
\pgfsetstrokecolor{currentstroke}%
\pgfsetdash{}{0pt}%
\pgfpathmoveto{\pgfqpoint{2.860000in}{1.109002in}}%
\pgfpathlineto{\pgfqpoint{2.860000in}{2.364012in}}%
\pgfusepath{stroke}%
\end{pgfscope}%
\begin{pgfscope}%
\pgfpathrectangle{\pgfqpoint{1.000000in}{0.720000in}}{\pgfqpoint{6.200000in}{4.620000in}}%
\pgfusepath{clip}%
\pgfsetbuttcap%
\pgfsetroundjoin%
\pgfsetlinewidth{0.501875pt}%
\definecolor{currentstroke}{rgb}{1.000000,0.000000,0.000000}%
\pgfsetstrokecolor{currentstroke}%
\pgfsetdash{}{0pt}%
\pgfpathmoveto{\pgfqpoint{3.728000in}{1.109002in}}%
\pgfpathlineto{\pgfqpoint{3.728000in}{3.984189in}}%
\pgfusepath{stroke}%
\end{pgfscope}%
\begin{pgfscope}%
\pgfsetrectcap%
\pgfsetmiterjoin%
\pgfsetlinewidth{0.803000pt}%
\definecolor{currentstroke}{rgb}{0.000000,0.000000,0.000000}%
\pgfsetstrokecolor{currentstroke}%
\pgfsetdash{}{0pt}%
\pgfpathmoveto{\pgfqpoint{1.000000in}{0.720000in}}%
\pgfpathlineto{\pgfqpoint{1.000000in}{5.340000in}}%
\pgfusepath{stroke}%
\end{pgfscope}%
\begin{pgfscope}%
\pgfsetrectcap%
\pgfsetmiterjoin%
\pgfsetlinewidth{0.803000pt}%
\definecolor{currentstroke}{rgb}{0.000000,0.000000,0.000000}%
\pgfsetstrokecolor{currentstroke}%
\pgfsetdash{}{0pt}%
\pgfpathmoveto{\pgfqpoint{7.200000in}{0.720000in}}%
\pgfpathlineto{\pgfqpoint{7.200000in}{5.340000in}}%
\pgfusepath{stroke}%
\end{pgfscope}%
\begin{pgfscope}%
\pgfsetrectcap%
\pgfsetmiterjoin%
\pgfsetlinewidth{0.803000pt}%
\definecolor{currentstroke}{rgb}{0.000000,0.000000,0.000000}%
\pgfsetstrokecolor{currentstroke}%
\pgfsetdash{}{0pt}%
\pgfpathmoveto{\pgfqpoint{1.000000in}{0.720000in}}%
\pgfpathlineto{\pgfqpoint{7.200000in}{0.720000in}}%
\pgfusepath{stroke}%
\end{pgfscope}%
\begin{pgfscope}%
\pgfsetrectcap%
\pgfsetmiterjoin%
\pgfsetlinewidth{0.803000pt}%
\definecolor{currentstroke}{rgb}{0.000000,0.000000,0.000000}%
\pgfsetstrokecolor{currentstroke}%
\pgfsetdash{}{0pt}%
\pgfpathmoveto{\pgfqpoint{1.000000in}{5.340000in}}%
\pgfpathlineto{\pgfqpoint{7.200000in}{5.340000in}}%
\pgfusepath{stroke}%
\end{pgfscope}%
\begin{pgfscope}%
\pgfsetbuttcap%
\pgfsetmiterjoin%
\definecolor{currentfill}{rgb}{1.000000,1.000000,1.000000}%
\pgfsetfillcolor{currentfill}%
\pgfsetfillopacity{0.800000}%
\pgfsetlinewidth{1.003750pt}%
\definecolor{currentstroke}{rgb}{0.800000,0.800000,0.800000}%
\pgfsetstrokecolor{currentstroke}%
\pgfsetstrokeopacity{0.800000}%
\pgfsetdash{}{0pt}%
\pgfpathmoveto{\pgfqpoint{4.976872in}{4.327865in}}%
\pgfpathlineto{\pgfqpoint{7.005556in}{4.327865in}}%
\pgfpathquadraticcurveto{\pgfqpoint{7.061111in}{4.327865in}}{\pgfqpoint{7.061111in}{4.383420in}}%
\pgfpathlineto{\pgfqpoint{7.061111in}{5.145556in}}%
\pgfpathquadraticcurveto{\pgfqpoint{7.061111in}{5.201111in}}{\pgfqpoint{7.005556in}{5.201111in}}%
\pgfpathlineto{\pgfqpoint{4.976872in}{5.201111in}}%
\pgfpathquadraticcurveto{\pgfqpoint{4.921317in}{5.201111in}}{\pgfqpoint{4.921317in}{5.145556in}}%
\pgfpathlineto{\pgfqpoint{4.921317in}{4.383420in}}%
\pgfpathquadraticcurveto{\pgfqpoint{4.921317in}{4.327865in}}{\pgfqpoint{4.976872in}{4.327865in}}%
\pgfpathlineto{\pgfqpoint{4.976872in}{4.327865in}}%
\pgfpathclose%
\pgfusepath{stroke,fill}%
\end{pgfscope}%
\begin{pgfscope}%
\pgfsetrectcap%
\pgfsetroundjoin%
\pgfsetlinewidth{2.007500pt}%
\definecolor{currentstroke}{rgb}{0.121569,0.466667,0.705882}%
\pgfsetstrokecolor{currentstroke}%
\pgfsetdash{}{0pt}%
\pgfpathmoveto{\pgfqpoint{5.032428in}{4.987184in}}%
\pgfpathlineto{\pgfqpoint{5.310206in}{4.987184in}}%
\pgfpathlineto{\pgfqpoint{5.587983in}{4.987184in}}%
\pgfusepath{stroke}%
\end{pgfscope}%
\begin{pgfscope}%
\definecolor{textcolor}{rgb}{0.000000,0.000000,0.000000}%
\pgfsetstrokecolor{textcolor}%
\pgfsetfillcolor{textcolor}%
\pgftext[x=5.810206in,y=4.889962in,left,base]{\color{textcolor}\sffamily\fontsize{20.000000}{24.000000}\selectfont Waveform}%
\end{pgfscope}%
\begin{pgfscope}%
\pgfsetbuttcap%
\pgfsetroundjoin%
\pgfsetlinewidth{0.501875pt}%
\definecolor{currentstroke}{rgb}{1.000000,0.000000,0.000000}%
\pgfsetstrokecolor{currentstroke}%
\pgfsetdash{}{0pt}%
\pgfpathmoveto{\pgfqpoint{5.032428in}{4.592227in}}%
\pgfpathlineto{\pgfqpoint{5.587983in}{4.592227in}}%
\pgfusepath{stroke}%
\end{pgfscope}%
\begin{pgfscope}%
\definecolor{textcolor}{rgb}{0.000000,0.000000,0.000000}%
\pgfsetstrokecolor{textcolor}%
\pgfsetfillcolor{textcolor}%
\pgftext[x=5.810206in,y=4.495005in,left,base]{\color{textcolor}\sffamily\fontsize{20.000000}{24.000000}\selectfont Charge}%
\end{pgfscope}%
\end{pgfpicture}%
\makeatother%
\endgroup%
}
    \caption{\label{fig:cnn}An example giving \\ $\Delta t_0=\SI{-3.05}{ns}$, $\mathrm{RSS}=\SI{10.0}{mV^2}$, $D_\mathrm{w}=\SI{0.64}{ns}$.}
  \end{subfigure}
  \caption{\label{fig:cnn-performance}Demonstration of CNN on $\num[retain-unity-mantissa=false]{1e4}$ waveforms in~\subref{fig:cnn-npe} and one waveform in~\subref{fig:cnn} sampled from the same setup as figure~\ref{fig:method}. In figure~\subref{fig:cnn-npe}, the middle line is the mean of the distribution. The size of errorbar is from \SIrange{15.8}{84.1}{\percent} quantiles, corresponding to $\SI{\pm 1}{\sigma}$ of a Gaussian distribution. }
\end{figure}

\subsection{Regression analysis}
\label{sec:regression}
With the generative waveform model in eq.~\eqref{eq:1}, regression is ideal for analysis. Although computational complexity hinders the applications of regression by the vast volumes of raw data, the advancement of sparse models and big data infrastructures strengthens the advantage of regression.

We replace $\hat{N}_\mathrm{PE}$ with a fixed sample size $N_\mathrm{s}$ and $\hat{t}_i$ with a fixed grid of times $t'_i$ in eq.~\eqref{eq:w-hat}, 
\begin{equation}
  \label{eq:gd}
  w'(t) = \sum_{i=1}^{N_\mathrm{s}}q'_iV_\mathrm{PE}(t-t'_i).
\end{equation}
When $\{t'_i\}$ is dense enough, $\{\hat{q}_i\}$ determines the inferred PE distribution $\hat{\phi}(t)$,
\begin{equation}
  \label{eq:gd-phi}
  \hat{\phi}(t) = \sum_{i=1}^{N_\mathrm{s}}\hat{q}_i\delta(t-t'_i).
\end{equation}
From the output $\hat{\phi}_\mathrm{dec}(t)$ of a deconvolution method in section~\ref{sec:lucyddm}, we confidently leave out all the $t'_i$ that $\hat{\phi}_\mathrm{dec}(t_i')=0$ in eq.~\eqref{eq:gd-phi} to reduce the number of parameters and the complexity.


We attempted to replace the dense $\bm{t'}$ grid in eq.~\eqref{eq:gd} with a length-varying vector of sparse PEs. However, the truth $N_\mathrm{PE}$ is unknown and formulating an explicit trans-dimensional model is expansive.  We shall leave it to section~\ref{subsec:fsmp}.

\subsubsection{Direct charge fitting}
\label{sec:dcf}

Fitting the charges $q'_i$ in eq.~\eqref{eq:gd} directly by minimizing RSS of $w'(t)$ and $w(t)$, we get
\begin{equation}
  \label{eq:gd-q}
  \bm{\hat{q}} = \arg \underset{q'_i \ge 0}{\min}~\mathrm{RSS}\left[w'(t),w(t)\right].
\end{equation}
RSS of eq.~\eqref{eq:gd-q} does not suffer at the sparse configuration in figure~\ref{fig:l2} provided that the dense grid in eq.~\eqref{eq:gd} covers all the PEs.

Slawski and Hein~\cite{slawski_non-negative_2013} proved that in deconvolution problems, the non-negative least-squares formulation in eq.~\eqref{eq:gd-q} is self-regularized and gives sparse solutions of $q'_i$.  Peterson~\cite{peterson_developments_2021} from IceCube used this technique for waveform analysis.  We optimize eq.~\eqref{eq:gd-q} by Broyden-Fletcher-\allowbreak{}Goldfarb-Shanno algorithm with a bound constraint~\cite{byrd_limited_1995}.  In figure~\ref{fig:fitting-npe}, charge fitting is consistent in $D_\mathrm{w}$ for different $N_\mathrm{PE}$'s, showing its resilience to pile-up.

\begin{figure}[H]
  \begin{subfigure}{.5\textwidth}
    \centering
    \resizebox{\textwidth}{!}{%% Creator: Matplotlib, PGF backend
%%
%% To include the figure in your LaTeX document, write
%%   \input{<filename>.pgf}
%%
%% Make sure the required packages are loaded in your preamble
%%   \usepackage{pgf}
%%
%% Also ensure that all the required font packages are loaded; for instance,
%% the lmodern package is sometimes necessary when using math font.
%%   \usepackage{lmodern}
%%
%% Figures using additional raster images can only be included by \input if
%% they are in the same directory as the main LaTeX file. For loading figures
%% from other directories you can use the `import` package
%%   \usepackage{import}
%%
%% and then include the figures with
%%   \import{<path to file>}{<filename>.pgf}
%%
%% Matplotlib used the following preamble
%%   \usepackage[detect-all,locale=DE]{siunitx}
%%
\begingroup%
\makeatletter%
\begin{pgfpicture}%
\pgfpathrectangle{\pgfpointorigin}{\pgfqpoint{8.000000in}{6.000000in}}%
\pgfusepath{use as bounding box, clip}%
\begin{pgfscope}%
\pgfsetbuttcap%
\pgfsetmiterjoin%
\definecolor{currentfill}{rgb}{1.000000,1.000000,1.000000}%
\pgfsetfillcolor{currentfill}%
\pgfsetlinewidth{0.000000pt}%
\definecolor{currentstroke}{rgb}{1.000000,1.000000,1.000000}%
\pgfsetstrokecolor{currentstroke}%
\pgfsetdash{}{0pt}%
\pgfpathmoveto{\pgfqpoint{0.000000in}{0.000000in}}%
\pgfpathlineto{\pgfqpoint{8.000000in}{0.000000in}}%
\pgfpathlineto{\pgfqpoint{8.000000in}{6.000000in}}%
\pgfpathlineto{\pgfqpoint{0.000000in}{6.000000in}}%
\pgfpathlineto{\pgfqpoint{0.000000in}{0.000000in}}%
\pgfpathclose%
\pgfusepath{fill}%
\end{pgfscope}%
\begin{pgfscope}%
\pgfsetbuttcap%
\pgfsetmiterjoin%
\definecolor{currentfill}{rgb}{1.000000,1.000000,1.000000}%
\pgfsetfillcolor{currentfill}%
\pgfsetlinewidth{0.000000pt}%
\definecolor{currentstroke}{rgb}{0.000000,0.000000,0.000000}%
\pgfsetstrokecolor{currentstroke}%
\pgfsetstrokeopacity{0.000000}%
\pgfsetdash{}{0pt}%
\pgfpathmoveto{\pgfqpoint{1.000000in}{0.720000in}}%
\pgfpathlineto{\pgfqpoint{5.800000in}{0.720000in}}%
\pgfpathlineto{\pgfqpoint{5.800000in}{5.340000in}}%
\pgfpathlineto{\pgfqpoint{1.000000in}{5.340000in}}%
\pgfpathlineto{\pgfqpoint{1.000000in}{0.720000in}}%
\pgfpathclose%
\pgfusepath{fill}%
\end{pgfscope}%
\begin{pgfscope}%
\pgfpathrectangle{\pgfqpoint{1.000000in}{0.720000in}}{\pgfqpoint{4.800000in}{4.620000in}}%
\pgfusepath{clip}%
\pgfsetbuttcap%
\pgfsetroundjoin%
\definecolor{currentfill}{rgb}{0.121569,0.466667,0.705882}%
\pgfsetfillcolor{currentfill}%
\pgfsetfillopacity{0.100000}%
\pgfsetlinewidth{0.000000pt}%
\definecolor{currentstroke}{rgb}{0.000000,0.000000,0.000000}%
\pgfsetstrokecolor{currentstroke}%
\pgfsetdash{}{0pt}%
\pgfpathmoveto{\pgfqpoint{1.342857in}{3.918517in}}%
\pgfpathlineto{\pgfqpoint{1.342857in}{1.746620in}}%
\pgfpathlineto{\pgfqpoint{1.685714in}{2.204821in}}%
\pgfpathlineto{\pgfqpoint{2.028571in}{2.437215in}}%
\pgfpathlineto{\pgfqpoint{2.371429in}{2.486519in}}%
\pgfpathlineto{\pgfqpoint{2.714286in}{2.520965in}}%
\pgfpathlineto{\pgfqpoint{3.057143in}{2.590286in}}%
\pgfpathlineto{\pgfqpoint{3.400000in}{2.581981in}}%
\pgfpathlineto{\pgfqpoint{3.742857in}{2.671545in}}%
\pgfpathlineto{\pgfqpoint{4.085714in}{2.513838in}}%
\pgfpathlineto{\pgfqpoint{4.428571in}{2.516011in}}%
\pgfpathlineto{\pgfqpoint{4.771429in}{2.616119in}}%
\pgfpathlineto{\pgfqpoint{5.114286in}{2.840190in}}%
\pgfpathlineto{\pgfqpoint{5.457143in}{3.291601in}}%
\pgfpathlineto{\pgfqpoint{5.457143in}{3.291601in}}%
\pgfpathlineto{\pgfqpoint{5.457143in}{3.291601in}}%
\pgfpathlineto{\pgfqpoint{5.114286in}{3.082018in}}%
\pgfpathlineto{\pgfqpoint{4.771429in}{3.459067in}}%
\pgfpathlineto{\pgfqpoint{4.428571in}{3.316363in}}%
\pgfpathlineto{\pgfqpoint{4.085714in}{3.560995in}}%
\pgfpathlineto{\pgfqpoint{3.742857in}{3.742335in}}%
\pgfpathlineto{\pgfqpoint{3.400000in}{3.772836in}}%
\pgfpathlineto{\pgfqpoint{3.057143in}{3.871905in}}%
\pgfpathlineto{\pgfqpoint{2.714286in}{3.965745in}}%
\pgfpathlineto{\pgfqpoint{2.371429in}{4.035488in}}%
\pgfpathlineto{\pgfqpoint{2.028571in}{4.129949in}}%
\pgfpathlineto{\pgfqpoint{1.685714in}{4.241646in}}%
\pgfpathlineto{\pgfqpoint{1.342857in}{3.918517in}}%
\pgfpathlineto{\pgfqpoint{1.342857in}{3.918517in}}%
\pgfpathclose%
\pgfusepath{fill}%
\end{pgfscope}%
\begin{pgfscope}%
\pgfsetbuttcap%
\pgfsetroundjoin%
\definecolor{currentfill}{rgb}{0.000000,0.000000,0.000000}%
\pgfsetfillcolor{currentfill}%
\pgfsetlinewidth{0.803000pt}%
\definecolor{currentstroke}{rgb}{0.000000,0.000000,0.000000}%
\pgfsetstrokecolor{currentstroke}%
\pgfsetdash{}{0pt}%
\pgfsys@defobject{currentmarker}{\pgfqpoint{0.000000in}{-0.048611in}}{\pgfqpoint{0.000000in}{0.000000in}}{%
\pgfpathmoveto{\pgfqpoint{0.000000in}{0.000000in}}%
\pgfpathlineto{\pgfqpoint{0.000000in}{-0.048611in}}%
\pgfusepath{stroke,fill}%
}%
\begin{pgfscope}%
\pgfsys@transformshift{1.342857in}{0.720000in}%
\pgfsys@useobject{currentmarker}{}%
\end{pgfscope}%
\end{pgfscope}%
\begin{pgfscope}%
\definecolor{textcolor}{rgb}{0.000000,0.000000,0.000000}%
\pgfsetstrokecolor{textcolor}%
\pgfsetfillcolor{textcolor}%
\pgftext[x=1.342857in,y=0.622778in,,top]{\color{textcolor}\sffamily\fontsize{20.000000}{24.000000}\selectfont 1}%
\end{pgfscope}%
\begin{pgfscope}%
\pgfsetbuttcap%
\pgfsetroundjoin%
\definecolor{currentfill}{rgb}{0.000000,0.000000,0.000000}%
\pgfsetfillcolor{currentfill}%
\pgfsetlinewidth{0.803000pt}%
\definecolor{currentstroke}{rgb}{0.000000,0.000000,0.000000}%
\pgfsetstrokecolor{currentstroke}%
\pgfsetdash{}{0pt}%
\pgfsys@defobject{currentmarker}{\pgfqpoint{0.000000in}{-0.048611in}}{\pgfqpoint{0.000000in}{0.000000in}}{%
\pgfpathmoveto{\pgfqpoint{0.000000in}{0.000000in}}%
\pgfpathlineto{\pgfqpoint{0.000000in}{-0.048611in}}%
\pgfusepath{stroke,fill}%
}%
\begin{pgfscope}%
\pgfsys@transformshift{2.028571in}{0.720000in}%
\pgfsys@useobject{currentmarker}{}%
\end{pgfscope}%
\end{pgfscope}%
\begin{pgfscope}%
\definecolor{textcolor}{rgb}{0.000000,0.000000,0.000000}%
\pgfsetstrokecolor{textcolor}%
\pgfsetfillcolor{textcolor}%
\pgftext[x=2.028571in,y=0.622778in,,top]{\color{textcolor}\sffamily\fontsize{20.000000}{24.000000}\selectfont 3}%
\end{pgfscope}%
\begin{pgfscope}%
\pgfsetbuttcap%
\pgfsetroundjoin%
\definecolor{currentfill}{rgb}{0.000000,0.000000,0.000000}%
\pgfsetfillcolor{currentfill}%
\pgfsetlinewidth{0.803000pt}%
\definecolor{currentstroke}{rgb}{0.000000,0.000000,0.000000}%
\pgfsetstrokecolor{currentstroke}%
\pgfsetdash{}{0pt}%
\pgfsys@defobject{currentmarker}{\pgfqpoint{0.000000in}{-0.048611in}}{\pgfqpoint{0.000000in}{0.000000in}}{%
\pgfpathmoveto{\pgfqpoint{0.000000in}{0.000000in}}%
\pgfpathlineto{\pgfqpoint{0.000000in}{-0.048611in}}%
\pgfusepath{stroke,fill}%
}%
\begin{pgfscope}%
\pgfsys@transformshift{2.714286in}{0.720000in}%
\pgfsys@useobject{currentmarker}{}%
\end{pgfscope}%
\end{pgfscope}%
\begin{pgfscope}%
\definecolor{textcolor}{rgb}{0.000000,0.000000,0.000000}%
\pgfsetstrokecolor{textcolor}%
\pgfsetfillcolor{textcolor}%
\pgftext[x=2.714286in,y=0.622778in,,top]{\color{textcolor}\sffamily\fontsize{20.000000}{24.000000}\selectfont 5}%
\end{pgfscope}%
\begin{pgfscope}%
\pgfsetbuttcap%
\pgfsetroundjoin%
\definecolor{currentfill}{rgb}{0.000000,0.000000,0.000000}%
\pgfsetfillcolor{currentfill}%
\pgfsetlinewidth{0.803000pt}%
\definecolor{currentstroke}{rgb}{0.000000,0.000000,0.000000}%
\pgfsetstrokecolor{currentstroke}%
\pgfsetdash{}{0pt}%
\pgfsys@defobject{currentmarker}{\pgfqpoint{0.000000in}{-0.048611in}}{\pgfqpoint{0.000000in}{0.000000in}}{%
\pgfpathmoveto{\pgfqpoint{0.000000in}{0.000000in}}%
\pgfpathlineto{\pgfqpoint{0.000000in}{-0.048611in}}%
\pgfusepath{stroke,fill}%
}%
\begin{pgfscope}%
\pgfsys@transformshift{3.400000in}{0.720000in}%
\pgfsys@useobject{currentmarker}{}%
\end{pgfscope}%
\end{pgfscope}%
\begin{pgfscope}%
\definecolor{textcolor}{rgb}{0.000000,0.000000,0.000000}%
\pgfsetstrokecolor{textcolor}%
\pgfsetfillcolor{textcolor}%
\pgftext[x=3.400000in,y=0.622778in,,top]{\color{textcolor}\sffamily\fontsize{20.000000}{24.000000}\selectfont 7}%
\end{pgfscope}%
\begin{pgfscope}%
\pgfsetbuttcap%
\pgfsetroundjoin%
\definecolor{currentfill}{rgb}{0.000000,0.000000,0.000000}%
\pgfsetfillcolor{currentfill}%
\pgfsetlinewidth{0.803000pt}%
\definecolor{currentstroke}{rgb}{0.000000,0.000000,0.000000}%
\pgfsetstrokecolor{currentstroke}%
\pgfsetdash{}{0pt}%
\pgfsys@defobject{currentmarker}{\pgfqpoint{0.000000in}{-0.048611in}}{\pgfqpoint{0.000000in}{0.000000in}}{%
\pgfpathmoveto{\pgfqpoint{0.000000in}{0.000000in}}%
\pgfpathlineto{\pgfqpoint{0.000000in}{-0.048611in}}%
\pgfusepath{stroke,fill}%
}%
\begin{pgfscope}%
\pgfsys@transformshift{4.085714in}{0.720000in}%
\pgfsys@useobject{currentmarker}{}%
\end{pgfscope}%
\end{pgfscope}%
\begin{pgfscope}%
\definecolor{textcolor}{rgb}{0.000000,0.000000,0.000000}%
\pgfsetstrokecolor{textcolor}%
\pgfsetfillcolor{textcolor}%
\pgftext[x=4.085714in,y=0.622778in,,top]{\color{textcolor}\sffamily\fontsize{20.000000}{24.000000}\selectfont 9}%
\end{pgfscope}%
\begin{pgfscope}%
\pgfsetbuttcap%
\pgfsetroundjoin%
\definecolor{currentfill}{rgb}{0.000000,0.000000,0.000000}%
\pgfsetfillcolor{currentfill}%
\pgfsetlinewidth{0.803000pt}%
\definecolor{currentstroke}{rgb}{0.000000,0.000000,0.000000}%
\pgfsetstrokecolor{currentstroke}%
\pgfsetdash{}{0pt}%
\pgfsys@defobject{currentmarker}{\pgfqpoint{0.000000in}{-0.048611in}}{\pgfqpoint{0.000000in}{0.000000in}}{%
\pgfpathmoveto{\pgfqpoint{0.000000in}{0.000000in}}%
\pgfpathlineto{\pgfqpoint{0.000000in}{-0.048611in}}%
\pgfusepath{stroke,fill}%
}%
\begin{pgfscope}%
\pgfsys@transformshift{4.771429in}{0.720000in}%
\pgfsys@useobject{currentmarker}{}%
\end{pgfscope}%
\end{pgfscope}%
\begin{pgfscope}%
\definecolor{textcolor}{rgb}{0.000000,0.000000,0.000000}%
\pgfsetstrokecolor{textcolor}%
\pgfsetfillcolor{textcolor}%
\pgftext[x=4.771429in,y=0.622778in,,top]{\color{textcolor}\sffamily\fontsize{20.000000}{24.000000}\selectfont 11}%
\end{pgfscope}%
\begin{pgfscope}%
\pgfsetbuttcap%
\pgfsetroundjoin%
\definecolor{currentfill}{rgb}{0.000000,0.000000,0.000000}%
\pgfsetfillcolor{currentfill}%
\pgfsetlinewidth{0.803000pt}%
\definecolor{currentstroke}{rgb}{0.000000,0.000000,0.000000}%
\pgfsetstrokecolor{currentstroke}%
\pgfsetdash{}{0pt}%
\pgfsys@defobject{currentmarker}{\pgfqpoint{0.000000in}{-0.048611in}}{\pgfqpoint{0.000000in}{0.000000in}}{%
\pgfpathmoveto{\pgfqpoint{0.000000in}{0.000000in}}%
\pgfpathlineto{\pgfqpoint{0.000000in}{-0.048611in}}%
\pgfusepath{stroke,fill}%
}%
\begin{pgfscope}%
\pgfsys@transformshift{5.457143in}{0.720000in}%
\pgfsys@useobject{currentmarker}{}%
\end{pgfscope}%
\end{pgfscope}%
\begin{pgfscope}%
\definecolor{textcolor}{rgb}{0.000000,0.000000,0.000000}%
\pgfsetstrokecolor{textcolor}%
\pgfsetfillcolor{textcolor}%
\pgftext[x=5.457143in,y=0.622778in,,top]{\color{textcolor}\sffamily\fontsize{20.000000}{24.000000}\selectfont 13}%
\end{pgfscope}%
\begin{pgfscope}%
\definecolor{textcolor}{rgb}{0.000000,0.000000,0.000000}%
\pgfsetstrokecolor{textcolor}%
\pgfsetfillcolor{textcolor}%
\pgftext[x=3.400000in,y=0.311155in,,top]{\color{textcolor}\sffamily\fontsize{20.000000}{24.000000}\selectfont \(\displaystyle N_{\mathrm{PE}}\)}%
\end{pgfscope}%
\begin{pgfscope}%
\pgfsetbuttcap%
\pgfsetroundjoin%
\definecolor{currentfill}{rgb}{0.000000,0.000000,0.000000}%
\pgfsetfillcolor{currentfill}%
\pgfsetlinewidth{0.803000pt}%
\definecolor{currentstroke}{rgb}{0.000000,0.000000,0.000000}%
\pgfsetstrokecolor{currentstroke}%
\pgfsetdash{}{0pt}%
\pgfsys@defobject{currentmarker}{\pgfqpoint{-0.048611in}{0.000000in}}{\pgfqpoint{-0.000000in}{0.000000in}}{%
\pgfpathmoveto{\pgfqpoint{-0.000000in}{0.000000in}}%
\pgfpathlineto{\pgfqpoint{-0.048611in}{0.000000in}}%
\pgfusepath{stroke,fill}%
}%
\begin{pgfscope}%
\pgfsys@transformshift{1.000000in}{0.720000in}%
\pgfsys@useobject{currentmarker}{}%
\end{pgfscope}%
\end{pgfscope}%
\begin{pgfscope}%
\definecolor{textcolor}{rgb}{0.000000,0.000000,0.000000}%
\pgfsetstrokecolor{textcolor}%
\pgfsetfillcolor{textcolor}%
\pgftext[x=0.560215in, y=0.619981in, left, base]{\color{textcolor}\sffamily\fontsize{20.000000}{24.000000}\selectfont \(\displaystyle {0.0}\)}%
\end{pgfscope}%
\begin{pgfscope}%
\pgfsetbuttcap%
\pgfsetroundjoin%
\definecolor{currentfill}{rgb}{0.000000,0.000000,0.000000}%
\pgfsetfillcolor{currentfill}%
\pgfsetlinewidth{0.803000pt}%
\definecolor{currentstroke}{rgb}{0.000000,0.000000,0.000000}%
\pgfsetstrokecolor{currentstroke}%
\pgfsetdash{}{0pt}%
\pgfsys@defobject{currentmarker}{\pgfqpoint{-0.048611in}{0.000000in}}{\pgfqpoint{-0.000000in}{0.000000in}}{%
\pgfpathmoveto{\pgfqpoint{-0.000000in}{0.000000in}}%
\pgfpathlineto{\pgfqpoint{-0.048611in}{0.000000in}}%
\pgfusepath{stroke,fill}%
}%
\begin{pgfscope}%
\pgfsys@transformshift{1.000000in}{1.377881in}%
\pgfsys@useobject{currentmarker}{}%
\end{pgfscope}%
\end{pgfscope}%
\begin{pgfscope}%
\definecolor{textcolor}{rgb}{0.000000,0.000000,0.000000}%
\pgfsetstrokecolor{textcolor}%
\pgfsetfillcolor{textcolor}%
\pgftext[x=0.560215in, y=1.277862in, left, base]{\color{textcolor}\sffamily\fontsize{20.000000}{24.000000}\selectfont \(\displaystyle {0.2}\)}%
\end{pgfscope}%
\begin{pgfscope}%
\pgfsetbuttcap%
\pgfsetroundjoin%
\definecolor{currentfill}{rgb}{0.000000,0.000000,0.000000}%
\pgfsetfillcolor{currentfill}%
\pgfsetlinewidth{0.803000pt}%
\definecolor{currentstroke}{rgb}{0.000000,0.000000,0.000000}%
\pgfsetstrokecolor{currentstroke}%
\pgfsetdash{}{0pt}%
\pgfsys@defobject{currentmarker}{\pgfqpoint{-0.048611in}{0.000000in}}{\pgfqpoint{-0.000000in}{0.000000in}}{%
\pgfpathmoveto{\pgfqpoint{-0.000000in}{0.000000in}}%
\pgfpathlineto{\pgfqpoint{-0.048611in}{0.000000in}}%
\pgfusepath{stroke,fill}%
}%
\begin{pgfscope}%
\pgfsys@transformshift{1.000000in}{2.035762in}%
\pgfsys@useobject{currentmarker}{}%
\end{pgfscope}%
\end{pgfscope}%
\begin{pgfscope}%
\definecolor{textcolor}{rgb}{0.000000,0.000000,0.000000}%
\pgfsetstrokecolor{textcolor}%
\pgfsetfillcolor{textcolor}%
\pgftext[x=0.560215in, y=1.935743in, left, base]{\color{textcolor}\sffamily\fontsize{20.000000}{24.000000}\selectfont \(\displaystyle {0.4}\)}%
\end{pgfscope}%
\begin{pgfscope}%
\pgfsetbuttcap%
\pgfsetroundjoin%
\definecolor{currentfill}{rgb}{0.000000,0.000000,0.000000}%
\pgfsetfillcolor{currentfill}%
\pgfsetlinewidth{0.803000pt}%
\definecolor{currentstroke}{rgb}{0.000000,0.000000,0.000000}%
\pgfsetstrokecolor{currentstroke}%
\pgfsetdash{}{0pt}%
\pgfsys@defobject{currentmarker}{\pgfqpoint{-0.048611in}{0.000000in}}{\pgfqpoint{-0.000000in}{0.000000in}}{%
\pgfpathmoveto{\pgfqpoint{-0.000000in}{0.000000in}}%
\pgfpathlineto{\pgfqpoint{-0.048611in}{0.000000in}}%
\pgfusepath{stroke,fill}%
}%
\begin{pgfscope}%
\pgfsys@transformshift{1.000000in}{2.693643in}%
\pgfsys@useobject{currentmarker}{}%
\end{pgfscope}%
\end{pgfscope}%
\begin{pgfscope}%
\definecolor{textcolor}{rgb}{0.000000,0.000000,0.000000}%
\pgfsetstrokecolor{textcolor}%
\pgfsetfillcolor{textcolor}%
\pgftext[x=0.560215in, y=2.593624in, left, base]{\color{textcolor}\sffamily\fontsize{20.000000}{24.000000}\selectfont \(\displaystyle {0.6}\)}%
\end{pgfscope}%
\begin{pgfscope}%
\pgfsetbuttcap%
\pgfsetroundjoin%
\definecolor{currentfill}{rgb}{0.000000,0.000000,0.000000}%
\pgfsetfillcolor{currentfill}%
\pgfsetlinewidth{0.803000pt}%
\definecolor{currentstroke}{rgb}{0.000000,0.000000,0.000000}%
\pgfsetstrokecolor{currentstroke}%
\pgfsetdash{}{0pt}%
\pgfsys@defobject{currentmarker}{\pgfqpoint{-0.048611in}{0.000000in}}{\pgfqpoint{-0.000000in}{0.000000in}}{%
\pgfpathmoveto{\pgfqpoint{-0.000000in}{0.000000in}}%
\pgfpathlineto{\pgfqpoint{-0.048611in}{0.000000in}}%
\pgfusepath{stroke,fill}%
}%
\begin{pgfscope}%
\pgfsys@transformshift{1.000000in}{3.351524in}%
\pgfsys@useobject{currentmarker}{}%
\end{pgfscope}%
\end{pgfscope}%
\begin{pgfscope}%
\definecolor{textcolor}{rgb}{0.000000,0.000000,0.000000}%
\pgfsetstrokecolor{textcolor}%
\pgfsetfillcolor{textcolor}%
\pgftext[x=0.560215in, y=3.251505in, left, base]{\color{textcolor}\sffamily\fontsize{20.000000}{24.000000}\selectfont \(\displaystyle {0.8}\)}%
\end{pgfscope}%
\begin{pgfscope}%
\pgfsetbuttcap%
\pgfsetroundjoin%
\definecolor{currentfill}{rgb}{0.000000,0.000000,0.000000}%
\pgfsetfillcolor{currentfill}%
\pgfsetlinewidth{0.803000pt}%
\definecolor{currentstroke}{rgb}{0.000000,0.000000,0.000000}%
\pgfsetstrokecolor{currentstroke}%
\pgfsetdash{}{0pt}%
\pgfsys@defobject{currentmarker}{\pgfqpoint{-0.048611in}{0.000000in}}{\pgfqpoint{-0.000000in}{0.000000in}}{%
\pgfpathmoveto{\pgfqpoint{-0.000000in}{0.000000in}}%
\pgfpathlineto{\pgfqpoint{-0.048611in}{0.000000in}}%
\pgfusepath{stroke,fill}%
}%
\begin{pgfscope}%
\pgfsys@transformshift{1.000000in}{4.009405in}%
\pgfsys@useobject{currentmarker}{}%
\end{pgfscope}%
\end{pgfscope}%
\begin{pgfscope}%
\definecolor{textcolor}{rgb}{0.000000,0.000000,0.000000}%
\pgfsetstrokecolor{textcolor}%
\pgfsetfillcolor{textcolor}%
\pgftext[x=0.560215in, y=3.909386in, left, base]{\color{textcolor}\sffamily\fontsize{20.000000}{24.000000}\selectfont \(\displaystyle {1.0}\)}%
\end{pgfscope}%
\begin{pgfscope}%
\pgfsetbuttcap%
\pgfsetroundjoin%
\definecolor{currentfill}{rgb}{0.000000,0.000000,0.000000}%
\pgfsetfillcolor{currentfill}%
\pgfsetlinewidth{0.803000pt}%
\definecolor{currentstroke}{rgb}{0.000000,0.000000,0.000000}%
\pgfsetstrokecolor{currentstroke}%
\pgfsetdash{}{0pt}%
\pgfsys@defobject{currentmarker}{\pgfqpoint{-0.048611in}{0.000000in}}{\pgfqpoint{-0.000000in}{0.000000in}}{%
\pgfpathmoveto{\pgfqpoint{-0.000000in}{0.000000in}}%
\pgfpathlineto{\pgfqpoint{-0.048611in}{0.000000in}}%
\pgfusepath{stroke,fill}%
}%
\begin{pgfscope}%
\pgfsys@transformshift{1.000000in}{4.667286in}%
\pgfsys@useobject{currentmarker}{}%
\end{pgfscope}%
\end{pgfscope}%
\begin{pgfscope}%
\definecolor{textcolor}{rgb}{0.000000,0.000000,0.000000}%
\pgfsetstrokecolor{textcolor}%
\pgfsetfillcolor{textcolor}%
\pgftext[x=0.560215in, y=4.567267in, left, base]{\color{textcolor}\sffamily\fontsize{20.000000}{24.000000}\selectfont \(\displaystyle {1.2}\)}%
\end{pgfscope}%
\begin{pgfscope}%
\pgfsetbuttcap%
\pgfsetroundjoin%
\definecolor{currentfill}{rgb}{0.000000,0.000000,0.000000}%
\pgfsetfillcolor{currentfill}%
\pgfsetlinewidth{0.803000pt}%
\definecolor{currentstroke}{rgb}{0.000000,0.000000,0.000000}%
\pgfsetstrokecolor{currentstroke}%
\pgfsetdash{}{0pt}%
\pgfsys@defobject{currentmarker}{\pgfqpoint{-0.048611in}{0.000000in}}{\pgfqpoint{-0.000000in}{0.000000in}}{%
\pgfpathmoveto{\pgfqpoint{-0.000000in}{0.000000in}}%
\pgfpathlineto{\pgfqpoint{-0.048611in}{0.000000in}}%
\pgfusepath{stroke,fill}%
}%
\begin{pgfscope}%
\pgfsys@transformshift{1.000000in}{5.325167in}%
\pgfsys@useobject{currentmarker}{}%
\end{pgfscope}%
\end{pgfscope}%
\begin{pgfscope}%
\definecolor{textcolor}{rgb}{0.000000,0.000000,0.000000}%
\pgfsetstrokecolor{textcolor}%
\pgfsetfillcolor{textcolor}%
\pgftext[x=0.560215in, y=5.225148in, left, base]{\color{textcolor}\sffamily\fontsize{20.000000}{24.000000}\selectfont \(\displaystyle {1.4}\)}%
\end{pgfscope}%
\begin{pgfscope}%
\definecolor{textcolor}{rgb}{0.000000,0.000000,0.000000}%
\pgfsetstrokecolor{textcolor}%
\pgfsetfillcolor{textcolor}%
\pgftext[x=0.504660in,y=3.030000in,,bottom,rotate=90.000000]{\color{textcolor}\sffamily\fontsize{20.000000}{24.000000}\selectfont \(\displaystyle \mathrm{Wasserstein\ Distance}/\si{ns}\)}%
\end{pgfscope}%
\begin{pgfscope}%
\pgfpathrectangle{\pgfqpoint{1.000000in}{0.720000in}}{\pgfqpoint{4.800000in}{4.620000in}}%
\pgfusepath{clip}%
\pgfsetrectcap%
\pgfsetroundjoin%
\pgfsetlinewidth{2.007500pt}%
\definecolor{currentstroke}{rgb}{0.000000,0.000000,1.000000}%
\pgfsetstrokecolor{currentstroke}%
\pgfsetdash{}{0pt}%
\pgfpathmoveto{\pgfqpoint{1.342857in}{2.534844in}}%
\pgfpathlineto{\pgfqpoint{1.685714in}{2.999071in}}%
\pgfpathlineto{\pgfqpoint{2.028571in}{3.133551in}}%
\pgfpathlineto{\pgfqpoint{2.371429in}{3.128694in}}%
\pgfpathlineto{\pgfqpoint{2.714286in}{3.118514in}}%
\pgfpathlineto{\pgfqpoint{3.057143in}{3.132624in}}%
\pgfpathlineto{\pgfqpoint{3.400000in}{3.122070in}}%
\pgfpathlineto{\pgfqpoint{3.742857in}{3.089822in}}%
\pgfpathlineto{\pgfqpoint{4.085714in}{3.014745in}}%
\pgfpathlineto{\pgfqpoint{4.428571in}{2.885237in}}%
\pgfpathlineto{\pgfqpoint{4.771429in}{2.934256in}}%
\pgfpathlineto{\pgfqpoint{5.114286in}{2.968502in}}%
\pgfpathlineto{\pgfqpoint{5.457143in}{3.291601in}}%
\pgfusepath{stroke}%
\end{pgfscope}%
\begin{pgfscope}%
\pgfpathrectangle{\pgfqpoint{1.000000in}{0.720000in}}{\pgfqpoint{4.800000in}{4.620000in}}%
\pgfusepath{clip}%
\pgfsetbuttcap%
\pgfsetroundjoin%
\pgfsetlinewidth{1.003750pt}%
\definecolor{currentstroke}{rgb}{0.000000,0.000000,1.000000}%
\pgfsetstrokecolor{currentstroke}%
\pgfsetdash{}{0pt}%
\pgfpathmoveto{\pgfqpoint{1.342857in}{1.746620in}}%
\pgfpathlineto{\pgfqpoint{1.342857in}{3.918517in}}%
\pgfusepath{stroke}%
\end{pgfscope}%
\begin{pgfscope}%
\pgfpathrectangle{\pgfqpoint{1.000000in}{0.720000in}}{\pgfqpoint{4.800000in}{4.620000in}}%
\pgfusepath{clip}%
\pgfsetbuttcap%
\pgfsetroundjoin%
\pgfsetlinewidth{1.003750pt}%
\definecolor{currentstroke}{rgb}{0.000000,0.000000,1.000000}%
\pgfsetstrokecolor{currentstroke}%
\pgfsetdash{}{0pt}%
\pgfpathmoveto{\pgfqpoint{1.685714in}{2.204821in}}%
\pgfpathlineto{\pgfqpoint{1.685714in}{4.241646in}}%
\pgfusepath{stroke}%
\end{pgfscope}%
\begin{pgfscope}%
\pgfpathrectangle{\pgfqpoint{1.000000in}{0.720000in}}{\pgfqpoint{4.800000in}{4.620000in}}%
\pgfusepath{clip}%
\pgfsetbuttcap%
\pgfsetroundjoin%
\pgfsetlinewidth{1.003750pt}%
\definecolor{currentstroke}{rgb}{0.000000,0.000000,1.000000}%
\pgfsetstrokecolor{currentstroke}%
\pgfsetdash{}{0pt}%
\pgfpathmoveto{\pgfqpoint{2.028571in}{2.437215in}}%
\pgfpathlineto{\pgfqpoint{2.028571in}{4.129949in}}%
\pgfusepath{stroke}%
\end{pgfscope}%
\begin{pgfscope}%
\pgfpathrectangle{\pgfqpoint{1.000000in}{0.720000in}}{\pgfqpoint{4.800000in}{4.620000in}}%
\pgfusepath{clip}%
\pgfsetbuttcap%
\pgfsetroundjoin%
\pgfsetlinewidth{1.003750pt}%
\definecolor{currentstroke}{rgb}{0.000000,0.000000,1.000000}%
\pgfsetstrokecolor{currentstroke}%
\pgfsetdash{}{0pt}%
\pgfpathmoveto{\pgfqpoint{2.371429in}{2.486519in}}%
\pgfpathlineto{\pgfqpoint{2.371429in}{4.035488in}}%
\pgfusepath{stroke}%
\end{pgfscope}%
\begin{pgfscope}%
\pgfpathrectangle{\pgfqpoint{1.000000in}{0.720000in}}{\pgfqpoint{4.800000in}{4.620000in}}%
\pgfusepath{clip}%
\pgfsetbuttcap%
\pgfsetroundjoin%
\pgfsetlinewidth{1.003750pt}%
\definecolor{currentstroke}{rgb}{0.000000,0.000000,1.000000}%
\pgfsetstrokecolor{currentstroke}%
\pgfsetdash{}{0pt}%
\pgfpathmoveto{\pgfqpoint{2.714286in}{2.520965in}}%
\pgfpathlineto{\pgfqpoint{2.714286in}{3.965745in}}%
\pgfusepath{stroke}%
\end{pgfscope}%
\begin{pgfscope}%
\pgfpathrectangle{\pgfqpoint{1.000000in}{0.720000in}}{\pgfqpoint{4.800000in}{4.620000in}}%
\pgfusepath{clip}%
\pgfsetbuttcap%
\pgfsetroundjoin%
\pgfsetlinewidth{1.003750pt}%
\definecolor{currentstroke}{rgb}{0.000000,0.000000,1.000000}%
\pgfsetstrokecolor{currentstroke}%
\pgfsetdash{}{0pt}%
\pgfpathmoveto{\pgfqpoint{3.057143in}{2.590286in}}%
\pgfpathlineto{\pgfqpoint{3.057143in}{3.871905in}}%
\pgfusepath{stroke}%
\end{pgfscope}%
\begin{pgfscope}%
\pgfpathrectangle{\pgfqpoint{1.000000in}{0.720000in}}{\pgfqpoint{4.800000in}{4.620000in}}%
\pgfusepath{clip}%
\pgfsetbuttcap%
\pgfsetroundjoin%
\pgfsetlinewidth{1.003750pt}%
\definecolor{currentstroke}{rgb}{0.000000,0.000000,1.000000}%
\pgfsetstrokecolor{currentstroke}%
\pgfsetdash{}{0pt}%
\pgfpathmoveto{\pgfqpoint{3.400000in}{2.581981in}}%
\pgfpathlineto{\pgfqpoint{3.400000in}{3.772836in}}%
\pgfusepath{stroke}%
\end{pgfscope}%
\begin{pgfscope}%
\pgfpathrectangle{\pgfqpoint{1.000000in}{0.720000in}}{\pgfqpoint{4.800000in}{4.620000in}}%
\pgfusepath{clip}%
\pgfsetbuttcap%
\pgfsetroundjoin%
\pgfsetlinewidth{1.003750pt}%
\definecolor{currentstroke}{rgb}{0.000000,0.000000,1.000000}%
\pgfsetstrokecolor{currentstroke}%
\pgfsetdash{}{0pt}%
\pgfpathmoveto{\pgfqpoint{3.742857in}{2.671545in}}%
\pgfpathlineto{\pgfqpoint{3.742857in}{3.742335in}}%
\pgfusepath{stroke}%
\end{pgfscope}%
\begin{pgfscope}%
\pgfpathrectangle{\pgfqpoint{1.000000in}{0.720000in}}{\pgfqpoint{4.800000in}{4.620000in}}%
\pgfusepath{clip}%
\pgfsetbuttcap%
\pgfsetroundjoin%
\pgfsetlinewidth{1.003750pt}%
\definecolor{currentstroke}{rgb}{0.000000,0.000000,1.000000}%
\pgfsetstrokecolor{currentstroke}%
\pgfsetdash{}{0pt}%
\pgfpathmoveto{\pgfqpoint{4.085714in}{2.513838in}}%
\pgfpathlineto{\pgfqpoint{4.085714in}{3.560995in}}%
\pgfusepath{stroke}%
\end{pgfscope}%
\begin{pgfscope}%
\pgfpathrectangle{\pgfqpoint{1.000000in}{0.720000in}}{\pgfqpoint{4.800000in}{4.620000in}}%
\pgfusepath{clip}%
\pgfsetbuttcap%
\pgfsetroundjoin%
\pgfsetlinewidth{1.003750pt}%
\definecolor{currentstroke}{rgb}{0.000000,0.000000,1.000000}%
\pgfsetstrokecolor{currentstroke}%
\pgfsetdash{}{0pt}%
\pgfpathmoveto{\pgfqpoint{4.428571in}{2.516011in}}%
\pgfpathlineto{\pgfqpoint{4.428571in}{3.316363in}}%
\pgfusepath{stroke}%
\end{pgfscope}%
\begin{pgfscope}%
\pgfpathrectangle{\pgfqpoint{1.000000in}{0.720000in}}{\pgfqpoint{4.800000in}{4.620000in}}%
\pgfusepath{clip}%
\pgfsetbuttcap%
\pgfsetroundjoin%
\pgfsetlinewidth{1.003750pt}%
\definecolor{currentstroke}{rgb}{0.000000,0.000000,1.000000}%
\pgfsetstrokecolor{currentstroke}%
\pgfsetdash{}{0pt}%
\pgfpathmoveto{\pgfqpoint{4.771429in}{2.616119in}}%
\pgfpathlineto{\pgfqpoint{4.771429in}{3.459067in}}%
\pgfusepath{stroke}%
\end{pgfscope}%
\begin{pgfscope}%
\pgfpathrectangle{\pgfqpoint{1.000000in}{0.720000in}}{\pgfqpoint{4.800000in}{4.620000in}}%
\pgfusepath{clip}%
\pgfsetbuttcap%
\pgfsetroundjoin%
\pgfsetlinewidth{1.003750pt}%
\definecolor{currentstroke}{rgb}{0.000000,0.000000,1.000000}%
\pgfsetstrokecolor{currentstroke}%
\pgfsetdash{}{0pt}%
\pgfpathmoveto{\pgfqpoint{5.114286in}{2.840190in}}%
\pgfpathlineto{\pgfqpoint{5.114286in}{3.082018in}}%
\pgfusepath{stroke}%
\end{pgfscope}%
\begin{pgfscope}%
\pgfpathrectangle{\pgfqpoint{1.000000in}{0.720000in}}{\pgfqpoint{4.800000in}{4.620000in}}%
\pgfusepath{clip}%
\pgfsetbuttcap%
\pgfsetroundjoin%
\pgfsetlinewidth{1.003750pt}%
\definecolor{currentstroke}{rgb}{0.000000,0.000000,1.000000}%
\pgfsetstrokecolor{currentstroke}%
\pgfsetdash{}{0pt}%
\pgfpathmoveto{\pgfqpoint{5.457143in}{3.291601in}}%
\pgfpathlineto{\pgfqpoint{5.457143in}{3.291601in}}%
\pgfusepath{stroke}%
\end{pgfscope}%
\begin{pgfscope}%
\pgfpathrectangle{\pgfqpoint{1.000000in}{0.720000in}}{\pgfqpoint{4.800000in}{4.620000in}}%
\pgfusepath{clip}%
\pgfsetbuttcap%
\pgfsetroundjoin%
\definecolor{currentfill}{rgb}{0.000000,0.000000,1.000000}%
\pgfsetfillcolor{currentfill}%
\pgfsetlinewidth{1.003750pt}%
\definecolor{currentstroke}{rgb}{0.000000,0.000000,1.000000}%
\pgfsetstrokecolor{currentstroke}%
\pgfsetdash{}{0pt}%
\pgfsys@defobject{currentmarker}{\pgfqpoint{-0.041667in}{-0.000000in}}{\pgfqpoint{0.041667in}{0.000000in}}{%
\pgfpathmoveto{\pgfqpoint{0.041667in}{-0.000000in}}%
\pgfpathlineto{\pgfqpoint{-0.041667in}{0.000000in}}%
\pgfusepath{stroke,fill}%
}%
\begin{pgfscope}%
\pgfsys@transformshift{1.342857in}{1.746620in}%
\pgfsys@useobject{currentmarker}{}%
\end{pgfscope}%
\begin{pgfscope}%
\pgfsys@transformshift{1.685714in}{2.204821in}%
\pgfsys@useobject{currentmarker}{}%
\end{pgfscope}%
\begin{pgfscope}%
\pgfsys@transformshift{2.028571in}{2.437215in}%
\pgfsys@useobject{currentmarker}{}%
\end{pgfscope}%
\begin{pgfscope}%
\pgfsys@transformshift{2.371429in}{2.486519in}%
\pgfsys@useobject{currentmarker}{}%
\end{pgfscope}%
\begin{pgfscope}%
\pgfsys@transformshift{2.714286in}{2.520965in}%
\pgfsys@useobject{currentmarker}{}%
\end{pgfscope}%
\begin{pgfscope}%
\pgfsys@transformshift{3.057143in}{2.590286in}%
\pgfsys@useobject{currentmarker}{}%
\end{pgfscope}%
\begin{pgfscope}%
\pgfsys@transformshift{3.400000in}{2.581981in}%
\pgfsys@useobject{currentmarker}{}%
\end{pgfscope}%
\begin{pgfscope}%
\pgfsys@transformshift{3.742857in}{2.671545in}%
\pgfsys@useobject{currentmarker}{}%
\end{pgfscope}%
\begin{pgfscope}%
\pgfsys@transformshift{4.085714in}{2.513838in}%
\pgfsys@useobject{currentmarker}{}%
\end{pgfscope}%
\begin{pgfscope}%
\pgfsys@transformshift{4.428571in}{2.516011in}%
\pgfsys@useobject{currentmarker}{}%
\end{pgfscope}%
\begin{pgfscope}%
\pgfsys@transformshift{4.771429in}{2.616119in}%
\pgfsys@useobject{currentmarker}{}%
\end{pgfscope}%
\begin{pgfscope}%
\pgfsys@transformshift{5.114286in}{2.840190in}%
\pgfsys@useobject{currentmarker}{}%
\end{pgfscope}%
\begin{pgfscope}%
\pgfsys@transformshift{5.457143in}{3.291601in}%
\pgfsys@useobject{currentmarker}{}%
\end{pgfscope}%
\end{pgfscope}%
\begin{pgfscope}%
\pgfpathrectangle{\pgfqpoint{1.000000in}{0.720000in}}{\pgfqpoint{4.800000in}{4.620000in}}%
\pgfusepath{clip}%
\pgfsetbuttcap%
\pgfsetroundjoin%
\definecolor{currentfill}{rgb}{0.000000,0.000000,1.000000}%
\pgfsetfillcolor{currentfill}%
\pgfsetlinewidth{1.003750pt}%
\definecolor{currentstroke}{rgb}{0.000000,0.000000,1.000000}%
\pgfsetstrokecolor{currentstroke}%
\pgfsetdash{}{0pt}%
\pgfsys@defobject{currentmarker}{\pgfqpoint{-0.041667in}{-0.000000in}}{\pgfqpoint{0.041667in}{0.000000in}}{%
\pgfpathmoveto{\pgfqpoint{0.041667in}{-0.000000in}}%
\pgfpathlineto{\pgfqpoint{-0.041667in}{0.000000in}}%
\pgfusepath{stroke,fill}%
}%
\begin{pgfscope}%
\pgfsys@transformshift{1.342857in}{3.918517in}%
\pgfsys@useobject{currentmarker}{}%
\end{pgfscope}%
\begin{pgfscope}%
\pgfsys@transformshift{1.685714in}{4.241646in}%
\pgfsys@useobject{currentmarker}{}%
\end{pgfscope}%
\begin{pgfscope}%
\pgfsys@transformshift{2.028571in}{4.129949in}%
\pgfsys@useobject{currentmarker}{}%
\end{pgfscope}%
\begin{pgfscope}%
\pgfsys@transformshift{2.371429in}{4.035488in}%
\pgfsys@useobject{currentmarker}{}%
\end{pgfscope}%
\begin{pgfscope}%
\pgfsys@transformshift{2.714286in}{3.965745in}%
\pgfsys@useobject{currentmarker}{}%
\end{pgfscope}%
\begin{pgfscope}%
\pgfsys@transformshift{3.057143in}{3.871905in}%
\pgfsys@useobject{currentmarker}{}%
\end{pgfscope}%
\begin{pgfscope}%
\pgfsys@transformshift{3.400000in}{3.772836in}%
\pgfsys@useobject{currentmarker}{}%
\end{pgfscope}%
\begin{pgfscope}%
\pgfsys@transformshift{3.742857in}{3.742335in}%
\pgfsys@useobject{currentmarker}{}%
\end{pgfscope}%
\begin{pgfscope}%
\pgfsys@transformshift{4.085714in}{3.560995in}%
\pgfsys@useobject{currentmarker}{}%
\end{pgfscope}%
\begin{pgfscope}%
\pgfsys@transformshift{4.428571in}{3.316363in}%
\pgfsys@useobject{currentmarker}{}%
\end{pgfscope}%
\begin{pgfscope}%
\pgfsys@transformshift{4.771429in}{3.459067in}%
\pgfsys@useobject{currentmarker}{}%
\end{pgfscope}%
\begin{pgfscope}%
\pgfsys@transformshift{5.114286in}{3.082018in}%
\pgfsys@useobject{currentmarker}{}%
\end{pgfscope}%
\begin{pgfscope}%
\pgfsys@transformshift{5.457143in}{3.291601in}%
\pgfsys@useobject{currentmarker}{}%
\end{pgfscope}%
\end{pgfscope}%
\begin{pgfscope}%
\pgfpathrectangle{\pgfqpoint{1.000000in}{0.720000in}}{\pgfqpoint{4.800000in}{4.620000in}}%
\pgfusepath{clip}%
\pgfsetbuttcap%
\pgfsetroundjoin%
\definecolor{currentfill}{rgb}{0.000000,0.000000,1.000000}%
\pgfsetfillcolor{currentfill}%
\pgfsetlinewidth{1.003750pt}%
\definecolor{currentstroke}{rgb}{0.000000,0.000000,1.000000}%
\pgfsetstrokecolor{currentstroke}%
\pgfsetdash{}{0pt}%
\pgfsys@defobject{currentmarker}{\pgfqpoint{-0.027778in}{-0.027778in}}{\pgfqpoint{0.027778in}{0.027778in}}{%
\pgfpathmoveto{\pgfqpoint{0.000000in}{-0.027778in}}%
\pgfpathcurveto{\pgfqpoint{0.007367in}{-0.027778in}}{\pgfqpoint{0.014433in}{-0.024851in}}{\pgfqpoint{0.019642in}{-0.019642in}}%
\pgfpathcurveto{\pgfqpoint{0.024851in}{-0.014433in}}{\pgfqpoint{0.027778in}{-0.007367in}}{\pgfqpoint{0.027778in}{0.000000in}}%
\pgfpathcurveto{\pgfqpoint{0.027778in}{0.007367in}}{\pgfqpoint{0.024851in}{0.014433in}}{\pgfqpoint{0.019642in}{0.019642in}}%
\pgfpathcurveto{\pgfqpoint{0.014433in}{0.024851in}}{\pgfqpoint{0.007367in}{0.027778in}}{\pgfqpoint{0.000000in}{0.027778in}}%
\pgfpathcurveto{\pgfqpoint{-0.007367in}{0.027778in}}{\pgfqpoint{-0.014433in}{0.024851in}}{\pgfqpoint{-0.019642in}{0.019642in}}%
\pgfpathcurveto{\pgfqpoint{-0.024851in}{0.014433in}}{\pgfqpoint{-0.027778in}{0.007367in}}{\pgfqpoint{-0.027778in}{0.000000in}}%
\pgfpathcurveto{\pgfqpoint{-0.027778in}{-0.007367in}}{\pgfqpoint{-0.024851in}{-0.014433in}}{\pgfqpoint{-0.019642in}{-0.019642in}}%
\pgfpathcurveto{\pgfqpoint{-0.014433in}{-0.024851in}}{\pgfqpoint{-0.007367in}{-0.027778in}}{\pgfqpoint{0.000000in}{-0.027778in}}%
\pgfpathlineto{\pgfqpoint{0.000000in}{-0.027778in}}%
\pgfpathclose%
\pgfusepath{stroke,fill}%
}%
\begin{pgfscope}%
\pgfsys@transformshift{1.342857in}{2.534844in}%
\pgfsys@useobject{currentmarker}{}%
\end{pgfscope}%
\begin{pgfscope}%
\pgfsys@transformshift{1.685714in}{2.999071in}%
\pgfsys@useobject{currentmarker}{}%
\end{pgfscope}%
\begin{pgfscope}%
\pgfsys@transformshift{2.028571in}{3.133551in}%
\pgfsys@useobject{currentmarker}{}%
\end{pgfscope}%
\begin{pgfscope}%
\pgfsys@transformshift{2.371429in}{3.128694in}%
\pgfsys@useobject{currentmarker}{}%
\end{pgfscope}%
\begin{pgfscope}%
\pgfsys@transformshift{2.714286in}{3.118514in}%
\pgfsys@useobject{currentmarker}{}%
\end{pgfscope}%
\begin{pgfscope}%
\pgfsys@transformshift{3.057143in}{3.132624in}%
\pgfsys@useobject{currentmarker}{}%
\end{pgfscope}%
\begin{pgfscope}%
\pgfsys@transformshift{3.400000in}{3.122070in}%
\pgfsys@useobject{currentmarker}{}%
\end{pgfscope}%
\begin{pgfscope}%
\pgfsys@transformshift{3.742857in}{3.089822in}%
\pgfsys@useobject{currentmarker}{}%
\end{pgfscope}%
\begin{pgfscope}%
\pgfsys@transformshift{4.085714in}{3.014745in}%
\pgfsys@useobject{currentmarker}{}%
\end{pgfscope}%
\begin{pgfscope}%
\pgfsys@transformshift{4.428571in}{2.885237in}%
\pgfsys@useobject{currentmarker}{}%
\end{pgfscope}%
\begin{pgfscope}%
\pgfsys@transformshift{4.771429in}{2.934256in}%
\pgfsys@useobject{currentmarker}{}%
\end{pgfscope}%
\begin{pgfscope}%
\pgfsys@transformshift{5.114286in}{2.968502in}%
\pgfsys@useobject{currentmarker}{}%
\end{pgfscope}%
\begin{pgfscope}%
\pgfsys@transformshift{5.457143in}{3.291601in}%
\pgfsys@useobject{currentmarker}{}%
\end{pgfscope}%
\end{pgfscope}%
\begin{pgfscope}%
\pgfsetrectcap%
\pgfsetmiterjoin%
\pgfsetlinewidth{0.803000pt}%
\definecolor{currentstroke}{rgb}{0.000000,0.000000,0.000000}%
\pgfsetstrokecolor{currentstroke}%
\pgfsetdash{}{0pt}%
\pgfpathmoveto{\pgfqpoint{1.000000in}{0.720000in}}%
\pgfpathlineto{\pgfqpoint{1.000000in}{5.340000in}}%
\pgfusepath{stroke}%
\end{pgfscope}%
\begin{pgfscope}%
\pgfsetrectcap%
\pgfsetmiterjoin%
\pgfsetlinewidth{0.803000pt}%
\definecolor{currentstroke}{rgb}{0.000000,0.000000,0.000000}%
\pgfsetstrokecolor{currentstroke}%
\pgfsetdash{}{0pt}%
\pgfpathmoveto{\pgfqpoint{5.800000in}{0.720000in}}%
\pgfpathlineto{\pgfqpoint{5.800000in}{5.340000in}}%
\pgfusepath{stroke}%
\end{pgfscope}%
\begin{pgfscope}%
\pgfsetrectcap%
\pgfsetmiterjoin%
\pgfsetlinewidth{0.803000pt}%
\definecolor{currentstroke}{rgb}{0.000000,0.000000,0.000000}%
\pgfsetstrokecolor{currentstroke}%
\pgfsetdash{}{0pt}%
\pgfpathmoveto{\pgfqpoint{1.000000in}{0.720000in}}%
\pgfpathlineto{\pgfqpoint{5.800000in}{0.720000in}}%
\pgfusepath{stroke}%
\end{pgfscope}%
\begin{pgfscope}%
\pgfsetrectcap%
\pgfsetmiterjoin%
\pgfsetlinewidth{0.803000pt}%
\definecolor{currentstroke}{rgb}{0.000000,0.000000,0.000000}%
\pgfsetstrokecolor{currentstroke}%
\pgfsetdash{}{0pt}%
\pgfpathmoveto{\pgfqpoint{1.000000in}{5.340000in}}%
\pgfpathlineto{\pgfqpoint{5.800000in}{5.340000in}}%
\pgfusepath{stroke}%
\end{pgfscope}%
\begin{pgfscope}%
\pgfsetbuttcap%
\pgfsetmiterjoin%
\definecolor{currentfill}{rgb}{1.000000,1.000000,1.000000}%
\pgfsetfillcolor{currentfill}%
\pgfsetfillopacity{0.800000}%
\pgfsetlinewidth{1.003750pt}%
\definecolor{currentstroke}{rgb}{0.800000,0.800000,0.800000}%
\pgfsetstrokecolor{currentstroke}%
\pgfsetstrokeopacity{0.800000}%
\pgfsetdash{}{0pt}%
\pgfpathmoveto{\pgfqpoint{4.331386in}{4.722821in}}%
\pgfpathlineto{\pgfqpoint{5.605556in}{4.722821in}}%
\pgfpathquadraticcurveto{\pgfqpoint{5.661111in}{4.722821in}}{\pgfqpoint{5.661111in}{4.778377in}}%
\pgfpathlineto{\pgfqpoint{5.661111in}{5.145556in}}%
\pgfpathquadraticcurveto{\pgfqpoint{5.661111in}{5.201111in}}{\pgfqpoint{5.605556in}{5.201111in}}%
\pgfpathlineto{\pgfqpoint{4.331386in}{5.201111in}}%
\pgfpathquadraticcurveto{\pgfqpoint{4.275830in}{5.201111in}}{\pgfqpoint{4.275830in}{5.145556in}}%
\pgfpathlineto{\pgfqpoint{4.275830in}{4.778377in}}%
\pgfpathquadraticcurveto{\pgfqpoint{4.275830in}{4.722821in}}{\pgfqpoint{4.331386in}{4.722821in}}%
\pgfpathlineto{\pgfqpoint{4.331386in}{4.722821in}}%
\pgfpathclose%
\pgfusepath{stroke,fill}%
\end{pgfscope}%
\begin{pgfscope}%
\pgfsetbuttcap%
\pgfsetroundjoin%
\pgfsetlinewidth{1.003750pt}%
\definecolor{currentstroke}{rgb}{0.000000,0.000000,1.000000}%
\pgfsetstrokecolor{currentstroke}%
\pgfsetdash{}{0pt}%
\pgfpathmoveto{\pgfqpoint{4.664719in}{4.848295in}}%
\pgfpathlineto{\pgfqpoint{4.664719in}{5.126073in}}%
\pgfusepath{stroke}%
\end{pgfscope}%
\begin{pgfscope}%
\pgfsetbuttcap%
\pgfsetroundjoin%
\definecolor{currentfill}{rgb}{0.000000,0.000000,1.000000}%
\pgfsetfillcolor{currentfill}%
\pgfsetlinewidth{1.003750pt}%
\definecolor{currentstroke}{rgb}{0.000000,0.000000,1.000000}%
\pgfsetstrokecolor{currentstroke}%
\pgfsetdash{}{0pt}%
\pgfsys@defobject{currentmarker}{\pgfqpoint{-0.041667in}{-0.000000in}}{\pgfqpoint{0.041667in}{0.000000in}}{%
\pgfpathmoveto{\pgfqpoint{0.041667in}{-0.000000in}}%
\pgfpathlineto{\pgfqpoint{-0.041667in}{0.000000in}}%
\pgfusepath{stroke,fill}%
}%
\begin{pgfscope}%
\pgfsys@transformshift{4.664719in}{4.848295in}%
\pgfsys@useobject{currentmarker}{}%
\end{pgfscope}%
\end{pgfscope}%
\begin{pgfscope}%
\pgfsetbuttcap%
\pgfsetroundjoin%
\definecolor{currentfill}{rgb}{0.000000,0.000000,1.000000}%
\pgfsetfillcolor{currentfill}%
\pgfsetlinewidth{1.003750pt}%
\definecolor{currentstroke}{rgb}{0.000000,0.000000,1.000000}%
\pgfsetstrokecolor{currentstroke}%
\pgfsetdash{}{0pt}%
\pgfsys@defobject{currentmarker}{\pgfqpoint{-0.041667in}{-0.000000in}}{\pgfqpoint{0.041667in}{0.000000in}}{%
\pgfpathmoveto{\pgfqpoint{0.041667in}{-0.000000in}}%
\pgfpathlineto{\pgfqpoint{-0.041667in}{0.000000in}}%
\pgfusepath{stroke,fill}%
}%
\begin{pgfscope}%
\pgfsys@transformshift{4.664719in}{5.126073in}%
\pgfsys@useobject{currentmarker}{}%
\end{pgfscope}%
\end{pgfscope}%
\begin{pgfscope}%
\pgfsetbuttcap%
\pgfsetroundjoin%
\definecolor{currentfill}{rgb}{0.000000,0.000000,1.000000}%
\pgfsetfillcolor{currentfill}%
\pgfsetlinewidth{1.003750pt}%
\definecolor{currentstroke}{rgb}{0.000000,0.000000,1.000000}%
\pgfsetstrokecolor{currentstroke}%
\pgfsetdash{}{0pt}%
\pgfsys@defobject{currentmarker}{\pgfqpoint{-0.027778in}{-0.027778in}}{\pgfqpoint{0.027778in}{0.027778in}}{%
\pgfpathmoveto{\pgfqpoint{0.000000in}{-0.027778in}}%
\pgfpathcurveto{\pgfqpoint{0.007367in}{-0.027778in}}{\pgfqpoint{0.014433in}{-0.024851in}}{\pgfqpoint{0.019642in}{-0.019642in}}%
\pgfpathcurveto{\pgfqpoint{0.024851in}{-0.014433in}}{\pgfqpoint{0.027778in}{-0.007367in}}{\pgfqpoint{0.027778in}{0.000000in}}%
\pgfpathcurveto{\pgfqpoint{0.027778in}{0.007367in}}{\pgfqpoint{0.024851in}{0.014433in}}{\pgfqpoint{0.019642in}{0.019642in}}%
\pgfpathcurveto{\pgfqpoint{0.014433in}{0.024851in}}{\pgfqpoint{0.007367in}{0.027778in}}{\pgfqpoint{0.000000in}{0.027778in}}%
\pgfpathcurveto{\pgfqpoint{-0.007367in}{0.027778in}}{\pgfqpoint{-0.014433in}{0.024851in}}{\pgfqpoint{-0.019642in}{0.019642in}}%
\pgfpathcurveto{\pgfqpoint{-0.024851in}{0.014433in}}{\pgfqpoint{-0.027778in}{0.007367in}}{\pgfqpoint{-0.027778in}{0.000000in}}%
\pgfpathcurveto{\pgfqpoint{-0.027778in}{-0.007367in}}{\pgfqpoint{-0.024851in}{-0.014433in}}{\pgfqpoint{-0.019642in}{-0.019642in}}%
\pgfpathcurveto{\pgfqpoint{-0.014433in}{-0.024851in}}{\pgfqpoint{-0.007367in}{-0.027778in}}{\pgfqpoint{0.000000in}{-0.027778in}}%
\pgfpathlineto{\pgfqpoint{0.000000in}{-0.027778in}}%
\pgfpathclose%
\pgfusepath{stroke,fill}%
}%
\begin{pgfscope}%
\pgfsys@transformshift{4.664719in}{4.987184in}%
\pgfsys@useobject{currentmarker}{}%
\end{pgfscope}%
\end{pgfscope}%
\begin{pgfscope}%
\definecolor{textcolor}{rgb}{0.000000,0.000000,0.000000}%
\pgfsetstrokecolor{textcolor}%
\pgfsetfillcolor{textcolor}%
\pgftext[x=5.164719in,y=4.889962in,left,base]{\color{textcolor}\sffamily\fontsize{20.000000}{24.000000}\selectfont \(\displaystyle D_\mathrm{w}\)}%
\end{pgfscope}%
\begin{pgfscope}%
\pgfsetbuttcap%
\pgfsetmiterjoin%
\definecolor{currentfill}{rgb}{1.000000,1.000000,1.000000}%
\pgfsetfillcolor{currentfill}%
\pgfsetlinewidth{0.000000pt}%
\definecolor{currentstroke}{rgb}{0.000000,0.000000,0.000000}%
\pgfsetstrokecolor{currentstroke}%
\pgfsetstrokeopacity{0.000000}%
\pgfsetdash{}{0pt}%
\pgfpathmoveto{\pgfqpoint{5.800000in}{0.720000in}}%
\pgfpathlineto{\pgfqpoint{7.200000in}{0.720000in}}%
\pgfpathlineto{\pgfqpoint{7.200000in}{5.340000in}}%
\pgfpathlineto{\pgfqpoint{5.800000in}{5.340000in}}%
\pgfpathlineto{\pgfqpoint{5.800000in}{0.720000in}}%
\pgfpathclose%
\pgfusepath{fill}%
\end{pgfscope}%
\begin{pgfscope}%
\pgfpathrectangle{\pgfqpoint{5.800000in}{0.720000in}}{\pgfqpoint{1.400000in}{4.620000in}}%
\pgfusepath{clip}%
\pgfsetbuttcap%
\pgfsetmiterjoin%
\definecolor{currentfill}{rgb}{0.121569,0.466667,0.705882}%
\pgfsetfillcolor{currentfill}%
\pgfsetlinewidth{0.000000pt}%
\definecolor{currentstroke}{rgb}{0.000000,0.000000,0.000000}%
\pgfsetstrokecolor{currentstroke}%
\pgfsetstrokeopacity{0.000000}%
\pgfsetdash{}{0pt}%
\pgfpathmoveto{\pgfqpoint{5.800000in}{0.720000in}}%
\pgfpathlineto{\pgfqpoint{5.805934in}{0.720000in}}%
\pgfpathlineto{\pgfqpoint{5.805934in}{0.835500in}}%
\pgfpathlineto{\pgfqpoint{5.800000in}{0.835500in}}%
\pgfpathlineto{\pgfqpoint{5.800000in}{0.720000in}}%
\pgfpathclose%
\pgfusepath{fill}%
\end{pgfscope}%
\begin{pgfscope}%
\pgfpathrectangle{\pgfqpoint{5.800000in}{0.720000in}}{\pgfqpoint{1.400000in}{4.620000in}}%
\pgfusepath{clip}%
\pgfsetbuttcap%
\pgfsetmiterjoin%
\definecolor{currentfill}{rgb}{0.121569,0.466667,0.705882}%
\pgfsetfillcolor{currentfill}%
\pgfsetlinewidth{0.000000pt}%
\definecolor{currentstroke}{rgb}{0.000000,0.000000,0.000000}%
\pgfsetstrokecolor{currentstroke}%
\pgfsetstrokeopacity{0.000000}%
\pgfsetdash{}{0pt}%
\pgfpathmoveto{\pgfqpoint{5.800000in}{0.835500in}}%
\pgfpathlineto{\pgfqpoint{5.811868in}{0.835500in}}%
\pgfpathlineto{\pgfqpoint{5.811868in}{0.951000in}}%
\pgfpathlineto{\pgfqpoint{5.800000in}{0.951000in}}%
\pgfpathlineto{\pgfqpoint{5.800000in}{0.835500in}}%
\pgfpathclose%
\pgfusepath{fill}%
\end{pgfscope}%
\begin{pgfscope}%
\pgfpathrectangle{\pgfqpoint{5.800000in}{0.720000in}}{\pgfqpoint{1.400000in}{4.620000in}}%
\pgfusepath{clip}%
\pgfsetbuttcap%
\pgfsetmiterjoin%
\definecolor{currentfill}{rgb}{0.121569,0.466667,0.705882}%
\pgfsetfillcolor{currentfill}%
\pgfsetlinewidth{0.000000pt}%
\definecolor{currentstroke}{rgb}{0.000000,0.000000,0.000000}%
\pgfsetstrokecolor{currentstroke}%
\pgfsetstrokeopacity{0.000000}%
\pgfsetdash{}{0pt}%
\pgfpathmoveto{\pgfqpoint{5.800000in}{0.951000in}}%
\pgfpathlineto{\pgfqpoint{5.831647in}{0.951000in}}%
\pgfpathlineto{\pgfqpoint{5.831647in}{1.066500in}}%
\pgfpathlineto{\pgfqpoint{5.800000in}{1.066500in}}%
\pgfpathlineto{\pgfqpoint{5.800000in}{0.951000in}}%
\pgfpathclose%
\pgfusepath{fill}%
\end{pgfscope}%
\begin{pgfscope}%
\pgfpathrectangle{\pgfqpoint{5.800000in}{0.720000in}}{\pgfqpoint{1.400000in}{4.620000in}}%
\pgfusepath{clip}%
\pgfsetbuttcap%
\pgfsetmiterjoin%
\definecolor{currentfill}{rgb}{0.121569,0.466667,0.705882}%
\pgfsetfillcolor{currentfill}%
\pgfsetlinewidth{0.000000pt}%
\definecolor{currentstroke}{rgb}{0.000000,0.000000,0.000000}%
\pgfsetstrokecolor{currentstroke}%
\pgfsetstrokeopacity{0.000000}%
\pgfsetdash{}{0pt}%
\pgfpathmoveto{\pgfqpoint{5.800000in}{1.066500in}}%
\pgfpathlineto{\pgfqpoint{5.827691in}{1.066500in}}%
\pgfpathlineto{\pgfqpoint{5.827691in}{1.182000in}}%
\pgfpathlineto{\pgfqpoint{5.800000in}{1.182000in}}%
\pgfpathlineto{\pgfqpoint{5.800000in}{1.066500in}}%
\pgfpathclose%
\pgfusepath{fill}%
\end{pgfscope}%
\begin{pgfscope}%
\pgfpathrectangle{\pgfqpoint{5.800000in}{0.720000in}}{\pgfqpoint{1.400000in}{4.620000in}}%
\pgfusepath{clip}%
\pgfsetbuttcap%
\pgfsetmiterjoin%
\definecolor{currentfill}{rgb}{0.121569,0.466667,0.705882}%
\pgfsetfillcolor{currentfill}%
\pgfsetlinewidth{0.000000pt}%
\definecolor{currentstroke}{rgb}{0.000000,0.000000,0.000000}%
\pgfsetstrokecolor{currentstroke}%
\pgfsetstrokeopacity{0.000000}%
\pgfsetdash{}{0pt}%
\pgfpathmoveto{\pgfqpoint{5.800000in}{1.182000in}}%
\pgfpathlineto{\pgfqpoint{5.833625in}{1.182000in}}%
\pgfpathlineto{\pgfqpoint{5.833625in}{1.297500in}}%
\pgfpathlineto{\pgfqpoint{5.800000in}{1.297500in}}%
\pgfpathlineto{\pgfqpoint{5.800000in}{1.182000in}}%
\pgfpathclose%
\pgfusepath{fill}%
\end{pgfscope}%
\begin{pgfscope}%
\pgfpathrectangle{\pgfqpoint{5.800000in}{0.720000in}}{\pgfqpoint{1.400000in}{4.620000in}}%
\pgfusepath{clip}%
\pgfsetbuttcap%
\pgfsetmiterjoin%
\definecolor{currentfill}{rgb}{0.121569,0.466667,0.705882}%
\pgfsetfillcolor{currentfill}%
\pgfsetlinewidth{0.000000pt}%
\definecolor{currentstroke}{rgb}{0.000000,0.000000,0.000000}%
\pgfsetstrokecolor{currentstroke}%
\pgfsetstrokeopacity{0.000000}%
\pgfsetdash{}{0pt}%
\pgfpathmoveto{\pgfqpoint{5.800000in}{1.297500in}}%
\pgfpathlineto{\pgfqpoint{5.857360in}{1.297500in}}%
\pgfpathlineto{\pgfqpoint{5.857360in}{1.413000in}}%
\pgfpathlineto{\pgfqpoint{5.800000in}{1.413000in}}%
\pgfpathlineto{\pgfqpoint{5.800000in}{1.297500in}}%
\pgfpathclose%
\pgfusepath{fill}%
\end{pgfscope}%
\begin{pgfscope}%
\pgfpathrectangle{\pgfqpoint{5.800000in}{0.720000in}}{\pgfqpoint{1.400000in}{4.620000in}}%
\pgfusepath{clip}%
\pgfsetbuttcap%
\pgfsetmiterjoin%
\definecolor{currentfill}{rgb}{0.121569,0.466667,0.705882}%
\pgfsetfillcolor{currentfill}%
\pgfsetlinewidth{0.000000pt}%
\definecolor{currentstroke}{rgb}{0.000000,0.000000,0.000000}%
\pgfsetstrokecolor{currentstroke}%
\pgfsetstrokeopacity{0.000000}%
\pgfsetdash{}{0pt}%
\pgfpathmoveto{\pgfqpoint{5.800000in}{1.413000in}}%
\pgfpathlineto{\pgfqpoint{5.871206in}{1.413000in}}%
\pgfpathlineto{\pgfqpoint{5.871206in}{1.528500in}}%
\pgfpathlineto{\pgfqpoint{5.800000in}{1.528500in}}%
\pgfpathlineto{\pgfqpoint{5.800000in}{1.413000in}}%
\pgfpathclose%
\pgfusepath{fill}%
\end{pgfscope}%
\begin{pgfscope}%
\pgfpathrectangle{\pgfqpoint{5.800000in}{0.720000in}}{\pgfqpoint{1.400000in}{4.620000in}}%
\pgfusepath{clip}%
\pgfsetbuttcap%
\pgfsetmiterjoin%
\definecolor{currentfill}{rgb}{0.121569,0.466667,0.705882}%
\pgfsetfillcolor{currentfill}%
\pgfsetlinewidth{0.000000pt}%
\definecolor{currentstroke}{rgb}{0.000000,0.000000,0.000000}%
\pgfsetstrokecolor{currentstroke}%
\pgfsetstrokeopacity{0.000000}%
\pgfsetdash{}{0pt}%
\pgfpathmoveto{\pgfqpoint{5.800000in}{1.528500in}}%
\pgfpathlineto{\pgfqpoint{5.898897in}{1.528500in}}%
\pgfpathlineto{\pgfqpoint{5.898897in}{1.644000in}}%
\pgfpathlineto{\pgfqpoint{5.800000in}{1.644000in}}%
\pgfpathlineto{\pgfqpoint{5.800000in}{1.528500in}}%
\pgfpathclose%
\pgfusepath{fill}%
\end{pgfscope}%
\begin{pgfscope}%
\pgfpathrectangle{\pgfqpoint{5.800000in}{0.720000in}}{\pgfqpoint{1.400000in}{4.620000in}}%
\pgfusepath{clip}%
\pgfsetbuttcap%
\pgfsetmiterjoin%
\definecolor{currentfill}{rgb}{0.121569,0.466667,0.705882}%
\pgfsetfillcolor{currentfill}%
\pgfsetlinewidth{0.000000pt}%
\definecolor{currentstroke}{rgb}{0.000000,0.000000,0.000000}%
\pgfsetstrokecolor{currentstroke}%
\pgfsetstrokeopacity{0.000000}%
\pgfsetdash{}{0pt}%
\pgfpathmoveto{\pgfqpoint{5.800000in}{1.644000in}}%
\pgfpathlineto{\pgfqpoint{5.942412in}{1.644000in}}%
\pgfpathlineto{\pgfqpoint{5.942412in}{1.759500in}}%
\pgfpathlineto{\pgfqpoint{5.800000in}{1.759500in}}%
\pgfpathlineto{\pgfqpoint{5.800000in}{1.644000in}}%
\pgfpathclose%
\pgfusepath{fill}%
\end{pgfscope}%
\begin{pgfscope}%
\pgfpathrectangle{\pgfqpoint{5.800000in}{0.720000in}}{\pgfqpoint{1.400000in}{4.620000in}}%
\pgfusepath{clip}%
\pgfsetbuttcap%
\pgfsetmiterjoin%
\definecolor{currentfill}{rgb}{0.121569,0.466667,0.705882}%
\pgfsetfillcolor{currentfill}%
\pgfsetlinewidth{0.000000pt}%
\definecolor{currentstroke}{rgb}{0.000000,0.000000,0.000000}%
\pgfsetstrokecolor{currentstroke}%
\pgfsetstrokeopacity{0.000000}%
\pgfsetdash{}{0pt}%
\pgfpathmoveto{\pgfqpoint{5.800000in}{1.759500in}}%
\pgfpathlineto{\pgfqpoint{5.981971in}{1.759500in}}%
\pgfpathlineto{\pgfqpoint{5.981971in}{1.875000in}}%
\pgfpathlineto{\pgfqpoint{5.800000in}{1.875000in}}%
\pgfpathlineto{\pgfqpoint{5.800000in}{1.759500in}}%
\pgfpathclose%
\pgfusepath{fill}%
\end{pgfscope}%
\begin{pgfscope}%
\pgfpathrectangle{\pgfqpoint{5.800000in}{0.720000in}}{\pgfqpoint{1.400000in}{4.620000in}}%
\pgfusepath{clip}%
\pgfsetbuttcap%
\pgfsetmiterjoin%
\definecolor{currentfill}{rgb}{0.121569,0.466667,0.705882}%
\pgfsetfillcolor{currentfill}%
\pgfsetlinewidth{0.000000pt}%
\definecolor{currentstroke}{rgb}{0.000000,0.000000,0.000000}%
\pgfsetstrokecolor{currentstroke}%
\pgfsetstrokeopacity{0.000000}%
\pgfsetdash{}{0pt}%
\pgfpathmoveto{\pgfqpoint{5.800000in}{1.875000in}}%
\pgfpathlineto{\pgfqpoint{6.090758in}{1.875000in}}%
\pgfpathlineto{\pgfqpoint{6.090758in}{1.990500in}}%
\pgfpathlineto{\pgfqpoint{5.800000in}{1.990500in}}%
\pgfpathlineto{\pgfqpoint{5.800000in}{1.875000in}}%
\pgfpathclose%
\pgfusepath{fill}%
\end{pgfscope}%
\begin{pgfscope}%
\pgfpathrectangle{\pgfqpoint{5.800000in}{0.720000in}}{\pgfqpoint{1.400000in}{4.620000in}}%
\pgfusepath{clip}%
\pgfsetbuttcap%
\pgfsetmiterjoin%
\definecolor{currentfill}{rgb}{0.121569,0.466667,0.705882}%
\pgfsetfillcolor{currentfill}%
\pgfsetlinewidth{0.000000pt}%
\definecolor{currentstroke}{rgb}{0.000000,0.000000,0.000000}%
\pgfsetstrokecolor{currentstroke}%
\pgfsetstrokeopacity{0.000000}%
\pgfsetdash{}{0pt}%
\pgfpathmoveto{\pgfqpoint{5.800000in}{1.990500in}}%
\pgfpathlineto{\pgfqpoint{6.248994in}{1.990500in}}%
\pgfpathlineto{\pgfqpoint{6.248994in}{2.106000in}}%
\pgfpathlineto{\pgfqpoint{5.800000in}{2.106000in}}%
\pgfpathlineto{\pgfqpoint{5.800000in}{1.990500in}}%
\pgfpathclose%
\pgfusepath{fill}%
\end{pgfscope}%
\begin{pgfscope}%
\pgfpathrectangle{\pgfqpoint{5.800000in}{0.720000in}}{\pgfqpoint{1.400000in}{4.620000in}}%
\pgfusepath{clip}%
\pgfsetbuttcap%
\pgfsetmiterjoin%
\definecolor{currentfill}{rgb}{0.121569,0.466667,0.705882}%
\pgfsetfillcolor{currentfill}%
\pgfsetlinewidth{0.000000pt}%
\definecolor{currentstroke}{rgb}{0.000000,0.000000,0.000000}%
\pgfsetstrokecolor{currentstroke}%
\pgfsetstrokeopacity{0.000000}%
\pgfsetdash{}{0pt}%
\pgfpathmoveto{\pgfqpoint{5.800000in}{2.106000in}}%
\pgfpathlineto{\pgfqpoint{6.369648in}{2.106000in}}%
\pgfpathlineto{\pgfqpoint{6.369648in}{2.221500in}}%
\pgfpathlineto{\pgfqpoint{5.800000in}{2.221500in}}%
\pgfpathlineto{\pgfqpoint{5.800000in}{2.106000in}}%
\pgfpathclose%
\pgfusepath{fill}%
\end{pgfscope}%
\begin{pgfscope}%
\pgfpathrectangle{\pgfqpoint{5.800000in}{0.720000in}}{\pgfqpoint{1.400000in}{4.620000in}}%
\pgfusepath{clip}%
\pgfsetbuttcap%
\pgfsetmiterjoin%
\definecolor{currentfill}{rgb}{0.121569,0.466667,0.705882}%
\pgfsetfillcolor{currentfill}%
\pgfsetlinewidth{0.000000pt}%
\definecolor{currentstroke}{rgb}{0.000000,0.000000,0.000000}%
\pgfsetstrokecolor{currentstroke}%
\pgfsetstrokeopacity{0.000000}%
\pgfsetdash{}{0pt}%
\pgfpathmoveto{\pgfqpoint{5.800000in}{2.221500in}}%
\pgfpathlineto{\pgfqpoint{6.494259in}{2.221500in}}%
\pgfpathlineto{\pgfqpoint{6.494259in}{2.337000in}}%
\pgfpathlineto{\pgfqpoint{5.800000in}{2.337000in}}%
\pgfpathlineto{\pgfqpoint{5.800000in}{2.221500in}}%
\pgfpathclose%
\pgfusepath{fill}%
\end{pgfscope}%
\begin{pgfscope}%
\pgfpathrectangle{\pgfqpoint{5.800000in}{0.720000in}}{\pgfqpoint{1.400000in}{4.620000in}}%
\pgfusepath{clip}%
\pgfsetbuttcap%
\pgfsetmiterjoin%
\definecolor{currentfill}{rgb}{0.121569,0.466667,0.705882}%
\pgfsetfillcolor{currentfill}%
\pgfsetlinewidth{0.000000pt}%
\definecolor{currentstroke}{rgb}{0.000000,0.000000,0.000000}%
\pgfsetstrokecolor{currentstroke}%
\pgfsetstrokeopacity{0.000000}%
\pgfsetdash{}{0pt}%
\pgfpathmoveto{\pgfqpoint{5.800000in}{2.337000in}}%
\pgfpathlineto{\pgfqpoint{6.607002in}{2.337000in}}%
\pgfpathlineto{\pgfqpoint{6.607002in}{2.452500in}}%
\pgfpathlineto{\pgfqpoint{5.800000in}{2.452500in}}%
\pgfpathlineto{\pgfqpoint{5.800000in}{2.337000in}}%
\pgfpathclose%
\pgfusepath{fill}%
\end{pgfscope}%
\begin{pgfscope}%
\pgfpathrectangle{\pgfqpoint{5.800000in}{0.720000in}}{\pgfqpoint{1.400000in}{4.620000in}}%
\pgfusepath{clip}%
\pgfsetbuttcap%
\pgfsetmiterjoin%
\definecolor{currentfill}{rgb}{0.121569,0.466667,0.705882}%
\pgfsetfillcolor{currentfill}%
\pgfsetlinewidth{0.000000pt}%
\definecolor{currentstroke}{rgb}{0.000000,0.000000,0.000000}%
\pgfsetstrokecolor{currentstroke}%
\pgfsetstrokeopacity{0.000000}%
\pgfsetdash{}{0pt}%
\pgfpathmoveto{\pgfqpoint{5.800000in}{2.452500in}}%
\pgfpathlineto{\pgfqpoint{6.840400in}{2.452500in}}%
\pgfpathlineto{\pgfqpoint{6.840400in}{2.568000in}}%
\pgfpathlineto{\pgfqpoint{5.800000in}{2.568000in}}%
\pgfpathlineto{\pgfqpoint{5.800000in}{2.452500in}}%
\pgfpathclose%
\pgfusepath{fill}%
\end{pgfscope}%
\begin{pgfscope}%
\pgfpathrectangle{\pgfqpoint{5.800000in}{0.720000in}}{\pgfqpoint{1.400000in}{4.620000in}}%
\pgfusepath{clip}%
\pgfsetbuttcap%
\pgfsetmiterjoin%
\definecolor{currentfill}{rgb}{0.121569,0.466667,0.705882}%
\pgfsetfillcolor{currentfill}%
\pgfsetlinewidth{0.000000pt}%
\definecolor{currentstroke}{rgb}{0.000000,0.000000,0.000000}%
\pgfsetstrokecolor{currentstroke}%
\pgfsetstrokeopacity{0.000000}%
\pgfsetdash{}{0pt}%
\pgfpathmoveto{\pgfqpoint{5.800000in}{2.568000in}}%
\pgfpathlineto{\pgfqpoint{6.945231in}{2.568000in}}%
\pgfpathlineto{\pgfqpoint{6.945231in}{2.683500in}}%
\pgfpathlineto{\pgfqpoint{5.800000in}{2.683500in}}%
\pgfpathlineto{\pgfqpoint{5.800000in}{2.568000in}}%
\pgfpathclose%
\pgfusepath{fill}%
\end{pgfscope}%
\begin{pgfscope}%
\pgfpathrectangle{\pgfqpoint{5.800000in}{0.720000in}}{\pgfqpoint{1.400000in}{4.620000in}}%
\pgfusepath{clip}%
\pgfsetbuttcap%
\pgfsetmiterjoin%
\definecolor{currentfill}{rgb}{0.121569,0.466667,0.705882}%
\pgfsetfillcolor{currentfill}%
\pgfsetlinewidth{0.000000pt}%
\definecolor{currentstroke}{rgb}{0.000000,0.000000,0.000000}%
\pgfsetstrokecolor{currentstroke}%
\pgfsetstrokeopacity{0.000000}%
\pgfsetdash{}{0pt}%
\pgfpathmoveto{\pgfqpoint{5.800000in}{2.683500in}}%
\pgfpathlineto{\pgfqpoint{7.016437in}{2.683500in}}%
\pgfpathlineto{\pgfqpoint{7.016437in}{2.799000in}}%
\pgfpathlineto{\pgfqpoint{5.800000in}{2.799000in}}%
\pgfpathlineto{\pgfqpoint{5.800000in}{2.683500in}}%
\pgfpathclose%
\pgfusepath{fill}%
\end{pgfscope}%
\begin{pgfscope}%
\pgfpathrectangle{\pgfqpoint{5.800000in}{0.720000in}}{\pgfqpoint{1.400000in}{4.620000in}}%
\pgfusepath{clip}%
\pgfsetbuttcap%
\pgfsetmiterjoin%
\definecolor{currentfill}{rgb}{0.121569,0.466667,0.705882}%
\pgfsetfillcolor{currentfill}%
\pgfsetlinewidth{0.000000pt}%
\definecolor{currentstroke}{rgb}{0.000000,0.000000,0.000000}%
\pgfsetstrokecolor{currentstroke}%
\pgfsetstrokeopacity{0.000000}%
\pgfsetdash{}{0pt}%
\pgfpathmoveto{\pgfqpoint{5.800000in}{2.799000in}}%
\pgfpathlineto{\pgfqpoint{7.069841in}{2.799000in}}%
\pgfpathlineto{\pgfqpoint{7.069841in}{2.914500in}}%
\pgfpathlineto{\pgfqpoint{5.800000in}{2.914500in}}%
\pgfpathlineto{\pgfqpoint{5.800000in}{2.799000in}}%
\pgfpathclose%
\pgfusepath{fill}%
\end{pgfscope}%
\begin{pgfscope}%
\pgfpathrectangle{\pgfqpoint{5.800000in}{0.720000in}}{\pgfqpoint{1.400000in}{4.620000in}}%
\pgfusepath{clip}%
\pgfsetbuttcap%
\pgfsetmiterjoin%
\definecolor{currentfill}{rgb}{0.121569,0.466667,0.705882}%
\pgfsetfillcolor{currentfill}%
\pgfsetlinewidth{0.000000pt}%
\definecolor{currentstroke}{rgb}{0.000000,0.000000,0.000000}%
\pgfsetstrokecolor{currentstroke}%
\pgfsetstrokeopacity{0.000000}%
\pgfsetdash{}{0pt}%
\pgfpathmoveto{\pgfqpoint{5.800000in}{2.914500in}}%
\pgfpathlineto{\pgfqpoint{7.014459in}{2.914500in}}%
\pgfpathlineto{\pgfqpoint{7.014459in}{3.030000in}}%
\pgfpathlineto{\pgfqpoint{5.800000in}{3.030000in}}%
\pgfpathlineto{\pgfqpoint{5.800000in}{2.914500in}}%
\pgfpathclose%
\pgfusepath{fill}%
\end{pgfscope}%
\begin{pgfscope}%
\pgfpathrectangle{\pgfqpoint{5.800000in}{0.720000in}}{\pgfqpoint{1.400000in}{4.620000in}}%
\pgfusepath{clip}%
\pgfsetbuttcap%
\pgfsetmiterjoin%
\definecolor{currentfill}{rgb}{0.121569,0.466667,0.705882}%
\pgfsetfillcolor{currentfill}%
\pgfsetlinewidth{0.000000pt}%
\definecolor{currentstroke}{rgb}{0.000000,0.000000,0.000000}%
\pgfsetstrokecolor{currentstroke}%
\pgfsetstrokeopacity{0.000000}%
\pgfsetdash{}{0pt}%
\pgfpathmoveto{\pgfqpoint{5.800000in}{3.030000in}}%
\pgfpathlineto{\pgfqpoint{7.050062in}{3.030000in}}%
\pgfpathlineto{\pgfqpoint{7.050062in}{3.145500in}}%
\pgfpathlineto{\pgfqpoint{5.800000in}{3.145500in}}%
\pgfpathlineto{\pgfqpoint{5.800000in}{3.030000in}}%
\pgfpathclose%
\pgfusepath{fill}%
\end{pgfscope}%
\begin{pgfscope}%
\pgfpathrectangle{\pgfqpoint{5.800000in}{0.720000in}}{\pgfqpoint{1.400000in}{4.620000in}}%
\pgfusepath{clip}%
\pgfsetbuttcap%
\pgfsetmiterjoin%
\definecolor{currentfill}{rgb}{0.121569,0.466667,0.705882}%
\pgfsetfillcolor{currentfill}%
\pgfsetlinewidth{0.000000pt}%
\definecolor{currentstroke}{rgb}{0.000000,0.000000,0.000000}%
\pgfsetstrokecolor{currentstroke}%
\pgfsetstrokeopacity{0.000000}%
\pgfsetdash{}{0pt}%
\pgfpathmoveto{\pgfqpoint{5.800000in}{3.145500in}}%
\pgfpathlineto{\pgfqpoint{6.943253in}{3.145500in}}%
\pgfpathlineto{\pgfqpoint{6.943253in}{3.261000in}}%
\pgfpathlineto{\pgfqpoint{5.800000in}{3.261000in}}%
\pgfpathlineto{\pgfqpoint{5.800000in}{3.145500in}}%
\pgfpathclose%
\pgfusepath{fill}%
\end{pgfscope}%
\begin{pgfscope}%
\pgfpathrectangle{\pgfqpoint{5.800000in}{0.720000in}}{\pgfqpoint{1.400000in}{4.620000in}}%
\pgfusepath{clip}%
\pgfsetbuttcap%
\pgfsetmiterjoin%
\definecolor{currentfill}{rgb}{0.121569,0.466667,0.705882}%
\pgfsetfillcolor{currentfill}%
\pgfsetlinewidth{0.000000pt}%
\definecolor{currentstroke}{rgb}{0.000000,0.000000,0.000000}%
\pgfsetstrokecolor{currentstroke}%
\pgfsetstrokeopacity{0.000000}%
\pgfsetdash{}{0pt}%
\pgfpathmoveto{\pgfqpoint{5.800000in}{3.261000in}}%
\pgfpathlineto{\pgfqpoint{6.895782in}{3.261000in}}%
\pgfpathlineto{\pgfqpoint{6.895782in}{3.376500in}}%
\pgfpathlineto{\pgfqpoint{5.800000in}{3.376500in}}%
\pgfpathlineto{\pgfqpoint{5.800000in}{3.261000in}}%
\pgfpathclose%
\pgfusepath{fill}%
\end{pgfscope}%
\begin{pgfscope}%
\pgfpathrectangle{\pgfqpoint{5.800000in}{0.720000in}}{\pgfqpoint{1.400000in}{4.620000in}}%
\pgfusepath{clip}%
\pgfsetbuttcap%
\pgfsetmiterjoin%
\definecolor{currentfill}{rgb}{0.121569,0.466667,0.705882}%
\pgfsetfillcolor{currentfill}%
\pgfsetlinewidth{0.000000pt}%
\definecolor{currentstroke}{rgb}{0.000000,0.000000,0.000000}%
\pgfsetstrokecolor{currentstroke}%
\pgfsetstrokeopacity{0.000000}%
\pgfsetdash{}{0pt}%
\pgfpathmoveto{\pgfqpoint{5.800000in}{3.376500in}}%
\pgfpathlineto{\pgfqpoint{6.781061in}{3.376500in}}%
\pgfpathlineto{\pgfqpoint{6.781061in}{3.492000in}}%
\pgfpathlineto{\pgfqpoint{5.800000in}{3.492000in}}%
\pgfpathlineto{\pgfqpoint{5.800000in}{3.376500in}}%
\pgfpathclose%
\pgfusepath{fill}%
\end{pgfscope}%
\begin{pgfscope}%
\pgfpathrectangle{\pgfqpoint{5.800000in}{0.720000in}}{\pgfqpoint{1.400000in}{4.620000in}}%
\pgfusepath{clip}%
\pgfsetbuttcap%
\pgfsetmiterjoin%
\definecolor{currentfill}{rgb}{0.121569,0.466667,0.705882}%
\pgfsetfillcolor{currentfill}%
\pgfsetlinewidth{0.000000pt}%
\definecolor{currentstroke}{rgb}{0.000000,0.000000,0.000000}%
\pgfsetstrokecolor{currentstroke}%
\pgfsetstrokeopacity{0.000000}%
\pgfsetdash{}{0pt}%
\pgfpathmoveto{\pgfqpoint{5.800000in}{3.492000in}}%
\pgfpathlineto{\pgfqpoint{6.620848in}{3.492000in}}%
\pgfpathlineto{\pgfqpoint{6.620848in}{3.607500in}}%
\pgfpathlineto{\pgfqpoint{5.800000in}{3.607500in}}%
\pgfpathlineto{\pgfqpoint{5.800000in}{3.492000in}}%
\pgfpathclose%
\pgfusepath{fill}%
\end{pgfscope}%
\begin{pgfscope}%
\pgfpathrectangle{\pgfqpoint{5.800000in}{0.720000in}}{\pgfqpoint{1.400000in}{4.620000in}}%
\pgfusepath{clip}%
\pgfsetbuttcap%
\pgfsetmiterjoin%
\definecolor{currentfill}{rgb}{0.121569,0.466667,0.705882}%
\pgfsetfillcolor{currentfill}%
\pgfsetlinewidth{0.000000pt}%
\definecolor{currentstroke}{rgb}{0.000000,0.000000,0.000000}%
\pgfsetstrokecolor{currentstroke}%
\pgfsetstrokeopacity{0.000000}%
\pgfsetdash{}{0pt}%
\pgfpathmoveto{\pgfqpoint{5.800000in}{3.607500in}}%
\pgfpathlineto{\pgfqpoint{6.480413in}{3.607500in}}%
\pgfpathlineto{\pgfqpoint{6.480413in}{3.723000in}}%
\pgfpathlineto{\pgfqpoint{5.800000in}{3.723000in}}%
\pgfpathlineto{\pgfqpoint{5.800000in}{3.607500in}}%
\pgfpathclose%
\pgfusepath{fill}%
\end{pgfscope}%
\begin{pgfscope}%
\pgfpathrectangle{\pgfqpoint{5.800000in}{0.720000in}}{\pgfqpoint{1.400000in}{4.620000in}}%
\pgfusepath{clip}%
\pgfsetbuttcap%
\pgfsetmiterjoin%
\definecolor{currentfill}{rgb}{0.121569,0.466667,0.705882}%
\pgfsetfillcolor{currentfill}%
\pgfsetlinewidth{0.000000pt}%
\definecolor{currentstroke}{rgb}{0.000000,0.000000,0.000000}%
\pgfsetstrokecolor{currentstroke}%
\pgfsetstrokeopacity{0.000000}%
\pgfsetdash{}{0pt}%
\pgfpathmoveto{\pgfqpoint{5.800000in}{3.723000in}}%
\pgfpathlineto{\pgfqpoint{6.385472in}{3.723000in}}%
\pgfpathlineto{\pgfqpoint{6.385472in}{3.838500in}}%
\pgfpathlineto{\pgfqpoint{5.800000in}{3.838500in}}%
\pgfpathlineto{\pgfqpoint{5.800000in}{3.723000in}}%
\pgfpathclose%
\pgfusepath{fill}%
\end{pgfscope}%
\begin{pgfscope}%
\pgfpathrectangle{\pgfqpoint{5.800000in}{0.720000in}}{\pgfqpoint{1.400000in}{4.620000in}}%
\pgfusepath{clip}%
\pgfsetbuttcap%
\pgfsetmiterjoin%
\definecolor{currentfill}{rgb}{0.121569,0.466667,0.705882}%
\pgfsetfillcolor{currentfill}%
\pgfsetlinewidth{0.000000pt}%
\definecolor{currentstroke}{rgb}{0.000000,0.000000,0.000000}%
\pgfsetstrokecolor{currentstroke}%
\pgfsetstrokeopacity{0.000000}%
\pgfsetdash{}{0pt}%
\pgfpathmoveto{\pgfqpoint{5.800000in}{3.838500in}}%
\pgfpathlineto{\pgfqpoint{6.288553in}{3.838500in}}%
\pgfpathlineto{\pgfqpoint{6.288553in}{3.954000in}}%
\pgfpathlineto{\pgfqpoint{5.800000in}{3.954000in}}%
\pgfpathlineto{\pgfqpoint{5.800000in}{3.838500in}}%
\pgfpathclose%
\pgfusepath{fill}%
\end{pgfscope}%
\begin{pgfscope}%
\pgfpathrectangle{\pgfqpoint{5.800000in}{0.720000in}}{\pgfqpoint{1.400000in}{4.620000in}}%
\pgfusepath{clip}%
\pgfsetbuttcap%
\pgfsetmiterjoin%
\definecolor{currentfill}{rgb}{0.121569,0.466667,0.705882}%
\pgfsetfillcolor{currentfill}%
\pgfsetlinewidth{0.000000pt}%
\definecolor{currentstroke}{rgb}{0.000000,0.000000,0.000000}%
\pgfsetstrokecolor{currentstroke}%
\pgfsetstrokeopacity{0.000000}%
\pgfsetdash{}{0pt}%
\pgfpathmoveto{\pgfqpoint{5.800000in}{3.954000in}}%
\pgfpathlineto{\pgfqpoint{6.241082in}{3.954000in}}%
\pgfpathlineto{\pgfqpoint{6.241082in}{4.069500in}}%
\pgfpathlineto{\pgfqpoint{5.800000in}{4.069500in}}%
\pgfpathlineto{\pgfqpoint{5.800000in}{3.954000in}}%
\pgfpathclose%
\pgfusepath{fill}%
\end{pgfscope}%
\begin{pgfscope}%
\pgfpathrectangle{\pgfqpoint{5.800000in}{0.720000in}}{\pgfqpoint{1.400000in}{4.620000in}}%
\pgfusepath{clip}%
\pgfsetbuttcap%
\pgfsetmiterjoin%
\definecolor{currentfill}{rgb}{0.121569,0.466667,0.705882}%
\pgfsetfillcolor{currentfill}%
\pgfsetlinewidth{0.000000pt}%
\definecolor{currentstroke}{rgb}{0.000000,0.000000,0.000000}%
\pgfsetstrokecolor{currentstroke}%
\pgfsetstrokeopacity{0.000000}%
\pgfsetdash{}{0pt}%
\pgfpathmoveto{\pgfqpoint{5.800000in}{4.069500in}}%
\pgfpathlineto{\pgfqpoint{6.183722in}{4.069500in}}%
\pgfpathlineto{\pgfqpoint{6.183722in}{4.185000in}}%
\pgfpathlineto{\pgfqpoint{5.800000in}{4.185000in}}%
\pgfpathlineto{\pgfqpoint{5.800000in}{4.069500in}}%
\pgfpathclose%
\pgfusepath{fill}%
\end{pgfscope}%
\begin{pgfscope}%
\pgfpathrectangle{\pgfqpoint{5.800000in}{0.720000in}}{\pgfqpoint{1.400000in}{4.620000in}}%
\pgfusepath{clip}%
\pgfsetbuttcap%
\pgfsetmiterjoin%
\definecolor{currentfill}{rgb}{0.121569,0.466667,0.705882}%
\pgfsetfillcolor{currentfill}%
\pgfsetlinewidth{0.000000pt}%
\definecolor{currentstroke}{rgb}{0.000000,0.000000,0.000000}%
\pgfsetstrokecolor{currentstroke}%
\pgfsetstrokeopacity{0.000000}%
\pgfsetdash{}{0pt}%
\pgfpathmoveto{\pgfqpoint{5.800000in}{4.185000in}}%
\pgfpathlineto{\pgfqpoint{6.181744in}{4.185000in}}%
\pgfpathlineto{\pgfqpoint{6.181744in}{4.300500in}}%
\pgfpathlineto{\pgfqpoint{5.800000in}{4.300500in}}%
\pgfpathlineto{\pgfqpoint{5.800000in}{4.185000in}}%
\pgfpathclose%
\pgfusepath{fill}%
\end{pgfscope}%
\begin{pgfscope}%
\pgfpathrectangle{\pgfqpoint{5.800000in}{0.720000in}}{\pgfqpoint{1.400000in}{4.620000in}}%
\pgfusepath{clip}%
\pgfsetbuttcap%
\pgfsetmiterjoin%
\definecolor{currentfill}{rgb}{0.121569,0.466667,0.705882}%
\pgfsetfillcolor{currentfill}%
\pgfsetlinewidth{0.000000pt}%
\definecolor{currentstroke}{rgb}{0.000000,0.000000,0.000000}%
\pgfsetstrokecolor{currentstroke}%
\pgfsetstrokeopacity{0.000000}%
\pgfsetdash{}{0pt}%
\pgfpathmoveto{\pgfqpoint{5.800000in}{4.300500in}}%
\pgfpathlineto{\pgfqpoint{6.001750in}{4.300500in}}%
\pgfpathlineto{\pgfqpoint{6.001750in}{4.416000in}}%
\pgfpathlineto{\pgfqpoint{5.800000in}{4.416000in}}%
\pgfpathlineto{\pgfqpoint{5.800000in}{4.300500in}}%
\pgfpathclose%
\pgfusepath{fill}%
\end{pgfscope}%
\begin{pgfscope}%
\pgfpathrectangle{\pgfqpoint{5.800000in}{0.720000in}}{\pgfqpoint{1.400000in}{4.620000in}}%
\pgfusepath{clip}%
\pgfsetbuttcap%
\pgfsetmiterjoin%
\definecolor{currentfill}{rgb}{0.121569,0.466667,0.705882}%
\pgfsetfillcolor{currentfill}%
\pgfsetlinewidth{0.000000pt}%
\definecolor{currentstroke}{rgb}{0.000000,0.000000,0.000000}%
\pgfsetstrokecolor{currentstroke}%
\pgfsetstrokeopacity{0.000000}%
\pgfsetdash{}{0pt}%
\pgfpathmoveto{\pgfqpoint{5.800000in}{4.416000in}}%
\pgfpathlineto{\pgfqpoint{6.023508in}{4.416000in}}%
\pgfpathlineto{\pgfqpoint{6.023508in}{4.531500in}}%
\pgfpathlineto{\pgfqpoint{5.800000in}{4.531500in}}%
\pgfpathlineto{\pgfqpoint{5.800000in}{4.416000in}}%
\pgfpathclose%
\pgfusepath{fill}%
\end{pgfscope}%
\begin{pgfscope}%
\pgfpathrectangle{\pgfqpoint{5.800000in}{0.720000in}}{\pgfqpoint{1.400000in}{4.620000in}}%
\pgfusepath{clip}%
\pgfsetbuttcap%
\pgfsetmiterjoin%
\definecolor{currentfill}{rgb}{0.121569,0.466667,0.705882}%
\pgfsetfillcolor{currentfill}%
\pgfsetlinewidth{0.000000pt}%
\definecolor{currentstroke}{rgb}{0.000000,0.000000,0.000000}%
\pgfsetstrokecolor{currentstroke}%
\pgfsetstrokeopacity{0.000000}%
\pgfsetdash{}{0pt}%
\pgfpathmoveto{\pgfqpoint{5.800000in}{4.531500in}}%
\pgfpathlineto{\pgfqpoint{5.972081in}{4.531500in}}%
\pgfpathlineto{\pgfqpoint{5.972081in}{4.647000in}}%
\pgfpathlineto{\pgfqpoint{5.800000in}{4.647000in}}%
\pgfpathlineto{\pgfqpoint{5.800000in}{4.531500in}}%
\pgfpathclose%
\pgfusepath{fill}%
\end{pgfscope}%
\begin{pgfscope}%
\pgfpathrectangle{\pgfqpoint{5.800000in}{0.720000in}}{\pgfqpoint{1.400000in}{4.620000in}}%
\pgfusepath{clip}%
\pgfsetbuttcap%
\pgfsetmiterjoin%
\definecolor{currentfill}{rgb}{0.121569,0.466667,0.705882}%
\pgfsetfillcolor{currentfill}%
\pgfsetlinewidth{0.000000pt}%
\definecolor{currentstroke}{rgb}{0.000000,0.000000,0.000000}%
\pgfsetstrokecolor{currentstroke}%
\pgfsetstrokeopacity{0.000000}%
\pgfsetdash{}{0pt}%
\pgfpathmoveto{\pgfqpoint{5.800000in}{4.647000in}}%
\pgfpathlineto{\pgfqpoint{5.944390in}{4.647000in}}%
\pgfpathlineto{\pgfqpoint{5.944390in}{4.762500in}}%
\pgfpathlineto{\pgfqpoint{5.800000in}{4.762500in}}%
\pgfpathlineto{\pgfqpoint{5.800000in}{4.647000in}}%
\pgfpathclose%
\pgfusepath{fill}%
\end{pgfscope}%
\begin{pgfscope}%
\pgfpathrectangle{\pgfqpoint{5.800000in}{0.720000in}}{\pgfqpoint{1.400000in}{4.620000in}}%
\pgfusepath{clip}%
\pgfsetbuttcap%
\pgfsetmiterjoin%
\definecolor{currentfill}{rgb}{0.121569,0.466667,0.705882}%
\pgfsetfillcolor{currentfill}%
\pgfsetlinewidth{0.000000pt}%
\definecolor{currentstroke}{rgb}{0.000000,0.000000,0.000000}%
\pgfsetstrokecolor{currentstroke}%
\pgfsetstrokeopacity{0.000000}%
\pgfsetdash{}{0pt}%
\pgfpathmoveto{\pgfqpoint{5.800000in}{4.762500in}}%
\pgfpathlineto{\pgfqpoint{5.928566in}{4.762500in}}%
\pgfpathlineto{\pgfqpoint{5.928566in}{4.878000in}}%
\pgfpathlineto{\pgfqpoint{5.800000in}{4.878000in}}%
\pgfpathlineto{\pgfqpoint{5.800000in}{4.762500in}}%
\pgfpathclose%
\pgfusepath{fill}%
\end{pgfscope}%
\begin{pgfscope}%
\pgfpathrectangle{\pgfqpoint{5.800000in}{0.720000in}}{\pgfqpoint{1.400000in}{4.620000in}}%
\pgfusepath{clip}%
\pgfsetbuttcap%
\pgfsetmiterjoin%
\definecolor{currentfill}{rgb}{0.121569,0.466667,0.705882}%
\pgfsetfillcolor{currentfill}%
\pgfsetlinewidth{0.000000pt}%
\definecolor{currentstroke}{rgb}{0.000000,0.000000,0.000000}%
\pgfsetstrokecolor{currentstroke}%
\pgfsetstrokeopacity{0.000000}%
\pgfsetdash{}{0pt}%
\pgfpathmoveto{\pgfqpoint{5.800000in}{4.878000in}}%
\pgfpathlineto{\pgfqpoint{5.904831in}{4.878000in}}%
\pgfpathlineto{\pgfqpoint{5.904831in}{4.993500in}}%
\pgfpathlineto{\pgfqpoint{5.800000in}{4.993500in}}%
\pgfpathlineto{\pgfqpoint{5.800000in}{4.878000in}}%
\pgfpathclose%
\pgfusepath{fill}%
\end{pgfscope}%
\begin{pgfscope}%
\pgfpathrectangle{\pgfqpoint{5.800000in}{0.720000in}}{\pgfqpoint{1.400000in}{4.620000in}}%
\pgfusepath{clip}%
\pgfsetbuttcap%
\pgfsetmiterjoin%
\definecolor{currentfill}{rgb}{0.121569,0.466667,0.705882}%
\pgfsetfillcolor{currentfill}%
\pgfsetlinewidth{0.000000pt}%
\definecolor{currentstroke}{rgb}{0.000000,0.000000,0.000000}%
\pgfsetstrokecolor{currentstroke}%
\pgfsetstrokeopacity{0.000000}%
\pgfsetdash{}{0pt}%
\pgfpathmoveto{\pgfqpoint{5.800000in}{4.993500in}}%
\pgfpathlineto{\pgfqpoint{5.894941in}{4.993500in}}%
\pgfpathlineto{\pgfqpoint{5.894941in}{5.109000in}}%
\pgfpathlineto{\pgfqpoint{5.800000in}{5.109000in}}%
\pgfpathlineto{\pgfqpoint{5.800000in}{4.993500in}}%
\pgfpathclose%
\pgfusepath{fill}%
\end{pgfscope}%
\begin{pgfscope}%
\pgfpathrectangle{\pgfqpoint{5.800000in}{0.720000in}}{\pgfqpoint{1.400000in}{4.620000in}}%
\pgfusepath{clip}%
\pgfsetbuttcap%
\pgfsetmiterjoin%
\definecolor{currentfill}{rgb}{0.121569,0.466667,0.705882}%
\pgfsetfillcolor{currentfill}%
\pgfsetlinewidth{0.000000pt}%
\definecolor{currentstroke}{rgb}{0.000000,0.000000,0.000000}%
\pgfsetstrokecolor{currentstroke}%
\pgfsetstrokeopacity{0.000000}%
\pgfsetdash{}{0pt}%
\pgfpathmoveto{\pgfqpoint{5.800000in}{5.109000in}}%
\pgfpathlineto{\pgfqpoint{5.896919in}{5.109000in}}%
\pgfpathlineto{\pgfqpoint{5.896919in}{5.224500in}}%
\pgfpathlineto{\pgfqpoint{5.800000in}{5.224500in}}%
\pgfpathlineto{\pgfqpoint{5.800000in}{5.109000in}}%
\pgfpathclose%
\pgfusepath{fill}%
\end{pgfscope}%
\begin{pgfscope}%
\pgfpathrectangle{\pgfqpoint{5.800000in}{0.720000in}}{\pgfqpoint{1.400000in}{4.620000in}}%
\pgfusepath{clip}%
\pgfsetbuttcap%
\pgfsetmiterjoin%
\definecolor{currentfill}{rgb}{0.121569,0.466667,0.705882}%
\pgfsetfillcolor{currentfill}%
\pgfsetlinewidth{0.000000pt}%
\definecolor{currentstroke}{rgb}{0.000000,0.000000,0.000000}%
\pgfsetstrokecolor{currentstroke}%
\pgfsetstrokeopacity{0.000000}%
\pgfsetdash{}{0pt}%
\pgfpathmoveto{\pgfqpoint{5.800000in}{5.224500in}}%
\pgfpathlineto{\pgfqpoint{5.865272in}{5.224500in}}%
\pgfpathlineto{\pgfqpoint{5.865272in}{5.340000in}}%
\pgfpathlineto{\pgfqpoint{5.800000in}{5.340000in}}%
\pgfpathlineto{\pgfqpoint{5.800000in}{5.224500in}}%
\pgfpathclose%
\pgfusepath{fill}%
\end{pgfscope}%
\begin{pgfscope}%
\definecolor{textcolor}{rgb}{0.000000,0.000000,0.000000}%
\pgfsetstrokecolor{textcolor}%
\pgfsetfillcolor{textcolor}%
\pgftext[x=6.500000in,y=0.664444in,,top]{\color{textcolor}\sffamily\fontsize{20.000000}{24.000000}\selectfont \(\displaystyle \mathrm{arb.\ unit}\)}%
\end{pgfscope}%
\begin{pgfscope}%
\pgfsetrectcap%
\pgfsetmiterjoin%
\pgfsetlinewidth{0.803000pt}%
\definecolor{currentstroke}{rgb}{0.000000,0.000000,0.000000}%
\pgfsetstrokecolor{currentstroke}%
\pgfsetdash{}{0pt}%
\pgfpathmoveto{\pgfqpoint{5.800000in}{0.720000in}}%
\pgfpathlineto{\pgfqpoint{5.800000in}{5.340000in}}%
\pgfusepath{stroke}%
\end{pgfscope}%
\begin{pgfscope}%
\pgfsetrectcap%
\pgfsetmiterjoin%
\pgfsetlinewidth{0.803000pt}%
\definecolor{currentstroke}{rgb}{0.000000,0.000000,0.000000}%
\pgfsetstrokecolor{currentstroke}%
\pgfsetdash{}{0pt}%
\pgfpathmoveto{\pgfqpoint{7.200000in}{0.720000in}}%
\pgfpathlineto{\pgfqpoint{7.200000in}{5.340000in}}%
\pgfusepath{stroke}%
\end{pgfscope}%
\begin{pgfscope}%
\pgfsetrectcap%
\pgfsetmiterjoin%
\pgfsetlinewidth{0.803000pt}%
\definecolor{currentstroke}{rgb}{0.000000,0.000000,0.000000}%
\pgfsetstrokecolor{currentstroke}%
\pgfsetdash{}{0pt}%
\pgfpathmoveto{\pgfqpoint{5.800000in}{0.720000in}}%
\pgfpathlineto{\pgfqpoint{7.200000in}{0.720000in}}%
\pgfusepath{stroke}%
\end{pgfscope}%
\begin{pgfscope}%
\pgfsetrectcap%
\pgfsetmiterjoin%
\pgfsetlinewidth{0.803000pt}%
\definecolor{currentstroke}{rgb}{0.000000,0.000000,0.000000}%
\pgfsetstrokecolor{currentstroke}%
\pgfsetdash{}{0pt}%
\pgfpathmoveto{\pgfqpoint{5.800000in}{5.340000in}}%
\pgfpathlineto{\pgfqpoint{7.200000in}{5.340000in}}%
\pgfusepath{stroke}%
\end{pgfscope}%
\end{pgfpicture}%
\makeatother%
\endgroup%
}
    \caption{\label{fig:fitting-npe} $D_\mathrm{w}$ histogram and its distributions conditioned \\ on $N_{\mathrm{PE}}$, errorbar explained in figure~\ref{fig:cnn-performance}.}
  \end{subfigure}
  \begin{subfigure}{.5\textwidth}
    \centering
    \resizebox{\textwidth}{!}{%% Creator: Matplotlib, PGF backend
%%
%% To include the figure in your LaTeX document, write
%%   \input{<filename>.pgf}
%%
%% Make sure the required packages are loaded in your preamble
%%   \usepackage{pgf}
%%
%% Also ensure that all the required font packages are loaded; for instance,
%% the lmodern package is sometimes necessary when using math font.
%%   \usepackage{lmodern}
%%
%% Figures using additional raster images can only be included by \input if
%% they are in the same directory as the main LaTeX file. For loading figures
%% from other directories you can use the `import` package
%%   \usepackage{import}
%%
%% and then include the figures with
%%   \import{<path to file>}{<filename>.pgf}
%%
%% Matplotlib used the following preamble
%%   \usepackage[detect-all,locale=DE]{siunitx}
%%
\begingroup%
\makeatletter%
\begin{pgfpicture}%
\pgfpathrectangle{\pgfpointorigin}{\pgfqpoint{8.000000in}{6.000000in}}%
\pgfusepath{use as bounding box, clip}%
\begin{pgfscope}%
\pgfsetbuttcap%
\pgfsetmiterjoin%
\definecolor{currentfill}{rgb}{1.000000,1.000000,1.000000}%
\pgfsetfillcolor{currentfill}%
\pgfsetlinewidth{0.000000pt}%
\definecolor{currentstroke}{rgb}{1.000000,1.000000,1.000000}%
\pgfsetstrokecolor{currentstroke}%
\pgfsetdash{}{0pt}%
\pgfpathmoveto{\pgfqpoint{0.000000in}{0.000000in}}%
\pgfpathlineto{\pgfqpoint{8.000000in}{0.000000in}}%
\pgfpathlineto{\pgfqpoint{8.000000in}{6.000000in}}%
\pgfpathlineto{\pgfqpoint{0.000000in}{6.000000in}}%
\pgfpathlineto{\pgfqpoint{0.000000in}{0.000000in}}%
\pgfpathclose%
\pgfusepath{fill}%
\end{pgfscope}%
\begin{pgfscope}%
\pgfsetbuttcap%
\pgfsetmiterjoin%
\definecolor{currentfill}{rgb}{1.000000,1.000000,1.000000}%
\pgfsetfillcolor{currentfill}%
\pgfsetlinewidth{0.000000pt}%
\definecolor{currentstroke}{rgb}{0.000000,0.000000,0.000000}%
\pgfsetstrokecolor{currentstroke}%
\pgfsetstrokeopacity{0.000000}%
\pgfsetdash{}{0pt}%
\pgfpathmoveto{\pgfqpoint{1.000000in}{0.720000in}}%
\pgfpathlineto{\pgfqpoint{7.200000in}{0.720000in}}%
\pgfpathlineto{\pgfqpoint{7.200000in}{5.340000in}}%
\pgfpathlineto{\pgfqpoint{1.000000in}{5.340000in}}%
\pgfpathlineto{\pgfqpoint{1.000000in}{0.720000in}}%
\pgfpathclose%
\pgfusepath{fill}%
\end{pgfscope}%
\begin{pgfscope}%
\pgfsetbuttcap%
\pgfsetroundjoin%
\definecolor{currentfill}{rgb}{0.000000,0.000000,0.000000}%
\pgfsetfillcolor{currentfill}%
\pgfsetlinewidth{0.803000pt}%
\definecolor{currentstroke}{rgb}{0.000000,0.000000,0.000000}%
\pgfsetstrokecolor{currentstroke}%
\pgfsetdash{}{0pt}%
\pgfsys@defobject{currentmarker}{\pgfqpoint{0.000000in}{-0.048611in}}{\pgfqpoint{0.000000in}{0.000000in}}{%
\pgfpathmoveto{\pgfqpoint{0.000000in}{0.000000in}}%
\pgfpathlineto{\pgfqpoint{0.000000in}{-0.048611in}}%
\pgfusepath{stroke,fill}%
}%
\begin{pgfscope}%
\pgfsys@transformshift{1.310000in}{0.720000in}%
\pgfsys@useobject{currentmarker}{}%
\end{pgfscope}%
\end{pgfscope}%
\begin{pgfscope}%
\definecolor{textcolor}{rgb}{0.000000,0.000000,0.000000}%
\pgfsetstrokecolor{textcolor}%
\pgfsetfillcolor{textcolor}%
\pgftext[x=1.310000in,y=0.622778in,,top]{\color{textcolor}\sffamily\fontsize{20.000000}{24.000000}\selectfont \(\displaystyle {450}\)}%
\end{pgfscope}%
\begin{pgfscope}%
\pgfsetbuttcap%
\pgfsetroundjoin%
\definecolor{currentfill}{rgb}{0.000000,0.000000,0.000000}%
\pgfsetfillcolor{currentfill}%
\pgfsetlinewidth{0.803000pt}%
\definecolor{currentstroke}{rgb}{0.000000,0.000000,0.000000}%
\pgfsetstrokecolor{currentstroke}%
\pgfsetdash{}{0pt}%
\pgfsys@defobject{currentmarker}{\pgfqpoint{0.000000in}{-0.048611in}}{\pgfqpoint{0.000000in}{0.000000in}}{%
\pgfpathmoveto{\pgfqpoint{0.000000in}{0.000000in}}%
\pgfpathlineto{\pgfqpoint{0.000000in}{-0.048611in}}%
\pgfusepath{stroke,fill}%
}%
\begin{pgfscope}%
\pgfsys@transformshift{2.860000in}{0.720000in}%
\pgfsys@useobject{currentmarker}{}%
\end{pgfscope}%
\end{pgfscope}%
\begin{pgfscope}%
\definecolor{textcolor}{rgb}{0.000000,0.000000,0.000000}%
\pgfsetstrokecolor{textcolor}%
\pgfsetfillcolor{textcolor}%
\pgftext[x=2.860000in,y=0.622778in,,top]{\color{textcolor}\sffamily\fontsize{20.000000}{24.000000}\selectfont \(\displaystyle {500}\)}%
\end{pgfscope}%
\begin{pgfscope}%
\pgfsetbuttcap%
\pgfsetroundjoin%
\definecolor{currentfill}{rgb}{0.000000,0.000000,0.000000}%
\pgfsetfillcolor{currentfill}%
\pgfsetlinewidth{0.803000pt}%
\definecolor{currentstroke}{rgb}{0.000000,0.000000,0.000000}%
\pgfsetstrokecolor{currentstroke}%
\pgfsetdash{}{0pt}%
\pgfsys@defobject{currentmarker}{\pgfqpoint{0.000000in}{-0.048611in}}{\pgfqpoint{0.000000in}{0.000000in}}{%
\pgfpathmoveto{\pgfqpoint{0.000000in}{0.000000in}}%
\pgfpathlineto{\pgfqpoint{0.000000in}{-0.048611in}}%
\pgfusepath{stroke,fill}%
}%
\begin{pgfscope}%
\pgfsys@transformshift{4.410000in}{0.720000in}%
\pgfsys@useobject{currentmarker}{}%
\end{pgfscope}%
\end{pgfscope}%
\begin{pgfscope}%
\definecolor{textcolor}{rgb}{0.000000,0.000000,0.000000}%
\pgfsetstrokecolor{textcolor}%
\pgfsetfillcolor{textcolor}%
\pgftext[x=4.410000in,y=0.622778in,,top]{\color{textcolor}\sffamily\fontsize{20.000000}{24.000000}\selectfont \(\displaystyle {550}\)}%
\end{pgfscope}%
\begin{pgfscope}%
\pgfsetbuttcap%
\pgfsetroundjoin%
\definecolor{currentfill}{rgb}{0.000000,0.000000,0.000000}%
\pgfsetfillcolor{currentfill}%
\pgfsetlinewidth{0.803000pt}%
\definecolor{currentstroke}{rgb}{0.000000,0.000000,0.000000}%
\pgfsetstrokecolor{currentstroke}%
\pgfsetdash{}{0pt}%
\pgfsys@defobject{currentmarker}{\pgfqpoint{0.000000in}{-0.048611in}}{\pgfqpoint{0.000000in}{0.000000in}}{%
\pgfpathmoveto{\pgfqpoint{0.000000in}{0.000000in}}%
\pgfpathlineto{\pgfqpoint{0.000000in}{-0.048611in}}%
\pgfusepath{stroke,fill}%
}%
\begin{pgfscope}%
\pgfsys@transformshift{5.960000in}{0.720000in}%
\pgfsys@useobject{currentmarker}{}%
\end{pgfscope}%
\end{pgfscope}%
\begin{pgfscope}%
\definecolor{textcolor}{rgb}{0.000000,0.000000,0.000000}%
\pgfsetstrokecolor{textcolor}%
\pgfsetfillcolor{textcolor}%
\pgftext[x=5.960000in,y=0.622778in,,top]{\color{textcolor}\sffamily\fontsize{20.000000}{24.000000}\selectfont \(\displaystyle {600}\)}%
\end{pgfscope}%
\begin{pgfscope}%
\definecolor{textcolor}{rgb}{0.000000,0.000000,0.000000}%
\pgfsetstrokecolor{textcolor}%
\pgfsetfillcolor{textcolor}%
\pgftext[x=4.100000in,y=0.311155in,,top]{\color{textcolor}\sffamily\fontsize{20.000000}{24.000000}\selectfont \(\displaystyle \mathrm{t}/\si{ns}\)}%
\end{pgfscope}%
\begin{pgfscope}%
\pgfsetbuttcap%
\pgfsetroundjoin%
\definecolor{currentfill}{rgb}{0.000000,0.000000,0.000000}%
\pgfsetfillcolor{currentfill}%
\pgfsetlinewidth{0.803000pt}%
\definecolor{currentstroke}{rgb}{0.000000,0.000000,0.000000}%
\pgfsetstrokecolor{currentstroke}%
\pgfsetdash{}{0pt}%
\pgfsys@defobject{currentmarker}{\pgfqpoint{-0.048611in}{0.000000in}}{\pgfqpoint{-0.000000in}{0.000000in}}{%
\pgfpathmoveto{\pgfqpoint{-0.000000in}{0.000000in}}%
\pgfpathlineto{\pgfqpoint{-0.048611in}{0.000000in}}%
\pgfusepath{stroke,fill}%
}%
\begin{pgfscope}%
\pgfsys@transformshift{1.000000in}{1.138093in}%
\pgfsys@useobject{currentmarker}{}%
\end{pgfscope}%
\end{pgfscope}%
\begin{pgfscope}%
\definecolor{textcolor}{rgb}{0.000000,0.000000,0.000000}%
\pgfsetstrokecolor{textcolor}%
\pgfsetfillcolor{textcolor}%
\pgftext[x=0.770670in, y=1.038074in, left, base]{\color{textcolor}\sffamily\fontsize{20.000000}{24.000000}\selectfont \(\displaystyle {0}\)}%
\end{pgfscope}%
\begin{pgfscope}%
\pgfsetbuttcap%
\pgfsetroundjoin%
\definecolor{currentfill}{rgb}{0.000000,0.000000,0.000000}%
\pgfsetfillcolor{currentfill}%
\pgfsetlinewidth{0.803000pt}%
\definecolor{currentstroke}{rgb}{0.000000,0.000000,0.000000}%
\pgfsetstrokecolor{currentstroke}%
\pgfsetdash{}{0pt}%
\pgfsys@defobject{currentmarker}{\pgfqpoint{-0.048611in}{0.000000in}}{\pgfqpoint{-0.000000in}{0.000000in}}{%
\pgfpathmoveto{\pgfqpoint{-0.000000in}{0.000000in}}%
\pgfpathlineto{\pgfqpoint{-0.048611in}{0.000000in}}%
\pgfusepath{stroke,fill}%
}%
\begin{pgfscope}%
\pgfsys@transformshift{1.000000in}{2.100312in}%
\pgfsys@useobject{currentmarker}{}%
\end{pgfscope}%
\end{pgfscope}%
\begin{pgfscope}%
\definecolor{textcolor}{rgb}{0.000000,0.000000,0.000000}%
\pgfsetstrokecolor{textcolor}%
\pgfsetfillcolor{textcolor}%
\pgftext[x=0.638563in, y=2.000293in, left, base]{\color{textcolor}\sffamily\fontsize{20.000000}{24.000000}\selectfont \(\displaystyle {10}\)}%
\end{pgfscope}%
\begin{pgfscope}%
\pgfsetbuttcap%
\pgfsetroundjoin%
\definecolor{currentfill}{rgb}{0.000000,0.000000,0.000000}%
\pgfsetfillcolor{currentfill}%
\pgfsetlinewidth{0.803000pt}%
\definecolor{currentstroke}{rgb}{0.000000,0.000000,0.000000}%
\pgfsetstrokecolor{currentstroke}%
\pgfsetdash{}{0pt}%
\pgfsys@defobject{currentmarker}{\pgfqpoint{-0.048611in}{0.000000in}}{\pgfqpoint{-0.000000in}{0.000000in}}{%
\pgfpathmoveto{\pgfqpoint{-0.000000in}{0.000000in}}%
\pgfpathlineto{\pgfqpoint{-0.048611in}{0.000000in}}%
\pgfusepath{stroke,fill}%
}%
\begin{pgfscope}%
\pgfsys@transformshift{1.000000in}{3.062531in}%
\pgfsys@useobject{currentmarker}{}%
\end{pgfscope}%
\end{pgfscope}%
\begin{pgfscope}%
\definecolor{textcolor}{rgb}{0.000000,0.000000,0.000000}%
\pgfsetstrokecolor{textcolor}%
\pgfsetfillcolor{textcolor}%
\pgftext[x=0.638563in, y=2.962512in, left, base]{\color{textcolor}\sffamily\fontsize{20.000000}{24.000000}\selectfont \(\displaystyle {20}\)}%
\end{pgfscope}%
\begin{pgfscope}%
\pgfsetbuttcap%
\pgfsetroundjoin%
\definecolor{currentfill}{rgb}{0.000000,0.000000,0.000000}%
\pgfsetfillcolor{currentfill}%
\pgfsetlinewidth{0.803000pt}%
\definecolor{currentstroke}{rgb}{0.000000,0.000000,0.000000}%
\pgfsetstrokecolor{currentstroke}%
\pgfsetdash{}{0pt}%
\pgfsys@defobject{currentmarker}{\pgfqpoint{-0.048611in}{0.000000in}}{\pgfqpoint{-0.000000in}{0.000000in}}{%
\pgfpathmoveto{\pgfqpoint{-0.000000in}{0.000000in}}%
\pgfpathlineto{\pgfqpoint{-0.048611in}{0.000000in}}%
\pgfusepath{stroke,fill}%
}%
\begin{pgfscope}%
\pgfsys@transformshift{1.000000in}{4.024750in}%
\pgfsys@useobject{currentmarker}{}%
\end{pgfscope}%
\end{pgfscope}%
\begin{pgfscope}%
\definecolor{textcolor}{rgb}{0.000000,0.000000,0.000000}%
\pgfsetstrokecolor{textcolor}%
\pgfsetfillcolor{textcolor}%
\pgftext[x=0.638563in, y=3.924731in, left, base]{\color{textcolor}\sffamily\fontsize{20.000000}{24.000000}\selectfont \(\displaystyle {30}\)}%
\end{pgfscope}%
\begin{pgfscope}%
\pgfsetbuttcap%
\pgfsetroundjoin%
\definecolor{currentfill}{rgb}{0.000000,0.000000,0.000000}%
\pgfsetfillcolor{currentfill}%
\pgfsetlinewidth{0.803000pt}%
\definecolor{currentstroke}{rgb}{0.000000,0.000000,0.000000}%
\pgfsetstrokecolor{currentstroke}%
\pgfsetdash{}{0pt}%
\pgfsys@defobject{currentmarker}{\pgfqpoint{-0.048611in}{0.000000in}}{\pgfqpoint{-0.000000in}{0.000000in}}{%
\pgfpathmoveto{\pgfqpoint{-0.000000in}{0.000000in}}%
\pgfpathlineto{\pgfqpoint{-0.048611in}{0.000000in}}%
\pgfusepath{stroke,fill}%
}%
\begin{pgfscope}%
\pgfsys@transformshift{1.000000in}{4.986969in}%
\pgfsys@useobject{currentmarker}{}%
\end{pgfscope}%
\end{pgfscope}%
\begin{pgfscope}%
\definecolor{textcolor}{rgb}{0.000000,0.000000,0.000000}%
\pgfsetstrokecolor{textcolor}%
\pgfsetfillcolor{textcolor}%
\pgftext[x=0.638563in, y=4.886950in, left, base]{\color{textcolor}\sffamily\fontsize{20.000000}{24.000000}\selectfont \(\displaystyle {40}\)}%
\end{pgfscope}%
\begin{pgfscope}%
\definecolor{textcolor}{rgb}{0.000000,0.000000,0.000000}%
\pgfsetstrokecolor{textcolor}%
\pgfsetfillcolor{textcolor}%
\pgftext[x=0.583007in,y=3.030000in,,bottom,rotate=90.000000]{\color{textcolor}\sffamily\fontsize{20.000000}{24.000000}\selectfont \(\displaystyle \mathrm{Voltage}/\si{mV}\)}%
\end{pgfscope}%
\begin{pgfscope}%
\pgfpathrectangle{\pgfqpoint{1.000000in}{0.720000in}}{\pgfqpoint{6.200000in}{4.620000in}}%
\pgfusepath{clip}%
\pgfsetrectcap%
\pgfsetroundjoin%
\pgfsetlinewidth{2.007500pt}%
\definecolor{currentstroke}{rgb}{0.121569,0.466667,0.705882}%
\pgfsetstrokecolor{currentstroke}%
\pgfsetdash{}{0pt}%
\pgfpathmoveto{\pgfqpoint{0.990000in}{1.198200in}}%
\pgfpathlineto{\pgfqpoint{1.000000in}{1.236022in}}%
\pgfpathlineto{\pgfqpoint{1.031000in}{1.099605in}}%
\pgfpathlineto{\pgfqpoint{1.062000in}{1.179883in}}%
\pgfpathlineto{\pgfqpoint{1.093000in}{1.166548in}}%
\pgfpathlineto{\pgfqpoint{1.124000in}{1.151460in}}%
\pgfpathlineto{\pgfqpoint{1.155000in}{1.246779in}}%
\pgfpathlineto{\pgfqpoint{1.186000in}{1.162871in}}%
\pgfpathlineto{\pgfqpoint{1.217000in}{1.095815in}}%
\pgfpathlineto{\pgfqpoint{1.248000in}{1.234109in}}%
\pgfpathlineto{\pgfqpoint{1.279000in}{1.163798in}}%
\pgfpathlineto{\pgfqpoint{1.310000in}{1.081082in}}%
\pgfpathlineto{\pgfqpoint{1.341000in}{1.185899in}}%
\pgfpathlineto{\pgfqpoint{1.372000in}{1.121694in}}%
\pgfpathlineto{\pgfqpoint{1.403000in}{1.349858in}}%
\pgfpathlineto{\pgfqpoint{1.434000in}{1.076658in}}%
\pgfpathlineto{\pgfqpoint{1.465000in}{1.015862in}}%
\pgfpathlineto{\pgfqpoint{1.496000in}{1.098724in}}%
\pgfpathlineto{\pgfqpoint{1.527000in}{1.043464in}}%
\pgfpathlineto{\pgfqpoint{1.558000in}{1.016923in}}%
\pgfpathlineto{\pgfqpoint{1.589000in}{1.129355in}}%
\pgfpathlineto{\pgfqpoint{1.620000in}{1.216799in}}%
\pgfpathlineto{\pgfqpoint{1.651000in}{1.193580in}}%
\pgfpathlineto{\pgfqpoint{1.682000in}{1.305667in}}%
\pgfpathlineto{\pgfqpoint{1.713000in}{1.220771in}}%
\pgfpathlineto{\pgfqpoint{1.744000in}{1.245455in}}%
\pgfpathlineto{\pgfqpoint{1.775000in}{1.259910in}}%
\pgfpathlineto{\pgfqpoint{1.806000in}{1.227172in}}%
\pgfpathlineto{\pgfqpoint{1.837000in}{1.119309in}}%
\pgfpathlineto{\pgfqpoint{1.868000in}{1.020815in}}%
\pgfpathlineto{\pgfqpoint{1.899000in}{1.157585in}}%
\pgfpathlineto{\pgfqpoint{1.930000in}{1.108898in}}%
\pgfpathlineto{\pgfqpoint{1.961000in}{1.222627in}}%
\pgfpathlineto{\pgfqpoint{1.992000in}{1.216911in}}%
\pgfpathlineto{\pgfqpoint{2.023000in}{1.434065in}}%
\pgfpathlineto{\pgfqpoint{2.054000in}{1.794623in}}%
\pgfpathlineto{\pgfqpoint{2.085000in}{2.211711in}}%
\pgfpathlineto{\pgfqpoint{2.116000in}{2.939053in}}%
\pgfpathlineto{\pgfqpoint{2.147000in}{3.122916in}}%
\pgfpathlineto{\pgfqpoint{2.178000in}{3.534362in}}%
\pgfpathlineto{\pgfqpoint{2.209000in}{3.664477in}}%
\pgfpathlineto{\pgfqpoint{2.240000in}{3.424525in}}%
\pgfpathlineto{\pgfqpoint{2.271000in}{3.407418in}}%
\pgfpathlineto{\pgfqpoint{2.302000in}{3.327000in}}%
\pgfpathlineto{\pgfqpoint{2.333000in}{2.824996in}}%
\pgfpathlineto{\pgfqpoint{2.364000in}{2.651454in}}%
\pgfpathlineto{\pgfqpoint{2.395000in}{2.640278in}}%
\pgfpathlineto{\pgfqpoint{2.426000in}{2.347307in}}%
\pgfpathlineto{\pgfqpoint{2.457000in}{1.964857in}}%
\pgfpathlineto{\pgfqpoint{2.488000in}{1.873927in}}%
\pgfpathlineto{\pgfqpoint{2.519000in}{1.791530in}}%
\pgfpathlineto{\pgfqpoint{2.550000in}{1.747096in}}%
\pgfpathlineto{\pgfqpoint{2.581000in}{1.966974in}}%
\pgfpathlineto{\pgfqpoint{2.612000in}{2.547045in}}%
\pgfpathlineto{\pgfqpoint{2.643000in}{2.790263in}}%
\pgfpathlineto{\pgfqpoint{2.674000in}{3.188882in}}%
\pgfpathlineto{\pgfqpoint{2.705000in}{3.279366in}}%
\pgfpathlineto{\pgfqpoint{2.736000in}{3.164397in}}%
\pgfpathlineto{\pgfqpoint{2.767000in}{2.999773in}}%
\pgfpathlineto{\pgfqpoint{2.798000in}{2.856473in}}%
\pgfpathlineto{\pgfqpoint{2.829000in}{2.689253in}}%
\pgfpathlineto{\pgfqpoint{2.860000in}{2.570875in}}%
\pgfpathlineto{\pgfqpoint{2.891000in}{2.287712in}}%
\pgfpathlineto{\pgfqpoint{2.922000in}{2.334664in}}%
\pgfpathlineto{\pgfqpoint{2.953000in}{2.132198in}}%
\pgfpathlineto{\pgfqpoint{2.984000in}{2.657852in}}%
\pgfpathlineto{\pgfqpoint{3.015000in}{2.664440in}}%
\pgfpathlineto{\pgfqpoint{3.046000in}{2.989300in}}%
\pgfpathlineto{\pgfqpoint{3.077000in}{2.859896in}}%
\pgfpathlineto{\pgfqpoint{3.108000in}{2.882247in}}%
\pgfpathlineto{\pgfqpoint{3.139000in}{2.802414in}}%
\pgfpathlineto{\pgfqpoint{3.170000in}{2.537908in}}%
\pgfpathlineto{\pgfqpoint{3.201000in}{2.423999in}}%
\pgfpathlineto{\pgfqpoint{3.232000in}{2.150211in}}%
\pgfpathlineto{\pgfqpoint{3.263000in}{2.005885in}}%
\pgfpathlineto{\pgfqpoint{3.294000in}{1.848422in}}%
\pgfpathlineto{\pgfqpoint{3.325000in}{1.628729in}}%
\pgfpathlineto{\pgfqpoint{3.356000in}{1.751142in}}%
\pgfpathlineto{\pgfqpoint{3.387000in}{1.765880in}}%
\pgfpathlineto{\pgfqpoint{3.418000in}{1.536049in}}%
\pgfpathlineto{\pgfqpoint{3.449000in}{1.333304in}}%
\pgfpathlineto{\pgfqpoint{3.480000in}{1.286616in}}%
\pgfpathlineto{\pgfqpoint{3.511000in}{1.500454in}}%
\pgfpathlineto{\pgfqpoint{3.542000in}{1.268876in}}%
\pgfpathlineto{\pgfqpoint{3.573000in}{1.246592in}}%
\pgfpathlineto{\pgfqpoint{3.604000in}{1.373978in}}%
\pgfpathlineto{\pgfqpoint{3.635000in}{1.204664in}}%
\pgfpathlineto{\pgfqpoint{3.666000in}{1.160858in}}%
\pgfpathlineto{\pgfqpoint{3.697000in}{1.229118in}}%
\pgfpathlineto{\pgfqpoint{3.728000in}{1.154302in}}%
\pgfpathlineto{\pgfqpoint{3.759000in}{1.096888in}}%
\pgfpathlineto{\pgfqpoint{3.790000in}{1.363101in}}%
\pgfpathlineto{\pgfqpoint{3.821000in}{1.461450in}}%
\pgfpathlineto{\pgfqpoint{3.852000in}{1.940584in}}%
\pgfpathlineto{\pgfqpoint{3.883000in}{2.162181in}}%
\pgfpathlineto{\pgfqpoint{3.914000in}{2.627114in}}%
\pgfpathlineto{\pgfqpoint{3.945000in}{2.586245in}}%
\pgfpathlineto{\pgfqpoint{3.976000in}{2.878269in}}%
\pgfpathlineto{\pgfqpoint{4.007000in}{2.748021in}}%
\pgfpathlineto{\pgfqpoint{4.038000in}{2.392908in}}%
\pgfpathlineto{\pgfqpoint{4.069000in}{2.391669in}}%
\pgfpathlineto{\pgfqpoint{4.100000in}{2.265457in}}%
\pgfpathlineto{\pgfqpoint{4.131000in}{2.050187in}}%
\pgfpathlineto{\pgfqpoint{4.162000in}{1.870113in}}%
\pgfpathlineto{\pgfqpoint{4.193000in}{1.807911in}}%
\pgfpathlineto{\pgfqpoint{4.224000in}{1.853158in}}%
\pgfpathlineto{\pgfqpoint{4.255000in}{1.787814in}}%
\pgfpathlineto{\pgfqpoint{4.286000in}{1.480142in}}%
\pgfpathlineto{\pgfqpoint{4.317000in}{1.448926in}}%
\pgfpathlineto{\pgfqpoint{4.348000in}{1.391990in}}%
\pgfpathlineto{\pgfqpoint{4.379000in}{1.310315in}}%
\pgfpathlineto{\pgfqpoint{4.410000in}{1.196321in}}%
\pgfpathlineto{\pgfqpoint{4.441000in}{1.371640in}}%
\pgfpathlineto{\pgfqpoint{4.472000in}{1.265609in}}%
\pgfpathlineto{\pgfqpoint{4.534000in}{1.139418in}}%
\pgfpathlineto{\pgfqpoint{4.565000in}{1.072433in}}%
\pgfpathlineto{\pgfqpoint{4.596000in}{1.292139in}}%
\pgfpathlineto{\pgfqpoint{4.627000in}{1.306758in}}%
\pgfpathlineto{\pgfqpoint{4.658000in}{1.127425in}}%
\pgfpathlineto{\pgfqpoint{4.689000in}{1.251963in}}%
\pgfpathlineto{\pgfqpoint{4.720000in}{0.986281in}}%
\pgfpathlineto{\pgfqpoint{4.751000in}{1.131557in}}%
\pgfpathlineto{\pgfqpoint{4.782000in}{1.154894in}}%
\pgfpathlineto{\pgfqpoint{4.813000in}{1.029853in}}%
\pgfpathlineto{\pgfqpoint{4.844000in}{1.088825in}}%
\pgfpathlineto{\pgfqpoint{4.875000in}{0.860213in}}%
\pgfpathlineto{\pgfqpoint{4.906000in}{1.032026in}}%
\pgfpathlineto{\pgfqpoint{4.937000in}{1.057972in}}%
\pgfpathlineto{\pgfqpoint{4.968000in}{0.957880in}}%
\pgfpathlineto{\pgfqpoint{4.999000in}{1.311552in}}%
\pgfpathlineto{\pgfqpoint{5.030000in}{1.194601in}}%
\pgfpathlineto{\pgfqpoint{5.061000in}{1.132455in}}%
\pgfpathlineto{\pgfqpoint{5.092000in}{1.213178in}}%
\pgfpathlineto{\pgfqpoint{5.123000in}{0.992327in}}%
\pgfpathlineto{\pgfqpoint{5.154000in}{1.079640in}}%
\pgfpathlineto{\pgfqpoint{5.185000in}{1.028729in}}%
\pgfpathlineto{\pgfqpoint{5.216000in}{0.953758in}}%
\pgfpathlineto{\pgfqpoint{5.247000in}{1.128676in}}%
\pgfpathlineto{\pgfqpoint{5.278000in}{1.151393in}}%
\pgfpathlineto{\pgfqpoint{5.309000in}{1.226037in}}%
\pgfpathlineto{\pgfqpoint{5.371000in}{1.116496in}}%
\pgfpathlineto{\pgfqpoint{5.402000in}{1.190882in}}%
\pgfpathlineto{\pgfqpoint{5.433000in}{1.214735in}}%
\pgfpathlineto{\pgfqpoint{5.464000in}{1.076841in}}%
\pgfpathlineto{\pgfqpoint{5.495000in}{1.047499in}}%
\pgfpathlineto{\pgfqpoint{5.526000in}{1.077993in}}%
\pgfpathlineto{\pgfqpoint{5.557000in}{1.110193in}}%
\pgfpathlineto{\pgfqpoint{5.588000in}{1.039882in}}%
\pgfpathlineto{\pgfqpoint{5.619000in}{0.942549in}}%
\pgfpathlineto{\pgfqpoint{5.650000in}{1.085781in}}%
\pgfpathlineto{\pgfqpoint{5.681000in}{1.089873in}}%
\pgfpathlineto{\pgfqpoint{5.712000in}{1.113993in}}%
\pgfpathlineto{\pgfqpoint{5.743000in}{1.157347in}}%
\pgfpathlineto{\pgfqpoint{5.774000in}{1.280928in}}%
\pgfpathlineto{\pgfqpoint{5.805000in}{1.127749in}}%
\pgfpathlineto{\pgfqpoint{5.836000in}{1.064021in}}%
\pgfpathlineto{\pgfqpoint{5.867000in}{1.091651in}}%
\pgfpathlineto{\pgfqpoint{5.898000in}{1.251217in}}%
\pgfpathlineto{\pgfqpoint{5.929000in}{1.240969in}}%
\pgfpathlineto{\pgfqpoint{5.960000in}{1.083470in}}%
\pgfpathlineto{\pgfqpoint{5.991000in}{1.051458in}}%
\pgfpathlineto{\pgfqpoint{6.022000in}{1.107580in}}%
\pgfpathlineto{\pgfqpoint{6.053000in}{1.030816in}}%
\pgfpathlineto{\pgfqpoint{6.084000in}{1.055144in}}%
\pgfpathlineto{\pgfqpoint{6.115000in}{1.118555in}}%
\pgfpathlineto{\pgfqpoint{6.146000in}{1.154678in}}%
\pgfpathlineto{\pgfqpoint{6.177000in}{1.101468in}}%
\pgfpathlineto{\pgfqpoint{6.208000in}{1.195512in}}%
\pgfpathlineto{\pgfqpoint{6.239000in}{1.047711in}}%
\pgfpathlineto{\pgfqpoint{6.270000in}{1.222265in}}%
\pgfpathlineto{\pgfqpoint{6.301000in}{1.008722in}}%
\pgfpathlineto{\pgfqpoint{6.332000in}{1.070775in}}%
\pgfpathlineto{\pgfqpoint{6.363000in}{1.090151in}}%
\pgfpathlineto{\pgfqpoint{6.394000in}{1.083919in}}%
\pgfpathlineto{\pgfqpoint{6.425000in}{1.287700in}}%
\pgfpathlineto{\pgfqpoint{6.456000in}{1.153262in}}%
\pgfpathlineto{\pgfqpoint{6.487000in}{1.236077in}}%
\pgfpathlineto{\pgfqpoint{6.518000in}{1.187929in}}%
\pgfpathlineto{\pgfqpoint{6.549000in}{1.195589in}}%
\pgfpathlineto{\pgfqpoint{6.580000in}{1.148104in}}%
\pgfpathlineto{\pgfqpoint{6.611000in}{1.055576in}}%
\pgfpathlineto{\pgfqpoint{6.642000in}{1.198618in}}%
\pgfpathlineto{\pgfqpoint{6.673000in}{1.101882in}}%
\pgfpathlineto{\pgfqpoint{6.704000in}{0.975630in}}%
\pgfpathlineto{\pgfqpoint{6.735000in}{1.135182in}}%
\pgfpathlineto{\pgfqpoint{6.766000in}{1.228368in}}%
\pgfpathlineto{\pgfqpoint{6.797000in}{1.198492in}}%
\pgfpathlineto{\pgfqpoint{6.828000in}{1.188794in}}%
\pgfpathlineto{\pgfqpoint{6.859000in}{0.966069in}}%
\pgfpathlineto{\pgfqpoint{6.890000in}{1.148773in}}%
\pgfpathlineto{\pgfqpoint{6.921000in}{0.907220in}}%
\pgfpathlineto{\pgfqpoint{6.952000in}{1.148749in}}%
\pgfpathlineto{\pgfqpoint{6.983000in}{1.018991in}}%
\pgfpathlineto{\pgfqpoint{7.014000in}{1.127595in}}%
\pgfpathlineto{\pgfqpoint{7.045000in}{1.155254in}}%
\pgfpathlineto{\pgfqpoint{7.076000in}{1.066620in}}%
\pgfpathlineto{\pgfqpoint{7.107000in}{1.054569in}}%
\pgfpathlineto{\pgfqpoint{7.138000in}{1.201487in}}%
\pgfpathlineto{\pgfqpoint{7.169000in}{1.313851in}}%
\pgfpathlineto{\pgfqpoint{7.200000in}{1.200376in}}%
\pgfpathlineto{\pgfqpoint{7.210000in}{1.133797in}}%
\pgfpathlineto{\pgfqpoint{7.210000in}{1.133797in}}%
\pgfusepath{stroke}%
\end{pgfscope}%
\begin{pgfscope}%
\pgfsetrectcap%
\pgfsetmiterjoin%
\pgfsetlinewidth{0.803000pt}%
\definecolor{currentstroke}{rgb}{0.000000,0.000000,0.000000}%
\pgfsetstrokecolor{currentstroke}%
\pgfsetdash{}{0pt}%
\pgfpathmoveto{\pgfqpoint{1.000000in}{0.720000in}}%
\pgfpathlineto{\pgfqpoint{1.000000in}{5.340000in}}%
\pgfusepath{stroke}%
\end{pgfscope}%
\begin{pgfscope}%
\pgfsetrectcap%
\pgfsetmiterjoin%
\pgfsetlinewidth{0.803000pt}%
\definecolor{currentstroke}{rgb}{0.000000,0.000000,0.000000}%
\pgfsetstrokecolor{currentstroke}%
\pgfsetdash{}{0pt}%
\pgfpathmoveto{\pgfqpoint{7.200000in}{0.720000in}}%
\pgfpathlineto{\pgfqpoint{7.200000in}{5.340000in}}%
\pgfusepath{stroke}%
\end{pgfscope}%
\begin{pgfscope}%
\pgfsetrectcap%
\pgfsetmiterjoin%
\pgfsetlinewidth{0.803000pt}%
\definecolor{currentstroke}{rgb}{0.000000,0.000000,0.000000}%
\pgfsetstrokecolor{currentstroke}%
\pgfsetdash{}{0pt}%
\pgfpathmoveto{\pgfqpoint{1.000000in}{0.720000in}}%
\pgfpathlineto{\pgfqpoint{7.200000in}{0.720000in}}%
\pgfusepath{stroke}%
\end{pgfscope}%
\begin{pgfscope}%
\pgfsetrectcap%
\pgfsetmiterjoin%
\pgfsetlinewidth{0.803000pt}%
\definecolor{currentstroke}{rgb}{0.000000,0.000000,0.000000}%
\pgfsetstrokecolor{currentstroke}%
\pgfsetdash{}{0pt}%
\pgfpathmoveto{\pgfqpoint{1.000000in}{5.340000in}}%
\pgfpathlineto{\pgfqpoint{7.200000in}{5.340000in}}%
\pgfusepath{stroke}%
\end{pgfscope}%
\begin{pgfscope}%
\pgfsetbuttcap%
\pgfsetroundjoin%
\definecolor{currentfill}{rgb}{0.000000,0.000000,0.000000}%
\pgfsetfillcolor{currentfill}%
\pgfsetlinewidth{0.803000pt}%
\definecolor{currentstroke}{rgb}{0.000000,0.000000,0.000000}%
\pgfsetstrokecolor{currentstroke}%
\pgfsetdash{}{0pt}%
\pgfsys@defobject{currentmarker}{\pgfqpoint{0.000000in}{0.000000in}}{\pgfqpoint{0.048611in}{0.000000in}}{%
\pgfpathmoveto{\pgfqpoint{0.000000in}{0.000000in}}%
\pgfpathlineto{\pgfqpoint{0.048611in}{0.000000in}}%
\pgfusepath{stroke,fill}%
}%
\begin{pgfscope}%
\pgfsys@transformshift{7.200000in}{1.138093in}%
\pgfsys@useobject{currentmarker}{}%
\end{pgfscope}%
\end{pgfscope}%
\begin{pgfscope}%
\definecolor{textcolor}{rgb}{0.000000,0.000000,0.000000}%
\pgfsetstrokecolor{textcolor}%
\pgfsetfillcolor{textcolor}%
\pgftext[x=7.297222in, y=1.038074in, left, base]{\color{textcolor}\sffamily\fontsize{20.000000}{24.000000}\selectfont 0.0}%
\end{pgfscope}%
\begin{pgfscope}%
\pgfsetbuttcap%
\pgfsetroundjoin%
\definecolor{currentfill}{rgb}{0.000000,0.000000,0.000000}%
\pgfsetfillcolor{currentfill}%
\pgfsetlinewidth{0.803000pt}%
\definecolor{currentstroke}{rgb}{0.000000,0.000000,0.000000}%
\pgfsetstrokecolor{currentstroke}%
\pgfsetdash{}{0pt}%
\pgfsys@defobject{currentmarker}{\pgfqpoint{0.000000in}{0.000000in}}{\pgfqpoint{0.048611in}{0.000000in}}{%
\pgfpathmoveto{\pgfqpoint{0.000000in}{0.000000in}}%
\pgfpathlineto{\pgfqpoint{0.048611in}{0.000000in}}%
\pgfusepath{stroke,fill}%
}%
\begin{pgfscope}%
\pgfsys@transformshift{7.200000in}{1.956670in}%
\pgfsys@useobject{currentmarker}{}%
\end{pgfscope}%
\end{pgfscope}%
\begin{pgfscope}%
\definecolor{textcolor}{rgb}{0.000000,0.000000,0.000000}%
\pgfsetstrokecolor{textcolor}%
\pgfsetfillcolor{textcolor}%
\pgftext[x=7.297222in, y=1.856651in, left, base]{\color{textcolor}\sffamily\fontsize{20.000000}{24.000000}\selectfont 0.2}%
\end{pgfscope}%
\begin{pgfscope}%
\pgfsetbuttcap%
\pgfsetroundjoin%
\definecolor{currentfill}{rgb}{0.000000,0.000000,0.000000}%
\pgfsetfillcolor{currentfill}%
\pgfsetlinewidth{0.803000pt}%
\definecolor{currentstroke}{rgb}{0.000000,0.000000,0.000000}%
\pgfsetstrokecolor{currentstroke}%
\pgfsetdash{}{0pt}%
\pgfsys@defobject{currentmarker}{\pgfqpoint{0.000000in}{0.000000in}}{\pgfqpoint{0.048611in}{0.000000in}}{%
\pgfpathmoveto{\pgfqpoint{0.000000in}{0.000000in}}%
\pgfpathlineto{\pgfqpoint{0.048611in}{0.000000in}}%
\pgfusepath{stroke,fill}%
}%
\begin{pgfscope}%
\pgfsys@transformshift{7.200000in}{2.775248in}%
\pgfsys@useobject{currentmarker}{}%
\end{pgfscope}%
\end{pgfscope}%
\begin{pgfscope}%
\definecolor{textcolor}{rgb}{0.000000,0.000000,0.000000}%
\pgfsetstrokecolor{textcolor}%
\pgfsetfillcolor{textcolor}%
\pgftext[x=7.297222in, y=2.675228in, left, base]{\color{textcolor}\sffamily\fontsize{20.000000}{24.000000}\selectfont 0.4}%
\end{pgfscope}%
\begin{pgfscope}%
\pgfsetbuttcap%
\pgfsetroundjoin%
\definecolor{currentfill}{rgb}{0.000000,0.000000,0.000000}%
\pgfsetfillcolor{currentfill}%
\pgfsetlinewidth{0.803000pt}%
\definecolor{currentstroke}{rgb}{0.000000,0.000000,0.000000}%
\pgfsetstrokecolor{currentstroke}%
\pgfsetdash{}{0pt}%
\pgfsys@defobject{currentmarker}{\pgfqpoint{0.000000in}{0.000000in}}{\pgfqpoint{0.048611in}{0.000000in}}{%
\pgfpathmoveto{\pgfqpoint{0.000000in}{0.000000in}}%
\pgfpathlineto{\pgfqpoint{0.048611in}{0.000000in}}%
\pgfusepath{stroke,fill}%
}%
\begin{pgfscope}%
\pgfsys@transformshift{7.200000in}{3.593825in}%
\pgfsys@useobject{currentmarker}{}%
\end{pgfscope}%
\end{pgfscope}%
\begin{pgfscope}%
\definecolor{textcolor}{rgb}{0.000000,0.000000,0.000000}%
\pgfsetstrokecolor{textcolor}%
\pgfsetfillcolor{textcolor}%
\pgftext[x=7.297222in, y=3.493806in, left, base]{\color{textcolor}\sffamily\fontsize{20.000000}{24.000000}\selectfont 0.6}%
\end{pgfscope}%
\begin{pgfscope}%
\pgfsetbuttcap%
\pgfsetroundjoin%
\definecolor{currentfill}{rgb}{0.000000,0.000000,0.000000}%
\pgfsetfillcolor{currentfill}%
\pgfsetlinewidth{0.803000pt}%
\definecolor{currentstroke}{rgb}{0.000000,0.000000,0.000000}%
\pgfsetstrokecolor{currentstroke}%
\pgfsetdash{}{0pt}%
\pgfsys@defobject{currentmarker}{\pgfqpoint{0.000000in}{0.000000in}}{\pgfqpoint{0.048611in}{0.000000in}}{%
\pgfpathmoveto{\pgfqpoint{0.000000in}{0.000000in}}%
\pgfpathlineto{\pgfqpoint{0.048611in}{0.000000in}}%
\pgfusepath{stroke,fill}%
}%
\begin{pgfscope}%
\pgfsys@transformshift{7.200000in}{4.412402in}%
\pgfsys@useobject{currentmarker}{}%
\end{pgfscope}%
\end{pgfscope}%
\begin{pgfscope}%
\definecolor{textcolor}{rgb}{0.000000,0.000000,0.000000}%
\pgfsetstrokecolor{textcolor}%
\pgfsetfillcolor{textcolor}%
\pgftext[x=7.297222in, y=4.312383in, left, base]{\color{textcolor}\sffamily\fontsize{20.000000}{24.000000}\selectfont 0.8}%
\end{pgfscope}%
\begin{pgfscope}%
\pgfsetbuttcap%
\pgfsetroundjoin%
\definecolor{currentfill}{rgb}{0.000000,0.000000,0.000000}%
\pgfsetfillcolor{currentfill}%
\pgfsetlinewidth{0.803000pt}%
\definecolor{currentstroke}{rgb}{0.000000,0.000000,0.000000}%
\pgfsetstrokecolor{currentstroke}%
\pgfsetdash{}{0pt}%
\pgfsys@defobject{currentmarker}{\pgfqpoint{0.000000in}{0.000000in}}{\pgfqpoint{0.048611in}{0.000000in}}{%
\pgfpathmoveto{\pgfqpoint{0.000000in}{0.000000in}}%
\pgfpathlineto{\pgfqpoint{0.048611in}{0.000000in}}%
\pgfusepath{stroke,fill}%
}%
\begin{pgfscope}%
\pgfsys@transformshift{7.200000in}{5.230980in}%
\pgfsys@useobject{currentmarker}{}%
\end{pgfscope}%
\end{pgfscope}%
\begin{pgfscope}%
\definecolor{textcolor}{rgb}{0.000000,0.000000,0.000000}%
\pgfsetstrokecolor{textcolor}%
\pgfsetfillcolor{textcolor}%
\pgftext[x=7.297222in, y=5.130960in, left, base]{\color{textcolor}\sffamily\fontsize{20.000000}{24.000000}\selectfont 1.0}%
\end{pgfscope}%
\begin{pgfscope}%
\definecolor{textcolor}{rgb}{0.000000,0.000000,0.000000}%
\pgfsetstrokecolor{textcolor}%
\pgfsetfillcolor{textcolor}%
\pgftext[x=7.698906in,y=3.030000in,,top,rotate=90.000000]{\color{textcolor}\sffamily\fontsize{20.000000}{24.000000}\selectfont \(\displaystyle \mathrm{Charge}\)}%
\end{pgfscope}%
\begin{pgfscope}%
\pgfpathrectangle{\pgfqpoint{1.000000in}{0.720000in}}{\pgfqpoint{6.200000in}{4.620000in}}%
\pgfusepath{clip}%
\pgfsetbuttcap%
\pgfsetroundjoin%
\pgfsetlinewidth{0.501875pt}%
\definecolor{currentstroke}{rgb}{1.000000,0.000000,0.000000}%
\pgfsetstrokecolor{currentstroke}%
\pgfsetdash{}{0pt}%
\pgfpathmoveto{\pgfqpoint{1.837000in}{1.138093in}}%
\pgfpathlineto{\pgfqpoint{1.837000in}{1.478236in}}%
\pgfusepath{stroke}%
\end{pgfscope}%
\begin{pgfscope}%
\pgfpathrectangle{\pgfqpoint{1.000000in}{0.720000in}}{\pgfqpoint{6.200000in}{4.620000in}}%
\pgfusepath{clip}%
\pgfsetbuttcap%
\pgfsetroundjoin%
\pgfsetlinewidth{0.501875pt}%
\definecolor{currentstroke}{rgb}{1.000000,0.000000,0.000000}%
\pgfsetstrokecolor{currentstroke}%
\pgfsetdash{}{0pt}%
\pgfpathmoveto{\pgfqpoint{1.868000in}{1.138093in}}%
\pgfpathlineto{\pgfqpoint{1.868000in}{1.139411in}}%
\pgfusepath{stroke}%
\end{pgfscope}%
\begin{pgfscope}%
\pgfpathrectangle{\pgfqpoint{1.000000in}{0.720000in}}{\pgfqpoint{6.200000in}{4.620000in}}%
\pgfusepath{clip}%
\pgfsetbuttcap%
\pgfsetroundjoin%
\pgfsetlinewidth{0.501875pt}%
\definecolor{currentstroke}{rgb}{1.000000,0.000000,0.000000}%
\pgfsetstrokecolor{currentstroke}%
\pgfsetdash{}{0pt}%
\pgfpathmoveto{\pgfqpoint{1.930000in}{1.138093in}}%
\pgfpathlineto{\pgfqpoint{1.930000in}{5.139909in}}%
\pgfusepath{stroke}%
\end{pgfscope}%
\begin{pgfscope}%
\pgfpathrectangle{\pgfqpoint{1.000000in}{0.720000in}}{\pgfqpoint{6.200000in}{4.620000in}}%
\pgfusepath{clip}%
\pgfsetbuttcap%
\pgfsetroundjoin%
\pgfsetlinewidth{0.501875pt}%
\definecolor{currentstroke}{rgb}{1.000000,0.000000,0.000000}%
\pgfsetstrokecolor{currentstroke}%
\pgfsetdash{}{0pt}%
\pgfpathmoveto{\pgfqpoint{1.961000in}{1.138093in}}%
\pgfpathlineto{\pgfqpoint{1.961000in}{1.202595in}}%
\pgfusepath{stroke}%
\end{pgfscope}%
\begin{pgfscope}%
\pgfpathrectangle{\pgfqpoint{1.000000in}{0.720000in}}{\pgfqpoint{6.200000in}{4.620000in}}%
\pgfusepath{clip}%
\pgfsetbuttcap%
\pgfsetroundjoin%
\pgfsetlinewidth{0.501875pt}%
\definecolor{currentstroke}{rgb}{1.000000,0.000000,0.000000}%
\pgfsetstrokecolor{currentstroke}%
\pgfsetdash{}{0pt}%
\pgfpathmoveto{\pgfqpoint{1.992000in}{1.138093in}}%
\pgfpathlineto{\pgfqpoint{1.992000in}{4.488283in}}%
\pgfusepath{stroke}%
\end{pgfscope}%
\begin{pgfscope}%
\pgfpathrectangle{\pgfqpoint{1.000000in}{0.720000in}}{\pgfqpoint{6.200000in}{4.620000in}}%
\pgfusepath{clip}%
\pgfsetbuttcap%
\pgfsetroundjoin%
\pgfsetlinewidth{0.501875pt}%
\definecolor{currentstroke}{rgb}{1.000000,0.000000,0.000000}%
\pgfsetstrokecolor{currentstroke}%
\pgfsetdash{}{0pt}%
\pgfpathmoveto{\pgfqpoint{2.457000in}{1.138093in}}%
\pgfpathlineto{\pgfqpoint{2.457000in}{3.571976in}}%
\pgfusepath{stroke}%
\end{pgfscope}%
\begin{pgfscope}%
\pgfpathrectangle{\pgfqpoint{1.000000in}{0.720000in}}{\pgfqpoint{6.200000in}{4.620000in}}%
\pgfusepath{clip}%
\pgfsetbuttcap%
\pgfsetroundjoin%
\pgfsetlinewidth{0.501875pt}%
\definecolor{currentstroke}{rgb}{1.000000,0.000000,0.000000}%
\pgfsetstrokecolor{currentstroke}%
\pgfsetdash{}{0pt}%
\pgfpathmoveto{\pgfqpoint{2.488000in}{1.138093in}}%
\pgfpathlineto{\pgfqpoint{2.488000in}{4.388176in}}%
\pgfusepath{stroke}%
\end{pgfscope}%
\begin{pgfscope}%
\pgfpathrectangle{\pgfqpoint{1.000000in}{0.720000in}}{\pgfqpoint{6.200000in}{4.620000in}}%
\pgfusepath{clip}%
\pgfsetbuttcap%
\pgfsetroundjoin%
\pgfsetlinewidth{0.501875pt}%
\definecolor{currentstroke}{rgb}{1.000000,0.000000,0.000000}%
\pgfsetstrokecolor{currentstroke}%
\pgfsetdash{}{0pt}%
\pgfpathmoveto{\pgfqpoint{2.829000in}{1.138093in}}%
\pgfpathlineto{\pgfqpoint{2.829000in}{3.157138in}}%
\pgfusepath{stroke}%
\end{pgfscope}%
\begin{pgfscope}%
\pgfpathrectangle{\pgfqpoint{1.000000in}{0.720000in}}{\pgfqpoint{6.200000in}{4.620000in}}%
\pgfusepath{clip}%
\pgfsetbuttcap%
\pgfsetroundjoin%
\pgfsetlinewidth{0.501875pt}%
\definecolor{currentstroke}{rgb}{1.000000,0.000000,0.000000}%
\pgfsetstrokecolor{currentstroke}%
\pgfsetdash{}{0pt}%
\pgfpathmoveto{\pgfqpoint{2.860000in}{1.138093in}}%
\pgfpathlineto{\pgfqpoint{2.860000in}{3.228380in}}%
\pgfusepath{stroke}%
\end{pgfscope}%
\begin{pgfscope}%
\pgfpathrectangle{\pgfqpoint{1.000000in}{0.720000in}}{\pgfqpoint{6.200000in}{4.620000in}}%
\pgfusepath{clip}%
\pgfsetbuttcap%
\pgfsetroundjoin%
\pgfsetlinewidth{0.501875pt}%
\definecolor{currentstroke}{rgb}{1.000000,0.000000,0.000000}%
\pgfsetstrokecolor{currentstroke}%
\pgfsetdash{}{0pt}%
\pgfpathmoveto{\pgfqpoint{3.697000in}{1.138093in}}%
\pgfpathlineto{\pgfqpoint{3.697000in}{2.475005in}}%
\pgfusepath{stroke}%
\end{pgfscope}%
\begin{pgfscope}%
\pgfpathrectangle{\pgfqpoint{1.000000in}{0.720000in}}{\pgfqpoint{6.200000in}{4.620000in}}%
\pgfusepath{clip}%
\pgfsetbuttcap%
\pgfsetroundjoin%
\pgfsetlinewidth{0.501875pt}%
\definecolor{currentstroke}{rgb}{1.000000,0.000000,0.000000}%
\pgfsetstrokecolor{currentstroke}%
\pgfsetdash{}{0pt}%
\pgfpathmoveto{\pgfqpoint{3.728000in}{1.138093in}}%
\pgfpathlineto{\pgfqpoint{3.728000in}{4.561296in}}%
\pgfusepath{stroke}%
\end{pgfscope}%
\begin{pgfscope}%
\pgfsetrectcap%
\pgfsetmiterjoin%
\pgfsetlinewidth{0.803000pt}%
\definecolor{currentstroke}{rgb}{0.000000,0.000000,0.000000}%
\pgfsetstrokecolor{currentstroke}%
\pgfsetdash{}{0pt}%
\pgfpathmoveto{\pgfqpoint{1.000000in}{0.720000in}}%
\pgfpathlineto{\pgfqpoint{1.000000in}{5.340000in}}%
\pgfusepath{stroke}%
\end{pgfscope}%
\begin{pgfscope}%
\pgfsetrectcap%
\pgfsetmiterjoin%
\pgfsetlinewidth{0.803000pt}%
\definecolor{currentstroke}{rgb}{0.000000,0.000000,0.000000}%
\pgfsetstrokecolor{currentstroke}%
\pgfsetdash{}{0pt}%
\pgfpathmoveto{\pgfqpoint{7.200000in}{0.720000in}}%
\pgfpathlineto{\pgfqpoint{7.200000in}{5.340000in}}%
\pgfusepath{stroke}%
\end{pgfscope}%
\begin{pgfscope}%
\pgfsetrectcap%
\pgfsetmiterjoin%
\pgfsetlinewidth{0.803000pt}%
\definecolor{currentstroke}{rgb}{0.000000,0.000000,0.000000}%
\pgfsetstrokecolor{currentstroke}%
\pgfsetdash{}{0pt}%
\pgfpathmoveto{\pgfqpoint{1.000000in}{0.720000in}}%
\pgfpathlineto{\pgfqpoint{7.200000in}{0.720000in}}%
\pgfusepath{stroke}%
\end{pgfscope}%
\begin{pgfscope}%
\pgfsetrectcap%
\pgfsetmiterjoin%
\pgfsetlinewidth{0.803000pt}%
\definecolor{currentstroke}{rgb}{0.000000,0.000000,0.000000}%
\pgfsetstrokecolor{currentstroke}%
\pgfsetdash{}{0pt}%
\pgfpathmoveto{\pgfqpoint{1.000000in}{5.340000in}}%
\pgfpathlineto{\pgfqpoint{7.200000in}{5.340000in}}%
\pgfusepath{stroke}%
\end{pgfscope}%
\begin{pgfscope}%
\pgfsetbuttcap%
\pgfsetmiterjoin%
\definecolor{currentfill}{rgb}{1.000000,1.000000,1.000000}%
\pgfsetfillcolor{currentfill}%
\pgfsetfillopacity{0.800000}%
\pgfsetlinewidth{1.003750pt}%
\definecolor{currentstroke}{rgb}{0.800000,0.800000,0.800000}%
\pgfsetstrokecolor{currentstroke}%
\pgfsetstrokeopacity{0.800000}%
\pgfsetdash{}{0pt}%
\pgfpathmoveto{\pgfqpoint{4.976872in}{4.327865in}}%
\pgfpathlineto{\pgfqpoint{7.005556in}{4.327865in}}%
\pgfpathquadraticcurveto{\pgfqpoint{7.061111in}{4.327865in}}{\pgfqpoint{7.061111in}{4.383420in}}%
\pgfpathlineto{\pgfqpoint{7.061111in}{5.145556in}}%
\pgfpathquadraticcurveto{\pgfqpoint{7.061111in}{5.201111in}}{\pgfqpoint{7.005556in}{5.201111in}}%
\pgfpathlineto{\pgfqpoint{4.976872in}{5.201111in}}%
\pgfpathquadraticcurveto{\pgfqpoint{4.921317in}{5.201111in}}{\pgfqpoint{4.921317in}{5.145556in}}%
\pgfpathlineto{\pgfqpoint{4.921317in}{4.383420in}}%
\pgfpathquadraticcurveto{\pgfqpoint{4.921317in}{4.327865in}}{\pgfqpoint{4.976872in}{4.327865in}}%
\pgfpathlineto{\pgfqpoint{4.976872in}{4.327865in}}%
\pgfpathclose%
\pgfusepath{stroke,fill}%
\end{pgfscope}%
\begin{pgfscope}%
\pgfsetrectcap%
\pgfsetroundjoin%
\pgfsetlinewidth{2.007500pt}%
\definecolor{currentstroke}{rgb}{0.121569,0.466667,0.705882}%
\pgfsetstrokecolor{currentstroke}%
\pgfsetdash{}{0pt}%
\pgfpathmoveto{\pgfqpoint{5.032428in}{4.987184in}}%
\pgfpathlineto{\pgfqpoint{5.310206in}{4.987184in}}%
\pgfpathlineto{\pgfqpoint{5.587983in}{4.987184in}}%
\pgfusepath{stroke}%
\end{pgfscope}%
\begin{pgfscope}%
\definecolor{textcolor}{rgb}{0.000000,0.000000,0.000000}%
\pgfsetstrokecolor{textcolor}%
\pgfsetfillcolor{textcolor}%
\pgftext[x=5.810206in,y=4.889962in,left,base]{\color{textcolor}\sffamily\fontsize{20.000000}{24.000000}\selectfont Waveform}%
\end{pgfscope}%
\begin{pgfscope}%
\pgfsetbuttcap%
\pgfsetroundjoin%
\pgfsetlinewidth{0.501875pt}%
\definecolor{currentstroke}{rgb}{1.000000,0.000000,0.000000}%
\pgfsetstrokecolor{currentstroke}%
\pgfsetdash{}{0pt}%
\pgfpathmoveto{\pgfqpoint{5.032428in}{4.592227in}}%
\pgfpathlineto{\pgfqpoint{5.587983in}{4.592227in}}%
\pgfusepath{stroke}%
\end{pgfscope}%
\begin{pgfscope}%
\definecolor{textcolor}{rgb}{0.000000,0.000000,0.000000}%
\pgfsetstrokecolor{textcolor}%
\pgfsetfillcolor{textcolor}%
\pgftext[x=5.810206in,y=4.495005in,left,base]{\color{textcolor}\sffamily\fontsize{20.000000}{24.000000}\selectfont Charge}%
\end{pgfscope}%
\end{pgfpicture}%
\makeatother%
\endgroup%
}
    \caption{\label{fig:fitting}An example giving \\ $\Delta{t_0}=\SI{-3.23}{ns}$, $\mathrm{RSS}=\SI{7.64}{mV^2}$,$D_\mathrm{w}=\SI{0.68}{ns}$.}
  \end{subfigure}
  \caption{\label{fig:dcf}Demonstration of direct charge fitting with $\num[retain-unity-mantissa=false]{1e4}$ waveforms in~\subref{fig:fitting-npe} and one waveform in~\subref{fig:fitting} sampled from the same setup as figure~\ref{fig:method}.  Direct charge fitting shows a better performance than LucyDDM in figure~\ref{fig:lucy} and a comparable $D_\mathrm{w}$ to CNN in figure~\ref{fig:cnn}.}
\end{figure}

The sparsity of $q'_i$ is evident in figure~\ref{fig:fitting}.  However, the majority of the $\hat{q}_i$ are less than 1.  This feature motivates us to incorporate prior knowledge of $q'_i$ towards a more dedicated model than directly fitting charges.


\subsubsection{Markov chain Monte Carlo}
\label{subsec:mcmc}
Chaining the $q'_i$ distribution~(section~\ref{subsec:spe}), the charge fitting eq.~\eqref{eq:gd-q} and the light curve eq.~\eqref{eq:time-pro}, we arrive at a hierarchical Bayesian model,
\begin{equation}
  \begin{aligned}
    t_{0} &\sim \mathcal{U}(0, \overline{t_0}) \\
    \mu_i &= \mu \int_{t'_i-\frac{\Delta t'}{2}}^{t'_i+\frac{\Delta t'}{2}} \phi(t' - t_0)\mathrm{d}t' \approx \mu\phi(t'_i - t_0)\Delta{t'} \\
    z_i &\sim \mathcal{B}(\mu_i) \\
    q'_{i,0}&=0\\
    q'_{i,1}& \sim \Gamma(k=1/0.4^2, \theta=0.4^2)\\
    q'_i &= q'_{i,z_i}\\
    w'(t) & = \sum_{i=1}^{N_\mathrm{s}}q'_iV_\mathrm{PE}(t-t'_i)\\
    w(t) &\sim \mathcal{N}(w'(t), \sigma_\epsilon^2)
  \end{aligned}
  \label{eq:mixnormal}
\end{equation}
where $\mathcal{U}$, $\mathcal{B}$ and $\Gamma$ stand for uniform, Bernoulli and gamma distributions, $\overline{t_0}$ is an upper bound of $t_0$, and $q'_i$ is a mixture of 0 (no PE) and normally distributed $q'_{i,1}$ (1 PE). When the expectation $\mu_i$ of a PE hitting $(t'_{i} - \frac{\Delta t'}{2}, t'_{i} + \frac{\Delta t'}{2})$ is small enough, that 0-1 approximation is valid.  The inferred waveform $w'(t)$ differs from observable $w(t)$ by a white noise $\epsilon(t) \sim \mathcal{N}(0, \sigma_\epsilon^2)$ motivated by eq.~\eqref{eq:1}.  When an indicator $z_i=0$, it turns off $q'_i$, reducing the number of parameters by one.  That is how eq.~\eqref{eq:mixnormal} achieves sparsity.

We generate posterior samples of $t_0$ and $\bm{q'}$ by Hamiltonian Monte Carlo~(HMC)~\cite{neal_mcmc_2012}, a variant of Markov chain Monte Carlo suitable for high-dimensional problems. Construct $\hat{t}$ and $\hat{q}_i$ as the mean estimators of posterior samples $t_0$ and $q'_i$ at $z_i=1$.  Unlike the $\hat{t}_\mathrm{KL}$ discussed in section~\ref{sec:pseudo}, $\hat{t}_0$ is a direct Bayesian estimator from eq.~\eqref{eq:mixnormal}.  We construct $\hat{\phi}(t)$ by eq.~\eqref{eq:gd-phi} and $\hat{w}(t)$ by $\hat{\phi} \otimes V_\mathrm{PE}$. RSS and $D_\mathrm{w}$ are then calculated according to eqs.~\eqref{eq:rss} and \eqref{eq:numerical}.

\begin{figure}[H]
  \begin{subfigure}{.5\textwidth}
    \centering
    \resizebox{\textwidth}{!}{%% Creator: Matplotlib, PGF backend
%%
%% To include the figure in your LaTeX document, write
%%   \input{<filename>.pgf}
%%
%% Make sure the required packages are loaded in your preamble
%%   \usepackage{pgf}
%%
%% Also ensure that all the required font packages are loaded; for instance,
%% the lmodern package is sometimes necessary when using math font.
%%   \usepackage{lmodern}
%%
%% Figures using additional raster images can only be included by \input if
%% they are in the same directory as the main LaTeX file. For loading figures
%% from other directories you can use the `import` package
%%   \usepackage{import}
%%
%% and then include the figures with
%%   \import{<path to file>}{<filename>.pgf}
%%
%% Matplotlib used the following preamble
%%   \usepackage[detect-all,locale=DE]{siunitx}
%%
\begingroup%
\makeatletter%
\begin{pgfpicture}%
\pgfpathrectangle{\pgfpointorigin}{\pgfqpoint{8.000000in}{6.000000in}}%
\pgfusepath{use as bounding box, clip}%
\begin{pgfscope}%
\pgfsetbuttcap%
\pgfsetmiterjoin%
\definecolor{currentfill}{rgb}{1.000000,1.000000,1.000000}%
\pgfsetfillcolor{currentfill}%
\pgfsetlinewidth{0.000000pt}%
\definecolor{currentstroke}{rgb}{1.000000,1.000000,1.000000}%
\pgfsetstrokecolor{currentstroke}%
\pgfsetdash{}{0pt}%
\pgfpathmoveto{\pgfqpoint{0.000000in}{0.000000in}}%
\pgfpathlineto{\pgfqpoint{8.000000in}{0.000000in}}%
\pgfpathlineto{\pgfqpoint{8.000000in}{6.000000in}}%
\pgfpathlineto{\pgfqpoint{0.000000in}{6.000000in}}%
\pgfpathlineto{\pgfqpoint{0.000000in}{0.000000in}}%
\pgfpathclose%
\pgfusepath{fill}%
\end{pgfscope}%
\begin{pgfscope}%
\pgfsetbuttcap%
\pgfsetmiterjoin%
\definecolor{currentfill}{rgb}{1.000000,1.000000,1.000000}%
\pgfsetfillcolor{currentfill}%
\pgfsetlinewidth{0.000000pt}%
\definecolor{currentstroke}{rgb}{0.000000,0.000000,0.000000}%
\pgfsetstrokecolor{currentstroke}%
\pgfsetstrokeopacity{0.000000}%
\pgfsetdash{}{0pt}%
\pgfpathmoveto{\pgfqpoint{1.000000in}{0.720000in}}%
\pgfpathlineto{\pgfqpoint{5.800000in}{0.720000in}}%
\pgfpathlineto{\pgfqpoint{5.800000in}{5.340000in}}%
\pgfpathlineto{\pgfqpoint{1.000000in}{5.340000in}}%
\pgfpathlineto{\pgfqpoint{1.000000in}{0.720000in}}%
\pgfpathclose%
\pgfusepath{fill}%
\end{pgfscope}%
\begin{pgfscope}%
\pgfpathrectangle{\pgfqpoint{1.000000in}{0.720000in}}{\pgfqpoint{4.800000in}{4.620000in}}%
\pgfusepath{clip}%
\pgfsetbuttcap%
\pgfsetroundjoin%
\definecolor{currentfill}{rgb}{0.121569,0.466667,0.705882}%
\pgfsetfillcolor{currentfill}%
\pgfsetfillopacity{0.100000}%
\pgfsetlinewidth{0.000000pt}%
\definecolor{currentstroke}{rgb}{0.000000,0.000000,0.000000}%
\pgfsetstrokecolor{currentstroke}%
\pgfsetdash{}{0pt}%
\pgfpathmoveto{\pgfqpoint{1.342857in}{3.413178in}}%
\pgfpathlineto{\pgfqpoint{1.342857in}{0.999226in}}%
\pgfpathlineto{\pgfqpoint{1.685714in}{1.691759in}}%
\pgfpathlineto{\pgfqpoint{2.028571in}{1.864827in}}%
\pgfpathlineto{\pgfqpoint{2.371429in}{1.964969in}}%
\pgfpathlineto{\pgfqpoint{2.714286in}{2.110513in}}%
\pgfpathlineto{\pgfqpoint{3.057143in}{2.255386in}}%
\pgfpathlineto{\pgfqpoint{3.400000in}{2.224485in}}%
\pgfpathlineto{\pgfqpoint{3.742857in}{2.300655in}}%
\pgfpathlineto{\pgfqpoint{4.085714in}{2.315543in}}%
\pgfpathlineto{\pgfqpoint{4.428571in}{2.284616in}}%
\pgfpathlineto{\pgfqpoint{4.771429in}{2.358620in}}%
\pgfpathlineto{\pgfqpoint{5.114286in}{2.460160in}}%
\pgfpathlineto{\pgfqpoint{5.457143in}{3.261881in}}%
\pgfpathlineto{\pgfqpoint{5.457143in}{3.261881in}}%
\pgfpathlineto{\pgfqpoint{5.457143in}{3.261881in}}%
\pgfpathlineto{\pgfqpoint{5.114286in}{2.981700in}}%
\pgfpathlineto{\pgfqpoint{4.771429in}{3.652490in}}%
\pgfpathlineto{\pgfqpoint{4.428571in}{3.276192in}}%
\pgfpathlineto{\pgfqpoint{4.085714in}{3.605453in}}%
\pgfpathlineto{\pgfqpoint{3.742857in}{3.746083in}}%
\pgfpathlineto{\pgfqpoint{3.400000in}{3.751019in}}%
\pgfpathlineto{\pgfqpoint{3.057143in}{3.773631in}}%
\pgfpathlineto{\pgfqpoint{2.714286in}{3.881706in}}%
\pgfpathlineto{\pgfqpoint{2.371429in}{3.927694in}}%
\pgfpathlineto{\pgfqpoint{2.028571in}{3.843035in}}%
\pgfpathlineto{\pgfqpoint{1.685714in}{3.981262in}}%
\pgfpathlineto{\pgfqpoint{1.342857in}{3.413178in}}%
\pgfpathlineto{\pgfqpoint{1.342857in}{3.413178in}}%
\pgfpathclose%
\pgfusepath{fill}%
\end{pgfscope}%
\begin{pgfscope}%
\pgfsetbuttcap%
\pgfsetroundjoin%
\definecolor{currentfill}{rgb}{0.000000,0.000000,0.000000}%
\pgfsetfillcolor{currentfill}%
\pgfsetlinewidth{0.803000pt}%
\definecolor{currentstroke}{rgb}{0.000000,0.000000,0.000000}%
\pgfsetstrokecolor{currentstroke}%
\pgfsetdash{}{0pt}%
\pgfsys@defobject{currentmarker}{\pgfqpoint{0.000000in}{-0.048611in}}{\pgfqpoint{0.000000in}{0.000000in}}{%
\pgfpathmoveto{\pgfqpoint{0.000000in}{0.000000in}}%
\pgfpathlineto{\pgfqpoint{0.000000in}{-0.048611in}}%
\pgfusepath{stroke,fill}%
}%
\begin{pgfscope}%
\pgfsys@transformshift{1.342857in}{0.720000in}%
\pgfsys@useobject{currentmarker}{}%
\end{pgfscope}%
\end{pgfscope}%
\begin{pgfscope}%
\definecolor{textcolor}{rgb}{0.000000,0.000000,0.000000}%
\pgfsetstrokecolor{textcolor}%
\pgfsetfillcolor{textcolor}%
\pgftext[x=1.342857in,y=0.622778in,,top]{\color{textcolor}\sffamily\fontsize{20.000000}{24.000000}\selectfont 1}%
\end{pgfscope}%
\begin{pgfscope}%
\pgfsetbuttcap%
\pgfsetroundjoin%
\definecolor{currentfill}{rgb}{0.000000,0.000000,0.000000}%
\pgfsetfillcolor{currentfill}%
\pgfsetlinewidth{0.803000pt}%
\definecolor{currentstroke}{rgb}{0.000000,0.000000,0.000000}%
\pgfsetstrokecolor{currentstroke}%
\pgfsetdash{}{0pt}%
\pgfsys@defobject{currentmarker}{\pgfqpoint{0.000000in}{-0.048611in}}{\pgfqpoint{0.000000in}{0.000000in}}{%
\pgfpathmoveto{\pgfqpoint{0.000000in}{0.000000in}}%
\pgfpathlineto{\pgfqpoint{0.000000in}{-0.048611in}}%
\pgfusepath{stroke,fill}%
}%
\begin{pgfscope}%
\pgfsys@transformshift{2.028571in}{0.720000in}%
\pgfsys@useobject{currentmarker}{}%
\end{pgfscope}%
\end{pgfscope}%
\begin{pgfscope}%
\definecolor{textcolor}{rgb}{0.000000,0.000000,0.000000}%
\pgfsetstrokecolor{textcolor}%
\pgfsetfillcolor{textcolor}%
\pgftext[x=2.028571in,y=0.622778in,,top]{\color{textcolor}\sffamily\fontsize{20.000000}{24.000000}\selectfont 3}%
\end{pgfscope}%
\begin{pgfscope}%
\pgfsetbuttcap%
\pgfsetroundjoin%
\definecolor{currentfill}{rgb}{0.000000,0.000000,0.000000}%
\pgfsetfillcolor{currentfill}%
\pgfsetlinewidth{0.803000pt}%
\definecolor{currentstroke}{rgb}{0.000000,0.000000,0.000000}%
\pgfsetstrokecolor{currentstroke}%
\pgfsetdash{}{0pt}%
\pgfsys@defobject{currentmarker}{\pgfqpoint{0.000000in}{-0.048611in}}{\pgfqpoint{0.000000in}{0.000000in}}{%
\pgfpathmoveto{\pgfqpoint{0.000000in}{0.000000in}}%
\pgfpathlineto{\pgfqpoint{0.000000in}{-0.048611in}}%
\pgfusepath{stroke,fill}%
}%
\begin{pgfscope}%
\pgfsys@transformshift{2.714286in}{0.720000in}%
\pgfsys@useobject{currentmarker}{}%
\end{pgfscope}%
\end{pgfscope}%
\begin{pgfscope}%
\definecolor{textcolor}{rgb}{0.000000,0.000000,0.000000}%
\pgfsetstrokecolor{textcolor}%
\pgfsetfillcolor{textcolor}%
\pgftext[x=2.714286in,y=0.622778in,,top]{\color{textcolor}\sffamily\fontsize{20.000000}{24.000000}\selectfont 5}%
\end{pgfscope}%
\begin{pgfscope}%
\pgfsetbuttcap%
\pgfsetroundjoin%
\definecolor{currentfill}{rgb}{0.000000,0.000000,0.000000}%
\pgfsetfillcolor{currentfill}%
\pgfsetlinewidth{0.803000pt}%
\definecolor{currentstroke}{rgb}{0.000000,0.000000,0.000000}%
\pgfsetstrokecolor{currentstroke}%
\pgfsetdash{}{0pt}%
\pgfsys@defobject{currentmarker}{\pgfqpoint{0.000000in}{-0.048611in}}{\pgfqpoint{0.000000in}{0.000000in}}{%
\pgfpathmoveto{\pgfqpoint{0.000000in}{0.000000in}}%
\pgfpathlineto{\pgfqpoint{0.000000in}{-0.048611in}}%
\pgfusepath{stroke,fill}%
}%
\begin{pgfscope}%
\pgfsys@transformshift{3.400000in}{0.720000in}%
\pgfsys@useobject{currentmarker}{}%
\end{pgfscope}%
\end{pgfscope}%
\begin{pgfscope}%
\definecolor{textcolor}{rgb}{0.000000,0.000000,0.000000}%
\pgfsetstrokecolor{textcolor}%
\pgfsetfillcolor{textcolor}%
\pgftext[x=3.400000in,y=0.622778in,,top]{\color{textcolor}\sffamily\fontsize{20.000000}{24.000000}\selectfont 7}%
\end{pgfscope}%
\begin{pgfscope}%
\pgfsetbuttcap%
\pgfsetroundjoin%
\definecolor{currentfill}{rgb}{0.000000,0.000000,0.000000}%
\pgfsetfillcolor{currentfill}%
\pgfsetlinewidth{0.803000pt}%
\definecolor{currentstroke}{rgb}{0.000000,0.000000,0.000000}%
\pgfsetstrokecolor{currentstroke}%
\pgfsetdash{}{0pt}%
\pgfsys@defobject{currentmarker}{\pgfqpoint{0.000000in}{-0.048611in}}{\pgfqpoint{0.000000in}{0.000000in}}{%
\pgfpathmoveto{\pgfqpoint{0.000000in}{0.000000in}}%
\pgfpathlineto{\pgfqpoint{0.000000in}{-0.048611in}}%
\pgfusepath{stroke,fill}%
}%
\begin{pgfscope}%
\pgfsys@transformshift{4.085714in}{0.720000in}%
\pgfsys@useobject{currentmarker}{}%
\end{pgfscope}%
\end{pgfscope}%
\begin{pgfscope}%
\definecolor{textcolor}{rgb}{0.000000,0.000000,0.000000}%
\pgfsetstrokecolor{textcolor}%
\pgfsetfillcolor{textcolor}%
\pgftext[x=4.085714in,y=0.622778in,,top]{\color{textcolor}\sffamily\fontsize{20.000000}{24.000000}\selectfont 9}%
\end{pgfscope}%
\begin{pgfscope}%
\pgfsetbuttcap%
\pgfsetroundjoin%
\definecolor{currentfill}{rgb}{0.000000,0.000000,0.000000}%
\pgfsetfillcolor{currentfill}%
\pgfsetlinewidth{0.803000pt}%
\definecolor{currentstroke}{rgb}{0.000000,0.000000,0.000000}%
\pgfsetstrokecolor{currentstroke}%
\pgfsetdash{}{0pt}%
\pgfsys@defobject{currentmarker}{\pgfqpoint{0.000000in}{-0.048611in}}{\pgfqpoint{0.000000in}{0.000000in}}{%
\pgfpathmoveto{\pgfqpoint{0.000000in}{0.000000in}}%
\pgfpathlineto{\pgfqpoint{0.000000in}{-0.048611in}}%
\pgfusepath{stroke,fill}%
}%
\begin{pgfscope}%
\pgfsys@transformshift{4.771429in}{0.720000in}%
\pgfsys@useobject{currentmarker}{}%
\end{pgfscope}%
\end{pgfscope}%
\begin{pgfscope}%
\definecolor{textcolor}{rgb}{0.000000,0.000000,0.000000}%
\pgfsetstrokecolor{textcolor}%
\pgfsetfillcolor{textcolor}%
\pgftext[x=4.771429in,y=0.622778in,,top]{\color{textcolor}\sffamily\fontsize{20.000000}{24.000000}\selectfont 11}%
\end{pgfscope}%
\begin{pgfscope}%
\pgfsetbuttcap%
\pgfsetroundjoin%
\definecolor{currentfill}{rgb}{0.000000,0.000000,0.000000}%
\pgfsetfillcolor{currentfill}%
\pgfsetlinewidth{0.803000pt}%
\definecolor{currentstroke}{rgb}{0.000000,0.000000,0.000000}%
\pgfsetstrokecolor{currentstroke}%
\pgfsetdash{}{0pt}%
\pgfsys@defobject{currentmarker}{\pgfqpoint{0.000000in}{-0.048611in}}{\pgfqpoint{0.000000in}{0.000000in}}{%
\pgfpathmoveto{\pgfqpoint{0.000000in}{0.000000in}}%
\pgfpathlineto{\pgfqpoint{0.000000in}{-0.048611in}}%
\pgfusepath{stroke,fill}%
}%
\begin{pgfscope}%
\pgfsys@transformshift{5.457143in}{0.720000in}%
\pgfsys@useobject{currentmarker}{}%
\end{pgfscope}%
\end{pgfscope}%
\begin{pgfscope}%
\definecolor{textcolor}{rgb}{0.000000,0.000000,0.000000}%
\pgfsetstrokecolor{textcolor}%
\pgfsetfillcolor{textcolor}%
\pgftext[x=5.457143in,y=0.622778in,,top]{\color{textcolor}\sffamily\fontsize{20.000000}{24.000000}\selectfont 13}%
\end{pgfscope}%
\begin{pgfscope}%
\definecolor{textcolor}{rgb}{0.000000,0.000000,0.000000}%
\pgfsetstrokecolor{textcolor}%
\pgfsetfillcolor{textcolor}%
\pgftext[x=3.400000in,y=0.311155in,,top]{\color{textcolor}\sffamily\fontsize{20.000000}{24.000000}\selectfont \(\displaystyle N_{\mathrm{PE}}\)}%
\end{pgfscope}%
\begin{pgfscope}%
\pgfsetbuttcap%
\pgfsetroundjoin%
\definecolor{currentfill}{rgb}{0.000000,0.000000,0.000000}%
\pgfsetfillcolor{currentfill}%
\pgfsetlinewidth{0.803000pt}%
\definecolor{currentstroke}{rgb}{0.000000,0.000000,0.000000}%
\pgfsetstrokecolor{currentstroke}%
\pgfsetdash{}{0pt}%
\pgfsys@defobject{currentmarker}{\pgfqpoint{-0.048611in}{0.000000in}}{\pgfqpoint{-0.000000in}{0.000000in}}{%
\pgfpathmoveto{\pgfqpoint{-0.000000in}{0.000000in}}%
\pgfpathlineto{\pgfqpoint{-0.048611in}{0.000000in}}%
\pgfusepath{stroke,fill}%
}%
\begin{pgfscope}%
\pgfsys@transformshift{1.000000in}{0.720000in}%
\pgfsys@useobject{currentmarker}{}%
\end{pgfscope}%
\end{pgfscope}%
\begin{pgfscope}%
\definecolor{textcolor}{rgb}{0.000000,0.000000,0.000000}%
\pgfsetstrokecolor{textcolor}%
\pgfsetfillcolor{textcolor}%
\pgftext[x=0.560215in, y=0.619981in, left, base]{\color{textcolor}\sffamily\fontsize{20.000000}{24.000000}\selectfont \(\displaystyle {0.0}\)}%
\end{pgfscope}%
\begin{pgfscope}%
\pgfsetbuttcap%
\pgfsetroundjoin%
\definecolor{currentfill}{rgb}{0.000000,0.000000,0.000000}%
\pgfsetfillcolor{currentfill}%
\pgfsetlinewidth{0.803000pt}%
\definecolor{currentstroke}{rgb}{0.000000,0.000000,0.000000}%
\pgfsetstrokecolor{currentstroke}%
\pgfsetdash{}{0pt}%
\pgfsys@defobject{currentmarker}{\pgfqpoint{-0.048611in}{0.000000in}}{\pgfqpoint{-0.000000in}{0.000000in}}{%
\pgfpathmoveto{\pgfqpoint{-0.000000in}{0.000000in}}%
\pgfpathlineto{\pgfqpoint{-0.048611in}{0.000000in}}%
\pgfusepath{stroke,fill}%
}%
\begin{pgfscope}%
\pgfsys@transformshift{1.000000in}{1.310338in}%
\pgfsys@useobject{currentmarker}{}%
\end{pgfscope}%
\end{pgfscope}%
\begin{pgfscope}%
\definecolor{textcolor}{rgb}{0.000000,0.000000,0.000000}%
\pgfsetstrokecolor{textcolor}%
\pgfsetfillcolor{textcolor}%
\pgftext[x=0.560215in, y=1.210318in, left, base]{\color{textcolor}\sffamily\fontsize{20.000000}{24.000000}\selectfont \(\displaystyle {0.2}\)}%
\end{pgfscope}%
\begin{pgfscope}%
\pgfsetbuttcap%
\pgfsetroundjoin%
\definecolor{currentfill}{rgb}{0.000000,0.000000,0.000000}%
\pgfsetfillcolor{currentfill}%
\pgfsetlinewidth{0.803000pt}%
\definecolor{currentstroke}{rgb}{0.000000,0.000000,0.000000}%
\pgfsetstrokecolor{currentstroke}%
\pgfsetdash{}{0pt}%
\pgfsys@defobject{currentmarker}{\pgfqpoint{-0.048611in}{0.000000in}}{\pgfqpoint{-0.000000in}{0.000000in}}{%
\pgfpathmoveto{\pgfqpoint{-0.000000in}{0.000000in}}%
\pgfpathlineto{\pgfqpoint{-0.048611in}{0.000000in}}%
\pgfusepath{stroke,fill}%
}%
\begin{pgfscope}%
\pgfsys@transformshift{1.000000in}{1.900675in}%
\pgfsys@useobject{currentmarker}{}%
\end{pgfscope}%
\end{pgfscope}%
\begin{pgfscope}%
\definecolor{textcolor}{rgb}{0.000000,0.000000,0.000000}%
\pgfsetstrokecolor{textcolor}%
\pgfsetfillcolor{textcolor}%
\pgftext[x=0.560215in, y=1.800656in, left, base]{\color{textcolor}\sffamily\fontsize{20.000000}{24.000000}\selectfont \(\displaystyle {0.4}\)}%
\end{pgfscope}%
\begin{pgfscope}%
\pgfsetbuttcap%
\pgfsetroundjoin%
\definecolor{currentfill}{rgb}{0.000000,0.000000,0.000000}%
\pgfsetfillcolor{currentfill}%
\pgfsetlinewidth{0.803000pt}%
\definecolor{currentstroke}{rgb}{0.000000,0.000000,0.000000}%
\pgfsetstrokecolor{currentstroke}%
\pgfsetdash{}{0pt}%
\pgfsys@defobject{currentmarker}{\pgfqpoint{-0.048611in}{0.000000in}}{\pgfqpoint{-0.000000in}{0.000000in}}{%
\pgfpathmoveto{\pgfqpoint{-0.000000in}{0.000000in}}%
\pgfpathlineto{\pgfqpoint{-0.048611in}{0.000000in}}%
\pgfusepath{stroke,fill}%
}%
\begin{pgfscope}%
\pgfsys@transformshift{1.000000in}{2.491013in}%
\pgfsys@useobject{currentmarker}{}%
\end{pgfscope}%
\end{pgfscope}%
\begin{pgfscope}%
\definecolor{textcolor}{rgb}{0.000000,0.000000,0.000000}%
\pgfsetstrokecolor{textcolor}%
\pgfsetfillcolor{textcolor}%
\pgftext[x=0.560215in, y=2.390993in, left, base]{\color{textcolor}\sffamily\fontsize{20.000000}{24.000000}\selectfont \(\displaystyle {0.6}\)}%
\end{pgfscope}%
\begin{pgfscope}%
\pgfsetbuttcap%
\pgfsetroundjoin%
\definecolor{currentfill}{rgb}{0.000000,0.000000,0.000000}%
\pgfsetfillcolor{currentfill}%
\pgfsetlinewidth{0.803000pt}%
\definecolor{currentstroke}{rgb}{0.000000,0.000000,0.000000}%
\pgfsetstrokecolor{currentstroke}%
\pgfsetdash{}{0pt}%
\pgfsys@defobject{currentmarker}{\pgfqpoint{-0.048611in}{0.000000in}}{\pgfqpoint{-0.000000in}{0.000000in}}{%
\pgfpathmoveto{\pgfqpoint{-0.000000in}{0.000000in}}%
\pgfpathlineto{\pgfqpoint{-0.048611in}{0.000000in}}%
\pgfusepath{stroke,fill}%
}%
\begin{pgfscope}%
\pgfsys@transformshift{1.000000in}{3.081350in}%
\pgfsys@useobject{currentmarker}{}%
\end{pgfscope}%
\end{pgfscope}%
\begin{pgfscope}%
\definecolor{textcolor}{rgb}{0.000000,0.000000,0.000000}%
\pgfsetstrokecolor{textcolor}%
\pgfsetfillcolor{textcolor}%
\pgftext[x=0.560215in, y=2.981331in, left, base]{\color{textcolor}\sffamily\fontsize{20.000000}{24.000000}\selectfont \(\displaystyle {0.8}\)}%
\end{pgfscope}%
\begin{pgfscope}%
\pgfsetbuttcap%
\pgfsetroundjoin%
\definecolor{currentfill}{rgb}{0.000000,0.000000,0.000000}%
\pgfsetfillcolor{currentfill}%
\pgfsetlinewidth{0.803000pt}%
\definecolor{currentstroke}{rgb}{0.000000,0.000000,0.000000}%
\pgfsetstrokecolor{currentstroke}%
\pgfsetdash{}{0pt}%
\pgfsys@defobject{currentmarker}{\pgfqpoint{-0.048611in}{0.000000in}}{\pgfqpoint{-0.000000in}{0.000000in}}{%
\pgfpathmoveto{\pgfqpoint{-0.000000in}{0.000000in}}%
\pgfpathlineto{\pgfqpoint{-0.048611in}{0.000000in}}%
\pgfusepath{stroke,fill}%
}%
\begin{pgfscope}%
\pgfsys@transformshift{1.000000in}{3.671688in}%
\pgfsys@useobject{currentmarker}{}%
\end{pgfscope}%
\end{pgfscope}%
\begin{pgfscope}%
\definecolor{textcolor}{rgb}{0.000000,0.000000,0.000000}%
\pgfsetstrokecolor{textcolor}%
\pgfsetfillcolor{textcolor}%
\pgftext[x=0.560215in, y=3.571668in, left, base]{\color{textcolor}\sffamily\fontsize{20.000000}{24.000000}\selectfont \(\displaystyle {1.0}\)}%
\end{pgfscope}%
\begin{pgfscope}%
\pgfsetbuttcap%
\pgfsetroundjoin%
\definecolor{currentfill}{rgb}{0.000000,0.000000,0.000000}%
\pgfsetfillcolor{currentfill}%
\pgfsetlinewidth{0.803000pt}%
\definecolor{currentstroke}{rgb}{0.000000,0.000000,0.000000}%
\pgfsetstrokecolor{currentstroke}%
\pgfsetdash{}{0pt}%
\pgfsys@defobject{currentmarker}{\pgfqpoint{-0.048611in}{0.000000in}}{\pgfqpoint{-0.000000in}{0.000000in}}{%
\pgfpathmoveto{\pgfqpoint{-0.000000in}{0.000000in}}%
\pgfpathlineto{\pgfqpoint{-0.048611in}{0.000000in}}%
\pgfusepath{stroke,fill}%
}%
\begin{pgfscope}%
\pgfsys@transformshift{1.000000in}{4.262025in}%
\pgfsys@useobject{currentmarker}{}%
\end{pgfscope}%
\end{pgfscope}%
\begin{pgfscope}%
\definecolor{textcolor}{rgb}{0.000000,0.000000,0.000000}%
\pgfsetstrokecolor{textcolor}%
\pgfsetfillcolor{textcolor}%
\pgftext[x=0.560215in, y=4.162006in, left, base]{\color{textcolor}\sffamily\fontsize{20.000000}{24.000000}\selectfont \(\displaystyle {1.2}\)}%
\end{pgfscope}%
\begin{pgfscope}%
\pgfsetbuttcap%
\pgfsetroundjoin%
\definecolor{currentfill}{rgb}{0.000000,0.000000,0.000000}%
\pgfsetfillcolor{currentfill}%
\pgfsetlinewidth{0.803000pt}%
\definecolor{currentstroke}{rgb}{0.000000,0.000000,0.000000}%
\pgfsetstrokecolor{currentstroke}%
\pgfsetdash{}{0pt}%
\pgfsys@defobject{currentmarker}{\pgfqpoint{-0.048611in}{0.000000in}}{\pgfqpoint{-0.000000in}{0.000000in}}{%
\pgfpathmoveto{\pgfqpoint{-0.000000in}{0.000000in}}%
\pgfpathlineto{\pgfqpoint{-0.048611in}{0.000000in}}%
\pgfusepath{stroke,fill}%
}%
\begin{pgfscope}%
\pgfsys@transformshift{1.000000in}{4.852363in}%
\pgfsys@useobject{currentmarker}{}%
\end{pgfscope}%
\end{pgfscope}%
\begin{pgfscope}%
\definecolor{textcolor}{rgb}{0.000000,0.000000,0.000000}%
\pgfsetstrokecolor{textcolor}%
\pgfsetfillcolor{textcolor}%
\pgftext[x=0.560215in, y=4.752343in, left, base]{\color{textcolor}\sffamily\fontsize{20.000000}{24.000000}\selectfont \(\displaystyle {1.4}\)}%
\end{pgfscope}%
\begin{pgfscope}%
\definecolor{textcolor}{rgb}{0.000000,0.000000,0.000000}%
\pgfsetstrokecolor{textcolor}%
\pgfsetfillcolor{textcolor}%
\pgftext[x=0.504660in,y=3.030000in,,bottom,rotate=90.000000]{\color{textcolor}\sffamily\fontsize{20.000000}{24.000000}\selectfont \(\displaystyle \mathrm{Wasserstein\ Distance}/\si{ns}\)}%
\end{pgfscope}%
\begin{pgfscope}%
\pgfpathrectangle{\pgfqpoint{1.000000in}{0.720000in}}{\pgfqpoint{4.800000in}{4.620000in}}%
\pgfusepath{clip}%
\pgfsetrectcap%
\pgfsetroundjoin%
\pgfsetlinewidth{2.007500pt}%
\definecolor{currentstroke}{rgb}{0.000000,0.000000,1.000000}%
\pgfsetstrokecolor{currentstroke}%
\pgfsetdash{}{0pt}%
\pgfpathmoveto{\pgfqpoint{1.342857in}{1.943302in}}%
\pgfpathlineto{\pgfqpoint{1.685714in}{2.512163in}}%
\pgfpathlineto{\pgfqpoint{2.028571in}{2.624157in}}%
\pgfpathlineto{\pgfqpoint{2.371429in}{2.663112in}}%
\pgfpathlineto{\pgfqpoint{2.714286in}{2.799138in}}%
\pgfpathlineto{\pgfqpoint{3.057143in}{2.804653in}}%
\pgfpathlineto{\pgfqpoint{3.400000in}{2.815932in}}%
\pgfpathlineto{\pgfqpoint{3.742857in}{2.969652in}}%
\pgfpathlineto{\pgfqpoint{4.085714in}{2.820098in}}%
\pgfpathlineto{\pgfqpoint{4.428571in}{2.743949in}}%
\pgfpathlineto{\pgfqpoint{4.771429in}{2.875480in}}%
\pgfpathlineto{\pgfqpoint{5.114286in}{2.763949in}}%
\pgfpathlineto{\pgfqpoint{5.457143in}{3.261881in}}%
\pgfusepath{stroke}%
\end{pgfscope}%
\begin{pgfscope}%
\pgfpathrectangle{\pgfqpoint{1.000000in}{0.720000in}}{\pgfqpoint{4.800000in}{4.620000in}}%
\pgfusepath{clip}%
\pgfsetbuttcap%
\pgfsetroundjoin%
\pgfsetlinewidth{1.003750pt}%
\definecolor{currentstroke}{rgb}{0.000000,0.000000,1.000000}%
\pgfsetstrokecolor{currentstroke}%
\pgfsetdash{}{0pt}%
\pgfpathmoveto{\pgfqpoint{1.342857in}{0.999226in}}%
\pgfpathlineto{\pgfqpoint{1.342857in}{3.413178in}}%
\pgfusepath{stroke}%
\end{pgfscope}%
\begin{pgfscope}%
\pgfpathrectangle{\pgfqpoint{1.000000in}{0.720000in}}{\pgfqpoint{4.800000in}{4.620000in}}%
\pgfusepath{clip}%
\pgfsetbuttcap%
\pgfsetroundjoin%
\pgfsetlinewidth{1.003750pt}%
\definecolor{currentstroke}{rgb}{0.000000,0.000000,1.000000}%
\pgfsetstrokecolor{currentstroke}%
\pgfsetdash{}{0pt}%
\pgfpathmoveto{\pgfqpoint{1.685714in}{1.691759in}}%
\pgfpathlineto{\pgfqpoint{1.685714in}{3.981262in}}%
\pgfusepath{stroke}%
\end{pgfscope}%
\begin{pgfscope}%
\pgfpathrectangle{\pgfqpoint{1.000000in}{0.720000in}}{\pgfqpoint{4.800000in}{4.620000in}}%
\pgfusepath{clip}%
\pgfsetbuttcap%
\pgfsetroundjoin%
\pgfsetlinewidth{1.003750pt}%
\definecolor{currentstroke}{rgb}{0.000000,0.000000,1.000000}%
\pgfsetstrokecolor{currentstroke}%
\pgfsetdash{}{0pt}%
\pgfpathmoveto{\pgfqpoint{2.028571in}{1.864827in}}%
\pgfpathlineto{\pgfqpoint{2.028571in}{3.843035in}}%
\pgfusepath{stroke}%
\end{pgfscope}%
\begin{pgfscope}%
\pgfpathrectangle{\pgfqpoint{1.000000in}{0.720000in}}{\pgfqpoint{4.800000in}{4.620000in}}%
\pgfusepath{clip}%
\pgfsetbuttcap%
\pgfsetroundjoin%
\pgfsetlinewidth{1.003750pt}%
\definecolor{currentstroke}{rgb}{0.000000,0.000000,1.000000}%
\pgfsetstrokecolor{currentstroke}%
\pgfsetdash{}{0pt}%
\pgfpathmoveto{\pgfqpoint{2.371429in}{1.964969in}}%
\pgfpathlineto{\pgfqpoint{2.371429in}{3.927694in}}%
\pgfusepath{stroke}%
\end{pgfscope}%
\begin{pgfscope}%
\pgfpathrectangle{\pgfqpoint{1.000000in}{0.720000in}}{\pgfqpoint{4.800000in}{4.620000in}}%
\pgfusepath{clip}%
\pgfsetbuttcap%
\pgfsetroundjoin%
\pgfsetlinewidth{1.003750pt}%
\definecolor{currentstroke}{rgb}{0.000000,0.000000,1.000000}%
\pgfsetstrokecolor{currentstroke}%
\pgfsetdash{}{0pt}%
\pgfpathmoveto{\pgfqpoint{2.714286in}{2.110513in}}%
\pgfpathlineto{\pgfqpoint{2.714286in}{3.881706in}}%
\pgfusepath{stroke}%
\end{pgfscope}%
\begin{pgfscope}%
\pgfpathrectangle{\pgfqpoint{1.000000in}{0.720000in}}{\pgfqpoint{4.800000in}{4.620000in}}%
\pgfusepath{clip}%
\pgfsetbuttcap%
\pgfsetroundjoin%
\pgfsetlinewidth{1.003750pt}%
\definecolor{currentstroke}{rgb}{0.000000,0.000000,1.000000}%
\pgfsetstrokecolor{currentstroke}%
\pgfsetdash{}{0pt}%
\pgfpathmoveto{\pgfqpoint{3.057143in}{2.255386in}}%
\pgfpathlineto{\pgfqpoint{3.057143in}{3.773631in}}%
\pgfusepath{stroke}%
\end{pgfscope}%
\begin{pgfscope}%
\pgfpathrectangle{\pgfqpoint{1.000000in}{0.720000in}}{\pgfqpoint{4.800000in}{4.620000in}}%
\pgfusepath{clip}%
\pgfsetbuttcap%
\pgfsetroundjoin%
\pgfsetlinewidth{1.003750pt}%
\definecolor{currentstroke}{rgb}{0.000000,0.000000,1.000000}%
\pgfsetstrokecolor{currentstroke}%
\pgfsetdash{}{0pt}%
\pgfpathmoveto{\pgfqpoint{3.400000in}{2.224485in}}%
\pgfpathlineto{\pgfqpoint{3.400000in}{3.751019in}}%
\pgfusepath{stroke}%
\end{pgfscope}%
\begin{pgfscope}%
\pgfpathrectangle{\pgfqpoint{1.000000in}{0.720000in}}{\pgfqpoint{4.800000in}{4.620000in}}%
\pgfusepath{clip}%
\pgfsetbuttcap%
\pgfsetroundjoin%
\pgfsetlinewidth{1.003750pt}%
\definecolor{currentstroke}{rgb}{0.000000,0.000000,1.000000}%
\pgfsetstrokecolor{currentstroke}%
\pgfsetdash{}{0pt}%
\pgfpathmoveto{\pgfqpoint{3.742857in}{2.300655in}}%
\pgfpathlineto{\pgfqpoint{3.742857in}{3.746083in}}%
\pgfusepath{stroke}%
\end{pgfscope}%
\begin{pgfscope}%
\pgfpathrectangle{\pgfqpoint{1.000000in}{0.720000in}}{\pgfqpoint{4.800000in}{4.620000in}}%
\pgfusepath{clip}%
\pgfsetbuttcap%
\pgfsetroundjoin%
\pgfsetlinewidth{1.003750pt}%
\definecolor{currentstroke}{rgb}{0.000000,0.000000,1.000000}%
\pgfsetstrokecolor{currentstroke}%
\pgfsetdash{}{0pt}%
\pgfpathmoveto{\pgfqpoint{4.085714in}{2.315543in}}%
\pgfpathlineto{\pgfqpoint{4.085714in}{3.605453in}}%
\pgfusepath{stroke}%
\end{pgfscope}%
\begin{pgfscope}%
\pgfpathrectangle{\pgfqpoint{1.000000in}{0.720000in}}{\pgfqpoint{4.800000in}{4.620000in}}%
\pgfusepath{clip}%
\pgfsetbuttcap%
\pgfsetroundjoin%
\pgfsetlinewidth{1.003750pt}%
\definecolor{currentstroke}{rgb}{0.000000,0.000000,1.000000}%
\pgfsetstrokecolor{currentstroke}%
\pgfsetdash{}{0pt}%
\pgfpathmoveto{\pgfqpoint{4.428571in}{2.284616in}}%
\pgfpathlineto{\pgfqpoint{4.428571in}{3.276192in}}%
\pgfusepath{stroke}%
\end{pgfscope}%
\begin{pgfscope}%
\pgfpathrectangle{\pgfqpoint{1.000000in}{0.720000in}}{\pgfqpoint{4.800000in}{4.620000in}}%
\pgfusepath{clip}%
\pgfsetbuttcap%
\pgfsetroundjoin%
\pgfsetlinewidth{1.003750pt}%
\definecolor{currentstroke}{rgb}{0.000000,0.000000,1.000000}%
\pgfsetstrokecolor{currentstroke}%
\pgfsetdash{}{0pt}%
\pgfpathmoveto{\pgfqpoint{4.771429in}{2.358620in}}%
\pgfpathlineto{\pgfqpoint{4.771429in}{3.652490in}}%
\pgfusepath{stroke}%
\end{pgfscope}%
\begin{pgfscope}%
\pgfpathrectangle{\pgfqpoint{1.000000in}{0.720000in}}{\pgfqpoint{4.800000in}{4.620000in}}%
\pgfusepath{clip}%
\pgfsetbuttcap%
\pgfsetroundjoin%
\pgfsetlinewidth{1.003750pt}%
\definecolor{currentstroke}{rgb}{0.000000,0.000000,1.000000}%
\pgfsetstrokecolor{currentstroke}%
\pgfsetdash{}{0pt}%
\pgfpathmoveto{\pgfqpoint{5.114286in}{2.460160in}}%
\pgfpathlineto{\pgfqpoint{5.114286in}{2.981700in}}%
\pgfusepath{stroke}%
\end{pgfscope}%
\begin{pgfscope}%
\pgfpathrectangle{\pgfqpoint{1.000000in}{0.720000in}}{\pgfqpoint{4.800000in}{4.620000in}}%
\pgfusepath{clip}%
\pgfsetbuttcap%
\pgfsetroundjoin%
\pgfsetlinewidth{1.003750pt}%
\definecolor{currentstroke}{rgb}{0.000000,0.000000,1.000000}%
\pgfsetstrokecolor{currentstroke}%
\pgfsetdash{}{0pt}%
\pgfpathmoveto{\pgfqpoint{5.457143in}{3.261881in}}%
\pgfpathlineto{\pgfqpoint{5.457143in}{3.261881in}}%
\pgfusepath{stroke}%
\end{pgfscope}%
\begin{pgfscope}%
\pgfpathrectangle{\pgfqpoint{1.000000in}{0.720000in}}{\pgfqpoint{4.800000in}{4.620000in}}%
\pgfusepath{clip}%
\pgfsetbuttcap%
\pgfsetroundjoin%
\definecolor{currentfill}{rgb}{0.000000,0.000000,1.000000}%
\pgfsetfillcolor{currentfill}%
\pgfsetlinewidth{1.003750pt}%
\definecolor{currentstroke}{rgb}{0.000000,0.000000,1.000000}%
\pgfsetstrokecolor{currentstroke}%
\pgfsetdash{}{0pt}%
\pgfsys@defobject{currentmarker}{\pgfqpoint{-0.041667in}{-0.000000in}}{\pgfqpoint{0.041667in}{0.000000in}}{%
\pgfpathmoveto{\pgfqpoint{0.041667in}{-0.000000in}}%
\pgfpathlineto{\pgfqpoint{-0.041667in}{0.000000in}}%
\pgfusepath{stroke,fill}%
}%
\begin{pgfscope}%
\pgfsys@transformshift{1.342857in}{0.999226in}%
\pgfsys@useobject{currentmarker}{}%
\end{pgfscope}%
\begin{pgfscope}%
\pgfsys@transformshift{1.685714in}{1.691759in}%
\pgfsys@useobject{currentmarker}{}%
\end{pgfscope}%
\begin{pgfscope}%
\pgfsys@transformshift{2.028571in}{1.864827in}%
\pgfsys@useobject{currentmarker}{}%
\end{pgfscope}%
\begin{pgfscope}%
\pgfsys@transformshift{2.371429in}{1.964969in}%
\pgfsys@useobject{currentmarker}{}%
\end{pgfscope}%
\begin{pgfscope}%
\pgfsys@transformshift{2.714286in}{2.110513in}%
\pgfsys@useobject{currentmarker}{}%
\end{pgfscope}%
\begin{pgfscope}%
\pgfsys@transformshift{3.057143in}{2.255386in}%
\pgfsys@useobject{currentmarker}{}%
\end{pgfscope}%
\begin{pgfscope}%
\pgfsys@transformshift{3.400000in}{2.224485in}%
\pgfsys@useobject{currentmarker}{}%
\end{pgfscope}%
\begin{pgfscope}%
\pgfsys@transformshift{3.742857in}{2.300655in}%
\pgfsys@useobject{currentmarker}{}%
\end{pgfscope}%
\begin{pgfscope}%
\pgfsys@transformshift{4.085714in}{2.315543in}%
\pgfsys@useobject{currentmarker}{}%
\end{pgfscope}%
\begin{pgfscope}%
\pgfsys@transformshift{4.428571in}{2.284616in}%
\pgfsys@useobject{currentmarker}{}%
\end{pgfscope}%
\begin{pgfscope}%
\pgfsys@transformshift{4.771429in}{2.358620in}%
\pgfsys@useobject{currentmarker}{}%
\end{pgfscope}%
\begin{pgfscope}%
\pgfsys@transformshift{5.114286in}{2.460160in}%
\pgfsys@useobject{currentmarker}{}%
\end{pgfscope}%
\begin{pgfscope}%
\pgfsys@transformshift{5.457143in}{3.261881in}%
\pgfsys@useobject{currentmarker}{}%
\end{pgfscope}%
\end{pgfscope}%
\begin{pgfscope}%
\pgfpathrectangle{\pgfqpoint{1.000000in}{0.720000in}}{\pgfqpoint{4.800000in}{4.620000in}}%
\pgfusepath{clip}%
\pgfsetbuttcap%
\pgfsetroundjoin%
\definecolor{currentfill}{rgb}{0.000000,0.000000,1.000000}%
\pgfsetfillcolor{currentfill}%
\pgfsetlinewidth{1.003750pt}%
\definecolor{currentstroke}{rgb}{0.000000,0.000000,1.000000}%
\pgfsetstrokecolor{currentstroke}%
\pgfsetdash{}{0pt}%
\pgfsys@defobject{currentmarker}{\pgfqpoint{-0.041667in}{-0.000000in}}{\pgfqpoint{0.041667in}{0.000000in}}{%
\pgfpathmoveto{\pgfqpoint{0.041667in}{-0.000000in}}%
\pgfpathlineto{\pgfqpoint{-0.041667in}{0.000000in}}%
\pgfusepath{stroke,fill}%
}%
\begin{pgfscope}%
\pgfsys@transformshift{1.342857in}{3.413178in}%
\pgfsys@useobject{currentmarker}{}%
\end{pgfscope}%
\begin{pgfscope}%
\pgfsys@transformshift{1.685714in}{3.981262in}%
\pgfsys@useobject{currentmarker}{}%
\end{pgfscope}%
\begin{pgfscope}%
\pgfsys@transformshift{2.028571in}{3.843035in}%
\pgfsys@useobject{currentmarker}{}%
\end{pgfscope}%
\begin{pgfscope}%
\pgfsys@transformshift{2.371429in}{3.927694in}%
\pgfsys@useobject{currentmarker}{}%
\end{pgfscope}%
\begin{pgfscope}%
\pgfsys@transformshift{2.714286in}{3.881706in}%
\pgfsys@useobject{currentmarker}{}%
\end{pgfscope}%
\begin{pgfscope}%
\pgfsys@transformshift{3.057143in}{3.773631in}%
\pgfsys@useobject{currentmarker}{}%
\end{pgfscope}%
\begin{pgfscope}%
\pgfsys@transformshift{3.400000in}{3.751019in}%
\pgfsys@useobject{currentmarker}{}%
\end{pgfscope}%
\begin{pgfscope}%
\pgfsys@transformshift{3.742857in}{3.746083in}%
\pgfsys@useobject{currentmarker}{}%
\end{pgfscope}%
\begin{pgfscope}%
\pgfsys@transformshift{4.085714in}{3.605453in}%
\pgfsys@useobject{currentmarker}{}%
\end{pgfscope}%
\begin{pgfscope}%
\pgfsys@transformshift{4.428571in}{3.276192in}%
\pgfsys@useobject{currentmarker}{}%
\end{pgfscope}%
\begin{pgfscope}%
\pgfsys@transformshift{4.771429in}{3.652490in}%
\pgfsys@useobject{currentmarker}{}%
\end{pgfscope}%
\begin{pgfscope}%
\pgfsys@transformshift{5.114286in}{2.981700in}%
\pgfsys@useobject{currentmarker}{}%
\end{pgfscope}%
\begin{pgfscope}%
\pgfsys@transformshift{5.457143in}{3.261881in}%
\pgfsys@useobject{currentmarker}{}%
\end{pgfscope}%
\end{pgfscope}%
\begin{pgfscope}%
\pgfpathrectangle{\pgfqpoint{1.000000in}{0.720000in}}{\pgfqpoint{4.800000in}{4.620000in}}%
\pgfusepath{clip}%
\pgfsetbuttcap%
\pgfsetroundjoin%
\definecolor{currentfill}{rgb}{0.000000,0.000000,1.000000}%
\pgfsetfillcolor{currentfill}%
\pgfsetlinewidth{1.003750pt}%
\definecolor{currentstroke}{rgb}{0.000000,0.000000,1.000000}%
\pgfsetstrokecolor{currentstroke}%
\pgfsetdash{}{0pt}%
\pgfsys@defobject{currentmarker}{\pgfqpoint{-0.027778in}{-0.027778in}}{\pgfqpoint{0.027778in}{0.027778in}}{%
\pgfpathmoveto{\pgfqpoint{0.000000in}{-0.027778in}}%
\pgfpathcurveto{\pgfqpoint{0.007367in}{-0.027778in}}{\pgfqpoint{0.014433in}{-0.024851in}}{\pgfqpoint{0.019642in}{-0.019642in}}%
\pgfpathcurveto{\pgfqpoint{0.024851in}{-0.014433in}}{\pgfqpoint{0.027778in}{-0.007367in}}{\pgfqpoint{0.027778in}{0.000000in}}%
\pgfpathcurveto{\pgfqpoint{0.027778in}{0.007367in}}{\pgfqpoint{0.024851in}{0.014433in}}{\pgfqpoint{0.019642in}{0.019642in}}%
\pgfpathcurveto{\pgfqpoint{0.014433in}{0.024851in}}{\pgfqpoint{0.007367in}{0.027778in}}{\pgfqpoint{0.000000in}{0.027778in}}%
\pgfpathcurveto{\pgfqpoint{-0.007367in}{0.027778in}}{\pgfqpoint{-0.014433in}{0.024851in}}{\pgfqpoint{-0.019642in}{0.019642in}}%
\pgfpathcurveto{\pgfqpoint{-0.024851in}{0.014433in}}{\pgfqpoint{-0.027778in}{0.007367in}}{\pgfqpoint{-0.027778in}{0.000000in}}%
\pgfpathcurveto{\pgfqpoint{-0.027778in}{-0.007367in}}{\pgfqpoint{-0.024851in}{-0.014433in}}{\pgfqpoint{-0.019642in}{-0.019642in}}%
\pgfpathcurveto{\pgfqpoint{-0.014433in}{-0.024851in}}{\pgfqpoint{-0.007367in}{-0.027778in}}{\pgfqpoint{0.000000in}{-0.027778in}}%
\pgfpathlineto{\pgfqpoint{0.000000in}{-0.027778in}}%
\pgfpathclose%
\pgfusepath{stroke,fill}%
}%
\begin{pgfscope}%
\pgfsys@transformshift{1.342857in}{1.943302in}%
\pgfsys@useobject{currentmarker}{}%
\end{pgfscope}%
\begin{pgfscope}%
\pgfsys@transformshift{1.685714in}{2.512163in}%
\pgfsys@useobject{currentmarker}{}%
\end{pgfscope}%
\begin{pgfscope}%
\pgfsys@transformshift{2.028571in}{2.624157in}%
\pgfsys@useobject{currentmarker}{}%
\end{pgfscope}%
\begin{pgfscope}%
\pgfsys@transformshift{2.371429in}{2.663112in}%
\pgfsys@useobject{currentmarker}{}%
\end{pgfscope}%
\begin{pgfscope}%
\pgfsys@transformshift{2.714286in}{2.799138in}%
\pgfsys@useobject{currentmarker}{}%
\end{pgfscope}%
\begin{pgfscope}%
\pgfsys@transformshift{3.057143in}{2.804653in}%
\pgfsys@useobject{currentmarker}{}%
\end{pgfscope}%
\begin{pgfscope}%
\pgfsys@transformshift{3.400000in}{2.815932in}%
\pgfsys@useobject{currentmarker}{}%
\end{pgfscope}%
\begin{pgfscope}%
\pgfsys@transformshift{3.742857in}{2.969652in}%
\pgfsys@useobject{currentmarker}{}%
\end{pgfscope}%
\begin{pgfscope}%
\pgfsys@transformshift{4.085714in}{2.820098in}%
\pgfsys@useobject{currentmarker}{}%
\end{pgfscope}%
\begin{pgfscope}%
\pgfsys@transformshift{4.428571in}{2.743949in}%
\pgfsys@useobject{currentmarker}{}%
\end{pgfscope}%
\begin{pgfscope}%
\pgfsys@transformshift{4.771429in}{2.875480in}%
\pgfsys@useobject{currentmarker}{}%
\end{pgfscope}%
\begin{pgfscope}%
\pgfsys@transformshift{5.114286in}{2.763949in}%
\pgfsys@useobject{currentmarker}{}%
\end{pgfscope}%
\begin{pgfscope}%
\pgfsys@transformshift{5.457143in}{3.261881in}%
\pgfsys@useobject{currentmarker}{}%
\end{pgfscope}%
\end{pgfscope}%
\begin{pgfscope}%
\pgfsetrectcap%
\pgfsetmiterjoin%
\pgfsetlinewidth{0.803000pt}%
\definecolor{currentstroke}{rgb}{0.000000,0.000000,0.000000}%
\pgfsetstrokecolor{currentstroke}%
\pgfsetdash{}{0pt}%
\pgfpathmoveto{\pgfqpoint{1.000000in}{0.720000in}}%
\pgfpathlineto{\pgfqpoint{1.000000in}{5.340000in}}%
\pgfusepath{stroke}%
\end{pgfscope}%
\begin{pgfscope}%
\pgfsetrectcap%
\pgfsetmiterjoin%
\pgfsetlinewidth{0.803000pt}%
\definecolor{currentstroke}{rgb}{0.000000,0.000000,0.000000}%
\pgfsetstrokecolor{currentstroke}%
\pgfsetdash{}{0pt}%
\pgfpathmoveto{\pgfqpoint{5.800000in}{0.720000in}}%
\pgfpathlineto{\pgfqpoint{5.800000in}{5.340000in}}%
\pgfusepath{stroke}%
\end{pgfscope}%
\begin{pgfscope}%
\pgfsetrectcap%
\pgfsetmiterjoin%
\pgfsetlinewidth{0.803000pt}%
\definecolor{currentstroke}{rgb}{0.000000,0.000000,0.000000}%
\pgfsetstrokecolor{currentstroke}%
\pgfsetdash{}{0pt}%
\pgfpathmoveto{\pgfqpoint{1.000000in}{0.720000in}}%
\pgfpathlineto{\pgfqpoint{5.800000in}{0.720000in}}%
\pgfusepath{stroke}%
\end{pgfscope}%
\begin{pgfscope}%
\pgfsetrectcap%
\pgfsetmiterjoin%
\pgfsetlinewidth{0.803000pt}%
\definecolor{currentstroke}{rgb}{0.000000,0.000000,0.000000}%
\pgfsetstrokecolor{currentstroke}%
\pgfsetdash{}{0pt}%
\pgfpathmoveto{\pgfqpoint{1.000000in}{5.340000in}}%
\pgfpathlineto{\pgfqpoint{5.800000in}{5.340000in}}%
\pgfusepath{stroke}%
\end{pgfscope}%
\begin{pgfscope}%
\pgfsetbuttcap%
\pgfsetmiterjoin%
\definecolor{currentfill}{rgb}{1.000000,1.000000,1.000000}%
\pgfsetfillcolor{currentfill}%
\pgfsetfillopacity{0.800000}%
\pgfsetlinewidth{1.003750pt}%
\definecolor{currentstroke}{rgb}{0.800000,0.800000,0.800000}%
\pgfsetstrokecolor{currentstroke}%
\pgfsetstrokeopacity{0.800000}%
\pgfsetdash{}{0pt}%
\pgfpathmoveto{\pgfqpoint{4.331386in}{4.722821in}}%
\pgfpathlineto{\pgfqpoint{5.605556in}{4.722821in}}%
\pgfpathquadraticcurveto{\pgfqpoint{5.661111in}{4.722821in}}{\pgfqpoint{5.661111in}{4.778377in}}%
\pgfpathlineto{\pgfqpoint{5.661111in}{5.145556in}}%
\pgfpathquadraticcurveto{\pgfqpoint{5.661111in}{5.201111in}}{\pgfqpoint{5.605556in}{5.201111in}}%
\pgfpathlineto{\pgfqpoint{4.331386in}{5.201111in}}%
\pgfpathquadraticcurveto{\pgfqpoint{4.275830in}{5.201111in}}{\pgfqpoint{4.275830in}{5.145556in}}%
\pgfpathlineto{\pgfqpoint{4.275830in}{4.778377in}}%
\pgfpathquadraticcurveto{\pgfqpoint{4.275830in}{4.722821in}}{\pgfqpoint{4.331386in}{4.722821in}}%
\pgfpathlineto{\pgfqpoint{4.331386in}{4.722821in}}%
\pgfpathclose%
\pgfusepath{stroke,fill}%
\end{pgfscope}%
\begin{pgfscope}%
\pgfsetbuttcap%
\pgfsetroundjoin%
\pgfsetlinewidth{1.003750pt}%
\definecolor{currentstroke}{rgb}{0.000000,0.000000,1.000000}%
\pgfsetstrokecolor{currentstroke}%
\pgfsetdash{}{0pt}%
\pgfpathmoveto{\pgfqpoint{4.664719in}{4.848295in}}%
\pgfpathlineto{\pgfqpoint{4.664719in}{5.126073in}}%
\pgfusepath{stroke}%
\end{pgfscope}%
\begin{pgfscope}%
\pgfsetbuttcap%
\pgfsetroundjoin%
\definecolor{currentfill}{rgb}{0.000000,0.000000,1.000000}%
\pgfsetfillcolor{currentfill}%
\pgfsetlinewidth{1.003750pt}%
\definecolor{currentstroke}{rgb}{0.000000,0.000000,1.000000}%
\pgfsetstrokecolor{currentstroke}%
\pgfsetdash{}{0pt}%
\pgfsys@defobject{currentmarker}{\pgfqpoint{-0.041667in}{-0.000000in}}{\pgfqpoint{0.041667in}{0.000000in}}{%
\pgfpathmoveto{\pgfqpoint{0.041667in}{-0.000000in}}%
\pgfpathlineto{\pgfqpoint{-0.041667in}{0.000000in}}%
\pgfusepath{stroke,fill}%
}%
\begin{pgfscope}%
\pgfsys@transformshift{4.664719in}{4.848295in}%
\pgfsys@useobject{currentmarker}{}%
\end{pgfscope}%
\end{pgfscope}%
\begin{pgfscope}%
\pgfsetbuttcap%
\pgfsetroundjoin%
\definecolor{currentfill}{rgb}{0.000000,0.000000,1.000000}%
\pgfsetfillcolor{currentfill}%
\pgfsetlinewidth{1.003750pt}%
\definecolor{currentstroke}{rgb}{0.000000,0.000000,1.000000}%
\pgfsetstrokecolor{currentstroke}%
\pgfsetdash{}{0pt}%
\pgfsys@defobject{currentmarker}{\pgfqpoint{-0.041667in}{-0.000000in}}{\pgfqpoint{0.041667in}{0.000000in}}{%
\pgfpathmoveto{\pgfqpoint{0.041667in}{-0.000000in}}%
\pgfpathlineto{\pgfqpoint{-0.041667in}{0.000000in}}%
\pgfusepath{stroke,fill}%
}%
\begin{pgfscope}%
\pgfsys@transformshift{4.664719in}{5.126073in}%
\pgfsys@useobject{currentmarker}{}%
\end{pgfscope}%
\end{pgfscope}%
\begin{pgfscope}%
\pgfsetbuttcap%
\pgfsetroundjoin%
\definecolor{currentfill}{rgb}{0.000000,0.000000,1.000000}%
\pgfsetfillcolor{currentfill}%
\pgfsetlinewidth{1.003750pt}%
\definecolor{currentstroke}{rgb}{0.000000,0.000000,1.000000}%
\pgfsetstrokecolor{currentstroke}%
\pgfsetdash{}{0pt}%
\pgfsys@defobject{currentmarker}{\pgfqpoint{-0.027778in}{-0.027778in}}{\pgfqpoint{0.027778in}{0.027778in}}{%
\pgfpathmoveto{\pgfqpoint{0.000000in}{-0.027778in}}%
\pgfpathcurveto{\pgfqpoint{0.007367in}{-0.027778in}}{\pgfqpoint{0.014433in}{-0.024851in}}{\pgfqpoint{0.019642in}{-0.019642in}}%
\pgfpathcurveto{\pgfqpoint{0.024851in}{-0.014433in}}{\pgfqpoint{0.027778in}{-0.007367in}}{\pgfqpoint{0.027778in}{0.000000in}}%
\pgfpathcurveto{\pgfqpoint{0.027778in}{0.007367in}}{\pgfqpoint{0.024851in}{0.014433in}}{\pgfqpoint{0.019642in}{0.019642in}}%
\pgfpathcurveto{\pgfqpoint{0.014433in}{0.024851in}}{\pgfqpoint{0.007367in}{0.027778in}}{\pgfqpoint{0.000000in}{0.027778in}}%
\pgfpathcurveto{\pgfqpoint{-0.007367in}{0.027778in}}{\pgfqpoint{-0.014433in}{0.024851in}}{\pgfqpoint{-0.019642in}{0.019642in}}%
\pgfpathcurveto{\pgfqpoint{-0.024851in}{0.014433in}}{\pgfqpoint{-0.027778in}{0.007367in}}{\pgfqpoint{-0.027778in}{0.000000in}}%
\pgfpathcurveto{\pgfqpoint{-0.027778in}{-0.007367in}}{\pgfqpoint{-0.024851in}{-0.014433in}}{\pgfqpoint{-0.019642in}{-0.019642in}}%
\pgfpathcurveto{\pgfqpoint{-0.014433in}{-0.024851in}}{\pgfqpoint{-0.007367in}{-0.027778in}}{\pgfqpoint{0.000000in}{-0.027778in}}%
\pgfpathlineto{\pgfqpoint{0.000000in}{-0.027778in}}%
\pgfpathclose%
\pgfusepath{stroke,fill}%
}%
\begin{pgfscope}%
\pgfsys@transformshift{4.664719in}{4.987184in}%
\pgfsys@useobject{currentmarker}{}%
\end{pgfscope}%
\end{pgfscope}%
\begin{pgfscope}%
\definecolor{textcolor}{rgb}{0.000000,0.000000,0.000000}%
\pgfsetstrokecolor{textcolor}%
\pgfsetfillcolor{textcolor}%
\pgftext[x=5.164719in,y=4.889962in,left,base]{\color{textcolor}\sffamily\fontsize{20.000000}{24.000000}\selectfont \(\displaystyle D_\mathrm{w}\)}%
\end{pgfscope}%
\begin{pgfscope}%
\pgfsetbuttcap%
\pgfsetmiterjoin%
\definecolor{currentfill}{rgb}{1.000000,1.000000,1.000000}%
\pgfsetfillcolor{currentfill}%
\pgfsetlinewidth{0.000000pt}%
\definecolor{currentstroke}{rgb}{0.000000,0.000000,0.000000}%
\pgfsetstrokecolor{currentstroke}%
\pgfsetstrokeopacity{0.000000}%
\pgfsetdash{}{0pt}%
\pgfpathmoveto{\pgfqpoint{5.800000in}{0.720000in}}%
\pgfpathlineto{\pgfqpoint{7.200000in}{0.720000in}}%
\pgfpathlineto{\pgfqpoint{7.200000in}{5.340000in}}%
\pgfpathlineto{\pgfqpoint{5.800000in}{5.340000in}}%
\pgfpathlineto{\pgfqpoint{5.800000in}{0.720000in}}%
\pgfpathclose%
\pgfusepath{fill}%
\end{pgfscope}%
\begin{pgfscope}%
\pgfpathrectangle{\pgfqpoint{5.800000in}{0.720000in}}{\pgfqpoint{1.400000in}{4.620000in}}%
\pgfusepath{clip}%
\pgfsetbuttcap%
\pgfsetmiterjoin%
\definecolor{currentfill}{rgb}{0.121569,0.466667,0.705882}%
\pgfsetfillcolor{currentfill}%
\pgfsetlinewidth{0.000000pt}%
\definecolor{currentstroke}{rgb}{0.000000,0.000000,0.000000}%
\pgfsetstrokecolor{currentstroke}%
\pgfsetstrokeopacity{0.000000}%
\pgfsetdash{}{0pt}%
\pgfpathmoveto{\pgfqpoint{5.800000in}{0.720000in}}%
\pgfpathlineto{\pgfqpoint{5.912975in}{0.720000in}}%
\pgfpathlineto{\pgfqpoint{5.912975in}{0.835500in}}%
\pgfpathlineto{\pgfqpoint{5.800000in}{0.835500in}}%
\pgfpathlineto{\pgfqpoint{5.800000in}{0.720000in}}%
\pgfpathclose%
\pgfusepath{fill}%
\end{pgfscope}%
\begin{pgfscope}%
\pgfpathrectangle{\pgfqpoint{5.800000in}{0.720000in}}{\pgfqpoint{1.400000in}{4.620000in}}%
\pgfusepath{clip}%
\pgfsetbuttcap%
\pgfsetmiterjoin%
\definecolor{currentfill}{rgb}{0.121569,0.466667,0.705882}%
\pgfsetfillcolor{currentfill}%
\pgfsetlinewidth{0.000000pt}%
\definecolor{currentstroke}{rgb}{0.000000,0.000000,0.000000}%
\pgfsetstrokecolor{currentstroke}%
\pgfsetstrokeopacity{0.000000}%
\pgfsetdash{}{0pt}%
\pgfpathmoveto{\pgfqpoint{5.800000in}{0.835500in}}%
\pgfpathlineto{\pgfqpoint{5.924273in}{0.835500in}}%
\pgfpathlineto{\pgfqpoint{5.924273in}{0.951000in}}%
\pgfpathlineto{\pgfqpoint{5.800000in}{0.951000in}}%
\pgfpathlineto{\pgfqpoint{5.800000in}{0.835500in}}%
\pgfpathclose%
\pgfusepath{fill}%
\end{pgfscope}%
\begin{pgfscope}%
\pgfpathrectangle{\pgfqpoint{5.800000in}{0.720000in}}{\pgfqpoint{1.400000in}{4.620000in}}%
\pgfusepath{clip}%
\pgfsetbuttcap%
\pgfsetmiterjoin%
\definecolor{currentfill}{rgb}{0.121569,0.466667,0.705882}%
\pgfsetfillcolor{currentfill}%
\pgfsetlinewidth{0.000000pt}%
\definecolor{currentstroke}{rgb}{0.000000,0.000000,0.000000}%
\pgfsetstrokecolor{currentstroke}%
\pgfsetstrokeopacity{0.000000}%
\pgfsetdash{}{0pt}%
\pgfpathmoveto{\pgfqpoint{5.800000in}{0.951000in}}%
\pgfpathlineto{\pgfqpoint{5.922013in}{0.951000in}}%
\pgfpathlineto{\pgfqpoint{5.922013in}{1.066500in}}%
\pgfpathlineto{\pgfqpoint{5.800000in}{1.066500in}}%
\pgfpathlineto{\pgfqpoint{5.800000in}{0.951000in}}%
\pgfpathclose%
\pgfusepath{fill}%
\end{pgfscope}%
\begin{pgfscope}%
\pgfpathrectangle{\pgfqpoint{5.800000in}{0.720000in}}{\pgfqpoint{1.400000in}{4.620000in}}%
\pgfusepath{clip}%
\pgfsetbuttcap%
\pgfsetmiterjoin%
\definecolor{currentfill}{rgb}{0.121569,0.466667,0.705882}%
\pgfsetfillcolor{currentfill}%
\pgfsetlinewidth{0.000000pt}%
\definecolor{currentstroke}{rgb}{0.000000,0.000000,0.000000}%
\pgfsetstrokecolor{currentstroke}%
\pgfsetstrokeopacity{0.000000}%
\pgfsetdash{}{0pt}%
\pgfpathmoveto{\pgfqpoint{5.800000in}{1.066500in}}%
\pgfpathlineto{\pgfqpoint{5.964944in}{1.066500in}}%
\pgfpathlineto{\pgfqpoint{5.964944in}{1.182000in}}%
\pgfpathlineto{\pgfqpoint{5.800000in}{1.182000in}}%
\pgfpathlineto{\pgfqpoint{5.800000in}{1.066500in}}%
\pgfpathclose%
\pgfusepath{fill}%
\end{pgfscope}%
\begin{pgfscope}%
\pgfpathrectangle{\pgfqpoint{5.800000in}{0.720000in}}{\pgfqpoint{1.400000in}{4.620000in}}%
\pgfusepath{clip}%
\pgfsetbuttcap%
\pgfsetmiterjoin%
\definecolor{currentfill}{rgb}{0.121569,0.466667,0.705882}%
\pgfsetfillcolor{currentfill}%
\pgfsetlinewidth{0.000000pt}%
\definecolor{currentstroke}{rgb}{0.000000,0.000000,0.000000}%
\pgfsetstrokecolor{currentstroke}%
\pgfsetstrokeopacity{0.000000}%
\pgfsetdash{}{0pt}%
\pgfpathmoveto{\pgfqpoint{5.800000in}{1.182000in}}%
\pgfpathlineto{\pgfqpoint{6.021431in}{1.182000in}}%
\pgfpathlineto{\pgfqpoint{6.021431in}{1.297500in}}%
\pgfpathlineto{\pgfqpoint{5.800000in}{1.297500in}}%
\pgfpathlineto{\pgfqpoint{5.800000in}{1.182000in}}%
\pgfpathclose%
\pgfusepath{fill}%
\end{pgfscope}%
\begin{pgfscope}%
\pgfpathrectangle{\pgfqpoint{5.800000in}{0.720000in}}{\pgfqpoint{1.400000in}{4.620000in}}%
\pgfusepath{clip}%
\pgfsetbuttcap%
\pgfsetmiterjoin%
\definecolor{currentfill}{rgb}{0.121569,0.466667,0.705882}%
\pgfsetfillcolor{currentfill}%
\pgfsetlinewidth{0.000000pt}%
\definecolor{currentstroke}{rgb}{0.000000,0.000000,0.000000}%
\pgfsetstrokecolor{currentstroke}%
\pgfsetstrokeopacity{0.000000}%
\pgfsetdash{}{0pt}%
\pgfpathmoveto{\pgfqpoint{5.800000in}{1.297500in}}%
\pgfpathlineto{\pgfqpoint{6.086957in}{1.297500in}}%
\pgfpathlineto{\pgfqpoint{6.086957in}{1.413000in}}%
\pgfpathlineto{\pgfqpoint{5.800000in}{1.413000in}}%
\pgfpathlineto{\pgfqpoint{5.800000in}{1.297500in}}%
\pgfpathclose%
\pgfusepath{fill}%
\end{pgfscope}%
\begin{pgfscope}%
\pgfpathrectangle{\pgfqpoint{5.800000in}{0.720000in}}{\pgfqpoint{1.400000in}{4.620000in}}%
\pgfusepath{clip}%
\pgfsetbuttcap%
\pgfsetmiterjoin%
\definecolor{currentfill}{rgb}{0.121569,0.466667,0.705882}%
\pgfsetfillcolor{currentfill}%
\pgfsetlinewidth{0.000000pt}%
\definecolor{currentstroke}{rgb}{0.000000,0.000000,0.000000}%
\pgfsetstrokecolor{currentstroke}%
\pgfsetstrokeopacity{0.000000}%
\pgfsetdash{}{0pt}%
\pgfpathmoveto{\pgfqpoint{5.800000in}{1.413000in}}%
\pgfpathlineto{\pgfqpoint{6.188635in}{1.413000in}}%
\pgfpathlineto{\pgfqpoint{6.188635in}{1.528500in}}%
\pgfpathlineto{\pgfqpoint{5.800000in}{1.528500in}}%
\pgfpathlineto{\pgfqpoint{5.800000in}{1.413000in}}%
\pgfpathclose%
\pgfusepath{fill}%
\end{pgfscope}%
\begin{pgfscope}%
\pgfpathrectangle{\pgfqpoint{5.800000in}{0.720000in}}{\pgfqpoint{1.400000in}{4.620000in}}%
\pgfusepath{clip}%
\pgfsetbuttcap%
\pgfsetmiterjoin%
\definecolor{currentfill}{rgb}{0.121569,0.466667,0.705882}%
\pgfsetfillcolor{currentfill}%
\pgfsetlinewidth{0.000000pt}%
\definecolor{currentstroke}{rgb}{0.000000,0.000000,0.000000}%
\pgfsetstrokecolor{currentstroke}%
\pgfsetstrokeopacity{0.000000}%
\pgfsetdash{}{0pt}%
\pgfpathmoveto{\pgfqpoint{5.800000in}{1.528500in}}%
\pgfpathlineto{\pgfqpoint{6.227046in}{1.528500in}}%
\pgfpathlineto{\pgfqpoint{6.227046in}{1.644000in}}%
\pgfpathlineto{\pgfqpoint{5.800000in}{1.644000in}}%
\pgfpathlineto{\pgfqpoint{5.800000in}{1.528500in}}%
\pgfpathclose%
\pgfusepath{fill}%
\end{pgfscope}%
\begin{pgfscope}%
\pgfpathrectangle{\pgfqpoint{5.800000in}{0.720000in}}{\pgfqpoint{1.400000in}{4.620000in}}%
\pgfusepath{clip}%
\pgfsetbuttcap%
\pgfsetmiterjoin%
\definecolor{currentfill}{rgb}{0.121569,0.466667,0.705882}%
\pgfsetfillcolor{currentfill}%
\pgfsetlinewidth{0.000000pt}%
\definecolor{currentstroke}{rgb}{0.000000,0.000000,0.000000}%
\pgfsetstrokecolor{currentstroke}%
\pgfsetstrokeopacity{0.000000}%
\pgfsetdash{}{0pt}%
\pgfpathmoveto{\pgfqpoint{5.800000in}{1.644000in}}%
\pgfpathlineto{\pgfqpoint{6.477851in}{1.644000in}}%
\pgfpathlineto{\pgfqpoint{6.477851in}{1.759500in}}%
\pgfpathlineto{\pgfqpoint{5.800000in}{1.759500in}}%
\pgfpathlineto{\pgfqpoint{5.800000in}{1.644000in}}%
\pgfpathclose%
\pgfusepath{fill}%
\end{pgfscope}%
\begin{pgfscope}%
\pgfpathrectangle{\pgfqpoint{5.800000in}{0.720000in}}{\pgfqpoint{1.400000in}{4.620000in}}%
\pgfusepath{clip}%
\pgfsetbuttcap%
\pgfsetmiterjoin%
\definecolor{currentfill}{rgb}{0.121569,0.466667,0.705882}%
\pgfsetfillcolor{currentfill}%
\pgfsetlinewidth{0.000000pt}%
\definecolor{currentstroke}{rgb}{0.000000,0.000000,0.000000}%
\pgfsetstrokecolor{currentstroke}%
\pgfsetstrokeopacity{0.000000}%
\pgfsetdash{}{0pt}%
\pgfpathmoveto{\pgfqpoint{5.800000in}{1.759500in}}%
\pgfpathlineto{\pgfqpoint{6.579529in}{1.759500in}}%
\pgfpathlineto{\pgfqpoint{6.579529in}{1.875000in}}%
\pgfpathlineto{\pgfqpoint{5.800000in}{1.875000in}}%
\pgfpathlineto{\pgfqpoint{5.800000in}{1.759500in}}%
\pgfpathclose%
\pgfusepath{fill}%
\end{pgfscope}%
\begin{pgfscope}%
\pgfpathrectangle{\pgfqpoint{5.800000in}{0.720000in}}{\pgfqpoint{1.400000in}{4.620000in}}%
\pgfusepath{clip}%
\pgfsetbuttcap%
\pgfsetmiterjoin%
\definecolor{currentfill}{rgb}{0.121569,0.466667,0.705882}%
\pgfsetfillcolor{currentfill}%
\pgfsetlinewidth{0.000000pt}%
\definecolor{currentstroke}{rgb}{0.000000,0.000000,0.000000}%
\pgfsetstrokecolor{currentstroke}%
\pgfsetstrokeopacity{0.000000}%
\pgfsetdash{}{0pt}%
\pgfpathmoveto{\pgfqpoint{5.800000in}{1.875000in}}%
\pgfpathlineto{\pgfqpoint{6.755770in}{1.875000in}}%
\pgfpathlineto{\pgfqpoint{6.755770in}{1.990500in}}%
\pgfpathlineto{\pgfqpoint{5.800000in}{1.990500in}}%
\pgfpathlineto{\pgfqpoint{5.800000in}{1.875000in}}%
\pgfpathclose%
\pgfusepath{fill}%
\end{pgfscope}%
\begin{pgfscope}%
\pgfpathrectangle{\pgfqpoint{5.800000in}{0.720000in}}{\pgfqpoint{1.400000in}{4.620000in}}%
\pgfusepath{clip}%
\pgfsetbuttcap%
\pgfsetmiterjoin%
\definecolor{currentfill}{rgb}{0.121569,0.466667,0.705882}%
\pgfsetfillcolor{currentfill}%
\pgfsetlinewidth{0.000000pt}%
\definecolor{currentstroke}{rgb}{0.000000,0.000000,0.000000}%
\pgfsetstrokecolor{currentstroke}%
\pgfsetstrokeopacity{0.000000}%
\pgfsetdash{}{0pt}%
\pgfpathmoveto{\pgfqpoint{5.800000in}{1.990500in}}%
\pgfpathlineto{\pgfqpoint{6.886821in}{1.990500in}}%
\pgfpathlineto{\pgfqpoint{6.886821in}{2.106000in}}%
\pgfpathlineto{\pgfqpoint{5.800000in}{2.106000in}}%
\pgfpathlineto{\pgfqpoint{5.800000in}{1.990500in}}%
\pgfpathclose%
\pgfusepath{fill}%
\end{pgfscope}%
\begin{pgfscope}%
\pgfpathrectangle{\pgfqpoint{5.800000in}{0.720000in}}{\pgfqpoint{1.400000in}{4.620000in}}%
\pgfusepath{clip}%
\pgfsetbuttcap%
\pgfsetmiterjoin%
\definecolor{currentfill}{rgb}{0.121569,0.466667,0.705882}%
\pgfsetfillcolor{currentfill}%
\pgfsetlinewidth{0.000000pt}%
\definecolor{currentstroke}{rgb}{0.000000,0.000000,0.000000}%
\pgfsetstrokecolor{currentstroke}%
\pgfsetstrokeopacity{0.000000}%
\pgfsetdash{}{0pt}%
\pgfpathmoveto{\pgfqpoint{5.800000in}{2.106000in}}%
\pgfpathlineto{\pgfqpoint{6.972683in}{2.106000in}}%
\pgfpathlineto{\pgfqpoint{6.972683in}{2.221500in}}%
\pgfpathlineto{\pgfqpoint{5.800000in}{2.221500in}}%
\pgfpathlineto{\pgfqpoint{5.800000in}{2.106000in}}%
\pgfpathclose%
\pgfusepath{fill}%
\end{pgfscope}%
\begin{pgfscope}%
\pgfpathrectangle{\pgfqpoint{5.800000in}{0.720000in}}{\pgfqpoint{1.400000in}{4.620000in}}%
\pgfusepath{clip}%
\pgfsetbuttcap%
\pgfsetmiterjoin%
\definecolor{currentfill}{rgb}{0.121569,0.466667,0.705882}%
\pgfsetfillcolor{currentfill}%
\pgfsetlinewidth{0.000000pt}%
\definecolor{currentstroke}{rgb}{0.000000,0.000000,0.000000}%
\pgfsetstrokecolor{currentstroke}%
\pgfsetstrokeopacity{0.000000}%
\pgfsetdash{}{0pt}%
\pgfpathmoveto{\pgfqpoint{5.800000in}{2.221500in}}%
\pgfpathlineto{\pgfqpoint{6.986240in}{2.221500in}}%
\pgfpathlineto{\pgfqpoint{6.986240in}{2.337000in}}%
\pgfpathlineto{\pgfqpoint{5.800000in}{2.337000in}}%
\pgfpathlineto{\pgfqpoint{5.800000in}{2.221500in}}%
\pgfpathclose%
\pgfusepath{fill}%
\end{pgfscope}%
\begin{pgfscope}%
\pgfpathrectangle{\pgfqpoint{5.800000in}{0.720000in}}{\pgfqpoint{1.400000in}{4.620000in}}%
\pgfusepath{clip}%
\pgfsetbuttcap%
\pgfsetmiterjoin%
\definecolor{currentfill}{rgb}{0.121569,0.466667,0.705882}%
\pgfsetfillcolor{currentfill}%
\pgfsetlinewidth{0.000000pt}%
\definecolor{currentstroke}{rgb}{0.000000,0.000000,0.000000}%
\pgfsetstrokecolor{currentstroke}%
\pgfsetstrokeopacity{0.000000}%
\pgfsetdash{}{0pt}%
\pgfpathmoveto{\pgfqpoint{5.800000in}{2.337000in}}%
\pgfpathlineto{\pgfqpoint{6.997537in}{2.337000in}}%
\pgfpathlineto{\pgfqpoint{6.997537in}{2.452500in}}%
\pgfpathlineto{\pgfqpoint{5.800000in}{2.452500in}}%
\pgfpathlineto{\pgfqpoint{5.800000in}{2.337000in}}%
\pgfpathclose%
\pgfusepath{fill}%
\end{pgfscope}%
\begin{pgfscope}%
\pgfpathrectangle{\pgfqpoint{5.800000in}{0.720000in}}{\pgfqpoint{1.400000in}{4.620000in}}%
\pgfusepath{clip}%
\pgfsetbuttcap%
\pgfsetmiterjoin%
\definecolor{currentfill}{rgb}{0.121569,0.466667,0.705882}%
\pgfsetfillcolor{currentfill}%
\pgfsetlinewidth{0.000000pt}%
\definecolor{currentstroke}{rgb}{0.000000,0.000000,0.000000}%
\pgfsetstrokecolor{currentstroke}%
\pgfsetstrokeopacity{0.000000}%
\pgfsetdash{}{0pt}%
\pgfpathmoveto{\pgfqpoint{5.800000in}{2.452500in}}%
\pgfpathlineto{\pgfqpoint{7.017873in}{2.452500in}}%
\pgfpathlineto{\pgfqpoint{7.017873in}{2.568000in}}%
\pgfpathlineto{\pgfqpoint{5.800000in}{2.568000in}}%
\pgfpathlineto{\pgfqpoint{5.800000in}{2.452500in}}%
\pgfpathclose%
\pgfusepath{fill}%
\end{pgfscope}%
\begin{pgfscope}%
\pgfpathrectangle{\pgfqpoint{5.800000in}{0.720000in}}{\pgfqpoint{1.400000in}{4.620000in}}%
\pgfusepath{clip}%
\pgfsetbuttcap%
\pgfsetmiterjoin%
\definecolor{currentfill}{rgb}{0.121569,0.466667,0.705882}%
\pgfsetfillcolor{currentfill}%
\pgfsetlinewidth{0.000000pt}%
\definecolor{currentstroke}{rgb}{0.000000,0.000000,0.000000}%
\pgfsetstrokecolor{currentstroke}%
\pgfsetstrokeopacity{0.000000}%
\pgfsetdash{}{0pt}%
\pgfpathmoveto{\pgfqpoint{5.800000in}{2.568000in}}%
\pgfpathlineto{\pgfqpoint{7.069841in}{2.568000in}}%
\pgfpathlineto{\pgfqpoint{7.069841in}{2.683500in}}%
\pgfpathlineto{\pgfqpoint{5.800000in}{2.683500in}}%
\pgfpathlineto{\pgfqpoint{5.800000in}{2.568000in}}%
\pgfpathclose%
\pgfusepath{fill}%
\end{pgfscope}%
\begin{pgfscope}%
\pgfpathrectangle{\pgfqpoint{5.800000in}{0.720000in}}{\pgfqpoint{1.400000in}{4.620000in}}%
\pgfusepath{clip}%
\pgfsetbuttcap%
\pgfsetmiterjoin%
\definecolor{currentfill}{rgb}{0.121569,0.466667,0.705882}%
\pgfsetfillcolor{currentfill}%
\pgfsetlinewidth{0.000000pt}%
\definecolor{currentstroke}{rgb}{0.000000,0.000000,0.000000}%
\pgfsetstrokecolor{currentstroke}%
\pgfsetstrokeopacity{0.000000}%
\pgfsetdash{}{0pt}%
\pgfpathmoveto{\pgfqpoint{5.800000in}{2.683500in}}%
\pgfpathlineto{\pgfqpoint{6.950088in}{2.683500in}}%
\pgfpathlineto{\pgfqpoint{6.950088in}{2.799000in}}%
\pgfpathlineto{\pgfqpoint{5.800000in}{2.799000in}}%
\pgfpathlineto{\pgfqpoint{5.800000in}{2.683500in}}%
\pgfpathclose%
\pgfusepath{fill}%
\end{pgfscope}%
\begin{pgfscope}%
\pgfpathrectangle{\pgfqpoint{5.800000in}{0.720000in}}{\pgfqpoint{1.400000in}{4.620000in}}%
\pgfusepath{clip}%
\pgfsetbuttcap%
\pgfsetmiterjoin%
\definecolor{currentfill}{rgb}{0.121569,0.466667,0.705882}%
\pgfsetfillcolor{currentfill}%
\pgfsetlinewidth{0.000000pt}%
\definecolor{currentstroke}{rgb}{0.000000,0.000000,0.000000}%
\pgfsetstrokecolor{currentstroke}%
\pgfsetstrokeopacity{0.000000}%
\pgfsetdash{}{0pt}%
\pgfpathmoveto{\pgfqpoint{5.800000in}{2.799000in}}%
\pgfpathlineto{\pgfqpoint{6.947828in}{2.799000in}}%
\pgfpathlineto{\pgfqpoint{6.947828in}{2.914500in}}%
\pgfpathlineto{\pgfqpoint{5.800000in}{2.914500in}}%
\pgfpathlineto{\pgfqpoint{5.800000in}{2.799000in}}%
\pgfpathclose%
\pgfusepath{fill}%
\end{pgfscope}%
\begin{pgfscope}%
\pgfpathrectangle{\pgfqpoint{5.800000in}{0.720000in}}{\pgfqpoint{1.400000in}{4.620000in}}%
\pgfusepath{clip}%
\pgfsetbuttcap%
\pgfsetmiterjoin%
\definecolor{currentfill}{rgb}{0.121569,0.466667,0.705882}%
\pgfsetfillcolor{currentfill}%
\pgfsetlinewidth{0.000000pt}%
\definecolor{currentstroke}{rgb}{0.000000,0.000000,0.000000}%
\pgfsetstrokecolor{currentstroke}%
\pgfsetstrokeopacity{0.000000}%
\pgfsetdash{}{0pt}%
\pgfpathmoveto{\pgfqpoint{5.800000in}{2.914500in}}%
\pgfpathlineto{\pgfqpoint{6.841631in}{2.914500in}}%
\pgfpathlineto{\pgfqpoint{6.841631in}{3.030000in}}%
\pgfpathlineto{\pgfqpoint{5.800000in}{3.030000in}}%
\pgfpathlineto{\pgfqpoint{5.800000in}{2.914500in}}%
\pgfpathclose%
\pgfusepath{fill}%
\end{pgfscope}%
\begin{pgfscope}%
\pgfpathrectangle{\pgfqpoint{5.800000in}{0.720000in}}{\pgfqpoint{1.400000in}{4.620000in}}%
\pgfusepath{clip}%
\pgfsetbuttcap%
\pgfsetmiterjoin%
\definecolor{currentfill}{rgb}{0.121569,0.466667,0.705882}%
\pgfsetfillcolor{currentfill}%
\pgfsetlinewidth{0.000000pt}%
\definecolor{currentstroke}{rgb}{0.000000,0.000000,0.000000}%
\pgfsetstrokecolor{currentstroke}%
\pgfsetstrokeopacity{0.000000}%
\pgfsetdash{}{0pt}%
\pgfpathmoveto{\pgfqpoint{5.800000in}{3.030000in}}%
\pgfpathlineto{\pgfqpoint{6.638276in}{3.030000in}}%
\pgfpathlineto{\pgfqpoint{6.638276in}{3.145500in}}%
\pgfpathlineto{\pgfqpoint{5.800000in}{3.145500in}}%
\pgfpathlineto{\pgfqpoint{5.800000in}{3.030000in}}%
\pgfpathclose%
\pgfusepath{fill}%
\end{pgfscope}%
\begin{pgfscope}%
\pgfpathrectangle{\pgfqpoint{5.800000in}{0.720000in}}{\pgfqpoint{1.400000in}{4.620000in}}%
\pgfusepath{clip}%
\pgfsetbuttcap%
\pgfsetmiterjoin%
\definecolor{currentfill}{rgb}{0.121569,0.466667,0.705882}%
\pgfsetfillcolor{currentfill}%
\pgfsetlinewidth{0.000000pt}%
\definecolor{currentstroke}{rgb}{0.000000,0.000000,0.000000}%
\pgfsetstrokecolor{currentstroke}%
\pgfsetstrokeopacity{0.000000}%
\pgfsetdash{}{0pt}%
\pgfpathmoveto{\pgfqpoint{5.800000in}{3.145500in}}%
\pgfpathlineto{\pgfqpoint{6.563712in}{3.145500in}}%
\pgfpathlineto{\pgfqpoint{6.563712in}{3.261000in}}%
\pgfpathlineto{\pgfqpoint{5.800000in}{3.261000in}}%
\pgfpathlineto{\pgfqpoint{5.800000in}{3.145500in}}%
\pgfpathclose%
\pgfusepath{fill}%
\end{pgfscope}%
\begin{pgfscope}%
\pgfpathrectangle{\pgfqpoint{5.800000in}{0.720000in}}{\pgfqpoint{1.400000in}{4.620000in}}%
\pgfusepath{clip}%
\pgfsetbuttcap%
\pgfsetmiterjoin%
\definecolor{currentfill}{rgb}{0.121569,0.466667,0.705882}%
\pgfsetfillcolor{currentfill}%
\pgfsetlinewidth{0.000000pt}%
\definecolor{currentstroke}{rgb}{0.000000,0.000000,0.000000}%
\pgfsetstrokecolor{currentstroke}%
\pgfsetstrokeopacity{0.000000}%
\pgfsetdash{}{0pt}%
\pgfpathmoveto{\pgfqpoint{5.800000in}{3.261000in}}%
\pgfpathlineto{\pgfqpoint{6.552415in}{3.261000in}}%
\pgfpathlineto{\pgfqpoint{6.552415in}{3.376500in}}%
\pgfpathlineto{\pgfqpoint{5.800000in}{3.376500in}}%
\pgfpathlineto{\pgfqpoint{5.800000in}{3.261000in}}%
\pgfpathclose%
\pgfusepath{fill}%
\end{pgfscope}%
\begin{pgfscope}%
\pgfpathrectangle{\pgfqpoint{5.800000in}{0.720000in}}{\pgfqpoint{1.400000in}{4.620000in}}%
\pgfusepath{clip}%
\pgfsetbuttcap%
\pgfsetmiterjoin%
\definecolor{currentfill}{rgb}{0.121569,0.466667,0.705882}%
\pgfsetfillcolor{currentfill}%
\pgfsetlinewidth{0.000000pt}%
\definecolor{currentstroke}{rgb}{0.000000,0.000000,0.000000}%
\pgfsetstrokecolor{currentstroke}%
\pgfsetstrokeopacity{0.000000}%
\pgfsetdash{}{0pt}%
\pgfpathmoveto{\pgfqpoint{5.800000in}{3.376500in}}%
\pgfpathlineto{\pgfqpoint{6.398769in}{3.376500in}}%
\pgfpathlineto{\pgfqpoint{6.398769in}{3.492000in}}%
\pgfpathlineto{\pgfqpoint{5.800000in}{3.492000in}}%
\pgfpathlineto{\pgfqpoint{5.800000in}{3.376500in}}%
\pgfpathclose%
\pgfusepath{fill}%
\end{pgfscope}%
\begin{pgfscope}%
\pgfpathrectangle{\pgfqpoint{5.800000in}{0.720000in}}{\pgfqpoint{1.400000in}{4.620000in}}%
\pgfusepath{clip}%
\pgfsetbuttcap%
\pgfsetmiterjoin%
\definecolor{currentfill}{rgb}{0.121569,0.466667,0.705882}%
\pgfsetfillcolor{currentfill}%
\pgfsetlinewidth{0.000000pt}%
\definecolor{currentstroke}{rgb}{0.000000,0.000000,0.000000}%
\pgfsetstrokecolor{currentstroke}%
\pgfsetstrokeopacity{0.000000}%
\pgfsetdash{}{0pt}%
\pgfpathmoveto{\pgfqpoint{5.800000in}{3.492000in}}%
\pgfpathlineto{\pgfqpoint{6.319686in}{3.492000in}}%
\pgfpathlineto{\pgfqpoint{6.319686in}{3.607500in}}%
\pgfpathlineto{\pgfqpoint{5.800000in}{3.607500in}}%
\pgfpathlineto{\pgfqpoint{5.800000in}{3.492000in}}%
\pgfpathclose%
\pgfusepath{fill}%
\end{pgfscope}%
\begin{pgfscope}%
\pgfpathrectangle{\pgfqpoint{5.800000in}{0.720000in}}{\pgfqpoint{1.400000in}{4.620000in}}%
\pgfusepath{clip}%
\pgfsetbuttcap%
\pgfsetmiterjoin%
\definecolor{currentfill}{rgb}{0.121569,0.466667,0.705882}%
\pgfsetfillcolor{currentfill}%
\pgfsetlinewidth{0.000000pt}%
\definecolor{currentstroke}{rgb}{0.000000,0.000000,0.000000}%
\pgfsetstrokecolor{currentstroke}%
\pgfsetstrokeopacity{0.000000}%
\pgfsetdash{}{0pt}%
\pgfpathmoveto{\pgfqpoint{5.800000in}{3.607500in}}%
\pgfpathlineto{\pgfqpoint{6.301610in}{3.607500in}}%
\pgfpathlineto{\pgfqpoint{6.301610in}{3.723000in}}%
\pgfpathlineto{\pgfqpoint{5.800000in}{3.723000in}}%
\pgfpathlineto{\pgfqpoint{5.800000in}{3.607500in}}%
\pgfpathclose%
\pgfusepath{fill}%
\end{pgfscope}%
\begin{pgfscope}%
\pgfpathrectangle{\pgfqpoint{5.800000in}{0.720000in}}{\pgfqpoint{1.400000in}{4.620000in}}%
\pgfusepath{clip}%
\pgfsetbuttcap%
\pgfsetmiterjoin%
\definecolor{currentfill}{rgb}{0.121569,0.466667,0.705882}%
\pgfsetfillcolor{currentfill}%
\pgfsetlinewidth{0.000000pt}%
\definecolor{currentstroke}{rgb}{0.000000,0.000000,0.000000}%
\pgfsetstrokecolor{currentstroke}%
\pgfsetstrokeopacity{0.000000}%
\pgfsetdash{}{0pt}%
\pgfpathmoveto{\pgfqpoint{5.800000in}{3.723000in}}%
\pgfpathlineto{\pgfqpoint{6.152483in}{3.723000in}}%
\pgfpathlineto{\pgfqpoint{6.152483in}{3.838500in}}%
\pgfpathlineto{\pgfqpoint{5.800000in}{3.838500in}}%
\pgfpathlineto{\pgfqpoint{5.800000in}{3.723000in}}%
\pgfpathclose%
\pgfusepath{fill}%
\end{pgfscope}%
\begin{pgfscope}%
\pgfpathrectangle{\pgfqpoint{5.800000in}{0.720000in}}{\pgfqpoint{1.400000in}{4.620000in}}%
\pgfusepath{clip}%
\pgfsetbuttcap%
\pgfsetmiterjoin%
\definecolor{currentfill}{rgb}{0.121569,0.466667,0.705882}%
\pgfsetfillcolor{currentfill}%
\pgfsetlinewidth{0.000000pt}%
\definecolor{currentstroke}{rgb}{0.000000,0.000000,0.000000}%
\pgfsetstrokecolor{currentstroke}%
\pgfsetstrokeopacity{0.000000}%
\pgfsetdash{}{0pt}%
\pgfpathmoveto{\pgfqpoint{5.800000in}{3.838500in}}%
\pgfpathlineto{\pgfqpoint{6.107293in}{3.838500in}}%
\pgfpathlineto{\pgfqpoint{6.107293in}{3.954000in}}%
\pgfpathlineto{\pgfqpoint{5.800000in}{3.954000in}}%
\pgfpathlineto{\pgfqpoint{5.800000in}{3.838500in}}%
\pgfpathclose%
\pgfusepath{fill}%
\end{pgfscope}%
\begin{pgfscope}%
\pgfpathrectangle{\pgfqpoint{5.800000in}{0.720000in}}{\pgfqpoint{1.400000in}{4.620000in}}%
\pgfusepath{clip}%
\pgfsetbuttcap%
\pgfsetmiterjoin%
\definecolor{currentfill}{rgb}{0.121569,0.466667,0.705882}%
\pgfsetfillcolor{currentfill}%
\pgfsetlinewidth{0.000000pt}%
\definecolor{currentstroke}{rgb}{0.000000,0.000000,0.000000}%
\pgfsetstrokecolor{currentstroke}%
\pgfsetstrokeopacity{0.000000}%
\pgfsetdash{}{0pt}%
\pgfpathmoveto{\pgfqpoint{5.800000in}{3.954000in}}%
\pgfpathlineto{\pgfqpoint{6.086957in}{3.954000in}}%
\pgfpathlineto{\pgfqpoint{6.086957in}{4.069500in}}%
\pgfpathlineto{\pgfqpoint{5.800000in}{4.069500in}}%
\pgfpathlineto{\pgfqpoint{5.800000in}{3.954000in}}%
\pgfpathclose%
\pgfusepath{fill}%
\end{pgfscope}%
\begin{pgfscope}%
\pgfpathrectangle{\pgfqpoint{5.800000in}{0.720000in}}{\pgfqpoint{1.400000in}{4.620000in}}%
\pgfusepath{clip}%
\pgfsetbuttcap%
\pgfsetmiterjoin%
\definecolor{currentfill}{rgb}{0.121569,0.466667,0.705882}%
\pgfsetfillcolor{currentfill}%
\pgfsetlinewidth{0.000000pt}%
\definecolor{currentstroke}{rgb}{0.000000,0.000000,0.000000}%
\pgfsetstrokecolor{currentstroke}%
\pgfsetstrokeopacity{0.000000}%
\pgfsetdash{}{0pt}%
\pgfpathmoveto{\pgfqpoint{5.800000in}{4.069500in}}%
\pgfpathlineto{\pgfqpoint{6.064362in}{4.069500in}}%
\pgfpathlineto{\pgfqpoint{6.064362in}{4.185000in}}%
\pgfpathlineto{\pgfqpoint{5.800000in}{4.185000in}}%
\pgfpathlineto{\pgfqpoint{5.800000in}{4.069500in}}%
\pgfpathclose%
\pgfusepath{fill}%
\end{pgfscope}%
\begin{pgfscope}%
\pgfpathrectangle{\pgfqpoint{5.800000in}{0.720000in}}{\pgfqpoint{1.400000in}{4.620000in}}%
\pgfusepath{clip}%
\pgfsetbuttcap%
\pgfsetmiterjoin%
\definecolor{currentfill}{rgb}{0.121569,0.466667,0.705882}%
\pgfsetfillcolor{currentfill}%
\pgfsetlinewidth{0.000000pt}%
\definecolor{currentstroke}{rgb}{0.000000,0.000000,0.000000}%
\pgfsetstrokecolor{currentstroke}%
\pgfsetstrokeopacity{0.000000}%
\pgfsetdash{}{0pt}%
\pgfpathmoveto{\pgfqpoint{5.800000in}{4.185000in}}%
\pgfpathlineto{\pgfqpoint{5.989798in}{4.185000in}}%
\pgfpathlineto{\pgfqpoint{5.989798in}{4.300500in}}%
\pgfpathlineto{\pgfqpoint{5.800000in}{4.300500in}}%
\pgfpathlineto{\pgfqpoint{5.800000in}{4.185000in}}%
\pgfpathclose%
\pgfusepath{fill}%
\end{pgfscope}%
\begin{pgfscope}%
\pgfpathrectangle{\pgfqpoint{5.800000in}{0.720000in}}{\pgfqpoint{1.400000in}{4.620000in}}%
\pgfusepath{clip}%
\pgfsetbuttcap%
\pgfsetmiterjoin%
\definecolor{currentfill}{rgb}{0.121569,0.466667,0.705882}%
\pgfsetfillcolor{currentfill}%
\pgfsetlinewidth{0.000000pt}%
\definecolor{currentstroke}{rgb}{0.000000,0.000000,0.000000}%
\pgfsetstrokecolor{currentstroke}%
\pgfsetstrokeopacity{0.000000}%
\pgfsetdash{}{0pt}%
\pgfpathmoveto{\pgfqpoint{5.800000in}{4.300500in}}%
\pgfpathlineto{\pgfqpoint{5.978501in}{4.300500in}}%
\pgfpathlineto{\pgfqpoint{5.978501in}{4.416000in}}%
\pgfpathlineto{\pgfqpoint{5.800000in}{4.416000in}}%
\pgfpathlineto{\pgfqpoint{5.800000in}{4.300500in}}%
\pgfpathclose%
\pgfusepath{fill}%
\end{pgfscope}%
\begin{pgfscope}%
\pgfpathrectangle{\pgfqpoint{5.800000in}{0.720000in}}{\pgfqpoint{1.400000in}{4.620000in}}%
\pgfusepath{clip}%
\pgfsetbuttcap%
\pgfsetmiterjoin%
\definecolor{currentfill}{rgb}{0.121569,0.466667,0.705882}%
\pgfsetfillcolor{currentfill}%
\pgfsetlinewidth{0.000000pt}%
\definecolor{currentstroke}{rgb}{0.000000,0.000000,0.000000}%
\pgfsetstrokecolor{currentstroke}%
\pgfsetstrokeopacity{0.000000}%
\pgfsetdash{}{0pt}%
\pgfpathmoveto{\pgfqpoint{5.800000in}{4.416000in}}%
\pgfpathlineto{\pgfqpoint{5.983020in}{4.416000in}}%
\pgfpathlineto{\pgfqpoint{5.983020in}{4.531500in}}%
\pgfpathlineto{\pgfqpoint{5.800000in}{4.531500in}}%
\pgfpathlineto{\pgfqpoint{5.800000in}{4.416000in}}%
\pgfpathclose%
\pgfusepath{fill}%
\end{pgfscope}%
\begin{pgfscope}%
\pgfpathrectangle{\pgfqpoint{5.800000in}{0.720000in}}{\pgfqpoint{1.400000in}{4.620000in}}%
\pgfusepath{clip}%
\pgfsetbuttcap%
\pgfsetmiterjoin%
\definecolor{currentfill}{rgb}{0.121569,0.466667,0.705882}%
\pgfsetfillcolor{currentfill}%
\pgfsetlinewidth{0.000000pt}%
\definecolor{currentstroke}{rgb}{0.000000,0.000000,0.000000}%
\pgfsetstrokecolor{currentstroke}%
\pgfsetstrokeopacity{0.000000}%
\pgfsetdash{}{0pt}%
\pgfpathmoveto{\pgfqpoint{5.800000in}{4.531500in}}%
\pgfpathlineto{\pgfqpoint{5.960425in}{4.531500in}}%
\pgfpathlineto{\pgfqpoint{5.960425in}{4.647000in}}%
\pgfpathlineto{\pgfqpoint{5.800000in}{4.647000in}}%
\pgfpathlineto{\pgfqpoint{5.800000in}{4.531500in}}%
\pgfpathclose%
\pgfusepath{fill}%
\end{pgfscope}%
\begin{pgfscope}%
\pgfpathrectangle{\pgfqpoint{5.800000in}{0.720000in}}{\pgfqpoint{1.400000in}{4.620000in}}%
\pgfusepath{clip}%
\pgfsetbuttcap%
\pgfsetmiterjoin%
\definecolor{currentfill}{rgb}{0.121569,0.466667,0.705882}%
\pgfsetfillcolor{currentfill}%
\pgfsetlinewidth{0.000000pt}%
\definecolor{currentstroke}{rgb}{0.000000,0.000000,0.000000}%
\pgfsetstrokecolor{currentstroke}%
\pgfsetstrokeopacity{0.000000}%
\pgfsetdash{}{0pt}%
\pgfpathmoveto{\pgfqpoint{5.800000in}{4.647000in}}%
\pgfpathlineto{\pgfqpoint{5.958165in}{4.647000in}}%
\pgfpathlineto{\pgfqpoint{5.958165in}{4.762500in}}%
\pgfpathlineto{\pgfqpoint{5.800000in}{4.762500in}}%
\pgfpathlineto{\pgfqpoint{5.800000in}{4.647000in}}%
\pgfpathclose%
\pgfusepath{fill}%
\end{pgfscope}%
\begin{pgfscope}%
\pgfpathrectangle{\pgfqpoint{5.800000in}{0.720000in}}{\pgfqpoint{1.400000in}{4.620000in}}%
\pgfusepath{clip}%
\pgfsetbuttcap%
\pgfsetmiterjoin%
\definecolor{currentfill}{rgb}{0.121569,0.466667,0.705882}%
\pgfsetfillcolor{currentfill}%
\pgfsetlinewidth{0.000000pt}%
\definecolor{currentstroke}{rgb}{0.000000,0.000000,0.000000}%
\pgfsetstrokecolor{currentstroke}%
\pgfsetstrokeopacity{0.000000}%
\pgfsetdash{}{0pt}%
\pgfpathmoveto{\pgfqpoint{5.800000in}{4.762500in}}%
\pgfpathlineto{\pgfqpoint{5.931051in}{4.762500in}}%
\pgfpathlineto{\pgfqpoint{5.931051in}{4.878000in}}%
\pgfpathlineto{\pgfqpoint{5.800000in}{4.878000in}}%
\pgfpathlineto{\pgfqpoint{5.800000in}{4.762500in}}%
\pgfpathclose%
\pgfusepath{fill}%
\end{pgfscope}%
\begin{pgfscope}%
\pgfpathrectangle{\pgfqpoint{5.800000in}{0.720000in}}{\pgfqpoint{1.400000in}{4.620000in}}%
\pgfusepath{clip}%
\pgfsetbuttcap%
\pgfsetmiterjoin%
\definecolor{currentfill}{rgb}{0.121569,0.466667,0.705882}%
\pgfsetfillcolor{currentfill}%
\pgfsetlinewidth{0.000000pt}%
\definecolor{currentstroke}{rgb}{0.000000,0.000000,0.000000}%
\pgfsetstrokecolor{currentstroke}%
\pgfsetstrokeopacity{0.000000}%
\pgfsetdash{}{0pt}%
\pgfpathmoveto{\pgfqpoint{5.800000in}{4.878000in}}%
\pgfpathlineto{\pgfqpoint{5.915235in}{4.878000in}}%
\pgfpathlineto{\pgfqpoint{5.915235in}{4.993500in}}%
\pgfpathlineto{\pgfqpoint{5.800000in}{4.993500in}}%
\pgfpathlineto{\pgfqpoint{5.800000in}{4.878000in}}%
\pgfpathclose%
\pgfusepath{fill}%
\end{pgfscope}%
\begin{pgfscope}%
\pgfpathrectangle{\pgfqpoint{5.800000in}{0.720000in}}{\pgfqpoint{1.400000in}{4.620000in}}%
\pgfusepath{clip}%
\pgfsetbuttcap%
\pgfsetmiterjoin%
\definecolor{currentfill}{rgb}{0.121569,0.466667,0.705882}%
\pgfsetfillcolor{currentfill}%
\pgfsetlinewidth{0.000000pt}%
\definecolor{currentstroke}{rgb}{0.000000,0.000000,0.000000}%
\pgfsetstrokecolor{currentstroke}%
\pgfsetstrokeopacity{0.000000}%
\pgfsetdash{}{0pt}%
\pgfpathmoveto{\pgfqpoint{5.800000in}{4.993500in}}%
\pgfpathlineto{\pgfqpoint{5.903937in}{4.993500in}}%
\pgfpathlineto{\pgfqpoint{5.903937in}{5.109000in}}%
\pgfpathlineto{\pgfqpoint{5.800000in}{5.109000in}}%
\pgfpathlineto{\pgfqpoint{5.800000in}{4.993500in}}%
\pgfpathclose%
\pgfusepath{fill}%
\end{pgfscope}%
\begin{pgfscope}%
\pgfpathrectangle{\pgfqpoint{5.800000in}{0.720000in}}{\pgfqpoint{1.400000in}{4.620000in}}%
\pgfusepath{clip}%
\pgfsetbuttcap%
\pgfsetmiterjoin%
\definecolor{currentfill}{rgb}{0.121569,0.466667,0.705882}%
\pgfsetfillcolor{currentfill}%
\pgfsetlinewidth{0.000000pt}%
\definecolor{currentstroke}{rgb}{0.000000,0.000000,0.000000}%
\pgfsetstrokecolor{currentstroke}%
\pgfsetstrokeopacity{0.000000}%
\pgfsetdash{}{0pt}%
\pgfpathmoveto{\pgfqpoint{5.800000in}{5.109000in}}%
\pgfpathlineto{\pgfqpoint{5.897159in}{5.109000in}}%
\pgfpathlineto{\pgfqpoint{5.897159in}{5.224500in}}%
\pgfpathlineto{\pgfqpoint{5.800000in}{5.224500in}}%
\pgfpathlineto{\pgfqpoint{5.800000in}{5.109000in}}%
\pgfpathclose%
\pgfusepath{fill}%
\end{pgfscope}%
\begin{pgfscope}%
\pgfpathrectangle{\pgfqpoint{5.800000in}{0.720000in}}{\pgfqpoint{1.400000in}{4.620000in}}%
\pgfusepath{clip}%
\pgfsetbuttcap%
\pgfsetmiterjoin%
\definecolor{currentfill}{rgb}{0.121569,0.466667,0.705882}%
\pgfsetfillcolor{currentfill}%
\pgfsetlinewidth{0.000000pt}%
\definecolor{currentstroke}{rgb}{0.000000,0.000000,0.000000}%
\pgfsetstrokecolor{currentstroke}%
\pgfsetstrokeopacity{0.000000}%
\pgfsetdash{}{0pt}%
\pgfpathmoveto{\pgfqpoint{5.800000in}{5.224500in}}%
\pgfpathlineto{\pgfqpoint{5.876823in}{5.224500in}}%
\pgfpathlineto{\pgfqpoint{5.876823in}{5.340000in}}%
\pgfpathlineto{\pgfqpoint{5.800000in}{5.340000in}}%
\pgfpathlineto{\pgfqpoint{5.800000in}{5.224500in}}%
\pgfpathclose%
\pgfusepath{fill}%
\end{pgfscope}%
\begin{pgfscope}%
\definecolor{textcolor}{rgb}{0.000000,0.000000,0.000000}%
\pgfsetstrokecolor{textcolor}%
\pgfsetfillcolor{textcolor}%
\pgftext[x=6.500000in,y=0.664444in,,top]{\color{textcolor}\sffamily\fontsize{20.000000}{24.000000}\selectfont \(\displaystyle \mathrm{arb.\ unit}\)}%
\end{pgfscope}%
\begin{pgfscope}%
\pgfsetrectcap%
\pgfsetmiterjoin%
\pgfsetlinewidth{0.803000pt}%
\definecolor{currentstroke}{rgb}{0.000000,0.000000,0.000000}%
\pgfsetstrokecolor{currentstroke}%
\pgfsetdash{}{0pt}%
\pgfpathmoveto{\pgfqpoint{5.800000in}{0.720000in}}%
\pgfpathlineto{\pgfqpoint{5.800000in}{5.340000in}}%
\pgfusepath{stroke}%
\end{pgfscope}%
\begin{pgfscope}%
\pgfsetrectcap%
\pgfsetmiterjoin%
\pgfsetlinewidth{0.803000pt}%
\definecolor{currentstroke}{rgb}{0.000000,0.000000,0.000000}%
\pgfsetstrokecolor{currentstroke}%
\pgfsetdash{}{0pt}%
\pgfpathmoveto{\pgfqpoint{7.200000in}{0.720000in}}%
\pgfpathlineto{\pgfqpoint{7.200000in}{5.340000in}}%
\pgfusepath{stroke}%
\end{pgfscope}%
\begin{pgfscope}%
\pgfsetrectcap%
\pgfsetmiterjoin%
\pgfsetlinewidth{0.803000pt}%
\definecolor{currentstroke}{rgb}{0.000000,0.000000,0.000000}%
\pgfsetstrokecolor{currentstroke}%
\pgfsetdash{}{0pt}%
\pgfpathmoveto{\pgfqpoint{5.800000in}{0.720000in}}%
\pgfpathlineto{\pgfqpoint{7.200000in}{0.720000in}}%
\pgfusepath{stroke}%
\end{pgfscope}%
\begin{pgfscope}%
\pgfsetrectcap%
\pgfsetmiterjoin%
\pgfsetlinewidth{0.803000pt}%
\definecolor{currentstroke}{rgb}{0.000000,0.000000,0.000000}%
\pgfsetstrokecolor{currentstroke}%
\pgfsetdash{}{0pt}%
\pgfpathmoveto{\pgfqpoint{5.800000in}{5.340000in}}%
\pgfpathlineto{\pgfqpoint{7.200000in}{5.340000in}}%
\pgfusepath{stroke}%
\end{pgfscope}%
\end{pgfpicture}%
\makeatother%
\endgroup%
}
    \caption{\label{fig:mcmc-npe} $D_\mathrm{w}$ histogram and its distributions conditioned \\ on $N_{\mathrm{PE}}$, errorbar explained in figure~\ref{fig:cnn-performance}.}
  \end{subfigure}
  \begin{subfigure}{.5\textwidth}
    \centering
    \resizebox{\textwidth}{!}{%% Creator: Matplotlib, PGF backend
%%
%% To include the figure in your LaTeX document, write
%%   \input{<filename>.pgf}
%%
%% Make sure the required packages are loaded in your preamble
%%   \usepackage{pgf}
%%
%% Also ensure that all the required font packages are loaded; for instance,
%% the lmodern package is sometimes necessary when using math font.
%%   \usepackage{lmodern}
%%
%% Figures using additional raster images can only be included by \input if
%% they are in the same directory as the main LaTeX file. For loading figures
%% from other directories you can use the `import` package
%%   \usepackage{import}
%%
%% and then include the figures with
%%   \import{<path to file>}{<filename>.pgf}
%%
%% Matplotlib used the following preamble
%%   \usepackage[detect-all,locale=DE]{siunitx}
%%
\begingroup%
\makeatletter%
\begin{pgfpicture}%
\pgfpathrectangle{\pgfpointorigin}{\pgfqpoint{8.000000in}{6.000000in}}%
\pgfusepath{use as bounding box, clip}%
\begin{pgfscope}%
\pgfsetbuttcap%
\pgfsetmiterjoin%
\definecolor{currentfill}{rgb}{1.000000,1.000000,1.000000}%
\pgfsetfillcolor{currentfill}%
\pgfsetlinewidth{0.000000pt}%
\definecolor{currentstroke}{rgb}{1.000000,1.000000,1.000000}%
\pgfsetstrokecolor{currentstroke}%
\pgfsetdash{}{0pt}%
\pgfpathmoveto{\pgfqpoint{0.000000in}{0.000000in}}%
\pgfpathlineto{\pgfqpoint{8.000000in}{0.000000in}}%
\pgfpathlineto{\pgfqpoint{8.000000in}{6.000000in}}%
\pgfpathlineto{\pgfqpoint{0.000000in}{6.000000in}}%
\pgfpathlineto{\pgfqpoint{0.000000in}{0.000000in}}%
\pgfpathclose%
\pgfusepath{fill}%
\end{pgfscope}%
\begin{pgfscope}%
\pgfsetbuttcap%
\pgfsetmiterjoin%
\definecolor{currentfill}{rgb}{1.000000,1.000000,1.000000}%
\pgfsetfillcolor{currentfill}%
\pgfsetlinewidth{0.000000pt}%
\definecolor{currentstroke}{rgb}{0.000000,0.000000,0.000000}%
\pgfsetstrokecolor{currentstroke}%
\pgfsetstrokeopacity{0.000000}%
\pgfsetdash{}{0pt}%
\pgfpathmoveto{\pgfqpoint{1.000000in}{0.720000in}}%
\pgfpathlineto{\pgfqpoint{7.200000in}{0.720000in}}%
\pgfpathlineto{\pgfqpoint{7.200000in}{5.340000in}}%
\pgfpathlineto{\pgfqpoint{1.000000in}{5.340000in}}%
\pgfpathlineto{\pgfqpoint{1.000000in}{0.720000in}}%
\pgfpathclose%
\pgfusepath{fill}%
\end{pgfscope}%
\begin{pgfscope}%
\pgfsetbuttcap%
\pgfsetroundjoin%
\definecolor{currentfill}{rgb}{0.000000,0.000000,0.000000}%
\pgfsetfillcolor{currentfill}%
\pgfsetlinewidth{0.803000pt}%
\definecolor{currentstroke}{rgb}{0.000000,0.000000,0.000000}%
\pgfsetstrokecolor{currentstroke}%
\pgfsetdash{}{0pt}%
\pgfsys@defobject{currentmarker}{\pgfqpoint{0.000000in}{-0.048611in}}{\pgfqpoint{0.000000in}{0.000000in}}{%
\pgfpathmoveto{\pgfqpoint{0.000000in}{0.000000in}}%
\pgfpathlineto{\pgfqpoint{0.000000in}{-0.048611in}}%
\pgfusepath{stroke,fill}%
}%
\begin{pgfscope}%
\pgfsys@transformshift{1.310000in}{0.720000in}%
\pgfsys@useobject{currentmarker}{}%
\end{pgfscope}%
\end{pgfscope}%
\begin{pgfscope}%
\definecolor{textcolor}{rgb}{0.000000,0.000000,0.000000}%
\pgfsetstrokecolor{textcolor}%
\pgfsetfillcolor{textcolor}%
\pgftext[x=1.310000in,y=0.622778in,,top]{\color{textcolor}\sffamily\fontsize{20.000000}{24.000000}\selectfont \(\displaystyle {450}\)}%
\end{pgfscope}%
\begin{pgfscope}%
\pgfsetbuttcap%
\pgfsetroundjoin%
\definecolor{currentfill}{rgb}{0.000000,0.000000,0.000000}%
\pgfsetfillcolor{currentfill}%
\pgfsetlinewidth{0.803000pt}%
\definecolor{currentstroke}{rgb}{0.000000,0.000000,0.000000}%
\pgfsetstrokecolor{currentstroke}%
\pgfsetdash{}{0pt}%
\pgfsys@defobject{currentmarker}{\pgfqpoint{0.000000in}{-0.048611in}}{\pgfqpoint{0.000000in}{0.000000in}}{%
\pgfpathmoveto{\pgfqpoint{0.000000in}{0.000000in}}%
\pgfpathlineto{\pgfqpoint{0.000000in}{-0.048611in}}%
\pgfusepath{stroke,fill}%
}%
\begin{pgfscope}%
\pgfsys@transformshift{2.860000in}{0.720000in}%
\pgfsys@useobject{currentmarker}{}%
\end{pgfscope}%
\end{pgfscope}%
\begin{pgfscope}%
\definecolor{textcolor}{rgb}{0.000000,0.000000,0.000000}%
\pgfsetstrokecolor{textcolor}%
\pgfsetfillcolor{textcolor}%
\pgftext[x=2.860000in,y=0.622778in,,top]{\color{textcolor}\sffamily\fontsize{20.000000}{24.000000}\selectfont \(\displaystyle {500}\)}%
\end{pgfscope}%
\begin{pgfscope}%
\pgfsetbuttcap%
\pgfsetroundjoin%
\definecolor{currentfill}{rgb}{0.000000,0.000000,0.000000}%
\pgfsetfillcolor{currentfill}%
\pgfsetlinewidth{0.803000pt}%
\definecolor{currentstroke}{rgb}{0.000000,0.000000,0.000000}%
\pgfsetstrokecolor{currentstroke}%
\pgfsetdash{}{0pt}%
\pgfsys@defobject{currentmarker}{\pgfqpoint{0.000000in}{-0.048611in}}{\pgfqpoint{0.000000in}{0.000000in}}{%
\pgfpathmoveto{\pgfqpoint{0.000000in}{0.000000in}}%
\pgfpathlineto{\pgfqpoint{0.000000in}{-0.048611in}}%
\pgfusepath{stroke,fill}%
}%
\begin{pgfscope}%
\pgfsys@transformshift{4.410000in}{0.720000in}%
\pgfsys@useobject{currentmarker}{}%
\end{pgfscope}%
\end{pgfscope}%
\begin{pgfscope}%
\definecolor{textcolor}{rgb}{0.000000,0.000000,0.000000}%
\pgfsetstrokecolor{textcolor}%
\pgfsetfillcolor{textcolor}%
\pgftext[x=4.410000in,y=0.622778in,,top]{\color{textcolor}\sffamily\fontsize{20.000000}{24.000000}\selectfont \(\displaystyle {550}\)}%
\end{pgfscope}%
\begin{pgfscope}%
\pgfsetbuttcap%
\pgfsetroundjoin%
\definecolor{currentfill}{rgb}{0.000000,0.000000,0.000000}%
\pgfsetfillcolor{currentfill}%
\pgfsetlinewidth{0.803000pt}%
\definecolor{currentstroke}{rgb}{0.000000,0.000000,0.000000}%
\pgfsetstrokecolor{currentstroke}%
\pgfsetdash{}{0pt}%
\pgfsys@defobject{currentmarker}{\pgfqpoint{0.000000in}{-0.048611in}}{\pgfqpoint{0.000000in}{0.000000in}}{%
\pgfpathmoveto{\pgfqpoint{0.000000in}{0.000000in}}%
\pgfpathlineto{\pgfqpoint{0.000000in}{-0.048611in}}%
\pgfusepath{stroke,fill}%
}%
\begin{pgfscope}%
\pgfsys@transformshift{5.960000in}{0.720000in}%
\pgfsys@useobject{currentmarker}{}%
\end{pgfscope}%
\end{pgfscope}%
\begin{pgfscope}%
\definecolor{textcolor}{rgb}{0.000000,0.000000,0.000000}%
\pgfsetstrokecolor{textcolor}%
\pgfsetfillcolor{textcolor}%
\pgftext[x=5.960000in,y=0.622778in,,top]{\color{textcolor}\sffamily\fontsize{20.000000}{24.000000}\selectfont \(\displaystyle {600}\)}%
\end{pgfscope}%
\begin{pgfscope}%
\definecolor{textcolor}{rgb}{0.000000,0.000000,0.000000}%
\pgfsetstrokecolor{textcolor}%
\pgfsetfillcolor{textcolor}%
\pgftext[x=4.100000in,y=0.311155in,,top]{\color{textcolor}\sffamily\fontsize{20.000000}{24.000000}\selectfont \(\displaystyle \mathrm{t}/\si{ns}\)}%
\end{pgfscope}%
\begin{pgfscope}%
\pgfsetbuttcap%
\pgfsetroundjoin%
\definecolor{currentfill}{rgb}{0.000000,0.000000,0.000000}%
\pgfsetfillcolor{currentfill}%
\pgfsetlinewidth{0.803000pt}%
\definecolor{currentstroke}{rgb}{0.000000,0.000000,0.000000}%
\pgfsetstrokecolor{currentstroke}%
\pgfsetdash{}{0pt}%
\pgfsys@defobject{currentmarker}{\pgfqpoint{-0.048611in}{0.000000in}}{\pgfqpoint{-0.000000in}{0.000000in}}{%
\pgfpathmoveto{\pgfqpoint{-0.000000in}{0.000000in}}%
\pgfpathlineto{\pgfqpoint{-0.048611in}{0.000000in}}%
\pgfusepath{stroke,fill}%
}%
\begin{pgfscope}%
\pgfsys@transformshift{1.000000in}{1.138093in}%
\pgfsys@useobject{currentmarker}{}%
\end{pgfscope}%
\end{pgfscope}%
\begin{pgfscope}%
\definecolor{textcolor}{rgb}{0.000000,0.000000,0.000000}%
\pgfsetstrokecolor{textcolor}%
\pgfsetfillcolor{textcolor}%
\pgftext[x=0.770670in, y=1.038074in, left, base]{\color{textcolor}\sffamily\fontsize{20.000000}{24.000000}\selectfont \(\displaystyle {0}\)}%
\end{pgfscope}%
\begin{pgfscope}%
\pgfsetbuttcap%
\pgfsetroundjoin%
\definecolor{currentfill}{rgb}{0.000000,0.000000,0.000000}%
\pgfsetfillcolor{currentfill}%
\pgfsetlinewidth{0.803000pt}%
\definecolor{currentstroke}{rgb}{0.000000,0.000000,0.000000}%
\pgfsetstrokecolor{currentstroke}%
\pgfsetdash{}{0pt}%
\pgfsys@defobject{currentmarker}{\pgfqpoint{-0.048611in}{0.000000in}}{\pgfqpoint{-0.000000in}{0.000000in}}{%
\pgfpathmoveto{\pgfqpoint{-0.000000in}{0.000000in}}%
\pgfpathlineto{\pgfqpoint{-0.048611in}{0.000000in}}%
\pgfusepath{stroke,fill}%
}%
\begin{pgfscope}%
\pgfsys@transformshift{1.000000in}{2.100312in}%
\pgfsys@useobject{currentmarker}{}%
\end{pgfscope}%
\end{pgfscope}%
\begin{pgfscope}%
\definecolor{textcolor}{rgb}{0.000000,0.000000,0.000000}%
\pgfsetstrokecolor{textcolor}%
\pgfsetfillcolor{textcolor}%
\pgftext[x=0.638563in, y=2.000293in, left, base]{\color{textcolor}\sffamily\fontsize{20.000000}{24.000000}\selectfont \(\displaystyle {10}\)}%
\end{pgfscope}%
\begin{pgfscope}%
\pgfsetbuttcap%
\pgfsetroundjoin%
\definecolor{currentfill}{rgb}{0.000000,0.000000,0.000000}%
\pgfsetfillcolor{currentfill}%
\pgfsetlinewidth{0.803000pt}%
\definecolor{currentstroke}{rgb}{0.000000,0.000000,0.000000}%
\pgfsetstrokecolor{currentstroke}%
\pgfsetdash{}{0pt}%
\pgfsys@defobject{currentmarker}{\pgfqpoint{-0.048611in}{0.000000in}}{\pgfqpoint{-0.000000in}{0.000000in}}{%
\pgfpathmoveto{\pgfqpoint{-0.000000in}{0.000000in}}%
\pgfpathlineto{\pgfqpoint{-0.048611in}{0.000000in}}%
\pgfusepath{stroke,fill}%
}%
\begin{pgfscope}%
\pgfsys@transformshift{1.000000in}{3.062531in}%
\pgfsys@useobject{currentmarker}{}%
\end{pgfscope}%
\end{pgfscope}%
\begin{pgfscope}%
\definecolor{textcolor}{rgb}{0.000000,0.000000,0.000000}%
\pgfsetstrokecolor{textcolor}%
\pgfsetfillcolor{textcolor}%
\pgftext[x=0.638563in, y=2.962512in, left, base]{\color{textcolor}\sffamily\fontsize{20.000000}{24.000000}\selectfont \(\displaystyle {20}\)}%
\end{pgfscope}%
\begin{pgfscope}%
\pgfsetbuttcap%
\pgfsetroundjoin%
\definecolor{currentfill}{rgb}{0.000000,0.000000,0.000000}%
\pgfsetfillcolor{currentfill}%
\pgfsetlinewidth{0.803000pt}%
\definecolor{currentstroke}{rgb}{0.000000,0.000000,0.000000}%
\pgfsetstrokecolor{currentstroke}%
\pgfsetdash{}{0pt}%
\pgfsys@defobject{currentmarker}{\pgfqpoint{-0.048611in}{0.000000in}}{\pgfqpoint{-0.000000in}{0.000000in}}{%
\pgfpathmoveto{\pgfqpoint{-0.000000in}{0.000000in}}%
\pgfpathlineto{\pgfqpoint{-0.048611in}{0.000000in}}%
\pgfusepath{stroke,fill}%
}%
\begin{pgfscope}%
\pgfsys@transformshift{1.000000in}{4.024750in}%
\pgfsys@useobject{currentmarker}{}%
\end{pgfscope}%
\end{pgfscope}%
\begin{pgfscope}%
\definecolor{textcolor}{rgb}{0.000000,0.000000,0.000000}%
\pgfsetstrokecolor{textcolor}%
\pgfsetfillcolor{textcolor}%
\pgftext[x=0.638563in, y=3.924731in, left, base]{\color{textcolor}\sffamily\fontsize{20.000000}{24.000000}\selectfont \(\displaystyle {30}\)}%
\end{pgfscope}%
\begin{pgfscope}%
\pgfsetbuttcap%
\pgfsetroundjoin%
\definecolor{currentfill}{rgb}{0.000000,0.000000,0.000000}%
\pgfsetfillcolor{currentfill}%
\pgfsetlinewidth{0.803000pt}%
\definecolor{currentstroke}{rgb}{0.000000,0.000000,0.000000}%
\pgfsetstrokecolor{currentstroke}%
\pgfsetdash{}{0pt}%
\pgfsys@defobject{currentmarker}{\pgfqpoint{-0.048611in}{0.000000in}}{\pgfqpoint{-0.000000in}{0.000000in}}{%
\pgfpathmoveto{\pgfqpoint{-0.000000in}{0.000000in}}%
\pgfpathlineto{\pgfqpoint{-0.048611in}{0.000000in}}%
\pgfusepath{stroke,fill}%
}%
\begin{pgfscope}%
\pgfsys@transformshift{1.000000in}{4.986969in}%
\pgfsys@useobject{currentmarker}{}%
\end{pgfscope}%
\end{pgfscope}%
\begin{pgfscope}%
\definecolor{textcolor}{rgb}{0.000000,0.000000,0.000000}%
\pgfsetstrokecolor{textcolor}%
\pgfsetfillcolor{textcolor}%
\pgftext[x=0.638563in, y=4.886950in, left, base]{\color{textcolor}\sffamily\fontsize{20.000000}{24.000000}\selectfont \(\displaystyle {40}\)}%
\end{pgfscope}%
\begin{pgfscope}%
\definecolor{textcolor}{rgb}{0.000000,0.000000,0.000000}%
\pgfsetstrokecolor{textcolor}%
\pgfsetfillcolor{textcolor}%
\pgftext[x=0.583007in,y=3.030000in,,bottom,rotate=90.000000]{\color{textcolor}\sffamily\fontsize{20.000000}{24.000000}\selectfont \(\displaystyle \mathrm{Voltage}/\si{mV}\)}%
\end{pgfscope}%
\begin{pgfscope}%
\pgfpathrectangle{\pgfqpoint{1.000000in}{0.720000in}}{\pgfqpoint{6.200000in}{4.620000in}}%
\pgfusepath{clip}%
\pgfsetrectcap%
\pgfsetroundjoin%
\pgfsetlinewidth{2.007500pt}%
\definecolor{currentstroke}{rgb}{0.121569,0.466667,0.705882}%
\pgfsetstrokecolor{currentstroke}%
\pgfsetdash{}{0pt}%
\pgfpathmoveto{\pgfqpoint{0.990000in}{1.198200in}}%
\pgfpathlineto{\pgfqpoint{1.000000in}{1.236022in}}%
\pgfpathlineto{\pgfqpoint{1.031000in}{1.099605in}}%
\pgfpathlineto{\pgfqpoint{1.062000in}{1.179883in}}%
\pgfpathlineto{\pgfqpoint{1.093000in}{1.166548in}}%
\pgfpathlineto{\pgfqpoint{1.124000in}{1.151460in}}%
\pgfpathlineto{\pgfqpoint{1.155000in}{1.246779in}}%
\pgfpathlineto{\pgfqpoint{1.186000in}{1.162871in}}%
\pgfpathlineto{\pgfqpoint{1.217000in}{1.095815in}}%
\pgfpathlineto{\pgfqpoint{1.248000in}{1.234109in}}%
\pgfpathlineto{\pgfqpoint{1.279000in}{1.163798in}}%
\pgfpathlineto{\pgfqpoint{1.310000in}{1.081082in}}%
\pgfpathlineto{\pgfqpoint{1.341000in}{1.185899in}}%
\pgfpathlineto{\pgfqpoint{1.372000in}{1.121694in}}%
\pgfpathlineto{\pgfqpoint{1.403000in}{1.349858in}}%
\pgfpathlineto{\pgfqpoint{1.434000in}{1.076658in}}%
\pgfpathlineto{\pgfqpoint{1.465000in}{1.015862in}}%
\pgfpathlineto{\pgfqpoint{1.496000in}{1.098724in}}%
\pgfpathlineto{\pgfqpoint{1.527000in}{1.043464in}}%
\pgfpathlineto{\pgfqpoint{1.558000in}{1.016923in}}%
\pgfpathlineto{\pgfqpoint{1.589000in}{1.129355in}}%
\pgfpathlineto{\pgfqpoint{1.620000in}{1.216799in}}%
\pgfpathlineto{\pgfqpoint{1.651000in}{1.193580in}}%
\pgfpathlineto{\pgfqpoint{1.682000in}{1.305667in}}%
\pgfpathlineto{\pgfqpoint{1.713000in}{1.220771in}}%
\pgfpathlineto{\pgfqpoint{1.744000in}{1.245455in}}%
\pgfpathlineto{\pgfqpoint{1.775000in}{1.259910in}}%
\pgfpathlineto{\pgfqpoint{1.806000in}{1.227172in}}%
\pgfpathlineto{\pgfqpoint{1.837000in}{1.119309in}}%
\pgfpathlineto{\pgfqpoint{1.868000in}{1.020815in}}%
\pgfpathlineto{\pgfqpoint{1.899000in}{1.157585in}}%
\pgfpathlineto{\pgfqpoint{1.930000in}{1.108898in}}%
\pgfpathlineto{\pgfqpoint{1.961000in}{1.222627in}}%
\pgfpathlineto{\pgfqpoint{1.992000in}{1.216911in}}%
\pgfpathlineto{\pgfqpoint{2.023000in}{1.434065in}}%
\pgfpathlineto{\pgfqpoint{2.054000in}{1.794623in}}%
\pgfpathlineto{\pgfqpoint{2.085000in}{2.211711in}}%
\pgfpathlineto{\pgfqpoint{2.116000in}{2.939053in}}%
\pgfpathlineto{\pgfqpoint{2.147000in}{3.122916in}}%
\pgfpathlineto{\pgfqpoint{2.178000in}{3.534362in}}%
\pgfpathlineto{\pgfqpoint{2.209000in}{3.664477in}}%
\pgfpathlineto{\pgfqpoint{2.240000in}{3.424525in}}%
\pgfpathlineto{\pgfqpoint{2.271000in}{3.407418in}}%
\pgfpathlineto{\pgfqpoint{2.302000in}{3.327000in}}%
\pgfpathlineto{\pgfqpoint{2.333000in}{2.824996in}}%
\pgfpathlineto{\pgfqpoint{2.364000in}{2.651454in}}%
\pgfpathlineto{\pgfqpoint{2.395000in}{2.640278in}}%
\pgfpathlineto{\pgfqpoint{2.426000in}{2.347307in}}%
\pgfpathlineto{\pgfqpoint{2.457000in}{1.964857in}}%
\pgfpathlineto{\pgfqpoint{2.488000in}{1.873927in}}%
\pgfpathlineto{\pgfqpoint{2.519000in}{1.791530in}}%
\pgfpathlineto{\pgfqpoint{2.550000in}{1.747096in}}%
\pgfpathlineto{\pgfqpoint{2.581000in}{1.966974in}}%
\pgfpathlineto{\pgfqpoint{2.612000in}{2.547045in}}%
\pgfpathlineto{\pgfqpoint{2.643000in}{2.790263in}}%
\pgfpathlineto{\pgfqpoint{2.674000in}{3.188882in}}%
\pgfpathlineto{\pgfqpoint{2.705000in}{3.279366in}}%
\pgfpathlineto{\pgfqpoint{2.736000in}{3.164397in}}%
\pgfpathlineto{\pgfqpoint{2.767000in}{2.999773in}}%
\pgfpathlineto{\pgfqpoint{2.798000in}{2.856473in}}%
\pgfpathlineto{\pgfqpoint{2.829000in}{2.689253in}}%
\pgfpathlineto{\pgfqpoint{2.860000in}{2.570875in}}%
\pgfpathlineto{\pgfqpoint{2.891000in}{2.287712in}}%
\pgfpathlineto{\pgfqpoint{2.922000in}{2.334664in}}%
\pgfpathlineto{\pgfqpoint{2.953000in}{2.132198in}}%
\pgfpathlineto{\pgfqpoint{2.984000in}{2.657852in}}%
\pgfpathlineto{\pgfqpoint{3.015000in}{2.664440in}}%
\pgfpathlineto{\pgfqpoint{3.046000in}{2.989300in}}%
\pgfpathlineto{\pgfqpoint{3.077000in}{2.859896in}}%
\pgfpathlineto{\pgfqpoint{3.108000in}{2.882247in}}%
\pgfpathlineto{\pgfqpoint{3.139000in}{2.802414in}}%
\pgfpathlineto{\pgfqpoint{3.170000in}{2.537908in}}%
\pgfpathlineto{\pgfqpoint{3.201000in}{2.423999in}}%
\pgfpathlineto{\pgfqpoint{3.232000in}{2.150211in}}%
\pgfpathlineto{\pgfqpoint{3.263000in}{2.005885in}}%
\pgfpathlineto{\pgfqpoint{3.294000in}{1.848422in}}%
\pgfpathlineto{\pgfqpoint{3.325000in}{1.628729in}}%
\pgfpathlineto{\pgfqpoint{3.356000in}{1.751142in}}%
\pgfpathlineto{\pgfqpoint{3.387000in}{1.765880in}}%
\pgfpathlineto{\pgfqpoint{3.418000in}{1.536049in}}%
\pgfpathlineto{\pgfqpoint{3.449000in}{1.333304in}}%
\pgfpathlineto{\pgfqpoint{3.480000in}{1.286616in}}%
\pgfpathlineto{\pgfqpoint{3.511000in}{1.500454in}}%
\pgfpathlineto{\pgfqpoint{3.542000in}{1.268876in}}%
\pgfpathlineto{\pgfqpoint{3.573000in}{1.246592in}}%
\pgfpathlineto{\pgfqpoint{3.604000in}{1.373978in}}%
\pgfpathlineto{\pgfqpoint{3.635000in}{1.204664in}}%
\pgfpathlineto{\pgfqpoint{3.666000in}{1.160858in}}%
\pgfpathlineto{\pgfqpoint{3.697000in}{1.229118in}}%
\pgfpathlineto{\pgfqpoint{3.728000in}{1.154302in}}%
\pgfpathlineto{\pgfqpoint{3.759000in}{1.096888in}}%
\pgfpathlineto{\pgfqpoint{3.790000in}{1.363101in}}%
\pgfpathlineto{\pgfqpoint{3.821000in}{1.461450in}}%
\pgfpathlineto{\pgfqpoint{3.852000in}{1.940584in}}%
\pgfpathlineto{\pgfqpoint{3.883000in}{2.162181in}}%
\pgfpathlineto{\pgfqpoint{3.914000in}{2.627114in}}%
\pgfpathlineto{\pgfqpoint{3.945000in}{2.586245in}}%
\pgfpathlineto{\pgfqpoint{3.976000in}{2.878269in}}%
\pgfpathlineto{\pgfqpoint{4.007000in}{2.748021in}}%
\pgfpathlineto{\pgfqpoint{4.038000in}{2.392908in}}%
\pgfpathlineto{\pgfqpoint{4.069000in}{2.391669in}}%
\pgfpathlineto{\pgfqpoint{4.100000in}{2.265457in}}%
\pgfpathlineto{\pgfqpoint{4.131000in}{2.050187in}}%
\pgfpathlineto{\pgfqpoint{4.162000in}{1.870113in}}%
\pgfpathlineto{\pgfqpoint{4.193000in}{1.807911in}}%
\pgfpathlineto{\pgfqpoint{4.224000in}{1.853158in}}%
\pgfpathlineto{\pgfqpoint{4.255000in}{1.787814in}}%
\pgfpathlineto{\pgfqpoint{4.286000in}{1.480142in}}%
\pgfpathlineto{\pgfqpoint{4.317000in}{1.448926in}}%
\pgfpathlineto{\pgfqpoint{4.348000in}{1.391990in}}%
\pgfpathlineto{\pgfqpoint{4.379000in}{1.310315in}}%
\pgfpathlineto{\pgfqpoint{4.410000in}{1.196321in}}%
\pgfpathlineto{\pgfqpoint{4.441000in}{1.371640in}}%
\pgfpathlineto{\pgfqpoint{4.472000in}{1.265609in}}%
\pgfpathlineto{\pgfqpoint{4.534000in}{1.139418in}}%
\pgfpathlineto{\pgfqpoint{4.565000in}{1.072433in}}%
\pgfpathlineto{\pgfqpoint{4.596000in}{1.292139in}}%
\pgfpathlineto{\pgfqpoint{4.627000in}{1.306758in}}%
\pgfpathlineto{\pgfqpoint{4.658000in}{1.127425in}}%
\pgfpathlineto{\pgfqpoint{4.689000in}{1.251963in}}%
\pgfpathlineto{\pgfqpoint{4.720000in}{0.986281in}}%
\pgfpathlineto{\pgfqpoint{4.751000in}{1.131557in}}%
\pgfpathlineto{\pgfqpoint{4.782000in}{1.154894in}}%
\pgfpathlineto{\pgfqpoint{4.813000in}{1.029853in}}%
\pgfpathlineto{\pgfqpoint{4.844000in}{1.088825in}}%
\pgfpathlineto{\pgfqpoint{4.875000in}{0.860213in}}%
\pgfpathlineto{\pgfqpoint{4.906000in}{1.032026in}}%
\pgfpathlineto{\pgfqpoint{4.937000in}{1.057972in}}%
\pgfpathlineto{\pgfqpoint{4.968000in}{0.957880in}}%
\pgfpathlineto{\pgfqpoint{4.999000in}{1.311552in}}%
\pgfpathlineto{\pgfqpoint{5.030000in}{1.194601in}}%
\pgfpathlineto{\pgfqpoint{5.061000in}{1.132455in}}%
\pgfpathlineto{\pgfqpoint{5.092000in}{1.213178in}}%
\pgfpathlineto{\pgfqpoint{5.123000in}{0.992327in}}%
\pgfpathlineto{\pgfqpoint{5.154000in}{1.079640in}}%
\pgfpathlineto{\pgfqpoint{5.185000in}{1.028729in}}%
\pgfpathlineto{\pgfqpoint{5.216000in}{0.953758in}}%
\pgfpathlineto{\pgfqpoint{5.247000in}{1.128676in}}%
\pgfpathlineto{\pgfqpoint{5.278000in}{1.151393in}}%
\pgfpathlineto{\pgfqpoint{5.309000in}{1.226037in}}%
\pgfpathlineto{\pgfqpoint{5.371000in}{1.116496in}}%
\pgfpathlineto{\pgfqpoint{5.402000in}{1.190882in}}%
\pgfpathlineto{\pgfqpoint{5.433000in}{1.214735in}}%
\pgfpathlineto{\pgfqpoint{5.464000in}{1.076841in}}%
\pgfpathlineto{\pgfqpoint{5.495000in}{1.047499in}}%
\pgfpathlineto{\pgfqpoint{5.526000in}{1.077993in}}%
\pgfpathlineto{\pgfqpoint{5.557000in}{1.110193in}}%
\pgfpathlineto{\pgfqpoint{5.588000in}{1.039882in}}%
\pgfpathlineto{\pgfqpoint{5.619000in}{0.942549in}}%
\pgfpathlineto{\pgfqpoint{5.650000in}{1.085781in}}%
\pgfpathlineto{\pgfqpoint{5.681000in}{1.089873in}}%
\pgfpathlineto{\pgfqpoint{5.712000in}{1.113993in}}%
\pgfpathlineto{\pgfqpoint{5.743000in}{1.157347in}}%
\pgfpathlineto{\pgfqpoint{5.774000in}{1.280928in}}%
\pgfpathlineto{\pgfqpoint{5.805000in}{1.127749in}}%
\pgfpathlineto{\pgfqpoint{5.836000in}{1.064021in}}%
\pgfpathlineto{\pgfqpoint{5.867000in}{1.091651in}}%
\pgfpathlineto{\pgfqpoint{5.898000in}{1.251217in}}%
\pgfpathlineto{\pgfqpoint{5.929000in}{1.240969in}}%
\pgfpathlineto{\pgfqpoint{5.960000in}{1.083470in}}%
\pgfpathlineto{\pgfqpoint{5.991000in}{1.051458in}}%
\pgfpathlineto{\pgfqpoint{6.022000in}{1.107580in}}%
\pgfpathlineto{\pgfqpoint{6.053000in}{1.030816in}}%
\pgfpathlineto{\pgfqpoint{6.084000in}{1.055144in}}%
\pgfpathlineto{\pgfqpoint{6.115000in}{1.118555in}}%
\pgfpathlineto{\pgfqpoint{6.146000in}{1.154678in}}%
\pgfpathlineto{\pgfqpoint{6.177000in}{1.101468in}}%
\pgfpathlineto{\pgfqpoint{6.208000in}{1.195512in}}%
\pgfpathlineto{\pgfqpoint{6.239000in}{1.047711in}}%
\pgfpathlineto{\pgfqpoint{6.270000in}{1.222265in}}%
\pgfpathlineto{\pgfqpoint{6.301000in}{1.008722in}}%
\pgfpathlineto{\pgfqpoint{6.332000in}{1.070775in}}%
\pgfpathlineto{\pgfqpoint{6.363000in}{1.090151in}}%
\pgfpathlineto{\pgfqpoint{6.394000in}{1.083919in}}%
\pgfpathlineto{\pgfqpoint{6.425000in}{1.287700in}}%
\pgfpathlineto{\pgfqpoint{6.456000in}{1.153262in}}%
\pgfpathlineto{\pgfqpoint{6.487000in}{1.236077in}}%
\pgfpathlineto{\pgfqpoint{6.518000in}{1.187929in}}%
\pgfpathlineto{\pgfqpoint{6.549000in}{1.195589in}}%
\pgfpathlineto{\pgfqpoint{6.580000in}{1.148104in}}%
\pgfpathlineto{\pgfqpoint{6.611000in}{1.055576in}}%
\pgfpathlineto{\pgfqpoint{6.642000in}{1.198618in}}%
\pgfpathlineto{\pgfqpoint{6.673000in}{1.101882in}}%
\pgfpathlineto{\pgfqpoint{6.704000in}{0.975630in}}%
\pgfpathlineto{\pgfqpoint{6.735000in}{1.135182in}}%
\pgfpathlineto{\pgfqpoint{6.766000in}{1.228368in}}%
\pgfpathlineto{\pgfqpoint{6.797000in}{1.198492in}}%
\pgfpathlineto{\pgfqpoint{6.828000in}{1.188794in}}%
\pgfpathlineto{\pgfqpoint{6.859000in}{0.966069in}}%
\pgfpathlineto{\pgfqpoint{6.890000in}{1.148773in}}%
\pgfpathlineto{\pgfqpoint{6.921000in}{0.907220in}}%
\pgfpathlineto{\pgfqpoint{6.952000in}{1.148749in}}%
\pgfpathlineto{\pgfqpoint{6.983000in}{1.018991in}}%
\pgfpathlineto{\pgfqpoint{7.014000in}{1.127595in}}%
\pgfpathlineto{\pgfqpoint{7.045000in}{1.155254in}}%
\pgfpathlineto{\pgfqpoint{7.076000in}{1.066620in}}%
\pgfpathlineto{\pgfqpoint{7.107000in}{1.054569in}}%
\pgfpathlineto{\pgfqpoint{7.138000in}{1.201487in}}%
\pgfpathlineto{\pgfqpoint{7.169000in}{1.313851in}}%
\pgfpathlineto{\pgfqpoint{7.200000in}{1.200376in}}%
\pgfpathlineto{\pgfqpoint{7.210000in}{1.133797in}}%
\pgfpathlineto{\pgfqpoint{7.210000in}{1.133797in}}%
\pgfusepath{stroke}%
\end{pgfscope}%
\begin{pgfscope}%
\pgfpathrectangle{\pgfqpoint{1.000000in}{0.720000in}}{\pgfqpoint{6.200000in}{4.620000in}}%
\pgfusepath{clip}%
\pgfsetbuttcap%
\pgfsetroundjoin%
\pgfsetlinewidth{2.007500pt}%
\definecolor{currentstroke}{rgb}{0.000000,0.500000,0.000000}%
\pgfsetstrokecolor{currentstroke}%
\pgfsetdash{}{0pt}%
\pgfpathmoveto{\pgfqpoint{0.990000in}{1.619202in}}%
\pgfpathlineto{\pgfqpoint{7.210000in}{1.619202in}}%
\pgfusepath{stroke}%
\end{pgfscope}%
\begin{pgfscope}%
\pgfsetrectcap%
\pgfsetmiterjoin%
\pgfsetlinewidth{0.803000pt}%
\definecolor{currentstroke}{rgb}{0.000000,0.000000,0.000000}%
\pgfsetstrokecolor{currentstroke}%
\pgfsetdash{}{0pt}%
\pgfpathmoveto{\pgfqpoint{1.000000in}{0.720000in}}%
\pgfpathlineto{\pgfqpoint{1.000000in}{5.340000in}}%
\pgfusepath{stroke}%
\end{pgfscope}%
\begin{pgfscope}%
\pgfsetrectcap%
\pgfsetmiterjoin%
\pgfsetlinewidth{0.803000pt}%
\definecolor{currentstroke}{rgb}{0.000000,0.000000,0.000000}%
\pgfsetstrokecolor{currentstroke}%
\pgfsetdash{}{0pt}%
\pgfpathmoveto{\pgfqpoint{7.200000in}{0.720000in}}%
\pgfpathlineto{\pgfqpoint{7.200000in}{5.340000in}}%
\pgfusepath{stroke}%
\end{pgfscope}%
\begin{pgfscope}%
\pgfsetrectcap%
\pgfsetmiterjoin%
\pgfsetlinewidth{0.803000pt}%
\definecolor{currentstroke}{rgb}{0.000000,0.000000,0.000000}%
\pgfsetstrokecolor{currentstroke}%
\pgfsetdash{}{0pt}%
\pgfpathmoveto{\pgfqpoint{1.000000in}{0.720000in}}%
\pgfpathlineto{\pgfqpoint{7.200000in}{0.720000in}}%
\pgfusepath{stroke}%
\end{pgfscope}%
\begin{pgfscope}%
\pgfsetrectcap%
\pgfsetmiterjoin%
\pgfsetlinewidth{0.803000pt}%
\definecolor{currentstroke}{rgb}{0.000000,0.000000,0.000000}%
\pgfsetstrokecolor{currentstroke}%
\pgfsetdash{}{0pt}%
\pgfpathmoveto{\pgfqpoint{1.000000in}{5.340000in}}%
\pgfpathlineto{\pgfqpoint{7.200000in}{5.340000in}}%
\pgfusepath{stroke}%
\end{pgfscope}%
\begin{pgfscope}%
\pgfsetbuttcap%
\pgfsetroundjoin%
\definecolor{currentfill}{rgb}{0.000000,0.000000,0.000000}%
\pgfsetfillcolor{currentfill}%
\pgfsetlinewidth{0.803000pt}%
\definecolor{currentstroke}{rgb}{0.000000,0.000000,0.000000}%
\pgfsetstrokecolor{currentstroke}%
\pgfsetdash{}{0pt}%
\pgfsys@defobject{currentmarker}{\pgfqpoint{0.000000in}{0.000000in}}{\pgfqpoint{0.048611in}{0.000000in}}{%
\pgfpathmoveto{\pgfqpoint{0.000000in}{0.000000in}}%
\pgfpathlineto{\pgfqpoint{0.048611in}{0.000000in}}%
\pgfusepath{stroke,fill}%
}%
\begin{pgfscope}%
\pgfsys@transformshift{7.200000in}{1.138093in}%
\pgfsys@useobject{currentmarker}{}%
\end{pgfscope}%
\end{pgfscope}%
\begin{pgfscope}%
\definecolor{textcolor}{rgb}{0.000000,0.000000,0.000000}%
\pgfsetstrokecolor{textcolor}%
\pgfsetfillcolor{textcolor}%
\pgftext[x=7.297222in, y=1.038074in, left, base]{\color{textcolor}\sffamily\fontsize{20.000000}{24.000000}\selectfont 0.0}%
\end{pgfscope}%
\begin{pgfscope}%
\pgfsetbuttcap%
\pgfsetroundjoin%
\definecolor{currentfill}{rgb}{0.000000,0.000000,0.000000}%
\pgfsetfillcolor{currentfill}%
\pgfsetlinewidth{0.803000pt}%
\definecolor{currentstroke}{rgb}{0.000000,0.000000,0.000000}%
\pgfsetstrokecolor{currentstroke}%
\pgfsetdash{}{0pt}%
\pgfsys@defobject{currentmarker}{\pgfqpoint{0.000000in}{0.000000in}}{\pgfqpoint{0.048611in}{0.000000in}}{%
\pgfpathmoveto{\pgfqpoint{0.000000in}{0.000000in}}%
\pgfpathlineto{\pgfqpoint{0.048611in}{0.000000in}}%
\pgfusepath{stroke,fill}%
}%
\begin{pgfscope}%
\pgfsys@transformshift{7.200000in}{1.855989in}%
\pgfsys@useobject{currentmarker}{}%
\end{pgfscope}%
\end{pgfscope}%
\begin{pgfscope}%
\definecolor{textcolor}{rgb}{0.000000,0.000000,0.000000}%
\pgfsetstrokecolor{textcolor}%
\pgfsetfillcolor{textcolor}%
\pgftext[x=7.297222in, y=1.755970in, left, base]{\color{textcolor}\sffamily\fontsize{20.000000}{24.000000}\selectfont 0.2}%
\end{pgfscope}%
\begin{pgfscope}%
\pgfsetbuttcap%
\pgfsetroundjoin%
\definecolor{currentfill}{rgb}{0.000000,0.000000,0.000000}%
\pgfsetfillcolor{currentfill}%
\pgfsetlinewidth{0.803000pt}%
\definecolor{currentstroke}{rgb}{0.000000,0.000000,0.000000}%
\pgfsetstrokecolor{currentstroke}%
\pgfsetdash{}{0pt}%
\pgfsys@defobject{currentmarker}{\pgfqpoint{0.000000in}{0.000000in}}{\pgfqpoint{0.048611in}{0.000000in}}{%
\pgfpathmoveto{\pgfqpoint{0.000000in}{0.000000in}}%
\pgfpathlineto{\pgfqpoint{0.048611in}{0.000000in}}%
\pgfusepath{stroke,fill}%
}%
\begin{pgfscope}%
\pgfsys@transformshift{7.200000in}{2.573885in}%
\pgfsys@useobject{currentmarker}{}%
\end{pgfscope}%
\end{pgfscope}%
\begin{pgfscope}%
\definecolor{textcolor}{rgb}{0.000000,0.000000,0.000000}%
\pgfsetstrokecolor{textcolor}%
\pgfsetfillcolor{textcolor}%
\pgftext[x=7.297222in, y=2.473866in, left, base]{\color{textcolor}\sffamily\fontsize{20.000000}{24.000000}\selectfont 0.5}%
\end{pgfscope}%
\begin{pgfscope}%
\pgfsetbuttcap%
\pgfsetroundjoin%
\definecolor{currentfill}{rgb}{0.000000,0.000000,0.000000}%
\pgfsetfillcolor{currentfill}%
\pgfsetlinewidth{0.803000pt}%
\definecolor{currentstroke}{rgb}{0.000000,0.000000,0.000000}%
\pgfsetstrokecolor{currentstroke}%
\pgfsetdash{}{0pt}%
\pgfsys@defobject{currentmarker}{\pgfqpoint{0.000000in}{0.000000in}}{\pgfqpoint{0.048611in}{0.000000in}}{%
\pgfpathmoveto{\pgfqpoint{0.000000in}{0.000000in}}%
\pgfpathlineto{\pgfqpoint{0.048611in}{0.000000in}}%
\pgfusepath{stroke,fill}%
}%
\begin{pgfscope}%
\pgfsys@transformshift{7.200000in}{3.291781in}%
\pgfsys@useobject{currentmarker}{}%
\end{pgfscope}%
\end{pgfscope}%
\begin{pgfscope}%
\definecolor{textcolor}{rgb}{0.000000,0.000000,0.000000}%
\pgfsetstrokecolor{textcolor}%
\pgfsetfillcolor{textcolor}%
\pgftext[x=7.297222in, y=3.191762in, left, base]{\color{textcolor}\sffamily\fontsize{20.000000}{24.000000}\selectfont 0.8}%
\end{pgfscope}%
\begin{pgfscope}%
\pgfsetbuttcap%
\pgfsetroundjoin%
\definecolor{currentfill}{rgb}{0.000000,0.000000,0.000000}%
\pgfsetfillcolor{currentfill}%
\pgfsetlinewidth{0.803000pt}%
\definecolor{currentstroke}{rgb}{0.000000,0.000000,0.000000}%
\pgfsetstrokecolor{currentstroke}%
\pgfsetdash{}{0pt}%
\pgfsys@defobject{currentmarker}{\pgfqpoint{0.000000in}{0.000000in}}{\pgfqpoint{0.048611in}{0.000000in}}{%
\pgfpathmoveto{\pgfqpoint{0.000000in}{0.000000in}}%
\pgfpathlineto{\pgfqpoint{0.048611in}{0.000000in}}%
\pgfusepath{stroke,fill}%
}%
\begin{pgfscope}%
\pgfsys@transformshift{7.200000in}{4.009677in}%
\pgfsys@useobject{currentmarker}{}%
\end{pgfscope}%
\end{pgfscope}%
\begin{pgfscope}%
\definecolor{textcolor}{rgb}{0.000000,0.000000,0.000000}%
\pgfsetstrokecolor{textcolor}%
\pgfsetfillcolor{textcolor}%
\pgftext[x=7.297222in, y=3.909658in, left, base]{\color{textcolor}\sffamily\fontsize{20.000000}{24.000000}\selectfont 1.0}%
\end{pgfscope}%
\begin{pgfscope}%
\pgfsetbuttcap%
\pgfsetroundjoin%
\definecolor{currentfill}{rgb}{0.000000,0.000000,0.000000}%
\pgfsetfillcolor{currentfill}%
\pgfsetlinewidth{0.803000pt}%
\definecolor{currentstroke}{rgb}{0.000000,0.000000,0.000000}%
\pgfsetstrokecolor{currentstroke}%
\pgfsetdash{}{0pt}%
\pgfsys@defobject{currentmarker}{\pgfqpoint{0.000000in}{0.000000in}}{\pgfqpoint{0.048611in}{0.000000in}}{%
\pgfpathmoveto{\pgfqpoint{0.000000in}{0.000000in}}%
\pgfpathlineto{\pgfqpoint{0.048611in}{0.000000in}}%
\pgfusepath{stroke,fill}%
}%
\begin{pgfscope}%
\pgfsys@transformshift{7.200000in}{4.727573in}%
\pgfsys@useobject{currentmarker}{}%
\end{pgfscope}%
\end{pgfscope}%
\begin{pgfscope}%
\definecolor{textcolor}{rgb}{0.000000,0.000000,0.000000}%
\pgfsetstrokecolor{textcolor}%
\pgfsetfillcolor{textcolor}%
\pgftext[x=7.297222in, y=4.627554in, left, base]{\color{textcolor}\sffamily\fontsize{20.000000}{24.000000}\selectfont 1.2}%
\end{pgfscope}%
\begin{pgfscope}%
\definecolor{textcolor}{rgb}{0.000000,0.000000,0.000000}%
\pgfsetstrokecolor{textcolor}%
\pgfsetfillcolor{textcolor}%
\pgftext[x=7.698906in,y=3.030000in,,top,rotate=90.000000]{\color{textcolor}\sffamily\fontsize{20.000000}{24.000000}\selectfont \(\displaystyle \mathrm{Charge}\)}%
\end{pgfscope}%
\begin{pgfscope}%
\pgfpathrectangle{\pgfqpoint{1.000000in}{0.720000in}}{\pgfqpoint{6.200000in}{4.620000in}}%
\pgfusepath{clip}%
\pgfsetbuttcap%
\pgfsetroundjoin%
\pgfsetlinewidth{0.501875pt}%
\definecolor{currentstroke}{rgb}{1.000000,0.000000,0.000000}%
\pgfsetstrokecolor{currentstroke}%
\pgfsetdash{}{0pt}%
\pgfpathmoveto{\pgfqpoint{1.775000in}{1.138093in}}%
\pgfpathlineto{\pgfqpoint{1.775000in}{1.138209in}}%
\pgfusepath{stroke}%
\end{pgfscope}%
\begin{pgfscope}%
\pgfpathrectangle{\pgfqpoint{1.000000in}{0.720000in}}{\pgfqpoint{6.200000in}{4.620000in}}%
\pgfusepath{clip}%
\pgfsetbuttcap%
\pgfsetroundjoin%
\pgfsetlinewidth{0.501875pt}%
\definecolor{currentstroke}{rgb}{1.000000,0.000000,0.000000}%
\pgfsetstrokecolor{currentstroke}%
\pgfsetdash{}{0pt}%
\pgfpathmoveto{\pgfqpoint{1.837000in}{1.138093in}}%
\pgfpathlineto{\pgfqpoint{1.837000in}{1.138097in}}%
\pgfusepath{stroke}%
\end{pgfscope}%
\begin{pgfscope}%
\pgfpathrectangle{\pgfqpoint{1.000000in}{0.720000in}}{\pgfqpoint{6.200000in}{4.620000in}}%
\pgfusepath{clip}%
\pgfsetbuttcap%
\pgfsetroundjoin%
\pgfsetlinewidth{0.501875pt}%
\definecolor{currentstroke}{rgb}{1.000000,0.000000,0.000000}%
\pgfsetstrokecolor{currentstroke}%
\pgfsetdash{}{0pt}%
\pgfpathmoveto{\pgfqpoint{1.883500in}{1.138093in}}%
\pgfpathlineto{\pgfqpoint{1.883500in}{2.592458in}}%
\pgfusepath{stroke}%
\end{pgfscope}%
\begin{pgfscope}%
\pgfpathrectangle{\pgfqpoint{1.000000in}{0.720000in}}{\pgfqpoint{6.200000in}{4.620000in}}%
\pgfusepath{clip}%
\pgfsetbuttcap%
\pgfsetroundjoin%
\pgfsetlinewidth{0.501875pt}%
\definecolor{currentstroke}{rgb}{1.000000,0.000000,0.000000}%
\pgfsetstrokecolor{currentstroke}%
\pgfsetdash{}{0pt}%
\pgfpathmoveto{\pgfqpoint{1.914500in}{1.138093in}}%
\pgfpathlineto{\pgfqpoint{1.914500in}{1.138110in}}%
\pgfusepath{stroke}%
\end{pgfscope}%
\begin{pgfscope}%
\pgfpathrectangle{\pgfqpoint{1.000000in}{0.720000in}}{\pgfqpoint{6.200000in}{4.620000in}}%
\pgfusepath{clip}%
\pgfsetbuttcap%
\pgfsetroundjoin%
\pgfsetlinewidth{0.501875pt}%
\definecolor{currentstroke}{rgb}{1.000000,0.000000,0.000000}%
\pgfsetstrokecolor{currentstroke}%
\pgfsetdash{}{0pt}%
\pgfpathmoveto{\pgfqpoint{1.976500in}{1.138093in}}%
\pgfpathlineto{\pgfqpoint{1.976500in}{5.139909in}}%
\pgfusepath{stroke}%
\end{pgfscope}%
\begin{pgfscope}%
\pgfpathrectangle{\pgfqpoint{1.000000in}{0.720000in}}{\pgfqpoint{6.200000in}{4.620000in}}%
\pgfusepath{clip}%
\pgfsetbuttcap%
\pgfsetroundjoin%
\pgfsetlinewidth{0.501875pt}%
\definecolor{currentstroke}{rgb}{1.000000,0.000000,0.000000}%
\pgfsetstrokecolor{currentstroke}%
\pgfsetdash{}{0pt}%
\pgfpathmoveto{\pgfqpoint{1.992000in}{1.138093in}}%
\pgfpathlineto{\pgfqpoint{1.992000in}{1.138160in}}%
\pgfusepath{stroke}%
\end{pgfscope}%
\begin{pgfscope}%
\pgfpathrectangle{\pgfqpoint{1.000000in}{0.720000in}}{\pgfqpoint{6.200000in}{4.620000in}}%
\pgfusepath{clip}%
\pgfsetbuttcap%
\pgfsetroundjoin%
\pgfsetlinewidth{0.501875pt}%
\definecolor{currentstroke}{rgb}{1.000000,0.000000,0.000000}%
\pgfsetstrokecolor{currentstroke}%
\pgfsetdash{}{0pt}%
\pgfpathmoveto{\pgfqpoint{2.007500in}{1.138093in}}%
\pgfpathlineto{\pgfqpoint{2.007500in}{1.138166in}}%
\pgfusepath{stroke}%
\end{pgfscope}%
\begin{pgfscope}%
\pgfpathrectangle{\pgfqpoint{1.000000in}{0.720000in}}{\pgfqpoint{6.200000in}{4.620000in}}%
\pgfusepath{clip}%
\pgfsetbuttcap%
\pgfsetroundjoin%
\pgfsetlinewidth{0.501875pt}%
\definecolor{currentstroke}{rgb}{1.000000,0.000000,0.000000}%
\pgfsetstrokecolor{currentstroke}%
\pgfsetdash{}{0pt}%
\pgfpathmoveto{\pgfqpoint{2.023000in}{1.138093in}}%
\pgfpathlineto{\pgfqpoint{2.023000in}{1.138169in}}%
\pgfusepath{stroke}%
\end{pgfscope}%
\begin{pgfscope}%
\pgfpathrectangle{\pgfqpoint{1.000000in}{0.720000in}}{\pgfqpoint{6.200000in}{4.620000in}}%
\pgfusepath{clip}%
\pgfsetbuttcap%
\pgfsetroundjoin%
\pgfsetlinewidth{0.501875pt}%
\definecolor{currentstroke}{rgb}{1.000000,0.000000,0.000000}%
\pgfsetstrokecolor{currentstroke}%
\pgfsetdash{}{0pt}%
\pgfpathmoveto{\pgfqpoint{2.054000in}{1.138093in}}%
\pgfpathlineto{\pgfqpoint{2.054000in}{1.138295in}}%
\pgfusepath{stroke}%
\end{pgfscope}%
\begin{pgfscope}%
\pgfpathrectangle{\pgfqpoint{1.000000in}{0.720000in}}{\pgfqpoint{6.200000in}{4.620000in}}%
\pgfusepath{clip}%
\pgfsetbuttcap%
\pgfsetroundjoin%
\pgfsetlinewidth{0.501875pt}%
\definecolor{currentstroke}{rgb}{1.000000,0.000000,0.000000}%
\pgfsetstrokecolor{currentstroke}%
\pgfsetdash{}{0pt}%
\pgfpathmoveto{\pgfqpoint{2.100500in}{1.138093in}}%
\pgfpathlineto{\pgfqpoint{2.100500in}{1.138156in}}%
\pgfusepath{stroke}%
\end{pgfscope}%
\begin{pgfscope}%
\pgfpathrectangle{\pgfqpoint{1.000000in}{0.720000in}}{\pgfqpoint{6.200000in}{4.620000in}}%
\pgfusepath{clip}%
\pgfsetbuttcap%
\pgfsetroundjoin%
\pgfsetlinewidth{0.501875pt}%
\definecolor{currentstroke}{rgb}{1.000000,0.000000,0.000000}%
\pgfsetstrokecolor{currentstroke}%
\pgfsetdash{}{0pt}%
\pgfpathmoveto{\pgfqpoint{2.116000in}{1.138093in}}%
\pgfpathlineto{\pgfqpoint{2.116000in}{1.138119in}}%
\pgfusepath{stroke}%
\end{pgfscope}%
\begin{pgfscope}%
\pgfpathrectangle{\pgfqpoint{1.000000in}{0.720000in}}{\pgfqpoint{6.200000in}{4.620000in}}%
\pgfusepath{clip}%
\pgfsetbuttcap%
\pgfsetroundjoin%
\pgfsetlinewidth{0.501875pt}%
\definecolor{currentstroke}{rgb}{1.000000,0.000000,0.000000}%
\pgfsetstrokecolor{currentstroke}%
\pgfsetdash{}{0pt}%
\pgfpathmoveto{\pgfqpoint{2.317500in}{1.138093in}}%
\pgfpathlineto{\pgfqpoint{2.317500in}{1.138134in}}%
\pgfusepath{stroke}%
\end{pgfscope}%
\begin{pgfscope}%
\pgfpathrectangle{\pgfqpoint{1.000000in}{0.720000in}}{\pgfqpoint{6.200000in}{4.620000in}}%
\pgfusepath{clip}%
\pgfsetbuttcap%
\pgfsetroundjoin%
\pgfsetlinewidth{0.501875pt}%
\definecolor{currentstroke}{rgb}{1.000000,0.000000,0.000000}%
\pgfsetstrokecolor{currentstroke}%
\pgfsetdash{}{0pt}%
\pgfpathmoveto{\pgfqpoint{2.348500in}{1.138093in}}%
\pgfpathlineto{\pgfqpoint{2.348500in}{1.138106in}}%
\pgfusepath{stroke}%
\end{pgfscope}%
\begin{pgfscope}%
\pgfpathrectangle{\pgfqpoint{1.000000in}{0.720000in}}{\pgfqpoint{6.200000in}{4.620000in}}%
\pgfusepath{clip}%
\pgfsetbuttcap%
\pgfsetroundjoin%
\pgfsetlinewidth{0.501875pt}%
\definecolor{currentstroke}{rgb}{1.000000,0.000000,0.000000}%
\pgfsetstrokecolor{currentstroke}%
\pgfsetdash{}{0pt}%
\pgfpathmoveto{\pgfqpoint{2.426000in}{1.138093in}}%
\pgfpathlineto{\pgfqpoint{2.426000in}{1.138116in}}%
\pgfusepath{stroke}%
\end{pgfscope}%
\begin{pgfscope}%
\pgfpathrectangle{\pgfqpoint{1.000000in}{0.720000in}}{\pgfqpoint{6.200000in}{4.620000in}}%
\pgfusepath{clip}%
\pgfsetbuttcap%
\pgfsetroundjoin%
\pgfsetlinewidth{0.501875pt}%
\definecolor{currentstroke}{rgb}{1.000000,0.000000,0.000000}%
\pgfsetstrokecolor{currentstroke}%
\pgfsetdash{}{0pt}%
\pgfpathmoveto{\pgfqpoint{2.441500in}{1.138093in}}%
\pgfpathlineto{\pgfqpoint{2.441500in}{1.138096in}}%
\pgfusepath{stroke}%
\end{pgfscope}%
\begin{pgfscope}%
\pgfpathrectangle{\pgfqpoint{1.000000in}{0.720000in}}{\pgfqpoint{6.200000in}{4.620000in}}%
\pgfusepath{clip}%
\pgfsetbuttcap%
\pgfsetroundjoin%
\pgfsetlinewidth{0.501875pt}%
\definecolor{currentstroke}{rgb}{1.000000,0.000000,0.000000}%
\pgfsetstrokecolor{currentstroke}%
\pgfsetdash{}{0pt}%
\pgfpathmoveto{\pgfqpoint{2.457000in}{1.138093in}}%
\pgfpathlineto{\pgfqpoint{2.457000in}{1.138262in}}%
\pgfusepath{stroke}%
\end{pgfscope}%
\begin{pgfscope}%
\pgfpathrectangle{\pgfqpoint{1.000000in}{0.720000in}}{\pgfqpoint{6.200000in}{4.620000in}}%
\pgfusepath{clip}%
\pgfsetbuttcap%
\pgfsetroundjoin%
\pgfsetlinewidth{0.501875pt}%
\definecolor{currentstroke}{rgb}{1.000000,0.000000,0.000000}%
\pgfsetstrokecolor{currentstroke}%
\pgfsetdash{}{0pt}%
\pgfpathmoveto{\pgfqpoint{2.472500in}{1.138093in}}%
\pgfpathlineto{\pgfqpoint{2.472500in}{5.092965in}}%
\pgfusepath{stroke}%
\end{pgfscope}%
\begin{pgfscope}%
\pgfpathrectangle{\pgfqpoint{1.000000in}{0.720000in}}{\pgfqpoint{6.200000in}{4.620000in}}%
\pgfusepath{clip}%
\pgfsetbuttcap%
\pgfsetroundjoin%
\pgfsetlinewidth{0.501875pt}%
\definecolor{currentstroke}{rgb}{1.000000,0.000000,0.000000}%
\pgfsetstrokecolor{currentstroke}%
\pgfsetdash{}{0pt}%
\pgfpathmoveto{\pgfqpoint{2.503500in}{1.138093in}}%
\pgfpathlineto{\pgfqpoint{2.503500in}{1.138133in}}%
\pgfusepath{stroke}%
\end{pgfscope}%
\begin{pgfscope}%
\pgfpathrectangle{\pgfqpoint{1.000000in}{0.720000in}}{\pgfqpoint{6.200000in}{4.620000in}}%
\pgfusepath{clip}%
\pgfsetbuttcap%
\pgfsetroundjoin%
\pgfsetlinewidth{0.501875pt}%
\definecolor{currentstroke}{rgb}{1.000000,0.000000,0.000000}%
\pgfsetstrokecolor{currentstroke}%
\pgfsetdash{}{0pt}%
\pgfpathmoveto{\pgfqpoint{2.519000in}{1.138093in}}%
\pgfpathlineto{\pgfqpoint{2.519000in}{1.138117in}}%
\pgfusepath{stroke}%
\end{pgfscope}%
\begin{pgfscope}%
\pgfpathrectangle{\pgfqpoint{1.000000in}{0.720000in}}{\pgfqpoint{6.200000in}{4.620000in}}%
\pgfusepath{clip}%
\pgfsetbuttcap%
\pgfsetroundjoin%
\pgfsetlinewidth{0.501875pt}%
\definecolor{currentstroke}{rgb}{1.000000,0.000000,0.000000}%
\pgfsetstrokecolor{currentstroke}%
\pgfsetdash{}{0pt}%
\pgfpathmoveto{\pgfqpoint{2.534500in}{1.138093in}}%
\pgfpathlineto{\pgfqpoint{2.534500in}{1.138131in}}%
\pgfusepath{stroke}%
\end{pgfscope}%
\begin{pgfscope}%
\pgfpathrectangle{\pgfqpoint{1.000000in}{0.720000in}}{\pgfqpoint{6.200000in}{4.620000in}}%
\pgfusepath{clip}%
\pgfsetbuttcap%
\pgfsetroundjoin%
\pgfsetlinewidth{0.501875pt}%
\definecolor{currentstroke}{rgb}{1.000000,0.000000,0.000000}%
\pgfsetstrokecolor{currentstroke}%
\pgfsetdash{}{0pt}%
\pgfpathmoveto{\pgfqpoint{2.565500in}{1.138093in}}%
\pgfpathlineto{\pgfqpoint{2.565500in}{1.138175in}}%
\pgfusepath{stroke}%
\end{pgfscope}%
\begin{pgfscope}%
\pgfpathrectangle{\pgfqpoint{1.000000in}{0.720000in}}{\pgfqpoint{6.200000in}{4.620000in}}%
\pgfusepath{clip}%
\pgfsetbuttcap%
\pgfsetroundjoin%
\pgfsetlinewidth{0.501875pt}%
\definecolor{currentstroke}{rgb}{1.000000,0.000000,0.000000}%
\pgfsetstrokecolor{currentstroke}%
\pgfsetdash{}{0pt}%
\pgfpathmoveto{\pgfqpoint{2.581000in}{1.138093in}}%
\pgfpathlineto{\pgfqpoint{2.581000in}{1.138166in}}%
\pgfusepath{stroke}%
\end{pgfscope}%
\begin{pgfscope}%
\pgfpathrectangle{\pgfqpoint{1.000000in}{0.720000in}}{\pgfqpoint{6.200000in}{4.620000in}}%
\pgfusepath{clip}%
\pgfsetbuttcap%
\pgfsetroundjoin%
\pgfsetlinewidth{0.501875pt}%
\definecolor{currentstroke}{rgb}{1.000000,0.000000,0.000000}%
\pgfsetstrokecolor{currentstroke}%
\pgfsetdash{}{0pt}%
\pgfpathmoveto{\pgfqpoint{2.612000in}{1.138093in}}%
\pgfpathlineto{\pgfqpoint{2.612000in}{1.138125in}}%
\pgfusepath{stroke}%
\end{pgfscope}%
\begin{pgfscope}%
\pgfpathrectangle{\pgfqpoint{1.000000in}{0.720000in}}{\pgfqpoint{6.200000in}{4.620000in}}%
\pgfusepath{clip}%
\pgfsetbuttcap%
\pgfsetroundjoin%
\pgfsetlinewidth{0.501875pt}%
\definecolor{currentstroke}{rgb}{1.000000,0.000000,0.000000}%
\pgfsetstrokecolor{currentstroke}%
\pgfsetdash{}{0pt}%
\pgfpathmoveto{\pgfqpoint{2.689500in}{1.138093in}}%
\pgfpathlineto{\pgfqpoint{2.689500in}{1.138094in}}%
\pgfusepath{stroke}%
\end{pgfscope}%
\begin{pgfscope}%
\pgfpathrectangle{\pgfqpoint{1.000000in}{0.720000in}}{\pgfqpoint{6.200000in}{4.620000in}}%
\pgfusepath{clip}%
\pgfsetbuttcap%
\pgfsetroundjoin%
\pgfsetlinewidth{0.501875pt}%
\definecolor{currentstroke}{rgb}{1.000000,0.000000,0.000000}%
\pgfsetstrokecolor{currentstroke}%
\pgfsetdash{}{0pt}%
\pgfpathmoveto{\pgfqpoint{2.705000in}{1.138093in}}%
\pgfpathlineto{\pgfqpoint{2.705000in}{1.138179in}}%
\pgfusepath{stroke}%
\end{pgfscope}%
\begin{pgfscope}%
\pgfpathrectangle{\pgfqpoint{1.000000in}{0.720000in}}{\pgfqpoint{6.200000in}{4.620000in}}%
\pgfusepath{clip}%
\pgfsetbuttcap%
\pgfsetroundjoin%
\pgfsetlinewidth{0.501875pt}%
\definecolor{currentstroke}{rgb}{1.000000,0.000000,0.000000}%
\pgfsetstrokecolor{currentstroke}%
\pgfsetdash{}{0pt}%
\pgfpathmoveto{\pgfqpoint{2.736000in}{1.138093in}}%
\pgfpathlineto{\pgfqpoint{2.736000in}{1.138203in}}%
\pgfusepath{stroke}%
\end{pgfscope}%
\begin{pgfscope}%
\pgfpathrectangle{\pgfqpoint{1.000000in}{0.720000in}}{\pgfqpoint{6.200000in}{4.620000in}}%
\pgfusepath{clip}%
\pgfsetbuttcap%
\pgfsetroundjoin%
\pgfsetlinewidth{0.501875pt}%
\definecolor{currentstroke}{rgb}{1.000000,0.000000,0.000000}%
\pgfsetstrokecolor{currentstroke}%
\pgfsetdash{}{0pt}%
\pgfpathmoveto{\pgfqpoint{2.767000in}{1.138093in}}%
\pgfpathlineto{\pgfqpoint{2.767000in}{1.138212in}}%
\pgfusepath{stroke}%
\end{pgfscope}%
\begin{pgfscope}%
\pgfpathrectangle{\pgfqpoint{1.000000in}{0.720000in}}{\pgfqpoint{6.200000in}{4.620000in}}%
\pgfusepath{clip}%
\pgfsetbuttcap%
\pgfsetroundjoin%
\pgfsetlinewidth{0.501875pt}%
\definecolor{currentstroke}{rgb}{1.000000,0.000000,0.000000}%
\pgfsetstrokecolor{currentstroke}%
\pgfsetdash{}{0pt}%
\pgfpathmoveto{\pgfqpoint{2.813500in}{1.138093in}}%
\pgfpathlineto{\pgfqpoint{2.813500in}{2.191299in}}%
\pgfusepath{stroke}%
\end{pgfscope}%
\begin{pgfscope}%
\pgfpathrectangle{\pgfqpoint{1.000000in}{0.720000in}}{\pgfqpoint{6.200000in}{4.620000in}}%
\pgfusepath{clip}%
\pgfsetbuttcap%
\pgfsetroundjoin%
\pgfsetlinewidth{0.501875pt}%
\definecolor{currentstroke}{rgb}{1.000000,0.000000,0.000000}%
\pgfsetstrokecolor{currentstroke}%
\pgfsetdash{}{0pt}%
\pgfpathmoveto{\pgfqpoint{2.860000in}{1.138093in}}%
\pgfpathlineto{\pgfqpoint{2.860000in}{3.011455in}}%
\pgfusepath{stroke}%
\end{pgfscope}%
\begin{pgfscope}%
\pgfpathrectangle{\pgfqpoint{1.000000in}{0.720000in}}{\pgfqpoint{6.200000in}{4.620000in}}%
\pgfusepath{clip}%
\pgfsetbuttcap%
\pgfsetroundjoin%
\pgfsetlinewidth{0.501875pt}%
\definecolor{currentstroke}{rgb}{1.000000,0.000000,0.000000}%
\pgfsetstrokecolor{currentstroke}%
\pgfsetdash{}{0pt}%
\pgfpathmoveto{\pgfqpoint{3.635000in}{1.138093in}}%
\pgfpathlineto{\pgfqpoint{3.635000in}{1.138113in}}%
\pgfusepath{stroke}%
\end{pgfscope}%
\begin{pgfscope}%
\pgfpathrectangle{\pgfqpoint{1.000000in}{0.720000in}}{\pgfqpoint{6.200000in}{4.620000in}}%
\pgfusepath{clip}%
\pgfsetbuttcap%
\pgfsetroundjoin%
\pgfsetlinewidth{0.501875pt}%
\definecolor{currentstroke}{rgb}{1.000000,0.000000,0.000000}%
\pgfsetstrokecolor{currentstroke}%
\pgfsetdash{}{0pt}%
\pgfpathmoveto{\pgfqpoint{3.697000in}{1.138093in}}%
\pgfpathlineto{\pgfqpoint{3.697000in}{1.138162in}}%
\pgfusepath{stroke}%
\end{pgfscope}%
\begin{pgfscope}%
\pgfpathrectangle{\pgfqpoint{1.000000in}{0.720000in}}{\pgfqpoint{6.200000in}{4.620000in}}%
\pgfusepath{clip}%
\pgfsetbuttcap%
\pgfsetroundjoin%
\pgfsetlinewidth{0.501875pt}%
\definecolor{currentstroke}{rgb}{1.000000,0.000000,0.000000}%
\pgfsetstrokecolor{currentstroke}%
\pgfsetdash{}{0pt}%
\pgfpathmoveto{\pgfqpoint{3.712500in}{1.138093in}}%
\pgfpathlineto{\pgfqpoint{3.712500in}{4.463161in}}%
\pgfusepath{stroke}%
\end{pgfscope}%
\begin{pgfscope}%
\pgfpathrectangle{\pgfqpoint{1.000000in}{0.720000in}}{\pgfqpoint{6.200000in}{4.620000in}}%
\pgfusepath{clip}%
\pgfsetbuttcap%
\pgfsetroundjoin%
\pgfsetlinewidth{0.501875pt}%
\definecolor{currentstroke}{rgb}{1.000000,0.000000,0.000000}%
\pgfsetstrokecolor{currentstroke}%
\pgfsetdash{}{0pt}%
\pgfpathmoveto{\pgfqpoint{3.728000in}{1.138093in}}%
\pgfpathlineto{\pgfqpoint{3.728000in}{1.138115in}}%
\pgfusepath{stroke}%
\end{pgfscope}%
\begin{pgfscope}%
\pgfpathrectangle{\pgfqpoint{1.000000in}{0.720000in}}{\pgfqpoint{6.200000in}{4.620000in}}%
\pgfusepath{clip}%
\pgfsetbuttcap%
\pgfsetroundjoin%
\pgfsetlinewidth{0.501875pt}%
\definecolor{currentstroke}{rgb}{1.000000,0.000000,0.000000}%
\pgfsetstrokecolor{currentstroke}%
\pgfsetdash{}{0pt}%
\pgfpathmoveto{\pgfqpoint{3.759000in}{1.138093in}}%
\pgfpathlineto{\pgfqpoint{3.759000in}{1.138207in}}%
\pgfusepath{stroke}%
\end{pgfscope}%
\begin{pgfscope}%
\pgfpathrectangle{\pgfqpoint{1.000000in}{0.720000in}}{\pgfqpoint{6.200000in}{4.620000in}}%
\pgfusepath{clip}%
\pgfsetbuttcap%
\pgfsetroundjoin%
\pgfsetlinewidth{0.501875pt}%
\definecolor{currentstroke}{rgb}{1.000000,0.000000,0.000000}%
\pgfsetstrokecolor{currentstroke}%
\pgfsetdash{}{0pt}%
\pgfpathmoveto{\pgfqpoint{3.774500in}{1.138093in}}%
\pgfpathlineto{\pgfqpoint{3.774500in}{1.138178in}}%
\pgfusepath{stroke}%
\end{pgfscope}%
\begin{pgfscope}%
\pgfpathrectangle{\pgfqpoint{1.000000in}{0.720000in}}{\pgfqpoint{6.200000in}{4.620000in}}%
\pgfusepath{clip}%
\pgfsetbuttcap%
\pgfsetroundjoin%
\pgfsetlinewidth{0.501875pt}%
\definecolor{currentstroke}{rgb}{1.000000,0.000000,0.000000}%
\pgfsetstrokecolor{currentstroke}%
\pgfsetdash{}{0pt}%
\pgfpathmoveto{\pgfqpoint{3.790000in}{1.138093in}}%
\pgfpathlineto{\pgfqpoint{3.790000in}{1.138170in}}%
\pgfusepath{stroke}%
\end{pgfscope}%
\begin{pgfscope}%
\pgfpathrectangle{\pgfqpoint{1.000000in}{0.720000in}}{\pgfqpoint{6.200000in}{4.620000in}}%
\pgfusepath{clip}%
\pgfsetbuttcap%
\pgfsetroundjoin%
\pgfsetlinewidth{0.501875pt}%
\definecolor{currentstroke}{rgb}{1.000000,0.000000,0.000000}%
\pgfsetstrokecolor{currentstroke}%
\pgfsetdash{}{0pt}%
\pgfpathmoveto{\pgfqpoint{3.805500in}{1.138093in}}%
\pgfpathlineto{\pgfqpoint{3.805500in}{1.138144in}}%
\pgfusepath{stroke}%
\end{pgfscope}%
\begin{pgfscope}%
\pgfpathrectangle{\pgfqpoint{1.000000in}{0.720000in}}{\pgfqpoint{6.200000in}{4.620000in}}%
\pgfusepath{clip}%
\pgfsetbuttcap%
\pgfsetroundjoin%
\pgfsetlinewidth{0.501875pt}%
\definecolor{currentstroke}{rgb}{1.000000,0.000000,0.000000}%
\pgfsetstrokecolor{currentstroke}%
\pgfsetdash{}{0pt}%
\pgfpathmoveto{\pgfqpoint{3.821000in}{1.138093in}}%
\pgfpathlineto{\pgfqpoint{3.821000in}{1.138111in}}%
\pgfusepath{stroke}%
\end{pgfscope}%
\begin{pgfscope}%
\pgfpathrectangle{\pgfqpoint{1.000000in}{0.720000in}}{\pgfqpoint{6.200000in}{4.620000in}}%
\pgfusepath{clip}%
\pgfsetbuttcap%
\pgfsetroundjoin%
\pgfsetlinewidth{0.501875pt}%
\definecolor{currentstroke}{rgb}{1.000000,0.000000,0.000000}%
\pgfsetstrokecolor{currentstroke}%
\pgfsetdash{}{0pt}%
\pgfpathmoveto{\pgfqpoint{3.836500in}{1.138093in}}%
\pgfpathlineto{\pgfqpoint{3.836500in}{1.138151in}}%
\pgfusepath{stroke}%
\end{pgfscope}%
\begin{pgfscope}%
\pgfpathrectangle{\pgfqpoint{1.000000in}{0.720000in}}{\pgfqpoint{6.200000in}{4.620000in}}%
\pgfusepath{clip}%
\pgfsetbuttcap%
\pgfsetroundjoin%
\pgfsetlinewidth{0.501875pt}%
\definecolor{currentstroke}{rgb}{1.000000,0.000000,0.000000}%
\pgfsetstrokecolor{currentstroke}%
\pgfsetdash{}{0pt}%
\pgfpathmoveto{\pgfqpoint{3.852000in}{1.138093in}}%
\pgfpathlineto{\pgfqpoint{3.852000in}{1.138229in}}%
\pgfusepath{stroke}%
\end{pgfscope}%
\begin{pgfscope}%
\pgfsetrectcap%
\pgfsetmiterjoin%
\pgfsetlinewidth{0.803000pt}%
\definecolor{currentstroke}{rgb}{0.000000,0.000000,0.000000}%
\pgfsetstrokecolor{currentstroke}%
\pgfsetdash{}{0pt}%
\pgfpathmoveto{\pgfqpoint{1.000000in}{0.720000in}}%
\pgfpathlineto{\pgfqpoint{1.000000in}{5.340000in}}%
\pgfusepath{stroke}%
\end{pgfscope}%
\begin{pgfscope}%
\pgfsetrectcap%
\pgfsetmiterjoin%
\pgfsetlinewidth{0.803000pt}%
\definecolor{currentstroke}{rgb}{0.000000,0.000000,0.000000}%
\pgfsetstrokecolor{currentstroke}%
\pgfsetdash{}{0pt}%
\pgfpathmoveto{\pgfqpoint{7.200000in}{0.720000in}}%
\pgfpathlineto{\pgfqpoint{7.200000in}{5.340000in}}%
\pgfusepath{stroke}%
\end{pgfscope}%
\begin{pgfscope}%
\pgfsetrectcap%
\pgfsetmiterjoin%
\pgfsetlinewidth{0.803000pt}%
\definecolor{currentstroke}{rgb}{0.000000,0.000000,0.000000}%
\pgfsetstrokecolor{currentstroke}%
\pgfsetdash{}{0pt}%
\pgfpathmoveto{\pgfqpoint{1.000000in}{0.720000in}}%
\pgfpathlineto{\pgfqpoint{7.200000in}{0.720000in}}%
\pgfusepath{stroke}%
\end{pgfscope}%
\begin{pgfscope}%
\pgfsetrectcap%
\pgfsetmiterjoin%
\pgfsetlinewidth{0.803000pt}%
\definecolor{currentstroke}{rgb}{0.000000,0.000000,0.000000}%
\pgfsetstrokecolor{currentstroke}%
\pgfsetdash{}{0pt}%
\pgfpathmoveto{\pgfqpoint{1.000000in}{5.340000in}}%
\pgfpathlineto{\pgfqpoint{7.200000in}{5.340000in}}%
\pgfusepath{stroke}%
\end{pgfscope}%
\begin{pgfscope}%
\pgfsetbuttcap%
\pgfsetmiterjoin%
\definecolor{currentfill}{rgb}{1.000000,1.000000,1.000000}%
\pgfsetfillcolor{currentfill}%
\pgfsetfillopacity{0.800000}%
\pgfsetlinewidth{1.003750pt}%
\definecolor{currentstroke}{rgb}{0.800000,0.800000,0.800000}%
\pgfsetstrokecolor{currentstroke}%
\pgfsetstrokeopacity{0.800000}%
\pgfsetdash{}{0pt}%
\pgfpathmoveto{\pgfqpoint{4.976872in}{3.932908in}}%
\pgfpathlineto{\pgfqpoint{7.005556in}{3.932908in}}%
\pgfpathquadraticcurveto{\pgfqpoint{7.061111in}{3.932908in}}{\pgfqpoint{7.061111in}{3.988464in}}%
\pgfpathlineto{\pgfqpoint{7.061111in}{5.145556in}}%
\pgfpathquadraticcurveto{\pgfqpoint{7.061111in}{5.201111in}}{\pgfqpoint{7.005556in}{5.201111in}}%
\pgfpathlineto{\pgfqpoint{4.976872in}{5.201111in}}%
\pgfpathquadraticcurveto{\pgfqpoint{4.921317in}{5.201111in}}{\pgfqpoint{4.921317in}{5.145556in}}%
\pgfpathlineto{\pgfqpoint{4.921317in}{3.988464in}}%
\pgfpathquadraticcurveto{\pgfqpoint{4.921317in}{3.932908in}}{\pgfqpoint{4.976872in}{3.932908in}}%
\pgfpathlineto{\pgfqpoint{4.976872in}{3.932908in}}%
\pgfpathclose%
\pgfusepath{stroke,fill}%
\end{pgfscope}%
\begin{pgfscope}%
\pgfsetrectcap%
\pgfsetroundjoin%
\pgfsetlinewidth{2.007500pt}%
\definecolor{currentstroke}{rgb}{0.121569,0.466667,0.705882}%
\pgfsetstrokecolor{currentstroke}%
\pgfsetdash{}{0pt}%
\pgfpathmoveto{\pgfqpoint{5.032428in}{4.987184in}}%
\pgfpathlineto{\pgfqpoint{5.310206in}{4.987184in}}%
\pgfpathlineto{\pgfqpoint{5.587983in}{4.987184in}}%
\pgfusepath{stroke}%
\end{pgfscope}%
\begin{pgfscope}%
\definecolor{textcolor}{rgb}{0.000000,0.000000,0.000000}%
\pgfsetstrokecolor{textcolor}%
\pgfsetfillcolor{textcolor}%
\pgftext[x=5.810206in,y=4.889962in,left,base]{\color{textcolor}\sffamily\fontsize{20.000000}{24.000000}\selectfont Waveform}%
\end{pgfscope}%
\begin{pgfscope}%
\pgfsetbuttcap%
\pgfsetroundjoin%
\pgfsetlinewidth{2.007500pt}%
\definecolor{currentstroke}{rgb}{0.000000,0.500000,0.000000}%
\pgfsetstrokecolor{currentstroke}%
\pgfsetdash{}{0pt}%
\pgfpathmoveto{\pgfqpoint{5.032428in}{4.592227in}}%
\pgfpathlineto{\pgfqpoint{5.587983in}{4.592227in}}%
\pgfusepath{stroke}%
\end{pgfscope}%
\begin{pgfscope}%
\definecolor{textcolor}{rgb}{0.000000,0.000000,0.000000}%
\pgfsetstrokecolor{textcolor}%
\pgfsetfillcolor{textcolor}%
\pgftext[x=5.810206in,y=4.495005in,left,base]{\color{textcolor}\sffamily\fontsize{20.000000}{24.000000}\selectfont Threshold}%
\end{pgfscope}%
\begin{pgfscope}%
\pgfsetbuttcap%
\pgfsetroundjoin%
\pgfsetlinewidth{0.501875pt}%
\definecolor{currentstroke}{rgb}{1.000000,0.000000,0.000000}%
\pgfsetstrokecolor{currentstroke}%
\pgfsetdash{}{0pt}%
\pgfpathmoveto{\pgfqpoint{5.032428in}{4.197271in}}%
\pgfpathlineto{\pgfqpoint{5.587983in}{4.197271in}}%
\pgfusepath{stroke}%
\end{pgfscope}%
\begin{pgfscope}%
\definecolor{textcolor}{rgb}{0.000000,0.000000,0.000000}%
\pgfsetstrokecolor{textcolor}%
\pgfsetfillcolor{textcolor}%
\pgftext[x=5.810206in,y=4.100048in,left,base]{\color{textcolor}\sffamily\fontsize{20.000000}{24.000000}\selectfont Charge}%
\end{pgfscope}%
\end{pgfpicture}%
\makeatother%
\endgroup%
}
    \caption{\label{fig:mcmc}An example with \\ $\Delta{t_0}=\SI{-2.48}{ns}$, $\mathrm{RSS}=\SI{16.25}{mV^2}$, $D_\mathrm{w}=\SI{0.76}{ns}$.}
  \end{subfigure}
  \caption{\label{fig:mcmc-performance}Demonstration of MCMC with $\num[retain-unity-mantissa=false]{1e4}$ waveforms in~\subref{fig:mcmc-npe} and one waveform in~\subref{fig:mcmc} sampled from the same setup as figure~\ref{fig:method}.  Although using a more dedicated model, MCMC performs worse than the direct charge fitting in figure~\ref{fig:dcf}. We suspect the Markov chain is not long enough.}
\end{figure}
Although we imposed a prior distribution in eq.~\eqref{eq:mixnormal} with $\E[q_i]=1$, The charges $\hat{q}_i$ in figure~\ref{fig:fitting} are still less than 1.  The $D_\mathrm{w}$ marginal distribution in figure~\ref{fig:mcmc-npe} is less smooth than that of the direct charge fitting in figure~\ref{fig:fitting-npe}.  Similarly, RSS in figure~\ref{fig:mcmc} is slightly worse than that in figure~\ref{fig:fitting}.  We suspect the Markov chain has not finally converged due to the trans-dimensional property of eq.~\eqref{eq:mixnormal}.  Extending the chain is not a solution because MCMC is already much slower than direct fitting in section~\ref{sec:dcf}.  We need an algorithm that pertains to the model of eq.~\eqref{eq:mixnormal} but much faster than MCMC.

\subsubsection{Fast stochastic matching pursuit}
\label{subsec:fsmp}
In reality, $w(t)$ is discretized as $\bm{w}$. If we rewrite the hierarchical model (eq.~\eqref{eq:mixnormal}) into a joint distribution, marginalizing out $\bm{q}'$ and $\bm{z}$ gives a flattened model,
\begin{equation}
  \label{eq:universe}
  \begin{aligned}
    p(\bm{w}, t_0, \mu) &= \sum_{\bm{z}} \int \mathrm{d}\bm{q}' p(\bm{w}, \bm{q}', \bm{z}, t_0, \mu) \\
    &= p(t_0, \mu) \sum_{\bm{z}} \left[\int \mathrm{d}\bm{q}' p(\bm{w}|\bm{q}') p(\bm{q}'|\bm{z}) \right] p(\bm{z}|t_0, \mu) \\
    &= p(t_0, \mu) \sum_{\bm{z}} p(\bm{w}|\bm{z}) p(\bm{z}|t_0, \mu) \\
    &= p(t_0, \mu) p(\bm{w}|t_0, \mu) \\
  \end{aligned}
\end{equation}
The integration over $\bm{q}'$ is the probability density of a multi-normal distribution $p(\bm{w}|\bm{z})$, with a fast algorithm to iteratively compute by Schniter~et al.~\cite{schniter_fast_2008}. The summation over $\bm{z}$, however, takes an exploding number of combinations.

Let's approximate the summation with a sample from $S = (\bm{s}_1, \bm{s}_2, \cdots, \bm{s}_M)$ by Metropolis-Hasings~\cite{metropolis_equation_1953, hastings_monte_1970, mackay_information_2003} from $p(\bm{z}) = C p(\bm{w} | \bm{z}) h( \bm{z})$. $C$ is independent of $\bm{z}$, and $h(\bm{z})$ is an educated guess for $p(\bm{z}|t_0, \mu)$ from a previous method like LucyDDM~(section~\ref{sec:lucyddm}). Then,
\begin{equation}
  \label{eq:mh}
  \begin{aligned}
    p(\bm{w}|\mu, t_0) &= \sum_{\bm{z}} p(\bm{w}|\bm{z}) p(\bm{z}|t_0, \mu) = \frac{1}{C}\sum_{\bm{z}} p(\bm{z}) \frac{p(\bm{z} | \mu, t_0)}{h(\bm{z})} \\
    &= \frac{1}{C} \E_{\bm{z}}\left[ \frac{p(\bm{z} | \mu, t_0)}{h(\bm{z})} \right] \approx \frac{1}{CM} \sum_{i=1}^M \frac{p(\bm{s}_i | \mu, t_0)}{h(\bm{s}_i)}. \\
  \end{aligned}
\end{equation}
Construct approximate MLEs for $t_0$, $\mu$ and $\bm{z}$, and the expectation estimation of $\hat{\bm{q}}$,
\begin{equation}
  \label{eq:fsmpcharge}
  \begin{aligned}
    \left(\hat{t}_0, \hat{\mu}\right) &= \arg\underset{t_0,\mu}{\max}~p(\bm{w}|\mu, t_0) = \arg\underset{t_0,\mu}{\max} \sum_{i=1}^M \frac{p(\bm{s}_i | \mu, t_0)}{h(\bm{s}_i)}\\
    \hat{\bm{z}} &= \arg \underset{\bm{s}_i \in S}{\max}~p(\bm{w}|\bm{s}_i) h(\bm{s}_i) \\
    \hat{\bm{q}}|{\hat{\bm{z}}} &= \E(\bm{q}'|\bm{w},\hat{\bm{z}})
  \end{aligned}
\end{equation}
RSS and $D_\mathrm{w}$ are calculated by eqs.~\eqref{eq:rss}, \eqref{eq:numerical}, \eqref{eq:gd-phi}.

We name the method \emph{fast stochastic matching pursuit}~(FSMP) after \emph{fast Bayesian matching pursuit}~(FBMP) by Schniter~et al.~\cite{schniter_fast_2008} and \emph{Bayesian stochastic matching pursuit} by Chen~et al.~\cite{chen_stochastic_2011}.  Here FSMP replaces the greedy search routine in FBMP with stochastic sampling.  With the help of Ekanadham~et al.'s function interpolation~\cite{ekanadham_recovery_2011}, FSMP straightforwardly extends $\bm{z}$ into an unbinned vector of PE locations $t_i$.  Geyer and Møller~\cite{geyer_simulation_1994} developed a similar sampler to handle trans-dimensionality in a Poisson point process.  $h(\bm{z})$ and the proposal distribution in Metropolis-Hastings steps could be tuned to improve sampling efficiency.  We shall leave the detailed study of the Markov chain convergence to our future publications.

\begin{figure}[H]
  \begin{subfigure}[b]{.45\textwidth}
    \centering
    \resizebox{1.05\textwidth}{!}{%% Creator: Matplotlib, PGF backend
%%
%% To include the figure in your LaTeX document, write
%%   \input{<filename>.pgf}
%%
%% Make sure the required packages are loaded in your preamble
%%   \usepackage{pgf}
%%
%% Also ensure that all the required font packages are loaded; for instance,
%% the lmodern package is sometimes necessary when using math font.
%%   \usepackage{lmodern}
%%
%% Figures using additional raster images can only be included by \input if
%% they are in the same directory as the main LaTeX file. For loading figures
%% from other directories you can use the `import` package
%%   \usepackage{import}
%%
%% and then include the figures with
%%   \import{<path to file>}{<filename>.pgf}
%%
%% Matplotlib used the following preamble
%%   \usepackage[detect-all,locale=DE]{siunitx}
%%
\begingroup%
\makeatletter%
\begin{pgfpicture}%
\pgfpathrectangle{\pgfpointorigin}{\pgfqpoint{8.000000in}{6.000000in}}%
\pgfusepath{use as bounding box, clip}%
\begin{pgfscope}%
\pgfsetbuttcap%
\pgfsetmiterjoin%
\definecolor{currentfill}{rgb}{1.000000,1.000000,1.000000}%
\pgfsetfillcolor{currentfill}%
\pgfsetlinewidth{0.000000pt}%
\definecolor{currentstroke}{rgb}{1.000000,1.000000,1.000000}%
\pgfsetstrokecolor{currentstroke}%
\pgfsetdash{}{0pt}%
\pgfpathmoveto{\pgfqpoint{0.000000in}{0.000000in}}%
\pgfpathlineto{\pgfqpoint{8.000000in}{0.000000in}}%
\pgfpathlineto{\pgfqpoint{8.000000in}{6.000000in}}%
\pgfpathlineto{\pgfqpoint{0.000000in}{6.000000in}}%
\pgfpathlineto{\pgfqpoint{0.000000in}{0.000000in}}%
\pgfpathclose%
\pgfusepath{fill}%
\end{pgfscope}%
\begin{pgfscope}%
\pgfsetbuttcap%
\pgfsetmiterjoin%
\definecolor{currentfill}{rgb}{1.000000,1.000000,1.000000}%
\pgfsetfillcolor{currentfill}%
\pgfsetlinewidth{0.000000pt}%
\definecolor{currentstroke}{rgb}{0.000000,0.000000,0.000000}%
\pgfsetstrokecolor{currentstroke}%
\pgfsetstrokeopacity{0.000000}%
\pgfsetdash{}{0pt}%
\pgfpathmoveto{\pgfqpoint{1.000000in}{0.720000in}}%
\pgfpathlineto{\pgfqpoint{5.800000in}{0.720000in}}%
\pgfpathlineto{\pgfqpoint{5.800000in}{5.340000in}}%
\pgfpathlineto{\pgfqpoint{1.000000in}{5.340000in}}%
\pgfpathlineto{\pgfqpoint{1.000000in}{0.720000in}}%
\pgfpathclose%
\pgfusepath{fill}%
\end{pgfscope}%
\begin{pgfscope}%
\pgfpathrectangle{\pgfqpoint{1.000000in}{0.720000in}}{\pgfqpoint{4.800000in}{4.620000in}}%
\pgfusepath{clip}%
\pgfsetbuttcap%
\pgfsetroundjoin%
\definecolor{currentfill}{rgb}{0.121569,0.466667,0.705882}%
\pgfsetfillcolor{currentfill}%
\pgfsetfillopacity{0.100000}%
\pgfsetlinewidth{0.000000pt}%
\definecolor{currentstroke}{rgb}{0.000000,0.000000,0.000000}%
\pgfsetstrokecolor{currentstroke}%
\pgfsetdash{}{0pt}%
\pgfpathmoveto{\pgfqpoint{1.300000in}{3.031456in}}%
\pgfpathlineto{\pgfqpoint{1.300000in}{1.191247in}}%
\pgfpathlineto{\pgfqpoint{1.600000in}{1.821221in}}%
\pgfpathlineto{\pgfqpoint{1.900000in}{2.190063in}}%
\pgfpathlineto{\pgfqpoint{2.200000in}{2.327301in}}%
\pgfpathlineto{\pgfqpoint{2.500000in}{2.428034in}}%
\pgfpathlineto{\pgfqpoint{2.800000in}{2.511331in}}%
\pgfpathlineto{\pgfqpoint{3.100000in}{2.528322in}}%
\pgfpathlineto{\pgfqpoint{3.400000in}{2.622883in}}%
\pgfpathlineto{\pgfqpoint{3.700000in}{2.648169in}}%
\pgfpathlineto{\pgfqpoint{4.000000in}{2.482835in}}%
\pgfpathlineto{\pgfqpoint{4.300000in}{2.532402in}}%
\pgfpathlineto{\pgfqpoint{4.600000in}{3.127639in}}%
\pgfpathlineto{\pgfqpoint{4.900000in}{3.598089in}}%
\pgfpathlineto{\pgfqpoint{5.200000in}{2.856242in}}%
\pgfpathlineto{\pgfqpoint{5.500000in}{3.493745in}}%
\pgfpathlineto{\pgfqpoint{5.500000in}{3.493745in}}%
\pgfpathlineto{\pgfqpoint{5.500000in}{3.493745in}}%
\pgfpathlineto{\pgfqpoint{5.200000in}{2.856242in}}%
\pgfpathlineto{\pgfqpoint{4.900000in}{3.598089in}}%
\pgfpathlineto{\pgfqpoint{4.600000in}{3.699964in}}%
\pgfpathlineto{\pgfqpoint{4.300000in}{3.695576in}}%
\pgfpathlineto{\pgfqpoint{4.000000in}{3.664786in}}%
\pgfpathlineto{\pgfqpoint{3.700000in}{3.741645in}}%
\pgfpathlineto{\pgfqpoint{3.400000in}{3.775141in}}%
\pgfpathlineto{\pgfqpoint{3.100000in}{3.900421in}}%
\pgfpathlineto{\pgfqpoint{2.800000in}{4.040753in}}%
\pgfpathlineto{\pgfqpoint{2.500000in}{3.983481in}}%
\pgfpathlineto{\pgfqpoint{2.200000in}{4.084868in}}%
\pgfpathlineto{\pgfqpoint{1.900000in}{3.996053in}}%
\pgfpathlineto{\pgfqpoint{1.600000in}{3.763289in}}%
\pgfpathlineto{\pgfqpoint{1.300000in}{3.031456in}}%
\pgfpathlineto{\pgfqpoint{1.300000in}{3.031456in}}%
\pgfpathclose%
\pgfusepath{fill}%
\end{pgfscope}%
\begin{pgfscope}%
\pgfsetbuttcap%
\pgfsetroundjoin%
\definecolor{currentfill}{rgb}{0.000000,0.000000,0.000000}%
\pgfsetfillcolor{currentfill}%
\pgfsetlinewidth{0.803000pt}%
\definecolor{currentstroke}{rgb}{0.000000,0.000000,0.000000}%
\pgfsetstrokecolor{currentstroke}%
\pgfsetdash{}{0pt}%
\pgfsys@defobject{currentmarker}{\pgfqpoint{0.000000in}{-0.048611in}}{\pgfqpoint{0.000000in}{0.000000in}}{%
\pgfpathmoveto{\pgfqpoint{0.000000in}{0.000000in}}%
\pgfpathlineto{\pgfqpoint{0.000000in}{-0.048611in}}%
\pgfusepath{stroke,fill}%
}%
\begin{pgfscope}%
\pgfsys@transformshift{1.300000in}{0.720000in}%
\pgfsys@useobject{currentmarker}{}%
\end{pgfscope}%
\end{pgfscope}%
\begin{pgfscope}%
\definecolor{textcolor}{rgb}{0.000000,0.000000,0.000000}%
\pgfsetstrokecolor{textcolor}%
\pgfsetfillcolor{textcolor}%
\pgftext[x=1.300000in,y=0.622778in,,top]{\color{textcolor}\sffamily\fontsize{20.000000}{24.000000}\selectfont 1}%
\end{pgfscope}%
\begin{pgfscope}%
\pgfsetbuttcap%
\pgfsetroundjoin%
\definecolor{currentfill}{rgb}{0.000000,0.000000,0.000000}%
\pgfsetfillcolor{currentfill}%
\pgfsetlinewidth{0.803000pt}%
\definecolor{currentstroke}{rgb}{0.000000,0.000000,0.000000}%
\pgfsetstrokecolor{currentstroke}%
\pgfsetdash{}{0pt}%
\pgfsys@defobject{currentmarker}{\pgfqpoint{0.000000in}{-0.048611in}}{\pgfqpoint{0.000000in}{0.000000in}}{%
\pgfpathmoveto{\pgfqpoint{0.000000in}{0.000000in}}%
\pgfpathlineto{\pgfqpoint{0.000000in}{-0.048611in}}%
\pgfusepath{stroke,fill}%
}%
\begin{pgfscope}%
\pgfsys@transformshift{1.900000in}{0.720000in}%
\pgfsys@useobject{currentmarker}{}%
\end{pgfscope}%
\end{pgfscope}%
\begin{pgfscope}%
\definecolor{textcolor}{rgb}{0.000000,0.000000,0.000000}%
\pgfsetstrokecolor{textcolor}%
\pgfsetfillcolor{textcolor}%
\pgftext[x=1.900000in,y=0.622778in,,top]{\color{textcolor}\sffamily\fontsize{20.000000}{24.000000}\selectfont 3}%
\end{pgfscope}%
\begin{pgfscope}%
\pgfsetbuttcap%
\pgfsetroundjoin%
\definecolor{currentfill}{rgb}{0.000000,0.000000,0.000000}%
\pgfsetfillcolor{currentfill}%
\pgfsetlinewidth{0.803000pt}%
\definecolor{currentstroke}{rgb}{0.000000,0.000000,0.000000}%
\pgfsetstrokecolor{currentstroke}%
\pgfsetdash{}{0pt}%
\pgfsys@defobject{currentmarker}{\pgfqpoint{0.000000in}{-0.048611in}}{\pgfqpoint{0.000000in}{0.000000in}}{%
\pgfpathmoveto{\pgfqpoint{0.000000in}{0.000000in}}%
\pgfpathlineto{\pgfqpoint{0.000000in}{-0.048611in}}%
\pgfusepath{stroke,fill}%
}%
\begin{pgfscope}%
\pgfsys@transformshift{2.500000in}{0.720000in}%
\pgfsys@useobject{currentmarker}{}%
\end{pgfscope}%
\end{pgfscope}%
\begin{pgfscope}%
\definecolor{textcolor}{rgb}{0.000000,0.000000,0.000000}%
\pgfsetstrokecolor{textcolor}%
\pgfsetfillcolor{textcolor}%
\pgftext[x=2.500000in,y=0.622778in,,top]{\color{textcolor}\sffamily\fontsize{20.000000}{24.000000}\selectfont 5}%
\end{pgfscope}%
\begin{pgfscope}%
\pgfsetbuttcap%
\pgfsetroundjoin%
\definecolor{currentfill}{rgb}{0.000000,0.000000,0.000000}%
\pgfsetfillcolor{currentfill}%
\pgfsetlinewidth{0.803000pt}%
\definecolor{currentstroke}{rgb}{0.000000,0.000000,0.000000}%
\pgfsetstrokecolor{currentstroke}%
\pgfsetdash{}{0pt}%
\pgfsys@defobject{currentmarker}{\pgfqpoint{0.000000in}{-0.048611in}}{\pgfqpoint{0.000000in}{0.000000in}}{%
\pgfpathmoveto{\pgfqpoint{0.000000in}{0.000000in}}%
\pgfpathlineto{\pgfqpoint{0.000000in}{-0.048611in}}%
\pgfusepath{stroke,fill}%
}%
\begin{pgfscope}%
\pgfsys@transformshift{3.100000in}{0.720000in}%
\pgfsys@useobject{currentmarker}{}%
\end{pgfscope}%
\end{pgfscope}%
\begin{pgfscope}%
\definecolor{textcolor}{rgb}{0.000000,0.000000,0.000000}%
\pgfsetstrokecolor{textcolor}%
\pgfsetfillcolor{textcolor}%
\pgftext[x=3.100000in,y=0.622778in,,top]{\color{textcolor}\sffamily\fontsize{20.000000}{24.000000}\selectfont 7}%
\end{pgfscope}%
\begin{pgfscope}%
\pgfsetbuttcap%
\pgfsetroundjoin%
\definecolor{currentfill}{rgb}{0.000000,0.000000,0.000000}%
\pgfsetfillcolor{currentfill}%
\pgfsetlinewidth{0.803000pt}%
\definecolor{currentstroke}{rgb}{0.000000,0.000000,0.000000}%
\pgfsetstrokecolor{currentstroke}%
\pgfsetdash{}{0pt}%
\pgfsys@defobject{currentmarker}{\pgfqpoint{0.000000in}{-0.048611in}}{\pgfqpoint{0.000000in}{0.000000in}}{%
\pgfpathmoveto{\pgfqpoint{0.000000in}{0.000000in}}%
\pgfpathlineto{\pgfqpoint{0.000000in}{-0.048611in}}%
\pgfusepath{stroke,fill}%
}%
\begin{pgfscope}%
\pgfsys@transformshift{3.700000in}{0.720000in}%
\pgfsys@useobject{currentmarker}{}%
\end{pgfscope}%
\end{pgfscope}%
\begin{pgfscope}%
\definecolor{textcolor}{rgb}{0.000000,0.000000,0.000000}%
\pgfsetstrokecolor{textcolor}%
\pgfsetfillcolor{textcolor}%
\pgftext[x=3.700000in,y=0.622778in,,top]{\color{textcolor}\sffamily\fontsize{20.000000}{24.000000}\selectfont 9}%
\end{pgfscope}%
\begin{pgfscope}%
\pgfsetbuttcap%
\pgfsetroundjoin%
\definecolor{currentfill}{rgb}{0.000000,0.000000,0.000000}%
\pgfsetfillcolor{currentfill}%
\pgfsetlinewidth{0.803000pt}%
\definecolor{currentstroke}{rgb}{0.000000,0.000000,0.000000}%
\pgfsetstrokecolor{currentstroke}%
\pgfsetdash{}{0pt}%
\pgfsys@defobject{currentmarker}{\pgfqpoint{0.000000in}{-0.048611in}}{\pgfqpoint{0.000000in}{0.000000in}}{%
\pgfpathmoveto{\pgfqpoint{0.000000in}{0.000000in}}%
\pgfpathlineto{\pgfqpoint{0.000000in}{-0.048611in}}%
\pgfusepath{stroke,fill}%
}%
\begin{pgfscope}%
\pgfsys@transformshift{4.300000in}{0.720000in}%
\pgfsys@useobject{currentmarker}{}%
\end{pgfscope}%
\end{pgfscope}%
\begin{pgfscope}%
\definecolor{textcolor}{rgb}{0.000000,0.000000,0.000000}%
\pgfsetstrokecolor{textcolor}%
\pgfsetfillcolor{textcolor}%
\pgftext[x=4.300000in,y=0.622778in,,top]{\color{textcolor}\sffamily\fontsize{20.000000}{24.000000}\selectfont 11}%
\end{pgfscope}%
\begin{pgfscope}%
\pgfsetbuttcap%
\pgfsetroundjoin%
\definecolor{currentfill}{rgb}{0.000000,0.000000,0.000000}%
\pgfsetfillcolor{currentfill}%
\pgfsetlinewidth{0.803000pt}%
\definecolor{currentstroke}{rgb}{0.000000,0.000000,0.000000}%
\pgfsetstrokecolor{currentstroke}%
\pgfsetdash{}{0pt}%
\pgfsys@defobject{currentmarker}{\pgfqpoint{0.000000in}{-0.048611in}}{\pgfqpoint{0.000000in}{0.000000in}}{%
\pgfpathmoveto{\pgfqpoint{0.000000in}{0.000000in}}%
\pgfpathlineto{\pgfqpoint{0.000000in}{-0.048611in}}%
\pgfusepath{stroke,fill}%
}%
\begin{pgfscope}%
\pgfsys@transformshift{4.900000in}{0.720000in}%
\pgfsys@useobject{currentmarker}{}%
\end{pgfscope}%
\end{pgfscope}%
\begin{pgfscope}%
\definecolor{textcolor}{rgb}{0.000000,0.000000,0.000000}%
\pgfsetstrokecolor{textcolor}%
\pgfsetfillcolor{textcolor}%
\pgftext[x=4.900000in,y=0.622778in,,top]{\color{textcolor}\sffamily\fontsize{20.000000}{24.000000}\selectfont 13}%
\end{pgfscope}%
\begin{pgfscope}%
\pgfsetbuttcap%
\pgfsetroundjoin%
\definecolor{currentfill}{rgb}{0.000000,0.000000,0.000000}%
\pgfsetfillcolor{currentfill}%
\pgfsetlinewidth{0.803000pt}%
\definecolor{currentstroke}{rgb}{0.000000,0.000000,0.000000}%
\pgfsetstrokecolor{currentstroke}%
\pgfsetdash{}{0pt}%
\pgfsys@defobject{currentmarker}{\pgfqpoint{0.000000in}{-0.048611in}}{\pgfqpoint{0.000000in}{0.000000in}}{%
\pgfpathmoveto{\pgfqpoint{0.000000in}{0.000000in}}%
\pgfpathlineto{\pgfqpoint{0.000000in}{-0.048611in}}%
\pgfusepath{stroke,fill}%
}%
\begin{pgfscope}%
\pgfsys@transformshift{5.500000in}{0.720000in}%
\pgfsys@useobject{currentmarker}{}%
\end{pgfscope}%
\end{pgfscope}%
\begin{pgfscope}%
\definecolor{textcolor}{rgb}{0.000000,0.000000,0.000000}%
\pgfsetstrokecolor{textcolor}%
\pgfsetfillcolor{textcolor}%
\pgftext[x=5.500000in,y=0.622778in,,top]{\color{textcolor}\sffamily\fontsize{20.000000}{24.000000}\selectfont 15}%
\end{pgfscope}%
\begin{pgfscope}%
\definecolor{textcolor}{rgb}{0.000000,0.000000,0.000000}%
\pgfsetstrokecolor{textcolor}%
\pgfsetfillcolor{textcolor}%
\pgftext[x=3.400000in,y=0.311155in,,top]{\color{textcolor}\sffamily\fontsize{20.000000}{24.000000}\selectfont \(\displaystyle N_{\mathrm{PE}}\)}%
\end{pgfscope}%
\begin{pgfscope}%
\pgfsetbuttcap%
\pgfsetroundjoin%
\definecolor{currentfill}{rgb}{0.000000,0.000000,0.000000}%
\pgfsetfillcolor{currentfill}%
\pgfsetlinewidth{0.803000pt}%
\definecolor{currentstroke}{rgb}{0.000000,0.000000,0.000000}%
\pgfsetstrokecolor{currentstroke}%
\pgfsetdash{}{0pt}%
\pgfsys@defobject{currentmarker}{\pgfqpoint{-0.048611in}{0.000000in}}{\pgfqpoint{-0.000000in}{0.000000in}}{%
\pgfpathmoveto{\pgfqpoint{-0.000000in}{0.000000in}}%
\pgfpathlineto{\pgfqpoint{-0.048611in}{0.000000in}}%
\pgfusepath{stroke,fill}%
}%
\begin{pgfscope}%
\pgfsys@transformshift{1.000000in}{0.720000in}%
\pgfsys@useobject{currentmarker}{}%
\end{pgfscope}%
\end{pgfscope}%
\begin{pgfscope}%
\definecolor{textcolor}{rgb}{0.000000,0.000000,0.000000}%
\pgfsetstrokecolor{textcolor}%
\pgfsetfillcolor{textcolor}%
\pgftext[x=0.428108in, y=0.619981in, left, base]{\color{textcolor}\sffamily\fontsize{20.000000}{24.000000}\selectfont \(\displaystyle {0.00}\)}%
\end{pgfscope}%
\begin{pgfscope}%
\pgfsetbuttcap%
\pgfsetroundjoin%
\definecolor{currentfill}{rgb}{0.000000,0.000000,0.000000}%
\pgfsetfillcolor{currentfill}%
\pgfsetlinewidth{0.803000pt}%
\definecolor{currentstroke}{rgb}{0.000000,0.000000,0.000000}%
\pgfsetstrokecolor{currentstroke}%
\pgfsetdash{}{0pt}%
\pgfsys@defobject{currentmarker}{\pgfqpoint{-0.048611in}{0.000000in}}{\pgfqpoint{-0.000000in}{0.000000in}}{%
\pgfpathmoveto{\pgfqpoint{-0.000000in}{0.000000in}}%
\pgfpathlineto{\pgfqpoint{-0.048611in}{0.000000in}}%
\pgfusepath{stroke,fill}%
}%
\begin{pgfscope}%
\pgfsys@transformshift{1.000000in}{1.396244in}%
\pgfsys@useobject{currentmarker}{}%
\end{pgfscope}%
\end{pgfscope}%
\begin{pgfscope}%
\definecolor{textcolor}{rgb}{0.000000,0.000000,0.000000}%
\pgfsetstrokecolor{textcolor}%
\pgfsetfillcolor{textcolor}%
\pgftext[x=0.428108in, y=1.296224in, left, base]{\color{textcolor}\sffamily\fontsize{20.000000}{24.000000}\selectfont \(\displaystyle {0.25}\)}%
\end{pgfscope}%
\begin{pgfscope}%
\pgfsetbuttcap%
\pgfsetroundjoin%
\definecolor{currentfill}{rgb}{0.000000,0.000000,0.000000}%
\pgfsetfillcolor{currentfill}%
\pgfsetlinewidth{0.803000pt}%
\definecolor{currentstroke}{rgb}{0.000000,0.000000,0.000000}%
\pgfsetstrokecolor{currentstroke}%
\pgfsetdash{}{0pt}%
\pgfsys@defobject{currentmarker}{\pgfqpoint{-0.048611in}{0.000000in}}{\pgfqpoint{-0.000000in}{0.000000in}}{%
\pgfpathmoveto{\pgfqpoint{-0.000000in}{0.000000in}}%
\pgfpathlineto{\pgfqpoint{-0.048611in}{0.000000in}}%
\pgfusepath{stroke,fill}%
}%
\begin{pgfscope}%
\pgfsys@transformshift{1.000000in}{2.072487in}%
\pgfsys@useobject{currentmarker}{}%
\end{pgfscope}%
\end{pgfscope}%
\begin{pgfscope}%
\definecolor{textcolor}{rgb}{0.000000,0.000000,0.000000}%
\pgfsetstrokecolor{textcolor}%
\pgfsetfillcolor{textcolor}%
\pgftext[x=0.428108in, y=1.972468in, left, base]{\color{textcolor}\sffamily\fontsize{20.000000}{24.000000}\selectfont \(\displaystyle {0.50}\)}%
\end{pgfscope}%
\begin{pgfscope}%
\pgfsetbuttcap%
\pgfsetroundjoin%
\definecolor{currentfill}{rgb}{0.000000,0.000000,0.000000}%
\pgfsetfillcolor{currentfill}%
\pgfsetlinewidth{0.803000pt}%
\definecolor{currentstroke}{rgb}{0.000000,0.000000,0.000000}%
\pgfsetstrokecolor{currentstroke}%
\pgfsetdash{}{0pt}%
\pgfsys@defobject{currentmarker}{\pgfqpoint{-0.048611in}{0.000000in}}{\pgfqpoint{-0.000000in}{0.000000in}}{%
\pgfpathmoveto{\pgfqpoint{-0.000000in}{0.000000in}}%
\pgfpathlineto{\pgfqpoint{-0.048611in}{0.000000in}}%
\pgfusepath{stroke,fill}%
}%
\begin{pgfscope}%
\pgfsys@transformshift{1.000000in}{2.748731in}%
\pgfsys@useobject{currentmarker}{}%
\end{pgfscope}%
\end{pgfscope}%
\begin{pgfscope}%
\definecolor{textcolor}{rgb}{0.000000,0.000000,0.000000}%
\pgfsetstrokecolor{textcolor}%
\pgfsetfillcolor{textcolor}%
\pgftext[x=0.428108in, y=2.648711in, left, base]{\color{textcolor}\sffamily\fontsize{20.000000}{24.000000}\selectfont \(\displaystyle {0.75}\)}%
\end{pgfscope}%
\begin{pgfscope}%
\pgfsetbuttcap%
\pgfsetroundjoin%
\definecolor{currentfill}{rgb}{0.000000,0.000000,0.000000}%
\pgfsetfillcolor{currentfill}%
\pgfsetlinewidth{0.803000pt}%
\definecolor{currentstroke}{rgb}{0.000000,0.000000,0.000000}%
\pgfsetstrokecolor{currentstroke}%
\pgfsetdash{}{0pt}%
\pgfsys@defobject{currentmarker}{\pgfqpoint{-0.048611in}{0.000000in}}{\pgfqpoint{-0.000000in}{0.000000in}}{%
\pgfpathmoveto{\pgfqpoint{-0.000000in}{0.000000in}}%
\pgfpathlineto{\pgfqpoint{-0.048611in}{0.000000in}}%
\pgfusepath{stroke,fill}%
}%
\begin{pgfscope}%
\pgfsys@transformshift{1.000000in}{3.424974in}%
\pgfsys@useobject{currentmarker}{}%
\end{pgfscope}%
\end{pgfscope}%
\begin{pgfscope}%
\definecolor{textcolor}{rgb}{0.000000,0.000000,0.000000}%
\pgfsetstrokecolor{textcolor}%
\pgfsetfillcolor{textcolor}%
\pgftext[x=0.428108in, y=3.324955in, left, base]{\color{textcolor}\sffamily\fontsize{20.000000}{24.000000}\selectfont \(\displaystyle {1.00}\)}%
\end{pgfscope}%
\begin{pgfscope}%
\pgfsetbuttcap%
\pgfsetroundjoin%
\definecolor{currentfill}{rgb}{0.000000,0.000000,0.000000}%
\pgfsetfillcolor{currentfill}%
\pgfsetlinewidth{0.803000pt}%
\definecolor{currentstroke}{rgb}{0.000000,0.000000,0.000000}%
\pgfsetstrokecolor{currentstroke}%
\pgfsetdash{}{0pt}%
\pgfsys@defobject{currentmarker}{\pgfqpoint{-0.048611in}{0.000000in}}{\pgfqpoint{-0.000000in}{0.000000in}}{%
\pgfpathmoveto{\pgfqpoint{-0.000000in}{0.000000in}}%
\pgfpathlineto{\pgfqpoint{-0.048611in}{0.000000in}}%
\pgfusepath{stroke,fill}%
}%
\begin{pgfscope}%
\pgfsys@transformshift{1.000000in}{4.101218in}%
\pgfsys@useobject{currentmarker}{}%
\end{pgfscope}%
\end{pgfscope}%
\begin{pgfscope}%
\definecolor{textcolor}{rgb}{0.000000,0.000000,0.000000}%
\pgfsetstrokecolor{textcolor}%
\pgfsetfillcolor{textcolor}%
\pgftext[x=0.428108in, y=4.001198in, left, base]{\color{textcolor}\sffamily\fontsize{20.000000}{24.000000}\selectfont \(\displaystyle {1.25}\)}%
\end{pgfscope}%
\begin{pgfscope}%
\pgfsetbuttcap%
\pgfsetroundjoin%
\definecolor{currentfill}{rgb}{0.000000,0.000000,0.000000}%
\pgfsetfillcolor{currentfill}%
\pgfsetlinewidth{0.803000pt}%
\definecolor{currentstroke}{rgb}{0.000000,0.000000,0.000000}%
\pgfsetstrokecolor{currentstroke}%
\pgfsetdash{}{0pt}%
\pgfsys@defobject{currentmarker}{\pgfqpoint{-0.048611in}{0.000000in}}{\pgfqpoint{-0.000000in}{0.000000in}}{%
\pgfpathmoveto{\pgfqpoint{-0.000000in}{0.000000in}}%
\pgfpathlineto{\pgfqpoint{-0.048611in}{0.000000in}}%
\pgfusepath{stroke,fill}%
}%
\begin{pgfscope}%
\pgfsys@transformshift{1.000000in}{4.777461in}%
\pgfsys@useobject{currentmarker}{}%
\end{pgfscope}%
\end{pgfscope}%
\begin{pgfscope}%
\definecolor{textcolor}{rgb}{0.000000,0.000000,0.000000}%
\pgfsetstrokecolor{textcolor}%
\pgfsetfillcolor{textcolor}%
\pgftext[x=0.428108in, y=4.677442in, left, base]{\color{textcolor}\sffamily\fontsize{20.000000}{24.000000}\selectfont \(\displaystyle {1.50}\)}%
\end{pgfscope}%
\begin{pgfscope}%
\definecolor{textcolor}{rgb}{0.000000,0.000000,0.000000}%
\pgfsetstrokecolor{textcolor}%
\pgfsetfillcolor{textcolor}%
\pgftext[x=0.372552in,y=3.030000in,,bottom,rotate=90.000000]{\color{textcolor}\sffamily\fontsize{20.000000}{24.000000}\selectfont \(\displaystyle \mathrm{Wasserstein\ Distance}/\si{ns}\)}%
\end{pgfscope}%
\begin{pgfscope}%
\pgfpathrectangle{\pgfqpoint{1.000000in}{0.720000in}}{\pgfqpoint{4.800000in}{4.620000in}}%
\pgfusepath{clip}%
\pgfsetrectcap%
\pgfsetroundjoin%
\pgfsetlinewidth{2.007500pt}%
\definecolor{currentstroke}{rgb}{0.000000,0.000000,1.000000}%
\pgfsetstrokecolor{currentstroke}%
\pgfsetdash{}{0pt}%
\pgfpathmoveto{\pgfqpoint{1.300000in}{2.131781in}}%
\pgfpathlineto{\pgfqpoint{1.600000in}{2.670089in}}%
\pgfpathlineto{\pgfqpoint{1.900000in}{2.983333in}}%
\pgfpathlineto{\pgfqpoint{2.200000in}{3.069187in}}%
\pgfpathlineto{\pgfqpoint{2.500000in}{3.119901in}}%
\pgfpathlineto{\pgfqpoint{2.800000in}{3.130642in}}%
\pgfpathlineto{\pgfqpoint{3.100000in}{3.150652in}}%
\pgfpathlineto{\pgfqpoint{3.400000in}{3.235519in}}%
\pgfpathlineto{\pgfqpoint{3.700000in}{3.177816in}}%
\pgfpathlineto{\pgfqpoint{4.000000in}{3.144073in}}%
\pgfpathlineto{\pgfqpoint{4.300000in}{3.215336in}}%
\pgfpathlineto{\pgfqpoint{4.600000in}{3.450986in}}%
\pgfpathlineto{\pgfqpoint{4.900000in}{3.598089in}}%
\pgfpathlineto{\pgfqpoint{5.200000in}{2.856242in}}%
\pgfpathlineto{\pgfqpoint{5.500000in}{3.493745in}}%
\pgfusepath{stroke}%
\end{pgfscope}%
\begin{pgfscope}%
\pgfpathrectangle{\pgfqpoint{1.000000in}{0.720000in}}{\pgfqpoint{4.800000in}{4.620000in}}%
\pgfusepath{clip}%
\pgfsetbuttcap%
\pgfsetroundjoin%
\pgfsetlinewidth{1.003750pt}%
\definecolor{currentstroke}{rgb}{0.000000,0.000000,1.000000}%
\pgfsetstrokecolor{currentstroke}%
\pgfsetdash{}{0pt}%
\pgfpathmoveto{\pgfqpoint{1.300000in}{1.191247in}}%
\pgfpathlineto{\pgfqpoint{1.300000in}{3.031456in}}%
\pgfusepath{stroke}%
\end{pgfscope}%
\begin{pgfscope}%
\pgfpathrectangle{\pgfqpoint{1.000000in}{0.720000in}}{\pgfqpoint{4.800000in}{4.620000in}}%
\pgfusepath{clip}%
\pgfsetbuttcap%
\pgfsetroundjoin%
\pgfsetlinewidth{1.003750pt}%
\definecolor{currentstroke}{rgb}{0.000000,0.000000,1.000000}%
\pgfsetstrokecolor{currentstroke}%
\pgfsetdash{}{0pt}%
\pgfpathmoveto{\pgfqpoint{1.600000in}{1.821221in}}%
\pgfpathlineto{\pgfqpoint{1.600000in}{3.763289in}}%
\pgfusepath{stroke}%
\end{pgfscope}%
\begin{pgfscope}%
\pgfpathrectangle{\pgfqpoint{1.000000in}{0.720000in}}{\pgfqpoint{4.800000in}{4.620000in}}%
\pgfusepath{clip}%
\pgfsetbuttcap%
\pgfsetroundjoin%
\pgfsetlinewidth{1.003750pt}%
\definecolor{currentstroke}{rgb}{0.000000,0.000000,1.000000}%
\pgfsetstrokecolor{currentstroke}%
\pgfsetdash{}{0pt}%
\pgfpathmoveto{\pgfqpoint{1.900000in}{2.190063in}}%
\pgfpathlineto{\pgfqpoint{1.900000in}{3.996053in}}%
\pgfusepath{stroke}%
\end{pgfscope}%
\begin{pgfscope}%
\pgfpathrectangle{\pgfqpoint{1.000000in}{0.720000in}}{\pgfqpoint{4.800000in}{4.620000in}}%
\pgfusepath{clip}%
\pgfsetbuttcap%
\pgfsetroundjoin%
\pgfsetlinewidth{1.003750pt}%
\definecolor{currentstroke}{rgb}{0.000000,0.000000,1.000000}%
\pgfsetstrokecolor{currentstroke}%
\pgfsetdash{}{0pt}%
\pgfpathmoveto{\pgfqpoint{2.200000in}{2.327301in}}%
\pgfpathlineto{\pgfqpoint{2.200000in}{4.084868in}}%
\pgfusepath{stroke}%
\end{pgfscope}%
\begin{pgfscope}%
\pgfpathrectangle{\pgfqpoint{1.000000in}{0.720000in}}{\pgfqpoint{4.800000in}{4.620000in}}%
\pgfusepath{clip}%
\pgfsetbuttcap%
\pgfsetroundjoin%
\pgfsetlinewidth{1.003750pt}%
\definecolor{currentstroke}{rgb}{0.000000,0.000000,1.000000}%
\pgfsetstrokecolor{currentstroke}%
\pgfsetdash{}{0pt}%
\pgfpathmoveto{\pgfqpoint{2.500000in}{2.428034in}}%
\pgfpathlineto{\pgfqpoint{2.500000in}{3.983481in}}%
\pgfusepath{stroke}%
\end{pgfscope}%
\begin{pgfscope}%
\pgfpathrectangle{\pgfqpoint{1.000000in}{0.720000in}}{\pgfqpoint{4.800000in}{4.620000in}}%
\pgfusepath{clip}%
\pgfsetbuttcap%
\pgfsetroundjoin%
\pgfsetlinewidth{1.003750pt}%
\definecolor{currentstroke}{rgb}{0.000000,0.000000,1.000000}%
\pgfsetstrokecolor{currentstroke}%
\pgfsetdash{}{0pt}%
\pgfpathmoveto{\pgfqpoint{2.800000in}{2.511331in}}%
\pgfpathlineto{\pgfqpoint{2.800000in}{4.040753in}}%
\pgfusepath{stroke}%
\end{pgfscope}%
\begin{pgfscope}%
\pgfpathrectangle{\pgfqpoint{1.000000in}{0.720000in}}{\pgfqpoint{4.800000in}{4.620000in}}%
\pgfusepath{clip}%
\pgfsetbuttcap%
\pgfsetroundjoin%
\pgfsetlinewidth{1.003750pt}%
\definecolor{currentstroke}{rgb}{0.000000,0.000000,1.000000}%
\pgfsetstrokecolor{currentstroke}%
\pgfsetdash{}{0pt}%
\pgfpathmoveto{\pgfqpoint{3.100000in}{2.528322in}}%
\pgfpathlineto{\pgfqpoint{3.100000in}{3.900421in}}%
\pgfusepath{stroke}%
\end{pgfscope}%
\begin{pgfscope}%
\pgfpathrectangle{\pgfqpoint{1.000000in}{0.720000in}}{\pgfqpoint{4.800000in}{4.620000in}}%
\pgfusepath{clip}%
\pgfsetbuttcap%
\pgfsetroundjoin%
\pgfsetlinewidth{1.003750pt}%
\definecolor{currentstroke}{rgb}{0.000000,0.000000,1.000000}%
\pgfsetstrokecolor{currentstroke}%
\pgfsetdash{}{0pt}%
\pgfpathmoveto{\pgfqpoint{3.400000in}{2.622883in}}%
\pgfpathlineto{\pgfqpoint{3.400000in}{3.775141in}}%
\pgfusepath{stroke}%
\end{pgfscope}%
\begin{pgfscope}%
\pgfpathrectangle{\pgfqpoint{1.000000in}{0.720000in}}{\pgfqpoint{4.800000in}{4.620000in}}%
\pgfusepath{clip}%
\pgfsetbuttcap%
\pgfsetroundjoin%
\pgfsetlinewidth{1.003750pt}%
\definecolor{currentstroke}{rgb}{0.000000,0.000000,1.000000}%
\pgfsetstrokecolor{currentstroke}%
\pgfsetdash{}{0pt}%
\pgfpathmoveto{\pgfqpoint{3.700000in}{2.648169in}}%
\pgfpathlineto{\pgfqpoint{3.700000in}{3.741645in}}%
\pgfusepath{stroke}%
\end{pgfscope}%
\begin{pgfscope}%
\pgfpathrectangle{\pgfqpoint{1.000000in}{0.720000in}}{\pgfqpoint{4.800000in}{4.620000in}}%
\pgfusepath{clip}%
\pgfsetbuttcap%
\pgfsetroundjoin%
\pgfsetlinewidth{1.003750pt}%
\definecolor{currentstroke}{rgb}{0.000000,0.000000,1.000000}%
\pgfsetstrokecolor{currentstroke}%
\pgfsetdash{}{0pt}%
\pgfpathmoveto{\pgfqpoint{4.000000in}{2.482835in}}%
\pgfpathlineto{\pgfqpoint{4.000000in}{3.664786in}}%
\pgfusepath{stroke}%
\end{pgfscope}%
\begin{pgfscope}%
\pgfpathrectangle{\pgfqpoint{1.000000in}{0.720000in}}{\pgfqpoint{4.800000in}{4.620000in}}%
\pgfusepath{clip}%
\pgfsetbuttcap%
\pgfsetroundjoin%
\pgfsetlinewidth{1.003750pt}%
\definecolor{currentstroke}{rgb}{0.000000,0.000000,1.000000}%
\pgfsetstrokecolor{currentstroke}%
\pgfsetdash{}{0pt}%
\pgfpathmoveto{\pgfqpoint{4.300000in}{2.532402in}}%
\pgfpathlineto{\pgfqpoint{4.300000in}{3.695576in}}%
\pgfusepath{stroke}%
\end{pgfscope}%
\begin{pgfscope}%
\pgfpathrectangle{\pgfqpoint{1.000000in}{0.720000in}}{\pgfqpoint{4.800000in}{4.620000in}}%
\pgfusepath{clip}%
\pgfsetbuttcap%
\pgfsetroundjoin%
\pgfsetlinewidth{1.003750pt}%
\definecolor{currentstroke}{rgb}{0.000000,0.000000,1.000000}%
\pgfsetstrokecolor{currentstroke}%
\pgfsetdash{}{0pt}%
\pgfpathmoveto{\pgfqpoint{4.600000in}{3.127639in}}%
\pgfpathlineto{\pgfqpoint{4.600000in}{3.699964in}}%
\pgfusepath{stroke}%
\end{pgfscope}%
\begin{pgfscope}%
\pgfpathrectangle{\pgfqpoint{1.000000in}{0.720000in}}{\pgfqpoint{4.800000in}{4.620000in}}%
\pgfusepath{clip}%
\pgfsetbuttcap%
\pgfsetroundjoin%
\pgfsetlinewidth{1.003750pt}%
\definecolor{currentstroke}{rgb}{0.000000,0.000000,1.000000}%
\pgfsetstrokecolor{currentstroke}%
\pgfsetdash{}{0pt}%
\pgfpathmoveto{\pgfqpoint{4.900000in}{3.598089in}}%
\pgfpathlineto{\pgfqpoint{4.900000in}{3.598089in}}%
\pgfusepath{stroke}%
\end{pgfscope}%
\begin{pgfscope}%
\pgfpathrectangle{\pgfqpoint{1.000000in}{0.720000in}}{\pgfqpoint{4.800000in}{4.620000in}}%
\pgfusepath{clip}%
\pgfsetbuttcap%
\pgfsetroundjoin%
\pgfsetlinewidth{1.003750pt}%
\definecolor{currentstroke}{rgb}{0.000000,0.000000,1.000000}%
\pgfsetstrokecolor{currentstroke}%
\pgfsetdash{}{0pt}%
\pgfpathmoveto{\pgfqpoint{5.200000in}{2.856242in}}%
\pgfpathlineto{\pgfqpoint{5.200000in}{2.856242in}}%
\pgfusepath{stroke}%
\end{pgfscope}%
\begin{pgfscope}%
\pgfpathrectangle{\pgfqpoint{1.000000in}{0.720000in}}{\pgfqpoint{4.800000in}{4.620000in}}%
\pgfusepath{clip}%
\pgfsetbuttcap%
\pgfsetroundjoin%
\pgfsetlinewidth{1.003750pt}%
\definecolor{currentstroke}{rgb}{0.000000,0.000000,1.000000}%
\pgfsetstrokecolor{currentstroke}%
\pgfsetdash{}{0pt}%
\pgfpathmoveto{\pgfqpoint{5.500000in}{3.493745in}}%
\pgfpathlineto{\pgfqpoint{5.500000in}{3.493745in}}%
\pgfusepath{stroke}%
\end{pgfscope}%
\begin{pgfscope}%
\pgfpathrectangle{\pgfqpoint{1.000000in}{0.720000in}}{\pgfqpoint{4.800000in}{4.620000in}}%
\pgfusepath{clip}%
\pgfsetbuttcap%
\pgfsetroundjoin%
\definecolor{currentfill}{rgb}{0.000000,0.000000,1.000000}%
\pgfsetfillcolor{currentfill}%
\pgfsetlinewidth{1.003750pt}%
\definecolor{currentstroke}{rgb}{0.000000,0.000000,1.000000}%
\pgfsetstrokecolor{currentstroke}%
\pgfsetdash{}{0pt}%
\pgfsys@defobject{currentmarker}{\pgfqpoint{-0.041667in}{-0.000000in}}{\pgfqpoint{0.041667in}{0.000000in}}{%
\pgfpathmoveto{\pgfqpoint{0.041667in}{-0.000000in}}%
\pgfpathlineto{\pgfqpoint{-0.041667in}{0.000000in}}%
\pgfusepath{stroke,fill}%
}%
\begin{pgfscope}%
\pgfsys@transformshift{1.300000in}{1.191247in}%
\pgfsys@useobject{currentmarker}{}%
\end{pgfscope}%
\begin{pgfscope}%
\pgfsys@transformshift{1.600000in}{1.821221in}%
\pgfsys@useobject{currentmarker}{}%
\end{pgfscope}%
\begin{pgfscope}%
\pgfsys@transformshift{1.900000in}{2.190063in}%
\pgfsys@useobject{currentmarker}{}%
\end{pgfscope}%
\begin{pgfscope}%
\pgfsys@transformshift{2.200000in}{2.327301in}%
\pgfsys@useobject{currentmarker}{}%
\end{pgfscope}%
\begin{pgfscope}%
\pgfsys@transformshift{2.500000in}{2.428034in}%
\pgfsys@useobject{currentmarker}{}%
\end{pgfscope}%
\begin{pgfscope}%
\pgfsys@transformshift{2.800000in}{2.511331in}%
\pgfsys@useobject{currentmarker}{}%
\end{pgfscope}%
\begin{pgfscope}%
\pgfsys@transformshift{3.100000in}{2.528322in}%
\pgfsys@useobject{currentmarker}{}%
\end{pgfscope}%
\begin{pgfscope}%
\pgfsys@transformshift{3.400000in}{2.622883in}%
\pgfsys@useobject{currentmarker}{}%
\end{pgfscope}%
\begin{pgfscope}%
\pgfsys@transformshift{3.700000in}{2.648169in}%
\pgfsys@useobject{currentmarker}{}%
\end{pgfscope}%
\begin{pgfscope}%
\pgfsys@transformshift{4.000000in}{2.482835in}%
\pgfsys@useobject{currentmarker}{}%
\end{pgfscope}%
\begin{pgfscope}%
\pgfsys@transformshift{4.300000in}{2.532402in}%
\pgfsys@useobject{currentmarker}{}%
\end{pgfscope}%
\begin{pgfscope}%
\pgfsys@transformshift{4.600000in}{3.127639in}%
\pgfsys@useobject{currentmarker}{}%
\end{pgfscope}%
\begin{pgfscope}%
\pgfsys@transformshift{4.900000in}{3.598089in}%
\pgfsys@useobject{currentmarker}{}%
\end{pgfscope}%
\begin{pgfscope}%
\pgfsys@transformshift{5.200000in}{2.856242in}%
\pgfsys@useobject{currentmarker}{}%
\end{pgfscope}%
\begin{pgfscope}%
\pgfsys@transformshift{5.500000in}{3.493745in}%
\pgfsys@useobject{currentmarker}{}%
\end{pgfscope}%
\end{pgfscope}%
\begin{pgfscope}%
\pgfpathrectangle{\pgfqpoint{1.000000in}{0.720000in}}{\pgfqpoint{4.800000in}{4.620000in}}%
\pgfusepath{clip}%
\pgfsetbuttcap%
\pgfsetroundjoin%
\definecolor{currentfill}{rgb}{0.000000,0.000000,1.000000}%
\pgfsetfillcolor{currentfill}%
\pgfsetlinewidth{1.003750pt}%
\definecolor{currentstroke}{rgb}{0.000000,0.000000,1.000000}%
\pgfsetstrokecolor{currentstroke}%
\pgfsetdash{}{0pt}%
\pgfsys@defobject{currentmarker}{\pgfqpoint{-0.041667in}{-0.000000in}}{\pgfqpoint{0.041667in}{0.000000in}}{%
\pgfpathmoveto{\pgfqpoint{0.041667in}{-0.000000in}}%
\pgfpathlineto{\pgfqpoint{-0.041667in}{0.000000in}}%
\pgfusepath{stroke,fill}%
}%
\begin{pgfscope}%
\pgfsys@transformshift{1.300000in}{3.031456in}%
\pgfsys@useobject{currentmarker}{}%
\end{pgfscope}%
\begin{pgfscope}%
\pgfsys@transformshift{1.600000in}{3.763289in}%
\pgfsys@useobject{currentmarker}{}%
\end{pgfscope}%
\begin{pgfscope}%
\pgfsys@transformshift{1.900000in}{3.996053in}%
\pgfsys@useobject{currentmarker}{}%
\end{pgfscope}%
\begin{pgfscope}%
\pgfsys@transformshift{2.200000in}{4.084868in}%
\pgfsys@useobject{currentmarker}{}%
\end{pgfscope}%
\begin{pgfscope}%
\pgfsys@transformshift{2.500000in}{3.983481in}%
\pgfsys@useobject{currentmarker}{}%
\end{pgfscope}%
\begin{pgfscope}%
\pgfsys@transformshift{2.800000in}{4.040753in}%
\pgfsys@useobject{currentmarker}{}%
\end{pgfscope}%
\begin{pgfscope}%
\pgfsys@transformshift{3.100000in}{3.900421in}%
\pgfsys@useobject{currentmarker}{}%
\end{pgfscope}%
\begin{pgfscope}%
\pgfsys@transformshift{3.400000in}{3.775141in}%
\pgfsys@useobject{currentmarker}{}%
\end{pgfscope}%
\begin{pgfscope}%
\pgfsys@transformshift{3.700000in}{3.741645in}%
\pgfsys@useobject{currentmarker}{}%
\end{pgfscope}%
\begin{pgfscope}%
\pgfsys@transformshift{4.000000in}{3.664786in}%
\pgfsys@useobject{currentmarker}{}%
\end{pgfscope}%
\begin{pgfscope}%
\pgfsys@transformshift{4.300000in}{3.695576in}%
\pgfsys@useobject{currentmarker}{}%
\end{pgfscope}%
\begin{pgfscope}%
\pgfsys@transformshift{4.600000in}{3.699964in}%
\pgfsys@useobject{currentmarker}{}%
\end{pgfscope}%
\begin{pgfscope}%
\pgfsys@transformshift{4.900000in}{3.598089in}%
\pgfsys@useobject{currentmarker}{}%
\end{pgfscope}%
\begin{pgfscope}%
\pgfsys@transformshift{5.200000in}{2.856242in}%
\pgfsys@useobject{currentmarker}{}%
\end{pgfscope}%
\begin{pgfscope}%
\pgfsys@transformshift{5.500000in}{3.493745in}%
\pgfsys@useobject{currentmarker}{}%
\end{pgfscope}%
\end{pgfscope}%
\begin{pgfscope}%
\pgfpathrectangle{\pgfqpoint{1.000000in}{0.720000in}}{\pgfqpoint{4.800000in}{4.620000in}}%
\pgfusepath{clip}%
\pgfsetbuttcap%
\pgfsetroundjoin%
\definecolor{currentfill}{rgb}{0.000000,0.000000,1.000000}%
\pgfsetfillcolor{currentfill}%
\pgfsetlinewidth{1.003750pt}%
\definecolor{currentstroke}{rgb}{0.000000,0.000000,1.000000}%
\pgfsetstrokecolor{currentstroke}%
\pgfsetdash{}{0pt}%
\pgfsys@defobject{currentmarker}{\pgfqpoint{-0.027778in}{-0.027778in}}{\pgfqpoint{0.027778in}{0.027778in}}{%
\pgfpathmoveto{\pgfqpoint{0.000000in}{-0.027778in}}%
\pgfpathcurveto{\pgfqpoint{0.007367in}{-0.027778in}}{\pgfqpoint{0.014433in}{-0.024851in}}{\pgfqpoint{0.019642in}{-0.019642in}}%
\pgfpathcurveto{\pgfqpoint{0.024851in}{-0.014433in}}{\pgfqpoint{0.027778in}{-0.007367in}}{\pgfqpoint{0.027778in}{0.000000in}}%
\pgfpathcurveto{\pgfqpoint{0.027778in}{0.007367in}}{\pgfqpoint{0.024851in}{0.014433in}}{\pgfqpoint{0.019642in}{0.019642in}}%
\pgfpathcurveto{\pgfqpoint{0.014433in}{0.024851in}}{\pgfqpoint{0.007367in}{0.027778in}}{\pgfqpoint{0.000000in}{0.027778in}}%
\pgfpathcurveto{\pgfqpoint{-0.007367in}{0.027778in}}{\pgfqpoint{-0.014433in}{0.024851in}}{\pgfqpoint{-0.019642in}{0.019642in}}%
\pgfpathcurveto{\pgfqpoint{-0.024851in}{0.014433in}}{\pgfqpoint{-0.027778in}{0.007367in}}{\pgfqpoint{-0.027778in}{0.000000in}}%
\pgfpathcurveto{\pgfqpoint{-0.027778in}{-0.007367in}}{\pgfqpoint{-0.024851in}{-0.014433in}}{\pgfqpoint{-0.019642in}{-0.019642in}}%
\pgfpathcurveto{\pgfqpoint{-0.014433in}{-0.024851in}}{\pgfqpoint{-0.007367in}{-0.027778in}}{\pgfqpoint{0.000000in}{-0.027778in}}%
\pgfpathlineto{\pgfqpoint{0.000000in}{-0.027778in}}%
\pgfpathclose%
\pgfusepath{stroke,fill}%
}%
\begin{pgfscope}%
\pgfsys@transformshift{1.300000in}{2.131781in}%
\pgfsys@useobject{currentmarker}{}%
\end{pgfscope}%
\begin{pgfscope}%
\pgfsys@transformshift{1.600000in}{2.670089in}%
\pgfsys@useobject{currentmarker}{}%
\end{pgfscope}%
\begin{pgfscope}%
\pgfsys@transformshift{1.900000in}{2.983333in}%
\pgfsys@useobject{currentmarker}{}%
\end{pgfscope}%
\begin{pgfscope}%
\pgfsys@transformshift{2.200000in}{3.069187in}%
\pgfsys@useobject{currentmarker}{}%
\end{pgfscope}%
\begin{pgfscope}%
\pgfsys@transformshift{2.500000in}{3.119901in}%
\pgfsys@useobject{currentmarker}{}%
\end{pgfscope}%
\begin{pgfscope}%
\pgfsys@transformshift{2.800000in}{3.130642in}%
\pgfsys@useobject{currentmarker}{}%
\end{pgfscope}%
\begin{pgfscope}%
\pgfsys@transformshift{3.100000in}{3.150652in}%
\pgfsys@useobject{currentmarker}{}%
\end{pgfscope}%
\begin{pgfscope}%
\pgfsys@transformshift{3.400000in}{3.235519in}%
\pgfsys@useobject{currentmarker}{}%
\end{pgfscope}%
\begin{pgfscope}%
\pgfsys@transformshift{3.700000in}{3.177816in}%
\pgfsys@useobject{currentmarker}{}%
\end{pgfscope}%
\begin{pgfscope}%
\pgfsys@transformshift{4.000000in}{3.144073in}%
\pgfsys@useobject{currentmarker}{}%
\end{pgfscope}%
\begin{pgfscope}%
\pgfsys@transformshift{4.300000in}{3.215336in}%
\pgfsys@useobject{currentmarker}{}%
\end{pgfscope}%
\begin{pgfscope}%
\pgfsys@transformshift{4.600000in}{3.450986in}%
\pgfsys@useobject{currentmarker}{}%
\end{pgfscope}%
\begin{pgfscope}%
\pgfsys@transformshift{4.900000in}{3.598089in}%
\pgfsys@useobject{currentmarker}{}%
\end{pgfscope}%
\begin{pgfscope}%
\pgfsys@transformshift{5.200000in}{2.856242in}%
\pgfsys@useobject{currentmarker}{}%
\end{pgfscope}%
\begin{pgfscope}%
\pgfsys@transformshift{5.500000in}{3.493745in}%
\pgfsys@useobject{currentmarker}{}%
\end{pgfscope}%
\end{pgfscope}%
\begin{pgfscope}%
\pgfsetrectcap%
\pgfsetmiterjoin%
\pgfsetlinewidth{0.803000pt}%
\definecolor{currentstroke}{rgb}{0.000000,0.000000,0.000000}%
\pgfsetstrokecolor{currentstroke}%
\pgfsetdash{}{0pt}%
\pgfpathmoveto{\pgfqpoint{1.000000in}{0.720000in}}%
\pgfpathlineto{\pgfqpoint{1.000000in}{5.340000in}}%
\pgfusepath{stroke}%
\end{pgfscope}%
\begin{pgfscope}%
\pgfsetrectcap%
\pgfsetmiterjoin%
\pgfsetlinewidth{0.803000pt}%
\definecolor{currentstroke}{rgb}{0.000000,0.000000,0.000000}%
\pgfsetstrokecolor{currentstroke}%
\pgfsetdash{}{0pt}%
\pgfpathmoveto{\pgfqpoint{5.800000in}{0.720000in}}%
\pgfpathlineto{\pgfqpoint{5.800000in}{5.340000in}}%
\pgfusepath{stroke}%
\end{pgfscope}%
\begin{pgfscope}%
\pgfsetrectcap%
\pgfsetmiterjoin%
\pgfsetlinewidth{0.803000pt}%
\definecolor{currentstroke}{rgb}{0.000000,0.000000,0.000000}%
\pgfsetstrokecolor{currentstroke}%
\pgfsetdash{}{0pt}%
\pgfpathmoveto{\pgfqpoint{1.000000in}{0.720000in}}%
\pgfpathlineto{\pgfqpoint{5.800000in}{0.720000in}}%
\pgfusepath{stroke}%
\end{pgfscope}%
\begin{pgfscope}%
\pgfsetrectcap%
\pgfsetmiterjoin%
\pgfsetlinewidth{0.803000pt}%
\definecolor{currentstroke}{rgb}{0.000000,0.000000,0.000000}%
\pgfsetstrokecolor{currentstroke}%
\pgfsetdash{}{0pt}%
\pgfpathmoveto{\pgfqpoint{1.000000in}{5.340000in}}%
\pgfpathlineto{\pgfqpoint{5.800000in}{5.340000in}}%
\pgfusepath{stroke}%
\end{pgfscope}%
\begin{pgfscope}%
\pgfsetbuttcap%
\pgfsetmiterjoin%
\definecolor{currentfill}{rgb}{1.000000,1.000000,1.000000}%
\pgfsetfillcolor{currentfill}%
\pgfsetfillopacity{0.800000}%
\pgfsetlinewidth{1.003750pt}%
\definecolor{currentstroke}{rgb}{0.800000,0.800000,0.800000}%
\pgfsetstrokecolor{currentstroke}%
\pgfsetstrokeopacity{0.800000}%
\pgfsetdash{}{0pt}%
\pgfpathmoveto{\pgfqpoint{4.331386in}{4.722821in}}%
\pgfpathlineto{\pgfqpoint{5.605556in}{4.722821in}}%
\pgfpathquadraticcurveto{\pgfqpoint{5.661111in}{4.722821in}}{\pgfqpoint{5.661111in}{4.778377in}}%
\pgfpathlineto{\pgfqpoint{5.661111in}{5.145556in}}%
\pgfpathquadraticcurveto{\pgfqpoint{5.661111in}{5.201111in}}{\pgfqpoint{5.605556in}{5.201111in}}%
\pgfpathlineto{\pgfqpoint{4.331386in}{5.201111in}}%
\pgfpathquadraticcurveto{\pgfqpoint{4.275830in}{5.201111in}}{\pgfqpoint{4.275830in}{5.145556in}}%
\pgfpathlineto{\pgfqpoint{4.275830in}{4.778377in}}%
\pgfpathquadraticcurveto{\pgfqpoint{4.275830in}{4.722821in}}{\pgfqpoint{4.331386in}{4.722821in}}%
\pgfpathlineto{\pgfqpoint{4.331386in}{4.722821in}}%
\pgfpathclose%
\pgfusepath{stroke,fill}%
\end{pgfscope}%
\begin{pgfscope}%
\pgfsetbuttcap%
\pgfsetroundjoin%
\pgfsetlinewidth{1.003750pt}%
\definecolor{currentstroke}{rgb}{0.000000,0.000000,1.000000}%
\pgfsetstrokecolor{currentstroke}%
\pgfsetdash{}{0pt}%
\pgfpathmoveto{\pgfqpoint{4.664719in}{4.848295in}}%
\pgfpathlineto{\pgfqpoint{4.664719in}{5.126073in}}%
\pgfusepath{stroke}%
\end{pgfscope}%
\begin{pgfscope}%
\pgfsetbuttcap%
\pgfsetroundjoin%
\definecolor{currentfill}{rgb}{0.000000,0.000000,1.000000}%
\pgfsetfillcolor{currentfill}%
\pgfsetlinewidth{1.003750pt}%
\definecolor{currentstroke}{rgb}{0.000000,0.000000,1.000000}%
\pgfsetstrokecolor{currentstroke}%
\pgfsetdash{}{0pt}%
\pgfsys@defobject{currentmarker}{\pgfqpoint{-0.041667in}{-0.000000in}}{\pgfqpoint{0.041667in}{0.000000in}}{%
\pgfpathmoveto{\pgfqpoint{0.041667in}{-0.000000in}}%
\pgfpathlineto{\pgfqpoint{-0.041667in}{0.000000in}}%
\pgfusepath{stroke,fill}%
}%
\begin{pgfscope}%
\pgfsys@transformshift{4.664719in}{4.848295in}%
\pgfsys@useobject{currentmarker}{}%
\end{pgfscope}%
\end{pgfscope}%
\begin{pgfscope}%
\pgfsetbuttcap%
\pgfsetroundjoin%
\definecolor{currentfill}{rgb}{0.000000,0.000000,1.000000}%
\pgfsetfillcolor{currentfill}%
\pgfsetlinewidth{1.003750pt}%
\definecolor{currentstroke}{rgb}{0.000000,0.000000,1.000000}%
\pgfsetstrokecolor{currentstroke}%
\pgfsetdash{}{0pt}%
\pgfsys@defobject{currentmarker}{\pgfqpoint{-0.041667in}{-0.000000in}}{\pgfqpoint{0.041667in}{0.000000in}}{%
\pgfpathmoveto{\pgfqpoint{0.041667in}{-0.000000in}}%
\pgfpathlineto{\pgfqpoint{-0.041667in}{0.000000in}}%
\pgfusepath{stroke,fill}%
}%
\begin{pgfscope}%
\pgfsys@transformshift{4.664719in}{5.126073in}%
\pgfsys@useobject{currentmarker}{}%
\end{pgfscope}%
\end{pgfscope}%
\begin{pgfscope}%
\pgfsetbuttcap%
\pgfsetroundjoin%
\definecolor{currentfill}{rgb}{0.000000,0.000000,1.000000}%
\pgfsetfillcolor{currentfill}%
\pgfsetlinewidth{1.003750pt}%
\definecolor{currentstroke}{rgb}{0.000000,0.000000,1.000000}%
\pgfsetstrokecolor{currentstroke}%
\pgfsetdash{}{0pt}%
\pgfsys@defobject{currentmarker}{\pgfqpoint{-0.027778in}{-0.027778in}}{\pgfqpoint{0.027778in}{0.027778in}}{%
\pgfpathmoveto{\pgfqpoint{0.000000in}{-0.027778in}}%
\pgfpathcurveto{\pgfqpoint{0.007367in}{-0.027778in}}{\pgfqpoint{0.014433in}{-0.024851in}}{\pgfqpoint{0.019642in}{-0.019642in}}%
\pgfpathcurveto{\pgfqpoint{0.024851in}{-0.014433in}}{\pgfqpoint{0.027778in}{-0.007367in}}{\pgfqpoint{0.027778in}{0.000000in}}%
\pgfpathcurveto{\pgfqpoint{0.027778in}{0.007367in}}{\pgfqpoint{0.024851in}{0.014433in}}{\pgfqpoint{0.019642in}{0.019642in}}%
\pgfpathcurveto{\pgfqpoint{0.014433in}{0.024851in}}{\pgfqpoint{0.007367in}{0.027778in}}{\pgfqpoint{0.000000in}{0.027778in}}%
\pgfpathcurveto{\pgfqpoint{-0.007367in}{0.027778in}}{\pgfqpoint{-0.014433in}{0.024851in}}{\pgfqpoint{-0.019642in}{0.019642in}}%
\pgfpathcurveto{\pgfqpoint{-0.024851in}{0.014433in}}{\pgfqpoint{-0.027778in}{0.007367in}}{\pgfqpoint{-0.027778in}{0.000000in}}%
\pgfpathcurveto{\pgfqpoint{-0.027778in}{-0.007367in}}{\pgfqpoint{-0.024851in}{-0.014433in}}{\pgfqpoint{-0.019642in}{-0.019642in}}%
\pgfpathcurveto{\pgfqpoint{-0.014433in}{-0.024851in}}{\pgfqpoint{-0.007367in}{-0.027778in}}{\pgfqpoint{0.000000in}{-0.027778in}}%
\pgfpathlineto{\pgfqpoint{0.000000in}{-0.027778in}}%
\pgfpathclose%
\pgfusepath{stroke,fill}%
}%
\begin{pgfscope}%
\pgfsys@transformshift{4.664719in}{4.987184in}%
\pgfsys@useobject{currentmarker}{}%
\end{pgfscope}%
\end{pgfscope}%
\begin{pgfscope}%
\definecolor{textcolor}{rgb}{0.000000,0.000000,0.000000}%
\pgfsetstrokecolor{textcolor}%
\pgfsetfillcolor{textcolor}%
\pgftext[x=5.164719in,y=4.889962in,left,base]{\color{textcolor}\sffamily\fontsize{20.000000}{24.000000}\selectfont \(\displaystyle D_\mathrm{w}\)}%
\end{pgfscope}%
\begin{pgfscope}%
\pgfsetbuttcap%
\pgfsetmiterjoin%
\definecolor{currentfill}{rgb}{1.000000,1.000000,1.000000}%
\pgfsetfillcolor{currentfill}%
\pgfsetlinewidth{0.000000pt}%
\definecolor{currentstroke}{rgb}{0.000000,0.000000,0.000000}%
\pgfsetstrokecolor{currentstroke}%
\pgfsetstrokeopacity{0.000000}%
\pgfsetdash{}{0pt}%
\pgfpathmoveto{\pgfqpoint{5.800000in}{0.720000in}}%
\pgfpathlineto{\pgfqpoint{7.200000in}{0.720000in}}%
\pgfpathlineto{\pgfqpoint{7.200000in}{5.340000in}}%
\pgfpathlineto{\pgfqpoint{5.800000in}{5.340000in}}%
\pgfpathlineto{\pgfqpoint{5.800000in}{0.720000in}}%
\pgfpathclose%
\pgfusepath{fill}%
\end{pgfscope}%
\begin{pgfscope}%
\pgfpathrectangle{\pgfqpoint{5.800000in}{0.720000in}}{\pgfqpoint{1.400000in}{4.620000in}}%
\pgfusepath{clip}%
\pgfsetbuttcap%
\pgfsetmiterjoin%
\definecolor{currentfill}{rgb}{0.121569,0.466667,0.705882}%
\pgfsetfillcolor{currentfill}%
\pgfsetlinewidth{0.000000pt}%
\definecolor{currentstroke}{rgb}{0.000000,0.000000,0.000000}%
\pgfsetstrokecolor{currentstroke}%
\pgfsetstrokeopacity{0.000000}%
\pgfsetdash{}{0pt}%
\pgfpathmoveto{\pgfqpoint{5.800000in}{0.720000in}}%
\pgfpathlineto{\pgfqpoint{5.864493in}{0.720000in}}%
\pgfpathlineto{\pgfqpoint{5.864493in}{0.835500in}}%
\pgfpathlineto{\pgfqpoint{5.800000in}{0.835500in}}%
\pgfpathlineto{\pgfqpoint{5.800000in}{0.720000in}}%
\pgfpathclose%
\pgfusepath{fill}%
\end{pgfscope}%
\begin{pgfscope}%
\pgfpathrectangle{\pgfqpoint{5.800000in}{0.720000in}}{\pgfqpoint{1.400000in}{4.620000in}}%
\pgfusepath{clip}%
\pgfsetbuttcap%
\pgfsetmiterjoin%
\definecolor{currentfill}{rgb}{0.121569,0.466667,0.705882}%
\pgfsetfillcolor{currentfill}%
\pgfsetlinewidth{0.000000pt}%
\definecolor{currentstroke}{rgb}{0.000000,0.000000,0.000000}%
\pgfsetstrokecolor{currentstroke}%
\pgfsetstrokeopacity{0.000000}%
\pgfsetdash{}{0pt}%
\pgfpathmoveto{\pgfqpoint{5.800000in}{0.835500in}}%
\pgfpathlineto{\pgfqpoint{5.866717in}{0.835500in}}%
\pgfpathlineto{\pgfqpoint{5.866717in}{0.951000in}}%
\pgfpathlineto{\pgfqpoint{5.800000in}{0.951000in}}%
\pgfpathlineto{\pgfqpoint{5.800000in}{0.835500in}}%
\pgfpathclose%
\pgfusepath{fill}%
\end{pgfscope}%
\begin{pgfscope}%
\pgfpathrectangle{\pgfqpoint{5.800000in}{0.720000in}}{\pgfqpoint{1.400000in}{4.620000in}}%
\pgfusepath{clip}%
\pgfsetbuttcap%
\pgfsetmiterjoin%
\definecolor{currentfill}{rgb}{0.121569,0.466667,0.705882}%
\pgfsetfillcolor{currentfill}%
\pgfsetlinewidth{0.000000pt}%
\definecolor{currentstroke}{rgb}{0.000000,0.000000,0.000000}%
\pgfsetstrokecolor{currentstroke}%
\pgfsetstrokeopacity{0.000000}%
\pgfsetdash{}{0pt}%
\pgfpathmoveto{\pgfqpoint{5.800000in}{0.951000in}}%
\pgfpathlineto{\pgfqpoint{5.902299in}{0.951000in}}%
\pgfpathlineto{\pgfqpoint{5.902299in}{1.066500in}}%
\pgfpathlineto{\pgfqpoint{5.800000in}{1.066500in}}%
\pgfpathlineto{\pgfqpoint{5.800000in}{0.951000in}}%
\pgfpathclose%
\pgfusepath{fill}%
\end{pgfscope}%
\begin{pgfscope}%
\pgfpathrectangle{\pgfqpoint{5.800000in}{0.720000in}}{\pgfqpoint{1.400000in}{4.620000in}}%
\pgfusepath{clip}%
\pgfsetbuttcap%
\pgfsetmiterjoin%
\definecolor{currentfill}{rgb}{0.121569,0.466667,0.705882}%
\pgfsetfillcolor{currentfill}%
\pgfsetlinewidth{0.000000pt}%
\definecolor{currentstroke}{rgb}{0.000000,0.000000,0.000000}%
\pgfsetstrokecolor{currentstroke}%
\pgfsetstrokeopacity{0.000000}%
\pgfsetdash{}{0pt}%
\pgfpathmoveto{\pgfqpoint{5.800000in}{1.066500in}}%
\pgfpathlineto{\pgfqpoint{5.886732in}{1.066500in}}%
\pgfpathlineto{\pgfqpoint{5.886732in}{1.182000in}}%
\pgfpathlineto{\pgfqpoint{5.800000in}{1.182000in}}%
\pgfpathlineto{\pgfqpoint{5.800000in}{1.066500in}}%
\pgfpathclose%
\pgfusepath{fill}%
\end{pgfscope}%
\begin{pgfscope}%
\pgfpathrectangle{\pgfqpoint{5.800000in}{0.720000in}}{\pgfqpoint{1.400000in}{4.620000in}}%
\pgfusepath{clip}%
\pgfsetbuttcap%
\pgfsetmiterjoin%
\definecolor{currentfill}{rgb}{0.121569,0.466667,0.705882}%
\pgfsetfillcolor{currentfill}%
\pgfsetlinewidth{0.000000pt}%
\definecolor{currentstroke}{rgb}{0.000000,0.000000,0.000000}%
\pgfsetstrokecolor{currentstroke}%
\pgfsetstrokeopacity{0.000000}%
\pgfsetdash{}{0pt}%
\pgfpathmoveto{\pgfqpoint{5.800000in}{1.182000in}}%
\pgfpathlineto{\pgfqpoint{5.915642in}{1.182000in}}%
\pgfpathlineto{\pgfqpoint{5.915642in}{1.297500in}}%
\pgfpathlineto{\pgfqpoint{5.800000in}{1.297500in}}%
\pgfpathlineto{\pgfqpoint{5.800000in}{1.182000in}}%
\pgfpathclose%
\pgfusepath{fill}%
\end{pgfscope}%
\begin{pgfscope}%
\pgfpathrectangle{\pgfqpoint{5.800000in}{0.720000in}}{\pgfqpoint{1.400000in}{4.620000in}}%
\pgfusepath{clip}%
\pgfsetbuttcap%
\pgfsetmiterjoin%
\definecolor{currentfill}{rgb}{0.121569,0.466667,0.705882}%
\pgfsetfillcolor{currentfill}%
\pgfsetlinewidth{0.000000pt}%
\definecolor{currentstroke}{rgb}{0.000000,0.000000,0.000000}%
\pgfsetstrokecolor{currentstroke}%
\pgfsetstrokeopacity{0.000000}%
\pgfsetdash{}{0pt}%
\pgfpathmoveto{\pgfqpoint{5.800000in}{1.297500in}}%
\pgfpathlineto{\pgfqpoint{5.957896in}{1.297500in}}%
\pgfpathlineto{\pgfqpoint{5.957896in}{1.413000in}}%
\pgfpathlineto{\pgfqpoint{5.800000in}{1.413000in}}%
\pgfpathlineto{\pgfqpoint{5.800000in}{1.297500in}}%
\pgfpathclose%
\pgfusepath{fill}%
\end{pgfscope}%
\begin{pgfscope}%
\pgfpathrectangle{\pgfqpoint{5.800000in}{0.720000in}}{\pgfqpoint{1.400000in}{4.620000in}}%
\pgfusepath{clip}%
\pgfsetbuttcap%
\pgfsetmiterjoin%
\definecolor{currentfill}{rgb}{0.121569,0.466667,0.705882}%
\pgfsetfillcolor{currentfill}%
\pgfsetlinewidth{0.000000pt}%
\definecolor{currentstroke}{rgb}{0.000000,0.000000,0.000000}%
\pgfsetstrokecolor{currentstroke}%
\pgfsetstrokeopacity{0.000000}%
\pgfsetdash{}{0pt}%
\pgfpathmoveto{\pgfqpoint{5.800000in}{1.413000in}}%
\pgfpathlineto{\pgfqpoint{5.982359in}{1.413000in}}%
\pgfpathlineto{\pgfqpoint{5.982359in}{1.528500in}}%
\pgfpathlineto{\pgfqpoint{5.800000in}{1.528500in}}%
\pgfpathlineto{\pgfqpoint{5.800000in}{1.413000in}}%
\pgfpathclose%
\pgfusepath{fill}%
\end{pgfscope}%
\begin{pgfscope}%
\pgfpathrectangle{\pgfqpoint{5.800000in}{0.720000in}}{\pgfqpoint{1.400000in}{4.620000in}}%
\pgfusepath{clip}%
\pgfsetbuttcap%
\pgfsetmiterjoin%
\definecolor{currentfill}{rgb}{0.121569,0.466667,0.705882}%
\pgfsetfillcolor{currentfill}%
\pgfsetlinewidth{0.000000pt}%
\definecolor{currentstroke}{rgb}{0.000000,0.000000,0.000000}%
\pgfsetstrokecolor{currentstroke}%
\pgfsetstrokeopacity{0.000000}%
\pgfsetdash{}{0pt}%
\pgfpathmoveto{\pgfqpoint{5.800000in}{1.528500in}}%
\pgfpathlineto{\pgfqpoint{6.075762in}{1.528500in}}%
\pgfpathlineto{\pgfqpoint{6.075762in}{1.644000in}}%
\pgfpathlineto{\pgfqpoint{5.800000in}{1.644000in}}%
\pgfpathlineto{\pgfqpoint{5.800000in}{1.528500in}}%
\pgfpathclose%
\pgfusepath{fill}%
\end{pgfscope}%
\begin{pgfscope}%
\pgfpathrectangle{\pgfqpoint{5.800000in}{0.720000in}}{\pgfqpoint{1.400000in}{4.620000in}}%
\pgfusepath{clip}%
\pgfsetbuttcap%
\pgfsetmiterjoin%
\definecolor{currentfill}{rgb}{0.121569,0.466667,0.705882}%
\pgfsetfillcolor{currentfill}%
\pgfsetlinewidth{0.000000pt}%
\definecolor{currentstroke}{rgb}{0.000000,0.000000,0.000000}%
\pgfsetstrokecolor{currentstroke}%
\pgfsetstrokeopacity{0.000000}%
\pgfsetdash{}{0pt}%
\pgfpathmoveto{\pgfqpoint{5.800000in}{1.644000in}}%
\pgfpathlineto{\pgfqpoint{6.135807in}{1.644000in}}%
\pgfpathlineto{\pgfqpoint{6.135807in}{1.759500in}}%
\pgfpathlineto{\pgfqpoint{5.800000in}{1.759500in}}%
\pgfpathlineto{\pgfqpoint{5.800000in}{1.644000in}}%
\pgfpathclose%
\pgfusepath{fill}%
\end{pgfscope}%
\begin{pgfscope}%
\pgfpathrectangle{\pgfqpoint{5.800000in}{0.720000in}}{\pgfqpoint{1.400000in}{4.620000in}}%
\pgfusepath{clip}%
\pgfsetbuttcap%
\pgfsetmiterjoin%
\definecolor{currentfill}{rgb}{0.121569,0.466667,0.705882}%
\pgfsetfillcolor{currentfill}%
\pgfsetlinewidth{0.000000pt}%
\definecolor{currentstroke}{rgb}{0.000000,0.000000,0.000000}%
\pgfsetstrokecolor{currentstroke}%
\pgfsetstrokeopacity{0.000000}%
\pgfsetdash{}{0pt}%
\pgfpathmoveto{\pgfqpoint{5.800000in}{1.759500in}}%
\pgfpathlineto{\pgfqpoint{6.224763in}{1.759500in}}%
\pgfpathlineto{\pgfqpoint{6.224763in}{1.875000in}}%
\pgfpathlineto{\pgfqpoint{5.800000in}{1.875000in}}%
\pgfpathlineto{\pgfqpoint{5.800000in}{1.759500in}}%
\pgfpathclose%
\pgfusepath{fill}%
\end{pgfscope}%
\begin{pgfscope}%
\pgfpathrectangle{\pgfqpoint{5.800000in}{0.720000in}}{\pgfqpoint{1.400000in}{4.620000in}}%
\pgfusepath{clip}%
\pgfsetbuttcap%
\pgfsetmiterjoin%
\definecolor{currentfill}{rgb}{0.121569,0.466667,0.705882}%
\pgfsetfillcolor{currentfill}%
\pgfsetlinewidth{0.000000pt}%
\definecolor{currentstroke}{rgb}{0.000000,0.000000,0.000000}%
\pgfsetstrokecolor{currentstroke}%
\pgfsetstrokeopacity{0.000000}%
\pgfsetdash{}{0pt}%
\pgfpathmoveto{\pgfqpoint{5.800000in}{1.875000in}}%
\pgfpathlineto{\pgfqpoint{6.360420in}{1.875000in}}%
\pgfpathlineto{\pgfqpoint{6.360420in}{1.990500in}}%
\pgfpathlineto{\pgfqpoint{5.800000in}{1.990500in}}%
\pgfpathlineto{\pgfqpoint{5.800000in}{1.875000in}}%
\pgfpathclose%
\pgfusepath{fill}%
\end{pgfscope}%
\begin{pgfscope}%
\pgfpathrectangle{\pgfqpoint{5.800000in}{0.720000in}}{\pgfqpoint{1.400000in}{4.620000in}}%
\pgfusepath{clip}%
\pgfsetbuttcap%
\pgfsetmiterjoin%
\definecolor{currentfill}{rgb}{0.121569,0.466667,0.705882}%
\pgfsetfillcolor{currentfill}%
\pgfsetlinewidth{0.000000pt}%
\definecolor{currentstroke}{rgb}{0.000000,0.000000,0.000000}%
\pgfsetstrokecolor{currentstroke}%
\pgfsetstrokeopacity{0.000000}%
\pgfsetdash{}{0pt}%
\pgfpathmoveto{\pgfqpoint{5.800000in}{1.990500in}}%
\pgfpathlineto{\pgfqpoint{6.449376in}{1.990500in}}%
\pgfpathlineto{\pgfqpoint{6.449376in}{2.106000in}}%
\pgfpathlineto{\pgfqpoint{5.800000in}{2.106000in}}%
\pgfpathlineto{\pgfqpoint{5.800000in}{1.990500in}}%
\pgfpathclose%
\pgfusepath{fill}%
\end{pgfscope}%
\begin{pgfscope}%
\pgfpathrectangle{\pgfqpoint{5.800000in}{0.720000in}}{\pgfqpoint{1.400000in}{4.620000in}}%
\pgfusepath{clip}%
\pgfsetbuttcap%
\pgfsetmiterjoin%
\definecolor{currentfill}{rgb}{0.121569,0.466667,0.705882}%
\pgfsetfillcolor{currentfill}%
\pgfsetlinewidth{0.000000pt}%
\definecolor{currentstroke}{rgb}{0.000000,0.000000,0.000000}%
\pgfsetstrokecolor{currentstroke}%
\pgfsetstrokeopacity{0.000000}%
\pgfsetdash{}{0pt}%
\pgfpathmoveto{\pgfqpoint{5.800000in}{2.106000in}}%
\pgfpathlineto{\pgfqpoint{6.507197in}{2.106000in}}%
\pgfpathlineto{\pgfqpoint{6.507197in}{2.221500in}}%
\pgfpathlineto{\pgfqpoint{5.800000in}{2.221500in}}%
\pgfpathlineto{\pgfqpoint{5.800000in}{2.106000in}}%
\pgfpathclose%
\pgfusepath{fill}%
\end{pgfscope}%
\begin{pgfscope}%
\pgfpathrectangle{\pgfqpoint{5.800000in}{0.720000in}}{\pgfqpoint{1.400000in}{4.620000in}}%
\pgfusepath{clip}%
\pgfsetbuttcap%
\pgfsetmiterjoin%
\definecolor{currentfill}{rgb}{0.121569,0.466667,0.705882}%
\pgfsetfillcolor{currentfill}%
\pgfsetlinewidth{0.000000pt}%
\definecolor{currentstroke}{rgb}{0.000000,0.000000,0.000000}%
\pgfsetstrokecolor{currentstroke}%
\pgfsetstrokeopacity{0.000000}%
\pgfsetdash{}{0pt}%
\pgfpathmoveto{\pgfqpoint{5.800000in}{2.221500in}}%
\pgfpathlineto{\pgfqpoint{6.680660in}{2.221500in}}%
\pgfpathlineto{\pgfqpoint{6.680660in}{2.337000in}}%
\pgfpathlineto{\pgfqpoint{5.800000in}{2.337000in}}%
\pgfpathlineto{\pgfqpoint{5.800000in}{2.221500in}}%
\pgfpathclose%
\pgfusepath{fill}%
\end{pgfscope}%
\begin{pgfscope}%
\pgfpathrectangle{\pgfqpoint{5.800000in}{0.720000in}}{\pgfqpoint{1.400000in}{4.620000in}}%
\pgfusepath{clip}%
\pgfsetbuttcap%
\pgfsetmiterjoin%
\definecolor{currentfill}{rgb}{0.121569,0.466667,0.705882}%
\pgfsetfillcolor{currentfill}%
\pgfsetlinewidth{0.000000pt}%
\definecolor{currentstroke}{rgb}{0.000000,0.000000,0.000000}%
\pgfsetstrokecolor{currentstroke}%
\pgfsetstrokeopacity{0.000000}%
\pgfsetdash{}{0pt}%
\pgfpathmoveto{\pgfqpoint{5.800000in}{2.337000in}}%
\pgfpathlineto{\pgfqpoint{6.809646in}{2.337000in}}%
\pgfpathlineto{\pgfqpoint{6.809646in}{2.452500in}}%
\pgfpathlineto{\pgfqpoint{5.800000in}{2.452500in}}%
\pgfpathlineto{\pgfqpoint{5.800000in}{2.337000in}}%
\pgfpathclose%
\pgfusepath{fill}%
\end{pgfscope}%
\begin{pgfscope}%
\pgfpathrectangle{\pgfqpoint{5.800000in}{0.720000in}}{\pgfqpoint{1.400000in}{4.620000in}}%
\pgfusepath{clip}%
\pgfsetbuttcap%
\pgfsetmiterjoin%
\definecolor{currentfill}{rgb}{0.121569,0.466667,0.705882}%
\pgfsetfillcolor{currentfill}%
\pgfsetlinewidth{0.000000pt}%
\definecolor{currentstroke}{rgb}{0.000000,0.000000,0.000000}%
\pgfsetstrokecolor{currentstroke}%
\pgfsetstrokeopacity{0.000000}%
\pgfsetdash{}{0pt}%
\pgfpathmoveto{\pgfqpoint{5.800000in}{2.452500in}}%
\pgfpathlineto{\pgfqpoint{6.936408in}{2.452500in}}%
\pgfpathlineto{\pgfqpoint{6.936408in}{2.568000in}}%
\pgfpathlineto{\pgfqpoint{5.800000in}{2.568000in}}%
\pgfpathlineto{\pgfqpoint{5.800000in}{2.452500in}}%
\pgfpathclose%
\pgfusepath{fill}%
\end{pgfscope}%
\begin{pgfscope}%
\pgfpathrectangle{\pgfqpoint{5.800000in}{0.720000in}}{\pgfqpoint{1.400000in}{4.620000in}}%
\pgfusepath{clip}%
\pgfsetbuttcap%
\pgfsetmiterjoin%
\definecolor{currentfill}{rgb}{0.121569,0.466667,0.705882}%
\pgfsetfillcolor{currentfill}%
\pgfsetlinewidth{0.000000pt}%
\definecolor{currentstroke}{rgb}{0.000000,0.000000,0.000000}%
\pgfsetstrokecolor{currentstroke}%
\pgfsetstrokeopacity{0.000000}%
\pgfsetdash{}{0pt}%
\pgfpathmoveto{\pgfqpoint{5.800000in}{2.568000in}}%
\pgfpathlineto{\pgfqpoint{6.945303in}{2.568000in}}%
\pgfpathlineto{\pgfqpoint{6.945303in}{2.683500in}}%
\pgfpathlineto{\pgfqpoint{5.800000in}{2.683500in}}%
\pgfpathlineto{\pgfqpoint{5.800000in}{2.568000in}}%
\pgfpathclose%
\pgfusepath{fill}%
\end{pgfscope}%
\begin{pgfscope}%
\pgfpathrectangle{\pgfqpoint{5.800000in}{0.720000in}}{\pgfqpoint{1.400000in}{4.620000in}}%
\pgfusepath{clip}%
\pgfsetbuttcap%
\pgfsetmiterjoin%
\definecolor{currentfill}{rgb}{0.121569,0.466667,0.705882}%
\pgfsetfillcolor{currentfill}%
\pgfsetlinewidth{0.000000pt}%
\definecolor{currentstroke}{rgb}{0.000000,0.000000,0.000000}%
\pgfsetstrokecolor{currentstroke}%
\pgfsetstrokeopacity{0.000000}%
\pgfsetdash{}{0pt}%
\pgfpathmoveto{\pgfqpoint{5.800000in}{2.683500in}}%
\pgfpathlineto{\pgfqpoint{7.005348in}{2.683500in}}%
\pgfpathlineto{\pgfqpoint{7.005348in}{2.799000in}}%
\pgfpathlineto{\pgfqpoint{5.800000in}{2.799000in}}%
\pgfpathlineto{\pgfqpoint{5.800000in}{2.683500in}}%
\pgfpathclose%
\pgfusepath{fill}%
\end{pgfscope}%
\begin{pgfscope}%
\pgfpathrectangle{\pgfqpoint{5.800000in}{0.720000in}}{\pgfqpoint{1.400000in}{4.620000in}}%
\pgfusepath{clip}%
\pgfsetbuttcap%
\pgfsetmiterjoin%
\definecolor{currentfill}{rgb}{0.121569,0.466667,0.705882}%
\pgfsetfillcolor{currentfill}%
\pgfsetlinewidth{0.000000pt}%
\definecolor{currentstroke}{rgb}{0.000000,0.000000,0.000000}%
\pgfsetstrokecolor{currentstroke}%
\pgfsetstrokeopacity{0.000000}%
\pgfsetdash{}{0pt}%
\pgfpathmoveto{\pgfqpoint{5.800000in}{2.799000in}}%
\pgfpathlineto{\pgfqpoint{7.025363in}{2.799000in}}%
\pgfpathlineto{\pgfqpoint{7.025363in}{2.914500in}}%
\pgfpathlineto{\pgfqpoint{5.800000in}{2.914500in}}%
\pgfpathlineto{\pgfqpoint{5.800000in}{2.799000in}}%
\pgfpathclose%
\pgfusepath{fill}%
\end{pgfscope}%
\begin{pgfscope}%
\pgfpathrectangle{\pgfqpoint{5.800000in}{0.720000in}}{\pgfqpoint{1.400000in}{4.620000in}}%
\pgfusepath{clip}%
\pgfsetbuttcap%
\pgfsetmiterjoin%
\definecolor{currentfill}{rgb}{0.121569,0.466667,0.705882}%
\pgfsetfillcolor{currentfill}%
\pgfsetlinewidth{0.000000pt}%
\definecolor{currentstroke}{rgb}{0.000000,0.000000,0.000000}%
\pgfsetstrokecolor{currentstroke}%
\pgfsetstrokeopacity{0.000000}%
\pgfsetdash{}{0pt}%
\pgfpathmoveto{\pgfqpoint{5.800000in}{2.914500in}}%
\pgfpathlineto{\pgfqpoint{7.069841in}{2.914500in}}%
\pgfpathlineto{\pgfqpoint{7.069841in}{3.030000in}}%
\pgfpathlineto{\pgfqpoint{5.800000in}{3.030000in}}%
\pgfpathlineto{\pgfqpoint{5.800000in}{2.914500in}}%
\pgfpathclose%
\pgfusepath{fill}%
\end{pgfscope}%
\begin{pgfscope}%
\pgfpathrectangle{\pgfqpoint{5.800000in}{0.720000in}}{\pgfqpoint{1.400000in}{4.620000in}}%
\pgfusepath{clip}%
\pgfsetbuttcap%
\pgfsetmiterjoin%
\definecolor{currentfill}{rgb}{0.121569,0.466667,0.705882}%
\pgfsetfillcolor{currentfill}%
\pgfsetlinewidth{0.000000pt}%
\definecolor{currentstroke}{rgb}{0.000000,0.000000,0.000000}%
\pgfsetstrokecolor{currentstroke}%
\pgfsetstrokeopacity{0.000000}%
\pgfsetdash{}{0pt}%
\pgfpathmoveto{\pgfqpoint{5.800000in}{3.030000in}}%
\pgfpathlineto{\pgfqpoint{7.018692in}{3.030000in}}%
\pgfpathlineto{\pgfqpoint{7.018692in}{3.145500in}}%
\pgfpathlineto{\pgfqpoint{5.800000in}{3.145500in}}%
\pgfpathlineto{\pgfqpoint{5.800000in}{3.030000in}}%
\pgfpathclose%
\pgfusepath{fill}%
\end{pgfscope}%
\begin{pgfscope}%
\pgfpathrectangle{\pgfqpoint{5.800000in}{0.720000in}}{\pgfqpoint{1.400000in}{4.620000in}}%
\pgfusepath{clip}%
\pgfsetbuttcap%
\pgfsetmiterjoin%
\definecolor{currentfill}{rgb}{0.121569,0.466667,0.705882}%
\pgfsetfillcolor{currentfill}%
\pgfsetlinewidth{0.000000pt}%
\definecolor{currentstroke}{rgb}{0.000000,0.000000,0.000000}%
\pgfsetstrokecolor{currentstroke}%
\pgfsetstrokeopacity{0.000000}%
\pgfsetdash{}{0pt}%
\pgfpathmoveto{\pgfqpoint{5.800000in}{3.145500in}}%
\pgfpathlineto{\pgfqpoint{6.983110in}{3.145500in}}%
\pgfpathlineto{\pgfqpoint{6.983110in}{3.261000in}}%
\pgfpathlineto{\pgfqpoint{5.800000in}{3.261000in}}%
\pgfpathlineto{\pgfqpoint{5.800000in}{3.145500in}}%
\pgfpathclose%
\pgfusepath{fill}%
\end{pgfscope}%
\begin{pgfscope}%
\pgfpathrectangle{\pgfqpoint{5.800000in}{0.720000in}}{\pgfqpoint{1.400000in}{4.620000in}}%
\pgfusepath{clip}%
\pgfsetbuttcap%
\pgfsetmiterjoin%
\definecolor{currentfill}{rgb}{0.121569,0.466667,0.705882}%
\pgfsetfillcolor{currentfill}%
\pgfsetlinewidth{0.000000pt}%
\definecolor{currentstroke}{rgb}{0.000000,0.000000,0.000000}%
\pgfsetstrokecolor{currentstroke}%
\pgfsetstrokeopacity{0.000000}%
\pgfsetdash{}{0pt}%
\pgfpathmoveto{\pgfqpoint{5.800000in}{3.261000in}}%
\pgfpathlineto{\pgfqpoint{6.920841in}{3.261000in}}%
\pgfpathlineto{\pgfqpoint{6.920841in}{3.376500in}}%
\pgfpathlineto{\pgfqpoint{5.800000in}{3.376500in}}%
\pgfpathlineto{\pgfqpoint{5.800000in}{3.261000in}}%
\pgfpathclose%
\pgfusepath{fill}%
\end{pgfscope}%
\begin{pgfscope}%
\pgfpathrectangle{\pgfqpoint{5.800000in}{0.720000in}}{\pgfqpoint{1.400000in}{4.620000in}}%
\pgfusepath{clip}%
\pgfsetbuttcap%
\pgfsetmiterjoin%
\definecolor{currentfill}{rgb}{0.121569,0.466667,0.705882}%
\pgfsetfillcolor{currentfill}%
\pgfsetlinewidth{0.000000pt}%
\definecolor{currentstroke}{rgb}{0.000000,0.000000,0.000000}%
\pgfsetstrokecolor{currentstroke}%
\pgfsetstrokeopacity{0.000000}%
\pgfsetdash{}{0pt}%
\pgfpathmoveto{\pgfqpoint{5.800000in}{3.376500in}}%
\pgfpathlineto{\pgfqpoint{6.762944in}{3.376500in}}%
\pgfpathlineto{\pgfqpoint{6.762944in}{3.492000in}}%
\pgfpathlineto{\pgfqpoint{5.800000in}{3.492000in}}%
\pgfpathlineto{\pgfqpoint{5.800000in}{3.376500in}}%
\pgfpathclose%
\pgfusepath{fill}%
\end{pgfscope}%
\begin{pgfscope}%
\pgfpathrectangle{\pgfqpoint{5.800000in}{0.720000in}}{\pgfqpoint{1.400000in}{4.620000in}}%
\pgfusepath{clip}%
\pgfsetbuttcap%
\pgfsetmiterjoin%
\definecolor{currentfill}{rgb}{0.121569,0.466667,0.705882}%
\pgfsetfillcolor{currentfill}%
\pgfsetlinewidth{0.000000pt}%
\definecolor{currentstroke}{rgb}{0.000000,0.000000,0.000000}%
\pgfsetstrokecolor{currentstroke}%
\pgfsetstrokeopacity{0.000000}%
\pgfsetdash{}{0pt}%
\pgfpathmoveto{\pgfqpoint{5.800000in}{3.492000in}}%
\pgfpathlineto{\pgfqpoint{6.609496in}{3.492000in}}%
\pgfpathlineto{\pgfqpoint{6.609496in}{3.607500in}}%
\pgfpathlineto{\pgfqpoint{5.800000in}{3.607500in}}%
\pgfpathlineto{\pgfqpoint{5.800000in}{3.492000in}}%
\pgfpathclose%
\pgfusepath{fill}%
\end{pgfscope}%
\begin{pgfscope}%
\pgfpathrectangle{\pgfqpoint{5.800000in}{0.720000in}}{\pgfqpoint{1.400000in}{4.620000in}}%
\pgfusepath{clip}%
\pgfsetbuttcap%
\pgfsetmiterjoin%
\definecolor{currentfill}{rgb}{0.121569,0.466667,0.705882}%
\pgfsetfillcolor{currentfill}%
\pgfsetlinewidth{0.000000pt}%
\definecolor{currentstroke}{rgb}{0.000000,0.000000,0.000000}%
\pgfsetstrokecolor{currentstroke}%
\pgfsetstrokeopacity{0.000000}%
\pgfsetdash{}{0pt}%
\pgfpathmoveto{\pgfqpoint{5.800000in}{3.607500in}}%
\pgfpathlineto{\pgfqpoint{6.571690in}{3.607500in}}%
\pgfpathlineto{\pgfqpoint{6.571690in}{3.723000in}}%
\pgfpathlineto{\pgfqpoint{5.800000in}{3.723000in}}%
\pgfpathlineto{\pgfqpoint{5.800000in}{3.607500in}}%
\pgfpathclose%
\pgfusepath{fill}%
\end{pgfscope}%
\begin{pgfscope}%
\pgfpathrectangle{\pgfqpoint{5.800000in}{0.720000in}}{\pgfqpoint{1.400000in}{4.620000in}}%
\pgfusepath{clip}%
\pgfsetbuttcap%
\pgfsetmiterjoin%
\definecolor{currentfill}{rgb}{0.121569,0.466667,0.705882}%
\pgfsetfillcolor{currentfill}%
\pgfsetlinewidth{0.000000pt}%
\definecolor{currentstroke}{rgb}{0.000000,0.000000,0.000000}%
\pgfsetstrokecolor{currentstroke}%
\pgfsetstrokeopacity{0.000000}%
\pgfsetdash{}{0pt}%
\pgfpathmoveto{\pgfqpoint{5.800000in}{3.723000in}}%
\pgfpathlineto{\pgfqpoint{6.444928in}{3.723000in}}%
\pgfpathlineto{\pgfqpoint{6.444928in}{3.838500in}}%
\pgfpathlineto{\pgfqpoint{5.800000in}{3.838500in}}%
\pgfpathlineto{\pgfqpoint{5.800000in}{3.723000in}}%
\pgfpathclose%
\pgfusepath{fill}%
\end{pgfscope}%
\begin{pgfscope}%
\pgfpathrectangle{\pgfqpoint{5.800000in}{0.720000in}}{\pgfqpoint{1.400000in}{4.620000in}}%
\pgfusepath{clip}%
\pgfsetbuttcap%
\pgfsetmiterjoin%
\definecolor{currentfill}{rgb}{0.121569,0.466667,0.705882}%
\pgfsetfillcolor{currentfill}%
\pgfsetlinewidth{0.000000pt}%
\definecolor{currentstroke}{rgb}{0.000000,0.000000,0.000000}%
\pgfsetstrokecolor{currentstroke}%
\pgfsetstrokeopacity{0.000000}%
\pgfsetdash{}{0pt}%
\pgfpathmoveto{\pgfqpoint{5.800000in}{3.838500in}}%
\pgfpathlineto{\pgfqpoint{6.318166in}{3.838500in}}%
\pgfpathlineto{\pgfqpoint{6.318166in}{3.954000in}}%
\pgfpathlineto{\pgfqpoint{5.800000in}{3.954000in}}%
\pgfpathlineto{\pgfqpoint{5.800000in}{3.838500in}}%
\pgfpathclose%
\pgfusepath{fill}%
\end{pgfscope}%
\begin{pgfscope}%
\pgfpathrectangle{\pgfqpoint{5.800000in}{0.720000in}}{\pgfqpoint{1.400000in}{4.620000in}}%
\pgfusepath{clip}%
\pgfsetbuttcap%
\pgfsetmiterjoin%
\definecolor{currentfill}{rgb}{0.121569,0.466667,0.705882}%
\pgfsetfillcolor{currentfill}%
\pgfsetlinewidth{0.000000pt}%
\definecolor{currentstroke}{rgb}{0.000000,0.000000,0.000000}%
\pgfsetstrokecolor{currentstroke}%
\pgfsetstrokeopacity{0.000000}%
\pgfsetdash{}{0pt}%
\pgfpathmoveto{\pgfqpoint{5.800000in}{3.954000in}}%
\pgfpathlineto{\pgfqpoint{6.318166in}{3.954000in}}%
\pgfpathlineto{\pgfqpoint{6.318166in}{4.069500in}}%
\pgfpathlineto{\pgfqpoint{5.800000in}{4.069500in}}%
\pgfpathlineto{\pgfqpoint{5.800000in}{3.954000in}}%
\pgfpathclose%
\pgfusepath{fill}%
\end{pgfscope}%
\begin{pgfscope}%
\pgfpathrectangle{\pgfqpoint{5.800000in}{0.720000in}}{\pgfqpoint{1.400000in}{4.620000in}}%
\pgfusepath{clip}%
\pgfsetbuttcap%
\pgfsetmiterjoin%
\definecolor{currentfill}{rgb}{0.121569,0.466667,0.705882}%
\pgfsetfillcolor{currentfill}%
\pgfsetlinewidth{0.000000pt}%
\definecolor{currentstroke}{rgb}{0.000000,0.000000,0.000000}%
\pgfsetstrokecolor{currentstroke}%
\pgfsetstrokeopacity{0.000000}%
\pgfsetdash{}{0pt}%
\pgfpathmoveto{\pgfqpoint{5.800000in}{4.069500in}}%
\pgfpathlineto{\pgfqpoint{6.198076in}{4.069500in}}%
\pgfpathlineto{\pgfqpoint{6.198076in}{4.185000in}}%
\pgfpathlineto{\pgfqpoint{5.800000in}{4.185000in}}%
\pgfpathlineto{\pgfqpoint{5.800000in}{4.069500in}}%
\pgfpathclose%
\pgfusepath{fill}%
\end{pgfscope}%
\begin{pgfscope}%
\pgfpathrectangle{\pgfqpoint{5.800000in}{0.720000in}}{\pgfqpoint{1.400000in}{4.620000in}}%
\pgfusepath{clip}%
\pgfsetbuttcap%
\pgfsetmiterjoin%
\definecolor{currentfill}{rgb}{0.121569,0.466667,0.705882}%
\pgfsetfillcolor{currentfill}%
\pgfsetlinewidth{0.000000pt}%
\definecolor{currentstroke}{rgb}{0.000000,0.000000,0.000000}%
\pgfsetstrokecolor{currentstroke}%
\pgfsetstrokeopacity{0.000000}%
\pgfsetdash{}{0pt}%
\pgfpathmoveto{\pgfqpoint{5.800000in}{4.185000in}}%
\pgfpathlineto{\pgfqpoint{6.146927in}{4.185000in}}%
\pgfpathlineto{\pgfqpoint{6.146927in}{4.300500in}}%
\pgfpathlineto{\pgfqpoint{5.800000in}{4.300500in}}%
\pgfpathlineto{\pgfqpoint{5.800000in}{4.185000in}}%
\pgfpathclose%
\pgfusepath{fill}%
\end{pgfscope}%
\begin{pgfscope}%
\pgfpathrectangle{\pgfqpoint{5.800000in}{0.720000in}}{\pgfqpoint{1.400000in}{4.620000in}}%
\pgfusepath{clip}%
\pgfsetbuttcap%
\pgfsetmiterjoin%
\definecolor{currentfill}{rgb}{0.121569,0.466667,0.705882}%
\pgfsetfillcolor{currentfill}%
\pgfsetlinewidth{0.000000pt}%
\definecolor{currentstroke}{rgb}{0.000000,0.000000,0.000000}%
\pgfsetstrokecolor{currentstroke}%
\pgfsetstrokeopacity{0.000000}%
\pgfsetdash{}{0pt}%
\pgfpathmoveto{\pgfqpoint{5.800000in}{4.300500in}}%
\pgfpathlineto{\pgfqpoint{6.062419in}{4.300500in}}%
\pgfpathlineto{\pgfqpoint{6.062419in}{4.416000in}}%
\pgfpathlineto{\pgfqpoint{5.800000in}{4.416000in}}%
\pgfpathlineto{\pgfqpoint{5.800000in}{4.300500in}}%
\pgfpathclose%
\pgfusepath{fill}%
\end{pgfscope}%
\begin{pgfscope}%
\pgfpathrectangle{\pgfqpoint{5.800000in}{0.720000in}}{\pgfqpoint{1.400000in}{4.620000in}}%
\pgfusepath{clip}%
\pgfsetbuttcap%
\pgfsetmiterjoin%
\definecolor{currentfill}{rgb}{0.121569,0.466667,0.705882}%
\pgfsetfillcolor{currentfill}%
\pgfsetlinewidth{0.000000pt}%
\definecolor{currentstroke}{rgb}{0.000000,0.000000,0.000000}%
\pgfsetstrokecolor{currentstroke}%
\pgfsetstrokeopacity{0.000000}%
\pgfsetdash{}{0pt}%
\pgfpathmoveto{\pgfqpoint{5.800000in}{4.416000in}}%
\pgfpathlineto{\pgfqpoint{6.062419in}{4.416000in}}%
\pgfpathlineto{\pgfqpoint{6.062419in}{4.531500in}}%
\pgfpathlineto{\pgfqpoint{5.800000in}{4.531500in}}%
\pgfpathlineto{\pgfqpoint{5.800000in}{4.416000in}}%
\pgfpathclose%
\pgfusepath{fill}%
\end{pgfscope}%
\begin{pgfscope}%
\pgfpathrectangle{\pgfqpoint{5.800000in}{0.720000in}}{\pgfqpoint{1.400000in}{4.620000in}}%
\pgfusepath{clip}%
\pgfsetbuttcap%
\pgfsetmiterjoin%
\definecolor{currentfill}{rgb}{0.121569,0.466667,0.705882}%
\pgfsetfillcolor{currentfill}%
\pgfsetlinewidth{0.000000pt}%
\definecolor{currentstroke}{rgb}{0.000000,0.000000,0.000000}%
\pgfsetstrokecolor{currentstroke}%
\pgfsetstrokeopacity{0.000000}%
\pgfsetdash{}{0pt}%
\pgfpathmoveto{\pgfqpoint{5.800000in}{4.531500in}}%
\pgfpathlineto{\pgfqpoint{5.991255in}{4.531500in}}%
\pgfpathlineto{\pgfqpoint{5.991255in}{4.647000in}}%
\pgfpathlineto{\pgfqpoint{5.800000in}{4.647000in}}%
\pgfpathlineto{\pgfqpoint{5.800000in}{4.531500in}}%
\pgfpathclose%
\pgfusepath{fill}%
\end{pgfscope}%
\begin{pgfscope}%
\pgfpathrectangle{\pgfqpoint{5.800000in}{0.720000in}}{\pgfqpoint{1.400000in}{4.620000in}}%
\pgfusepath{clip}%
\pgfsetbuttcap%
\pgfsetmiterjoin%
\definecolor{currentfill}{rgb}{0.121569,0.466667,0.705882}%
\pgfsetfillcolor{currentfill}%
\pgfsetlinewidth{0.000000pt}%
\definecolor{currentstroke}{rgb}{0.000000,0.000000,0.000000}%
\pgfsetstrokecolor{currentstroke}%
\pgfsetstrokeopacity{0.000000}%
\pgfsetdash{}{0pt}%
\pgfpathmoveto{\pgfqpoint{5.800000in}{4.647000in}}%
\pgfpathlineto{\pgfqpoint{5.966792in}{4.647000in}}%
\pgfpathlineto{\pgfqpoint{5.966792in}{4.762500in}}%
\pgfpathlineto{\pgfqpoint{5.800000in}{4.762500in}}%
\pgfpathlineto{\pgfqpoint{5.800000in}{4.647000in}}%
\pgfpathclose%
\pgfusepath{fill}%
\end{pgfscope}%
\begin{pgfscope}%
\pgfpathrectangle{\pgfqpoint{5.800000in}{0.720000in}}{\pgfqpoint{1.400000in}{4.620000in}}%
\pgfusepath{clip}%
\pgfsetbuttcap%
\pgfsetmiterjoin%
\definecolor{currentfill}{rgb}{0.121569,0.466667,0.705882}%
\pgfsetfillcolor{currentfill}%
\pgfsetlinewidth{0.000000pt}%
\definecolor{currentstroke}{rgb}{0.000000,0.000000,0.000000}%
\pgfsetstrokecolor{currentstroke}%
\pgfsetstrokeopacity{0.000000}%
\pgfsetdash{}{0pt}%
\pgfpathmoveto{\pgfqpoint{5.800000in}{4.762500in}}%
\pgfpathlineto{\pgfqpoint{5.915642in}{4.762500in}}%
\pgfpathlineto{\pgfqpoint{5.915642in}{4.878000in}}%
\pgfpathlineto{\pgfqpoint{5.800000in}{4.878000in}}%
\pgfpathlineto{\pgfqpoint{5.800000in}{4.762500in}}%
\pgfpathclose%
\pgfusepath{fill}%
\end{pgfscope}%
\begin{pgfscope}%
\pgfpathrectangle{\pgfqpoint{5.800000in}{0.720000in}}{\pgfqpoint{1.400000in}{4.620000in}}%
\pgfusepath{clip}%
\pgfsetbuttcap%
\pgfsetmiterjoin%
\definecolor{currentfill}{rgb}{0.121569,0.466667,0.705882}%
\pgfsetfillcolor{currentfill}%
\pgfsetlinewidth{0.000000pt}%
\definecolor{currentstroke}{rgb}{0.000000,0.000000,0.000000}%
\pgfsetstrokecolor{currentstroke}%
\pgfsetstrokeopacity{0.000000}%
\pgfsetdash{}{0pt}%
\pgfpathmoveto{\pgfqpoint{5.800000in}{4.878000in}}%
\pgfpathlineto{\pgfqpoint{5.920090in}{4.878000in}}%
\pgfpathlineto{\pgfqpoint{5.920090in}{4.993500in}}%
\pgfpathlineto{\pgfqpoint{5.800000in}{4.993500in}}%
\pgfpathlineto{\pgfqpoint{5.800000in}{4.878000in}}%
\pgfpathclose%
\pgfusepath{fill}%
\end{pgfscope}%
\begin{pgfscope}%
\pgfpathrectangle{\pgfqpoint{5.800000in}{0.720000in}}{\pgfqpoint{1.400000in}{4.620000in}}%
\pgfusepath{clip}%
\pgfsetbuttcap%
\pgfsetmiterjoin%
\definecolor{currentfill}{rgb}{0.121569,0.466667,0.705882}%
\pgfsetfillcolor{currentfill}%
\pgfsetlinewidth{0.000000pt}%
\definecolor{currentstroke}{rgb}{0.000000,0.000000,0.000000}%
\pgfsetstrokecolor{currentstroke}%
\pgfsetstrokeopacity{0.000000}%
\pgfsetdash{}{0pt}%
\pgfpathmoveto{\pgfqpoint{5.800000in}{4.993500in}}%
\pgfpathlineto{\pgfqpoint{5.886732in}{4.993500in}}%
\pgfpathlineto{\pgfqpoint{5.886732in}{5.109000in}}%
\pgfpathlineto{\pgfqpoint{5.800000in}{5.109000in}}%
\pgfpathlineto{\pgfqpoint{5.800000in}{4.993500in}}%
\pgfpathclose%
\pgfusepath{fill}%
\end{pgfscope}%
\begin{pgfscope}%
\pgfpathrectangle{\pgfqpoint{5.800000in}{0.720000in}}{\pgfqpoint{1.400000in}{4.620000in}}%
\pgfusepath{clip}%
\pgfsetbuttcap%
\pgfsetmiterjoin%
\definecolor{currentfill}{rgb}{0.121569,0.466667,0.705882}%
\pgfsetfillcolor{currentfill}%
\pgfsetlinewidth{0.000000pt}%
\definecolor{currentstroke}{rgb}{0.000000,0.000000,0.000000}%
\pgfsetstrokecolor{currentstroke}%
\pgfsetstrokeopacity{0.000000}%
\pgfsetdash{}{0pt}%
\pgfpathmoveto{\pgfqpoint{5.800000in}{5.109000in}}%
\pgfpathlineto{\pgfqpoint{5.875612in}{5.109000in}}%
\pgfpathlineto{\pgfqpoint{5.875612in}{5.224500in}}%
\pgfpathlineto{\pgfqpoint{5.800000in}{5.224500in}}%
\pgfpathlineto{\pgfqpoint{5.800000in}{5.109000in}}%
\pgfpathclose%
\pgfusepath{fill}%
\end{pgfscope}%
\begin{pgfscope}%
\pgfpathrectangle{\pgfqpoint{5.800000in}{0.720000in}}{\pgfqpoint{1.400000in}{4.620000in}}%
\pgfusepath{clip}%
\pgfsetbuttcap%
\pgfsetmiterjoin%
\definecolor{currentfill}{rgb}{0.121569,0.466667,0.705882}%
\pgfsetfillcolor{currentfill}%
\pgfsetlinewidth{0.000000pt}%
\definecolor{currentstroke}{rgb}{0.000000,0.000000,0.000000}%
\pgfsetstrokecolor{currentstroke}%
\pgfsetstrokeopacity{0.000000}%
\pgfsetdash{}{0pt}%
\pgfpathmoveto{\pgfqpoint{5.800000in}{5.224500in}}%
\pgfpathlineto{\pgfqpoint{5.873388in}{5.224500in}}%
\pgfpathlineto{\pgfqpoint{5.873388in}{5.340000in}}%
\pgfpathlineto{\pgfqpoint{5.800000in}{5.340000in}}%
\pgfpathlineto{\pgfqpoint{5.800000in}{5.224500in}}%
\pgfpathclose%
\pgfusepath{fill}%
\end{pgfscope}%
\begin{pgfscope}%
\definecolor{textcolor}{rgb}{0.000000,0.000000,0.000000}%
\pgfsetstrokecolor{textcolor}%
\pgfsetfillcolor{textcolor}%
\pgftext[x=6.500000in,y=0.664444in,,top]{\color{textcolor}\sffamily\fontsize{20.000000}{24.000000}\selectfont \(\displaystyle \mathrm{arb.\ unit}\)}%
\end{pgfscope}%
\begin{pgfscope}%
\pgfsetrectcap%
\pgfsetmiterjoin%
\pgfsetlinewidth{0.803000pt}%
\definecolor{currentstroke}{rgb}{0.000000,0.000000,0.000000}%
\pgfsetstrokecolor{currentstroke}%
\pgfsetdash{}{0pt}%
\pgfpathmoveto{\pgfqpoint{5.800000in}{0.720000in}}%
\pgfpathlineto{\pgfqpoint{5.800000in}{5.340000in}}%
\pgfusepath{stroke}%
\end{pgfscope}%
\begin{pgfscope}%
\pgfsetrectcap%
\pgfsetmiterjoin%
\pgfsetlinewidth{0.803000pt}%
\definecolor{currentstroke}{rgb}{0.000000,0.000000,0.000000}%
\pgfsetstrokecolor{currentstroke}%
\pgfsetdash{}{0pt}%
\pgfpathmoveto{\pgfqpoint{7.200000in}{0.720000in}}%
\pgfpathlineto{\pgfqpoint{7.200000in}{5.340000in}}%
\pgfusepath{stroke}%
\end{pgfscope}%
\begin{pgfscope}%
\pgfsetrectcap%
\pgfsetmiterjoin%
\pgfsetlinewidth{0.803000pt}%
\definecolor{currentstroke}{rgb}{0.000000,0.000000,0.000000}%
\pgfsetstrokecolor{currentstroke}%
\pgfsetdash{}{0pt}%
\pgfpathmoveto{\pgfqpoint{5.800000in}{0.720000in}}%
\pgfpathlineto{\pgfqpoint{7.200000in}{0.720000in}}%
\pgfusepath{stroke}%
\end{pgfscope}%
\begin{pgfscope}%
\pgfsetrectcap%
\pgfsetmiterjoin%
\pgfsetlinewidth{0.803000pt}%
\definecolor{currentstroke}{rgb}{0.000000,0.000000,0.000000}%
\pgfsetstrokecolor{currentstroke}%
\pgfsetdash{}{0pt}%
\pgfpathmoveto{\pgfqpoint{5.800000in}{5.340000in}}%
\pgfpathlineto{\pgfqpoint{7.200000in}{5.340000in}}%
\pgfusepath{stroke}%
\end{pgfscope}%
\end{pgfpicture}%
\makeatother%
\endgroup%
}
    \caption{\label{fig:fsmp-npe} $D_\mathrm{w}$ histogram and distributions conditioned on $N_{\mathrm{PE}}$, errorbar explained in figure~\ref{fig:cnn-performance}.}
  \end{subfigure}
  \hspace{0.5em}
  \begin{subfigure}[b]{.55\textwidth}
    \centering
    \resizebox{\textwidth}{!}{%% Creator: Matplotlib, PGF backend
%%
%% To include the figure in your LaTeX document, write
%%   \input{<filename>.pgf}
%%
%% Make sure the required packages are loaded in your preamble
%%   \usepackage{pgf}
%%
%% Also ensure that all the required font packages are loaded; for instance,
%% the lmodern package is sometimes necessary when using math font.
%%   \usepackage{lmodern}
%%
%% Figures using additional raster images can only be included by \input if
%% they are in the same directory as the main LaTeX file. For loading figures
%% from other directories you can use the `import` package
%%   \usepackage{import}
%%
%% and then include the figures with
%%   \import{<path to file>}{<filename>.pgf}
%%
%% Matplotlib used the following preamble
%%   \usepackage[detect-all,locale=DE]{siunitx}
%%
\begingroup%
\makeatletter%
\begin{pgfpicture}%
\pgfpathrectangle{\pgfpointorigin}{\pgfqpoint{10.000000in}{10.000000in}}%
\pgfusepath{use as bounding box, clip}%
\begin{pgfscope}%
\pgfsetbuttcap%
\pgfsetmiterjoin%
\definecolor{currentfill}{rgb}{1.000000,1.000000,1.000000}%
\pgfsetfillcolor{currentfill}%
\pgfsetlinewidth{0.000000pt}%
\definecolor{currentstroke}{rgb}{1.000000,1.000000,1.000000}%
\pgfsetstrokecolor{currentstroke}%
\pgfsetdash{}{0pt}%
\pgfpathmoveto{\pgfqpoint{0.000000in}{0.000000in}}%
\pgfpathlineto{\pgfqpoint{10.000000in}{0.000000in}}%
\pgfpathlineto{\pgfqpoint{10.000000in}{10.000000in}}%
\pgfpathlineto{\pgfqpoint{0.000000in}{10.000000in}}%
\pgfpathlineto{\pgfqpoint{0.000000in}{0.000000in}}%
\pgfpathclose%
\pgfusepath{fill}%
\end{pgfscope}%
\begin{pgfscope}%
\pgfsetbuttcap%
\pgfsetmiterjoin%
\definecolor{currentfill}{rgb}{1.000000,1.000000,1.000000}%
\pgfsetfillcolor{currentfill}%
\pgfsetlinewidth{0.000000pt}%
\definecolor{currentstroke}{rgb}{0.000000,0.000000,0.000000}%
\pgfsetstrokecolor{currentstroke}%
\pgfsetstrokeopacity{0.000000}%
\pgfsetdash{}{0pt}%
\pgfpathmoveto{\pgfqpoint{1.000000in}{2.000000in}}%
\pgfpathlineto{\pgfqpoint{9.500000in}{2.000000in}}%
\pgfpathlineto{\pgfqpoint{9.500000in}{5.000000in}}%
\pgfpathlineto{\pgfqpoint{1.000000in}{5.000000in}}%
\pgfpathlineto{\pgfqpoint{1.000000in}{2.000000in}}%
\pgfpathclose%
\pgfusepath{fill}%
\end{pgfscope}%
\begin{pgfscope}%
\pgfpathrectangle{\pgfqpoint{1.000000in}{2.000000in}}{\pgfqpoint{8.500000in}{3.000000in}}%
\pgfusepath{clip}%
\pgfsetrectcap%
\pgfsetroundjoin%
\pgfsetlinewidth{0.803000pt}%
\definecolor{currentstroke}{rgb}{0.690196,0.690196,0.690196}%
\pgfsetstrokecolor{currentstroke}%
\pgfsetdash{}{0pt}%
\pgfpathmoveto{\pgfqpoint{2.014084in}{2.000000in}}%
\pgfpathlineto{\pgfqpoint{2.014084in}{5.000000in}}%
\pgfusepath{stroke}%
\end{pgfscope}%
\begin{pgfscope}%
\pgfsetbuttcap%
\pgfsetroundjoin%
\definecolor{currentfill}{rgb}{0.000000,0.000000,0.000000}%
\pgfsetfillcolor{currentfill}%
\pgfsetlinewidth{0.803000pt}%
\definecolor{currentstroke}{rgb}{0.000000,0.000000,0.000000}%
\pgfsetstrokecolor{currentstroke}%
\pgfsetdash{}{0pt}%
\pgfsys@defobject{currentmarker}{\pgfqpoint{0.000000in}{-0.048611in}}{\pgfqpoint{0.000000in}{0.000000in}}{%
\pgfpathmoveto{\pgfqpoint{0.000000in}{0.000000in}}%
\pgfpathlineto{\pgfqpoint{0.000000in}{-0.048611in}}%
\pgfusepath{stroke,fill}%
}%
\begin{pgfscope}%
\pgfsys@transformshift{2.014084in}{2.000000in}%
\pgfsys@useobject{currentmarker}{}%
\end{pgfscope}%
\end{pgfscope}%
\begin{pgfscope}%
\pgfpathrectangle{\pgfqpoint{1.000000in}{2.000000in}}{\pgfqpoint{8.500000in}{3.000000in}}%
\pgfusepath{clip}%
\pgfsetrectcap%
\pgfsetroundjoin%
\pgfsetlinewidth{0.803000pt}%
\definecolor{currentstroke}{rgb}{0.690196,0.690196,0.690196}%
\pgfsetstrokecolor{currentstroke}%
\pgfsetdash{}{0pt}%
\pgfpathmoveto{\pgfqpoint{3.657947in}{2.000000in}}%
\pgfpathlineto{\pgfqpoint{3.657947in}{5.000000in}}%
\pgfusepath{stroke}%
\end{pgfscope}%
\begin{pgfscope}%
\pgfsetbuttcap%
\pgfsetroundjoin%
\definecolor{currentfill}{rgb}{0.000000,0.000000,0.000000}%
\pgfsetfillcolor{currentfill}%
\pgfsetlinewidth{0.803000pt}%
\definecolor{currentstroke}{rgb}{0.000000,0.000000,0.000000}%
\pgfsetstrokecolor{currentstroke}%
\pgfsetdash{}{0pt}%
\pgfsys@defobject{currentmarker}{\pgfqpoint{0.000000in}{-0.048611in}}{\pgfqpoint{0.000000in}{0.000000in}}{%
\pgfpathmoveto{\pgfqpoint{0.000000in}{0.000000in}}%
\pgfpathlineto{\pgfqpoint{0.000000in}{-0.048611in}}%
\pgfusepath{stroke,fill}%
}%
\begin{pgfscope}%
\pgfsys@transformshift{3.657947in}{2.000000in}%
\pgfsys@useobject{currentmarker}{}%
\end{pgfscope}%
\end{pgfscope}%
\begin{pgfscope}%
\pgfpathrectangle{\pgfqpoint{1.000000in}{2.000000in}}{\pgfqpoint{8.500000in}{3.000000in}}%
\pgfusepath{clip}%
\pgfsetrectcap%
\pgfsetroundjoin%
\pgfsetlinewidth{0.803000pt}%
\definecolor{currentstroke}{rgb}{0.690196,0.690196,0.690196}%
\pgfsetstrokecolor{currentstroke}%
\pgfsetdash{}{0pt}%
\pgfpathmoveto{\pgfqpoint{5.301810in}{2.000000in}}%
\pgfpathlineto{\pgfqpoint{5.301810in}{5.000000in}}%
\pgfusepath{stroke}%
\end{pgfscope}%
\begin{pgfscope}%
\pgfsetbuttcap%
\pgfsetroundjoin%
\definecolor{currentfill}{rgb}{0.000000,0.000000,0.000000}%
\pgfsetfillcolor{currentfill}%
\pgfsetlinewidth{0.803000pt}%
\definecolor{currentstroke}{rgb}{0.000000,0.000000,0.000000}%
\pgfsetstrokecolor{currentstroke}%
\pgfsetdash{}{0pt}%
\pgfsys@defobject{currentmarker}{\pgfqpoint{0.000000in}{-0.048611in}}{\pgfqpoint{0.000000in}{0.000000in}}{%
\pgfpathmoveto{\pgfqpoint{0.000000in}{0.000000in}}%
\pgfpathlineto{\pgfqpoint{0.000000in}{-0.048611in}}%
\pgfusepath{stroke,fill}%
}%
\begin{pgfscope}%
\pgfsys@transformshift{5.301810in}{2.000000in}%
\pgfsys@useobject{currentmarker}{}%
\end{pgfscope}%
\end{pgfscope}%
\begin{pgfscope}%
\pgfpathrectangle{\pgfqpoint{1.000000in}{2.000000in}}{\pgfqpoint{8.500000in}{3.000000in}}%
\pgfusepath{clip}%
\pgfsetrectcap%
\pgfsetroundjoin%
\pgfsetlinewidth{0.803000pt}%
\definecolor{currentstroke}{rgb}{0.690196,0.690196,0.690196}%
\pgfsetstrokecolor{currentstroke}%
\pgfsetdash{}{0pt}%
\pgfpathmoveto{\pgfqpoint{6.945673in}{2.000000in}}%
\pgfpathlineto{\pgfqpoint{6.945673in}{5.000000in}}%
\pgfusepath{stroke}%
\end{pgfscope}%
\begin{pgfscope}%
\pgfsetbuttcap%
\pgfsetroundjoin%
\definecolor{currentfill}{rgb}{0.000000,0.000000,0.000000}%
\pgfsetfillcolor{currentfill}%
\pgfsetlinewidth{0.803000pt}%
\definecolor{currentstroke}{rgb}{0.000000,0.000000,0.000000}%
\pgfsetstrokecolor{currentstroke}%
\pgfsetdash{}{0pt}%
\pgfsys@defobject{currentmarker}{\pgfqpoint{0.000000in}{-0.048611in}}{\pgfqpoint{0.000000in}{0.000000in}}{%
\pgfpathmoveto{\pgfqpoint{0.000000in}{0.000000in}}%
\pgfpathlineto{\pgfqpoint{0.000000in}{-0.048611in}}%
\pgfusepath{stroke,fill}%
}%
\begin{pgfscope}%
\pgfsys@transformshift{6.945673in}{2.000000in}%
\pgfsys@useobject{currentmarker}{}%
\end{pgfscope}%
\end{pgfscope}%
\begin{pgfscope}%
\pgfpathrectangle{\pgfqpoint{1.000000in}{2.000000in}}{\pgfqpoint{8.500000in}{3.000000in}}%
\pgfusepath{clip}%
\pgfsetrectcap%
\pgfsetroundjoin%
\pgfsetlinewidth{0.803000pt}%
\definecolor{currentstroke}{rgb}{0.690196,0.690196,0.690196}%
\pgfsetstrokecolor{currentstroke}%
\pgfsetdash{}{0pt}%
\pgfpathmoveto{\pgfqpoint{8.589536in}{2.000000in}}%
\pgfpathlineto{\pgfqpoint{8.589536in}{5.000000in}}%
\pgfusepath{stroke}%
\end{pgfscope}%
\begin{pgfscope}%
\pgfsetbuttcap%
\pgfsetroundjoin%
\definecolor{currentfill}{rgb}{0.000000,0.000000,0.000000}%
\pgfsetfillcolor{currentfill}%
\pgfsetlinewidth{0.803000pt}%
\definecolor{currentstroke}{rgb}{0.000000,0.000000,0.000000}%
\pgfsetstrokecolor{currentstroke}%
\pgfsetdash{}{0pt}%
\pgfsys@defobject{currentmarker}{\pgfqpoint{0.000000in}{-0.048611in}}{\pgfqpoint{0.000000in}{0.000000in}}{%
\pgfpathmoveto{\pgfqpoint{0.000000in}{0.000000in}}%
\pgfpathlineto{\pgfqpoint{0.000000in}{-0.048611in}}%
\pgfusepath{stroke,fill}%
}%
\begin{pgfscope}%
\pgfsys@transformshift{8.589536in}{2.000000in}%
\pgfsys@useobject{currentmarker}{}%
\end{pgfscope}%
\end{pgfscope}%
\begin{pgfscope}%
\pgfpathrectangle{\pgfqpoint{1.000000in}{2.000000in}}{\pgfqpoint{8.500000in}{3.000000in}}%
\pgfusepath{clip}%
\pgfsetrectcap%
\pgfsetroundjoin%
\pgfsetlinewidth{0.803000pt}%
\definecolor{currentstroke}{rgb}{0.690196,0.690196,0.690196}%
\pgfsetstrokecolor{currentstroke}%
\pgfsetdash{}{0pt}%
\pgfpathmoveto{\pgfqpoint{1.000000in}{2.604534in}}%
\pgfpathlineto{\pgfqpoint{9.500000in}{2.604534in}}%
\pgfusepath{stroke}%
\end{pgfscope}%
\begin{pgfscope}%
\pgfsetbuttcap%
\pgfsetroundjoin%
\definecolor{currentfill}{rgb}{0.000000,0.000000,0.000000}%
\pgfsetfillcolor{currentfill}%
\pgfsetlinewidth{0.803000pt}%
\definecolor{currentstroke}{rgb}{0.000000,0.000000,0.000000}%
\pgfsetstrokecolor{currentstroke}%
\pgfsetdash{}{0pt}%
\pgfsys@defobject{currentmarker}{\pgfqpoint{-0.048611in}{0.000000in}}{\pgfqpoint{-0.000000in}{0.000000in}}{%
\pgfpathmoveto{\pgfqpoint{-0.000000in}{0.000000in}}%
\pgfpathlineto{\pgfqpoint{-0.048611in}{0.000000in}}%
\pgfusepath{stroke,fill}%
}%
\begin{pgfscope}%
\pgfsys@transformshift{1.000000in}{2.604534in}%
\pgfsys@useobject{currentmarker}{}%
\end{pgfscope}%
\end{pgfscope}%
\begin{pgfscope}%
\definecolor{textcolor}{rgb}{0.000000,0.000000,0.000000}%
\pgfsetstrokecolor{textcolor}%
\pgfsetfillcolor{textcolor}%
\pgftext[x=0.770670in, y=2.504515in, left, base]{\color{textcolor}\sffamily\fontsize{20.000000}{24.000000}\selectfont \(\displaystyle {0}\)}%
\end{pgfscope}%
\begin{pgfscope}%
\pgfpathrectangle{\pgfqpoint{1.000000in}{2.000000in}}{\pgfqpoint{8.500000in}{3.000000in}}%
\pgfusepath{clip}%
\pgfsetrectcap%
\pgfsetroundjoin%
\pgfsetlinewidth{0.803000pt}%
\definecolor{currentstroke}{rgb}{0.690196,0.690196,0.690196}%
\pgfsetstrokecolor{currentstroke}%
\pgfsetdash{}{0pt}%
\pgfpathmoveto{\pgfqpoint{1.000000in}{3.370941in}}%
\pgfpathlineto{\pgfqpoint{9.500000in}{3.370941in}}%
\pgfusepath{stroke}%
\end{pgfscope}%
\begin{pgfscope}%
\pgfsetbuttcap%
\pgfsetroundjoin%
\definecolor{currentfill}{rgb}{0.000000,0.000000,0.000000}%
\pgfsetfillcolor{currentfill}%
\pgfsetlinewidth{0.803000pt}%
\definecolor{currentstroke}{rgb}{0.000000,0.000000,0.000000}%
\pgfsetstrokecolor{currentstroke}%
\pgfsetdash{}{0pt}%
\pgfsys@defobject{currentmarker}{\pgfqpoint{-0.048611in}{0.000000in}}{\pgfqpoint{-0.000000in}{0.000000in}}{%
\pgfpathmoveto{\pgfqpoint{-0.000000in}{0.000000in}}%
\pgfpathlineto{\pgfqpoint{-0.048611in}{0.000000in}}%
\pgfusepath{stroke,fill}%
}%
\begin{pgfscope}%
\pgfsys@transformshift{1.000000in}{3.370941in}%
\pgfsys@useobject{currentmarker}{}%
\end{pgfscope}%
\end{pgfscope}%
\begin{pgfscope}%
\definecolor{textcolor}{rgb}{0.000000,0.000000,0.000000}%
\pgfsetstrokecolor{textcolor}%
\pgfsetfillcolor{textcolor}%
\pgftext[x=0.638563in, y=3.270921in, left, base]{\color{textcolor}\sffamily\fontsize{20.000000}{24.000000}\selectfont \(\displaystyle {10}\)}%
\end{pgfscope}%
\begin{pgfscope}%
\pgfpathrectangle{\pgfqpoint{1.000000in}{2.000000in}}{\pgfqpoint{8.500000in}{3.000000in}}%
\pgfusepath{clip}%
\pgfsetrectcap%
\pgfsetroundjoin%
\pgfsetlinewidth{0.803000pt}%
\definecolor{currentstroke}{rgb}{0.690196,0.690196,0.690196}%
\pgfsetstrokecolor{currentstroke}%
\pgfsetdash{}{0pt}%
\pgfpathmoveto{\pgfqpoint{1.000000in}{4.137347in}}%
\pgfpathlineto{\pgfqpoint{9.500000in}{4.137347in}}%
\pgfusepath{stroke}%
\end{pgfscope}%
\begin{pgfscope}%
\pgfsetbuttcap%
\pgfsetroundjoin%
\definecolor{currentfill}{rgb}{0.000000,0.000000,0.000000}%
\pgfsetfillcolor{currentfill}%
\pgfsetlinewidth{0.803000pt}%
\definecolor{currentstroke}{rgb}{0.000000,0.000000,0.000000}%
\pgfsetstrokecolor{currentstroke}%
\pgfsetdash{}{0pt}%
\pgfsys@defobject{currentmarker}{\pgfqpoint{-0.048611in}{0.000000in}}{\pgfqpoint{-0.000000in}{0.000000in}}{%
\pgfpathmoveto{\pgfqpoint{-0.000000in}{0.000000in}}%
\pgfpathlineto{\pgfqpoint{-0.048611in}{0.000000in}}%
\pgfusepath{stroke,fill}%
}%
\begin{pgfscope}%
\pgfsys@transformshift{1.000000in}{4.137347in}%
\pgfsys@useobject{currentmarker}{}%
\end{pgfscope}%
\end{pgfscope}%
\begin{pgfscope}%
\definecolor{textcolor}{rgb}{0.000000,0.000000,0.000000}%
\pgfsetstrokecolor{textcolor}%
\pgfsetfillcolor{textcolor}%
\pgftext[x=0.638563in, y=4.037328in, left, base]{\color{textcolor}\sffamily\fontsize{20.000000}{24.000000}\selectfont \(\displaystyle {20}\)}%
\end{pgfscope}%
\begin{pgfscope}%
\pgfpathrectangle{\pgfqpoint{1.000000in}{2.000000in}}{\pgfqpoint{8.500000in}{3.000000in}}%
\pgfusepath{clip}%
\pgfsetrectcap%
\pgfsetroundjoin%
\pgfsetlinewidth{0.803000pt}%
\definecolor{currentstroke}{rgb}{0.690196,0.690196,0.690196}%
\pgfsetstrokecolor{currentstroke}%
\pgfsetdash{}{0pt}%
\pgfpathmoveto{\pgfqpoint{1.000000in}{4.903754in}}%
\pgfpathlineto{\pgfqpoint{9.500000in}{4.903754in}}%
\pgfusepath{stroke}%
\end{pgfscope}%
\begin{pgfscope}%
\pgfsetbuttcap%
\pgfsetroundjoin%
\definecolor{currentfill}{rgb}{0.000000,0.000000,0.000000}%
\pgfsetfillcolor{currentfill}%
\pgfsetlinewidth{0.803000pt}%
\definecolor{currentstroke}{rgb}{0.000000,0.000000,0.000000}%
\pgfsetstrokecolor{currentstroke}%
\pgfsetdash{}{0pt}%
\pgfsys@defobject{currentmarker}{\pgfqpoint{-0.048611in}{0.000000in}}{\pgfqpoint{-0.000000in}{0.000000in}}{%
\pgfpathmoveto{\pgfqpoint{-0.000000in}{0.000000in}}%
\pgfpathlineto{\pgfqpoint{-0.048611in}{0.000000in}}%
\pgfusepath{stroke,fill}%
}%
\begin{pgfscope}%
\pgfsys@transformshift{1.000000in}{4.903754in}%
\pgfsys@useobject{currentmarker}{}%
\end{pgfscope}%
\end{pgfscope}%
\begin{pgfscope}%
\definecolor{textcolor}{rgb}{0.000000,0.000000,0.000000}%
\pgfsetstrokecolor{textcolor}%
\pgfsetfillcolor{textcolor}%
\pgftext[x=0.638563in, y=4.803734in, left, base]{\color{textcolor}\sffamily\fontsize{20.000000}{24.000000}\selectfont \(\displaystyle {30}\)}%
\end{pgfscope}%
\begin{pgfscope}%
\definecolor{textcolor}{rgb}{0.000000,0.000000,0.000000}%
\pgfsetstrokecolor{textcolor}%
\pgfsetfillcolor{textcolor}%
\pgftext[x=0.583007in,y=3.500000in,,bottom,rotate=90.000000]{\color{textcolor}\sffamily\fontsize{20.000000}{24.000000}\selectfont \(\displaystyle \mathrm{Voltage}/\si{mV}\)}%
\end{pgfscope}%
\begin{pgfscope}%
\pgfpathrectangle{\pgfqpoint{1.000000in}{2.000000in}}{\pgfqpoint{8.500000in}{3.000000in}}%
\pgfusepath{clip}%
\pgfsetrectcap%
\pgfsetroundjoin%
\pgfsetlinewidth{2.007500pt}%
\definecolor{currentstroke}{rgb}{0.000000,0.000000,1.000000}%
\pgfsetstrokecolor{currentstroke}%
\pgfsetdash{}{0pt}%
\pgfpathmoveto{\pgfqpoint{0.994888in}{2.536413in}}%
\pgfpathlineto{\pgfqpoint{1.027766in}{2.603761in}}%
\pgfpathlineto{\pgfqpoint{1.060643in}{2.501875in}}%
\pgfpathlineto{\pgfqpoint{1.093520in}{2.501829in}}%
\pgfpathlineto{\pgfqpoint{1.126397in}{2.533059in}}%
\pgfpathlineto{\pgfqpoint{1.159275in}{2.612422in}}%
\pgfpathlineto{\pgfqpoint{1.192152in}{2.768639in}}%
\pgfpathlineto{\pgfqpoint{1.225029in}{2.600580in}}%
\pgfpathlineto{\pgfqpoint{1.257906in}{2.657488in}}%
\pgfpathlineto{\pgfqpoint{1.290784in}{2.600062in}}%
\pgfpathlineto{\pgfqpoint{1.323661in}{2.653123in}}%
\pgfpathlineto{\pgfqpoint{1.356538in}{2.668734in}}%
\pgfpathlineto{\pgfqpoint{1.389415in}{2.625930in}}%
\pgfpathlineto{\pgfqpoint{1.422293in}{2.617155in}}%
\pgfpathlineto{\pgfqpoint{1.455170in}{2.683191in}}%
\pgfpathlineto{\pgfqpoint{1.488047in}{2.574846in}}%
\pgfpathlineto{\pgfqpoint{1.520925in}{2.641956in}}%
\pgfpathlineto{\pgfqpoint{1.553802in}{2.609008in}}%
\pgfpathlineto{\pgfqpoint{1.586679in}{2.695659in}}%
\pgfpathlineto{\pgfqpoint{1.619556in}{2.682701in}}%
\pgfpathlineto{\pgfqpoint{1.652434in}{2.589145in}}%
\pgfpathlineto{\pgfqpoint{1.685311in}{2.682535in}}%
\pgfpathlineto{\pgfqpoint{1.718188in}{2.573878in}}%
\pgfpathlineto{\pgfqpoint{1.751065in}{2.637820in}}%
\pgfpathlineto{\pgfqpoint{1.783943in}{2.627199in}}%
\pgfpathlineto{\pgfqpoint{1.816820in}{2.615181in}}%
\pgfpathlineto{\pgfqpoint{1.849697in}{2.691103in}}%
\pgfpathlineto{\pgfqpoint{1.882574in}{2.624270in}}%
\pgfpathlineto{\pgfqpoint{1.915452in}{2.570860in}}%
\pgfpathlineto{\pgfqpoint{1.948329in}{2.681011in}}%
\pgfpathlineto{\pgfqpoint{1.981206in}{2.625009in}}%
\pgfpathlineto{\pgfqpoint{2.014084in}{2.559125in}}%
\pgfpathlineto{\pgfqpoint{2.046961in}{2.642612in}}%
\pgfpathlineto{\pgfqpoint{2.079838in}{2.591472in}}%
\pgfpathlineto{\pgfqpoint{2.112715in}{2.773204in}}%
\pgfpathlineto{\pgfqpoint{2.145593in}{2.555601in}}%
\pgfpathlineto{\pgfqpoint{2.178470in}{2.507177in}}%
\pgfpathlineto{\pgfqpoint{2.211347in}{2.573177in}}%
\pgfpathlineto{\pgfqpoint{2.244224in}{2.529162in}}%
\pgfpathlineto{\pgfqpoint{2.277102in}{2.508023in}}%
\pgfpathlineto{\pgfqpoint{2.309979in}{2.597575in}}%
\pgfpathlineto{\pgfqpoint{2.342856in}{2.667224in}}%
\pgfpathlineto{\pgfqpoint{2.375733in}{2.648730in}}%
\pgfpathlineto{\pgfqpoint{2.408611in}{2.738006in}}%
\pgfpathlineto{\pgfqpoint{2.441488in}{2.670387in}}%
\pgfpathlineto{\pgfqpoint{2.474365in}{2.690048in}}%
\pgfpathlineto{\pgfqpoint{2.507242in}{2.701562in}}%
\pgfpathlineto{\pgfqpoint{2.540120in}{2.675485in}}%
\pgfpathlineto{\pgfqpoint{2.572997in}{2.589573in}}%
\pgfpathlineto{\pgfqpoint{2.605874in}{2.511122in}}%
\pgfpathlineto{\pgfqpoint{2.638752in}{2.620060in}}%
\pgfpathlineto{\pgfqpoint{2.671629in}{2.581281in}}%
\pgfpathlineto{\pgfqpoint{2.704506in}{2.671866in}}%
\pgfpathlineto{\pgfqpoint{2.737383in}{2.667313in}}%
\pgfpathlineto{\pgfqpoint{2.770261in}{2.840275in}}%
\pgfpathlineto{\pgfqpoint{2.803138in}{3.127460in}}%
\pgfpathlineto{\pgfqpoint{2.836015in}{3.459670in}}%
\pgfpathlineto{\pgfqpoint{2.868892in}{4.038997in}}%
\pgfpathlineto{\pgfqpoint{2.901770in}{4.185444in}}%
\pgfpathlineto{\pgfqpoint{2.934647in}{4.513160in}}%
\pgfpathlineto{\pgfqpoint{2.967524in}{4.616797in}}%
\pgfpathlineto{\pgfqpoint{3.000401in}{4.425675in}}%
\pgfpathlineto{\pgfqpoint{3.033279in}{4.412049in}}%
\pgfpathlineto{\pgfqpoint{3.066156in}{4.347996in}}%
\pgfpathlineto{\pgfqpoint{3.099033in}{3.948151in}}%
\pgfpathlineto{\pgfqpoint{3.131910in}{3.809925in}}%
\pgfpathlineto{\pgfqpoint{3.164788in}{3.801023in}}%
\pgfpathlineto{\pgfqpoint{3.197665in}{3.567672in}}%
\pgfpathlineto{\pgfqpoint{3.230542in}{3.263051in}}%
\pgfpathlineto{\pgfqpoint{3.263420in}{3.190625in}}%
\pgfpathlineto{\pgfqpoint{3.296297in}{3.124997in}}%
\pgfpathlineto{\pgfqpoint{3.329174in}{3.089604in}}%
\pgfpathlineto{\pgfqpoint{3.362051in}{3.264737in}}%
\pgfpathlineto{\pgfqpoint{3.394929in}{3.726763in}}%
\pgfpathlineto{\pgfqpoint{3.427806in}{3.920486in}}%
\pgfpathlineto{\pgfqpoint{3.460683in}{4.237986in}}%
\pgfpathlineto{\pgfqpoint{3.493560in}{4.310056in}}%
\pgfpathlineto{\pgfqpoint{3.526438in}{4.218483in}}%
\pgfpathlineto{\pgfqpoint{3.559315in}{4.087360in}}%
\pgfpathlineto{\pgfqpoint{3.592192in}{3.973223in}}%
\pgfpathlineto{\pgfqpoint{3.625069in}{3.840031in}}%
\pgfpathlineto{\pgfqpoint{3.657947in}{3.745744in}}%
\pgfpathlineto{\pgfqpoint{3.690824in}{3.520204in}}%
\pgfpathlineto{\pgfqpoint{3.723701in}{3.557602in}}%
\pgfpathlineto{\pgfqpoint{3.756578in}{3.396338in}}%
\pgfpathlineto{\pgfqpoint{3.789456in}{3.815021in}}%
\pgfpathlineto{\pgfqpoint{3.822333in}{3.820268in}}%
\pgfpathlineto{\pgfqpoint{3.855210in}{4.079018in}}%
\pgfpathlineto{\pgfqpoint{3.888088in}{3.975948in}}%
\pgfpathlineto{\pgfqpoint{3.920965in}{3.993751in}}%
\pgfpathlineto{\pgfqpoint{3.953842in}{3.930164in}}%
\pgfpathlineto{\pgfqpoint{3.986719in}{3.719486in}}%
\pgfpathlineto{\pgfqpoint{4.019597in}{3.628757in}}%
\pgfpathlineto{\pgfqpoint{4.052474in}{3.410685in}}%
\pgfpathlineto{\pgfqpoint{4.085351in}{3.295730in}}%
\pgfpathlineto{\pgfqpoint{4.118228in}{3.170311in}}%
\pgfpathlineto{\pgfqpoint{4.151106in}{2.995325in}}%
\pgfpathlineto{\pgfqpoint{4.183983in}{3.092827in}}%
\pgfpathlineto{\pgfqpoint{4.216860in}{3.104566in}}%
\pgfpathlineto{\pgfqpoint{4.249737in}{2.921506in}}%
\pgfpathlineto{\pgfqpoint{4.282615in}{2.760020in}}%
\pgfpathlineto{\pgfqpoint{4.315492in}{2.722833in}}%
\pgfpathlineto{\pgfqpoint{4.348369in}{2.893154in}}%
\pgfpathlineto{\pgfqpoint{4.381246in}{2.708703in}}%
\pgfpathlineto{\pgfqpoint{4.414124in}{2.690953in}}%
\pgfpathlineto{\pgfqpoint{4.447001in}{2.792417in}}%
\pgfpathlineto{\pgfqpoint{4.479878in}{2.657558in}}%
\pgfpathlineto{\pgfqpoint{4.512756in}{2.622667in}}%
\pgfpathlineto{\pgfqpoint{4.545633in}{2.677036in}}%
\pgfpathlineto{\pgfqpoint{4.578510in}{2.617444in}}%
\pgfpathlineto{\pgfqpoint{4.611387in}{2.571715in}}%
\pgfpathlineto{\pgfqpoint{4.644265in}{2.783753in}}%
\pgfpathlineto{\pgfqpoint{4.677142in}{2.862088in}}%
\pgfpathlineto{\pgfqpoint{4.710019in}{3.243718in}}%
\pgfpathlineto{\pgfqpoint{4.742896in}{3.420219in}}%
\pgfpathlineto{\pgfqpoint{4.775774in}{3.790538in}}%
\pgfpathlineto{\pgfqpoint{4.808651in}{3.757986in}}%
\pgfpathlineto{\pgfqpoint{4.841528in}{3.990583in}}%
\pgfpathlineto{\pgfqpoint{4.874405in}{3.886840in}}%
\pgfpathlineto{\pgfqpoint{4.907283in}{3.603993in}}%
\pgfpathlineto{\pgfqpoint{4.940160in}{3.603007in}}%
\pgfpathlineto{\pgfqpoint{4.973037in}{3.502479in}}%
\pgfpathlineto{\pgfqpoint{5.005915in}{3.331016in}}%
\pgfpathlineto{\pgfqpoint{5.038792in}{3.187587in}}%
\pgfpathlineto{\pgfqpoint{5.071669in}{3.138044in}}%
\pgfpathlineto{\pgfqpoint{5.104546in}{3.174083in}}%
\pgfpathlineto{\pgfqpoint{5.137424in}{3.122036in}}%
\pgfpathlineto{\pgfqpoint{5.170301in}{2.876976in}}%
\pgfpathlineto{\pgfqpoint{5.203178in}{2.852112in}}%
\pgfpathlineto{\pgfqpoint{5.236055in}{2.806763in}}%
\pgfpathlineto{\pgfqpoint{5.268933in}{2.741709in}}%
\pgfpathlineto{\pgfqpoint{5.301810in}{2.650913in}}%
\pgfpathlineto{\pgfqpoint{5.334687in}{2.790554in}}%
\pgfpathlineto{\pgfqpoint{5.367564in}{2.706100in}}%
\pgfpathlineto{\pgfqpoint{5.433319in}{2.605590in}}%
\pgfpathlineto{\pgfqpoint{5.466196in}{2.552236in}}%
\pgfpathlineto{\pgfqpoint{5.499073in}{2.727232in}}%
\pgfpathlineto{\pgfqpoint{5.531951in}{2.738876in}}%
\pgfpathlineto{\pgfqpoint{5.564828in}{2.596037in}}%
\pgfpathlineto{\pgfqpoint{5.597705in}{2.695232in}}%
\pgfpathlineto{\pgfqpoint{5.630583in}{2.483616in}}%
\pgfpathlineto{\pgfqpoint{5.663460in}{2.599328in}}%
\pgfpathlineto{\pgfqpoint{5.696337in}{2.617917in}}%
\pgfpathlineto{\pgfqpoint{5.729214in}{2.518321in}}%
\pgfpathlineto{\pgfqpoint{5.762092in}{2.565292in}}%
\pgfpathlineto{\pgfqpoint{5.794969in}{2.383203in}}%
\pgfpathlineto{\pgfqpoint{5.827846in}{2.520052in}}%
\pgfpathlineto{\pgfqpoint{5.860723in}{2.540718in}}%
\pgfpathlineto{\pgfqpoint{5.893601in}{2.460995in}}%
\pgfpathlineto{\pgfqpoint{5.926478in}{2.742694in}}%
\pgfpathlineto{\pgfqpoint{5.959355in}{2.649543in}}%
\pgfpathlineto{\pgfqpoint{5.992232in}{2.600044in}}%
\pgfpathlineto{\pgfqpoint{6.025110in}{2.664340in}}%
\pgfpathlineto{\pgfqpoint{6.057987in}{2.488431in}}%
\pgfpathlineto{\pgfqpoint{6.090864in}{2.557977in}}%
\pgfpathlineto{\pgfqpoint{6.123741in}{2.517426in}}%
\pgfpathlineto{\pgfqpoint{6.156619in}{2.457712in}}%
\pgfpathlineto{\pgfqpoint{6.189496in}{2.597034in}}%
\pgfpathlineto{\pgfqpoint{6.222373in}{2.615128in}}%
\pgfpathlineto{\pgfqpoint{6.255251in}{2.674582in}}%
\pgfpathlineto{\pgfqpoint{6.321005in}{2.587333in}}%
\pgfpathlineto{\pgfqpoint{6.353882in}{2.646580in}}%
\pgfpathlineto{\pgfqpoint{6.386760in}{2.665579in}}%
\pgfpathlineto{\pgfqpoint{6.419637in}{2.555747in}}%
\pgfpathlineto{\pgfqpoint{6.452514in}{2.532376in}}%
\pgfpathlineto{\pgfqpoint{6.518269in}{2.582312in}}%
\pgfpathlineto{\pgfqpoint{6.551146in}{2.526309in}}%
\pgfpathlineto{\pgfqpoint{6.584023in}{2.448784in}}%
\pgfpathlineto{\pgfqpoint{6.616900in}{2.562867in}}%
\pgfpathlineto{\pgfqpoint{6.649778in}{2.566127in}}%
\pgfpathlineto{\pgfqpoint{6.682655in}{2.585339in}}%
\pgfpathlineto{\pgfqpoint{6.715532in}{2.619870in}}%
\pgfpathlineto{\pgfqpoint{6.748409in}{2.718302in}}%
\pgfpathlineto{\pgfqpoint{6.781287in}{2.596296in}}%
\pgfpathlineto{\pgfqpoint{6.814164in}{2.545536in}}%
\pgfpathlineto{\pgfqpoint{6.847041in}{2.567543in}}%
\pgfpathlineto{\pgfqpoint{6.879919in}{2.694638in}}%
\pgfpathlineto{\pgfqpoint{6.912796in}{2.686475in}}%
\pgfpathlineto{\pgfqpoint{6.945673in}{2.561027in}}%
\pgfpathlineto{\pgfqpoint{6.978550in}{2.535530in}}%
\pgfpathlineto{\pgfqpoint{7.011428in}{2.580230in}}%
\pgfpathlineto{\pgfqpoint{7.044305in}{2.519088in}}%
\pgfpathlineto{\pgfqpoint{7.077182in}{2.538465in}}%
\pgfpathlineto{\pgfqpoint{7.110059in}{2.588972in}}%
\pgfpathlineto{\pgfqpoint{7.142937in}{2.617744in}}%
\pgfpathlineto{\pgfqpoint{7.175814in}{2.575363in}}%
\pgfpathlineto{\pgfqpoint{7.208691in}{2.650268in}}%
\pgfpathlineto{\pgfqpoint{7.241568in}{2.532545in}}%
\pgfpathlineto{\pgfqpoint{7.274446in}{2.671577in}}%
\pgfpathlineto{\pgfqpoint{7.307323in}{2.501490in}}%
\pgfpathlineto{\pgfqpoint{7.340200in}{2.550915in}}%
\pgfpathlineto{\pgfqpoint{7.373077in}{2.566349in}}%
\pgfpathlineto{\pgfqpoint{7.405955in}{2.561385in}}%
\pgfpathlineto{\pgfqpoint{7.438832in}{2.723696in}}%
\pgfpathlineto{\pgfqpoint{7.471709in}{2.616616in}}%
\pgfpathlineto{\pgfqpoint{7.504587in}{2.682578in}}%
\pgfpathlineto{\pgfqpoint{7.537464in}{2.644229in}}%
\pgfpathlineto{\pgfqpoint{7.570341in}{2.650330in}}%
\pgfpathlineto{\pgfqpoint{7.603218in}{2.612508in}}%
\pgfpathlineto{\pgfqpoint{7.636096in}{2.538810in}}%
\pgfpathlineto{\pgfqpoint{7.668973in}{2.652743in}}%
\pgfpathlineto{\pgfqpoint{7.701850in}{2.575692in}}%
\pgfpathlineto{\pgfqpoint{7.734727in}{2.475133in}}%
\pgfpathlineto{\pgfqpoint{7.767605in}{2.602215in}}%
\pgfpathlineto{\pgfqpoint{7.800482in}{2.676438in}}%
\pgfpathlineto{\pgfqpoint{7.833359in}{2.652642in}}%
\pgfpathlineto{\pgfqpoint{7.866236in}{2.644918in}}%
\pgfpathlineto{\pgfqpoint{7.899114in}{2.467517in}}%
\pgfpathlineto{\pgfqpoint{7.931991in}{2.613041in}}%
\pgfpathlineto{\pgfqpoint{7.964868in}{2.420644in}}%
\pgfpathlineto{\pgfqpoint{7.997746in}{2.613021in}}%
\pgfpathlineto{\pgfqpoint{8.030623in}{2.509669in}}%
\pgfpathlineto{\pgfqpoint{8.063500in}{2.596173in}}%
\pgfpathlineto{\pgfqpoint{8.096377in}{2.618203in}}%
\pgfpathlineto{\pgfqpoint{8.129255in}{2.547606in}}%
\pgfpathlineto{\pgfqpoint{8.162132in}{2.538008in}}%
\pgfpathlineto{\pgfqpoint{8.195009in}{2.655028in}}%
\pgfpathlineto{\pgfqpoint{8.227886in}{2.744526in}}%
\pgfpathlineto{\pgfqpoint{8.260764in}{2.654142in}}%
\pgfpathlineto{\pgfqpoint{8.293641in}{2.489750in}}%
\pgfpathlineto{\pgfqpoint{8.326518in}{2.662746in}}%
\pgfpathlineto{\pgfqpoint{8.359395in}{2.502816in}}%
\pgfpathlineto{\pgfqpoint{8.392273in}{2.558689in}}%
\pgfpathlineto{\pgfqpoint{8.425150in}{2.588060in}}%
\pgfpathlineto{\pgfqpoint{8.458027in}{2.558029in}}%
\pgfpathlineto{\pgfqpoint{8.490904in}{2.491519in}}%
\pgfpathlineto{\pgfqpoint{8.523782in}{2.651752in}}%
\pgfpathlineto{\pgfqpoint{8.556659in}{2.534628in}}%
\pgfpathlineto{\pgfqpoint{8.589536in}{2.666431in}}%
\pgfpathlineto{\pgfqpoint{8.622414in}{2.607255in}}%
\pgfpathlineto{\pgfqpoint{8.655291in}{2.690909in}}%
\pgfpathlineto{\pgfqpoint{8.688168in}{2.580678in}}%
\pgfpathlineto{\pgfqpoint{8.721045in}{2.606806in}}%
\pgfpathlineto{\pgfqpoint{8.753923in}{2.758102in}}%
\pgfpathlineto{\pgfqpoint{8.786800in}{2.509101in}}%
\pgfpathlineto{\pgfqpoint{8.819677in}{2.522510in}}%
\pgfpathlineto{\pgfqpoint{8.852554in}{2.686123in}}%
\pgfpathlineto{\pgfqpoint{8.885432in}{2.630770in}}%
\pgfpathlineto{\pgfqpoint{8.918309in}{2.673832in}}%
\pgfpathlineto{\pgfqpoint{8.951186in}{2.605243in}}%
\pgfpathlineto{\pgfqpoint{8.984063in}{2.591680in}}%
\pgfpathlineto{\pgfqpoint{9.016941in}{2.551346in}}%
\pgfpathlineto{\pgfqpoint{9.049818in}{2.818084in}}%
\pgfpathlineto{\pgfqpoint{9.082695in}{2.541807in}}%
\pgfpathlineto{\pgfqpoint{9.115572in}{2.631749in}}%
\pgfpathlineto{\pgfqpoint{9.148450in}{2.702113in}}%
\pgfpathlineto{\pgfqpoint{9.181327in}{2.676599in}}%
\pgfpathlineto{\pgfqpoint{9.214204in}{2.706566in}}%
\pgfpathlineto{\pgfqpoint{9.247082in}{2.527200in}}%
\pgfpathlineto{\pgfqpoint{9.279959in}{2.594899in}}%
\pgfpathlineto{\pgfqpoint{9.312836in}{2.615903in}}%
\pgfpathlineto{\pgfqpoint{9.345713in}{2.504064in}}%
\pgfpathlineto{\pgfqpoint{9.378591in}{2.605356in}}%
\pgfpathlineto{\pgfqpoint{9.411468in}{2.661173in}}%
\pgfpathlineto{\pgfqpoint{9.444345in}{2.650616in}}%
\pgfpathlineto{\pgfqpoint{9.477222in}{2.551195in}}%
\pgfpathlineto{\pgfqpoint{9.510000in}{2.693207in}}%
\pgfpathlineto{\pgfqpoint{9.510000in}{2.693207in}}%
\pgfusepath{stroke}%
\end{pgfscope}%
\begin{pgfscope}%
\pgfpathrectangle{\pgfqpoint{1.000000in}{2.000000in}}{\pgfqpoint{8.500000in}{3.000000in}}%
\pgfusepath{clip}%
\pgfsetrectcap%
\pgfsetroundjoin%
\pgfsetlinewidth{2.007500pt}%
\definecolor{currentstroke}{rgb}{0.000000,0.000000,0.000000}%
\pgfsetstrokecolor{currentstroke}%
\pgfsetdash{}{0pt}%
\pgfpathmoveto{\pgfqpoint{0.994888in}{2.604534in}}%
\pgfpathlineto{\pgfqpoint{2.671629in}{2.604561in}}%
\pgfpathlineto{\pgfqpoint{2.704506in}{2.613324in}}%
\pgfpathlineto{\pgfqpoint{2.737383in}{2.681284in}}%
\pgfpathlineto{\pgfqpoint{2.770261in}{2.833915in}}%
\pgfpathlineto{\pgfqpoint{2.803138in}{3.101768in}}%
\pgfpathlineto{\pgfqpoint{2.836015in}{3.508895in}}%
\pgfpathlineto{\pgfqpoint{2.868892in}{3.949581in}}%
\pgfpathlineto{\pgfqpoint{2.901770in}{4.298159in}}%
\pgfpathlineto{\pgfqpoint{2.934647in}{4.496376in}}%
\pgfpathlineto{\pgfqpoint{2.967524in}{4.547997in}}%
\pgfpathlineto{\pgfqpoint{3.000401in}{4.486810in}}%
\pgfpathlineto{\pgfqpoint{3.033279in}{4.352558in}}%
\pgfpathlineto{\pgfqpoint{3.066156in}{4.179328in}}%
\pgfpathlineto{\pgfqpoint{3.131910in}{3.807317in}}%
\pgfpathlineto{\pgfqpoint{3.164788in}{3.634586in}}%
\pgfpathlineto{\pgfqpoint{3.197665in}{3.478798in}}%
\pgfpathlineto{\pgfqpoint{3.230542in}{3.341698in}}%
\pgfpathlineto{\pgfqpoint{3.263420in}{3.223132in}}%
\pgfpathlineto{\pgfqpoint{3.296297in}{3.130433in}}%
\pgfpathlineto{\pgfqpoint{3.329174in}{3.156145in}}%
\pgfpathlineto{\pgfqpoint{3.362051in}{3.381130in}}%
\pgfpathlineto{\pgfqpoint{3.394929in}{3.710794in}}%
\pgfpathlineto{\pgfqpoint{3.427806in}{4.006270in}}%
\pgfpathlineto{\pgfqpoint{3.460683in}{4.191158in}}%
\pgfpathlineto{\pgfqpoint{3.493560in}{4.253770in}}%
\pgfpathlineto{\pgfqpoint{3.526438in}{4.216745in}}%
\pgfpathlineto{\pgfqpoint{3.559315in}{4.112481in}}%
\pgfpathlineto{\pgfqpoint{3.592192in}{3.970545in}}%
\pgfpathlineto{\pgfqpoint{3.657947in}{3.655959in}}%
\pgfpathlineto{\pgfqpoint{3.690824in}{3.511672in}}%
\pgfpathlineto{\pgfqpoint{3.723701in}{3.447844in}}%
\pgfpathlineto{\pgfqpoint{3.756578in}{3.539021in}}%
\pgfpathlineto{\pgfqpoint{3.789456in}{3.726563in}}%
\pgfpathlineto{\pgfqpoint{3.822333in}{3.904629in}}%
\pgfpathlineto{\pgfqpoint{3.855210in}{4.009883in}}%
\pgfpathlineto{\pgfqpoint{3.888088in}{4.029028in}}%
\pgfpathlineto{\pgfqpoint{3.920965in}{3.976747in}}%
\pgfpathlineto{\pgfqpoint{3.953842in}{3.876465in}}%
\pgfpathlineto{\pgfqpoint{3.986719in}{3.750125in}}%
\pgfpathlineto{\pgfqpoint{4.052474in}{3.480862in}}%
\pgfpathlineto{\pgfqpoint{4.085351in}{3.355945in}}%
\pgfpathlineto{\pgfqpoint{4.118228in}{3.243193in}}%
\pgfpathlineto{\pgfqpoint{4.151106in}{3.143859in}}%
\pgfpathlineto{\pgfqpoint{4.183983in}{3.057841in}}%
\pgfpathlineto{\pgfqpoint{4.216860in}{2.984279in}}%
\pgfpathlineto{\pgfqpoint{4.249737in}{2.921938in}}%
\pgfpathlineto{\pgfqpoint{4.282615in}{2.869454in}}%
\pgfpathlineto{\pgfqpoint{4.315492in}{2.825480in}}%
\pgfpathlineto{\pgfqpoint{4.348369in}{2.788759in}}%
\pgfpathlineto{\pgfqpoint{4.381246in}{2.758167in}}%
\pgfpathlineto{\pgfqpoint{4.414124in}{2.732718in}}%
\pgfpathlineto{\pgfqpoint{4.447001in}{2.711564in}}%
\pgfpathlineto{\pgfqpoint{4.479878in}{2.693987in}}%
\pgfpathlineto{\pgfqpoint{4.512756in}{2.679379in}}%
\pgfpathlineto{\pgfqpoint{4.545633in}{2.667236in}}%
\pgfpathlineto{\pgfqpoint{4.578510in}{2.657133in}}%
\pgfpathlineto{\pgfqpoint{4.611387in}{2.650515in}}%
\pgfpathlineto{\pgfqpoint{4.644265in}{2.699457in}}%
\pgfpathlineto{\pgfqpoint{4.677142in}{2.902968in}}%
\pgfpathlineto{\pgfqpoint{4.742896in}{3.535044in}}%
\pgfpathlineto{\pgfqpoint{4.775774in}{3.759370in}}%
\pgfpathlineto{\pgfqpoint{4.808651in}{3.871749in}}%
\pgfpathlineto{\pgfqpoint{4.841528in}{3.886427in}}%
\pgfpathlineto{\pgfqpoint{4.874405in}{3.831046in}}%
\pgfpathlineto{\pgfqpoint{4.907283in}{3.732997in}}%
\pgfpathlineto{\pgfqpoint{4.940160in}{3.614023in}}%
\pgfpathlineto{\pgfqpoint{5.005915in}{3.368454in}}%
\pgfpathlineto{\pgfqpoint{5.038792in}{3.256838in}}%
\pgfpathlineto{\pgfqpoint{5.071669in}{3.157002in}}%
\pgfpathlineto{\pgfqpoint{5.104546in}{3.069659in}}%
\pgfpathlineto{\pgfqpoint{5.137424in}{2.994442in}}%
\pgfpathlineto{\pgfqpoint{5.170301in}{2.930400in}}%
\pgfpathlineto{\pgfqpoint{5.203178in}{2.876325in}}%
\pgfpathlineto{\pgfqpoint{5.236055in}{2.830939in}}%
\pgfpathlineto{\pgfqpoint{5.268933in}{2.793011in}}%
\pgfpathlineto{\pgfqpoint{5.301810in}{2.761412in}}%
\pgfpathlineto{\pgfqpoint{5.334687in}{2.735141in}}%
\pgfpathlineto{\pgfqpoint{5.367564in}{2.713327in}}%
\pgfpathlineto{\pgfqpoint{5.400442in}{2.695226in}}%
\pgfpathlineto{\pgfqpoint{5.433319in}{2.680211in}}%
\pgfpathlineto{\pgfqpoint{5.466196in}{2.667752in}}%
\pgfpathlineto{\pgfqpoint{5.499073in}{2.657411in}}%
\pgfpathlineto{\pgfqpoint{5.531951in}{2.648820in}}%
\pgfpathlineto{\pgfqpoint{5.564828in}{2.641677in}}%
\pgfpathlineto{\pgfqpoint{5.630583in}{2.630779in}}%
\pgfpathlineto{\pgfqpoint{5.696337in}{2.623192in}}%
\pgfpathlineto{\pgfqpoint{5.762092in}{2.617882in}}%
\pgfpathlineto{\pgfqpoint{5.860723in}{2.612707in}}%
\pgfpathlineto{\pgfqpoint{5.992232in}{2.608876in}}%
\pgfpathlineto{\pgfqpoint{6.156619in}{2.606570in}}%
\pgfpathlineto{\pgfqpoint{6.452514in}{2.605102in}}%
\pgfpathlineto{\pgfqpoint{7.175814in}{2.604571in}}%
\pgfpathlineto{\pgfqpoint{9.510000in}{2.604534in}}%
\pgfpathlineto{\pgfqpoint{9.510000in}{2.604534in}}%
\pgfusepath{stroke}%
\end{pgfscope}%
\begin{pgfscope}%
\pgfpathrectangle{\pgfqpoint{1.000000in}{2.000000in}}{\pgfqpoint{8.500000in}{3.000000in}}%
\pgfusepath{clip}%
\pgfsetrectcap%
\pgfsetroundjoin%
\pgfsetlinewidth{2.007500pt}%
\definecolor{currentstroke}{rgb}{0.000000,0.500000,0.000000}%
\pgfsetstrokecolor{currentstroke}%
\pgfsetdash{}{0pt}%
\pgfpathmoveto{\pgfqpoint{0.994888in}{2.604534in}}%
\pgfpathlineto{\pgfqpoint{2.671629in}{2.604598in}}%
\pgfpathlineto{\pgfqpoint{2.704506in}{2.612281in}}%
\pgfpathlineto{\pgfqpoint{2.737383in}{2.657642in}}%
\pgfpathlineto{\pgfqpoint{2.770261in}{2.778016in}}%
\pgfpathlineto{\pgfqpoint{2.803138in}{3.076383in}}%
\pgfpathlineto{\pgfqpoint{2.868892in}{4.007166in}}%
\pgfpathlineto{\pgfqpoint{2.901770in}{4.355106in}}%
\pgfpathlineto{\pgfqpoint{2.934647in}{4.537796in}}%
\pgfpathlineto{\pgfqpoint{2.967524in}{4.570818in}}%
\pgfpathlineto{\pgfqpoint{3.000401in}{4.493853in}}%
\pgfpathlineto{\pgfqpoint{3.033279in}{4.348399in}}%
\pgfpathlineto{\pgfqpoint{3.066156in}{4.168274in}}%
\pgfpathlineto{\pgfqpoint{3.131910in}{3.791463in}}%
\pgfpathlineto{\pgfqpoint{3.164788in}{3.618945in}}%
\pgfpathlineto{\pgfqpoint{3.197665in}{3.464203in}}%
\pgfpathlineto{\pgfqpoint{3.230542in}{3.328560in}}%
\pgfpathlineto{\pgfqpoint{3.263420in}{3.211844in}}%
\pgfpathlineto{\pgfqpoint{3.296297in}{3.143309in}}%
\pgfpathlineto{\pgfqpoint{3.329174in}{3.241828in}}%
\pgfpathlineto{\pgfqpoint{3.362051in}{3.518027in}}%
\pgfpathlineto{\pgfqpoint{3.394929in}{3.840736in}}%
\pgfpathlineto{\pgfqpoint{3.427806in}{4.091537in}}%
\pgfpathlineto{\pgfqpoint{3.460683in}{4.222949in}}%
\pgfpathlineto{\pgfqpoint{3.493560in}{4.240336in}}%
\pgfpathlineto{\pgfqpoint{3.526438in}{4.171998in}}%
\pgfpathlineto{\pgfqpoint{3.559315in}{4.049717in}}%
\pgfpathlineto{\pgfqpoint{3.592192in}{3.900086in}}%
\pgfpathlineto{\pgfqpoint{3.657947in}{3.588747in}}%
\pgfpathlineto{\pgfqpoint{3.690824in}{3.470009in}}%
\pgfpathlineto{\pgfqpoint{3.723701in}{3.481408in}}%
\pgfpathlineto{\pgfqpoint{3.756578in}{3.633588in}}%
\pgfpathlineto{\pgfqpoint{3.789456in}{3.827724in}}%
\pgfpathlineto{\pgfqpoint{3.822333in}{3.973807in}}%
\pgfpathlineto{\pgfqpoint{3.855210in}{4.035360in}}%
\pgfpathlineto{\pgfqpoint{3.888088in}{4.015894in}}%
\pgfpathlineto{\pgfqpoint{3.920965in}{3.936292in}}%
\pgfpathlineto{\pgfqpoint{3.953842in}{3.820073in}}%
\pgfpathlineto{\pgfqpoint{4.052474in}{3.420208in}}%
\pgfpathlineto{\pgfqpoint{4.085351in}{3.300643in}}%
\pgfpathlineto{\pgfqpoint{4.118228in}{3.194133in}}%
\pgfpathlineto{\pgfqpoint{4.151106in}{3.101168in}}%
\pgfpathlineto{\pgfqpoint{4.183983in}{3.021206in}}%
\pgfpathlineto{\pgfqpoint{4.216860in}{2.953156in}}%
\pgfpathlineto{\pgfqpoint{4.249737in}{2.895691in}}%
\pgfpathlineto{\pgfqpoint{4.282615in}{2.847438in}}%
\pgfpathlineto{\pgfqpoint{4.315492in}{2.807082in}}%
\pgfpathlineto{\pgfqpoint{4.348369in}{2.773426in}}%
\pgfpathlineto{\pgfqpoint{4.381246in}{2.745409in}}%
\pgfpathlineto{\pgfqpoint{4.414124in}{2.722114in}}%
\pgfpathlineto{\pgfqpoint{4.447001in}{2.702755in}}%
\pgfpathlineto{\pgfqpoint{4.479878in}{2.686669in}}%
\pgfpathlineto{\pgfqpoint{4.512756in}{2.673299in}}%
\pgfpathlineto{\pgfqpoint{4.545633in}{2.662180in}}%
\pgfpathlineto{\pgfqpoint{4.578510in}{2.653143in}}%
\pgfpathlineto{\pgfqpoint{4.611387in}{2.671775in}}%
\pgfpathlineto{\pgfqpoint{4.644265in}{2.819869in}}%
\pgfpathlineto{\pgfqpoint{4.677142in}{3.107840in}}%
\pgfpathlineto{\pgfqpoint{4.710019in}{3.426202in}}%
\pgfpathlineto{\pgfqpoint{4.742896in}{3.676155in}}%
\pgfpathlineto{\pgfqpoint{4.775774in}{3.818743in}}%
\pgfpathlineto{\pgfqpoint{4.808651in}{3.859542in}}%
\pgfpathlineto{\pgfqpoint{4.841528in}{3.823380in}}%
\pgfpathlineto{\pgfqpoint{4.874405in}{3.737887in}}%
\pgfpathlineto{\pgfqpoint{4.907283in}{3.626188in}}%
\pgfpathlineto{\pgfqpoint{4.973037in}{3.385216in}}%
\pgfpathlineto{\pgfqpoint{5.005915in}{3.273187in}}%
\pgfpathlineto{\pgfqpoint{5.038792in}{3.172123in}}%
\pgfpathlineto{\pgfqpoint{5.071669in}{3.083176in}}%
\pgfpathlineto{\pgfqpoint{5.104546in}{3.006247in}}%
\pgfpathlineto{\pgfqpoint{5.137424in}{2.940544in}}%
\pgfpathlineto{\pgfqpoint{5.170301in}{2.884939in}}%
\pgfpathlineto{\pgfqpoint{5.203178in}{2.838191in}}%
\pgfpathlineto{\pgfqpoint{5.236055in}{2.799079in}}%
\pgfpathlineto{\pgfqpoint{5.268933in}{2.766467in}}%
\pgfpathlineto{\pgfqpoint{5.301810in}{2.739338in}}%
\pgfpathlineto{\pgfqpoint{5.334687in}{2.716804in}}%
\pgfpathlineto{\pgfqpoint{5.367564in}{2.698103in}}%
\pgfpathlineto{\pgfqpoint{5.400442in}{2.682589in}}%
\pgfpathlineto{\pgfqpoint{5.433319in}{2.669717in}}%
\pgfpathlineto{\pgfqpoint{5.466196in}{2.659034in}}%
\pgfpathlineto{\pgfqpoint{5.499073in}{2.650162in}}%
\pgfpathlineto{\pgfqpoint{5.531951in}{2.642787in}}%
\pgfpathlineto{\pgfqpoint{5.597705in}{2.631540in}}%
\pgfpathlineto{\pgfqpoint{5.663460in}{2.623716in}}%
\pgfpathlineto{\pgfqpoint{5.729214in}{2.618245in}}%
\pgfpathlineto{\pgfqpoint{5.827846in}{2.612918in}}%
\pgfpathlineto{\pgfqpoint{5.959355in}{2.608981in}}%
\pgfpathlineto{\pgfqpoint{6.123741in}{2.606615in}}%
\pgfpathlineto{\pgfqpoint{6.419637in}{2.605112in}}%
\pgfpathlineto{\pgfqpoint{7.142937in}{2.604571in}}%
\pgfpathlineto{\pgfqpoint{9.510000in}{2.604534in}}%
\pgfpathlineto{\pgfqpoint{9.510000in}{2.604534in}}%
\pgfusepath{stroke}%
\end{pgfscope}%
\begin{pgfscope}%
\pgfpathrectangle{\pgfqpoint{1.000000in}{2.000000in}}{\pgfqpoint{8.500000in}{3.000000in}}%
\pgfusepath{clip}%
\pgfsetbuttcap%
\pgfsetroundjoin%
\pgfsetlinewidth{2.007500pt}%
\definecolor{currentstroke}{rgb}{0.000000,0.750000,0.750000}%
\pgfsetstrokecolor{currentstroke}%
\pgfsetdash{}{0pt}%
\pgfpathmoveto{\pgfqpoint{0.990000in}{2.987737in}}%
\pgfpathlineto{\pgfqpoint{9.510000in}{2.987737in}}%
\pgfusepath{stroke}%
\end{pgfscope}%
\begin{pgfscope}%
\pgfsetrectcap%
\pgfsetmiterjoin%
\pgfsetlinewidth{0.803000pt}%
\definecolor{currentstroke}{rgb}{0.000000,0.000000,0.000000}%
\pgfsetstrokecolor{currentstroke}%
\pgfsetdash{}{0pt}%
\pgfpathmoveto{\pgfqpoint{1.000000in}{2.000000in}}%
\pgfpathlineto{\pgfqpoint{1.000000in}{5.000000in}}%
\pgfusepath{stroke}%
\end{pgfscope}%
\begin{pgfscope}%
\pgfsetrectcap%
\pgfsetmiterjoin%
\pgfsetlinewidth{0.803000pt}%
\definecolor{currentstroke}{rgb}{0.000000,0.000000,0.000000}%
\pgfsetstrokecolor{currentstroke}%
\pgfsetdash{}{0pt}%
\pgfpathmoveto{\pgfqpoint{9.500000in}{2.000000in}}%
\pgfpathlineto{\pgfqpoint{9.500000in}{5.000000in}}%
\pgfusepath{stroke}%
\end{pgfscope}%
\begin{pgfscope}%
\pgfsetrectcap%
\pgfsetmiterjoin%
\pgfsetlinewidth{0.803000pt}%
\definecolor{currentstroke}{rgb}{0.000000,0.000000,0.000000}%
\pgfsetstrokecolor{currentstroke}%
\pgfsetdash{}{0pt}%
\pgfpathmoveto{\pgfqpoint{1.000000in}{2.000000in}}%
\pgfpathlineto{\pgfqpoint{9.500000in}{2.000000in}}%
\pgfusepath{stroke}%
\end{pgfscope}%
\begin{pgfscope}%
\pgfsetrectcap%
\pgfsetmiterjoin%
\pgfsetlinewidth{0.803000pt}%
\definecolor{currentstroke}{rgb}{0.000000,0.000000,0.000000}%
\pgfsetstrokecolor{currentstroke}%
\pgfsetdash{}{0pt}%
\pgfpathmoveto{\pgfqpoint{1.000000in}{5.000000in}}%
\pgfpathlineto{\pgfqpoint{9.500000in}{5.000000in}}%
\pgfusepath{stroke}%
\end{pgfscope}%
\begin{pgfscope}%
\pgfsetbuttcap%
\pgfsetmiterjoin%
\definecolor{currentfill}{rgb}{1.000000,1.000000,1.000000}%
\pgfsetfillcolor{currentfill}%
\pgfsetfillopacity{0.800000}%
\pgfsetlinewidth{1.003750pt}%
\definecolor{currentstroke}{rgb}{0.800000,0.800000,0.800000}%
\pgfsetstrokecolor{currentstroke}%
\pgfsetstrokeopacity{0.800000}%
\pgfsetdash{}{0pt}%
\pgfpathmoveto{\pgfqpoint{5.738045in}{3.197952in}}%
\pgfpathlineto{\pgfqpoint{9.305556in}{3.197952in}}%
\pgfpathquadraticcurveto{\pgfqpoint{9.361111in}{3.197952in}}{\pgfqpoint{9.361111in}{3.253507in}}%
\pgfpathlineto{\pgfqpoint{9.361111in}{4.805556in}}%
\pgfpathquadraticcurveto{\pgfqpoint{9.361111in}{4.861111in}}{\pgfqpoint{9.305556in}{4.861111in}}%
\pgfpathlineto{\pgfqpoint{5.738045in}{4.861111in}}%
\pgfpathquadraticcurveto{\pgfqpoint{5.682490in}{4.861111in}}{\pgfqpoint{5.682490in}{4.805556in}}%
\pgfpathlineto{\pgfqpoint{5.682490in}{3.253507in}}%
\pgfpathquadraticcurveto{\pgfqpoint{5.682490in}{3.197952in}}{\pgfqpoint{5.738045in}{3.197952in}}%
\pgfpathlineto{\pgfqpoint{5.738045in}{3.197952in}}%
\pgfpathclose%
\pgfusepath{stroke,fill}%
\end{pgfscope}%
\begin{pgfscope}%
\pgfsetrectcap%
\pgfsetroundjoin%
\pgfsetlinewidth{2.007500pt}%
\definecolor{currentstroke}{rgb}{0.000000,0.000000,1.000000}%
\pgfsetstrokecolor{currentstroke}%
\pgfsetdash{}{0pt}%
\pgfpathmoveto{\pgfqpoint{5.793601in}{4.647184in}}%
\pgfpathlineto{\pgfqpoint{6.071379in}{4.647184in}}%
\pgfpathlineto{\pgfqpoint{6.349157in}{4.647184in}}%
\pgfusepath{stroke}%
\end{pgfscope}%
\begin{pgfscope}%
\definecolor{textcolor}{rgb}{0.000000,0.000000,0.000000}%
\pgfsetstrokecolor{textcolor}%
\pgfsetfillcolor{textcolor}%
\pgftext[x=6.571379in,y=4.549962in,left,base]{\color{textcolor}\sffamily\fontsize{20.000000}{24.000000}\selectfont noisy waveform}%
\end{pgfscope}%
\begin{pgfscope}%
\pgfsetrectcap%
\pgfsetroundjoin%
\pgfsetlinewidth{2.007500pt}%
\definecolor{currentstroke}{rgb}{0.000000,0.000000,0.000000}%
\pgfsetstrokecolor{currentstroke}%
\pgfsetdash{}{0pt}%
\pgfpathmoveto{\pgfqpoint{5.793601in}{4.252227in}}%
\pgfpathlineto{\pgfqpoint{6.071379in}{4.252227in}}%
\pgfpathlineto{\pgfqpoint{6.349157in}{4.252227in}}%
\pgfusepath{stroke}%
\end{pgfscope}%
\begin{pgfscope}%
\definecolor{textcolor}{rgb}{0.000000,0.000000,0.000000}%
\pgfsetstrokecolor{textcolor}%
\pgfsetfillcolor{textcolor}%
\pgftext[x=6.571379in,y=4.155005in,left,base]{\color{textcolor}\sffamily\fontsize{20.000000}{24.000000}\selectfont true waveform}%
\end{pgfscope}%
\begin{pgfscope}%
\pgfsetrectcap%
\pgfsetroundjoin%
\pgfsetlinewidth{2.007500pt}%
\definecolor{currentstroke}{rgb}{0.000000,0.500000,0.000000}%
\pgfsetstrokecolor{currentstroke}%
\pgfsetdash{}{0pt}%
\pgfpathmoveto{\pgfqpoint{5.793601in}{3.857271in}}%
\pgfpathlineto{\pgfqpoint{6.071379in}{3.857271in}}%
\pgfpathlineto{\pgfqpoint{6.349157in}{3.857271in}}%
\pgfusepath{stroke}%
\end{pgfscope}%
\begin{pgfscope}%
\definecolor{textcolor}{rgb}{0.000000,0.000000,0.000000}%
\pgfsetstrokecolor{textcolor}%
\pgfsetfillcolor{textcolor}%
\pgftext[x=6.571379in,y=3.760048in,left,base]{\color{textcolor}\sffamily\fontsize{20.000000}{24.000000}\selectfont reconstructed waveform}%
\end{pgfscope}%
\begin{pgfscope}%
\pgfsetbuttcap%
\pgfsetroundjoin%
\pgfsetlinewidth{2.007500pt}%
\definecolor{currentstroke}{rgb}{0.000000,0.750000,0.750000}%
\pgfsetstrokecolor{currentstroke}%
\pgfsetdash{}{0pt}%
\pgfpathmoveto{\pgfqpoint{5.793601in}{3.462314in}}%
\pgfpathlineto{\pgfqpoint{6.349157in}{3.462314in}}%
\pgfusepath{stroke}%
\end{pgfscope}%
\begin{pgfscope}%
\definecolor{textcolor}{rgb}{0.000000,0.000000,0.000000}%
\pgfsetstrokecolor{textcolor}%
\pgfsetfillcolor{textcolor}%
\pgftext[x=6.571379in,y=3.365092in,left,base]{\color{textcolor}\sffamily\fontsize{20.000000}{24.000000}\selectfont threshold}%
\end{pgfscope}%
\begin{pgfscope}%
\pgfsetbuttcap%
\pgfsetmiterjoin%
\definecolor{currentfill}{rgb}{1.000000,1.000000,1.000000}%
\pgfsetfillcolor{currentfill}%
\pgfsetlinewidth{0.000000pt}%
\definecolor{currentstroke}{rgb}{0.000000,0.000000,0.000000}%
\pgfsetstrokecolor{currentstroke}%
\pgfsetstrokeopacity{0.000000}%
\pgfsetdash{}{0pt}%
\pgfpathmoveto{\pgfqpoint{1.000000in}{5.000000in}}%
\pgfpathlineto{\pgfqpoint{9.500000in}{5.000000in}}%
\pgfpathlineto{\pgfqpoint{9.500000in}{7.000000in}}%
\pgfpathlineto{\pgfqpoint{1.000000in}{7.000000in}}%
\pgfpathlineto{\pgfqpoint{1.000000in}{5.000000in}}%
\pgfpathclose%
\pgfusepath{fill}%
\end{pgfscope}%
\begin{pgfscope}%
\pgfpathrectangle{\pgfqpoint{1.000000in}{5.000000in}}{\pgfqpoint{8.500000in}{2.000000in}}%
\pgfusepath{clip}%
\pgfsetrectcap%
\pgfsetroundjoin%
\pgfsetlinewidth{0.803000pt}%
\definecolor{currentstroke}{rgb}{0.690196,0.690196,0.690196}%
\pgfsetstrokecolor{currentstroke}%
\pgfsetdash{}{0pt}%
\pgfpathmoveto{\pgfqpoint{2.014084in}{5.000000in}}%
\pgfpathlineto{\pgfqpoint{2.014084in}{7.000000in}}%
\pgfusepath{stroke}%
\end{pgfscope}%
\begin{pgfscope}%
\pgfsetbuttcap%
\pgfsetroundjoin%
\definecolor{currentfill}{rgb}{0.000000,0.000000,0.000000}%
\pgfsetfillcolor{currentfill}%
\pgfsetlinewidth{0.803000pt}%
\definecolor{currentstroke}{rgb}{0.000000,0.000000,0.000000}%
\pgfsetstrokecolor{currentstroke}%
\pgfsetdash{}{0pt}%
\pgfsys@defobject{currentmarker}{\pgfqpoint{0.000000in}{-0.048611in}}{\pgfqpoint{0.000000in}{0.000000in}}{%
\pgfpathmoveto{\pgfqpoint{0.000000in}{0.000000in}}%
\pgfpathlineto{\pgfqpoint{0.000000in}{-0.048611in}}%
\pgfusepath{stroke,fill}%
}%
\begin{pgfscope}%
\pgfsys@transformshift{2.014084in}{5.000000in}%
\pgfsys@useobject{currentmarker}{}%
\end{pgfscope}%
\end{pgfscope}%
\begin{pgfscope}%
\pgfpathrectangle{\pgfqpoint{1.000000in}{5.000000in}}{\pgfqpoint{8.500000in}{2.000000in}}%
\pgfusepath{clip}%
\pgfsetrectcap%
\pgfsetroundjoin%
\pgfsetlinewidth{0.803000pt}%
\definecolor{currentstroke}{rgb}{0.690196,0.690196,0.690196}%
\pgfsetstrokecolor{currentstroke}%
\pgfsetdash{}{0pt}%
\pgfpathmoveto{\pgfqpoint{3.657947in}{5.000000in}}%
\pgfpathlineto{\pgfqpoint{3.657947in}{7.000000in}}%
\pgfusepath{stroke}%
\end{pgfscope}%
\begin{pgfscope}%
\pgfsetbuttcap%
\pgfsetroundjoin%
\definecolor{currentfill}{rgb}{0.000000,0.000000,0.000000}%
\pgfsetfillcolor{currentfill}%
\pgfsetlinewidth{0.803000pt}%
\definecolor{currentstroke}{rgb}{0.000000,0.000000,0.000000}%
\pgfsetstrokecolor{currentstroke}%
\pgfsetdash{}{0pt}%
\pgfsys@defobject{currentmarker}{\pgfqpoint{0.000000in}{-0.048611in}}{\pgfqpoint{0.000000in}{0.000000in}}{%
\pgfpathmoveto{\pgfqpoint{0.000000in}{0.000000in}}%
\pgfpathlineto{\pgfqpoint{0.000000in}{-0.048611in}}%
\pgfusepath{stroke,fill}%
}%
\begin{pgfscope}%
\pgfsys@transformshift{3.657947in}{5.000000in}%
\pgfsys@useobject{currentmarker}{}%
\end{pgfscope}%
\end{pgfscope}%
\begin{pgfscope}%
\pgfpathrectangle{\pgfqpoint{1.000000in}{5.000000in}}{\pgfqpoint{8.500000in}{2.000000in}}%
\pgfusepath{clip}%
\pgfsetrectcap%
\pgfsetroundjoin%
\pgfsetlinewidth{0.803000pt}%
\definecolor{currentstroke}{rgb}{0.690196,0.690196,0.690196}%
\pgfsetstrokecolor{currentstroke}%
\pgfsetdash{}{0pt}%
\pgfpathmoveto{\pgfqpoint{5.301810in}{5.000000in}}%
\pgfpathlineto{\pgfqpoint{5.301810in}{7.000000in}}%
\pgfusepath{stroke}%
\end{pgfscope}%
\begin{pgfscope}%
\pgfsetbuttcap%
\pgfsetroundjoin%
\definecolor{currentfill}{rgb}{0.000000,0.000000,0.000000}%
\pgfsetfillcolor{currentfill}%
\pgfsetlinewidth{0.803000pt}%
\definecolor{currentstroke}{rgb}{0.000000,0.000000,0.000000}%
\pgfsetstrokecolor{currentstroke}%
\pgfsetdash{}{0pt}%
\pgfsys@defobject{currentmarker}{\pgfqpoint{0.000000in}{-0.048611in}}{\pgfqpoint{0.000000in}{0.000000in}}{%
\pgfpathmoveto{\pgfqpoint{0.000000in}{0.000000in}}%
\pgfpathlineto{\pgfqpoint{0.000000in}{-0.048611in}}%
\pgfusepath{stroke,fill}%
}%
\begin{pgfscope}%
\pgfsys@transformshift{5.301810in}{5.000000in}%
\pgfsys@useobject{currentmarker}{}%
\end{pgfscope}%
\end{pgfscope}%
\begin{pgfscope}%
\pgfpathrectangle{\pgfqpoint{1.000000in}{5.000000in}}{\pgfqpoint{8.500000in}{2.000000in}}%
\pgfusepath{clip}%
\pgfsetrectcap%
\pgfsetroundjoin%
\pgfsetlinewidth{0.803000pt}%
\definecolor{currentstroke}{rgb}{0.690196,0.690196,0.690196}%
\pgfsetstrokecolor{currentstroke}%
\pgfsetdash{}{0pt}%
\pgfpathmoveto{\pgfqpoint{6.945673in}{5.000000in}}%
\pgfpathlineto{\pgfqpoint{6.945673in}{7.000000in}}%
\pgfusepath{stroke}%
\end{pgfscope}%
\begin{pgfscope}%
\pgfsetbuttcap%
\pgfsetroundjoin%
\definecolor{currentfill}{rgb}{0.000000,0.000000,0.000000}%
\pgfsetfillcolor{currentfill}%
\pgfsetlinewidth{0.803000pt}%
\definecolor{currentstroke}{rgb}{0.000000,0.000000,0.000000}%
\pgfsetstrokecolor{currentstroke}%
\pgfsetdash{}{0pt}%
\pgfsys@defobject{currentmarker}{\pgfqpoint{0.000000in}{-0.048611in}}{\pgfqpoint{0.000000in}{0.000000in}}{%
\pgfpathmoveto{\pgfqpoint{0.000000in}{0.000000in}}%
\pgfpathlineto{\pgfqpoint{0.000000in}{-0.048611in}}%
\pgfusepath{stroke,fill}%
}%
\begin{pgfscope}%
\pgfsys@transformshift{6.945673in}{5.000000in}%
\pgfsys@useobject{currentmarker}{}%
\end{pgfscope}%
\end{pgfscope}%
\begin{pgfscope}%
\pgfpathrectangle{\pgfqpoint{1.000000in}{5.000000in}}{\pgfqpoint{8.500000in}{2.000000in}}%
\pgfusepath{clip}%
\pgfsetrectcap%
\pgfsetroundjoin%
\pgfsetlinewidth{0.803000pt}%
\definecolor{currentstroke}{rgb}{0.690196,0.690196,0.690196}%
\pgfsetstrokecolor{currentstroke}%
\pgfsetdash{}{0pt}%
\pgfpathmoveto{\pgfqpoint{8.589536in}{5.000000in}}%
\pgfpathlineto{\pgfqpoint{8.589536in}{7.000000in}}%
\pgfusepath{stroke}%
\end{pgfscope}%
\begin{pgfscope}%
\pgfsetbuttcap%
\pgfsetroundjoin%
\definecolor{currentfill}{rgb}{0.000000,0.000000,0.000000}%
\pgfsetfillcolor{currentfill}%
\pgfsetlinewidth{0.803000pt}%
\definecolor{currentstroke}{rgb}{0.000000,0.000000,0.000000}%
\pgfsetstrokecolor{currentstroke}%
\pgfsetdash{}{0pt}%
\pgfsys@defobject{currentmarker}{\pgfqpoint{0.000000in}{-0.048611in}}{\pgfqpoint{0.000000in}{0.000000in}}{%
\pgfpathmoveto{\pgfqpoint{0.000000in}{0.000000in}}%
\pgfpathlineto{\pgfqpoint{0.000000in}{-0.048611in}}%
\pgfusepath{stroke,fill}%
}%
\begin{pgfscope}%
\pgfsys@transformshift{8.589536in}{5.000000in}%
\pgfsys@useobject{currentmarker}{}%
\end{pgfscope}%
\end{pgfscope}%
\begin{pgfscope}%
\pgfpathrectangle{\pgfqpoint{1.000000in}{5.000000in}}{\pgfqpoint{8.500000in}{2.000000in}}%
\pgfusepath{clip}%
\pgfsetrectcap%
\pgfsetroundjoin%
\pgfsetlinewidth{0.803000pt}%
\definecolor{currentstroke}{rgb}{0.690196,0.690196,0.690196}%
\pgfsetstrokecolor{currentstroke}%
\pgfsetdash{}{0pt}%
\pgfpathmoveto{\pgfqpoint{1.000000in}{5.000000in}}%
\pgfpathlineto{\pgfqpoint{9.500000in}{5.000000in}}%
\pgfusepath{stroke}%
\end{pgfscope}%
\begin{pgfscope}%
\pgfsetbuttcap%
\pgfsetroundjoin%
\definecolor{currentfill}{rgb}{0.000000,0.000000,0.000000}%
\pgfsetfillcolor{currentfill}%
\pgfsetlinewidth{0.803000pt}%
\definecolor{currentstroke}{rgb}{0.000000,0.000000,0.000000}%
\pgfsetstrokecolor{currentstroke}%
\pgfsetdash{}{0pt}%
\pgfsys@defobject{currentmarker}{\pgfqpoint{-0.048611in}{0.000000in}}{\pgfqpoint{-0.000000in}{0.000000in}}{%
\pgfpathmoveto{\pgfqpoint{-0.000000in}{0.000000in}}%
\pgfpathlineto{\pgfqpoint{-0.048611in}{0.000000in}}%
\pgfusepath{stroke,fill}%
}%
\begin{pgfscope}%
\pgfsys@transformshift{1.000000in}{5.000000in}%
\pgfsys@useobject{currentmarker}{}%
\end{pgfscope}%
\end{pgfscope}%
\begin{pgfscope}%
\definecolor{textcolor}{rgb}{0.000000,0.000000,0.000000}%
\pgfsetstrokecolor{textcolor}%
\pgfsetfillcolor{textcolor}%
\pgftext[x=0.560215in, y=4.899981in, left, base]{\color{textcolor}\sffamily\fontsize{20.000000}{24.000000}\selectfont \(\displaystyle {0.0}\)}%
\end{pgfscope}%
\begin{pgfscope}%
\pgfpathrectangle{\pgfqpoint{1.000000in}{5.000000in}}{\pgfqpoint{8.500000in}{2.000000in}}%
\pgfusepath{clip}%
\pgfsetrectcap%
\pgfsetroundjoin%
\pgfsetlinewidth{0.803000pt}%
\definecolor{currentstroke}{rgb}{0.690196,0.690196,0.690196}%
\pgfsetstrokecolor{currentstroke}%
\pgfsetdash{}{0pt}%
\pgfpathmoveto{\pgfqpoint{1.000000in}{5.239485in}}%
\pgfpathlineto{\pgfqpoint{9.500000in}{5.239485in}}%
\pgfusepath{stroke}%
\end{pgfscope}%
\begin{pgfscope}%
\pgfsetbuttcap%
\pgfsetroundjoin%
\definecolor{currentfill}{rgb}{0.000000,0.000000,0.000000}%
\pgfsetfillcolor{currentfill}%
\pgfsetlinewidth{0.803000pt}%
\definecolor{currentstroke}{rgb}{0.000000,0.000000,0.000000}%
\pgfsetstrokecolor{currentstroke}%
\pgfsetdash{}{0pt}%
\pgfsys@defobject{currentmarker}{\pgfqpoint{-0.048611in}{0.000000in}}{\pgfqpoint{-0.000000in}{0.000000in}}{%
\pgfpathmoveto{\pgfqpoint{-0.000000in}{0.000000in}}%
\pgfpathlineto{\pgfqpoint{-0.048611in}{0.000000in}}%
\pgfusepath{stroke,fill}%
}%
\begin{pgfscope}%
\pgfsys@transformshift{1.000000in}{5.239485in}%
\pgfsys@useobject{currentmarker}{}%
\end{pgfscope}%
\end{pgfscope}%
\begin{pgfscope}%
\definecolor{textcolor}{rgb}{0.000000,0.000000,0.000000}%
\pgfsetstrokecolor{textcolor}%
\pgfsetfillcolor{textcolor}%
\pgftext[x=0.560215in, y=5.139465in, left, base]{\color{textcolor}\sffamily\fontsize{20.000000}{24.000000}\selectfont \(\displaystyle {0.2}\)}%
\end{pgfscope}%
\begin{pgfscope}%
\pgfpathrectangle{\pgfqpoint{1.000000in}{5.000000in}}{\pgfqpoint{8.500000in}{2.000000in}}%
\pgfusepath{clip}%
\pgfsetrectcap%
\pgfsetroundjoin%
\pgfsetlinewidth{0.803000pt}%
\definecolor{currentstroke}{rgb}{0.690196,0.690196,0.690196}%
\pgfsetstrokecolor{currentstroke}%
\pgfsetdash{}{0pt}%
\pgfpathmoveto{\pgfqpoint{1.000000in}{5.478969in}}%
\pgfpathlineto{\pgfqpoint{9.500000in}{5.478969in}}%
\pgfusepath{stroke}%
\end{pgfscope}%
\begin{pgfscope}%
\pgfsetbuttcap%
\pgfsetroundjoin%
\definecolor{currentfill}{rgb}{0.000000,0.000000,0.000000}%
\pgfsetfillcolor{currentfill}%
\pgfsetlinewidth{0.803000pt}%
\definecolor{currentstroke}{rgb}{0.000000,0.000000,0.000000}%
\pgfsetstrokecolor{currentstroke}%
\pgfsetdash{}{0pt}%
\pgfsys@defobject{currentmarker}{\pgfqpoint{-0.048611in}{0.000000in}}{\pgfqpoint{-0.000000in}{0.000000in}}{%
\pgfpathmoveto{\pgfqpoint{-0.000000in}{0.000000in}}%
\pgfpathlineto{\pgfqpoint{-0.048611in}{0.000000in}}%
\pgfusepath{stroke,fill}%
}%
\begin{pgfscope}%
\pgfsys@transformshift{1.000000in}{5.478969in}%
\pgfsys@useobject{currentmarker}{}%
\end{pgfscope}%
\end{pgfscope}%
\begin{pgfscope}%
\definecolor{textcolor}{rgb}{0.000000,0.000000,0.000000}%
\pgfsetstrokecolor{textcolor}%
\pgfsetfillcolor{textcolor}%
\pgftext[x=0.560215in, y=5.378950in, left, base]{\color{textcolor}\sffamily\fontsize{20.000000}{24.000000}\selectfont \(\displaystyle {0.4}\)}%
\end{pgfscope}%
\begin{pgfscope}%
\pgfpathrectangle{\pgfqpoint{1.000000in}{5.000000in}}{\pgfqpoint{8.500000in}{2.000000in}}%
\pgfusepath{clip}%
\pgfsetrectcap%
\pgfsetroundjoin%
\pgfsetlinewidth{0.803000pt}%
\definecolor{currentstroke}{rgb}{0.690196,0.690196,0.690196}%
\pgfsetstrokecolor{currentstroke}%
\pgfsetdash{}{0pt}%
\pgfpathmoveto{\pgfqpoint{1.000000in}{5.718454in}}%
\pgfpathlineto{\pgfqpoint{9.500000in}{5.718454in}}%
\pgfusepath{stroke}%
\end{pgfscope}%
\begin{pgfscope}%
\pgfsetbuttcap%
\pgfsetroundjoin%
\definecolor{currentfill}{rgb}{0.000000,0.000000,0.000000}%
\pgfsetfillcolor{currentfill}%
\pgfsetlinewidth{0.803000pt}%
\definecolor{currentstroke}{rgb}{0.000000,0.000000,0.000000}%
\pgfsetstrokecolor{currentstroke}%
\pgfsetdash{}{0pt}%
\pgfsys@defobject{currentmarker}{\pgfqpoint{-0.048611in}{0.000000in}}{\pgfqpoint{-0.000000in}{0.000000in}}{%
\pgfpathmoveto{\pgfqpoint{-0.000000in}{0.000000in}}%
\pgfpathlineto{\pgfqpoint{-0.048611in}{0.000000in}}%
\pgfusepath{stroke,fill}%
}%
\begin{pgfscope}%
\pgfsys@transformshift{1.000000in}{5.718454in}%
\pgfsys@useobject{currentmarker}{}%
\end{pgfscope}%
\end{pgfscope}%
\begin{pgfscope}%
\definecolor{textcolor}{rgb}{0.000000,0.000000,0.000000}%
\pgfsetstrokecolor{textcolor}%
\pgfsetfillcolor{textcolor}%
\pgftext[x=0.560215in, y=5.618435in, left, base]{\color{textcolor}\sffamily\fontsize{20.000000}{24.000000}\selectfont \(\displaystyle {0.6}\)}%
\end{pgfscope}%
\begin{pgfscope}%
\pgfpathrectangle{\pgfqpoint{1.000000in}{5.000000in}}{\pgfqpoint{8.500000in}{2.000000in}}%
\pgfusepath{clip}%
\pgfsetrectcap%
\pgfsetroundjoin%
\pgfsetlinewidth{0.803000pt}%
\definecolor{currentstroke}{rgb}{0.690196,0.690196,0.690196}%
\pgfsetstrokecolor{currentstroke}%
\pgfsetdash{}{0pt}%
\pgfpathmoveto{\pgfqpoint{1.000000in}{5.957939in}}%
\pgfpathlineto{\pgfqpoint{9.500000in}{5.957939in}}%
\pgfusepath{stroke}%
\end{pgfscope}%
\begin{pgfscope}%
\pgfsetbuttcap%
\pgfsetroundjoin%
\definecolor{currentfill}{rgb}{0.000000,0.000000,0.000000}%
\pgfsetfillcolor{currentfill}%
\pgfsetlinewidth{0.803000pt}%
\definecolor{currentstroke}{rgb}{0.000000,0.000000,0.000000}%
\pgfsetstrokecolor{currentstroke}%
\pgfsetdash{}{0pt}%
\pgfsys@defobject{currentmarker}{\pgfqpoint{-0.048611in}{0.000000in}}{\pgfqpoint{-0.000000in}{0.000000in}}{%
\pgfpathmoveto{\pgfqpoint{-0.000000in}{0.000000in}}%
\pgfpathlineto{\pgfqpoint{-0.048611in}{0.000000in}}%
\pgfusepath{stroke,fill}%
}%
\begin{pgfscope}%
\pgfsys@transformshift{1.000000in}{5.957939in}%
\pgfsys@useobject{currentmarker}{}%
\end{pgfscope}%
\end{pgfscope}%
\begin{pgfscope}%
\definecolor{textcolor}{rgb}{0.000000,0.000000,0.000000}%
\pgfsetstrokecolor{textcolor}%
\pgfsetfillcolor{textcolor}%
\pgftext[x=0.560215in, y=5.857920in, left, base]{\color{textcolor}\sffamily\fontsize{20.000000}{24.000000}\selectfont \(\displaystyle {0.8}\)}%
\end{pgfscope}%
\begin{pgfscope}%
\pgfpathrectangle{\pgfqpoint{1.000000in}{5.000000in}}{\pgfqpoint{8.500000in}{2.000000in}}%
\pgfusepath{clip}%
\pgfsetrectcap%
\pgfsetroundjoin%
\pgfsetlinewidth{0.803000pt}%
\definecolor{currentstroke}{rgb}{0.690196,0.690196,0.690196}%
\pgfsetstrokecolor{currentstroke}%
\pgfsetdash{}{0pt}%
\pgfpathmoveto{\pgfqpoint{1.000000in}{6.197424in}}%
\pgfpathlineto{\pgfqpoint{9.500000in}{6.197424in}}%
\pgfusepath{stroke}%
\end{pgfscope}%
\begin{pgfscope}%
\pgfsetbuttcap%
\pgfsetroundjoin%
\definecolor{currentfill}{rgb}{0.000000,0.000000,0.000000}%
\pgfsetfillcolor{currentfill}%
\pgfsetlinewidth{0.803000pt}%
\definecolor{currentstroke}{rgb}{0.000000,0.000000,0.000000}%
\pgfsetstrokecolor{currentstroke}%
\pgfsetdash{}{0pt}%
\pgfsys@defobject{currentmarker}{\pgfqpoint{-0.048611in}{0.000000in}}{\pgfqpoint{-0.000000in}{0.000000in}}{%
\pgfpathmoveto{\pgfqpoint{-0.000000in}{0.000000in}}%
\pgfpathlineto{\pgfqpoint{-0.048611in}{0.000000in}}%
\pgfusepath{stroke,fill}%
}%
\begin{pgfscope}%
\pgfsys@transformshift{1.000000in}{6.197424in}%
\pgfsys@useobject{currentmarker}{}%
\end{pgfscope}%
\end{pgfscope}%
\begin{pgfscope}%
\definecolor{textcolor}{rgb}{0.000000,0.000000,0.000000}%
\pgfsetstrokecolor{textcolor}%
\pgfsetfillcolor{textcolor}%
\pgftext[x=0.560215in, y=6.097404in, left, base]{\color{textcolor}\sffamily\fontsize{20.000000}{24.000000}\selectfont \(\displaystyle {1.0}\)}%
\end{pgfscope}%
\begin{pgfscope}%
\pgfpathrectangle{\pgfqpoint{1.000000in}{5.000000in}}{\pgfqpoint{8.500000in}{2.000000in}}%
\pgfusepath{clip}%
\pgfsetrectcap%
\pgfsetroundjoin%
\pgfsetlinewidth{0.803000pt}%
\definecolor{currentstroke}{rgb}{0.690196,0.690196,0.690196}%
\pgfsetstrokecolor{currentstroke}%
\pgfsetdash{}{0pt}%
\pgfpathmoveto{\pgfqpoint{1.000000in}{6.436908in}}%
\pgfpathlineto{\pgfqpoint{9.500000in}{6.436908in}}%
\pgfusepath{stroke}%
\end{pgfscope}%
\begin{pgfscope}%
\pgfsetbuttcap%
\pgfsetroundjoin%
\definecolor{currentfill}{rgb}{0.000000,0.000000,0.000000}%
\pgfsetfillcolor{currentfill}%
\pgfsetlinewidth{0.803000pt}%
\definecolor{currentstroke}{rgb}{0.000000,0.000000,0.000000}%
\pgfsetstrokecolor{currentstroke}%
\pgfsetdash{}{0pt}%
\pgfsys@defobject{currentmarker}{\pgfqpoint{-0.048611in}{0.000000in}}{\pgfqpoint{-0.000000in}{0.000000in}}{%
\pgfpathmoveto{\pgfqpoint{-0.000000in}{0.000000in}}%
\pgfpathlineto{\pgfqpoint{-0.048611in}{0.000000in}}%
\pgfusepath{stroke,fill}%
}%
\begin{pgfscope}%
\pgfsys@transformshift{1.000000in}{6.436908in}%
\pgfsys@useobject{currentmarker}{}%
\end{pgfscope}%
\end{pgfscope}%
\begin{pgfscope}%
\definecolor{textcolor}{rgb}{0.000000,0.000000,0.000000}%
\pgfsetstrokecolor{textcolor}%
\pgfsetfillcolor{textcolor}%
\pgftext[x=0.560215in, y=6.336889in, left, base]{\color{textcolor}\sffamily\fontsize{20.000000}{24.000000}\selectfont \(\displaystyle {1.2}\)}%
\end{pgfscope}%
\begin{pgfscope}%
\pgfpathrectangle{\pgfqpoint{1.000000in}{5.000000in}}{\pgfqpoint{8.500000in}{2.000000in}}%
\pgfusepath{clip}%
\pgfsetrectcap%
\pgfsetroundjoin%
\pgfsetlinewidth{0.803000pt}%
\definecolor{currentstroke}{rgb}{0.690196,0.690196,0.690196}%
\pgfsetstrokecolor{currentstroke}%
\pgfsetdash{}{0pt}%
\pgfpathmoveto{\pgfqpoint{1.000000in}{6.676393in}}%
\pgfpathlineto{\pgfqpoint{9.500000in}{6.676393in}}%
\pgfusepath{stroke}%
\end{pgfscope}%
\begin{pgfscope}%
\pgfsetbuttcap%
\pgfsetroundjoin%
\definecolor{currentfill}{rgb}{0.000000,0.000000,0.000000}%
\pgfsetfillcolor{currentfill}%
\pgfsetlinewidth{0.803000pt}%
\definecolor{currentstroke}{rgb}{0.000000,0.000000,0.000000}%
\pgfsetstrokecolor{currentstroke}%
\pgfsetdash{}{0pt}%
\pgfsys@defobject{currentmarker}{\pgfqpoint{-0.048611in}{0.000000in}}{\pgfqpoint{-0.000000in}{0.000000in}}{%
\pgfpathmoveto{\pgfqpoint{-0.000000in}{0.000000in}}%
\pgfpathlineto{\pgfqpoint{-0.048611in}{0.000000in}}%
\pgfusepath{stroke,fill}%
}%
\begin{pgfscope}%
\pgfsys@transformshift{1.000000in}{6.676393in}%
\pgfsys@useobject{currentmarker}{}%
\end{pgfscope}%
\end{pgfscope}%
\begin{pgfscope}%
\definecolor{textcolor}{rgb}{0.000000,0.000000,0.000000}%
\pgfsetstrokecolor{textcolor}%
\pgfsetfillcolor{textcolor}%
\pgftext[x=0.560215in, y=6.576374in, left, base]{\color{textcolor}\sffamily\fontsize{20.000000}{24.000000}\selectfont \(\displaystyle {1.4}\)}%
\end{pgfscope}%
\begin{pgfscope}%
\definecolor{textcolor}{rgb}{0.000000,0.000000,0.000000}%
\pgfsetstrokecolor{textcolor}%
\pgfsetfillcolor{textcolor}%
\pgftext[x=0.504660in,y=6.000000in,,bottom,rotate=90.000000]{\color{textcolor}\sffamily\fontsize{20.000000}{24.000000}\selectfont \(\displaystyle \mathrm{Charge}\)}%
\end{pgfscope}%
\begin{pgfscope}%
\pgfpathrectangle{\pgfqpoint{1.000000in}{5.000000in}}{\pgfqpoint{8.500000in}{2.000000in}}%
\pgfusepath{clip}%
\pgfsetbuttcap%
\pgfsetroundjoin%
\pgfsetlinewidth{2.007500pt}%
\definecolor{currentstroke}{rgb}{0.000000,0.000000,0.000000}%
\pgfsetstrokecolor{currentstroke}%
\pgfsetdash{}{0pt}%
\pgfpathmoveto{\pgfqpoint{2.643863in}{5.000000in}}%
\pgfpathlineto{\pgfqpoint{2.643863in}{5.722630in}}%
\pgfusepath{stroke}%
\end{pgfscope}%
\begin{pgfscope}%
\pgfpathrectangle{\pgfqpoint{1.000000in}{5.000000in}}{\pgfqpoint{8.500000in}{2.000000in}}%
\pgfusepath{clip}%
\pgfsetbuttcap%
\pgfsetroundjoin%
\pgfsetlinewidth{2.007500pt}%
\definecolor{currentstroke}{rgb}{0.000000,0.000000,0.000000}%
\pgfsetstrokecolor{currentstroke}%
\pgfsetdash{}{0pt}%
\pgfpathmoveto{\pgfqpoint{2.718512in}{5.000000in}}%
\pgfpathlineto{\pgfqpoint{2.718512in}{6.502106in}}%
\pgfusepath{stroke}%
\end{pgfscope}%
\begin{pgfscope}%
\pgfpathrectangle{\pgfqpoint{1.000000in}{5.000000in}}{\pgfqpoint{8.500000in}{2.000000in}}%
\pgfusepath{clip}%
\pgfsetbuttcap%
\pgfsetroundjoin%
\pgfsetlinewidth{2.007500pt}%
\definecolor{currentstroke}{rgb}{0.000000,0.000000,0.000000}%
\pgfsetstrokecolor{currentstroke}%
\pgfsetdash{}{0pt}%
\pgfpathmoveto{\pgfqpoint{3.243482in}{5.000000in}}%
\pgfpathlineto{\pgfqpoint{3.243482in}{6.646606in}}%
\pgfusepath{stroke}%
\end{pgfscope}%
\begin{pgfscope}%
\pgfpathrectangle{\pgfqpoint{1.000000in}{5.000000in}}{\pgfqpoint{8.500000in}{2.000000in}}%
\pgfusepath{clip}%
\pgfsetbuttcap%
\pgfsetroundjoin%
\pgfsetlinewidth{2.007500pt}%
\definecolor{currentstroke}{rgb}{0.000000,0.000000,0.000000}%
\pgfsetstrokecolor{currentstroke}%
\pgfsetdash{}{0pt}%
\pgfpathmoveto{\pgfqpoint{3.641356in}{5.000000in}}%
\pgfpathlineto{\pgfqpoint{3.641356in}{6.237118in}}%
\pgfusepath{stroke}%
\end{pgfscope}%
\begin{pgfscope}%
\pgfpathrectangle{\pgfqpoint{1.000000in}{5.000000in}}{\pgfqpoint{8.500000in}{2.000000in}}%
\pgfusepath{clip}%
\pgfsetbuttcap%
\pgfsetroundjoin%
\pgfsetlinewidth{2.007500pt}%
\definecolor{currentstroke}{rgb}{0.000000,0.000000,0.000000}%
\pgfsetstrokecolor{currentstroke}%
\pgfsetdash{}{0pt}%
\pgfpathmoveto{\pgfqpoint{4.568410in}{5.000000in}}%
\pgfpathlineto{\pgfqpoint{4.568410in}{6.411076in}}%
\pgfusepath{stroke}%
\end{pgfscope}%
\begin{pgfscope}%
\pgfsetrectcap%
\pgfsetmiterjoin%
\pgfsetlinewidth{0.803000pt}%
\definecolor{currentstroke}{rgb}{0.000000,0.000000,0.000000}%
\pgfsetstrokecolor{currentstroke}%
\pgfsetdash{}{0pt}%
\pgfpathmoveto{\pgfqpoint{1.000000in}{5.000000in}}%
\pgfpathlineto{\pgfqpoint{1.000000in}{7.000000in}}%
\pgfusepath{stroke}%
\end{pgfscope}%
\begin{pgfscope}%
\pgfsetrectcap%
\pgfsetmiterjoin%
\pgfsetlinewidth{0.803000pt}%
\definecolor{currentstroke}{rgb}{0.000000,0.000000,0.000000}%
\pgfsetstrokecolor{currentstroke}%
\pgfsetdash{}{0pt}%
\pgfpathmoveto{\pgfqpoint{9.500000in}{5.000000in}}%
\pgfpathlineto{\pgfqpoint{9.500000in}{7.000000in}}%
\pgfusepath{stroke}%
\end{pgfscope}%
\begin{pgfscope}%
\pgfsetrectcap%
\pgfsetmiterjoin%
\pgfsetlinewidth{0.803000pt}%
\definecolor{currentstroke}{rgb}{0.000000,0.000000,0.000000}%
\pgfsetstrokecolor{currentstroke}%
\pgfsetdash{}{0pt}%
\pgfpathmoveto{\pgfqpoint{1.000000in}{5.000000in}}%
\pgfpathlineto{\pgfqpoint{9.500000in}{5.000000in}}%
\pgfusepath{stroke}%
\end{pgfscope}%
\begin{pgfscope}%
\pgfsetrectcap%
\pgfsetmiterjoin%
\pgfsetlinewidth{0.803000pt}%
\definecolor{currentstroke}{rgb}{0.000000,0.000000,0.000000}%
\pgfsetstrokecolor{currentstroke}%
\pgfsetdash{}{0pt}%
\pgfpathmoveto{\pgfqpoint{1.000000in}{7.000000in}}%
\pgfpathlineto{\pgfqpoint{9.500000in}{7.000000in}}%
\pgfusepath{stroke}%
\end{pgfscope}%
\begin{pgfscope}%
\pgfsetbuttcap%
\pgfsetmiterjoin%
\definecolor{currentfill}{rgb}{1.000000,1.000000,1.000000}%
\pgfsetfillcolor{currentfill}%
\pgfsetfillopacity{0.800000}%
\pgfsetlinewidth{1.003750pt}%
\definecolor{currentstroke}{rgb}{0.800000,0.800000,0.800000}%
\pgfsetstrokecolor{currentstroke}%
\pgfsetstrokeopacity{0.800000}%
\pgfsetdash{}{0pt}%
\pgfpathmoveto{\pgfqpoint{7.146293in}{6.382821in}}%
\pgfpathlineto{\pgfqpoint{9.305556in}{6.382821in}}%
\pgfpathquadraticcurveto{\pgfqpoint{9.361111in}{6.382821in}}{\pgfqpoint{9.361111in}{6.438377in}}%
\pgfpathlineto{\pgfqpoint{9.361111in}{6.805556in}}%
\pgfpathquadraticcurveto{\pgfqpoint{9.361111in}{6.861111in}}{\pgfqpoint{9.305556in}{6.861111in}}%
\pgfpathlineto{\pgfqpoint{7.146293in}{6.861111in}}%
\pgfpathquadraticcurveto{\pgfqpoint{7.090737in}{6.861111in}}{\pgfqpoint{7.090737in}{6.805556in}}%
\pgfpathlineto{\pgfqpoint{7.090737in}{6.438377in}}%
\pgfpathquadraticcurveto{\pgfqpoint{7.090737in}{6.382821in}}{\pgfqpoint{7.146293in}{6.382821in}}%
\pgfpathlineto{\pgfqpoint{7.146293in}{6.382821in}}%
\pgfpathclose%
\pgfusepath{stroke,fill}%
\end{pgfscope}%
\begin{pgfscope}%
\pgfsetbuttcap%
\pgfsetroundjoin%
\pgfsetlinewidth{2.007500pt}%
\definecolor{currentstroke}{rgb}{0.000000,0.000000,0.000000}%
\pgfsetstrokecolor{currentstroke}%
\pgfsetdash{}{0pt}%
\pgfpathmoveto{\pgfqpoint{7.201848in}{6.647184in}}%
\pgfpathlineto{\pgfqpoint{7.757404in}{6.647184in}}%
\pgfusepath{stroke}%
\end{pgfscope}%
\begin{pgfscope}%
\definecolor{textcolor}{rgb}{0.000000,0.000000,0.000000}%
\pgfsetstrokecolor{textcolor}%
\pgfsetfillcolor{textcolor}%
\pgftext[x=7.979626in,y=6.549962in,left,base]{\color{textcolor}\sffamily\fontsize{20.000000}{24.000000}\selectfont true charge}%
\end{pgfscope}%
\begin{pgfscope}%
\pgfsetbuttcap%
\pgfsetmiterjoin%
\definecolor{currentfill}{rgb}{1.000000,1.000000,1.000000}%
\pgfsetfillcolor{currentfill}%
\pgfsetlinewidth{0.000000pt}%
\definecolor{currentstroke}{rgb}{0.000000,0.000000,0.000000}%
\pgfsetstrokecolor{currentstroke}%
\pgfsetstrokeopacity{0.000000}%
\pgfsetdash{}{0pt}%
\pgfpathmoveto{\pgfqpoint{1.000000in}{7.000000in}}%
\pgfpathlineto{\pgfqpoint{9.500000in}{7.000000in}}%
\pgfpathlineto{\pgfqpoint{9.500000in}{9.000000in}}%
\pgfpathlineto{\pgfqpoint{1.000000in}{9.000000in}}%
\pgfpathlineto{\pgfqpoint{1.000000in}{7.000000in}}%
\pgfpathclose%
\pgfusepath{fill}%
\end{pgfscope}%
\begin{pgfscope}%
\pgfpathrectangle{\pgfqpoint{1.000000in}{7.000000in}}{\pgfqpoint{8.500000in}{2.000000in}}%
\pgfusepath{clip}%
\pgfsetrectcap%
\pgfsetroundjoin%
\pgfsetlinewidth{0.803000pt}%
\definecolor{currentstroke}{rgb}{0.690196,0.690196,0.690196}%
\pgfsetstrokecolor{currentstroke}%
\pgfsetdash{}{0pt}%
\pgfpathmoveto{\pgfqpoint{2.014084in}{7.000000in}}%
\pgfpathlineto{\pgfqpoint{2.014084in}{9.000000in}}%
\pgfusepath{stroke}%
\end{pgfscope}%
\begin{pgfscope}%
\pgfsetbuttcap%
\pgfsetroundjoin%
\definecolor{currentfill}{rgb}{0.000000,0.000000,0.000000}%
\pgfsetfillcolor{currentfill}%
\pgfsetlinewidth{0.803000pt}%
\definecolor{currentstroke}{rgb}{0.000000,0.000000,0.000000}%
\pgfsetstrokecolor{currentstroke}%
\pgfsetdash{}{0pt}%
\pgfsys@defobject{currentmarker}{\pgfqpoint{0.000000in}{-0.048611in}}{\pgfqpoint{0.000000in}{0.000000in}}{%
\pgfpathmoveto{\pgfqpoint{0.000000in}{0.000000in}}%
\pgfpathlineto{\pgfqpoint{0.000000in}{-0.048611in}}%
\pgfusepath{stroke,fill}%
}%
\begin{pgfscope}%
\pgfsys@transformshift{2.014084in}{7.000000in}%
\pgfsys@useobject{currentmarker}{}%
\end{pgfscope}%
\end{pgfscope}%
\begin{pgfscope}%
\pgfpathrectangle{\pgfqpoint{1.000000in}{7.000000in}}{\pgfqpoint{8.500000in}{2.000000in}}%
\pgfusepath{clip}%
\pgfsetrectcap%
\pgfsetroundjoin%
\pgfsetlinewidth{0.803000pt}%
\definecolor{currentstroke}{rgb}{0.690196,0.690196,0.690196}%
\pgfsetstrokecolor{currentstroke}%
\pgfsetdash{}{0pt}%
\pgfpathmoveto{\pgfqpoint{3.657947in}{7.000000in}}%
\pgfpathlineto{\pgfqpoint{3.657947in}{9.000000in}}%
\pgfusepath{stroke}%
\end{pgfscope}%
\begin{pgfscope}%
\pgfsetbuttcap%
\pgfsetroundjoin%
\definecolor{currentfill}{rgb}{0.000000,0.000000,0.000000}%
\pgfsetfillcolor{currentfill}%
\pgfsetlinewidth{0.803000pt}%
\definecolor{currentstroke}{rgb}{0.000000,0.000000,0.000000}%
\pgfsetstrokecolor{currentstroke}%
\pgfsetdash{}{0pt}%
\pgfsys@defobject{currentmarker}{\pgfqpoint{0.000000in}{-0.048611in}}{\pgfqpoint{0.000000in}{0.000000in}}{%
\pgfpathmoveto{\pgfqpoint{0.000000in}{0.000000in}}%
\pgfpathlineto{\pgfqpoint{0.000000in}{-0.048611in}}%
\pgfusepath{stroke,fill}%
}%
\begin{pgfscope}%
\pgfsys@transformshift{3.657947in}{7.000000in}%
\pgfsys@useobject{currentmarker}{}%
\end{pgfscope}%
\end{pgfscope}%
\begin{pgfscope}%
\pgfpathrectangle{\pgfqpoint{1.000000in}{7.000000in}}{\pgfqpoint{8.500000in}{2.000000in}}%
\pgfusepath{clip}%
\pgfsetrectcap%
\pgfsetroundjoin%
\pgfsetlinewidth{0.803000pt}%
\definecolor{currentstroke}{rgb}{0.690196,0.690196,0.690196}%
\pgfsetstrokecolor{currentstroke}%
\pgfsetdash{}{0pt}%
\pgfpathmoveto{\pgfqpoint{5.301810in}{7.000000in}}%
\pgfpathlineto{\pgfqpoint{5.301810in}{9.000000in}}%
\pgfusepath{stroke}%
\end{pgfscope}%
\begin{pgfscope}%
\pgfsetbuttcap%
\pgfsetroundjoin%
\definecolor{currentfill}{rgb}{0.000000,0.000000,0.000000}%
\pgfsetfillcolor{currentfill}%
\pgfsetlinewidth{0.803000pt}%
\definecolor{currentstroke}{rgb}{0.000000,0.000000,0.000000}%
\pgfsetstrokecolor{currentstroke}%
\pgfsetdash{}{0pt}%
\pgfsys@defobject{currentmarker}{\pgfqpoint{0.000000in}{-0.048611in}}{\pgfqpoint{0.000000in}{0.000000in}}{%
\pgfpathmoveto{\pgfqpoint{0.000000in}{0.000000in}}%
\pgfpathlineto{\pgfqpoint{0.000000in}{-0.048611in}}%
\pgfusepath{stroke,fill}%
}%
\begin{pgfscope}%
\pgfsys@transformshift{5.301810in}{7.000000in}%
\pgfsys@useobject{currentmarker}{}%
\end{pgfscope}%
\end{pgfscope}%
\begin{pgfscope}%
\pgfpathrectangle{\pgfqpoint{1.000000in}{7.000000in}}{\pgfqpoint{8.500000in}{2.000000in}}%
\pgfusepath{clip}%
\pgfsetrectcap%
\pgfsetroundjoin%
\pgfsetlinewidth{0.803000pt}%
\definecolor{currentstroke}{rgb}{0.690196,0.690196,0.690196}%
\pgfsetstrokecolor{currentstroke}%
\pgfsetdash{}{0pt}%
\pgfpathmoveto{\pgfqpoint{6.945673in}{7.000000in}}%
\pgfpathlineto{\pgfqpoint{6.945673in}{9.000000in}}%
\pgfusepath{stroke}%
\end{pgfscope}%
\begin{pgfscope}%
\pgfsetbuttcap%
\pgfsetroundjoin%
\definecolor{currentfill}{rgb}{0.000000,0.000000,0.000000}%
\pgfsetfillcolor{currentfill}%
\pgfsetlinewidth{0.803000pt}%
\definecolor{currentstroke}{rgb}{0.000000,0.000000,0.000000}%
\pgfsetstrokecolor{currentstroke}%
\pgfsetdash{}{0pt}%
\pgfsys@defobject{currentmarker}{\pgfqpoint{0.000000in}{-0.048611in}}{\pgfqpoint{0.000000in}{0.000000in}}{%
\pgfpathmoveto{\pgfqpoint{0.000000in}{0.000000in}}%
\pgfpathlineto{\pgfqpoint{0.000000in}{-0.048611in}}%
\pgfusepath{stroke,fill}%
}%
\begin{pgfscope}%
\pgfsys@transformshift{6.945673in}{7.000000in}%
\pgfsys@useobject{currentmarker}{}%
\end{pgfscope}%
\end{pgfscope}%
\begin{pgfscope}%
\pgfpathrectangle{\pgfqpoint{1.000000in}{7.000000in}}{\pgfqpoint{8.500000in}{2.000000in}}%
\pgfusepath{clip}%
\pgfsetrectcap%
\pgfsetroundjoin%
\pgfsetlinewidth{0.803000pt}%
\definecolor{currentstroke}{rgb}{0.690196,0.690196,0.690196}%
\pgfsetstrokecolor{currentstroke}%
\pgfsetdash{}{0pt}%
\pgfpathmoveto{\pgfqpoint{8.589536in}{7.000000in}}%
\pgfpathlineto{\pgfqpoint{8.589536in}{9.000000in}}%
\pgfusepath{stroke}%
\end{pgfscope}%
\begin{pgfscope}%
\pgfsetbuttcap%
\pgfsetroundjoin%
\definecolor{currentfill}{rgb}{0.000000,0.000000,0.000000}%
\pgfsetfillcolor{currentfill}%
\pgfsetlinewidth{0.803000pt}%
\definecolor{currentstroke}{rgb}{0.000000,0.000000,0.000000}%
\pgfsetstrokecolor{currentstroke}%
\pgfsetdash{}{0pt}%
\pgfsys@defobject{currentmarker}{\pgfqpoint{0.000000in}{-0.048611in}}{\pgfqpoint{0.000000in}{0.000000in}}{%
\pgfpathmoveto{\pgfqpoint{0.000000in}{0.000000in}}%
\pgfpathlineto{\pgfqpoint{0.000000in}{-0.048611in}}%
\pgfusepath{stroke,fill}%
}%
\begin{pgfscope}%
\pgfsys@transformshift{8.589536in}{7.000000in}%
\pgfsys@useobject{currentmarker}{}%
\end{pgfscope}%
\end{pgfscope}%
\begin{pgfscope}%
\pgfpathrectangle{\pgfqpoint{1.000000in}{7.000000in}}{\pgfqpoint{8.500000in}{2.000000in}}%
\pgfusepath{clip}%
\pgfsetrectcap%
\pgfsetroundjoin%
\pgfsetlinewidth{0.803000pt}%
\definecolor{currentstroke}{rgb}{0.690196,0.690196,0.690196}%
\pgfsetstrokecolor{currentstroke}%
\pgfsetdash{}{0pt}%
\pgfpathmoveto{\pgfqpoint{1.000000in}{7.000000in}}%
\pgfpathlineto{\pgfqpoint{9.500000in}{7.000000in}}%
\pgfusepath{stroke}%
\end{pgfscope}%
\begin{pgfscope}%
\pgfsetbuttcap%
\pgfsetroundjoin%
\definecolor{currentfill}{rgb}{0.000000,0.000000,0.000000}%
\pgfsetfillcolor{currentfill}%
\pgfsetlinewidth{0.803000pt}%
\definecolor{currentstroke}{rgb}{0.000000,0.000000,0.000000}%
\pgfsetstrokecolor{currentstroke}%
\pgfsetdash{}{0pt}%
\pgfsys@defobject{currentmarker}{\pgfqpoint{-0.048611in}{0.000000in}}{\pgfqpoint{-0.000000in}{0.000000in}}{%
\pgfpathmoveto{\pgfqpoint{-0.000000in}{0.000000in}}%
\pgfpathlineto{\pgfqpoint{-0.048611in}{0.000000in}}%
\pgfusepath{stroke,fill}%
}%
\begin{pgfscope}%
\pgfsys@transformshift{1.000000in}{7.000000in}%
\pgfsys@useobject{currentmarker}{}%
\end{pgfscope}%
\end{pgfscope}%
\begin{pgfscope}%
\definecolor{textcolor}{rgb}{0.000000,0.000000,0.000000}%
\pgfsetstrokecolor{textcolor}%
\pgfsetfillcolor{textcolor}%
\pgftext[x=0.560215in, y=6.899981in, left, base]{\color{textcolor}\sffamily\fontsize{20.000000}{24.000000}\selectfont \(\displaystyle {0.0}\)}%
\end{pgfscope}%
\begin{pgfscope}%
\pgfpathrectangle{\pgfqpoint{1.000000in}{7.000000in}}{\pgfqpoint{8.500000in}{2.000000in}}%
\pgfusepath{clip}%
\pgfsetrectcap%
\pgfsetroundjoin%
\pgfsetlinewidth{0.803000pt}%
\definecolor{currentstroke}{rgb}{0.690196,0.690196,0.690196}%
\pgfsetstrokecolor{currentstroke}%
\pgfsetdash{}{0pt}%
\pgfpathmoveto{\pgfqpoint{1.000000in}{7.239485in}}%
\pgfpathlineto{\pgfqpoint{9.500000in}{7.239485in}}%
\pgfusepath{stroke}%
\end{pgfscope}%
\begin{pgfscope}%
\pgfsetbuttcap%
\pgfsetroundjoin%
\definecolor{currentfill}{rgb}{0.000000,0.000000,0.000000}%
\pgfsetfillcolor{currentfill}%
\pgfsetlinewidth{0.803000pt}%
\definecolor{currentstroke}{rgb}{0.000000,0.000000,0.000000}%
\pgfsetstrokecolor{currentstroke}%
\pgfsetdash{}{0pt}%
\pgfsys@defobject{currentmarker}{\pgfqpoint{-0.048611in}{0.000000in}}{\pgfqpoint{-0.000000in}{0.000000in}}{%
\pgfpathmoveto{\pgfqpoint{-0.000000in}{0.000000in}}%
\pgfpathlineto{\pgfqpoint{-0.048611in}{0.000000in}}%
\pgfusepath{stroke,fill}%
}%
\begin{pgfscope}%
\pgfsys@transformshift{1.000000in}{7.239485in}%
\pgfsys@useobject{currentmarker}{}%
\end{pgfscope}%
\end{pgfscope}%
\begin{pgfscope}%
\definecolor{textcolor}{rgb}{0.000000,0.000000,0.000000}%
\pgfsetstrokecolor{textcolor}%
\pgfsetfillcolor{textcolor}%
\pgftext[x=0.560215in, y=7.139465in, left, base]{\color{textcolor}\sffamily\fontsize{20.000000}{24.000000}\selectfont \(\displaystyle {0.2}\)}%
\end{pgfscope}%
\begin{pgfscope}%
\pgfpathrectangle{\pgfqpoint{1.000000in}{7.000000in}}{\pgfqpoint{8.500000in}{2.000000in}}%
\pgfusepath{clip}%
\pgfsetrectcap%
\pgfsetroundjoin%
\pgfsetlinewidth{0.803000pt}%
\definecolor{currentstroke}{rgb}{0.690196,0.690196,0.690196}%
\pgfsetstrokecolor{currentstroke}%
\pgfsetdash{}{0pt}%
\pgfpathmoveto{\pgfqpoint{1.000000in}{7.478969in}}%
\pgfpathlineto{\pgfqpoint{9.500000in}{7.478969in}}%
\pgfusepath{stroke}%
\end{pgfscope}%
\begin{pgfscope}%
\pgfsetbuttcap%
\pgfsetroundjoin%
\definecolor{currentfill}{rgb}{0.000000,0.000000,0.000000}%
\pgfsetfillcolor{currentfill}%
\pgfsetlinewidth{0.803000pt}%
\definecolor{currentstroke}{rgb}{0.000000,0.000000,0.000000}%
\pgfsetstrokecolor{currentstroke}%
\pgfsetdash{}{0pt}%
\pgfsys@defobject{currentmarker}{\pgfqpoint{-0.048611in}{0.000000in}}{\pgfqpoint{-0.000000in}{0.000000in}}{%
\pgfpathmoveto{\pgfqpoint{-0.000000in}{0.000000in}}%
\pgfpathlineto{\pgfqpoint{-0.048611in}{0.000000in}}%
\pgfusepath{stroke,fill}%
}%
\begin{pgfscope}%
\pgfsys@transformshift{1.000000in}{7.478969in}%
\pgfsys@useobject{currentmarker}{}%
\end{pgfscope}%
\end{pgfscope}%
\begin{pgfscope}%
\definecolor{textcolor}{rgb}{0.000000,0.000000,0.000000}%
\pgfsetstrokecolor{textcolor}%
\pgfsetfillcolor{textcolor}%
\pgftext[x=0.560215in, y=7.378950in, left, base]{\color{textcolor}\sffamily\fontsize{20.000000}{24.000000}\selectfont \(\displaystyle {0.4}\)}%
\end{pgfscope}%
\begin{pgfscope}%
\pgfpathrectangle{\pgfqpoint{1.000000in}{7.000000in}}{\pgfqpoint{8.500000in}{2.000000in}}%
\pgfusepath{clip}%
\pgfsetrectcap%
\pgfsetroundjoin%
\pgfsetlinewidth{0.803000pt}%
\definecolor{currentstroke}{rgb}{0.690196,0.690196,0.690196}%
\pgfsetstrokecolor{currentstroke}%
\pgfsetdash{}{0pt}%
\pgfpathmoveto{\pgfqpoint{1.000000in}{7.718454in}}%
\pgfpathlineto{\pgfqpoint{9.500000in}{7.718454in}}%
\pgfusepath{stroke}%
\end{pgfscope}%
\begin{pgfscope}%
\pgfsetbuttcap%
\pgfsetroundjoin%
\definecolor{currentfill}{rgb}{0.000000,0.000000,0.000000}%
\pgfsetfillcolor{currentfill}%
\pgfsetlinewidth{0.803000pt}%
\definecolor{currentstroke}{rgb}{0.000000,0.000000,0.000000}%
\pgfsetstrokecolor{currentstroke}%
\pgfsetdash{}{0pt}%
\pgfsys@defobject{currentmarker}{\pgfqpoint{-0.048611in}{0.000000in}}{\pgfqpoint{-0.000000in}{0.000000in}}{%
\pgfpathmoveto{\pgfqpoint{-0.000000in}{0.000000in}}%
\pgfpathlineto{\pgfqpoint{-0.048611in}{0.000000in}}%
\pgfusepath{stroke,fill}%
}%
\begin{pgfscope}%
\pgfsys@transformshift{1.000000in}{7.718454in}%
\pgfsys@useobject{currentmarker}{}%
\end{pgfscope}%
\end{pgfscope}%
\begin{pgfscope}%
\definecolor{textcolor}{rgb}{0.000000,0.000000,0.000000}%
\pgfsetstrokecolor{textcolor}%
\pgfsetfillcolor{textcolor}%
\pgftext[x=0.560215in, y=7.618435in, left, base]{\color{textcolor}\sffamily\fontsize{20.000000}{24.000000}\selectfont \(\displaystyle {0.6}\)}%
\end{pgfscope}%
\begin{pgfscope}%
\pgfpathrectangle{\pgfqpoint{1.000000in}{7.000000in}}{\pgfqpoint{8.500000in}{2.000000in}}%
\pgfusepath{clip}%
\pgfsetrectcap%
\pgfsetroundjoin%
\pgfsetlinewidth{0.803000pt}%
\definecolor{currentstroke}{rgb}{0.690196,0.690196,0.690196}%
\pgfsetstrokecolor{currentstroke}%
\pgfsetdash{}{0pt}%
\pgfpathmoveto{\pgfqpoint{1.000000in}{7.957939in}}%
\pgfpathlineto{\pgfqpoint{9.500000in}{7.957939in}}%
\pgfusepath{stroke}%
\end{pgfscope}%
\begin{pgfscope}%
\pgfsetbuttcap%
\pgfsetroundjoin%
\definecolor{currentfill}{rgb}{0.000000,0.000000,0.000000}%
\pgfsetfillcolor{currentfill}%
\pgfsetlinewidth{0.803000pt}%
\definecolor{currentstroke}{rgb}{0.000000,0.000000,0.000000}%
\pgfsetstrokecolor{currentstroke}%
\pgfsetdash{}{0pt}%
\pgfsys@defobject{currentmarker}{\pgfqpoint{-0.048611in}{0.000000in}}{\pgfqpoint{-0.000000in}{0.000000in}}{%
\pgfpathmoveto{\pgfqpoint{-0.000000in}{0.000000in}}%
\pgfpathlineto{\pgfqpoint{-0.048611in}{0.000000in}}%
\pgfusepath{stroke,fill}%
}%
\begin{pgfscope}%
\pgfsys@transformshift{1.000000in}{7.957939in}%
\pgfsys@useobject{currentmarker}{}%
\end{pgfscope}%
\end{pgfscope}%
\begin{pgfscope}%
\definecolor{textcolor}{rgb}{0.000000,0.000000,0.000000}%
\pgfsetstrokecolor{textcolor}%
\pgfsetfillcolor{textcolor}%
\pgftext[x=0.560215in, y=7.857920in, left, base]{\color{textcolor}\sffamily\fontsize{20.000000}{24.000000}\selectfont \(\displaystyle {0.8}\)}%
\end{pgfscope}%
\begin{pgfscope}%
\pgfpathrectangle{\pgfqpoint{1.000000in}{7.000000in}}{\pgfqpoint{8.500000in}{2.000000in}}%
\pgfusepath{clip}%
\pgfsetrectcap%
\pgfsetroundjoin%
\pgfsetlinewidth{0.803000pt}%
\definecolor{currentstroke}{rgb}{0.690196,0.690196,0.690196}%
\pgfsetstrokecolor{currentstroke}%
\pgfsetdash{}{0pt}%
\pgfpathmoveto{\pgfqpoint{1.000000in}{8.197424in}}%
\pgfpathlineto{\pgfqpoint{9.500000in}{8.197424in}}%
\pgfusepath{stroke}%
\end{pgfscope}%
\begin{pgfscope}%
\pgfsetbuttcap%
\pgfsetroundjoin%
\definecolor{currentfill}{rgb}{0.000000,0.000000,0.000000}%
\pgfsetfillcolor{currentfill}%
\pgfsetlinewidth{0.803000pt}%
\definecolor{currentstroke}{rgb}{0.000000,0.000000,0.000000}%
\pgfsetstrokecolor{currentstroke}%
\pgfsetdash{}{0pt}%
\pgfsys@defobject{currentmarker}{\pgfqpoint{-0.048611in}{0.000000in}}{\pgfqpoint{-0.000000in}{0.000000in}}{%
\pgfpathmoveto{\pgfqpoint{-0.000000in}{0.000000in}}%
\pgfpathlineto{\pgfqpoint{-0.048611in}{0.000000in}}%
\pgfusepath{stroke,fill}%
}%
\begin{pgfscope}%
\pgfsys@transformshift{1.000000in}{8.197424in}%
\pgfsys@useobject{currentmarker}{}%
\end{pgfscope}%
\end{pgfscope}%
\begin{pgfscope}%
\definecolor{textcolor}{rgb}{0.000000,0.000000,0.000000}%
\pgfsetstrokecolor{textcolor}%
\pgfsetfillcolor{textcolor}%
\pgftext[x=0.560215in, y=8.097404in, left, base]{\color{textcolor}\sffamily\fontsize{20.000000}{24.000000}\selectfont \(\displaystyle {1.0}\)}%
\end{pgfscope}%
\begin{pgfscope}%
\pgfpathrectangle{\pgfqpoint{1.000000in}{7.000000in}}{\pgfqpoint{8.500000in}{2.000000in}}%
\pgfusepath{clip}%
\pgfsetrectcap%
\pgfsetroundjoin%
\pgfsetlinewidth{0.803000pt}%
\definecolor{currentstroke}{rgb}{0.690196,0.690196,0.690196}%
\pgfsetstrokecolor{currentstroke}%
\pgfsetdash{}{0pt}%
\pgfpathmoveto{\pgfqpoint{1.000000in}{8.436908in}}%
\pgfpathlineto{\pgfqpoint{9.500000in}{8.436908in}}%
\pgfusepath{stroke}%
\end{pgfscope}%
\begin{pgfscope}%
\pgfsetbuttcap%
\pgfsetroundjoin%
\definecolor{currentfill}{rgb}{0.000000,0.000000,0.000000}%
\pgfsetfillcolor{currentfill}%
\pgfsetlinewidth{0.803000pt}%
\definecolor{currentstroke}{rgb}{0.000000,0.000000,0.000000}%
\pgfsetstrokecolor{currentstroke}%
\pgfsetdash{}{0pt}%
\pgfsys@defobject{currentmarker}{\pgfqpoint{-0.048611in}{0.000000in}}{\pgfqpoint{-0.000000in}{0.000000in}}{%
\pgfpathmoveto{\pgfqpoint{-0.000000in}{0.000000in}}%
\pgfpathlineto{\pgfqpoint{-0.048611in}{0.000000in}}%
\pgfusepath{stroke,fill}%
}%
\begin{pgfscope}%
\pgfsys@transformshift{1.000000in}{8.436908in}%
\pgfsys@useobject{currentmarker}{}%
\end{pgfscope}%
\end{pgfscope}%
\begin{pgfscope}%
\definecolor{textcolor}{rgb}{0.000000,0.000000,0.000000}%
\pgfsetstrokecolor{textcolor}%
\pgfsetfillcolor{textcolor}%
\pgftext[x=0.560215in, y=8.336889in, left, base]{\color{textcolor}\sffamily\fontsize{20.000000}{24.000000}\selectfont \(\displaystyle {1.2}\)}%
\end{pgfscope}%
\begin{pgfscope}%
\pgfpathrectangle{\pgfqpoint{1.000000in}{7.000000in}}{\pgfqpoint{8.500000in}{2.000000in}}%
\pgfusepath{clip}%
\pgfsetrectcap%
\pgfsetroundjoin%
\pgfsetlinewidth{0.803000pt}%
\definecolor{currentstroke}{rgb}{0.690196,0.690196,0.690196}%
\pgfsetstrokecolor{currentstroke}%
\pgfsetdash{}{0pt}%
\pgfpathmoveto{\pgfqpoint{1.000000in}{8.676393in}}%
\pgfpathlineto{\pgfqpoint{9.500000in}{8.676393in}}%
\pgfusepath{stroke}%
\end{pgfscope}%
\begin{pgfscope}%
\pgfsetbuttcap%
\pgfsetroundjoin%
\definecolor{currentfill}{rgb}{0.000000,0.000000,0.000000}%
\pgfsetfillcolor{currentfill}%
\pgfsetlinewidth{0.803000pt}%
\definecolor{currentstroke}{rgb}{0.000000,0.000000,0.000000}%
\pgfsetstrokecolor{currentstroke}%
\pgfsetdash{}{0pt}%
\pgfsys@defobject{currentmarker}{\pgfqpoint{-0.048611in}{0.000000in}}{\pgfqpoint{-0.000000in}{0.000000in}}{%
\pgfpathmoveto{\pgfqpoint{-0.000000in}{0.000000in}}%
\pgfpathlineto{\pgfqpoint{-0.048611in}{0.000000in}}%
\pgfusepath{stroke,fill}%
}%
\begin{pgfscope}%
\pgfsys@transformshift{1.000000in}{8.676393in}%
\pgfsys@useobject{currentmarker}{}%
\end{pgfscope}%
\end{pgfscope}%
\begin{pgfscope}%
\definecolor{textcolor}{rgb}{0.000000,0.000000,0.000000}%
\pgfsetstrokecolor{textcolor}%
\pgfsetfillcolor{textcolor}%
\pgftext[x=0.560215in, y=8.576374in, left, base]{\color{textcolor}\sffamily\fontsize{20.000000}{24.000000}\selectfont \(\displaystyle {1.4}\)}%
\end{pgfscope}%
\begin{pgfscope}%
\definecolor{textcolor}{rgb}{0.000000,0.000000,0.000000}%
\pgfsetstrokecolor{textcolor}%
\pgfsetfillcolor{textcolor}%
\pgftext[x=0.504660in,y=8.000000in,,bottom,rotate=90.000000]{\color{textcolor}\sffamily\fontsize{20.000000}{24.000000}\selectfont \(\displaystyle \mathrm{Charge}\)}%
\end{pgfscope}%
\begin{pgfscope}%
\pgfpathrectangle{\pgfqpoint{1.000000in}{7.000000in}}{\pgfqpoint{8.500000in}{2.000000in}}%
\pgfusepath{clip}%
\pgfsetbuttcap%
\pgfsetroundjoin%
\pgfsetlinewidth{2.007500pt}%
\definecolor{currentstroke}{rgb}{0.000000,0.500000,0.000000}%
\pgfsetstrokecolor{currentstroke}%
\pgfsetdash{}{0pt}%
\pgfpathmoveto{\pgfqpoint{2.638752in}{7.000000in}}%
\pgfpathlineto{\pgfqpoint{2.638752in}{7.401336in}}%
\pgfusepath{stroke}%
\end{pgfscope}%
\begin{pgfscope}%
\pgfpathrectangle{\pgfqpoint{1.000000in}{7.000000in}}{\pgfqpoint{8.500000in}{2.000000in}}%
\pgfusepath{clip}%
\pgfsetbuttcap%
\pgfsetroundjoin%
\pgfsetlinewidth{2.007500pt}%
\definecolor{currentstroke}{rgb}{0.000000,0.500000,0.000000}%
\pgfsetstrokecolor{currentstroke}%
\pgfsetdash{}{0pt}%
\pgfpathmoveto{\pgfqpoint{2.704506in}{7.000000in}}%
\pgfpathlineto{\pgfqpoint{2.704506in}{8.818182in}}%
\pgfusepath{stroke}%
\end{pgfscope}%
\begin{pgfscope}%
\pgfpathrectangle{\pgfqpoint{1.000000in}{7.000000in}}{\pgfqpoint{8.500000in}{2.000000in}}%
\pgfusepath{clip}%
\pgfsetbuttcap%
\pgfsetroundjoin%
\pgfsetlinewidth{2.007500pt}%
\definecolor{currentstroke}{rgb}{0.000000,0.500000,0.000000}%
\pgfsetstrokecolor{currentstroke}%
\pgfsetdash{}{0pt}%
\pgfpathmoveto{\pgfqpoint{3.230542in}{7.000000in}}%
\pgfpathlineto{\pgfqpoint{3.230542in}{8.627223in}}%
\pgfusepath{stroke}%
\end{pgfscope}%
\begin{pgfscope}%
\pgfpathrectangle{\pgfqpoint{1.000000in}{7.000000in}}{\pgfqpoint{8.500000in}{2.000000in}}%
\pgfusepath{clip}%
\pgfsetbuttcap%
\pgfsetroundjoin%
\pgfsetlinewidth{2.007500pt}%
\definecolor{currentstroke}{rgb}{0.000000,0.500000,0.000000}%
\pgfsetstrokecolor{currentstroke}%
\pgfsetdash{}{0pt}%
\pgfpathmoveto{\pgfqpoint{3.625069in}{7.000000in}}%
\pgfpathlineto{\pgfqpoint{3.625069in}{8.239466in}}%
\pgfusepath{stroke}%
\end{pgfscope}%
\begin{pgfscope}%
\pgfpathrectangle{\pgfqpoint{1.000000in}{7.000000in}}{\pgfqpoint{8.500000in}{2.000000in}}%
\pgfusepath{clip}%
\pgfsetbuttcap%
\pgfsetroundjoin%
\pgfsetlinewidth{2.007500pt}%
\definecolor{currentstroke}{rgb}{0.000000,0.500000,0.000000}%
\pgfsetstrokecolor{currentstroke}%
\pgfsetdash{}{0pt}%
\pgfpathmoveto{\pgfqpoint{4.545633in}{7.000000in}}%
\pgfpathlineto{\pgfqpoint{4.545633in}{8.375900in}}%
\pgfusepath{stroke}%
\end{pgfscope}%
\begin{pgfscope}%
\pgfsetrectcap%
\pgfsetmiterjoin%
\pgfsetlinewidth{0.803000pt}%
\definecolor{currentstroke}{rgb}{0.000000,0.000000,0.000000}%
\pgfsetstrokecolor{currentstroke}%
\pgfsetdash{}{0pt}%
\pgfpathmoveto{\pgfqpoint{1.000000in}{7.000000in}}%
\pgfpathlineto{\pgfqpoint{1.000000in}{9.000000in}}%
\pgfusepath{stroke}%
\end{pgfscope}%
\begin{pgfscope}%
\pgfsetrectcap%
\pgfsetmiterjoin%
\pgfsetlinewidth{0.803000pt}%
\definecolor{currentstroke}{rgb}{0.000000,0.000000,0.000000}%
\pgfsetstrokecolor{currentstroke}%
\pgfsetdash{}{0pt}%
\pgfpathmoveto{\pgfqpoint{9.500000in}{7.000000in}}%
\pgfpathlineto{\pgfqpoint{9.500000in}{9.000000in}}%
\pgfusepath{stroke}%
\end{pgfscope}%
\begin{pgfscope}%
\pgfsetrectcap%
\pgfsetmiterjoin%
\pgfsetlinewidth{0.803000pt}%
\definecolor{currentstroke}{rgb}{0.000000,0.000000,0.000000}%
\pgfsetstrokecolor{currentstroke}%
\pgfsetdash{}{0pt}%
\pgfpathmoveto{\pgfqpoint{1.000000in}{7.000000in}}%
\pgfpathlineto{\pgfqpoint{9.500000in}{7.000000in}}%
\pgfusepath{stroke}%
\end{pgfscope}%
\begin{pgfscope}%
\pgfsetrectcap%
\pgfsetmiterjoin%
\pgfsetlinewidth{0.803000pt}%
\definecolor{currentstroke}{rgb}{0.000000,0.000000,0.000000}%
\pgfsetstrokecolor{currentstroke}%
\pgfsetdash{}{0pt}%
\pgfpathmoveto{\pgfqpoint{1.000000in}{9.000000in}}%
\pgfpathlineto{\pgfqpoint{9.500000in}{9.000000in}}%
\pgfusepath{stroke}%
\end{pgfscope}%
\begin{pgfscope}%
\pgfsetbuttcap%
\pgfsetmiterjoin%
\definecolor{currentfill}{rgb}{1.000000,1.000000,1.000000}%
\pgfsetfillcolor{currentfill}%
\pgfsetfillopacity{0.800000}%
\pgfsetlinewidth{1.003750pt}%
\definecolor{currentstroke}{rgb}{0.800000,0.800000,0.800000}%
\pgfsetstrokecolor{currentstroke}%
\pgfsetstrokeopacity{0.800000}%
\pgfsetdash{}{0pt}%
\pgfpathmoveto{\pgfqpoint{6.076997in}{8.382821in}}%
\pgfpathlineto{\pgfqpoint{9.305556in}{8.382821in}}%
\pgfpathquadraticcurveto{\pgfqpoint{9.361111in}{8.382821in}}{\pgfqpoint{9.361111in}{8.438377in}}%
\pgfpathlineto{\pgfqpoint{9.361111in}{8.805556in}}%
\pgfpathquadraticcurveto{\pgfqpoint{9.361111in}{8.861111in}}{\pgfqpoint{9.305556in}{8.861111in}}%
\pgfpathlineto{\pgfqpoint{6.076997in}{8.861111in}}%
\pgfpathquadraticcurveto{\pgfqpoint{6.021442in}{8.861111in}}{\pgfqpoint{6.021442in}{8.805556in}}%
\pgfpathlineto{\pgfqpoint{6.021442in}{8.438377in}}%
\pgfpathquadraticcurveto{\pgfqpoint{6.021442in}{8.382821in}}{\pgfqpoint{6.076997in}{8.382821in}}%
\pgfpathlineto{\pgfqpoint{6.076997in}{8.382821in}}%
\pgfpathclose%
\pgfusepath{stroke,fill}%
\end{pgfscope}%
\begin{pgfscope}%
\pgfsetbuttcap%
\pgfsetroundjoin%
\pgfsetlinewidth{2.007500pt}%
\definecolor{currentstroke}{rgb}{0.000000,0.500000,0.000000}%
\pgfsetstrokecolor{currentstroke}%
\pgfsetdash{}{0pt}%
\pgfpathmoveto{\pgfqpoint{6.132553in}{8.647184in}}%
\pgfpathlineto{\pgfqpoint{6.688108in}{8.647184in}}%
\pgfusepath{stroke}%
\end{pgfscope}%
\begin{pgfscope}%
\definecolor{textcolor}{rgb}{0.000000,0.000000,0.000000}%
\pgfsetstrokecolor{textcolor}%
\pgfsetfillcolor{textcolor}%
\pgftext[x=6.910331in,y=8.549962in,left,base]{\color{textcolor}\sffamily\fontsize{20.000000}{24.000000}\selectfont reconstructed charge}%
\end{pgfscope}%
\begin{pgfscope}%
\pgfsetbuttcap%
\pgfsetmiterjoin%
\definecolor{currentfill}{rgb}{1.000000,1.000000,1.000000}%
\pgfsetfillcolor{currentfill}%
\pgfsetlinewidth{0.000000pt}%
\definecolor{currentstroke}{rgb}{0.000000,0.000000,0.000000}%
\pgfsetstrokecolor{currentstroke}%
\pgfsetstrokeopacity{0.000000}%
\pgfsetdash{}{0pt}%
\pgfpathmoveto{\pgfqpoint{1.000000in}{1.000000in}}%
\pgfpathlineto{\pgfqpoint{9.500000in}{1.000000in}}%
\pgfpathlineto{\pgfqpoint{9.500000in}{2.000000in}}%
\pgfpathlineto{\pgfqpoint{1.000000in}{2.000000in}}%
\pgfpathlineto{\pgfqpoint{1.000000in}{1.000000in}}%
\pgfpathclose%
\pgfusepath{fill}%
\end{pgfscope}%
\begin{pgfscope}%
\pgfpathrectangle{\pgfqpoint{1.000000in}{1.000000in}}{\pgfqpoint{8.500000in}{1.000000in}}%
\pgfusepath{clip}%
\pgfsetbuttcap%
\pgfsetroundjoin%
\definecolor{currentfill}{rgb}{0.000000,0.000000,0.000000}%
\pgfsetfillcolor{currentfill}%
\pgfsetlinewidth{1.003750pt}%
\definecolor{currentstroke}{rgb}{0.000000,0.000000,0.000000}%
\pgfsetstrokecolor{currentstroke}%
\pgfsetdash{}{0pt}%
\pgfsys@defobject{currentmarker}{\pgfqpoint{-0.013889in}{-0.013889in}}{\pgfqpoint{0.013889in}{0.013889in}}{%
\pgfpathmoveto{\pgfqpoint{0.000000in}{-0.013889in}}%
\pgfpathcurveto{\pgfqpoint{0.003683in}{-0.013889in}}{\pgfqpoint{0.007216in}{-0.012425in}}{\pgfqpoint{0.009821in}{-0.009821in}}%
\pgfpathcurveto{\pgfqpoint{0.012425in}{-0.007216in}}{\pgfqpoint{0.013889in}{-0.003683in}}{\pgfqpoint{0.013889in}{0.000000in}}%
\pgfpathcurveto{\pgfqpoint{0.013889in}{0.003683in}}{\pgfqpoint{0.012425in}{0.007216in}}{\pgfqpoint{0.009821in}{0.009821in}}%
\pgfpathcurveto{\pgfqpoint{0.007216in}{0.012425in}}{\pgfqpoint{0.003683in}{0.013889in}}{\pgfqpoint{0.000000in}{0.013889in}}%
\pgfpathcurveto{\pgfqpoint{-0.003683in}{0.013889in}}{\pgfqpoint{-0.007216in}{0.012425in}}{\pgfqpoint{-0.009821in}{0.009821in}}%
\pgfpathcurveto{\pgfqpoint{-0.012425in}{0.007216in}}{\pgfqpoint{-0.013889in}{0.003683in}}{\pgfqpoint{-0.013889in}{0.000000in}}%
\pgfpathcurveto{\pgfqpoint{-0.013889in}{-0.003683in}}{\pgfqpoint{-0.012425in}{-0.007216in}}{\pgfqpoint{-0.009821in}{-0.009821in}}%
\pgfpathcurveto{\pgfqpoint{-0.007216in}{-0.012425in}}{\pgfqpoint{-0.003683in}{-0.013889in}}{\pgfqpoint{0.000000in}{-0.013889in}}%
\pgfpathlineto{\pgfqpoint{0.000000in}{-0.013889in}}%
\pgfpathclose%
\pgfusepath{stroke,fill}%
}%
\begin{pgfscope}%
\pgfsys@transformshift{-12.780685in}{1.609814in}%
\pgfsys@useobject{currentmarker}{}%
\end{pgfscope}%
\begin{pgfscope}%
\pgfsys@transformshift{-12.747808in}{1.696999in}%
\pgfsys@useobject{currentmarker}{}%
\end{pgfscope}%
\begin{pgfscope}%
\pgfsys@transformshift{-12.714931in}{1.512976in}%
\pgfsys@useobject{currentmarker}{}%
\end{pgfscope}%
\begin{pgfscope}%
\pgfsys@transformshift{-12.682053in}{1.590772in}%
\pgfsys@useobject{currentmarker}{}%
\end{pgfscope}%
\begin{pgfscope}%
\pgfsys@transformshift{-12.649176in}{1.378312in}%
\pgfsys@useobject{currentmarker}{}%
\end{pgfscope}%
\begin{pgfscope}%
\pgfsys@transformshift{-12.616299in}{1.408033in}%
\pgfsys@useobject{currentmarker}{}%
\end{pgfscope}%
\begin{pgfscope}%
\pgfsys@transformshift{-12.583422in}{1.451860in}%
\pgfsys@useobject{currentmarker}{}%
\end{pgfscope}%
\begin{pgfscope}%
\pgfsys@transformshift{-12.550544in}{1.401370in}%
\pgfsys@useobject{currentmarker}{}%
\end{pgfscope}%
\begin{pgfscope}%
\pgfsys@transformshift{-12.517667in}{1.579952in}%
\pgfsys@useobject{currentmarker}{}%
\end{pgfscope}%
\begin{pgfscope}%
\pgfsys@transformshift{-12.484790in}{1.569617in}%
\pgfsys@useobject{currentmarker}{}%
\end{pgfscope}%
\begin{pgfscope}%
\pgfsys@transformshift{-12.451913in}{1.381866in}%
\pgfsys@useobject{currentmarker}{}%
\end{pgfscope}%
\begin{pgfscope}%
\pgfsys@transformshift{-12.419035in}{1.533488in}%
\pgfsys@useobject{currentmarker}{}%
\end{pgfscope}%
\begin{pgfscope}%
\pgfsys@transformshift{-12.386158in}{1.492236in}%
\pgfsys@useobject{currentmarker}{}%
\end{pgfscope}%
\begin{pgfscope}%
\pgfsys@transformshift{-12.353281in}{1.315195in}%
\pgfsys@useobject{currentmarker}{}%
\end{pgfscope}%
\begin{pgfscope}%
\pgfsys@transformshift{-12.320404in}{1.304638in}%
\pgfsys@useobject{currentmarker}{}%
\end{pgfscope}%
\begin{pgfscope}%
\pgfsys@transformshift{-12.287526in}{1.451390in}%
\pgfsys@useobject{currentmarker}{}%
\end{pgfscope}%
\begin{pgfscope}%
\pgfsys@transformshift{-12.254649in}{1.371091in}%
\pgfsys@useobject{currentmarker}{}%
\end{pgfscope}%
\begin{pgfscope}%
\pgfsys@transformshift{-12.221772in}{1.534653in}%
\pgfsys@useobject{currentmarker}{}%
\end{pgfscope}%
\begin{pgfscope}%
\pgfsys@transformshift{-12.188894in}{1.476342in}%
\pgfsys@useobject{currentmarker}{}%
\end{pgfscope}%
\begin{pgfscope}%
\pgfsys@transformshift{-12.156017in}{1.519732in}%
\pgfsys@useobject{currentmarker}{}%
\end{pgfscope}%
\begin{pgfscope}%
\pgfsys@transformshift{-12.123140in}{1.470210in}%
\pgfsys@useobject{currentmarker}{}%
\end{pgfscope}%
\begin{pgfscope}%
\pgfsys@transformshift{-12.090263in}{1.391248in}%
\pgfsys@useobject{currentmarker}{}%
\end{pgfscope}%
\begin{pgfscope}%
\pgfsys@transformshift{-12.057385in}{1.605552in}%
\pgfsys@useobject{currentmarker}{}%
\end{pgfscope}%
\begin{pgfscope}%
\pgfsys@transformshift{-12.024508in}{1.477559in}%
\pgfsys@useobject{currentmarker}{}%
\end{pgfscope}%
\begin{pgfscope}%
\pgfsys@transformshift{-11.991631in}{1.547480in}%
\pgfsys@useobject{currentmarker}{}%
\end{pgfscope}%
\begin{pgfscope}%
\pgfsys@transformshift{-11.958754in}{1.479420in}%
\pgfsys@useobject{currentmarker}{}%
\end{pgfscope}%
\begin{pgfscope}%
\pgfsys@transformshift{-11.925876in}{1.438342in}%
\pgfsys@useobject{currentmarker}{}%
\end{pgfscope}%
\begin{pgfscope}%
\pgfsys@transformshift{-11.892999in}{1.503219in}%
\pgfsys@useobject{currentmarker}{}%
\end{pgfscope}%
\begin{pgfscope}%
\pgfsys@transformshift{-11.860122in}{1.568491in}%
\pgfsys@useobject{currentmarker}{}%
\end{pgfscope}%
\begin{pgfscope}%
\pgfsys@transformshift{-11.827245in}{1.672148in}%
\pgfsys@useobject{currentmarker}{}%
\end{pgfscope}%
\begin{pgfscope}%
\pgfsys@transformshift{-11.794367in}{1.331794in}%
\pgfsys@useobject{currentmarker}{}%
\end{pgfscope}%
\begin{pgfscope}%
\pgfsys@transformshift{-11.761490in}{1.473329in}%
\pgfsys@useobject{currentmarker}{}%
\end{pgfscope}%
\begin{pgfscope}%
\pgfsys@transformshift{-11.728613in}{1.405252in}%
\pgfsys@useobject{currentmarker}{}%
\end{pgfscope}%
\begin{pgfscope}%
\pgfsys@transformshift{-11.695735in}{1.649950in}%
\pgfsys@useobject{currentmarker}{}%
\end{pgfscope}%
\begin{pgfscope}%
\pgfsys@transformshift{-11.662858in}{1.463532in}%
\pgfsys@useobject{currentmarker}{}%
\end{pgfscope}%
\begin{pgfscope}%
\pgfsys@transformshift{-11.629981in}{1.689261in}%
\pgfsys@useobject{currentmarker}{}%
\end{pgfscope}%
\begin{pgfscope}%
\pgfsys@transformshift{-11.597104in}{1.708935in}%
\pgfsys@useobject{currentmarker}{}%
\end{pgfscope}%
\begin{pgfscope}%
\pgfsys@transformshift{-11.564226in}{1.459233in}%
\pgfsys@useobject{currentmarker}{}%
\end{pgfscope}%
\begin{pgfscope}%
\pgfsys@transformshift{-11.531349in}{1.399884in}%
\pgfsys@useobject{currentmarker}{}%
\end{pgfscope}%
\begin{pgfscope}%
\pgfsys@transformshift{-11.498472in}{1.583236in}%
\pgfsys@useobject{currentmarker}{}%
\end{pgfscope}%
\begin{pgfscope}%
\pgfsys@transformshift{-11.465595in}{1.454440in}%
\pgfsys@useobject{currentmarker}{}%
\end{pgfscope}%
\begin{pgfscope}%
\pgfsys@transformshift{-11.432717in}{1.366191in}%
\pgfsys@useobject{currentmarker}{}%
\end{pgfscope}%
\begin{pgfscope}%
\pgfsys@transformshift{-11.399840in}{1.492039in}%
\pgfsys@useobject{currentmarker}{}%
\end{pgfscope}%
\begin{pgfscope}%
\pgfsys@transformshift{-11.366963in}{1.383382in}%
\pgfsys@useobject{currentmarker}{}%
\end{pgfscope}%
\begin{pgfscope}%
\pgfsys@transformshift{-11.334086in}{1.498839in}%
\pgfsys@useobject{currentmarker}{}%
\end{pgfscope}%
\begin{pgfscope}%
\pgfsys@transformshift{-11.301208in}{1.560510in}%
\pgfsys@useobject{currentmarker}{}%
\end{pgfscope}%
\begin{pgfscope}%
\pgfsys@transformshift{-11.268331in}{1.441414in}%
\pgfsys@useobject{currentmarker}{}%
\end{pgfscope}%
\begin{pgfscope}%
\pgfsys@transformshift{-11.235454in}{1.649350in}%
\pgfsys@useobject{currentmarker}{}%
\end{pgfscope}%
\begin{pgfscope}%
\pgfsys@transformshift{-11.202577in}{1.430068in}%
\pgfsys@useobject{currentmarker}{}%
\end{pgfscope}%
\begin{pgfscope}%
\pgfsys@transformshift{-11.169699in}{1.564831in}%
\pgfsys@useobject{currentmarker}{}%
\end{pgfscope}%
\begin{pgfscope}%
\pgfsys@transformshift{-11.136822in}{1.686565in}%
\pgfsys@useobject{currentmarker}{}%
\end{pgfscope}%
\begin{pgfscope}%
\pgfsys@transformshift{-11.103945in}{1.430341in}%
\pgfsys@useobject{currentmarker}{}%
\end{pgfscope}%
\begin{pgfscope}%
\pgfsys@transformshift{-11.071067in}{1.546148in}%
\pgfsys@useobject{currentmarker}{}%
\end{pgfscope}%
\begin{pgfscope}%
\pgfsys@transformshift{-11.038190in}{1.447454in}%
\pgfsys@useobject{currentmarker}{}%
\end{pgfscope}%
\begin{pgfscope}%
\pgfsys@transformshift{-11.005313in}{1.566320in}%
\pgfsys@useobject{currentmarker}{}%
\end{pgfscope}%
\begin{pgfscope}%
\pgfsys@transformshift{-10.972436in}{1.392726in}%
\pgfsys@useobject{currentmarker}{}%
\end{pgfscope}%
\begin{pgfscope}%
\pgfsys@transformshift{-10.939558in}{1.498411in}%
\pgfsys@useobject{currentmarker}{}%
\end{pgfscope}%
\begin{pgfscope}%
\pgfsys@transformshift{-10.906681in}{1.580783in}%
\pgfsys@useobject{currentmarker}{}%
\end{pgfscope}%
\begin{pgfscope}%
\pgfsys@transformshift{-10.873804in}{1.267620in}%
\pgfsys@useobject{currentmarker}{}%
\end{pgfscope}%
\begin{pgfscope}%
\pgfsys@transformshift{-10.840927in}{1.524752in}%
\pgfsys@useobject{currentmarker}{}%
\end{pgfscope}%
\begin{pgfscope}%
\pgfsys@transformshift{-10.808049in}{1.379593in}%
\pgfsys@useobject{currentmarker}{}%
\end{pgfscope}%
\begin{pgfscope}%
\pgfsys@transformshift{-10.775172in}{1.323367in}%
\pgfsys@useobject{currentmarker}{}%
\end{pgfscope}%
\begin{pgfscope}%
\pgfsys@transformshift{-10.742295in}{1.531321in}%
\pgfsys@useobject{currentmarker}{}%
\end{pgfscope}%
\begin{pgfscope}%
\pgfsys@transformshift{-10.709418in}{1.524432in}%
\pgfsys@useobject{currentmarker}{}%
\end{pgfscope}%
\begin{pgfscope}%
\pgfsys@transformshift{-10.676540in}{1.612247in}%
\pgfsys@useobject{currentmarker}{}%
\end{pgfscope}%
\begin{pgfscope}%
\pgfsys@transformshift{-10.643663in}{1.311697in}%
\pgfsys@useobject{currentmarker}{}%
\end{pgfscope}%
\begin{pgfscope}%
\pgfsys@transformshift{-10.610786in}{1.422365in}%
\pgfsys@useobject{currentmarker}{}%
\end{pgfscope}%
\begin{pgfscope}%
\pgfsys@transformshift{-10.577909in}{1.476763in}%
\pgfsys@useobject{currentmarker}{}%
\end{pgfscope}%
\begin{pgfscope}%
\pgfsys@transformshift{-10.545031in}{1.627709in}%
\pgfsys@useobject{currentmarker}{}%
\end{pgfscope}%
\begin{pgfscope}%
\pgfsys@transformshift{-10.512154in}{1.447949in}%
\pgfsys@useobject{currentmarker}{}%
\end{pgfscope}%
\begin{pgfscope}%
\pgfsys@transformshift{-10.479277in}{1.322107in}%
\pgfsys@useobject{currentmarker}{}%
\end{pgfscope}%
\begin{pgfscope}%
\pgfsys@transformshift{-10.446399in}{1.464246in}%
\pgfsys@useobject{currentmarker}{}%
\end{pgfscope}%
\begin{pgfscope}%
\pgfsys@transformshift{-10.413522in}{1.587453in}%
\pgfsys@useobject{currentmarker}{}%
\end{pgfscope}%
\begin{pgfscope}%
\pgfsys@transformshift{-10.380645in}{1.693830in}%
\pgfsys@useobject{currentmarker}{}%
\end{pgfscope}%
\begin{pgfscope}%
\pgfsys@transformshift{-10.347768in}{1.535656in}%
\pgfsys@useobject{currentmarker}{}%
\end{pgfscope}%
\begin{pgfscope}%
\pgfsys@transformshift{-10.314890in}{1.487892in}%
\pgfsys@useobject{currentmarker}{}%
\end{pgfscope}%
\begin{pgfscope}%
\pgfsys@transformshift{-10.282013in}{1.514453in}%
\pgfsys@useobject{currentmarker}{}%
\end{pgfscope}%
\begin{pgfscope}%
\pgfsys@transformshift{-10.249136in}{1.410411in}%
\pgfsys@useobject{currentmarker}{}%
\end{pgfscope}%
\begin{pgfscope}%
\pgfsys@transformshift{-10.216259in}{1.612798in}%
\pgfsys@useobject{currentmarker}{}%
\end{pgfscope}%
\begin{pgfscope}%
\pgfsys@transformshift{-10.183381in}{1.484939in}%
\pgfsys@useobject{currentmarker}{}%
\end{pgfscope}%
\begin{pgfscope}%
\pgfsys@transformshift{-10.150504in}{1.413287in}%
\pgfsys@useobject{currentmarker}{}%
\end{pgfscope}%
\begin{pgfscope}%
\pgfsys@transformshift{-10.117627in}{1.570072in}%
\pgfsys@useobject{currentmarker}{}%
\end{pgfscope}%
\begin{pgfscope}%
\pgfsys@transformshift{-10.084750in}{1.673199in}%
\pgfsys@useobject{currentmarker}{}%
\end{pgfscope}%
\begin{pgfscope}%
\pgfsys@transformshift{-10.051872in}{1.423673in}%
\pgfsys@useobject{currentmarker}{}%
\end{pgfscope}%
\begin{pgfscope}%
\pgfsys@transformshift{-10.018995in}{1.676161in}%
\pgfsys@useobject{currentmarker}{}%
\end{pgfscope}%
\begin{pgfscope}%
\pgfsys@transformshift{-9.986118in}{1.441922in}%
\pgfsys@useobject{currentmarker}{}%
\end{pgfscope}%
\begin{pgfscope}%
\pgfsys@transformshift{-9.953241in}{1.473359in}%
\pgfsys@useobject{currentmarker}{}%
\end{pgfscope}%
\begin{pgfscope}%
\pgfsys@transformshift{-9.920363in}{1.550225in}%
\pgfsys@useobject{currentmarker}{}%
\end{pgfscope}%
\begin{pgfscope}%
\pgfsys@transformshift{-9.887486in}{1.354871in}%
\pgfsys@useobject{currentmarker}{}%
\end{pgfscope}%
\begin{pgfscope}%
\pgfsys@transformshift{-9.854609in}{1.359954in}%
\pgfsys@useobject{currentmarker}{}%
\end{pgfscope}%
\begin{pgfscope}%
\pgfsys@transformshift{-9.821731in}{1.644339in}%
\pgfsys@useobject{currentmarker}{}%
\end{pgfscope}%
\begin{pgfscope}%
\pgfsys@transformshift{-9.788854in}{1.422623in}%
\pgfsys@useobject{currentmarker}{}%
\end{pgfscope}%
\begin{pgfscope}%
\pgfsys@transformshift{-9.755977in}{1.334372in}%
\pgfsys@useobject{currentmarker}{}%
\end{pgfscope}%
\begin{pgfscope}%
\pgfsys@transformshift{-9.723100in}{1.566374in}%
\pgfsys@useobject{currentmarker}{}%
\end{pgfscope}%
\begin{pgfscope}%
\pgfsys@transformshift{-9.690222in}{1.493456in}%
\pgfsys@useobject{currentmarker}{}%
\end{pgfscope}%
\begin{pgfscope}%
\pgfsys@transformshift{-9.657345in}{1.410858in}%
\pgfsys@useobject{currentmarker}{}%
\end{pgfscope}%
\begin{pgfscope}%
\pgfsys@transformshift{-9.624468in}{1.382231in}%
\pgfsys@useobject{currentmarker}{}%
\end{pgfscope}%
\begin{pgfscope}%
\pgfsys@transformshift{-9.591591in}{1.482862in}%
\pgfsys@useobject{currentmarker}{}%
\end{pgfscope}%
\begin{pgfscope}%
\pgfsys@transformshift{-9.558713in}{1.503593in}%
\pgfsys@useobject{currentmarker}{}%
\end{pgfscope}%
\begin{pgfscope}%
\pgfsys@transformshift{-9.525836in}{1.484928in}%
\pgfsys@useobject{currentmarker}{}%
\end{pgfscope}%
\begin{pgfscope}%
\pgfsys@transformshift{-9.492959in}{1.520345in}%
\pgfsys@useobject{currentmarker}{}%
\end{pgfscope}%
\begin{pgfscope}%
\pgfsys@transformshift{-9.460082in}{1.544830in}%
\pgfsys@useobject{currentmarker}{}%
\end{pgfscope}%
\begin{pgfscope}%
\pgfsys@transformshift{-9.427204in}{1.658184in}%
\pgfsys@useobject{currentmarker}{}%
\end{pgfscope}%
\begin{pgfscope}%
\pgfsys@transformshift{-9.394327in}{1.696931in}%
\pgfsys@useobject{currentmarker}{}%
\end{pgfscope}%
\begin{pgfscope}%
\pgfsys@transformshift{-9.361450in}{1.495691in}%
\pgfsys@useobject{currentmarker}{}%
\end{pgfscope}%
\begin{pgfscope}%
\pgfsys@transformshift{-9.328572in}{1.467964in}%
\pgfsys@useobject{currentmarker}{}%
\end{pgfscope}%
\begin{pgfscope}%
\pgfsys@transformshift{-9.295695in}{1.327490in}%
\pgfsys@useobject{currentmarker}{}%
\end{pgfscope}%
\begin{pgfscope}%
\pgfsys@transformshift{-9.262818in}{1.543453in}%
\pgfsys@useobject{currentmarker}{}%
\end{pgfscope}%
\begin{pgfscope}%
\pgfsys@transformshift{-9.229941in}{1.468431in}%
\pgfsys@useobject{currentmarker}{}%
\end{pgfscope}%
\begin{pgfscope}%
\pgfsys@transformshift{-9.197063in}{1.414008in}%
\pgfsys@useobject{currentmarker}{}%
\end{pgfscope}%
\begin{pgfscope}%
\pgfsys@transformshift{-9.164186in}{1.722183in}%
\pgfsys@useobject{currentmarker}{}%
\end{pgfscope}%
\begin{pgfscope}%
\pgfsys@transformshift{-9.131309in}{1.675677in}%
\pgfsys@useobject{currentmarker}{}%
\end{pgfscope}%
\begin{pgfscope}%
\pgfsys@transformshift{-9.098432in}{1.477952in}%
\pgfsys@useobject{currentmarker}{}%
\end{pgfscope}%
\begin{pgfscope}%
\pgfsys@transformshift{-9.065554in}{1.607555in}%
\pgfsys@useobject{currentmarker}{}%
\end{pgfscope}%
\begin{pgfscope}%
\pgfsys@transformshift{-9.032677in}{1.401309in}%
\pgfsys@useobject{currentmarker}{}%
\end{pgfscope}%
\begin{pgfscope}%
\pgfsys@transformshift{-8.999800in}{1.274886in}%
\pgfsys@useobject{currentmarker}{}%
\end{pgfscope}%
\begin{pgfscope}%
\pgfsys@transformshift{-8.966923in}{1.284697in}%
\pgfsys@useobject{currentmarker}{}%
\end{pgfscope}%
\begin{pgfscope}%
\pgfsys@transformshift{-8.934045in}{1.407201in}%
\pgfsys@useobject{currentmarker}{}%
\end{pgfscope}%
\begin{pgfscope}%
\pgfsys@transformshift{-8.901168in}{1.563016in}%
\pgfsys@useobject{currentmarker}{}%
\end{pgfscope}%
\begin{pgfscope}%
\pgfsys@transformshift{-8.868291in}{1.431040in}%
\pgfsys@useobject{currentmarker}{}%
\end{pgfscope}%
\begin{pgfscope}%
\pgfsys@transformshift{-8.835414in}{1.412179in}%
\pgfsys@useobject{currentmarker}{}%
\end{pgfscope}%
\begin{pgfscope}%
\pgfsys@transformshift{-8.802536in}{1.354400in}%
\pgfsys@useobject{currentmarker}{}%
\end{pgfscope}%
\begin{pgfscope}%
\pgfsys@transformshift{-8.769659in}{1.465954in}%
\pgfsys@useobject{currentmarker}{}%
\end{pgfscope}%
\begin{pgfscope}%
\pgfsys@transformshift{-8.736782in}{1.592725in}%
\pgfsys@useobject{currentmarker}{}%
\end{pgfscope}%
\begin{pgfscope}%
\pgfsys@transformshift{-8.703904in}{1.505476in}%
\pgfsys@useobject{currentmarker}{}%
\end{pgfscope}%
\begin{pgfscope}%
\pgfsys@transformshift{-8.671027in}{1.413549in}%
\pgfsys@useobject{currentmarker}{}%
\end{pgfscope}%
\begin{pgfscope}%
\pgfsys@transformshift{-8.638150in}{1.723193in}%
\pgfsys@useobject{currentmarker}{}%
\end{pgfscope}%
\begin{pgfscope}%
\pgfsys@transformshift{-8.605273in}{1.481145in}%
\pgfsys@useobject{currentmarker}{}%
\end{pgfscope}%
\begin{pgfscope}%
\pgfsys@transformshift{-8.572395in}{1.613046in}%
\pgfsys@useobject{currentmarker}{}%
\end{pgfscope}%
\begin{pgfscope}%
\pgfsys@transformshift{-8.539518in}{1.669957in}%
\pgfsys@useobject{currentmarker}{}%
\end{pgfscope}%
\begin{pgfscope}%
\pgfsys@transformshift{-8.506641in}{1.486313in}%
\pgfsys@useobject{currentmarker}{}%
\end{pgfscope}%
\begin{pgfscope}%
\pgfsys@transformshift{-8.473764in}{1.299449in}%
\pgfsys@useobject{currentmarker}{}%
\end{pgfscope}%
\begin{pgfscope}%
\pgfsys@transformshift{-8.440886in}{1.456271in}%
\pgfsys@useobject{currentmarker}{}%
\end{pgfscope}%
\begin{pgfscope}%
\pgfsys@transformshift{-8.408009in}{1.519689in}%
\pgfsys@useobject{currentmarker}{}%
\end{pgfscope}%
\begin{pgfscope}%
\pgfsys@transformshift{-8.375132in}{1.562117in}%
\pgfsys@useobject{currentmarker}{}%
\end{pgfscope}%
\begin{pgfscope}%
\pgfsys@transformshift{-8.342255in}{1.401177in}%
\pgfsys@useobject{currentmarker}{}%
\end{pgfscope}%
\begin{pgfscope}%
\pgfsys@transformshift{-8.309377in}{1.276236in}%
\pgfsys@useobject{currentmarker}{}%
\end{pgfscope}%
\begin{pgfscope}%
\pgfsys@transformshift{-8.276500in}{1.589566in}%
\pgfsys@useobject{currentmarker}{}%
\end{pgfscope}%
\begin{pgfscope}%
\pgfsys@transformshift{-8.243623in}{1.581184in}%
\pgfsys@useobject{currentmarker}{}%
\end{pgfscope}%
\begin{pgfscope}%
\pgfsys@transformshift{-8.210746in}{1.517387in}%
\pgfsys@useobject{currentmarker}{}%
\end{pgfscope}%
\begin{pgfscope}%
\pgfsys@transformshift{-8.177868in}{1.343820in}%
\pgfsys@useobject{currentmarker}{}%
\end{pgfscope}%
\begin{pgfscope}%
\pgfsys@transformshift{-8.144991in}{1.660001in}%
\pgfsys@useobject{currentmarker}{}%
\end{pgfscope}%
\begin{pgfscope}%
\pgfsys@transformshift{-8.112114in}{1.523433in}%
\pgfsys@useobject{currentmarker}{}%
\end{pgfscope}%
\begin{pgfscope}%
\pgfsys@transformshift{-8.079236in}{1.458094in}%
\pgfsys@useobject{currentmarker}{}%
\end{pgfscope}%
\begin{pgfscope}%
\pgfsys@transformshift{-8.046359in}{1.585710in}%
\pgfsys@useobject{currentmarker}{}%
\end{pgfscope}%
\begin{pgfscope}%
\pgfsys@transformshift{-8.013482in}{1.496827in}%
\pgfsys@useobject{currentmarker}{}%
\end{pgfscope}%
\begin{pgfscope}%
\pgfsys@transformshift{-7.980605in}{1.350735in}%
\pgfsys@useobject{currentmarker}{}%
\end{pgfscope}%
\begin{pgfscope}%
\pgfsys@transformshift{-7.947727in}{1.671527in}%
\pgfsys@useobject{currentmarker}{}%
\end{pgfscope}%
\begin{pgfscope}%
\pgfsys@transformshift{-7.914850in}{1.433902in}%
\pgfsys@useobject{currentmarker}{}%
\end{pgfscope}%
\begin{pgfscope}%
\pgfsys@transformshift{-7.881973in}{1.358736in}%
\pgfsys@useobject{currentmarker}{}%
\end{pgfscope}%
\begin{pgfscope}%
\pgfsys@transformshift{-7.849096in}{1.511245in}%
\pgfsys@useobject{currentmarker}{}%
\end{pgfscope}%
\begin{pgfscope}%
\pgfsys@transformshift{-7.816218in}{1.436880in}%
\pgfsys@useobject{currentmarker}{}%
\end{pgfscope}%
\begin{pgfscope}%
\pgfsys@transformshift{-7.783341in}{1.767946in}%
\pgfsys@useobject{currentmarker}{}%
\end{pgfscope}%
\begin{pgfscope}%
\pgfsys@transformshift{-7.750464in}{1.485724in}%
\pgfsys@useobject{currentmarker}{}%
\end{pgfscope}%
\begin{pgfscope}%
\pgfsys@transformshift{-7.717587in}{1.588928in}%
\pgfsys@useobject{currentmarker}{}%
\end{pgfscope}%
\begin{pgfscope}%
\pgfsys@transformshift{-7.684709in}{1.524612in}%
\pgfsys@useobject{currentmarker}{}%
\end{pgfscope}%
\begin{pgfscope}%
\pgfsys@transformshift{-7.651832in}{1.583402in}%
\pgfsys@useobject{currentmarker}{}%
\end{pgfscope}%
\begin{pgfscope}%
\pgfsys@transformshift{-7.618955in}{1.591609in}%
\pgfsys@useobject{currentmarker}{}%
\end{pgfscope}%
\begin{pgfscope}%
\pgfsys@transformshift{-7.586078in}{1.574437in}%
\pgfsys@useobject{currentmarker}{}%
\end{pgfscope}%
\begin{pgfscope}%
\pgfsys@transformshift{-7.553200in}{1.203825in}%
\pgfsys@useobject{currentmarker}{}%
\end{pgfscope}%
\begin{pgfscope}%
\pgfsys@transformshift{-7.520323in}{1.310688in}%
\pgfsys@useobject{currentmarker}{}%
\end{pgfscope}%
\begin{pgfscope}%
\pgfsys@transformshift{-7.487446in}{1.606085in}%
\pgfsys@useobject{currentmarker}{}%
\end{pgfscope}%
\begin{pgfscope}%
\pgfsys@transformshift{-7.454568in}{1.378009in}%
\pgfsys@useobject{currentmarker}{}%
\end{pgfscope}%
\begin{pgfscope}%
\pgfsys@transformshift{-7.421691in}{1.411850in}%
\pgfsys@useobject{currentmarker}{}%
\end{pgfscope}%
\begin{pgfscope}%
\pgfsys@transformshift{-7.388814in}{1.410519in}%
\pgfsys@useobject{currentmarker}{}%
\end{pgfscope}%
\begin{pgfscope}%
\pgfsys@transformshift{-7.355937in}{1.528228in}%
\pgfsys@useobject{currentmarker}{}%
\end{pgfscope}%
\begin{pgfscope}%
\pgfsys@transformshift{-7.323059in}{1.586829in}%
\pgfsys@useobject{currentmarker}{}%
\end{pgfscope}%
\begin{pgfscope}%
\pgfsys@transformshift{-7.290182in}{1.485667in}%
\pgfsys@useobject{currentmarker}{}%
\end{pgfscope}%
\begin{pgfscope}%
\pgfsys@transformshift{-7.257305in}{1.477310in}%
\pgfsys@useobject{currentmarker}{}%
\end{pgfscope}%
\begin{pgfscope}%
\pgfsys@transformshift{-7.224428in}{1.552503in}%
\pgfsys@useobject{currentmarker}{}%
\end{pgfscope}%
\begin{pgfscope}%
\pgfsys@transformshift{-7.191550in}{1.467101in}%
\pgfsys@useobject{currentmarker}{}%
\end{pgfscope}%
\begin{pgfscope}%
\pgfsys@transformshift{-7.158673in}{1.580001in}%
\pgfsys@useobject{currentmarker}{}%
\end{pgfscope}%
\begin{pgfscope}%
\pgfsys@transformshift{-7.125796in}{1.571310in}%
\pgfsys@useobject{currentmarker}{}%
\end{pgfscope}%
\begin{pgfscope}%
\pgfsys@transformshift{-7.092919in}{1.448627in}%
\pgfsys@useobject{currentmarker}{}%
\end{pgfscope}%
\begin{pgfscope}%
\pgfsys@transformshift{-7.060041in}{1.509475in}%
\pgfsys@useobject{currentmarker}{}%
\end{pgfscope}%
\begin{pgfscope}%
\pgfsys@transformshift{-7.027164in}{1.469877in}%
\pgfsys@useobject{currentmarker}{}%
\end{pgfscope}%
\begin{pgfscope}%
\pgfsys@transformshift{-6.994287in}{1.545692in}%
\pgfsys@useobject{currentmarker}{}%
\end{pgfscope}%
\begin{pgfscope}%
\pgfsys@transformshift{-6.961410in}{1.489566in}%
\pgfsys@useobject{currentmarker}{}%
\end{pgfscope}%
\begin{pgfscope}%
\pgfsys@transformshift{-6.928532in}{1.544202in}%
\pgfsys@useobject{currentmarker}{}%
\end{pgfscope}%
\begin{pgfscope}%
\pgfsys@transformshift{-6.895655in}{1.589199in}%
\pgfsys@useobject{currentmarker}{}%
\end{pgfscope}%
\begin{pgfscope}%
\pgfsys@transformshift{-6.862778in}{1.512531in}%
\pgfsys@useobject{currentmarker}{}%
\end{pgfscope}%
\begin{pgfscope}%
\pgfsys@transformshift{-6.829900in}{1.439458in}%
\pgfsys@useobject{currentmarker}{}%
\end{pgfscope}%
\begin{pgfscope}%
\pgfsys@transformshift{-6.797023in}{1.627717in}%
\pgfsys@useobject{currentmarker}{}%
\end{pgfscope}%
\begin{pgfscope}%
\pgfsys@transformshift{-6.764146in}{1.596380in}%
\pgfsys@useobject{currentmarker}{}%
\end{pgfscope}%
\begin{pgfscope}%
\pgfsys@transformshift{-6.731269in}{1.512372in}%
\pgfsys@useobject{currentmarker}{}%
\end{pgfscope}%
\begin{pgfscope}%
\pgfsys@transformshift{-6.698391in}{1.394914in}%
\pgfsys@useobject{currentmarker}{}%
\end{pgfscope}%
\begin{pgfscope}%
\pgfsys@transformshift{-6.665514in}{1.485800in}%
\pgfsys@useobject{currentmarker}{}%
\end{pgfscope}%
\begin{pgfscope}%
\pgfsys@transformshift{-6.632637in}{1.499315in}%
\pgfsys@useobject{currentmarker}{}%
\end{pgfscope}%
\begin{pgfscope}%
\pgfsys@transformshift{-6.599760in}{1.569875in}%
\pgfsys@useobject{currentmarker}{}%
\end{pgfscope}%
\begin{pgfscope}%
\pgfsys@transformshift{-6.566882in}{1.662914in}%
\pgfsys@useobject{currentmarker}{}%
\end{pgfscope}%
\begin{pgfscope}%
\pgfsys@transformshift{-6.534005in}{1.449620in}%
\pgfsys@useobject{currentmarker}{}%
\end{pgfscope}%
\begin{pgfscope}%
\pgfsys@transformshift{-6.501128in}{1.555619in}%
\pgfsys@useobject{currentmarker}{}%
\end{pgfscope}%
\begin{pgfscope}%
\pgfsys@transformshift{-6.468251in}{1.607290in}%
\pgfsys@useobject{currentmarker}{}%
\end{pgfscope}%
\begin{pgfscope}%
\pgfsys@transformshift{-6.435373in}{1.489879in}%
\pgfsys@useobject{currentmarker}{}%
\end{pgfscope}%
\begin{pgfscope}%
\pgfsys@transformshift{-6.402496in}{1.520910in}%
\pgfsys@useobject{currentmarker}{}%
\end{pgfscope}%
\begin{pgfscope}%
\pgfsys@transformshift{-6.369619in}{1.393396in}%
\pgfsys@useobject{currentmarker}{}%
\end{pgfscope}%
\begin{pgfscope}%
\pgfsys@transformshift{-6.336741in}{1.615707in}%
\pgfsys@useobject{currentmarker}{}%
\end{pgfscope}%
\begin{pgfscope}%
\pgfsys@transformshift{-6.303864in}{1.435974in}%
\pgfsys@useobject{currentmarker}{}%
\end{pgfscope}%
\begin{pgfscope}%
\pgfsys@transformshift{-6.270987in}{1.605494in}%
\pgfsys@useobject{currentmarker}{}%
\end{pgfscope}%
\begin{pgfscope}%
\pgfsys@transformshift{-6.238110in}{1.494873in}%
\pgfsys@useobject{currentmarker}{}%
\end{pgfscope}%
\begin{pgfscope}%
\pgfsys@transformshift{-6.205232in}{1.429378in}%
\pgfsys@useobject{currentmarker}{}%
\end{pgfscope}%
\begin{pgfscope}%
\pgfsys@transformshift{-6.172355in}{1.522704in}%
\pgfsys@useobject{currentmarker}{}%
\end{pgfscope}%
\begin{pgfscope}%
\pgfsys@transformshift{-6.139478in}{1.726175in}%
\pgfsys@useobject{currentmarker}{}%
\end{pgfscope}%
\begin{pgfscope}%
\pgfsys@transformshift{-6.106601in}{1.517658in}%
\pgfsys@useobject{currentmarker}{}%
\end{pgfscope}%
\begin{pgfscope}%
\pgfsys@transformshift{-6.073723in}{1.440086in}%
\pgfsys@useobject{currentmarker}{}%
\end{pgfscope}%
\begin{pgfscope}%
\pgfsys@transformshift{-6.040846in}{1.401621in}%
\pgfsys@useobject{currentmarker}{}%
\end{pgfscope}%
\begin{pgfscope}%
\pgfsys@transformshift{-6.007969in}{1.449078in}%
\pgfsys@useobject{currentmarker}{}%
\end{pgfscope}%
\begin{pgfscope}%
\pgfsys@transformshift{-5.975092in}{1.623152in}%
\pgfsys@useobject{currentmarker}{}%
\end{pgfscope}%
\begin{pgfscope}%
\pgfsys@transformshift{-5.942214in}{1.527351in}%
\pgfsys@useobject{currentmarker}{}%
\end{pgfscope}%
\begin{pgfscope}%
\pgfsys@transformshift{-5.909337in}{1.419481in}%
\pgfsys@useobject{currentmarker}{}%
\end{pgfscope}%
\begin{pgfscope}%
\pgfsys@transformshift{-5.876460in}{1.504895in}%
\pgfsys@useobject{currentmarker}{}%
\end{pgfscope}%
\begin{pgfscope}%
\pgfsys@transformshift{-5.843583in}{1.332771in}%
\pgfsys@useobject{currentmarker}{}%
\end{pgfscope}%
\begin{pgfscope}%
\pgfsys@transformshift{-5.810705in}{1.532316in}%
\pgfsys@useobject{currentmarker}{}%
\end{pgfscope}%
\begin{pgfscope}%
\pgfsys@transformshift{-5.777828in}{1.472860in}%
\pgfsys@useobject{currentmarker}{}%
\end{pgfscope}%
\begin{pgfscope}%
\pgfsys@transformshift{-5.744951in}{1.602243in}%
\pgfsys@useobject{currentmarker}{}%
\end{pgfscope}%
\begin{pgfscope}%
\pgfsys@transformshift{-5.712073in}{1.409223in}%
\pgfsys@useobject{currentmarker}{}%
\end{pgfscope}%
\begin{pgfscope}%
\pgfsys@transformshift{-5.679196in}{1.592440in}%
\pgfsys@useobject{currentmarker}{}%
\end{pgfscope}%
\begin{pgfscope}%
\pgfsys@transformshift{-5.646319in}{1.422596in}%
\pgfsys@useobject{currentmarker}{}%
\end{pgfscope}%
\begin{pgfscope}%
\pgfsys@transformshift{-5.613442in}{1.585164in}%
\pgfsys@useobject{currentmarker}{}%
\end{pgfscope}%
\begin{pgfscope}%
\pgfsys@transformshift{-5.580564in}{1.514316in}%
\pgfsys@useobject{currentmarker}{}%
\end{pgfscope}%
\begin{pgfscope}%
\pgfsys@transformshift{-5.547687in}{1.516061in}%
\pgfsys@useobject{currentmarker}{}%
\end{pgfscope}%
\begin{pgfscope}%
\pgfsys@transformshift{-5.514810in}{1.739256in}%
\pgfsys@useobject{currentmarker}{}%
\end{pgfscope}%
\begin{pgfscope}%
\pgfsys@transformshift{-5.481933in}{1.491440in}%
\pgfsys@useobject{currentmarker}{}%
\end{pgfscope}%
\begin{pgfscope}%
\pgfsys@transformshift{-5.449055in}{1.511493in}%
\pgfsys@useobject{currentmarker}{}%
\end{pgfscope}%
\begin{pgfscope}%
\pgfsys@transformshift{-5.416178in}{1.268783in}%
\pgfsys@useobject{currentmarker}{}%
\end{pgfscope}%
\begin{pgfscope}%
\pgfsys@transformshift{-5.383301in}{1.600339in}%
\pgfsys@useobject{currentmarker}{}%
\end{pgfscope}%
\begin{pgfscope}%
\pgfsys@transformshift{-5.350424in}{1.412464in}%
\pgfsys@useobject{currentmarker}{}%
\end{pgfscope}%
\begin{pgfscope}%
\pgfsys@transformshift{-5.317546in}{1.640531in}%
\pgfsys@useobject{currentmarker}{}%
\end{pgfscope}%
\begin{pgfscope}%
\pgfsys@transformshift{-5.284669in}{1.484019in}%
\pgfsys@useobject{currentmarker}{}%
\end{pgfscope}%
\begin{pgfscope}%
\pgfsys@transformshift{-5.251792in}{1.413464in}%
\pgfsys@useobject{currentmarker}{}%
\end{pgfscope}%
\begin{pgfscope}%
\pgfsys@transformshift{-5.218915in}{1.401346in}%
\pgfsys@useobject{currentmarker}{}%
\end{pgfscope}%
\begin{pgfscope}%
\pgfsys@transformshift{-5.186037in}{1.384369in}%
\pgfsys@useobject{currentmarker}{}%
\end{pgfscope}%
\begin{pgfscope}%
\pgfsys@transformshift{-5.153160in}{1.596943in}%
\pgfsys@useobject{currentmarker}{}%
\end{pgfscope}%
\begin{pgfscope}%
\pgfsys@transformshift{-5.120283in}{1.477402in}%
\pgfsys@useobject{currentmarker}{}%
\end{pgfscope}%
\begin{pgfscope}%
\pgfsys@transformshift{-5.087405in}{1.326636in}%
\pgfsys@useobject{currentmarker}{}%
\end{pgfscope}%
\begin{pgfscope}%
\pgfsys@transformshift{-5.054528in}{1.527598in}%
\pgfsys@useobject{currentmarker}{}%
\end{pgfscope}%
\begin{pgfscope}%
\pgfsys@transformshift{-5.021651in}{1.465526in}%
\pgfsys@useobject{currentmarker}{}%
\end{pgfscope}%
\begin{pgfscope}%
\pgfsys@transformshift{-4.988774in}{1.327581in}%
\pgfsys@useobject{currentmarker}{}%
\end{pgfscope}%
\begin{pgfscope}%
\pgfsys@transformshift{-4.955896in}{1.540126in}%
\pgfsys@useobject{currentmarker}{}%
\end{pgfscope}%
\begin{pgfscope}%
\pgfsys@transformshift{-4.923019in}{1.257567in}%
\pgfsys@useobject{currentmarker}{}%
\end{pgfscope}%
\begin{pgfscope}%
\pgfsys@transformshift{-4.890142in}{1.541210in}%
\pgfsys@useobject{currentmarker}{}%
\end{pgfscope}%
\begin{pgfscope}%
\pgfsys@transformshift{-4.857265in}{1.396737in}%
\pgfsys@useobject{currentmarker}{}%
\end{pgfscope}%
\begin{pgfscope}%
\pgfsys@transformshift{-4.824387in}{1.404700in}%
\pgfsys@useobject{currentmarker}{}%
\end{pgfscope}%
\begin{pgfscope}%
\pgfsys@transformshift{-4.791510in}{1.574381in}%
\pgfsys@useobject{currentmarker}{}%
\end{pgfscope}%
\begin{pgfscope}%
\pgfsys@transformshift{-4.758633in}{1.424252in}%
\pgfsys@useobject{currentmarker}{}%
\end{pgfscope}%
\begin{pgfscope}%
\pgfsys@transformshift{-4.725756in}{1.632420in}%
\pgfsys@useobject{currentmarker}{}%
\end{pgfscope}%
\begin{pgfscope}%
\pgfsys@transformshift{-4.692878in}{1.565356in}%
\pgfsys@useobject{currentmarker}{}%
\end{pgfscope}%
\begin{pgfscope}%
\pgfsys@transformshift{-4.660001in}{1.595597in}%
\pgfsys@useobject{currentmarker}{}%
\end{pgfscope}%
\begin{pgfscope}%
\pgfsys@transformshift{-4.627124in}{1.586392in}%
\pgfsys@useobject{currentmarker}{}%
\end{pgfscope}%
\begin{pgfscope}%
\pgfsys@transformshift{-4.594247in}{1.539959in}%
\pgfsys@useobject{currentmarker}{}%
\end{pgfscope}%
\begin{pgfscope}%
\pgfsys@transformshift{-4.561369in}{1.507569in}%
\pgfsys@useobject{currentmarker}{}%
\end{pgfscope}%
\begin{pgfscope}%
\pgfsys@transformshift{-4.528492in}{1.591925in}%
\pgfsys@useobject{currentmarker}{}%
\end{pgfscope}%
\begin{pgfscope}%
\pgfsys@transformshift{-4.495615in}{1.606098in}%
\pgfsys@useobject{currentmarker}{}%
\end{pgfscope}%
\begin{pgfscope}%
\pgfsys@transformshift{-4.462737in}{1.358122in}%
\pgfsys@useobject{currentmarker}{}%
\end{pgfscope}%
\begin{pgfscope}%
\pgfsys@transformshift{-4.429860in}{1.452186in}%
\pgfsys@useobject{currentmarker}{}%
\end{pgfscope}%
\begin{pgfscope}%
\pgfsys@transformshift{-4.396983in}{1.571142in}%
\pgfsys@useobject{currentmarker}{}%
\end{pgfscope}%
\begin{pgfscope}%
\pgfsys@transformshift{-4.364106in}{1.393689in}%
\pgfsys@useobject{currentmarker}{}%
\end{pgfscope}%
\begin{pgfscope}%
\pgfsys@transformshift{-4.331228in}{1.469148in}%
\pgfsys@useobject{currentmarker}{}%
\end{pgfscope}%
\begin{pgfscope}%
\pgfsys@transformshift{-4.298351in}{1.456118in}%
\pgfsys@useobject{currentmarker}{}%
\end{pgfscope}%
\begin{pgfscope}%
\pgfsys@transformshift{-4.265474in}{1.634431in}%
\pgfsys@useobject{currentmarker}{}%
\end{pgfscope}%
\begin{pgfscope}%
\pgfsys@transformshift{-4.232597in}{1.489743in}%
\pgfsys@useobject{currentmarker}{}%
\end{pgfscope}%
\begin{pgfscope}%
\pgfsys@transformshift{-4.199719in}{1.367816in}%
\pgfsys@useobject{currentmarker}{}%
\end{pgfscope}%
\begin{pgfscope}%
\pgfsys@transformshift{-4.166842in}{1.478636in}%
\pgfsys@useobject{currentmarker}{}%
\end{pgfscope}%
\begin{pgfscope}%
\pgfsys@transformshift{-4.133965in}{1.665338in}%
\pgfsys@useobject{currentmarker}{}%
\end{pgfscope}%
\begin{pgfscope}%
\pgfsys@transformshift{-4.101088in}{1.448215in}%
\pgfsys@useobject{currentmarker}{}%
\end{pgfscope}%
\begin{pgfscope}%
\pgfsys@transformshift{-4.068210in}{1.296558in}%
\pgfsys@useobject{currentmarker}{}%
\end{pgfscope}%
\begin{pgfscope}%
\pgfsys@transformshift{-4.035333in}{1.499585in}%
\pgfsys@useobject{currentmarker}{}%
\end{pgfscope}%
\begin{pgfscope}%
\pgfsys@transformshift{-4.002456in}{1.442656in}%
\pgfsys@useobject{currentmarker}{}%
\end{pgfscope}%
\begin{pgfscope}%
\pgfsys@transformshift{-3.969579in}{1.572189in}%
\pgfsys@useobject{currentmarker}{}%
\end{pgfscope}%
\begin{pgfscope}%
\pgfsys@transformshift{-3.936701in}{1.417236in}%
\pgfsys@useobject{currentmarker}{}%
\end{pgfscope}%
\begin{pgfscope}%
\pgfsys@transformshift{-3.903824in}{1.646567in}%
\pgfsys@useobject{currentmarker}{}%
\end{pgfscope}%
\begin{pgfscope}%
\pgfsys@transformshift{-3.870947in}{1.490271in}%
\pgfsys@useobject{currentmarker}{}%
\end{pgfscope}%
\begin{pgfscope}%
\pgfsys@transformshift{-3.838069in}{1.639659in}%
\pgfsys@useobject{currentmarker}{}%
\end{pgfscope}%
\begin{pgfscope}%
\pgfsys@transformshift{-3.805192in}{1.575976in}%
\pgfsys@useobject{currentmarker}{}%
\end{pgfscope}%
\begin{pgfscope}%
\pgfsys@transformshift{-3.772315in}{1.461729in}%
\pgfsys@useobject{currentmarker}{}%
\end{pgfscope}%
\begin{pgfscope}%
\pgfsys@transformshift{-3.739438in}{1.533463in}%
\pgfsys@useobject{currentmarker}{}%
\end{pgfscope}%
\begin{pgfscope}%
\pgfsys@transformshift{-3.706560in}{1.521104in}%
\pgfsys@useobject{currentmarker}{}%
\end{pgfscope}%
\begin{pgfscope}%
\pgfsys@transformshift{-3.673683in}{1.478229in}%
\pgfsys@useobject{currentmarker}{}%
\end{pgfscope}%
\begin{pgfscope}%
\pgfsys@transformshift{-3.640806in}{1.388501in}%
\pgfsys@useobject{currentmarker}{}%
\end{pgfscope}%
\begin{pgfscope}%
\pgfsys@transformshift{-3.607929in}{1.326210in}%
\pgfsys@useobject{currentmarker}{}%
\end{pgfscope}%
\begin{pgfscope}%
\pgfsys@transformshift{-3.575051in}{1.515485in}%
\pgfsys@useobject{currentmarker}{}%
\end{pgfscope}%
\begin{pgfscope}%
\pgfsys@transformshift{-3.542174in}{1.498305in}%
\pgfsys@useobject{currentmarker}{}%
\end{pgfscope}%
\begin{pgfscope}%
\pgfsys@transformshift{-3.509297in}{1.607568in}%
\pgfsys@useobject{currentmarker}{}%
\end{pgfscope}%
\begin{pgfscope}%
\pgfsys@transformshift{-3.476420in}{1.635718in}%
\pgfsys@useobject{currentmarker}{}%
\end{pgfscope}%
\begin{pgfscope}%
\pgfsys@transformshift{-3.443542in}{1.464029in}%
\pgfsys@useobject{currentmarker}{}%
\end{pgfscope}%
\begin{pgfscope}%
\pgfsys@transformshift{-3.410665in}{1.540737in}%
\pgfsys@useobject{currentmarker}{}%
\end{pgfscope}%
\begin{pgfscope}%
\pgfsys@transformshift{-3.377788in}{1.716169in}%
\pgfsys@useobject{currentmarker}{}%
\end{pgfscope}%
\begin{pgfscope}%
\pgfsys@transformshift{-3.344910in}{1.281236in}%
\pgfsys@useobject{currentmarker}{}%
\end{pgfscope}%
\begin{pgfscope}%
\pgfsys@transformshift{-3.312033in}{1.524445in}%
\pgfsys@useobject{currentmarker}{}%
\end{pgfscope}%
\begin{pgfscope}%
\pgfsys@transformshift{-3.279156in}{1.471023in}%
\pgfsys@useobject{currentmarker}{}%
\end{pgfscope}%
\begin{pgfscope}%
\pgfsys@transformshift{-3.246279in}{1.475701in}%
\pgfsys@useobject{currentmarker}{}%
\end{pgfscope}%
\begin{pgfscope}%
\pgfsys@transformshift{-3.213401in}{1.459974in}%
\pgfsys@useobject{currentmarker}{}%
\end{pgfscope}%
\begin{pgfscope}%
\pgfsys@transformshift{-3.180524in}{1.466764in}%
\pgfsys@useobject{currentmarker}{}%
\end{pgfscope}%
\begin{pgfscope}%
\pgfsys@transformshift{-3.147647in}{1.617779in}%
\pgfsys@useobject{currentmarker}{}%
\end{pgfscope}%
\begin{pgfscope}%
\pgfsys@transformshift{-3.114770in}{1.617692in}%
\pgfsys@useobject{currentmarker}{}%
\end{pgfscope}%
\begin{pgfscope}%
\pgfsys@transformshift{-3.081892in}{1.611685in}%
\pgfsys@useobject{currentmarker}{}%
\end{pgfscope}%
\begin{pgfscope}%
\pgfsys@transformshift{-3.049015in}{1.401471in}%
\pgfsys@useobject{currentmarker}{}%
\end{pgfscope}%
\begin{pgfscope}%
\pgfsys@transformshift{-3.016138in}{1.452443in}%
\pgfsys@useobject{currentmarker}{}%
\end{pgfscope}%
\begin{pgfscope}%
\pgfsys@transformshift{-2.983261in}{1.416337in}%
\pgfsys@useobject{currentmarker}{}%
\end{pgfscope}%
\begin{pgfscope}%
\pgfsys@transformshift{-2.950383in}{1.365899in}%
\pgfsys@useobject{currentmarker}{}%
\end{pgfscope}%
\begin{pgfscope}%
\pgfsys@transformshift{-2.917506in}{1.519173in}%
\pgfsys@useobject{currentmarker}{}%
\end{pgfscope}%
\begin{pgfscope}%
\pgfsys@transformshift{-2.884629in}{1.586208in}%
\pgfsys@useobject{currentmarker}{}%
\end{pgfscope}%
\begin{pgfscope}%
\pgfsys@transformshift{-2.851752in}{1.511848in}%
\pgfsys@useobject{currentmarker}{}%
\end{pgfscope}%
\begin{pgfscope}%
\pgfsys@transformshift{-2.818874in}{1.484680in}%
\pgfsys@useobject{currentmarker}{}%
\end{pgfscope}%
\begin{pgfscope}%
\pgfsys@transformshift{-2.785997in}{1.419002in}%
\pgfsys@useobject{currentmarker}{}%
\end{pgfscope}%
\begin{pgfscope}%
\pgfsys@transformshift{-2.753120in}{1.549307in}%
\pgfsys@useobject{currentmarker}{}%
\end{pgfscope}%
\begin{pgfscope}%
\pgfsys@transformshift{-2.720242in}{1.381232in}%
\pgfsys@useobject{currentmarker}{}%
\end{pgfscope}%
\begin{pgfscope}%
\pgfsys@transformshift{-2.687365in}{1.439433in}%
\pgfsys@useobject{currentmarker}{}%
\end{pgfscope}%
\begin{pgfscope}%
\pgfsys@transformshift{-2.654488in}{1.481102in}%
\pgfsys@useobject{currentmarker}{}%
\end{pgfscope}%
\begin{pgfscope}%
\pgfsys@transformshift{-2.621611in}{1.477199in}%
\pgfsys@useobject{currentmarker}{}%
\end{pgfscope}%
\begin{pgfscope}%
\pgfsys@transformshift{-2.588733in}{1.449991in}%
\pgfsys@useobject{currentmarker}{}%
\end{pgfscope}%
\begin{pgfscope}%
\pgfsys@transformshift{-2.555856in}{1.536146in}%
\pgfsys@useobject{currentmarker}{}%
\end{pgfscope}%
\begin{pgfscope}%
\pgfsys@transformshift{-2.522979in}{1.651281in}%
\pgfsys@useobject{currentmarker}{}%
\end{pgfscope}%
\begin{pgfscope}%
\pgfsys@transformshift{-2.490102in}{1.724585in}%
\pgfsys@useobject{currentmarker}{}%
\end{pgfscope}%
\begin{pgfscope}%
\pgfsys@transformshift{-2.457224in}{1.425904in}%
\pgfsys@useobject{currentmarker}{}%
\end{pgfscope}%
\begin{pgfscope}%
\pgfsys@transformshift{-2.424347in}{1.482549in}%
\pgfsys@useobject{currentmarker}{}%
\end{pgfscope}%
\begin{pgfscope}%
\pgfsys@transformshift{-2.391470in}{1.565975in}%
\pgfsys@useobject{currentmarker}{}%
\end{pgfscope}%
\begin{pgfscope}%
\pgfsys@transformshift{-2.358593in}{1.634649in}%
\pgfsys@useobject{currentmarker}{}%
\end{pgfscope}%
\begin{pgfscope}%
\pgfsys@transformshift{-2.325715in}{1.341199in}%
\pgfsys@useobject{currentmarker}{}%
\end{pgfscope}%
\begin{pgfscope}%
\pgfsys@transformshift{-2.292838in}{1.410567in}%
\pgfsys@useobject{currentmarker}{}%
\end{pgfscope}%
\begin{pgfscope}%
\pgfsys@transformshift{-2.259961in}{1.386641in}%
\pgfsys@useobject{currentmarker}{}%
\end{pgfscope}%
\begin{pgfscope}%
\pgfsys@transformshift{-2.227084in}{1.627102in}%
\pgfsys@useobject{currentmarker}{}%
\end{pgfscope}%
\begin{pgfscope}%
\pgfsys@transformshift{-2.194206in}{1.597499in}%
\pgfsys@useobject{currentmarker}{}%
\end{pgfscope}%
\begin{pgfscope}%
\pgfsys@transformshift{-2.161329in}{1.456077in}%
\pgfsys@useobject{currentmarker}{}%
\end{pgfscope}%
\begin{pgfscope}%
\pgfsys@transformshift{-2.128452in}{1.498731in}%
\pgfsys@useobject{currentmarker}{}%
\end{pgfscope}%
\begin{pgfscope}%
\pgfsys@transformshift{-2.095574in}{1.568592in}%
\pgfsys@useobject{currentmarker}{}%
\end{pgfscope}%
\begin{pgfscope}%
\pgfsys@transformshift{-2.062697in}{1.391026in}%
\pgfsys@useobject{currentmarker}{}%
\end{pgfscope}%
\begin{pgfscope}%
\pgfsys@transformshift{-2.029820in}{1.432832in}%
\pgfsys@useobject{currentmarker}{}%
\end{pgfscope}%
\begin{pgfscope}%
\pgfsys@transformshift{-1.996943in}{1.530271in}%
\pgfsys@useobject{currentmarker}{}%
\end{pgfscope}%
\begin{pgfscope}%
\pgfsys@transformshift{-1.964065in}{1.508858in}%
\pgfsys@useobject{currentmarker}{}%
\end{pgfscope}%
\begin{pgfscope}%
\pgfsys@transformshift{-1.931188in}{1.604044in}%
\pgfsys@useobject{currentmarker}{}%
\end{pgfscope}%
\begin{pgfscope}%
\pgfsys@transformshift{-1.898311in}{1.522022in}%
\pgfsys@useobject{currentmarker}{}%
\end{pgfscope}%
\begin{pgfscope}%
\pgfsys@transformshift{-1.865434in}{1.521515in}%
\pgfsys@useobject{currentmarker}{}%
\end{pgfscope}%
\begin{pgfscope}%
\pgfsys@transformshift{-1.832556in}{1.433365in}%
\pgfsys@useobject{currentmarker}{}%
\end{pgfscope}%
\begin{pgfscope}%
\pgfsys@transformshift{-1.799679in}{1.486579in}%
\pgfsys@useobject{currentmarker}{}%
\end{pgfscope}%
\begin{pgfscope}%
\pgfsys@transformshift{-1.766802in}{1.508521in}%
\pgfsys@useobject{currentmarker}{}%
\end{pgfscope}%
\begin{pgfscope}%
\pgfsys@transformshift{-1.733925in}{1.530443in}%
\pgfsys@useobject{currentmarker}{}%
\end{pgfscope}%
\begin{pgfscope}%
\pgfsys@transformshift{-1.701047in}{1.415952in}%
\pgfsys@useobject{currentmarker}{}%
\end{pgfscope}%
\begin{pgfscope}%
\pgfsys@transformshift{-1.668170in}{1.591776in}%
\pgfsys@useobject{currentmarker}{}%
\end{pgfscope}%
\begin{pgfscope}%
\pgfsys@transformshift{-1.635293in}{1.536194in}%
\pgfsys@useobject{currentmarker}{}%
\end{pgfscope}%
\begin{pgfscope}%
\pgfsys@transformshift{-1.602416in}{1.690642in}%
\pgfsys@useobject{currentmarker}{}%
\end{pgfscope}%
\begin{pgfscope}%
\pgfsys@transformshift{-1.569538in}{1.554441in}%
\pgfsys@useobject{currentmarker}{}%
\end{pgfscope}%
\begin{pgfscope}%
\pgfsys@transformshift{-1.536661in}{1.524182in}%
\pgfsys@useobject{currentmarker}{}%
\end{pgfscope}%
\begin{pgfscope}%
\pgfsys@transformshift{-1.503784in}{1.529038in}%
\pgfsys@useobject{currentmarker}{}%
\end{pgfscope}%
\begin{pgfscope}%
\pgfsys@transformshift{-1.470906in}{1.503136in}%
\pgfsys@useobject{currentmarker}{}%
\end{pgfscope}%
\begin{pgfscope}%
\pgfsys@transformshift{-1.438029in}{1.716910in}%
\pgfsys@useobject{currentmarker}{}%
\end{pgfscope}%
\begin{pgfscope}%
\pgfsys@transformshift{-1.405152in}{1.477734in}%
\pgfsys@useobject{currentmarker}{}%
\end{pgfscope}%
\begin{pgfscope}%
\pgfsys@transformshift{-1.372275in}{1.477000in}%
\pgfsys@useobject{currentmarker}{}%
\end{pgfscope}%
\begin{pgfscope}%
\pgfsys@transformshift{-1.339397in}{1.358037in}%
\pgfsys@useobject{currentmarker}{}%
\end{pgfscope}%
\begin{pgfscope}%
\pgfsys@transformshift{-1.306520in}{1.775456in}%
\pgfsys@useobject{currentmarker}{}%
\end{pgfscope}%
\begin{pgfscope}%
\pgfsys@transformshift{-1.273643in}{1.460944in}%
\pgfsys@useobject{currentmarker}{}%
\end{pgfscope}%
\begin{pgfscope}%
\pgfsys@transformshift{-1.240766in}{1.367439in}%
\pgfsys@useobject{currentmarker}{}%
\end{pgfscope}%
\begin{pgfscope}%
\pgfsys@transformshift{-1.207888in}{1.686429in}%
\pgfsys@useobject{currentmarker}{}%
\end{pgfscope}%
\begin{pgfscope}%
\pgfsys@transformshift{-1.175011in}{1.482471in}%
\pgfsys@useobject{currentmarker}{}%
\end{pgfscope}%
\begin{pgfscope}%
\pgfsys@transformshift{-1.142134in}{1.455099in}%
\pgfsys@useobject{currentmarker}{}%
\end{pgfscope}%
\begin{pgfscope}%
\pgfsys@transformshift{-1.109257in}{1.453298in}%
\pgfsys@useobject{currentmarker}{}%
\end{pgfscope}%
\begin{pgfscope}%
\pgfsys@transformshift{-1.076379in}{1.682781in}%
\pgfsys@useobject{currentmarker}{}%
\end{pgfscope}%
\begin{pgfscope}%
\pgfsys@transformshift{-1.043502in}{1.449909in}%
\pgfsys@useobject{currentmarker}{}%
\end{pgfscope}%
\begin{pgfscope}%
\pgfsys@transformshift{-1.010625in}{1.656665in}%
\pgfsys@useobject{currentmarker}{}%
\end{pgfscope}%
\begin{pgfscope}%
\pgfsys@transformshift{-0.977747in}{1.480673in}%
\pgfsys@useobject{currentmarker}{}%
\end{pgfscope}%
\begin{pgfscope}%
\pgfsys@transformshift{-0.944870in}{1.417224in}%
\pgfsys@useobject{currentmarker}{}%
\end{pgfscope}%
\begin{pgfscope}%
\pgfsys@transformshift{-0.911993in}{1.552614in}%
\pgfsys@useobject{currentmarker}{}%
\end{pgfscope}%
\begin{pgfscope}%
\pgfsys@transformshift{-0.879116in}{1.349461in}%
\pgfsys@useobject{currentmarker}{}%
\end{pgfscope}%
\begin{pgfscope}%
\pgfsys@transformshift{-0.846238in}{1.534759in}%
\pgfsys@useobject{currentmarker}{}%
\end{pgfscope}%
\begin{pgfscope}%
\pgfsys@transformshift{-0.813361in}{1.549191in}%
\pgfsys@useobject{currentmarker}{}%
\end{pgfscope}%
\begin{pgfscope}%
\pgfsys@transformshift{-0.780484in}{1.461047in}%
\pgfsys@useobject{currentmarker}{}%
\end{pgfscope}%
\begin{pgfscope}%
\pgfsys@transformshift{-0.747607in}{1.441794in}%
\pgfsys@useobject{currentmarker}{}%
\end{pgfscope}%
\begin{pgfscope}%
\pgfsys@transformshift{-0.714729in}{1.494150in}%
\pgfsys@useobject{currentmarker}{}%
\end{pgfscope}%
\begin{pgfscope}%
\pgfsys@transformshift{-0.681852in}{1.532181in}%
\pgfsys@useobject{currentmarker}{}%
\end{pgfscope}%
\begin{pgfscope}%
\pgfsys@transformshift{-0.648975in}{1.338643in}%
\pgfsys@useobject{currentmarker}{}%
\end{pgfscope}%
\begin{pgfscope}%
\pgfsys@transformshift{-0.616098in}{1.486464in}%
\pgfsys@useobject{currentmarker}{}%
\end{pgfscope}%
\begin{pgfscope}%
\pgfsys@transformshift{-0.583220in}{1.721974in}%
\pgfsys@useobject{currentmarker}{}%
\end{pgfscope}%
\begin{pgfscope}%
\pgfsys@transformshift{-0.550343in}{1.505363in}%
\pgfsys@useobject{currentmarker}{}%
\end{pgfscope}%
\begin{pgfscope}%
\pgfsys@transformshift{-0.517466in}{1.513116in}%
\pgfsys@useobject{currentmarker}{}%
\end{pgfscope}%
\begin{pgfscope}%
\pgfsys@transformshift{-0.484589in}{1.434306in}%
\pgfsys@useobject{currentmarker}{}%
\end{pgfscope}%
\begin{pgfscope}%
\pgfsys@transformshift{-0.451711in}{1.769267in}%
\pgfsys@useobject{currentmarker}{}%
\end{pgfscope}%
\begin{pgfscope}%
\pgfsys@transformshift{-0.418834in}{1.394322in}%
\pgfsys@useobject{currentmarker}{}%
\end{pgfscope}%
\begin{pgfscope}%
\pgfsys@transformshift{-0.385957in}{1.387757in}%
\pgfsys@useobject{currentmarker}{}%
\end{pgfscope}%
\begin{pgfscope}%
\pgfsys@transformshift{-0.353079in}{1.281177in}%
\pgfsys@useobject{currentmarker}{}%
\end{pgfscope}%
\begin{pgfscope}%
\pgfsys@transformshift{-0.320202in}{1.394554in}%
\pgfsys@useobject{currentmarker}{}%
\end{pgfscope}%
\begin{pgfscope}%
\pgfsys@transformshift{-0.287325in}{1.290129in}%
\pgfsys@useobject{currentmarker}{}%
\end{pgfscope}%
\begin{pgfscope}%
\pgfsys@transformshift{-0.254448in}{1.586107in}%
\pgfsys@useobject{currentmarker}{}%
\end{pgfscope}%
\begin{pgfscope}%
\pgfsys@transformshift{-0.221570in}{1.524948in}%
\pgfsys@useobject{currentmarker}{}%
\end{pgfscope}%
\begin{pgfscope}%
\pgfsys@transformshift{-0.188693in}{1.478027in}%
\pgfsys@useobject{currentmarker}{}%
\end{pgfscope}%
\begin{pgfscope}%
\pgfsys@transformshift{-0.155816in}{1.503103in}%
\pgfsys@useobject{currentmarker}{}%
\end{pgfscope}%
\begin{pgfscope}%
\pgfsys@transformshift{-0.122939in}{1.420395in}%
\pgfsys@useobject{currentmarker}{}%
\end{pgfscope}%
\begin{pgfscope}%
\pgfsys@transformshift{-0.090061in}{1.467946in}%
\pgfsys@useobject{currentmarker}{}%
\end{pgfscope}%
\begin{pgfscope}%
\pgfsys@transformshift{-0.057184in}{1.443934in}%
\pgfsys@useobject{currentmarker}{}%
\end{pgfscope}%
\begin{pgfscope}%
\pgfsys@transformshift{-0.024307in}{1.369310in}%
\pgfsys@useobject{currentmarker}{}%
\end{pgfscope}%
\begin{pgfscope}%
\pgfsys@transformshift{0.008570in}{1.595653in}%
\pgfsys@useobject{currentmarker}{}%
\end{pgfscope}%
\begin{pgfscope}%
\pgfsys@transformshift{0.041448in}{1.559798in}%
\pgfsys@useobject{currentmarker}{}%
\end{pgfscope}%
\begin{pgfscope}%
\pgfsys@transformshift{0.074325in}{1.441888in}%
\pgfsys@useobject{currentmarker}{}%
\end{pgfscope}%
\begin{pgfscope}%
\pgfsys@transformshift{0.107202in}{1.615076in}%
\pgfsys@useobject{currentmarker}{}%
\end{pgfscope}%
\begin{pgfscope}%
\pgfsys@transformshift{0.140079in}{1.566608in}%
\pgfsys@useobject{currentmarker}{}%
\end{pgfscope}%
\begin{pgfscope}%
\pgfsys@transformshift{0.172957in}{1.477060in}%
\pgfsys@useobject{currentmarker}{}%
\end{pgfscope}%
\begin{pgfscope}%
\pgfsys@transformshift{0.205834in}{1.455375in}%
\pgfsys@useobject{currentmarker}{}%
\end{pgfscope}%
\begin{pgfscope}%
\pgfsys@transformshift{0.238711in}{1.427161in}%
\pgfsys@useobject{currentmarker}{}%
\end{pgfscope}%
\begin{pgfscope}%
\pgfsys@transformshift{0.271589in}{1.627597in}%
\pgfsys@useobject{currentmarker}{}%
\end{pgfscope}%
\begin{pgfscope}%
\pgfsys@transformshift{0.304466in}{1.576134in}%
\pgfsys@useobject{currentmarker}{}%
\end{pgfscope}%
\begin{pgfscope}%
\pgfsys@transformshift{0.337343in}{1.572697in}%
\pgfsys@useobject{currentmarker}{}%
\end{pgfscope}%
\begin{pgfscope}%
\pgfsys@transformshift{0.370220in}{1.600264in}%
\pgfsys@useobject{currentmarker}{}%
\end{pgfscope}%
\begin{pgfscope}%
\pgfsys@transformshift{0.403098in}{1.409489in}%
\pgfsys@useobject{currentmarker}{}%
\end{pgfscope}%
\begin{pgfscope}%
\pgfsys@transformshift{0.435975in}{1.558032in}%
\pgfsys@useobject{currentmarker}{}%
\end{pgfscope}%
\begin{pgfscope}%
\pgfsys@transformshift{0.468852in}{1.432063in}%
\pgfsys@useobject{currentmarker}{}%
\end{pgfscope}%
\begin{pgfscope}%
\pgfsys@transformshift{0.501729in}{1.396205in}%
\pgfsys@useobject{currentmarker}{}%
\end{pgfscope}%
\begin{pgfscope}%
\pgfsys@transformshift{0.534607in}{1.492476in}%
\pgfsys@useobject{currentmarker}{}%
\end{pgfscope}%
\begin{pgfscope}%
\pgfsys@transformshift{0.567484in}{1.473088in}%
\pgfsys@useobject{currentmarker}{}%
\end{pgfscope}%
\begin{pgfscope}%
\pgfsys@transformshift{0.600361in}{1.419831in}%
\pgfsys@useobject{currentmarker}{}%
\end{pgfscope}%
\begin{pgfscope}%
\pgfsys@transformshift{0.633238in}{1.471623in}%
\pgfsys@useobject{currentmarker}{}%
\end{pgfscope}%
\begin{pgfscope}%
\pgfsys@transformshift{0.666116in}{1.385478in}%
\pgfsys@useobject{currentmarker}{}%
\end{pgfscope}%
\begin{pgfscope}%
\pgfsys@transformshift{0.698993in}{1.517946in}%
\pgfsys@useobject{currentmarker}{}%
\end{pgfscope}%
\begin{pgfscope}%
\pgfsys@transformshift{0.731870in}{1.577882in}%
\pgfsys@useobject{currentmarker}{}%
\end{pgfscope}%
\begin{pgfscope}%
\pgfsys@transformshift{0.764747in}{1.506683in}%
\pgfsys@useobject{currentmarker}{}%
\end{pgfscope}%
\begin{pgfscope}%
\pgfsys@transformshift{0.797625in}{1.662422in}%
\pgfsys@useobject{currentmarker}{}%
\end{pgfscope}%
\begin{pgfscope}%
\pgfsys@transformshift{0.830502in}{1.564510in}%
\pgfsys@useobject{currentmarker}{}%
\end{pgfscope}%
\begin{pgfscope}%
\pgfsys@transformshift{0.863379in}{1.554707in}%
\pgfsys@useobject{currentmarker}{}%
\end{pgfscope}%
\begin{pgfscope}%
\pgfsys@transformshift{0.896257in}{1.484043in}%
\pgfsys@useobject{currentmarker}{}%
\end{pgfscope}%
\begin{pgfscope}%
\pgfsys@transformshift{0.929134in}{1.486100in}%
\pgfsys@useobject{currentmarker}{}%
\end{pgfscope}%
\begin{pgfscope}%
\pgfsys@transformshift{0.962011in}{1.559655in}%
\pgfsys@useobject{currentmarker}{}%
\end{pgfscope}%
\begin{pgfscope}%
\pgfsys@transformshift{0.994888in}{1.588884in}%
\pgfsys@useobject{currentmarker}{}%
\end{pgfscope}%
\begin{pgfscope}%
\pgfsys@transformshift{1.027766in}{1.501009in}%
\pgfsys@useobject{currentmarker}{}%
\end{pgfscope}%
\begin{pgfscope}%
\pgfsys@transformshift{1.060643in}{1.633948in}%
\pgfsys@useobject{currentmarker}{}%
\end{pgfscope}%
\begin{pgfscope}%
\pgfsys@transformshift{1.093520in}{1.634009in}%
\pgfsys@useobject{currentmarker}{}%
\end{pgfscope}%
\begin{pgfscope}%
\pgfsys@transformshift{1.126397in}{1.593260in}%
\pgfsys@useobject{currentmarker}{}%
\end{pgfscope}%
\begin{pgfscope}%
\pgfsys@transformshift{1.159275in}{1.489708in}%
\pgfsys@useobject{currentmarker}{}%
\end{pgfscope}%
\begin{pgfscope}%
\pgfsys@transformshift{1.192152in}{1.285877in}%
\pgfsys@useobject{currentmarker}{}%
\end{pgfscope}%
\begin{pgfscope}%
\pgfsys@transformshift{1.225029in}{1.505159in}%
\pgfsys@useobject{currentmarker}{}%
\end{pgfscope}%
\begin{pgfscope}%
\pgfsys@transformshift{1.257906in}{1.430907in}%
\pgfsys@useobject{currentmarker}{}%
\end{pgfscope}%
\begin{pgfscope}%
\pgfsys@transformshift{1.290784in}{1.505836in}%
\pgfsys@useobject{currentmarker}{}%
\end{pgfscope}%
\begin{pgfscope}%
\pgfsys@transformshift{1.323661in}{1.436602in}%
\pgfsys@useobject{currentmarker}{}%
\end{pgfscope}%
\begin{pgfscope}%
\pgfsys@transformshift{1.356538in}{1.416233in}%
\pgfsys@useobject{currentmarker}{}%
\end{pgfscope}%
\begin{pgfscope}%
\pgfsys@transformshift{1.389415in}{1.472082in}%
\pgfsys@useobject{currentmarker}{}%
\end{pgfscope}%
\begin{pgfscope}%
\pgfsys@transformshift{1.422293in}{1.483532in}%
\pgfsys@useobject{currentmarker}{}%
\end{pgfscope}%
\begin{pgfscope}%
\pgfsys@transformshift{1.455170in}{1.397370in}%
\pgfsys@useobject{currentmarker}{}%
\end{pgfscope}%
\begin{pgfscope}%
\pgfsys@transformshift{1.488047in}{1.538737in}%
\pgfsys@useobject{currentmarker}{}%
\end{pgfscope}%
\begin{pgfscope}%
\pgfsys@transformshift{1.520925in}{1.451173in}%
\pgfsys@useobject{currentmarker}{}%
\end{pgfscope}%
\begin{pgfscope}%
\pgfsys@transformshift{1.553802in}{1.494163in}%
\pgfsys@useobject{currentmarker}{}%
\end{pgfscope}%
\begin{pgfscope}%
\pgfsys@transformshift{1.586679in}{1.381101in}%
\pgfsys@useobject{currentmarker}{}%
\end{pgfscope}%
\begin{pgfscope}%
\pgfsys@transformshift{1.619556in}{1.398008in}%
\pgfsys@useobject{currentmarker}{}%
\end{pgfscope}%
\begin{pgfscope}%
\pgfsys@transformshift{1.652434in}{1.520079in}%
\pgfsys@useobject{currentmarker}{}%
\end{pgfscope}%
\begin{pgfscope}%
\pgfsys@transformshift{1.685311in}{1.398226in}%
\pgfsys@useobject{currentmarker}{}%
\end{pgfscope}%
\begin{pgfscope}%
\pgfsys@transformshift{1.718188in}{1.539999in}%
\pgfsys@useobject{currentmarker}{}%
\end{pgfscope}%
\begin{pgfscope}%
\pgfsys@transformshift{1.751065in}{1.456568in}%
\pgfsys@useobject{currentmarker}{}%
\end{pgfscope}%
\begin{pgfscope}%
\pgfsys@transformshift{1.783943in}{1.470427in}%
\pgfsys@useobject{currentmarker}{}%
\end{pgfscope}%
\begin{pgfscope}%
\pgfsys@transformshift{1.816820in}{1.486108in}%
\pgfsys@useobject{currentmarker}{}%
\end{pgfscope}%
\begin{pgfscope}%
\pgfsys@transformshift{1.849697in}{1.387046in}%
\pgfsys@useobject{currentmarker}{}%
\end{pgfscope}%
\begin{pgfscope}%
\pgfsys@transformshift{1.882574in}{1.474249in}%
\pgfsys@useobject{currentmarker}{}%
\end{pgfscope}%
\begin{pgfscope}%
\pgfsys@transformshift{1.915452in}{1.543938in}%
\pgfsys@useobject{currentmarker}{}%
\end{pgfscope}%
\begin{pgfscope}%
\pgfsys@transformshift{1.948329in}{1.400214in}%
\pgfsys@useobject{currentmarker}{}%
\end{pgfscope}%
\begin{pgfscope}%
\pgfsys@transformshift{1.981206in}{1.473285in}%
\pgfsys@useobject{currentmarker}{}%
\end{pgfscope}%
\begin{pgfscope}%
\pgfsys@transformshift{2.014084in}{1.559249in}%
\pgfsys@useobject{currentmarker}{}%
\end{pgfscope}%
\begin{pgfscope}%
\pgfsys@transformshift{2.046961in}{1.450317in}%
\pgfsys@useobject{currentmarker}{}%
\end{pgfscope}%
\begin{pgfscope}%
\pgfsys@transformshift{2.079838in}{1.517043in}%
\pgfsys@useobject{currentmarker}{}%
\end{pgfscope}%
\begin{pgfscope}%
\pgfsys@transformshift{2.112715in}{1.279921in}%
\pgfsys@useobject{currentmarker}{}%
\end{pgfscope}%
\begin{pgfscope}%
\pgfsys@transformshift{2.145593in}{1.563847in}%
\pgfsys@useobject{currentmarker}{}%
\end{pgfscope}%
\begin{pgfscope}%
\pgfsys@transformshift{2.178470in}{1.627030in}%
\pgfsys@useobject{currentmarker}{}%
\end{pgfscope}%
\begin{pgfscope}%
\pgfsys@transformshift{2.211347in}{1.540914in}%
\pgfsys@useobject{currentmarker}{}%
\end{pgfscope}%
\begin{pgfscope}%
\pgfsys@transformshift{2.244224in}{1.598344in}%
\pgfsys@useobject{currentmarker}{}%
\end{pgfscope}%
\begin{pgfscope}%
\pgfsys@transformshift{2.277102in}{1.625927in}%
\pgfsys@useobject{currentmarker}{}%
\end{pgfscope}%
\begin{pgfscope}%
\pgfsys@transformshift{2.309979in}{1.509080in}%
\pgfsys@useobject{currentmarker}{}%
\end{pgfscope}%
\begin{pgfscope}%
\pgfsys@transformshift{2.342856in}{1.418203in}%
\pgfsys@useobject{currentmarker}{}%
\end{pgfscope}%
\begin{pgfscope}%
\pgfsys@transformshift{2.375733in}{1.442334in}%
\pgfsys@useobject{currentmarker}{}%
\end{pgfscope}%
\begin{pgfscope}%
\pgfsys@transformshift{2.408611in}{1.325847in}%
\pgfsys@useobject{currentmarker}{}%
\end{pgfscope}%
\begin{pgfscope}%
\pgfsys@transformshift{2.441488in}{1.414076in}%
\pgfsys@useobject{currentmarker}{}%
\end{pgfscope}%
\begin{pgfscope}%
\pgfsys@transformshift{2.474365in}{1.388422in}%
\pgfsys@useobject{currentmarker}{}%
\end{pgfscope}%
\begin{pgfscope}%
\pgfsys@transformshift{2.507242in}{1.373400in}%
\pgfsys@useobject{currentmarker}{}%
\end{pgfscope}%
\begin{pgfscope}%
\pgfsys@transformshift{2.540120in}{1.407424in}%
\pgfsys@useobject{currentmarker}{}%
\end{pgfscope}%
\begin{pgfscope}%
\pgfsys@transformshift{2.572997in}{1.519522in}%
\pgfsys@useobject{currentmarker}{}%
\end{pgfscope}%
\begin{pgfscope}%
\pgfsys@transformshift{2.605874in}{1.621883in}%
\pgfsys@useobject{currentmarker}{}%
\end{pgfscope}%
\begin{pgfscope}%
\pgfsys@transformshift{2.638752in}{1.479742in}%
\pgfsys@useobject{currentmarker}{}%
\end{pgfscope}%
\begin{pgfscope}%
\pgfsys@transformshift{2.671629in}{1.530424in}%
\pgfsys@useobject{currentmarker}{}%
\end{pgfscope}%
\begin{pgfscope}%
\pgfsys@transformshift{2.704506in}{1.422255in}%
\pgfsys@useobject{currentmarker}{}%
\end{pgfscope}%
\begin{pgfscope}%
\pgfsys@transformshift{2.737383in}{1.487381in}%
\pgfsys@useobject{currentmarker}{}%
\end{pgfscope}%
\begin{pgfscope}%
\pgfsys@transformshift{2.770261in}{1.418764in}%
\pgfsys@useobject{currentmarker}{}%
\end{pgfscope}%
\begin{pgfscope}%
\pgfsys@transformshift{2.803138in}{1.433356in}%
\pgfsys@useobject{currentmarker}{}%
\end{pgfscope}%
\begin{pgfscope}%
\pgfsys@transformshift{2.836015in}{1.602235in}%
\pgfsys@useobject{currentmarker}{}%
\end{pgfscope}%
\begin{pgfscope}%
\pgfsys@transformshift{2.868892in}{1.458468in}%
\pgfsys@useobject{currentmarker}{}%
\end{pgfscope}%
\begin{pgfscope}%
\pgfsys@transformshift{2.901770in}{1.721374in}%
\pgfsys@useobject{currentmarker}{}%
\end{pgfscope}%
\begin{pgfscope}%
\pgfsys@transformshift{2.934647in}{1.532144in}%
\pgfsys@useobject{currentmarker}{}%
\end{pgfscope}%
\begin{pgfscope}%
\pgfsys@transformshift{2.967524in}{1.440008in}%
\pgfsys@useobject{currentmarker}{}%
\end{pgfscope}%
\begin{pgfscope}%
\pgfsys@transformshift{3.000401in}{1.588958in}%
\pgfsys@useobject{currentmarker}{}%
\end{pgfscope}%
\begin{pgfscope}%
\pgfsys@transformshift{3.033279in}{1.416950in}%
\pgfsys@useobject{currentmarker}{}%
\end{pgfscope}%
\begin{pgfscope}%
\pgfsys@transformshift{3.066156in}{1.265500in}%
\pgfsys@useobject{currentmarker}{}%
\end{pgfscope}%
\begin{pgfscope}%
\pgfsys@transformshift{3.099033in}{1.538302in}%
\pgfsys@useobject{currentmarker}{}%
\end{pgfscope}%
\begin{pgfscope}%
\pgfsys@transformshift{3.131910in}{1.475912in}%
\pgfsys@useobject{currentmarker}{}%
\end{pgfscope}%
\begin{pgfscope}%
\pgfsys@transformshift{3.164788in}{1.262426in}%
\pgfsys@useobject{currentmarker}{}%
\end{pgfscope}%
\begin{pgfscope}%
\pgfsys@transformshift{3.197665in}{1.364994in}%
\pgfsys@useobject{currentmarker}{}%
\end{pgfscope}%
\begin{pgfscope}%
\pgfsys@transformshift{3.230542in}{1.585475in}%
\pgfsys@useobject{currentmarker}{}%
\end{pgfscope}%
\begin{pgfscope}%
\pgfsys@transformshift{3.263420in}{1.527685in}%
\pgfsys@useobject{currentmarker}{}%
\end{pgfscope}%
\begin{pgfscope}%
\pgfsys@transformshift{3.296297in}{1.523893in}%
\pgfsys@useobject{currentmarker}{}%
\end{pgfscope}%
\begin{pgfscope}%
\pgfsys@transformshift{3.329174in}{1.698620in}%
\pgfsys@useobject{currentmarker}{}%
\end{pgfscope}%
\begin{pgfscope}%
\pgfsys@transformshift{3.362051in}{1.830490in}%
\pgfsys@useobject{currentmarker}{}%
\end{pgfscope}%
\begin{pgfscope}%
\pgfsys@transformshift{3.394929in}{1.648711in}%
\pgfsys@useobject{currentmarker}{}%
\end{pgfscope}%
\begin{pgfscope}%
\pgfsys@transformshift{3.427806in}{1.723186in}%
\pgfsys@useobject{currentmarker}{}%
\end{pgfscope}%
\begin{pgfscope}%
\pgfsys@transformshift{3.460683in}{1.480381in}%
\pgfsys@useobject{currentmarker}{}%
\end{pgfscope}%
\begin{pgfscope}%
\pgfsys@transformshift{3.493560in}{1.409030in}%
\pgfsys@useobject{currentmarker}{}%
\end{pgfscope}%
\begin{pgfscope}%
\pgfsys@transformshift{3.526438in}{1.439346in}%
\pgfsys@useobject{currentmarker}{}%
\end{pgfscope}%
\begin{pgfscope}%
\pgfsys@transformshift{3.559315in}{1.450884in}%
\pgfsys@useobject{currentmarker}{}%
\end{pgfscope}%
\begin{pgfscope}%
\pgfsys@transformshift{3.592192in}{1.404572in}%
\pgfsys@useobject{currentmarker}{}%
\end{pgfscope}%
\begin{pgfscope}%
\pgfsys@transformshift{3.625069in}{1.372416in}%
\pgfsys@useobject{currentmarker}{}%
\end{pgfscope}%
\begin{pgfscope}%
\pgfsys@transformshift{3.657947in}{1.295153in}%
\pgfsys@useobject{currentmarker}{}%
\end{pgfscope}%
\begin{pgfscope}%
\pgfsys@transformshift{3.690824in}{1.434506in}%
\pgfsys@useobject{currentmarker}{}%
\end{pgfscope}%
\begin{pgfscope}%
\pgfsys@transformshift{3.723701in}{1.400582in}%
\pgfsys@useobject{currentmarker}{}%
\end{pgfscope}%
\begin{pgfscope}%
\pgfsys@transformshift{3.756578in}{1.809562in}%
\pgfsys@useobject{currentmarker}{}%
\end{pgfscope}%
\begin{pgfscope}%
\pgfsys@transformshift{3.789456in}{1.516574in}%
\pgfsys@useobject{currentmarker}{}%
\end{pgfscope}%
\begin{pgfscope}%
\pgfsys@transformshift{3.822333in}{1.700337in}%
\pgfsys@useobject{currentmarker}{}%
\end{pgfscope}%
\begin{pgfscope}%
\pgfsys@transformshift{3.855210in}{1.443035in}%
\pgfsys@useobject{currentmarker}{}%
\end{pgfscope}%
\begin{pgfscope}%
\pgfsys@transformshift{3.888088in}{1.552121in}%
\pgfsys@useobject{currentmarker}{}%
\end{pgfscope}%
\begin{pgfscope}%
\pgfsys@transformshift{3.920965in}{1.425027in}%
\pgfsys@useobject{currentmarker}{}%
\end{pgfscope}%
\begin{pgfscope}%
\pgfsys@transformshift{3.953842in}{1.356354in}%
\pgfsys@useobject{currentmarker}{}%
\end{pgfscope}%
\begin{pgfscope}%
\pgfsys@transformshift{3.986719in}{1.457407in}%
\pgfsys@useobject{currentmarker}{}%
\end{pgfscope}%
\begin{pgfscope}%
\pgfsys@transformshift{4.019597in}{1.398013in}%
\pgfsys@useobject{currentmarker}{}%
\end{pgfscope}%
\begin{pgfscope}%
\pgfsys@transformshift{4.052474in}{1.512425in}%
\pgfsys@useobject{currentmarker}{}%
\end{pgfscope}%
\begin{pgfscope}%
\pgfsys@transformshift{4.085351in}{1.506410in}%
\pgfsys@useobject{currentmarker}{}%
\end{pgfscope}%
\begin{pgfscope}%
\pgfsys@transformshift{4.118228in}{1.531083in}%
\pgfsys@useobject{currentmarker}{}%
\end{pgfscope}%
\begin{pgfscope}%
\pgfsys@transformshift{4.151106in}{1.638103in}%
\pgfsys@useobject{currentmarker}{}%
\end{pgfscope}%
\begin{pgfscope}%
\pgfsys@transformshift{4.183983in}{1.406549in}%
\pgfsys@useobject{currentmarker}{}%
\end{pgfscope}%
\begin{pgfscope}%
\pgfsys@transformshift{4.216860in}{1.302441in}%
\pgfsys@useobject{currentmarker}{}%
\end{pgfscope}%
\begin{pgfscope}%
\pgfsys@transformshift{4.249737in}{1.466317in}%
\pgfsys@useobject{currentmarker}{}%
\end{pgfscope}%
\begin{pgfscope}%
\pgfsys@transformshift{4.282615in}{1.614062in}%
\pgfsys@useobject{currentmarker}{}%
\end{pgfscope}%
\begin{pgfscope}%
\pgfsys@transformshift{4.315492in}{1.609928in}%
\pgfsys@useobject{currentmarker}{}%
\end{pgfscope}%
\begin{pgfscope}%
\pgfsys@transformshift{4.348369in}{1.343779in}%
\pgfsys@useobject{currentmarker}{}%
\end{pgfscope}%
\begin{pgfscope}%
\pgfsys@transformshift{4.381246in}{1.547895in}%
\pgfsys@useobject{currentmarker}{}%
\end{pgfscope}%
\begin{pgfscope}%
\pgfsys@transformshift{4.414124in}{1.540659in}%
\pgfsys@useobject{currentmarker}{}%
\end{pgfscope}%
\begin{pgfscope}%
\pgfsys@transformshift{4.447001in}{1.383011in}%
\pgfsys@useobject{currentmarker}{}%
\end{pgfscope}%
\begin{pgfscope}%
\pgfsys@transformshift{4.479878in}{1.537984in}%
\pgfsys@useobject{currentmarker}{}%
\end{pgfscope}%
\begin{pgfscope}%
\pgfsys@transformshift{4.512756in}{1.566064in}%
\pgfsys@useobject{currentmarker}{}%
\end{pgfscope}%
\begin{pgfscope}%
\pgfsys@transformshift{4.545633in}{1.480616in}%
\pgfsys@useobject{currentmarker}{}%
\end{pgfscope}%
\begin{pgfscope}%
\pgfsys@transformshift{4.578510in}{1.546579in}%
\pgfsys@useobject{currentmarker}{}%
\end{pgfscope}%
\begin{pgfscope}%
\pgfsys@transformshift{4.611387in}{1.630558in}%
\pgfsys@useobject{currentmarker}{}%
\end{pgfscope}%
\begin{pgfscope}%
\pgfsys@transformshift{4.644265in}{1.547124in}%
\pgfsys@useobject{currentmarker}{}%
\end{pgfscope}%
\begin{pgfscope}%
\pgfsys@transformshift{4.677142in}{1.820655in}%
\pgfsys@useobject{currentmarker}{}%
\end{pgfscope}%
\begin{pgfscope}%
\pgfsys@transformshift{4.710019in}{1.738104in}%
\pgfsys@useobject{currentmarker}{}%
\end{pgfscope}%
\begin{pgfscope}%
\pgfsys@transformshift{4.742896in}{1.833942in}%
\pgfsys@useobject{currentmarker}{}%
\end{pgfscope}%
\begin{pgfscope}%
\pgfsys@transformshift{4.775774in}{1.536801in}%
\pgfsys@useobject{currentmarker}{}%
\end{pgfscope}%
\begin{pgfscope}%
\pgfsys@transformshift{4.808651in}{1.632510in}%
\pgfsys@useobject{currentmarker}{}%
\end{pgfscope}%
\begin{pgfscope}%
\pgfsys@transformshift{4.841528in}{1.281835in}%
\pgfsys@useobject{currentmarker}{}%
\end{pgfscope}%
\begin{pgfscope}%
\pgfsys@transformshift{4.874405in}{1.305648in}%
\pgfsys@useobject{currentmarker}{}%
\end{pgfscope}%
\begin{pgfscope}%
\pgfsys@transformshift{4.907283in}{1.528960in}%
\pgfsys@useobject{currentmarker}{}%
\end{pgfscope}%
\begin{pgfscope}%
\pgfsys@transformshift{4.940160in}{1.372129in}%
\pgfsys@useobject{currentmarker}{}%
\end{pgfscope}%
\begin{pgfscope}%
\pgfsys@transformshift{4.973037in}{1.346997in}%
\pgfsys@useobject{currentmarker}{}%
\end{pgfscope}%
\begin{pgfscope}%
\pgfsys@transformshift{5.005915in}{1.424545in}%
\pgfsys@useobject{currentmarker}{}%
\end{pgfscope}%
\begin{pgfscope}%
\pgfsys@transformshift{5.038792in}{1.479823in}%
\pgfsys@useobject{currentmarker}{}%
\end{pgfscope}%
\begin{pgfscope}%
\pgfsys@transformshift{5.071669in}{1.428409in}%
\pgfsys@useobject{currentmarker}{}%
\end{pgfscope}%
\begin{pgfscope}%
\pgfsys@transformshift{5.104546in}{1.281010in}%
\pgfsys@useobject{currentmarker}{}%
\end{pgfscope}%
\begin{pgfscope}%
\pgfsys@transformshift{5.137424in}{1.263190in}%
\pgfsys@useobject{currentmarker}{}%
\end{pgfscope}%
\begin{pgfscope}%
\pgfsys@transformshift{5.170301in}{1.510390in}%
\pgfsys@useobject{currentmarker}{}%
\end{pgfscope}%
\begin{pgfscope}%
\pgfsys@transformshift{5.203178in}{1.481836in}%
\pgfsys@useobject{currentmarker}{}%
\end{pgfscope}%
\begin{pgfscope}%
\pgfsys@transformshift{5.236055in}{1.489975in}%
\pgfsys@useobject{currentmarker}{}%
\end{pgfscope}%
\begin{pgfscope}%
\pgfsys@transformshift{5.268933in}{1.532305in}%
\pgfsys@useobject{currentmarker}{}%
\end{pgfscope}%
\begin{pgfscope}%
\pgfsys@transformshift{5.301810in}{1.615376in}%
\pgfsys@useobject{currentmarker}{}%
\end{pgfscope}%
\begin{pgfscope}%
\pgfsys@transformshift{5.334687in}{1.403772in}%
\pgfsys@useobject{currentmarker}{}%
\end{pgfscope}%
\begin{pgfscope}%
\pgfsys@transformshift{5.367564in}{1.489565in}%
\pgfsys@useobject{currentmarker}{}%
\end{pgfscope}%
\begin{pgfscope}%
\pgfsys@transformshift{5.400442in}{1.535190in}%
\pgfsys@useobject{currentmarker}{}%
\end{pgfscope}%
\begin{pgfscope}%
\pgfsys@transformshift{5.433319in}{1.583673in}%
\pgfsys@useobject{currentmarker}{}%
\end{pgfscope}%
\begin{pgfscope}%
\pgfsys@transformshift{5.466196in}{1.639349in}%
\pgfsys@useobject{currentmarker}{}%
\end{pgfscope}%
\begin{pgfscope}%
\pgfsys@transformshift{5.499073in}{1.399440in}%
\pgfsys@useobject{currentmarker}{}%
\end{pgfscope}%
\begin{pgfscope}%
\pgfsys@transformshift{5.531951in}{1.374624in}%
\pgfsys@useobject{currentmarker}{}%
\end{pgfscope}%
\begin{pgfscope}%
\pgfsys@transformshift{5.564828in}{1.552993in}%
\pgfsys@useobject{currentmarker}{}%
\end{pgfscope}%
\begin{pgfscope}%
\pgfsys@transformshift{5.597705in}{1.416895in}%
\pgfsys@useobject{currentmarker}{}%
\end{pgfscope}%
\begin{pgfscope}%
\pgfsys@transformshift{5.630583in}{1.687446in}%
\pgfsys@useobject{currentmarker}{}%
\end{pgfscope}%
\begin{pgfscope}%
\pgfsys@transformshift{5.663460in}{1.531821in}%
\pgfsys@useobject{currentmarker}{}%
\end{pgfscope}%
\begin{pgfscope}%
\pgfsys@transformshift{5.696337in}{1.503682in}%
\pgfsys@useobject{currentmarker}{}%
\end{pgfscope}%
\begin{pgfscope}%
\pgfsys@transformshift{5.729214in}{1.630379in}%
\pgfsys@useobject{currentmarker}{}%
\end{pgfscope}%
\begin{pgfscope}%
\pgfsys@transformshift{5.762092in}{1.566363in}%
\pgfsys@useobject{currentmarker}{}%
\end{pgfscope}%
\begin{pgfscope}%
\pgfsys@transformshift{5.794969in}{1.801658in}%
\pgfsys@useobject{currentmarker}{}%
\end{pgfscope}%
\begin{pgfscope}%
\pgfsys@transformshift{5.827846in}{1.621170in}%
\pgfsys@useobject{currentmarker}{}%
\end{pgfscope}%
\begin{pgfscope}%
\pgfsys@transformshift{5.860723in}{1.592580in}%
\pgfsys@useobject{currentmarker}{}%
\end{pgfscope}%
\begin{pgfscope}%
\pgfsys@transformshift{5.893601in}{1.695231in}%
\pgfsys@useobject{currentmarker}{}%
\end{pgfscope}%
\begin{pgfscope}%
\pgfsys@transformshift{5.926478in}{1.326513in}%
\pgfsys@useobject{currentmarker}{}%
\end{pgfscope}%
\begin{pgfscope}%
\pgfsys@transformshift{5.959355in}{1.447075in}%
\pgfsys@useobject{currentmarker}{}%
\end{pgfscope}%
\begin{pgfscope}%
\pgfsys@transformshift{5.992232in}{1.510829in}%
\pgfsys@useobject{currentmarker}{}%
\end{pgfscope}%
\begin{pgfscope}%
\pgfsys@transformshift{6.025110in}{1.426230in}%
\pgfsys@useobject{currentmarker}{}%
\end{pgfscope}%
\begin{pgfscope}%
\pgfsys@transformshift{6.057987in}{1.655153in}%
\pgfsys@useobject{currentmarker}{}%
\end{pgfscope}%
\begin{pgfscope}%
\pgfsys@transformshift{6.090864in}{1.563899in}%
\pgfsys@useobject{currentmarker}{}%
\end{pgfscope}%
\begin{pgfscope}%
\pgfsys@transformshift{6.123741in}{1.616373in}%
\pgfsys@useobject{currentmarker}{}%
\end{pgfscope}%
\begin{pgfscope}%
\pgfsys@transformshift{6.156619in}{1.693915in}%
\pgfsys@useobject{currentmarker}{}%
\end{pgfscope}%
\begin{pgfscope}%
\pgfsys@transformshift{6.189496in}{1.511811in}%
\pgfsys@useobject{currentmarker}{}%
\end{pgfscope}%
\begin{pgfscope}%
\pgfsys@transformshift{6.222373in}{1.487928in}%
\pgfsys@useobject{currentmarker}{}%
\end{pgfscope}%
\begin{pgfscope}%
\pgfsys@transformshift{6.255251in}{1.410120in}%
\pgfsys@useobject{currentmarker}{}%
\end{pgfscope}%
\begin{pgfscope}%
\pgfsys@transformshift{6.288128in}{1.467343in}%
\pgfsys@useobject{currentmarker}{}%
\end{pgfscope}%
\begin{pgfscope}%
\pgfsys@transformshift{6.321005in}{1.523587in}%
\pgfsys@useobject{currentmarker}{}%
\end{pgfscope}%
\begin{pgfscope}%
\pgfsys@transformshift{6.353882in}{1.446132in}%
\pgfsys@useobject{currentmarker}{}%
\end{pgfscope}%
\begin{pgfscope}%
\pgfsys@transformshift{6.386760in}{1.421214in}%
\pgfsys@useobject{currentmarker}{}%
\end{pgfscope}%
\begin{pgfscope}%
\pgfsys@transformshift{6.419637in}{1.564411in}%
\pgfsys@useobject{currentmarker}{}%
\end{pgfscope}%
\begin{pgfscope}%
\pgfsys@transformshift{6.452514in}{1.594809in}%
\pgfsys@useobject{currentmarker}{}%
\end{pgfscope}%
\begin{pgfscope}%
\pgfsys@transformshift{6.485391in}{1.563035in}%
\pgfsys@useobject{currentmarker}{}%
\end{pgfscope}%
\begin{pgfscope}%
\pgfsys@transformshift{6.518269in}{1.529499in}%
\pgfsys@useobject{currentmarker}{}%
\end{pgfscope}%
\begin{pgfscope}%
\pgfsys@transformshift{6.551146in}{1.602508in}%
\pgfsys@useobject{currentmarker}{}%
\end{pgfscope}%
\begin{pgfscope}%
\pgfsys@transformshift{6.584023in}{1.703608in}%
\pgfsys@useobject{currentmarker}{}%
\end{pgfscope}%
\begin{pgfscope}%
\pgfsys@transformshift{6.616900in}{1.554706in}%
\pgfsys@useobject{currentmarker}{}%
\end{pgfscope}%
\begin{pgfscope}%
\pgfsys@transformshift{6.649778in}{1.550412in}%
\pgfsys@useobject{currentmarker}{}%
\end{pgfscope}%
\begin{pgfscope}%
\pgfsys@transformshift{6.682655in}{1.525308in}%
\pgfsys@useobject{currentmarker}{}%
\end{pgfscope}%
\begin{pgfscope}%
\pgfsys@transformshift{6.715532in}{1.480221in}%
\pgfsys@useobject{currentmarker}{}%
\end{pgfscope}%
\begin{pgfscope}%
\pgfsys@transformshift{6.748409in}{1.351760in}%
\pgfsys@useobject{currentmarker}{}%
\end{pgfscope}%
\begin{pgfscope}%
\pgfsys@transformshift{6.781287in}{1.510929in}%
\pgfsys@useobject{currentmarker}{}%
\end{pgfscope}%
\begin{pgfscope}%
\pgfsys@transformshift{6.814164in}{1.577139in}%
\pgfsys@useobject{currentmarker}{}%
\end{pgfscope}%
\begin{pgfscope}%
\pgfsys@transformshift{6.847041in}{1.548406in}%
\pgfsys@useobject{currentmarker}{}%
\end{pgfscope}%
\begin{pgfscope}%
\pgfsys@transformshift{6.879919in}{1.382558in}%
\pgfsys@useobject{currentmarker}{}%
\end{pgfscope}%
\begin{pgfscope}%
\pgfsys@transformshift{6.912796in}{1.393194in}%
\pgfsys@useobject{currentmarker}{}%
\end{pgfscope}%
\begin{pgfscope}%
\pgfsys@transformshift{6.945673in}{1.556865in}%
\pgfsys@useobject{currentmarker}{}%
\end{pgfscope}%
\begin{pgfscope}%
\pgfsys@transformshift{6.978550in}{1.590123in}%
\pgfsys@useobject{currentmarker}{}%
\end{pgfscope}%
\begin{pgfscope}%
\pgfsys@transformshift{7.011428in}{1.531788in}%
\pgfsys@useobject{currentmarker}{}%
\end{pgfscope}%
\begin{pgfscope}%
\pgfsys@transformshift{7.044305in}{1.611558in}%
\pgfsys@useobject{currentmarker}{}%
\end{pgfscope}%
\begin{pgfscope}%
\pgfsys@transformshift{7.077182in}{1.586267in}%
\pgfsys@useobject{currentmarker}{}%
\end{pgfscope}%
\begin{pgfscope}%
\pgfsys@transformshift{7.110059in}{1.520359in}%
\pgfsys@useobject{currentmarker}{}%
\end{pgfscope}%
\begin{pgfscope}%
\pgfsys@transformshift{7.142937in}{1.482812in}%
\pgfsys@useobject{currentmarker}{}%
\end{pgfscope}%
\begin{pgfscope}%
\pgfsys@transformshift{7.175814in}{1.538106in}%
\pgfsys@useobject{currentmarker}{}%
\end{pgfscope}%
\begin{pgfscope}%
\pgfsys@transformshift{7.208691in}{1.440365in}%
\pgfsys@useobject{currentmarker}{}%
\end{pgfscope}%
\begin{pgfscope}%
\pgfsys@transformshift{7.241568in}{1.593965in}%
\pgfsys@useobject{currentmarker}{}%
\end{pgfscope}%
\begin{pgfscope}%
\pgfsys@transformshift{7.274446in}{1.412554in}%
\pgfsys@useobject{currentmarker}{}%
\end{pgfscope}%
\begin{pgfscope}%
\pgfsys@transformshift{7.307323in}{1.634478in}%
\pgfsys@useobject{currentmarker}{}%
\end{pgfscope}%
\begin{pgfscope}%
\pgfsys@transformshift{7.340200in}{1.569986in}%
\pgfsys@useobject{currentmarker}{}%
\end{pgfscope}%
\begin{pgfscope}%
\pgfsys@transformshift{7.373077in}{1.549846in}%
\pgfsys@useobject{currentmarker}{}%
\end{pgfscope}%
\begin{pgfscope}%
\pgfsys@transformshift{7.405955in}{1.556321in}%
\pgfsys@useobject{currentmarker}{}%
\end{pgfscope}%
\begin{pgfscope}%
\pgfsys@transformshift{7.438832in}{1.344537in}%
\pgfsys@useobject{currentmarker}{}%
\end{pgfscope}%
\begin{pgfscope}%
\pgfsys@transformshift{7.471709in}{1.484251in}%
\pgfsys@useobject{currentmarker}{}%
\end{pgfscope}%
\begin{pgfscope}%
\pgfsys@transformshift{7.504587in}{1.398183in}%
\pgfsys@useobject{currentmarker}{}%
\end{pgfscope}%
\begin{pgfscope}%
\pgfsys@transformshift{7.537464in}{1.448220in}%
\pgfsys@useobject{currentmarker}{}%
\end{pgfscope}%
\begin{pgfscope}%
\pgfsys@transformshift{7.570341in}{1.440258in}%
\pgfsys@useobject{currentmarker}{}%
\end{pgfscope}%
\begin{pgfscope}%
\pgfsys@transformshift{7.603218in}{1.489606in}%
\pgfsys@useobject{currentmarker}{}%
\end{pgfscope}%
\begin{pgfscope}%
\pgfsys@transformshift{7.636096in}{1.585766in}%
\pgfsys@useobject{currentmarker}{}%
\end{pgfscope}%
\begin{pgfscope}%
\pgfsys@transformshift{7.668973in}{1.437107in}%
\pgfsys@useobject{currentmarker}{}%
\end{pgfscope}%
\begin{pgfscope}%
\pgfsys@transformshift{7.701850in}{1.537641in}%
\pgfsys@useobject{currentmarker}{}%
\end{pgfscope}%
\begin{pgfscope}%
\pgfsys@transformshift{7.734727in}{1.668849in}%
\pgfsys@useobject{currentmarker}{}%
\end{pgfscope}%
\begin{pgfscope}%
\pgfsys@transformshift{7.767605in}{1.503032in}%
\pgfsys@useobject{currentmarker}{}%
\end{pgfscope}%
\begin{pgfscope}%
\pgfsys@transformshift{7.800482in}{1.406186in}%
\pgfsys@useobject{currentmarker}{}%
\end{pgfscope}%
\begin{pgfscope}%
\pgfsys@transformshift{7.833359in}{1.437234in}%
\pgfsys@useobject{currentmarker}{}%
\end{pgfscope}%
\begin{pgfscope}%
\pgfsys@transformshift{7.866236in}{1.447312in}%
\pgfsys@useobject{currentmarker}{}%
\end{pgfscope}%
\begin{pgfscope}%
\pgfsys@transformshift{7.899114in}{1.678783in}%
\pgfsys@useobject{currentmarker}{}%
\end{pgfscope}%
\begin{pgfscope}%
\pgfsys@transformshift{7.931991in}{1.488904in}%
\pgfsys@useobject{currentmarker}{}%
\end{pgfscope}%
\begin{pgfscope}%
\pgfsys@transformshift{7.964868in}{1.739942in}%
\pgfsys@useobject{currentmarker}{}%
\end{pgfscope}%
\begin{pgfscope}%
\pgfsys@transformshift{7.997746in}{1.488929in}%
\pgfsys@useobject{currentmarker}{}%
\end{pgfscope}%
\begin{pgfscope}%
\pgfsys@transformshift{8.030623in}{1.623781in}%
\pgfsys@useobject{currentmarker}{}%
\end{pgfscope}%
\begin{pgfscope}%
\pgfsys@transformshift{8.063500in}{1.510912in}%
\pgfsys@useobject{currentmarker}{}%
\end{pgfscope}%
\begin{pgfscope}%
\pgfsys@transformshift{8.096377in}{1.482168in}%
\pgfsys@useobject{currentmarker}{}%
\end{pgfscope}%
\begin{pgfscope}%
\pgfsys@transformshift{8.129255in}{1.574281in}%
\pgfsys@useobject{currentmarker}{}%
\end{pgfscope}%
\begin{pgfscope}%
\pgfsys@transformshift{8.162132in}{1.586805in}%
\pgfsys@useobject{currentmarker}{}%
\end{pgfscope}%
\begin{pgfscope}%
\pgfsys@transformshift{8.195009in}{1.434118in}%
\pgfsys@useobject{currentmarker}{}%
\end{pgfscope}%
\begin{pgfscope}%
\pgfsys@transformshift{8.227886in}{1.317342in}%
\pgfsys@useobject{currentmarker}{}%
\end{pgfscope}%
\begin{pgfscope}%
\pgfsys@transformshift{8.260764in}{1.435273in}%
\pgfsys@useobject{currentmarker}{}%
\end{pgfscope}%
\begin{pgfscope}%
\pgfsys@transformshift{8.293641in}{1.649770in}%
\pgfsys@useobject{currentmarker}{}%
\end{pgfscope}%
\begin{pgfscope}%
\pgfsys@transformshift{8.326518in}{1.424047in}%
\pgfsys@useobject{currentmarker}{}%
\end{pgfscope}%
\begin{pgfscope}%
\pgfsys@transformshift{8.359395in}{1.632722in}%
\pgfsys@useobject{currentmarker}{}%
\end{pgfscope}%
\begin{pgfscope}%
\pgfsys@transformshift{8.392273in}{1.559819in}%
\pgfsys@useobject{currentmarker}{}%
\end{pgfscope}%
\begin{pgfscope}%
\pgfsys@transformshift{8.425150in}{1.521496in}%
\pgfsys@useobject{currentmarker}{}%
\end{pgfscope}%
\begin{pgfscope}%
\pgfsys@transformshift{8.458027in}{1.560681in}%
\pgfsys@useobject{currentmarker}{}%
\end{pgfscope}%
\begin{pgfscope}%
\pgfsys@transformshift{8.490904in}{1.647462in}%
\pgfsys@useobject{currentmarker}{}%
\end{pgfscope}%
\begin{pgfscope}%
\pgfsys@transformshift{8.523782in}{1.438391in}%
\pgfsys@useobject{currentmarker}{}%
\end{pgfscope}%
\begin{pgfscope}%
\pgfsys@transformshift{8.556659in}{1.591213in}%
\pgfsys@useobject{currentmarker}{}%
\end{pgfscope}%
\begin{pgfscope}%
\pgfsys@transformshift{8.589536in}{1.419238in}%
\pgfsys@useobject{currentmarker}{}%
\end{pgfscope}%
\begin{pgfscope}%
\pgfsys@transformshift{8.622414in}{1.496451in}%
\pgfsys@useobject{currentmarker}{}%
\end{pgfscope}%
\begin{pgfscope}%
\pgfsys@transformshift{8.655291in}{1.387299in}%
\pgfsys@useobject{currentmarker}{}%
\end{pgfscope}%
\begin{pgfscope}%
\pgfsys@transformshift{8.688168in}{1.531128in}%
\pgfsys@useobject{currentmarker}{}%
\end{pgfscope}%
\begin{pgfscope}%
\pgfsys@transformshift{8.721045in}{1.497037in}%
\pgfsys@useobject{currentmarker}{}%
\end{pgfscope}%
\begin{pgfscope}%
\pgfsys@transformshift{8.753923in}{1.299627in}%
\pgfsys@useobject{currentmarker}{}%
\end{pgfscope}%
\begin{pgfscope}%
\pgfsys@transformshift{8.786800in}{1.624521in}%
\pgfsys@useobject{currentmarker}{}%
\end{pgfscope}%
\begin{pgfscope}%
\pgfsys@transformshift{8.819677in}{1.607025in}%
\pgfsys@useobject{currentmarker}{}%
\end{pgfscope}%
\begin{pgfscope}%
\pgfsys@transformshift{8.852554in}{1.393544in}%
\pgfsys@useobject{currentmarker}{}%
\end{pgfscope}%
\begin{pgfscope}%
\pgfsys@transformshift{8.885432in}{1.465768in}%
\pgfsys@useobject{currentmarker}{}%
\end{pgfscope}%
\begin{pgfscope}%
\pgfsys@transformshift{8.918309in}{1.409581in}%
\pgfsys@useobject{currentmarker}{}%
\end{pgfscope}%
\begin{pgfscope}%
\pgfsys@transformshift{8.951186in}{1.499076in}%
\pgfsys@useobject{currentmarker}{}%
\end{pgfscope}%
\begin{pgfscope}%
\pgfsys@transformshift{8.984063in}{1.516772in}%
\pgfsys@useobject{currentmarker}{}%
\end{pgfscope}%
\begin{pgfscope}%
\pgfsys@transformshift{9.016941in}{1.569399in}%
\pgfsys@useobject{currentmarker}{}%
\end{pgfscope}%
\begin{pgfscope}%
\pgfsys@transformshift{9.049818in}{1.221362in}%
\pgfsys@useobject{currentmarker}{}%
\end{pgfscope}%
\begin{pgfscope}%
\pgfsys@transformshift{9.082695in}{1.581846in}%
\pgfsys@useobject{currentmarker}{}%
\end{pgfscope}%
\begin{pgfscope}%
\pgfsys@transformshift{9.115572in}{1.464491in}%
\pgfsys@useobject{currentmarker}{}%
\end{pgfscope}%
\begin{pgfscope}%
\pgfsys@transformshift{9.148450in}{1.372681in}%
\pgfsys@useobject{currentmarker}{}%
\end{pgfscope}%
\begin{pgfscope}%
\pgfsys@transformshift{9.181327in}{1.405971in}%
\pgfsys@useobject{currentmarker}{}%
\end{pgfscope}%
\begin{pgfscope}%
\pgfsys@transformshift{9.214204in}{1.366870in}%
\pgfsys@useobject{currentmarker}{}%
\end{pgfscope}%
\begin{pgfscope}%
\pgfsys@transformshift{9.247082in}{1.600906in}%
\pgfsys@useobject{currentmarker}{}%
\end{pgfscope}%
\begin{pgfscope}%
\pgfsys@transformshift{9.279959in}{1.512572in}%
\pgfsys@useobject{currentmarker}{}%
\end{pgfscope}%
\begin{pgfscope}%
\pgfsys@transformshift{9.312836in}{1.485166in}%
\pgfsys@useobject{currentmarker}{}%
\end{pgfscope}%
\begin{pgfscope}%
\pgfsys@transformshift{9.345713in}{1.631092in}%
\pgfsys@useobject{currentmarker}{}%
\end{pgfscope}%
\begin{pgfscope}%
\pgfsys@transformshift{9.378591in}{1.498928in}%
\pgfsys@useobject{currentmarker}{}%
\end{pgfscope}%
\begin{pgfscope}%
\pgfsys@transformshift{9.411468in}{1.426098in}%
\pgfsys@useobject{currentmarker}{}%
\end{pgfscope}%
\begin{pgfscope}%
\pgfsys@transformshift{9.444345in}{1.439873in}%
\pgfsys@useobject{currentmarker}{}%
\end{pgfscope}%
\begin{pgfscope}%
\pgfsys@transformshift{9.477222in}{1.569596in}%
\pgfsys@useobject{currentmarker}{}%
\end{pgfscope}%
\begin{pgfscope}%
\pgfsys@transformshift{9.510100in}{1.383738in}%
\pgfsys@useobject{currentmarker}{}%
\end{pgfscope}%
\begin{pgfscope}%
\pgfsys@transformshift{9.542977in}{1.361238in}%
\pgfsys@useobject{currentmarker}{}%
\end{pgfscope}%
\begin{pgfscope}%
\pgfsys@transformshift{9.575854in}{1.489448in}%
\pgfsys@useobject{currentmarker}{}%
\end{pgfscope}%
\begin{pgfscope}%
\pgfsys@transformshift{9.608731in}{1.395034in}%
\pgfsys@useobject{currentmarker}{}%
\end{pgfscope}%
\begin{pgfscope}%
\pgfsys@transformshift{9.641609in}{1.456746in}%
\pgfsys@useobject{currentmarker}{}%
\end{pgfscope}%
\begin{pgfscope}%
\pgfsys@transformshift{9.674486in}{1.497169in}%
\pgfsys@useobject{currentmarker}{}%
\end{pgfscope}%
\begin{pgfscope}%
\pgfsys@transformshift{9.707363in}{1.354814in}%
\pgfsys@useobject{currentmarker}{}%
\end{pgfscope}%
\begin{pgfscope}%
\pgfsys@transformshift{9.740240in}{1.683439in}%
\pgfsys@useobject{currentmarker}{}%
\end{pgfscope}%
\begin{pgfscope}%
\pgfsys@transformshift{9.773118in}{1.456297in}%
\pgfsys@useobject{currentmarker}{}%
\end{pgfscope}%
\begin{pgfscope}%
\pgfsys@transformshift{9.805995in}{1.513445in}%
\pgfsys@useobject{currentmarker}{}%
\end{pgfscope}%
\begin{pgfscope}%
\pgfsys@transformshift{9.838872in}{1.480666in}%
\pgfsys@useobject{currentmarker}{}%
\end{pgfscope}%
\begin{pgfscope}%
\pgfsys@transformshift{9.871750in}{1.558101in}%
\pgfsys@useobject{currentmarker}{}%
\end{pgfscope}%
\begin{pgfscope}%
\pgfsys@transformshift{9.904627in}{1.486395in}%
\pgfsys@useobject{currentmarker}{}%
\end{pgfscope}%
\begin{pgfscope}%
\pgfsys@transformshift{9.937504in}{1.558807in}%
\pgfsys@useobject{currentmarker}{}%
\end{pgfscope}%
\begin{pgfscope}%
\pgfsys@transformshift{9.970381in}{1.634507in}%
\pgfsys@useobject{currentmarker}{}%
\end{pgfscope}%
\begin{pgfscope}%
\pgfsys@transformshift{10.003259in}{1.563748in}%
\pgfsys@useobject{currentmarker}{}%
\end{pgfscope}%
\begin{pgfscope}%
\pgfsys@transformshift{10.036136in}{1.449349in}%
\pgfsys@useobject{currentmarker}{}%
\end{pgfscope}%
\begin{pgfscope}%
\pgfsys@transformshift{10.069013in}{1.593716in}%
\pgfsys@useobject{currentmarker}{}%
\end{pgfscope}%
\begin{pgfscope}%
\pgfsys@transformshift{10.101890in}{1.547269in}%
\pgfsys@useobject{currentmarker}{}%
\end{pgfscope}%
\begin{pgfscope}%
\pgfsys@transformshift{10.134768in}{1.458477in}%
\pgfsys@useobject{currentmarker}{}%
\end{pgfscope}%
\begin{pgfscope}%
\pgfsys@transformshift{10.167645in}{1.398816in}%
\pgfsys@useobject{currentmarker}{}%
\end{pgfscope}%
\begin{pgfscope}%
\pgfsys@transformshift{10.200522in}{1.643958in}%
\pgfsys@useobject{currentmarker}{}%
\end{pgfscope}%
\begin{pgfscope}%
\pgfsys@transformshift{10.233399in}{1.515859in}%
\pgfsys@useobject{currentmarker}{}%
\end{pgfscope}%
\begin{pgfscope}%
\pgfsys@transformshift{10.266277in}{1.652299in}%
\pgfsys@useobject{currentmarker}{}%
\end{pgfscope}%
\begin{pgfscope}%
\pgfsys@transformshift{10.299154in}{1.336838in}%
\pgfsys@useobject{currentmarker}{}%
\end{pgfscope}%
\begin{pgfscope}%
\pgfsys@transformshift{10.332031in}{1.408825in}%
\pgfsys@useobject{currentmarker}{}%
\end{pgfscope}%
\begin{pgfscope}%
\pgfsys@transformshift{10.364909in}{1.351176in}%
\pgfsys@useobject{currentmarker}{}%
\end{pgfscope}%
\begin{pgfscope}%
\pgfsys@transformshift{10.397786in}{1.394835in}%
\pgfsys@useobject{currentmarker}{}%
\end{pgfscope}%
\begin{pgfscope}%
\pgfsys@transformshift{10.430663in}{1.422838in}%
\pgfsys@useobject{currentmarker}{}%
\end{pgfscope}%
\begin{pgfscope}%
\pgfsys@transformshift{10.463540in}{1.459485in}%
\pgfsys@useobject{currentmarker}{}%
\end{pgfscope}%
\begin{pgfscope}%
\pgfsys@transformshift{10.496418in}{1.573730in}%
\pgfsys@useobject{currentmarker}{}%
\end{pgfscope}%
\begin{pgfscope}%
\pgfsys@transformshift{10.529295in}{1.471158in}%
\pgfsys@useobject{currentmarker}{}%
\end{pgfscope}%
\begin{pgfscope}%
\pgfsys@transformshift{10.562172in}{1.607557in}%
\pgfsys@useobject{currentmarker}{}%
\end{pgfscope}%
\begin{pgfscope}%
\pgfsys@transformshift{10.595049in}{1.564763in}%
\pgfsys@useobject{currentmarker}{}%
\end{pgfscope}%
\begin{pgfscope}%
\pgfsys@transformshift{10.627927in}{1.505573in}%
\pgfsys@useobject{currentmarker}{}%
\end{pgfscope}%
\begin{pgfscope}%
\pgfsys@transformshift{10.660804in}{1.259193in}%
\pgfsys@useobject{currentmarker}{}%
\end{pgfscope}%
\begin{pgfscope}%
\pgfsys@transformshift{10.693681in}{1.521176in}%
\pgfsys@useobject{currentmarker}{}%
\end{pgfscope}%
\begin{pgfscope}%
\pgfsys@transformshift{10.726558in}{1.556263in}%
\pgfsys@useobject{currentmarker}{}%
\end{pgfscope}%
\begin{pgfscope}%
\pgfsys@transformshift{10.759436in}{1.441765in}%
\pgfsys@useobject{currentmarker}{}%
\end{pgfscope}%
\begin{pgfscope}%
\pgfsys@transformshift{10.792313in}{1.485965in}%
\pgfsys@useobject{currentmarker}{}%
\end{pgfscope}%
\begin{pgfscope}%
\pgfsys@transformshift{10.825190in}{1.601597in}%
\pgfsys@useobject{currentmarker}{}%
\end{pgfscope}%
\begin{pgfscope}%
\pgfsys@transformshift{10.858067in}{1.596019in}%
\pgfsys@useobject{currentmarker}{}%
\end{pgfscope}%
\begin{pgfscope}%
\pgfsys@transformshift{10.890945in}{1.433636in}%
\pgfsys@useobject{currentmarker}{}%
\end{pgfscope}%
\begin{pgfscope}%
\pgfsys@transformshift{10.923822in}{1.370982in}%
\pgfsys@useobject{currentmarker}{}%
\end{pgfscope}%
\begin{pgfscope}%
\pgfsys@transformshift{10.956699in}{1.553886in}%
\pgfsys@useobject{currentmarker}{}%
\end{pgfscope}%
\begin{pgfscope}%
\pgfsys@transformshift{10.989577in}{1.506412in}%
\pgfsys@useobject{currentmarker}{}%
\end{pgfscope}%
\begin{pgfscope}%
\pgfsys@transformshift{11.022454in}{1.576295in}%
\pgfsys@useobject{currentmarker}{}%
\end{pgfscope}%
\begin{pgfscope}%
\pgfsys@transformshift{11.055331in}{1.437271in}%
\pgfsys@useobject{currentmarker}{}%
\end{pgfscope}%
\begin{pgfscope}%
\pgfsys@transformshift{11.088208in}{1.599103in}%
\pgfsys@useobject{currentmarker}{}%
\end{pgfscope}%
\begin{pgfscope}%
\pgfsys@transformshift{11.121086in}{1.423807in}%
\pgfsys@useobject{currentmarker}{}%
\end{pgfscope}%
\begin{pgfscope}%
\pgfsys@transformshift{11.153963in}{1.471509in}%
\pgfsys@useobject{currentmarker}{}%
\end{pgfscope}%
\begin{pgfscope}%
\pgfsys@transformshift{11.186840in}{1.377645in}%
\pgfsys@useobject{currentmarker}{}%
\end{pgfscope}%
\begin{pgfscope}%
\pgfsys@transformshift{11.219717in}{1.673243in}%
\pgfsys@useobject{currentmarker}{}%
\end{pgfscope}%
\begin{pgfscope}%
\pgfsys@transformshift{11.252595in}{1.455753in}%
\pgfsys@useobject{currentmarker}{}%
\end{pgfscope}%
\begin{pgfscope}%
\pgfsys@transformshift{11.285472in}{1.467538in}%
\pgfsys@useobject{currentmarker}{}%
\end{pgfscope}%
\begin{pgfscope}%
\pgfsys@transformshift{11.318349in}{1.541347in}%
\pgfsys@useobject{currentmarker}{}%
\end{pgfscope}%
\begin{pgfscope}%
\pgfsys@transformshift{11.351226in}{1.489520in}%
\pgfsys@useobject{currentmarker}{}%
\end{pgfscope}%
\begin{pgfscope}%
\pgfsys@transformshift{11.384104in}{1.504926in}%
\pgfsys@useobject{currentmarker}{}%
\end{pgfscope}%
\begin{pgfscope}%
\pgfsys@transformshift{11.416981in}{1.413569in}%
\pgfsys@useobject{currentmarker}{}%
\end{pgfscope}%
\begin{pgfscope}%
\pgfsys@transformshift{11.449858in}{1.685555in}%
\pgfsys@useobject{currentmarker}{}%
\end{pgfscope}%
\begin{pgfscope}%
\pgfsys@transformshift{11.482735in}{1.659305in}%
\pgfsys@useobject{currentmarker}{}%
\end{pgfscope}%
\begin{pgfscope}%
\pgfsys@transformshift{11.515613in}{1.429316in}%
\pgfsys@useobject{currentmarker}{}%
\end{pgfscope}%
\begin{pgfscope}%
\pgfsys@transformshift{11.548490in}{1.527638in}%
\pgfsys@useobject{currentmarker}{}%
\end{pgfscope}%
\begin{pgfscope}%
\pgfsys@transformshift{11.581367in}{1.509192in}%
\pgfsys@useobject{currentmarker}{}%
\end{pgfscope}%
\begin{pgfscope}%
\pgfsys@transformshift{11.614245in}{1.482222in}%
\pgfsys@useobject{currentmarker}{}%
\end{pgfscope}%
\begin{pgfscope}%
\pgfsys@transformshift{11.647122in}{1.424735in}%
\pgfsys@useobject{currentmarker}{}%
\end{pgfscope}%
\begin{pgfscope}%
\pgfsys@transformshift{11.679999in}{1.329983in}%
\pgfsys@useobject{currentmarker}{}%
\end{pgfscope}%
\begin{pgfscope}%
\pgfsys@transformshift{11.712876in}{1.301116in}%
\pgfsys@useobject{currentmarker}{}%
\end{pgfscope}%
\begin{pgfscope}%
\pgfsys@transformshift{11.745754in}{1.491775in}%
\pgfsys@useobject{currentmarker}{}%
\end{pgfscope}%
\begin{pgfscope}%
\pgfsys@transformshift{11.778631in}{1.566057in}%
\pgfsys@useobject{currentmarker}{}%
\end{pgfscope}%
\begin{pgfscope}%
\pgfsys@transformshift{11.811508in}{1.384456in}%
\pgfsys@useobject{currentmarker}{}%
\end{pgfscope}%
\begin{pgfscope}%
\pgfsys@transformshift{11.844385in}{1.528464in}%
\pgfsys@useobject{currentmarker}{}%
\end{pgfscope}%
\begin{pgfscope}%
\pgfsys@transformshift{11.877263in}{1.444970in}%
\pgfsys@useobject{currentmarker}{}%
\end{pgfscope}%
\begin{pgfscope}%
\pgfsys@transformshift{11.910140in}{1.547903in}%
\pgfsys@useobject{currentmarker}{}%
\end{pgfscope}%
\begin{pgfscope}%
\pgfsys@transformshift{11.943017in}{1.454421in}%
\pgfsys@useobject{currentmarker}{}%
\end{pgfscope}%
\begin{pgfscope}%
\pgfsys@transformshift{11.975894in}{1.499683in}%
\pgfsys@useobject{currentmarker}{}%
\end{pgfscope}%
\begin{pgfscope}%
\pgfsys@transformshift{12.008772in}{1.537173in}%
\pgfsys@useobject{currentmarker}{}%
\end{pgfscope}%
\begin{pgfscope}%
\pgfsys@transformshift{12.041649in}{1.497078in}%
\pgfsys@useobject{currentmarker}{}%
\end{pgfscope}%
\begin{pgfscope}%
\pgfsys@transformshift{12.074526in}{1.386812in}%
\pgfsys@useobject{currentmarker}{}%
\end{pgfscope}%
\begin{pgfscope}%
\pgfsys@transformshift{12.107403in}{1.411625in}%
\pgfsys@useobject{currentmarker}{}%
\end{pgfscope}%
\begin{pgfscope}%
\pgfsys@transformshift{12.140281in}{1.598302in}%
\pgfsys@useobject{currentmarker}{}%
\end{pgfscope}%
\begin{pgfscope}%
\pgfsys@transformshift{12.173158in}{1.611612in}%
\pgfsys@useobject{currentmarker}{}%
\end{pgfscope}%
\begin{pgfscope}%
\pgfsys@transformshift{12.206035in}{1.491811in}%
\pgfsys@useobject{currentmarker}{}%
\end{pgfscope}%
\begin{pgfscope}%
\pgfsys@transformshift{12.238913in}{1.486090in}%
\pgfsys@useobject{currentmarker}{}%
\end{pgfscope}%
\begin{pgfscope}%
\pgfsys@transformshift{12.271790in}{1.527273in}%
\pgfsys@useobject{currentmarker}{}%
\end{pgfscope}%
\begin{pgfscope}%
\pgfsys@transformshift{12.304667in}{1.473548in}%
\pgfsys@useobject{currentmarker}{}%
\end{pgfscope}%
\begin{pgfscope}%
\pgfsys@transformshift{12.337544in}{1.505611in}%
\pgfsys@useobject{currentmarker}{}%
\end{pgfscope}%
\begin{pgfscope}%
\pgfsys@transformshift{12.370422in}{1.424334in}%
\pgfsys@useobject{currentmarker}{}%
\end{pgfscope}%
\begin{pgfscope}%
\pgfsys@transformshift{12.403299in}{1.536954in}%
\pgfsys@useobject{currentmarker}{}%
\end{pgfscope}%
\begin{pgfscope}%
\pgfsys@transformshift{12.436176in}{1.458631in}%
\pgfsys@useobject{currentmarker}{}%
\end{pgfscope}%
\begin{pgfscope}%
\pgfsys@transformshift{12.469053in}{1.463941in}%
\pgfsys@useobject{currentmarker}{}%
\end{pgfscope}%
\begin{pgfscope}%
\pgfsys@transformshift{12.501931in}{1.373173in}%
\pgfsys@useobject{currentmarker}{}%
\end{pgfscope}%
\begin{pgfscope}%
\pgfsys@transformshift{12.534808in}{1.478223in}%
\pgfsys@useobject{currentmarker}{}%
\end{pgfscope}%
\begin{pgfscope}%
\pgfsys@transformshift{12.567685in}{1.415248in}%
\pgfsys@useobject{currentmarker}{}%
\end{pgfscope}%
\begin{pgfscope}%
\pgfsys@transformshift{12.600562in}{1.344908in}%
\pgfsys@useobject{currentmarker}{}%
\end{pgfscope}%
\begin{pgfscope}%
\pgfsys@transformshift{12.633440in}{1.579027in}%
\pgfsys@useobject{currentmarker}{}%
\end{pgfscope}%
\begin{pgfscope}%
\pgfsys@transformshift{12.666317in}{1.593111in}%
\pgfsys@useobject{currentmarker}{}%
\end{pgfscope}%
\begin{pgfscope}%
\pgfsys@transformshift{12.699194in}{1.484247in}%
\pgfsys@useobject{currentmarker}{}%
\end{pgfscope}%
\begin{pgfscope}%
\pgfsys@transformshift{12.732071in}{1.610142in}%
\pgfsys@useobject{currentmarker}{}%
\end{pgfscope}%
\begin{pgfscope}%
\pgfsys@transformshift{12.764949in}{1.358313in}%
\pgfsys@useobject{currentmarker}{}%
\end{pgfscope}%
\begin{pgfscope}%
\pgfsys@transformshift{12.797826in}{1.441716in}%
\pgfsys@useobject{currentmarker}{}%
\end{pgfscope}%
\begin{pgfscope}%
\pgfsys@transformshift{12.830703in}{1.384596in}%
\pgfsys@useobject{currentmarker}{}%
\end{pgfscope}%
\begin{pgfscope}%
\pgfsys@transformshift{12.863581in}{1.565123in}%
\pgfsys@useobject{currentmarker}{}%
\end{pgfscope}%
\begin{pgfscope}%
\pgfsys@transformshift{12.896458in}{1.428297in}%
\pgfsys@useobject{currentmarker}{}%
\end{pgfscope}%
\begin{pgfscope}%
\pgfsys@transformshift{12.929335in}{1.472098in}%
\pgfsys@useobject{currentmarker}{}%
\end{pgfscope}%
\begin{pgfscope}%
\pgfsys@transformshift{12.962212in}{1.709795in}%
\pgfsys@useobject{currentmarker}{}%
\end{pgfscope}%
\begin{pgfscope}%
\pgfsys@transformshift{12.995090in}{1.514003in}%
\pgfsys@useobject{currentmarker}{}%
\end{pgfscope}%
\begin{pgfscope}%
\pgfsys@transformshift{13.027967in}{1.646398in}%
\pgfsys@useobject{currentmarker}{}%
\end{pgfscope}%
\begin{pgfscope}%
\pgfsys@transformshift{13.060844in}{1.313984in}%
\pgfsys@useobject{currentmarker}{}%
\end{pgfscope}%
\begin{pgfscope}%
\pgfsys@transformshift{13.093721in}{1.424137in}%
\pgfsys@useobject{currentmarker}{}%
\end{pgfscope}%
\begin{pgfscope}%
\pgfsys@transformshift{13.126599in}{1.550505in}%
\pgfsys@useobject{currentmarker}{}%
\end{pgfscope}%
\begin{pgfscope}%
\pgfsys@transformshift{13.159476in}{1.578761in}%
\pgfsys@useobject{currentmarker}{}%
\end{pgfscope}%
\begin{pgfscope}%
\pgfsys@transformshift{13.192353in}{1.611386in}%
\pgfsys@useobject{currentmarker}{}%
\end{pgfscope}%
\begin{pgfscope}%
\pgfsys@transformshift{13.225230in}{1.519865in}%
\pgfsys@useobject{currentmarker}{}%
\end{pgfscope}%
\begin{pgfscope}%
\pgfsys@transformshift{13.258108in}{1.706771in}%
\pgfsys@useobject{currentmarker}{}%
\end{pgfscope}%
\begin{pgfscope}%
\pgfsys@transformshift{13.290985in}{1.587699in}%
\pgfsys@useobject{currentmarker}{}%
\end{pgfscope}%
\begin{pgfscope}%
\pgfsys@transformshift{13.323862in}{1.526319in}%
\pgfsys@useobject{currentmarker}{}%
\end{pgfscope}%
\begin{pgfscope}%
\pgfsys@transformshift{13.356740in}{1.274373in}%
\pgfsys@useobject{currentmarker}{}%
\end{pgfscope}%
\begin{pgfscope}%
\pgfsys@transformshift{13.389617in}{1.435810in}%
\pgfsys@useobject{currentmarker}{}%
\end{pgfscope}%
\begin{pgfscope}%
\pgfsys@transformshift{13.422494in}{1.568789in}%
\pgfsys@useobject{currentmarker}{}%
\end{pgfscope}%
\begin{pgfscope}%
\pgfsys@transformshift{13.455371in}{1.344449in}%
\pgfsys@useobject{currentmarker}{}%
\end{pgfscope}%
\begin{pgfscope}%
\pgfsys@transformshift{13.488249in}{1.668309in}%
\pgfsys@useobject{currentmarker}{}%
\end{pgfscope}%
\begin{pgfscope}%
\pgfsys@transformshift{13.521126in}{1.566864in}%
\pgfsys@useobject{currentmarker}{}%
\end{pgfscope}%
\begin{pgfscope}%
\pgfsys@transformshift{13.554003in}{1.492675in}%
\pgfsys@useobject{currentmarker}{}%
\end{pgfscope}%
\begin{pgfscope}%
\pgfsys@transformshift{13.586880in}{1.471091in}%
\pgfsys@useobject{currentmarker}{}%
\end{pgfscope}%
\begin{pgfscope}%
\pgfsys@transformshift{13.619758in}{1.650707in}%
\pgfsys@useobject{currentmarker}{}%
\end{pgfscope}%
\begin{pgfscope}%
\pgfsys@transformshift{13.652635in}{1.646265in}%
\pgfsys@useobject{currentmarker}{}%
\end{pgfscope}%
\begin{pgfscope}%
\pgfsys@transformshift{13.685512in}{1.436364in}%
\pgfsys@useobject{currentmarker}{}%
\end{pgfscope}%
\begin{pgfscope}%
\pgfsys@transformshift{13.718389in}{1.413467in}%
\pgfsys@useobject{currentmarker}{}%
\end{pgfscope}%
\begin{pgfscope}%
\pgfsys@transformshift{13.751267in}{1.402549in}%
\pgfsys@useobject{currentmarker}{}%
\end{pgfscope}%
\begin{pgfscope}%
\pgfsys@transformshift{13.784144in}{1.504574in}%
\pgfsys@useobject{currentmarker}{}%
\end{pgfscope}%
\begin{pgfscope}%
\pgfsys@transformshift{13.817021in}{1.543976in}%
\pgfsys@useobject{currentmarker}{}%
\end{pgfscope}%
\begin{pgfscope}%
\pgfsys@transformshift{13.849898in}{1.554312in}%
\pgfsys@useobject{currentmarker}{}%
\end{pgfscope}%
\begin{pgfscope}%
\pgfsys@transformshift{13.882776in}{1.439561in}%
\pgfsys@useobject{currentmarker}{}%
\end{pgfscope}%
\begin{pgfscope}%
\pgfsys@transformshift{13.915653in}{1.438069in}%
\pgfsys@useobject{currentmarker}{}%
\end{pgfscope}%
\begin{pgfscope}%
\pgfsys@transformshift{13.948530in}{1.647896in}%
\pgfsys@useobject{currentmarker}{}%
\end{pgfscope}%
\begin{pgfscope}%
\pgfsys@transformshift{13.981408in}{1.510785in}%
\pgfsys@useobject{currentmarker}{}%
\end{pgfscope}%
\begin{pgfscope}%
\pgfsys@transformshift{14.014285in}{1.517148in}%
\pgfsys@useobject{currentmarker}{}%
\end{pgfscope}%
\begin{pgfscope}%
\pgfsys@transformshift{14.047162in}{1.382292in}%
\pgfsys@useobject{currentmarker}{}%
\end{pgfscope}%
\begin{pgfscope}%
\pgfsys@transformshift{14.080039in}{1.338375in}%
\pgfsys@useobject{currentmarker}{}%
\end{pgfscope}%
\begin{pgfscope}%
\pgfsys@transformshift{14.112917in}{1.589372in}%
\pgfsys@useobject{currentmarker}{}%
\end{pgfscope}%
\begin{pgfscope}%
\pgfsys@transformshift{14.145794in}{1.372307in}%
\pgfsys@useobject{currentmarker}{}%
\end{pgfscope}%
\begin{pgfscope}%
\pgfsys@transformshift{14.178671in}{1.570002in}%
\pgfsys@useobject{currentmarker}{}%
\end{pgfscope}%
\begin{pgfscope}%
\pgfsys@transformshift{14.211548in}{1.439908in}%
\pgfsys@useobject{currentmarker}{}%
\end{pgfscope}%
\begin{pgfscope}%
\pgfsys@transformshift{14.244426in}{1.494120in}%
\pgfsys@useobject{currentmarker}{}%
\end{pgfscope}%
\begin{pgfscope}%
\pgfsys@transformshift{14.277303in}{1.515113in}%
\pgfsys@useobject{currentmarker}{}%
\end{pgfscope}%
\begin{pgfscope}%
\pgfsys@transformshift{14.310180in}{1.569350in}%
\pgfsys@useobject{currentmarker}{}%
\end{pgfscope}%
\begin{pgfscope}%
\pgfsys@transformshift{14.343057in}{1.497796in}%
\pgfsys@useobject{currentmarker}{}%
\end{pgfscope}%
\begin{pgfscope}%
\pgfsys@transformshift{14.375935in}{1.577008in}%
\pgfsys@useobject{currentmarker}{}%
\end{pgfscope}%
\begin{pgfscope}%
\pgfsys@transformshift{14.408812in}{1.639146in}%
\pgfsys@useobject{currentmarker}{}%
\end{pgfscope}%
\begin{pgfscope}%
\pgfsys@transformshift{14.441689in}{1.561961in}%
\pgfsys@useobject{currentmarker}{}%
\end{pgfscope}%
\begin{pgfscope}%
\pgfsys@transformshift{14.474566in}{1.604949in}%
\pgfsys@useobject{currentmarker}{}%
\end{pgfscope}%
\begin{pgfscope}%
\pgfsys@transformshift{14.507444in}{1.531373in}%
\pgfsys@useobject{currentmarker}{}%
\end{pgfscope}%
\begin{pgfscope}%
\pgfsys@transformshift{14.540321in}{1.368527in}%
\pgfsys@useobject{currentmarker}{}%
\end{pgfscope}%
\begin{pgfscope}%
\pgfsys@transformshift{14.573198in}{1.455340in}%
\pgfsys@useobject{currentmarker}{}%
\end{pgfscope}%
\begin{pgfscope}%
\pgfsys@transformshift{14.606076in}{1.426049in}%
\pgfsys@useobject{currentmarker}{}%
\end{pgfscope}%
\begin{pgfscope}%
\pgfsys@transformshift{14.638953in}{1.567025in}%
\pgfsys@useobject{currentmarker}{}%
\end{pgfscope}%
\begin{pgfscope}%
\pgfsys@transformshift{14.671830in}{1.423020in}%
\pgfsys@useobject{currentmarker}{}%
\end{pgfscope}%
\begin{pgfscope}%
\pgfsys@transformshift{14.704707in}{1.574034in}%
\pgfsys@useobject{currentmarker}{}%
\end{pgfscope}%
\begin{pgfscope}%
\pgfsys@transformshift{14.737585in}{1.643545in}%
\pgfsys@useobject{currentmarker}{}%
\end{pgfscope}%
\begin{pgfscope}%
\pgfsys@transformshift{14.770462in}{1.298663in}%
\pgfsys@useobject{currentmarker}{}%
\end{pgfscope}%
\begin{pgfscope}%
\pgfsys@transformshift{14.803339in}{1.481464in}%
\pgfsys@useobject{currentmarker}{}%
\end{pgfscope}%
\begin{pgfscope}%
\pgfsys@transformshift{14.836216in}{1.576069in}%
\pgfsys@useobject{currentmarker}{}%
\end{pgfscope}%
\begin{pgfscope}%
\pgfsys@transformshift{14.869094in}{1.305617in}%
\pgfsys@useobject{currentmarker}{}%
\end{pgfscope}%
\begin{pgfscope}%
\pgfsys@transformshift{14.901971in}{1.607357in}%
\pgfsys@useobject{currentmarker}{}%
\end{pgfscope}%
\begin{pgfscope}%
\pgfsys@transformshift{14.934848in}{1.715657in}%
\pgfsys@useobject{currentmarker}{}%
\end{pgfscope}%
\begin{pgfscope}%
\pgfsys@transformshift{14.967725in}{1.367470in}%
\pgfsys@useobject{currentmarker}{}%
\end{pgfscope}%
\begin{pgfscope}%
\pgfsys@transformshift{15.000603in}{1.537170in}%
\pgfsys@useobject{currentmarker}{}%
\end{pgfscope}%
\begin{pgfscope}%
\pgfsys@transformshift{15.033480in}{1.474710in}%
\pgfsys@useobject{currentmarker}{}%
\end{pgfscope}%
\begin{pgfscope}%
\pgfsys@transformshift{15.066357in}{1.636791in}%
\pgfsys@useobject{currentmarker}{}%
\end{pgfscope}%
\begin{pgfscope}%
\pgfsys@transformshift{15.099234in}{1.485232in}%
\pgfsys@useobject{currentmarker}{}%
\end{pgfscope}%
\begin{pgfscope}%
\pgfsys@transformshift{15.132112in}{1.478440in}%
\pgfsys@useobject{currentmarker}{}%
\end{pgfscope}%
\begin{pgfscope}%
\pgfsys@transformshift{15.164989in}{1.612128in}%
\pgfsys@useobject{currentmarker}{}%
\end{pgfscope}%
\begin{pgfscope}%
\pgfsys@transformshift{15.197866in}{1.512360in}%
\pgfsys@useobject{currentmarker}{}%
\end{pgfscope}%
\begin{pgfscope}%
\pgfsys@transformshift{15.230744in}{1.428052in}%
\pgfsys@useobject{currentmarker}{}%
\end{pgfscope}%
\begin{pgfscope}%
\pgfsys@transformshift{15.263621in}{1.476259in}%
\pgfsys@useobject{currentmarker}{}%
\end{pgfscope}%
\begin{pgfscope}%
\pgfsys@transformshift{15.296498in}{1.512179in}%
\pgfsys@useobject{currentmarker}{}%
\end{pgfscope}%
\begin{pgfscope}%
\pgfsys@transformshift{15.329375in}{1.516082in}%
\pgfsys@useobject{currentmarker}{}%
\end{pgfscope}%
\begin{pgfscope}%
\pgfsys@transformshift{15.362253in}{1.413670in}%
\pgfsys@useobject{currentmarker}{}%
\end{pgfscope}%
\begin{pgfscope}%
\pgfsys@transformshift{15.395130in}{1.389542in}%
\pgfsys@useobject{currentmarker}{}%
\end{pgfscope}%
\begin{pgfscope}%
\pgfsys@transformshift{15.428007in}{1.407060in}%
\pgfsys@useobject{currentmarker}{}%
\end{pgfscope}%
\begin{pgfscope}%
\pgfsys@transformshift{15.460884in}{1.490650in}%
\pgfsys@useobject{currentmarker}{}%
\end{pgfscope}%
\begin{pgfscope}%
\pgfsys@transformshift{15.493762in}{1.456410in}%
\pgfsys@useobject{currentmarker}{}%
\end{pgfscope}%
\begin{pgfscope}%
\pgfsys@transformshift{15.526639in}{1.549861in}%
\pgfsys@useobject{currentmarker}{}%
\end{pgfscope}%
\begin{pgfscope}%
\pgfsys@transformshift{15.559516in}{1.506610in}%
\pgfsys@useobject{currentmarker}{}%
\end{pgfscope}%
\begin{pgfscope}%
\pgfsys@transformshift{15.592393in}{1.389354in}%
\pgfsys@useobject{currentmarker}{}%
\end{pgfscope}%
\begin{pgfscope}%
\pgfsys@transformshift{15.625271in}{1.635256in}%
\pgfsys@useobject{currentmarker}{}%
\end{pgfscope}%
\begin{pgfscope}%
\pgfsys@transformshift{15.658148in}{1.455516in}%
\pgfsys@useobject{currentmarker}{}%
\end{pgfscope}%
\begin{pgfscope}%
\pgfsys@transformshift{15.691025in}{1.491827in}%
\pgfsys@useobject{currentmarker}{}%
\end{pgfscope}%
\begin{pgfscope}%
\pgfsys@transformshift{15.723902in}{1.543433in}%
\pgfsys@useobject{currentmarker}{}%
\end{pgfscope}%
\begin{pgfscope}%
\pgfsys@transformshift{15.756780in}{1.383592in}%
\pgfsys@useobject{currentmarker}{}%
\end{pgfscope}%
\begin{pgfscope}%
\pgfsys@transformshift{15.789657in}{1.535237in}%
\pgfsys@useobject{currentmarker}{}%
\end{pgfscope}%
\begin{pgfscope}%
\pgfsys@transformshift{15.822534in}{1.542724in}%
\pgfsys@useobject{currentmarker}{}%
\end{pgfscope}%
\begin{pgfscope}%
\pgfsys@transformshift{15.855412in}{1.668953in}%
\pgfsys@useobject{currentmarker}{}%
\end{pgfscope}%
\begin{pgfscope}%
\pgfsys@transformshift{15.888289in}{1.417186in}%
\pgfsys@useobject{currentmarker}{}%
\end{pgfscope}%
\begin{pgfscope}%
\pgfsys@transformshift{15.921166in}{1.407724in}%
\pgfsys@useobject{currentmarker}{}%
\end{pgfscope}%
\begin{pgfscope}%
\pgfsys@transformshift{15.954043in}{1.688811in}%
\pgfsys@useobject{currentmarker}{}%
\end{pgfscope}%
\begin{pgfscope}%
\pgfsys@transformshift{15.986921in}{1.413829in}%
\pgfsys@useobject{currentmarker}{}%
\end{pgfscope}%
\begin{pgfscope}%
\pgfsys@transformshift{16.019798in}{1.437880in}%
\pgfsys@useobject{currentmarker}{}%
\end{pgfscope}%
\begin{pgfscope}%
\pgfsys@transformshift{16.052675in}{1.308510in}%
\pgfsys@useobject{currentmarker}{}%
\end{pgfscope}%
\begin{pgfscope}%
\pgfsys@transformshift{16.085552in}{1.354617in}%
\pgfsys@useobject{currentmarker}{}%
\end{pgfscope}%
\begin{pgfscope}%
\pgfsys@transformshift{16.118430in}{1.426617in}%
\pgfsys@useobject{currentmarker}{}%
\end{pgfscope}%
\begin{pgfscope}%
\pgfsys@transformshift{16.151307in}{1.579509in}%
\pgfsys@useobject{currentmarker}{}%
\end{pgfscope}%
\begin{pgfscope}%
\pgfsys@transformshift{16.184184in}{1.560449in}%
\pgfsys@useobject{currentmarker}{}%
\end{pgfscope}%
\begin{pgfscope}%
\pgfsys@transformshift{16.217061in}{1.648349in}%
\pgfsys@useobject{currentmarker}{}%
\end{pgfscope}%
\begin{pgfscope}%
\pgfsys@transformshift{16.249939in}{1.410685in}%
\pgfsys@useobject{currentmarker}{}%
\end{pgfscope}%
\begin{pgfscope}%
\pgfsys@transformshift{16.282816in}{1.506470in}%
\pgfsys@useobject{currentmarker}{}%
\end{pgfscope}%
\begin{pgfscope}%
\pgfsys@transformshift{16.315693in}{1.464461in}%
\pgfsys@useobject{currentmarker}{}%
\end{pgfscope}%
\begin{pgfscope}%
\pgfsys@transformshift{16.348571in}{1.439819in}%
\pgfsys@useobject{currentmarker}{}%
\end{pgfscope}%
\begin{pgfscope}%
\pgfsys@transformshift{16.381448in}{1.643182in}%
\pgfsys@useobject{currentmarker}{}%
\end{pgfscope}%
\begin{pgfscope}%
\pgfsys@transformshift{16.414325in}{1.544386in}%
\pgfsys@useobject{currentmarker}{}%
\end{pgfscope}%
\begin{pgfscope}%
\pgfsys@transformshift{16.447202in}{1.521518in}%
\pgfsys@useobject{currentmarker}{}%
\end{pgfscope}%
\begin{pgfscope}%
\pgfsys@transformshift{16.480080in}{1.488930in}%
\pgfsys@useobject{currentmarker}{}%
\end{pgfscope}%
\begin{pgfscope}%
\pgfsys@transformshift{16.512957in}{1.573518in}%
\pgfsys@useobject{currentmarker}{}%
\end{pgfscope}%
\begin{pgfscope}%
\pgfsys@transformshift{16.545834in}{1.494164in}%
\pgfsys@useobject{currentmarker}{}%
\end{pgfscope}%
\begin{pgfscope}%
\pgfsys@transformshift{16.578711in}{1.426764in}%
\pgfsys@useobject{currentmarker}{}%
\end{pgfscope}%
\begin{pgfscope}%
\pgfsys@transformshift{16.611589in}{1.480779in}%
\pgfsys@useobject{currentmarker}{}%
\end{pgfscope}%
\begin{pgfscope}%
\pgfsys@transformshift{16.644466in}{1.445876in}%
\pgfsys@useobject{currentmarker}{}%
\end{pgfscope}%
\begin{pgfscope}%
\pgfsys@transformshift{16.677343in}{1.501149in}%
\pgfsys@useobject{currentmarker}{}%
\end{pgfscope}%
\begin{pgfscope}%
\pgfsys@transformshift{16.710220in}{1.448431in}%
\pgfsys@useobject{currentmarker}{}%
\end{pgfscope}%
\begin{pgfscope}%
\pgfsys@transformshift{16.743098in}{1.510602in}%
\pgfsys@useobject{currentmarker}{}%
\end{pgfscope}%
\begin{pgfscope}%
\pgfsys@transformshift{16.775975in}{1.517288in}%
\pgfsys@useobject{currentmarker}{}%
\end{pgfscope}%
\begin{pgfscope}%
\pgfsys@transformshift{16.808852in}{1.458200in}%
\pgfsys@useobject{currentmarker}{}%
\end{pgfscope}%
\begin{pgfscope}%
\pgfsys@transformshift{16.841729in}{1.418056in}%
\pgfsys@useobject{currentmarker}{}%
\end{pgfscope}%
\begin{pgfscope}%
\pgfsys@transformshift{16.874607in}{1.463489in}%
\pgfsys@useobject{currentmarker}{}%
\end{pgfscope}%
\begin{pgfscope}%
\pgfsys@transformshift{16.907484in}{1.607409in}%
\pgfsys@useobject{currentmarker}{}%
\end{pgfscope}%
\begin{pgfscope}%
\pgfsys@transformshift{16.940361in}{1.603316in}%
\pgfsys@useobject{currentmarker}{}%
\end{pgfscope}%
\begin{pgfscope}%
\pgfsys@transformshift{16.973239in}{1.486980in}%
\pgfsys@useobject{currentmarker}{}%
\end{pgfscope}%
\begin{pgfscope}%
\pgfsys@transformshift{17.006116in}{1.552584in}%
\pgfsys@useobject{currentmarker}{}%
\end{pgfscope}%
\begin{pgfscope}%
\pgfsys@transformshift{17.038993in}{1.645593in}%
\pgfsys@useobject{currentmarker}{}%
\end{pgfscope}%
\begin{pgfscope}%
\pgfsys@transformshift{17.071870in}{1.425076in}%
\pgfsys@useobject{currentmarker}{}%
\end{pgfscope}%
\begin{pgfscope}%
\pgfsys@transformshift{17.104748in}{1.551379in}%
\pgfsys@useobject{currentmarker}{}%
\end{pgfscope}%
\begin{pgfscope}%
\pgfsys@transformshift{17.137625in}{1.455674in}%
\pgfsys@useobject{currentmarker}{}%
\end{pgfscope}%
\begin{pgfscope}%
\pgfsys@transformshift{17.170502in}{1.478369in}%
\pgfsys@useobject{currentmarker}{}%
\end{pgfscope}%
\begin{pgfscope}%
\pgfsys@transformshift{17.203379in}{1.406096in}%
\pgfsys@useobject{currentmarker}{}%
\end{pgfscope}%
\begin{pgfscope}%
\pgfsys@transformshift{17.236257in}{1.515824in}%
\pgfsys@useobject{currentmarker}{}%
\end{pgfscope}%
\begin{pgfscope}%
\pgfsys@transformshift{17.269134in}{1.448195in}%
\pgfsys@useobject{currentmarker}{}%
\end{pgfscope}%
\begin{pgfscope}%
\pgfsys@transformshift{17.302011in}{1.637092in}%
\pgfsys@useobject{currentmarker}{}%
\end{pgfscope}%
\begin{pgfscope}%
\pgfsys@transformshift{17.334888in}{1.506135in}%
\pgfsys@useobject{currentmarker}{}%
\end{pgfscope}%
\begin{pgfscope}%
\pgfsys@transformshift{17.367766in}{1.617743in}%
\pgfsys@useobject{currentmarker}{}%
\end{pgfscope}%
\begin{pgfscope}%
\pgfsys@transformshift{17.400643in}{1.541271in}%
\pgfsys@useobject{currentmarker}{}%
\end{pgfscope}%
\begin{pgfscope}%
\pgfsys@transformshift{17.433520in}{1.629750in}%
\pgfsys@useobject{currentmarker}{}%
\end{pgfscope}%
\begin{pgfscope}%
\pgfsys@transformshift{17.466397in}{1.300951in}%
\pgfsys@useobject{currentmarker}{}%
\end{pgfscope}%
\begin{pgfscope}%
\pgfsys@transformshift{17.499275in}{1.568219in}%
\pgfsys@useobject{currentmarker}{}%
\end{pgfscope}%
\begin{pgfscope}%
\pgfsys@transformshift{17.532152in}{1.718699in}%
\pgfsys@useobject{currentmarker}{}%
\end{pgfscope}%
\begin{pgfscope}%
\pgfsys@transformshift{17.565029in}{1.497363in}%
\pgfsys@useobject{currentmarker}{}%
\end{pgfscope}%
\begin{pgfscope}%
\pgfsys@transformshift{17.597907in}{1.390381in}%
\pgfsys@useobject{currentmarker}{}%
\end{pgfscope}%
\begin{pgfscope}%
\pgfsys@transformshift{17.630784in}{1.466392in}%
\pgfsys@useobject{currentmarker}{}%
\end{pgfscope}%
\begin{pgfscope}%
\pgfsys@transformshift{17.663661in}{1.543401in}%
\pgfsys@useobject{currentmarker}{}%
\end{pgfscope}%
\begin{pgfscope}%
\pgfsys@transformshift{17.696538in}{1.462385in}%
\pgfsys@useobject{currentmarker}{}%
\end{pgfscope}%
\begin{pgfscope}%
\pgfsys@transformshift{17.729416in}{1.633481in}%
\pgfsys@useobject{currentmarker}{}%
\end{pgfscope}%
\begin{pgfscope}%
\pgfsys@transformshift{17.762293in}{1.270458in}%
\pgfsys@useobject{currentmarker}{}%
\end{pgfscope}%
\begin{pgfscope}%
\pgfsys@transformshift{17.795170in}{1.424362in}%
\pgfsys@useobject{currentmarker}{}%
\end{pgfscope}%
\begin{pgfscope}%
\pgfsys@transformshift{17.828047in}{1.777380in}%
\pgfsys@useobject{currentmarker}{}%
\end{pgfscope}%
\begin{pgfscope}%
\pgfsys@transformshift{17.860925in}{1.659744in}%
\pgfsys@useobject{currentmarker}{}%
\end{pgfscope}%
\begin{pgfscope}%
\pgfsys@transformshift{17.893802in}{1.341354in}%
\pgfsys@useobject{currentmarker}{}%
\end{pgfscope}%
\begin{pgfscope}%
\pgfsys@transformshift{17.926679in}{1.408579in}%
\pgfsys@useobject{currentmarker}{}%
\end{pgfscope}%
\begin{pgfscope}%
\pgfsys@transformshift{17.959556in}{1.632266in}%
\pgfsys@useobject{currentmarker}{}%
\end{pgfscope}%
\begin{pgfscope}%
\pgfsys@transformshift{17.992434in}{1.312198in}%
\pgfsys@useobject{currentmarker}{}%
\end{pgfscope}%
\begin{pgfscope}%
\pgfsys@transformshift{18.025311in}{1.589630in}%
\pgfsys@useobject{currentmarker}{}%
\end{pgfscope}%
\begin{pgfscope}%
\pgfsys@transformshift{18.058188in}{1.550925in}%
\pgfsys@useobject{currentmarker}{}%
\end{pgfscope}%
\begin{pgfscope}%
\pgfsys@transformshift{18.091065in}{1.658194in}%
\pgfsys@useobject{currentmarker}{}%
\end{pgfscope}%
\begin{pgfscope}%
\pgfsys@transformshift{18.123943in}{1.401315in}%
\pgfsys@useobject{currentmarker}{}%
\end{pgfscope}%
\begin{pgfscope}%
\pgfsys@transformshift{18.156820in}{1.538689in}%
\pgfsys@useobject{currentmarker}{}%
\end{pgfscope}%
\begin{pgfscope}%
\pgfsys@transformshift{18.189697in}{1.406119in}%
\pgfsys@useobject{currentmarker}{}%
\end{pgfscope}%
\begin{pgfscope}%
\pgfsys@transformshift{18.222575in}{1.559178in}%
\pgfsys@useobject{currentmarker}{}%
\end{pgfscope}%
\begin{pgfscope}%
\pgfsys@transformshift{18.255452in}{1.553387in}%
\pgfsys@useobject{currentmarker}{}%
\end{pgfscope}%
\begin{pgfscope}%
\pgfsys@transformshift{18.288329in}{1.547558in}%
\pgfsys@useobject{currentmarker}{}%
\end{pgfscope}%
\begin{pgfscope}%
\pgfsys@transformshift{18.321206in}{1.527402in}%
\pgfsys@useobject{currentmarker}{}%
\end{pgfscope}%
\begin{pgfscope}%
\pgfsys@transformshift{18.354084in}{1.581170in}%
\pgfsys@useobject{currentmarker}{}%
\end{pgfscope}%
\begin{pgfscope}%
\pgfsys@transformshift{18.386961in}{1.409693in}%
\pgfsys@useobject{currentmarker}{}%
\end{pgfscope}%
\begin{pgfscope}%
\pgfsys@transformshift{18.419838in}{1.371706in}%
\pgfsys@useobject{currentmarker}{}%
\end{pgfscope}%
\begin{pgfscope}%
\pgfsys@transformshift{18.452715in}{1.538917in}%
\pgfsys@useobject{currentmarker}{}%
\end{pgfscope}%
\begin{pgfscope}%
\pgfsys@transformshift{18.485593in}{1.424931in}%
\pgfsys@useobject{currentmarker}{}%
\end{pgfscope}%
\begin{pgfscope}%
\pgfsys@transformshift{18.518470in}{1.392777in}%
\pgfsys@useobject{currentmarker}{}%
\end{pgfscope}%
\begin{pgfscope}%
\pgfsys@transformshift{18.551347in}{1.395899in}%
\pgfsys@useobject{currentmarker}{}%
\end{pgfscope}%
\begin{pgfscope}%
\pgfsys@transformshift{18.584224in}{1.489355in}%
\pgfsys@useobject{currentmarker}{}%
\end{pgfscope}%
\begin{pgfscope}%
\pgfsys@transformshift{18.617102in}{1.540009in}%
\pgfsys@useobject{currentmarker}{}%
\end{pgfscope}%
\begin{pgfscope}%
\pgfsys@transformshift{18.649979in}{1.475936in}%
\pgfsys@useobject{currentmarker}{}%
\end{pgfscope}%
\begin{pgfscope}%
\pgfsys@transformshift{18.682856in}{1.613702in}%
\pgfsys@useobject{currentmarker}{}%
\end{pgfscope}%
\begin{pgfscope}%
\pgfsys@transformshift{18.715733in}{1.456518in}%
\pgfsys@useobject{currentmarker}{}%
\end{pgfscope}%
\begin{pgfscope}%
\pgfsys@transformshift{18.748611in}{1.511027in}%
\pgfsys@useobject{currentmarker}{}%
\end{pgfscope}%
\begin{pgfscope}%
\pgfsys@transformshift{18.781488in}{1.389865in}%
\pgfsys@useobject{currentmarker}{}%
\end{pgfscope}%
\begin{pgfscope}%
\pgfsys@transformshift{18.814365in}{1.437710in}%
\pgfsys@useobject{currentmarker}{}%
\end{pgfscope}%
\begin{pgfscope}%
\pgfsys@transformshift{18.847243in}{1.473056in}%
\pgfsys@useobject{currentmarker}{}%
\end{pgfscope}%
\begin{pgfscope}%
\pgfsys@transformshift{18.880120in}{1.538515in}%
\pgfsys@useobject{currentmarker}{}%
\end{pgfscope}%
\begin{pgfscope}%
\pgfsys@transformshift{18.912997in}{1.462043in}%
\pgfsys@useobject{currentmarker}{}%
\end{pgfscope}%
\begin{pgfscope}%
\pgfsys@transformshift{18.945874in}{1.471707in}%
\pgfsys@useobject{currentmarker}{}%
\end{pgfscope}%
\begin{pgfscope}%
\pgfsys@transformshift{18.978752in}{1.416868in}%
\pgfsys@useobject{currentmarker}{}%
\end{pgfscope}%
\begin{pgfscope}%
\pgfsys@transformshift{19.011629in}{1.453307in}%
\pgfsys@useobject{currentmarker}{}%
\end{pgfscope}%
\begin{pgfscope}%
\pgfsys@transformshift{19.044506in}{1.495907in}%
\pgfsys@useobject{currentmarker}{}%
\end{pgfscope}%
\begin{pgfscope}%
\pgfsys@transformshift{19.077383in}{1.434700in}%
\pgfsys@useobject{currentmarker}{}%
\end{pgfscope}%
\begin{pgfscope}%
\pgfsys@transformshift{19.110261in}{1.517240in}%
\pgfsys@useobject{currentmarker}{}%
\end{pgfscope}%
\begin{pgfscope}%
\pgfsys@transformshift{19.143138in}{1.539333in}%
\pgfsys@useobject{currentmarker}{}%
\end{pgfscope}%
\begin{pgfscope}%
\pgfsys@transformshift{19.176015in}{1.417228in}%
\pgfsys@useobject{currentmarker}{}%
\end{pgfscope}%
\begin{pgfscope}%
\pgfsys@transformshift{19.208892in}{1.618805in}%
\pgfsys@useobject{currentmarker}{}%
\end{pgfscope}%
\begin{pgfscope}%
\pgfsys@transformshift{19.241770in}{1.270669in}%
\pgfsys@useobject{currentmarker}{}%
\end{pgfscope}%
\begin{pgfscope}%
\pgfsys@transformshift{19.274647in}{1.500816in}%
\pgfsys@useobject{currentmarker}{}%
\end{pgfscope}%
\begin{pgfscope}%
\pgfsys@transformshift{19.307524in}{1.578203in}%
\pgfsys@useobject{currentmarker}{}%
\end{pgfscope}%
\begin{pgfscope}%
\pgfsys@transformshift{19.340402in}{1.362596in}%
\pgfsys@useobject{currentmarker}{}%
\end{pgfscope}%
\begin{pgfscope}%
\pgfsys@transformshift{19.373279in}{1.355036in}%
\pgfsys@useobject{currentmarker}{}%
\end{pgfscope}%
\begin{pgfscope}%
\pgfsys@transformshift{19.406156in}{1.610054in}%
\pgfsys@useobject{currentmarker}{}%
\end{pgfscope}%
\begin{pgfscope}%
\pgfsys@transformshift{19.439033in}{1.330090in}%
\pgfsys@useobject{currentmarker}{}%
\end{pgfscope}%
\begin{pgfscope}%
\pgfsys@transformshift{19.471911in}{1.353706in}%
\pgfsys@useobject{currentmarker}{}%
\end{pgfscope}%
\begin{pgfscope}%
\pgfsys@transformshift{19.504788in}{1.498244in}%
\pgfsys@useobject{currentmarker}{}%
\end{pgfscope}%
\begin{pgfscope}%
\pgfsys@transformshift{19.537665in}{1.432062in}%
\pgfsys@useobject{currentmarker}{}%
\end{pgfscope}%
\begin{pgfscope}%
\pgfsys@transformshift{19.570542in}{1.454558in}%
\pgfsys@useobject{currentmarker}{}%
\end{pgfscope}%
\begin{pgfscope}%
\pgfsys@transformshift{19.603420in}{1.412562in}%
\pgfsys@useobject{currentmarker}{}%
\end{pgfscope}%
\begin{pgfscope}%
\pgfsys@transformshift{19.636297in}{1.727985in}%
\pgfsys@useobject{currentmarker}{}%
\end{pgfscope}%
\begin{pgfscope}%
\pgfsys@transformshift{19.669174in}{1.729869in}%
\pgfsys@useobject{currentmarker}{}%
\end{pgfscope}%
\begin{pgfscope}%
\pgfsys@transformshift{19.702051in}{1.664375in}%
\pgfsys@useobject{currentmarker}{}%
\end{pgfscope}%
\begin{pgfscope}%
\pgfsys@transformshift{19.734929in}{1.415575in}%
\pgfsys@useobject{currentmarker}{}%
\end{pgfscope}%
\begin{pgfscope}%
\pgfsys@transformshift{19.767806in}{1.423087in}%
\pgfsys@useobject{currentmarker}{}%
\end{pgfscope}%
\begin{pgfscope}%
\pgfsys@transformshift{19.800683in}{1.568191in}%
\pgfsys@useobject{currentmarker}{}%
\end{pgfscope}%
\begin{pgfscope}%
\pgfsys@transformshift{19.833560in}{1.580334in}%
\pgfsys@useobject{currentmarker}{}%
\end{pgfscope}%
\begin{pgfscope}%
\pgfsys@transformshift{19.866438in}{1.497294in}%
\pgfsys@useobject{currentmarker}{}%
\end{pgfscope}%
\begin{pgfscope}%
\pgfsys@transformshift{19.899315in}{1.544812in}%
\pgfsys@useobject{currentmarker}{}%
\end{pgfscope}%
\begin{pgfscope}%
\pgfsys@transformshift{19.932192in}{1.421992in}%
\pgfsys@useobject{currentmarker}{}%
\end{pgfscope}%
\begin{pgfscope}%
\pgfsys@transformshift{19.965070in}{1.443850in}%
\pgfsys@useobject{currentmarker}{}%
\end{pgfscope}%
\begin{pgfscope}%
\pgfsys@transformshift{19.997947in}{1.625603in}%
\pgfsys@useobject{currentmarker}{}%
\end{pgfscope}%
\begin{pgfscope}%
\pgfsys@transformshift{20.030824in}{1.429984in}%
\pgfsys@useobject{currentmarker}{}%
\end{pgfscope}%
\begin{pgfscope}%
\pgfsys@transformshift{20.063701in}{1.553585in}%
\pgfsys@useobject{currentmarker}{}%
\end{pgfscope}%
\begin{pgfscope}%
\pgfsys@transformshift{20.096579in}{1.506238in}%
\pgfsys@useobject{currentmarker}{}%
\end{pgfscope}%
\begin{pgfscope}%
\pgfsys@transformshift{20.129456in}{1.419132in}%
\pgfsys@useobject{currentmarker}{}%
\end{pgfscope}%
\begin{pgfscope}%
\pgfsys@transformshift{20.162333in}{1.374817in}%
\pgfsys@useobject{currentmarker}{}%
\end{pgfscope}%
\begin{pgfscope}%
\pgfsys@transformshift{20.195210in}{1.594848in}%
\pgfsys@useobject{currentmarker}{}%
\end{pgfscope}%
\begin{pgfscope}%
\pgfsys@transformshift{20.228088in}{1.515037in}%
\pgfsys@useobject{currentmarker}{}%
\end{pgfscope}%
\begin{pgfscope}%
\pgfsys@transformshift{20.260965in}{1.256911in}%
\pgfsys@useobject{currentmarker}{}%
\end{pgfscope}%
\begin{pgfscope}%
\pgfsys@transformshift{20.293842in}{1.520494in}%
\pgfsys@useobject{currentmarker}{}%
\end{pgfscope}%
\begin{pgfscope}%
\pgfsys@transformshift{20.326719in}{1.677118in}%
\pgfsys@useobject{currentmarker}{}%
\end{pgfscope}%
\begin{pgfscope}%
\pgfsys@transformshift{20.359597in}{1.459150in}%
\pgfsys@useobject{currentmarker}{}%
\end{pgfscope}%
\begin{pgfscope}%
\pgfsys@transformshift{20.392474in}{1.479954in}%
\pgfsys@useobject{currentmarker}{}%
\end{pgfscope}%
\begin{pgfscope}%
\pgfsys@transformshift{20.425351in}{1.348647in}%
\pgfsys@useobject{currentmarker}{}%
\end{pgfscope}%
\begin{pgfscope}%
\pgfsys@transformshift{20.458228in}{1.363421in}%
\pgfsys@useobject{currentmarker}{}%
\end{pgfscope}%
\begin{pgfscope}%
\pgfsys@transformshift{20.491106in}{1.587929in}%
\pgfsys@useobject{currentmarker}{}%
\end{pgfscope}%
\begin{pgfscope}%
\pgfsys@transformshift{20.523983in}{1.532476in}%
\pgfsys@useobject{currentmarker}{}%
\end{pgfscope}%
\begin{pgfscope}%
\pgfsys@transformshift{20.556860in}{1.640664in}%
\pgfsys@useobject{currentmarker}{}%
\end{pgfscope}%
\begin{pgfscope}%
\pgfsys@transformshift{20.589738in}{1.429459in}%
\pgfsys@useobject{currentmarker}{}%
\end{pgfscope}%
\begin{pgfscope}%
\pgfsys@transformshift{20.622615in}{1.564721in}%
\pgfsys@useobject{currentmarker}{}%
\end{pgfscope}%
\begin{pgfscope}%
\pgfsys@transformshift{20.655492in}{1.582790in}%
\pgfsys@useobject{currentmarker}{}%
\end{pgfscope}%
\begin{pgfscope}%
\pgfsys@transformshift{20.688369in}{1.397655in}%
\pgfsys@useobject{currentmarker}{}%
\end{pgfscope}%
\begin{pgfscope}%
\pgfsys@transformshift{20.721247in}{1.498069in}%
\pgfsys@useobject{currentmarker}{}%
\end{pgfscope}%
\begin{pgfscope}%
\pgfsys@transformshift{20.754124in}{1.525278in}%
\pgfsys@useobject{currentmarker}{}%
\end{pgfscope}%
\begin{pgfscope}%
\pgfsys@transformshift{20.787001in}{1.441409in}%
\pgfsys@useobject{currentmarker}{}%
\end{pgfscope}%
\begin{pgfscope}%
\pgfsys@transformshift{20.819878in}{1.503933in}%
\pgfsys@useobject{currentmarker}{}%
\end{pgfscope}%
\begin{pgfscope}%
\pgfsys@transformshift{20.852756in}{1.530001in}%
\pgfsys@useobject{currentmarker}{}%
\end{pgfscope}%
\begin{pgfscope}%
\pgfsys@transformshift{20.885633in}{1.443374in}%
\pgfsys@useobject{currentmarker}{}%
\end{pgfscope}%
\begin{pgfscope}%
\pgfsys@transformshift{20.918510in}{1.593133in}%
\pgfsys@useobject{currentmarker}{}%
\end{pgfscope}%
\begin{pgfscope}%
\pgfsys@transformshift{20.951387in}{1.580275in}%
\pgfsys@useobject{currentmarker}{}%
\end{pgfscope}%
\begin{pgfscope}%
\pgfsys@transformshift{20.984265in}{1.486043in}%
\pgfsys@useobject{currentmarker}{}%
\end{pgfscope}%
\begin{pgfscope}%
\pgfsys@transformshift{21.017142in}{1.472392in}%
\pgfsys@useobject{currentmarker}{}%
\end{pgfscope}%
\end{pgfscope}%
\begin{pgfscope}%
\pgfpathrectangle{\pgfqpoint{1.000000in}{1.000000in}}{\pgfqpoint{8.500000in}{1.000000in}}%
\pgfusepath{clip}%
\pgfsetrectcap%
\pgfsetroundjoin%
\pgfsetlinewidth{0.803000pt}%
\definecolor{currentstroke}{rgb}{0.690196,0.690196,0.690196}%
\pgfsetstrokecolor{currentstroke}%
\pgfsetdash{}{0pt}%
\pgfpathmoveto{\pgfqpoint{2.014084in}{1.000000in}}%
\pgfpathlineto{\pgfqpoint{2.014084in}{2.000000in}}%
\pgfusepath{stroke}%
\end{pgfscope}%
\begin{pgfscope}%
\pgfsetbuttcap%
\pgfsetroundjoin%
\definecolor{currentfill}{rgb}{0.000000,0.000000,0.000000}%
\pgfsetfillcolor{currentfill}%
\pgfsetlinewidth{0.803000pt}%
\definecolor{currentstroke}{rgb}{0.000000,0.000000,0.000000}%
\pgfsetstrokecolor{currentstroke}%
\pgfsetdash{}{0pt}%
\pgfsys@defobject{currentmarker}{\pgfqpoint{0.000000in}{-0.048611in}}{\pgfqpoint{0.000000in}{0.000000in}}{%
\pgfpathmoveto{\pgfqpoint{0.000000in}{0.000000in}}%
\pgfpathlineto{\pgfqpoint{0.000000in}{-0.048611in}}%
\pgfusepath{stroke,fill}%
}%
\begin{pgfscope}%
\pgfsys@transformshift{2.014084in}{1.000000in}%
\pgfsys@useobject{currentmarker}{}%
\end{pgfscope}%
\end{pgfscope}%
\begin{pgfscope}%
\definecolor{textcolor}{rgb}{0.000000,0.000000,0.000000}%
\pgfsetstrokecolor{textcolor}%
\pgfsetfillcolor{textcolor}%
\pgftext[x=2.014084in,y=0.902778in,,top]{\color{textcolor}\sffamily\fontsize{20.000000}{24.000000}\selectfont \(\displaystyle {450}\)}%
\end{pgfscope}%
\begin{pgfscope}%
\pgfpathrectangle{\pgfqpoint{1.000000in}{1.000000in}}{\pgfqpoint{8.500000in}{1.000000in}}%
\pgfusepath{clip}%
\pgfsetrectcap%
\pgfsetroundjoin%
\pgfsetlinewidth{0.803000pt}%
\definecolor{currentstroke}{rgb}{0.690196,0.690196,0.690196}%
\pgfsetstrokecolor{currentstroke}%
\pgfsetdash{}{0pt}%
\pgfpathmoveto{\pgfqpoint{3.657947in}{1.000000in}}%
\pgfpathlineto{\pgfqpoint{3.657947in}{2.000000in}}%
\pgfusepath{stroke}%
\end{pgfscope}%
\begin{pgfscope}%
\pgfsetbuttcap%
\pgfsetroundjoin%
\definecolor{currentfill}{rgb}{0.000000,0.000000,0.000000}%
\pgfsetfillcolor{currentfill}%
\pgfsetlinewidth{0.803000pt}%
\definecolor{currentstroke}{rgb}{0.000000,0.000000,0.000000}%
\pgfsetstrokecolor{currentstroke}%
\pgfsetdash{}{0pt}%
\pgfsys@defobject{currentmarker}{\pgfqpoint{0.000000in}{-0.048611in}}{\pgfqpoint{0.000000in}{0.000000in}}{%
\pgfpathmoveto{\pgfqpoint{0.000000in}{0.000000in}}%
\pgfpathlineto{\pgfqpoint{0.000000in}{-0.048611in}}%
\pgfusepath{stroke,fill}%
}%
\begin{pgfscope}%
\pgfsys@transformshift{3.657947in}{1.000000in}%
\pgfsys@useobject{currentmarker}{}%
\end{pgfscope}%
\end{pgfscope}%
\begin{pgfscope}%
\definecolor{textcolor}{rgb}{0.000000,0.000000,0.000000}%
\pgfsetstrokecolor{textcolor}%
\pgfsetfillcolor{textcolor}%
\pgftext[x=3.657947in,y=0.902778in,,top]{\color{textcolor}\sffamily\fontsize{20.000000}{24.000000}\selectfont \(\displaystyle {500}\)}%
\end{pgfscope}%
\begin{pgfscope}%
\pgfpathrectangle{\pgfqpoint{1.000000in}{1.000000in}}{\pgfqpoint{8.500000in}{1.000000in}}%
\pgfusepath{clip}%
\pgfsetrectcap%
\pgfsetroundjoin%
\pgfsetlinewidth{0.803000pt}%
\definecolor{currentstroke}{rgb}{0.690196,0.690196,0.690196}%
\pgfsetstrokecolor{currentstroke}%
\pgfsetdash{}{0pt}%
\pgfpathmoveto{\pgfqpoint{5.301810in}{1.000000in}}%
\pgfpathlineto{\pgfqpoint{5.301810in}{2.000000in}}%
\pgfusepath{stroke}%
\end{pgfscope}%
\begin{pgfscope}%
\pgfsetbuttcap%
\pgfsetroundjoin%
\definecolor{currentfill}{rgb}{0.000000,0.000000,0.000000}%
\pgfsetfillcolor{currentfill}%
\pgfsetlinewidth{0.803000pt}%
\definecolor{currentstroke}{rgb}{0.000000,0.000000,0.000000}%
\pgfsetstrokecolor{currentstroke}%
\pgfsetdash{}{0pt}%
\pgfsys@defobject{currentmarker}{\pgfqpoint{0.000000in}{-0.048611in}}{\pgfqpoint{0.000000in}{0.000000in}}{%
\pgfpathmoveto{\pgfqpoint{0.000000in}{0.000000in}}%
\pgfpathlineto{\pgfqpoint{0.000000in}{-0.048611in}}%
\pgfusepath{stroke,fill}%
}%
\begin{pgfscope}%
\pgfsys@transformshift{5.301810in}{1.000000in}%
\pgfsys@useobject{currentmarker}{}%
\end{pgfscope}%
\end{pgfscope}%
\begin{pgfscope}%
\definecolor{textcolor}{rgb}{0.000000,0.000000,0.000000}%
\pgfsetstrokecolor{textcolor}%
\pgfsetfillcolor{textcolor}%
\pgftext[x=5.301810in,y=0.902778in,,top]{\color{textcolor}\sffamily\fontsize{20.000000}{24.000000}\selectfont \(\displaystyle {550}\)}%
\end{pgfscope}%
\begin{pgfscope}%
\pgfpathrectangle{\pgfqpoint{1.000000in}{1.000000in}}{\pgfqpoint{8.500000in}{1.000000in}}%
\pgfusepath{clip}%
\pgfsetrectcap%
\pgfsetroundjoin%
\pgfsetlinewidth{0.803000pt}%
\definecolor{currentstroke}{rgb}{0.690196,0.690196,0.690196}%
\pgfsetstrokecolor{currentstroke}%
\pgfsetdash{}{0pt}%
\pgfpathmoveto{\pgfqpoint{6.945673in}{1.000000in}}%
\pgfpathlineto{\pgfqpoint{6.945673in}{2.000000in}}%
\pgfusepath{stroke}%
\end{pgfscope}%
\begin{pgfscope}%
\pgfsetbuttcap%
\pgfsetroundjoin%
\definecolor{currentfill}{rgb}{0.000000,0.000000,0.000000}%
\pgfsetfillcolor{currentfill}%
\pgfsetlinewidth{0.803000pt}%
\definecolor{currentstroke}{rgb}{0.000000,0.000000,0.000000}%
\pgfsetstrokecolor{currentstroke}%
\pgfsetdash{}{0pt}%
\pgfsys@defobject{currentmarker}{\pgfqpoint{0.000000in}{-0.048611in}}{\pgfqpoint{0.000000in}{0.000000in}}{%
\pgfpathmoveto{\pgfqpoint{0.000000in}{0.000000in}}%
\pgfpathlineto{\pgfqpoint{0.000000in}{-0.048611in}}%
\pgfusepath{stroke,fill}%
}%
\begin{pgfscope}%
\pgfsys@transformshift{6.945673in}{1.000000in}%
\pgfsys@useobject{currentmarker}{}%
\end{pgfscope}%
\end{pgfscope}%
\begin{pgfscope}%
\definecolor{textcolor}{rgb}{0.000000,0.000000,0.000000}%
\pgfsetstrokecolor{textcolor}%
\pgfsetfillcolor{textcolor}%
\pgftext[x=6.945673in,y=0.902778in,,top]{\color{textcolor}\sffamily\fontsize{20.000000}{24.000000}\selectfont \(\displaystyle {600}\)}%
\end{pgfscope}%
\begin{pgfscope}%
\pgfpathrectangle{\pgfqpoint{1.000000in}{1.000000in}}{\pgfqpoint{8.500000in}{1.000000in}}%
\pgfusepath{clip}%
\pgfsetrectcap%
\pgfsetroundjoin%
\pgfsetlinewidth{0.803000pt}%
\definecolor{currentstroke}{rgb}{0.690196,0.690196,0.690196}%
\pgfsetstrokecolor{currentstroke}%
\pgfsetdash{}{0pt}%
\pgfpathmoveto{\pgfqpoint{8.589536in}{1.000000in}}%
\pgfpathlineto{\pgfqpoint{8.589536in}{2.000000in}}%
\pgfusepath{stroke}%
\end{pgfscope}%
\begin{pgfscope}%
\pgfsetbuttcap%
\pgfsetroundjoin%
\definecolor{currentfill}{rgb}{0.000000,0.000000,0.000000}%
\pgfsetfillcolor{currentfill}%
\pgfsetlinewidth{0.803000pt}%
\definecolor{currentstroke}{rgb}{0.000000,0.000000,0.000000}%
\pgfsetstrokecolor{currentstroke}%
\pgfsetdash{}{0pt}%
\pgfsys@defobject{currentmarker}{\pgfqpoint{0.000000in}{-0.048611in}}{\pgfqpoint{0.000000in}{0.000000in}}{%
\pgfpathmoveto{\pgfqpoint{0.000000in}{0.000000in}}%
\pgfpathlineto{\pgfqpoint{0.000000in}{-0.048611in}}%
\pgfusepath{stroke,fill}%
}%
\begin{pgfscope}%
\pgfsys@transformshift{8.589536in}{1.000000in}%
\pgfsys@useobject{currentmarker}{}%
\end{pgfscope}%
\end{pgfscope}%
\begin{pgfscope}%
\definecolor{textcolor}{rgb}{0.000000,0.000000,0.000000}%
\pgfsetstrokecolor{textcolor}%
\pgfsetfillcolor{textcolor}%
\pgftext[x=8.589536in,y=0.902778in,,top]{\color{textcolor}\sffamily\fontsize{20.000000}{24.000000}\selectfont \(\displaystyle {650}\)}%
\end{pgfscope}%
\begin{pgfscope}%
\definecolor{textcolor}{rgb}{0.000000,0.000000,0.000000}%
\pgfsetstrokecolor{textcolor}%
\pgfsetfillcolor{textcolor}%
\pgftext[x=5.250000in,y=0.591155in,,top]{\color{textcolor}\sffamily\fontsize{20.000000}{24.000000}\selectfont \(\displaystyle \mathrm{t}/\si{ns}\)}%
\end{pgfscope}%
\begin{pgfscope}%
\pgfpathrectangle{\pgfqpoint{1.000000in}{1.000000in}}{\pgfqpoint{8.500000in}{1.000000in}}%
\pgfusepath{clip}%
\pgfsetrectcap%
\pgfsetroundjoin%
\pgfsetlinewidth{0.803000pt}%
\definecolor{currentstroke}{rgb}{0.690196,0.690196,0.690196}%
\pgfsetstrokecolor{currentstroke}%
\pgfsetdash{}{0pt}%
\pgfpathmoveto{\pgfqpoint{1.000000in}{1.000000in}}%
\pgfpathlineto{\pgfqpoint{9.500000in}{1.000000in}}%
\pgfusepath{stroke}%
\end{pgfscope}%
\begin{pgfscope}%
\pgfsetbuttcap%
\pgfsetroundjoin%
\definecolor{currentfill}{rgb}{0.000000,0.000000,0.000000}%
\pgfsetfillcolor{currentfill}%
\pgfsetlinewidth{0.803000pt}%
\definecolor{currentstroke}{rgb}{0.000000,0.000000,0.000000}%
\pgfsetstrokecolor{currentstroke}%
\pgfsetdash{}{0pt}%
\pgfsys@defobject{currentmarker}{\pgfqpoint{-0.048611in}{0.000000in}}{\pgfqpoint{-0.000000in}{0.000000in}}{%
\pgfpathmoveto{\pgfqpoint{-0.000000in}{0.000000in}}%
\pgfpathlineto{\pgfqpoint{-0.048611in}{0.000000in}}%
\pgfusepath{stroke,fill}%
}%
\begin{pgfscope}%
\pgfsys@transformshift{1.000000in}{1.000000in}%
\pgfsys@useobject{currentmarker}{}%
\end{pgfscope}%
\end{pgfscope}%
\begin{pgfscope}%
\definecolor{textcolor}{rgb}{0.000000,0.000000,0.000000}%
\pgfsetstrokecolor{textcolor}%
\pgfsetfillcolor{textcolor}%
\pgftext[x=0.546626in, y=0.899981in, left, base]{\color{textcolor}\sffamily\fontsize{20.000000}{24.000000}\selectfont \(\displaystyle {-5}\)}%
\end{pgfscope}%
\begin{pgfscope}%
\pgfpathrectangle{\pgfqpoint{1.000000in}{1.000000in}}{\pgfqpoint{8.500000in}{1.000000in}}%
\pgfusepath{clip}%
\pgfsetrectcap%
\pgfsetroundjoin%
\pgfsetlinewidth{0.803000pt}%
\definecolor{currentstroke}{rgb}{0.690196,0.690196,0.690196}%
\pgfsetstrokecolor{currentstroke}%
\pgfsetdash{}{0pt}%
\pgfpathmoveto{\pgfqpoint{1.000000in}{1.500000in}}%
\pgfpathlineto{\pgfqpoint{9.500000in}{1.500000in}}%
\pgfusepath{stroke}%
\end{pgfscope}%
\begin{pgfscope}%
\pgfsetbuttcap%
\pgfsetroundjoin%
\definecolor{currentfill}{rgb}{0.000000,0.000000,0.000000}%
\pgfsetfillcolor{currentfill}%
\pgfsetlinewidth{0.803000pt}%
\definecolor{currentstroke}{rgb}{0.000000,0.000000,0.000000}%
\pgfsetstrokecolor{currentstroke}%
\pgfsetdash{}{0pt}%
\pgfsys@defobject{currentmarker}{\pgfqpoint{-0.048611in}{0.000000in}}{\pgfqpoint{-0.000000in}{0.000000in}}{%
\pgfpathmoveto{\pgfqpoint{-0.000000in}{0.000000in}}%
\pgfpathlineto{\pgfqpoint{-0.048611in}{0.000000in}}%
\pgfusepath{stroke,fill}%
}%
\begin{pgfscope}%
\pgfsys@transformshift{1.000000in}{1.500000in}%
\pgfsys@useobject{currentmarker}{}%
\end{pgfscope}%
\end{pgfscope}%
\begin{pgfscope}%
\definecolor{textcolor}{rgb}{0.000000,0.000000,0.000000}%
\pgfsetstrokecolor{textcolor}%
\pgfsetfillcolor{textcolor}%
\pgftext[x=0.770670in, y=1.399981in, left, base]{\color{textcolor}\sffamily\fontsize{20.000000}{24.000000}\selectfont \(\displaystyle {0}\)}%
\end{pgfscope}%
\begin{pgfscope}%
\pgfpathrectangle{\pgfqpoint{1.000000in}{1.000000in}}{\pgfqpoint{8.500000in}{1.000000in}}%
\pgfusepath{clip}%
\pgfsetrectcap%
\pgfsetroundjoin%
\pgfsetlinewidth{0.803000pt}%
\definecolor{currentstroke}{rgb}{0.690196,0.690196,0.690196}%
\pgfsetstrokecolor{currentstroke}%
\pgfsetdash{}{0pt}%
\pgfpathmoveto{\pgfqpoint{1.000000in}{2.000000in}}%
\pgfpathlineto{\pgfqpoint{9.500000in}{2.000000in}}%
\pgfusepath{stroke}%
\end{pgfscope}%
\begin{pgfscope}%
\pgfsetbuttcap%
\pgfsetroundjoin%
\definecolor{currentfill}{rgb}{0.000000,0.000000,0.000000}%
\pgfsetfillcolor{currentfill}%
\pgfsetlinewidth{0.803000pt}%
\definecolor{currentstroke}{rgb}{0.000000,0.000000,0.000000}%
\pgfsetstrokecolor{currentstroke}%
\pgfsetdash{}{0pt}%
\pgfsys@defobject{currentmarker}{\pgfqpoint{-0.048611in}{0.000000in}}{\pgfqpoint{-0.000000in}{0.000000in}}{%
\pgfpathmoveto{\pgfqpoint{-0.000000in}{0.000000in}}%
\pgfpathlineto{\pgfqpoint{-0.048611in}{0.000000in}}%
\pgfusepath{stroke,fill}%
}%
\begin{pgfscope}%
\pgfsys@transformshift{1.000000in}{2.000000in}%
\pgfsys@useobject{currentmarker}{}%
\end{pgfscope}%
\end{pgfscope}%
\begin{pgfscope}%
\definecolor{textcolor}{rgb}{0.000000,0.000000,0.000000}%
\pgfsetstrokecolor{textcolor}%
\pgfsetfillcolor{textcolor}%
\pgftext[x=0.770670in, y=1.899981in, left, base]{\color{textcolor}\sffamily\fontsize{20.000000}{24.000000}\selectfont \(\displaystyle {5}\)}%
\end{pgfscope}%
\begin{pgfscope}%
\definecolor{textcolor}{rgb}{0.000000,0.000000,0.000000}%
\pgfsetstrokecolor{textcolor}%
\pgfsetfillcolor{textcolor}%
\pgftext[x=0.491071in,y=1.500000in,,bottom,rotate=90.000000]{\color{textcolor}\sffamily\fontsize{20.000000}{24.000000}\selectfont \(\displaystyle \mathrm{Voltage}/\si{mV}\)}%
\end{pgfscope}%
\begin{pgfscope}%
\pgfsetrectcap%
\pgfsetmiterjoin%
\pgfsetlinewidth{0.803000pt}%
\definecolor{currentstroke}{rgb}{0.000000,0.000000,0.000000}%
\pgfsetstrokecolor{currentstroke}%
\pgfsetdash{}{0pt}%
\pgfpathmoveto{\pgfqpoint{1.000000in}{1.000000in}}%
\pgfpathlineto{\pgfqpoint{1.000000in}{2.000000in}}%
\pgfusepath{stroke}%
\end{pgfscope}%
\begin{pgfscope}%
\pgfsetrectcap%
\pgfsetmiterjoin%
\pgfsetlinewidth{0.803000pt}%
\definecolor{currentstroke}{rgb}{0.000000,0.000000,0.000000}%
\pgfsetstrokecolor{currentstroke}%
\pgfsetdash{}{0pt}%
\pgfpathmoveto{\pgfqpoint{9.500000in}{1.000000in}}%
\pgfpathlineto{\pgfqpoint{9.500000in}{2.000000in}}%
\pgfusepath{stroke}%
\end{pgfscope}%
\begin{pgfscope}%
\pgfsetrectcap%
\pgfsetmiterjoin%
\pgfsetlinewidth{0.803000pt}%
\definecolor{currentstroke}{rgb}{0.000000,0.000000,0.000000}%
\pgfsetstrokecolor{currentstroke}%
\pgfsetdash{}{0pt}%
\pgfpathmoveto{\pgfqpoint{1.000000in}{1.000000in}}%
\pgfpathlineto{\pgfqpoint{9.500000in}{1.000000in}}%
\pgfusepath{stroke}%
\end{pgfscope}%
\begin{pgfscope}%
\pgfsetrectcap%
\pgfsetmiterjoin%
\pgfsetlinewidth{0.803000pt}%
\definecolor{currentstroke}{rgb}{0.000000,0.000000,0.000000}%
\pgfsetstrokecolor{currentstroke}%
\pgfsetdash{}{0pt}%
\pgfpathmoveto{\pgfqpoint{1.000000in}{2.000000in}}%
\pgfpathlineto{\pgfqpoint{9.500000in}{2.000000in}}%
\pgfusepath{stroke}%
\end{pgfscope}%
\begin{pgfscope}%
\pgfsetbuttcap%
\pgfsetmiterjoin%
\definecolor{currentfill}{rgb}{1.000000,1.000000,1.000000}%
\pgfsetfillcolor{currentfill}%
\pgfsetfillopacity{0.800000}%
\pgfsetlinewidth{1.003750pt}%
\definecolor{currentstroke}{rgb}{0.800000,0.800000,0.800000}%
\pgfsetstrokecolor{currentstroke}%
\pgfsetstrokeopacity{0.800000}%
\pgfsetdash{}{0pt}%
\pgfpathmoveto{\pgfqpoint{7.459429in}{1.382821in}}%
\pgfpathlineto{\pgfqpoint{9.305556in}{1.382821in}}%
\pgfpathquadraticcurveto{\pgfqpoint{9.361111in}{1.382821in}}{\pgfqpoint{9.361111in}{1.438377in}}%
\pgfpathlineto{\pgfqpoint{9.361111in}{1.805556in}}%
\pgfpathquadraticcurveto{\pgfqpoint{9.361111in}{1.861111in}}{\pgfqpoint{9.305556in}{1.861111in}}%
\pgfpathlineto{\pgfqpoint{7.459429in}{1.861111in}}%
\pgfpathquadraticcurveto{\pgfqpoint{7.403873in}{1.861111in}}{\pgfqpoint{7.403873in}{1.805556in}}%
\pgfpathlineto{\pgfqpoint{7.403873in}{1.438377in}}%
\pgfpathquadraticcurveto{\pgfqpoint{7.403873in}{1.382821in}}{\pgfqpoint{7.459429in}{1.382821in}}%
\pgfpathlineto{\pgfqpoint{7.459429in}{1.382821in}}%
\pgfpathclose%
\pgfusepath{stroke,fill}%
\end{pgfscope}%
\begin{pgfscope}%
\pgfsetbuttcap%
\pgfsetroundjoin%
\definecolor{currentfill}{rgb}{0.000000,0.000000,0.000000}%
\pgfsetfillcolor{currentfill}%
\pgfsetlinewidth{1.003750pt}%
\definecolor{currentstroke}{rgb}{0.000000,0.000000,0.000000}%
\pgfsetstrokecolor{currentstroke}%
\pgfsetdash{}{0pt}%
\pgfsys@defobject{currentmarker}{\pgfqpoint{-0.013889in}{-0.013889in}}{\pgfqpoint{0.013889in}{0.013889in}}{%
\pgfpathmoveto{\pgfqpoint{0.000000in}{-0.013889in}}%
\pgfpathcurveto{\pgfqpoint{0.003683in}{-0.013889in}}{\pgfqpoint{0.007216in}{-0.012425in}}{\pgfqpoint{0.009821in}{-0.009821in}}%
\pgfpathcurveto{\pgfqpoint{0.012425in}{-0.007216in}}{\pgfqpoint{0.013889in}{-0.003683in}}{\pgfqpoint{0.013889in}{0.000000in}}%
\pgfpathcurveto{\pgfqpoint{0.013889in}{0.003683in}}{\pgfqpoint{0.012425in}{0.007216in}}{\pgfqpoint{0.009821in}{0.009821in}}%
\pgfpathcurveto{\pgfqpoint{0.007216in}{0.012425in}}{\pgfqpoint{0.003683in}{0.013889in}}{\pgfqpoint{0.000000in}{0.013889in}}%
\pgfpathcurveto{\pgfqpoint{-0.003683in}{0.013889in}}{\pgfqpoint{-0.007216in}{0.012425in}}{\pgfqpoint{-0.009821in}{0.009821in}}%
\pgfpathcurveto{\pgfqpoint{-0.012425in}{0.007216in}}{\pgfqpoint{-0.013889in}{0.003683in}}{\pgfqpoint{-0.013889in}{0.000000in}}%
\pgfpathcurveto{\pgfqpoint{-0.013889in}{-0.003683in}}{\pgfqpoint{-0.012425in}{-0.007216in}}{\pgfqpoint{-0.009821in}{-0.009821in}}%
\pgfpathcurveto{\pgfqpoint{-0.007216in}{-0.012425in}}{\pgfqpoint{-0.003683in}{-0.013889in}}{\pgfqpoint{0.000000in}{-0.013889in}}%
\pgfpathlineto{\pgfqpoint{0.000000in}{-0.013889in}}%
\pgfpathclose%
\pgfusepath{stroke,fill}%
}%
\begin{pgfscope}%
\pgfsys@transformshift{7.792762in}{1.622878in}%
\pgfsys@useobject{currentmarker}{}%
\end{pgfscope}%
\end{pgfscope}%
\begin{pgfscope}%
\definecolor{textcolor}{rgb}{0.000000,0.000000,0.000000}%
\pgfsetstrokecolor{textcolor}%
\pgfsetfillcolor{textcolor}%
\pgftext[x=8.292762in,y=1.549962in,left,base]{\color{textcolor}\sffamily\fontsize{20.000000}{24.000000}\selectfont residuals}%
\end{pgfscope}%
\end{pgfpicture}%
\makeatother%
\endgroup%
}
    \caption{\label{fig:fsmp} An example giving \\ $\Delta{t_0}=\SI{-3.97}{ns}$, $\mathrm{RSS}=\SI{17.8}{mV^2}$, $D_\mathrm{w}=\SI{0.64}{ns}$.}
  \end{subfigure}
  \caption{\label{fig:fsmp-performance}Demonstration of FSMP with $\num[retain-unity-mantissa=false]{1e4}$ waveforms in~\subref{fig:fsmp-npe} and one waveform in~\subref{fig:fsmp} sampled from the same setup as figure~\ref{fig:method}.  FSMP reconstructs the waveform and charges flawlessly.}
\end{figure}
In terms of $D_\mathrm{w}$, figure~\ref{fig:fsmp-npe} shows that FSMP is on par with CNN in figure~\ref{fig:cnn-npe}.  Figure~\ref{fig:fsmp} is a perfect reconstruction example where the true and reconstructed charges and waveforms overlap.  A bias of $\Delta{t_0}=\SI{-3.97}{ns}$ aligns with $\hat{t}_\mathrm{ALL}$ in eq.~\eqref{eq:2}, which will be covered in section~\ref{subsec:timeresolution}.  The superior performance of FSMP attributes to sparsity and positiveness of $q'_i$, correct modeling of $V_\mathrm{PE}$, $q'$ distribution and white noise.

Estimators for $t_0$ and $\mu$ in eq.~\eqref{eq:fsmpcharge} is an elegant interface to event reconstruction, eliminating the need of $\hat{t}_\mathrm{KL}$ and $\hat{\mu}_\mathrm{KL}$ in section~\ref{sec:pseudo}.

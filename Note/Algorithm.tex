\section{Algorithms and their performance}
\label{sec:algorithm}

Waveform analysis is to obtain $t_i$ and $q_i$ estimators $\hat{t}_i$ and $\hat{q}_i$ from waveform $w(t)$, where the output indices $i$ are from 1 to $\hat{N}_\mathrm{PE}$ and $\hat{N}_\mathrm{PE}$ is an estimator of $N_\mathrm{PE}$ in eq.~\eqref{eq:lc-sample}. Figure~\ref{fig:pile} illustrates the input waveform $w(t)$ and the outputs charge $\bm{\hat{t}}, \hat{\bm{q}}$ obtained from $w(t)$, where boldface $\hat{\bm{t}}$ denotes the vector $\hat{t}_i$. 

$\hat{N}_\mathrm{PE}$ may fail to estimate $N_\mathrm{PE}$ due to the fluctuation of $q_i$ and the ambiguity of $\hat{t}_i$. For example, 1, 2 and even 3~PEs can generate the same charge as $1.6$ units.  A single PE charged $1$ might be misinterpreted as 2~PEs at consecutive $\hat{t}_i$ and $\hat{t}_{i+1}$ with $\hat{q}_i=\hat{q}_{i+1}=0.5$.

\subsection{Evaluation criteria}
\label{sec:criteria}
Subject to such ambiguity of $t_i/q_i$, we introduce a set of evaluation criteria to assess the algorithms' performance.

\subsubsection{Kullback-Leibler divergence}
\label{sec:pseudo}

We construct a light curve estimator $\hat{\phi}(t)$ from $\bm{\hat{t}}$, $\bm{\hat{q}}$ and $\hat{N}_\mathrm{PE}$,
\begin{equation}
  \label{eq:lc}
  \hat{\phi}(t) = \sum_{i=1}^{\hat{N}_\mathrm{PE}} \hat{q}_i\delta(t-\hat{t}_i),
\end{equation}
which resembles eq.~\eqref{eq:lc-sample}.

Basu et al.'s \textit{density power divergence}~\cite{basu_robust_1998} contains the classical Kullback-Leibler~(KL) divergence~\cite{kullback_information_1951} as a special case.  Non-normalized KL divergence is defined accordingly if we do not normalize $\hat{\phi}(t)$ and $\mu \phi(t-t_{0})$ to 1 when considering their divergence in eq.~\eqref{eq:kl},
\begin{equation}
  \begin{aligned}
    D_\mathrm{KL}\left[\hat{\phi}(t) \parallel \mu\phi(t-t_0)\right] & =\int \left[\hat{\phi}(t) \log\frac{\hat{\phi}(t)}{\mu\phi(t-t_0)} + \mu\phi(t-t_0) - \hat{\phi}(t) \right]\mathrm{d}t \\
    & = - \int \hat{\phi}(t) \log\phi(t-t_0)\mathrm{d}t - \log(\mu)\int\hat{\phi}(t)\mathrm{d}t + \mu + \int \left[\hat{\phi}(t) \log\hat{\phi}(t) - \hat{\phi}(t) \right]\mathrm{d}t \\
    & = - \sum_{i=1}^{\hat{N}_\mathrm{PE}}\left[\int \hat{q}_i\delta(t-\hat{t_i}) \log\phi(t-t_0)\mathrm{d}t - \log(\mu)\int\hat{q}_i\delta(t-\hat{t_i})\mathrm{d}t\right] + \mu +  C \\
    & = -\log \left\{\prod_{i=1}^{\hat{N}_\mathrm{PE}} \left[\phi(\hat{t}_i-t_0)\right]^{\hat{q}_i}\right\} - \log(\mu)\sum_{i=1}^{\hat{N}_\mathrm{PE}} \hat{q}_i + \mu + C
  \label{eq:kl}
  \end{aligned}
\end{equation}
where $C$ is a constant regarding $t_0$ and $\mu$.  Define the time KL estimator as
\begin{equation}
  \begin{aligned}
  \label{eq:pseudo}
  \hat{t}_\mathrm{KL} &= \arg\underset{t_0}{\min}~D_\mathrm{KL}\left[\hat{\phi}(t) \parallel \mu\phi(t-t_0)\right] \\
  &= \arg\underset{t_0}{\max} \prod_{i=1}^{\hat{N}_\mathrm{PE}} \left[\phi(\hat{t}_i-t_0)\right]^{\hat{q}_i},
  \end{aligned}
\end{equation}
which reduces to an MLE like eq.~\eqref{eq:2} if $\hat{q}_i\equiv 1$.  $\hat{t}_\mathrm{KL}$ estimates $t_0$ when $t_i, q_i, N_\mathrm{PE}$ are all uncertain.
Similar to $\hat{t}_\mathrm{1st}$ and $\hat{t}_\mathrm{ALL}$, we define the standard deviation $\sqrt{\Var[\hat{t}_\mathrm{KL} - t_0]}$ to the resolution of an algorithm via KL divergence.

The intensity KL estimator is,
\begin{equation}
  \label{eq:pseudo-mu}
  \hat{\mu}_\mathrm{KL} = \arg\underset{\mu}{\min}~D_\mathrm{KL}\left[\hat{\phi}(t) \parallel \mu\phi(t-t_0)\right] = \sum_{i=1}^{\hat{N}_\mathrm{PE}} \hat{q}_i.
\end{equation}


\subsubsection{Residual sum of squares}
\label{sec:rss}

Following eqs.~\eqref{eq:1} and~\eqref{eq:lc}, we construct an estimator of a waveform,
\begin{equation}
  \label{eq:w-hat}
  \hat{w}(t) = \sum_{i=1}^{\hat{N}_\mathrm{PE}}\hat{q}_i V_\mathrm{PE}(t-\hat{t}_i) = \hat{\phi}(t) \otimes V_\mathrm{PE}(t).
\end{equation}

For a noise-free evaluation of $\hat{w}(t)$, residual sum of squares~(RSS) is a $\ell_2$-distance of it to $\tilde{w}(t)$,
\begin{equation}
  \label{eq:rss}
  \mathrm{RSS} \coloneqq\int\left[\hat{w}(t) - \tilde{w}(t)\right]^2\mathrm{d}t.
\end{equation}
We choose $\tilde{w}(t)$ for evaluating algorithms because otherwise with the raw waveform $w(t)$ RSS will be dominated by the white noise term $\epsilon(t)$.

Figure~\ref{fig:l2} demonstrates that if two functions do not overlap, their $\mathrm{RSS}$ remain constant regardless of relative positions.  The delta functions in the sampled light curves $\hat{\phi}(t)$ and $\tilde{\phi}(t)$ hardly overlap, rendering $\mathrm{RSS}$ useless.  Furthermore, RSS cannot compare a discrete function with a continuous one.  We shall only consider the $\mathrm{RSS}$ of waveforms.

\begin{figure}[H]
  \centering
  \resizebox{0.6\textwidth}{!}{version https://git-lfs.github.com/spec/v1
oid sha256:89fccffb60ec36abfc346cf0582b28d165f2dbfa373f1db789d373157dc2446a
size 66545
}
  \caption{\label{fig:l2} The $\mathrm{RSS}$ of red and blue curves is a function of the two shaded regions. It is a constant when the curves shift horizontally when they do not overlap.  In contrast, the Wasserstein distance $D_\mathrm{w}$ of the two curves is associated with their separation.  It complements $\mathrm{RSS}$ and offers a time-sensitive metric suitable for the sparse PE space.}
\end{figure}

\subsubsection{Wasserstein distance}
\label{sec:W-dist}

Wasserstein distance $D_\mathrm{w}$ is a metric between two distributions, either of which can be discrete or continuous. It can capture the difference between a waveform analysis result $\hat{\phi}$ and the sampled light curve $\tilde{\phi}(t)$ in eq.~\eqref{eq:lc-sample}.
\begin{equation}
  D_\mathrm{w}\left[\hat{\phi}_*, \tilde{\phi}_*\right] = \inf_{\gamma \in \Gamma} \left[\int \left\vert t_1 - t_2 \right\vert \gamma(t_1, t_2)\mathrm{d}t_1\mathrm{d}t_2\right],
\end{equation}
where $*$ denotes the normalized light curves and $\Gamma$ is the collection of joint distributions with marginals $\hat{\phi}_*(t)$ and $\tilde{\phi}_*(t)$,
\begin{equation*}
  \label{eq:joint}
  \Gamma = \left\{\gamma(t_1, t_2) ~\middle\vert~ \int\gamma(t_1,t_2)\mathrm{d}t_1 = \tilde{\phi}_*(t_2) , \int\gamma(t_1,t_2)\mathrm{d}t_2 = \hat{\phi}_*(t_1) \right\}.
\end{equation*}
It is also known as the \textit{earth mover's distance}~\cite{levina_earth_2001}, ecoding the minimum cost to transport mass from one distribution to the other, as shown in figure~\ref{fig:l2}.

Alternatively, we can calculate $D_\mathrm{w}$ from cumulative distribution functions (CDF). Let $\hat\Phi(t)$ and $\tilde\Phi(t)$ denote the CDF of $\hat{\phi}_*(t)$ and $\tilde{\phi}_*(t)$, respectively. Then $D_w$ is equivalent to the $\ell_1$-metric between the two CDFs,
\begin{equation}
    D_\mathrm{w}\left[\hat{\phi}_*, \tilde{\phi}_*\right] = \int\left|\hat{\Phi}(t) - \tilde{\Phi}(t)\right| \mathrm{d}t.
    \label{eq:numerical}
\end{equation}

In the following, we assess the performance of waveform analysis algorithms ranging from heuristics, deconvolution, neural network to regression by the criteria discussed in this section.

\subsection{Heuristic methods}
By directly extracting the patterns in the waveforms, \textit{heuristics} refer to the methods making minimal assumptions of the instrumental and statistical features.  Straightforward to implement and widely deployed in neutrino and dark matter experiments~\cite{students22}, they are poorly documented in the literature.  In this section, we try to formulate the heuristics actually have been used in the experiments so as to make an objective comparison with more advanced techniques.

\subsubsection{Waveform shifting}
\label{sec:shifting}
Some experiments use waveforms as direct input of analysis. Proton decay search at KamLAND~\cite{kamland_collaboration_search_2015} sums up all the PMT waveforms after shifting by time-of-flight for each event candidate.  The total waveform shape is used for a $\chi^2$-based particle identification (PID). The Double Chooz experiment also superposes waveforms to extract PID information by Fourier transformation~\cite{chooz_2018}. Samani~et~al.\cite{samani_pulse_2020} extracts pulse width from a raw waveform and use it as a PID discriminator.  Such techniques are extensions of pulse shape discrimination~(PSD) to large neutrino and dark matter experiments.  In the view of this study, extended PSD uses shifted waveform to approximate PE hit pattern, thus named \textit{waveform shifting}.

As illustrated in figure~\ref{fig:shifting}, we firstly select all the $t_i$'s where the waveform $w(t_i)$ exceeds a threshold $V_\mathrm{th}$ to suppress noise, shift them by a constant $\Delta t$. For a SER pulse $V_\mathrm{PE}(t)$ whose truth PE time is $t=0$, $\Delta t$ should minimize the Wasserstein distance $D_\mathrm{w}$. Thus,
\begin{equation}
    \Delta t \equiv \arg\underset{\Delta t'}{\min} D_\mathrm{w}\left[ V_\mathrm{PE*}(t), \delta(t-\Delta t') \right] \implies \int_{0}^{\Delta t} V_\mathrm{PE}(t) \mathrm{d}t = \frac{1}{2} \int_{0}^{\infty} V_\mathrm{PE}(t) \mathrm{d}t.
  \label{eq:waveform-shift-dt}
\end{equation}
The PE times are inferred as $\hat{t}_i = t_i - \Delta t$.  Corresponding $w(t_i)$'s are scaled by $\alpha$ to minimize RSS:
\begin{equation}
  \hat{\alpha} = \arg\underset{\alpha}{\min}~\mathrm{RSS}\left[ \alpha \sum_iw(t_i) \otimes V_\mathrm{PE}(t-\hat{t}_i), w(t) \right] .
  \label{eq:alpha}
\end{equation}
The charges are determined as $\hat{q}_i = \hat{\alpha} w(t_i)$.  Notice the difference from eq.~\eqref{eq:rss}: $\tilde{w}(t)$ unknown in data analysis, we replace it with $w(t)$.

Since the whole over-threshold waveform sample points are treated as PEs, one PE can be split into many. Thus, the obtained $\hat{q}_i$ are smaller than truth PE charges. The waveform shifting model formulated above captures the logic behind waveform superposition methods.  The underlying assumption to treat a waveform as PEs is simply not true and time precision suffers.  It works only if the width of $V_\mathrm{PE}$ is negligible for the purpose, sometimes when classifying events.

\begin{figure}[H]
  \begin{subfigure}{.5\textwidth}
    \centering
    \resizebox{\textwidth}{!}{version https://git-lfs.github.com/spec/v1
oid sha256:4e3ae678726ac99b039f986925475b3087d6237bca7e161114d220cb576afc3e
size 77712
}
    \caption{\label{fig:shifting} A waveform shifting example gives \\ $\hat{t}_\mathrm{KL}-t_0=\SI{2.70}{ns}$, $\mathrm{RSS}=\SI{948.5}{mV^2}$, $D_\mathrm{w}=\SI{3.20}{ns}$.}
  \end{subfigure}
  \begin{subfigure}{.5\textwidth}
    \centering
    \resizebox{\textwidth}{!}{%% Creator: Matplotlib, PGF backend
%%
%% To include the figure in your LaTeX document, write
%%   \input{<filename>.pgf}
%%
%% Make sure the required packages are loaded in your preamble
%%   \usepackage{pgf}
%%
%% Also ensure that all the required font packages are loaded; for instance,
%% the lmodern package is sometimes necessary when using math font.
%%   \usepackage{lmodern}
%%
%% Figures using additional raster images can only be included by \input if
%% they are in the same directory as the main LaTeX file. For loading figures
%% from other directories you can use the `import` package
%%   \usepackage{import}
%%
%% and then include the figures with
%%   \import{<path to file>}{<filename>.pgf}
%%
%% Matplotlib used the following preamble
%%   \usepackage[detect-all,locale=DE]{siunitx}
%%
\begingroup%
\makeatletter%
\begin{pgfpicture}%
\pgfpathrectangle{\pgfpointorigin}{\pgfqpoint{8.000000in}{6.000000in}}%
\pgfusepath{use as bounding box, clip}%
\begin{pgfscope}%
\pgfsetbuttcap%
\pgfsetmiterjoin%
\definecolor{currentfill}{rgb}{1.000000,1.000000,1.000000}%
\pgfsetfillcolor{currentfill}%
\pgfsetlinewidth{0.000000pt}%
\definecolor{currentstroke}{rgb}{1.000000,1.000000,1.000000}%
\pgfsetstrokecolor{currentstroke}%
\pgfsetdash{}{0pt}%
\pgfpathmoveto{\pgfqpoint{0.000000in}{0.000000in}}%
\pgfpathlineto{\pgfqpoint{8.000000in}{0.000000in}}%
\pgfpathlineto{\pgfqpoint{8.000000in}{6.000000in}}%
\pgfpathlineto{\pgfqpoint{0.000000in}{6.000000in}}%
\pgfpathlineto{\pgfqpoint{0.000000in}{0.000000in}}%
\pgfpathclose%
\pgfusepath{fill}%
\end{pgfscope}%
\begin{pgfscope}%
\pgfsetbuttcap%
\pgfsetmiterjoin%
\definecolor{currentfill}{rgb}{1.000000,1.000000,1.000000}%
\pgfsetfillcolor{currentfill}%
\pgfsetlinewidth{0.000000pt}%
\definecolor{currentstroke}{rgb}{0.000000,0.000000,0.000000}%
\pgfsetstrokecolor{currentstroke}%
\pgfsetstrokeopacity{0.000000}%
\pgfsetdash{}{0pt}%
\pgfpathmoveto{\pgfqpoint{1.000000in}{0.720000in}}%
\pgfpathlineto{\pgfqpoint{7.200000in}{0.720000in}}%
\pgfpathlineto{\pgfqpoint{7.200000in}{5.340000in}}%
\pgfpathlineto{\pgfqpoint{1.000000in}{5.340000in}}%
\pgfpathlineto{\pgfqpoint{1.000000in}{0.720000in}}%
\pgfpathclose%
\pgfusepath{fill}%
\end{pgfscope}%
\begin{pgfscope}%
\pgfsetbuttcap%
\pgfsetroundjoin%
\definecolor{currentfill}{rgb}{0.000000,0.000000,0.000000}%
\pgfsetfillcolor{currentfill}%
\pgfsetlinewidth{0.803000pt}%
\definecolor{currentstroke}{rgb}{0.000000,0.000000,0.000000}%
\pgfsetstrokecolor{currentstroke}%
\pgfsetdash{}{0pt}%
\pgfsys@defobject{currentmarker}{\pgfqpoint{0.000000in}{-0.048611in}}{\pgfqpoint{0.000000in}{0.000000in}}{%
\pgfpathmoveto{\pgfqpoint{0.000000in}{0.000000in}}%
\pgfpathlineto{\pgfqpoint{0.000000in}{-0.048611in}}%
\pgfusepath{stroke,fill}%
}%
\begin{pgfscope}%
\pgfsys@transformshift{1.310000in}{0.720000in}%
\pgfsys@useobject{currentmarker}{}%
\end{pgfscope}%
\end{pgfscope}%
\begin{pgfscope}%
\definecolor{textcolor}{rgb}{0.000000,0.000000,0.000000}%
\pgfsetstrokecolor{textcolor}%
\pgfsetfillcolor{textcolor}%
\pgftext[x=1.310000in,y=0.622778in,,top]{\color{textcolor}\sffamily\fontsize{20.000000}{24.000000}\selectfont \(\displaystyle {450}\)}%
\end{pgfscope}%
\begin{pgfscope}%
\pgfsetbuttcap%
\pgfsetroundjoin%
\definecolor{currentfill}{rgb}{0.000000,0.000000,0.000000}%
\pgfsetfillcolor{currentfill}%
\pgfsetlinewidth{0.803000pt}%
\definecolor{currentstroke}{rgb}{0.000000,0.000000,0.000000}%
\pgfsetstrokecolor{currentstroke}%
\pgfsetdash{}{0pt}%
\pgfsys@defobject{currentmarker}{\pgfqpoint{0.000000in}{-0.048611in}}{\pgfqpoint{0.000000in}{0.000000in}}{%
\pgfpathmoveto{\pgfqpoint{0.000000in}{0.000000in}}%
\pgfpathlineto{\pgfqpoint{0.000000in}{-0.048611in}}%
\pgfusepath{stroke,fill}%
}%
\begin{pgfscope}%
\pgfsys@transformshift{2.860000in}{0.720000in}%
\pgfsys@useobject{currentmarker}{}%
\end{pgfscope}%
\end{pgfscope}%
\begin{pgfscope}%
\definecolor{textcolor}{rgb}{0.000000,0.000000,0.000000}%
\pgfsetstrokecolor{textcolor}%
\pgfsetfillcolor{textcolor}%
\pgftext[x=2.860000in,y=0.622778in,,top]{\color{textcolor}\sffamily\fontsize{20.000000}{24.000000}\selectfont \(\displaystyle {500}\)}%
\end{pgfscope}%
\begin{pgfscope}%
\pgfsetbuttcap%
\pgfsetroundjoin%
\definecolor{currentfill}{rgb}{0.000000,0.000000,0.000000}%
\pgfsetfillcolor{currentfill}%
\pgfsetlinewidth{0.803000pt}%
\definecolor{currentstroke}{rgb}{0.000000,0.000000,0.000000}%
\pgfsetstrokecolor{currentstroke}%
\pgfsetdash{}{0pt}%
\pgfsys@defobject{currentmarker}{\pgfqpoint{0.000000in}{-0.048611in}}{\pgfqpoint{0.000000in}{0.000000in}}{%
\pgfpathmoveto{\pgfqpoint{0.000000in}{0.000000in}}%
\pgfpathlineto{\pgfqpoint{0.000000in}{-0.048611in}}%
\pgfusepath{stroke,fill}%
}%
\begin{pgfscope}%
\pgfsys@transformshift{4.410000in}{0.720000in}%
\pgfsys@useobject{currentmarker}{}%
\end{pgfscope}%
\end{pgfscope}%
\begin{pgfscope}%
\definecolor{textcolor}{rgb}{0.000000,0.000000,0.000000}%
\pgfsetstrokecolor{textcolor}%
\pgfsetfillcolor{textcolor}%
\pgftext[x=4.410000in,y=0.622778in,,top]{\color{textcolor}\sffamily\fontsize{20.000000}{24.000000}\selectfont \(\displaystyle {550}\)}%
\end{pgfscope}%
\begin{pgfscope}%
\pgfsetbuttcap%
\pgfsetroundjoin%
\definecolor{currentfill}{rgb}{0.000000,0.000000,0.000000}%
\pgfsetfillcolor{currentfill}%
\pgfsetlinewidth{0.803000pt}%
\definecolor{currentstroke}{rgb}{0.000000,0.000000,0.000000}%
\pgfsetstrokecolor{currentstroke}%
\pgfsetdash{}{0pt}%
\pgfsys@defobject{currentmarker}{\pgfqpoint{0.000000in}{-0.048611in}}{\pgfqpoint{0.000000in}{0.000000in}}{%
\pgfpathmoveto{\pgfqpoint{0.000000in}{0.000000in}}%
\pgfpathlineto{\pgfqpoint{0.000000in}{-0.048611in}}%
\pgfusepath{stroke,fill}%
}%
\begin{pgfscope}%
\pgfsys@transformshift{5.960000in}{0.720000in}%
\pgfsys@useobject{currentmarker}{}%
\end{pgfscope}%
\end{pgfscope}%
\begin{pgfscope}%
\definecolor{textcolor}{rgb}{0.000000,0.000000,0.000000}%
\pgfsetstrokecolor{textcolor}%
\pgfsetfillcolor{textcolor}%
\pgftext[x=5.960000in,y=0.622778in,,top]{\color{textcolor}\sffamily\fontsize{20.000000}{24.000000}\selectfont \(\displaystyle {600}\)}%
\end{pgfscope}%
\begin{pgfscope}%
\definecolor{textcolor}{rgb}{0.000000,0.000000,0.000000}%
\pgfsetstrokecolor{textcolor}%
\pgfsetfillcolor{textcolor}%
\pgftext[x=4.100000in,y=0.311155in,,top]{\color{textcolor}\sffamily\fontsize{20.000000}{24.000000}\selectfont \(\displaystyle \mathrm{t}/\si{ns}\)}%
\end{pgfscope}%
\begin{pgfscope}%
\pgfsetbuttcap%
\pgfsetroundjoin%
\definecolor{currentfill}{rgb}{0.000000,0.000000,0.000000}%
\pgfsetfillcolor{currentfill}%
\pgfsetlinewidth{0.803000pt}%
\definecolor{currentstroke}{rgb}{0.000000,0.000000,0.000000}%
\pgfsetstrokecolor{currentstroke}%
\pgfsetdash{}{0pt}%
\pgfsys@defobject{currentmarker}{\pgfqpoint{-0.048611in}{0.000000in}}{\pgfqpoint{-0.000000in}{0.000000in}}{%
\pgfpathmoveto{\pgfqpoint{-0.000000in}{0.000000in}}%
\pgfpathlineto{\pgfqpoint{-0.048611in}{0.000000in}}%
\pgfusepath{stroke,fill}%
}%
\begin{pgfscope}%
\pgfsys@transformshift{1.000000in}{1.125493in}%
\pgfsys@useobject{currentmarker}{}%
\end{pgfscope}%
\end{pgfscope}%
\begin{pgfscope}%
\definecolor{textcolor}{rgb}{0.000000,0.000000,0.000000}%
\pgfsetstrokecolor{textcolor}%
\pgfsetfillcolor{textcolor}%
\pgftext[x=0.770670in, y=1.025474in, left, base]{\color{textcolor}\sffamily\fontsize{20.000000}{24.000000}\selectfont \(\displaystyle {0}\)}%
\end{pgfscope}%
\begin{pgfscope}%
\pgfsetbuttcap%
\pgfsetroundjoin%
\definecolor{currentfill}{rgb}{0.000000,0.000000,0.000000}%
\pgfsetfillcolor{currentfill}%
\pgfsetlinewidth{0.803000pt}%
\definecolor{currentstroke}{rgb}{0.000000,0.000000,0.000000}%
\pgfsetstrokecolor{currentstroke}%
\pgfsetdash{}{0pt}%
\pgfsys@defobject{currentmarker}{\pgfqpoint{-0.048611in}{0.000000in}}{\pgfqpoint{-0.000000in}{0.000000in}}{%
\pgfpathmoveto{\pgfqpoint{-0.000000in}{0.000000in}}%
\pgfpathlineto{\pgfqpoint{-0.048611in}{0.000000in}}%
\pgfusepath{stroke,fill}%
}%
\begin{pgfscope}%
\pgfsys@transformshift{1.000000in}{2.058714in}%
\pgfsys@useobject{currentmarker}{}%
\end{pgfscope}%
\end{pgfscope}%
\begin{pgfscope}%
\definecolor{textcolor}{rgb}{0.000000,0.000000,0.000000}%
\pgfsetstrokecolor{textcolor}%
\pgfsetfillcolor{textcolor}%
\pgftext[x=0.638563in, y=1.958694in, left, base]{\color{textcolor}\sffamily\fontsize{20.000000}{24.000000}\selectfont \(\displaystyle {10}\)}%
\end{pgfscope}%
\begin{pgfscope}%
\pgfsetbuttcap%
\pgfsetroundjoin%
\definecolor{currentfill}{rgb}{0.000000,0.000000,0.000000}%
\pgfsetfillcolor{currentfill}%
\pgfsetlinewidth{0.803000pt}%
\definecolor{currentstroke}{rgb}{0.000000,0.000000,0.000000}%
\pgfsetstrokecolor{currentstroke}%
\pgfsetdash{}{0pt}%
\pgfsys@defobject{currentmarker}{\pgfqpoint{-0.048611in}{0.000000in}}{\pgfqpoint{-0.000000in}{0.000000in}}{%
\pgfpathmoveto{\pgfqpoint{-0.000000in}{0.000000in}}%
\pgfpathlineto{\pgfqpoint{-0.048611in}{0.000000in}}%
\pgfusepath{stroke,fill}%
}%
\begin{pgfscope}%
\pgfsys@transformshift{1.000000in}{2.991934in}%
\pgfsys@useobject{currentmarker}{}%
\end{pgfscope}%
\end{pgfscope}%
\begin{pgfscope}%
\definecolor{textcolor}{rgb}{0.000000,0.000000,0.000000}%
\pgfsetstrokecolor{textcolor}%
\pgfsetfillcolor{textcolor}%
\pgftext[x=0.638563in, y=2.891915in, left, base]{\color{textcolor}\sffamily\fontsize{20.000000}{24.000000}\selectfont \(\displaystyle {20}\)}%
\end{pgfscope}%
\begin{pgfscope}%
\pgfsetbuttcap%
\pgfsetroundjoin%
\definecolor{currentfill}{rgb}{0.000000,0.000000,0.000000}%
\pgfsetfillcolor{currentfill}%
\pgfsetlinewidth{0.803000pt}%
\definecolor{currentstroke}{rgb}{0.000000,0.000000,0.000000}%
\pgfsetstrokecolor{currentstroke}%
\pgfsetdash{}{0pt}%
\pgfsys@defobject{currentmarker}{\pgfqpoint{-0.048611in}{0.000000in}}{\pgfqpoint{-0.000000in}{0.000000in}}{%
\pgfpathmoveto{\pgfqpoint{-0.000000in}{0.000000in}}%
\pgfpathlineto{\pgfqpoint{-0.048611in}{0.000000in}}%
\pgfusepath{stroke,fill}%
}%
\begin{pgfscope}%
\pgfsys@transformshift{1.000000in}{3.925155in}%
\pgfsys@useobject{currentmarker}{}%
\end{pgfscope}%
\end{pgfscope}%
\begin{pgfscope}%
\definecolor{textcolor}{rgb}{0.000000,0.000000,0.000000}%
\pgfsetstrokecolor{textcolor}%
\pgfsetfillcolor{textcolor}%
\pgftext[x=0.638563in, y=3.825136in, left, base]{\color{textcolor}\sffamily\fontsize{20.000000}{24.000000}\selectfont \(\displaystyle {30}\)}%
\end{pgfscope}%
\begin{pgfscope}%
\pgfsetbuttcap%
\pgfsetroundjoin%
\definecolor{currentfill}{rgb}{0.000000,0.000000,0.000000}%
\pgfsetfillcolor{currentfill}%
\pgfsetlinewidth{0.803000pt}%
\definecolor{currentstroke}{rgb}{0.000000,0.000000,0.000000}%
\pgfsetstrokecolor{currentstroke}%
\pgfsetdash{}{0pt}%
\pgfsys@defobject{currentmarker}{\pgfqpoint{-0.048611in}{0.000000in}}{\pgfqpoint{-0.000000in}{0.000000in}}{%
\pgfpathmoveto{\pgfqpoint{-0.000000in}{0.000000in}}%
\pgfpathlineto{\pgfqpoint{-0.048611in}{0.000000in}}%
\pgfusepath{stroke,fill}%
}%
\begin{pgfscope}%
\pgfsys@transformshift{1.000000in}{4.858376in}%
\pgfsys@useobject{currentmarker}{}%
\end{pgfscope}%
\end{pgfscope}%
\begin{pgfscope}%
\definecolor{textcolor}{rgb}{0.000000,0.000000,0.000000}%
\pgfsetstrokecolor{textcolor}%
\pgfsetfillcolor{textcolor}%
\pgftext[x=0.638563in, y=4.758357in, left, base]{\color{textcolor}\sffamily\fontsize{20.000000}{24.000000}\selectfont \(\displaystyle {40}\)}%
\end{pgfscope}%
\begin{pgfscope}%
\definecolor{textcolor}{rgb}{0.000000,0.000000,0.000000}%
\pgfsetstrokecolor{textcolor}%
\pgfsetfillcolor{textcolor}%
\pgftext[x=0.583007in,y=3.030000in,,bottom,rotate=90.000000]{\color{textcolor}\sffamily\fontsize{20.000000}{24.000000}\selectfont \(\displaystyle \mathrm{Voltage}/\si{mV}\)}%
\end{pgfscope}%
\begin{pgfscope}%
\pgfpathrectangle{\pgfqpoint{1.000000in}{0.720000in}}{\pgfqpoint{6.200000in}{4.620000in}}%
\pgfusepath{clip}%
\pgfsetrectcap%
\pgfsetroundjoin%
\pgfsetlinewidth{2.007500pt}%
\definecolor{currentstroke}{rgb}{0.121569,0.466667,0.705882}%
\pgfsetstrokecolor{currentstroke}%
\pgfsetdash{}{0pt}%
\pgfpathmoveto{\pgfqpoint{0.990000in}{1.183788in}}%
\pgfpathlineto{\pgfqpoint{1.000000in}{1.220471in}}%
\pgfpathlineto{\pgfqpoint{1.031000in}{1.088165in}}%
\pgfpathlineto{\pgfqpoint{1.062000in}{1.166024in}}%
\pgfpathlineto{\pgfqpoint{1.093000in}{1.153091in}}%
\pgfpathlineto{\pgfqpoint{1.124000in}{1.138457in}}%
\pgfpathlineto{\pgfqpoint{1.155000in}{1.230904in}}%
\pgfpathlineto{\pgfqpoint{1.186000in}{1.149524in}}%
\pgfpathlineto{\pgfqpoint{1.217000in}{1.084489in}}%
\pgfpathlineto{\pgfqpoint{1.248000in}{1.218615in}}%
\pgfpathlineto{\pgfqpoint{1.279000in}{1.150424in}}%
\pgfpathlineto{\pgfqpoint{1.310000in}{1.070200in}}%
\pgfpathlineto{\pgfqpoint{1.341000in}{1.171858in}}%
\pgfpathlineto{\pgfqpoint{1.372000in}{1.109588in}}%
\pgfpathlineto{\pgfqpoint{1.403000in}{1.330876in}}%
\pgfpathlineto{\pgfqpoint{1.434000in}{1.065910in}}%
\pgfpathlineto{\pgfqpoint{1.465000in}{1.006946in}}%
\pgfpathlineto{\pgfqpoint{1.496000in}{1.087311in}}%
\pgfpathlineto{\pgfqpoint{1.527000in}{1.033716in}}%
\pgfpathlineto{\pgfqpoint{1.558000in}{1.007975in}}%
\pgfpathlineto{\pgfqpoint{1.589000in}{1.117019in}}%
\pgfpathlineto{\pgfqpoint{1.620000in}{1.201827in}}%
\pgfpathlineto{\pgfqpoint{1.651000in}{1.179308in}}%
\pgfpathlineto{\pgfqpoint{1.682000in}{1.288016in}}%
\pgfpathlineto{\pgfqpoint{1.713000in}{1.205679in}}%
\pgfpathlineto{\pgfqpoint{1.744000in}{1.229620in}}%
\pgfpathlineto{\pgfqpoint{1.775000in}{1.243639in}}%
\pgfpathlineto{\pgfqpoint{1.806000in}{1.211887in}}%
\pgfpathlineto{\pgfqpoint{1.837000in}{1.107275in}}%
\pgfpathlineto{\pgfqpoint{1.868000in}{1.011749in}}%
\pgfpathlineto{\pgfqpoint{1.899000in}{1.144398in}}%
\pgfpathlineto{\pgfqpoint{1.930000in}{1.097178in}}%
\pgfpathlineto{\pgfqpoint{1.961000in}{1.207480in}}%
\pgfpathlineto{\pgfqpoint{1.992000in}{1.201936in}}%
\pgfpathlineto{\pgfqpoint{2.023000in}{1.412545in}}%
\pgfpathlineto{\pgfqpoint{2.054000in}{1.762238in}}%
\pgfpathlineto{\pgfqpoint{2.085000in}{2.166756in}}%
\pgfpathlineto{\pgfqpoint{2.116000in}{2.872177in}}%
\pgfpathlineto{\pgfqpoint{2.147000in}{3.050499in}}%
\pgfpathlineto{\pgfqpoint{2.178000in}{3.449546in}}%
\pgfpathlineto{\pgfqpoint{2.209000in}{3.575740in}}%
\pgfpathlineto{\pgfqpoint{2.240000in}{3.343019in}}%
\pgfpathlineto{\pgfqpoint{2.271000in}{3.326427in}}%
\pgfpathlineto{\pgfqpoint{2.302000in}{3.248433in}}%
\pgfpathlineto{\pgfqpoint{2.333000in}{2.761558in}}%
\pgfpathlineto{\pgfqpoint{2.364000in}{2.593246in}}%
\pgfpathlineto{\pgfqpoint{2.395000in}{2.582407in}}%
\pgfpathlineto{\pgfqpoint{2.426000in}{2.298265in}}%
\pgfpathlineto{\pgfqpoint{2.457000in}{1.927341in}}%
\pgfpathlineto{\pgfqpoint{2.488000in}{1.839151in}}%
\pgfpathlineto{\pgfqpoint{2.519000in}{1.759238in}}%
\pgfpathlineto{\pgfqpoint{2.550000in}{1.716142in}}%
\pgfpathlineto{\pgfqpoint{2.581000in}{1.929394in}}%
\pgfpathlineto{\pgfqpoint{2.612000in}{2.491984in}}%
\pgfpathlineto{\pgfqpoint{2.643000in}{2.727872in}}%
\pgfpathlineto{\pgfqpoint{2.674000in}{3.114478in}}%
\pgfpathlineto{\pgfqpoint{2.705000in}{3.202234in}}%
\pgfpathlineto{\pgfqpoint{2.736000in}{3.090730in}}%
\pgfpathlineto{\pgfqpoint{2.767000in}{2.931067in}}%
\pgfpathlineto{\pgfqpoint{2.798000in}{2.792087in}}%
\pgfpathlineto{\pgfqpoint{2.829000in}{2.629906in}}%
\pgfpathlineto{\pgfqpoint{2.860000in}{2.515095in}}%
\pgfpathlineto{\pgfqpoint{2.891000in}{2.240466in}}%
\pgfpathlineto{\pgfqpoint{2.922000in}{2.286004in}}%
\pgfpathlineto{\pgfqpoint{2.953000in}{2.089639in}}%
\pgfpathlineto{\pgfqpoint{2.984000in}{2.599451in}}%
\pgfpathlineto{\pgfqpoint{3.015000in}{2.605840in}}%
\pgfpathlineto{\pgfqpoint{3.046000in}{2.920910in}}%
\pgfpathlineto{\pgfqpoint{3.077000in}{2.795406in}}%
\pgfpathlineto{\pgfqpoint{3.108000in}{2.817084in}}%
\pgfpathlineto{\pgfqpoint{3.139000in}{2.739656in}}%
\pgfpathlineto{\pgfqpoint{3.170000in}{2.483122in}}%
\pgfpathlineto{\pgfqpoint{3.201000in}{2.372645in}}%
\pgfpathlineto{\pgfqpoint{3.232000in}{2.107109in}}%
\pgfpathlineto{\pgfqpoint{3.263000in}{1.967133in}}%
\pgfpathlineto{\pgfqpoint{3.294000in}{1.814415in}}%
\pgfpathlineto{\pgfqpoint{3.325000in}{1.601343in}}%
\pgfpathlineto{\pgfqpoint{3.356000in}{1.720067in}}%
\pgfpathlineto{\pgfqpoint{3.387000in}{1.734361in}}%
\pgfpathlineto{\pgfqpoint{3.418000in}{1.511456in}}%
\pgfpathlineto{\pgfqpoint{3.449000in}{1.314821in}}%
\pgfpathlineto{\pgfqpoint{3.480000in}{1.269540in}}%
\pgfpathlineto{\pgfqpoint{3.511000in}{1.476934in}}%
\pgfpathlineto{\pgfqpoint{3.542000in}{1.252334in}}%
\pgfpathlineto{\pgfqpoint{3.573000in}{1.230722in}}%
\pgfpathlineto{\pgfqpoint{3.604000in}{1.354269in}}%
\pgfpathlineto{\pgfqpoint{3.635000in}{1.190058in}}%
\pgfpathlineto{\pgfqpoint{3.666000in}{1.147572in}}%
\pgfpathlineto{\pgfqpoint{3.697000in}{1.213775in}}%
\pgfpathlineto{\pgfqpoint{3.728000in}{1.141213in}}%
\pgfpathlineto{\pgfqpoint{3.759000in}{1.085530in}}%
\pgfpathlineto{\pgfqpoint{3.790000in}{1.343720in}}%
\pgfpathlineto{\pgfqpoint{3.821000in}{1.439105in}}%
\pgfpathlineto{\pgfqpoint{3.852000in}{1.903800in}}%
\pgfpathlineto{\pgfqpoint{3.883000in}{2.118718in}}%
\pgfpathlineto{\pgfqpoint{3.914000in}{2.569640in}}%
\pgfpathlineto{\pgfqpoint{3.945000in}{2.530002in}}%
\pgfpathlineto{\pgfqpoint{3.976000in}{2.813226in}}%
\pgfpathlineto{\pgfqpoint{4.007000in}{2.686903in}}%
\pgfpathlineto{\pgfqpoint{4.038000in}{2.342492in}}%
\pgfpathlineto{\pgfqpoint{4.069000in}{2.341291in}}%
\pgfpathlineto{\pgfqpoint{4.100000in}{2.218882in}}%
\pgfpathlineto{\pgfqpoint{4.131000in}{2.010099in}}%
\pgfpathlineto{\pgfqpoint{4.162000in}{1.835452in}}%
\pgfpathlineto{\pgfqpoint{4.193000in}{1.775125in}}%
\pgfpathlineto{\pgfqpoint{4.224000in}{1.819008in}}%
\pgfpathlineto{\pgfqpoint{4.255000in}{1.755634in}}%
\pgfpathlineto{\pgfqpoint{4.286000in}{1.457233in}}%
\pgfpathlineto{\pgfqpoint{4.317000in}{1.426958in}}%
\pgfpathlineto{\pgfqpoint{4.348000in}{1.371738in}}%
\pgfpathlineto{\pgfqpoint{4.379000in}{1.292524in}}%
\pgfpathlineto{\pgfqpoint{4.410000in}{1.181966in}}%
\pgfpathlineto{\pgfqpoint{4.441000in}{1.352001in}}%
\pgfpathlineto{\pgfqpoint{4.472000in}{1.249166in}}%
\pgfpathlineto{\pgfqpoint{4.534000in}{1.126778in}}%
\pgfpathlineto{\pgfqpoint{4.565000in}{1.061812in}}%
\pgfpathlineto{\pgfqpoint{4.596000in}{1.274897in}}%
\pgfpathlineto{\pgfqpoint{4.627000in}{1.289075in}}%
\pgfpathlineto{\pgfqpoint{4.658000in}{1.115146in}}%
\pgfpathlineto{\pgfqpoint{4.689000in}{1.235931in}}%
\pgfpathlineto{\pgfqpoint{4.720000in}{0.978256in}}%
\pgfpathlineto{\pgfqpoint{4.751000in}{1.119154in}}%
\pgfpathlineto{\pgfqpoint{4.782000in}{1.141788in}}%
\pgfpathlineto{\pgfqpoint{4.813000in}{1.020515in}}%
\pgfpathlineto{\pgfqpoint{4.844000in}{1.077710in}}%
\pgfpathlineto{\pgfqpoint{4.875000in}{0.855988in}}%
\pgfpathlineto{\pgfqpoint{4.906000in}{1.022623in}}%
\pgfpathlineto{\pgfqpoint{4.937000in}{1.047787in}}%
\pgfpathlineto{\pgfqpoint{4.968000in}{0.950711in}}%
\pgfpathlineto{\pgfqpoint{4.999000in}{1.293724in}}%
\pgfpathlineto{\pgfqpoint{5.030000in}{1.180298in}}%
\pgfpathlineto{\pgfqpoint{5.061000in}{1.120025in}}%
\pgfpathlineto{\pgfqpoint{5.092000in}{1.198315in}}%
\pgfpathlineto{\pgfqpoint{5.123000in}{0.984120in}}%
\pgfpathlineto{\pgfqpoint{5.154000in}{1.068802in}}%
\pgfpathlineto{\pgfqpoint{5.185000in}{1.019425in}}%
\pgfpathlineto{\pgfqpoint{5.216000in}{0.946714in}}%
\pgfpathlineto{\pgfqpoint{5.247000in}{1.116360in}}%
\pgfpathlineto{\pgfqpoint{5.278000in}{1.138393in}}%
\pgfpathlineto{\pgfqpoint{5.309000in}{1.210787in}}%
\pgfpathlineto{\pgfqpoint{5.371000in}{1.104547in}}%
\pgfpathlineto{\pgfqpoint{5.402000in}{1.176691in}}%
\pgfpathlineto{\pgfqpoint{5.433000in}{1.199825in}}%
\pgfpathlineto{\pgfqpoint{5.464000in}{1.066087in}}%
\pgfpathlineto{\pgfqpoint{5.495000in}{1.037629in}}%
\pgfpathlineto{\pgfqpoint{5.526000in}{1.067204in}}%
\pgfpathlineto{\pgfqpoint{5.557000in}{1.098434in}}%
\pgfpathlineto{\pgfqpoint{5.588000in}{1.030242in}}%
\pgfpathlineto{\pgfqpoint{5.619000in}{0.935843in}}%
\pgfpathlineto{\pgfqpoint{5.650000in}{1.074757in}}%
\pgfpathlineto{\pgfqpoint{5.681000in}{1.078726in}}%
\pgfpathlineto{\pgfqpoint{5.712000in}{1.102120in}}%
\pgfpathlineto{\pgfqpoint{5.743000in}{1.144167in}}%
\pgfpathlineto{\pgfqpoint{5.774000in}{1.264024in}}%
\pgfpathlineto{\pgfqpoint{5.805000in}{1.115461in}}%
\pgfpathlineto{\pgfqpoint{5.836000in}{1.053653in}}%
\pgfpathlineto{\pgfqpoint{5.867000in}{1.080451in}}%
\pgfpathlineto{\pgfqpoint{5.898000in}{1.235208in}}%
\pgfpathlineto{\pgfqpoint{5.929000in}{1.225269in}}%
\pgfpathlineto{\pgfqpoint{5.960000in}{1.072516in}}%
\pgfpathlineto{\pgfqpoint{5.991000in}{1.041469in}}%
\pgfpathlineto{\pgfqpoint{6.022000in}{1.095899in}}%
\pgfpathlineto{\pgfqpoint{6.053000in}{1.021449in}}%
\pgfpathlineto{\pgfqpoint{6.084000in}{1.045043in}}%
\pgfpathlineto{\pgfqpoint{6.115000in}{1.106544in}}%
\pgfpathlineto{\pgfqpoint{6.146000in}{1.141578in}}%
\pgfpathlineto{\pgfqpoint{6.177000in}{1.089972in}}%
\pgfpathlineto{\pgfqpoint{6.208000in}{1.181181in}}%
\pgfpathlineto{\pgfqpoint{6.239000in}{1.037835in}}%
\pgfpathlineto{\pgfqpoint{6.270000in}{1.207128in}}%
\pgfpathlineto{\pgfqpoint{6.301000in}{1.000021in}}%
\pgfpathlineto{\pgfqpoint{6.332000in}{1.060203in}}%
\pgfpathlineto{\pgfqpoint{6.363000in}{1.078996in}}%
\pgfpathlineto{\pgfqpoint{6.394000in}{1.072952in}}%
\pgfpathlineto{\pgfqpoint{6.425000in}{1.270591in}}%
\pgfpathlineto{\pgfqpoint{6.456000in}{1.140205in}}%
\pgfpathlineto{\pgfqpoint{6.487000in}{1.220524in}}%
\pgfpathlineto{\pgfqpoint{6.518000in}{1.173827in}}%
\pgfpathlineto{\pgfqpoint{6.549000in}{1.181256in}}%
\pgfpathlineto{\pgfqpoint{6.580000in}{1.135203in}}%
\pgfpathlineto{\pgfqpoint{6.611000in}{1.045463in}}%
\pgfpathlineto{\pgfqpoint{6.642000in}{1.184194in}}%
\pgfpathlineto{\pgfqpoint{6.673000in}{1.090373in}}%
\pgfpathlineto{\pgfqpoint{6.704000in}{0.967926in}}%
\pgfpathlineto{\pgfqpoint{6.735000in}{1.122669in}}%
\pgfpathlineto{\pgfqpoint{6.766000in}{1.213047in}}%
\pgfpathlineto{\pgfqpoint{6.797000in}{1.184072in}}%
\pgfpathlineto{\pgfqpoint{6.828000in}{1.174666in}}%
\pgfpathlineto{\pgfqpoint{6.859000in}{0.958653in}}%
\pgfpathlineto{\pgfqpoint{6.890000in}{1.135852in}}%
\pgfpathlineto{\pgfqpoint{6.921000in}{0.901578in}}%
\pgfpathlineto{\pgfqpoint{6.952000in}{1.135827in}}%
\pgfpathlineto{\pgfqpoint{6.983000in}{1.009980in}}%
\pgfpathlineto{\pgfqpoint{7.014000in}{1.115312in}}%
\pgfpathlineto{\pgfqpoint{7.045000in}{1.142137in}}%
\pgfpathlineto{\pgfqpoint{7.076000in}{1.056174in}}%
\pgfpathlineto{\pgfqpoint{7.107000in}{1.044486in}}%
\pgfpathlineto{\pgfqpoint{7.138000in}{1.186977in}}%
\pgfpathlineto{\pgfqpoint{7.169000in}{1.295955in}}%
\pgfpathlineto{\pgfqpoint{7.200000in}{1.185899in}}%
\pgfpathlineto{\pgfqpoint{7.210000in}{1.121327in}}%
\pgfpathlineto{\pgfqpoint{7.210000in}{1.121327in}}%
\pgfusepath{stroke}%
\end{pgfscope}%
\begin{pgfscope}%
\pgfpathrectangle{\pgfqpoint{1.000000in}{0.720000in}}{\pgfqpoint{6.200000in}{4.620000in}}%
\pgfusepath{clip}%
\pgfsetbuttcap%
\pgfsetroundjoin%
\pgfsetlinewidth{2.007500pt}%
\definecolor{currentstroke}{rgb}{0.000000,0.500000,0.000000}%
\pgfsetstrokecolor{currentstroke}%
\pgfsetdash{}{0pt}%
\pgfpathmoveto{\pgfqpoint{0.990000in}{1.592103in}}%
\pgfpathlineto{\pgfqpoint{7.210000in}{1.592103in}}%
\pgfusepath{stroke}%
\end{pgfscope}%
\begin{pgfscope}%
\pgfsetrectcap%
\pgfsetmiterjoin%
\pgfsetlinewidth{0.803000pt}%
\definecolor{currentstroke}{rgb}{0.000000,0.000000,0.000000}%
\pgfsetstrokecolor{currentstroke}%
\pgfsetdash{}{0pt}%
\pgfpathmoveto{\pgfqpoint{1.000000in}{0.720000in}}%
\pgfpathlineto{\pgfqpoint{1.000000in}{5.340000in}}%
\pgfusepath{stroke}%
\end{pgfscope}%
\begin{pgfscope}%
\pgfsetrectcap%
\pgfsetmiterjoin%
\pgfsetlinewidth{0.803000pt}%
\definecolor{currentstroke}{rgb}{0.000000,0.000000,0.000000}%
\pgfsetstrokecolor{currentstroke}%
\pgfsetdash{}{0pt}%
\pgfpathmoveto{\pgfqpoint{7.200000in}{0.720000in}}%
\pgfpathlineto{\pgfqpoint{7.200000in}{5.340000in}}%
\pgfusepath{stroke}%
\end{pgfscope}%
\begin{pgfscope}%
\pgfsetrectcap%
\pgfsetmiterjoin%
\pgfsetlinewidth{0.803000pt}%
\definecolor{currentstroke}{rgb}{0.000000,0.000000,0.000000}%
\pgfsetstrokecolor{currentstroke}%
\pgfsetdash{}{0pt}%
\pgfpathmoveto{\pgfqpoint{1.000000in}{0.720000in}}%
\pgfpathlineto{\pgfqpoint{7.200000in}{0.720000in}}%
\pgfusepath{stroke}%
\end{pgfscope}%
\begin{pgfscope}%
\pgfsetrectcap%
\pgfsetmiterjoin%
\pgfsetlinewidth{0.803000pt}%
\definecolor{currentstroke}{rgb}{0.000000,0.000000,0.000000}%
\pgfsetstrokecolor{currentstroke}%
\pgfsetdash{}{0pt}%
\pgfpathmoveto{\pgfqpoint{1.000000in}{5.340000in}}%
\pgfpathlineto{\pgfqpoint{7.200000in}{5.340000in}}%
\pgfusepath{stroke}%
\end{pgfscope}%
\begin{pgfscope}%
\pgfsetroundcap%
\pgfsetroundjoin%
\definecolor{currentfill}{rgb}{0.000000,0.000000,0.000000}%
\pgfsetfillcolor{currentfill}%
\pgfsetlinewidth{1.003750pt}%
\definecolor{currentstroke}{rgb}{0.000000,0.000000,0.000000}%
\pgfsetstrokecolor{currentstroke}%
\pgfsetdash{}{0pt}%
\pgfpathmoveto{\pgfqpoint{2.243472in}{4.733537in}}%
\pgfpathquadraticcurveto{\pgfqpoint{2.243472in}{4.359577in}}{\pgfqpoint{2.243472in}{3.985616in}}%
\pgfpathlineto{\pgfqpoint{2.253889in}{3.985616in}}%
\pgfpathquadraticcurveto{\pgfqpoint{2.246944in}{3.902289in}}{\pgfqpoint{2.240000in}{3.818962in}}%
\pgfpathquadraticcurveto{\pgfqpoint{2.233056in}{3.902289in}}{\pgfqpoint{2.226111in}{3.985616in}}%
\pgfpathlineto{\pgfqpoint{2.236528in}{3.985616in}}%
\pgfpathquadraticcurveto{\pgfqpoint{2.236528in}{4.359577in}}{\pgfqpoint{2.236528in}{4.733537in}}%
\pgfpathlineto{\pgfqpoint{2.243472in}{4.733537in}}%
\pgfpathlineto{\pgfqpoint{2.243472in}{4.733537in}}%
\pgfpathclose%
\pgfusepath{stroke,fill}%
\end{pgfscope}%
\begin{pgfscope}%
\pgfsetroundcap%
\pgfsetroundjoin%
\definecolor{currentfill}{rgb}{0.000000,0.000000,0.000000}%
\pgfsetfillcolor{currentfill}%
\pgfsetlinewidth{1.003750pt}%
\definecolor{currentstroke}{rgb}{0.000000,0.000000,0.000000}%
\pgfsetstrokecolor{currentstroke}%
\pgfsetdash{}{0pt}%
\pgfpathmoveto{\pgfqpoint{2.770472in}{4.321586in}}%
\pgfpathquadraticcurveto{\pgfqpoint{2.770472in}{3.947625in}}{\pgfqpoint{2.770472in}{3.573665in}}%
\pgfpathlineto{\pgfqpoint{2.780889in}{3.573665in}}%
\pgfpathquadraticcurveto{\pgfqpoint{2.773944in}{3.490338in}}{\pgfqpoint{2.767000in}{3.407011in}}%
\pgfpathquadraticcurveto{\pgfqpoint{2.760056in}{3.490338in}}{\pgfqpoint{2.753111in}{3.573665in}}%
\pgfpathlineto{\pgfqpoint{2.763528in}{3.573665in}}%
\pgfpathquadraticcurveto{\pgfqpoint{2.763528in}{3.947625in}}{\pgfqpoint{2.763528in}{4.321586in}}%
\pgfpathlineto{\pgfqpoint{2.770472in}{4.321586in}}%
\pgfpathlineto{\pgfqpoint{2.770472in}{4.321586in}}%
\pgfpathclose%
\pgfusepath{stroke,fill}%
\end{pgfscope}%
\begin{pgfscope}%
\pgfsetroundcap%
\pgfsetroundjoin%
\definecolor{currentfill}{rgb}{0.000000,0.000000,0.000000}%
\pgfsetfillcolor{currentfill}%
\pgfsetlinewidth{1.003750pt}%
\definecolor{currentstroke}{rgb}{0.000000,0.000000,0.000000}%
\pgfsetstrokecolor{currentstroke}%
\pgfsetdash{}{0pt}%
\pgfpathmoveto{\pgfqpoint{3.111472in}{4.207602in}}%
\pgfpathquadraticcurveto{\pgfqpoint{3.111472in}{3.833642in}}{\pgfqpoint{3.111472in}{3.459681in}}%
\pgfpathlineto{\pgfqpoint{3.121889in}{3.459681in}}%
\pgfpathquadraticcurveto{\pgfqpoint{3.114944in}{3.376354in}}{\pgfqpoint{3.108000in}{3.293027in}}%
\pgfpathquadraticcurveto{\pgfqpoint{3.101056in}{3.376354in}}{\pgfqpoint{3.094111in}{3.459681in}}%
\pgfpathlineto{\pgfqpoint{3.104528in}{3.459681in}}%
\pgfpathquadraticcurveto{\pgfqpoint{3.104528in}{3.833642in}}{\pgfqpoint{3.104528in}{4.207602in}}%
\pgfpathlineto{\pgfqpoint{3.111472in}{4.207602in}}%
\pgfpathlineto{\pgfqpoint{3.111472in}{4.207602in}}%
\pgfpathclose%
\pgfusepath{stroke,fill}%
\end{pgfscope}%
\begin{pgfscope}%
\pgfsetroundcap%
\pgfsetroundjoin%
\definecolor{currentfill}{rgb}{0.000000,0.000000,0.000000}%
\pgfsetfillcolor{currentfill}%
\pgfsetlinewidth{1.003750pt}%
\definecolor{currentstroke}{rgb}{0.000000,0.000000,0.000000}%
\pgfsetstrokecolor{currentstroke}%
\pgfsetdash{}{0pt}%
\pgfpathmoveto{\pgfqpoint{4.010472in}{4.077421in}}%
\pgfpathquadraticcurveto{\pgfqpoint{4.010472in}{3.703460in}}{\pgfqpoint{4.010472in}{3.329500in}}%
\pgfpathlineto{\pgfqpoint{4.020889in}{3.329500in}}%
\pgfpathquadraticcurveto{\pgfqpoint{4.013944in}{3.246173in}}{\pgfqpoint{4.007000in}{3.162846in}}%
\pgfpathquadraticcurveto{\pgfqpoint{4.000056in}{3.246173in}}{\pgfqpoint{3.993111in}{3.329500in}}%
\pgfpathlineto{\pgfqpoint{4.003528in}{3.329500in}}%
\pgfpathquadraticcurveto{\pgfqpoint{4.003528in}{3.703460in}}{\pgfqpoint{4.003528in}{4.077421in}}%
\pgfpathlineto{\pgfqpoint{4.010472in}{4.077421in}}%
\pgfpathlineto{\pgfqpoint{4.010472in}{4.077421in}}%
\pgfpathclose%
\pgfusepath{stroke,fill}%
\end{pgfscope}%
\begin{pgfscope}%
\pgfsetbuttcap%
\pgfsetroundjoin%
\definecolor{currentfill}{rgb}{0.000000,0.000000,0.000000}%
\pgfsetfillcolor{currentfill}%
\pgfsetlinewidth{0.803000pt}%
\definecolor{currentstroke}{rgb}{0.000000,0.000000,0.000000}%
\pgfsetstrokecolor{currentstroke}%
\pgfsetdash{}{0pt}%
\pgfsys@defobject{currentmarker}{\pgfqpoint{0.000000in}{0.000000in}}{\pgfqpoint{0.048611in}{0.000000in}}{%
\pgfpathmoveto{\pgfqpoint{0.000000in}{0.000000in}}%
\pgfpathlineto{\pgfqpoint{0.048611in}{0.000000in}}%
\pgfusepath{stroke,fill}%
}%
\begin{pgfscope}%
\pgfsys@transformshift{7.200000in}{1.125493in}%
\pgfsys@useobject{currentmarker}{}%
\end{pgfscope}%
\end{pgfscope}%
\begin{pgfscope}%
\definecolor{textcolor}{rgb}{0.000000,0.000000,0.000000}%
\pgfsetstrokecolor{textcolor}%
\pgfsetfillcolor{textcolor}%
\pgftext[x=7.297222in, y=1.025474in, left, base]{\color{textcolor}\sffamily\fontsize{20.000000}{24.000000}\selectfont 0.0}%
\end{pgfscope}%
\begin{pgfscope}%
\pgfsetbuttcap%
\pgfsetroundjoin%
\definecolor{currentfill}{rgb}{0.000000,0.000000,0.000000}%
\pgfsetfillcolor{currentfill}%
\pgfsetlinewidth{0.803000pt}%
\definecolor{currentstroke}{rgb}{0.000000,0.000000,0.000000}%
\pgfsetstrokecolor{currentstroke}%
\pgfsetdash{}{0pt}%
\pgfsys@defobject{currentmarker}{\pgfqpoint{0.000000in}{0.000000in}}{\pgfqpoint{0.048611in}{0.000000in}}{%
\pgfpathmoveto{\pgfqpoint{0.000000in}{0.000000in}}%
\pgfpathlineto{\pgfqpoint{0.048611in}{0.000000in}}%
\pgfusepath{stroke,fill}%
}%
\begin{pgfscope}%
\pgfsys@transformshift{7.200000in}{2.217318in}%
\pgfsys@useobject{currentmarker}{}%
\end{pgfscope}%
\end{pgfscope}%
\begin{pgfscope}%
\definecolor{textcolor}{rgb}{0.000000,0.000000,0.000000}%
\pgfsetstrokecolor{textcolor}%
\pgfsetfillcolor{textcolor}%
\pgftext[x=7.297222in, y=2.117299in, left, base]{\color{textcolor}\sffamily\fontsize{20.000000}{24.000000}\selectfont 0.5}%
\end{pgfscope}%
\begin{pgfscope}%
\pgfsetbuttcap%
\pgfsetroundjoin%
\definecolor{currentfill}{rgb}{0.000000,0.000000,0.000000}%
\pgfsetfillcolor{currentfill}%
\pgfsetlinewidth{0.803000pt}%
\definecolor{currentstroke}{rgb}{0.000000,0.000000,0.000000}%
\pgfsetstrokecolor{currentstroke}%
\pgfsetdash{}{0pt}%
\pgfsys@defobject{currentmarker}{\pgfqpoint{0.000000in}{0.000000in}}{\pgfqpoint{0.048611in}{0.000000in}}{%
\pgfpathmoveto{\pgfqpoint{0.000000in}{0.000000in}}%
\pgfpathlineto{\pgfqpoint{0.048611in}{0.000000in}}%
\pgfusepath{stroke,fill}%
}%
\begin{pgfscope}%
\pgfsys@transformshift{7.200000in}{3.309144in}%
\pgfsys@useobject{currentmarker}{}%
\end{pgfscope}%
\end{pgfscope}%
\begin{pgfscope}%
\definecolor{textcolor}{rgb}{0.000000,0.000000,0.000000}%
\pgfsetstrokecolor{textcolor}%
\pgfsetfillcolor{textcolor}%
\pgftext[x=7.297222in, y=3.209124in, left, base]{\color{textcolor}\sffamily\fontsize{20.000000}{24.000000}\selectfont 1.0}%
\end{pgfscope}%
\begin{pgfscope}%
\pgfsetbuttcap%
\pgfsetroundjoin%
\definecolor{currentfill}{rgb}{0.000000,0.000000,0.000000}%
\pgfsetfillcolor{currentfill}%
\pgfsetlinewidth{0.803000pt}%
\definecolor{currentstroke}{rgb}{0.000000,0.000000,0.000000}%
\pgfsetstrokecolor{currentstroke}%
\pgfsetdash{}{0pt}%
\pgfsys@defobject{currentmarker}{\pgfqpoint{0.000000in}{0.000000in}}{\pgfqpoint{0.048611in}{0.000000in}}{%
\pgfpathmoveto{\pgfqpoint{0.000000in}{0.000000in}}%
\pgfpathlineto{\pgfqpoint{0.048611in}{0.000000in}}%
\pgfusepath{stroke,fill}%
}%
\begin{pgfscope}%
\pgfsys@transformshift{7.200000in}{4.400969in}%
\pgfsys@useobject{currentmarker}{}%
\end{pgfscope}%
\end{pgfscope}%
\begin{pgfscope}%
\definecolor{textcolor}{rgb}{0.000000,0.000000,0.000000}%
\pgfsetstrokecolor{textcolor}%
\pgfsetfillcolor{textcolor}%
\pgftext[x=7.297222in, y=4.300950in, left, base]{\color{textcolor}\sffamily\fontsize{20.000000}{24.000000}\selectfont 1.5}%
\end{pgfscope}%
\begin{pgfscope}%
\definecolor{textcolor}{rgb}{0.000000,0.000000,0.000000}%
\pgfsetstrokecolor{textcolor}%
\pgfsetfillcolor{textcolor}%
\pgftext[x=7.698906in,y=3.030000in,,top,rotate=90.000000]{\color{textcolor}\sffamily\fontsize{20.000000}{24.000000}\selectfont \(\displaystyle \mathrm{Charge}\)}%
\end{pgfscope}%
\begin{pgfscope}%
\pgfpathrectangle{\pgfqpoint{1.000000in}{0.720000in}}{\pgfqpoint{6.200000in}{4.620000in}}%
\pgfusepath{clip}%
\pgfsetbuttcap%
\pgfsetroundjoin%
\pgfsetlinewidth{1.505625pt}%
\definecolor{currentstroke}{rgb}{1.000000,0.000000,0.000000}%
\pgfsetstrokecolor{currentstroke}%
\pgfsetdash{}{0pt}%
\pgfpathmoveto{\pgfqpoint{1.992000in}{1.125493in}}%
\pgfpathlineto{\pgfqpoint{1.992000in}{4.637582in}}%
\pgfusepath{stroke}%
\end{pgfscope}%
\begin{pgfscope}%
\pgfpathrectangle{\pgfqpoint{1.000000in}{0.720000in}}{\pgfqpoint{6.200000in}{4.620000in}}%
\pgfusepath{clip}%
\pgfsetbuttcap%
\pgfsetroundjoin%
\pgfsetlinewidth{1.505625pt}%
\definecolor{currentstroke}{rgb}{1.000000,0.000000,0.000000}%
\pgfsetstrokecolor{currentstroke}%
\pgfsetdash{}{0pt}%
\pgfpathmoveto{\pgfqpoint{2.519000in}{1.125493in}}%
\pgfpathlineto{\pgfqpoint{2.519000in}{3.985139in}}%
\pgfusepath{stroke}%
\end{pgfscope}%
\begin{pgfscope}%
\pgfpathrectangle{\pgfqpoint{1.000000in}{0.720000in}}{\pgfqpoint{6.200000in}{4.620000in}}%
\pgfusepath{clip}%
\pgfsetbuttcap%
\pgfsetroundjoin%
\pgfsetlinewidth{1.505625pt}%
\definecolor{currentstroke}{rgb}{1.000000,0.000000,0.000000}%
\pgfsetstrokecolor{currentstroke}%
\pgfsetdash{}{0pt}%
\pgfpathmoveto{\pgfqpoint{2.860000in}{1.125493in}}%
\pgfpathlineto{\pgfqpoint{2.860000in}{3.804613in}}%
\pgfusepath{stroke}%
\end{pgfscope}%
\begin{pgfscope}%
\pgfpathrectangle{\pgfqpoint{1.000000in}{0.720000in}}{\pgfqpoint{6.200000in}{4.620000in}}%
\pgfusepath{clip}%
\pgfsetbuttcap%
\pgfsetroundjoin%
\pgfsetlinewidth{1.505625pt}%
\definecolor{currentstroke}{rgb}{1.000000,0.000000,0.000000}%
\pgfsetstrokecolor{currentstroke}%
\pgfsetdash{}{0pt}%
\pgfpathmoveto{\pgfqpoint{3.759000in}{1.125493in}}%
\pgfpathlineto{\pgfqpoint{3.759000in}{3.598433in}}%
\pgfusepath{stroke}%
\end{pgfscope}%
\begin{pgfscope}%
\pgfsetrectcap%
\pgfsetmiterjoin%
\pgfsetlinewidth{0.803000pt}%
\definecolor{currentstroke}{rgb}{0.000000,0.000000,0.000000}%
\pgfsetstrokecolor{currentstroke}%
\pgfsetdash{}{0pt}%
\pgfpathmoveto{\pgfqpoint{1.000000in}{0.720000in}}%
\pgfpathlineto{\pgfqpoint{1.000000in}{5.340000in}}%
\pgfusepath{stroke}%
\end{pgfscope}%
\begin{pgfscope}%
\pgfsetrectcap%
\pgfsetmiterjoin%
\pgfsetlinewidth{0.803000pt}%
\definecolor{currentstroke}{rgb}{0.000000,0.000000,0.000000}%
\pgfsetstrokecolor{currentstroke}%
\pgfsetdash{}{0pt}%
\pgfpathmoveto{\pgfqpoint{7.200000in}{0.720000in}}%
\pgfpathlineto{\pgfqpoint{7.200000in}{5.340000in}}%
\pgfusepath{stroke}%
\end{pgfscope}%
\begin{pgfscope}%
\pgfsetrectcap%
\pgfsetmiterjoin%
\pgfsetlinewidth{0.803000pt}%
\definecolor{currentstroke}{rgb}{0.000000,0.000000,0.000000}%
\pgfsetstrokecolor{currentstroke}%
\pgfsetdash{}{0pt}%
\pgfpathmoveto{\pgfqpoint{1.000000in}{0.720000in}}%
\pgfpathlineto{\pgfqpoint{7.200000in}{0.720000in}}%
\pgfusepath{stroke}%
\end{pgfscope}%
\begin{pgfscope}%
\pgfsetrectcap%
\pgfsetmiterjoin%
\pgfsetlinewidth{0.803000pt}%
\definecolor{currentstroke}{rgb}{0.000000,0.000000,0.000000}%
\pgfsetstrokecolor{currentstroke}%
\pgfsetdash{}{0pt}%
\pgfpathmoveto{\pgfqpoint{1.000000in}{5.340000in}}%
\pgfpathlineto{\pgfqpoint{7.200000in}{5.340000in}}%
\pgfusepath{stroke}%
\end{pgfscope}%
\begin{pgfscope}%
\pgfsetroundcap%
\pgfsetroundjoin%
\definecolor{currentfill}{rgb}{0.000000,0.000000,0.000000}%
\pgfsetfillcolor{currentfill}%
\pgfsetlinewidth{1.003750pt}%
\definecolor{currentstroke}{rgb}{0.000000,0.000000,0.000000}%
\pgfsetstrokecolor{currentstroke}%
\pgfsetdash{}{0pt}%
\pgfpathmoveto{\pgfqpoint{4.531840in}{4.974902in}}%
\pgfpathquadraticcurveto{\pgfqpoint{3.749319in}{4.974902in}}{\pgfqpoint{2.966797in}{4.974902in}}%
\pgfpathlineto{\pgfqpoint{2.966797in}{4.961013in}}%
\pgfpathquadraticcurveto{\pgfqpoint{2.883471in}{4.974902in}}{\pgfqpoint{2.800144in}{4.988791in}}%
\pgfpathquadraticcurveto{\pgfqpoint{2.883471in}{5.002680in}}{\pgfqpoint{2.966797in}{5.016569in}}%
\pgfpathlineto{\pgfqpoint{2.966797in}{5.002680in}}%
\pgfpathquadraticcurveto{\pgfqpoint{3.749319in}{5.002680in}}{\pgfqpoint{4.531840in}{5.002680in}}%
\pgfpathlineto{\pgfqpoint{4.531840in}{4.974902in}}%
\pgfpathlineto{\pgfqpoint{4.531840in}{4.974902in}}%
\pgfpathclose%
\pgfusepath{stroke,fill}%
\end{pgfscope}%
\begin{pgfscope}%
\pgfsetbuttcap%
\pgfsetmiterjoin%
\definecolor{currentfill}{rgb}{1.000000,1.000000,1.000000}%
\pgfsetfillcolor{currentfill}%
\pgfsetfillopacity{0.800000}%
\pgfsetlinewidth{1.003750pt}%
\definecolor{currentstroke}{rgb}{0.800000,0.800000,0.800000}%
\pgfsetstrokecolor{currentstroke}%
\pgfsetstrokeopacity{0.800000}%
\pgfsetdash{}{0pt}%
\pgfpathmoveto{\pgfqpoint{4.976872in}{3.932908in}}%
\pgfpathlineto{\pgfqpoint{7.005556in}{3.932908in}}%
\pgfpathquadraticcurveto{\pgfqpoint{7.061111in}{3.932908in}}{\pgfqpoint{7.061111in}{3.988464in}}%
\pgfpathlineto{\pgfqpoint{7.061111in}{5.145556in}}%
\pgfpathquadraticcurveto{\pgfqpoint{7.061111in}{5.201111in}}{\pgfqpoint{7.005556in}{5.201111in}}%
\pgfpathlineto{\pgfqpoint{4.976872in}{5.201111in}}%
\pgfpathquadraticcurveto{\pgfqpoint{4.921317in}{5.201111in}}{\pgfqpoint{4.921317in}{5.145556in}}%
\pgfpathlineto{\pgfqpoint{4.921317in}{3.988464in}}%
\pgfpathquadraticcurveto{\pgfqpoint{4.921317in}{3.932908in}}{\pgfqpoint{4.976872in}{3.932908in}}%
\pgfpathlineto{\pgfqpoint{4.976872in}{3.932908in}}%
\pgfpathclose%
\pgfusepath{stroke,fill}%
\end{pgfscope}%
\begin{pgfscope}%
\pgfsetrectcap%
\pgfsetroundjoin%
\pgfsetlinewidth{2.007500pt}%
\definecolor{currentstroke}{rgb}{0.121569,0.466667,0.705882}%
\pgfsetstrokecolor{currentstroke}%
\pgfsetdash{}{0pt}%
\pgfpathmoveto{\pgfqpoint{5.032428in}{4.987184in}}%
\pgfpathlineto{\pgfqpoint{5.310206in}{4.987184in}}%
\pgfpathlineto{\pgfqpoint{5.587983in}{4.987184in}}%
\pgfusepath{stroke}%
\end{pgfscope}%
\begin{pgfscope}%
\definecolor{textcolor}{rgb}{0.000000,0.000000,0.000000}%
\pgfsetstrokecolor{textcolor}%
\pgfsetfillcolor{textcolor}%
\pgftext[x=5.810206in,y=4.889962in,left,base]{\color{textcolor}\sffamily\fontsize{20.000000}{24.000000}\selectfont Waveform}%
\end{pgfscope}%
\begin{pgfscope}%
\pgfsetbuttcap%
\pgfsetroundjoin%
\pgfsetlinewidth{2.007500pt}%
\definecolor{currentstroke}{rgb}{0.000000,0.500000,0.000000}%
\pgfsetstrokecolor{currentstroke}%
\pgfsetdash{}{0pt}%
\pgfpathmoveto{\pgfqpoint{5.032428in}{4.592227in}}%
\pgfpathlineto{\pgfqpoint{5.587983in}{4.592227in}}%
\pgfusepath{stroke}%
\end{pgfscope}%
\begin{pgfscope}%
\definecolor{textcolor}{rgb}{0.000000,0.000000,0.000000}%
\pgfsetstrokecolor{textcolor}%
\pgfsetfillcolor{textcolor}%
\pgftext[x=5.810206in,y=4.495005in,left,base]{\color{textcolor}\sffamily\fontsize{20.000000}{24.000000}\selectfont Threshold}%
\end{pgfscope}%
\begin{pgfscope}%
\pgfsetbuttcap%
\pgfsetroundjoin%
\pgfsetlinewidth{1.505625pt}%
\definecolor{currentstroke}{rgb}{1.000000,0.000000,0.000000}%
\pgfsetstrokecolor{currentstroke}%
\pgfsetdash{}{0pt}%
\pgfpathmoveto{\pgfqpoint{5.032428in}{4.197271in}}%
\pgfpathlineto{\pgfqpoint{5.587983in}{4.197271in}}%
\pgfusepath{stroke}%
\end{pgfscope}%
\begin{pgfscope}%
\definecolor{textcolor}{rgb}{0.000000,0.000000,0.000000}%
\pgfsetstrokecolor{textcolor}%
\pgfsetfillcolor{textcolor}%
\pgftext[x=5.810206in,y=4.100048in,left,base]{\color{textcolor}\sffamily\fontsize{20.000000}{24.000000}\selectfont Charge}%
\end{pgfscope}%
\end{pgfpicture}%
\makeatother%
\endgroup%
}
    \caption{\label{fig:peak} A peak finding example gives \\ $\hat{t}_\mathrm{KL} - t_0=\SI{4.85}{ns}$, $\mathrm{RSS}=\SI{739.9}{mV^2}$, $D_\mathrm{w}=\SI{2.35}{ns}$.}
  \end{subfigure}
  \caption{\label{fig:method}Demonstrations of heuristic methods on a waveform sampled from $\mu=4$, $\tau_\ell=\SI{20}{ns}$, $\sigma_\ell=\SI{5}{ns}$ light curve conditions.  Peak finding in~\subref{fig:peak} handles charges more realistically than waveform shifting in~\subref{fig:shifting}, giving better numbers by the $\mathrm{RSS}$ and $D_\mathrm{w}$ criteria in section \ref{sec:criteria}. $\Delta t_0$ is the bias of the $\hat{t}_\mathrm{KL}$ estimator.}
\end{figure}

\subsubsection{Peak finding}
\label{sec:findpeak}

The peak of $V_\mathrm{PE}$ is a distinct feature in waveforms, making \textit{Peak finding} more effective than waveform shifting.  We smooth a waveform by a low-pass Savitzky-Golay filter~\cite{savitzky_smoothing_1964} and find all the peaks at $t_i$'s.  The following resembles waveform shifting: apply a constant shift $\Delta t \equiv \arg\underset{t}{\max} V_\mathrm{PE}(t)$ to get $\hat{t}_i = t_i - \Delta t$, and calculate a scaling factor $\alpha$ to get $\hat{q_i}=\hat{\alpha} w(t_i)$ in the same way as eq.~\eqref{eq:alpha}.  As shown in figure~\ref{fig:peak}, peak finding outputs charges close to 1 and works well for lower PE counts.  But when PEs pile up closely, peaks overlap intensively, making this method unreliable.  Peak finding is usually too trivial to be documented but found almost everywhere~\cite{students22}.

\subsection{Deconvolution}
\label{sec:deconv}
Deconvolution is motivated by viewing the waveform as a convolution of sparse spike train $\tilde{\phi}$ and $V_\mathrm{PE}$ in eq.~\eqref{eq:1}.  Huang et al.~\cite{huang_flash_2018} from DayaBay and Grassi et al.~\cite{grassi_charge_2018} introduced deconvolution-based waveform analysis in charge reconstruction and linearity studies.  Zhang et al.~\cite{zhang_comparison_2019} then applied it to the JUNO prototype.  Deconvolution methods are better than heuristic ones by using the full shape of $V_\mathrm{PE}(t)$, thus can accommodate overshoots and pile-ups.  Noise and Nyquist limit make deconvolution sensitive to fluctuations in real-world applications.  A carefully selected low-pass filter mitigates the difficulty but might introduce Gibbs ringing artifacts in the smoothed waveforms and the deconvoluted results. Despite such drawbacks, deconvolution algorithms are fast and useful to give initial crude solutions for the more advanced algorithms.  Deployed in running experiments, they are discussed in this section to make an objective evaluation. 

\subsubsection{Fourier deconvolution}
\label{sec:fourier}
The deconvolution relation is evident in the Fourier transform $\mathcal{F}$ to eq.~\eqref{eq:1},
\begin{equation}
  \label{eq:fourier}
  \mathcal{F}[w]  = \mathcal{F}[\tilde{\phi}]\mathcal{F}[V_\mathrm{PE}] + \mathcal{F}[\epsilon]
  \implies \mathcal{F}[\tilde{\phi}]  = \frac{\mathcal{F}[w]}{\mathcal{F}[V_\mathrm{PE}]} - \frac{\mathcal{F}[\epsilon]}{\mathcal{F}[V_\mathrm{PE}]}.
\end{equation}
By low-pass filtering the waveform $w(t)$ to get $\tilde{w}(t)$, we suppress the noise term $\epsilon$.  In the inverse Fourier transform $\hat{\phi}_1(t) = \mathcal{F}^{-1}\left[\frac{\mathcal{F}[\tilde{w}]}{\mathcal{F}[V_\mathrm{PE}]}\right](t)$, remaining noise and limited bandwidth lead to smaller and even negative $\hat{q}_i$.  We apply a $q_\mathrm{th}$ threshold regularizer to cut off the unphysical parts of $\hat{\phi}_1(t)$,
\begin{equation}
  \label{eq:fdconv2}
    \hat{\phi}(t) = \hat{\alpha}\underbrace{\hat{\phi}_1(t) I\left(\hat{\phi}_1(t) - q_\mathrm{th}\right)}_{\text{over-threshold part of} \hat{\phi}_1(t)}  
\end{equation}
where $I(x)$ is the indicator function, and $\hat{\alpha}$ is the scaling factor to minimize $\mathrm{RSS}$ like in eq.~\eqref{eq:alpha},
\begin{equation*}
  \begin{aligned}
  \label{eq:id}
  I(x) = \left\{
    \begin{array}{ll}
      1 & \mbox{, if $x\ge0$}, \\
      0 & \mbox{, otherwise}
    \end{array}
    \right.
    \quad~~~
    \hat{\alpha} = \arg \underset{\alpha}{\min}\mathrm{RSS}\left[\alpha \hat{\phi} \otimes V_\mathrm{PE}, w\right]. \\
  \end{aligned}
\end{equation*}

Figure~\ref{fig:fd} illustrates that Fourier deconvolution outperforms heuristic methods, but still with a lot of small-charged PEs.

\begin{figure}[H]
  \begin{subfigure}{0.5\textwidth}
    \centering
    \resizebox{\textwidth}{!}{%% Creator: Matplotlib, PGF backend
%%
%% To include the figure in your LaTeX document, write
%%   \input{<filename>.pgf}
%%
%% Make sure the required packages are loaded in your preamble
%%   \usepackage{pgf}
%%
%% Also ensure that all the required font packages are loaded; for instance,
%% the lmodern package is sometimes necessary when using math font.
%%   \usepackage{lmodern}
%%
%% Figures using additional raster images can only be included by \input if
%% they are in the same directory as the main LaTeX file. For loading figures
%% from other directories you can use the `import` package
%%   \usepackage{import}
%%
%% and then include the figures with
%%   \import{<path to file>}{<filename>.pgf}
%%
%% Matplotlib used the following preamble
%%   \usepackage[detect-all,locale=DE]{siunitx}
%%
\begingroup%
\makeatletter%
\begin{pgfpicture}%
\pgfpathrectangle{\pgfpointorigin}{\pgfqpoint{8.000000in}{6.000000in}}%
\pgfusepath{use as bounding box, clip}%
\begin{pgfscope}%
\pgfsetbuttcap%
\pgfsetmiterjoin%
\definecolor{currentfill}{rgb}{1.000000,1.000000,1.000000}%
\pgfsetfillcolor{currentfill}%
\pgfsetlinewidth{0.000000pt}%
\definecolor{currentstroke}{rgb}{1.000000,1.000000,1.000000}%
\pgfsetstrokecolor{currentstroke}%
\pgfsetdash{}{0pt}%
\pgfpathmoveto{\pgfqpoint{0.000000in}{0.000000in}}%
\pgfpathlineto{\pgfqpoint{8.000000in}{0.000000in}}%
\pgfpathlineto{\pgfqpoint{8.000000in}{6.000000in}}%
\pgfpathlineto{\pgfqpoint{0.000000in}{6.000000in}}%
\pgfpathlineto{\pgfqpoint{0.000000in}{0.000000in}}%
\pgfpathclose%
\pgfusepath{fill}%
\end{pgfscope}%
\begin{pgfscope}%
\pgfsetbuttcap%
\pgfsetmiterjoin%
\definecolor{currentfill}{rgb}{1.000000,1.000000,1.000000}%
\pgfsetfillcolor{currentfill}%
\pgfsetlinewidth{0.000000pt}%
\definecolor{currentstroke}{rgb}{0.000000,0.000000,0.000000}%
\pgfsetstrokecolor{currentstroke}%
\pgfsetstrokeopacity{0.000000}%
\pgfsetdash{}{0pt}%
\pgfpathmoveto{\pgfqpoint{1.000000in}{0.720000in}}%
\pgfpathlineto{\pgfqpoint{7.200000in}{0.720000in}}%
\pgfpathlineto{\pgfqpoint{7.200000in}{5.340000in}}%
\pgfpathlineto{\pgfqpoint{1.000000in}{5.340000in}}%
\pgfpathlineto{\pgfqpoint{1.000000in}{0.720000in}}%
\pgfpathclose%
\pgfusepath{fill}%
\end{pgfscope}%
\begin{pgfscope}%
\pgfsetbuttcap%
\pgfsetroundjoin%
\definecolor{currentfill}{rgb}{0.000000,0.000000,0.000000}%
\pgfsetfillcolor{currentfill}%
\pgfsetlinewidth{0.803000pt}%
\definecolor{currentstroke}{rgb}{0.000000,0.000000,0.000000}%
\pgfsetstrokecolor{currentstroke}%
\pgfsetdash{}{0pt}%
\pgfsys@defobject{currentmarker}{\pgfqpoint{0.000000in}{-0.048611in}}{\pgfqpoint{0.000000in}{0.000000in}}{%
\pgfpathmoveto{\pgfqpoint{0.000000in}{0.000000in}}%
\pgfpathlineto{\pgfqpoint{0.000000in}{-0.048611in}}%
\pgfusepath{stroke,fill}%
}%
\begin{pgfscope}%
\pgfsys@transformshift{1.310000in}{0.720000in}%
\pgfsys@useobject{currentmarker}{}%
\end{pgfscope}%
\end{pgfscope}%
\begin{pgfscope}%
\definecolor{textcolor}{rgb}{0.000000,0.000000,0.000000}%
\pgfsetstrokecolor{textcolor}%
\pgfsetfillcolor{textcolor}%
\pgftext[x=1.310000in,y=0.622778in,,top]{\color{textcolor}\sffamily\fontsize{20.000000}{24.000000}\selectfont \(\displaystyle {450}\)}%
\end{pgfscope}%
\begin{pgfscope}%
\pgfsetbuttcap%
\pgfsetroundjoin%
\definecolor{currentfill}{rgb}{0.000000,0.000000,0.000000}%
\pgfsetfillcolor{currentfill}%
\pgfsetlinewidth{0.803000pt}%
\definecolor{currentstroke}{rgb}{0.000000,0.000000,0.000000}%
\pgfsetstrokecolor{currentstroke}%
\pgfsetdash{}{0pt}%
\pgfsys@defobject{currentmarker}{\pgfqpoint{0.000000in}{-0.048611in}}{\pgfqpoint{0.000000in}{0.000000in}}{%
\pgfpathmoveto{\pgfqpoint{0.000000in}{0.000000in}}%
\pgfpathlineto{\pgfqpoint{0.000000in}{-0.048611in}}%
\pgfusepath{stroke,fill}%
}%
\begin{pgfscope}%
\pgfsys@transformshift{2.860000in}{0.720000in}%
\pgfsys@useobject{currentmarker}{}%
\end{pgfscope}%
\end{pgfscope}%
\begin{pgfscope}%
\definecolor{textcolor}{rgb}{0.000000,0.000000,0.000000}%
\pgfsetstrokecolor{textcolor}%
\pgfsetfillcolor{textcolor}%
\pgftext[x=2.860000in,y=0.622778in,,top]{\color{textcolor}\sffamily\fontsize{20.000000}{24.000000}\selectfont \(\displaystyle {500}\)}%
\end{pgfscope}%
\begin{pgfscope}%
\pgfsetbuttcap%
\pgfsetroundjoin%
\definecolor{currentfill}{rgb}{0.000000,0.000000,0.000000}%
\pgfsetfillcolor{currentfill}%
\pgfsetlinewidth{0.803000pt}%
\definecolor{currentstroke}{rgb}{0.000000,0.000000,0.000000}%
\pgfsetstrokecolor{currentstroke}%
\pgfsetdash{}{0pt}%
\pgfsys@defobject{currentmarker}{\pgfqpoint{0.000000in}{-0.048611in}}{\pgfqpoint{0.000000in}{0.000000in}}{%
\pgfpathmoveto{\pgfqpoint{0.000000in}{0.000000in}}%
\pgfpathlineto{\pgfqpoint{0.000000in}{-0.048611in}}%
\pgfusepath{stroke,fill}%
}%
\begin{pgfscope}%
\pgfsys@transformshift{4.410000in}{0.720000in}%
\pgfsys@useobject{currentmarker}{}%
\end{pgfscope}%
\end{pgfscope}%
\begin{pgfscope}%
\definecolor{textcolor}{rgb}{0.000000,0.000000,0.000000}%
\pgfsetstrokecolor{textcolor}%
\pgfsetfillcolor{textcolor}%
\pgftext[x=4.410000in,y=0.622778in,,top]{\color{textcolor}\sffamily\fontsize{20.000000}{24.000000}\selectfont \(\displaystyle {550}\)}%
\end{pgfscope}%
\begin{pgfscope}%
\pgfsetbuttcap%
\pgfsetroundjoin%
\definecolor{currentfill}{rgb}{0.000000,0.000000,0.000000}%
\pgfsetfillcolor{currentfill}%
\pgfsetlinewidth{0.803000pt}%
\definecolor{currentstroke}{rgb}{0.000000,0.000000,0.000000}%
\pgfsetstrokecolor{currentstroke}%
\pgfsetdash{}{0pt}%
\pgfsys@defobject{currentmarker}{\pgfqpoint{0.000000in}{-0.048611in}}{\pgfqpoint{0.000000in}{0.000000in}}{%
\pgfpathmoveto{\pgfqpoint{0.000000in}{0.000000in}}%
\pgfpathlineto{\pgfqpoint{0.000000in}{-0.048611in}}%
\pgfusepath{stroke,fill}%
}%
\begin{pgfscope}%
\pgfsys@transformshift{5.960000in}{0.720000in}%
\pgfsys@useobject{currentmarker}{}%
\end{pgfscope}%
\end{pgfscope}%
\begin{pgfscope}%
\definecolor{textcolor}{rgb}{0.000000,0.000000,0.000000}%
\pgfsetstrokecolor{textcolor}%
\pgfsetfillcolor{textcolor}%
\pgftext[x=5.960000in,y=0.622778in,,top]{\color{textcolor}\sffamily\fontsize{20.000000}{24.000000}\selectfont \(\displaystyle {600}\)}%
\end{pgfscope}%
\begin{pgfscope}%
\definecolor{textcolor}{rgb}{0.000000,0.000000,0.000000}%
\pgfsetstrokecolor{textcolor}%
\pgfsetfillcolor{textcolor}%
\pgftext[x=4.100000in,y=0.311155in,,top]{\color{textcolor}\sffamily\fontsize{20.000000}{24.000000}\selectfont \(\displaystyle \mathrm{t}/\si{ns}\)}%
\end{pgfscope}%
\begin{pgfscope}%
\pgfsetbuttcap%
\pgfsetroundjoin%
\definecolor{currentfill}{rgb}{0.000000,0.000000,0.000000}%
\pgfsetfillcolor{currentfill}%
\pgfsetlinewidth{0.803000pt}%
\definecolor{currentstroke}{rgb}{0.000000,0.000000,0.000000}%
\pgfsetstrokecolor{currentstroke}%
\pgfsetdash{}{0pt}%
\pgfsys@defobject{currentmarker}{\pgfqpoint{-0.048611in}{0.000000in}}{\pgfqpoint{-0.000000in}{0.000000in}}{%
\pgfpathmoveto{\pgfqpoint{-0.000000in}{0.000000in}}%
\pgfpathlineto{\pgfqpoint{-0.048611in}{0.000000in}}%
\pgfusepath{stroke,fill}%
}%
\begin{pgfscope}%
\pgfsys@transformshift{1.000000in}{1.052772in}%
\pgfsys@useobject{currentmarker}{}%
\end{pgfscope}%
\end{pgfscope}%
\begin{pgfscope}%
\definecolor{textcolor}{rgb}{0.000000,0.000000,0.000000}%
\pgfsetstrokecolor{textcolor}%
\pgfsetfillcolor{textcolor}%
\pgftext[x=0.770670in, y=0.952752in, left, base]{\color{textcolor}\sffamily\fontsize{20.000000}{24.000000}\selectfont \(\displaystyle {0}\)}%
\end{pgfscope}%
\begin{pgfscope}%
\pgfsetbuttcap%
\pgfsetroundjoin%
\definecolor{currentfill}{rgb}{0.000000,0.000000,0.000000}%
\pgfsetfillcolor{currentfill}%
\pgfsetlinewidth{0.803000pt}%
\definecolor{currentstroke}{rgb}{0.000000,0.000000,0.000000}%
\pgfsetstrokecolor{currentstroke}%
\pgfsetdash{}{0pt}%
\pgfsys@defobject{currentmarker}{\pgfqpoint{-0.048611in}{0.000000in}}{\pgfqpoint{-0.000000in}{0.000000in}}{%
\pgfpathmoveto{\pgfqpoint{-0.000000in}{0.000000in}}%
\pgfpathlineto{\pgfqpoint{-0.048611in}{0.000000in}}%
\pgfusepath{stroke,fill}%
}%
\begin{pgfscope}%
\pgfsys@transformshift{1.000000in}{1.818629in}%
\pgfsys@useobject{currentmarker}{}%
\end{pgfscope}%
\end{pgfscope}%
\begin{pgfscope}%
\definecolor{textcolor}{rgb}{0.000000,0.000000,0.000000}%
\pgfsetstrokecolor{textcolor}%
\pgfsetfillcolor{textcolor}%
\pgftext[x=0.638563in, y=1.718609in, left, base]{\color{textcolor}\sffamily\fontsize{20.000000}{24.000000}\selectfont \(\displaystyle {10}\)}%
\end{pgfscope}%
\begin{pgfscope}%
\pgfsetbuttcap%
\pgfsetroundjoin%
\definecolor{currentfill}{rgb}{0.000000,0.000000,0.000000}%
\pgfsetfillcolor{currentfill}%
\pgfsetlinewidth{0.803000pt}%
\definecolor{currentstroke}{rgb}{0.000000,0.000000,0.000000}%
\pgfsetstrokecolor{currentstroke}%
\pgfsetdash{}{0pt}%
\pgfsys@defobject{currentmarker}{\pgfqpoint{-0.048611in}{0.000000in}}{\pgfqpoint{-0.000000in}{0.000000in}}{%
\pgfpathmoveto{\pgfqpoint{-0.000000in}{0.000000in}}%
\pgfpathlineto{\pgfqpoint{-0.048611in}{0.000000in}}%
\pgfusepath{stroke,fill}%
}%
\begin{pgfscope}%
\pgfsys@transformshift{1.000000in}{2.584485in}%
\pgfsys@useobject{currentmarker}{}%
\end{pgfscope}%
\end{pgfscope}%
\begin{pgfscope}%
\definecolor{textcolor}{rgb}{0.000000,0.000000,0.000000}%
\pgfsetstrokecolor{textcolor}%
\pgfsetfillcolor{textcolor}%
\pgftext[x=0.638563in, y=2.484466in, left, base]{\color{textcolor}\sffamily\fontsize{20.000000}{24.000000}\selectfont \(\displaystyle {20}\)}%
\end{pgfscope}%
\begin{pgfscope}%
\pgfsetbuttcap%
\pgfsetroundjoin%
\definecolor{currentfill}{rgb}{0.000000,0.000000,0.000000}%
\pgfsetfillcolor{currentfill}%
\pgfsetlinewidth{0.803000pt}%
\definecolor{currentstroke}{rgb}{0.000000,0.000000,0.000000}%
\pgfsetstrokecolor{currentstroke}%
\pgfsetdash{}{0pt}%
\pgfsys@defobject{currentmarker}{\pgfqpoint{-0.048611in}{0.000000in}}{\pgfqpoint{-0.000000in}{0.000000in}}{%
\pgfpathmoveto{\pgfqpoint{-0.000000in}{0.000000in}}%
\pgfpathlineto{\pgfqpoint{-0.048611in}{0.000000in}}%
\pgfusepath{stroke,fill}%
}%
\begin{pgfscope}%
\pgfsys@transformshift{1.000000in}{3.350342in}%
\pgfsys@useobject{currentmarker}{}%
\end{pgfscope}%
\end{pgfscope}%
\begin{pgfscope}%
\definecolor{textcolor}{rgb}{0.000000,0.000000,0.000000}%
\pgfsetstrokecolor{textcolor}%
\pgfsetfillcolor{textcolor}%
\pgftext[x=0.638563in, y=3.250323in, left, base]{\color{textcolor}\sffamily\fontsize{20.000000}{24.000000}\selectfont \(\displaystyle {30}\)}%
\end{pgfscope}%
\begin{pgfscope}%
\pgfsetbuttcap%
\pgfsetroundjoin%
\definecolor{currentfill}{rgb}{0.000000,0.000000,0.000000}%
\pgfsetfillcolor{currentfill}%
\pgfsetlinewidth{0.803000pt}%
\definecolor{currentstroke}{rgb}{0.000000,0.000000,0.000000}%
\pgfsetstrokecolor{currentstroke}%
\pgfsetdash{}{0pt}%
\pgfsys@defobject{currentmarker}{\pgfqpoint{-0.048611in}{0.000000in}}{\pgfqpoint{-0.000000in}{0.000000in}}{%
\pgfpathmoveto{\pgfqpoint{-0.000000in}{0.000000in}}%
\pgfpathlineto{\pgfqpoint{-0.048611in}{0.000000in}}%
\pgfusepath{stroke,fill}%
}%
\begin{pgfscope}%
\pgfsys@transformshift{1.000000in}{4.116199in}%
\pgfsys@useobject{currentmarker}{}%
\end{pgfscope}%
\end{pgfscope}%
\begin{pgfscope}%
\definecolor{textcolor}{rgb}{0.000000,0.000000,0.000000}%
\pgfsetstrokecolor{textcolor}%
\pgfsetfillcolor{textcolor}%
\pgftext[x=0.638563in, y=4.016180in, left, base]{\color{textcolor}\sffamily\fontsize{20.000000}{24.000000}\selectfont \(\displaystyle {40}\)}%
\end{pgfscope}%
\begin{pgfscope}%
\pgfsetbuttcap%
\pgfsetroundjoin%
\definecolor{currentfill}{rgb}{0.000000,0.000000,0.000000}%
\pgfsetfillcolor{currentfill}%
\pgfsetlinewidth{0.803000pt}%
\definecolor{currentstroke}{rgb}{0.000000,0.000000,0.000000}%
\pgfsetstrokecolor{currentstroke}%
\pgfsetdash{}{0pt}%
\pgfsys@defobject{currentmarker}{\pgfqpoint{-0.048611in}{0.000000in}}{\pgfqpoint{-0.000000in}{0.000000in}}{%
\pgfpathmoveto{\pgfqpoint{-0.000000in}{0.000000in}}%
\pgfpathlineto{\pgfqpoint{-0.048611in}{0.000000in}}%
\pgfusepath{stroke,fill}%
}%
\begin{pgfscope}%
\pgfsys@transformshift{1.000000in}{4.882056in}%
\pgfsys@useobject{currentmarker}{}%
\end{pgfscope}%
\end{pgfscope}%
\begin{pgfscope}%
\definecolor{textcolor}{rgb}{0.000000,0.000000,0.000000}%
\pgfsetstrokecolor{textcolor}%
\pgfsetfillcolor{textcolor}%
\pgftext[x=0.638563in, y=4.782037in, left, base]{\color{textcolor}\sffamily\fontsize{20.000000}{24.000000}\selectfont \(\displaystyle {50}\)}%
\end{pgfscope}%
\begin{pgfscope}%
\definecolor{textcolor}{rgb}{0.000000,0.000000,0.000000}%
\pgfsetstrokecolor{textcolor}%
\pgfsetfillcolor{textcolor}%
\pgftext[x=0.583007in,y=3.030000in,,bottom,rotate=90.000000]{\color{textcolor}\sffamily\fontsize{20.000000}{24.000000}\selectfont \(\displaystyle \mathrm{Voltage}/\si{mV}\)}%
\end{pgfscope}%
\begin{pgfscope}%
\pgfpathrectangle{\pgfqpoint{1.000000in}{0.720000in}}{\pgfqpoint{6.200000in}{4.620000in}}%
\pgfusepath{clip}%
\pgfsetrectcap%
\pgfsetroundjoin%
\pgfsetlinewidth{2.007500pt}%
\definecolor{currentstroke}{rgb}{0.121569,0.466667,0.705882}%
\pgfsetstrokecolor{currentstroke}%
\pgfsetdash{}{0pt}%
\pgfpathmoveto{\pgfqpoint{0.990000in}{1.100612in}}%
\pgfpathlineto{\pgfqpoint{1.000000in}{1.130716in}}%
\pgfpathlineto{\pgfqpoint{1.031000in}{1.022138in}}%
\pgfpathlineto{\pgfqpoint{1.062000in}{1.086034in}}%
\pgfpathlineto{\pgfqpoint{1.093000in}{1.075420in}}%
\pgfpathlineto{\pgfqpoint{1.124000in}{1.063411in}}%
\pgfpathlineto{\pgfqpoint{1.155000in}{1.139278in}}%
\pgfpathlineto{\pgfqpoint{1.186000in}{1.072493in}}%
\pgfpathlineto{\pgfqpoint{1.217000in}{1.019122in}}%
\pgfpathlineto{\pgfqpoint{1.248000in}{1.129193in}}%
\pgfpathlineto{\pgfqpoint{1.279000in}{1.073232in}}%
\pgfpathlineto{\pgfqpoint{1.310000in}{1.007395in}}%
\pgfpathlineto{\pgfqpoint{1.341000in}{1.090822in}}%
\pgfpathlineto{\pgfqpoint{1.372000in}{1.039719in}}%
\pgfpathlineto{\pgfqpoint{1.403000in}{1.221321in}}%
\pgfpathlineto{\pgfqpoint{1.434000in}{1.003874in}}%
\pgfpathlineto{\pgfqpoint{1.465000in}{0.955485in}}%
\pgfpathlineto{\pgfqpoint{1.496000in}{1.021437in}}%
\pgfpathlineto{\pgfqpoint{1.527000in}{0.977454in}}%
\pgfpathlineto{\pgfqpoint{1.558000in}{0.956329in}}%
\pgfpathlineto{\pgfqpoint{1.589000in}{1.045817in}}%
\pgfpathlineto{\pgfqpoint{1.620000in}{1.115416in}}%
\pgfpathlineto{\pgfqpoint{1.651000in}{1.096936in}}%
\pgfpathlineto{\pgfqpoint{1.682000in}{1.186148in}}%
\pgfpathlineto{\pgfqpoint{1.713000in}{1.118577in}}%
\pgfpathlineto{\pgfqpoint{1.744000in}{1.138225in}}%
\pgfpathlineto{\pgfqpoint{1.775000in}{1.149730in}}%
\pgfpathlineto{\pgfqpoint{1.806000in}{1.123672in}}%
\pgfpathlineto{\pgfqpoint{1.837000in}{1.037821in}}%
\pgfpathlineto{\pgfqpoint{1.868000in}{0.959427in}}%
\pgfpathlineto{\pgfqpoint{1.899000in}{1.068286in}}%
\pgfpathlineto{\pgfqpoint{1.930000in}{1.029535in}}%
\pgfpathlineto{\pgfqpoint{1.961000in}{1.120055in}}%
\pgfpathlineto{\pgfqpoint{1.992000in}{1.115505in}}%
\pgfpathlineto{\pgfqpoint{2.023000in}{1.288344in}}%
\pgfpathlineto{\pgfqpoint{2.054000in}{1.575323in}}%
\pgfpathlineto{\pgfqpoint{2.085000in}{1.907294in}}%
\pgfpathlineto{\pgfqpoint{2.116000in}{2.486206in}}%
\pgfpathlineto{\pgfqpoint{2.147000in}{2.632547in}}%
\pgfpathlineto{\pgfqpoint{2.178000in}{2.960029in}}%
\pgfpathlineto{\pgfqpoint{2.209000in}{3.063591in}}%
\pgfpathlineto{\pgfqpoint{2.240000in}{2.872606in}}%
\pgfpathlineto{\pgfqpoint{2.271000in}{2.858990in}}%
\pgfpathlineto{\pgfqpoint{2.302000in}{2.794983in}}%
\pgfpathlineto{\pgfqpoint{2.333000in}{2.395425in}}%
\pgfpathlineto{\pgfqpoint{2.364000in}{2.257298in}}%
\pgfpathlineto{\pgfqpoint{2.395000in}{2.248403in}}%
\pgfpathlineto{\pgfqpoint{2.426000in}{2.015219in}}%
\pgfpathlineto{\pgfqpoint{2.457000in}{1.710816in}}%
\pgfpathlineto{\pgfqpoint{2.488000in}{1.638443in}}%
\pgfpathlineto{\pgfqpoint{2.519000in}{1.572861in}}%
\pgfpathlineto{\pgfqpoint{2.550000in}{1.537494in}}%
\pgfpathlineto{\pgfqpoint{2.581000in}{1.712502in}}%
\pgfpathlineto{\pgfqpoint{2.612000in}{2.174196in}}%
\pgfpathlineto{\pgfqpoint{2.643000in}{2.367780in}}%
\pgfpathlineto{\pgfqpoint{2.674000in}{2.685052in}}%
\pgfpathlineto{\pgfqpoint{2.705000in}{2.757070in}}%
\pgfpathlineto{\pgfqpoint{2.736000in}{2.665563in}}%
\pgfpathlineto{\pgfqpoint{2.767000in}{2.534534in}}%
\pgfpathlineto{\pgfqpoint{2.798000in}{2.420479in}}%
\pgfpathlineto{\pgfqpoint{2.829000in}{2.287383in}}%
\pgfpathlineto{\pgfqpoint{2.860000in}{2.193163in}}%
\pgfpathlineto{\pgfqpoint{2.891000in}{1.967785in}}%
\pgfpathlineto{\pgfqpoint{2.922000in}{2.005156in}}%
\pgfpathlineto{\pgfqpoint{2.953000in}{1.844007in}}%
\pgfpathlineto{\pgfqpoint{2.984000in}{2.262390in}}%
\pgfpathlineto{\pgfqpoint{3.015000in}{2.267634in}}%
\pgfpathlineto{\pgfqpoint{3.046000in}{2.526199in}}%
\pgfpathlineto{\pgfqpoint{3.077000in}{2.423202in}}%
\pgfpathlineto{\pgfqpoint{3.108000in}{2.440993in}}%
\pgfpathlineto{\pgfqpoint{3.139000in}{2.377451in}}%
\pgfpathlineto{\pgfqpoint{3.170000in}{2.166923in}}%
\pgfpathlineto{\pgfqpoint{3.201000in}{2.076260in}}%
\pgfpathlineto{\pgfqpoint{3.232000in}{1.858344in}}%
\pgfpathlineto{\pgfqpoint{3.263000in}{1.743472in}}%
\pgfpathlineto{\pgfqpoint{3.294000in}{1.618143in}}%
\pgfpathlineto{\pgfqpoint{3.325000in}{1.443283in}}%
\pgfpathlineto{\pgfqpoint{3.356000in}{1.540715in}}%
\pgfpathlineto{\pgfqpoint{3.387000in}{1.552445in}}%
\pgfpathlineto{\pgfqpoint{3.418000in}{1.369516in}}%
\pgfpathlineto{\pgfqpoint{3.449000in}{1.208146in}}%
\pgfpathlineto{\pgfqpoint{3.480000in}{1.170985in}}%
\pgfpathlineto{\pgfqpoint{3.511000in}{1.341185in}}%
\pgfpathlineto{\pgfqpoint{3.542000in}{1.156865in}}%
\pgfpathlineto{\pgfqpoint{3.573000in}{1.139129in}}%
\pgfpathlineto{\pgfqpoint{3.604000in}{1.240519in}}%
\pgfpathlineto{\pgfqpoint{3.635000in}{1.105758in}}%
\pgfpathlineto{\pgfqpoint{3.666000in}{1.070891in}}%
\pgfpathlineto{\pgfqpoint{3.697000in}{1.125221in}}%
\pgfpathlineto{\pgfqpoint{3.728000in}{1.065673in}}%
\pgfpathlineto{\pgfqpoint{3.759000in}{1.019976in}}%
\pgfpathlineto{\pgfqpoint{3.790000in}{1.231862in}}%
\pgfpathlineto{\pgfqpoint{3.821000in}{1.310141in}}%
\pgfpathlineto{\pgfqpoint{3.852000in}{1.691497in}}%
\pgfpathlineto{\pgfqpoint{3.883000in}{1.867872in}}%
\pgfpathlineto{\pgfqpoint{3.914000in}{2.237925in}}%
\pgfpathlineto{\pgfqpoint{3.945000in}{2.205396in}}%
\pgfpathlineto{\pgfqpoint{3.976000in}{2.437826in}}%
\pgfpathlineto{\pgfqpoint{4.007000in}{2.334158in}}%
\pgfpathlineto{\pgfqpoint{4.038000in}{2.051514in}}%
\pgfpathlineto{\pgfqpoint{4.069000in}{2.050528in}}%
\pgfpathlineto{\pgfqpoint{4.100000in}{1.950072in}}%
\pgfpathlineto{\pgfqpoint{4.131000in}{1.778733in}}%
\pgfpathlineto{\pgfqpoint{4.162000in}{1.635407in}}%
\pgfpathlineto{\pgfqpoint{4.193000in}{1.585899in}}%
\pgfpathlineto{\pgfqpoint{4.224000in}{1.621912in}}%
\pgfpathlineto{\pgfqpoint{4.255000in}{1.569903in}}%
\pgfpathlineto{\pgfqpoint{4.286000in}{1.325018in}}%
\pgfpathlineto{\pgfqpoint{4.317000in}{1.300172in}}%
\pgfpathlineto{\pgfqpoint{4.348000in}{1.254855in}}%
\pgfpathlineto{\pgfqpoint{4.379000in}{1.189848in}}%
\pgfpathlineto{\pgfqpoint{4.410000in}{1.099117in}}%
\pgfpathlineto{\pgfqpoint{4.441000in}{1.238658in}}%
\pgfpathlineto{\pgfqpoint{4.472000in}{1.154265in}}%
\pgfpathlineto{\pgfqpoint{4.534000in}{1.053827in}}%
\pgfpathlineto{\pgfqpoint{4.565000in}{1.000511in}}%
\pgfpathlineto{\pgfqpoint{4.596000in}{1.175382in}}%
\pgfpathlineto{\pgfqpoint{4.627000in}{1.187017in}}%
\pgfpathlineto{\pgfqpoint{4.658000in}{1.044281in}}%
\pgfpathlineto{\pgfqpoint{4.689000in}{1.143404in}}%
\pgfpathlineto{\pgfqpoint{4.720000in}{0.931941in}}%
\pgfpathlineto{\pgfqpoint{4.751000in}{1.047570in}}%
\pgfpathlineto{\pgfqpoint{4.782000in}{1.066145in}}%
\pgfpathlineto{\pgfqpoint{4.813000in}{0.966620in}}%
\pgfpathlineto{\pgfqpoint{4.844000in}{1.013558in}}%
\pgfpathlineto{\pgfqpoint{4.875000in}{0.831600in}}%
\pgfpathlineto{\pgfqpoint{4.906000in}{0.968350in}}%
\pgfpathlineto{\pgfqpoint{4.937000in}{0.989001in}}%
\pgfpathlineto{\pgfqpoint{4.968000in}{0.909335in}}%
\pgfpathlineto{\pgfqpoint{4.999000in}{1.190833in}}%
\pgfpathlineto{\pgfqpoint{5.030000in}{1.097748in}}%
\pgfpathlineto{\pgfqpoint{5.061000in}{1.048285in}}%
\pgfpathlineto{\pgfqpoint{5.092000in}{1.112534in}}%
\pgfpathlineto{\pgfqpoint{5.123000in}{0.936752in}}%
\pgfpathlineto{\pgfqpoint{5.154000in}{1.006248in}}%
\pgfpathlineto{\pgfqpoint{5.185000in}{0.965726in}}%
\pgfpathlineto{\pgfqpoint{5.216000in}{0.906055in}}%
\pgfpathlineto{\pgfqpoint{5.247000in}{1.045277in}}%
\pgfpathlineto{\pgfqpoint{5.278000in}{1.063358in}}%
\pgfpathlineto{\pgfqpoint{5.309000in}{1.122769in}}%
\pgfpathlineto{\pgfqpoint{5.371000in}{1.035582in}}%
\pgfpathlineto{\pgfqpoint{5.402000in}{1.094788in}}%
\pgfpathlineto{\pgfqpoint{5.433000in}{1.113773in}}%
\pgfpathlineto{\pgfqpoint{5.464000in}{1.004020in}}%
\pgfpathlineto{\pgfqpoint{5.495000in}{0.980666in}}%
\pgfpathlineto{\pgfqpoint{5.557000in}{1.030566in}}%
\pgfpathlineto{\pgfqpoint{5.588000in}{0.974603in}}%
\pgfpathlineto{\pgfqpoint{5.619000in}{0.897133in}}%
\pgfpathlineto{\pgfqpoint{5.650000in}{1.011135in}}%
\pgfpathlineto{\pgfqpoint{5.681000in}{1.014392in}}%
\pgfpathlineto{\pgfqpoint{5.712000in}{1.033590in}}%
\pgfpathlineto{\pgfqpoint{5.743000in}{1.068097in}}%
\pgfpathlineto{\pgfqpoint{5.774000in}{1.166458in}}%
\pgfpathlineto{\pgfqpoint{5.805000in}{1.044539in}}%
\pgfpathlineto{\pgfqpoint{5.836000in}{0.993816in}}%
\pgfpathlineto{\pgfqpoint{5.867000in}{1.015808in}}%
\pgfpathlineto{\pgfqpoint{5.898000in}{1.142811in}}%
\pgfpathlineto{\pgfqpoint{5.929000in}{1.134654in}}%
\pgfpathlineto{\pgfqpoint{5.960000in}{1.009296in}}%
\pgfpathlineto{\pgfqpoint{5.991000in}{0.983817in}}%
\pgfpathlineto{\pgfqpoint{6.022000in}{1.028485in}}%
\pgfpathlineto{\pgfqpoint{6.053000in}{0.967387in}}%
\pgfpathlineto{\pgfqpoint{6.084000in}{0.986750in}}%
\pgfpathlineto{\pgfqpoint{6.115000in}{1.037221in}}%
\pgfpathlineto{\pgfqpoint{6.146000in}{1.065972in}}%
\pgfpathlineto{\pgfqpoint{6.177000in}{1.023621in}}%
\pgfpathlineto{\pgfqpoint{6.208000in}{1.098473in}}%
\pgfpathlineto{\pgfqpoint{6.239000in}{0.980835in}}%
\pgfpathlineto{\pgfqpoint{6.270000in}{1.119766in}}%
\pgfpathlineto{\pgfqpoint{6.301000in}{0.949802in}}%
\pgfpathlineto{\pgfqpoint{6.332000in}{0.999191in}}%
\pgfpathlineto{\pgfqpoint{6.363000in}{1.014614in}}%
\pgfpathlineto{\pgfqpoint{6.394000in}{1.009653in}}%
\pgfpathlineto{\pgfqpoint{6.425000in}{1.171848in}}%
\pgfpathlineto{\pgfqpoint{6.456000in}{1.064845in}}%
\pgfpathlineto{\pgfqpoint{6.487000in}{1.130760in}}%
\pgfpathlineto{\pgfqpoint{6.518000in}{1.092438in}}%
\pgfpathlineto{\pgfqpoint{6.549000in}{1.098535in}}%
\pgfpathlineto{\pgfqpoint{6.580000in}{1.060740in}}%
\pgfpathlineto{\pgfqpoint{6.611000in}{0.987094in}}%
\pgfpathlineto{\pgfqpoint{6.642000in}{1.100946in}}%
\pgfpathlineto{\pgfqpoint{6.673000in}{1.023950in}}%
\pgfpathlineto{\pgfqpoint{6.704000in}{0.923463in}}%
\pgfpathlineto{\pgfqpoint{6.735000in}{1.050455in}}%
\pgfpathlineto{\pgfqpoint{6.766000in}{1.124624in}}%
\pgfpathlineto{\pgfqpoint{6.797000in}{1.100845in}}%
\pgfpathlineto{\pgfqpoint{6.828000in}{1.093127in}}%
\pgfpathlineto{\pgfqpoint{6.859000in}{0.915853in}}%
\pgfpathlineto{\pgfqpoint{6.890000in}{1.061273in}}%
\pgfpathlineto{\pgfqpoint{6.921000in}{0.869013in}}%
\pgfpathlineto{\pgfqpoint{6.952000in}{1.061253in}}%
\pgfpathlineto{\pgfqpoint{6.983000in}{0.957975in}}%
\pgfpathlineto{\pgfqpoint{7.014000in}{1.044416in}}%
\pgfpathlineto{\pgfqpoint{7.045000in}{1.066431in}}%
\pgfpathlineto{\pgfqpoint{7.076000in}{0.995885in}}%
\pgfpathlineto{\pgfqpoint{7.107000in}{0.986293in}}%
\pgfpathlineto{\pgfqpoint{7.138000in}{1.103229in}}%
\pgfpathlineto{\pgfqpoint{7.169000in}{1.192663in}}%
\pgfpathlineto{\pgfqpoint{7.200000in}{1.102344in}}%
\pgfpathlineto{\pgfqpoint{7.210000in}{1.049353in}}%
\pgfpathlineto{\pgfqpoint{7.210000in}{1.049353in}}%
\pgfusepath{stroke}%
\end{pgfscope}%
\begin{pgfscope}%
\pgfpathrectangle{\pgfqpoint{1.000000in}{0.720000in}}{\pgfqpoint{6.200000in}{4.620000in}}%
\pgfusepath{clip}%
\pgfsetbuttcap%
\pgfsetroundjoin%
\pgfsetlinewidth{2.007500pt}%
\definecolor{currentstroke}{rgb}{0.000000,0.500000,0.000000}%
\pgfsetstrokecolor{currentstroke}%
\pgfsetdash{}{0pt}%
\pgfpathmoveto{\pgfqpoint{0.990000in}{1.435700in}}%
\pgfpathlineto{\pgfqpoint{7.210000in}{1.435700in}}%
\pgfusepath{stroke}%
\end{pgfscope}%
\begin{pgfscope}%
\pgfsetrectcap%
\pgfsetmiterjoin%
\pgfsetlinewidth{0.803000pt}%
\definecolor{currentstroke}{rgb}{0.000000,0.000000,0.000000}%
\pgfsetstrokecolor{currentstroke}%
\pgfsetdash{}{0pt}%
\pgfpathmoveto{\pgfqpoint{1.000000in}{0.720000in}}%
\pgfpathlineto{\pgfqpoint{1.000000in}{5.340000in}}%
\pgfusepath{stroke}%
\end{pgfscope}%
\begin{pgfscope}%
\pgfsetrectcap%
\pgfsetmiterjoin%
\pgfsetlinewidth{0.803000pt}%
\definecolor{currentstroke}{rgb}{0.000000,0.000000,0.000000}%
\pgfsetstrokecolor{currentstroke}%
\pgfsetdash{}{0pt}%
\pgfpathmoveto{\pgfqpoint{7.200000in}{0.720000in}}%
\pgfpathlineto{\pgfqpoint{7.200000in}{5.340000in}}%
\pgfusepath{stroke}%
\end{pgfscope}%
\begin{pgfscope}%
\pgfsetrectcap%
\pgfsetmiterjoin%
\pgfsetlinewidth{0.803000pt}%
\definecolor{currentstroke}{rgb}{0.000000,0.000000,0.000000}%
\pgfsetstrokecolor{currentstroke}%
\pgfsetdash{}{0pt}%
\pgfpathmoveto{\pgfqpoint{1.000000in}{0.720000in}}%
\pgfpathlineto{\pgfqpoint{7.200000in}{0.720000in}}%
\pgfusepath{stroke}%
\end{pgfscope}%
\begin{pgfscope}%
\pgfsetrectcap%
\pgfsetmiterjoin%
\pgfsetlinewidth{0.803000pt}%
\definecolor{currentstroke}{rgb}{0.000000,0.000000,0.000000}%
\pgfsetstrokecolor{currentstroke}%
\pgfsetdash{}{0pt}%
\pgfpathmoveto{\pgfqpoint{1.000000in}{5.340000in}}%
\pgfpathlineto{\pgfqpoint{7.200000in}{5.340000in}}%
\pgfusepath{stroke}%
\end{pgfscope}%
\begin{pgfscope}%
\pgfsetbuttcap%
\pgfsetroundjoin%
\definecolor{currentfill}{rgb}{0.000000,0.000000,0.000000}%
\pgfsetfillcolor{currentfill}%
\pgfsetlinewidth{0.803000pt}%
\definecolor{currentstroke}{rgb}{0.000000,0.000000,0.000000}%
\pgfsetstrokecolor{currentstroke}%
\pgfsetdash{}{0pt}%
\pgfsys@defobject{currentmarker}{\pgfqpoint{0.000000in}{0.000000in}}{\pgfqpoint{0.048611in}{0.000000in}}{%
\pgfpathmoveto{\pgfqpoint{0.000000in}{0.000000in}}%
\pgfpathlineto{\pgfqpoint{0.048611in}{0.000000in}}%
\pgfusepath{stroke,fill}%
}%
\begin{pgfscope}%
\pgfsys@transformshift{7.200000in}{1.052772in}%
\pgfsys@useobject{currentmarker}{}%
\end{pgfscope}%
\end{pgfscope}%
\begin{pgfscope}%
\definecolor{textcolor}{rgb}{0.000000,0.000000,0.000000}%
\pgfsetstrokecolor{textcolor}%
\pgfsetfillcolor{textcolor}%
\pgftext[x=7.297222in, y=0.952752in, left, base]{\color{textcolor}\sffamily\fontsize{20.000000}{24.000000}\selectfont 0.0}%
\end{pgfscope}%
\begin{pgfscope}%
\pgfsetbuttcap%
\pgfsetroundjoin%
\definecolor{currentfill}{rgb}{0.000000,0.000000,0.000000}%
\pgfsetfillcolor{currentfill}%
\pgfsetlinewidth{0.803000pt}%
\definecolor{currentstroke}{rgb}{0.000000,0.000000,0.000000}%
\pgfsetstrokecolor{currentstroke}%
\pgfsetdash{}{0pt}%
\pgfsys@defobject{currentmarker}{\pgfqpoint{0.000000in}{0.000000in}}{\pgfqpoint{0.048611in}{0.000000in}}{%
\pgfpathmoveto{\pgfqpoint{0.000000in}{0.000000in}}%
\pgfpathlineto{\pgfqpoint{0.048611in}{0.000000in}}%
\pgfusepath{stroke,fill}%
}%
\begin{pgfscope}%
\pgfsys@transformshift{7.200000in}{1.683246in}%
\pgfsys@useobject{currentmarker}{}%
\end{pgfscope}%
\end{pgfscope}%
\begin{pgfscope}%
\definecolor{textcolor}{rgb}{0.000000,0.000000,0.000000}%
\pgfsetstrokecolor{textcolor}%
\pgfsetfillcolor{textcolor}%
\pgftext[x=7.297222in, y=1.583227in, left, base]{\color{textcolor}\sffamily\fontsize{20.000000}{24.000000}\selectfont 0.2}%
\end{pgfscope}%
\begin{pgfscope}%
\pgfsetbuttcap%
\pgfsetroundjoin%
\definecolor{currentfill}{rgb}{0.000000,0.000000,0.000000}%
\pgfsetfillcolor{currentfill}%
\pgfsetlinewidth{0.803000pt}%
\definecolor{currentstroke}{rgb}{0.000000,0.000000,0.000000}%
\pgfsetstrokecolor{currentstroke}%
\pgfsetdash{}{0pt}%
\pgfsys@defobject{currentmarker}{\pgfqpoint{0.000000in}{0.000000in}}{\pgfqpoint{0.048611in}{0.000000in}}{%
\pgfpathmoveto{\pgfqpoint{0.000000in}{0.000000in}}%
\pgfpathlineto{\pgfqpoint{0.048611in}{0.000000in}}%
\pgfusepath{stroke,fill}%
}%
\begin{pgfscope}%
\pgfsys@transformshift{7.200000in}{2.313721in}%
\pgfsys@useobject{currentmarker}{}%
\end{pgfscope}%
\end{pgfscope}%
\begin{pgfscope}%
\definecolor{textcolor}{rgb}{0.000000,0.000000,0.000000}%
\pgfsetstrokecolor{textcolor}%
\pgfsetfillcolor{textcolor}%
\pgftext[x=7.297222in, y=2.213702in, left, base]{\color{textcolor}\sffamily\fontsize{20.000000}{24.000000}\selectfont 0.5}%
\end{pgfscope}%
\begin{pgfscope}%
\pgfsetbuttcap%
\pgfsetroundjoin%
\definecolor{currentfill}{rgb}{0.000000,0.000000,0.000000}%
\pgfsetfillcolor{currentfill}%
\pgfsetlinewidth{0.803000pt}%
\definecolor{currentstroke}{rgb}{0.000000,0.000000,0.000000}%
\pgfsetstrokecolor{currentstroke}%
\pgfsetdash{}{0pt}%
\pgfsys@defobject{currentmarker}{\pgfqpoint{0.000000in}{0.000000in}}{\pgfqpoint{0.048611in}{0.000000in}}{%
\pgfpathmoveto{\pgfqpoint{0.000000in}{0.000000in}}%
\pgfpathlineto{\pgfqpoint{0.048611in}{0.000000in}}%
\pgfusepath{stroke,fill}%
}%
\begin{pgfscope}%
\pgfsys@transformshift{7.200000in}{2.944196in}%
\pgfsys@useobject{currentmarker}{}%
\end{pgfscope}%
\end{pgfscope}%
\begin{pgfscope}%
\definecolor{textcolor}{rgb}{0.000000,0.000000,0.000000}%
\pgfsetstrokecolor{textcolor}%
\pgfsetfillcolor{textcolor}%
\pgftext[x=7.297222in, y=2.844177in, left, base]{\color{textcolor}\sffamily\fontsize{20.000000}{24.000000}\selectfont 0.8}%
\end{pgfscope}%
\begin{pgfscope}%
\pgfsetbuttcap%
\pgfsetroundjoin%
\definecolor{currentfill}{rgb}{0.000000,0.000000,0.000000}%
\pgfsetfillcolor{currentfill}%
\pgfsetlinewidth{0.803000pt}%
\definecolor{currentstroke}{rgb}{0.000000,0.000000,0.000000}%
\pgfsetstrokecolor{currentstroke}%
\pgfsetdash{}{0pt}%
\pgfsys@defobject{currentmarker}{\pgfqpoint{0.000000in}{0.000000in}}{\pgfqpoint{0.048611in}{0.000000in}}{%
\pgfpathmoveto{\pgfqpoint{0.000000in}{0.000000in}}%
\pgfpathlineto{\pgfqpoint{0.048611in}{0.000000in}}%
\pgfusepath{stroke,fill}%
}%
\begin{pgfscope}%
\pgfsys@transformshift{7.200000in}{3.574671in}%
\pgfsys@useobject{currentmarker}{}%
\end{pgfscope}%
\end{pgfscope}%
\begin{pgfscope}%
\definecolor{textcolor}{rgb}{0.000000,0.000000,0.000000}%
\pgfsetstrokecolor{textcolor}%
\pgfsetfillcolor{textcolor}%
\pgftext[x=7.297222in, y=3.474651in, left, base]{\color{textcolor}\sffamily\fontsize{20.000000}{24.000000}\selectfont 1.0}%
\end{pgfscope}%
\begin{pgfscope}%
\pgfsetbuttcap%
\pgfsetroundjoin%
\definecolor{currentfill}{rgb}{0.000000,0.000000,0.000000}%
\pgfsetfillcolor{currentfill}%
\pgfsetlinewidth{0.803000pt}%
\definecolor{currentstroke}{rgb}{0.000000,0.000000,0.000000}%
\pgfsetstrokecolor{currentstroke}%
\pgfsetdash{}{0pt}%
\pgfsys@defobject{currentmarker}{\pgfqpoint{0.000000in}{0.000000in}}{\pgfqpoint{0.048611in}{0.000000in}}{%
\pgfpathmoveto{\pgfqpoint{0.000000in}{0.000000in}}%
\pgfpathlineto{\pgfqpoint{0.048611in}{0.000000in}}%
\pgfusepath{stroke,fill}%
}%
\begin{pgfscope}%
\pgfsys@transformshift{7.200000in}{4.205145in}%
\pgfsys@useobject{currentmarker}{}%
\end{pgfscope}%
\end{pgfscope}%
\begin{pgfscope}%
\definecolor{textcolor}{rgb}{0.000000,0.000000,0.000000}%
\pgfsetstrokecolor{textcolor}%
\pgfsetfillcolor{textcolor}%
\pgftext[x=7.297222in, y=4.105126in, left, base]{\color{textcolor}\sffamily\fontsize{20.000000}{24.000000}\selectfont 1.2}%
\end{pgfscope}%
\begin{pgfscope}%
\pgfsetbuttcap%
\pgfsetroundjoin%
\definecolor{currentfill}{rgb}{0.000000,0.000000,0.000000}%
\pgfsetfillcolor{currentfill}%
\pgfsetlinewidth{0.803000pt}%
\definecolor{currentstroke}{rgb}{0.000000,0.000000,0.000000}%
\pgfsetstrokecolor{currentstroke}%
\pgfsetdash{}{0pt}%
\pgfsys@defobject{currentmarker}{\pgfqpoint{0.000000in}{0.000000in}}{\pgfqpoint{0.048611in}{0.000000in}}{%
\pgfpathmoveto{\pgfqpoint{0.000000in}{0.000000in}}%
\pgfpathlineto{\pgfqpoint{0.048611in}{0.000000in}}%
\pgfusepath{stroke,fill}%
}%
\begin{pgfscope}%
\pgfsys@transformshift{7.200000in}{4.835620in}%
\pgfsys@useobject{currentmarker}{}%
\end{pgfscope}%
\end{pgfscope}%
\begin{pgfscope}%
\definecolor{textcolor}{rgb}{0.000000,0.000000,0.000000}%
\pgfsetstrokecolor{textcolor}%
\pgfsetfillcolor{textcolor}%
\pgftext[x=7.297222in, y=4.735601in, left, base]{\color{textcolor}\sffamily\fontsize{20.000000}{24.000000}\selectfont 1.5}%
\end{pgfscope}%
\begin{pgfscope}%
\definecolor{textcolor}{rgb}{0.000000,0.000000,0.000000}%
\pgfsetstrokecolor{textcolor}%
\pgfsetfillcolor{textcolor}%
\pgftext[x=7.698906in,y=3.030000in,,top,rotate=90.000000]{\color{textcolor}\sffamily\fontsize{20.000000}{24.000000}\selectfont \(\displaystyle \mathrm{Charge}\)}%
\end{pgfscope}%
\begin{pgfscope}%
\pgfpathrectangle{\pgfqpoint{1.000000in}{0.720000in}}{\pgfqpoint{6.200000in}{4.620000in}}%
\pgfusepath{clip}%
\pgfsetbuttcap%
\pgfsetroundjoin%
\pgfsetlinewidth{0.501875pt}%
\definecolor{currentstroke}{rgb}{1.000000,0.000000,0.000000}%
\pgfsetstrokecolor{currentstroke}%
\pgfsetdash{}{0pt}%
\pgfpathmoveto{\pgfqpoint{1.837000in}{1.052772in}}%
\pgfpathlineto{\pgfqpoint{1.837000in}{1.392998in}}%
\pgfusepath{stroke}%
\end{pgfscope}%
\begin{pgfscope}%
\pgfpathrectangle{\pgfqpoint{1.000000in}{0.720000in}}{\pgfqpoint{6.200000in}{4.620000in}}%
\pgfusepath{clip}%
\pgfsetbuttcap%
\pgfsetroundjoin%
\pgfsetlinewidth{0.501875pt}%
\definecolor{currentstroke}{rgb}{1.000000,0.000000,0.000000}%
\pgfsetstrokecolor{currentstroke}%
\pgfsetdash{}{0pt}%
\pgfpathmoveto{\pgfqpoint{1.868000in}{1.052772in}}%
\pgfpathlineto{\pgfqpoint{1.868000in}{1.558565in}}%
\pgfusepath{stroke}%
\end{pgfscope}%
\begin{pgfscope}%
\pgfpathrectangle{\pgfqpoint{1.000000in}{0.720000in}}{\pgfqpoint{6.200000in}{4.620000in}}%
\pgfusepath{clip}%
\pgfsetbuttcap%
\pgfsetroundjoin%
\pgfsetlinewidth{0.501875pt}%
\definecolor{currentstroke}{rgb}{1.000000,0.000000,0.000000}%
\pgfsetstrokecolor{currentstroke}%
\pgfsetdash{}{0pt}%
\pgfpathmoveto{\pgfqpoint{1.899000in}{1.052772in}}%
\pgfpathlineto{\pgfqpoint{1.899000in}{1.691569in}}%
\pgfusepath{stroke}%
\end{pgfscope}%
\begin{pgfscope}%
\pgfpathrectangle{\pgfqpoint{1.000000in}{0.720000in}}{\pgfqpoint{6.200000in}{4.620000in}}%
\pgfusepath{clip}%
\pgfsetbuttcap%
\pgfsetroundjoin%
\pgfsetlinewidth{0.501875pt}%
\definecolor{currentstroke}{rgb}{1.000000,0.000000,0.000000}%
\pgfsetstrokecolor{currentstroke}%
\pgfsetdash{}{0pt}%
\pgfpathmoveto{\pgfqpoint{1.930000in}{1.052772in}}%
\pgfpathlineto{\pgfqpoint{1.930000in}{1.771864in}}%
\pgfusepath{stroke}%
\end{pgfscope}%
\begin{pgfscope}%
\pgfpathrectangle{\pgfqpoint{1.000000in}{0.720000in}}{\pgfqpoint{6.200000in}{4.620000in}}%
\pgfusepath{clip}%
\pgfsetbuttcap%
\pgfsetroundjoin%
\pgfsetlinewidth{0.501875pt}%
\definecolor{currentstroke}{rgb}{1.000000,0.000000,0.000000}%
\pgfsetstrokecolor{currentstroke}%
\pgfsetdash{}{0pt}%
\pgfpathmoveto{\pgfqpoint{1.961000in}{1.052772in}}%
\pgfpathlineto{\pgfqpoint{1.961000in}{1.789505in}}%
\pgfusepath{stroke}%
\end{pgfscope}%
\begin{pgfscope}%
\pgfpathrectangle{\pgfqpoint{1.000000in}{0.720000in}}{\pgfqpoint{6.200000in}{4.620000in}}%
\pgfusepath{clip}%
\pgfsetbuttcap%
\pgfsetroundjoin%
\pgfsetlinewidth{0.501875pt}%
\definecolor{currentstroke}{rgb}{1.000000,0.000000,0.000000}%
\pgfsetstrokecolor{currentstroke}%
\pgfsetdash{}{0pt}%
\pgfpathmoveto{\pgfqpoint{1.992000in}{1.052772in}}%
\pgfpathlineto{\pgfqpoint{1.992000in}{1.745187in}}%
\pgfusepath{stroke}%
\end{pgfscope}%
\begin{pgfscope}%
\pgfpathrectangle{\pgfqpoint{1.000000in}{0.720000in}}{\pgfqpoint{6.200000in}{4.620000in}}%
\pgfusepath{clip}%
\pgfsetbuttcap%
\pgfsetroundjoin%
\pgfsetlinewidth{0.501875pt}%
\definecolor{currentstroke}{rgb}{1.000000,0.000000,0.000000}%
\pgfsetstrokecolor{currentstroke}%
\pgfsetdash{}{0pt}%
\pgfpathmoveto{\pgfqpoint{2.023000in}{1.052772in}}%
\pgfpathlineto{\pgfqpoint{2.023000in}{1.648230in}}%
\pgfusepath{stroke}%
\end{pgfscope}%
\begin{pgfscope}%
\pgfpathrectangle{\pgfqpoint{1.000000in}{0.720000in}}{\pgfqpoint{6.200000in}{4.620000in}}%
\pgfusepath{clip}%
\pgfsetbuttcap%
\pgfsetroundjoin%
\pgfsetlinewidth{0.501875pt}%
\definecolor{currentstroke}{rgb}{1.000000,0.000000,0.000000}%
\pgfsetstrokecolor{currentstroke}%
\pgfsetdash{}{0pt}%
\pgfpathmoveto{\pgfqpoint{2.054000in}{1.052772in}}%
\pgfpathlineto{\pgfqpoint{2.054000in}{1.513298in}}%
\pgfusepath{stroke}%
\end{pgfscope}%
\begin{pgfscope}%
\pgfpathrectangle{\pgfqpoint{1.000000in}{0.720000in}}{\pgfqpoint{6.200000in}{4.620000in}}%
\pgfusepath{clip}%
\pgfsetbuttcap%
\pgfsetroundjoin%
\pgfsetlinewidth{0.501875pt}%
\definecolor{currentstroke}{rgb}{1.000000,0.000000,0.000000}%
\pgfsetstrokecolor{currentstroke}%
\pgfsetdash{}{0pt}%
\pgfpathmoveto{\pgfqpoint{2.085000in}{1.052772in}}%
\pgfpathlineto{\pgfqpoint{2.085000in}{1.357257in}}%
\pgfusepath{stroke}%
\end{pgfscope}%
\begin{pgfscope}%
\pgfpathrectangle{\pgfqpoint{1.000000in}{0.720000in}}{\pgfqpoint{6.200000in}{4.620000in}}%
\pgfusepath{clip}%
\pgfsetbuttcap%
\pgfsetroundjoin%
\pgfsetlinewidth{0.501875pt}%
\definecolor{currentstroke}{rgb}{1.000000,0.000000,0.000000}%
\pgfsetstrokecolor{currentstroke}%
\pgfsetdash{}{0pt}%
\pgfpathmoveto{\pgfqpoint{2.364000in}{1.052772in}}%
\pgfpathlineto{\pgfqpoint{2.364000in}{1.335911in}}%
\pgfusepath{stroke}%
\end{pgfscope}%
\begin{pgfscope}%
\pgfpathrectangle{\pgfqpoint{1.000000in}{0.720000in}}{\pgfqpoint{6.200000in}{4.620000in}}%
\pgfusepath{clip}%
\pgfsetbuttcap%
\pgfsetroundjoin%
\pgfsetlinewidth{0.501875pt}%
\definecolor{currentstroke}{rgb}{1.000000,0.000000,0.000000}%
\pgfsetstrokecolor{currentstroke}%
\pgfsetdash{}{0pt}%
\pgfpathmoveto{\pgfqpoint{2.395000in}{1.052772in}}%
\pgfpathlineto{\pgfqpoint{2.395000in}{1.518930in}}%
\pgfusepath{stroke}%
\end{pgfscope}%
\begin{pgfscope}%
\pgfpathrectangle{\pgfqpoint{1.000000in}{0.720000in}}{\pgfqpoint{6.200000in}{4.620000in}}%
\pgfusepath{clip}%
\pgfsetbuttcap%
\pgfsetroundjoin%
\pgfsetlinewidth{0.501875pt}%
\definecolor{currentstroke}{rgb}{1.000000,0.000000,0.000000}%
\pgfsetstrokecolor{currentstroke}%
\pgfsetdash{}{0pt}%
\pgfpathmoveto{\pgfqpoint{2.426000in}{1.052772in}}%
\pgfpathlineto{\pgfqpoint{2.426000in}{1.660348in}}%
\pgfusepath{stroke}%
\end{pgfscope}%
\begin{pgfscope}%
\pgfpathrectangle{\pgfqpoint{1.000000in}{0.720000in}}{\pgfqpoint{6.200000in}{4.620000in}}%
\pgfusepath{clip}%
\pgfsetbuttcap%
\pgfsetroundjoin%
\pgfsetlinewidth{0.501875pt}%
\definecolor{currentstroke}{rgb}{1.000000,0.000000,0.000000}%
\pgfsetstrokecolor{currentstroke}%
\pgfsetdash{}{0pt}%
\pgfpathmoveto{\pgfqpoint{2.457000in}{1.052772in}}%
\pgfpathlineto{\pgfqpoint{2.457000in}{1.729364in}}%
\pgfusepath{stroke}%
\end{pgfscope}%
\begin{pgfscope}%
\pgfpathrectangle{\pgfqpoint{1.000000in}{0.720000in}}{\pgfqpoint{6.200000in}{4.620000in}}%
\pgfusepath{clip}%
\pgfsetbuttcap%
\pgfsetroundjoin%
\pgfsetlinewidth{0.501875pt}%
\definecolor{currentstroke}{rgb}{1.000000,0.000000,0.000000}%
\pgfsetstrokecolor{currentstroke}%
\pgfsetdash{}{0pt}%
\pgfpathmoveto{\pgfqpoint{2.488000in}{1.052772in}}%
\pgfpathlineto{\pgfqpoint{2.488000in}{1.709178in}}%
\pgfusepath{stroke}%
\end{pgfscope}%
\begin{pgfscope}%
\pgfpathrectangle{\pgfqpoint{1.000000in}{0.720000in}}{\pgfqpoint{6.200000in}{4.620000in}}%
\pgfusepath{clip}%
\pgfsetbuttcap%
\pgfsetroundjoin%
\pgfsetlinewidth{0.501875pt}%
\definecolor{currentstroke}{rgb}{1.000000,0.000000,0.000000}%
\pgfsetstrokecolor{currentstroke}%
\pgfsetdash{}{0pt}%
\pgfpathmoveto{\pgfqpoint{2.519000in}{1.052772in}}%
\pgfpathlineto{\pgfqpoint{2.519000in}{1.602789in}}%
\pgfusepath{stroke}%
\end{pgfscope}%
\begin{pgfscope}%
\pgfpathrectangle{\pgfqpoint{1.000000in}{0.720000in}}{\pgfqpoint{6.200000in}{4.620000in}}%
\pgfusepath{clip}%
\pgfsetbuttcap%
\pgfsetroundjoin%
\pgfsetlinewidth{0.501875pt}%
\definecolor{currentstroke}{rgb}{1.000000,0.000000,0.000000}%
\pgfsetstrokecolor{currentstroke}%
\pgfsetdash{}{0pt}%
\pgfpathmoveto{\pgfqpoint{2.550000in}{1.052772in}}%
\pgfpathlineto{\pgfqpoint{2.550000in}{1.433895in}}%
\pgfusepath{stroke}%
\end{pgfscope}%
\begin{pgfscope}%
\pgfpathrectangle{\pgfqpoint{1.000000in}{0.720000in}}{\pgfqpoint{6.200000in}{4.620000in}}%
\pgfusepath{clip}%
\pgfsetbuttcap%
\pgfsetroundjoin%
\pgfsetlinewidth{0.501875pt}%
\definecolor{currentstroke}{rgb}{1.000000,0.000000,0.000000}%
\pgfsetstrokecolor{currentstroke}%
\pgfsetdash{}{0pt}%
\pgfpathmoveto{\pgfqpoint{2.767000in}{1.052772in}}%
\pgfpathlineto{\pgfqpoint{2.767000in}{1.317021in}}%
\pgfusepath{stroke}%
\end{pgfscope}%
\begin{pgfscope}%
\pgfpathrectangle{\pgfqpoint{1.000000in}{0.720000in}}{\pgfqpoint{6.200000in}{4.620000in}}%
\pgfusepath{clip}%
\pgfsetbuttcap%
\pgfsetroundjoin%
\pgfsetlinewidth{0.501875pt}%
\definecolor{currentstroke}{rgb}{1.000000,0.000000,0.000000}%
\pgfsetstrokecolor{currentstroke}%
\pgfsetdash{}{0pt}%
\pgfpathmoveto{\pgfqpoint{2.798000in}{1.052772in}}%
\pgfpathlineto{\pgfqpoint{2.798000in}{1.463499in}}%
\pgfusepath{stroke}%
\end{pgfscope}%
\begin{pgfscope}%
\pgfpathrectangle{\pgfqpoint{1.000000in}{0.720000in}}{\pgfqpoint{6.200000in}{4.620000in}}%
\pgfusepath{clip}%
\pgfsetbuttcap%
\pgfsetroundjoin%
\pgfsetlinewidth{0.501875pt}%
\definecolor{currentstroke}{rgb}{1.000000,0.000000,0.000000}%
\pgfsetstrokecolor{currentstroke}%
\pgfsetdash{}{0pt}%
\pgfpathmoveto{\pgfqpoint{2.829000in}{1.052772in}}%
\pgfpathlineto{\pgfqpoint{2.829000in}{1.555210in}}%
\pgfusepath{stroke}%
\end{pgfscope}%
\begin{pgfscope}%
\pgfpathrectangle{\pgfqpoint{1.000000in}{0.720000in}}{\pgfqpoint{6.200000in}{4.620000in}}%
\pgfusepath{clip}%
\pgfsetbuttcap%
\pgfsetroundjoin%
\pgfsetlinewidth{0.501875pt}%
\definecolor{currentstroke}{rgb}{1.000000,0.000000,0.000000}%
\pgfsetstrokecolor{currentstroke}%
\pgfsetdash{}{0pt}%
\pgfpathmoveto{\pgfqpoint{2.860000in}{1.052772in}}%
\pgfpathlineto{\pgfqpoint{2.860000in}{1.570312in}}%
\pgfusepath{stroke}%
\end{pgfscope}%
\begin{pgfscope}%
\pgfpathrectangle{\pgfqpoint{1.000000in}{0.720000in}}{\pgfqpoint{6.200000in}{4.620000in}}%
\pgfusepath{clip}%
\pgfsetbuttcap%
\pgfsetroundjoin%
\pgfsetlinewidth{0.501875pt}%
\definecolor{currentstroke}{rgb}{1.000000,0.000000,0.000000}%
\pgfsetstrokecolor{currentstroke}%
\pgfsetdash{}{0pt}%
\pgfpathmoveto{\pgfqpoint{2.891000in}{1.052772in}}%
\pgfpathlineto{\pgfqpoint{2.891000in}{1.507960in}}%
\pgfusepath{stroke}%
\end{pgfscope}%
\begin{pgfscope}%
\pgfpathrectangle{\pgfqpoint{1.000000in}{0.720000in}}{\pgfqpoint{6.200000in}{4.620000in}}%
\pgfusepath{clip}%
\pgfsetbuttcap%
\pgfsetroundjoin%
\pgfsetlinewidth{0.501875pt}%
\definecolor{currentstroke}{rgb}{1.000000,0.000000,0.000000}%
\pgfsetstrokecolor{currentstroke}%
\pgfsetdash{}{0pt}%
\pgfpathmoveto{\pgfqpoint{2.922000in}{1.052772in}}%
\pgfpathlineto{\pgfqpoint{2.922000in}{1.386910in}}%
\pgfusepath{stroke}%
\end{pgfscope}%
\begin{pgfscope}%
\pgfpathrectangle{\pgfqpoint{1.000000in}{0.720000in}}{\pgfqpoint{6.200000in}{4.620000in}}%
\pgfusepath{clip}%
\pgfsetbuttcap%
\pgfsetroundjoin%
\pgfsetlinewidth{0.501875pt}%
\definecolor{currentstroke}{rgb}{1.000000,0.000000,0.000000}%
\pgfsetstrokecolor{currentstroke}%
\pgfsetdash{}{0pt}%
\pgfpathmoveto{\pgfqpoint{3.635000in}{1.052772in}}%
\pgfpathlineto{\pgfqpoint{3.635000in}{1.355506in}}%
\pgfusepath{stroke}%
\end{pgfscope}%
\begin{pgfscope}%
\pgfpathrectangle{\pgfqpoint{1.000000in}{0.720000in}}{\pgfqpoint{6.200000in}{4.620000in}}%
\pgfusepath{clip}%
\pgfsetbuttcap%
\pgfsetroundjoin%
\pgfsetlinewidth{0.501875pt}%
\definecolor{currentstroke}{rgb}{1.000000,0.000000,0.000000}%
\pgfsetstrokecolor{currentstroke}%
\pgfsetdash{}{0pt}%
\pgfpathmoveto{\pgfqpoint{3.666000in}{1.052772in}}%
\pgfpathlineto{\pgfqpoint{3.666000in}{1.468666in}}%
\pgfusepath{stroke}%
\end{pgfscope}%
\begin{pgfscope}%
\pgfpathrectangle{\pgfqpoint{1.000000in}{0.720000in}}{\pgfqpoint{6.200000in}{4.620000in}}%
\pgfusepath{clip}%
\pgfsetbuttcap%
\pgfsetroundjoin%
\pgfsetlinewidth{0.501875pt}%
\definecolor{currentstroke}{rgb}{1.000000,0.000000,0.000000}%
\pgfsetstrokecolor{currentstroke}%
\pgfsetdash{}{0pt}%
\pgfpathmoveto{\pgfqpoint{3.697000in}{1.052772in}}%
\pgfpathlineto{\pgfqpoint{3.697000in}{1.550501in}}%
\pgfusepath{stroke}%
\end{pgfscope}%
\begin{pgfscope}%
\pgfpathrectangle{\pgfqpoint{1.000000in}{0.720000in}}{\pgfqpoint{6.200000in}{4.620000in}}%
\pgfusepath{clip}%
\pgfsetbuttcap%
\pgfsetroundjoin%
\pgfsetlinewidth{0.501875pt}%
\definecolor{currentstroke}{rgb}{1.000000,0.000000,0.000000}%
\pgfsetstrokecolor{currentstroke}%
\pgfsetdash{}{0pt}%
\pgfpathmoveto{\pgfqpoint{3.728000in}{1.052772in}}%
\pgfpathlineto{\pgfqpoint{3.728000in}{1.580546in}}%
\pgfusepath{stroke}%
\end{pgfscope}%
\begin{pgfscope}%
\pgfpathrectangle{\pgfqpoint{1.000000in}{0.720000in}}{\pgfqpoint{6.200000in}{4.620000in}}%
\pgfusepath{clip}%
\pgfsetbuttcap%
\pgfsetroundjoin%
\pgfsetlinewidth{0.501875pt}%
\definecolor{currentstroke}{rgb}{1.000000,0.000000,0.000000}%
\pgfsetstrokecolor{currentstroke}%
\pgfsetdash{}{0pt}%
\pgfpathmoveto{\pgfqpoint{3.759000in}{1.052772in}}%
\pgfpathlineto{\pgfqpoint{3.759000in}{1.548245in}}%
\pgfusepath{stroke}%
\end{pgfscope}%
\begin{pgfscope}%
\pgfpathrectangle{\pgfqpoint{1.000000in}{0.720000in}}{\pgfqpoint{6.200000in}{4.620000in}}%
\pgfusepath{clip}%
\pgfsetbuttcap%
\pgfsetroundjoin%
\pgfsetlinewidth{0.501875pt}%
\definecolor{currentstroke}{rgb}{1.000000,0.000000,0.000000}%
\pgfsetstrokecolor{currentstroke}%
\pgfsetdash{}{0pt}%
\pgfpathmoveto{\pgfqpoint{3.790000in}{1.052772in}}%
\pgfpathlineto{\pgfqpoint{3.790000in}{1.457067in}}%
\pgfusepath{stroke}%
\end{pgfscope}%
\begin{pgfscope}%
\pgfpathrectangle{\pgfqpoint{1.000000in}{0.720000in}}{\pgfqpoint{6.200000in}{4.620000in}}%
\pgfusepath{clip}%
\pgfsetbuttcap%
\pgfsetroundjoin%
\pgfsetlinewidth{0.501875pt}%
\definecolor{currentstroke}{rgb}{1.000000,0.000000,0.000000}%
\pgfsetstrokecolor{currentstroke}%
\pgfsetdash{}{0pt}%
\pgfpathmoveto{\pgfqpoint{3.821000in}{1.052772in}}%
\pgfpathlineto{\pgfqpoint{3.821000in}{1.324688in}}%
\pgfusepath{stroke}%
\end{pgfscope}%
\begin{pgfscope}%
\pgfsetrectcap%
\pgfsetmiterjoin%
\pgfsetlinewidth{0.803000pt}%
\definecolor{currentstroke}{rgb}{0.000000,0.000000,0.000000}%
\pgfsetstrokecolor{currentstroke}%
\pgfsetdash{}{0pt}%
\pgfpathmoveto{\pgfqpoint{1.000000in}{0.720000in}}%
\pgfpathlineto{\pgfqpoint{1.000000in}{5.340000in}}%
\pgfusepath{stroke}%
\end{pgfscope}%
\begin{pgfscope}%
\pgfsetrectcap%
\pgfsetmiterjoin%
\pgfsetlinewidth{0.803000pt}%
\definecolor{currentstroke}{rgb}{0.000000,0.000000,0.000000}%
\pgfsetstrokecolor{currentstroke}%
\pgfsetdash{}{0pt}%
\pgfpathmoveto{\pgfqpoint{7.200000in}{0.720000in}}%
\pgfpathlineto{\pgfqpoint{7.200000in}{5.340000in}}%
\pgfusepath{stroke}%
\end{pgfscope}%
\begin{pgfscope}%
\pgfsetrectcap%
\pgfsetmiterjoin%
\pgfsetlinewidth{0.803000pt}%
\definecolor{currentstroke}{rgb}{0.000000,0.000000,0.000000}%
\pgfsetstrokecolor{currentstroke}%
\pgfsetdash{}{0pt}%
\pgfpathmoveto{\pgfqpoint{1.000000in}{0.720000in}}%
\pgfpathlineto{\pgfqpoint{7.200000in}{0.720000in}}%
\pgfusepath{stroke}%
\end{pgfscope}%
\begin{pgfscope}%
\pgfsetrectcap%
\pgfsetmiterjoin%
\pgfsetlinewidth{0.803000pt}%
\definecolor{currentstroke}{rgb}{0.000000,0.000000,0.000000}%
\pgfsetstrokecolor{currentstroke}%
\pgfsetdash{}{0pt}%
\pgfpathmoveto{\pgfqpoint{1.000000in}{5.340000in}}%
\pgfpathlineto{\pgfqpoint{7.200000in}{5.340000in}}%
\pgfusepath{stroke}%
\end{pgfscope}%
\begin{pgfscope}%
\pgfsetbuttcap%
\pgfsetmiterjoin%
\definecolor{currentfill}{rgb}{1.000000,1.000000,1.000000}%
\pgfsetfillcolor{currentfill}%
\pgfsetfillopacity{0.800000}%
\pgfsetlinewidth{1.003750pt}%
\definecolor{currentstroke}{rgb}{0.800000,0.800000,0.800000}%
\pgfsetstrokecolor{currentstroke}%
\pgfsetstrokeopacity{0.800000}%
\pgfsetdash{}{0pt}%
\pgfpathmoveto{\pgfqpoint{4.976872in}{3.932908in}}%
\pgfpathlineto{\pgfqpoint{7.005556in}{3.932908in}}%
\pgfpathquadraticcurveto{\pgfqpoint{7.061111in}{3.932908in}}{\pgfqpoint{7.061111in}{3.988464in}}%
\pgfpathlineto{\pgfqpoint{7.061111in}{5.145556in}}%
\pgfpathquadraticcurveto{\pgfqpoint{7.061111in}{5.201111in}}{\pgfqpoint{7.005556in}{5.201111in}}%
\pgfpathlineto{\pgfqpoint{4.976872in}{5.201111in}}%
\pgfpathquadraticcurveto{\pgfqpoint{4.921317in}{5.201111in}}{\pgfqpoint{4.921317in}{5.145556in}}%
\pgfpathlineto{\pgfqpoint{4.921317in}{3.988464in}}%
\pgfpathquadraticcurveto{\pgfqpoint{4.921317in}{3.932908in}}{\pgfqpoint{4.976872in}{3.932908in}}%
\pgfpathlineto{\pgfqpoint{4.976872in}{3.932908in}}%
\pgfpathclose%
\pgfusepath{stroke,fill}%
\end{pgfscope}%
\begin{pgfscope}%
\pgfsetrectcap%
\pgfsetroundjoin%
\pgfsetlinewidth{2.007500pt}%
\definecolor{currentstroke}{rgb}{0.121569,0.466667,0.705882}%
\pgfsetstrokecolor{currentstroke}%
\pgfsetdash{}{0pt}%
\pgfpathmoveto{\pgfqpoint{5.032428in}{4.987184in}}%
\pgfpathlineto{\pgfqpoint{5.310206in}{4.987184in}}%
\pgfpathlineto{\pgfqpoint{5.587983in}{4.987184in}}%
\pgfusepath{stroke}%
\end{pgfscope}%
\begin{pgfscope}%
\definecolor{textcolor}{rgb}{0.000000,0.000000,0.000000}%
\pgfsetstrokecolor{textcolor}%
\pgfsetfillcolor{textcolor}%
\pgftext[x=5.810206in,y=4.889962in,left,base]{\color{textcolor}\sffamily\fontsize{20.000000}{24.000000}\selectfont Waveform}%
\end{pgfscope}%
\begin{pgfscope}%
\pgfsetbuttcap%
\pgfsetroundjoin%
\pgfsetlinewidth{2.007500pt}%
\definecolor{currentstroke}{rgb}{0.000000,0.500000,0.000000}%
\pgfsetstrokecolor{currentstroke}%
\pgfsetdash{}{0pt}%
\pgfpathmoveto{\pgfqpoint{5.032428in}{4.592227in}}%
\pgfpathlineto{\pgfqpoint{5.587983in}{4.592227in}}%
\pgfusepath{stroke}%
\end{pgfscope}%
\begin{pgfscope}%
\definecolor{textcolor}{rgb}{0.000000,0.000000,0.000000}%
\pgfsetstrokecolor{textcolor}%
\pgfsetfillcolor{textcolor}%
\pgftext[x=5.810206in,y=4.495005in,left,base]{\color{textcolor}\sffamily\fontsize{20.000000}{24.000000}\selectfont Threshold}%
\end{pgfscope}%
\begin{pgfscope}%
\pgfsetbuttcap%
\pgfsetroundjoin%
\pgfsetlinewidth{0.501875pt}%
\definecolor{currentstroke}{rgb}{1.000000,0.000000,0.000000}%
\pgfsetstrokecolor{currentstroke}%
\pgfsetdash{}{0pt}%
\pgfpathmoveto{\pgfqpoint{5.032428in}{4.197271in}}%
\pgfpathlineto{\pgfqpoint{5.587983in}{4.197271in}}%
\pgfusepath{stroke}%
\end{pgfscope}%
\begin{pgfscope}%
\definecolor{textcolor}{rgb}{0.000000,0.000000,0.000000}%
\pgfsetstrokecolor{textcolor}%
\pgfsetfillcolor{textcolor}%
\pgftext[x=5.810206in,y=4.100048in,left,base]{\color{textcolor}\sffamily\fontsize{20.000000}{24.000000}\selectfont Charge}%
\end{pgfscope}%
\end{pgfpicture}%
\makeatother%
\endgroup%
}
    \caption{\label{fig:fd} A Fourier deconvolution example: \\ $\hat{t}_\mathrm{KL} - t_0=\SI{2.61}{ns}$, $\mathrm{RSS}=\SI{153.7}{mV^2}$, $D_\mathrm{w}=\SI{1.87}{ns}$.}
  \end{subfigure}
  \begin{subfigure}{0.5\textwidth}
    \centering
    \resizebox{\textwidth}{!}{%% Creator: Matplotlib, PGF backend
%%
%% To include the figure in your LaTeX document, write
%%   \input{<filename>.pgf}
%%
%% Make sure the required packages are loaded in your preamble
%%   \usepackage{pgf}
%%
%% Also ensure that all the required font packages are loaded; for instance,
%% the lmodern package is sometimes necessary when using math font.
%%   \usepackage{lmodern}
%%
%% Figures using additional raster images can only be included by \input if
%% they are in the same directory as the main LaTeX file. For loading figures
%% from other directories you can use the `import` package
%%   \usepackage{import}
%%
%% and then include the figures with
%%   \import{<path to file>}{<filename>.pgf}
%%
%% Matplotlib used the following preamble
%%   \usepackage[detect-all,locale=DE]{siunitx}
%%
\begingroup%
\makeatletter%
\begin{pgfpicture}%
\pgfpathrectangle{\pgfpointorigin}{\pgfqpoint{8.000000in}{6.000000in}}%
\pgfusepath{use as bounding box, clip}%
\begin{pgfscope}%
\pgfsetbuttcap%
\pgfsetmiterjoin%
\definecolor{currentfill}{rgb}{1.000000,1.000000,1.000000}%
\pgfsetfillcolor{currentfill}%
\pgfsetlinewidth{0.000000pt}%
\definecolor{currentstroke}{rgb}{1.000000,1.000000,1.000000}%
\pgfsetstrokecolor{currentstroke}%
\pgfsetdash{}{0pt}%
\pgfpathmoveto{\pgfqpoint{0.000000in}{0.000000in}}%
\pgfpathlineto{\pgfqpoint{8.000000in}{0.000000in}}%
\pgfpathlineto{\pgfqpoint{8.000000in}{6.000000in}}%
\pgfpathlineto{\pgfqpoint{0.000000in}{6.000000in}}%
\pgfpathlineto{\pgfqpoint{0.000000in}{0.000000in}}%
\pgfpathclose%
\pgfusepath{fill}%
\end{pgfscope}%
\begin{pgfscope}%
\pgfsetbuttcap%
\pgfsetmiterjoin%
\definecolor{currentfill}{rgb}{1.000000,1.000000,1.000000}%
\pgfsetfillcolor{currentfill}%
\pgfsetlinewidth{0.000000pt}%
\definecolor{currentstroke}{rgb}{0.000000,0.000000,0.000000}%
\pgfsetstrokecolor{currentstroke}%
\pgfsetstrokeopacity{0.000000}%
\pgfsetdash{}{0pt}%
\pgfpathmoveto{\pgfqpoint{1.000000in}{0.720000in}}%
\pgfpathlineto{\pgfqpoint{7.200000in}{0.720000in}}%
\pgfpathlineto{\pgfqpoint{7.200000in}{5.340000in}}%
\pgfpathlineto{\pgfqpoint{1.000000in}{5.340000in}}%
\pgfpathlineto{\pgfqpoint{1.000000in}{0.720000in}}%
\pgfpathclose%
\pgfusepath{fill}%
\end{pgfscope}%
\begin{pgfscope}%
\pgfsetbuttcap%
\pgfsetroundjoin%
\definecolor{currentfill}{rgb}{0.000000,0.000000,0.000000}%
\pgfsetfillcolor{currentfill}%
\pgfsetlinewidth{0.803000pt}%
\definecolor{currentstroke}{rgb}{0.000000,0.000000,0.000000}%
\pgfsetstrokecolor{currentstroke}%
\pgfsetdash{}{0pt}%
\pgfsys@defobject{currentmarker}{\pgfqpoint{0.000000in}{-0.048611in}}{\pgfqpoint{0.000000in}{0.000000in}}{%
\pgfpathmoveto{\pgfqpoint{0.000000in}{0.000000in}}%
\pgfpathlineto{\pgfqpoint{0.000000in}{-0.048611in}}%
\pgfusepath{stroke,fill}%
}%
\begin{pgfscope}%
\pgfsys@transformshift{1.310000in}{0.720000in}%
\pgfsys@useobject{currentmarker}{}%
\end{pgfscope}%
\end{pgfscope}%
\begin{pgfscope}%
\definecolor{textcolor}{rgb}{0.000000,0.000000,0.000000}%
\pgfsetstrokecolor{textcolor}%
\pgfsetfillcolor{textcolor}%
\pgftext[x=1.310000in,y=0.622778in,,top]{\color{textcolor}\sffamily\fontsize{20.000000}{24.000000}\selectfont \(\displaystyle {450}\)}%
\end{pgfscope}%
\begin{pgfscope}%
\pgfsetbuttcap%
\pgfsetroundjoin%
\definecolor{currentfill}{rgb}{0.000000,0.000000,0.000000}%
\pgfsetfillcolor{currentfill}%
\pgfsetlinewidth{0.803000pt}%
\definecolor{currentstroke}{rgb}{0.000000,0.000000,0.000000}%
\pgfsetstrokecolor{currentstroke}%
\pgfsetdash{}{0pt}%
\pgfsys@defobject{currentmarker}{\pgfqpoint{0.000000in}{-0.048611in}}{\pgfqpoint{0.000000in}{0.000000in}}{%
\pgfpathmoveto{\pgfqpoint{0.000000in}{0.000000in}}%
\pgfpathlineto{\pgfqpoint{0.000000in}{-0.048611in}}%
\pgfusepath{stroke,fill}%
}%
\begin{pgfscope}%
\pgfsys@transformshift{2.860000in}{0.720000in}%
\pgfsys@useobject{currentmarker}{}%
\end{pgfscope}%
\end{pgfscope}%
\begin{pgfscope}%
\definecolor{textcolor}{rgb}{0.000000,0.000000,0.000000}%
\pgfsetstrokecolor{textcolor}%
\pgfsetfillcolor{textcolor}%
\pgftext[x=2.860000in,y=0.622778in,,top]{\color{textcolor}\sffamily\fontsize{20.000000}{24.000000}\selectfont \(\displaystyle {500}\)}%
\end{pgfscope}%
\begin{pgfscope}%
\pgfsetbuttcap%
\pgfsetroundjoin%
\definecolor{currentfill}{rgb}{0.000000,0.000000,0.000000}%
\pgfsetfillcolor{currentfill}%
\pgfsetlinewidth{0.803000pt}%
\definecolor{currentstroke}{rgb}{0.000000,0.000000,0.000000}%
\pgfsetstrokecolor{currentstroke}%
\pgfsetdash{}{0pt}%
\pgfsys@defobject{currentmarker}{\pgfqpoint{0.000000in}{-0.048611in}}{\pgfqpoint{0.000000in}{0.000000in}}{%
\pgfpathmoveto{\pgfqpoint{0.000000in}{0.000000in}}%
\pgfpathlineto{\pgfqpoint{0.000000in}{-0.048611in}}%
\pgfusepath{stroke,fill}%
}%
\begin{pgfscope}%
\pgfsys@transformshift{4.410000in}{0.720000in}%
\pgfsys@useobject{currentmarker}{}%
\end{pgfscope}%
\end{pgfscope}%
\begin{pgfscope}%
\definecolor{textcolor}{rgb}{0.000000,0.000000,0.000000}%
\pgfsetstrokecolor{textcolor}%
\pgfsetfillcolor{textcolor}%
\pgftext[x=4.410000in,y=0.622778in,,top]{\color{textcolor}\sffamily\fontsize{20.000000}{24.000000}\selectfont \(\displaystyle {550}\)}%
\end{pgfscope}%
\begin{pgfscope}%
\pgfsetbuttcap%
\pgfsetroundjoin%
\definecolor{currentfill}{rgb}{0.000000,0.000000,0.000000}%
\pgfsetfillcolor{currentfill}%
\pgfsetlinewidth{0.803000pt}%
\definecolor{currentstroke}{rgb}{0.000000,0.000000,0.000000}%
\pgfsetstrokecolor{currentstroke}%
\pgfsetdash{}{0pt}%
\pgfsys@defobject{currentmarker}{\pgfqpoint{0.000000in}{-0.048611in}}{\pgfqpoint{0.000000in}{0.000000in}}{%
\pgfpathmoveto{\pgfqpoint{0.000000in}{0.000000in}}%
\pgfpathlineto{\pgfqpoint{0.000000in}{-0.048611in}}%
\pgfusepath{stroke,fill}%
}%
\begin{pgfscope}%
\pgfsys@transformshift{5.960000in}{0.720000in}%
\pgfsys@useobject{currentmarker}{}%
\end{pgfscope}%
\end{pgfscope}%
\begin{pgfscope}%
\definecolor{textcolor}{rgb}{0.000000,0.000000,0.000000}%
\pgfsetstrokecolor{textcolor}%
\pgfsetfillcolor{textcolor}%
\pgftext[x=5.960000in,y=0.622778in,,top]{\color{textcolor}\sffamily\fontsize{20.000000}{24.000000}\selectfont \(\displaystyle {600}\)}%
\end{pgfscope}%
\begin{pgfscope}%
\definecolor{textcolor}{rgb}{0.000000,0.000000,0.000000}%
\pgfsetstrokecolor{textcolor}%
\pgfsetfillcolor{textcolor}%
\pgftext[x=4.100000in,y=0.311155in,,top]{\color{textcolor}\sffamily\fontsize{20.000000}{24.000000}\selectfont \(\displaystyle \mathrm{t}/\si{ns}\)}%
\end{pgfscope}%
\begin{pgfscope}%
\pgfsetbuttcap%
\pgfsetroundjoin%
\definecolor{currentfill}{rgb}{0.000000,0.000000,0.000000}%
\pgfsetfillcolor{currentfill}%
\pgfsetlinewidth{0.803000pt}%
\definecolor{currentstroke}{rgb}{0.000000,0.000000,0.000000}%
\pgfsetstrokecolor{currentstroke}%
\pgfsetdash{}{0pt}%
\pgfsys@defobject{currentmarker}{\pgfqpoint{-0.048611in}{0.000000in}}{\pgfqpoint{-0.000000in}{0.000000in}}{%
\pgfpathmoveto{\pgfqpoint{-0.000000in}{0.000000in}}%
\pgfpathlineto{\pgfqpoint{-0.048611in}{0.000000in}}%
\pgfusepath{stroke,fill}%
}%
\begin{pgfscope}%
\pgfsys@transformshift{1.000000in}{1.145834in}%
\pgfsys@useobject{currentmarker}{}%
\end{pgfscope}%
\end{pgfscope}%
\begin{pgfscope}%
\definecolor{textcolor}{rgb}{0.000000,0.000000,0.000000}%
\pgfsetstrokecolor{textcolor}%
\pgfsetfillcolor{textcolor}%
\pgftext[x=0.770670in, y=1.045815in, left, base]{\color{textcolor}\sffamily\fontsize{20.000000}{24.000000}\selectfont \(\displaystyle {0}\)}%
\end{pgfscope}%
\begin{pgfscope}%
\pgfsetbuttcap%
\pgfsetroundjoin%
\definecolor{currentfill}{rgb}{0.000000,0.000000,0.000000}%
\pgfsetfillcolor{currentfill}%
\pgfsetlinewidth{0.803000pt}%
\definecolor{currentstroke}{rgb}{0.000000,0.000000,0.000000}%
\pgfsetstrokecolor{currentstroke}%
\pgfsetdash{}{0pt}%
\pgfsys@defobject{currentmarker}{\pgfqpoint{-0.048611in}{0.000000in}}{\pgfqpoint{-0.000000in}{0.000000in}}{%
\pgfpathmoveto{\pgfqpoint{-0.000000in}{0.000000in}}%
\pgfpathlineto{\pgfqpoint{-0.048611in}{0.000000in}}%
\pgfusepath{stroke,fill}%
}%
\begin{pgfscope}%
\pgfsys@transformshift{1.000000in}{1.760228in}%
\pgfsys@useobject{currentmarker}{}%
\end{pgfscope}%
\end{pgfscope}%
\begin{pgfscope}%
\definecolor{textcolor}{rgb}{0.000000,0.000000,0.000000}%
\pgfsetstrokecolor{textcolor}%
\pgfsetfillcolor{textcolor}%
\pgftext[x=0.770670in, y=1.660209in, left, base]{\color{textcolor}\sffamily\fontsize{20.000000}{24.000000}\selectfont \(\displaystyle {5}\)}%
\end{pgfscope}%
\begin{pgfscope}%
\pgfsetbuttcap%
\pgfsetroundjoin%
\definecolor{currentfill}{rgb}{0.000000,0.000000,0.000000}%
\pgfsetfillcolor{currentfill}%
\pgfsetlinewidth{0.803000pt}%
\definecolor{currentstroke}{rgb}{0.000000,0.000000,0.000000}%
\pgfsetstrokecolor{currentstroke}%
\pgfsetdash{}{0pt}%
\pgfsys@defobject{currentmarker}{\pgfqpoint{-0.048611in}{0.000000in}}{\pgfqpoint{-0.000000in}{0.000000in}}{%
\pgfpathmoveto{\pgfqpoint{-0.000000in}{0.000000in}}%
\pgfpathlineto{\pgfqpoint{-0.048611in}{0.000000in}}%
\pgfusepath{stroke,fill}%
}%
\begin{pgfscope}%
\pgfsys@transformshift{1.000000in}{2.374622in}%
\pgfsys@useobject{currentmarker}{}%
\end{pgfscope}%
\end{pgfscope}%
\begin{pgfscope}%
\definecolor{textcolor}{rgb}{0.000000,0.000000,0.000000}%
\pgfsetstrokecolor{textcolor}%
\pgfsetfillcolor{textcolor}%
\pgftext[x=0.638563in, y=2.274603in, left, base]{\color{textcolor}\sffamily\fontsize{20.000000}{24.000000}\selectfont \(\displaystyle {10}\)}%
\end{pgfscope}%
\begin{pgfscope}%
\pgfsetbuttcap%
\pgfsetroundjoin%
\definecolor{currentfill}{rgb}{0.000000,0.000000,0.000000}%
\pgfsetfillcolor{currentfill}%
\pgfsetlinewidth{0.803000pt}%
\definecolor{currentstroke}{rgb}{0.000000,0.000000,0.000000}%
\pgfsetstrokecolor{currentstroke}%
\pgfsetdash{}{0pt}%
\pgfsys@defobject{currentmarker}{\pgfqpoint{-0.048611in}{0.000000in}}{\pgfqpoint{-0.000000in}{0.000000in}}{%
\pgfpathmoveto{\pgfqpoint{-0.000000in}{0.000000in}}%
\pgfpathlineto{\pgfqpoint{-0.048611in}{0.000000in}}%
\pgfusepath{stroke,fill}%
}%
\begin{pgfscope}%
\pgfsys@transformshift{1.000000in}{2.989016in}%
\pgfsys@useobject{currentmarker}{}%
\end{pgfscope}%
\end{pgfscope}%
\begin{pgfscope}%
\definecolor{textcolor}{rgb}{0.000000,0.000000,0.000000}%
\pgfsetstrokecolor{textcolor}%
\pgfsetfillcolor{textcolor}%
\pgftext[x=0.638563in, y=2.888996in, left, base]{\color{textcolor}\sffamily\fontsize{20.000000}{24.000000}\selectfont \(\displaystyle {15}\)}%
\end{pgfscope}%
\begin{pgfscope}%
\pgfsetbuttcap%
\pgfsetroundjoin%
\definecolor{currentfill}{rgb}{0.000000,0.000000,0.000000}%
\pgfsetfillcolor{currentfill}%
\pgfsetlinewidth{0.803000pt}%
\definecolor{currentstroke}{rgb}{0.000000,0.000000,0.000000}%
\pgfsetstrokecolor{currentstroke}%
\pgfsetdash{}{0pt}%
\pgfsys@defobject{currentmarker}{\pgfqpoint{-0.048611in}{0.000000in}}{\pgfqpoint{-0.000000in}{0.000000in}}{%
\pgfpathmoveto{\pgfqpoint{-0.000000in}{0.000000in}}%
\pgfpathlineto{\pgfqpoint{-0.048611in}{0.000000in}}%
\pgfusepath{stroke,fill}%
}%
\begin{pgfscope}%
\pgfsys@transformshift{1.000000in}{3.603409in}%
\pgfsys@useobject{currentmarker}{}%
\end{pgfscope}%
\end{pgfscope}%
\begin{pgfscope}%
\definecolor{textcolor}{rgb}{0.000000,0.000000,0.000000}%
\pgfsetstrokecolor{textcolor}%
\pgfsetfillcolor{textcolor}%
\pgftext[x=0.638563in, y=3.503390in, left, base]{\color{textcolor}\sffamily\fontsize{20.000000}{24.000000}\selectfont \(\displaystyle {20}\)}%
\end{pgfscope}%
\begin{pgfscope}%
\pgfsetbuttcap%
\pgfsetroundjoin%
\definecolor{currentfill}{rgb}{0.000000,0.000000,0.000000}%
\pgfsetfillcolor{currentfill}%
\pgfsetlinewidth{0.803000pt}%
\definecolor{currentstroke}{rgb}{0.000000,0.000000,0.000000}%
\pgfsetstrokecolor{currentstroke}%
\pgfsetdash{}{0pt}%
\pgfsys@defobject{currentmarker}{\pgfqpoint{-0.048611in}{0.000000in}}{\pgfqpoint{-0.000000in}{0.000000in}}{%
\pgfpathmoveto{\pgfqpoint{-0.000000in}{0.000000in}}%
\pgfpathlineto{\pgfqpoint{-0.048611in}{0.000000in}}%
\pgfusepath{stroke,fill}%
}%
\begin{pgfscope}%
\pgfsys@transformshift{1.000000in}{4.217803in}%
\pgfsys@useobject{currentmarker}{}%
\end{pgfscope}%
\end{pgfscope}%
\begin{pgfscope}%
\definecolor{textcolor}{rgb}{0.000000,0.000000,0.000000}%
\pgfsetstrokecolor{textcolor}%
\pgfsetfillcolor{textcolor}%
\pgftext[x=0.638563in, y=4.117784in, left, base]{\color{textcolor}\sffamily\fontsize{20.000000}{24.000000}\selectfont \(\displaystyle {25}\)}%
\end{pgfscope}%
\begin{pgfscope}%
\pgfsetbuttcap%
\pgfsetroundjoin%
\definecolor{currentfill}{rgb}{0.000000,0.000000,0.000000}%
\pgfsetfillcolor{currentfill}%
\pgfsetlinewidth{0.803000pt}%
\definecolor{currentstroke}{rgb}{0.000000,0.000000,0.000000}%
\pgfsetstrokecolor{currentstroke}%
\pgfsetdash{}{0pt}%
\pgfsys@defobject{currentmarker}{\pgfqpoint{-0.048611in}{0.000000in}}{\pgfqpoint{-0.000000in}{0.000000in}}{%
\pgfpathmoveto{\pgfqpoint{-0.000000in}{0.000000in}}%
\pgfpathlineto{\pgfqpoint{-0.048611in}{0.000000in}}%
\pgfusepath{stroke,fill}%
}%
\begin{pgfscope}%
\pgfsys@transformshift{1.000000in}{4.832197in}%
\pgfsys@useobject{currentmarker}{}%
\end{pgfscope}%
\end{pgfscope}%
\begin{pgfscope}%
\definecolor{textcolor}{rgb}{0.000000,0.000000,0.000000}%
\pgfsetstrokecolor{textcolor}%
\pgfsetfillcolor{textcolor}%
\pgftext[x=0.638563in, y=4.732177in, left, base]{\color{textcolor}\sffamily\fontsize{20.000000}{24.000000}\selectfont \(\displaystyle {30}\)}%
\end{pgfscope}%
\begin{pgfscope}%
\definecolor{textcolor}{rgb}{0.000000,0.000000,0.000000}%
\pgfsetstrokecolor{textcolor}%
\pgfsetfillcolor{textcolor}%
\pgftext[x=0.583007in,y=3.030000in,,bottom,rotate=90.000000]{\color{textcolor}\sffamily\fontsize{20.000000}{24.000000}\selectfont \(\displaystyle \mathrm{Voltage}/\si{mV}\)}%
\end{pgfscope}%
\begin{pgfscope}%
\pgfpathrectangle{\pgfqpoint{1.000000in}{0.720000in}}{\pgfqpoint{6.200000in}{4.620000in}}%
\pgfusepath{clip}%
\pgfsetrectcap%
\pgfsetroundjoin%
\pgfsetlinewidth{2.007500pt}%
\definecolor{currentstroke}{rgb}{0.121569,0.466667,0.705882}%
\pgfsetstrokecolor{currentstroke}%
\pgfsetdash{}{0pt}%
\pgfpathmoveto{\pgfqpoint{0.990000in}{1.222593in}}%
\pgfpathlineto{\pgfqpoint{1.000000in}{1.270894in}}%
\pgfpathlineto{\pgfqpoint{1.031000in}{1.096684in}}%
\pgfpathlineto{\pgfqpoint{1.062000in}{1.199203in}}%
\pgfpathlineto{\pgfqpoint{1.093000in}{1.182173in}}%
\pgfpathlineto{\pgfqpoint{1.124000in}{1.162905in}}%
\pgfpathlineto{\pgfqpoint{1.155000in}{1.284631in}}%
\pgfpathlineto{\pgfqpoint{1.186000in}{1.177477in}}%
\pgfpathlineto{\pgfqpoint{1.217000in}{1.091844in}}%
\pgfpathlineto{\pgfqpoint{1.248000in}{1.268450in}}%
\pgfpathlineto{\pgfqpoint{1.279000in}{1.178661in}}%
\pgfpathlineto{\pgfqpoint{1.310000in}{1.073030in}}%
\pgfpathlineto{\pgfqpoint{1.341000in}{1.206884in}}%
\pgfpathlineto{\pgfqpoint{1.372000in}{1.124892in}}%
\pgfpathlineto{\pgfqpoint{1.403000in}{1.416265in}}%
\pgfpathlineto{\pgfqpoint{1.434000in}{1.067380in}}%
\pgfpathlineto{\pgfqpoint{1.465000in}{0.989741in}}%
\pgfpathlineto{\pgfqpoint{1.496000in}{1.095559in}}%
\pgfpathlineto{\pgfqpoint{1.527000in}{1.024990in}}%
\pgfpathlineto{\pgfqpoint{1.558000in}{0.991097in}}%
\pgfpathlineto{\pgfqpoint{1.589000in}{1.134677in}}%
\pgfpathlineto{\pgfqpoint{1.620000in}{1.246345in}}%
\pgfpathlineto{\pgfqpoint{1.651000in}{1.216694in}}%
\pgfpathlineto{\pgfqpoint{1.682000in}{1.359832in}}%
\pgfpathlineto{\pgfqpoint{1.713000in}{1.251417in}}%
\pgfpathlineto{\pgfqpoint{1.744000in}{1.282940in}}%
\pgfpathlineto{\pgfqpoint{1.775000in}{1.301399in}}%
\pgfpathlineto{\pgfqpoint{1.806000in}{1.259591in}}%
\pgfpathlineto{\pgfqpoint{1.837000in}{1.121847in}}%
\pgfpathlineto{\pgfqpoint{1.868000in}{0.996067in}}%
\pgfpathlineto{\pgfqpoint{1.899000in}{1.170727in}}%
\pgfpathlineto{\pgfqpoint{1.930000in}{1.108552in}}%
\pgfpathlineto{\pgfqpoint{1.961000in}{1.253788in}}%
\pgfpathlineto{\pgfqpoint{1.992000in}{1.246488in}}%
\pgfpathlineto{\pgfqpoint{2.023000in}{1.523801in}}%
\pgfpathlineto{\pgfqpoint{2.054000in}{1.984247in}}%
\pgfpathlineto{\pgfqpoint{2.085000in}{2.516883in}}%
\pgfpathlineto{\pgfqpoint{2.116000in}{3.445723in}}%
\pgfpathlineto{\pgfqpoint{2.147000in}{3.680523in}}%
\pgfpathlineto{\pgfqpoint{2.178000in}{4.205954in}}%
\pgfpathlineto{\pgfqpoint{2.209000in}{4.372116in}}%
\pgfpathlineto{\pgfqpoint{2.240000in}{4.065688in}}%
\pgfpathlineto{\pgfqpoint{2.271000in}{4.043842in}}%
\pgfpathlineto{\pgfqpoint{2.302000in}{3.941145in}}%
\pgfpathlineto{\pgfqpoint{2.333000in}{3.300069in}}%
\pgfpathlineto{\pgfqpoint{2.364000in}{3.078449in}}%
\pgfpathlineto{\pgfqpoint{2.395000in}{3.064178in}}%
\pgfpathlineto{\pgfqpoint{2.426000in}{2.690044in}}%
\pgfpathlineto{\pgfqpoint{2.457000in}{2.201641in}}%
\pgfpathlineto{\pgfqpoint{2.488000in}{2.085521in}}%
\pgfpathlineto{\pgfqpoint{2.519000in}{1.980297in}}%
\pgfpathlineto{\pgfqpoint{2.550000in}{1.923553in}}%
\pgfpathlineto{\pgfqpoint{2.581000in}{2.204345in}}%
\pgfpathlineto{\pgfqpoint{2.612000in}{2.945116in}}%
\pgfpathlineto{\pgfqpoint{2.643000in}{3.255714in}}%
\pgfpathlineto{\pgfqpoint{2.674000in}{3.764764in}}%
\pgfpathlineto{\pgfqpoint{2.705000in}{3.880315in}}%
\pgfpathlineto{\pgfqpoint{2.736000in}{3.733495in}}%
\pgfpathlineto{\pgfqpoint{2.767000in}{3.523265in}}%
\pgfpathlineto{\pgfqpoint{2.798000in}{3.340267in}}%
\pgfpathlineto{\pgfqpoint{2.829000in}{3.126720in}}%
\pgfpathlineto{\pgfqpoint{2.860000in}{2.975547in}}%
\pgfpathlineto{\pgfqpoint{2.891000in}{2.613938in}}%
\pgfpathlineto{\pgfqpoint{2.922000in}{2.673898in}}%
\pgfpathlineto{\pgfqpoint{2.953000in}{2.415341in}}%
\pgfpathlineto{\pgfqpoint{2.984000in}{3.086620in}}%
\pgfpathlineto{\pgfqpoint{3.015000in}{3.095033in}}%
\pgfpathlineto{\pgfqpoint{3.046000in}{3.509890in}}%
\pgfpathlineto{\pgfqpoint{3.077000in}{3.344637in}}%
\pgfpathlineto{\pgfqpoint{3.108000in}{3.373180in}}%
\pgfpathlineto{\pgfqpoint{3.139000in}{3.271231in}}%
\pgfpathlineto{\pgfqpoint{3.170000in}{2.933447in}}%
\pgfpathlineto{\pgfqpoint{3.201000in}{2.787981in}}%
\pgfpathlineto{\pgfqpoint{3.232000in}{2.438344in}}%
\pgfpathlineto{\pgfqpoint{3.263000in}{2.254036in}}%
\pgfpathlineto{\pgfqpoint{3.294000in}{2.052950in}}%
\pgfpathlineto{\pgfqpoint{3.325000in}{1.772394in}}%
\pgfpathlineto{\pgfqpoint{3.356000in}{1.928720in}}%
\pgfpathlineto{\pgfqpoint{3.387000in}{1.947541in}}%
\pgfpathlineto{\pgfqpoint{3.418000in}{1.654039in}}%
\pgfpathlineto{\pgfqpoint{3.449000in}{1.395126in}}%
\pgfpathlineto{\pgfqpoint{3.480000in}{1.335504in}}%
\pgfpathlineto{\pgfqpoint{3.511000in}{1.608582in}}%
\pgfpathlineto{\pgfqpoint{3.542000in}{1.312849in}}%
\pgfpathlineto{\pgfqpoint{3.573000in}{1.284391in}}%
\pgfpathlineto{\pgfqpoint{3.604000in}{1.447068in}}%
\pgfpathlineto{\pgfqpoint{3.635000in}{1.230849in}}%
\pgfpathlineto{\pgfqpoint{3.666000in}{1.174907in}}%
\pgfpathlineto{\pgfqpoint{3.697000in}{1.262077in}}%
\pgfpathlineto{\pgfqpoint{3.728000in}{1.166534in}}%
\pgfpathlineto{\pgfqpoint{3.759000in}{1.093215in}}%
\pgfpathlineto{\pgfqpoint{3.790000in}{1.433177in}}%
\pgfpathlineto{\pgfqpoint{3.821000in}{1.558773in}}%
\pgfpathlineto{\pgfqpoint{3.852000in}{2.170644in}}%
\pgfpathlineto{\pgfqpoint{3.883000in}{2.453631in}}%
\pgfpathlineto{\pgfqpoint{3.914000in}{3.047367in}}%
\pgfpathlineto{\pgfqpoint{3.945000in}{2.995175in}}%
\pgfpathlineto{\pgfqpoint{3.976000in}{3.368100in}}%
\pgfpathlineto{\pgfqpoint{4.007000in}{3.201769in}}%
\pgfpathlineto{\pgfqpoint{4.038000in}{2.748277in}}%
\pgfpathlineto{\pgfqpoint{4.069000in}{2.746696in}}%
\pgfpathlineto{\pgfqpoint{4.100000in}{2.585518in}}%
\pgfpathlineto{\pgfqpoint{4.131000in}{2.310610in}}%
\pgfpathlineto{\pgfqpoint{4.162000in}{2.080649in}}%
\pgfpathlineto{\pgfqpoint{4.193000in}{2.001216in}}%
\pgfpathlineto{\pgfqpoint{4.224000in}{2.058998in}}%
\pgfpathlineto{\pgfqpoint{4.255000in}{1.975551in}}%
\pgfpathlineto{\pgfqpoint{4.286000in}{1.582643in}}%
\pgfpathlineto{\pgfqpoint{4.317000in}{1.542779in}}%
\pgfpathlineto{\pgfqpoint{4.348000in}{1.470070in}}%
\pgfpathlineto{\pgfqpoint{4.379000in}{1.365768in}}%
\pgfpathlineto{\pgfqpoint{4.410000in}{1.220194in}}%
\pgfpathlineto{\pgfqpoint{4.441000in}{1.444082in}}%
\pgfpathlineto{\pgfqpoint{4.472000in}{1.308676in}}%
\pgfpathlineto{\pgfqpoint{4.534000in}{1.147527in}}%
\pgfpathlineto{\pgfqpoint{4.565000in}{1.061984in}}%
\pgfpathlineto{\pgfqpoint{4.596000in}{1.342557in}}%
\pgfpathlineto{\pgfqpoint{4.627000in}{1.361226in}}%
\pgfpathlineto{\pgfqpoint{4.658000in}{1.132211in}}%
\pgfpathlineto{\pgfqpoint{4.689000in}{1.291251in}}%
\pgfpathlineto{\pgfqpoint{4.720000in}{0.951966in}}%
\pgfpathlineto{\pgfqpoint{4.751000in}{1.137488in}}%
\pgfpathlineto{\pgfqpoint{4.782000in}{1.167291in}}%
\pgfpathlineto{\pgfqpoint{4.813000in}{1.007608in}}%
\pgfpathlineto{\pgfqpoint{4.844000in}{1.082917in}}%
\pgfpathlineto{\pgfqpoint{4.875000in}{0.790972in}}%
\pgfpathlineto{\pgfqpoint{4.906000in}{1.010384in}}%
\pgfpathlineto{\pgfqpoint{4.937000in}{1.043517in}}%
\pgfpathlineto{\pgfqpoint{4.968000in}{0.915696in}}%
\pgfpathlineto{\pgfqpoint{4.999000in}{1.367348in}}%
\pgfpathlineto{\pgfqpoint{5.030000in}{1.217998in}}%
\pgfpathlineto{\pgfqpoint{5.061000in}{1.138635in}}%
\pgfpathlineto{\pgfqpoint{5.092000in}{1.241721in}}%
\pgfpathlineto{\pgfqpoint{5.123000in}{0.959686in}}%
\pgfpathlineto{\pgfqpoint{5.154000in}{1.071188in}}%
\pgfpathlineto{\pgfqpoint{5.185000in}{1.006173in}}%
\pgfpathlineto{\pgfqpoint{5.216000in}{0.910433in}}%
\pgfpathlineto{\pgfqpoint{5.247000in}{1.133809in}}%
\pgfpathlineto{\pgfqpoint{5.278000in}{1.162820in}}%
\pgfpathlineto{\pgfqpoint{5.309000in}{1.258142in}}%
\pgfpathlineto{\pgfqpoint{5.371000in}{1.118255in}}%
\pgfpathlineto{\pgfqpoint{5.402000in}{1.213248in}}%
\pgfpathlineto{\pgfqpoint{5.433000in}{1.243709in}}%
\pgfpathlineto{\pgfqpoint{5.464000in}{1.067614in}}%
\pgfpathlineto{\pgfqpoint{5.495000in}{1.030143in}}%
\pgfpathlineto{\pgfqpoint{5.526000in}{1.069084in}}%
\pgfpathlineto{\pgfqpoint{5.557000in}{1.110206in}}%
\pgfpathlineto{\pgfqpoint{5.588000in}{1.020416in}}%
\pgfpathlineto{\pgfqpoint{5.619000in}{0.896119in}}%
\pgfpathlineto{\pgfqpoint{5.650000in}{1.079030in}}%
\pgfpathlineto{\pgfqpoint{5.681000in}{1.084256in}}%
\pgfpathlineto{\pgfqpoint{5.712000in}{1.115058in}}%
\pgfpathlineto{\pgfqpoint{5.743000in}{1.170423in}}%
\pgfpathlineto{\pgfqpoint{5.774000in}{1.328240in}}%
\pgfpathlineto{\pgfqpoint{5.805000in}{1.132625in}}%
\pgfpathlineto{\pgfqpoint{5.836000in}{1.051242in}}%
\pgfpathlineto{\pgfqpoint{5.867000in}{1.086527in}}%
\pgfpathlineto{\pgfqpoint{5.898000in}{1.290298in}}%
\pgfpathlineto{\pgfqpoint{5.929000in}{1.277211in}}%
\pgfpathlineto{\pgfqpoint{5.960000in}{1.076079in}}%
\pgfpathlineto{\pgfqpoint{5.991000in}{1.035199in}}%
\pgfpathlineto{\pgfqpoint{6.022000in}{1.106868in}}%
\pgfpathlineto{\pgfqpoint{6.053000in}{1.008838in}}%
\pgfpathlineto{\pgfqpoint{6.084000in}{1.039905in}}%
\pgfpathlineto{\pgfqpoint{6.115000in}{1.120884in}}%
\pgfpathlineto{\pgfqpoint{6.146000in}{1.167014in}}%
\pgfpathlineto{\pgfqpoint{6.177000in}{1.099064in}}%
\pgfpathlineto{\pgfqpoint{6.208000in}{1.219160in}}%
\pgfpathlineto{\pgfqpoint{6.239000in}{1.030414in}}%
\pgfpathlineto{\pgfqpoint{6.270000in}{1.253325in}}%
\pgfpathlineto{\pgfqpoint{6.301000in}{0.980623in}}%
\pgfpathlineto{\pgfqpoint{6.332000in}{1.059867in}}%
\pgfpathlineto{\pgfqpoint{6.363000in}{1.084611in}}%
\pgfpathlineto{\pgfqpoint{6.394000in}{1.076653in}}%
\pgfpathlineto{\pgfqpoint{6.425000in}{1.336888in}}%
\pgfpathlineto{\pgfqpoint{6.456000in}{1.165206in}}%
\pgfpathlineto{\pgfqpoint{6.487000in}{1.270963in}}%
\pgfpathlineto{\pgfqpoint{6.518000in}{1.209477in}}%
\pgfpathlineto{\pgfqpoint{6.549000in}{1.219259in}}%
\pgfpathlineto{\pgfqpoint{6.580000in}{1.158620in}}%
\pgfpathlineto{\pgfqpoint{6.611000in}{1.040458in}}%
\pgfpathlineto{\pgfqpoint{6.642000in}{1.223128in}}%
\pgfpathlineto{\pgfqpoint{6.673000in}{1.099591in}}%
\pgfpathlineto{\pgfqpoint{6.704000in}{0.938364in}}%
\pgfpathlineto{\pgfqpoint{6.735000in}{1.142117in}}%
\pgfpathlineto{\pgfqpoint{6.766000in}{1.261119in}}%
\pgfpathlineto{\pgfqpoint{6.797000in}{1.222967in}}%
\pgfpathlineto{\pgfqpoint{6.828000in}{1.210582in}}%
\pgfpathlineto{\pgfqpoint{6.859000in}{0.926154in}}%
\pgfpathlineto{\pgfqpoint{6.890000in}{1.159474in}}%
\pgfpathlineto{\pgfqpoint{6.921000in}{0.851001in}}%
\pgfpathlineto{\pgfqpoint{6.952000in}{1.159442in}}%
\pgfpathlineto{\pgfqpoint{6.983000in}{0.993737in}}%
\pgfpathlineto{\pgfqpoint{7.014000in}{1.132429in}}%
\pgfpathlineto{\pgfqpoint{7.045000in}{1.167750in}}%
\pgfpathlineto{\pgfqpoint{7.076000in}{1.054561in}}%
\pgfpathlineto{\pgfqpoint{7.107000in}{1.039172in}}%
\pgfpathlineto{\pgfqpoint{7.138000in}{1.226791in}}%
\pgfpathlineto{\pgfqpoint{7.169000in}{1.370284in}}%
\pgfpathlineto{\pgfqpoint{7.200000in}{1.225372in}}%
\pgfpathlineto{\pgfqpoint{7.210000in}{1.140349in}}%
\pgfpathlineto{\pgfqpoint{7.210000in}{1.140349in}}%
\pgfusepath{stroke}%
\end{pgfscope}%
\begin{pgfscope}%
\pgfpathrectangle{\pgfqpoint{1.000000in}{0.720000in}}{\pgfqpoint{6.200000in}{4.620000in}}%
\pgfusepath{clip}%
\pgfsetbuttcap%
\pgfsetroundjoin%
\pgfsetlinewidth{2.007500pt}%
\definecolor{currentstroke}{rgb}{0.000000,0.500000,0.000000}%
\pgfsetstrokecolor{currentstroke}%
\pgfsetdash{}{0pt}%
\pgfpathmoveto{\pgfqpoint{0.990000in}{1.760228in}}%
\pgfpathlineto{\pgfqpoint{7.210000in}{1.760228in}}%
\pgfusepath{stroke}%
\end{pgfscope}%
\begin{pgfscope}%
\pgfsetrectcap%
\pgfsetmiterjoin%
\pgfsetlinewidth{0.803000pt}%
\definecolor{currentstroke}{rgb}{0.000000,0.000000,0.000000}%
\pgfsetstrokecolor{currentstroke}%
\pgfsetdash{}{0pt}%
\pgfpathmoveto{\pgfqpoint{1.000000in}{0.720000in}}%
\pgfpathlineto{\pgfqpoint{1.000000in}{5.340000in}}%
\pgfusepath{stroke}%
\end{pgfscope}%
\begin{pgfscope}%
\pgfsetrectcap%
\pgfsetmiterjoin%
\pgfsetlinewidth{0.803000pt}%
\definecolor{currentstroke}{rgb}{0.000000,0.000000,0.000000}%
\pgfsetstrokecolor{currentstroke}%
\pgfsetdash{}{0pt}%
\pgfpathmoveto{\pgfqpoint{7.200000in}{0.720000in}}%
\pgfpathlineto{\pgfqpoint{7.200000in}{5.340000in}}%
\pgfusepath{stroke}%
\end{pgfscope}%
\begin{pgfscope}%
\pgfsetrectcap%
\pgfsetmiterjoin%
\pgfsetlinewidth{0.803000pt}%
\definecolor{currentstroke}{rgb}{0.000000,0.000000,0.000000}%
\pgfsetstrokecolor{currentstroke}%
\pgfsetdash{}{0pt}%
\pgfpathmoveto{\pgfqpoint{1.000000in}{0.720000in}}%
\pgfpathlineto{\pgfqpoint{7.200000in}{0.720000in}}%
\pgfusepath{stroke}%
\end{pgfscope}%
\begin{pgfscope}%
\pgfsetrectcap%
\pgfsetmiterjoin%
\pgfsetlinewidth{0.803000pt}%
\definecolor{currentstroke}{rgb}{0.000000,0.000000,0.000000}%
\pgfsetstrokecolor{currentstroke}%
\pgfsetdash{}{0pt}%
\pgfpathmoveto{\pgfqpoint{1.000000in}{5.340000in}}%
\pgfpathlineto{\pgfqpoint{7.200000in}{5.340000in}}%
\pgfusepath{stroke}%
\end{pgfscope}%
\begin{pgfscope}%
\pgfsetbuttcap%
\pgfsetroundjoin%
\definecolor{currentfill}{rgb}{0.000000,0.000000,0.000000}%
\pgfsetfillcolor{currentfill}%
\pgfsetlinewidth{0.803000pt}%
\definecolor{currentstroke}{rgb}{0.000000,0.000000,0.000000}%
\pgfsetstrokecolor{currentstroke}%
\pgfsetdash{}{0pt}%
\pgfsys@defobject{currentmarker}{\pgfqpoint{0.000000in}{0.000000in}}{\pgfqpoint{0.048611in}{0.000000in}}{%
\pgfpathmoveto{\pgfqpoint{0.000000in}{0.000000in}}%
\pgfpathlineto{\pgfqpoint{0.048611in}{0.000000in}}%
\pgfusepath{stroke,fill}%
}%
\begin{pgfscope}%
\pgfsys@transformshift{7.200000in}{1.145834in}%
\pgfsys@useobject{currentmarker}{}%
\end{pgfscope}%
\end{pgfscope}%
\begin{pgfscope}%
\definecolor{textcolor}{rgb}{0.000000,0.000000,0.000000}%
\pgfsetstrokecolor{textcolor}%
\pgfsetfillcolor{textcolor}%
\pgftext[x=7.297222in, y=1.045815in, left, base]{\color{textcolor}\sffamily\fontsize{20.000000}{24.000000}\selectfont 0.0}%
\end{pgfscope}%
\begin{pgfscope}%
\pgfsetbuttcap%
\pgfsetroundjoin%
\definecolor{currentfill}{rgb}{0.000000,0.000000,0.000000}%
\pgfsetfillcolor{currentfill}%
\pgfsetlinewidth{0.803000pt}%
\definecolor{currentstroke}{rgb}{0.000000,0.000000,0.000000}%
\pgfsetstrokecolor{currentstroke}%
\pgfsetdash{}{0pt}%
\pgfsys@defobject{currentmarker}{\pgfqpoint{0.000000in}{0.000000in}}{\pgfqpoint{0.048611in}{0.000000in}}{%
\pgfpathmoveto{\pgfqpoint{0.000000in}{0.000000in}}%
\pgfpathlineto{\pgfqpoint{0.048611in}{0.000000in}}%
\pgfusepath{stroke,fill}%
}%
\begin{pgfscope}%
\pgfsys@transformshift{7.200000in}{1.762623in}%
\pgfsys@useobject{currentmarker}{}%
\end{pgfscope}%
\end{pgfscope}%
\begin{pgfscope}%
\definecolor{textcolor}{rgb}{0.000000,0.000000,0.000000}%
\pgfsetstrokecolor{textcolor}%
\pgfsetfillcolor{textcolor}%
\pgftext[x=7.297222in, y=1.662604in, left, base]{\color{textcolor}\sffamily\fontsize{20.000000}{24.000000}\selectfont 0.2}%
\end{pgfscope}%
\begin{pgfscope}%
\pgfsetbuttcap%
\pgfsetroundjoin%
\definecolor{currentfill}{rgb}{0.000000,0.000000,0.000000}%
\pgfsetfillcolor{currentfill}%
\pgfsetlinewidth{0.803000pt}%
\definecolor{currentstroke}{rgb}{0.000000,0.000000,0.000000}%
\pgfsetstrokecolor{currentstroke}%
\pgfsetdash{}{0pt}%
\pgfsys@defobject{currentmarker}{\pgfqpoint{0.000000in}{0.000000in}}{\pgfqpoint{0.048611in}{0.000000in}}{%
\pgfpathmoveto{\pgfqpoint{0.000000in}{0.000000in}}%
\pgfpathlineto{\pgfqpoint{0.048611in}{0.000000in}}%
\pgfusepath{stroke,fill}%
}%
\begin{pgfscope}%
\pgfsys@transformshift{7.200000in}{2.379413in}%
\pgfsys@useobject{currentmarker}{}%
\end{pgfscope}%
\end{pgfscope}%
\begin{pgfscope}%
\definecolor{textcolor}{rgb}{0.000000,0.000000,0.000000}%
\pgfsetstrokecolor{textcolor}%
\pgfsetfillcolor{textcolor}%
\pgftext[x=7.297222in, y=2.279393in, left, base]{\color{textcolor}\sffamily\fontsize{20.000000}{24.000000}\selectfont 0.5}%
\end{pgfscope}%
\begin{pgfscope}%
\pgfsetbuttcap%
\pgfsetroundjoin%
\definecolor{currentfill}{rgb}{0.000000,0.000000,0.000000}%
\pgfsetfillcolor{currentfill}%
\pgfsetlinewidth{0.803000pt}%
\definecolor{currentstroke}{rgb}{0.000000,0.000000,0.000000}%
\pgfsetstrokecolor{currentstroke}%
\pgfsetdash{}{0pt}%
\pgfsys@defobject{currentmarker}{\pgfqpoint{0.000000in}{0.000000in}}{\pgfqpoint{0.048611in}{0.000000in}}{%
\pgfpathmoveto{\pgfqpoint{0.000000in}{0.000000in}}%
\pgfpathlineto{\pgfqpoint{0.048611in}{0.000000in}}%
\pgfusepath{stroke,fill}%
}%
\begin{pgfscope}%
\pgfsys@transformshift{7.200000in}{2.996202in}%
\pgfsys@useobject{currentmarker}{}%
\end{pgfscope}%
\end{pgfscope}%
\begin{pgfscope}%
\definecolor{textcolor}{rgb}{0.000000,0.000000,0.000000}%
\pgfsetstrokecolor{textcolor}%
\pgfsetfillcolor{textcolor}%
\pgftext[x=7.297222in, y=2.896182in, left, base]{\color{textcolor}\sffamily\fontsize{20.000000}{24.000000}\selectfont 0.8}%
\end{pgfscope}%
\begin{pgfscope}%
\pgfsetbuttcap%
\pgfsetroundjoin%
\definecolor{currentfill}{rgb}{0.000000,0.000000,0.000000}%
\pgfsetfillcolor{currentfill}%
\pgfsetlinewidth{0.803000pt}%
\definecolor{currentstroke}{rgb}{0.000000,0.000000,0.000000}%
\pgfsetstrokecolor{currentstroke}%
\pgfsetdash{}{0pt}%
\pgfsys@defobject{currentmarker}{\pgfqpoint{0.000000in}{0.000000in}}{\pgfqpoint{0.048611in}{0.000000in}}{%
\pgfpathmoveto{\pgfqpoint{0.000000in}{0.000000in}}%
\pgfpathlineto{\pgfqpoint{0.048611in}{0.000000in}}%
\pgfusepath{stroke,fill}%
}%
\begin{pgfscope}%
\pgfsys@transformshift{7.200000in}{3.612991in}%
\pgfsys@useobject{currentmarker}{}%
\end{pgfscope}%
\end{pgfscope}%
\begin{pgfscope}%
\definecolor{textcolor}{rgb}{0.000000,0.000000,0.000000}%
\pgfsetstrokecolor{textcolor}%
\pgfsetfillcolor{textcolor}%
\pgftext[x=7.297222in, y=3.512971in, left, base]{\color{textcolor}\sffamily\fontsize{20.000000}{24.000000}\selectfont 1.0}%
\end{pgfscope}%
\begin{pgfscope}%
\pgfsetbuttcap%
\pgfsetroundjoin%
\definecolor{currentfill}{rgb}{0.000000,0.000000,0.000000}%
\pgfsetfillcolor{currentfill}%
\pgfsetlinewidth{0.803000pt}%
\definecolor{currentstroke}{rgb}{0.000000,0.000000,0.000000}%
\pgfsetstrokecolor{currentstroke}%
\pgfsetdash{}{0pt}%
\pgfsys@defobject{currentmarker}{\pgfqpoint{0.000000in}{0.000000in}}{\pgfqpoint{0.048611in}{0.000000in}}{%
\pgfpathmoveto{\pgfqpoint{0.000000in}{0.000000in}}%
\pgfpathlineto{\pgfqpoint{0.048611in}{0.000000in}}%
\pgfusepath{stroke,fill}%
}%
\begin{pgfscope}%
\pgfsys@transformshift{7.200000in}{4.229780in}%
\pgfsys@useobject{currentmarker}{}%
\end{pgfscope}%
\end{pgfscope}%
\begin{pgfscope}%
\definecolor{textcolor}{rgb}{0.000000,0.000000,0.000000}%
\pgfsetstrokecolor{textcolor}%
\pgfsetfillcolor{textcolor}%
\pgftext[x=7.297222in, y=4.129760in, left, base]{\color{textcolor}\sffamily\fontsize{20.000000}{24.000000}\selectfont 1.2}%
\end{pgfscope}%
\begin{pgfscope}%
\pgfsetbuttcap%
\pgfsetroundjoin%
\definecolor{currentfill}{rgb}{0.000000,0.000000,0.000000}%
\pgfsetfillcolor{currentfill}%
\pgfsetlinewidth{0.803000pt}%
\definecolor{currentstroke}{rgb}{0.000000,0.000000,0.000000}%
\pgfsetstrokecolor{currentstroke}%
\pgfsetdash{}{0pt}%
\pgfsys@defobject{currentmarker}{\pgfqpoint{0.000000in}{0.000000in}}{\pgfqpoint{0.048611in}{0.000000in}}{%
\pgfpathmoveto{\pgfqpoint{0.000000in}{0.000000in}}%
\pgfpathlineto{\pgfqpoint{0.048611in}{0.000000in}}%
\pgfusepath{stroke,fill}%
}%
\begin{pgfscope}%
\pgfsys@transformshift{7.200000in}{4.846569in}%
\pgfsys@useobject{currentmarker}{}%
\end{pgfscope}%
\end{pgfscope}%
\begin{pgfscope}%
\definecolor{textcolor}{rgb}{0.000000,0.000000,0.000000}%
\pgfsetstrokecolor{textcolor}%
\pgfsetfillcolor{textcolor}%
\pgftext[x=7.297222in, y=4.746550in, left, base]{\color{textcolor}\sffamily\fontsize{20.000000}{24.000000}\selectfont 1.5}%
\end{pgfscope}%
\begin{pgfscope}%
\definecolor{textcolor}{rgb}{0.000000,0.000000,0.000000}%
\pgfsetstrokecolor{textcolor}%
\pgfsetfillcolor{textcolor}%
\pgftext[x=7.698906in,y=3.030000in,,top,rotate=90.000000]{\color{textcolor}\sffamily\fontsize{20.000000}{24.000000}\selectfont \(\displaystyle \mathrm{Charge}\)}%
\end{pgfscope}%
\begin{pgfscope}%
\pgfpathrectangle{\pgfqpoint{1.000000in}{0.720000in}}{\pgfqpoint{6.200000in}{4.620000in}}%
\pgfusepath{clip}%
\pgfsetbuttcap%
\pgfsetroundjoin%
\pgfsetlinewidth{0.501875pt}%
\definecolor{currentstroke}{rgb}{1.000000,0.000000,0.000000}%
\pgfsetstrokecolor{currentstroke}%
\pgfsetdash{}{0pt}%
\pgfpathmoveto{\pgfqpoint{1.899000in}{1.145834in}}%
\pgfpathlineto{\pgfqpoint{1.899000in}{2.266516in}}%
\pgfusepath{stroke}%
\end{pgfscope}%
\begin{pgfscope}%
\pgfpathrectangle{\pgfqpoint{1.000000in}{0.720000in}}{\pgfqpoint{6.200000in}{4.620000in}}%
\pgfusepath{clip}%
\pgfsetbuttcap%
\pgfsetroundjoin%
\pgfsetlinewidth{0.501875pt}%
\definecolor{currentstroke}{rgb}{1.000000,0.000000,0.000000}%
\pgfsetstrokecolor{currentstroke}%
\pgfsetdash{}{0pt}%
\pgfpathmoveto{\pgfqpoint{1.961000in}{1.145834in}}%
\pgfpathlineto{\pgfqpoint{1.961000in}{4.051816in}}%
\pgfusepath{stroke}%
\end{pgfscope}%
\begin{pgfscope}%
\pgfpathrectangle{\pgfqpoint{1.000000in}{0.720000in}}{\pgfqpoint{6.200000in}{4.620000in}}%
\pgfusepath{clip}%
\pgfsetbuttcap%
\pgfsetroundjoin%
\pgfsetlinewidth{0.501875pt}%
\definecolor{currentstroke}{rgb}{1.000000,0.000000,0.000000}%
\pgfsetstrokecolor{currentstroke}%
\pgfsetdash{}{0pt}%
\pgfpathmoveto{\pgfqpoint{1.992000in}{1.145834in}}%
\pgfpathlineto{\pgfqpoint{1.992000in}{1.681152in}}%
\pgfusepath{stroke}%
\end{pgfscope}%
\begin{pgfscope}%
\pgfpathrectangle{\pgfqpoint{1.000000in}{0.720000in}}{\pgfqpoint{6.200000in}{4.620000in}}%
\pgfusepath{clip}%
\pgfsetbuttcap%
\pgfsetroundjoin%
\pgfsetlinewidth{0.501875pt}%
\definecolor{currentstroke}{rgb}{1.000000,0.000000,0.000000}%
\pgfsetstrokecolor{currentstroke}%
\pgfsetdash{}{0pt}%
\pgfpathmoveto{\pgfqpoint{2.457000in}{1.145834in}}%
\pgfpathlineto{\pgfqpoint{2.457000in}{2.562393in}}%
\pgfusepath{stroke}%
\end{pgfscope}%
\begin{pgfscope}%
\pgfpathrectangle{\pgfqpoint{1.000000in}{0.720000in}}{\pgfqpoint{6.200000in}{4.620000in}}%
\pgfusepath{clip}%
\pgfsetbuttcap%
\pgfsetroundjoin%
\pgfsetlinewidth{0.501875pt}%
\definecolor{currentstroke}{rgb}{1.000000,0.000000,0.000000}%
\pgfsetstrokecolor{currentstroke}%
\pgfsetdash{}{0pt}%
\pgfpathmoveto{\pgfqpoint{2.488000in}{1.145834in}}%
\pgfpathlineto{\pgfqpoint{2.488000in}{3.248536in}}%
\pgfusepath{stroke}%
\end{pgfscope}%
\begin{pgfscope}%
\pgfpathrectangle{\pgfqpoint{1.000000in}{0.720000in}}{\pgfqpoint{6.200000in}{4.620000in}}%
\pgfusepath{clip}%
\pgfsetbuttcap%
\pgfsetroundjoin%
\pgfsetlinewidth{0.501875pt}%
\definecolor{currentstroke}{rgb}{1.000000,0.000000,0.000000}%
\pgfsetstrokecolor{currentstroke}%
\pgfsetdash{}{0pt}%
\pgfpathmoveto{\pgfqpoint{2.829000in}{1.145834in}}%
\pgfpathlineto{\pgfqpoint{2.829000in}{2.394289in}}%
\pgfusepath{stroke}%
\end{pgfscope}%
\begin{pgfscope}%
\pgfpathrectangle{\pgfqpoint{1.000000in}{0.720000in}}{\pgfqpoint{6.200000in}{4.620000in}}%
\pgfusepath{clip}%
\pgfsetbuttcap%
\pgfsetroundjoin%
\pgfsetlinewidth{0.501875pt}%
\definecolor{currentstroke}{rgb}{1.000000,0.000000,0.000000}%
\pgfsetstrokecolor{currentstroke}%
\pgfsetdash{}{0pt}%
\pgfpathmoveto{\pgfqpoint{2.860000in}{1.145834in}}%
\pgfpathlineto{\pgfqpoint{2.860000in}{2.337619in}}%
\pgfusepath{stroke}%
\end{pgfscope}%
\begin{pgfscope}%
\pgfpathrectangle{\pgfqpoint{1.000000in}{0.720000in}}{\pgfqpoint{6.200000in}{4.620000in}}%
\pgfusepath{clip}%
\pgfsetbuttcap%
\pgfsetroundjoin%
\pgfsetlinewidth{0.501875pt}%
\definecolor{currentstroke}{rgb}{1.000000,0.000000,0.000000}%
\pgfsetstrokecolor{currentstroke}%
\pgfsetdash{}{0pt}%
\pgfpathmoveto{\pgfqpoint{3.697000in}{1.145834in}}%
\pgfpathlineto{\pgfqpoint{3.697000in}{2.265162in}}%
\pgfusepath{stroke}%
\end{pgfscope}%
\begin{pgfscope}%
\pgfpathrectangle{\pgfqpoint{1.000000in}{0.720000in}}{\pgfqpoint{6.200000in}{4.620000in}}%
\pgfusepath{clip}%
\pgfsetbuttcap%
\pgfsetroundjoin%
\pgfsetlinewidth{0.501875pt}%
\definecolor{currentstroke}{rgb}{1.000000,0.000000,0.000000}%
\pgfsetstrokecolor{currentstroke}%
\pgfsetdash{}{0pt}%
\pgfpathmoveto{\pgfqpoint{3.728000in}{1.145834in}}%
\pgfpathlineto{\pgfqpoint{3.728000in}{2.931684in}}%
\pgfusepath{stroke}%
\end{pgfscope}%
\begin{pgfscope}%
\pgfsetrectcap%
\pgfsetmiterjoin%
\pgfsetlinewidth{0.803000pt}%
\definecolor{currentstroke}{rgb}{0.000000,0.000000,0.000000}%
\pgfsetstrokecolor{currentstroke}%
\pgfsetdash{}{0pt}%
\pgfpathmoveto{\pgfqpoint{1.000000in}{0.720000in}}%
\pgfpathlineto{\pgfqpoint{1.000000in}{5.340000in}}%
\pgfusepath{stroke}%
\end{pgfscope}%
\begin{pgfscope}%
\pgfsetrectcap%
\pgfsetmiterjoin%
\pgfsetlinewidth{0.803000pt}%
\definecolor{currentstroke}{rgb}{0.000000,0.000000,0.000000}%
\pgfsetstrokecolor{currentstroke}%
\pgfsetdash{}{0pt}%
\pgfpathmoveto{\pgfqpoint{7.200000in}{0.720000in}}%
\pgfpathlineto{\pgfqpoint{7.200000in}{5.340000in}}%
\pgfusepath{stroke}%
\end{pgfscope}%
\begin{pgfscope}%
\pgfsetrectcap%
\pgfsetmiterjoin%
\pgfsetlinewidth{0.803000pt}%
\definecolor{currentstroke}{rgb}{0.000000,0.000000,0.000000}%
\pgfsetstrokecolor{currentstroke}%
\pgfsetdash{}{0pt}%
\pgfpathmoveto{\pgfqpoint{1.000000in}{0.720000in}}%
\pgfpathlineto{\pgfqpoint{7.200000in}{0.720000in}}%
\pgfusepath{stroke}%
\end{pgfscope}%
\begin{pgfscope}%
\pgfsetrectcap%
\pgfsetmiterjoin%
\pgfsetlinewidth{0.803000pt}%
\definecolor{currentstroke}{rgb}{0.000000,0.000000,0.000000}%
\pgfsetstrokecolor{currentstroke}%
\pgfsetdash{}{0pt}%
\pgfpathmoveto{\pgfqpoint{1.000000in}{5.340000in}}%
\pgfpathlineto{\pgfqpoint{7.200000in}{5.340000in}}%
\pgfusepath{stroke}%
\end{pgfscope}%
\begin{pgfscope}%
\pgfsetbuttcap%
\pgfsetmiterjoin%
\definecolor{currentfill}{rgb}{1.000000,1.000000,1.000000}%
\pgfsetfillcolor{currentfill}%
\pgfsetfillopacity{0.800000}%
\pgfsetlinewidth{1.003750pt}%
\definecolor{currentstroke}{rgb}{0.800000,0.800000,0.800000}%
\pgfsetstrokecolor{currentstroke}%
\pgfsetstrokeopacity{0.800000}%
\pgfsetdash{}{0pt}%
\pgfpathmoveto{\pgfqpoint{4.976872in}{3.932908in}}%
\pgfpathlineto{\pgfqpoint{7.005556in}{3.932908in}}%
\pgfpathquadraticcurveto{\pgfqpoint{7.061111in}{3.932908in}}{\pgfqpoint{7.061111in}{3.988464in}}%
\pgfpathlineto{\pgfqpoint{7.061111in}{5.145556in}}%
\pgfpathquadraticcurveto{\pgfqpoint{7.061111in}{5.201111in}}{\pgfqpoint{7.005556in}{5.201111in}}%
\pgfpathlineto{\pgfqpoint{4.976872in}{5.201111in}}%
\pgfpathquadraticcurveto{\pgfqpoint{4.921317in}{5.201111in}}{\pgfqpoint{4.921317in}{5.145556in}}%
\pgfpathlineto{\pgfqpoint{4.921317in}{3.988464in}}%
\pgfpathquadraticcurveto{\pgfqpoint{4.921317in}{3.932908in}}{\pgfqpoint{4.976872in}{3.932908in}}%
\pgfpathlineto{\pgfqpoint{4.976872in}{3.932908in}}%
\pgfpathclose%
\pgfusepath{stroke,fill}%
\end{pgfscope}%
\begin{pgfscope}%
\pgfsetrectcap%
\pgfsetroundjoin%
\pgfsetlinewidth{2.007500pt}%
\definecolor{currentstroke}{rgb}{0.121569,0.466667,0.705882}%
\pgfsetstrokecolor{currentstroke}%
\pgfsetdash{}{0pt}%
\pgfpathmoveto{\pgfqpoint{5.032428in}{4.987184in}}%
\pgfpathlineto{\pgfqpoint{5.310206in}{4.987184in}}%
\pgfpathlineto{\pgfqpoint{5.587983in}{4.987184in}}%
\pgfusepath{stroke}%
\end{pgfscope}%
\begin{pgfscope}%
\definecolor{textcolor}{rgb}{0.000000,0.000000,0.000000}%
\pgfsetstrokecolor{textcolor}%
\pgfsetfillcolor{textcolor}%
\pgftext[x=5.810206in,y=4.889962in,left,base]{\color{textcolor}\sffamily\fontsize{20.000000}{24.000000}\selectfont Waveform}%
\end{pgfscope}%
\begin{pgfscope}%
\pgfsetbuttcap%
\pgfsetroundjoin%
\pgfsetlinewidth{2.007500pt}%
\definecolor{currentstroke}{rgb}{0.000000,0.500000,0.000000}%
\pgfsetstrokecolor{currentstroke}%
\pgfsetdash{}{0pt}%
\pgfpathmoveto{\pgfqpoint{5.032428in}{4.592227in}}%
\pgfpathlineto{\pgfqpoint{5.587983in}{4.592227in}}%
\pgfusepath{stroke}%
\end{pgfscope}%
\begin{pgfscope}%
\definecolor{textcolor}{rgb}{0.000000,0.000000,0.000000}%
\pgfsetstrokecolor{textcolor}%
\pgfsetfillcolor{textcolor}%
\pgftext[x=5.810206in,y=4.495005in,left,base]{\color{textcolor}\sffamily\fontsize{20.000000}{24.000000}\selectfont Threshold}%
\end{pgfscope}%
\begin{pgfscope}%
\pgfsetbuttcap%
\pgfsetroundjoin%
\pgfsetlinewidth{0.501875pt}%
\definecolor{currentstroke}{rgb}{1.000000,0.000000,0.000000}%
\pgfsetstrokecolor{currentstroke}%
\pgfsetdash{}{0pt}%
\pgfpathmoveto{\pgfqpoint{5.032428in}{4.197271in}}%
\pgfpathlineto{\pgfqpoint{5.587983in}{4.197271in}}%
\pgfusepath{stroke}%
\end{pgfscope}%
\begin{pgfscope}%
\definecolor{textcolor}{rgb}{0.000000,0.000000,0.000000}%
\pgfsetstrokecolor{textcolor}%
\pgfsetfillcolor{textcolor}%
\pgftext[x=5.810206in,y=4.100048in,left,base]{\color{textcolor}\sffamily\fontsize{20.000000}{24.000000}\selectfont Charge}%
\end{pgfscope}%
\end{pgfpicture}%
\makeatother%
\endgroup%
}
    \caption{\label{fig:lucy} A Richardson-Lucy direct demodulation example:\\ $\hat{t}_\mathrm{KL} - t_0=\SI{2.77}{ns}$, $\mathrm{RSS}=\SI{10.0}{mV^2}$, $D_\mathrm{w}=\SI{0.60}{ns}$.}
  \end{subfigure}
  \caption{\label{fig:deconv}Demonstrations of deconvolution methods on a waveform sampled from the same setup as figure~\ref{fig:method}. Richardson-Lucy direct demodulation in~\subref{fig:lucy} imposes positive charges in iterations and obtains better results than Fourier deconvolution in~\subref{fig:fd}.}
\end{figure}

\subsubsection{Richardson-Lucy direct demodulation}
\label{sec:lucyddm}

\textit{Richardson-Lucy direct demodulation}~(LucyDDM)~\cite{lucy_iterative_1974} with a non-linear iteration to calculate deconvolution has a wide application in astronomy~\cite{li_richardson-lucy_2019} and image processing. We view $V_{\mathrm{PE}*}(t-s)$ as a conditional probability distribution $p(t|s)$ where $t$ denotes PMT amplified electron time, and $s$ represents the given PE time. By the Bayesian rule,
\begin{equation}
  \label{eq:lucy}
  \tilde{\phi}_*(s) V_{\mathrm{PE}*}(t-s) = \tilde{\phi}_*(s)p(t|s) = p(t,s) = \tilde{w}_*(t)p(s|t),
\end{equation}
where $p(t, s)$ is the joint distribution of amplified electron $t$ and PE time $s$, and $\tilde{w}$ is the smoothed $w$.  Cancel out the normalization factors,
\begin{equation}
  \label{eq:ptt}
  p(s|t) = \frac{\tilde{\phi}_*(s) V_{\mathrm{PE}*}(t-s)}{\tilde{w}_*(t)} = \frac{\tilde{\phi}(s) V_{\mathrm{PE}}(t-s)}{\int\tilde{\phi}(s') V_{\mathrm{PE}}(t-s')\mathrm{d}s'}.
\end{equation}
Then a recurrence relation $\phi_*$ is,
\begin{equation}
  \label{eq:iter}
  \begin{aligned}
    \tilde{\phi}_*(s) & = \int p(s|t) \tilde{w}_*(t)\mathrm{d}t = \int \frac{\tilde{\phi}(s) V_{\mathrm{PE}}(t-s)}{\int\tilde{\phi}(s') V_{\mathrm{PE}}(t-s')\mathrm{d}s'} \tilde{w}_*(t) \mathrm{d}t \\
    \implies \hat{\phi}^{n+1}(s) & = \int \frac{\hat{\phi}^n(s) V_{\mathrm{PE}*}(t-s)}{\int\hat{\phi}^n(s') V_{\mathrm{PE}}(t-s')\mathrm{d}s'} \tilde{w}(t) \mathrm{d}t,
  \end{aligned}
\end{equation}
where only $V_{\mathrm{PE}*}$ in the numerator is normalized, and superscript $n$ denotes the iteration step.
Like Fourier deconvolution in eq.~\eqref{eq:fdconv2}, we threshold and scale the converged $\hat{\phi}^\infty$ to get $\hat{\phi}$.  As shown in figure~\ref{fig:lucy}, the positive constraint of $\hat{\phi}$ makes LucyDDM more resilient to noise.

The remaining noise in the smoothed $\tilde{w}$ crucially influences deconvolution.  A probabilistic method will correctly model the noise term $\epsilon$, as we shall see in section \ref{sec:regression}.

\subsection{Convolutional neural network}
\label{sec:cnn}
Convolutional neural networks~(CNN) made breakthroughs in various fields like computer vision~\cite{he_deep_2016} and natural language processing~\cite{vaswani_attention_2017}.  As an efficient composition of weighted additions and non-linear functions, neural networks outperform many traditional algorithms.  The success of CNN induces many ongoing efforts to apply it to waveform analysis~\cite{students22}.  It is thus interesting and insightful to make a comparison of CNN with the remaining traditional methods.

The input discretized waveform $\bm{w}$ is an 1 dimensional vector.  However, $\hat{q}_i$ and $\hat{t}_i$ are two variable length ($\hat{N}_\mathrm{PE}$) vectors, which is not a well-defined output for a 1 dimensional CNN~(1D-CNN).  Instead, we replace $\hat{N}_\mathrm{PE}$ with a fixed sample size $N_\mathrm{s}$ and $\hat{t}_i$ with a fixed grid of times $t'_j$ and associating $q'_j$. For most $j$, $q'_j = 0$, meaning there is no PE on time grid $t'_j$. By stripping out $j$ where $q'_j=0$, the remaining $q'_j$, $t'_j$ are $\hat{q}_i$ and $\hat{t}_i$.  Now $q'_j$ is a 1D vector with fixed length $N_\mathrm{s}$, suitable for 1D-CNN.

We choose a shallow network structure of 5 layers to recognize patterns as shown in figure~\ref{fig:struct}, motivated by the pulse shape and universality of $V_\mathrm{PE}(t)$ for all the PEs. The input waveform vector $\bm{w}$ is convoluted by several kernels $\bm{K}^{(1)}_m$ into new vectors $v_m$:
\begin{equation}
  \bm{v}^{(1)}_m = \bm{K}^{(1)}_m \otimes \bm{w},\ m\in \{1,\ldots,M\}
  \label{eq:1DCNN-11}
\end{equation}
As 1D vectors, $\bm{K}^{(1)}_m$ share the same length called \textit{kernel size}.  $M$ is the \textit{number of channels}. As shown in figure~\ref{fig:struct}, considering the localized nature of $V_\mathrm{PE}(t)$,  we choose the kernel size to be $21$ and $M=25$.

After the above linear \textit{link} operations, a point-wise nonlinear \textit{activation} transformation, leaky rectified linear function $\mathrm{LReL}(\cdot)$\cite{leakyReLU} is used:
\begin{equation}
  \begin{aligned}
    & \bm{v'}^{(1)}_m = \mathrm{LReL}(\bm{v}^{(1)}_m) \\
    \text{where  } & \mathrm{LReL}(x) = \left\{ \begin{aligned}
      & 0.05 x & x<0 \\
      & x & x\geqslant 0 \\
    \end{aligned} \right.
  \end{aligned}
  \label{eq:1DCNN-12}
\end{equation}
The two operations form the first layer. The second layer is similar,
\begin{equation}
  \bm{v}'^{(2)}_n = \mathrm{LReL}\left(\sum_{m=1}^{M} \bm{K}^{(2)}_{nm} \otimes \bm{v'}^{(1)}_m\right),\ n\in \{1,\ldots,N\}
  \label{eq:1DCNN-2}
\end{equation}
mapping $M$-channelled $\bm{v'}^{(1)}_m$ to $N$-channelled $\bm{v'}^{(2)}_n$.

\begin{figure}[H]
  \begin{subfigure}{.4\textwidth}
    \centering
    \begin{adjustbox}{width=0.5\textwidth}
      \tikzstyle{block} = [rectangle, rounded corners, minimum width=2cm, minimum height=1cm, text centered, draw=black]
\tikzstyle{arrow} = [thick, ->, >=stealth]
\begin{tikzpicture}[node distance=2cm]
    \node (0) [block] {$1\times1029$};
    \node (1) [block, below of=0] {$25\times1029$};
    \node (2) [block, below of=1] {$20\times1029$};
    \node (3) [block, below of=2] {$15\times1029$};
    \node (4) [block, below of=3] {$10\times1029$};
    \node (5) [block, below of=4] {$1\times1029$};
    \draw [arrow] (0) -- node [midway](0to1) {} (1);
    \draw [arrow] (1) -- node [midway](1to2) {} (2);
    \draw [arrow] (2) -- node [midway](2to3) {} (3);
    \draw [arrow] (3) -- node [midway](3to4) {} (4);
    \draw [arrow] (4) -- node [midway](4to5) {} (5);
    \node (a) [block, right of=0to1, xshift=1.2cm] {kernel=$21$};
    \node (b) [block, right of=1to2, xshift=1.2cm] {kernel=$17$};
    \node (c) [block, right of=2to3, xshift=1.2cm] {kernel=$13$};
    \node (d) [block, right of=3to4, xshift=1.2cm] {kernel=$9$};
    \node (e) [block, right of=4to5, xshift=1.2cm] {kernel=$1$};
    \draw [arrow] (a) -- (0to1) {};
    \draw [arrow] (b) -- (1to2) {};
    \draw [arrow] (c) -- (2to3) {};
    \draw [arrow] (d) -- (3to4) {};
    \draw [arrow] (e) -- (4to5) {};
\end{tikzpicture}
    \end{adjustbox}
    \caption{\label{fig:struct} Structure of the neural network.}
  \end{subfigure}
  \begin{subfigure}{.5\textwidth}
    \centering
    \resizebox{\textwidth}{!}{version https://git-lfs.github.com/spec/v1
oid sha256:4c9f0405e0a17c89e29f023642ea05c51d350a8fd1adf6cfa82f2fda57f5a7a7
size 35431
}
    \caption{\label{fig:loss} Evolution of loss.}
  \end{subfigure}
  \caption{\label{fig:CNN} Training process of a CNN. A shallow network structure of 5 layers in~\subref{fig:struct} is trained to converge in Wasserstein distance as shown in~\subref{fig:loss}.  ``kernel=21'' stands for a 1-D convolutional kernel size 21. ``1029'' is the number of voltage samples in a waveform.  $1\times$ represents the number of channels in each layer.}
\end{figure}

At the bottom of figure~\ref{fig:struct}, 1D-CNN gives the desired output, a one-channelled vector $q'_j$, which determines the PE distribution $\phi'(t)$ by
\begin{equation}
  \label{eq:gd-phi}
  \phi'(t) = \sum_{j=1}^{N_\mathrm{s}}q'_j\delta(t-t'_j).
\end{equation}

The whole network is a non-linear function $\mathrm{CNN}(\cdot)$ from $\bm{w}$ to $\phi'$ with numerous free parameters $\bm{K}^{(1)}_m, \bm{K}^{(2)}_{mn}, \ldots$ which we denote as $\mathcal{K}$. We \textit{train} to fit the parameters against true $\tilde{\phi}$,
\begin{equation}
  \hat{\mathcal{K}} = \arg\underset{\mathcal{K}}{\min} D_\mathrm{w}\left[\mathrm{CNN}(\bm{w}; \mathcal{K}), \tilde{\phi}\right]
  \label{eq:CNN-train}
\end{equation}
by back-propagation. Figure~\ref{fig:loss} shows the convergence of Wasserstein distance during training. Such fitting process is an example of \textit{supervised learning}. As explained in figure~\ref{fig:l2}, $D_\mathrm{w}$ can naturally measure the time difference between two sparse $\phi'$ and $\tilde{\phi}$ in eq.~\eqref{eq:CNN-train}, making 1D-CNN not need to split a PE into smaller ones to fit waveform fluctuations.  This gives sparser results in contrast to deconvolution methods in section~\ref{sec:deconv} and direct charge fitting in section~\ref{sec:dcf}, which shall be further  discussed in section~\ref{sec:sparsity}.

In figure~\ref{fig:cnn-npe}, $D_\mathrm{w}$ is the smallest for one PE.  $D_\mathrm{w}$ stops increasing with $N_\mathrm{PE}$ at about 6 PEs, where the PE times are the most challenging to extract.  When $N_\mathrm{PE}$ is more than 6, pile-ups tend to produce a continuous waveform and the average PE time accuracy stays flat. Similar to eq.~\eqref{eq:fdconv2}, the output of CNN should be scaled by $\hat{\alpha}$ to get $\hat{\phi}$. Such small $D_\mathrm{w}$ in figure~\ref{fig:cnn-npe} provides a precise matching of waveforms horizontally to guarantee effective $\hat{\alpha}$ scaling, explaining why $\mathrm{RSS}$ is also small in figure~\ref{fig:cnn}.

\begin{figure}[H]
  \begin{subfigure}{.5\textwidth}
    \centering
    \resizebox{\textwidth}{!}{%% Creator: Matplotlib, PGF backend
%%
%% To include the figure in your LaTeX document, write
%%   \input{<filename>.pgf}
%%
%% Make sure the required packages are loaded in your preamble
%%   \usepackage{pgf}
%%
%% and, on pdftex
%%   \usepackage[utf8]{inputenc}\DeclareUnicodeCharacter{2212}{-}
%%
%% or, on luatex and xetex
%%   \usepackage{unicode-math}
%%
%% Figures using additional raster images can only be included by \input if
%% they are in the same directory as the main LaTeX file. For loading figures
%% from other directories you can use the `import` package
%%   \usepackage{import}
%%
%% and then include the figures with
%%   \import{<path to file>}{<filename>.pgf}
%%
%% Matplotlib used the following preamble
%%   \usepackage[detect-all,locale=DE]{siunitx}
%%
\begingroup%
\makeatletter%
\begin{pgfpicture}%
\pgfpathrectangle{\pgfpointorigin}{\pgfqpoint{8.000000in}{6.000000in}}%
\pgfusepath{use as bounding box, clip}%
\begin{pgfscope}%
\pgfsetbuttcap%
\pgfsetmiterjoin%
\definecolor{currentfill}{rgb}{1.000000,1.000000,1.000000}%
\pgfsetfillcolor{currentfill}%
\pgfsetlinewidth{0.000000pt}%
\definecolor{currentstroke}{rgb}{1.000000,1.000000,1.000000}%
\pgfsetstrokecolor{currentstroke}%
\pgfsetdash{}{0pt}%
\pgfpathmoveto{\pgfqpoint{0.000000in}{0.000000in}}%
\pgfpathlineto{\pgfqpoint{8.000000in}{0.000000in}}%
\pgfpathlineto{\pgfqpoint{8.000000in}{6.000000in}}%
\pgfpathlineto{\pgfqpoint{0.000000in}{6.000000in}}%
\pgfpathclose%
\pgfusepath{fill}%
\end{pgfscope}%
\begin{pgfscope}%
\pgfsetbuttcap%
\pgfsetmiterjoin%
\definecolor{currentfill}{rgb}{1.000000,1.000000,1.000000}%
\pgfsetfillcolor{currentfill}%
\pgfsetlinewidth{0.000000pt}%
\definecolor{currentstroke}{rgb}{0.000000,0.000000,0.000000}%
\pgfsetstrokecolor{currentstroke}%
\pgfsetstrokeopacity{0.000000}%
\pgfsetdash{}{0pt}%
\pgfpathmoveto{\pgfqpoint{1.000000in}{0.720000in}}%
\pgfpathlineto{\pgfqpoint{5.800000in}{0.720000in}}%
\pgfpathlineto{\pgfqpoint{5.800000in}{5.340000in}}%
\pgfpathlineto{\pgfqpoint{1.000000in}{5.340000in}}%
\pgfpathclose%
\pgfusepath{fill}%
\end{pgfscope}%
\begin{pgfscope}%
\pgfsetbuttcap%
\pgfsetroundjoin%
\definecolor{currentfill}{rgb}{0.000000,0.000000,0.000000}%
\pgfsetfillcolor{currentfill}%
\pgfsetlinewidth{0.803000pt}%
\definecolor{currentstroke}{rgb}{0.000000,0.000000,0.000000}%
\pgfsetstrokecolor{currentstroke}%
\pgfsetdash{}{0pt}%
\pgfsys@defobject{currentmarker}{\pgfqpoint{0.000000in}{-0.048611in}}{\pgfqpoint{0.000000in}{0.000000in}}{%
\pgfpathmoveto{\pgfqpoint{0.000000in}{0.000000in}}%
\pgfpathlineto{\pgfqpoint{0.000000in}{-0.048611in}}%
\pgfusepath{stroke,fill}%
}%
\begin{pgfscope}%
\pgfsys@transformshift{1.300000in}{0.720000in}%
\pgfsys@useobject{currentmarker}{}%
\end{pgfscope}%
\end{pgfscope}%
\begin{pgfscope}%
\definecolor{textcolor}{rgb}{0.000000,0.000000,0.000000}%
\pgfsetstrokecolor{textcolor}%
\pgfsetfillcolor{textcolor}%
\pgftext[x=1.300000in,y=0.622778in,,top]{\color{textcolor}\sffamily\fontsize{20.000000}{24.000000}\selectfont 1}%
\end{pgfscope}%
\begin{pgfscope}%
\pgfsetbuttcap%
\pgfsetroundjoin%
\definecolor{currentfill}{rgb}{0.000000,0.000000,0.000000}%
\pgfsetfillcolor{currentfill}%
\pgfsetlinewidth{0.803000pt}%
\definecolor{currentstroke}{rgb}{0.000000,0.000000,0.000000}%
\pgfsetstrokecolor{currentstroke}%
\pgfsetdash{}{0pt}%
\pgfsys@defobject{currentmarker}{\pgfqpoint{0.000000in}{-0.048611in}}{\pgfqpoint{0.000000in}{0.000000in}}{%
\pgfpathmoveto{\pgfqpoint{0.000000in}{0.000000in}}%
\pgfpathlineto{\pgfqpoint{0.000000in}{-0.048611in}}%
\pgfusepath{stroke,fill}%
}%
\begin{pgfscope}%
\pgfsys@transformshift{1.900000in}{0.720000in}%
\pgfsys@useobject{currentmarker}{}%
\end{pgfscope}%
\end{pgfscope}%
\begin{pgfscope}%
\definecolor{textcolor}{rgb}{0.000000,0.000000,0.000000}%
\pgfsetstrokecolor{textcolor}%
\pgfsetfillcolor{textcolor}%
\pgftext[x=1.900000in,y=0.622778in,,top]{\color{textcolor}\sffamily\fontsize{20.000000}{24.000000}\selectfont 3}%
\end{pgfscope}%
\begin{pgfscope}%
\pgfsetbuttcap%
\pgfsetroundjoin%
\definecolor{currentfill}{rgb}{0.000000,0.000000,0.000000}%
\pgfsetfillcolor{currentfill}%
\pgfsetlinewidth{0.803000pt}%
\definecolor{currentstroke}{rgb}{0.000000,0.000000,0.000000}%
\pgfsetstrokecolor{currentstroke}%
\pgfsetdash{}{0pt}%
\pgfsys@defobject{currentmarker}{\pgfqpoint{0.000000in}{-0.048611in}}{\pgfqpoint{0.000000in}{0.000000in}}{%
\pgfpathmoveto{\pgfqpoint{0.000000in}{0.000000in}}%
\pgfpathlineto{\pgfqpoint{0.000000in}{-0.048611in}}%
\pgfusepath{stroke,fill}%
}%
\begin{pgfscope}%
\pgfsys@transformshift{2.500000in}{0.720000in}%
\pgfsys@useobject{currentmarker}{}%
\end{pgfscope}%
\end{pgfscope}%
\begin{pgfscope}%
\definecolor{textcolor}{rgb}{0.000000,0.000000,0.000000}%
\pgfsetstrokecolor{textcolor}%
\pgfsetfillcolor{textcolor}%
\pgftext[x=2.500000in,y=0.622778in,,top]{\color{textcolor}\sffamily\fontsize{20.000000}{24.000000}\selectfont 5}%
\end{pgfscope}%
\begin{pgfscope}%
\pgfsetbuttcap%
\pgfsetroundjoin%
\definecolor{currentfill}{rgb}{0.000000,0.000000,0.000000}%
\pgfsetfillcolor{currentfill}%
\pgfsetlinewidth{0.803000pt}%
\definecolor{currentstroke}{rgb}{0.000000,0.000000,0.000000}%
\pgfsetstrokecolor{currentstroke}%
\pgfsetdash{}{0pt}%
\pgfsys@defobject{currentmarker}{\pgfqpoint{0.000000in}{-0.048611in}}{\pgfqpoint{0.000000in}{0.000000in}}{%
\pgfpathmoveto{\pgfqpoint{0.000000in}{0.000000in}}%
\pgfpathlineto{\pgfqpoint{0.000000in}{-0.048611in}}%
\pgfusepath{stroke,fill}%
}%
\begin{pgfscope}%
\pgfsys@transformshift{3.100000in}{0.720000in}%
\pgfsys@useobject{currentmarker}{}%
\end{pgfscope}%
\end{pgfscope}%
\begin{pgfscope}%
\definecolor{textcolor}{rgb}{0.000000,0.000000,0.000000}%
\pgfsetstrokecolor{textcolor}%
\pgfsetfillcolor{textcolor}%
\pgftext[x=3.100000in,y=0.622778in,,top]{\color{textcolor}\sffamily\fontsize{20.000000}{24.000000}\selectfont 7}%
\end{pgfscope}%
\begin{pgfscope}%
\pgfsetbuttcap%
\pgfsetroundjoin%
\definecolor{currentfill}{rgb}{0.000000,0.000000,0.000000}%
\pgfsetfillcolor{currentfill}%
\pgfsetlinewidth{0.803000pt}%
\definecolor{currentstroke}{rgb}{0.000000,0.000000,0.000000}%
\pgfsetstrokecolor{currentstroke}%
\pgfsetdash{}{0pt}%
\pgfsys@defobject{currentmarker}{\pgfqpoint{0.000000in}{-0.048611in}}{\pgfqpoint{0.000000in}{0.000000in}}{%
\pgfpathmoveto{\pgfqpoint{0.000000in}{0.000000in}}%
\pgfpathlineto{\pgfqpoint{0.000000in}{-0.048611in}}%
\pgfusepath{stroke,fill}%
}%
\begin{pgfscope}%
\pgfsys@transformshift{3.700000in}{0.720000in}%
\pgfsys@useobject{currentmarker}{}%
\end{pgfscope}%
\end{pgfscope}%
\begin{pgfscope}%
\definecolor{textcolor}{rgb}{0.000000,0.000000,0.000000}%
\pgfsetstrokecolor{textcolor}%
\pgfsetfillcolor{textcolor}%
\pgftext[x=3.700000in,y=0.622778in,,top]{\color{textcolor}\sffamily\fontsize{20.000000}{24.000000}\selectfont 9}%
\end{pgfscope}%
\begin{pgfscope}%
\pgfsetbuttcap%
\pgfsetroundjoin%
\definecolor{currentfill}{rgb}{0.000000,0.000000,0.000000}%
\pgfsetfillcolor{currentfill}%
\pgfsetlinewidth{0.803000pt}%
\definecolor{currentstroke}{rgb}{0.000000,0.000000,0.000000}%
\pgfsetstrokecolor{currentstroke}%
\pgfsetdash{}{0pt}%
\pgfsys@defobject{currentmarker}{\pgfqpoint{0.000000in}{-0.048611in}}{\pgfqpoint{0.000000in}{0.000000in}}{%
\pgfpathmoveto{\pgfqpoint{0.000000in}{0.000000in}}%
\pgfpathlineto{\pgfqpoint{0.000000in}{-0.048611in}}%
\pgfusepath{stroke,fill}%
}%
\begin{pgfscope}%
\pgfsys@transformshift{4.300000in}{0.720000in}%
\pgfsys@useobject{currentmarker}{}%
\end{pgfscope}%
\end{pgfscope}%
\begin{pgfscope}%
\definecolor{textcolor}{rgb}{0.000000,0.000000,0.000000}%
\pgfsetstrokecolor{textcolor}%
\pgfsetfillcolor{textcolor}%
\pgftext[x=4.300000in,y=0.622778in,,top]{\color{textcolor}\sffamily\fontsize{20.000000}{24.000000}\selectfont 11}%
\end{pgfscope}%
\begin{pgfscope}%
\pgfsetbuttcap%
\pgfsetroundjoin%
\definecolor{currentfill}{rgb}{0.000000,0.000000,0.000000}%
\pgfsetfillcolor{currentfill}%
\pgfsetlinewidth{0.803000pt}%
\definecolor{currentstroke}{rgb}{0.000000,0.000000,0.000000}%
\pgfsetstrokecolor{currentstroke}%
\pgfsetdash{}{0pt}%
\pgfsys@defobject{currentmarker}{\pgfqpoint{0.000000in}{-0.048611in}}{\pgfqpoint{0.000000in}{0.000000in}}{%
\pgfpathmoveto{\pgfqpoint{0.000000in}{0.000000in}}%
\pgfpathlineto{\pgfqpoint{0.000000in}{-0.048611in}}%
\pgfusepath{stroke,fill}%
}%
\begin{pgfscope}%
\pgfsys@transformshift{4.900000in}{0.720000in}%
\pgfsys@useobject{currentmarker}{}%
\end{pgfscope}%
\end{pgfscope}%
\begin{pgfscope}%
\definecolor{textcolor}{rgb}{0.000000,0.000000,0.000000}%
\pgfsetstrokecolor{textcolor}%
\pgfsetfillcolor{textcolor}%
\pgftext[x=4.900000in,y=0.622778in,,top]{\color{textcolor}\sffamily\fontsize{20.000000}{24.000000}\selectfont 13}%
\end{pgfscope}%
\begin{pgfscope}%
\pgfsetbuttcap%
\pgfsetroundjoin%
\definecolor{currentfill}{rgb}{0.000000,0.000000,0.000000}%
\pgfsetfillcolor{currentfill}%
\pgfsetlinewidth{0.803000pt}%
\definecolor{currentstroke}{rgb}{0.000000,0.000000,0.000000}%
\pgfsetstrokecolor{currentstroke}%
\pgfsetdash{}{0pt}%
\pgfsys@defobject{currentmarker}{\pgfqpoint{0.000000in}{-0.048611in}}{\pgfqpoint{0.000000in}{0.000000in}}{%
\pgfpathmoveto{\pgfqpoint{0.000000in}{0.000000in}}%
\pgfpathlineto{\pgfqpoint{0.000000in}{-0.048611in}}%
\pgfusepath{stroke,fill}%
}%
\begin{pgfscope}%
\pgfsys@transformshift{5.500000in}{0.720000in}%
\pgfsys@useobject{currentmarker}{}%
\end{pgfscope}%
\end{pgfscope}%
\begin{pgfscope}%
\definecolor{textcolor}{rgb}{0.000000,0.000000,0.000000}%
\pgfsetstrokecolor{textcolor}%
\pgfsetfillcolor{textcolor}%
\pgftext[x=5.500000in,y=0.622778in,,top]{\color{textcolor}\sffamily\fontsize{20.000000}{24.000000}\selectfont 15}%
\end{pgfscope}%
\begin{pgfscope}%
\definecolor{textcolor}{rgb}{0.000000,0.000000,0.000000}%
\pgfsetstrokecolor{textcolor}%
\pgfsetfillcolor{textcolor}%
\pgftext[x=3.400000in,y=0.311155in,,top]{\color{textcolor}\sffamily\fontsize{20.000000}{24.000000}\selectfont \(\displaystyle N_{\mathrm{PE}}\)}%
\end{pgfscope}%
\begin{pgfscope}%
\pgfsetbuttcap%
\pgfsetroundjoin%
\definecolor{currentfill}{rgb}{0.000000,0.000000,0.000000}%
\pgfsetfillcolor{currentfill}%
\pgfsetlinewidth{0.803000pt}%
\definecolor{currentstroke}{rgb}{0.000000,0.000000,0.000000}%
\pgfsetstrokecolor{currentstroke}%
\pgfsetdash{}{0pt}%
\pgfsys@defobject{currentmarker}{\pgfqpoint{-0.048611in}{0.000000in}}{\pgfqpoint{-0.000000in}{0.000000in}}{%
\pgfpathmoveto{\pgfqpoint{-0.000000in}{0.000000in}}%
\pgfpathlineto{\pgfqpoint{-0.048611in}{0.000000in}}%
\pgfusepath{stroke,fill}%
}%
\begin{pgfscope}%
\pgfsys@transformshift{1.000000in}{0.720000in}%
\pgfsys@useobject{currentmarker}{}%
\end{pgfscope}%
\end{pgfscope}%
\begin{pgfscope}%
\definecolor{textcolor}{rgb}{0.000000,0.000000,0.000000}%
\pgfsetstrokecolor{textcolor}%
\pgfsetfillcolor{textcolor}%
\pgftext[x=0.560215in, y=0.619981in, left, base]{\color{textcolor}\sffamily\fontsize{20.000000}{24.000000}\selectfont \(\displaystyle {0.0}\)}%
\end{pgfscope}%
\begin{pgfscope}%
\pgfsetbuttcap%
\pgfsetroundjoin%
\definecolor{currentfill}{rgb}{0.000000,0.000000,0.000000}%
\pgfsetfillcolor{currentfill}%
\pgfsetlinewidth{0.803000pt}%
\definecolor{currentstroke}{rgb}{0.000000,0.000000,0.000000}%
\pgfsetstrokecolor{currentstroke}%
\pgfsetdash{}{0pt}%
\pgfsys@defobject{currentmarker}{\pgfqpoint{-0.048611in}{0.000000in}}{\pgfqpoint{-0.000000in}{0.000000in}}{%
\pgfpathmoveto{\pgfqpoint{-0.000000in}{0.000000in}}%
\pgfpathlineto{\pgfqpoint{-0.048611in}{0.000000in}}%
\pgfusepath{stroke,fill}%
}%
\begin{pgfscope}%
\pgfsys@transformshift{1.000000in}{1.598723in}%
\pgfsys@useobject{currentmarker}{}%
\end{pgfscope}%
\end{pgfscope}%
\begin{pgfscope}%
\definecolor{textcolor}{rgb}{0.000000,0.000000,0.000000}%
\pgfsetstrokecolor{textcolor}%
\pgfsetfillcolor{textcolor}%
\pgftext[x=0.560215in, y=1.498704in, left, base]{\color{textcolor}\sffamily\fontsize{20.000000}{24.000000}\selectfont \(\displaystyle {0.2}\)}%
\end{pgfscope}%
\begin{pgfscope}%
\pgfsetbuttcap%
\pgfsetroundjoin%
\definecolor{currentfill}{rgb}{0.000000,0.000000,0.000000}%
\pgfsetfillcolor{currentfill}%
\pgfsetlinewidth{0.803000pt}%
\definecolor{currentstroke}{rgb}{0.000000,0.000000,0.000000}%
\pgfsetstrokecolor{currentstroke}%
\pgfsetdash{}{0pt}%
\pgfsys@defobject{currentmarker}{\pgfqpoint{-0.048611in}{0.000000in}}{\pgfqpoint{-0.000000in}{0.000000in}}{%
\pgfpathmoveto{\pgfqpoint{-0.000000in}{0.000000in}}%
\pgfpathlineto{\pgfqpoint{-0.048611in}{0.000000in}}%
\pgfusepath{stroke,fill}%
}%
\begin{pgfscope}%
\pgfsys@transformshift{1.000000in}{2.477446in}%
\pgfsys@useobject{currentmarker}{}%
\end{pgfscope}%
\end{pgfscope}%
\begin{pgfscope}%
\definecolor{textcolor}{rgb}{0.000000,0.000000,0.000000}%
\pgfsetstrokecolor{textcolor}%
\pgfsetfillcolor{textcolor}%
\pgftext[x=0.560215in, y=2.377426in, left, base]{\color{textcolor}\sffamily\fontsize{20.000000}{24.000000}\selectfont \(\displaystyle {0.4}\)}%
\end{pgfscope}%
\begin{pgfscope}%
\pgfsetbuttcap%
\pgfsetroundjoin%
\definecolor{currentfill}{rgb}{0.000000,0.000000,0.000000}%
\pgfsetfillcolor{currentfill}%
\pgfsetlinewidth{0.803000pt}%
\definecolor{currentstroke}{rgb}{0.000000,0.000000,0.000000}%
\pgfsetstrokecolor{currentstroke}%
\pgfsetdash{}{0pt}%
\pgfsys@defobject{currentmarker}{\pgfqpoint{-0.048611in}{0.000000in}}{\pgfqpoint{-0.000000in}{0.000000in}}{%
\pgfpathmoveto{\pgfqpoint{-0.000000in}{0.000000in}}%
\pgfpathlineto{\pgfqpoint{-0.048611in}{0.000000in}}%
\pgfusepath{stroke,fill}%
}%
\begin{pgfscope}%
\pgfsys@transformshift{1.000000in}{3.356169in}%
\pgfsys@useobject{currentmarker}{}%
\end{pgfscope}%
\end{pgfscope}%
\begin{pgfscope}%
\definecolor{textcolor}{rgb}{0.000000,0.000000,0.000000}%
\pgfsetstrokecolor{textcolor}%
\pgfsetfillcolor{textcolor}%
\pgftext[x=0.560215in, y=3.256149in, left, base]{\color{textcolor}\sffamily\fontsize{20.000000}{24.000000}\selectfont \(\displaystyle {0.6}\)}%
\end{pgfscope}%
\begin{pgfscope}%
\pgfsetbuttcap%
\pgfsetroundjoin%
\definecolor{currentfill}{rgb}{0.000000,0.000000,0.000000}%
\pgfsetfillcolor{currentfill}%
\pgfsetlinewidth{0.803000pt}%
\definecolor{currentstroke}{rgb}{0.000000,0.000000,0.000000}%
\pgfsetstrokecolor{currentstroke}%
\pgfsetdash{}{0pt}%
\pgfsys@defobject{currentmarker}{\pgfqpoint{-0.048611in}{0.000000in}}{\pgfqpoint{-0.000000in}{0.000000in}}{%
\pgfpathmoveto{\pgfqpoint{-0.000000in}{0.000000in}}%
\pgfpathlineto{\pgfqpoint{-0.048611in}{0.000000in}}%
\pgfusepath{stroke,fill}%
}%
\begin{pgfscope}%
\pgfsys@transformshift{1.000000in}{4.234891in}%
\pgfsys@useobject{currentmarker}{}%
\end{pgfscope}%
\end{pgfscope}%
\begin{pgfscope}%
\definecolor{textcolor}{rgb}{0.000000,0.000000,0.000000}%
\pgfsetstrokecolor{textcolor}%
\pgfsetfillcolor{textcolor}%
\pgftext[x=0.560215in, y=4.134872in, left, base]{\color{textcolor}\sffamily\fontsize{20.000000}{24.000000}\selectfont \(\displaystyle {0.8}\)}%
\end{pgfscope}%
\begin{pgfscope}%
\pgfsetbuttcap%
\pgfsetroundjoin%
\definecolor{currentfill}{rgb}{0.000000,0.000000,0.000000}%
\pgfsetfillcolor{currentfill}%
\pgfsetlinewidth{0.803000pt}%
\definecolor{currentstroke}{rgb}{0.000000,0.000000,0.000000}%
\pgfsetstrokecolor{currentstroke}%
\pgfsetdash{}{0pt}%
\pgfsys@defobject{currentmarker}{\pgfqpoint{-0.048611in}{0.000000in}}{\pgfqpoint{-0.000000in}{0.000000in}}{%
\pgfpathmoveto{\pgfqpoint{-0.000000in}{0.000000in}}%
\pgfpathlineto{\pgfqpoint{-0.048611in}{0.000000in}}%
\pgfusepath{stroke,fill}%
}%
\begin{pgfscope}%
\pgfsys@transformshift{1.000000in}{5.113614in}%
\pgfsys@useobject{currentmarker}{}%
\end{pgfscope}%
\end{pgfscope}%
\begin{pgfscope}%
\definecolor{textcolor}{rgb}{0.000000,0.000000,0.000000}%
\pgfsetstrokecolor{textcolor}%
\pgfsetfillcolor{textcolor}%
\pgftext[x=0.560215in, y=5.013595in, left, base]{\color{textcolor}\sffamily\fontsize{20.000000}{24.000000}\selectfont \(\displaystyle {1.0}\)}%
\end{pgfscope}%
\begin{pgfscope}%
\definecolor{textcolor}{rgb}{0.000000,0.000000,0.000000}%
\pgfsetstrokecolor{textcolor}%
\pgfsetfillcolor{textcolor}%
\pgftext[x=0.504660in,y=3.030000in,,bottom,rotate=90.000000]{\color{textcolor}\sffamily\fontsize{20.000000}{24.000000}\selectfont \(\displaystyle \mathrm{Wasserstein\ Distance}/\si{ns}\)}%
\end{pgfscope}%
\begin{pgfscope}%
\pgfpathrectangle{\pgfqpoint{1.000000in}{0.720000in}}{\pgfqpoint{4.800000in}{4.620000in}}%
\pgfusepath{clip}%
\pgfsetrectcap%
\pgfsetroundjoin%
\pgfsetlinewidth{1.003750pt}%
\definecolor{currentstroke}{rgb}{0.000000,0.000000,0.000000}%
\pgfsetstrokecolor{currentstroke}%
\pgfsetdash{}{0pt}%
\pgfpathmoveto{\pgfqpoint{1.300000in}{1.275034in}}%
\pgfpathlineto{\pgfqpoint{1.300000in}{0.720538in}}%
\pgfusepath{stroke}%
\end{pgfscope}%
\begin{pgfscope}%
\pgfpathrectangle{\pgfqpoint{1.000000in}{0.720000in}}{\pgfqpoint{4.800000in}{4.620000in}}%
\pgfusepath{clip}%
\pgfsetrectcap%
\pgfsetroundjoin%
\pgfsetlinewidth{1.003750pt}%
\definecolor{currentstroke}{rgb}{0.000000,0.000000,0.000000}%
\pgfsetstrokecolor{currentstroke}%
\pgfsetdash{}{0pt}%
\pgfpathmoveto{\pgfqpoint{1.300000in}{2.458391in}}%
\pgfpathlineto{\pgfqpoint{1.300000in}{3.826356in}}%
\pgfusepath{stroke}%
\end{pgfscope}%
\begin{pgfscope}%
\pgfpathrectangle{\pgfqpoint{1.000000in}{0.720000in}}{\pgfqpoint{4.800000in}{4.620000in}}%
\pgfusepath{clip}%
\pgfsetrectcap%
\pgfsetroundjoin%
\pgfsetlinewidth{1.003750pt}%
\definecolor{currentstroke}{rgb}{0.000000,0.000000,0.000000}%
\pgfsetstrokecolor{currentstroke}%
\pgfsetdash{}{0pt}%
\pgfpathmoveto{\pgfqpoint{1.262500in}{0.720538in}}%
\pgfpathlineto{\pgfqpoint{1.337500in}{0.720538in}}%
\pgfusepath{stroke}%
\end{pgfscope}%
\begin{pgfscope}%
\pgfpathrectangle{\pgfqpoint{1.000000in}{0.720000in}}{\pgfqpoint{4.800000in}{4.620000in}}%
\pgfusepath{clip}%
\pgfsetrectcap%
\pgfsetroundjoin%
\pgfsetlinewidth{1.003750pt}%
\definecolor{currentstroke}{rgb}{0.000000,0.000000,0.000000}%
\pgfsetstrokecolor{currentstroke}%
\pgfsetdash{}{0pt}%
\pgfpathmoveto{\pgfqpoint{1.262500in}{3.826356in}}%
\pgfpathlineto{\pgfqpoint{1.337500in}{3.826356in}}%
\pgfusepath{stroke}%
\end{pgfscope}%
\begin{pgfscope}%
\pgfpathrectangle{\pgfqpoint{1.000000in}{0.720000in}}{\pgfqpoint{4.800000in}{4.620000in}}%
\pgfusepath{clip}%
\pgfsetrectcap%
\pgfsetroundjoin%
\pgfsetlinewidth{1.003750pt}%
\definecolor{currentstroke}{rgb}{0.000000,0.000000,0.000000}%
\pgfsetstrokecolor{currentstroke}%
\pgfsetdash{}{0pt}%
\pgfpathmoveto{\pgfqpoint{1.600000in}{2.052040in}}%
\pgfpathlineto{\pgfqpoint{1.600000in}{0.935591in}}%
\pgfusepath{stroke}%
\end{pgfscope}%
\begin{pgfscope}%
\pgfpathrectangle{\pgfqpoint{1.000000in}{0.720000in}}{\pgfqpoint{4.800000in}{4.620000in}}%
\pgfusepath{clip}%
\pgfsetrectcap%
\pgfsetroundjoin%
\pgfsetlinewidth{1.003750pt}%
\definecolor{currentstroke}{rgb}{0.000000,0.000000,0.000000}%
\pgfsetstrokecolor{currentstroke}%
\pgfsetdash{}{0pt}%
\pgfpathmoveto{\pgfqpoint{1.600000in}{3.291574in}}%
\pgfpathlineto{\pgfqpoint{1.600000in}{5.120000in}}%
\pgfusepath{stroke}%
\end{pgfscope}%
\begin{pgfscope}%
\pgfpathrectangle{\pgfqpoint{1.000000in}{0.720000in}}{\pgfqpoint{4.800000in}{4.620000in}}%
\pgfusepath{clip}%
\pgfsetrectcap%
\pgfsetroundjoin%
\pgfsetlinewidth{1.003750pt}%
\definecolor{currentstroke}{rgb}{0.000000,0.000000,0.000000}%
\pgfsetstrokecolor{currentstroke}%
\pgfsetdash{}{0pt}%
\pgfpathmoveto{\pgfqpoint{1.562500in}{0.935591in}}%
\pgfpathlineto{\pgfqpoint{1.637500in}{0.935591in}}%
\pgfusepath{stroke}%
\end{pgfscope}%
\begin{pgfscope}%
\pgfpathrectangle{\pgfqpoint{1.000000in}{0.720000in}}{\pgfqpoint{4.800000in}{4.620000in}}%
\pgfusepath{clip}%
\pgfsetrectcap%
\pgfsetroundjoin%
\pgfsetlinewidth{1.003750pt}%
\definecolor{currentstroke}{rgb}{0.000000,0.000000,0.000000}%
\pgfsetstrokecolor{currentstroke}%
\pgfsetdash{}{0pt}%
\pgfpathmoveto{\pgfqpoint{1.562500in}{5.120000in}}%
\pgfpathlineto{\pgfqpoint{1.637500in}{5.120000in}}%
\pgfusepath{stroke}%
\end{pgfscope}%
\begin{pgfscope}%
\pgfpathrectangle{\pgfqpoint{1.000000in}{0.720000in}}{\pgfqpoint{4.800000in}{4.620000in}}%
\pgfusepath{clip}%
\pgfsetrectcap%
\pgfsetroundjoin%
\pgfsetlinewidth{1.003750pt}%
\definecolor{currentstroke}{rgb}{0.000000,0.000000,0.000000}%
\pgfsetstrokecolor{currentstroke}%
\pgfsetdash{}{0pt}%
\pgfpathmoveto{\pgfqpoint{1.900000in}{2.377197in}}%
\pgfpathlineto{\pgfqpoint{1.900000in}{1.232889in}}%
\pgfusepath{stroke}%
\end{pgfscope}%
\begin{pgfscope}%
\pgfpathrectangle{\pgfqpoint{1.000000in}{0.720000in}}{\pgfqpoint{4.800000in}{4.620000in}}%
\pgfusepath{clip}%
\pgfsetrectcap%
\pgfsetroundjoin%
\pgfsetlinewidth{1.003750pt}%
\definecolor{currentstroke}{rgb}{0.000000,0.000000,0.000000}%
\pgfsetstrokecolor{currentstroke}%
\pgfsetdash{}{0pt}%
\pgfpathmoveto{\pgfqpoint{1.900000in}{3.484357in}}%
\pgfpathlineto{\pgfqpoint{1.900000in}{5.119703in}}%
\pgfusepath{stroke}%
\end{pgfscope}%
\begin{pgfscope}%
\pgfpathrectangle{\pgfqpoint{1.000000in}{0.720000in}}{\pgfqpoint{4.800000in}{4.620000in}}%
\pgfusepath{clip}%
\pgfsetrectcap%
\pgfsetroundjoin%
\pgfsetlinewidth{1.003750pt}%
\definecolor{currentstroke}{rgb}{0.000000,0.000000,0.000000}%
\pgfsetstrokecolor{currentstroke}%
\pgfsetdash{}{0pt}%
\pgfpathmoveto{\pgfqpoint{1.862500in}{1.232889in}}%
\pgfpathlineto{\pgfqpoint{1.937500in}{1.232889in}}%
\pgfusepath{stroke}%
\end{pgfscope}%
\begin{pgfscope}%
\pgfpathrectangle{\pgfqpoint{1.000000in}{0.720000in}}{\pgfqpoint{4.800000in}{4.620000in}}%
\pgfusepath{clip}%
\pgfsetrectcap%
\pgfsetroundjoin%
\pgfsetlinewidth{1.003750pt}%
\definecolor{currentstroke}{rgb}{0.000000,0.000000,0.000000}%
\pgfsetstrokecolor{currentstroke}%
\pgfsetdash{}{0pt}%
\pgfpathmoveto{\pgfqpoint{1.862500in}{5.119703in}}%
\pgfpathlineto{\pgfqpoint{1.937500in}{5.119703in}}%
\pgfusepath{stroke}%
\end{pgfscope}%
\begin{pgfscope}%
\pgfpathrectangle{\pgfqpoint{1.000000in}{0.720000in}}{\pgfqpoint{4.800000in}{4.620000in}}%
\pgfusepath{clip}%
\pgfsetrectcap%
\pgfsetroundjoin%
\pgfsetlinewidth{1.003750pt}%
\definecolor{currentstroke}{rgb}{0.000000,0.000000,0.000000}%
\pgfsetstrokecolor{currentstroke}%
\pgfsetdash{}{0pt}%
\pgfpathmoveto{\pgfqpoint{2.200000in}{2.587973in}}%
\pgfpathlineto{\pgfqpoint{2.200000in}{1.138166in}}%
\pgfusepath{stroke}%
\end{pgfscope}%
\begin{pgfscope}%
\pgfpathrectangle{\pgfqpoint{1.000000in}{0.720000in}}{\pgfqpoint{4.800000in}{4.620000in}}%
\pgfusepath{clip}%
\pgfsetrectcap%
\pgfsetroundjoin%
\pgfsetlinewidth{1.003750pt}%
\definecolor{currentstroke}{rgb}{0.000000,0.000000,0.000000}%
\pgfsetstrokecolor{currentstroke}%
\pgfsetdash{}{0pt}%
\pgfpathmoveto{\pgfqpoint{2.200000in}{3.606409in}}%
\pgfpathlineto{\pgfqpoint{2.200000in}{5.119000in}}%
\pgfusepath{stroke}%
\end{pgfscope}%
\begin{pgfscope}%
\pgfpathrectangle{\pgfqpoint{1.000000in}{0.720000in}}{\pgfqpoint{4.800000in}{4.620000in}}%
\pgfusepath{clip}%
\pgfsetrectcap%
\pgfsetroundjoin%
\pgfsetlinewidth{1.003750pt}%
\definecolor{currentstroke}{rgb}{0.000000,0.000000,0.000000}%
\pgfsetstrokecolor{currentstroke}%
\pgfsetdash{}{0pt}%
\pgfpathmoveto{\pgfqpoint{2.162500in}{1.138166in}}%
\pgfpathlineto{\pgfqpoint{2.237500in}{1.138166in}}%
\pgfusepath{stroke}%
\end{pgfscope}%
\begin{pgfscope}%
\pgfpathrectangle{\pgfqpoint{1.000000in}{0.720000in}}{\pgfqpoint{4.800000in}{4.620000in}}%
\pgfusepath{clip}%
\pgfsetrectcap%
\pgfsetroundjoin%
\pgfsetlinewidth{1.003750pt}%
\definecolor{currentstroke}{rgb}{0.000000,0.000000,0.000000}%
\pgfsetstrokecolor{currentstroke}%
\pgfsetdash{}{0pt}%
\pgfpathmoveto{\pgfqpoint{2.162500in}{5.119000in}}%
\pgfpathlineto{\pgfqpoint{2.237500in}{5.119000in}}%
\pgfusepath{stroke}%
\end{pgfscope}%
\begin{pgfscope}%
\pgfpathrectangle{\pgfqpoint{1.000000in}{0.720000in}}{\pgfqpoint{4.800000in}{4.620000in}}%
\pgfusepath{clip}%
\pgfsetrectcap%
\pgfsetroundjoin%
\pgfsetlinewidth{1.003750pt}%
\definecolor{currentstroke}{rgb}{0.000000,0.000000,0.000000}%
\pgfsetstrokecolor{currentstroke}%
\pgfsetdash{}{0pt}%
\pgfpathmoveto{\pgfqpoint{2.500000in}{2.733762in}}%
\pgfpathlineto{\pgfqpoint{2.500000in}{1.537923in}}%
\pgfusepath{stroke}%
\end{pgfscope}%
\begin{pgfscope}%
\pgfpathrectangle{\pgfqpoint{1.000000in}{0.720000in}}{\pgfqpoint{4.800000in}{4.620000in}}%
\pgfusepath{clip}%
\pgfsetrectcap%
\pgfsetroundjoin%
\pgfsetlinewidth{1.003750pt}%
\definecolor{currentstroke}{rgb}{0.000000,0.000000,0.000000}%
\pgfsetstrokecolor{currentstroke}%
\pgfsetdash{}{0pt}%
\pgfpathmoveto{\pgfqpoint{2.500000in}{3.663233in}}%
\pgfpathlineto{\pgfqpoint{2.500000in}{5.028135in}}%
\pgfusepath{stroke}%
\end{pgfscope}%
\begin{pgfscope}%
\pgfpathrectangle{\pgfqpoint{1.000000in}{0.720000in}}{\pgfqpoint{4.800000in}{4.620000in}}%
\pgfusepath{clip}%
\pgfsetrectcap%
\pgfsetroundjoin%
\pgfsetlinewidth{1.003750pt}%
\definecolor{currentstroke}{rgb}{0.000000,0.000000,0.000000}%
\pgfsetstrokecolor{currentstroke}%
\pgfsetdash{}{0pt}%
\pgfpathmoveto{\pgfqpoint{2.462500in}{1.537923in}}%
\pgfpathlineto{\pgfqpoint{2.537500in}{1.537923in}}%
\pgfusepath{stroke}%
\end{pgfscope}%
\begin{pgfscope}%
\pgfpathrectangle{\pgfqpoint{1.000000in}{0.720000in}}{\pgfqpoint{4.800000in}{4.620000in}}%
\pgfusepath{clip}%
\pgfsetrectcap%
\pgfsetroundjoin%
\pgfsetlinewidth{1.003750pt}%
\definecolor{currentstroke}{rgb}{0.000000,0.000000,0.000000}%
\pgfsetstrokecolor{currentstroke}%
\pgfsetdash{}{0pt}%
\pgfpathmoveto{\pgfqpoint{2.462500in}{5.028135in}}%
\pgfpathlineto{\pgfqpoint{2.537500in}{5.028135in}}%
\pgfusepath{stroke}%
\end{pgfscope}%
\begin{pgfscope}%
\pgfpathrectangle{\pgfqpoint{1.000000in}{0.720000in}}{\pgfqpoint{4.800000in}{4.620000in}}%
\pgfusepath{clip}%
\pgfsetrectcap%
\pgfsetroundjoin%
\pgfsetlinewidth{1.003750pt}%
\definecolor{currentstroke}{rgb}{0.000000,0.000000,0.000000}%
\pgfsetstrokecolor{currentstroke}%
\pgfsetdash{}{0pt}%
\pgfpathmoveto{\pgfqpoint{2.800000in}{2.764618in}}%
\pgfpathlineto{\pgfqpoint{2.800000in}{1.500258in}}%
\pgfusepath{stroke}%
\end{pgfscope}%
\begin{pgfscope}%
\pgfpathrectangle{\pgfqpoint{1.000000in}{0.720000in}}{\pgfqpoint{4.800000in}{4.620000in}}%
\pgfusepath{clip}%
\pgfsetrectcap%
\pgfsetroundjoin%
\pgfsetlinewidth{1.003750pt}%
\definecolor{currentstroke}{rgb}{0.000000,0.000000,0.000000}%
\pgfsetstrokecolor{currentstroke}%
\pgfsetdash{}{0pt}%
\pgfpathmoveto{\pgfqpoint{2.800000in}{3.681932in}}%
\pgfpathlineto{\pgfqpoint{2.800000in}{5.053071in}}%
\pgfusepath{stroke}%
\end{pgfscope}%
\begin{pgfscope}%
\pgfpathrectangle{\pgfqpoint{1.000000in}{0.720000in}}{\pgfqpoint{4.800000in}{4.620000in}}%
\pgfusepath{clip}%
\pgfsetrectcap%
\pgfsetroundjoin%
\pgfsetlinewidth{1.003750pt}%
\definecolor{currentstroke}{rgb}{0.000000,0.000000,0.000000}%
\pgfsetstrokecolor{currentstroke}%
\pgfsetdash{}{0pt}%
\pgfpathmoveto{\pgfqpoint{2.762500in}{1.500258in}}%
\pgfpathlineto{\pgfqpoint{2.837500in}{1.500258in}}%
\pgfusepath{stroke}%
\end{pgfscope}%
\begin{pgfscope}%
\pgfpathrectangle{\pgfqpoint{1.000000in}{0.720000in}}{\pgfqpoint{4.800000in}{4.620000in}}%
\pgfusepath{clip}%
\pgfsetrectcap%
\pgfsetroundjoin%
\pgfsetlinewidth{1.003750pt}%
\definecolor{currentstroke}{rgb}{0.000000,0.000000,0.000000}%
\pgfsetstrokecolor{currentstroke}%
\pgfsetdash{}{0pt}%
\pgfpathmoveto{\pgfqpoint{2.762500in}{5.053071in}}%
\pgfpathlineto{\pgfqpoint{2.837500in}{5.053071in}}%
\pgfusepath{stroke}%
\end{pgfscope}%
\begin{pgfscope}%
\pgfpathrectangle{\pgfqpoint{1.000000in}{0.720000in}}{\pgfqpoint{4.800000in}{4.620000in}}%
\pgfusepath{clip}%
\pgfsetrectcap%
\pgfsetroundjoin%
\pgfsetlinewidth{1.003750pt}%
\definecolor{currentstroke}{rgb}{0.000000,0.000000,0.000000}%
\pgfsetstrokecolor{currentstroke}%
\pgfsetdash{}{0pt}%
\pgfpathmoveto{\pgfqpoint{3.100000in}{2.887424in}}%
\pgfpathlineto{\pgfqpoint{3.100000in}{1.647440in}}%
\pgfusepath{stroke}%
\end{pgfscope}%
\begin{pgfscope}%
\pgfpathrectangle{\pgfqpoint{1.000000in}{0.720000in}}{\pgfqpoint{4.800000in}{4.620000in}}%
\pgfusepath{clip}%
\pgfsetrectcap%
\pgfsetroundjoin%
\pgfsetlinewidth{1.003750pt}%
\definecolor{currentstroke}{rgb}{0.000000,0.000000,0.000000}%
\pgfsetstrokecolor{currentstroke}%
\pgfsetdash{}{0pt}%
\pgfpathmoveto{\pgfqpoint{3.100000in}{3.737343in}}%
\pgfpathlineto{\pgfqpoint{3.100000in}{5.003119in}}%
\pgfusepath{stroke}%
\end{pgfscope}%
\begin{pgfscope}%
\pgfpathrectangle{\pgfqpoint{1.000000in}{0.720000in}}{\pgfqpoint{4.800000in}{4.620000in}}%
\pgfusepath{clip}%
\pgfsetrectcap%
\pgfsetroundjoin%
\pgfsetlinewidth{1.003750pt}%
\definecolor{currentstroke}{rgb}{0.000000,0.000000,0.000000}%
\pgfsetstrokecolor{currentstroke}%
\pgfsetdash{}{0pt}%
\pgfpathmoveto{\pgfqpoint{3.062500in}{1.647440in}}%
\pgfpathlineto{\pgfqpoint{3.137500in}{1.647440in}}%
\pgfusepath{stroke}%
\end{pgfscope}%
\begin{pgfscope}%
\pgfpathrectangle{\pgfqpoint{1.000000in}{0.720000in}}{\pgfqpoint{4.800000in}{4.620000in}}%
\pgfusepath{clip}%
\pgfsetrectcap%
\pgfsetroundjoin%
\pgfsetlinewidth{1.003750pt}%
\definecolor{currentstroke}{rgb}{0.000000,0.000000,0.000000}%
\pgfsetstrokecolor{currentstroke}%
\pgfsetdash{}{0pt}%
\pgfpathmoveto{\pgfqpoint{3.062500in}{5.003119in}}%
\pgfpathlineto{\pgfqpoint{3.137500in}{5.003119in}}%
\pgfusepath{stroke}%
\end{pgfscope}%
\begin{pgfscope}%
\pgfpathrectangle{\pgfqpoint{1.000000in}{0.720000in}}{\pgfqpoint{4.800000in}{4.620000in}}%
\pgfusepath{clip}%
\pgfsetrectcap%
\pgfsetroundjoin%
\pgfsetlinewidth{1.003750pt}%
\definecolor{currentstroke}{rgb}{0.000000,0.000000,0.000000}%
\pgfsetstrokecolor{currentstroke}%
\pgfsetdash{}{0pt}%
\pgfpathmoveto{\pgfqpoint{3.400000in}{2.913642in}}%
\pgfpathlineto{\pgfqpoint{3.400000in}{2.092707in}}%
\pgfusepath{stroke}%
\end{pgfscope}%
\begin{pgfscope}%
\pgfpathrectangle{\pgfqpoint{1.000000in}{0.720000in}}{\pgfqpoint{4.800000in}{4.620000in}}%
\pgfusepath{clip}%
\pgfsetrectcap%
\pgfsetroundjoin%
\pgfsetlinewidth{1.003750pt}%
\definecolor{currentstroke}{rgb}{0.000000,0.000000,0.000000}%
\pgfsetstrokecolor{currentstroke}%
\pgfsetdash{}{0pt}%
\pgfpathmoveto{\pgfqpoint{3.400000in}{3.674032in}}%
\pgfpathlineto{\pgfqpoint{3.400000in}{4.741288in}}%
\pgfusepath{stroke}%
\end{pgfscope}%
\begin{pgfscope}%
\pgfpathrectangle{\pgfqpoint{1.000000in}{0.720000in}}{\pgfqpoint{4.800000in}{4.620000in}}%
\pgfusepath{clip}%
\pgfsetrectcap%
\pgfsetroundjoin%
\pgfsetlinewidth{1.003750pt}%
\definecolor{currentstroke}{rgb}{0.000000,0.000000,0.000000}%
\pgfsetstrokecolor{currentstroke}%
\pgfsetdash{}{0pt}%
\pgfpathmoveto{\pgfqpoint{3.362500in}{2.092707in}}%
\pgfpathlineto{\pgfqpoint{3.437500in}{2.092707in}}%
\pgfusepath{stroke}%
\end{pgfscope}%
\begin{pgfscope}%
\pgfpathrectangle{\pgfqpoint{1.000000in}{0.720000in}}{\pgfqpoint{4.800000in}{4.620000in}}%
\pgfusepath{clip}%
\pgfsetrectcap%
\pgfsetroundjoin%
\pgfsetlinewidth{1.003750pt}%
\definecolor{currentstroke}{rgb}{0.000000,0.000000,0.000000}%
\pgfsetstrokecolor{currentstroke}%
\pgfsetdash{}{0pt}%
\pgfpathmoveto{\pgfqpoint{3.362500in}{4.741288in}}%
\pgfpathlineto{\pgfqpoint{3.437500in}{4.741288in}}%
\pgfusepath{stroke}%
\end{pgfscope}%
\begin{pgfscope}%
\pgfpathrectangle{\pgfqpoint{1.000000in}{0.720000in}}{\pgfqpoint{4.800000in}{4.620000in}}%
\pgfusepath{clip}%
\pgfsetrectcap%
\pgfsetroundjoin%
\pgfsetlinewidth{1.003750pt}%
\definecolor{currentstroke}{rgb}{0.000000,0.000000,0.000000}%
\pgfsetstrokecolor{currentstroke}%
\pgfsetdash{}{0pt}%
\pgfpathmoveto{\pgfqpoint{3.700000in}{2.840050in}}%
\pgfpathlineto{\pgfqpoint{3.700000in}{2.194448in}}%
\pgfusepath{stroke}%
\end{pgfscope}%
\begin{pgfscope}%
\pgfpathrectangle{\pgfqpoint{1.000000in}{0.720000in}}{\pgfqpoint{4.800000in}{4.620000in}}%
\pgfusepath{clip}%
\pgfsetrectcap%
\pgfsetroundjoin%
\pgfsetlinewidth{1.003750pt}%
\definecolor{currentstroke}{rgb}{0.000000,0.000000,0.000000}%
\pgfsetstrokecolor{currentstroke}%
\pgfsetdash{}{0pt}%
\pgfpathmoveto{\pgfqpoint{3.700000in}{3.655616in}}%
\pgfpathlineto{\pgfqpoint{3.700000in}{4.727067in}}%
\pgfusepath{stroke}%
\end{pgfscope}%
\begin{pgfscope}%
\pgfpathrectangle{\pgfqpoint{1.000000in}{0.720000in}}{\pgfqpoint{4.800000in}{4.620000in}}%
\pgfusepath{clip}%
\pgfsetrectcap%
\pgfsetroundjoin%
\pgfsetlinewidth{1.003750pt}%
\definecolor{currentstroke}{rgb}{0.000000,0.000000,0.000000}%
\pgfsetstrokecolor{currentstroke}%
\pgfsetdash{}{0pt}%
\pgfpathmoveto{\pgfqpoint{3.662500in}{2.194448in}}%
\pgfpathlineto{\pgfqpoint{3.737500in}{2.194448in}}%
\pgfusepath{stroke}%
\end{pgfscope}%
\begin{pgfscope}%
\pgfpathrectangle{\pgfqpoint{1.000000in}{0.720000in}}{\pgfqpoint{4.800000in}{4.620000in}}%
\pgfusepath{clip}%
\pgfsetrectcap%
\pgfsetroundjoin%
\pgfsetlinewidth{1.003750pt}%
\definecolor{currentstroke}{rgb}{0.000000,0.000000,0.000000}%
\pgfsetstrokecolor{currentstroke}%
\pgfsetdash{}{0pt}%
\pgfpathmoveto{\pgfqpoint{3.662500in}{4.727067in}}%
\pgfpathlineto{\pgfqpoint{3.737500in}{4.727067in}}%
\pgfusepath{stroke}%
\end{pgfscope}%
\begin{pgfscope}%
\pgfpathrectangle{\pgfqpoint{1.000000in}{0.720000in}}{\pgfqpoint{4.800000in}{4.620000in}}%
\pgfusepath{clip}%
\pgfsetrectcap%
\pgfsetroundjoin%
\pgfsetlinewidth{1.003750pt}%
\definecolor{currentstroke}{rgb}{0.000000,0.000000,0.000000}%
\pgfsetstrokecolor{currentstroke}%
\pgfsetdash{}{0pt}%
\pgfpathmoveto{\pgfqpoint{4.000000in}{2.930521in}}%
\pgfpathlineto{\pgfqpoint{4.000000in}{2.488672in}}%
\pgfusepath{stroke}%
\end{pgfscope}%
\begin{pgfscope}%
\pgfpathrectangle{\pgfqpoint{1.000000in}{0.720000in}}{\pgfqpoint{4.800000in}{4.620000in}}%
\pgfusepath{clip}%
\pgfsetrectcap%
\pgfsetroundjoin%
\pgfsetlinewidth{1.003750pt}%
\definecolor{currentstroke}{rgb}{0.000000,0.000000,0.000000}%
\pgfsetstrokecolor{currentstroke}%
\pgfsetdash{}{0pt}%
\pgfpathmoveto{\pgfqpoint{4.000000in}{3.490054in}}%
\pgfpathlineto{\pgfqpoint{4.000000in}{4.255666in}}%
\pgfusepath{stroke}%
\end{pgfscope}%
\begin{pgfscope}%
\pgfpathrectangle{\pgfqpoint{1.000000in}{0.720000in}}{\pgfqpoint{4.800000in}{4.620000in}}%
\pgfusepath{clip}%
\pgfsetrectcap%
\pgfsetroundjoin%
\pgfsetlinewidth{1.003750pt}%
\definecolor{currentstroke}{rgb}{0.000000,0.000000,0.000000}%
\pgfsetstrokecolor{currentstroke}%
\pgfsetdash{}{0pt}%
\pgfpathmoveto{\pgfqpoint{3.962500in}{2.488672in}}%
\pgfpathlineto{\pgfqpoint{4.037500in}{2.488672in}}%
\pgfusepath{stroke}%
\end{pgfscope}%
\begin{pgfscope}%
\pgfpathrectangle{\pgfqpoint{1.000000in}{0.720000in}}{\pgfqpoint{4.800000in}{4.620000in}}%
\pgfusepath{clip}%
\pgfsetrectcap%
\pgfsetroundjoin%
\pgfsetlinewidth{1.003750pt}%
\definecolor{currentstroke}{rgb}{0.000000,0.000000,0.000000}%
\pgfsetstrokecolor{currentstroke}%
\pgfsetdash{}{0pt}%
\pgfpathmoveto{\pgfqpoint{3.962500in}{4.255666in}}%
\pgfpathlineto{\pgfqpoint{4.037500in}{4.255666in}}%
\pgfusepath{stroke}%
\end{pgfscope}%
\begin{pgfscope}%
\pgfpathrectangle{\pgfqpoint{1.000000in}{0.720000in}}{\pgfqpoint{4.800000in}{4.620000in}}%
\pgfusepath{clip}%
\pgfsetrectcap%
\pgfsetroundjoin%
\pgfsetlinewidth{1.003750pt}%
\definecolor{currentstroke}{rgb}{0.000000,0.000000,0.000000}%
\pgfsetstrokecolor{currentstroke}%
\pgfsetdash{}{0pt}%
\pgfpathmoveto{\pgfqpoint{4.300000in}{2.918050in}}%
\pgfpathlineto{\pgfqpoint{4.300000in}{2.432390in}}%
\pgfusepath{stroke}%
\end{pgfscope}%
\begin{pgfscope}%
\pgfpathrectangle{\pgfqpoint{1.000000in}{0.720000in}}{\pgfqpoint{4.800000in}{4.620000in}}%
\pgfusepath{clip}%
\pgfsetrectcap%
\pgfsetroundjoin%
\pgfsetlinewidth{1.003750pt}%
\definecolor{currentstroke}{rgb}{0.000000,0.000000,0.000000}%
\pgfsetstrokecolor{currentstroke}%
\pgfsetdash{}{0pt}%
\pgfpathmoveto{\pgfqpoint{4.300000in}{3.613869in}}%
\pgfpathlineto{\pgfqpoint{4.300000in}{4.557049in}}%
\pgfusepath{stroke}%
\end{pgfscope}%
\begin{pgfscope}%
\pgfpathrectangle{\pgfqpoint{1.000000in}{0.720000in}}{\pgfqpoint{4.800000in}{4.620000in}}%
\pgfusepath{clip}%
\pgfsetrectcap%
\pgfsetroundjoin%
\pgfsetlinewidth{1.003750pt}%
\definecolor{currentstroke}{rgb}{0.000000,0.000000,0.000000}%
\pgfsetstrokecolor{currentstroke}%
\pgfsetdash{}{0pt}%
\pgfpathmoveto{\pgfqpoint{4.262500in}{2.432390in}}%
\pgfpathlineto{\pgfqpoint{4.337500in}{2.432390in}}%
\pgfusepath{stroke}%
\end{pgfscope}%
\begin{pgfscope}%
\pgfpathrectangle{\pgfqpoint{1.000000in}{0.720000in}}{\pgfqpoint{4.800000in}{4.620000in}}%
\pgfusepath{clip}%
\pgfsetrectcap%
\pgfsetroundjoin%
\pgfsetlinewidth{1.003750pt}%
\definecolor{currentstroke}{rgb}{0.000000,0.000000,0.000000}%
\pgfsetstrokecolor{currentstroke}%
\pgfsetdash{}{0pt}%
\pgfpathmoveto{\pgfqpoint{4.262500in}{4.557049in}}%
\pgfpathlineto{\pgfqpoint{4.337500in}{4.557049in}}%
\pgfusepath{stroke}%
\end{pgfscope}%
\begin{pgfscope}%
\pgfpathrectangle{\pgfqpoint{1.000000in}{0.720000in}}{\pgfqpoint{4.800000in}{4.620000in}}%
\pgfusepath{clip}%
\pgfsetrectcap%
\pgfsetroundjoin%
\pgfsetlinewidth{1.003750pt}%
\definecolor{currentstroke}{rgb}{0.000000,0.000000,0.000000}%
\pgfsetstrokecolor{currentstroke}%
\pgfsetdash{}{0pt}%
\pgfpathmoveto{\pgfqpoint{4.600000in}{3.632982in}}%
\pgfpathlineto{\pgfqpoint{4.600000in}{2.870742in}}%
\pgfusepath{stroke}%
\end{pgfscope}%
\begin{pgfscope}%
\pgfpathrectangle{\pgfqpoint{1.000000in}{0.720000in}}{\pgfqpoint{4.800000in}{4.620000in}}%
\pgfusepath{clip}%
\pgfsetrectcap%
\pgfsetroundjoin%
\pgfsetlinewidth{1.003750pt}%
\definecolor{currentstroke}{rgb}{0.000000,0.000000,0.000000}%
\pgfsetstrokecolor{currentstroke}%
\pgfsetdash{}{0pt}%
\pgfpathmoveto{\pgfqpoint{4.600000in}{4.241231in}}%
\pgfpathlineto{\pgfqpoint{4.600000in}{4.885439in}}%
\pgfusepath{stroke}%
\end{pgfscope}%
\begin{pgfscope}%
\pgfpathrectangle{\pgfqpoint{1.000000in}{0.720000in}}{\pgfqpoint{4.800000in}{4.620000in}}%
\pgfusepath{clip}%
\pgfsetrectcap%
\pgfsetroundjoin%
\pgfsetlinewidth{1.003750pt}%
\definecolor{currentstroke}{rgb}{0.000000,0.000000,0.000000}%
\pgfsetstrokecolor{currentstroke}%
\pgfsetdash{}{0pt}%
\pgfpathmoveto{\pgfqpoint{4.562500in}{2.870742in}}%
\pgfpathlineto{\pgfqpoint{4.637500in}{2.870742in}}%
\pgfusepath{stroke}%
\end{pgfscope}%
\begin{pgfscope}%
\pgfpathrectangle{\pgfqpoint{1.000000in}{0.720000in}}{\pgfqpoint{4.800000in}{4.620000in}}%
\pgfusepath{clip}%
\pgfsetrectcap%
\pgfsetroundjoin%
\pgfsetlinewidth{1.003750pt}%
\definecolor{currentstroke}{rgb}{0.000000,0.000000,0.000000}%
\pgfsetstrokecolor{currentstroke}%
\pgfsetdash{}{0pt}%
\pgfpathmoveto{\pgfqpoint{4.562500in}{4.885439in}}%
\pgfpathlineto{\pgfqpoint{4.637500in}{4.885439in}}%
\pgfusepath{stroke}%
\end{pgfscope}%
\begin{pgfscope}%
\pgfpathrectangle{\pgfqpoint{1.000000in}{0.720000in}}{\pgfqpoint{4.800000in}{4.620000in}}%
\pgfusepath{clip}%
\pgfsetrectcap%
\pgfsetroundjoin%
\pgfsetlinewidth{1.003750pt}%
\definecolor{currentstroke}{rgb}{0.000000,0.000000,0.000000}%
\pgfsetstrokecolor{currentstroke}%
\pgfsetdash{}{0pt}%
\pgfpathmoveto{\pgfqpoint{4.900000in}{2.631498in}}%
\pgfpathlineto{\pgfqpoint{4.900000in}{2.631498in}}%
\pgfusepath{stroke}%
\end{pgfscope}%
\begin{pgfscope}%
\pgfpathrectangle{\pgfqpoint{1.000000in}{0.720000in}}{\pgfqpoint{4.800000in}{4.620000in}}%
\pgfusepath{clip}%
\pgfsetrectcap%
\pgfsetroundjoin%
\pgfsetlinewidth{1.003750pt}%
\definecolor{currentstroke}{rgb}{0.000000,0.000000,0.000000}%
\pgfsetstrokecolor{currentstroke}%
\pgfsetdash{}{0pt}%
\pgfpathmoveto{\pgfqpoint{4.900000in}{2.631498in}}%
\pgfpathlineto{\pgfqpoint{4.900000in}{2.631498in}}%
\pgfusepath{stroke}%
\end{pgfscope}%
\begin{pgfscope}%
\pgfpathrectangle{\pgfqpoint{1.000000in}{0.720000in}}{\pgfqpoint{4.800000in}{4.620000in}}%
\pgfusepath{clip}%
\pgfsetrectcap%
\pgfsetroundjoin%
\pgfsetlinewidth{1.003750pt}%
\definecolor{currentstroke}{rgb}{0.000000,0.000000,0.000000}%
\pgfsetstrokecolor{currentstroke}%
\pgfsetdash{}{0pt}%
\pgfpathmoveto{\pgfqpoint{4.862500in}{2.631498in}}%
\pgfpathlineto{\pgfqpoint{4.937500in}{2.631498in}}%
\pgfusepath{stroke}%
\end{pgfscope}%
\begin{pgfscope}%
\pgfpathrectangle{\pgfqpoint{1.000000in}{0.720000in}}{\pgfqpoint{4.800000in}{4.620000in}}%
\pgfusepath{clip}%
\pgfsetrectcap%
\pgfsetroundjoin%
\pgfsetlinewidth{1.003750pt}%
\definecolor{currentstroke}{rgb}{0.000000,0.000000,0.000000}%
\pgfsetstrokecolor{currentstroke}%
\pgfsetdash{}{0pt}%
\pgfpathmoveto{\pgfqpoint{4.862500in}{2.631498in}}%
\pgfpathlineto{\pgfqpoint{4.937500in}{2.631498in}}%
\pgfusepath{stroke}%
\end{pgfscope}%
\begin{pgfscope}%
\pgfpathrectangle{\pgfqpoint{1.000000in}{0.720000in}}{\pgfqpoint{4.800000in}{4.620000in}}%
\pgfusepath{clip}%
\pgfsetrectcap%
\pgfsetroundjoin%
\pgfsetlinewidth{1.003750pt}%
\definecolor{currentstroke}{rgb}{0.000000,0.000000,0.000000}%
\pgfsetstrokecolor{currentstroke}%
\pgfsetdash{}{0pt}%
\pgfpathmoveto{\pgfqpoint{5.200000in}{3.483259in}}%
\pgfpathlineto{\pgfqpoint{5.200000in}{3.483259in}}%
\pgfusepath{stroke}%
\end{pgfscope}%
\begin{pgfscope}%
\pgfpathrectangle{\pgfqpoint{1.000000in}{0.720000in}}{\pgfqpoint{4.800000in}{4.620000in}}%
\pgfusepath{clip}%
\pgfsetrectcap%
\pgfsetroundjoin%
\pgfsetlinewidth{1.003750pt}%
\definecolor{currentstroke}{rgb}{0.000000,0.000000,0.000000}%
\pgfsetstrokecolor{currentstroke}%
\pgfsetdash{}{0pt}%
\pgfpathmoveto{\pgfqpoint{5.200000in}{3.483259in}}%
\pgfpathlineto{\pgfqpoint{5.200000in}{3.483259in}}%
\pgfusepath{stroke}%
\end{pgfscope}%
\begin{pgfscope}%
\pgfpathrectangle{\pgfqpoint{1.000000in}{0.720000in}}{\pgfqpoint{4.800000in}{4.620000in}}%
\pgfusepath{clip}%
\pgfsetrectcap%
\pgfsetroundjoin%
\pgfsetlinewidth{1.003750pt}%
\definecolor{currentstroke}{rgb}{0.000000,0.000000,0.000000}%
\pgfsetstrokecolor{currentstroke}%
\pgfsetdash{}{0pt}%
\pgfpathmoveto{\pgfqpoint{5.162500in}{3.483259in}}%
\pgfpathlineto{\pgfqpoint{5.237500in}{3.483259in}}%
\pgfusepath{stroke}%
\end{pgfscope}%
\begin{pgfscope}%
\pgfpathrectangle{\pgfqpoint{1.000000in}{0.720000in}}{\pgfqpoint{4.800000in}{4.620000in}}%
\pgfusepath{clip}%
\pgfsetrectcap%
\pgfsetroundjoin%
\pgfsetlinewidth{1.003750pt}%
\definecolor{currentstroke}{rgb}{0.000000,0.000000,0.000000}%
\pgfsetstrokecolor{currentstroke}%
\pgfsetdash{}{0pt}%
\pgfpathmoveto{\pgfqpoint{5.162500in}{3.483259in}}%
\pgfpathlineto{\pgfqpoint{5.237500in}{3.483259in}}%
\pgfusepath{stroke}%
\end{pgfscope}%
\begin{pgfscope}%
\pgfpathrectangle{\pgfqpoint{1.000000in}{0.720000in}}{\pgfqpoint{4.800000in}{4.620000in}}%
\pgfusepath{clip}%
\pgfsetrectcap%
\pgfsetroundjoin%
\pgfsetlinewidth{1.003750pt}%
\definecolor{currentstroke}{rgb}{0.000000,0.000000,0.000000}%
\pgfsetstrokecolor{currentstroke}%
\pgfsetdash{}{0pt}%
\pgfpathmoveto{\pgfqpoint{5.500000in}{3.271319in}}%
\pgfpathlineto{\pgfqpoint{5.500000in}{3.271319in}}%
\pgfusepath{stroke}%
\end{pgfscope}%
\begin{pgfscope}%
\pgfpathrectangle{\pgfqpoint{1.000000in}{0.720000in}}{\pgfqpoint{4.800000in}{4.620000in}}%
\pgfusepath{clip}%
\pgfsetrectcap%
\pgfsetroundjoin%
\pgfsetlinewidth{1.003750pt}%
\definecolor{currentstroke}{rgb}{0.000000,0.000000,0.000000}%
\pgfsetstrokecolor{currentstroke}%
\pgfsetdash{}{0pt}%
\pgfpathmoveto{\pgfqpoint{5.500000in}{3.271319in}}%
\pgfpathlineto{\pgfqpoint{5.500000in}{3.271319in}}%
\pgfusepath{stroke}%
\end{pgfscope}%
\begin{pgfscope}%
\pgfpathrectangle{\pgfqpoint{1.000000in}{0.720000in}}{\pgfqpoint{4.800000in}{4.620000in}}%
\pgfusepath{clip}%
\pgfsetrectcap%
\pgfsetroundjoin%
\pgfsetlinewidth{1.003750pt}%
\definecolor{currentstroke}{rgb}{0.000000,0.000000,0.000000}%
\pgfsetstrokecolor{currentstroke}%
\pgfsetdash{}{0pt}%
\pgfpathmoveto{\pgfqpoint{5.462500in}{3.271319in}}%
\pgfpathlineto{\pgfqpoint{5.537500in}{3.271319in}}%
\pgfusepath{stroke}%
\end{pgfscope}%
\begin{pgfscope}%
\pgfpathrectangle{\pgfqpoint{1.000000in}{0.720000in}}{\pgfqpoint{4.800000in}{4.620000in}}%
\pgfusepath{clip}%
\pgfsetrectcap%
\pgfsetroundjoin%
\pgfsetlinewidth{1.003750pt}%
\definecolor{currentstroke}{rgb}{0.000000,0.000000,0.000000}%
\pgfsetstrokecolor{currentstroke}%
\pgfsetdash{}{0pt}%
\pgfpathmoveto{\pgfqpoint{5.462500in}{3.271319in}}%
\pgfpathlineto{\pgfqpoint{5.537500in}{3.271319in}}%
\pgfusepath{stroke}%
\end{pgfscope}%
\begin{pgfscope}%
\pgfpathrectangle{\pgfqpoint{1.000000in}{0.720000in}}{\pgfqpoint{4.800000in}{4.620000in}}%
\pgfusepath{clip}%
\pgfsetrectcap%
\pgfsetroundjoin%
\pgfsetlinewidth{2.007500pt}%
\definecolor{currentstroke}{rgb}{0.121569,0.466667,0.705882}%
\pgfsetstrokecolor{currentstroke}%
\pgfsetdash{}{0pt}%
\pgfpathmoveto{\pgfqpoint{1.300000in}{1.882181in}}%
\pgfpathlineto{\pgfqpoint{1.600000in}{2.628237in}}%
\pgfpathlineto{\pgfqpoint{1.900000in}{2.895617in}}%
\pgfpathlineto{\pgfqpoint{2.200000in}{3.051319in}}%
\pgfpathlineto{\pgfqpoint{2.500000in}{3.175738in}}%
\pgfpathlineto{\pgfqpoint{2.800000in}{3.151728in}}%
\pgfpathlineto{\pgfqpoint{3.100000in}{3.287234in}}%
\pgfpathlineto{\pgfqpoint{3.400000in}{3.276514in}}%
\pgfpathlineto{\pgfqpoint{3.700000in}{3.237637in}}%
\pgfpathlineto{\pgfqpoint{4.000000in}{3.273046in}}%
\pgfpathlineto{\pgfqpoint{4.300000in}{3.207480in}}%
\pgfpathlineto{\pgfqpoint{4.600000in}{4.069136in}}%
\pgfpathlineto{\pgfqpoint{4.900000in}{2.631498in}}%
\pgfpathlineto{\pgfqpoint{5.200000in}{3.483259in}}%
\pgfpathlineto{\pgfqpoint{5.500000in}{3.271319in}}%
\pgfusepath{stroke}%
\end{pgfscope}%
\begin{pgfscope}%
\pgfpathrectangle{\pgfqpoint{1.000000in}{0.720000in}}{\pgfqpoint{4.800000in}{4.620000in}}%
\pgfusepath{clip}%
\pgfsetbuttcap%
\pgfsetmiterjoin%
\definecolor{currentfill}{rgb}{0.121569,0.466667,0.705882}%
\pgfsetfillcolor{currentfill}%
\pgfsetlinewidth{1.003750pt}%
\definecolor{currentstroke}{rgb}{0.000000,0.000000,0.000000}%
\pgfsetstrokecolor{currentstroke}%
\pgfsetdash{}{0pt}%
\pgfpathmoveto{\pgfqpoint{1.225000in}{1.275034in}}%
\pgfpathlineto{\pgfqpoint{1.375000in}{1.275034in}}%
\pgfpathlineto{\pgfqpoint{1.375000in}{2.458391in}}%
\pgfpathlineto{\pgfqpoint{1.225000in}{2.458391in}}%
\pgfpathlineto{\pgfqpoint{1.225000in}{1.275034in}}%
\pgfpathclose%
\pgfusepath{stroke,fill}%
\end{pgfscope}%
\begin{pgfscope}%
\pgfpathrectangle{\pgfqpoint{1.000000in}{0.720000in}}{\pgfqpoint{4.800000in}{4.620000in}}%
\pgfusepath{clip}%
\pgfsetbuttcap%
\pgfsetmiterjoin%
\definecolor{currentfill}{rgb}{0.121569,0.466667,0.705882}%
\pgfsetfillcolor{currentfill}%
\pgfsetlinewidth{1.003750pt}%
\definecolor{currentstroke}{rgb}{0.000000,0.000000,0.000000}%
\pgfsetstrokecolor{currentstroke}%
\pgfsetdash{}{0pt}%
\pgfpathmoveto{\pgfqpoint{1.525000in}{2.052040in}}%
\pgfpathlineto{\pgfqpoint{1.675000in}{2.052040in}}%
\pgfpathlineto{\pgfqpoint{1.675000in}{3.291574in}}%
\pgfpathlineto{\pgfqpoint{1.525000in}{3.291574in}}%
\pgfpathlineto{\pgfqpoint{1.525000in}{2.052040in}}%
\pgfpathclose%
\pgfusepath{stroke,fill}%
\end{pgfscope}%
\begin{pgfscope}%
\pgfpathrectangle{\pgfqpoint{1.000000in}{0.720000in}}{\pgfqpoint{4.800000in}{4.620000in}}%
\pgfusepath{clip}%
\pgfsetbuttcap%
\pgfsetmiterjoin%
\definecolor{currentfill}{rgb}{0.121569,0.466667,0.705882}%
\pgfsetfillcolor{currentfill}%
\pgfsetlinewidth{1.003750pt}%
\definecolor{currentstroke}{rgb}{0.000000,0.000000,0.000000}%
\pgfsetstrokecolor{currentstroke}%
\pgfsetdash{}{0pt}%
\pgfpathmoveto{\pgfqpoint{1.825000in}{2.377197in}}%
\pgfpathlineto{\pgfqpoint{1.975000in}{2.377197in}}%
\pgfpathlineto{\pgfqpoint{1.975000in}{3.484357in}}%
\pgfpathlineto{\pgfqpoint{1.825000in}{3.484357in}}%
\pgfpathlineto{\pgfqpoint{1.825000in}{2.377197in}}%
\pgfpathclose%
\pgfusepath{stroke,fill}%
\end{pgfscope}%
\begin{pgfscope}%
\pgfpathrectangle{\pgfqpoint{1.000000in}{0.720000in}}{\pgfqpoint{4.800000in}{4.620000in}}%
\pgfusepath{clip}%
\pgfsetbuttcap%
\pgfsetmiterjoin%
\definecolor{currentfill}{rgb}{0.121569,0.466667,0.705882}%
\pgfsetfillcolor{currentfill}%
\pgfsetlinewidth{1.003750pt}%
\definecolor{currentstroke}{rgb}{0.000000,0.000000,0.000000}%
\pgfsetstrokecolor{currentstroke}%
\pgfsetdash{}{0pt}%
\pgfpathmoveto{\pgfqpoint{2.125000in}{2.587973in}}%
\pgfpathlineto{\pgfqpoint{2.275000in}{2.587973in}}%
\pgfpathlineto{\pgfqpoint{2.275000in}{3.606409in}}%
\pgfpathlineto{\pgfqpoint{2.125000in}{3.606409in}}%
\pgfpathlineto{\pgfqpoint{2.125000in}{2.587973in}}%
\pgfpathclose%
\pgfusepath{stroke,fill}%
\end{pgfscope}%
\begin{pgfscope}%
\pgfpathrectangle{\pgfqpoint{1.000000in}{0.720000in}}{\pgfqpoint{4.800000in}{4.620000in}}%
\pgfusepath{clip}%
\pgfsetbuttcap%
\pgfsetmiterjoin%
\definecolor{currentfill}{rgb}{0.121569,0.466667,0.705882}%
\pgfsetfillcolor{currentfill}%
\pgfsetlinewidth{1.003750pt}%
\definecolor{currentstroke}{rgb}{0.000000,0.000000,0.000000}%
\pgfsetstrokecolor{currentstroke}%
\pgfsetdash{}{0pt}%
\pgfpathmoveto{\pgfqpoint{2.425000in}{2.733762in}}%
\pgfpathlineto{\pgfqpoint{2.575000in}{2.733762in}}%
\pgfpathlineto{\pgfqpoint{2.575000in}{3.663233in}}%
\pgfpathlineto{\pgfqpoint{2.425000in}{3.663233in}}%
\pgfpathlineto{\pgfqpoint{2.425000in}{2.733762in}}%
\pgfpathclose%
\pgfusepath{stroke,fill}%
\end{pgfscope}%
\begin{pgfscope}%
\pgfpathrectangle{\pgfqpoint{1.000000in}{0.720000in}}{\pgfqpoint{4.800000in}{4.620000in}}%
\pgfusepath{clip}%
\pgfsetbuttcap%
\pgfsetmiterjoin%
\definecolor{currentfill}{rgb}{0.121569,0.466667,0.705882}%
\pgfsetfillcolor{currentfill}%
\pgfsetlinewidth{1.003750pt}%
\definecolor{currentstroke}{rgb}{0.000000,0.000000,0.000000}%
\pgfsetstrokecolor{currentstroke}%
\pgfsetdash{}{0pt}%
\pgfpathmoveto{\pgfqpoint{2.725000in}{2.764618in}}%
\pgfpathlineto{\pgfqpoint{2.875000in}{2.764618in}}%
\pgfpathlineto{\pgfqpoint{2.875000in}{3.681932in}}%
\pgfpathlineto{\pgfqpoint{2.725000in}{3.681932in}}%
\pgfpathlineto{\pgfqpoint{2.725000in}{2.764618in}}%
\pgfpathclose%
\pgfusepath{stroke,fill}%
\end{pgfscope}%
\begin{pgfscope}%
\pgfpathrectangle{\pgfqpoint{1.000000in}{0.720000in}}{\pgfqpoint{4.800000in}{4.620000in}}%
\pgfusepath{clip}%
\pgfsetbuttcap%
\pgfsetmiterjoin%
\definecolor{currentfill}{rgb}{0.121569,0.466667,0.705882}%
\pgfsetfillcolor{currentfill}%
\pgfsetlinewidth{1.003750pt}%
\definecolor{currentstroke}{rgb}{0.000000,0.000000,0.000000}%
\pgfsetstrokecolor{currentstroke}%
\pgfsetdash{}{0pt}%
\pgfpathmoveto{\pgfqpoint{3.025000in}{2.887424in}}%
\pgfpathlineto{\pgfqpoint{3.175000in}{2.887424in}}%
\pgfpathlineto{\pgfqpoint{3.175000in}{3.737343in}}%
\pgfpathlineto{\pgfqpoint{3.025000in}{3.737343in}}%
\pgfpathlineto{\pgfqpoint{3.025000in}{2.887424in}}%
\pgfpathclose%
\pgfusepath{stroke,fill}%
\end{pgfscope}%
\begin{pgfscope}%
\pgfpathrectangle{\pgfqpoint{1.000000in}{0.720000in}}{\pgfqpoint{4.800000in}{4.620000in}}%
\pgfusepath{clip}%
\pgfsetbuttcap%
\pgfsetmiterjoin%
\definecolor{currentfill}{rgb}{0.121569,0.466667,0.705882}%
\pgfsetfillcolor{currentfill}%
\pgfsetlinewidth{1.003750pt}%
\definecolor{currentstroke}{rgb}{0.000000,0.000000,0.000000}%
\pgfsetstrokecolor{currentstroke}%
\pgfsetdash{}{0pt}%
\pgfpathmoveto{\pgfqpoint{3.325000in}{2.913642in}}%
\pgfpathlineto{\pgfqpoint{3.475000in}{2.913642in}}%
\pgfpathlineto{\pgfqpoint{3.475000in}{3.674032in}}%
\pgfpathlineto{\pgfqpoint{3.325000in}{3.674032in}}%
\pgfpathlineto{\pgfqpoint{3.325000in}{2.913642in}}%
\pgfpathclose%
\pgfusepath{stroke,fill}%
\end{pgfscope}%
\begin{pgfscope}%
\pgfpathrectangle{\pgfqpoint{1.000000in}{0.720000in}}{\pgfqpoint{4.800000in}{4.620000in}}%
\pgfusepath{clip}%
\pgfsetbuttcap%
\pgfsetmiterjoin%
\definecolor{currentfill}{rgb}{0.121569,0.466667,0.705882}%
\pgfsetfillcolor{currentfill}%
\pgfsetlinewidth{1.003750pt}%
\definecolor{currentstroke}{rgb}{0.000000,0.000000,0.000000}%
\pgfsetstrokecolor{currentstroke}%
\pgfsetdash{}{0pt}%
\pgfpathmoveto{\pgfqpoint{3.625000in}{2.840050in}}%
\pgfpathlineto{\pgfqpoint{3.775000in}{2.840050in}}%
\pgfpathlineto{\pgfqpoint{3.775000in}{3.655616in}}%
\pgfpathlineto{\pgfqpoint{3.625000in}{3.655616in}}%
\pgfpathlineto{\pgfqpoint{3.625000in}{2.840050in}}%
\pgfpathclose%
\pgfusepath{stroke,fill}%
\end{pgfscope}%
\begin{pgfscope}%
\pgfpathrectangle{\pgfqpoint{1.000000in}{0.720000in}}{\pgfqpoint{4.800000in}{4.620000in}}%
\pgfusepath{clip}%
\pgfsetbuttcap%
\pgfsetmiterjoin%
\definecolor{currentfill}{rgb}{0.121569,0.466667,0.705882}%
\pgfsetfillcolor{currentfill}%
\pgfsetlinewidth{1.003750pt}%
\definecolor{currentstroke}{rgb}{0.000000,0.000000,0.000000}%
\pgfsetstrokecolor{currentstroke}%
\pgfsetdash{}{0pt}%
\pgfpathmoveto{\pgfqpoint{3.925000in}{2.930521in}}%
\pgfpathlineto{\pgfqpoint{4.075000in}{2.930521in}}%
\pgfpathlineto{\pgfqpoint{4.075000in}{3.490054in}}%
\pgfpathlineto{\pgfqpoint{3.925000in}{3.490054in}}%
\pgfpathlineto{\pgfqpoint{3.925000in}{2.930521in}}%
\pgfpathclose%
\pgfusepath{stroke,fill}%
\end{pgfscope}%
\begin{pgfscope}%
\pgfpathrectangle{\pgfqpoint{1.000000in}{0.720000in}}{\pgfqpoint{4.800000in}{4.620000in}}%
\pgfusepath{clip}%
\pgfsetbuttcap%
\pgfsetmiterjoin%
\definecolor{currentfill}{rgb}{0.121569,0.466667,0.705882}%
\pgfsetfillcolor{currentfill}%
\pgfsetlinewidth{1.003750pt}%
\definecolor{currentstroke}{rgb}{0.000000,0.000000,0.000000}%
\pgfsetstrokecolor{currentstroke}%
\pgfsetdash{}{0pt}%
\pgfpathmoveto{\pgfqpoint{4.225000in}{2.918050in}}%
\pgfpathlineto{\pgfqpoint{4.375000in}{2.918050in}}%
\pgfpathlineto{\pgfqpoint{4.375000in}{3.613869in}}%
\pgfpathlineto{\pgfqpoint{4.225000in}{3.613869in}}%
\pgfpathlineto{\pgfqpoint{4.225000in}{2.918050in}}%
\pgfpathclose%
\pgfusepath{stroke,fill}%
\end{pgfscope}%
\begin{pgfscope}%
\pgfpathrectangle{\pgfqpoint{1.000000in}{0.720000in}}{\pgfqpoint{4.800000in}{4.620000in}}%
\pgfusepath{clip}%
\pgfsetbuttcap%
\pgfsetmiterjoin%
\definecolor{currentfill}{rgb}{0.121569,0.466667,0.705882}%
\pgfsetfillcolor{currentfill}%
\pgfsetlinewidth{1.003750pt}%
\definecolor{currentstroke}{rgb}{0.000000,0.000000,0.000000}%
\pgfsetstrokecolor{currentstroke}%
\pgfsetdash{}{0pt}%
\pgfpathmoveto{\pgfqpoint{4.525000in}{3.632982in}}%
\pgfpathlineto{\pgfqpoint{4.675000in}{3.632982in}}%
\pgfpathlineto{\pgfqpoint{4.675000in}{4.241231in}}%
\pgfpathlineto{\pgfqpoint{4.525000in}{4.241231in}}%
\pgfpathlineto{\pgfqpoint{4.525000in}{3.632982in}}%
\pgfpathclose%
\pgfusepath{stroke,fill}%
\end{pgfscope}%
\begin{pgfscope}%
\pgfpathrectangle{\pgfqpoint{1.000000in}{0.720000in}}{\pgfqpoint{4.800000in}{4.620000in}}%
\pgfusepath{clip}%
\pgfsetbuttcap%
\pgfsetmiterjoin%
\definecolor{currentfill}{rgb}{0.121569,0.466667,0.705882}%
\pgfsetfillcolor{currentfill}%
\pgfsetlinewidth{1.003750pt}%
\definecolor{currentstroke}{rgb}{0.000000,0.000000,0.000000}%
\pgfsetstrokecolor{currentstroke}%
\pgfsetdash{}{0pt}%
\pgfpathmoveto{\pgfqpoint{4.825000in}{2.631498in}}%
\pgfpathlineto{\pgfqpoint{4.975000in}{2.631498in}}%
\pgfpathlineto{\pgfqpoint{4.975000in}{2.631498in}}%
\pgfpathlineto{\pgfqpoint{4.825000in}{2.631498in}}%
\pgfpathlineto{\pgfqpoint{4.825000in}{2.631498in}}%
\pgfpathclose%
\pgfusepath{stroke,fill}%
\end{pgfscope}%
\begin{pgfscope}%
\pgfpathrectangle{\pgfqpoint{1.000000in}{0.720000in}}{\pgfqpoint{4.800000in}{4.620000in}}%
\pgfusepath{clip}%
\pgfsetbuttcap%
\pgfsetmiterjoin%
\definecolor{currentfill}{rgb}{0.121569,0.466667,0.705882}%
\pgfsetfillcolor{currentfill}%
\pgfsetlinewidth{1.003750pt}%
\definecolor{currentstroke}{rgb}{0.000000,0.000000,0.000000}%
\pgfsetstrokecolor{currentstroke}%
\pgfsetdash{}{0pt}%
\pgfpathmoveto{\pgfqpoint{5.125000in}{3.483259in}}%
\pgfpathlineto{\pgfqpoint{5.275000in}{3.483259in}}%
\pgfpathlineto{\pgfqpoint{5.275000in}{3.483259in}}%
\pgfpathlineto{\pgfqpoint{5.125000in}{3.483259in}}%
\pgfpathlineto{\pgfqpoint{5.125000in}{3.483259in}}%
\pgfpathclose%
\pgfusepath{stroke,fill}%
\end{pgfscope}%
\begin{pgfscope}%
\pgfpathrectangle{\pgfqpoint{1.000000in}{0.720000in}}{\pgfqpoint{4.800000in}{4.620000in}}%
\pgfusepath{clip}%
\pgfsetbuttcap%
\pgfsetmiterjoin%
\definecolor{currentfill}{rgb}{0.121569,0.466667,0.705882}%
\pgfsetfillcolor{currentfill}%
\pgfsetlinewidth{1.003750pt}%
\definecolor{currentstroke}{rgb}{0.000000,0.000000,0.000000}%
\pgfsetstrokecolor{currentstroke}%
\pgfsetdash{}{0pt}%
\pgfpathmoveto{\pgfqpoint{5.425000in}{3.271319in}}%
\pgfpathlineto{\pgfqpoint{5.575000in}{3.271319in}}%
\pgfpathlineto{\pgfqpoint{5.575000in}{3.271319in}}%
\pgfpathlineto{\pgfqpoint{5.425000in}{3.271319in}}%
\pgfpathlineto{\pgfqpoint{5.425000in}{3.271319in}}%
\pgfpathclose%
\pgfusepath{stroke,fill}%
\end{pgfscope}%
\begin{pgfscope}%
\pgfpathrectangle{\pgfqpoint{1.000000in}{0.720000in}}{\pgfqpoint{4.800000in}{4.620000in}}%
\pgfusepath{clip}%
\pgfsetrectcap%
\pgfsetroundjoin%
\pgfsetlinewidth{1.003750pt}%
\definecolor{currentstroke}{rgb}{1.000000,0.498039,0.054902}%
\pgfsetstrokecolor{currentstroke}%
\pgfsetdash{}{0pt}%
\pgfpathmoveto{\pgfqpoint{1.225000in}{1.882181in}}%
\pgfpathlineto{\pgfqpoint{1.375000in}{1.882181in}}%
\pgfusepath{stroke}%
\end{pgfscope}%
\begin{pgfscope}%
\pgfpathrectangle{\pgfqpoint{1.000000in}{0.720000in}}{\pgfqpoint{4.800000in}{4.620000in}}%
\pgfusepath{clip}%
\pgfsetrectcap%
\pgfsetroundjoin%
\pgfsetlinewidth{1.003750pt}%
\definecolor{currentstroke}{rgb}{1.000000,0.498039,0.054902}%
\pgfsetstrokecolor{currentstroke}%
\pgfsetdash{}{0pt}%
\pgfpathmoveto{\pgfqpoint{1.525000in}{2.628237in}}%
\pgfpathlineto{\pgfqpoint{1.675000in}{2.628237in}}%
\pgfusepath{stroke}%
\end{pgfscope}%
\begin{pgfscope}%
\pgfpathrectangle{\pgfqpoint{1.000000in}{0.720000in}}{\pgfqpoint{4.800000in}{4.620000in}}%
\pgfusepath{clip}%
\pgfsetrectcap%
\pgfsetroundjoin%
\pgfsetlinewidth{1.003750pt}%
\definecolor{currentstroke}{rgb}{1.000000,0.498039,0.054902}%
\pgfsetstrokecolor{currentstroke}%
\pgfsetdash{}{0pt}%
\pgfpathmoveto{\pgfqpoint{1.825000in}{2.895617in}}%
\pgfpathlineto{\pgfqpoint{1.975000in}{2.895617in}}%
\pgfusepath{stroke}%
\end{pgfscope}%
\begin{pgfscope}%
\pgfpathrectangle{\pgfqpoint{1.000000in}{0.720000in}}{\pgfqpoint{4.800000in}{4.620000in}}%
\pgfusepath{clip}%
\pgfsetrectcap%
\pgfsetroundjoin%
\pgfsetlinewidth{1.003750pt}%
\definecolor{currentstroke}{rgb}{1.000000,0.498039,0.054902}%
\pgfsetstrokecolor{currentstroke}%
\pgfsetdash{}{0pt}%
\pgfpathmoveto{\pgfqpoint{2.125000in}{3.051319in}}%
\pgfpathlineto{\pgfqpoint{2.275000in}{3.051319in}}%
\pgfusepath{stroke}%
\end{pgfscope}%
\begin{pgfscope}%
\pgfpathrectangle{\pgfqpoint{1.000000in}{0.720000in}}{\pgfqpoint{4.800000in}{4.620000in}}%
\pgfusepath{clip}%
\pgfsetrectcap%
\pgfsetroundjoin%
\pgfsetlinewidth{1.003750pt}%
\definecolor{currentstroke}{rgb}{1.000000,0.498039,0.054902}%
\pgfsetstrokecolor{currentstroke}%
\pgfsetdash{}{0pt}%
\pgfpathmoveto{\pgfqpoint{2.425000in}{3.175738in}}%
\pgfpathlineto{\pgfqpoint{2.575000in}{3.175738in}}%
\pgfusepath{stroke}%
\end{pgfscope}%
\begin{pgfscope}%
\pgfpathrectangle{\pgfqpoint{1.000000in}{0.720000in}}{\pgfqpoint{4.800000in}{4.620000in}}%
\pgfusepath{clip}%
\pgfsetrectcap%
\pgfsetroundjoin%
\pgfsetlinewidth{1.003750pt}%
\definecolor{currentstroke}{rgb}{1.000000,0.498039,0.054902}%
\pgfsetstrokecolor{currentstroke}%
\pgfsetdash{}{0pt}%
\pgfpathmoveto{\pgfqpoint{2.725000in}{3.151728in}}%
\pgfpathlineto{\pgfqpoint{2.875000in}{3.151728in}}%
\pgfusepath{stroke}%
\end{pgfscope}%
\begin{pgfscope}%
\pgfpathrectangle{\pgfqpoint{1.000000in}{0.720000in}}{\pgfqpoint{4.800000in}{4.620000in}}%
\pgfusepath{clip}%
\pgfsetrectcap%
\pgfsetroundjoin%
\pgfsetlinewidth{1.003750pt}%
\definecolor{currentstroke}{rgb}{1.000000,0.498039,0.054902}%
\pgfsetstrokecolor{currentstroke}%
\pgfsetdash{}{0pt}%
\pgfpathmoveto{\pgfqpoint{3.025000in}{3.287234in}}%
\pgfpathlineto{\pgfqpoint{3.175000in}{3.287234in}}%
\pgfusepath{stroke}%
\end{pgfscope}%
\begin{pgfscope}%
\pgfpathrectangle{\pgfqpoint{1.000000in}{0.720000in}}{\pgfqpoint{4.800000in}{4.620000in}}%
\pgfusepath{clip}%
\pgfsetrectcap%
\pgfsetroundjoin%
\pgfsetlinewidth{1.003750pt}%
\definecolor{currentstroke}{rgb}{1.000000,0.498039,0.054902}%
\pgfsetstrokecolor{currentstroke}%
\pgfsetdash{}{0pt}%
\pgfpathmoveto{\pgfqpoint{3.325000in}{3.276514in}}%
\pgfpathlineto{\pgfqpoint{3.475000in}{3.276514in}}%
\pgfusepath{stroke}%
\end{pgfscope}%
\begin{pgfscope}%
\pgfpathrectangle{\pgfqpoint{1.000000in}{0.720000in}}{\pgfqpoint{4.800000in}{4.620000in}}%
\pgfusepath{clip}%
\pgfsetrectcap%
\pgfsetroundjoin%
\pgfsetlinewidth{1.003750pt}%
\definecolor{currentstroke}{rgb}{1.000000,0.498039,0.054902}%
\pgfsetstrokecolor{currentstroke}%
\pgfsetdash{}{0pt}%
\pgfpathmoveto{\pgfqpoint{3.625000in}{3.237637in}}%
\pgfpathlineto{\pgfqpoint{3.775000in}{3.237637in}}%
\pgfusepath{stroke}%
\end{pgfscope}%
\begin{pgfscope}%
\pgfpathrectangle{\pgfqpoint{1.000000in}{0.720000in}}{\pgfqpoint{4.800000in}{4.620000in}}%
\pgfusepath{clip}%
\pgfsetrectcap%
\pgfsetroundjoin%
\pgfsetlinewidth{1.003750pt}%
\definecolor{currentstroke}{rgb}{1.000000,0.498039,0.054902}%
\pgfsetstrokecolor{currentstroke}%
\pgfsetdash{}{0pt}%
\pgfpathmoveto{\pgfqpoint{3.925000in}{3.273046in}}%
\pgfpathlineto{\pgfqpoint{4.075000in}{3.273046in}}%
\pgfusepath{stroke}%
\end{pgfscope}%
\begin{pgfscope}%
\pgfpathrectangle{\pgfqpoint{1.000000in}{0.720000in}}{\pgfqpoint{4.800000in}{4.620000in}}%
\pgfusepath{clip}%
\pgfsetrectcap%
\pgfsetroundjoin%
\pgfsetlinewidth{1.003750pt}%
\definecolor{currentstroke}{rgb}{1.000000,0.498039,0.054902}%
\pgfsetstrokecolor{currentstroke}%
\pgfsetdash{}{0pt}%
\pgfpathmoveto{\pgfqpoint{4.225000in}{3.207480in}}%
\pgfpathlineto{\pgfqpoint{4.375000in}{3.207480in}}%
\pgfusepath{stroke}%
\end{pgfscope}%
\begin{pgfscope}%
\pgfpathrectangle{\pgfqpoint{1.000000in}{0.720000in}}{\pgfqpoint{4.800000in}{4.620000in}}%
\pgfusepath{clip}%
\pgfsetrectcap%
\pgfsetroundjoin%
\pgfsetlinewidth{1.003750pt}%
\definecolor{currentstroke}{rgb}{1.000000,0.498039,0.054902}%
\pgfsetstrokecolor{currentstroke}%
\pgfsetdash{}{0pt}%
\pgfpathmoveto{\pgfqpoint{4.525000in}{4.069136in}}%
\pgfpathlineto{\pgfqpoint{4.675000in}{4.069136in}}%
\pgfusepath{stroke}%
\end{pgfscope}%
\begin{pgfscope}%
\pgfpathrectangle{\pgfqpoint{1.000000in}{0.720000in}}{\pgfqpoint{4.800000in}{4.620000in}}%
\pgfusepath{clip}%
\pgfsetrectcap%
\pgfsetroundjoin%
\pgfsetlinewidth{1.003750pt}%
\definecolor{currentstroke}{rgb}{1.000000,0.498039,0.054902}%
\pgfsetstrokecolor{currentstroke}%
\pgfsetdash{}{0pt}%
\pgfpathmoveto{\pgfqpoint{4.825000in}{2.631498in}}%
\pgfpathlineto{\pgfqpoint{4.975000in}{2.631498in}}%
\pgfusepath{stroke}%
\end{pgfscope}%
\begin{pgfscope}%
\pgfpathrectangle{\pgfqpoint{1.000000in}{0.720000in}}{\pgfqpoint{4.800000in}{4.620000in}}%
\pgfusepath{clip}%
\pgfsetrectcap%
\pgfsetroundjoin%
\pgfsetlinewidth{1.003750pt}%
\definecolor{currentstroke}{rgb}{1.000000,0.498039,0.054902}%
\pgfsetstrokecolor{currentstroke}%
\pgfsetdash{}{0pt}%
\pgfpathmoveto{\pgfqpoint{5.125000in}{3.483259in}}%
\pgfpathlineto{\pgfqpoint{5.275000in}{3.483259in}}%
\pgfusepath{stroke}%
\end{pgfscope}%
\begin{pgfscope}%
\pgfpathrectangle{\pgfqpoint{1.000000in}{0.720000in}}{\pgfqpoint{4.800000in}{4.620000in}}%
\pgfusepath{clip}%
\pgfsetrectcap%
\pgfsetroundjoin%
\pgfsetlinewidth{1.003750pt}%
\definecolor{currentstroke}{rgb}{1.000000,0.498039,0.054902}%
\pgfsetstrokecolor{currentstroke}%
\pgfsetdash{}{0pt}%
\pgfpathmoveto{\pgfqpoint{5.425000in}{3.271319in}}%
\pgfpathlineto{\pgfqpoint{5.575000in}{3.271319in}}%
\pgfusepath{stroke}%
\end{pgfscope}%
\begin{pgfscope}%
\pgfsetrectcap%
\pgfsetmiterjoin%
\pgfsetlinewidth{0.803000pt}%
\definecolor{currentstroke}{rgb}{0.000000,0.000000,0.000000}%
\pgfsetstrokecolor{currentstroke}%
\pgfsetdash{}{0pt}%
\pgfpathmoveto{\pgfqpoint{1.000000in}{0.720000in}}%
\pgfpathlineto{\pgfqpoint{1.000000in}{5.340000in}}%
\pgfusepath{stroke}%
\end{pgfscope}%
\begin{pgfscope}%
\pgfsetrectcap%
\pgfsetmiterjoin%
\pgfsetlinewidth{0.803000pt}%
\definecolor{currentstroke}{rgb}{0.000000,0.000000,0.000000}%
\pgfsetstrokecolor{currentstroke}%
\pgfsetdash{}{0pt}%
\pgfpathmoveto{\pgfqpoint{5.800000in}{0.720000in}}%
\pgfpathlineto{\pgfqpoint{5.800000in}{5.340000in}}%
\pgfusepath{stroke}%
\end{pgfscope}%
\begin{pgfscope}%
\pgfsetrectcap%
\pgfsetmiterjoin%
\pgfsetlinewidth{0.803000pt}%
\definecolor{currentstroke}{rgb}{0.000000,0.000000,0.000000}%
\pgfsetstrokecolor{currentstroke}%
\pgfsetdash{}{0pt}%
\pgfpathmoveto{\pgfqpoint{1.000000in}{0.720000in}}%
\pgfpathlineto{\pgfqpoint{5.800000in}{0.720000in}}%
\pgfusepath{stroke}%
\end{pgfscope}%
\begin{pgfscope}%
\pgfsetrectcap%
\pgfsetmiterjoin%
\pgfsetlinewidth{0.803000pt}%
\definecolor{currentstroke}{rgb}{0.000000,0.000000,0.000000}%
\pgfsetstrokecolor{currentstroke}%
\pgfsetdash{}{0pt}%
\pgfpathmoveto{\pgfqpoint{1.000000in}{5.340000in}}%
\pgfpathlineto{\pgfqpoint{5.800000in}{5.340000in}}%
\pgfusepath{stroke}%
\end{pgfscope}%
\begin{pgfscope}%
\pgfsetbuttcap%
\pgfsetmiterjoin%
\definecolor{currentfill}{rgb}{1.000000,1.000000,1.000000}%
\pgfsetfillcolor{currentfill}%
\pgfsetfillopacity{0.800000}%
\pgfsetlinewidth{1.003750pt}%
\definecolor{currentstroke}{rgb}{0.800000,0.800000,0.800000}%
\pgfsetstrokecolor{currentstroke}%
\pgfsetstrokeopacity{0.800000}%
\pgfsetdash{}{0pt}%
\pgfpathmoveto{\pgfqpoint{4.330170in}{0.858889in}}%
\pgfpathlineto{\pgfqpoint{5.605556in}{0.858889in}}%
\pgfpathquadraticcurveto{\pgfqpoint{5.661111in}{0.858889in}}{\pgfqpoint{5.661111in}{0.914444in}}%
\pgfpathlineto{\pgfqpoint{5.661111in}{1.281623in}}%
\pgfpathquadraticcurveto{\pgfqpoint{5.661111in}{1.337179in}}{\pgfqpoint{5.605556in}{1.337179in}}%
\pgfpathlineto{\pgfqpoint{4.330170in}{1.337179in}}%
\pgfpathquadraticcurveto{\pgfqpoint{4.274615in}{1.337179in}}{\pgfqpoint{4.274615in}{1.281623in}}%
\pgfpathlineto{\pgfqpoint{4.274615in}{0.914444in}}%
\pgfpathquadraticcurveto{\pgfqpoint{4.274615in}{0.858889in}}{\pgfqpoint{4.330170in}{0.858889in}}%
\pgfpathclose%
\pgfusepath{stroke,fill}%
\end{pgfscope}%
\begin{pgfscope}%
\pgfsetrectcap%
\pgfsetroundjoin%
\pgfsetlinewidth{2.007500pt}%
\definecolor{currentstroke}{rgb}{0.121569,0.466667,0.705882}%
\pgfsetstrokecolor{currentstroke}%
\pgfsetdash{}{0pt}%
\pgfpathmoveto{\pgfqpoint{4.385726in}{1.123251in}}%
\pgfpathlineto{\pgfqpoint{4.941281in}{1.123251in}}%
\pgfusepath{stroke}%
\end{pgfscope}%
\begin{pgfscope}%
\definecolor{textcolor}{rgb}{0.000000,0.000000,0.000000}%
\pgfsetstrokecolor{textcolor}%
\pgfsetfillcolor{textcolor}%
\pgftext[x=5.163504in,y=1.026029in,left,base]{\color{textcolor}\sffamily\fontsize{20.000000}{24.000000}\selectfont \(\displaystyle D_w\)}%
\end{pgfscope}%
\begin{pgfscope}%
\pgfsetbuttcap%
\pgfsetmiterjoin%
\definecolor{currentfill}{rgb}{1.000000,1.000000,1.000000}%
\pgfsetfillcolor{currentfill}%
\pgfsetlinewidth{0.000000pt}%
\definecolor{currentstroke}{rgb}{0.000000,0.000000,0.000000}%
\pgfsetstrokecolor{currentstroke}%
\pgfsetstrokeopacity{0.000000}%
\pgfsetdash{}{0pt}%
\pgfpathmoveto{\pgfqpoint{5.800000in}{0.720000in}}%
\pgfpathlineto{\pgfqpoint{7.200000in}{0.720000in}}%
\pgfpathlineto{\pgfqpoint{7.200000in}{5.340000in}}%
\pgfpathlineto{\pgfqpoint{5.800000in}{5.340000in}}%
\pgfpathclose%
\pgfusepath{fill}%
\end{pgfscope}%
\begin{pgfscope}%
\pgfpathrectangle{\pgfqpoint{5.800000in}{0.720000in}}{\pgfqpoint{1.400000in}{4.620000in}}%
\pgfusepath{clip}%
\pgfsetbuttcap%
\pgfsetmiterjoin%
\definecolor{currentfill}{rgb}{0.121569,0.466667,0.705882}%
\pgfsetfillcolor{currentfill}%
\pgfsetlinewidth{0.000000pt}%
\definecolor{currentstroke}{rgb}{0.000000,0.000000,0.000000}%
\pgfsetstrokecolor{currentstroke}%
\pgfsetstrokeopacity{0.000000}%
\pgfsetdash{}{0pt}%
\pgfpathmoveto{\pgfqpoint{5.800000in}{0.720000in}}%
\pgfpathlineto{\pgfqpoint{5.881793in}{0.720000in}}%
\pgfpathlineto{\pgfqpoint{5.881793in}{0.835730in}}%
\pgfpathlineto{\pgfqpoint{5.800000in}{0.835730in}}%
\pgfpathclose%
\pgfusepath{fill}%
\end{pgfscope}%
\begin{pgfscope}%
\pgfpathrectangle{\pgfqpoint{5.800000in}{0.720000in}}{\pgfqpoint{1.400000in}{4.620000in}}%
\pgfusepath{clip}%
\pgfsetbuttcap%
\pgfsetmiterjoin%
\definecolor{currentfill}{rgb}{0.121569,0.466667,0.705882}%
\pgfsetfillcolor{currentfill}%
\pgfsetlinewidth{0.000000pt}%
\definecolor{currentstroke}{rgb}{0.000000,0.000000,0.000000}%
\pgfsetstrokecolor{currentstroke}%
\pgfsetstrokeopacity{0.000000}%
\pgfsetdash{}{0pt}%
\pgfpathmoveto{\pgfqpoint{5.800000in}{0.835730in}}%
\pgfpathlineto{\pgfqpoint{5.879748in}{0.835730in}}%
\pgfpathlineto{\pgfqpoint{5.879748in}{0.951459in}}%
\pgfpathlineto{\pgfqpoint{5.800000in}{0.951459in}}%
\pgfpathclose%
\pgfusepath{fill}%
\end{pgfscope}%
\begin{pgfscope}%
\pgfpathrectangle{\pgfqpoint{5.800000in}{0.720000in}}{\pgfqpoint{1.400000in}{4.620000in}}%
\pgfusepath{clip}%
\pgfsetbuttcap%
\pgfsetmiterjoin%
\definecolor{currentfill}{rgb}{0.121569,0.466667,0.705882}%
\pgfsetfillcolor{currentfill}%
\pgfsetlinewidth{0.000000pt}%
\definecolor{currentstroke}{rgb}{0.000000,0.000000,0.000000}%
\pgfsetstrokecolor{currentstroke}%
\pgfsetstrokeopacity{0.000000}%
\pgfsetdash{}{0pt}%
\pgfpathmoveto{\pgfqpoint{5.800000in}{0.951459in}}%
\pgfpathlineto{\pgfqpoint{5.869524in}{0.951459in}}%
\pgfpathlineto{\pgfqpoint{5.869524in}{1.067189in}}%
\pgfpathlineto{\pgfqpoint{5.800000in}{1.067189in}}%
\pgfpathclose%
\pgfusepath{fill}%
\end{pgfscope}%
\begin{pgfscope}%
\pgfpathrectangle{\pgfqpoint{5.800000in}{0.720000in}}{\pgfqpoint{1.400000in}{4.620000in}}%
\pgfusepath{clip}%
\pgfsetbuttcap%
\pgfsetmiterjoin%
\definecolor{currentfill}{rgb}{0.121569,0.466667,0.705882}%
\pgfsetfillcolor{currentfill}%
\pgfsetlinewidth{0.000000pt}%
\definecolor{currentstroke}{rgb}{0.000000,0.000000,0.000000}%
\pgfsetstrokecolor{currentstroke}%
\pgfsetstrokeopacity{0.000000}%
\pgfsetdash{}{0pt}%
\pgfpathmoveto{\pgfqpoint{5.800000in}{1.067189in}}%
\pgfpathlineto{\pgfqpoint{5.898152in}{1.067189in}}%
\pgfpathlineto{\pgfqpoint{5.898152in}{1.182919in}}%
\pgfpathlineto{\pgfqpoint{5.800000in}{1.182919in}}%
\pgfpathclose%
\pgfusepath{fill}%
\end{pgfscope}%
\begin{pgfscope}%
\pgfpathrectangle{\pgfqpoint{5.800000in}{0.720000in}}{\pgfqpoint{1.400000in}{4.620000in}}%
\pgfusepath{clip}%
\pgfsetbuttcap%
\pgfsetmiterjoin%
\definecolor{currentfill}{rgb}{0.121569,0.466667,0.705882}%
\pgfsetfillcolor{currentfill}%
\pgfsetlinewidth{0.000000pt}%
\definecolor{currentstroke}{rgb}{0.000000,0.000000,0.000000}%
\pgfsetstrokecolor{currentstroke}%
\pgfsetstrokeopacity{0.000000}%
\pgfsetdash{}{0pt}%
\pgfpathmoveto{\pgfqpoint{5.800000in}{1.182919in}}%
\pgfpathlineto{\pgfqpoint{5.920645in}{1.182919in}}%
\pgfpathlineto{\pgfqpoint{5.920645in}{1.298648in}}%
\pgfpathlineto{\pgfqpoint{5.800000in}{1.298648in}}%
\pgfpathclose%
\pgfusepath{fill}%
\end{pgfscope}%
\begin{pgfscope}%
\pgfpathrectangle{\pgfqpoint{5.800000in}{0.720000in}}{\pgfqpoint{1.400000in}{4.620000in}}%
\pgfusepath{clip}%
\pgfsetbuttcap%
\pgfsetmiterjoin%
\definecolor{currentfill}{rgb}{0.121569,0.466667,0.705882}%
\pgfsetfillcolor{currentfill}%
\pgfsetlinewidth{0.000000pt}%
\definecolor{currentstroke}{rgb}{0.000000,0.000000,0.000000}%
\pgfsetstrokecolor{currentstroke}%
\pgfsetstrokeopacity{0.000000}%
\pgfsetdash{}{0pt}%
\pgfpathmoveto{\pgfqpoint{5.800000in}{1.298648in}}%
\pgfpathlineto{\pgfqpoint{5.969721in}{1.298648in}}%
\pgfpathlineto{\pgfqpoint{5.969721in}{1.414378in}}%
\pgfpathlineto{\pgfqpoint{5.800000in}{1.414378in}}%
\pgfpathclose%
\pgfusepath{fill}%
\end{pgfscope}%
\begin{pgfscope}%
\pgfpathrectangle{\pgfqpoint{5.800000in}{0.720000in}}{\pgfqpoint{1.400000in}{4.620000in}}%
\pgfusepath{clip}%
\pgfsetbuttcap%
\pgfsetmiterjoin%
\definecolor{currentfill}{rgb}{0.121569,0.466667,0.705882}%
\pgfsetfillcolor{currentfill}%
\pgfsetlinewidth{0.000000pt}%
\definecolor{currentstroke}{rgb}{0.000000,0.000000,0.000000}%
\pgfsetstrokecolor{currentstroke}%
\pgfsetstrokeopacity{0.000000}%
\pgfsetdash{}{0pt}%
\pgfpathmoveto{\pgfqpoint{5.800000in}{1.414378in}}%
\pgfpathlineto{\pgfqpoint{5.975856in}{1.414378in}}%
\pgfpathlineto{\pgfqpoint{5.975856in}{1.530108in}}%
\pgfpathlineto{\pgfqpoint{5.800000in}{1.530108in}}%
\pgfpathclose%
\pgfusepath{fill}%
\end{pgfscope}%
\begin{pgfscope}%
\pgfpathrectangle{\pgfqpoint{5.800000in}{0.720000in}}{\pgfqpoint{1.400000in}{4.620000in}}%
\pgfusepath{clip}%
\pgfsetbuttcap%
\pgfsetmiterjoin%
\definecolor{currentfill}{rgb}{0.121569,0.466667,0.705882}%
\pgfsetfillcolor{currentfill}%
\pgfsetlinewidth{0.000000pt}%
\definecolor{currentstroke}{rgb}{0.000000,0.000000,0.000000}%
\pgfsetstrokecolor{currentstroke}%
\pgfsetstrokeopacity{0.000000}%
\pgfsetdash{}{0pt}%
\pgfpathmoveto{\pgfqpoint{5.800000in}{1.530108in}}%
\pgfpathlineto{\pgfqpoint{6.080142in}{1.530108in}}%
\pgfpathlineto{\pgfqpoint{6.080142in}{1.645837in}}%
\pgfpathlineto{\pgfqpoint{5.800000in}{1.645837in}}%
\pgfpathclose%
\pgfusepath{fill}%
\end{pgfscope}%
\begin{pgfscope}%
\pgfpathrectangle{\pgfqpoint{5.800000in}{0.720000in}}{\pgfqpoint{1.400000in}{4.620000in}}%
\pgfusepath{clip}%
\pgfsetbuttcap%
\pgfsetmiterjoin%
\definecolor{currentfill}{rgb}{0.121569,0.466667,0.705882}%
\pgfsetfillcolor{currentfill}%
\pgfsetlinewidth{0.000000pt}%
\definecolor{currentstroke}{rgb}{0.000000,0.000000,0.000000}%
\pgfsetstrokecolor{currentstroke}%
\pgfsetstrokeopacity{0.000000}%
\pgfsetdash{}{0pt}%
\pgfpathmoveto{\pgfqpoint{5.800000in}{1.645837in}}%
\pgfpathlineto{\pgfqpoint{6.051514in}{1.645837in}}%
\pgfpathlineto{\pgfqpoint{6.051514in}{1.761567in}}%
\pgfpathlineto{\pgfqpoint{5.800000in}{1.761567in}}%
\pgfpathclose%
\pgfusepath{fill}%
\end{pgfscope}%
\begin{pgfscope}%
\pgfpathrectangle{\pgfqpoint{5.800000in}{0.720000in}}{\pgfqpoint{1.400000in}{4.620000in}}%
\pgfusepath{clip}%
\pgfsetbuttcap%
\pgfsetmiterjoin%
\definecolor{currentfill}{rgb}{0.121569,0.466667,0.705882}%
\pgfsetfillcolor{currentfill}%
\pgfsetlinewidth{0.000000pt}%
\definecolor{currentstroke}{rgb}{0.000000,0.000000,0.000000}%
\pgfsetstrokecolor{currentstroke}%
\pgfsetstrokeopacity{0.000000}%
\pgfsetdash{}{0pt}%
\pgfpathmoveto{\pgfqpoint{5.800000in}{1.761567in}}%
\pgfpathlineto{\pgfqpoint{6.184429in}{1.761567in}}%
\pgfpathlineto{\pgfqpoint{6.184429in}{1.877297in}}%
\pgfpathlineto{\pgfqpoint{5.800000in}{1.877297in}}%
\pgfpathclose%
\pgfusepath{fill}%
\end{pgfscope}%
\begin{pgfscope}%
\pgfpathrectangle{\pgfqpoint{5.800000in}{0.720000in}}{\pgfqpoint{1.400000in}{4.620000in}}%
\pgfusepath{clip}%
\pgfsetbuttcap%
\pgfsetmiterjoin%
\definecolor{currentfill}{rgb}{0.121569,0.466667,0.705882}%
\pgfsetfillcolor{currentfill}%
\pgfsetlinewidth{0.000000pt}%
\definecolor{currentstroke}{rgb}{0.000000,0.000000,0.000000}%
\pgfsetstrokecolor{currentstroke}%
\pgfsetstrokeopacity{0.000000}%
\pgfsetdash{}{0pt}%
\pgfpathmoveto{\pgfqpoint{5.800000in}{1.877297in}}%
\pgfpathlineto{\pgfqpoint{6.235549in}{1.877297in}}%
\pgfpathlineto{\pgfqpoint{6.235549in}{1.993026in}}%
\pgfpathlineto{\pgfqpoint{5.800000in}{1.993026in}}%
\pgfpathclose%
\pgfusepath{fill}%
\end{pgfscope}%
\begin{pgfscope}%
\pgfpathrectangle{\pgfqpoint{5.800000in}{0.720000in}}{\pgfqpoint{1.400000in}{4.620000in}}%
\pgfusepath{clip}%
\pgfsetbuttcap%
\pgfsetmiterjoin%
\definecolor{currentfill}{rgb}{0.121569,0.466667,0.705882}%
\pgfsetfillcolor{currentfill}%
\pgfsetlinewidth{0.000000pt}%
\definecolor{currentstroke}{rgb}{0.000000,0.000000,0.000000}%
\pgfsetstrokecolor{currentstroke}%
\pgfsetstrokeopacity{0.000000}%
\pgfsetdash{}{0pt}%
\pgfpathmoveto{\pgfqpoint{5.800000in}{1.993026in}}%
\pgfpathlineto{\pgfqpoint{6.341881in}{1.993026in}}%
\pgfpathlineto{\pgfqpoint{6.341881in}{2.108756in}}%
\pgfpathlineto{\pgfqpoint{5.800000in}{2.108756in}}%
\pgfpathclose%
\pgfusepath{fill}%
\end{pgfscope}%
\begin{pgfscope}%
\pgfpathrectangle{\pgfqpoint{5.800000in}{0.720000in}}{\pgfqpoint{1.400000in}{4.620000in}}%
\pgfusepath{clip}%
\pgfsetbuttcap%
\pgfsetmiterjoin%
\definecolor{currentfill}{rgb}{0.121569,0.466667,0.705882}%
\pgfsetfillcolor{currentfill}%
\pgfsetlinewidth{0.000000pt}%
\definecolor{currentstroke}{rgb}{0.000000,0.000000,0.000000}%
\pgfsetstrokecolor{currentstroke}%
\pgfsetstrokeopacity{0.000000}%
\pgfsetdash{}{0pt}%
\pgfpathmoveto{\pgfqpoint{5.800000in}{2.108756in}}%
\pgfpathlineto{\pgfqpoint{6.519781in}{2.108756in}}%
\pgfpathlineto{\pgfqpoint{6.519781in}{2.224486in}}%
\pgfpathlineto{\pgfqpoint{5.800000in}{2.224486in}}%
\pgfpathclose%
\pgfusepath{fill}%
\end{pgfscope}%
\begin{pgfscope}%
\pgfpathrectangle{\pgfqpoint{5.800000in}{0.720000in}}{\pgfqpoint{1.400000in}{4.620000in}}%
\pgfusepath{clip}%
\pgfsetbuttcap%
\pgfsetmiterjoin%
\definecolor{currentfill}{rgb}{0.121569,0.466667,0.705882}%
\pgfsetfillcolor{currentfill}%
\pgfsetlinewidth{0.000000pt}%
\definecolor{currentstroke}{rgb}{0.000000,0.000000,0.000000}%
\pgfsetstrokecolor{currentstroke}%
\pgfsetstrokeopacity{0.000000}%
\pgfsetdash{}{0pt}%
\pgfpathmoveto{\pgfqpoint{5.800000in}{2.224486in}}%
\pgfpathlineto{\pgfqpoint{6.560678in}{2.224486in}}%
\pgfpathlineto{\pgfqpoint{6.560678in}{2.340216in}}%
\pgfpathlineto{\pgfqpoint{5.800000in}{2.340216in}}%
\pgfpathclose%
\pgfusepath{fill}%
\end{pgfscope}%
\begin{pgfscope}%
\pgfpathrectangle{\pgfqpoint{5.800000in}{0.720000in}}{\pgfqpoint{1.400000in}{4.620000in}}%
\pgfusepath{clip}%
\pgfsetbuttcap%
\pgfsetmiterjoin%
\definecolor{currentfill}{rgb}{0.121569,0.466667,0.705882}%
\pgfsetfillcolor{currentfill}%
\pgfsetlinewidth{0.000000pt}%
\definecolor{currentstroke}{rgb}{0.000000,0.000000,0.000000}%
\pgfsetstrokecolor{currentstroke}%
\pgfsetstrokeopacity{0.000000}%
\pgfsetdash{}{0pt}%
\pgfpathmoveto{\pgfqpoint{5.800000in}{2.340216in}}%
\pgfpathlineto{\pgfqpoint{6.675189in}{2.340216in}}%
\pgfpathlineto{\pgfqpoint{6.675189in}{2.455945in}}%
\pgfpathlineto{\pgfqpoint{5.800000in}{2.455945in}}%
\pgfpathclose%
\pgfusepath{fill}%
\end{pgfscope}%
\begin{pgfscope}%
\pgfpathrectangle{\pgfqpoint{5.800000in}{0.720000in}}{\pgfqpoint{1.400000in}{4.620000in}}%
\pgfusepath{clip}%
\pgfsetbuttcap%
\pgfsetmiterjoin%
\definecolor{currentfill}{rgb}{0.121569,0.466667,0.705882}%
\pgfsetfillcolor{currentfill}%
\pgfsetlinewidth{0.000000pt}%
\definecolor{currentstroke}{rgb}{0.000000,0.000000,0.000000}%
\pgfsetstrokecolor{currentstroke}%
\pgfsetstrokeopacity{0.000000}%
\pgfsetdash{}{0pt}%
\pgfpathmoveto{\pgfqpoint{5.800000in}{2.455945in}}%
\pgfpathlineto{\pgfqpoint{6.812192in}{2.455945in}}%
\pgfpathlineto{\pgfqpoint{6.812192in}{2.571675in}}%
\pgfpathlineto{\pgfqpoint{5.800000in}{2.571675in}}%
\pgfpathclose%
\pgfusepath{fill}%
\end{pgfscope}%
\begin{pgfscope}%
\pgfpathrectangle{\pgfqpoint{5.800000in}{0.720000in}}{\pgfqpoint{1.400000in}{4.620000in}}%
\pgfusepath{clip}%
\pgfsetbuttcap%
\pgfsetmiterjoin%
\definecolor{currentfill}{rgb}{0.121569,0.466667,0.705882}%
\pgfsetfillcolor{currentfill}%
\pgfsetlinewidth{0.000000pt}%
\definecolor{currentstroke}{rgb}{0.000000,0.000000,0.000000}%
\pgfsetstrokecolor{currentstroke}%
\pgfsetstrokeopacity{0.000000}%
\pgfsetdash{}{0pt}%
\pgfpathmoveto{\pgfqpoint{5.800000in}{2.571675in}}%
\pgfpathlineto{\pgfqpoint{6.926703in}{2.571675in}}%
\pgfpathlineto{\pgfqpoint{6.926703in}{2.687405in}}%
\pgfpathlineto{\pgfqpoint{5.800000in}{2.687405in}}%
\pgfpathclose%
\pgfusepath{fill}%
\end{pgfscope}%
\begin{pgfscope}%
\pgfpathrectangle{\pgfqpoint{5.800000in}{0.720000in}}{\pgfqpoint{1.400000in}{4.620000in}}%
\pgfusepath{clip}%
\pgfsetbuttcap%
\pgfsetmiterjoin%
\definecolor{currentfill}{rgb}{0.121569,0.466667,0.705882}%
\pgfsetfillcolor{currentfill}%
\pgfsetlinewidth{0.000000pt}%
\definecolor{currentstroke}{rgb}{0.000000,0.000000,0.000000}%
\pgfsetstrokecolor{currentstroke}%
\pgfsetstrokeopacity{0.000000}%
\pgfsetdash{}{0pt}%
\pgfpathmoveto{\pgfqpoint{5.800000in}{2.687405in}}%
\pgfpathlineto{\pgfqpoint{6.965555in}{2.687405in}}%
\pgfpathlineto{\pgfqpoint{6.965555in}{2.803134in}}%
\pgfpathlineto{\pgfqpoint{5.800000in}{2.803134in}}%
\pgfpathclose%
\pgfusepath{fill}%
\end{pgfscope}%
\begin{pgfscope}%
\pgfpathrectangle{\pgfqpoint{5.800000in}{0.720000in}}{\pgfqpoint{1.400000in}{4.620000in}}%
\pgfusepath{clip}%
\pgfsetbuttcap%
\pgfsetmiterjoin%
\definecolor{currentfill}{rgb}{0.121569,0.466667,0.705882}%
\pgfsetfillcolor{currentfill}%
\pgfsetlinewidth{0.000000pt}%
\definecolor{currentstroke}{rgb}{0.000000,0.000000,0.000000}%
\pgfsetstrokecolor{currentstroke}%
\pgfsetstrokeopacity{0.000000}%
\pgfsetdash{}{0pt}%
\pgfpathmoveto{\pgfqpoint{5.800000in}{2.803134in}}%
\pgfpathlineto{\pgfqpoint{7.016676in}{2.803134in}}%
\pgfpathlineto{\pgfqpoint{7.016676in}{2.918864in}}%
\pgfpathlineto{\pgfqpoint{5.800000in}{2.918864in}}%
\pgfpathclose%
\pgfusepath{fill}%
\end{pgfscope}%
\begin{pgfscope}%
\pgfpathrectangle{\pgfqpoint{5.800000in}{0.720000in}}{\pgfqpoint{1.400000in}{4.620000in}}%
\pgfusepath{clip}%
\pgfsetbuttcap%
\pgfsetmiterjoin%
\definecolor{currentfill}{rgb}{0.121569,0.466667,0.705882}%
\pgfsetfillcolor{currentfill}%
\pgfsetlinewidth{0.000000pt}%
\definecolor{currentstroke}{rgb}{0.000000,0.000000,0.000000}%
\pgfsetstrokecolor{currentstroke}%
\pgfsetstrokeopacity{0.000000}%
\pgfsetdash{}{0pt}%
\pgfpathmoveto{\pgfqpoint{5.800000in}{2.918864in}}%
\pgfpathlineto{\pgfqpoint{7.069841in}{2.918864in}}%
\pgfpathlineto{\pgfqpoint{7.069841in}{3.034594in}}%
\pgfpathlineto{\pgfqpoint{5.800000in}{3.034594in}}%
\pgfpathclose%
\pgfusepath{fill}%
\end{pgfscope}%
\begin{pgfscope}%
\pgfpathrectangle{\pgfqpoint{5.800000in}{0.720000in}}{\pgfqpoint{1.400000in}{4.620000in}}%
\pgfusepath{clip}%
\pgfsetbuttcap%
\pgfsetmiterjoin%
\definecolor{currentfill}{rgb}{0.121569,0.466667,0.705882}%
\pgfsetfillcolor{currentfill}%
\pgfsetlinewidth{0.000000pt}%
\definecolor{currentstroke}{rgb}{0.000000,0.000000,0.000000}%
\pgfsetstrokecolor{currentstroke}%
\pgfsetstrokeopacity{0.000000}%
\pgfsetdash{}{0pt}%
\pgfpathmoveto{\pgfqpoint{5.800000in}{3.034594in}}%
\pgfpathlineto{\pgfqpoint{6.988048in}{3.034594in}}%
\pgfpathlineto{\pgfqpoint{6.988048in}{3.150323in}}%
\pgfpathlineto{\pgfqpoint{5.800000in}{3.150323in}}%
\pgfpathclose%
\pgfusepath{fill}%
\end{pgfscope}%
\begin{pgfscope}%
\pgfpathrectangle{\pgfqpoint{5.800000in}{0.720000in}}{\pgfqpoint{1.400000in}{4.620000in}}%
\pgfusepath{clip}%
\pgfsetbuttcap%
\pgfsetmiterjoin%
\definecolor{currentfill}{rgb}{0.121569,0.466667,0.705882}%
\pgfsetfillcolor{currentfill}%
\pgfsetlinewidth{0.000000pt}%
\definecolor{currentstroke}{rgb}{0.000000,0.000000,0.000000}%
\pgfsetstrokecolor{currentstroke}%
\pgfsetstrokeopacity{0.000000}%
\pgfsetdash{}{0pt}%
\pgfpathmoveto{\pgfqpoint{5.800000in}{3.150323in}}%
\pgfpathlineto{\pgfqpoint{6.828551in}{3.150323in}}%
\pgfpathlineto{\pgfqpoint{6.828551in}{3.266053in}}%
\pgfpathlineto{\pgfqpoint{5.800000in}{3.266053in}}%
\pgfpathclose%
\pgfusepath{fill}%
\end{pgfscope}%
\begin{pgfscope}%
\pgfpathrectangle{\pgfqpoint{5.800000in}{0.720000in}}{\pgfqpoint{1.400000in}{4.620000in}}%
\pgfusepath{clip}%
\pgfsetbuttcap%
\pgfsetmiterjoin%
\definecolor{currentfill}{rgb}{0.121569,0.466667,0.705882}%
\pgfsetfillcolor{currentfill}%
\pgfsetlinewidth{0.000000pt}%
\definecolor{currentstroke}{rgb}{0.000000,0.000000,0.000000}%
\pgfsetstrokecolor{currentstroke}%
\pgfsetstrokeopacity{0.000000}%
\pgfsetdash{}{0pt}%
\pgfpathmoveto{\pgfqpoint{5.800000in}{3.266053in}}%
\pgfpathlineto{\pgfqpoint{6.756982in}{3.266053in}}%
\pgfpathlineto{\pgfqpoint{6.756982in}{3.381783in}}%
\pgfpathlineto{\pgfqpoint{5.800000in}{3.381783in}}%
\pgfpathclose%
\pgfusepath{fill}%
\end{pgfscope}%
\begin{pgfscope}%
\pgfpathrectangle{\pgfqpoint{5.800000in}{0.720000in}}{\pgfqpoint{1.400000in}{4.620000in}}%
\pgfusepath{clip}%
\pgfsetbuttcap%
\pgfsetmiterjoin%
\definecolor{currentfill}{rgb}{0.121569,0.466667,0.705882}%
\pgfsetfillcolor{currentfill}%
\pgfsetlinewidth{0.000000pt}%
\definecolor{currentstroke}{rgb}{0.000000,0.000000,0.000000}%
\pgfsetstrokecolor{currentstroke}%
\pgfsetstrokeopacity{0.000000}%
\pgfsetdash{}{0pt}%
\pgfpathmoveto{\pgfqpoint{5.800000in}{3.381783in}}%
\pgfpathlineto{\pgfqpoint{6.722220in}{3.381783in}}%
\pgfpathlineto{\pgfqpoint{6.722220in}{3.497512in}}%
\pgfpathlineto{\pgfqpoint{5.800000in}{3.497512in}}%
\pgfpathclose%
\pgfusepath{fill}%
\end{pgfscope}%
\begin{pgfscope}%
\pgfpathrectangle{\pgfqpoint{5.800000in}{0.720000in}}{\pgfqpoint{1.400000in}{4.620000in}}%
\pgfusepath{clip}%
\pgfsetbuttcap%
\pgfsetmiterjoin%
\definecolor{currentfill}{rgb}{0.121569,0.466667,0.705882}%
\pgfsetfillcolor{currentfill}%
\pgfsetlinewidth{0.000000pt}%
\definecolor{currentstroke}{rgb}{0.000000,0.000000,0.000000}%
\pgfsetstrokecolor{currentstroke}%
\pgfsetstrokeopacity{0.000000}%
\pgfsetdash{}{0pt}%
\pgfpathmoveto{\pgfqpoint{5.800000in}{3.497512in}}%
\pgfpathlineto{\pgfqpoint{6.622023in}{3.497512in}}%
\pgfpathlineto{\pgfqpoint{6.622023in}{3.613242in}}%
\pgfpathlineto{\pgfqpoint{5.800000in}{3.613242in}}%
\pgfpathclose%
\pgfusepath{fill}%
\end{pgfscope}%
\begin{pgfscope}%
\pgfpathrectangle{\pgfqpoint{5.800000in}{0.720000in}}{\pgfqpoint{1.400000in}{4.620000in}}%
\pgfusepath{clip}%
\pgfsetbuttcap%
\pgfsetmiterjoin%
\definecolor{currentfill}{rgb}{0.121569,0.466667,0.705882}%
\pgfsetfillcolor{currentfill}%
\pgfsetlinewidth{0.000000pt}%
\definecolor{currentstroke}{rgb}{0.000000,0.000000,0.000000}%
\pgfsetstrokecolor{currentstroke}%
\pgfsetstrokeopacity{0.000000}%
\pgfsetdash{}{0pt}%
\pgfpathmoveto{\pgfqpoint{5.800000in}{3.613242in}}%
\pgfpathlineto{\pgfqpoint{6.583171in}{3.613242in}}%
\pgfpathlineto{\pgfqpoint{6.583171in}{3.728972in}}%
\pgfpathlineto{\pgfqpoint{5.800000in}{3.728972in}}%
\pgfpathclose%
\pgfusepath{fill}%
\end{pgfscope}%
\begin{pgfscope}%
\pgfpathrectangle{\pgfqpoint{5.800000in}{0.720000in}}{\pgfqpoint{1.400000in}{4.620000in}}%
\pgfusepath{clip}%
\pgfsetbuttcap%
\pgfsetmiterjoin%
\definecolor{currentfill}{rgb}{0.121569,0.466667,0.705882}%
\pgfsetfillcolor{currentfill}%
\pgfsetlinewidth{0.000000pt}%
\definecolor{currentstroke}{rgb}{0.000000,0.000000,0.000000}%
\pgfsetstrokecolor{currentstroke}%
\pgfsetstrokeopacity{0.000000}%
\pgfsetdash{}{0pt}%
\pgfpathmoveto{\pgfqpoint{5.800000in}{3.728972in}}%
\pgfpathlineto{\pgfqpoint{6.378688in}{3.728972in}}%
\pgfpathlineto{\pgfqpoint{6.378688in}{3.844701in}}%
\pgfpathlineto{\pgfqpoint{5.800000in}{3.844701in}}%
\pgfpathclose%
\pgfusepath{fill}%
\end{pgfscope}%
\begin{pgfscope}%
\pgfpathrectangle{\pgfqpoint{5.800000in}{0.720000in}}{\pgfqpoint{1.400000in}{4.620000in}}%
\pgfusepath{clip}%
\pgfsetbuttcap%
\pgfsetmiterjoin%
\definecolor{currentfill}{rgb}{0.121569,0.466667,0.705882}%
\pgfsetfillcolor{currentfill}%
\pgfsetlinewidth{0.000000pt}%
\definecolor{currentstroke}{rgb}{0.000000,0.000000,0.000000}%
\pgfsetstrokecolor{currentstroke}%
\pgfsetstrokeopacity{0.000000}%
\pgfsetdash{}{0pt}%
\pgfpathmoveto{\pgfqpoint{5.800000in}{3.844701in}}%
\pgfpathlineto{\pgfqpoint{6.268267in}{3.844701in}}%
\pgfpathlineto{\pgfqpoint{6.268267in}{3.960431in}}%
\pgfpathlineto{\pgfqpoint{5.800000in}{3.960431in}}%
\pgfpathclose%
\pgfusepath{fill}%
\end{pgfscope}%
\begin{pgfscope}%
\pgfpathrectangle{\pgfqpoint{5.800000in}{0.720000in}}{\pgfqpoint{1.400000in}{4.620000in}}%
\pgfusepath{clip}%
\pgfsetbuttcap%
\pgfsetmiterjoin%
\definecolor{currentfill}{rgb}{0.121569,0.466667,0.705882}%
\pgfsetfillcolor{currentfill}%
\pgfsetlinewidth{0.000000pt}%
\definecolor{currentstroke}{rgb}{0.000000,0.000000,0.000000}%
\pgfsetstrokecolor{currentstroke}%
\pgfsetstrokeopacity{0.000000}%
\pgfsetdash{}{0pt}%
\pgfpathmoveto{\pgfqpoint{5.800000in}{3.960431in}}%
\pgfpathlineto{\pgfqpoint{6.264177in}{3.960431in}}%
\pgfpathlineto{\pgfqpoint{6.264177in}{4.076161in}}%
\pgfpathlineto{\pgfqpoint{5.800000in}{4.076161in}}%
\pgfpathclose%
\pgfusepath{fill}%
\end{pgfscope}%
\begin{pgfscope}%
\pgfpathrectangle{\pgfqpoint{5.800000in}{0.720000in}}{\pgfqpoint{1.400000in}{4.620000in}}%
\pgfusepath{clip}%
\pgfsetbuttcap%
\pgfsetmiterjoin%
\definecolor{currentfill}{rgb}{0.121569,0.466667,0.705882}%
\pgfsetfillcolor{currentfill}%
\pgfsetlinewidth{0.000000pt}%
\definecolor{currentstroke}{rgb}{0.000000,0.000000,0.000000}%
\pgfsetstrokecolor{currentstroke}%
\pgfsetstrokeopacity{0.000000}%
\pgfsetdash{}{0pt}%
\pgfpathmoveto{\pgfqpoint{5.800000in}{4.076161in}}%
\pgfpathlineto{\pgfqpoint{6.135353in}{4.076161in}}%
\pgfpathlineto{\pgfqpoint{6.135353in}{4.191890in}}%
\pgfpathlineto{\pgfqpoint{5.800000in}{4.191890in}}%
\pgfpathclose%
\pgfusepath{fill}%
\end{pgfscope}%
\begin{pgfscope}%
\pgfpathrectangle{\pgfqpoint{5.800000in}{0.720000in}}{\pgfqpoint{1.400000in}{4.620000in}}%
\pgfusepath{clip}%
\pgfsetbuttcap%
\pgfsetmiterjoin%
\definecolor{currentfill}{rgb}{0.121569,0.466667,0.705882}%
\pgfsetfillcolor{currentfill}%
\pgfsetlinewidth{0.000000pt}%
\definecolor{currentstroke}{rgb}{0.000000,0.000000,0.000000}%
\pgfsetstrokecolor{currentstroke}%
\pgfsetstrokeopacity{0.000000}%
\pgfsetdash{}{0pt}%
\pgfpathmoveto{\pgfqpoint{5.800000in}{4.191890in}}%
\pgfpathlineto{\pgfqpoint{6.114904in}{4.191890in}}%
\pgfpathlineto{\pgfqpoint{6.114904in}{4.307620in}}%
\pgfpathlineto{\pgfqpoint{5.800000in}{4.307620in}}%
\pgfpathclose%
\pgfusepath{fill}%
\end{pgfscope}%
\begin{pgfscope}%
\pgfpathrectangle{\pgfqpoint{5.800000in}{0.720000in}}{\pgfqpoint{1.400000in}{4.620000in}}%
\pgfusepath{clip}%
\pgfsetbuttcap%
\pgfsetmiterjoin%
\definecolor{currentfill}{rgb}{0.121569,0.466667,0.705882}%
\pgfsetfillcolor{currentfill}%
\pgfsetlinewidth{0.000000pt}%
\definecolor{currentstroke}{rgb}{0.000000,0.000000,0.000000}%
\pgfsetstrokecolor{currentstroke}%
\pgfsetstrokeopacity{0.000000}%
\pgfsetdash{}{0pt}%
\pgfpathmoveto{\pgfqpoint{5.800000in}{4.307620in}}%
\pgfpathlineto{\pgfqpoint{6.039245in}{4.307620in}}%
\pgfpathlineto{\pgfqpoint{6.039245in}{4.423350in}}%
\pgfpathlineto{\pgfqpoint{5.800000in}{4.423350in}}%
\pgfpathclose%
\pgfusepath{fill}%
\end{pgfscope}%
\begin{pgfscope}%
\pgfpathrectangle{\pgfqpoint{5.800000in}{0.720000in}}{\pgfqpoint{1.400000in}{4.620000in}}%
\pgfusepath{clip}%
\pgfsetbuttcap%
\pgfsetmiterjoin%
\definecolor{currentfill}{rgb}{0.121569,0.466667,0.705882}%
\pgfsetfillcolor{currentfill}%
\pgfsetlinewidth{0.000000pt}%
\definecolor{currentstroke}{rgb}{0.000000,0.000000,0.000000}%
\pgfsetstrokecolor{currentstroke}%
\pgfsetstrokeopacity{0.000000}%
\pgfsetdash{}{0pt}%
\pgfpathmoveto{\pgfqpoint{5.800000in}{4.423350in}}%
\pgfpathlineto{\pgfqpoint{6.031066in}{4.423350in}}%
\pgfpathlineto{\pgfqpoint{6.031066in}{4.539079in}}%
\pgfpathlineto{\pgfqpoint{5.800000in}{4.539079in}}%
\pgfpathclose%
\pgfusepath{fill}%
\end{pgfscope}%
\begin{pgfscope}%
\pgfpathrectangle{\pgfqpoint{5.800000in}{0.720000in}}{\pgfqpoint{1.400000in}{4.620000in}}%
\pgfusepath{clip}%
\pgfsetbuttcap%
\pgfsetmiterjoin%
\definecolor{currentfill}{rgb}{0.121569,0.466667,0.705882}%
\pgfsetfillcolor{currentfill}%
\pgfsetlinewidth{0.000000pt}%
\definecolor{currentstroke}{rgb}{0.000000,0.000000,0.000000}%
\pgfsetstrokecolor{currentstroke}%
\pgfsetstrokeopacity{0.000000}%
\pgfsetdash{}{0pt}%
\pgfpathmoveto{\pgfqpoint{5.800000in}{4.539079in}}%
\pgfpathlineto{\pgfqpoint{5.961542in}{4.539079in}}%
\pgfpathlineto{\pgfqpoint{5.961542in}{4.654809in}}%
\pgfpathlineto{\pgfqpoint{5.800000in}{4.654809in}}%
\pgfpathclose%
\pgfusepath{fill}%
\end{pgfscope}%
\begin{pgfscope}%
\pgfpathrectangle{\pgfqpoint{5.800000in}{0.720000in}}{\pgfqpoint{1.400000in}{4.620000in}}%
\pgfusepath{clip}%
\pgfsetbuttcap%
\pgfsetmiterjoin%
\definecolor{currentfill}{rgb}{0.121569,0.466667,0.705882}%
\pgfsetfillcolor{currentfill}%
\pgfsetlinewidth{0.000000pt}%
\definecolor{currentstroke}{rgb}{0.000000,0.000000,0.000000}%
\pgfsetstrokecolor{currentstroke}%
\pgfsetstrokeopacity{0.000000}%
\pgfsetdash{}{0pt}%
\pgfpathmoveto{\pgfqpoint{5.800000in}{4.654809in}}%
\pgfpathlineto{\pgfqpoint{5.957452in}{4.654809in}}%
\pgfpathlineto{\pgfqpoint{5.957452in}{4.770539in}}%
\pgfpathlineto{\pgfqpoint{5.800000in}{4.770539in}}%
\pgfpathclose%
\pgfusepath{fill}%
\end{pgfscope}%
\begin{pgfscope}%
\pgfpathrectangle{\pgfqpoint{5.800000in}{0.720000in}}{\pgfqpoint{1.400000in}{4.620000in}}%
\pgfusepath{clip}%
\pgfsetbuttcap%
\pgfsetmiterjoin%
\definecolor{currentfill}{rgb}{0.121569,0.466667,0.705882}%
\pgfsetfillcolor{currentfill}%
\pgfsetlinewidth{0.000000pt}%
\definecolor{currentstroke}{rgb}{0.000000,0.000000,0.000000}%
\pgfsetstrokecolor{currentstroke}%
\pgfsetstrokeopacity{0.000000}%
\pgfsetdash{}{0pt}%
\pgfpathmoveto{\pgfqpoint{5.800000in}{4.770539in}}%
\pgfpathlineto{\pgfqpoint{5.934959in}{4.770539in}}%
\pgfpathlineto{\pgfqpoint{5.934959in}{4.886268in}}%
\pgfpathlineto{\pgfqpoint{5.800000in}{4.886268in}}%
\pgfpathclose%
\pgfusepath{fill}%
\end{pgfscope}%
\begin{pgfscope}%
\pgfpathrectangle{\pgfqpoint{5.800000in}{0.720000in}}{\pgfqpoint{1.400000in}{4.620000in}}%
\pgfusepath{clip}%
\pgfsetbuttcap%
\pgfsetmiterjoin%
\definecolor{currentfill}{rgb}{0.121569,0.466667,0.705882}%
\pgfsetfillcolor{currentfill}%
\pgfsetlinewidth{0.000000pt}%
\definecolor{currentstroke}{rgb}{0.000000,0.000000,0.000000}%
\pgfsetstrokecolor{currentstroke}%
\pgfsetstrokeopacity{0.000000}%
\pgfsetdash{}{0pt}%
\pgfpathmoveto{\pgfqpoint{5.800000in}{4.886268in}}%
\pgfpathlineto{\pgfqpoint{5.928824in}{4.886268in}}%
\pgfpathlineto{\pgfqpoint{5.928824in}{5.001998in}}%
\pgfpathlineto{\pgfqpoint{5.800000in}{5.001998in}}%
\pgfpathclose%
\pgfusepath{fill}%
\end{pgfscope}%
\begin{pgfscope}%
\pgfpathrectangle{\pgfqpoint{5.800000in}{0.720000in}}{\pgfqpoint{1.400000in}{4.620000in}}%
\pgfusepath{clip}%
\pgfsetbuttcap%
\pgfsetmiterjoin%
\definecolor{currentfill}{rgb}{0.121569,0.466667,0.705882}%
\pgfsetfillcolor{currentfill}%
\pgfsetlinewidth{0.000000pt}%
\definecolor{currentstroke}{rgb}{0.000000,0.000000,0.000000}%
\pgfsetstrokecolor{currentstroke}%
\pgfsetstrokeopacity{0.000000}%
\pgfsetdash{}{0pt}%
\pgfpathmoveto{\pgfqpoint{5.800000in}{5.001998in}}%
\pgfpathlineto{\pgfqpoint{5.892017in}{5.001998in}}%
\pgfpathlineto{\pgfqpoint{5.892017in}{5.117728in}}%
\pgfpathlineto{\pgfqpoint{5.800000in}{5.117728in}}%
\pgfpathclose%
\pgfusepath{fill}%
\end{pgfscope}%
\begin{pgfscope}%
\pgfpathrectangle{\pgfqpoint{5.800000in}{0.720000in}}{\pgfqpoint{1.400000in}{4.620000in}}%
\pgfusepath{clip}%
\pgfsetbuttcap%
\pgfsetmiterjoin%
\definecolor{currentfill}{rgb}{0.121569,0.466667,0.705882}%
\pgfsetfillcolor{currentfill}%
\pgfsetlinewidth{0.000000pt}%
\definecolor{currentstroke}{rgb}{0.000000,0.000000,0.000000}%
\pgfsetstrokecolor{currentstroke}%
\pgfsetstrokeopacity{0.000000}%
\pgfsetdash{}{0pt}%
\pgfpathmoveto{\pgfqpoint{5.800000in}{5.117728in}}%
\pgfpathlineto{\pgfqpoint{5.889973in}{5.117728in}}%
\pgfpathlineto{\pgfqpoint{5.889973in}{5.233458in}}%
\pgfpathlineto{\pgfqpoint{5.800000in}{5.233458in}}%
\pgfpathclose%
\pgfusepath{fill}%
\end{pgfscope}%
\begin{pgfscope}%
\pgfpathrectangle{\pgfqpoint{5.800000in}{0.720000in}}{\pgfqpoint{1.400000in}{4.620000in}}%
\pgfusepath{clip}%
\pgfsetbuttcap%
\pgfsetmiterjoin%
\definecolor{currentfill}{rgb}{0.121569,0.466667,0.705882}%
\pgfsetfillcolor{currentfill}%
\pgfsetlinewidth{0.000000pt}%
\definecolor{currentstroke}{rgb}{0.000000,0.000000,0.000000}%
\pgfsetstrokecolor{currentstroke}%
\pgfsetstrokeopacity{0.000000}%
\pgfsetdash{}{0pt}%
\pgfpathmoveto{\pgfqpoint{5.800000in}{5.233458in}}%
\pgfpathlineto{\pgfqpoint{5.867479in}{5.233458in}}%
\pgfpathlineto{\pgfqpoint{5.867479in}{5.349187in}}%
\pgfpathlineto{\pgfqpoint{5.800000in}{5.349187in}}%
\pgfpathclose%
\pgfusepath{fill}%
\end{pgfscope}%
\begin{pgfscope}%
\pgfpathrectangle{\pgfqpoint{5.800000in}{0.720000in}}{\pgfqpoint{1.400000in}{4.620000in}}%
\pgfusepath{clip}%
\pgfsetbuttcap%
\pgfsetmiterjoin%
\definecolor{currentfill}{rgb}{0.121569,0.466667,0.705882}%
\pgfsetfillcolor{currentfill}%
\pgfsetlinewidth{0.000000pt}%
\definecolor{currentstroke}{rgb}{0.000000,0.000000,0.000000}%
\pgfsetstrokecolor{currentstroke}%
\pgfsetstrokeopacity{0.000000}%
\pgfsetdash{}{0pt}%
\pgfpathmoveto{\pgfqpoint{5.800000in}{5.349187in}}%
\pgfpathlineto{\pgfqpoint{5.849076in}{5.349187in}}%
\pgfpathlineto{\pgfqpoint{5.849076in}{5.464917in}}%
\pgfpathlineto{\pgfqpoint{5.800000in}{5.464917in}}%
\pgfpathclose%
\pgfusepath{fill}%
\end{pgfscope}%
\begin{pgfscope}%
\pgfpathrectangle{\pgfqpoint{5.800000in}{0.720000in}}{\pgfqpoint{1.400000in}{4.620000in}}%
\pgfusepath{clip}%
\pgfsetbuttcap%
\pgfsetmiterjoin%
\definecolor{currentfill}{rgb}{0.121569,0.466667,0.705882}%
\pgfsetfillcolor{currentfill}%
\pgfsetlinewidth{0.000000pt}%
\definecolor{currentstroke}{rgb}{0.000000,0.000000,0.000000}%
\pgfsetstrokecolor{currentstroke}%
\pgfsetstrokeopacity{0.000000}%
\pgfsetdash{}{0pt}%
\pgfpathmoveto{\pgfqpoint{5.800000in}{5.464917in}}%
\pgfpathlineto{\pgfqpoint{5.826583in}{5.464917in}}%
\pgfpathlineto{\pgfqpoint{5.826583in}{5.580647in}}%
\pgfpathlineto{\pgfqpoint{5.800000in}{5.580647in}}%
\pgfpathclose%
\pgfusepath{fill}%
\end{pgfscope}%
\begin{pgfscope}%
\pgfpathrectangle{\pgfqpoint{5.800000in}{0.720000in}}{\pgfqpoint{1.400000in}{4.620000in}}%
\pgfusepath{clip}%
\pgfsetbuttcap%
\pgfsetmiterjoin%
\definecolor{currentfill}{rgb}{0.121569,0.466667,0.705882}%
\pgfsetfillcolor{currentfill}%
\pgfsetlinewidth{0.000000pt}%
\definecolor{currentstroke}{rgb}{0.000000,0.000000,0.000000}%
\pgfsetstrokecolor{currentstroke}%
\pgfsetstrokeopacity{0.000000}%
\pgfsetdash{}{0pt}%
\pgfpathmoveto{\pgfqpoint{5.800000in}{5.580647in}}%
\pgfpathlineto{\pgfqpoint{5.844986in}{5.580647in}}%
\pgfpathlineto{\pgfqpoint{5.844986in}{5.696376in}}%
\pgfpathlineto{\pgfqpoint{5.800000in}{5.696376in}}%
\pgfpathclose%
\pgfusepath{fill}%
\end{pgfscope}%
\begin{pgfscope}%
\pgfpathrectangle{\pgfqpoint{5.800000in}{0.720000in}}{\pgfqpoint{1.400000in}{4.620000in}}%
\pgfusepath{clip}%
\pgfsetbuttcap%
\pgfsetmiterjoin%
\definecolor{currentfill}{rgb}{0.121569,0.466667,0.705882}%
\pgfsetfillcolor{currentfill}%
\pgfsetlinewidth{0.000000pt}%
\definecolor{currentstroke}{rgb}{0.000000,0.000000,0.000000}%
\pgfsetstrokecolor{currentstroke}%
\pgfsetstrokeopacity{0.000000}%
\pgfsetdash{}{0pt}%
\pgfpathmoveto{\pgfqpoint{5.800000in}{5.696376in}}%
\pgfpathlineto{\pgfqpoint{5.836807in}{5.696376in}}%
\pgfpathlineto{\pgfqpoint{5.836807in}{5.812106in}}%
\pgfpathlineto{\pgfqpoint{5.800000in}{5.812106in}}%
\pgfpathclose%
\pgfusepath{fill}%
\end{pgfscope}%
\begin{pgfscope}%
\pgfpathrectangle{\pgfqpoint{5.800000in}{0.720000in}}{\pgfqpoint{1.400000in}{4.620000in}}%
\pgfusepath{clip}%
\pgfsetbuttcap%
\pgfsetmiterjoin%
\definecolor{currentfill}{rgb}{0.121569,0.466667,0.705882}%
\pgfsetfillcolor{currentfill}%
\pgfsetlinewidth{0.000000pt}%
\definecolor{currentstroke}{rgb}{0.000000,0.000000,0.000000}%
\pgfsetstrokecolor{currentstroke}%
\pgfsetstrokeopacity{0.000000}%
\pgfsetdash{}{0pt}%
\pgfpathmoveto{\pgfqpoint{5.800000in}{5.812106in}}%
\pgfpathlineto{\pgfqpoint{5.820448in}{5.812106in}}%
\pgfpathlineto{\pgfqpoint{5.820448in}{5.927836in}}%
\pgfpathlineto{\pgfqpoint{5.800000in}{5.927836in}}%
\pgfpathclose%
\pgfusepath{fill}%
\end{pgfscope}%
\begin{pgfscope}%
\pgfpathrectangle{\pgfqpoint{5.800000in}{0.720000in}}{\pgfqpoint{1.400000in}{4.620000in}}%
\pgfusepath{clip}%
\pgfsetbuttcap%
\pgfsetmiterjoin%
\definecolor{currentfill}{rgb}{0.121569,0.466667,0.705882}%
\pgfsetfillcolor{currentfill}%
\pgfsetlinewidth{0.000000pt}%
\definecolor{currentstroke}{rgb}{0.000000,0.000000,0.000000}%
\pgfsetstrokecolor{currentstroke}%
\pgfsetstrokeopacity{0.000000}%
\pgfsetdash{}{0pt}%
\pgfpathmoveto{\pgfqpoint{5.800000in}{5.927836in}}%
\pgfpathlineto{\pgfqpoint{5.824538in}{5.927836in}}%
\pgfpathlineto{\pgfqpoint{5.824538in}{6.043565in}}%
\pgfpathlineto{\pgfqpoint{5.800000in}{6.043565in}}%
\pgfpathclose%
\pgfusepath{fill}%
\end{pgfscope}%
\begin{pgfscope}%
\pgfpathrectangle{\pgfqpoint{5.800000in}{0.720000in}}{\pgfqpoint{1.400000in}{4.620000in}}%
\pgfusepath{clip}%
\pgfsetbuttcap%
\pgfsetmiterjoin%
\definecolor{currentfill}{rgb}{0.121569,0.466667,0.705882}%
\pgfsetfillcolor{currentfill}%
\pgfsetlinewidth{0.000000pt}%
\definecolor{currentstroke}{rgb}{0.000000,0.000000,0.000000}%
\pgfsetstrokecolor{currentstroke}%
\pgfsetstrokeopacity{0.000000}%
\pgfsetdash{}{0pt}%
\pgfpathmoveto{\pgfqpoint{5.800000in}{6.043565in}}%
\pgfpathlineto{\pgfqpoint{5.816359in}{6.043565in}}%
\pgfpathlineto{\pgfqpoint{5.816359in}{6.159295in}}%
\pgfpathlineto{\pgfqpoint{5.800000in}{6.159295in}}%
\pgfpathclose%
\pgfusepath{fill}%
\end{pgfscope}%
\begin{pgfscope}%
\pgfpathrectangle{\pgfqpoint{5.800000in}{0.720000in}}{\pgfqpoint{1.400000in}{4.620000in}}%
\pgfusepath{clip}%
\pgfsetbuttcap%
\pgfsetmiterjoin%
\definecolor{currentfill}{rgb}{0.121569,0.466667,0.705882}%
\pgfsetfillcolor{currentfill}%
\pgfsetlinewidth{0.000000pt}%
\definecolor{currentstroke}{rgb}{0.000000,0.000000,0.000000}%
\pgfsetstrokecolor{currentstroke}%
\pgfsetstrokeopacity{0.000000}%
\pgfsetdash{}{0pt}%
\pgfpathmoveto{\pgfqpoint{5.800000in}{6.159295in}}%
\pgfpathlineto{\pgfqpoint{5.814314in}{6.159295in}}%
\pgfpathlineto{\pgfqpoint{5.814314in}{6.275025in}}%
\pgfpathlineto{\pgfqpoint{5.800000in}{6.275025in}}%
\pgfpathclose%
\pgfusepath{fill}%
\end{pgfscope}%
\begin{pgfscope}%
\pgfpathrectangle{\pgfqpoint{5.800000in}{0.720000in}}{\pgfqpoint{1.400000in}{4.620000in}}%
\pgfusepath{clip}%
\pgfsetbuttcap%
\pgfsetmiterjoin%
\definecolor{currentfill}{rgb}{0.121569,0.466667,0.705882}%
\pgfsetfillcolor{currentfill}%
\pgfsetlinewidth{0.000000pt}%
\definecolor{currentstroke}{rgb}{0.000000,0.000000,0.000000}%
\pgfsetstrokecolor{currentstroke}%
\pgfsetstrokeopacity{0.000000}%
\pgfsetdash{}{0pt}%
\pgfpathmoveto{\pgfqpoint{5.800000in}{6.275025in}}%
\pgfpathlineto{\pgfqpoint{5.814314in}{6.275025in}}%
\pgfpathlineto{\pgfqpoint{5.814314in}{6.390754in}}%
\pgfpathlineto{\pgfqpoint{5.800000in}{6.390754in}}%
\pgfpathclose%
\pgfusepath{fill}%
\end{pgfscope}%
\begin{pgfscope}%
\pgfpathrectangle{\pgfqpoint{5.800000in}{0.720000in}}{\pgfqpoint{1.400000in}{4.620000in}}%
\pgfusepath{clip}%
\pgfsetbuttcap%
\pgfsetmiterjoin%
\definecolor{currentfill}{rgb}{0.121569,0.466667,0.705882}%
\pgfsetfillcolor{currentfill}%
\pgfsetlinewidth{0.000000pt}%
\definecolor{currentstroke}{rgb}{0.000000,0.000000,0.000000}%
\pgfsetstrokecolor{currentstroke}%
\pgfsetstrokeopacity{0.000000}%
\pgfsetdash{}{0pt}%
\pgfpathmoveto{\pgfqpoint{5.800000in}{6.390754in}}%
\pgfpathlineto{\pgfqpoint{5.820448in}{6.390754in}}%
\pgfpathlineto{\pgfqpoint{5.820448in}{6.506484in}}%
\pgfpathlineto{\pgfqpoint{5.800000in}{6.506484in}}%
\pgfpathclose%
\pgfusepath{fill}%
\end{pgfscope}%
\begin{pgfscope}%
\pgfpathrectangle{\pgfqpoint{5.800000in}{0.720000in}}{\pgfqpoint{1.400000in}{4.620000in}}%
\pgfusepath{clip}%
\pgfsetbuttcap%
\pgfsetmiterjoin%
\definecolor{currentfill}{rgb}{0.121569,0.466667,0.705882}%
\pgfsetfillcolor{currentfill}%
\pgfsetlinewidth{0.000000pt}%
\definecolor{currentstroke}{rgb}{0.000000,0.000000,0.000000}%
\pgfsetstrokecolor{currentstroke}%
\pgfsetstrokeopacity{0.000000}%
\pgfsetdash{}{0pt}%
\pgfpathmoveto{\pgfqpoint{5.800000in}{6.506484in}}%
\pgfpathlineto{\pgfqpoint{5.816359in}{6.506484in}}%
\pgfpathlineto{\pgfqpoint{5.816359in}{6.622214in}}%
\pgfpathlineto{\pgfqpoint{5.800000in}{6.622214in}}%
\pgfpathclose%
\pgfusepath{fill}%
\end{pgfscope}%
\begin{pgfscope}%
\pgfpathrectangle{\pgfqpoint{5.800000in}{0.720000in}}{\pgfqpoint{1.400000in}{4.620000in}}%
\pgfusepath{clip}%
\pgfsetbuttcap%
\pgfsetmiterjoin%
\definecolor{currentfill}{rgb}{0.121569,0.466667,0.705882}%
\pgfsetfillcolor{currentfill}%
\pgfsetlinewidth{0.000000pt}%
\definecolor{currentstroke}{rgb}{0.000000,0.000000,0.000000}%
\pgfsetstrokecolor{currentstroke}%
\pgfsetstrokeopacity{0.000000}%
\pgfsetdash{}{0pt}%
\pgfpathmoveto{\pgfqpoint{5.800000in}{6.622214in}}%
\pgfpathlineto{\pgfqpoint{5.810224in}{6.622214in}}%
\pgfpathlineto{\pgfqpoint{5.810224in}{6.737943in}}%
\pgfpathlineto{\pgfqpoint{5.800000in}{6.737943in}}%
\pgfpathclose%
\pgfusepath{fill}%
\end{pgfscope}%
\begin{pgfscope}%
\pgfpathrectangle{\pgfqpoint{5.800000in}{0.720000in}}{\pgfqpoint{1.400000in}{4.620000in}}%
\pgfusepath{clip}%
\pgfsetbuttcap%
\pgfsetmiterjoin%
\definecolor{currentfill}{rgb}{0.121569,0.466667,0.705882}%
\pgfsetfillcolor{currentfill}%
\pgfsetlinewidth{0.000000pt}%
\definecolor{currentstroke}{rgb}{0.000000,0.000000,0.000000}%
\pgfsetstrokecolor{currentstroke}%
\pgfsetstrokeopacity{0.000000}%
\pgfsetdash{}{0pt}%
\pgfpathmoveto{\pgfqpoint{5.800000in}{6.737943in}}%
\pgfpathlineto{\pgfqpoint{5.808179in}{6.737943in}}%
\pgfpathlineto{\pgfqpoint{5.808179in}{6.853673in}}%
\pgfpathlineto{\pgfqpoint{5.800000in}{6.853673in}}%
\pgfpathclose%
\pgfusepath{fill}%
\end{pgfscope}%
\begin{pgfscope}%
\pgfpathrectangle{\pgfqpoint{5.800000in}{0.720000in}}{\pgfqpoint{1.400000in}{4.620000in}}%
\pgfusepath{clip}%
\pgfsetbuttcap%
\pgfsetmiterjoin%
\definecolor{currentfill}{rgb}{0.121569,0.466667,0.705882}%
\pgfsetfillcolor{currentfill}%
\pgfsetlinewidth{0.000000pt}%
\definecolor{currentstroke}{rgb}{0.000000,0.000000,0.000000}%
\pgfsetstrokecolor{currentstroke}%
\pgfsetstrokeopacity{0.000000}%
\pgfsetdash{}{0pt}%
\pgfpathmoveto{\pgfqpoint{5.800000in}{6.853673in}}%
\pgfpathlineto{\pgfqpoint{5.810224in}{6.853673in}}%
\pgfpathlineto{\pgfqpoint{5.810224in}{6.969403in}}%
\pgfpathlineto{\pgfqpoint{5.800000in}{6.969403in}}%
\pgfpathclose%
\pgfusepath{fill}%
\end{pgfscope}%
\begin{pgfscope}%
\pgfpathrectangle{\pgfqpoint{5.800000in}{0.720000in}}{\pgfqpoint{1.400000in}{4.620000in}}%
\pgfusepath{clip}%
\pgfsetbuttcap%
\pgfsetmiterjoin%
\definecolor{currentfill}{rgb}{0.121569,0.466667,0.705882}%
\pgfsetfillcolor{currentfill}%
\pgfsetlinewidth{0.000000pt}%
\definecolor{currentstroke}{rgb}{0.000000,0.000000,0.000000}%
\pgfsetstrokecolor{currentstroke}%
\pgfsetstrokeopacity{0.000000}%
\pgfsetdash{}{0pt}%
\pgfpathmoveto{\pgfqpoint{5.800000in}{6.969403in}}%
\pgfpathlineto{\pgfqpoint{5.806134in}{6.969403in}}%
\pgfpathlineto{\pgfqpoint{5.806134in}{7.085132in}}%
\pgfpathlineto{\pgfqpoint{5.800000in}{7.085132in}}%
\pgfpathclose%
\pgfusepath{fill}%
\end{pgfscope}%
\begin{pgfscope}%
\pgfpathrectangle{\pgfqpoint{5.800000in}{0.720000in}}{\pgfqpoint{1.400000in}{4.620000in}}%
\pgfusepath{clip}%
\pgfsetbuttcap%
\pgfsetmiterjoin%
\definecolor{currentfill}{rgb}{0.121569,0.466667,0.705882}%
\pgfsetfillcolor{currentfill}%
\pgfsetlinewidth{0.000000pt}%
\definecolor{currentstroke}{rgb}{0.000000,0.000000,0.000000}%
\pgfsetstrokecolor{currentstroke}%
\pgfsetstrokeopacity{0.000000}%
\pgfsetdash{}{0pt}%
\pgfpathmoveto{\pgfqpoint{5.800000in}{7.085132in}}%
\pgfpathlineto{\pgfqpoint{5.808179in}{7.085132in}}%
\pgfpathlineto{\pgfqpoint{5.808179in}{7.200862in}}%
\pgfpathlineto{\pgfqpoint{5.800000in}{7.200862in}}%
\pgfpathclose%
\pgfusepath{fill}%
\end{pgfscope}%
\begin{pgfscope}%
\pgfpathrectangle{\pgfqpoint{5.800000in}{0.720000in}}{\pgfqpoint{1.400000in}{4.620000in}}%
\pgfusepath{clip}%
\pgfsetbuttcap%
\pgfsetmiterjoin%
\definecolor{currentfill}{rgb}{0.121569,0.466667,0.705882}%
\pgfsetfillcolor{currentfill}%
\pgfsetlinewidth{0.000000pt}%
\definecolor{currentstroke}{rgb}{0.000000,0.000000,0.000000}%
\pgfsetstrokecolor{currentstroke}%
\pgfsetstrokeopacity{0.000000}%
\pgfsetdash{}{0pt}%
\pgfpathmoveto{\pgfqpoint{5.800000in}{7.200862in}}%
\pgfpathlineto{\pgfqpoint{5.804090in}{7.200862in}}%
\pgfpathlineto{\pgfqpoint{5.804090in}{7.316592in}}%
\pgfpathlineto{\pgfqpoint{5.800000in}{7.316592in}}%
\pgfpathclose%
\pgfusepath{fill}%
\end{pgfscope}%
\begin{pgfscope}%
\pgfpathrectangle{\pgfqpoint{5.800000in}{0.720000in}}{\pgfqpoint{1.400000in}{4.620000in}}%
\pgfusepath{clip}%
\pgfsetbuttcap%
\pgfsetmiterjoin%
\definecolor{currentfill}{rgb}{0.121569,0.466667,0.705882}%
\pgfsetfillcolor{currentfill}%
\pgfsetlinewidth{0.000000pt}%
\definecolor{currentstroke}{rgb}{0.000000,0.000000,0.000000}%
\pgfsetstrokecolor{currentstroke}%
\pgfsetstrokeopacity{0.000000}%
\pgfsetdash{}{0pt}%
\pgfpathmoveto{\pgfqpoint{5.800000in}{7.316592in}}%
\pgfpathlineto{\pgfqpoint{5.810224in}{7.316592in}}%
\pgfpathlineto{\pgfqpoint{5.810224in}{7.432321in}}%
\pgfpathlineto{\pgfqpoint{5.800000in}{7.432321in}}%
\pgfpathclose%
\pgfusepath{fill}%
\end{pgfscope}%
\begin{pgfscope}%
\pgfpathrectangle{\pgfqpoint{5.800000in}{0.720000in}}{\pgfqpoint{1.400000in}{4.620000in}}%
\pgfusepath{clip}%
\pgfsetbuttcap%
\pgfsetmiterjoin%
\definecolor{currentfill}{rgb}{0.121569,0.466667,0.705882}%
\pgfsetfillcolor{currentfill}%
\pgfsetlinewidth{0.000000pt}%
\definecolor{currentstroke}{rgb}{0.000000,0.000000,0.000000}%
\pgfsetstrokecolor{currentstroke}%
\pgfsetstrokeopacity{0.000000}%
\pgfsetdash{}{0pt}%
\pgfpathmoveto{\pgfqpoint{5.800000in}{7.432321in}}%
\pgfpathlineto{\pgfqpoint{5.800000in}{7.432321in}}%
\pgfpathlineto{\pgfqpoint{5.800000in}{7.548051in}}%
\pgfpathlineto{\pgfqpoint{5.800000in}{7.548051in}}%
\pgfpathclose%
\pgfusepath{fill}%
\end{pgfscope}%
\begin{pgfscope}%
\pgfpathrectangle{\pgfqpoint{5.800000in}{0.720000in}}{\pgfqpoint{1.400000in}{4.620000in}}%
\pgfusepath{clip}%
\pgfsetbuttcap%
\pgfsetmiterjoin%
\definecolor{currentfill}{rgb}{0.121569,0.466667,0.705882}%
\pgfsetfillcolor{currentfill}%
\pgfsetlinewidth{0.000000pt}%
\definecolor{currentstroke}{rgb}{0.000000,0.000000,0.000000}%
\pgfsetstrokecolor{currentstroke}%
\pgfsetstrokeopacity{0.000000}%
\pgfsetdash{}{0pt}%
\pgfpathmoveto{\pgfqpoint{5.800000in}{7.548051in}}%
\pgfpathlineto{\pgfqpoint{5.802045in}{7.548051in}}%
\pgfpathlineto{\pgfqpoint{5.802045in}{7.663781in}}%
\pgfpathlineto{\pgfqpoint{5.800000in}{7.663781in}}%
\pgfpathclose%
\pgfusepath{fill}%
\end{pgfscope}%
\begin{pgfscope}%
\pgfpathrectangle{\pgfqpoint{5.800000in}{0.720000in}}{\pgfqpoint{1.400000in}{4.620000in}}%
\pgfusepath{clip}%
\pgfsetbuttcap%
\pgfsetmiterjoin%
\definecolor{currentfill}{rgb}{0.121569,0.466667,0.705882}%
\pgfsetfillcolor{currentfill}%
\pgfsetlinewidth{0.000000pt}%
\definecolor{currentstroke}{rgb}{0.000000,0.000000,0.000000}%
\pgfsetstrokecolor{currentstroke}%
\pgfsetstrokeopacity{0.000000}%
\pgfsetdash{}{0pt}%
\pgfpathmoveto{\pgfqpoint{5.800000in}{7.663781in}}%
\pgfpathlineto{\pgfqpoint{5.806134in}{7.663781in}}%
\pgfpathlineto{\pgfqpoint{5.806134in}{7.779510in}}%
\pgfpathlineto{\pgfqpoint{5.800000in}{7.779510in}}%
\pgfpathclose%
\pgfusepath{fill}%
\end{pgfscope}%
\begin{pgfscope}%
\pgfpathrectangle{\pgfqpoint{5.800000in}{0.720000in}}{\pgfqpoint{1.400000in}{4.620000in}}%
\pgfusepath{clip}%
\pgfsetbuttcap%
\pgfsetmiterjoin%
\definecolor{currentfill}{rgb}{0.121569,0.466667,0.705882}%
\pgfsetfillcolor{currentfill}%
\pgfsetlinewidth{0.000000pt}%
\definecolor{currentstroke}{rgb}{0.000000,0.000000,0.000000}%
\pgfsetstrokecolor{currentstroke}%
\pgfsetstrokeopacity{0.000000}%
\pgfsetdash{}{0pt}%
\pgfpathmoveto{\pgfqpoint{5.800000in}{7.779510in}}%
\pgfpathlineto{\pgfqpoint{5.804090in}{7.779510in}}%
\pgfpathlineto{\pgfqpoint{5.804090in}{7.895240in}}%
\pgfpathlineto{\pgfqpoint{5.800000in}{7.895240in}}%
\pgfpathclose%
\pgfusepath{fill}%
\end{pgfscope}%
\begin{pgfscope}%
\pgfpathrectangle{\pgfqpoint{5.800000in}{0.720000in}}{\pgfqpoint{1.400000in}{4.620000in}}%
\pgfusepath{clip}%
\pgfsetbuttcap%
\pgfsetmiterjoin%
\definecolor{currentfill}{rgb}{0.121569,0.466667,0.705882}%
\pgfsetfillcolor{currentfill}%
\pgfsetlinewidth{0.000000pt}%
\definecolor{currentstroke}{rgb}{0.000000,0.000000,0.000000}%
\pgfsetstrokecolor{currentstroke}%
\pgfsetstrokeopacity{0.000000}%
\pgfsetdash{}{0pt}%
\pgfpathmoveto{\pgfqpoint{5.800000in}{7.895240in}}%
\pgfpathlineto{\pgfqpoint{5.804090in}{7.895240in}}%
\pgfpathlineto{\pgfqpoint{5.804090in}{8.010970in}}%
\pgfpathlineto{\pgfqpoint{5.800000in}{8.010970in}}%
\pgfpathclose%
\pgfusepath{fill}%
\end{pgfscope}%
\begin{pgfscope}%
\pgfpathrectangle{\pgfqpoint{5.800000in}{0.720000in}}{\pgfqpoint{1.400000in}{4.620000in}}%
\pgfusepath{clip}%
\pgfsetbuttcap%
\pgfsetmiterjoin%
\definecolor{currentfill}{rgb}{0.121569,0.466667,0.705882}%
\pgfsetfillcolor{currentfill}%
\pgfsetlinewidth{0.000000pt}%
\definecolor{currentstroke}{rgb}{0.000000,0.000000,0.000000}%
\pgfsetstrokecolor{currentstroke}%
\pgfsetstrokeopacity{0.000000}%
\pgfsetdash{}{0pt}%
\pgfpathmoveto{\pgfqpoint{5.800000in}{8.010970in}}%
\pgfpathlineto{\pgfqpoint{5.806134in}{8.010970in}}%
\pgfpathlineto{\pgfqpoint{5.806134in}{8.126700in}}%
\pgfpathlineto{\pgfqpoint{5.800000in}{8.126700in}}%
\pgfpathclose%
\pgfusepath{fill}%
\end{pgfscope}%
\begin{pgfscope}%
\pgfpathrectangle{\pgfqpoint{5.800000in}{0.720000in}}{\pgfqpoint{1.400000in}{4.620000in}}%
\pgfusepath{clip}%
\pgfsetbuttcap%
\pgfsetmiterjoin%
\definecolor{currentfill}{rgb}{0.121569,0.466667,0.705882}%
\pgfsetfillcolor{currentfill}%
\pgfsetlinewidth{0.000000pt}%
\definecolor{currentstroke}{rgb}{0.000000,0.000000,0.000000}%
\pgfsetstrokecolor{currentstroke}%
\pgfsetstrokeopacity{0.000000}%
\pgfsetdash{}{0pt}%
\pgfpathmoveto{\pgfqpoint{5.800000in}{8.126700in}}%
\pgfpathlineto{\pgfqpoint{5.804090in}{8.126700in}}%
\pgfpathlineto{\pgfqpoint{5.804090in}{8.242429in}}%
\pgfpathlineto{\pgfqpoint{5.800000in}{8.242429in}}%
\pgfpathclose%
\pgfusepath{fill}%
\end{pgfscope}%
\begin{pgfscope}%
\pgfpathrectangle{\pgfqpoint{5.800000in}{0.720000in}}{\pgfqpoint{1.400000in}{4.620000in}}%
\pgfusepath{clip}%
\pgfsetbuttcap%
\pgfsetmiterjoin%
\definecolor{currentfill}{rgb}{0.121569,0.466667,0.705882}%
\pgfsetfillcolor{currentfill}%
\pgfsetlinewidth{0.000000pt}%
\definecolor{currentstroke}{rgb}{0.000000,0.000000,0.000000}%
\pgfsetstrokecolor{currentstroke}%
\pgfsetstrokeopacity{0.000000}%
\pgfsetdash{}{0pt}%
\pgfpathmoveto{\pgfqpoint{5.800000in}{8.242429in}}%
\pgfpathlineto{\pgfqpoint{5.810224in}{8.242429in}}%
\pgfpathlineto{\pgfqpoint{5.810224in}{8.358159in}}%
\pgfpathlineto{\pgfqpoint{5.800000in}{8.358159in}}%
\pgfpathclose%
\pgfusepath{fill}%
\end{pgfscope}%
\begin{pgfscope}%
\pgfpathrectangle{\pgfqpoint{5.800000in}{0.720000in}}{\pgfqpoint{1.400000in}{4.620000in}}%
\pgfusepath{clip}%
\pgfsetbuttcap%
\pgfsetmiterjoin%
\definecolor{currentfill}{rgb}{0.121569,0.466667,0.705882}%
\pgfsetfillcolor{currentfill}%
\pgfsetlinewidth{0.000000pt}%
\definecolor{currentstroke}{rgb}{0.000000,0.000000,0.000000}%
\pgfsetstrokecolor{currentstroke}%
\pgfsetstrokeopacity{0.000000}%
\pgfsetdash{}{0pt}%
\pgfpathmoveto{\pgfqpoint{5.800000in}{8.358159in}}%
\pgfpathlineto{\pgfqpoint{5.802045in}{8.358159in}}%
\pgfpathlineto{\pgfqpoint{5.802045in}{8.473889in}}%
\pgfpathlineto{\pgfqpoint{5.800000in}{8.473889in}}%
\pgfpathclose%
\pgfusepath{fill}%
\end{pgfscope}%
\begin{pgfscope}%
\pgfpathrectangle{\pgfqpoint{5.800000in}{0.720000in}}{\pgfqpoint{1.400000in}{4.620000in}}%
\pgfusepath{clip}%
\pgfsetbuttcap%
\pgfsetmiterjoin%
\definecolor{currentfill}{rgb}{0.121569,0.466667,0.705882}%
\pgfsetfillcolor{currentfill}%
\pgfsetlinewidth{0.000000pt}%
\definecolor{currentstroke}{rgb}{0.000000,0.000000,0.000000}%
\pgfsetstrokecolor{currentstroke}%
\pgfsetstrokeopacity{0.000000}%
\pgfsetdash{}{0pt}%
\pgfpathmoveto{\pgfqpoint{5.800000in}{8.473889in}}%
\pgfpathlineto{\pgfqpoint{5.804090in}{8.473889in}}%
\pgfpathlineto{\pgfqpoint{5.804090in}{8.589618in}}%
\pgfpathlineto{\pgfqpoint{5.800000in}{8.589618in}}%
\pgfpathclose%
\pgfusepath{fill}%
\end{pgfscope}%
\begin{pgfscope}%
\pgfpathrectangle{\pgfqpoint{5.800000in}{0.720000in}}{\pgfqpoint{1.400000in}{4.620000in}}%
\pgfusepath{clip}%
\pgfsetbuttcap%
\pgfsetmiterjoin%
\definecolor{currentfill}{rgb}{0.121569,0.466667,0.705882}%
\pgfsetfillcolor{currentfill}%
\pgfsetlinewidth{0.000000pt}%
\definecolor{currentstroke}{rgb}{0.000000,0.000000,0.000000}%
\pgfsetstrokecolor{currentstroke}%
\pgfsetstrokeopacity{0.000000}%
\pgfsetdash{}{0pt}%
\pgfpathmoveto{\pgfqpoint{5.800000in}{8.589618in}}%
\pgfpathlineto{\pgfqpoint{5.802045in}{8.589618in}}%
\pgfpathlineto{\pgfqpoint{5.802045in}{8.705348in}}%
\pgfpathlineto{\pgfqpoint{5.800000in}{8.705348in}}%
\pgfpathclose%
\pgfusepath{fill}%
\end{pgfscope}%
\begin{pgfscope}%
\pgfpathrectangle{\pgfqpoint{5.800000in}{0.720000in}}{\pgfqpoint{1.400000in}{4.620000in}}%
\pgfusepath{clip}%
\pgfsetbuttcap%
\pgfsetmiterjoin%
\definecolor{currentfill}{rgb}{0.121569,0.466667,0.705882}%
\pgfsetfillcolor{currentfill}%
\pgfsetlinewidth{0.000000pt}%
\definecolor{currentstroke}{rgb}{0.000000,0.000000,0.000000}%
\pgfsetstrokecolor{currentstroke}%
\pgfsetstrokeopacity{0.000000}%
\pgfsetdash{}{0pt}%
\pgfpathmoveto{\pgfqpoint{5.800000in}{8.705348in}}%
\pgfpathlineto{\pgfqpoint{5.802045in}{8.705348in}}%
\pgfpathlineto{\pgfqpoint{5.802045in}{8.821078in}}%
\pgfpathlineto{\pgfqpoint{5.800000in}{8.821078in}}%
\pgfpathclose%
\pgfusepath{fill}%
\end{pgfscope}%
\begin{pgfscope}%
\pgfpathrectangle{\pgfqpoint{5.800000in}{0.720000in}}{\pgfqpoint{1.400000in}{4.620000in}}%
\pgfusepath{clip}%
\pgfsetbuttcap%
\pgfsetmiterjoin%
\definecolor{currentfill}{rgb}{0.121569,0.466667,0.705882}%
\pgfsetfillcolor{currentfill}%
\pgfsetlinewidth{0.000000pt}%
\definecolor{currentstroke}{rgb}{0.000000,0.000000,0.000000}%
\pgfsetstrokecolor{currentstroke}%
\pgfsetstrokeopacity{0.000000}%
\pgfsetdash{}{0pt}%
\pgfpathmoveto{\pgfqpoint{5.800000in}{8.821078in}}%
\pgfpathlineto{\pgfqpoint{5.800000in}{8.821078in}}%
\pgfpathlineto{\pgfqpoint{5.800000in}{8.936807in}}%
\pgfpathlineto{\pgfqpoint{5.800000in}{8.936807in}}%
\pgfpathclose%
\pgfusepath{fill}%
\end{pgfscope}%
\begin{pgfscope}%
\pgfpathrectangle{\pgfqpoint{5.800000in}{0.720000in}}{\pgfqpoint{1.400000in}{4.620000in}}%
\pgfusepath{clip}%
\pgfsetbuttcap%
\pgfsetmiterjoin%
\definecolor{currentfill}{rgb}{0.121569,0.466667,0.705882}%
\pgfsetfillcolor{currentfill}%
\pgfsetlinewidth{0.000000pt}%
\definecolor{currentstroke}{rgb}{0.000000,0.000000,0.000000}%
\pgfsetstrokecolor{currentstroke}%
\pgfsetstrokeopacity{0.000000}%
\pgfsetdash{}{0pt}%
\pgfpathmoveto{\pgfqpoint{5.800000in}{8.936807in}}%
\pgfpathlineto{\pgfqpoint{5.800000in}{8.936807in}}%
\pgfpathlineto{\pgfqpoint{5.800000in}{9.052537in}}%
\pgfpathlineto{\pgfqpoint{5.800000in}{9.052537in}}%
\pgfpathclose%
\pgfusepath{fill}%
\end{pgfscope}%
\begin{pgfscope}%
\pgfpathrectangle{\pgfqpoint{5.800000in}{0.720000in}}{\pgfqpoint{1.400000in}{4.620000in}}%
\pgfusepath{clip}%
\pgfsetbuttcap%
\pgfsetmiterjoin%
\definecolor{currentfill}{rgb}{0.121569,0.466667,0.705882}%
\pgfsetfillcolor{currentfill}%
\pgfsetlinewidth{0.000000pt}%
\definecolor{currentstroke}{rgb}{0.000000,0.000000,0.000000}%
\pgfsetstrokecolor{currentstroke}%
\pgfsetstrokeopacity{0.000000}%
\pgfsetdash{}{0pt}%
\pgfpathmoveto{\pgfqpoint{5.800000in}{9.052537in}}%
\pgfpathlineto{\pgfqpoint{5.800000in}{9.052537in}}%
\pgfpathlineto{\pgfqpoint{5.800000in}{9.168267in}}%
\pgfpathlineto{\pgfqpoint{5.800000in}{9.168267in}}%
\pgfpathclose%
\pgfusepath{fill}%
\end{pgfscope}%
\begin{pgfscope}%
\pgfpathrectangle{\pgfqpoint{5.800000in}{0.720000in}}{\pgfqpoint{1.400000in}{4.620000in}}%
\pgfusepath{clip}%
\pgfsetbuttcap%
\pgfsetmiterjoin%
\definecolor{currentfill}{rgb}{0.121569,0.466667,0.705882}%
\pgfsetfillcolor{currentfill}%
\pgfsetlinewidth{0.000000pt}%
\definecolor{currentstroke}{rgb}{0.000000,0.000000,0.000000}%
\pgfsetstrokecolor{currentstroke}%
\pgfsetstrokeopacity{0.000000}%
\pgfsetdash{}{0pt}%
\pgfpathmoveto{\pgfqpoint{5.800000in}{9.168267in}}%
\pgfpathlineto{\pgfqpoint{5.802045in}{9.168267in}}%
\pgfpathlineto{\pgfqpoint{5.802045in}{9.283996in}}%
\pgfpathlineto{\pgfqpoint{5.800000in}{9.283996in}}%
\pgfpathclose%
\pgfusepath{fill}%
\end{pgfscope}%
\begin{pgfscope}%
\pgfpathrectangle{\pgfqpoint{5.800000in}{0.720000in}}{\pgfqpoint{1.400000in}{4.620000in}}%
\pgfusepath{clip}%
\pgfsetbuttcap%
\pgfsetmiterjoin%
\definecolor{currentfill}{rgb}{0.121569,0.466667,0.705882}%
\pgfsetfillcolor{currentfill}%
\pgfsetlinewidth{0.000000pt}%
\definecolor{currentstroke}{rgb}{0.000000,0.000000,0.000000}%
\pgfsetstrokecolor{currentstroke}%
\pgfsetstrokeopacity{0.000000}%
\pgfsetdash{}{0pt}%
\pgfpathmoveto{\pgfqpoint{5.800000in}{9.283996in}}%
\pgfpathlineto{\pgfqpoint{5.800000in}{9.283996in}}%
\pgfpathlineto{\pgfqpoint{5.800000in}{9.399726in}}%
\pgfpathlineto{\pgfqpoint{5.800000in}{9.399726in}}%
\pgfpathclose%
\pgfusepath{fill}%
\end{pgfscope}%
\begin{pgfscope}%
\pgfpathrectangle{\pgfqpoint{5.800000in}{0.720000in}}{\pgfqpoint{1.400000in}{4.620000in}}%
\pgfusepath{clip}%
\pgfsetbuttcap%
\pgfsetmiterjoin%
\definecolor{currentfill}{rgb}{0.121569,0.466667,0.705882}%
\pgfsetfillcolor{currentfill}%
\pgfsetlinewidth{0.000000pt}%
\definecolor{currentstroke}{rgb}{0.000000,0.000000,0.000000}%
\pgfsetstrokecolor{currentstroke}%
\pgfsetstrokeopacity{0.000000}%
\pgfsetdash{}{0pt}%
\pgfpathmoveto{\pgfqpoint{5.800000in}{9.399726in}}%
\pgfpathlineto{\pgfqpoint{5.800000in}{9.399726in}}%
\pgfpathlineto{\pgfqpoint{5.800000in}{9.515456in}}%
\pgfpathlineto{\pgfqpoint{5.800000in}{9.515456in}}%
\pgfpathclose%
\pgfusepath{fill}%
\end{pgfscope}%
\begin{pgfscope}%
\pgfpathrectangle{\pgfqpoint{5.800000in}{0.720000in}}{\pgfqpoint{1.400000in}{4.620000in}}%
\pgfusepath{clip}%
\pgfsetbuttcap%
\pgfsetmiterjoin%
\definecolor{currentfill}{rgb}{0.121569,0.466667,0.705882}%
\pgfsetfillcolor{currentfill}%
\pgfsetlinewidth{0.000000pt}%
\definecolor{currentstroke}{rgb}{0.000000,0.000000,0.000000}%
\pgfsetstrokecolor{currentstroke}%
\pgfsetstrokeopacity{0.000000}%
\pgfsetdash{}{0pt}%
\pgfpathmoveto{\pgfqpoint{5.800000in}{9.515456in}}%
\pgfpathlineto{\pgfqpoint{5.800000in}{9.515456in}}%
\pgfpathlineto{\pgfqpoint{5.800000in}{9.631185in}}%
\pgfpathlineto{\pgfqpoint{5.800000in}{9.631185in}}%
\pgfpathclose%
\pgfusepath{fill}%
\end{pgfscope}%
\begin{pgfscope}%
\pgfpathrectangle{\pgfqpoint{5.800000in}{0.720000in}}{\pgfqpoint{1.400000in}{4.620000in}}%
\pgfusepath{clip}%
\pgfsetbuttcap%
\pgfsetmiterjoin%
\definecolor{currentfill}{rgb}{0.121569,0.466667,0.705882}%
\pgfsetfillcolor{currentfill}%
\pgfsetlinewidth{0.000000pt}%
\definecolor{currentstroke}{rgb}{0.000000,0.000000,0.000000}%
\pgfsetstrokecolor{currentstroke}%
\pgfsetstrokeopacity{0.000000}%
\pgfsetdash{}{0pt}%
\pgfpathmoveto{\pgfqpoint{5.800000in}{9.631185in}}%
\pgfpathlineto{\pgfqpoint{5.800000in}{9.631185in}}%
\pgfpathlineto{\pgfqpoint{5.800000in}{9.746915in}}%
\pgfpathlineto{\pgfqpoint{5.800000in}{9.746915in}}%
\pgfpathclose%
\pgfusepath{fill}%
\end{pgfscope}%
\begin{pgfscope}%
\pgfpathrectangle{\pgfqpoint{5.800000in}{0.720000in}}{\pgfqpoint{1.400000in}{4.620000in}}%
\pgfusepath{clip}%
\pgfsetbuttcap%
\pgfsetmiterjoin%
\definecolor{currentfill}{rgb}{0.121569,0.466667,0.705882}%
\pgfsetfillcolor{currentfill}%
\pgfsetlinewidth{0.000000pt}%
\definecolor{currentstroke}{rgb}{0.000000,0.000000,0.000000}%
\pgfsetstrokecolor{currentstroke}%
\pgfsetstrokeopacity{0.000000}%
\pgfsetdash{}{0pt}%
\pgfpathmoveto{\pgfqpoint{5.800000in}{9.746915in}}%
\pgfpathlineto{\pgfqpoint{5.800000in}{9.746915in}}%
\pgfpathlineto{\pgfqpoint{5.800000in}{9.862645in}}%
\pgfpathlineto{\pgfqpoint{5.800000in}{9.862645in}}%
\pgfpathclose%
\pgfusepath{fill}%
\end{pgfscope}%
\begin{pgfscope}%
\pgfpathrectangle{\pgfqpoint{5.800000in}{0.720000in}}{\pgfqpoint{1.400000in}{4.620000in}}%
\pgfusepath{clip}%
\pgfsetbuttcap%
\pgfsetmiterjoin%
\definecolor{currentfill}{rgb}{0.121569,0.466667,0.705882}%
\pgfsetfillcolor{currentfill}%
\pgfsetlinewidth{0.000000pt}%
\definecolor{currentstroke}{rgb}{0.000000,0.000000,0.000000}%
\pgfsetstrokecolor{currentstroke}%
\pgfsetstrokeopacity{0.000000}%
\pgfsetdash{}{0pt}%
\pgfpathmoveto{\pgfqpoint{5.800000in}{9.862645in}}%
\pgfpathlineto{\pgfqpoint{5.804090in}{9.862645in}}%
\pgfpathlineto{\pgfqpoint{5.804090in}{9.978374in}}%
\pgfpathlineto{\pgfqpoint{5.800000in}{9.978374in}}%
\pgfpathclose%
\pgfusepath{fill}%
\end{pgfscope}%
\begin{pgfscope}%
\pgfpathrectangle{\pgfqpoint{5.800000in}{0.720000in}}{\pgfqpoint{1.400000in}{4.620000in}}%
\pgfusepath{clip}%
\pgfsetbuttcap%
\pgfsetmiterjoin%
\definecolor{currentfill}{rgb}{0.121569,0.466667,0.705882}%
\pgfsetfillcolor{currentfill}%
\pgfsetlinewidth{0.000000pt}%
\definecolor{currentstroke}{rgb}{0.000000,0.000000,0.000000}%
\pgfsetstrokecolor{currentstroke}%
\pgfsetstrokeopacity{0.000000}%
\pgfsetdash{}{0pt}%
\pgfpathmoveto{\pgfqpoint{5.800000in}{9.978374in}}%
\pgfpathlineto{\pgfqpoint{5.800000in}{9.978374in}}%
\pgfpathlineto{\pgfqpoint{5.800000in}{10.094104in}}%
\pgfpathlineto{\pgfqpoint{5.800000in}{10.094104in}}%
\pgfpathclose%
\pgfusepath{fill}%
\end{pgfscope}%
\begin{pgfscope}%
\pgfpathrectangle{\pgfqpoint{5.800000in}{0.720000in}}{\pgfqpoint{1.400000in}{4.620000in}}%
\pgfusepath{clip}%
\pgfsetbuttcap%
\pgfsetmiterjoin%
\definecolor{currentfill}{rgb}{0.121569,0.466667,0.705882}%
\pgfsetfillcolor{currentfill}%
\pgfsetlinewidth{0.000000pt}%
\definecolor{currentstroke}{rgb}{0.000000,0.000000,0.000000}%
\pgfsetstrokecolor{currentstroke}%
\pgfsetstrokeopacity{0.000000}%
\pgfsetdash{}{0pt}%
\pgfpathmoveto{\pgfqpoint{5.800000in}{10.094104in}}%
\pgfpathlineto{\pgfqpoint{5.800000in}{10.094104in}}%
\pgfpathlineto{\pgfqpoint{5.800000in}{10.209834in}}%
\pgfpathlineto{\pgfqpoint{5.800000in}{10.209834in}}%
\pgfpathclose%
\pgfusepath{fill}%
\end{pgfscope}%
\begin{pgfscope}%
\pgfpathrectangle{\pgfqpoint{5.800000in}{0.720000in}}{\pgfqpoint{1.400000in}{4.620000in}}%
\pgfusepath{clip}%
\pgfsetbuttcap%
\pgfsetmiterjoin%
\definecolor{currentfill}{rgb}{0.121569,0.466667,0.705882}%
\pgfsetfillcolor{currentfill}%
\pgfsetlinewidth{0.000000pt}%
\definecolor{currentstroke}{rgb}{0.000000,0.000000,0.000000}%
\pgfsetstrokecolor{currentstroke}%
\pgfsetstrokeopacity{0.000000}%
\pgfsetdash{}{0pt}%
\pgfpathmoveto{\pgfqpoint{5.800000in}{10.209834in}}%
\pgfpathlineto{\pgfqpoint{5.802045in}{10.209834in}}%
\pgfpathlineto{\pgfqpoint{5.802045in}{10.325563in}}%
\pgfpathlineto{\pgfqpoint{5.800000in}{10.325563in}}%
\pgfpathclose%
\pgfusepath{fill}%
\end{pgfscope}%
\begin{pgfscope}%
\pgfpathrectangle{\pgfqpoint{5.800000in}{0.720000in}}{\pgfqpoint{1.400000in}{4.620000in}}%
\pgfusepath{clip}%
\pgfsetbuttcap%
\pgfsetmiterjoin%
\definecolor{currentfill}{rgb}{0.121569,0.466667,0.705882}%
\pgfsetfillcolor{currentfill}%
\pgfsetlinewidth{0.000000pt}%
\definecolor{currentstroke}{rgb}{0.000000,0.000000,0.000000}%
\pgfsetstrokecolor{currentstroke}%
\pgfsetstrokeopacity{0.000000}%
\pgfsetdash{}{0pt}%
\pgfpathmoveto{\pgfqpoint{5.800000in}{10.325563in}}%
\pgfpathlineto{\pgfqpoint{5.802045in}{10.325563in}}%
\pgfpathlineto{\pgfqpoint{5.802045in}{10.441293in}}%
\pgfpathlineto{\pgfqpoint{5.800000in}{10.441293in}}%
\pgfpathclose%
\pgfusepath{fill}%
\end{pgfscope}%
\begin{pgfscope}%
\pgfpathrectangle{\pgfqpoint{5.800000in}{0.720000in}}{\pgfqpoint{1.400000in}{4.620000in}}%
\pgfusepath{clip}%
\pgfsetbuttcap%
\pgfsetmiterjoin%
\definecolor{currentfill}{rgb}{0.121569,0.466667,0.705882}%
\pgfsetfillcolor{currentfill}%
\pgfsetlinewidth{0.000000pt}%
\definecolor{currentstroke}{rgb}{0.000000,0.000000,0.000000}%
\pgfsetstrokecolor{currentstroke}%
\pgfsetstrokeopacity{0.000000}%
\pgfsetdash{}{0pt}%
\pgfpathmoveto{\pgfqpoint{5.800000in}{10.441293in}}%
\pgfpathlineto{\pgfqpoint{5.800000in}{10.441293in}}%
\pgfpathlineto{\pgfqpoint{5.800000in}{10.557023in}}%
\pgfpathlineto{\pgfqpoint{5.800000in}{10.557023in}}%
\pgfpathclose%
\pgfusepath{fill}%
\end{pgfscope}%
\begin{pgfscope}%
\pgfpathrectangle{\pgfqpoint{5.800000in}{0.720000in}}{\pgfqpoint{1.400000in}{4.620000in}}%
\pgfusepath{clip}%
\pgfsetbuttcap%
\pgfsetmiterjoin%
\definecolor{currentfill}{rgb}{0.121569,0.466667,0.705882}%
\pgfsetfillcolor{currentfill}%
\pgfsetlinewidth{0.000000pt}%
\definecolor{currentstroke}{rgb}{0.000000,0.000000,0.000000}%
\pgfsetstrokecolor{currentstroke}%
\pgfsetstrokeopacity{0.000000}%
\pgfsetdash{}{0pt}%
\pgfpathmoveto{\pgfqpoint{5.800000in}{10.557023in}}%
\pgfpathlineto{\pgfqpoint{5.800000in}{10.557023in}}%
\pgfpathlineto{\pgfqpoint{5.800000in}{10.672752in}}%
\pgfpathlineto{\pgfqpoint{5.800000in}{10.672752in}}%
\pgfpathclose%
\pgfusepath{fill}%
\end{pgfscope}%
\begin{pgfscope}%
\pgfpathrectangle{\pgfqpoint{5.800000in}{0.720000in}}{\pgfqpoint{1.400000in}{4.620000in}}%
\pgfusepath{clip}%
\pgfsetbuttcap%
\pgfsetmiterjoin%
\definecolor{currentfill}{rgb}{0.121569,0.466667,0.705882}%
\pgfsetfillcolor{currentfill}%
\pgfsetlinewidth{0.000000pt}%
\definecolor{currentstroke}{rgb}{0.000000,0.000000,0.000000}%
\pgfsetstrokecolor{currentstroke}%
\pgfsetstrokeopacity{0.000000}%
\pgfsetdash{}{0pt}%
\pgfpathmoveto{\pgfqpoint{5.800000in}{10.672752in}}%
\pgfpathlineto{\pgfqpoint{5.802045in}{10.672752in}}%
\pgfpathlineto{\pgfqpoint{5.802045in}{10.788482in}}%
\pgfpathlineto{\pgfqpoint{5.800000in}{10.788482in}}%
\pgfpathclose%
\pgfusepath{fill}%
\end{pgfscope}%
\begin{pgfscope}%
\pgfpathrectangle{\pgfqpoint{5.800000in}{0.720000in}}{\pgfqpoint{1.400000in}{4.620000in}}%
\pgfusepath{clip}%
\pgfsetbuttcap%
\pgfsetmiterjoin%
\definecolor{currentfill}{rgb}{0.121569,0.466667,0.705882}%
\pgfsetfillcolor{currentfill}%
\pgfsetlinewidth{0.000000pt}%
\definecolor{currentstroke}{rgb}{0.000000,0.000000,0.000000}%
\pgfsetstrokecolor{currentstroke}%
\pgfsetstrokeopacity{0.000000}%
\pgfsetdash{}{0pt}%
\pgfpathmoveto{\pgfqpoint{5.800000in}{10.788482in}}%
\pgfpathlineto{\pgfqpoint{5.800000in}{10.788482in}}%
\pgfpathlineto{\pgfqpoint{5.800000in}{10.904212in}}%
\pgfpathlineto{\pgfqpoint{5.800000in}{10.904212in}}%
\pgfpathclose%
\pgfusepath{fill}%
\end{pgfscope}%
\begin{pgfscope}%
\pgfpathrectangle{\pgfqpoint{5.800000in}{0.720000in}}{\pgfqpoint{1.400000in}{4.620000in}}%
\pgfusepath{clip}%
\pgfsetbuttcap%
\pgfsetmiterjoin%
\definecolor{currentfill}{rgb}{0.121569,0.466667,0.705882}%
\pgfsetfillcolor{currentfill}%
\pgfsetlinewidth{0.000000pt}%
\definecolor{currentstroke}{rgb}{0.000000,0.000000,0.000000}%
\pgfsetstrokecolor{currentstroke}%
\pgfsetstrokeopacity{0.000000}%
\pgfsetdash{}{0pt}%
\pgfpathmoveto{\pgfqpoint{5.800000in}{10.904212in}}%
\pgfpathlineto{\pgfqpoint{5.800000in}{10.904212in}}%
\pgfpathlineto{\pgfqpoint{5.800000in}{11.019942in}}%
\pgfpathlineto{\pgfqpoint{5.800000in}{11.019942in}}%
\pgfpathclose%
\pgfusepath{fill}%
\end{pgfscope}%
\begin{pgfscope}%
\pgfpathrectangle{\pgfqpoint{5.800000in}{0.720000in}}{\pgfqpoint{1.400000in}{4.620000in}}%
\pgfusepath{clip}%
\pgfsetbuttcap%
\pgfsetmiterjoin%
\definecolor{currentfill}{rgb}{0.121569,0.466667,0.705882}%
\pgfsetfillcolor{currentfill}%
\pgfsetlinewidth{0.000000pt}%
\definecolor{currentstroke}{rgb}{0.000000,0.000000,0.000000}%
\pgfsetstrokecolor{currentstroke}%
\pgfsetstrokeopacity{0.000000}%
\pgfsetdash{}{0pt}%
\pgfpathmoveto{\pgfqpoint{5.800000in}{11.019942in}}%
\pgfpathlineto{\pgfqpoint{5.800000in}{11.019942in}}%
\pgfpathlineto{\pgfqpoint{5.800000in}{11.135671in}}%
\pgfpathlineto{\pgfqpoint{5.800000in}{11.135671in}}%
\pgfpathclose%
\pgfusepath{fill}%
\end{pgfscope}%
\begin{pgfscope}%
\pgfpathrectangle{\pgfqpoint{5.800000in}{0.720000in}}{\pgfqpoint{1.400000in}{4.620000in}}%
\pgfusepath{clip}%
\pgfsetbuttcap%
\pgfsetmiterjoin%
\definecolor{currentfill}{rgb}{0.121569,0.466667,0.705882}%
\pgfsetfillcolor{currentfill}%
\pgfsetlinewidth{0.000000pt}%
\definecolor{currentstroke}{rgb}{0.000000,0.000000,0.000000}%
\pgfsetstrokecolor{currentstroke}%
\pgfsetstrokeopacity{0.000000}%
\pgfsetdash{}{0pt}%
\pgfpathmoveto{\pgfqpoint{5.800000in}{11.135671in}}%
\pgfpathlineto{\pgfqpoint{5.800000in}{11.135671in}}%
\pgfpathlineto{\pgfqpoint{5.800000in}{11.251401in}}%
\pgfpathlineto{\pgfqpoint{5.800000in}{11.251401in}}%
\pgfpathclose%
\pgfusepath{fill}%
\end{pgfscope}%
\begin{pgfscope}%
\pgfpathrectangle{\pgfqpoint{5.800000in}{0.720000in}}{\pgfqpoint{1.400000in}{4.620000in}}%
\pgfusepath{clip}%
\pgfsetbuttcap%
\pgfsetmiterjoin%
\definecolor{currentfill}{rgb}{0.121569,0.466667,0.705882}%
\pgfsetfillcolor{currentfill}%
\pgfsetlinewidth{0.000000pt}%
\definecolor{currentstroke}{rgb}{0.000000,0.000000,0.000000}%
\pgfsetstrokecolor{currentstroke}%
\pgfsetstrokeopacity{0.000000}%
\pgfsetdash{}{0pt}%
\pgfpathmoveto{\pgfqpoint{5.800000in}{11.251401in}}%
\pgfpathlineto{\pgfqpoint{5.802045in}{11.251401in}}%
\pgfpathlineto{\pgfqpoint{5.802045in}{11.367131in}}%
\pgfpathlineto{\pgfqpoint{5.800000in}{11.367131in}}%
\pgfpathclose%
\pgfusepath{fill}%
\end{pgfscope}%
\begin{pgfscope}%
\pgfpathrectangle{\pgfqpoint{5.800000in}{0.720000in}}{\pgfqpoint{1.400000in}{4.620000in}}%
\pgfusepath{clip}%
\pgfsetbuttcap%
\pgfsetmiterjoin%
\definecolor{currentfill}{rgb}{0.121569,0.466667,0.705882}%
\pgfsetfillcolor{currentfill}%
\pgfsetlinewidth{0.000000pt}%
\definecolor{currentstroke}{rgb}{0.000000,0.000000,0.000000}%
\pgfsetstrokecolor{currentstroke}%
\pgfsetstrokeopacity{0.000000}%
\pgfsetdash{}{0pt}%
\pgfpathmoveto{\pgfqpoint{5.800000in}{11.367131in}}%
\pgfpathlineto{\pgfqpoint{5.800000in}{11.367131in}}%
\pgfpathlineto{\pgfqpoint{5.800000in}{11.482860in}}%
\pgfpathlineto{\pgfqpoint{5.800000in}{11.482860in}}%
\pgfpathclose%
\pgfusepath{fill}%
\end{pgfscope}%
\begin{pgfscope}%
\pgfpathrectangle{\pgfqpoint{5.800000in}{0.720000in}}{\pgfqpoint{1.400000in}{4.620000in}}%
\pgfusepath{clip}%
\pgfsetbuttcap%
\pgfsetmiterjoin%
\definecolor{currentfill}{rgb}{0.121569,0.466667,0.705882}%
\pgfsetfillcolor{currentfill}%
\pgfsetlinewidth{0.000000pt}%
\definecolor{currentstroke}{rgb}{0.000000,0.000000,0.000000}%
\pgfsetstrokecolor{currentstroke}%
\pgfsetstrokeopacity{0.000000}%
\pgfsetdash{}{0pt}%
\pgfpathmoveto{\pgfqpoint{5.800000in}{11.482860in}}%
\pgfpathlineto{\pgfqpoint{5.800000in}{11.482860in}}%
\pgfpathlineto{\pgfqpoint{5.800000in}{11.598590in}}%
\pgfpathlineto{\pgfqpoint{5.800000in}{11.598590in}}%
\pgfpathclose%
\pgfusepath{fill}%
\end{pgfscope}%
\begin{pgfscope}%
\pgfpathrectangle{\pgfqpoint{5.800000in}{0.720000in}}{\pgfqpoint{1.400000in}{4.620000in}}%
\pgfusepath{clip}%
\pgfsetbuttcap%
\pgfsetmiterjoin%
\definecolor{currentfill}{rgb}{0.121569,0.466667,0.705882}%
\pgfsetfillcolor{currentfill}%
\pgfsetlinewidth{0.000000pt}%
\definecolor{currentstroke}{rgb}{0.000000,0.000000,0.000000}%
\pgfsetstrokecolor{currentstroke}%
\pgfsetstrokeopacity{0.000000}%
\pgfsetdash{}{0pt}%
\pgfpathmoveto{\pgfqpoint{5.800000in}{11.598590in}}%
\pgfpathlineto{\pgfqpoint{5.800000in}{11.598590in}}%
\pgfpathlineto{\pgfqpoint{5.800000in}{11.714320in}}%
\pgfpathlineto{\pgfqpoint{5.800000in}{11.714320in}}%
\pgfpathclose%
\pgfusepath{fill}%
\end{pgfscope}%
\begin{pgfscope}%
\pgfpathrectangle{\pgfqpoint{5.800000in}{0.720000in}}{\pgfqpoint{1.400000in}{4.620000in}}%
\pgfusepath{clip}%
\pgfsetbuttcap%
\pgfsetmiterjoin%
\definecolor{currentfill}{rgb}{0.121569,0.466667,0.705882}%
\pgfsetfillcolor{currentfill}%
\pgfsetlinewidth{0.000000pt}%
\definecolor{currentstroke}{rgb}{0.000000,0.000000,0.000000}%
\pgfsetstrokecolor{currentstroke}%
\pgfsetstrokeopacity{0.000000}%
\pgfsetdash{}{0pt}%
\pgfpathmoveto{\pgfqpoint{5.800000in}{11.714320in}}%
\pgfpathlineto{\pgfqpoint{5.800000in}{11.714320in}}%
\pgfpathlineto{\pgfqpoint{5.800000in}{11.830049in}}%
\pgfpathlineto{\pgfqpoint{5.800000in}{11.830049in}}%
\pgfpathclose%
\pgfusepath{fill}%
\end{pgfscope}%
\begin{pgfscope}%
\pgfpathrectangle{\pgfqpoint{5.800000in}{0.720000in}}{\pgfqpoint{1.400000in}{4.620000in}}%
\pgfusepath{clip}%
\pgfsetbuttcap%
\pgfsetmiterjoin%
\definecolor{currentfill}{rgb}{0.121569,0.466667,0.705882}%
\pgfsetfillcolor{currentfill}%
\pgfsetlinewidth{0.000000pt}%
\definecolor{currentstroke}{rgb}{0.000000,0.000000,0.000000}%
\pgfsetstrokecolor{currentstroke}%
\pgfsetstrokeopacity{0.000000}%
\pgfsetdash{}{0pt}%
\pgfpathmoveto{\pgfqpoint{5.800000in}{11.830049in}}%
\pgfpathlineto{\pgfqpoint{5.800000in}{11.830049in}}%
\pgfpathlineto{\pgfqpoint{5.800000in}{11.945779in}}%
\pgfpathlineto{\pgfqpoint{5.800000in}{11.945779in}}%
\pgfpathclose%
\pgfusepath{fill}%
\end{pgfscope}%
\begin{pgfscope}%
\pgfpathrectangle{\pgfqpoint{5.800000in}{0.720000in}}{\pgfqpoint{1.400000in}{4.620000in}}%
\pgfusepath{clip}%
\pgfsetbuttcap%
\pgfsetmiterjoin%
\definecolor{currentfill}{rgb}{0.121569,0.466667,0.705882}%
\pgfsetfillcolor{currentfill}%
\pgfsetlinewidth{0.000000pt}%
\definecolor{currentstroke}{rgb}{0.000000,0.000000,0.000000}%
\pgfsetstrokecolor{currentstroke}%
\pgfsetstrokeopacity{0.000000}%
\pgfsetdash{}{0pt}%
\pgfpathmoveto{\pgfqpoint{5.800000in}{11.945779in}}%
\pgfpathlineto{\pgfqpoint{5.800000in}{11.945779in}}%
\pgfpathlineto{\pgfqpoint{5.800000in}{12.061509in}}%
\pgfpathlineto{\pgfqpoint{5.800000in}{12.061509in}}%
\pgfpathclose%
\pgfusepath{fill}%
\end{pgfscope}%
\begin{pgfscope}%
\pgfpathrectangle{\pgfqpoint{5.800000in}{0.720000in}}{\pgfqpoint{1.400000in}{4.620000in}}%
\pgfusepath{clip}%
\pgfsetbuttcap%
\pgfsetmiterjoin%
\definecolor{currentfill}{rgb}{0.121569,0.466667,0.705882}%
\pgfsetfillcolor{currentfill}%
\pgfsetlinewidth{0.000000pt}%
\definecolor{currentstroke}{rgb}{0.000000,0.000000,0.000000}%
\pgfsetstrokecolor{currentstroke}%
\pgfsetstrokeopacity{0.000000}%
\pgfsetdash{}{0pt}%
\pgfpathmoveto{\pgfqpoint{5.800000in}{12.061509in}}%
\pgfpathlineto{\pgfqpoint{5.800000in}{12.061509in}}%
\pgfpathlineto{\pgfqpoint{5.800000in}{12.177238in}}%
\pgfpathlineto{\pgfqpoint{5.800000in}{12.177238in}}%
\pgfpathclose%
\pgfusepath{fill}%
\end{pgfscope}%
\begin{pgfscope}%
\pgfpathrectangle{\pgfqpoint{5.800000in}{0.720000in}}{\pgfqpoint{1.400000in}{4.620000in}}%
\pgfusepath{clip}%
\pgfsetbuttcap%
\pgfsetmiterjoin%
\definecolor{currentfill}{rgb}{0.121569,0.466667,0.705882}%
\pgfsetfillcolor{currentfill}%
\pgfsetlinewidth{0.000000pt}%
\definecolor{currentstroke}{rgb}{0.000000,0.000000,0.000000}%
\pgfsetstrokecolor{currentstroke}%
\pgfsetstrokeopacity{0.000000}%
\pgfsetdash{}{0pt}%
\pgfpathmoveto{\pgfqpoint{5.800000in}{12.177238in}}%
\pgfpathlineto{\pgfqpoint{5.800000in}{12.177238in}}%
\pgfpathlineto{\pgfqpoint{5.800000in}{12.292968in}}%
\pgfpathlineto{\pgfqpoint{5.800000in}{12.292968in}}%
\pgfpathclose%
\pgfusepath{fill}%
\end{pgfscope}%
\begin{pgfscope}%
\pgfpathrectangle{\pgfqpoint{5.800000in}{0.720000in}}{\pgfqpoint{1.400000in}{4.620000in}}%
\pgfusepath{clip}%
\pgfsetbuttcap%
\pgfsetmiterjoin%
\definecolor{currentfill}{rgb}{0.121569,0.466667,0.705882}%
\pgfsetfillcolor{currentfill}%
\pgfsetlinewidth{0.000000pt}%
\definecolor{currentstroke}{rgb}{0.000000,0.000000,0.000000}%
\pgfsetstrokecolor{currentstroke}%
\pgfsetstrokeopacity{0.000000}%
\pgfsetdash{}{0pt}%
\pgfpathmoveto{\pgfqpoint{5.800000in}{12.292968in}}%
\pgfpathlineto{\pgfqpoint{5.800000in}{12.292968in}}%
\pgfpathlineto{\pgfqpoint{5.800000in}{12.408698in}}%
\pgfpathlineto{\pgfqpoint{5.800000in}{12.408698in}}%
\pgfpathclose%
\pgfusepath{fill}%
\end{pgfscope}%
\begin{pgfscope}%
\pgfpathrectangle{\pgfqpoint{5.800000in}{0.720000in}}{\pgfqpoint{1.400000in}{4.620000in}}%
\pgfusepath{clip}%
\pgfsetbuttcap%
\pgfsetmiterjoin%
\definecolor{currentfill}{rgb}{0.121569,0.466667,0.705882}%
\pgfsetfillcolor{currentfill}%
\pgfsetlinewidth{0.000000pt}%
\definecolor{currentstroke}{rgb}{0.000000,0.000000,0.000000}%
\pgfsetstrokecolor{currentstroke}%
\pgfsetstrokeopacity{0.000000}%
\pgfsetdash{}{0pt}%
\pgfpathmoveto{\pgfqpoint{5.800000in}{12.408698in}}%
\pgfpathlineto{\pgfqpoint{5.800000in}{12.408698in}}%
\pgfpathlineto{\pgfqpoint{5.800000in}{12.524427in}}%
\pgfpathlineto{\pgfqpoint{5.800000in}{12.524427in}}%
\pgfpathclose%
\pgfusepath{fill}%
\end{pgfscope}%
\begin{pgfscope}%
\pgfpathrectangle{\pgfqpoint{5.800000in}{0.720000in}}{\pgfqpoint{1.400000in}{4.620000in}}%
\pgfusepath{clip}%
\pgfsetbuttcap%
\pgfsetmiterjoin%
\definecolor{currentfill}{rgb}{0.121569,0.466667,0.705882}%
\pgfsetfillcolor{currentfill}%
\pgfsetlinewidth{0.000000pt}%
\definecolor{currentstroke}{rgb}{0.000000,0.000000,0.000000}%
\pgfsetstrokecolor{currentstroke}%
\pgfsetstrokeopacity{0.000000}%
\pgfsetdash{}{0pt}%
\pgfpathmoveto{\pgfqpoint{5.800000in}{12.524427in}}%
\pgfpathlineto{\pgfqpoint{5.800000in}{12.524427in}}%
\pgfpathlineto{\pgfqpoint{5.800000in}{12.640157in}}%
\pgfpathlineto{\pgfqpoint{5.800000in}{12.640157in}}%
\pgfpathclose%
\pgfusepath{fill}%
\end{pgfscope}%
\begin{pgfscope}%
\pgfpathrectangle{\pgfqpoint{5.800000in}{0.720000in}}{\pgfqpoint{1.400000in}{4.620000in}}%
\pgfusepath{clip}%
\pgfsetbuttcap%
\pgfsetmiterjoin%
\definecolor{currentfill}{rgb}{0.121569,0.466667,0.705882}%
\pgfsetfillcolor{currentfill}%
\pgfsetlinewidth{0.000000pt}%
\definecolor{currentstroke}{rgb}{0.000000,0.000000,0.000000}%
\pgfsetstrokecolor{currentstroke}%
\pgfsetstrokeopacity{0.000000}%
\pgfsetdash{}{0pt}%
\pgfpathmoveto{\pgfqpoint{5.800000in}{12.640157in}}%
\pgfpathlineto{\pgfqpoint{5.800000in}{12.640157in}}%
\pgfpathlineto{\pgfqpoint{5.800000in}{12.755887in}}%
\pgfpathlineto{\pgfqpoint{5.800000in}{12.755887in}}%
\pgfpathclose%
\pgfusepath{fill}%
\end{pgfscope}%
\begin{pgfscope}%
\pgfpathrectangle{\pgfqpoint{5.800000in}{0.720000in}}{\pgfqpoint{1.400000in}{4.620000in}}%
\pgfusepath{clip}%
\pgfsetbuttcap%
\pgfsetmiterjoin%
\definecolor{currentfill}{rgb}{0.121569,0.466667,0.705882}%
\pgfsetfillcolor{currentfill}%
\pgfsetlinewidth{0.000000pt}%
\definecolor{currentstroke}{rgb}{0.000000,0.000000,0.000000}%
\pgfsetstrokecolor{currentstroke}%
\pgfsetstrokeopacity{0.000000}%
\pgfsetdash{}{0pt}%
\pgfpathmoveto{\pgfqpoint{5.800000in}{12.755887in}}%
\pgfpathlineto{\pgfqpoint{5.802045in}{12.755887in}}%
\pgfpathlineto{\pgfqpoint{5.802045in}{12.871616in}}%
\pgfpathlineto{\pgfqpoint{5.800000in}{12.871616in}}%
\pgfpathclose%
\pgfusepath{fill}%
\end{pgfscope}%
\begin{pgfscope}%
\pgfpathrectangle{\pgfqpoint{5.800000in}{0.720000in}}{\pgfqpoint{1.400000in}{4.620000in}}%
\pgfusepath{clip}%
\pgfsetbuttcap%
\pgfsetmiterjoin%
\definecolor{currentfill}{rgb}{0.121569,0.466667,0.705882}%
\pgfsetfillcolor{currentfill}%
\pgfsetlinewidth{0.000000pt}%
\definecolor{currentstroke}{rgb}{0.000000,0.000000,0.000000}%
\pgfsetstrokecolor{currentstroke}%
\pgfsetstrokeopacity{0.000000}%
\pgfsetdash{}{0pt}%
\pgfpathmoveto{\pgfqpoint{5.800000in}{12.871616in}}%
\pgfpathlineto{\pgfqpoint{5.800000in}{12.871616in}}%
\pgfpathlineto{\pgfqpoint{5.800000in}{12.987346in}}%
\pgfpathlineto{\pgfqpoint{5.800000in}{12.987346in}}%
\pgfpathclose%
\pgfusepath{fill}%
\end{pgfscope}%
\begin{pgfscope}%
\pgfpathrectangle{\pgfqpoint{5.800000in}{0.720000in}}{\pgfqpoint{1.400000in}{4.620000in}}%
\pgfusepath{clip}%
\pgfsetbuttcap%
\pgfsetmiterjoin%
\definecolor{currentfill}{rgb}{0.121569,0.466667,0.705882}%
\pgfsetfillcolor{currentfill}%
\pgfsetlinewidth{0.000000pt}%
\definecolor{currentstroke}{rgb}{0.000000,0.000000,0.000000}%
\pgfsetstrokecolor{currentstroke}%
\pgfsetstrokeopacity{0.000000}%
\pgfsetdash{}{0pt}%
\pgfpathmoveto{\pgfqpoint{5.800000in}{12.987346in}}%
\pgfpathlineto{\pgfqpoint{5.800000in}{12.987346in}}%
\pgfpathlineto{\pgfqpoint{5.800000in}{13.103076in}}%
\pgfpathlineto{\pgfqpoint{5.800000in}{13.103076in}}%
\pgfpathclose%
\pgfusepath{fill}%
\end{pgfscope}%
\begin{pgfscope}%
\pgfpathrectangle{\pgfqpoint{5.800000in}{0.720000in}}{\pgfqpoint{1.400000in}{4.620000in}}%
\pgfusepath{clip}%
\pgfsetbuttcap%
\pgfsetmiterjoin%
\definecolor{currentfill}{rgb}{0.121569,0.466667,0.705882}%
\pgfsetfillcolor{currentfill}%
\pgfsetlinewidth{0.000000pt}%
\definecolor{currentstroke}{rgb}{0.000000,0.000000,0.000000}%
\pgfsetstrokecolor{currentstroke}%
\pgfsetstrokeopacity{0.000000}%
\pgfsetdash{}{0pt}%
\pgfpathmoveto{\pgfqpoint{5.800000in}{13.103076in}}%
\pgfpathlineto{\pgfqpoint{5.800000in}{13.103076in}}%
\pgfpathlineto{\pgfqpoint{5.800000in}{13.218805in}}%
\pgfpathlineto{\pgfqpoint{5.800000in}{13.218805in}}%
\pgfpathclose%
\pgfusepath{fill}%
\end{pgfscope}%
\begin{pgfscope}%
\pgfpathrectangle{\pgfqpoint{5.800000in}{0.720000in}}{\pgfqpoint{1.400000in}{4.620000in}}%
\pgfusepath{clip}%
\pgfsetbuttcap%
\pgfsetmiterjoin%
\definecolor{currentfill}{rgb}{0.121569,0.466667,0.705882}%
\pgfsetfillcolor{currentfill}%
\pgfsetlinewidth{0.000000pt}%
\definecolor{currentstroke}{rgb}{0.000000,0.000000,0.000000}%
\pgfsetstrokecolor{currentstroke}%
\pgfsetstrokeopacity{0.000000}%
\pgfsetdash{}{0pt}%
\pgfpathmoveto{\pgfqpoint{5.800000in}{13.218805in}}%
\pgfpathlineto{\pgfqpoint{5.800000in}{13.218805in}}%
\pgfpathlineto{\pgfqpoint{5.800000in}{13.334535in}}%
\pgfpathlineto{\pgfqpoint{5.800000in}{13.334535in}}%
\pgfpathclose%
\pgfusepath{fill}%
\end{pgfscope}%
\begin{pgfscope}%
\pgfpathrectangle{\pgfqpoint{5.800000in}{0.720000in}}{\pgfqpoint{1.400000in}{4.620000in}}%
\pgfusepath{clip}%
\pgfsetbuttcap%
\pgfsetmiterjoin%
\definecolor{currentfill}{rgb}{0.121569,0.466667,0.705882}%
\pgfsetfillcolor{currentfill}%
\pgfsetlinewidth{0.000000pt}%
\definecolor{currentstroke}{rgb}{0.000000,0.000000,0.000000}%
\pgfsetstrokecolor{currentstroke}%
\pgfsetstrokeopacity{0.000000}%
\pgfsetdash{}{0pt}%
\pgfpathmoveto{\pgfqpoint{5.800000in}{13.334535in}}%
\pgfpathlineto{\pgfqpoint{5.800000in}{13.334535in}}%
\pgfpathlineto{\pgfqpoint{5.800000in}{13.450265in}}%
\pgfpathlineto{\pgfqpoint{5.800000in}{13.450265in}}%
\pgfpathclose%
\pgfusepath{fill}%
\end{pgfscope}%
\begin{pgfscope}%
\pgfpathrectangle{\pgfqpoint{5.800000in}{0.720000in}}{\pgfqpoint{1.400000in}{4.620000in}}%
\pgfusepath{clip}%
\pgfsetbuttcap%
\pgfsetmiterjoin%
\definecolor{currentfill}{rgb}{0.121569,0.466667,0.705882}%
\pgfsetfillcolor{currentfill}%
\pgfsetlinewidth{0.000000pt}%
\definecolor{currentstroke}{rgb}{0.000000,0.000000,0.000000}%
\pgfsetstrokecolor{currentstroke}%
\pgfsetstrokeopacity{0.000000}%
\pgfsetdash{}{0pt}%
\pgfpathmoveto{\pgfqpoint{5.800000in}{13.450265in}}%
\pgfpathlineto{\pgfqpoint{5.800000in}{13.450265in}}%
\pgfpathlineto{\pgfqpoint{5.800000in}{13.565994in}}%
\pgfpathlineto{\pgfqpoint{5.800000in}{13.565994in}}%
\pgfpathclose%
\pgfusepath{fill}%
\end{pgfscope}%
\begin{pgfscope}%
\pgfpathrectangle{\pgfqpoint{5.800000in}{0.720000in}}{\pgfqpoint{1.400000in}{4.620000in}}%
\pgfusepath{clip}%
\pgfsetbuttcap%
\pgfsetmiterjoin%
\definecolor{currentfill}{rgb}{0.121569,0.466667,0.705882}%
\pgfsetfillcolor{currentfill}%
\pgfsetlinewidth{0.000000pt}%
\definecolor{currentstroke}{rgb}{0.000000,0.000000,0.000000}%
\pgfsetstrokecolor{currentstroke}%
\pgfsetstrokeopacity{0.000000}%
\pgfsetdash{}{0pt}%
\pgfpathmoveto{\pgfqpoint{5.800000in}{13.565994in}}%
\pgfpathlineto{\pgfqpoint{5.800000in}{13.565994in}}%
\pgfpathlineto{\pgfqpoint{5.800000in}{13.681724in}}%
\pgfpathlineto{\pgfqpoint{5.800000in}{13.681724in}}%
\pgfpathclose%
\pgfusepath{fill}%
\end{pgfscope}%
\begin{pgfscope}%
\pgfpathrectangle{\pgfqpoint{5.800000in}{0.720000in}}{\pgfqpoint{1.400000in}{4.620000in}}%
\pgfusepath{clip}%
\pgfsetbuttcap%
\pgfsetmiterjoin%
\definecolor{currentfill}{rgb}{0.121569,0.466667,0.705882}%
\pgfsetfillcolor{currentfill}%
\pgfsetlinewidth{0.000000pt}%
\definecolor{currentstroke}{rgb}{0.000000,0.000000,0.000000}%
\pgfsetstrokecolor{currentstroke}%
\pgfsetstrokeopacity{0.000000}%
\pgfsetdash{}{0pt}%
\pgfpathmoveto{\pgfqpoint{5.800000in}{13.681724in}}%
\pgfpathlineto{\pgfqpoint{5.800000in}{13.681724in}}%
\pgfpathlineto{\pgfqpoint{5.800000in}{13.797454in}}%
\pgfpathlineto{\pgfqpoint{5.800000in}{13.797454in}}%
\pgfpathclose%
\pgfusepath{fill}%
\end{pgfscope}%
\begin{pgfscope}%
\pgfpathrectangle{\pgfqpoint{5.800000in}{0.720000in}}{\pgfqpoint{1.400000in}{4.620000in}}%
\pgfusepath{clip}%
\pgfsetbuttcap%
\pgfsetmiterjoin%
\definecolor{currentfill}{rgb}{0.121569,0.466667,0.705882}%
\pgfsetfillcolor{currentfill}%
\pgfsetlinewidth{0.000000pt}%
\definecolor{currentstroke}{rgb}{0.000000,0.000000,0.000000}%
\pgfsetstrokecolor{currentstroke}%
\pgfsetstrokeopacity{0.000000}%
\pgfsetdash{}{0pt}%
\pgfpathmoveto{\pgfqpoint{5.800000in}{13.797454in}}%
\pgfpathlineto{\pgfqpoint{5.800000in}{13.797454in}}%
\pgfpathlineto{\pgfqpoint{5.800000in}{13.913184in}}%
\pgfpathlineto{\pgfqpoint{5.800000in}{13.913184in}}%
\pgfpathclose%
\pgfusepath{fill}%
\end{pgfscope}%
\begin{pgfscope}%
\pgfpathrectangle{\pgfqpoint{5.800000in}{0.720000in}}{\pgfqpoint{1.400000in}{4.620000in}}%
\pgfusepath{clip}%
\pgfsetbuttcap%
\pgfsetmiterjoin%
\definecolor{currentfill}{rgb}{0.121569,0.466667,0.705882}%
\pgfsetfillcolor{currentfill}%
\pgfsetlinewidth{0.000000pt}%
\definecolor{currentstroke}{rgb}{0.000000,0.000000,0.000000}%
\pgfsetstrokecolor{currentstroke}%
\pgfsetstrokeopacity{0.000000}%
\pgfsetdash{}{0pt}%
\pgfpathmoveto{\pgfqpoint{5.800000in}{13.913184in}}%
\pgfpathlineto{\pgfqpoint{5.800000in}{13.913184in}}%
\pgfpathlineto{\pgfqpoint{5.800000in}{14.028913in}}%
\pgfpathlineto{\pgfqpoint{5.800000in}{14.028913in}}%
\pgfpathclose%
\pgfusepath{fill}%
\end{pgfscope}%
\begin{pgfscope}%
\pgfpathrectangle{\pgfqpoint{5.800000in}{0.720000in}}{\pgfqpoint{1.400000in}{4.620000in}}%
\pgfusepath{clip}%
\pgfsetbuttcap%
\pgfsetmiterjoin%
\definecolor{currentfill}{rgb}{0.121569,0.466667,0.705882}%
\pgfsetfillcolor{currentfill}%
\pgfsetlinewidth{0.000000pt}%
\definecolor{currentstroke}{rgb}{0.000000,0.000000,0.000000}%
\pgfsetstrokecolor{currentstroke}%
\pgfsetstrokeopacity{0.000000}%
\pgfsetdash{}{0pt}%
\pgfpathmoveto{\pgfqpoint{5.800000in}{14.028913in}}%
\pgfpathlineto{\pgfqpoint{5.800000in}{14.028913in}}%
\pgfpathlineto{\pgfqpoint{5.800000in}{14.144643in}}%
\pgfpathlineto{\pgfqpoint{5.800000in}{14.144643in}}%
\pgfpathclose%
\pgfusepath{fill}%
\end{pgfscope}%
\begin{pgfscope}%
\pgfpathrectangle{\pgfqpoint{5.800000in}{0.720000in}}{\pgfqpoint{1.400000in}{4.620000in}}%
\pgfusepath{clip}%
\pgfsetbuttcap%
\pgfsetmiterjoin%
\definecolor{currentfill}{rgb}{0.121569,0.466667,0.705882}%
\pgfsetfillcolor{currentfill}%
\pgfsetlinewidth{0.000000pt}%
\definecolor{currentstroke}{rgb}{0.000000,0.000000,0.000000}%
\pgfsetstrokecolor{currentstroke}%
\pgfsetstrokeopacity{0.000000}%
\pgfsetdash{}{0pt}%
\pgfpathmoveto{\pgfqpoint{5.800000in}{14.144643in}}%
\pgfpathlineto{\pgfqpoint{5.800000in}{14.144643in}}%
\pgfpathlineto{\pgfqpoint{5.800000in}{14.260373in}}%
\pgfpathlineto{\pgfqpoint{5.800000in}{14.260373in}}%
\pgfpathclose%
\pgfusepath{fill}%
\end{pgfscope}%
\begin{pgfscope}%
\pgfpathrectangle{\pgfqpoint{5.800000in}{0.720000in}}{\pgfqpoint{1.400000in}{4.620000in}}%
\pgfusepath{clip}%
\pgfsetbuttcap%
\pgfsetmiterjoin%
\definecolor{currentfill}{rgb}{0.121569,0.466667,0.705882}%
\pgfsetfillcolor{currentfill}%
\pgfsetlinewidth{0.000000pt}%
\definecolor{currentstroke}{rgb}{0.000000,0.000000,0.000000}%
\pgfsetstrokecolor{currentstroke}%
\pgfsetstrokeopacity{0.000000}%
\pgfsetdash{}{0pt}%
\pgfpathmoveto{\pgfqpoint{5.800000in}{14.260373in}}%
\pgfpathlineto{\pgfqpoint{5.800000in}{14.260373in}}%
\pgfpathlineto{\pgfqpoint{5.800000in}{14.376102in}}%
\pgfpathlineto{\pgfqpoint{5.800000in}{14.376102in}}%
\pgfpathclose%
\pgfusepath{fill}%
\end{pgfscope}%
\begin{pgfscope}%
\pgfpathrectangle{\pgfqpoint{5.800000in}{0.720000in}}{\pgfqpoint{1.400000in}{4.620000in}}%
\pgfusepath{clip}%
\pgfsetbuttcap%
\pgfsetmiterjoin%
\definecolor{currentfill}{rgb}{0.121569,0.466667,0.705882}%
\pgfsetfillcolor{currentfill}%
\pgfsetlinewidth{0.000000pt}%
\definecolor{currentstroke}{rgb}{0.000000,0.000000,0.000000}%
\pgfsetstrokecolor{currentstroke}%
\pgfsetstrokeopacity{0.000000}%
\pgfsetdash{}{0pt}%
\pgfpathmoveto{\pgfqpoint{5.800000in}{14.376102in}}%
\pgfpathlineto{\pgfqpoint{5.800000in}{14.376102in}}%
\pgfpathlineto{\pgfqpoint{5.800000in}{14.491832in}}%
\pgfpathlineto{\pgfqpoint{5.800000in}{14.491832in}}%
\pgfpathclose%
\pgfusepath{fill}%
\end{pgfscope}%
\begin{pgfscope}%
\pgfpathrectangle{\pgfqpoint{5.800000in}{0.720000in}}{\pgfqpoint{1.400000in}{4.620000in}}%
\pgfusepath{clip}%
\pgfsetbuttcap%
\pgfsetmiterjoin%
\definecolor{currentfill}{rgb}{0.121569,0.466667,0.705882}%
\pgfsetfillcolor{currentfill}%
\pgfsetlinewidth{0.000000pt}%
\definecolor{currentstroke}{rgb}{0.000000,0.000000,0.000000}%
\pgfsetstrokecolor{currentstroke}%
\pgfsetstrokeopacity{0.000000}%
\pgfsetdash{}{0pt}%
\pgfpathmoveto{\pgfqpoint{5.800000in}{14.491832in}}%
\pgfpathlineto{\pgfqpoint{5.802045in}{14.491832in}}%
\pgfpathlineto{\pgfqpoint{5.802045in}{14.607562in}}%
\pgfpathlineto{\pgfqpoint{5.800000in}{14.607562in}}%
\pgfpathclose%
\pgfusepath{fill}%
\end{pgfscope}%
\begin{pgfscope}%
\pgfpathrectangle{\pgfqpoint{5.800000in}{0.720000in}}{\pgfqpoint{1.400000in}{4.620000in}}%
\pgfusepath{clip}%
\pgfsetbuttcap%
\pgfsetmiterjoin%
\definecolor{currentfill}{rgb}{0.121569,0.466667,0.705882}%
\pgfsetfillcolor{currentfill}%
\pgfsetlinewidth{0.000000pt}%
\definecolor{currentstroke}{rgb}{0.000000,0.000000,0.000000}%
\pgfsetstrokecolor{currentstroke}%
\pgfsetstrokeopacity{0.000000}%
\pgfsetdash{}{0pt}%
\pgfpathmoveto{\pgfqpoint{5.800000in}{14.607562in}}%
\pgfpathlineto{\pgfqpoint{5.800000in}{14.607562in}}%
\pgfpathlineto{\pgfqpoint{5.800000in}{14.723291in}}%
\pgfpathlineto{\pgfqpoint{5.800000in}{14.723291in}}%
\pgfpathclose%
\pgfusepath{fill}%
\end{pgfscope}%
\begin{pgfscope}%
\pgfpathrectangle{\pgfqpoint{5.800000in}{0.720000in}}{\pgfqpoint{1.400000in}{4.620000in}}%
\pgfusepath{clip}%
\pgfsetbuttcap%
\pgfsetmiterjoin%
\definecolor{currentfill}{rgb}{0.121569,0.466667,0.705882}%
\pgfsetfillcolor{currentfill}%
\pgfsetlinewidth{0.000000pt}%
\definecolor{currentstroke}{rgb}{0.000000,0.000000,0.000000}%
\pgfsetstrokecolor{currentstroke}%
\pgfsetstrokeopacity{0.000000}%
\pgfsetdash{}{0pt}%
\pgfpathmoveto{\pgfqpoint{5.800000in}{14.723291in}}%
\pgfpathlineto{\pgfqpoint{5.800000in}{14.723291in}}%
\pgfpathlineto{\pgfqpoint{5.800000in}{14.839021in}}%
\pgfpathlineto{\pgfqpoint{5.800000in}{14.839021in}}%
\pgfpathclose%
\pgfusepath{fill}%
\end{pgfscope}%
\begin{pgfscope}%
\pgfpathrectangle{\pgfqpoint{5.800000in}{0.720000in}}{\pgfqpoint{1.400000in}{4.620000in}}%
\pgfusepath{clip}%
\pgfsetbuttcap%
\pgfsetmiterjoin%
\definecolor{currentfill}{rgb}{0.121569,0.466667,0.705882}%
\pgfsetfillcolor{currentfill}%
\pgfsetlinewidth{0.000000pt}%
\definecolor{currentstroke}{rgb}{0.000000,0.000000,0.000000}%
\pgfsetstrokecolor{currentstroke}%
\pgfsetstrokeopacity{0.000000}%
\pgfsetdash{}{0pt}%
\pgfpathmoveto{\pgfqpoint{5.800000in}{14.839021in}}%
\pgfpathlineto{\pgfqpoint{5.800000in}{14.839021in}}%
\pgfpathlineto{\pgfqpoint{5.800000in}{14.954751in}}%
\pgfpathlineto{\pgfqpoint{5.800000in}{14.954751in}}%
\pgfpathclose%
\pgfusepath{fill}%
\end{pgfscope}%
\begin{pgfscope}%
\pgfpathrectangle{\pgfqpoint{5.800000in}{0.720000in}}{\pgfqpoint{1.400000in}{4.620000in}}%
\pgfusepath{clip}%
\pgfsetbuttcap%
\pgfsetmiterjoin%
\definecolor{currentfill}{rgb}{0.121569,0.466667,0.705882}%
\pgfsetfillcolor{currentfill}%
\pgfsetlinewidth{0.000000pt}%
\definecolor{currentstroke}{rgb}{0.000000,0.000000,0.000000}%
\pgfsetstrokecolor{currentstroke}%
\pgfsetstrokeopacity{0.000000}%
\pgfsetdash{}{0pt}%
\pgfpathmoveto{\pgfqpoint{5.800000in}{14.954751in}}%
\pgfpathlineto{\pgfqpoint{5.800000in}{14.954751in}}%
\pgfpathlineto{\pgfqpoint{5.800000in}{15.070480in}}%
\pgfpathlineto{\pgfqpoint{5.800000in}{15.070480in}}%
\pgfpathclose%
\pgfusepath{fill}%
\end{pgfscope}%
\begin{pgfscope}%
\pgfpathrectangle{\pgfqpoint{5.800000in}{0.720000in}}{\pgfqpoint{1.400000in}{4.620000in}}%
\pgfusepath{clip}%
\pgfsetbuttcap%
\pgfsetmiterjoin%
\definecolor{currentfill}{rgb}{0.121569,0.466667,0.705882}%
\pgfsetfillcolor{currentfill}%
\pgfsetlinewidth{0.000000pt}%
\definecolor{currentstroke}{rgb}{0.000000,0.000000,0.000000}%
\pgfsetstrokecolor{currentstroke}%
\pgfsetstrokeopacity{0.000000}%
\pgfsetdash{}{0pt}%
\pgfpathmoveto{\pgfqpoint{5.800000in}{15.070480in}}%
\pgfpathlineto{\pgfqpoint{5.800000in}{15.070480in}}%
\pgfpathlineto{\pgfqpoint{5.800000in}{15.186210in}}%
\pgfpathlineto{\pgfqpoint{5.800000in}{15.186210in}}%
\pgfpathclose%
\pgfusepath{fill}%
\end{pgfscope}%
\begin{pgfscope}%
\pgfpathrectangle{\pgfqpoint{5.800000in}{0.720000in}}{\pgfqpoint{1.400000in}{4.620000in}}%
\pgfusepath{clip}%
\pgfsetbuttcap%
\pgfsetmiterjoin%
\definecolor{currentfill}{rgb}{0.121569,0.466667,0.705882}%
\pgfsetfillcolor{currentfill}%
\pgfsetlinewidth{0.000000pt}%
\definecolor{currentstroke}{rgb}{0.000000,0.000000,0.000000}%
\pgfsetstrokecolor{currentstroke}%
\pgfsetstrokeopacity{0.000000}%
\pgfsetdash{}{0pt}%
\pgfpathmoveto{\pgfqpoint{5.800000in}{15.186210in}}%
\pgfpathlineto{\pgfqpoint{5.800000in}{15.186210in}}%
\pgfpathlineto{\pgfqpoint{5.800000in}{15.301940in}}%
\pgfpathlineto{\pgfqpoint{5.800000in}{15.301940in}}%
\pgfpathclose%
\pgfusepath{fill}%
\end{pgfscope}%
\begin{pgfscope}%
\pgfpathrectangle{\pgfqpoint{5.800000in}{0.720000in}}{\pgfqpoint{1.400000in}{4.620000in}}%
\pgfusepath{clip}%
\pgfsetbuttcap%
\pgfsetmiterjoin%
\definecolor{currentfill}{rgb}{0.121569,0.466667,0.705882}%
\pgfsetfillcolor{currentfill}%
\pgfsetlinewidth{0.000000pt}%
\definecolor{currentstroke}{rgb}{0.000000,0.000000,0.000000}%
\pgfsetstrokecolor{currentstroke}%
\pgfsetstrokeopacity{0.000000}%
\pgfsetdash{}{0pt}%
\pgfpathmoveto{\pgfqpoint{5.800000in}{15.301940in}}%
\pgfpathlineto{\pgfqpoint{5.800000in}{15.301940in}}%
\pgfpathlineto{\pgfqpoint{5.800000in}{15.417669in}}%
\pgfpathlineto{\pgfqpoint{5.800000in}{15.417669in}}%
\pgfpathclose%
\pgfusepath{fill}%
\end{pgfscope}%
\begin{pgfscope}%
\pgfpathrectangle{\pgfqpoint{5.800000in}{0.720000in}}{\pgfqpoint{1.400000in}{4.620000in}}%
\pgfusepath{clip}%
\pgfsetbuttcap%
\pgfsetmiterjoin%
\definecolor{currentfill}{rgb}{0.121569,0.466667,0.705882}%
\pgfsetfillcolor{currentfill}%
\pgfsetlinewidth{0.000000pt}%
\definecolor{currentstroke}{rgb}{0.000000,0.000000,0.000000}%
\pgfsetstrokecolor{currentstroke}%
\pgfsetstrokeopacity{0.000000}%
\pgfsetdash{}{0pt}%
\pgfpathmoveto{\pgfqpoint{5.800000in}{15.417669in}}%
\pgfpathlineto{\pgfqpoint{5.800000in}{15.417669in}}%
\pgfpathlineto{\pgfqpoint{5.800000in}{15.533399in}}%
\pgfpathlineto{\pgfqpoint{5.800000in}{15.533399in}}%
\pgfpathclose%
\pgfusepath{fill}%
\end{pgfscope}%
\begin{pgfscope}%
\pgfpathrectangle{\pgfqpoint{5.800000in}{0.720000in}}{\pgfqpoint{1.400000in}{4.620000in}}%
\pgfusepath{clip}%
\pgfsetbuttcap%
\pgfsetmiterjoin%
\definecolor{currentfill}{rgb}{0.121569,0.466667,0.705882}%
\pgfsetfillcolor{currentfill}%
\pgfsetlinewidth{0.000000pt}%
\definecolor{currentstroke}{rgb}{0.000000,0.000000,0.000000}%
\pgfsetstrokecolor{currentstroke}%
\pgfsetstrokeopacity{0.000000}%
\pgfsetdash{}{0pt}%
\pgfpathmoveto{\pgfqpoint{5.800000in}{15.533399in}}%
\pgfpathlineto{\pgfqpoint{5.800000in}{15.533399in}}%
\pgfpathlineto{\pgfqpoint{5.800000in}{15.649129in}}%
\pgfpathlineto{\pgfqpoint{5.800000in}{15.649129in}}%
\pgfpathclose%
\pgfusepath{fill}%
\end{pgfscope}%
\begin{pgfscope}%
\pgfpathrectangle{\pgfqpoint{5.800000in}{0.720000in}}{\pgfqpoint{1.400000in}{4.620000in}}%
\pgfusepath{clip}%
\pgfsetbuttcap%
\pgfsetmiterjoin%
\definecolor{currentfill}{rgb}{0.121569,0.466667,0.705882}%
\pgfsetfillcolor{currentfill}%
\pgfsetlinewidth{0.000000pt}%
\definecolor{currentstroke}{rgb}{0.000000,0.000000,0.000000}%
\pgfsetstrokecolor{currentstroke}%
\pgfsetstrokeopacity{0.000000}%
\pgfsetdash{}{0pt}%
\pgfpathmoveto{\pgfqpoint{5.800000in}{15.649129in}}%
\pgfpathlineto{\pgfqpoint{5.800000in}{15.649129in}}%
\pgfpathlineto{\pgfqpoint{5.800000in}{15.764858in}}%
\pgfpathlineto{\pgfqpoint{5.800000in}{15.764858in}}%
\pgfpathclose%
\pgfusepath{fill}%
\end{pgfscope}%
\begin{pgfscope}%
\pgfpathrectangle{\pgfqpoint{5.800000in}{0.720000in}}{\pgfqpoint{1.400000in}{4.620000in}}%
\pgfusepath{clip}%
\pgfsetbuttcap%
\pgfsetmiterjoin%
\definecolor{currentfill}{rgb}{0.121569,0.466667,0.705882}%
\pgfsetfillcolor{currentfill}%
\pgfsetlinewidth{0.000000pt}%
\definecolor{currentstroke}{rgb}{0.000000,0.000000,0.000000}%
\pgfsetstrokecolor{currentstroke}%
\pgfsetstrokeopacity{0.000000}%
\pgfsetdash{}{0pt}%
\pgfpathmoveto{\pgfqpoint{5.800000in}{15.764858in}}%
\pgfpathlineto{\pgfqpoint{5.800000in}{15.764858in}}%
\pgfpathlineto{\pgfqpoint{5.800000in}{15.880588in}}%
\pgfpathlineto{\pgfqpoint{5.800000in}{15.880588in}}%
\pgfpathclose%
\pgfusepath{fill}%
\end{pgfscope}%
\begin{pgfscope}%
\pgfpathrectangle{\pgfqpoint{5.800000in}{0.720000in}}{\pgfqpoint{1.400000in}{4.620000in}}%
\pgfusepath{clip}%
\pgfsetbuttcap%
\pgfsetmiterjoin%
\definecolor{currentfill}{rgb}{0.121569,0.466667,0.705882}%
\pgfsetfillcolor{currentfill}%
\pgfsetlinewidth{0.000000pt}%
\definecolor{currentstroke}{rgb}{0.000000,0.000000,0.000000}%
\pgfsetstrokecolor{currentstroke}%
\pgfsetstrokeopacity{0.000000}%
\pgfsetdash{}{0pt}%
\pgfpathmoveto{\pgfqpoint{5.800000in}{15.880588in}}%
\pgfpathlineto{\pgfqpoint{5.800000in}{15.880588in}}%
\pgfpathlineto{\pgfqpoint{5.800000in}{15.996318in}}%
\pgfpathlineto{\pgfqpoint{5.800000in}{15.996318in}}%
\pgfpathclose%
\pgfusepath{fill}%
\end{pgfscope}%
\begin{pgfscope}%
\pgfpathrectangle{\pgfqpoint{5.800000in}{0.720000in}}{\pgfqpoint{1.400000in}{4.620000in}}%
\pgfusepath{clip}%
\pgfsetbuttcap%
\pgfsetmiterjoin%
\definecolor{currentfill}{rgb}{0.121569,0.466667,0.705882}%
\pgfsetfillcolor{currentfill}%
\pgfsetlinewidth{0.000000pt}%
\definecolor{currentstroke}{rgb}{0.000000,0.000000,0.000000}%
\pgfsetstrokecolor{currentstroke}%
\pgfsetstrokeopacity{0.000000}%
\pgfsetdash{}{0pt}%
\pgfpathmoveto{\pgfqpoint{5.800000in}{15.996318in}}%
\pgfpathlineto{\pgfqpoint{5.800000in}{15.996318in}}%
\pgfpathlineto{\pgfqpoint{5.800000in}{16.112047in}}%
\pgfpathlineto{\pgfqpoint{5.800000in}{16.112047in}}%
\pgfpathclose%
\pgfusepath{fill}%
\end{pgfscope}%
\begin{pgfscope}%
\pgfpathrectangle{\pgfqpoint{5.800000in}{0.720000in}}{\pgfqpoint{1.400000in}{4.620000in}}%
\pgfusepath{clip}%
\pgfsetbuttcap%
\pgfsetmiterjoin%
\definecolor{currentfill}{rgb}{0.121569,0.466667,0.705882}%
\pgfsetfillcolor{currentfill}%
\pgfsetlinewidth{0.000000pt}%
\definecolor{currentstroke}{rgb}{0.000000,0.000000,0.000000}%
\pgfsetstrokecolor{currentstroke}%
\pgfsetstrokeopacity{0.000000}%
\pgfsetdash{}{0pt}%
\pgfpathmoveto{\pgfqpoint{5.800000in}{16.112047in}}%
\pgfpathlineto{\pgfqpoint{5.800000in}{16.112047in}}%
\pgfpathlineto{\pgfqpoint{5.800000in}{16.227777in}}%
\pgfpathlineto{\pgfqpoint{5.800000in}{16.227777in}}%
\pgfpathclose%
\pgfusepath{fill}%
\end{pgfscope}%
\begin{pgfscope}%
\pgfpathrectangle{\pgfqpoint{5.800000in}{0.720000in}}{\pgfqpoint{1.400000in}{4.620000in}}%
\pgfusepath{clip}%
\pgfsetbuttcap%
\pgfsetmiterjoin%
\definecolor{currentfill}{rgb}{0.121569,0.466667,0.705882}%
\pgfsetfillcolor{currentfill}%
\pgfsetlinewidth{0.000000pt}%
\definecolor{currentstroke}{rgb}{0.000000,0.000000,0.000000}%
\pgfsetstrokecolor{currentstroke}%
\pgfsetstrokeopacity{0.000000}%
\pgfsetdash{}{0pt}%
\pgfpathmoveto{\pgfqpoint{5.800000in}{16.227777in}}%
\pgfpathlineto{\pgfqpoint{5.800000in}{16.227777in}}%
\pgfpathlineto{\pgfqpoint{5.800000in}{16.343507in}}%
\pgfpathlineto{\pgfqpoint{5.800000in}{16.343507in}}%
\pgfpathclose%
\pgfusepath{fill}%
\end{pgfscope}%
\begin{pgfscope}%
\pgfpathrectangle{\pgfqpoint{5.800000in}{0.720000in}}{\pgfqpoint{1.400000in}{4.620000in}}%
\pgfusepath{clip}%
\pgfsetbuttcap%
\pgfsetmiterjoin%
\definecolor{currentfill}{rgb}{0.121569,0.466667,0.705882}%
\pgfsetfillcolor{currentfill}%
\pgfsetlinewidth{0.000000pt}%
\definecolor{currentstroke}{rgb}{0.000000,0.000000,0.000000}%
\pgfsetstrokecolor{currentstroke}%
\pgfsetstrokeopacity{0.000000}%
\pgfsetdash{}{0pt}%
\pgfpathmoveto{\pgfqpoint{5.800000in}{16.343507in}}%
\pgfpathlineto{\pgfqpoint{5.800000in}{16.343507in}}%
\pgfpathlineto{\pgfqpoint{5.800000in}{16.459236in}}%
\pgfpathlineto{\pgfqpoint{5.800000in}{16.459236in}}%
\pgfpathclose%
\pgfusepath{fill}%
\end{pgfscope}%
\begin{pgfscope}%
\pgfpathrectangle{\pgfqpoint{5.800000in}{0.720000in}}{\pgfqpoint{1.400000in}{4.620000in}}%
\pgfusepath{clip}%
\pgfsetbuttcap%
\pgfsetmiterjoin%
\definecolor{currentfill}{rgb}{0.121569,0.466667,0.705882}%
\pgfsetfillcolor{currentfill}%
\pgfsetlinewidth{0.000000pt}%
\definecolor{currentstroke}{rgb}{0.000000,0.000000,0.000000}%
\pgfsetstrokecolor{currentstroke}%
\pgfsetstrokeopacity{0.000000}%
\pgfsetdash{}{0pt}%
\pgfpathmoveto{\pgfqpoint{5.800000in}{16.459236in}}%
\pgfpathlineto{\pgfqpoint{5.800000in}{16.459236in}}%
\pgfpathlineto{\pgfqpoint{5.800000in}{16.574966in}}%
\pgfpathlineto{\pgfqpoint{5.800000in}{16.574966in}}%
\pgfpathclose%
\pgfusepath{fill}%
\end{pgfscope}%
\begin{pgfscope}%
\pgfpathrectangle{\pgfqpoint{5.800000in}{0.720000in}}{\pgfqpoint{1.400000in}{4.620000in}}%
\pgfusepath{clip}%
\pgfsetbuttcap%
\pgfsetmiterjoin%
\definecolor{currentfill}{rgb}{0.121569,0.466667,0.705882}%
\pgfsetfillcolor{currentfill}%
\pgfsetlinewidth{0.000000pt}%
\definecolor{currentstroke}{rgb}{0.000000,0.000000,0.000000}%
\pgfsetstrokecolor{currentstroke}%
\pgfsetstrokeopacity{0.000000}%
\pgfsetdash{}{0pt}%
\pgfpathmoveto{\pgfqpoint{5.800000in}{16.574966in}}%
\pgfpathlineto{\pgfqpoint{5.800000in}{16.574966in}}%
\pgfpathlineto{\pgfqpoint{5.800000in}{16.690696in}}%
\pgfpathlineto{\pgfqpoint{5.800000in}{16.690696in}}%
\pgfpathclose%
\pgfusepath{fill}%
\end{pgfscope}%
\begin{pgfscope}%
\pgfpathrectangle{\pgfqpoint{5.800000in}{0.720000in}}{\pgfqpoint{1.400000in}{4.620000in}}%
\pgfusepath{clip}%
\pgfsetbuttcap%
\pgfsetmiterjoin%
\definecolor{currentfill}{rgb}{0.121569,0.466667,0.705882}%
\pgfsetfillcolor{currentfill}%
\pgfsetlinewidth{0.000000pt}%
\definecolor{currentstroke}{rgb}{0.000000,0.000000,0.000000}%
\pgfsetstrokecolor{currentstroke}%
\pgfsetstrokeopacity{0.000000}%
\pgfsetdash{}{0pt}%
\pgfpathmoveto{\pgfqpoint{5.800000in}{16.690696in}}%
\pgfpathlineto{\pgfqpoint{5.800000in}{16.690696in}}%
\pgfpathlineto{\pgfqpoint{5.800000in}{16.806426in}}%
\pgfpathlineto{\pgfqpoint{5.800000in}{16.806426in}}%
\pgfpathclose%
\pgfusepath{fill}%
\end{pgfscope}%
\begin{pgfscope}%
\pgfpathrectangle{\pgfqpoint{5.800000in}{0.720000in}}{\pgfqpoint{1.400000in}{4.620000in}}%
\pgfusepath{clip}%
\pgfsetbuttcap%
\pgfsetmiterjoin%
\definecolor{currentfill}{rgb}{0.121569,0.466667,0.705882}%
\pgfsetfillcolor{currentfill}%
\pgfsetlinewidth{0.000000pt}%
\definecolor{currentstroke}{rgb}{0.000000,0.000000,0.000000}%
\pgfsetstrokecolor{currentstroke}%
\pgfsetstrokeopacity{0.000000}%
\pgfsetdash{}{0pt}%
\pgfpathmoveto{\pgfqpoint{5.800000in}{16.806426in}}%
\pgfpathlineto{\pgfqpoint{5.800000in}{16.806426in}}%
\pgfpathlineto{\pgfqpoint{5.800000in}{16.922155in}}%
\pgfpathlineto{\pgfqpoint{5.800000in}{16.922155in}}%
\pgfpathclose%
\pgfusepath{fill}%
\end{pgfscope}%
\begin{pgfscope}%
\pgfpathrectangle{\pgfqpoint{5.800000in}{0.720000in}}{\pgfqpoint{1.400000in}{4.620000in}}%
\pgfusepath{clip}%
\pgfsetbuttcap%
\pgfsetmiterjoin%
\definecolor{currentfill}{rgb}{0.121569,0.466667,0.705882}%
\pgfsetfillcolor{currentfill}%
\pgfsetlinewidth{0.000000pt}%
\definecolor{currentstroke}{rgb}{0.000000,0.000000,0.000000}%
\pgfsetstrokecolor{currentstroke}%
\pgfsetstrokeopacity{0.000000}%
\pgfsetdash{}{0pt}%
\pgfpathmoveto{\pgfqpoint{5.800000in}{16.922155in}}%
\pgfpathlineto{\pgfqpoint{5.800000in}{16.922155in}}%
\pgfpathlineto{\pgfqpoint{5.800000in}{17.037885in}}%
\pgfpathlineto{\pgfqpoint{5.800000in}{17.037885in}}%
\pgfpathclose%
\pgfusepath{fill}%
\end{pgfscope}%
\begin{pgfscope}%
\pgfpathrectangle{\pgfqpoint{5.800000in}{0.720000in}}{\pgfqpoint{1.400000in}{4.620000in}}%
\pgfusepath{clip}%
\pgfsetbuttcap%
\pgfsetmiterjoin%
\definecolor{currentfill}{rgb}{0.121569,0.466667,0.705882}%
\pgfsetfillcolor{currentfill}%
\pgfsetlinewidth{0.000000pt}%
\definecolor{currentstroke}{rgb}{0.000000,0.000000,0.000000}%
\pgfsetstrokecolor{currentstroke}%
\pgfsetstrokeopacity{0.000000}%
\pgfsetdash{}{0pt}%
\pgfpathmoveto{\pgfqpoint{5.800000in}{17.037885in}}%
\pgfpathlineto{\pgfqpoint{5.800000in}{17.037885in}}%
\pgfpathlineto{\pgfqpoint{5.800000in}{17.153615in}}%
\pgfpathlineto{\pgfqpoint{5.800000in}{17.153615in}}%
\pgfpathclose%
\pgfusepath{fill}%
\end{pgfscope}%
\begin{pgfscope}%
\pgfpathrectangle{\pgfqpoint{5.800000in}{0.720000in}}{\pgfqpoint{1.400000in}{4.620000in}}%
\pgfusepath{clip}%
\pgfsetbuttcap%
\pgfsetmiterjoin%
\definecolor{currentfill}{rgb}{0.121569,0.466667,0.705882}%
\pgfsetfillcolor{currentfill}%
\pgfsetlinewidth{0.000000pt}%
\definecolor{currentstroke}{rgb}{0.000000,0.000000,0.000000}%
\pgfsetstrokecolor{currentstroke}%
\pgfsetstrokeopacity{0.000000}%
\pgfsetdash{}{0pt}%
\pgfpathmoveto{\pgfqpoint{5.800000in}{17.153615in}}%
\pgfpathlineto{\pgfqpoint{5.800000in}{17.153615in}}%
\pgfpathlineto{\pgfqpoint{5.800000in}{17.269344in}}%
\pgfpathlineto{\pgfqpoint{5.800000in}{17.269344in}}%
\pgfpathclose%
\pgfusepath{fill}%
\end{pgfscope}%
\begin{pgfscope}%
\pgfpathrectangle{\pgfqpoint{5.800000in}{0.720000in}}{\pgfqpoint{1.400000in}{4.620000in}}%
\pgfusepath{clip}%
\pgfsetbuttcap%
\pgfsetmiterjoin%
\definecolor{currentfill}{rgb}{0.121569,0.466667,0.705882}%
\pgfsetfillcolor{currentfill}%
\pgfsetlinewidth{0.000000pt}%
\definecolor{currentstroke}{rgb}{0.000000,0.000000,0.000000}%
\pgfsetstrokecolor{currentstroke}%
\pgfsetstrokeopacity{0.000000}%
\pgfsetdash{}{0pt}%
\pgfpathmoveto{\pgfqpoint{5.800000in}{17.269344in}}%
\pgfpathlineto{\pgfqpoint{5.800000in}{17.269344in}}%
\pgfpathlineto{\pgfqpoint{5.800000in}{17.385074in}}%
\pgfpathlineto{\pgfqpoint{5.800000in}{17.385074in}}%
\pgfpathclose%
\pgfusepath{fill}%
\end{pgfscope}%
\begin{pgfscope}%
\pgfpathrectangle{\pgfqpoint{5.800000in}{0.720000in}}{\pgfqpoint{1.400000in}{4.620000in}}%
\pgfusepath{clip}%
\pgfsetbuttcap%
\pgfsetmiterjoin%
\definecolor{currentfill}{rgb}{0.121569,0.466667,0.705882}%
\pgfsetfillcolor{currentfill}%
\pgfsetlinewidth{0.000000pt}%
\definecolor{currentstroke}{rgb}{0.000000,0.000000,0.000000}%
\pgfsetstrokecolor{currentstroke}%
\pgfsetstrokeopacity{0.000000}%
\pgfsetdash{}{0pt}%
\pgfpathmoveto{\pgfqpoint{5.800000in}{17.385074in}}%
\pgfpathlineto{\pgfqpoint{5.800000in}{17.385074in}}%
\pgfpathlineto{\pgfqpoint{5.800000in}{17.500804in}}%
\pgfpathlineto{\pgfqpoint{5.800000in}{17.500804in}}%
\pgfpathclose%
\pgfusepath{fill}%
\end{pgfscope}%
\begin{pgfscope}%
\pgfpathrectangle{\pgfqpoint{5.800000in}{0.720000in}}{\pgfqpoint{1.400000in}{4.620000in}}%
\pgfusepath{clip}%
\pgfsetbuttcap%
\pgfsetmiterjoin%
\definecolor{currentfill}{rgb}{0.121569,0.466667,0.705882}%
\pgfsetfillcolor{currentfill}%
\pgfsetlinewidth{0.000000pt}%
\definecolor{currentstroke}{rgb}{0.000000,0.000000,0.000000}%
\pgfsetstrokecolor{currentstroke}%
\pgfsetstrokeopacity{0.000000}%
\pgfsetdash{}{0pt}%
\pgfpathmoveto{\pgfqpoint{5.800000in}{17.500804in}}%
\pgfpathlineto{\pgfqpoint{5.800000in}{17.500804in}}%
\pgfpathlineto{\pgfqpoint{5.800000in}{17.616533in}}%
\pgfpathlineto{\pgfqpoint{5.800000in}{17.616533in}}%
\pgfpathclose%
\pgfusepath{fill}%
\end{pgfscope}%
\begin{pgfscope}%
\pgfpathrectangle{\pgfqpoint{5.800000in}{0.720000in}}{\pgfqpoint{1.400000in}{4.620000in}}%
\pgfusepath{clip}%
\pgfsetbuttcap%
\pgfsetmiterjoin%
\definecolor{currentfill}{rgb}{0.121569,0.466667,0.705882}%
\pgfsetfillcolor{currentfill}%
\pgfsetlinewidth{0.000000pt}%
\definecolor{currentstroke}{rgb}{0.000000,0.000000,0.000000}%
\pgfsetstrokecolor{currentstroke}%
\pgfsetstrokeopacity{0.000000}%
\pgfsetdash{}{0pt}%
\pgfpathmoveto{\pgfqpoint{5.800000in}{17.616533in}}%
\pgfpathlineto{\pgfqpoint{5.800000in}{17.616533in}}%
\pgfpathlineto{\pgfqpoint{5.800000in}{17.732263in}}%
\pgfpathlineto{\pgfqpoint{5.800000in}{17.732263in}}%
\pgfpathclose%
\pgfusepath{fill}%
\end{pgfscope}%
\begin{pgfscope}%
\pgfpathrectangle{\pgfqpoint{5.800000in}{0.720000in}}{\pgfqpoint{1.400000in}{4.620000in}}%
\pgfusepath{clip}%
\pgfsetbuttcap%
\pgfsetmiterjoin%
\definecolor{currentfill}{rgb}{0.121569,0.466667,0.705882}%
\pgfsetfillcolor{currentfill}%
\pgfsetlinewidth{0.000000pt}%
\definecolor{currentstroke}{rgb}{0.000000,0.000000,0.000000}%
\pgfsetstrokecolor{currentstroke}%
\pgfsetstrokeopacity{0.000000}%
\pgfsetdash{}{0pt}%
\pgfpathmoveto{\pgfqpoint{5.800000in}{17.732263in}}%
\pgfpathlineto{\pgfqpoint{5.800000in}{17.732263in}}%
\pgfpathlineto{\pgfqpoint{5.800000in}{17.847993in}}%
\pgfpathlineto{\pgfqpoint{5.800000in}{17.847993in}}%
\pgfpathclose%
\pgfusepath{fill}%
\end{pgfscope}%
\begin{pgfscope}%
\pgfpathrectangle{\pgfqpoint{5.800000in}{0.720000in}}{\pgfqpoint{1.400000in}{4.620000in}}%
\pgfusepath{clip}%
\pgfsetbuttcap%
\pgfsetmiterjoin%
\definecolor{currentfill}{rgb}{0.121569,0.466667,0.705882}%
\pgfsetfillcolor{currentfill}%
\pgfsetlinewidth{0.000000pt}%
\definecolor{currentstroke}{rgb}{0.000000,0.000000,0.000000}%
\pgfsetstrokecolor{currentstroke}%
\pgfsetstrokeopacity{0.000000}%
\pgfsetdash{}{0pt}%
\pgfpathmoveto{\pgfqpoint{5.800000in}{17.847993in}}%
\pgfpathlineto{\pgfqpoint{5.800000in}{17.847993in}}%
\pgfpathlineto{\pgfqpoint{5.800000in}{17.963722in}}%
\pgfpathlineto{\pgfqpoint{5.800000in}{17.963722in}}%
\pgfpathclose%
\pgfusepath{fill}%
\end{pgfscope}%
\begin{pgfscope}%
\pgfpathrectangle{\pgfqpoint{5.800000in}{0.720000in}}{\pgfqpoint{1.400000in}{4.620000in}}%
\pgfusepath{clip}%
\pgfsetbuttcap%
\pgfsetmiterjoin%
\definecolor{currentfill}{rgb}{0.121569,0.466667,0.705882}%
\pgfsetfillcolor{currentfill}%
\pgfsetlinewidth{0.000000pt}%
\definecolor{currentstroke}{rgb}{0.000000,0.000000,0.000000}%
\pgfsetstrokecolor{currentstroke}%
\pgfsetstrokeopacity{0.000000}%
\pgfsetdash{}{0pt}%
\pgfpathmoveto{\pgfqpoint{5.800000in}{17.963722in}}%
\pgfpathlineto{\pgfqpoint{5.800000in}{17.963722in}}%
\pgfpathlineto{\pgfqpoint{5.800000in}{18.079452in}}%
\pgfpathlineto{\pgfqpoint{5.800000in}{18.079452in}}%
\pgfpathclose%
\pgfusepath{fill}%
\end{pgfscope}%
\begin{pgfscope}%
\pgfpathrectangle{\pgfqpoint{5.800000in}{0.720000in}}{\pgfqpoint{1.400000in}{4.620000in}}%
\pgfusepath{clip}%
\pgfsetbuttcap%
\pgfsetmiterjoin%
\definecolor{currentfill}{rgb}{0.121569,0.466667,0.705882}%
\pgfsetfillcolor{currentfill}%
\pgfsetlinewidth{0.000000pt}%
\definecolor{currentstroke}{rgb}{0.000000,0.000000,0.000000}%
\pgfsetstrokecolor{currentstroke}%
\pgfsetstrokeopacity{0.000000}%
\pgfsetdash{}{0pt}%
\pgfpathmoveto{\pgfqpoint{5.800000in}{18.079452in}}%
\pgfpathlineto{\pgfqpoint{5.800000in}{18.079452in}}%
\pgfpathlineto{\pgfqpoint{5.800000in}{18.195182in}}%
\pgfpathlineto{\pgfqpoint{5.800000in}{18.195182in}}%
\pgfpathclose%
\pgfusepath{fill}%
\end{pgfscope}%
\begin{pgfscope}%
\pgfpathrectangle{\pgfqpoint{5.800000in}{0.720000in}}{\pgfqpoint{1.400000in}{4.620000in}}%
\pgfusepath{clip}%
\pgfsetbuttcap%
\pgfsetmiterjoin%
\definecolor{currentfill}{rgb}{0.121569,0.466667,0.705882}%
\pgfsetfillcolor{currentfill}%
\pgfsetlinewidth{0.000000pt}%
\definecolor{currentstroke}{rgb}{0.000000,0.000000,0.000000}%
\pgfsetstrokecolor{currentstroke}%
\pgfsetstrokeopacity{0.000000}%
\pgfsetdash{}{0pt}%
\pgfpathmoveto{\pgfqpoint{5.800000in}{18.195182in}}%
\pgfpathlineto{\pgfqpoint{5.800000in}{18.195182in}}%
\pgfpathlineto{\pgfqpoint{5.800000in}{18.310911in}}%
\pgfpathlineto{\pgfqpoint{5.800000in}{18.310911in}}%
\pgfpathclose%
\pgfusepath{fill}%
\end{pgfscope}%
\begin{pgfscope}%
\pgfpathrectangle{\pgfqpoint{5.800000in}{0.720000in}}{\pgfqpoint{1.400000in}{4.620000in}}%
\pgfusepath{clip}%
\pgfsetbuttcap%
\pgfsetmiterjoin%
\definecolor{currentfill}{rgb}{0.121569,0.466667,0.705882}%
\pgfsetfillcolor{currentfill}%
\pgfsetlinewidth{0.000000pt}%
\definecolor{currentstroke}{rgb}{0.000000,0.000000,0.000000}%
\pgfsetstrokecolor{currentstroke}%
\pgfsetstrokeopacity{0.000000}%
\pgfsetdash{}{0pt}%
\pgfpathmoveto{\pgfqpoint{5.800000in}{18.310911in}}%
\pgfpathlineto{\pgfqpoint{5.800000in}{18.310911in}}%
\pgfpathlineto{\pgfqpoint{5.800000in}{18.426641in}}%
\pgfpathlineto{\pgfqpoint{5.800000in}{18.426641in}}%
\pgfpathclose%
\pgfusepath{fill}%
\end{pgfscope}%
\begin{pgfscope}%
\pgfpathrectangle{\pgfqpoint{5.800000in}{0.720000in}}{\pgfqpoint{1.400000in}{4.620000in}}%
\pgfusepath{clip}%
\pgfsetbuttcap%
\pgfsetmiterjoin%
\definecolor{currentfill}{rgb}{0.121569,0.466667,0.705882}%
\pgfsetfillcolor{currentfill}%
\pgfsetlinewidth{0.000000pt}%
\definecolor{currentstroke}{rgb}{0.000000,0.000000,0.000000}%
\pgfsetstrokecolor{currentstroke}%
\pgfsetstrokeopacity{0.000000}%
\pgfsetdash{}{0pt}%
\pgfpathmoveto{\pgfqpoint{5.800000in}{18.426641in}}%
\pgfpathlineto{\pgfqpoint{5.800000in}{18.426641in}}%
\pgfpathlineto{\pgfqpoint{5.800000in}{18.542371in}}%
\pgfpathlineto{\pgfqpoint{5.800000in}{18.542371in}}%
\pgfpathclose%
\pgfusepath{fill}%
\end{pgfscope}%
\begin{pgfscope}%
\pgfpathrectangle{\pgfqpoint{5.800000in}{0.720000in}}{\pgfqpoint{1.400000in}{4.620000in}}%
\pgfusepath{clip}%
\pgfsetbuttcap%
\pgfsetmiterjoin%
\definecolor{currentfill}{rgb}{0.121569,0.466667,0.705882}%
\pgfsetfillcolor{currentfill}%
\pgfsetlinewidth{0.000000pt}%
\definecolor{currentstroke}{rgb}{0.000000,0.000000,0.000000}%
\pgfsetstrokecolor{currentstroke}%
\pgfsetstrokeopacity{0.000000}%
\pgfsetdash{}{0pt}%
\pgfpathmoveto{\pgfqpoint{5.800000in}{18.542371in}}%
\pgfpathlineto{\pgfqpoint{5.800000in}{18.542371in}}%
\pgfpathlineto{\pgfqpoint{5.800000in}{18.658100in}}%
\pgfpathlineto{\pgfqpoint{5.800000in}{18.658100in}}%
\pgfpathclose%
\pgfusepath{fill}%
\end{pgfscope}%
\begin{pgfscope}%
\pgfpathrectangle{\pgfqpoint{5.800000in}{0.720000in}}{\pgfqpoint{1.400000in}{4.620000in}}%
\pgfusepath{clip}%
\pgfsetbuttcap%
\pgfsetmiterjoin%
\definecolor{currentfill}{rgb}{0.121569,0.466667,0.705882}%
\pgfsetfillcolor{currentfill}%
\pgfsetlinewidth{0.000000pt}%
\definecolor{currentstroke}{rgb}{0.000000,0.000000,0.000000}%
\pgfsetstrokecolor{currentstroke}%
\pgfsetstrokeopacity{0.000000}%
\pgfsetdash{}{0pt}%
\pgfpathmoveto{\pgfqpoint{5.800000in}{18.658100in}}%
\pgfpathlineto{\pgfqpoint{5.800000in}{18.658100in}}%
\pgfpathlineto{\pgfqpoint{5.800000in}{18.773830in}}%
\pgfpathlineto{\pgfqpoint{5.800000in}{18.773830in}}%
\pgfpathclose%
\pgfusepath{fill}%
\end{pgfscope}%
\begin{pgfscope}%
\pgfpathrectangle{\pgfqpoint{5.800000in}{0.720000in}}{\pgfqpoint{1.400000in}{4.620000in}}%
\pgfusepath{clip}%
\pgfsetbuttcap%
\pgfsetmiterjoin%
\definecolor{currentfill}{rgb}{0.121569,0.466667,0.705882}%
\pgfsetfillcolor{currentfill}%
\pgfsetlinewidth{0.000000pt}%
\definecolor{currentstroke}{rgb}{0.000000,0.000000,0.000000}%
\pgfsetstrokecolor{currentstroke}%
\pgfsetstrokeopacity{0.000000}%
\pgfsetdash{}{0pt}%
\pgfpathmoveto{\pgfqpoint{5.800000in}{18.773830in}}%
\pgfpathlineto{\pgfqpoint{5.800000in}{18.773830in}}%
\pgfpathlineto{\pgfqpoint{5.800000in}{18.889560in}}%
\pgfpathlineto{\pgfqpoint{5.800000in}{18.889560in}}%
\pgfpathclose%
\pgfusepath{fill}%
\end{pgfscope}%
\begin{pgfscope}%
\pgfpathrectangle{\pgfqpoint{5.800000in}{0.720000in}}{\pgfqpoint{1.400000in}{4.620000in}}%
\pgfusepath{clip}%
\pgfsetbuttcap%
\pgfsetmiterjoin%
\definecolor{currentfill}{rgb}{0.121569,0.466667,0.705882}%
\pgfsetfillcolor{currentfill}%
\pgfsetlinewidth{0.000000pt}%
\definecolor{currentstroke}{rgb}{0.000000,0.000000,0.000000}%
\pgfsetstrokecolor{currentstroke}%
\pgfsetstrokeopacity{0.000000}%
\pgfsetdash{}{0pt}%
\pgfpathmoveto{\pgfqpoint{5.800000in}{18.889560in}}%
\pgfpathlineto{\pgfqpoint{5.800000in}{18.889560in}}%
\pgfpathlineto{\pgfqpoint{5.800000in}{19.005289in}}%
\pgfpathlineto{\pgfqpoint{5.800000in}{19.005289in}}%
\pgfpathclose%
\pgfusepath{fill}%
\end{pgfscope}%
\begin{pgfscope}%
\definecolor{textcolor}{rgb}{0.000000,0.000000,0.000000}%
\pgfsetstrokecolor{textcolor}%
\pgfsetfillcolor{textcolor}%
\pgftext[x=6.500000in,y=0.664444in,,top]{\color{textcolor}\sffamily\fontsize{20.000000}{24.000000}\selectfont \(\displaystyle \mathrm{arb.\ unit}\)}%
\end{pgfscope}%
\begin{pgfscope}%
\pgfsetrectcap%
\pgfsetmiterjoin%
\pgfsetlinewidth{0.803000pt}%
\definecolor{currentstroke}{rgb}{0.000000,0.000000,0.000000}%
\pgfsetstrokecolor{currentstroke}%
\pgfsetdash{}{0pt}%
\pgfpathmoveto{\pgfqpoint{5.800000in}{0.720000in}}%
\pgfpathlineto{\pgfqpoint{5.800000in}{5.340000in}}%
\pgfusepath{stroke}%
\end{pgfscope}%
\begin{pgfscope}%
\pgfsetrectcap%
\pgfsetmiterjoin%
\pgfsetlinewidth{0.803000pt}%
\definecolor{currentstroke}{rgb}{0.000000,0.000000,0.000000}%
\pgfsetstrokecolor{currentstroke}%
\pgfsetdash{}{0pt}%
\pgfpathmoveto{\pgfqpoint{7.200000in}{0.720000in}}%
\pgfpathlineto{\pgfqpoint{7.200000in}{5.340000in}}%
\pgfusepath{stroke}%
\end{pgfscope}%
\begin{pgfscope}%
\pgfsetrectcap%
\pgfsetmiterjoin%
\pgfsetlinewidth{0.803000pt}%
\definecolor{currentstroke}{rgb}{0.000000,0.000000,0.000000}%
\pgfsetstrokecolor{currentstroke}%
\pgfsetdash{}{0pt}%
\pgfpathmoveto{\pgfqpoint{5.800000in}{0.720000in}}%
\pgfpathlineto{\pgfqpoint{7.200000in}{0.720000in}}%
\pgfusepath{stroke}%
\end{pgfscope}%
\begin{pgfscope}%
\pgfsetrectcap%
\pgfsetmiterjoin%
\pgfsetlinewidth{0.803000pt}%
\definecolor{currentstroke}{rgb}{0.000000,0.000000,0.000000}%
\pgfsetstrokecolor{currentstroke}%
\pgfsetdash{}{0pt}%
\pgfpathmoveto{\pgfqpoint{5.800000in}{5.340000in}}%
\pgfpathlineto{\pgfqpoint{7.200000in}{5.340000in}}%
\pgfusepath{stroke}%
\end{pgfscope}%
\end{pgfpicture}%
\makeatother%
\endgroup%
}
    \caption{\label{fig:cnn-npe} $D_\mathrm{w}$ histogram and its distributions conditioned \\ on $N_{\mathrm{PE}}$. ``arbi. unit'' means arbitrary unit.}
  \end{subfigure}
  \begin{subfigure}{.5\textwidth}
    \centering
    \resizebox{\textwidth}{!}{%% Creator: Matplotlib, PGF backend
%%
%% To include the figure in your LaTeX document, write
%%   \input{<filename>.pgf}
%%
%% Make sure the required packages are loaded in your preamble
%%   \usepackage{pgf}
%%
%% Also ensure that all the required font packages are loaded; for instance,
%% the lmodern package is sometimes necessary when using math font.
%%   \usepackage{lmodern}
%%
%% Figures using additional raster images can only be included by \input if
%% they are in the same directory as the main LaTeX file. For loading figures
%% from other directories you can use the `import` package
%%   \usepackage{import}
%%
%% and then include the figures with
%%   \import{<path to file>}{<filename>.pgf}
%%
%% Matplotlib used the following preamble
%%   \usepackage[detect-all,locale=DE]{siunitx}
%%
\begingroup%
\makeatletter%
\begin{pgfpicture}%
\pgfpathrectangle{\pgfpointorigin}{\pgfqpoint{8.000000in}{6.000000in}}%
\pgfusepath{use as bounding box, clip}%
\begin{pgfscope}%
\pgfsetbuttcap%
\pgfsetmiterjoin%
\definecolor{currentfill}{rgb}{1.000000,1.000000,1.000000}%
\pgfsetfillcolor{currentfill}%
\pgfsetlinewidth{0.000000pt}%
\definecolor{currentstroke}{rgb}{1.000000,1.000000,1.000000}%
\pgfsetstrokecolor{currentstroke}%
\pgfsetdash{}{0pt}%
\pgfpathmoveto{\pgfqpoint{0.000000in}{0.000000in}}%
\pgfpathlineto{\pgfqpoint{8.000000in}{0.000000in}}%
\pgfpathlineto{\pgfqpoint{8.000000in}{6.000000in}}%
\pgfpathlineto{\pgfqpoint{0.000000in}{6.000000in}}%
\pgfpathlineto{\pgfqpoint{0.000000in}{0.000000in}}%
\pgfpathclose%
\pgfusepath{fill}%
\end{pgfscope}%
\begin{pgfscope}%
\pgfsetbuttcap%
\pgfsetmiterjoin%
\definecolor{currentfill}{rgb}{1.000000,1.000000,1.000000}%
\pgfsetfillcolor{currentfill}%
\pgfsetlinewidth{0.000000pt}%
\definecolor{currentstroke}{rgb}{0.000000,0.000000,0.000000}%
\pgfsetstrokecolor{currentstroke}%
\pgfsetstrokeopacity{0.000000}%
\pgfsetdash{}{0pt}%
\pgfpathmoveto{\pgfqpoint{1.000000in}{0.720000in}}%
\pgfpathlineto{\pgfqpoint{7.200000in}{0.720000in}}%
\pgfpathlineto{\pgfqpoint{7.200000in}{5.340000in}}%
\pgfpathlineto{\pgfqpoint{1.000000in}{5.340000in}}%
\pgfpathlineto{\pgfqpoint{1.000000in}{0.720000in}}%
\pgfpathclose%
\pgfusepath{fill}%
\end{pgfscope}%
\begin{pgfscope}%
\pgfsetbuttcap%
\pgfsetroundjoin%
\definecolor{currentfill}{rgb}{0.000000,0.000000,0.000000}%
\pgfsetfillcolor{currentfill}%
\pgfsetlinewidth{0.803000pt}%
\definecolor{currentstroke}{rgb}{0.000000,0.000000,0.000000}%
\pgfsetstrokecolor{currentstroke}%
\pgfsetdash{}{0pt}%
\pgfsys@defobject{currentmarker}{\pgfqpoint{0.000000in}{-0.048611in}}{\pgfqpoint{0.000000in}{0.000000in}}{%
\pgfpathmoveto{\pgfqpoint{0.000000in}{0.000000in}}%
\pgfpathlineto{\pgfqpoint{0.000000in}{-0.048611in}}%
\pgfusepath{stroke,fill}%
}%
\begin{pgfscope}%
\pgfsys@transformshift{1.310000in}{0.720000in}%
\pgfsys@useobject{currentmarker}{}%
\end{pgfscope}%
\end{pgfscope}%
\begin{pgfscope}%
\definecolor{textcolor}{rgb}{0.000000,0.000000,0.000000}%
\pgfsetstrokecolor{textcolor}%
\pgfsetfillcolor{textcolor}%
\pgftext[x=1.310000in,y=0.622778in,,top]{\color{textcolor}\sffamily\fontsize{20.000000}{24.000000}\selectfont \(\displaystyle {450}\)}%
\end{pgfscope}%
\begin{pgfscope}%
\pgfsetbuttcap%
\pgfsetroundjoin%
\definecolor{currentfill}{rgb}{0.000000,0.000000,0.000000}%
\pgfsetfillcolor{currentfill}%
\pgfsetlinewidth{0.803000pt}%
\definecolor{currentstroke}{rgb}{0.000000,0.000000,0.000000}%
\pgfsetstrokecolor{currentstroke}%
\pgfsetdash{}{0pt}%
\pgfsys@defobject{currentmarker}{\pgfqpoint{0.000000in}{-0.048611in}}{\pgfqpoint{0.000000in}{0.000000in}}{%
\pgfpathmoveto{\pgfqpoint{0.000000in}{0.000000in}}%
\pgfpathlineto{\pgfqpoint{0.000000in}{-0.048611in}}%
\pgfusepath{stroke,fill}%
}%
\begin{pgfscope}%
\pgfsys@transformshift{2.860000in}{0.720000in}%
\pgfsys@useobject{currentmarker}{}%
\end{pgfscope}%
\end{pgfscope}%
\begin{pgfscope}%
\definecolor{textcolor}{rgb}{0.000000,0.000000,0.000000}%
\pgfsetstrokecolor{textcolor}%
\pgfsetfillcolor{textcolor}%
\pgftext[x=2.860000in,y=0.622778in,,top]{\color{textcolor}\sffamily\fontsize{20.000000}{24.000000}\selectfont \(\displaystyle {500}\)}%
\end{pgfscope}%
\begin{pgfscope}%
\pgfsetbuttcap%
\pgfsetroundjoin%
\definecolor{currentfill}{rgb}{0.000000,0.000000,0.000000}%
\pgfsetfillcolor{currentfill}%
\pgfsetlinewidth{0.803000pt}%
\definecolor{currentstroke}{rgb}{0.000000,0.000000,0.000000}%
\pgfsetstrokecolor{currentstroke}%
\pgfsetdash{}{0pt}%
\pgfsys@defobject{currentmarker}{\pgfqpoint{0.000000in}{-0.048611in}}{\pgfqpoint{0.000000in}{0.000000in}}{%
\pgfpathmoveto{\pgfqpoint{0.000000in}{0.000000in}}%
\pgfpathlineto{\pgfqpoint{0.000000in}{-0.048611in}}%
\pgfusepath{stroke,fill}%
}%
\begin{pgfscope}%
\pgfsys@transformshift{4.410000in}{0.720000in}%
\pgfsys@useobject{currentmarker}{}%
\end{pgfscope}%
\end{pgfscope}%
\begin{pgfscope}%
\definecolor{textcolor}{rgb}{0.000000,0.000000,0.000000}%
\pgfsetstrokecolor{textcolor}%
\pgfsetfillcolor{textcolor}%
\pgftext[x=4.410000in,y=0.622778in,,top]{\color{textcolor}\sffamily\fontsize{20.000000}{24.000000}\selectfont \(\displaystyle {550}\)}%
\end{pgfscope}%
\begin{pgfscope}%
\pgfsetbuttcap%
\pgfsetroundjoin%
\definecolor{currentfill}{rgb}{0.000000,0.000000,0.000000}%
\pgfsetfillcolor{currentfill}%
\pgfsetlinewidth{0.803000pt}%
\definecolor{currentstroke}{rgb}{0.000000,0.000000,0.000000}%
\pgfsetstrokecolor{currentstroke}%
\pgfsetdash{}{0pt}%
\pgfsys@defobject{currentmarker}{\pgfqpoint{0.000000in}{-0.048611in}}{\pgfqpoint{0.000000in}{0.000000in}}{%
\pgfpathmoveto{\pgfqpoint{0.000000in}{0.000000in}}%
\pgfpathlineto{\pgfqpoint{0.000000in}{-0.048611in}}%
\pgfusepath{stroke,fill}%
}%
\begin{pgfscope}%
\pgfsys@transformshift{5.960000in}{0.720000in}%
\pgfsys@useobject{currentmarker}{}%
\end{pgfscope}%
\end{pgfscope}%
\begin{pgfscope}%
\definecolor{textcolor}{rgb}{0.000000,0.000000,0.000000}%
\pgfsetstrokecolor{textcolor}%
\pgfsetfillcolor{textcolor}%
\pgftext[x=5.960000in,y=0.622778in,,top]{\color{textcolor}\sffamily\fontsize{20.000000}{24.000000}\selectfont \(\displaystyle {600}\)}%
\end{pgfscope}%
\begin{pgfscope}%
\definecolor{textcolor}{rgb}{0.000000,0.000000,0.000000}%
\pgfsetstrokecolor{textcolor}%
\pgfsetfillcolor{textcolor}%
\pgftext[x=4.100000in,y=0.311155in,,top]{\color{textcolor}\sffamily\fontsize{20.000000}{24.000000}\selectfont \(\displaystyle \mathrm{t}/\si{ns}\)}%
\end{pgfscope}%
\begin{pgfscope}%
\pgfsetbuttcap%
\pgfsetroundjoin%
\definecolor{currentfill}{rgb}{0.000000,0.000000,0.000000}%
\pgfsetfillcolor{currentfill}%
\pgfsetlinewidth{0.803000pt}%
\definecolor{currentstroke}{rgb}{0.000000,0.000000,0.000000}%
\pgfsetstrokecolor{currentstroke}%
\pgfsetdash{}{0pt}%
\pgfsys@defobject{currentmarker}{\pgfqpoint{-0.048611in}{0.000000in}}{\pgfqpoint{-0.000000in}{0.000000in}}{%
\pgfpathmoveto{\pgfqpoint{-0.000000in}{0.000000in}}%
\pgfpathlineto{\pgfqpoint{-0.048611in}{0.000000in}}%
\pgfusepath{stroke,fill}%
}%
\begin{pgfscope}%
\pgfsys@transformshift{1.000000in}{1.109002in}%
\pgfsys@useobject{currentmarker}{}%
\end{pgfscope}%
\end{pgfscope}%
\begin{pgfscope}%
\definecolor{textcolor}{rgb}{0.000000,0.000000,0.000000}%
\pgfsetstrokecolor{textcolor}%
\pgfsetfillcolor{textcolor}%
\pgftext[x=0.770670in, y=1.008983in, left, base]{\color{textcolor}\sffamily\fontsize{20.000000}{24.000000}\selectfont \(\displaystyle {0}\)}%
\end{pgfscope}%
\begin{pgfscope}%
\pgfsetbuttcap%
\pgfsetroundjoin%
\definecolor{currentfill}{rgb}{0.000000,0.000000,0.000000}%
\pgfsetfillcolor{currentfill}%
\pgfsetlinewidth{0.803000pt}%
\definecolor{currentstroke}{rgb}{0.000000,0.000000,0.000000}%
\pgfsetstrokecolor{currentstroke}%
\pgfsetdash{}{0pt}%
\pgfsys@defobject{currentmarker}{\pgfqpoint{-0.048611in}{0.000000in}}{\pgfqpoint{-0.000000in}{0.000000in}}{%
\pgfpathmoveto{\pgfqpoint{-0.000000in}{0.000000in}}%
\pgfpathlineto{\pgfqpoint{-0.048611in}{0.000000in}}%
\pgfusepath{stroke,fill}%
}%
\begin{pgfscope}%
\pgfsys@transformshift{1.000000in}{2.004270in}%
\pgfsys@useobject{currentmarker}{}%
\end{pgfscope}%
\end{pgfscope}%
\begin{pgfscope}%
\definecolor{textcolor}{rgb}{0.000000,0.000000,0.000000}%
\pgfsetstrokecolor{textcolor}%
\pgfsetfillcolor{textcolor}%
\pgftext[x=0.638563in, y=1.904251in, left, base]{\color{textcolor}\sffamily\fontsize{20.000000}{24.000000}\selectfont \(\displaystyle {10}\)}%
\end{pgfscope}%
\begin{pgfscope}%
\pgfsetbuttcap%
\pgfsetroundjoin%
\definecolor{currentfill}{rgb}{0.000000,0.000000,0.000000}%
\pgfsetfillcolor{currentfill}%
\pgfsetlinewidth{0.803000pt}%
\definecolor{currentstroke}{rgb}{0.000000,0.000000,0.000000}%
\pgfsetstrokecolor{currentstroke}%
\pgfsetdash{}{0pt}%
\pgfsys@defobject{currentmarker}{\pgfqpoint{-0.048611in}{0.000000in}}{\pgfqpoint{-0.000000in}{0.000000in}}{%
\pgfpathmoveto{\pgfqpoint{-0.000000in}{0.000000in}}%
\pgfpathlineto{\pgfqpoint{-0.048611in}{0.000000in}}%
\pgfusepath{stroke,fill}%
}%
\begin{pgfscope}%
\pgfsys@transformshift{1.000000in}{2.899538in}%
\pgfsys@useobject{currentmarker}{}%
\end{pgfscope}%
\end{pgfscope}%
\begin{pgfscope}%
\definecolor{textcolor}{rgb}{0.000000,0.000000,0.000000}%
\pgfsetstrokecolor{textcolor}%
\pgfsetfillcolor{textcolor}%
\pgftext[x=0.638563in, y=2.799519in, left, base]{\color{textcolor}\sffamily\fontsize{20.000000}{24.000000}\selectfont \(\displaystyle {20}\)}%
\end{pgfscope}%
\begin{pgfscope}%
\pgfsetbuttcap%
\pgfsetroundjoin%
\definecolor{currentfill}{rgb}{0.000000,0.000000,0.000000}%
\pgfsetfillcolor{currentfill}%
\pgfsetlinewidth{0.803000pt}%
\definecolor{currentstroke}{rgb}{0.000000,0.000000,0.000000}%
\pgfsetstrokecolor{currentstroke}%
\pgfsetdash{}{0pt}%
\pgfsys@defobject{currentmarker}{\pgfqpoint{-0.048611in}{0.000000in}}{\pgfqpoint{-0.000000in}{0.000000in}}{%
\pgfpathmoveto{\pgfqpoint{-0.000000in}{0.000000in}}%
\pgfpathlineto{\pgfqpoint{-0.048611in}{0.000000in}}%
\pgfusepath{stroke,fill}%
}%
\begin{pgfscope}%
\pgfsys@transformshift{1.000000in}{3.794806in}%
\pgfsys@useobject{currentmarker}{}%
\end{pgfscope}%
\end{pgfscope}%
\begin{pgfscope}%
\definecolor{textcolor}{rgb}{0.000000,0.000000,0.000000}%
\pgfsetstrokecolor{textcolor}%
\pgfsetfillcolor{textcolor}%
\pgftext[x=0.638563in, y=3.694787in, left, base]{\color{textcolor}\sffamily\fontsize{20.000000}{24.000000}\selectfont \(\displaystyle {30}\)}%
\end{pgfscope}%
\begin{pgfscope}%
\pgfsetbuttcap%
\pgfsetroundjoin%
\definecolor{currentfill}{rgb}{0.000000,0.000000,0.000000}%
\pgfsetfillcolor{currentfill}%
\pgfsetlinewidth{0.803000pt}%
\definecolor{currentstroke}{rgb}{0.000000,0.000000,0.000000}%
\pgfsetstrokecolor{currentstroke}%
\pgfsetdash{}{0pt}%
\pgfsys@defobject{currentmarker}{\pgfqpoint{-0.048611in}{0.000000in}}{\pgfqpoint{-0.000000in}{0.000000in}}{%
\pgfpathmoveto{\pgfqpoint{-0.000000in}{0.000000in}}%
\pgfpathlineto{\pgfqpoint{-0.048611in}{0.000000in}}%
\pgfusepath{stroke,fill}%
}%
\begin{pgfscope}%
\pgfsys@transformshift{1.000000in}{4.690074in}%
\pgfsys@useobject{currentmarker}{}%
\end{pgfscope}%
\end{pgfscope}%
\begin{pgfscope}%
\definecolor{textcolor}{rgb}{0.000000,0.000000,0.000000}%
\pgfsetstrokecolor{textcolor}%
\pgfsetfillcolor{textcolor}%
\pgftext[x=0.638563in, y=4.590054in, left, base]{\color{textcolor}\sffamily\fontsize{20.000000}{24.000000}\selectfont \(\displaystyle {40}\)}%
\end{pgfscope}%
\begin{pgfscope}%
\definecolor{textcolor}{rgb}{0.000000,0.000000,0.000000}%
\pgfsetstrokecolor{textcolor}%
\pgfsetfillcolor{textcolor}%
\pgftext[x=0.583007in,y=3.030000in,,bottom,rotate=90.000000]{\color{textcolor}\sffamily\fontsize{20.000000}{24.000000}\selectfont \(\displaystyle \mathrm{Voltage}/\si{mV}\)}%
\end{pgfscope}%
\begin{pgfscope}%
\pgfpathrectangle{\pgfqpoint{1.000000in}{0.720000in}}{\pgfqpoint{6.200000in}{4.620000in}}%
\pgfusepath{clip}%
\pgfsetrectcap%
\pgfsetroundjoin%
\pgfsetlinewidth{2.007500pt}%
\definecolor{currentstroke}{rgb}{0.121569,0.466667,0.705882}%
\pgfsetstrokecolor{currentstroke}%
\pgfsetdash{}{0pt}%
\pgfpathmoveto{\pgfqpoint{0.990000in}{1.164926in}}%
\pgfpathlineto{\pgfqpoint{1.000000in}{1.200117in}}%
\pgfpathlineto{\pgfqpoint{1.031000in}{1.073192in}}%
\pgfpathlineto{\pgfqpoint{1.062000in}{1.147885in}}%
\pgfpathlineto{\pgfqpoint{1.093000in}{1.135477in}}%
\pgfpathlineto{\pgfqpoint{1.124000in}{1.121439in}}%
\pgfpathlineto{\pgfqpoint{1.155000in}{1.210126in}}%
\pgfpathlineto{\pgfqpoint{1.186000in}{1.132056in}}%
\pgfpathlineto{\pgfqpoint{1.217000in}{1.069666in}}%
\pgfpathlineto{\pgfqpoint{1.248000in}{1.198337in}}%
\pgfpathlineto{\pgfqpoint{1.279000in}{1.132919in}}%
\pgfpathlineto{\pgfqpoint{1.310000in}{1.055958in}}%
\pgfpathlineto{\pgfqpoint{1.341000in}{1.153482in}}%
\pgfpathlineto{\pgfqpoint{1.372000in}{1.093744in}}%
\pgfpathlineto{\pgfqpoint{1.403000in}{1.306032in}}%
\pgfpathlineto{\pgfqpoint{1.434000in}{1.051842in}}%
\pgfpathlineto{\pgfqpoint{1.465000in}{0.995276in}}%
\pgfpathlineto{\pgfqpoint{1.496000in}{1.072373in}}%
\pgfpathlineto{\pgfqpoint{1.527000in}{1.020957in}}%
\pgfpathlineto{\pgfqpoint{1.558000in}{0.996263in}}%
\pgfpathlineto{\pgfqpoint{1.589000in}{1.100873in}}%
\pgfpathlineto{\pgfqpoint{1.620000in}{1.182232in}}%
\pgfpathlineto{\pgfqpoint{1.651000in}{1.160629in}}%
\pgfpathlineto{\pgfqpoint{1.682000in}{1.264916in}}%
\pgfpathlineto{\pgfqpoint{1.713000in}{1.185927in}}%
\pgfpathlineto{\pgfqpoint{1.744000in}{1.208894in}}%
\pgfpathlineto{\pgfqpoint{1.775000in}{1.222343in}}%
\pgfpathlineto{\pgfqpoint{1.806000in}{1.191883in}}%
\pgfpathlineto{\pgfqpoint{1.837000in}{1.091525in}}%
\pgfpathlineto{\pgfqpoint{1.868000in}{0.999884in}}%
\pgfpathlineto{\pgfqpoint{1.899000in}{1.127138in}}%
\pgfpathlineto{\pgfqpoint{1.930000in}{1.081839in}}%
\pgfpathlineto{\pgfqpoint{1.961000in}{1.187655in}}%
\pgfpathlineto{\pgfqpoint{1.992000in}{1.182336in}}%
\pgfpathlineto{\pgfqpoint{2.023000in}{1.384380in}}%
\pgfpathlineto{\pgfqpoint{2.054000in}{1.719851in}}%
\pgfpathlineto{\pgfqpoint{2.085000in}{2.107918in}}%
\pgfpathlineto{\pgfqpoint{2.116000in}{2.784651in}}%
\pgfpathlineto{\pgfqpoint{2.147000in}{2.955721in}}%
\pgfpathlineto{\pgfqpoint{2.178000in}{3.338539in}}%
\pgfpathlineto{\pgfqpoint{2.209000in}{3.459601in}}%
\pgfpathlineto{\pgfqpoint{2.240000in}{3.236344in}}%
\pgfpathlineto{\pgfqpoint{2.271000in}{3.220427in}}%
\pgfpathlineto{\pgfqpoint{2.302000in}{3.145605in}}%
\pgfpathlineto{\pgfqpoint{2.333000in}{2.678530in}}%
\pgfpathlineto{\pgfqpoint{2.364000in}{2.517063in}}%
\pgfpathlineto{\pgfqpoint{2.395000in}{2.506665in}}%
\pgfpathlineto{\pgfqpoint{2.426000in}{2.234079in}}%
\pgfpathlineto{\pgfqpoint{2.457000in}{1.878240in}}%
\pgfpathlineto{\pgfqpoint{2.488000in}{1.793637in}}%
\pgfpathlineto{\pgfqpoint{2.519000in}{1.716973in}}%
\pgfpathlineto{\pgfqpoint{2.550000in}{1.675631in}}%
\pgfpathlineto{\pgfqpoint{2.581000in}{1.880210in}}%
\pgfpathlineto{\pgfqpoint{2.612000in}{2.419920in}}%
\pgfpathlineto{\pgfqpoint{2.643000in}{2.646214in}}%
\pgfpathlineto{\pgfqpoint{2.674000in}{3.017097in}}%
\pgfpathlineto{\pgfqpoint{2.705000in}{3.101285in}}%
\pgfpathlineto{\pgfqpoint{2.736000in}{2.994316in}}%
\pgfpathlineto{\pgfqpoint{2.767000in}{2.841146in}}%
\pgfpathlineto{\pgfqpoint{2.798000in}{2.707818in}}%
\pgfpathlineto{\pgfqpoint{2.829000in}{2.552232in}}%
\pgfpathlineto{\pgfqpoint{2.860000in}{2.442091in}}%
\pgfpathlineto{\pgfqpoint{2.891000in}{2.178630in}}%
\pgfpathlineto{\pgfqpoint{2.922000in}{2.222316in}}%
\pgfpathlineto{\pgfqpoint{2.953000in}{2.033937in}}%
\pgfpathlineto{\pgfqpoint{2.984000in}{2.523017in}}%
\pgfpathlineto{\pgfqpoint{3.015000in}{2.529146in}}%
\pgfpathlineto{\pgfqpoint{3.046000in}{2.831402in}}%
\pgfpathlineto{\pgfqpoint{3.077000in}{2.711002in}}%
\pgfpathlineto{\pgfqpoint{3.108000in}{2.731798in}}%
\pgfpathlineto{\pgfqpoint{3.139000in}{2.657520in}}%
\pgfpathlineto{\pgfqpoint{3.170000in}{2.411418in}}%
\pgfpathlineto{\pgfqpoint{3.201000in}{2.305435in}}%
\pgfpathlineto{\pgfqpoint{3.232000in}{2.050697in}}%
\pgfpathlineto{\pgfqpoint{3.263000in}{1.916413in}}%
\pgfpathlineto{\pgfqpoint{3.294000in}{1.769907in}}%
\pgfpathlineto{\pgfqpoint{3.325000in}{1.565500in}}%
\pgfpathlineto{\pgfqpoint{3.356000in}{1.679396in}}%
\pgfpathlineto{\pgfqpoint{3.387000in}{1.693108in}}%
\pgfpathlineto{\pgfqpoint{3.418000in}{1.479269in}}%
\pgfpathlineto{\pgfqpoint{3.449000in}{1.290630in}}%
\pgfpathlineto{\pgfqpoint{3.480000in}{1.247191in}}%
\pgfpathlineto{\pgfqpoint{3.511000in}{1.446150in}}%
\pgfpathlineto{\pgfqpoint{3.542000in}{1.230685in}}%
\pgfpathlineto{\pgfqpoint{3.573000in}{1.209951in}}%
\pgfpathlineto{\pgfqpoint{3.604000in}{1.328474in}}%
\pgfpathlineto{\pgfqpoint{3.635000in}{1.170942in}}%
\pgfpathlineto{\pgfqpoint{3.666000in}{1.130183in}}%
\pgfpathlineto{\pgfqpoint{3.697000in}{1.193694in}}%
\pgfpathlineto{\pgfqpoint{3.728000in}{1.124083in}}%
\pgfpathlineto{\pgfqpoint{3.759000in}{1.070664in}}%
\pgfpathlineto{\pgfqpoint{3.790000in}{1.318354in}}%
\pgfpathlineto{\pgfqpoint{3.821000in}{1.409860in}}%
\pgfpathlineto{\pgfqpoint{3.852000in}{1.855656in}}%
\pgfpathlineto{\pgfqpoint{3.883000in}{2.061834in}}%
\pgfpathlineto{\pgfqpoint{3.914000in}{2.494417in}}%
\pgfpathlineto{\pgfqpoint{3.945000in}{2.456392in}}%
\pgfpathlineto{\pgfqpoint{3.976000in}{2.728097in}}%
\pgfpathlineto{\pgfqpoint{4.007000in}{2.606911in}}%
\pgfpathlineto{\pgfqpoint{4.038000in}{2.276507in}}%
\pgfpathlineto{\pgfqpoint{4.069000in}{2.275355in}}%
\pgfpathlineto{\pgfqpoint{4.100000in}{2.157924in}}%
\pgfpathlineto{\pgfqpoint{4.131000in}{1.957633in}}%
\pgfpathlineto{\pgfqpoint{4.162000in}{1.790088in}}%
\pgfpathlineto{\pgfqpoint{4.193000in}{1.732214in}}%
\pgfpathlineto{\pgfqpoint{4.224000in}{1.774313in}}%
\pgfpathlineto{\pgfqpoint{4.255000in}{1.713516in}}%
\pgfpathlineto{\pgfqpoint{4.286000in}{1.427251in}}%
\pgfpathlineto{\pgfqpoint{4.317000in}{1.398207in}}%
\pgfpathlineto{\pgfqpoint{4.348000in}{1.345233in}}%
\pgfpathlineto{\pgfqpoint{4.379000in}{1.269241in}}%
\pgfpathlineto{\pgfqpoint{4.410000in}{1.163179in}}%
\pgfpathlineto{\pgfqpoint{4.441000in}{1.326299in}}%
\pgfpathlineto{\pgfqpoint{4.472000in}{1.227645in}}%
\pgfpathlineto{\pgfqpoint{4.534000in}{1.110235in}}%
\pgfpathlineto{\pgfqpoint{4.565000in}{1.047911in}}%
\pgfpathlineto{\pgfqpoint{4.596000in}{1.252330in}}%
\pgfpathlineto{\pgfqpoint{4.627000in}{1.265932in}}%
\pgfpathlineto{\pgfqpoint{4.658000in}{1.099076in}}%
\pgfpathlineto{\pgfqpoint{4.689000in}{1.214949in}}%
\pgfpathlineto{\pgfqpoint{4.720000in}{0.967753in}}%
\pgfpathlineto{\pgfqpoint{4.751000in}{1.102921in}}%
\pgfpathlineto{\pgfqpoint{4.782000in}{1.124635in}}%
\pgfpathlineto{\pgfqpoint{4.813000in}{1.008293in}}%
\pgfpathlineto{\pgfqpoint{4.844000in}{1.063162in}}%
\pgfpathlineto{\pgfqpoint{4.875000in}{0.850457in}}%
\pgfpathlineto{\pgfqpoint{4.906000in}{1.010315in}}%
\pgfpathlineto{\pgfqpoint{4.937000in}{1.034456in}}%
\pgfpathlineto{\pgfqpoint{4.968000in}{0.941328in}}%
\pgfpathlineto{\pgfqpoint{4.999000in}{1.270392in}}%
\pgfpathlineto{\pgfqpoint{5.030000in}{1.161578in}}%
\pgfpathlineto{\pgfqpoint{5.061000in}{1.103757in}}%
\pgfpathlineto{\pgfqpoint{5.092000in}{1.178863in}}%
\pgfpathlineto{\pgfqpoint{5.123000in}{0.973378in}}%
\pgfpathlineto{\pgfqpoint{5.154000in}{1.054616in}}%
\pgfpathlineto{\pgfqpoint{5.185000in}{1.007248in}}%
\pgfpathlineto{\pgfqpoint{5.216000in}{0.937494in}}%
\pgfpathlineto{\pgfqpoint{5.247000in}{1.100240in}}%
\pgfpathlineto{\pgfqpoint{5.278000in}{1.121377in}}%
\pgfpathlineto{\pgfqpoint{5.309000in}{1.190827in}}%
\pgfpathlineto{\pgfqpoint{5.371000in}{1.088908in}}%
\pgfpathlineto{\pgfqpoint{5.402000in}{1.158118in}}%
\pgfpathlineto{\pgfqpoint{5.433000in}{1.180311in}}%
\pgfpathlineto{\pgfqpoint{5.464000in}{1.052012in}}%
\pgfpathlineto{\pgfqpoint{5.495000in}{1.024712in}}%
\pgfpathlineto{\pgfqpoint{5.526000in}{1.053084in}}%
\pgfpathlineto{\pgfqpoint{5.557000in}{1.083044in}}%
\pgfpathlineto{\pgfqpoint{5.588000in}{1.017625in}}%
\pgfpathlineto{\pgfqpoint{5.619000in}{0.927064in}}%
\pgfpathlineto{\pgfqpoint{5.650000in}{1.060330in}}%
\pgfpathlineto{\pgfqpoint{5.681000in}{1.064137in}}%
\pgfpathlineto{\pgfqpoint{5.712000in}{1.086579in}}%
\pgfpathlineto{\pgfqpoint{5.743000in}{1.126917in}}%
\pgfpathlineto{\pgfqpoint{5.774000in}{1.241899in}}%
\pgfpathlineto{\pgfqpoint{5.805000in}{1.099378in}}%
\pgfpathlineto{\pgfqpoint{5.836000in}{1.040084in}}%
\pgfpathlineto{\pgfqpoint{5.867000in}{1.065792in}}%
\pgfpathlineto{\pgfqpoint{5.898000in}{1.214255in}}%
\pgfpathlineto{\pgfqpoint{5.929000in}{1.204720in}}%
\pgfpathlineto{\pgfqpoint{5.960000in}{1.058180in}}%
\pgfpathlineto{\pgfqpoint{5.991000in}{1.028396in}}%
\pgfpathlineto{\pgfqpoint{6.022000in}{1.080612in}}%
\pgfpathlineto{\pgfqpoint{6.053000in}{1.009189in}}%
\pgfpathlineto{\pgfqpoint{6.084000in}{1.031824in}}%
\pgfpathlineto{\pgfqpoint{6.115000in}{1.090823in}}%
\pgfpathlineto{\pgfqpoint{6.146000in}{1.124433in}}%
\pgfpathlineto{\pgfqpoint{6.177000in}{1.074926in}}%
\pgfpathlineto{\pgfqpoint{6.208000in}{1.162426in}}%
\pgfpathlineto{\pgfqpoint{6.239000in}{1.024909in}}%
\pgfpathlineto{\pgfqpoint{6.270000in}{1.187317in}}%
\pgfpathlineto{\pgfqpoint{6.301000in}{0.988633in}}%
\pgfpathlineto{\pgfqpoint{6.332000in}{1.046368in}}%
\pgfpathlineto{\pgfqpoint{6.363000in}{1.064396in}}%
\pgfpathlineto{\pgfqpoint{6.394000in}{1.058598in}}%
\pgfpathlineto{\pgfqpoint{6.425000in}{1.248199in}}%
\pgfpathlineto{\pgfqpoint{6.456000in}{1.123116in}}%
\pgfpathlineto{\pgfqpoint{6.487000in}{1.200168in}}%
\pgfpathlineto{\pgfqpoint{6.518000in}{1.155370in}}%
\pgfpathlineto{\pgfqpoint{6.549000in}{1.162498in}}%
\pgfpathlineto{\pgfqpoint{6.580000in}{1.118317in}}%
\pgfpathlineto{\pgfqpoint{6.611000in}{1.032227in}}%
\pgfpathlineto{\pgfqpoint{6.642000in}{1.165316in}}%
\pgfpathlineto{\pgfqpoint{6.673000in}{1.075310in}}%
\pgfpathlineto{\pgfqpoint{6.704000in}{0.957843in}}%
\pgfpathlineto{\pgfqpoint{6.735000in}{1.106293in}}%
\pgfpathlineto{\pgfqpoint{6.766000in}{1.192996in}}%
\pgfpathlineto{\pgfqpoint{6.797000in}{1.165199in}}%
\pgfpathlineto{\pgfqpoint{6.828000in}{1.156176in}}%
\pgfpathlineto{\pgfqpoint{6.859000in}{0.948947in}}%
\pgfpathlineto{\pgfqpoint{6.890000in}{1.118939in}}%
\pgfpathlineto{\pgfqpoint{6.921000in}{0.894193in}}%
\pgfpathlineto{\pgfqpoint{6.952000in}{1.118916in}}%
\pgfpathlineto{\pgfqpoint{6.983000in}{0.998187in}}%
\pgfpathlineto{\pgfqpoint{7.014000in}{1.099235in}}%
\pgfpathlineto{\pgfqpoint{7.045000in}{1.124969in}}%
\pgfpathlineto{\pgfqpoint{7.076000in}{1.042502in}}%
\pgfpathlineto{\pgfqpoint{7.107000in}{1.031290in}}%
\pgfpathlineto{\pgfqpoint{7.138000in}{1.167985in}}%
\pgfpathlineto{\pgfqpoint{7.169000in}{1.272531in}}%
\pgfpathlineto{\pgfqpoint{7.200000in}{1.166951in}}%
\pgfpathlineto{\pgfqpoint{7.210000in}{1.105005in}}%
\pgfpathlineto{\pgfqpoint{7.210000in}{1.105005in}}%
\pgfusepath{stroke}%
\end{pgfscope}%
\begin{pgfscope}%
\pgfsetrectcap%
\pgfsetmiterjoin%
\pgfsetlinewidth{0.803000pt}%
\definecolor{currentstroke}{rgb}{0.000000,0.000000,0.000000}%
\pgfsetstrokecolor{currentstroke}%
\pgfsetdash{}{0pt}%
\pgfpathmoveto{\pgfqpoint{1.000000in}{0.720000in}}%
\pgfpathlineto{\pgfqpoint{1.000000in}{5.340000in}}%
\pgfusepath{stroke}%
\end{pgfscope}%
\begin{pgfscope}%
\pgfsetrectcap%
\pgfsetmiterjoin%
\pgfsetlinewidth{0.803000pt}%
\definecolor{currentstroke}{rgb}{0.000000,0.000000,0.000000}%
\pgfsetstrokecolor{currentstroke}%
\pgfsetdash{}{0pt}%
\pgfpathmoveto{\pgfqpoint{7.200000in}{0.720000in}}%
\pgfpathlineto{\pgfqpoint{7.200000in}{5.340000in}}%
\pgfusepath{stroke}%
\end{pgfscope}%
\begin{pgfscope}%
\pgfsetrectcap%
\pgfsetmiterjoin%
\pgfsetlinewidth{0.803000pt}%
\definecolor{currentstroke}{rgb}{0.000000,0.000000,0.000000}%
\pgfsetstrokecolor{currentstroke}%
\pgfsetdash{}{0pt}%
\pgfpathmoveto{\pgfqpoint{1.000000in}{0.720000in}}%
\pgfpathlineto{\pgfqpoint{7.200000in}{0.720000in}}%
\pgfusepath{stroke}%
\end{pgfscope}%
\begin{pgfscope}%
\pgfsetrectcap%
\pgfsetmiterjoin%
\pgfsetlinewidth{0.803000pt}%
\definecolor{currentstroke}{rgb}{0.000000,0.000000,0.000000}%
\pgfsetstrokecolor{currentstroke}%
\pgfsetdash{}{0pt}%
\pgfpathmoveto{\pgfqpoint{1.000000in}{5.340000in}}%
\pgfpathlineto{\pgfqpoint{7.200000in}{5.340000in}}%
\pgfusepath{stroke}%
\end{pgfscope}%
\begin{pgfscope}%
\pgfsetbuttcap%
\pgfsetroundjoin%
\definecolor{currentfill}{rgb}{0.000000,0.000000,0.000000}%
\pgfsetfillcolor{currentfill}%
\pgfsetlinewidth{0.803000pt}%
\definecolor{currentstroke}{rgb}{0.000000,0.000000,0.000000}%
\pgfsetstrokecolor{currentstroke}%
\pgfsetdash{}{0pt}%
\pgfsys@defobject{currentmarker}{\pgfqpoint{0.000000in}{0.000000in}}{\pgfqpoint{0.048611in}{0.000000in}}{%
\pgfpathmoveto{\pgfqpoint{0.000000in}{0.000000in}}%
\pgfpathlineto{\pgfqpoint{0.048611in}{0.000000in}}%
\pgfusepath{stroke,fill}%
}%
\begin{pgfscope}%
\pgfsys@transformshift{7.200000in}{1.109002in}%
\pgfsys@useobject{currentmarker}{}%
\end{pgfscope}%
\end{pgfscope}%
\begin{pgfscope}%
\definecolor{textcolor}{rgb}{0.000000,0.000000,0.000000}%
\pgfsetstrokecolor{textcolor}%
\pgfsetfillcolor{textcolor}%
\pgftext[x=7.297222in, y=1.008983in, left, base]{\color{textcolor}\sffamily\fontsize{20.000000}{24.000000}\selectfont 0.0}%
\end{pgfscope}%
\begin{pgfscope}%
\pgfsetbuttcap%
\pgfsetroundjoin%
\definecolor{currentfill}{rgb}{0.000000,0.000000,0.000000}%
\pgfsetfillcolor{currentfill}%
\pgfsetlinewidth{0.803000pt}%
\definecolor{currentstroke}{rgb}{0.000000,0.000000,0.000000}%
\pgfsetstrokecolor{currentstroke}%
\pgfsetdash{}{0pt}%
\pgfsys@defobject{currentmarker}{\pgfqpoint{0.000000in}{0.000000in}}{\pgfqpoint{0.048611in}{0.000000in}}{%
\pgfpathmoveto{\pgfqpoint{0.000000in}{0.000000in}}%
\pgfpathlineto{\pgfqpoint{0.048611in}{0.000000in}}%
\pgfusepath{stroke,fill}%
}%
\begin{pgfscope}%
\pgfsys@transformshift{7.200000in}{1.731208in}%
\pgfsys@useobject{currentmarker}{}%
\end{pgfscope}%
\end{pgfscope}%
\begin{pgfscope}%
\definecolor{textcolor}{rgb}{0.000000,0.000000,0.000000}%
\pgfsetstrokecolor{textcolor}%
\pgfsetfillcolor{textcolor}%
\pgftext[x=7.297222in, y=1.631188in, left, base]{\color{textcolor}\sffamily\fontsize{20.000000}{24.000000}\selectfont 0.2}%
\end{pgfscope}%
\begin{pgfscope}%
\pgfsetbuttcap%
\pgfsetroundjoin%
\definecolor{currentfill}{rgb}{0.000000,0.000000,0.000000}%
\pgfsetfillcolor{currentfill}%
\pgfsetlinewidth{0.803000pt}%
\definecolor{currentstroke}{rgb}{0.000000,0.000000,0.000000}%
\pgfsetstrokecolor{currentstroke}%
\pgfsetdash{}{0pt}%
\pgfsys@defobject{currentmarker}{\pgfqpoint{0.000000in}{0.000000in}}{\pgfqpoint{0.048611in}{0.000000in}}{%
\pgfpathmoveto{\pgfqpoint{0.000000in}{0.000000in}}%
\pgfpathlineto{\pgfqpoint{0.048611in}{0.000000in}}%
\pgfusepath{stroke,fill}%
}%
\begin{pgfscope}%
\pgfsys@transformshift{7.200000in}{2.353413in}%
\pgfsys@useobject{currentmarker}{}%
\end{pgfscope}%
\end{pgfscope}%
\begin{pgfscope}%
\definecolor{textcolor}{rgb}{0.000000,0.000000,0.000000}%
\pgfsetstrokecolor{textcolor}%
\pgfsetfillcolor{textcolor}%
\pgftext[x=7.297222in, y=2.253394in, left, base]{\color{textcolor}\sffamily\fontsize{20.000000}{24.000000}\selectfont 0.5}%
\end{pgfscope}%
\begin{pgfscope}%
\pgfsetbuttcap%
\pgfsetroundjoin%
\definecolor{currentfill}{rgb}{0.000000,0.000000,0.000000}%
\pgfsetfillcolor{currentfill}%
\pgfsetlinewidth{0.803000pt}%
\definecolor{currentstroke}{rgb}{0.000000,0.000000,0.000000}%
\pgfsetstrokecolor{currentstroke}%
\pgfsetdash{}{0pt}%
\pgfsys@defobject{currentmarker}{\pgfqpoint{0.000000in}{0.000000in}}{\pgfqpoint{0.048611in}{0.000000in}}{%
\pgfpathmoveto{\pgfqpoint{0.000000in}{0.000000in}}%
\pgfpathlineto{\pgfqpoint{0.048611in}{0.000000in}}%
\pgfusepath{stroke,fill}%
}%
\begin{pgfscope}%
\pgfsys@transformshift{7.200000in}{2.975619in}%
\pgfsys@useobject{currentmarker}{}%
\end{pgfscope}%
\end{pgfscope}%
\begin{pgfscope}%
\definecolor{textcolor}{rgb}{0.000000,0.000000,0.000000}%
\pgfsetstrokecolor{textcolor}%
\pgfsetfillcolor{textcolor}%
\pgftext[x=7.297222in, y=2.875600in, left, base]{\color{textcolor}\sffamily\fontsize{20.000000}{24.000000}\selectfont 0.8}%
\end{pgfscope}%
\begin{pgfscope}%
\pgfsetbuttcap%
\pgfsetroundjoin%
\definecolor{currentfill}{rgb}{0.000000,0.000000,0.000000}%
\pgfsetfillcolor{currentfill}%
\pgfsetlinewidth{0.803000pt}%
\definecolor{currentstroke}{rgb}{0.000000,0.000000,0.000000}%
\pgfsetstrokecolor{currentstroke}%
\pgfsetdash{}{0pt}%
\pgfsys@defobject{currentmarker}{\pgfqpoint{0.000000in}{0.000000in}}{\pgfqpoint{0.048611in}{0.000000in}}{%
\pgfpathmoveto{\pgfqpoint{0.000000in}{0.000000in}}%
\pgfpathlineto{\pgfqpoint{0.048611in}{0.000000in}}%
\pgfusepath{stroke,fill}%
}%
\begin{pgfscope}%
\pgfsys@transformshift{7.200000in}{3.597824in}%
\pgfsys@useobject{currentmarker}{}%
\end{pgfscope}%
\end{pgfscope}%
\begin{pgfscope}%
\definecolor{textcolor}{rgb}{0.000000,0.000000,0.000000}%
\pgfsetstrokecolor{textcolor}%
\pgfsetfillcolor{textcolor}%
\pgftext[x=7.297222in, y=3.497805in, left, base]{\color{textcolor}\sffamily\fontsize{20.000000}{24.000000}\selectfont 1.0}%
\end{pgfscope}%
\begin{pgfscope}%
\pgfsetbuttcap%
\pgfsetroundjoin%
\definecolor{currentfill}{rgb}{0.000000,0.000000,0.000000}%
\pgfsetfillcolor{currentfill}%
\pgfsetlinewidth{0.803000pt}%
\definecolor{currentstroke}{rgb}{0.000000,0.000000,0.000000}%
\pgfsetstrokecolor{currentstroke}%
\pgfsetdash{}{0pt}%
\pgfsys@defobject{currentmarker}{\pgfqpoint{0.000000in}{0.000000in}}{\pgfqpoint{0.048611in}{0.000000in}}{%
\pgfpathmoveto{\pgfqpoint{0.000000in}{0.000000in}}%
\pgfpathlineto{\pgfqpoint{0.048611in}{0.000000in}}%
\pgfusepath{stroke,fill}%
}%
\begin{pgfscope}%
\pgfsys@transformshift{7.200000in}{4.220030in}%
\pgfsys@useobject{currentmarker}{}%
\end{pgfscope}%
\end{pgfscope}%
\begin{pgfscope}%
\definecolor{textcolor}{rgb}{0.000000,0.000000,0.000000}%
\pgfsetstrokecolor{textcolor}%
\pgfsetfillcolor{textcolor}%
\pgftext[x=7.297222in, y=4.120011in, left, base]{\color{textcolor}\sffamily\fontsize{20.000000}{24.000000}\selectfont 1.2}%
\end{pgfscope}%
\begin{pgfscope}%
\pgfsetbuttcap%
\pgfsetroundjoin%
\definecolor{currentfill}{rgb}{0.000000,0.000000,0.000000}%
\pgfsetfillcolor{currentfill}%
\pgfsetlinewidth{0.803000pt}%
\definecolor{currentstroke}{rgb}{0.000000,0.000000,0.000000}%
\pgfsetstrokecolor{currentstroke}%
\pgfsetdash{}{0pt}%
\pgfsys@defobject{currentmarker}{\pgfqpoint{0.000000in}{0.000000in}}{\pgfqpoint{0.048611in}{0.000000in}}{%
\pgfpathmoveto{\pgfqpoint{0.000000in}{0.000000in}}%
\pgfpathlineto{\pgfqpoint{0.048611in}{0.000000in}}%
\pgfusepath{stroke,fill}%
}%
\begin{pgfscope}%
\pgfsys@transformshift{7.200000in}{4.842236in}%
\pgfsys@useobject{currentmarker}{}%
\end{pgfscope}%
\end{pgfscope}%
\begin{pgfscope}%
\definecolor{textcolor}{rgb}{0.000000,0.000000,0.000000}%
\pgfsetstrokecolor{textcolor}%
\pgfsetfillcolor{textcolor}%
\pgftext[x=7.297222in, y=4.742216in, left, base]{\color{textcolor}\sffamily\fontsize{20.000000}{24.000000}\selectfont 1.5}%
\end{pgfscope}%
\begin{pgfscope}%
\definecolor{textcolor}{rgb}{0.000000,0.000000,0.000000}%
\pgfsetstrokecolor{textcolor}%
\pgfsetfillcolor{textcolor}%
\pgftext[x=7.698906in,y=3.030000in,,top,rotate=90.000000]{\color{textcolor}\sffamily\fontsize{20.000000}{24.000000}\selectfont \(\displaystyle \mathrm{Charge}\)}%
\end{pgfscope}%
\begin{pgfscope}%
\pgfpathrectangle{\pgfqpoint{1.000000in}{0.720000in}}{\pgfqpoint{6.200000in}{4.620000in}}%
\pgfusepath{clip}%
\pgfsetbuttcap%
\pgfsetroundjoin%
\pgfsetlinewidth{0.501875pt}%
\definecolor{currentstroke}{rgb}{1.000000,0.000000,0.000000}%
\pgfsetstrokecolor{currentstroke}%
\pgfsetdash{}{0pt}%
\pgfpathmoveto{\pgfqpoint{1.930000in}{1.109002in}}%
\pgfpathlineto{\pgfqpoint{1.930000in}{3.270061in}}%
\pgfusepath{stroke}%
\end{pgfscope}%
\begin{pgfscope}%
\pgfpathrectangle{\pgfqpoint{1.000000in}{0.720000in}}{\pgfqpoint{6.200000in}{4.620000in}}%
\pgfusepath{clip}%
\pgfsetbuttcap%
\pgfsetroundjoin%
\pgfsetlinewidth{0.501875pt}%
\definecolor{currentstroke}{rgb}{1.000000,0.000000,0.000000}%
\pgfsetstrokecolor{currentstroke}%
\pgfsetdash{}{0pt}%
\pgfpathmoveto{\pgfqpoint{1.961000in}{1.109002in}}%
\pgfpathlineto{\pgfqpoint{1.961000in}{1.752433in}}%
\pgfusepath{stroke}%
\end{pgfscope}%
\begin{pgfscope}%
\pgfpathrectangle{\pgfqpoint{1.000000in}{0.720000in}}{\pgfqpoint{6.200000in}{4.620000in}}%
\pgfusepath{clip}%
\pgfsetbuttcap%
\pgfsetroundjoin%
\pgfsetlinewidth{0.501875pt}%
\definecolor{currentstroke}{rgb}{1.000000,0.000000,0.000000}%
\pgfsetstrokecolor{currentstroke}%
\pgfsetdash{}{0pt}%
\pgfpathmoveto{\pgfqpoint{1.992000in}{1.109002in}}%
\pgfpathlineto{\pgfqpoint{1.992000in}{2.981556in}}%
\pgfusepath{stroke}%
\end{pgfscope}%
\begin{pgfscope}%
\pgfpathrectangle{\pgfqpoint{1.000000in}{0.720000in}}{\pgfqpoint{6.200000in}{4.620000in}}%
\pgfusepath{clip}%
\pgfsetbuttcap%
\pgfsetroundjoin%
\pgfsetlinewidth{0.501875pt}%
\definecolor{currentstroke}{rgb}{1.000000,0.000000,0.000000}%
\pgfsetstrokecolor{currentstroke}%
\pgfsetdash{}{0pt}%
\pgfpathmoveto{\pgfqpoint{2.457000in}{1.109002in}}%
\pgfpathlineto{\pgfqpoint{2.457000in}{2.945749in}}%
\pgfusepath{stroke}%
\end{pgfscope}%
\begin{pgfscope}%
\pgfpathrectangle{\pgfqpoint{1.000000in}{0.720000in}}{\pgfqpoint{6.200000in}{4.620000in}}%
\pgfusepath{clip}%
\pgfsetbuttcap%
\pgfsetroundjoin%
\pgfsetlinewidth{0.501875pt}%
\definecolor{currentstroke}{rgb}{1.000000,0.000000,0.000000}%
\pgfsetstrokecolor{currentstroke}%
\pgfsetdash{}{0pt}%
\pgfpathmoveto{\pgfqpoint{2.488000in}{1.109002in}}%
\pgfpathlineto{\pgfqpoint{2.488000in}{2.713041in}}%
\pgfusepath{stroke}%
\end{pgfscope}%
\begin{pgfscope}%
\pgfpathrectangle{\pgfqpoint{1.000000in}{0.720000in}}{\pgfqpoint{6.200000in}{4.620000in}}%
\pgfusepath{clip}%
\pgfsetbuttcap%
\pgfsetroundjoin%
\pgfsetlinewidth{0.501875pt}%
\definecolor{currentstroke}{rgb}{1.000000,0.000000,0.000000}%
\pgfsetstrokecolor{currentstroke}%
\pgfsetdash{}{0pt}%
\pgfpathmoveto{\pgfqpoint{2.829000in}{1.109002in}}%
\pgfpathlineto{\pgfqpoint{2.829000in}{2.333591in}}%
\pgfusepath{stroke}%
\end{pgfscope}%
\begin{pgfscope}%
\pgfpathrectangle{\pgfqpoint{1.000000in}{0.720000in}}{\pgfqpoint{6.200000in}{4.620000in}}%
\pgfusepath{clip}%
\pgfsetbuttcap%
\pgfsetroundjoin%
\pgfsetlinewidth{0.501875pt}%
\definecolor{currentstroke}{rgb}{1.000000,0.000000,0.000000}%
\pgfsetstrokecolor{currentstroke}%
\pgfsetdash{}{0pt}%
\pgfpathmoveto{\pgfqpoint{2.860000in}{1.109002in}}%
\pgfpathlineto{\pgfqpoint{2.860000in}{2.364012in}}%
\pgfusepath{stroke}%
\end{pgfscope}%
\begin{pgfscope}%
\pgfpathrectangle{\pgfqpoint{1.000000in}{0.720000in}}{\pgfqpoint{6.200000in}{4.620000in}}%
\pgfusepath{clip}%
\pgfsetbuttcap%
\pgfsetroundjoin%
\pgfsetlinewidth{0.501875pt}%
\definecolor{currentstroke}{rgb}{1.000000,0.000000,0.000000}%
\pgfsetstrokecolor{currentstroke}%
\pgfsetdash{}{0pt}%
\pgfpathmoveto{\pgfqpoint{3.728000in}{1.109002in}}%
\pgfpathlineto{\pgfqpoint{3.728000in}{3.984189in}}%
\pgfusepath{stroke}%
\end{pgfscope}%
\begin{pgfscope}%
\pgfsetrectcap%
\pgfsetmiterjoin%
\pgfsetlinewidth{0.803000pt}%
\definecolor{currentstroke}{rgb}{0.000000,0.000000,0.000000}%
\pgfsetstrokecolor{currentstroke}%
\pgfsetdash{}{0pt}%
\pgfpathmoveto{\pgfqpoint{1.000000in}{0.720000in}}%
\pgfpathlineto{\pgfqpoint{1.000000in}{5.340000in}}%
\pgfusepath{stroke}%
\end{pgfscope}%
\begin{pgfscope}%
\pgfsetrectcap%
\pgfsetmiterjoin%
\pgfsetlinewidth{0.803000pt}%
\definecolor{currentstroke}{rgb}{0.000000,0.000000,0.000000}%
\pgfsetstrokecolor{currentstroke}%
\pgfsetdash{}{0pt}%
\pgfpathmoveto{\pgfqpoint{7.200000in}{0.720000in}}%
\pgfpathlineto{\pgfqpoint{7.200000in}{5.340000in}}%
\pgfusepath{stroke}%
\end{pgfscope}%
\begin{pgfscope}%
\pgfsetrectcap%
\pgfsetmiterjoin%
\pgfsetlinewidth{0.803000pt}%
\definecolor{currentstroke}{rgb}{0.000000,0.000000,0.000000}%
\pgfsetstrokecolor{currentstroke}%
\pgfsetdash{}{0pt}%
\pgfpathmoveto{\pgfqpoint{1.000000in}{0.720000in}}%
\pgfpathlineto{\pgfqpoint{7.200000in}{0.720000in}}%
\pgfusepath{stroke}%
\end{pgfscope}%
\begin{pgfscope}%
\pgfsetrectcap%
\pgfsetmiterjoin%
\pgfsetlinewidth{0.803000pt}%
\definecolor{currentstroke}{rgb}{0.000000,0.000000,0.000000}%
\pgfsetstrokecolor{currentstroke}%
\pgfsetdash{}{0pt}%
\pgfpathmoveto{\pgfqpoint{1.000000in}{5.340000in}}%
\pgfpathlineto{\pgfqpoint{7.200000in}{5.340000in}}%
\pgfusepath{stroke}%
\end{pgfscope}%
\begin{pgfscope}%
\pgfsetbuttcap%
\pgfsetmiterjoin%
\definecolor{currentfill}{rgb}{1.000000,1.000000,1.000000}%
\pgfsetfillcolor{currentfill}%
\pgfsetfillopacity{0.800000}%
\pgfsetlinewidth{1.003750pt}%
\definecolor{currentstroke}{rgb}{0.800000,0.800000,0.800000}%
\pgfsetstrokecolor{currentstroke}%
\pgfsetstrokeopacity{0.800000}%
\pgfsetdash{}{0pt}%
\pgfpathmoveto{\pgfqpoint{4.976872in}{4.327865in}}%
\pgfpathlineto{\pgfqpoint{7.005556in}{4.327865in}}%
\pgfpathquadraticcurveto{\pgfqpoint{7.061111in}{4.327865in}}{\pgfqpoint{7.061111in}{4.383420in}}%
\pgfpathlineto{\pgfqpoint{7.061111in}{5.145556in}}%
\pgfpathquadraticcurveto{\pgfqpoint{7.061111in}{5.201111in}}{\pgfqpoint{7.005556in}{5.201111in}}%
\pgfpathlineto{\pgfqpoint{4.976872in}{5.201111in}}%
\pgfpathquadraticcurveto{\pgfqpoint{4.921317in}{5.201111in}}{\pgfqpoint{4.921317in}{5.145556in}}%
\pgfpathlineto{\pgfqpoint{4.921317in}{4.383420in}}%
\pgfpathquadraticcurveto{\pgfqpoint{4.921317in}{4.327865in}}{\pgfqpoint{4.976872in}{4.327865in}}%
\pgfpathlineto{\pgfqpoint{4.976872in}{4.327865in}}%
\pgfpathclose%
\pgfusepath{stroke,fill}%
\end{pgfscope}%
\begin{pgfscope}%
\pgfsetrectcap%
\pgfsetroundjoin%
\pgfsetlinewidth{2.007500pt}%
\definecolor{currentstroke}{rgb}{0.121569,0.466667,0.705882}%
\pgfsetstrokecolor{currentstroke}%
\pgfsetdash{}{0pt}%
\pgfpathmoveto{\pgfqpoint{5.032428in}{4.987184in}}%
\pgfpathlineto{\pgfqpoint{5.310206in}{4.987184in}}%
\pgfpathlineto{\pgfqpoint{5.587983in}{4.987184in}}%
\pgfusepath{stroke}%
\end{pgfscope}%
\begin{pgfscope}%
\definecolor{textcolor}{rgb}{0.000000,0.000000,0.000000}%
\pgfsetstrokecolor{textcolor}%
\pgfsetfillcolor{textcolor}%
\pgftext[x=5.810206in,y=4.889962in,left,base]{\color{textcolor}\sffamily\fontsize{20.000000}{24.000000}\selectfont Waveform}%
\end{pgfscope}%
\begin{pgfscope}%
\pgfsetbuttcap%
\pgfsetroundjoin%
\pgfsetlinewidth{0.501875pt}%
\definecolor{currentstroke}{rgb}{1.000000,0.000000,0.000000}%
\pgfsetstrokecolor{currentstroke}%
\pgfsetdash{}{0pt}%
\pgfpathmoveto{\pgfqpoint{5.032428in}{4.592227in}}%
\pgfpathlineto{\pgfqpoint{5.587983in}{4.592227in}}%
\pgfusepath{stroke}%
\end{pgfscope}%
\begin{pgfscope}%
\definecolor{textcolor}{rgb}{0.000000,0.000000,0.000000}%
\pgfsetstrokecolor{textcolor}%
\pgfsetfillcolor{textcolor}%
\pgftext[x=5.810206in,y=4.495005in,left,base]{\color{textcolor}\sffamily\fontsize{20.000000}{24.000000}\selectfont Charge}%
\end{pgfscope}%
\end{pgfpicture}%
\makeatother%
\endgroup%
}
    \caption{\label{fig:cnn}An example giving \\ $\hat{t}_\mathrm{KL} - t_0=\SI{2.96}{ns}$, $\mathrm{RSS}=\SI{19.5}{mV^2}$, $D_\mathrm{w}=\SI{0.81}{ns}$.}
  \end{subfigure}
  \caption{\label{fig:cnn-performance}Demonstration of CNN on $\num[retain-unity-mantissa=false]{1e4}$ waveforms in~\subref{fig:cnn-npe} and one waveform in~\subref{fig:cnn} sampled from the same setup as figure~\ref{fig:method}. In figure~\subref{fig:cnn-npe}, the middle line is the mean of the distribution. The size of errorbar is from \SIrange{15.8}{84.1}{\percent} quantiles, corresponding to $\SI{\pm 1}{\sigma}$ of a Gaussian distribution. }
\end{figure}

\subsection{Regression analysis}
\label{sec:regression}
With the generative waveform model in eq.~\eqref{eq:1}, regression is ideal for analysis. Although computational complexity hinders the applications of regression by the vast volumes of raw data, the advancement of sparse models and big data infrastructures strengthens the advantage of regression.

The truth $N_\mathrm{PE}$ is unknown and formulating an explicit trans-dimensional model is expansive.  So, in the first two methods, we use the grid representation of PE sequence $q'_j, j\in \{1\cdots N_\mathrm{s}\}$ introduced in \ref{sec:cnn} in order to avoid cross-dimensional issue.  We shall solve the issue and turn back to length-varying representation in section~\ref{subsec:fsmp}.

Regression methods adjust $\{q'_j\}$ to fit eq.~\eqref{eq:gd}:
\begin{equation}
  \label{eq:gd}
  w'(t) = \sum_{j=1}^{N_\mathrm{s}}q'_jV_\mathrm{PE}(t-t'_j).
\end{equation}

From the output $\hat{\phi}_\mathrm{dec}(t)$ of a deconvolution method in section~\ref{sec:lucyddm}, we confidently leave out all the $t'_j$ that $\hat{\phi}_\mathrm{dec}(t_j')=0$ in eq.~\eqref{eq:gd-phi} to reduce the number of parameters and the complexity.

\subsubsection{Direct charge fitting}
\label{sec:dcf}

Fitting the charges $q'_j$ in eq.~\eqref{eq:gd} directly by minimizing RSS of $w'(t)$ and $w(t)$, we get
\begin{equation}
  \label{eq:gd-q}
  \bm{\hat{q}} = \arg \underset{q'_j \ge 0}{\min}~\mathrm{RSS}\left[w'(t),w(t)\right].
\end{equation}
RSS of eq.~\eqref{eq:gd-q} does not suffer at the sparse configuration in figure~\ref{fig:l2} provided that the dense grid in eq.~\eqref{eq:gd} covers all the PEs.

Slawski and Hein~\cite{slawski_non-negative_2013} proved that in deconvolution problems, the non-negative least-squares formulation in eq.~\eqref{eq:gd-q} is self-regularized and gives sparse solutions of $q'_i$.  Peterson~\cite{peterson_developments_2021} from IceCube used this technique for waveform analysis.  We optimize eq.~\eqref{eq:gd-q} by Broyden-Fletcher-\allowbreak{}Goldfarb-Shanno algorithm with a bound constraint~\cite{byrd_limited_1995}.  In figure~\ref{fig:fitting-npe}, charge fitting is consistent in $D_\mathrm{w}$ for different $N_\mathrm{PE}$'s, showing its resilience to pile-up.

\begin{figure}[H]
  \begin{subfigure}{.5\textwidth}
    \centering
    \resizebox{\textwidth}{!}{%% Creator: Matplotlib, PGF backend
%%
%% To include the figure in your LaTeX document, write
%%   \input{<filename>.pgf}
%%
%% Make sure the required packages are loaded in your preamble
%%   \usepackage{pgf}
%%
%% and, on pdftex
%%   \usepackage[utf8]{inputenc}\DeclareUnicodeCharacter{2212}{-}
%%
%% or, on luatex and xetex
%%   \usepackage{unicode-math}
%%
%% Figures using additional raster images can only be included by \input if
%% they are in the same directory as the main LaTeX file. For loading figures
%% from other directories you can use the `import` package
%%   \usepackage{import}
%%
%% and then include the figures with
%%   \import{<path to file>}{<filename>.pgf}
%%
%% Matplotlib used the following preamble
%%   \usepackage[detect-all,locale=DE]{siunitx}
%%
\begingroup%
\makeatletter%
\begin{pgfpicture}%
\pgfpathrectangle{\pgfpointorigin}{\pgfqpoint{8.000000in}{6.000000in}}%
\pgfusepath{use as bounding box, clip}%
\begin{pgfscope}%
\pgfsetbuttcap%
\pgfsetmiterjoin%
\definecolor{currentfill}{rgb}{1.000000,1.000000,1.000000}%
\pgfsetfillcolor{currentfill}%
\pgfsetlinewidth{0.000000pt}%
\definecolor{currentstroke}{rgb}{1.000000,1.000000,1.000000}%
\pgfsetstrokecolor{currentstroke}%
\pgfsetdash{}{0pt}%
\pgfpathmoveto{\pgfqpoint{0.000000in}{0.000000in}}%
\pgfpathlineto{\pgfqpoint{8.000000in}{0.000000in}}%
\pgfpathlineto{\pgfqpoint{8.000000in}{6.000000in}}%
\pgfpathlineto{\pgfqpoint{0.000000in}{6.000000in}}%
\pgfpathclose%
\pgfusepath{fill}%
\end{pgfscope}%
\begin{pgfscope}%
\pgfsetbuttcap%
\pgfsetmiterjoin%
\definecolor{currentfill}{rgb}{1.000000,1.000000,1.000000}%
\pgfsetfillcolor{currentfill}%
\pgfsetlinewidth{0.000000pt}%
\definecolor{currentstroke}{rgb}{0.000000,0.000000,0.000000}%
\pgfsetstrokecolor{currentstroke}%
\pgfsetstrokeopacity{0.000000}%
\pgfsetdash{}{0pt}%
\pgfpathmoveto{\pgfqpoint{0.800000in}{0.720000in}}%
\pgfpathlineto{\pgfqpoint{5.200000in}{0.720000in}}%
\pgfpathlineto{\pgfqpoint{5.200000in}{5.520000in}}%
\pgfpathlineto{\pgfqpoint{0.800000in}{5.520000in}}%
\pgfpathclose%
\pgfusepath{fill}%
\end{pgfscope}%
\begin{pgfscope}%
\pgfsetbuttcap%
\pgfsetroundjoin%
\definecolor{currentfill}{rgb}{0.000000,0.000000,0.000000}%
\pgfsetfillcolor{currentfill}%
\pgfsetlinewidth{0.803000pt}%
\definecolor{currentstroke}{rgb}{0.000000,0.000000,0.000000}%
\pgfsetstrokecolor{currentstroke}%
\pgfsetdash{}{0pt}%
\pgfsys@defobject{currentmarker}{\pgfqpoint{0.000000in}{-0.048611in}}{\pgfqpoint{0.000000in}{0.000000in}}{%
\pgfpathmoveto{\pgfqpoint{0.000000in}{0.000000in}}%
\pgfpathlineto{\pgfqpoint{0.000000in}{-0.048611in}}%
\pgfusepath{stroke,fill}%
}%
\begin{pgfscope}%
\pgfsys@transformshift{1.075000in}{0.720000in}%
\pgfsys@useobject{currentmarker}{}%
\end{pgfscope}%
\end{pgfscope}%
\begin{pgfscope}%
\definecolor{textcolor}{rgb}{0.000000,0.000000,0.000000}%
\pgfsetstrokecolor{textcolor}%
\pgfsetfillcolor{textcolor}%
\pgftext[x=1.075000in,y=0.622778in,,top]{\color{textcolor}\sffamily\fontsize{20.000000}{24.000000}\selectfont 1}%
\end{pgfscope}%
\begin{pgfscope}%
\pgfsetbuttcap%
\pgfsetroundjoin%
\definecolor{currentfill}{rgb}{0.000000,0.000000,0.000000}%
\pgfsetfillcolor{currentfill}%
\pgfsetlinewidth{0.803000pt}%
\definecolor{currentstroke}{rgb}{0.000000,0.000000,0.000000}%
\pgfsetstrokecolor{currentstroke}%
\pgfsetdash{}{0pt}%
\pgfsys@defobject{currentmarker}{\pgfqpoint{0.000000in}{-0.048611in}}{\pgfqpoint{0.000000in}{0.000000in}}{%
\pgfpathmoveto{\pgfqpoint{0.000000in}{0.000000in}}%
\pgfpathlineto{\pgfqpoint{0.000000in}{-0.048611in}}%
\pgfusepath{stroke,fill}%
}%
\begin{pgfscope}%
\pgfsys@transformshift{1.350000in}{0.720000in}%
\pgfsys@useobject{currentmarker}{}%
\end{pgfscope}%
\end{pgfscope}%
\begin{pgfscope}%
\definecolor{textcolor}{rgb}{0.000000,0.000000,0.000000}%
\pgfsetstrokecolor{textcolor}%
\pgfsetfillcolor{textcolor}%
\pgftext[x=1.350000in,y=0.622778in,,top]{\color{textcolor}\sffamily\fontsize{20.000000}{24.000000}\selectfont 2}%
\end{pgfscope}%
\begin{pgfscope}%
\pgfsetbuttcap%
\pgfsetroundjoin%
\definecolor{currentfill}{rgb}{0.000000,0.000000,0.000000}%
\pgfsetfillcolor{currentfill}%
\pgfsetlinewidth{0.803000pt}%
\definecolor{currentstroke}{rgb}{0.000000,0.000000,0.000000}%
\pgfsetstrokecolor{currentstroke}%
\pgfsetdash{}{0pt}%
\pgfsys@defobject{currentmarker}{\pgfqpoint{0.000000in}{-0.048611in}}{\pgfqpoint{0.000000in}{0.000000in}}{%
\pgfpathmoveto{\pgfqpoint{0.000000in}{0.000000in}}%
\pgfpathlineto{\pgfqpoint{0.000000in}{-0.048611in}}%
\pgfusepath{stroke,fill}%
}%
\begin{pgfscope}%
\pgfsys@transformshift{1.625000in}{0.720000in}%
\pgfsys@useobject{currentmarker}{}%
\end{pgfscope}%
\end{pgfscope}%
\begin{pgfscope}%
\definecolor{textcolor}{rgb}{0.000000,0.000000,0.000000}%
\pgfsetstrokecolor{textcolor}%
\pgfsetfillcolor{textcolor}%
\pgftext[x=1.625000in,y=0.622778in,,top]{\color{textcolor}\sffamily\fontsize{20.000000}{24.000000}\selectfont 3}%
\end{pgfscope}%
\begin{pgfscope}%
\pgfsetbuttcap%
\pgfsetroundjoin%
\definecolor{currentfill}{rgb}{0.000000,0.000000,0.000000}%
\pgfsetfillcolor{currentfill}%
\pgfsetlinewidth{0.803000pt}%
\definecolor{currentstroke}{rgb}{0.000000,0.000000,0.000000}%
\pgfsetstrokecolor{currentstroke}%
\pgfsetdash{}{0pt}%
\pgfsys@defobject{currentmarker}{\pgfqpoint{0.000000in}{-0.048611in}}{\pgfqpoint{0.000000in}{0.000000in}}{%
\pgfpathmoveto{\pgfqpoint{0.000000in}{0.000000in}}%
\pgfpathlineto{\pgfqpoint{0.000000in}{-0.048611in}}%
\pgfusepath{stroke,fill}%
}%
\begin{pgfscope}%
\pgfsys@transformshift{1.900000in}{0.720000in}%
\pgfsys@useobject{currentmarker}{}%
\end{pgfscope}%
\end{pgfscope}%
\begin{pgfscope}%
\definecolor{textcolor}{rgb}{0.000000,0.000000,0.000000}%
\pgfsetstrokecolor{textcolor}%
\pgfsetfillcolor{textcolor}%
\pgftext[x=1.900000in,y=0.622778in,,top]{\color{textcolor}\sffamily\fontsize{20.000000}{24.000000}\selectfont 4}%
\end{pgfscope}%
\begin{pgfscope}%
\pgfsetbuttcap%
\pgfsetroundjoin%
\definecolor{currentfill}{rgb}{0.000000,0.000000,0.000000}%
\pgfsetfillcolor{currentfill}%
\pgfsetlinewidth{0.803000pt}%
\definecolor{currentstroke}{rgb}{0.000000,0.000000,0.000000}%
\pgfsetstrokecolor{currentstroke}%
\pgfsetdash{}{0pt}%
\pgfsys@defobject{currentmarker}{\pgfqpoint{0.000000in}{-0.048611in}}{\pgfqpoint{0.000000in}{0.000000in}}{%
\pgfpathmoveto{\pgfqpoint{0.000000in}{0.000000in}}%
\pgfpathlineto{\pgfqpoint{0.000000in}{-0.048611in}}%
\pgfusepath{stroke,fill}%
}%
\begin{pgfscope}%
\pgfsys@transformshift{2.175000in}{0.720000in}%
\pgfsys@useobject{currentmarker}{}%
\end{pgfscope}%
\end{pgfscope}%
\begin{pgfscope}%
\definecolor{textcolor}{rgb}{0.000000,0.000000,0.000000}%
\pgfsetstrokecolor{textcolor}%
\pgfsetfillcolor{textcolor}%
\pgftext[x=2.175000in,y=0.622778in,,top]{\color{textcolor}\sffamily\fontsize{20.000000}{24.000000}\selectfont 5}%
\end{pgfscope}%
\begin{pgfscope}%
\pgfsetbuttcap%
\pgfsetroundjoin%
\definecolor{currentfill}{rgb}{0.000000,0.000000,0.000000}%
\pgfsetfillcolor{currentfill}%
\pgfsetlinewidth{0.803000pt}%
\definecolor{currentstroke}{rgb}{0.000000,0.000000,0.000000}%
\pgfsetstrokecolor{currentstroke}%
\pgfsetdash{}{0pt}%
\pgfsys@defobject{currentmarker}{\pgfqpoint{0.000000in}{-0.048611in}}{\pgfqpoint{0.000000in}{0.000000in}}{%
\pgfpathmoveto{\pgfqpoint{0.000000in}{0.000000in}}%
\pgfpathlineto{\pgfqpoint{0.000000in}{-0.048611in}}%
\pgfusepath{stroke,fill}%
}%
\begin{pgfscope}%
\pgfsys@transformshift{2.450000in}{0.720000in}%
\pgfsys@useobject{currentmarker}{}%
\end{pgfscope}%
\end{pgfscope}%
\begin{pgfscope}%
\definecolor{textcolor}{rgb}{0.000000,0.000000,0.000000}%
\pgfsetstrokecolor{textcolor}%
\pgfsetfillcolor{textcolor}%
\pgftext[x=2.450000in,y=0.622778in,,top]{\color{textcolor}\sffamily\fontsize{20.000000}{24.000000}\selectfont 6}%
\end{pgfscope}%
\begin{pgfscope}%
\pgfsetbuttcap%
\pgfsetroundjoin%
\definecolor{currentfill}{rgb}{0.000000,0.000000,0.000000}%
\pgfsetfillcolor{currentfill}%
\pgfsetlinewidth{0.803000pt}%
\definecolor{currentstroke}{rgb}{0.000000,0.000000,0.000000}%
\pgfsetstrokecolor{currentstroke}%
\pgfsetdash{}{0pt}%
\pgfsys@defobject{currentmarker}{\pgfqpoint{0.000000in}{-0.048611in}}{\pgfqpoint{0.000000in}{0.000000in}}{%
\pgfpathmoveto{\pgfqpoint{0.000000in}{0.000000in}}%
\pgfpathlineto{\pgfqpoint{0.000000in}{-0.048611in}}%
\pgfusepath{stroke,fill}%
}%
\begin{pgfscope}%
\pgfsys@transformshift{2.725000in}{0.720000in}%
\pgfsys@useobject{currentmarker}{}%
\end{pgfscope}%
\end{pgfscope}%
\begin{pgfscope}%
\definecolor{textcolor}{rgb}{0.000000,0.000000,0.000000}%
\pgfsetstrokecolor{textcolor}%
\pgfsetfillcolor{textcolor}%
\pgftext[x=2.725000in,y=0.622778in,,top]{\color{textcolor}\sffamily\fontsize{20.000000}{24.000000}\selectfont 7}%
\end{pgfscope}%
\begin{pgfscope}%
\pgfsetbuttcap%
\pgfsetroundjoin%
\definecolor{currentfill}{rgb}{0.000000,0.000000,0.000000}%
\pgfsetfillcolor{currentfill}%
\pgfsetlinewidth{0.803000pt}%
\definecolor{currentstroke}{rgb}{0.000000,0.000000,0.000000}%
\pgfsetstrokecolor{currentstroke}%
\pgfsetdash{}{0pt}%
\pgfsys@defobject{currentmarker}{\pgfqpoint{0.000000in}{-0.048611in}}{\pgfqpoint{0.000000in}{0.000000in}}{%
\pgfpathmoveto{\pgfqpoint{0.000000in}{0.000000in}}%
\pgfpathlineto{\pgfqpoint{0.000000in}{-0.048611in}}%
\pgfusepath{stroke,fill}%
}%
\begin{pgfscope}%
\pgfsys@transformshift{3.000000in}{0.720000in}%
\pgfsys@useobject{currentmarker}{}%
\end{pgfscope}%
\end{pgfscope}%
\begin{pgfscope}%
\definecolor{textcolor}{rgb}{0.000000,0.000000,0.000000}%
\pgfsetstrokecolor{textcolor}%
\pgfsetfillcolor{textcolor}%
\pgftext[x=3.000000in,y=0.622778in,,top]{\color{textcolor}\sffamily\fontsize{20.000000}{24.000000}\selectfont 8}%
\end{pgfscope}%
\begin{pgfscope}%
\pgfsetbuttcap%
\pgfsetroundjoin%
\definecolor{currentfill}{rgb}{0.000000,0.000000,0.000000}%
\pgfsetfillcolor{currentfill}%
\pgfsetlinewidth{0.803000pt}%
\definecolor{currentstroke}{rgb}{0.000000,0.000000,0.000000}%
\pgfsetstrokecolor{currentstroke}%
\pgfsetdash{}{0pt}%
\pgfsys@defobject{currentmarker}{\pgfqpoint{0.000000in}{-0.048611in}}{\pgfqpoint{0.000000in}{0.000000in}}{%
\pgfpathmoveto{\pgfqpoint{0.000000in}{0.000000in}}%
\pgfpathlineto{\pgfqpoint{0.000000in}{-0.048611in}}%
\pgfusepath{stroke,fill}%
}%
\begin{pgfscope}%
\pgfsys@transformshift{3.275000in}{0.720000in}%
\pgfsys@useobject{currentmarker}{}%
\end{pgfscope}%
\end{pgfscope}%
\begin{pgfscope}%
\definecolor{textcolor}{rgb}{0.000000,0.000000,0.000000}%
\pgfsetstrokecolor{textcolor}%
\pgfsetfillcolor{textcolor}%
\pgftext[x=3.275000in,y=0.622778in,,top]{\color{textcolor}\sffamily\fontsize{20.000000}{24.000000}\selectfont 9}%
\end{pgfscope}%
\begin{pgfscope}%
\pgfsetbuttcap%
\pgfsetroundjoin%
\definecolor{currentfill}{rgb}{0.000000,0.000000,0.000000}%
\pgfsetfillcolor{currentfill}%
\pgfsetlinewidth{0.803000pt}%
\definecolor{currentstroke}{rgb}{0.000000,0.000000,0.000000}%
\pgfsetstrokecolor{currentstroke}%
\pgfsetdash{}{0pt}%
\pgfsys@defobject{currentmarker}{\pgfqpoint{0.000000in}{-0.048611in}}{\pgfqpoint{0.000000in}{0.000000in}}{%
\pgfpathmoveto{\pgfqpoint{0.000000in}{0.000000in}}%
\pgfpathlineto{\pgfqpoint{0.000000in}{-0.048611in}}%
\pgfusepath{stroke,fill}%
}%
\begin{pgfscope}%
\pgfsys@transformshift{3.550000in}{0.720000in}%
\pgfsys@useobject{currentmarker}{}%
\end{pgfscope}%
\end{pgfscope}%
\begin{pgfscope}%
\definecolor{textcolor}{rgb}{0.000000,0.000000,0.000000}%
\pgfsetstrokecolor{textcolor}%
\pgfsetfillcolor{textcolor}%
\pgftext[x=3.550000in,y=0.622778in,,top]{\color{textcolor}\sffamily\fontsize{20.000000}{24.000000}\selectfont 10}%
\end{pgfscope}%
\begin{pgfscope}%
\pgfsetbuttcap%
\pgfsetroundjoin%
\definecolor{currentfill}{rgb}{0.000000,0.000000,0.000000}%
\pgfsetfillcolor{currentfill}%
\pgfsetlinewidth{0.803000pt}%
\definecolor{currentstroke}{rgb}{0.000000,0.000000,0.000000}%
\pgfsetstrokecolor{currentstroke}%
\pgfsetdash{}{0pt}%
\pgfsys@defobject{currentmarker}{\pgfqpoint{0.000000in}{-0.048611in}}{\pgfqpoint{0.000000in}{0.000000in}}{%
\pgfpathmoveto{\pgfqpoint{0.000000in}{0.000000in}}%
\pgfpathlineto{\pgfqpoint{0.000000in}{-0.048611in}}%
\pgfusepath{stroke,fill}%
}%
\begin{pgfscope}%
\pgfsys@transformshift{3.825000in}{0.720000in}%
\pgfsys@useobject{currentmarker}{}%
\end{pgfscope}%
\end{pgfscope}%
\begin{pgfscope}%
\definecolor{textcolor}{rgb}{0.000000,0.000000,0.000000}%
\pgfsetstrokecolor{textcolor}%
\pgfsetfillcolor{textcolor}%
\pgftext[x=3.825000in,y=0.622778in,,top]{\color{textcolor}\sffamily\fontsize{20.000000}{24.000000}\selectfont 11}%
\end{pgfscope}%
\begin{pgfscope}%
\pgfsetbuttcap%
\pgfsetroundjoin%
\definecolor{currentfill}{rgb}{0.000000,0.000000,0.000000}%
\pgfsetfillcolor{currentfill}%
\pgfsetlinewidth{0.803000pt}%
\definecolor{currentstroke}{rgb}{0.000000,0.000000,0.000000}%
\pgfsetstrokecolor{currentstroke}%
\pgfsetdash{}{0pt}%
\pgfsys@defobject{currentmarker}{\pgfqpoint{0.000000in}{-0.048611in}}{\pgfqpoint{0.000000in}{0.000000in}}{%
\pgfpathmoveto{\pgfqpoint{0.000000in}{0.000000in}}%
\pgfpathlineto{\pgfqpoint{0.000000in}{-0.048611in}}%
\pgfusepath{stroke,fill}%
}%
\begin{pgfscope}%
\pgfsys@transformshift{4.100000in}{0.720000in}%
\pgfsys@useobject{currentmarker}{}%
\end{pgfscope}%
\end{pgfscope}%
\begin{pgfscope}%
\definecolor{textcolor}{rgb}{0.000000,0.000000,0.000000}%
\pgfsetstrokecolor{textcolor}%
\pgfsetfillcolor{textcolor}%
\pgftext[x=4.100000in,y=0.622778in,,top]{\color{textcolor}\sffamily\fontsize{20.000000}{24.000000}\selectfont 12}%
\end{pgfscope}%
\begin{pgfscope}%
\pgfsetbuttcap%
\pgfsetroundjoin%
\definecolor{currentfill}{rgb}{0.000000,0.000000,0.000000}%
\pgfsetfillcolor{currentfill}%
\pgfsetlinewidth{0.803000pt}%
\definecolor{currentstroke}{rgb}{0.000000,0.000000,0.000000}%
\pgfsetstrokecolor{currentstroke}%
\pgfsetdash{}{0pt}%
\pgfsys@defobject{currentmarker}{\pgfqpoint{0.000000in}{-0.048611in}}{\pgfqpoint{0.000000in}{0.000000in}}{%
\pgfpathmoveto{\pgfqpoint{0.000000in}{0.000000in}}%
\pgfpathlineto{\pgfqpoint{0.000000in}{-0.048611in}}%
\pgfusepath{stroke,fill}%
}%
\begin{pgfscope}%
\pgfsys@transformshift{4.375000in}{0.720000in}%
\pgfsys@useobject{currentmarker}{}%
\end{pgfscope}%
\end{pgfscope}%
\begin{pgfscope}%
\definecolor{textcolor}{rgb}{0.000000,0.000000,0.000000}%
\pgfsetstrokecolor{textcolor}%
\pgfsetfillcolor{textcolor}%
\pgftext[x=4.375000in,y=0.622778in,,top]{\color{textcolor}\sffamily\fontsize{20.000000}{24.000000}\selectfont 13}%
\end{pgfscope}%
\begin{pgfscope}%
\pgfsetbuttcap%
\pgfsetroundjoin%
\definecolor{currentfill}{rgb}{0.000000,0.000000,0.000000}%
\pgfsetfillcolor{currentfill}%
\pgfsetlinewidth{0.803000pt}%
\definecolor{currentstroke}{rgb}{0.000000,0.000000,0.000000}%
\pgfsetstrokecolor{currentstroke}%
\pgfsetdash{}{0pt}%
\pgfsys@defobject{currentmarker}{\pgfqpoint{0.000000in}{-0.048611in}}{\pgfqpoint{0.000000in}{0.000000in}}{%
\pgfpathmoveto{\pgfqpoint{0.000000in}{0.000000in}}%
\pgfpathlineto{\pgfqpoint{0.000000in}{-0.048611in}}%
\pgfusepath{stroke,fill}%
}%
\begin{pgfscope}%
\pgfsys@transformshift{4.650000in}{0.720000in}%
\pgfsys@useobject{currentmarker}{}%
\end{pgfscope}%
\end{pgfscope}%
\begin{pgfscope}%
\definecolor{textcolor}{rgb}{0.000000,0.000000,0.000000}%
\pgfsetstrokecolor{textcolor}%
\pgfsetfillcolor{textcolor}%
\pgftext[x=4.650000in,y=0.622778in,,top]{\color{textcolor}\sffamily\fontsize{20.000000}{24.000000}\selectfont 14}%
\end{pgfscope}%
\begin{pgfscope}%
\pgfsetbuttcap%
\pgfsetroundjoin%
\definecolor{currentfill}{rgb}{0.000000,0.000000,0.000000}%
\pgfsetfillcolor{currentfill}%
\pgfsetlinewidth{0.803000pt}%
\definecolor{currentstroke}{rgb}{0.000000,0.000000,0.000000}%
\pgfsetstrokecolor{currentstroke}%
\pgfsetdash{}{0pt}%
\pgfsys@defobject{currentmarker}{\pgfqpoint{0.000000in}{-0.048611in}}{\pgfqpoint{0.000000in}{0.000000in}}{%
\pgfpathmoveto{\pgfqpoint{0.000000in}{0.000000in}}%
\pgfpathlineto{\pgfqpoint{0.000000in}{-0.048611in}}%
\pgfusepath{stroke,fill}%
}%
\begin{pgfscope}%
\pgfsys@transformshift{4.925000in}{0.720000in}%
\pgfsys@useobject{currentmarker}{}%
\end{pgfscope}%
\end{pgfscope}%
\begin{pgfscope}%
\definecolor{textcolor}{rgb}{0.000000,0.000000,0.000000}%
\pgfsetstrokecolor{textcolor}%
\pgfsetfillcolor{textcolor}%
\pgftext[x=4.925000in,y=0.622778in,,top]{\color{textcolor}\sffamily\fontsize{20.000000}{24.000000}\selectfont 15}%
\end{pgfscope}%
\begin{pgfscope}%
\definecolor{textcolor}{rgb}{0.000000,0.000000,0.000000}%
\pgfsetstrokecolor{textcolor}%
\pgfsetfillcolor{textcolor}%
\pgftext[x=3.000000in,y=0.311155in,,top]{\color{textcolor}\sffamily\fontsize{20.000000}{24.000000}\selectfont \(\displaystyle N_{pe}\)}%
\end{pgfscope}%
\begin{pgfscope}%
\pgfsetbuttcap%
\pgfsetroundjoin%
\definecolor{currentfill}{rgb}{0.000000,0.000000,0.000000}%
\pgfsetfillcolor{currentfill}%
\pgfsetlinewidth{0.803000pt}%
\definecolor{currentstroke}{rgb}{0.000000,0.000000,0.000000}%
\pgfsetstrokecolor{currentstroke}%
\pgfsetdash{}{0pt}%
\pgfsys@defobject{currentmarker}{\pgfqpoint{-0.048611in}{0.000000in}}{\pgfqpoint{-0.000000in}{0.000000in}}{%
\pgfpathmoveto{\pgfqpoint{-0.000000in}{0.000000in}}%
\pgfpathlineto{\pgfqpoint{-0.048611in}{0.000000in}}%
\pgfusepath{stroke,fill}%
}%
\begin{pgfscope}%
\pgfsys@transformshift{0.800000in}{0.720000in}%
\pgfsys@useobject{currentmarker}{}%
\end{pgfscope}%
\end{pgfscope}%
\begin{pgfscope}%
\definecolor{textcolor}{rgb}{0.000000,0.000000,0.000000}%
\pgfsetstrokecolor{textcolor}%
\pgfsetfillcolor{textcolor}%
\pgftext[x=0.360215in, y=0.619981in, left, base]{\color{textcolor}\sffamily\fontsize{20.000000}{24.000000}\selectfont \(\displaystyle {0.0}\)}%
\end{pgfscope}%
\begin{pgfscope}%
\pgfsetbuttcap%
\pgfsetroundjoin%
\definecolor{currentfill}{rgb}{0.000000,0.000000,0.000000}%
\pgfsetfillcolor{currentfill}%
\pgfsetlinewidth{0.803000pt}%
\definecolor{currentstroke}{rgb}{0.000000,0.000000,0.000000}%
\pgfsetstrokecolor{currentstroke}%
\pgfsetdash{}{0pt}%
\pgfsys@defobject{currentmarker}{\pgfqpoint{-0.048611in}{0.000000in}}{\pgfqpoint{-0.000000in}{0.000000in}}{%
\pgfpathmoveto{\pgfqpoint{-0.000000in}{0.000000in}}%
\pgfpathlineto{\pgfqpoint{-0.048611in}{0.000000in}}%
\pgfusepath{stroke,fill}%
}%
\begin{pgfscope}%
\pgfsys@transformshift{0.800000in}{1.599141in}%
\pgfsys@useobject{currentmarker}{}%
\end{pgfscope}%
\end{pgfscope}%
\begin{pgfscope}%
\definecolor{textcolor}{rgb}{0.000000,0.000000,0.000000}%
\pgfsetstrokecolor{textcolor}%
\pgfsetfillcolor{textcolor}%
\pgftext[x=0.360215in, y=1.499122in, left, base]{\color{textcolor}\sffamily\fontsize{20.000000}{24.000000}\selectfont \(\displaystyle {0.5}\)}%
\end{pgfscope}%
\begin{pgfscope}%
\pgfsetbuttcap%
\pgfsetroundjoin%
\definecolor{currentfill}{rgb}{0.000000,0.000000,0.000000}%
\pgfsetfillcolor{currentfill}%
\pgfsetlinewidth{0.803000pt}%
\definecolor{currentstroke}{rgb}{0.000000,0.000000,0.000000}%
\pgfsetstrokecolor{currentstroke}%
\pgfsetdash{}{0pt}%
\pgfsys@defobject{currentmarker}{\pgfqpoint{-0.048611in}{0.000000in}}{\pgfqpoint{-0.000000in}{0.000000in}}{%
\pgfpathmoveto{\pgfqpoint{-0.000000in}{0.000000in}}%
\pgfpathlineto{\pgfqpoint{-0.048611in}{0.000000in}}%
\pgfusepath{stroke,fill}%
}%
\begin{pgfscope}%
\pgfsys@transformshift{0.800000in}{2.478282in}%
\pgfsys@useobject{currentmarker}{}%
\end{pgfscope}%
\end{pgfscope}%
\begin{pgfscope}%
\definecolor{textcolor}{rgb}{0.000000,0.000000,0.000000}%
\pgfsetstrokecolor{textcolor}%
\pgfsetfillcolor{textcolor}%
\pgftext[x=0.360215in, y=2.378263in, left, base]{\color{textcolor}\sffamily\fontsize{20.000000}{24.000000}\selectfont \(\displaystyle {1.0}\)}%
\end{pgfscope}%
\begin{pgfscope}%
\pgfsetbuttcap%
\pgfsetroundjoin%
\definecolor{currentfill}{rgb}{0.000000,0.000000,0.000000}%
\pgfsetfillcolor{currentfill}%
\pgfsetlinewidth{0.803000pt}%
\definecolor{currentstroke}{rgb}{0.000000,0.000000,0.000000}%
\pgfsetstrokecolor{currentstroke}%
\pgfsetdash{}{0pt}%
\pgfsys@defobject{currentmarker}{\pgfqpoint{-0.048611in}{0.000000in}}{\pgfqpoint{-0.000000in}{0.000000in}}{%
\pgfpathmoveto{\pgfqpoint{-0.000000in}{0.000000in}}%
\pgfpathlineto{\pgfqpoint{-0.048611in}{0.000000in}}%
\pgfusepath{stroke,fill}%
}%
\begin{pgfscope}%
\pgfsys@transformshift{0.800000in}{3.357423in}%
\pgfsys@useobject{currentmarker}{}%
\end{pgfscope}%
\end{pgfscope}%
\begin{pgfscope}%
\definecolor{textcolor}{rgb}{0.000000,0.000000,0.000000}%
\pgfsetstrokecolor{textcolor}%
\pgfsetfillcolor{textcolor}%
\pgftext[x=0.360215in, y=3.257404in, left, base]{\color{textcolor}\sffamily\fontsize{20.000000}{24.000000}\selectfont \(\displaystyle {1.5}\)}%
\end{pgfscope}%
\begin{pgfscope}%
\pgfsetbuttcap%
\pgfsetroundjoin%
\definecolor{currentfill}{rgb}{0.000000,0.000000,0.000000}%
\pgfsetfillcolor{currentfill}%
\pgfsetlinewidth{0.803000pt}%
\definecolor{currentstroke}{rgb}{0.000000,0.000000,0.000000}%
\pgfsetstrokecolor{currentstroke}%
\pgfsetdash{}{0pt}%
\pgfsys@defobject{currentmarker}{\pgfqpoint{-0.048611in}{0.000000in}}{\pgfqpoint{-0.000000in}{0.000000in}}{%
\pgfpathmoveto{\pgfqpoint{-0.000000in}{0.000000in}}%
\pgfpathlineto{\pgfqpoint{-0.048611in}{0.000000in}}%
\pgfusepath{stroke,fill}%
}%
\begin{pgfscope}%
\pgfsys@transformshift{0.800000in}{4.236564in}%
\pgfsys@useobject{currentmarker}{}%
\end{pgfscope}%
\end{pgfscope}%
\begin{pgfscope}%
\definecolor{textcolor}{rgb}{0.000000,0.000000,0.000000}%
\pgfsetstrokecolor{textcolor}%
\pgfsetfillcolor{textcolor}%
\pgftext[x=0.360215in, y=4.136545in, left, base]{\color{textcolor}\sffamily\fontsize{20.000000}{24.000000}\selectfont \(\displaystyle {2.0}\)}%
\end{pgfscope}%
\begin{pgfscope}%
\pgfsetbuttcap%
\pgfsetroundjoin%
\definecolor{currentfill}{rgb}{0.000000,0.000000,0.000000}%
\pgfsetfillcolor{currentfill}%
\pgfsetlinewidth{0.803000pt}%
\definecolor{currentstroke}{rgb}{0.000000,0.000000,0.000000}%
\pgfsetstrokecolor{currentstroke}%
\pgfsetdash{}{0pt}%
\pgfsys@defobject{currentmarker}{\pgfqpoint{-0.048611in}{0.000000in}}{\pgfqpoint{-0.000000in}{0.000000in}}{%
\pgfpathmoveto{\pgfqpoint{-0.000000in}{0.000000in}}%
\pgfpathlineto{\pgfqpoint{-0.048611in}{0.000000in}}%
\pgfusepath{stroke,fill}%
}%
\begin{pgfscope}%
\pgfsys@transformshift{0.800000in}{5.115705in}%
\pgfsys@useobject{currentmarker}{}%
\end{pgfscope}%
\end{pgfscope}%
\begin{pgfscope}%
\definecolor{textcolor}{rgb}{0.000000,0.000000,0.000000}%
\pgfsetstrokecolor{textcolor}%
\pgfsetfillcolor{textcolor}%
\pgftext[x=0.360215in, y=5.015686in, left, base]{\color{textcolor}\sffamily\fontsize{20.000000}{24.000000}\selectfont \(\displaystyle {2.5}\)}%
\end{pgfscope}%
\begin{pgfscope}%
\definecolor{textcolor}{rgb}{0.000000,0.000000,0.000000}%
\pgfsetstrokecolor{textcolor}%
\pgfsetfillcolor{textcolor}%
\pgftext[x=0.304660in,y=3.120000in,,bottom,rotate=90.000000]{\color{textcolor}\sffamily\fontsize{20.000000}{24.000000}\selectfont \(\displaystyle \mathrm{W-dist}/\si{ns}\)}%
\end{pgfscope}%
\begin{pgfscope}%
\pgfpathrectangle{\pgfqpoint{0.800000in}{0.720000in}}{\pgfqpoint{4.400000in}{4.800000in}}%
\pgfusepath{clip}%
\pgfsetrectcap%
\pgfsetroundjoin%
\pgfsetlinewidth{1.003750pt}%
\definecolor{currentstroke}{rgb}{0.000000,0.000000,0.000000}%
\pgfsetstrokecolor{currentstroke}%
\pgfsetdash{}{0pt}%
\pgfpathmoveto{\pgfqpoint{1.075000in}{1.137443in}}%
\pgfpathlineto{\pgfqpoint{1.075000in}{0.720215in}}%
\pgfusepath{stroke}%
\end{pgfscope}%
\begin{pgfscope}%
\pgfpathrectangle{\pgfqpoint{0.800000in}{0.720000in}}{\pgfqpoint{4.400000in}{4.800000in}}%
\pgfusepath{clip}%
\pgfsetrectcap%
\pgfsetroundjoin%
\pgfsetlinewidth{1.003750pt}%
\definecolor{currentstroke}{rgb}{0.000000,0.000000,0.000000}%
\pgfsetstrokecolor{currentstroke}%
\pgfsetdash{}{0pt}%
\pgfpathmoveto{\pgfqpoint{1.075000in}{1.602816in}}%
\pgfpathlineto{\pgfqpoint{1.075000in}{2.283821in}}%
\pgfusepath{stroke}%
\end{pgfscope}%
\begin{pgfscope}%
\pgfpathrectangle{\pgfqpoint{0.800000in}{0.720000in}}{\pgfqpoint{4.400000in}{4.800000in}}%
\pgfusepath{clip}%
\pgfsetrectcap%
\pgfsetroundjoin%
\pgfsetlinewidth{1.003750pt}%
\definecolor{currentstroke}{rgb}{0.000000,0.000000,0.000000}%
\pgfsetstrokecolor{currentstroke}%
\pgfsetdash{}{0pt}%
\pgfpathmoveto{\pgfqpoint{1.040625in}{0.720215in}}%
\pgfpathlineto{\pgfqpoint{1.109375in}{0.720215in}}%
\pgfusepath{stroke}%
\end{pgfscope}%
\begin{pgfscope}%
\pgfpathrectangle{\pgfqpoint{0.800000in}{0.720000in}}{\pgfqpoint{4.400000in}{4.800000in}}%
\pgfusepath{clip}%
\pgfsetrectcap%
\pgfsetroundjoin%
\pgfsetlinewidth{1.003750pt}%
\definecolor{currentstroke}{rgb}{0.000000,0.000000,0.000000}%
\pgfsetstrokecolor{currentstroke}%
\pgfsetdash{}{0pt}%
\pgfpathmoveto{\pgfqpoint{1.040625in}{2.283821in}}%
\pgfpathlineto{\pgfqpoint{1.109375in}{2.283821in}}%
\pgfusepath{stroke}%
\end{pgfscope}%
\begin{pgfscope}%
\pgfpathrectangle{\pgfqpoint{0.800000in}{0.720000in}}{\pgfqpoint{4.400000in}{4.800000in}}%
\pgfusepath{clip}%
\pgfsetrectcap%
\pgfsetroundjoin%
\pgfsetlinewidth{1.003750pt}%
\definecolor{currentstroke}{rgb}{0.000000,0.000000,0.000000}%
\pgfsetstrokecolor{currentstroke}%
\pgfsetdash{}{0pt}%
\pgfpathmoveto{\pgfqpoint{1.350000in}{1.687260in}}%
\pgfpathlineto{\pgfqpoint{1.350000in}{0.817347in}}%
\pgfusepath{stroke}%
\end{pgfscope}%
\begin{pgfscope}%
\pgfpathrectangle{\pgfqpoint{0.800000in}{0.720000in}}{\pgfqpoint{4.400000in}{4.800000in}}%
\pgfusepath{clip}%
\pgfsetrectcap%
\pgfsetroundjoin%
\pgfsetlinewidth{1.003750pt}%
\definecolor{currentstroke}{rgb}{0.000000,0.000000,0.000000}%
\pgfsetstrokecolor{currentstroke}%
\pgfsetdash{}{0pt}%
\pgfpathmoveto{\pgfqpoint{1.350000in}{2.997316in}}%
\pgfpathlineto{\pgfqpoint{1.350000in}{4.940300in}}%
\pgfusepath{stroke}%
\end{pgfscope}%
\begin{pgfscope}%
\pgfpathrectangle{\pgfqpoint{0.800000in}{0.720000in}}{\pgfqpoint{4.400000in}{4.800000in}}%
\pgfusepath{clip}%
\pgfsetrectcap%
\pgfsetroundjoin%
\pgfsetlinewidth{1.003750pt}%
\definecolor{currentstroke}{rgb}{0.000000,0.000000,0.000000}%
\pgfsetstrokecolor{currentstroke}%
\pgfsetdash{}{0pt}%
\pgfpathmoveto{\pgfqpoint{1.315625in}{0.817347in}}%
\pgfpathlineto{\pgfqpoint{1.384375in}{0.817347in}}%
\pgfusepath{stroke}%
\end{pgfscope}%
\begin{pgfscope}%
\pgfpathrectangle{\pgfqpoint{0.800000in}{0.720000in}}{\pgfqpoint{4.400000in}{4.800000in}}%
\pgfusepath{clip}%
\pgfsetrectcap%
\pgfsetroundjoin%
\pgfsetlinewidth{1.003750pt}%
\definecolor{currentstroke}{rgb}{0.000000,0.000000,0.000000}%
\pgfsetstrokecolor{currentstroke}%
\pgfsetdash{}{0pt}%
\pgfpathmoveto{\pgfqpoint{1.315625in}{4.940300in}}%
\pgfpathlineto{\pgfqpoint{1.384375in}{4.940300in}}%
\pgfusepath{stroke}%
\end{pgfscope}%
\begin{pgfscope}%
\pgfpathrectangle{\pgfqpoint{0.800000in}{0.720000in}}{\pgfqpoint{4.400000in}{4.800000in}}%
\pgfusepath{clip}%
\pgfsetrectcap%
\pgfsetroundjoin%
\pgfsetlinewidth{1.003750pt}%
\definecolor{currentstroke}{rgb}{0.000000,0.000000,0.000000}%
\pgfsetstrokecolor{currentstroke}%
\pgfsetdash{}{0pt}%
\pgfpathmoveto{\pgfqpoint{1.625000in}{1.867618in}}%
\pgfpathlineto{\pgfqpoint{1.625000in}{0.839857in}}%
\pgfusepath{stroke}%
\end{pgfscope}%
\begin{pgfscope}%
\pgfpathrectangle{\pgfqpoint{0.800000in}{0.720000in}}{\pgfqpoint{4.400000in}{4.800000in}}%
\pgfusepath{clip}%
\pgfsetrectcap%
\pgfsetroundjoin%
\pgfsetlinewidth{1.003750pt}%
\definecolor{currentstroke}{rgb}{0.000000,0.000000,0.000000}%
\pgfsetstrokecolor{currentstroke}%
\pgfsetdash{}{0pt}%
\pgfpathmoveto{\pgfqpoint{1.625000in}{3.247630in}}%
\pgfpathlineto{\pgfqpoint{1.625000in}{5.291429in}}%
\pgfusepath{stroke}%
\end{pgfscope}%
\begin{pgfscope}%
\pgfpathrectangle{\pgfqpoint{0.800000in}{0.720000in}}{\pgfqpoint{4.400000in}{4.800000in}}%
\pgfusepath{clip}%
\pgfsetrectcap%
\pgfsetroundjoin%
\pgfsetlinewidth{1.003750pt}%
\definecolor{currentstroke}{rgb}{0.000000,0.000000,0.000000}%
\pgfsetstrokecolor{currentstroke}%
\pgfsetdash{}{0pt}%
\pgfpathmoveto{\pgfqpoint{1.590625in}{0.839857in}}%
\pgfpathlineto{\pgfqpoint{1.659375in}{0.839857in}}%
\pgfusepath{stroke}%
\end{pgfscope}%
\begin{pgfscope}%
\pgfpathrectangle{\pgfqpoint{0.800000in}{0.720000in}}{\pgfqpoint{4.400000in}{4.800000in}}%
\pgfusepath{clip}%
\pgfsetrectcap%
\pgfsetroundjoin%
\pgfsetlinewidth{1.003750pt}%
\definecolor{currentstroke}{rgb}{0.000000,0.000000,0.000000}%
\pgfsetstrokecolor{currentstroke}%
\pgfsetdash{}{0pt}%
\pgfpathmoveto{\pgfqpoint{1.590625in}{5.291429in}}%
\pgfpathlineto{\pgfqpoint{1.659375in}{5.291429in}}%
\pgfusepath{stroke}%
\end{pgfscope}%
\begin{pgfscope}%
\pgfpathrectangle{\pgfqpoint{0.800000in}{0.720000in}}{\pgfqpoint{4.400000in}{4.800000in}}%
\pgfusepath{clip}%
\pgfsetrectcap%
\pgfsetroundjoin%
\pgfsetlinewidth{1.003750pt}%
\definecolor{currentstroke}{rgb}{0.000000,0.000000,0.000000}%
\pgfsetstrokecolor{currentstroke}%
\pgfsetdash{}{0pt}%
\pgfpathmoveto{\pgfqpoint{1.900000in}{1.935387in}}%
\pgfpathlineto{\pgfqpoint{1.900000in}{0.943107in}}%
\pgfusepath{stroke}%
\end{pgfscope}%
\begin{pgfscope}%
\pgfpathrectangle{\pgfqpoint{0.800000in}{0.720000in}}{\pgfqpoint{4.400000in}{4.800000in}}%
\pgfusepath{clip}%
\pgfsetrectcap%
\pgfsetroundjoin%
\pgfsetlinewidth{1.003750pt}%
\definecolor{currentstroke}{rgb}{0.000000,0.000000,0.000000}%
\pgfsetstrokecolor{currentstroke}%
\pgfsetdash{}{0pt}%
\pgfpathmoveto{\pgfqpoint{1.900000in}{3.263401in}}%
\pgfpathlineto{\pgfqpoint{1.900000in}{5.232818in}}%
\pgfusepath{stroke}%
\end{pgfscope}%
\begin{pgfscope}%
\pgfpathrectangle{\pgfqpoint{0.800000in}{0.720000in}}{\pgfqpoint{4.400000in}{4.800000in}}%
\pgfusepath{clip}%
\pgfsetrectcap%
\pgfsetroundjoin%
\pgfsetlinewidth{1.003750pt}%
\definecolor{currentstroke}{rgb}{0.000000,0.000000,0.000000}%
\pgfsetstrokecolor{currentstroke}%
\pgfsetdash{}{0pt}%
\pgfpathmoveto{\pgfqpoint{1.865625in}{0.943107in}}%
\pgfpathlineto{\pgfqpoint{1.934375in}{0.943107in}}%
\pgfusepath{stroke}%
\end{pgfscope}%
\begin{pgfscope}%
\pgfpathrectangle{\pgfqpoint{0.800000in}{0.720000in}}{\pgfqpoint{4.400000in}{4.800000in}}%
\pgfusepath{clip}%
\pgfsetrectcap%
\pgfsetroundjoin%
\pgfsetlinewidth{1.003750pt}%
\definecolor{currentstroke}{rgb}{0.000000,0.000000,0.000000}%
\pgfsetstrokecolor{currentstroke}%
\pgfsetdash{}{0pt}%
\pgfpathmoveto{\pgfqpoint{1.865625in}{5.232818in}}%
\pgfpathlineto{\pgfqpoint{1.934375in}{5.232818in}}%
\pgfusepath{stroke}%
\end{pgfscope}%
\begin{pgfscope}%
\pgfpathrectangle{\pgfqpoint{0.800000in}{0.720000in}}{\pgfqpoint{4.400000in}{4.800000in}}%
\pgfusepath{clip}%
\pgfsetrectcap%
\pgfsetroundjoin%
\pgfsetlinewidth{1.003750pt}%
\definecolor{currentstroke}{rgb}{0.000000,0.000000,0.000000}%
\pgfsetstrokecolor{currentstroke}%
\pgfsetdash{}{0pt}%
\pgfpathmoveto{\pgfqpoint{2.175000in}{2.042205in}}%
\pgfpathlineto{\pgfqpoint{2.175000in}{1.155015in}}%
\pgfusepath{stroke}%
\end{pgfscope}%
\begin{pgfscope}%
\pgfpathrectangle{\pgfqpoint{0.800000in}{0.720000in}}{\pgfqpoint{4.400000in}{4.800000in}}%
\pgfusepath{clip}%
\pgfsetrectcap%
\pgfsetroundjoin%
\pgfsetlinewidth{1.003750pt}%
\definecolor{currentstroke}{rgb}{0.000000,0.000000,0.000000}%
\pgfsetstrokecolor{currentstroke}%
\pgfsetdash{}{0pt}%
\pgfpathmoveto{\pgfqpoint{2.175000in}{3.178667in}}%
\pgfpathlineto{\pgfqpoint{2.175000in}{4.861529in}}%
\pgfusepath{stroke}%
\end{pgfscope}%
\begin{pgfscope}%
\pgfpathrectangle{\pgfqpoint{0.800000in}{0.720000in}}{\pgfqpoint{4.400000in}{4.800000in}}%
\pgfusepath{clip}%
\pgfsetrectcap%
\pgfsetroundjoin%
\pgfsetlinewidth{1.003750pt}%
\definecolor{currentstroke}{rgb}{0.000000,0.000000,0.000000}%
\pgfsetstrokecolor{currentstroke}%
\pgfsetdash{}{0pt}%
\pgfpathmoveto{\pgfqpoint{2.140625in}{1.155015in}}%
\pgfpathlineto{\pgfqpoint{2.209375in}{1.155015in}}%
\pgfusepath{stroke}%
\end{pgfscope}%
\begin{pgfscope}%
\pgfpathrectangle{\pgfqpoint{0.800000in}{0.720000in}}{\pgfqpoint{4.400000in}{4.800000in}}%
\pgfusepath{clip}%
\pgfsetrectcap%
\pgfsetroundjoin%
\pgfsetlinewidth{1.003750pt}%
\definecolor{currentstroke}{rgb}{0.000000,0.000000,0.000000}%
\pgfsetstrokecolor{currentstroke}%
\pgfsetdash{}{0pt}%
\pgfpathmoveto{\pgfqpoint{2.140625in}{4.861529in}}%
\pgfpathlineto{\pgfqpoint{2.209375in}{4.861529in}}%
\pgfusepath{stroke}%
\end{pgfscope}%
\begin{pgfscope}%
\pgfpathrectangle{\pgfqpoint{0.800000in}{0.720000in}}{\pgfqpoint{4.400000in}{4.800000in}}%
\pgfusepath{clip}%
\pgfsetrectcap%
\pgfsetroundjoin%
\pgfsetlinewidth{1.003750pt}%
\definecolor{currentstroke}{rgb}{0.000000,0.000000,0.000000}%
\pgfsetstrokecolor{currentstroke}%
\pgfsetdash{}{0pt}%
\pgfpathmoveto{\pgfqpoint{2.450000in}{2.037654in}}%
\pgfpathlineto{\pgfqpoint{2.450000in}{1.195451in}}%
\pgfusepath{stroke}%
\end{pgfscope}%
\begin{pgfscope}%
\pgfpathrectangle{\pgfqpoint{0.800000in}{0.720000in}}{\pgfqpoint{4.400000in}{4.800000in}}%
\pgfusepath{clip}%
\pgfsetrectcap%
\pgfsetroundjoin%
\pgfsetlinewidth{1.003750pt}%
\definecolor{currentstroke}{rgb}{0.000000,0.000000,0.000000}%
\pgfsetstrokecolor{currentstroke}%
\pgfsetdash{}{0pt}%
\pgfpathmoveto{\pgfqpoint{2.450000in}{3.036130in}}%
\pgfpathlineto{\pgfqpoint{2.450000in}{4.503463in}}%
\pgfusepath{stroke}%
\end{pgfscope}%
\begin{pgfscope}%
\pgfpathrectangle{\pgfqpoint{0.800000in}{0.720000in}}{\pgfqpoint{4.400000in}{4.800000in}}%
\pgfusepath{clip}%
\pgfsetrectcap%
\pgfsetroundjoin%
\pgfsetlinewidth{1.003750pt}%
\definecolor{currentstroke}{rgb}{0.000000,0.000000,0.000000}%
\pgfsetstrokecolor{currentstroke}%
\pgfsetdash{}{0pt}%
\pgfpathmoveto{\pgfqpoint{2.415625in}{1.195451in}}%
\pgfpathlineto{\pgfqpoint{2.484375in}{1.195451in}}%
\pgfusepath{stroke}%
\end{pgfscope}%
\begin{pgfscope}%
\pgfpathrectangle{\pgfqpoint{0.800000in}{0.720000in}}{\pgfqpoint{4.400000in}{4.800000in}}%
\pgfusepath{clip}%
\pgfsetrectcap%
\pgfsetroundjoin%
\pgfsetlinewidth{1.003750pt}%
\definecolor{currentstroke}{rgb}{0.000000,0.000000,0.000000}%
\pgfsetstrokecolor{currentstroke}%
\pgfsetdash{}{0pt}%
\pgfpathmoveto{\pgfqpoint{2.415625in}{4.503463in}}%
\pgfpathlineto{\pgfqpoint{2.484375in}{4.503463in}}%
\pgfusepath{stroke}%
\end{pgfscope}%
\begin{pgfscope}%
\pgfpathrectangle{\pgfqpoint{0.800000in}{0.720000in}}{\pgfqpoint{4.400000in}{4.800000in}}%
\pgfusepath{clip}%
\pgfsetrectcap%
\pgfsetroundjoin%
\pgfsetlinewidth{1.003750pt}%
\definecolor{currentstroke}{rgb}{0.000000,0.000000,0.000000}%
\pgfsetstrokecolor{currentstroke}%
\pgfsetdash{}{0pt}%
\pgfpathmoveto{\pgfqpoint{2.725000in}{2.045823in}}%
\pgfpathlineto{\pgfqpoint{2.725000in}{1.236004in}}%
\pgfusepath{stroke}%
\end{pgfscope}%
\begin{pgfscope}%
\pgfpathrectangle{\pgfqpoint{0.800000in}{0.720000in}}{\pgfqpoint{4.400000in}{4.800000in}}%
\pgfusepath{clip}%
\pgfsetrectcap%
\pgfsetroundjoin%
\pgfsetlinewidth{1.003750pt}%
\definecolor{currentstroke}{rgb}{0.000000,0.000000,0.000000}%
\pgfsetstrokecolor{currentstroke}%
\pgfsetdash{}{0pt}%
\pgfpathmoveto{\pgfqpoint{2.725000in}{3.074083in}}%
\pgfpathlineto{\pgfqpoint{2.725000in}{4.613233in}}%
\pgfusepath{stroke}%
\end{pgfscope}%
\begin{pgfscope}%
\pgfpathrectangle{\pgfqpoint{0.800000in}{0.720000in}}{\pgfqpoint{4.400000in}{4.800000in}}%
\pgfusepath{clip}%
\pgfsetrectcap%
\pgfsetroundjoin%
\pgfsetlinewidth{1.003750pt}%
\definecolor{currentstroke}{rgb}{0.000000,0.000000,0.000000}%
\pgfsetstrokecolor{currentstroke}%
\pgfsetdash{}{0pt}%
\pgfpathmoveto{\pgfqpoint{2.690625in}{1.236004in}}%
\pgfpathlineto{\pgfqpoint{2.759375in}{1.236004in}}%
\pgfusepath{stroke}%
\end{pgfscope}%
\begin{pgfscope}%
\pgfpathrectangle{\pgfqpoint{0.800000in}{0.720000in}}{\pgfqpoint{4.400000in}{4.800000in}}%
\pgfusepath{clip}%
\pgfsetrectcap%
\pgfsetroundjoin%
\pgfsetlinewidth{1.003750pt}%
\definecolor{currentstroke}{rgb}{0.000000,0.000000,0.000000}%
\pgfsetstrokecolor{currentstroke}%
\pgfsetdash{}{0pt}%
\pgfpathmoveto{\pgfqpoint{2.690625in}{4.613233in}}%
\pgfpathlineto{\pgfqpoint{2.759375in}{4.613233in}}%
\pgfusepath{stroke}%
\end{pgfscope}%
\begin{pgfscope}%
\pgfpathrectangle{\pgfqpoint{0.800000in}{0.720000in}}{\pgfqpoint{4.400000in}{4.800000in}}%
\pgfusepath{clip}%
\pgfsetrectcap%
\pgfsetroundjoin%
\pgfsetlinewidth{1.003750pt}%
\definecolor{currentstroke}{rgb}{0.000000,0.000000,0.000000}%
\pgfsetstrokecolor{currentstroke}%
\pgfsetdash{}{0pt}%
\pgfpathmoveto{\pgfqpoint{3.000000in}{2.063607in}}%
\pgfpathlineto{\pgfqpoint{3.000000in}{1.449237in}}%
\pgfusepath{stroke}%
\end{pgfscope}%
\begin{pgfscope}%
\pgfpathrectangle{\pgfqpoint{0.800000in}{0.720000in}}{\pgfqpoint{4.400000in}{4.800000in}}%
\pgfusepath{clip}%
\pgfsetrectcap%
\pgfsetroundjoin%
\pgfsetlinewidth{1.003750pt}%
\definecolor{currentstroke}{rgb}{0.000000,0.000000,0.000000}%
\pgfsetstrokecolor{currentstroke}%
\pgfsetdash{}{0pt}%
\pgfpathmoveto{\pgfqpoint{3.000000in}{2.960546in}}%
\pgfpathlineto{\pgfqpoint{3.000000in}{4.288691in}}%
\pgfusepath{stroke}%
\end{pgfscope}%
\begin{pgfscope}%
\pgfpathrectangle{\pgfqpoint{0.800000in}{0.720000in}}{\pgfqpoint{4.400000in}{4.800000in}}%
\pgfusepath{clip}%
\pgfsetrectcap%
\pgfsetroundjoin%
\pgfsetlinewidth{1.003750pt}%
\definecolor{currentstroke}{rgb}{0.000000,0.000000,0.000000}%
\pgfsetstrokecolor{currentstroke}%
\pgfsetdash{}{0pt}%
\pgfpathmoveto{\pgfqpoint{2.965625in}{1.449237in}}%
\pgfpathlineto{\pgfqpoint{3.034375in}{1.449237in}}%
\pgfusepath{stroke}%
\end{pgfscope}%
\begin{pgfscope}%
\pgfpathrectangle{\pgfqpoint{0.800000in}{0.720000in}}{\pgfqpoint{4.400000in}{4.800000in}}%
\pgfusepath{clip}%
\pgfsetrectcap%
\pgfsetroundjoin%
\pgfsetlinewidth{1.003750pt}%
\definecolor{currentstroke}{rgb}{0.000000,0.000000,0.000000}%
\pgfsetstrokecolor{currentstroke}%
\pgfsetdash{}{0pt}%
\pgfpathmoveto{\pgfqpoint{2.965625in}{4.288691in}}%
\pgfpathlineto{\pgfqpoint{3.034375in}{4.288691in}}%
\pgfusepath{stroke}%
\end{pgfscope}%
\begin{pgfscope}%
\pgfpathrectangle{\pgfqpoint{0.800000in}{0.720000in}}{\pgfqpoint{4.400000in}{4.800000in}}%
\pgfusepath{clip}%
\pgfsetrectcap%
\pgfsetroundjoin%
\pgfsetlinewidth{1.003750pt}%
\definecolor{currentstroke}{rgb}{0.000000,0.000000,0.000000}%
\pgfsetstrokecolor{currentstroke}%
\pgfsetdash{}{0pt}%
\pgfpathmoveto{\pgfqpoint{3.275000in}{2.067746in}}%
\pgfpathlineto{\pgfqpoint{3.275000in}{1.582132in}}%
\pgfusepath{stroke}%
\end{pgfscope}%
\begin{pgfscope}%
\pgfpathrectangle{\pgfqpoint{0.800000in}{0.720000in}}{\pgfqpoint{4.400000in}{4.800000in}}%
\pgfusepath{clip}%
\pgfsetrectcap%
\pgfsetroundjoin%
\pgfsetlinewidth{1.003750pt}%
\definecolor{currentstroke}{rgb}{0.000000,0.000000,0.000000}%
\pgfsetstrokecolor{currentstroke}%
\pgfsetdash{}{0pt}%
\pgfpathmoveto{\pgfqpoint{3.275000in}{2.838840in}}%
\pgfpathlineto{\pgfqpoint{3.275000in}{3.958049in}}%
\pgfusepath{stroke}%
\end{pgfscope}%
\begin{pgfscope}%
\pgfpathrectangle{\pgfqpoint{0.800000in}{0.720000in}}{\pgfqpoint{4.400000in}{4.800000in}}%
\pgfusepath{clip}%
\pgfsetrectcap%
\pgfsetroundjoin%
\pgfsetlinewidth{1.003750pt}%
\definecolor{currentstroke}{rgb}{0.000000,0.000000,0.000000}%
\pgfsetstrokecolor{currentstroke}%
\pgfsetdash{}{0pt}%
\pgfpathmoveto{\pgfqpoint{3.240625in}{1.582132in}}%
\pgfpathlineto{\pgfqpoint{3.309375in}{1.582132in}}%
\pgfusepath{stroke}%
\end{pgfscope}%
\begin{pgfscope}%
\pgfpathrectangle{\pgfqpoint{0.800000in}{0.720000in}}{\pgfqpoint{4.400000in}{4.800000in}}%
\pgfusepath{clip}%
\pgfsetrectcap%
\pgfsetroundjoin%
\pgfsetlinewidth{1.003750pt}%
\definecolor{currentstroke}{rgb}{0.000000,0.000000,0.000000}%
\pgfsetstrokecolor{currentstroke}%
\pgfsetdash{}{0pt}%
\pgfpathmoveto{\pgfqpoint{3.240625in}{3.958049in}}%
\pgfpathlineto{\pgfqpoint{3.309375in}{3.958049in}}%
\pgfusepath{stroke}%
\end{pgfscope}%
\begin{pgfscope}%
\pgfpathrectangle{\pgfqpoint{0.800000in}{0.720000in}}{\pgfqpoint{4.400000in}{4.800000in}}%
\pgfusepath{clip}%
\pgfsetrectcap%
\pgfsetroundjoin%
\pgfsetlinewidth{1.003750pt}%
\definecolor{currentstroke}{rgb}{0.000000,0.000000,0.000000}%
\pgfsetstrokecolor{currentstroke}%
\pgfsetdash{}{0pt}%
\pgfpathmoveto{\pgfqpoint{3.550000in}{1.955702in}}%
\pgfpathlineto{\pgfqpoint{3.550000in}{1.526458in}}%
\pgfusepath{stroke}%
\end{pgfscope}%
\begin{pgfscope}%
\pgfpathrectangle{\pgfqpoint{0.800000in}{0.720000in}}{\pgfqpoint{4.400000in}{4.800000in}}%
\pgfusepath{clip}%
\pgfsetrectcap%
\pgfsetroundjoin%
\pgfsetlinewidth{1.003750pt}%
\definecolor{currentstroke}{rgb}{0.000000,0.000000,0.000000}%
\pgfsetstrokecolor{currentstroke}%
\pgfsetdash{}{0pt}%
\pgfpathmoveto{\pgfqpoint{3.550000in}{3.103282in}}%
\pgfpathlineto{\pgfqpoint{3.550000in}{4.398040in}}%
\pgfusepath{stroke}%
\end{pgfscope}%
\begin{pgfscope}%
\pgfpathrectangle{\pgfqpoint{0.800000in}{0.720000in}}{\pgfqpoint{4.400000in}{4.800000in}}%
\pgfusepath{clip}%
\pgfsetrectcap%
\pgfsetroundjoin%
\pgfsetlinewidth{1.003750pt}%
\definecolor{currentstroke}{rgb}{0.000000,0.000000,0.000000}%
\pgfsetstrokecolor{currentstroke}%
\pgfsetdash{}{0pt}%
\pgfpathmoveto{\pgfqpoint{3.515625in}{1.526458in}}%
\pgfpathlineto{\pgfqpoint{3.584375in}{1.526458in}}%
\pgfusepath{stroke}%
\end{pgfscope}%
\begin{pgfscope}%
\pgfpathrectangle{\pgfqpoint{0.800000in}{0.720000in}}{\pgfqpoint{4.400000in}{4.800000in}}%
\pgfusepath{clip}%
\pgfsetrectcap%
\pgfsetroundjoin%
\pgfsetlinewidth{1.003750pt}%
\definecolor{currentstroke}{rgb}{0.000000,0.000000,0.000000}%
\pgfsetstrokecolor{currentstroke}%
\pgfsetdash{}{0pt}%
\pgfpathmoveto{\pgfqpoint{3.515625in}{4.398040in}}%
\pgfpathlineto{\pgfqpoint{3.584375in}{4.398040in}}%
\pgfusepath{stroke}%
\end{pgfscope}%
\begin{pgfscope}%
\pgfpathrectangle{\pgfqpoint{0.800000in}{0.720000in}}{\pgfqpoint{4.400000in}{4.800000in}}%
\pgfusepath{clip}%
\pgfsetrectcap%
\pgfsetroundjoin%
\pgfsetlinewidth{1.003750pt}%
\definecolor{currentstroke}{rgb}{0.000000,0.000000,0.000000}%
\pgfsetstrokecolor{currentstroke}%
\pgfsetdash{}{0pt}%
\pgfpathmoveto{\pgfqpoint{3.825000in}{1.910933in}}%
\pgfpathlineto{\pgfqpoint{3.825000in}{1.703481in}}%
\pgfusepath{stroke}%
\end{pgfscope}%
\begin{pgfscope}%
\pgfpathrectangle{\pgfqpoint{0.800000in}{0.720000in}}{\pgfqpoint{4.400000in}{4.800000in}}%
\pgfusepath{clip}%
\pgfsetrectcap%
\pgfsetroundjoin%
\pgfsetlinewidth{1.003750pt}%
\definecolor{currentstroke}{rgb}{0.000000,0.000000,0.000000}%
\pgfsetstrokecolor{currentstroke}%
\pgfsetdash{}{0pt}%
\pgfpathmoveto{\pgfqpoint{3.825000in}{2.383939in}}%
\pgfpathlineto{\pgfqpoint{3.825000in}{2.711145in}}%
\pgfusepath{stroke}%
\end{pgfscope}%
\begin{pgfscope}%
\pgfpathrectangle{\pgfqpoint{0.800000in}{0.720000in}}{\pgfqpoint{4.400000in}{4.800000in}}%
\pgfusepath{clip}%
\pgfsetrectcap%
\pgfsetroundjoin%
\pgfsetlinewidth{1.003750pt}%
\definecolor{currentstroke}{rgb}{0.000000,0.000000,0.000000}%
\pgfsetstrokecolor{currentstroke}%
\pgfsetdash{}{0pt}%
\pgfpathmoveto{\pgfqpoint{3.790625in}{1.703481in}}%
\pgfpathlineto{\pgfqpoint{3.859375in}{1.703481in}}%
\pgfusepath{stroke}%
\end{pgfscope}%
\begin{pgfscope}%
\pgfpathrectangle{\pgfqpoint{0.800000in}{0.720000in}}{\pgfqpoint{4.400000in}{4.800000in}}%
\pgfusepath{clip}%
\pgfsetrectcap%
\pgfsetroundjoin%
\pgfsetlinewidth{1.003750pt}%
\definecolor{currentstroke}{rgb}{0.000000,0.000000,0.000000}%
\pgfsetstrokecolor{currentstroke}%
\pgfsetdash{}{0pt}%
\pgfpathmoveto{\pgfqpoint{3.790625in}{2.711145in}}%
\pgfpathlineto{\pgfqpoint{3.859375in}{2.711145in}}%
\pgfusepath{stroke}%
\end{pgfscope}%
\begin{pgfscope}%
\pgfpathrectangle{\pgfqpoint{0.800000in}{0.720000in}}{\pgfqpoint{4.400000in}{4.800000in}}%
\pgfusepath{clip}%
\pgfsetrectcap%
\pgfsetroundjoin%
\pgfsetlinewidth{1.003750pt}%
\definecolor{currentstroke}{rgb}{0.000000,0.000000,0.000000}%
\pgfsetstrokecolor{currentstroke}%
\pgfsetdash{}{0pt}%
\pgfpathmoveto{\pgfqpoint{4.100000in}{2.065336in}}%
\pgfpathlineto{\pgfqpoint{4.100000in}{1.884212in}}%
\pgfusepath{stroke}%
\end{pgfscope}%
\begin{pgfscope}%
\pgfpathrectangle{\pgfqpoint{0.800000in}{0.720000in}}{\pgfqpoint{4.400000in}{4.800000in}}%
\pgfusepath{clip}%
\pgfsetrectcap%
\pgfsetroundjoin%
\pgfsetlinewidth{1.003750pt}%
\definecolor{currentstroke}{rgb}{0.000000,0.000000,0.000000}%
\pgfsetstrokecolor{currentstroke}%
\pgfsetdash{}{0pt}%
\pgfpathmoveto{\pgfqpoint{4.100000in}{2.549325in}}%
\pgfpathlineto{\pgfqpoint{4.100000in}{2.697450in}}%
\pgfusepath{stroke}%
\end{pgfscope}%
\begin{pgfscope}%
\pgfpathrectangle{\pgfqpoint{0.800000in}{0.720000in}}{\pgfqpoint{4.400000in}{4.800000in}}%
\pgfusepath{clip}%
\pgfsetrectcap%
\pgfsetroundjoin%
\pgfsetlinewidth{1.003750pt}%
\definecolor{currentstroke}{rgb}{0.000000,0.000000,0.000000}%
\pgfsetstrokecolor{currentstroke}%
\pgfsetdash{}{0pt}%
\pgfpathmoveto{\pgfqpoint{4.065625in}{1.884212in}}%
\pgfpathlineto{\pgfqpoint{4.134375in}{1.884212in}}%
\pgfusepath{stroke}%
\end{pgfscope}%
\begin{pgfscope}%
\pgfpathrectangle{\pgfqpoint{0.800000in}{0.720000in}}{\pgfqpoint{4.400000in}{4.800000in}}%
\pgfusepath{clip}%
\pgfsetrectcap%
\pgfsetroundjoin%
\pgfsetlinewidth{1.003750pt}%
\definecolor{currentstroke}{rgb}{0.000000,0.000000,0.000000}%
\pgfsetstrokecolor{currentstroke}%
\pgfsetdash{}{0pt}%
\pgfpathmoveto{\pgfqpoint{4.065625in}{2.697450in}}%
\pgfpathlineto{\pgfqpoint{4.134375in}{2.697450in}}%
\pgfusepath{stroke}%
\end{pgfscope}%
\begin{pgfscope}%
\pgfpathrectangle{\pgfqpoint{0.800000in}{0.720000in}}{\pgfqpoint{4.400000in}{4.800000in}}%
\pgfusepath{clip}%
\pgfsetrectcap%
\pgfsetroundjoin%
\pgfsetlinewidth{1.003750pt}%
\definecolor{currentstroke}{rgb}{0.000000,0.000000,0.000000}%
\pgfsetstrokecolor{currentstroke}%
\pgfsetdash{}{0pt}%
\pgfpathmoveto{\pgfqpoint{4.375000in}{2.612017in}}%
\pgfpathlineto{\pgfqpoint{4.375000in}{2.612017in}}%
\pgfusepath{stroke}%
\end{pgfscope}%
\begin{pgfscope}%
\pgfpathrectangle{\pgfqpoint{0.800000in}{0.720000in}}{\pgfqpoint{4.400000in}{4.800000in}}%
\pgfusepath{clip}%
\pgfsetrectcap%
\pgfsetroundjoin%
\pgfsetlinewidth{1.003750pt}%
\definecolor{currentstroke}{rgb}{0.000000,0.000000,0.000000}%
\pgfsetstrokecolor{currentstroke}%
\pgfsetdash{}{0pt}%
\pgfpathmoveto{\pgfqpoint{4.375000in}{2.612017in}}%
\pgfpathlineto{\pgfqpoint{4.375000in}{2.612017in}}%
\pgfusepath{stroke}%
\end{pgfscope}%
\begin{pgfscope}%
\pgfpathrectangle{\pgfqpoint{0.800000in}{0.720000in}}{\pgfqpoint{4.400000in}{4.800000in}}%
\pgfusepath{clip}%
\pgfsetrectcap%
\pgfsetroundjoin%
\pgfsetlinewidth{1.003750pt}%
\definecolor{currentstroke}{rgb}{0.000000,0.000000,0.000000}%
\pgfsetstrokecolor{currentstroke}%
\pgfsetdash{}{0pt}%
\pgfpathmoveto{\pgfqpoint{4.340625in}{2.612017in}}%
\pgfpathlineto{\pgfqpoint{4.409375in}{2.612017in}}%
\pgfusepath{stroke}%
\end{pgfscope}%
\begin{pgfscope}%
\pgfpathrectangle{\pgfqpoint{0.800000in}{0.720000in}}{\pgfqpoint{4.400000in}{4.800000in}}%
\pgfusepath{clip}%
\pgfsetrectcap%
\pgfsetroundjoin%
\pgfsetlinewidth{1.003750pt}%
\definecolor{currentstroke}{rgb}{0.000000,0.000000,0.000000}%
\pgfsetstrokecolor{currentstroke}%
\pgfsetdash{}{0pt}%
\pgfpathmoveto{\pgfqpoint{4.340625in}{2.612017in}}%
\pgfpathlineto{\pgfqpoint{4.409375in}{2.612017in}}%
\pgfusepath{stroke}%
\end{pgfscope}%
\begin{pgfscope}%
\pgfpathrectangle{\pgfqpoint{0.800000in}{0.720000in}}{\pgfqpoint{4.400000in}{4.800000in}}%
\pgfusepath{clip}%
\pgfsetrectcap%
\pgfsetroundjoin%
\pgfsetlinewidth{1.003750pt}%
\definecolor{currentstroke}{rgb}{0.000000,0.000000,0.000000}%
\pgfsetstrokecolor{currentstroke}%
\pgfsetdash{}{0pt}%
\pgfpathmoveto{\pgfqpoint{4.650000in}{3.219781in}}%
\pgfpathlineto{\pgfqpoint{4.650000in}{3.219781in}}%
\pgfusepath{stroke}%
\end{pgfscope}%
\begin{pgfscope}%
\pgfpathrectangle{\pgfqpoint{0.800000in}{0.720000in}}{\pgfqpoint{4.400000in}{4.800000in}}%
\pgfusepath{clip}%
\pgfsetrectcap%
\pgfsetroundjoin%
\pgfsetlinewidth{1.003750pt}%
\definecolor{currentstroke}{rgb}{0.000000,0.000000,0.000000}%
\pgfsetstrokecolor{currentstroke}%
\pgfsetdash{}{0pt}%
\pgfpathmoveto{\pgfqpoint{4.650000in}{3.219781in}}%
\pgfpathlineto{\pgfqpoint{4.650000in}{3.219781in}}%
\pgfusepath{stroke}%
\end{pgfscope}%
\begin{pgfscope}%
\pgfpathrectangle{\pgfqpoint{0.800000in}{0.720000in}}{\pgfqpoint{4.400000in}{4.800000in}}%
\pgfusepath{clip}%
\pgfsetrectcap%
\pgfsetroundjoin%
\pgfsetlinewidth{1.003750pt}%
\definecolor{currentstroke}{rgb}{0.000000,0.000000,0.000000}%
\pgfsetstrokecolor{currentstroke}%
\pgfsetdash{}{0pt}%
\pgfpathmoveto{\pgfqpoint{4.615625in}{3.219781in}}%
\pgfpathlineto{\pgfqpoint{4.684375in}{3.219781in}}%
\pgfusepath{stroke}%
\end{pgfscope}%
\begin{pgfscope}%
\pgfpathrectangle{\pgfqpoint{0.800000in}{0.720000in}}{\pgfqpoint{4.400000in}{4.800000in}}%
\pgfusepath{clip}%
\pgfsetrectcap%
\pgfsetroundjoin%
\pgfsetlinewidth{1.003750pt}%
\definecolor{currentstroke}{rgb}{0.000000,0.000000,0.000000}%
\pgfsetstrokecolor{currentstroke}%
\pgfsetdash{}{0pt}%
\pgfpathmoveto{\pgfqpoint{4.615625in}{3.219781in}}%
\pgfpathlineto{\pgfqpoint{4.684375in}{3.219781in}}%
\pgfusepath{stroke}%
\end{pgfscope}%
\begin{pgfscope}%
\pgfpathrectangle{\pgfqpoint{0.800000in}{0.720000in}}{\pgfqpoint{4.400000in}{4.800000in}}%
\pgfusepath{clip}%
\pgfsetrectcap%
\pgfsetroundjoin%
\pgfsetlinewidth{1.003750pt}%
\definecolor{currentstroke}{rgb}{0.000000,0.000000,0.000000}%
\pgfsetstrokecolor{currentstroke}%
\pgfsetdash{}{0pt}%
\pgfpathmoveto{\pgfqpoint{4.925000in}{4.311097in}}%
\pgfpathlineto{\pgfqpoint{4.925000in}{4.311097in}}%
\pgfusepath{stroke}%
\end{pgfscope}%
\begin{pgfscope}%
\pgfpathrectangle{\pgfqpoint{0.800000in}{0.720000in}}{\pgfqpoint{4.400000in}{4.800000in}}%
\pgfusepath{clip}%
\pgfsetrectcap%
\pgfsetroundjoin%
\pgfsetlinewidth{1.003750pt}%
\definecolor{currentstroke}{rgb}{0.000000,0.000000,0.000000}%
\pgfsetstrokecolor{currentstroke}%
\pgfsetdash{}{0pt}%
\pgfpathmoveto{\pgfqpoint{4.925000in}{4.311097in}}%
\pgfpathlineto{\pgfqpoint{4.925000in}{4.311097in}}%
\pgfusepath{stroke}%
\end{pgfscope}%
\begin{pgfscope}%
\pgfpathrectangle{\pgfqpoint{0.800000in}{0.720000in}}{\pgfqpoint{4.400000in}{4.800000in}}%
\pgfusepath{clip}%
\pgfsetrectcap%
\pgfsetroundjoin%
\pgfsetlinewidth{1.003750pt}%
\definecolor{currentstroke}{rgb}{0.000000,0.000000,0.000000}%
\pgfsetstrokecolor{currentstroke}%
\pgfsetdash{}{0pt}%
\pgfpathmoveto{\pgfqpoint{4.890625in}{4.311097in}}%
\pgfpathlineto{\pgfqpoint{4.959375in}{4.311097in}}%
\pgfusepath{stroke}%
\end{pgfscope}%
\begin{pgfscope}%
\pgfpathrectangle{\pgfqpoint{0.800000in}{0.720000in}}{\pgfqpoint{4.400000in}{4.800000in}}%
\pgfusepath{clip}%
\pgfsetrectcap%
\pgfsetroundjoin%
\pgfsetlinewidth{1.003750pt}%
\definecolor{currentstroke}{rgb}{0.000000,0.000000,0.000000}%
\pgfsetstrokecolor{currentstroke}%
\pgfsetdash{}{0pt}%
\pgfpathmoveto{\pgfqpoint{4.890625in}{4.311097in}}%
\pgfpathlineto{\pgfqpoint{4.959375in}{4.311097in}}%
\pgfusepath{stroke}%
\end{pgfscope}%
\begin{pgfscope}%
\pgfpathrectangle{\pgfqpoint{0.800000in}{0.720000in}}{\pgfqpoint{4.400000in}{4.800000in}}%
\pgfusepath{clip}%
\pgfsetrectcap%
\pgfsetroundjoin%
\pgfsetlinewidth{2.007500pt}%
\definecolor{currentstroke}{rgb}{0.121569,0.466667,0.705882}%
\pgfsetstrokecolor{currentstroke}%
\pgfsetdash{}{0pt}%
\pgfpathmoveto{\pgfqpoint{1.075000in}{1.495020in}}%
\pgfpathlineto{\pgfqpoint{1.350000in}{2.119255in}}%
\pgfpathlineto{\pgfqpoint{1.625000in}{2.372613in}}%
\pgfpathlineto{\pgfqpoint{1.900000in}{2.402056in}}%
\pgfpathlineto{\pgfqpoint{2.175000in}{2.475098in}}%
\pgfpathlineto{\pgfqpoint{2.450000in}{2.412981in}}%
\pgfpathlineto{\pgfqpoint{2.725000in}{2.424576in}}%
\pgfpathlineto{\pgfqpoint{3.000000in}{2.419029in}}%
\pgfpathlineto{\pgfqpoint{3.275000in}{2.391766in}}%
\pgfpathlineto{\pgfqpoint{3.550000in}{2.204143in}}%
\pgfpathlineto{\pgfqpoint{3.825000in}{2.226955in}}%
\pgfpathlineto{\pgfqpoint{4.100000in}{2.319927in}}%
\pgfpathlineto{\pgfqpoint{4.375000in}{2.612017in}}%
\pgfpathlineto{\pgfqpoint{4.650000in}{3.219781in}}%
\pgfpathlineto{\pgfqpoint{4.925000in}{4.311097in}}%
\pgfusepath{stroke}%
\end{pgfscope}%
\begin{pgfscope}%
\pgfpathrectangle{\pgfqpoint{0.800000in}{0.720000in}}{\pgfqpoint{4.400000in}{4.800000in}}%
\pgfusepath{clip}%
\pgfsetbuttcap%
\pgfsetmiterjoin%
\definecolor{currentfill}{rgb}{0.121569,0.466667,0.705882}%
\pgfsetfillcolor{currentfill}%
\pgfsetlinewidth{1.003750pt}%
\definecolor{currentstroke}{rgb}{0.000000,0.000000,0.000000}%
\pgfsetstrokecolor{currentstroke}%
\pgfsetdash{}{0pt}%
\pgfpathmoveto{\pgfqpoint{1.006250in}{1.137443in}}%
\pgfpathlineto{\pgfqpoint{1.143750in}{1.137443in}}%
\pgfpathlineto{\pgfqpoint{1.143750in}{1.602816in}}%
\pgfpathlineto{\pgfqpoint{1.006250in}{1.602816in}}%
\pgfpathlineto{\pgfqpoint{1.006250in}{1.137443in}}%
\pgfpathclose%
\pgfusepath{stroke,fill}%
\end{pgfscope}%
\begin{pgfscope}%
\pgfpathrectangle{\pgfqpoint{0.800000in}{0.720000in}}{\pgfqpoint{4.400000in}{4.800000in}}%
\pgfusepath{clip}%
\pgfsetbuttcap%
\pgfsetmiterjoin%
\definecolor{currentfill}{rgb}{0.121569,0.466667,0.705882}%
\pgfsetfillcolor{currentfill}%
\pgfsetlinewidth{1.003750pt}%
\definecolor{currentstroke}{rgb}{0.000000,0.000000,0.000000}%
\pgfsetstrokecolor{currentstroke}%
\pgfsetdash{}{0pt}%
\pgfpathmoveto{\pgfqpoint{1.281250in}{1.687260in}}%
\pgfpathlineto{\pgfqpoint{1.418750in}{1.687260in}}%
\pgfpathlineto{\pgfqpoint{1.418750in}{2.997316in}}%
\pgfpathlineto{\pgfqpoint{1.281250in}{2.997316in}}%
\pgfpathlineto{\pgfqpoint{1.281250in}{1.687260in}}%
\pgfpathclose%
\pgfusepath{stroke,fill}%
\end{pgfscope}%
\begin{pgfscope}%
\pgfpathrectangle{\pgfqpoint{0.800000in}{0.720000in}}{\pgfqpoint{4.400000in}{4.800000in}}%
\pgfusepath{clip}%
\pgfsetbuttcap%
\pgfsetmiterjoin%
\definecolor{currentfill}{rgb}{0.121569,0.466667,0.705882}%
\pgfsetfillcolor{currentfill}%
\pgfsetlinewidth{1.003750pt}%
\definecolor{currentstroke}{rgb}{0.000000,0.000000,0.000000}%
\pgfsetstrokecolor{currentstroke}%
\pgfsetdash{}{0pt}%
\pgfpathmoveto{\pgfqpoint{1.556250in}{1.867618in}}%
\pgfpathlineto{\pgfqpoint{1.693750in}{1.867618in}}%
\pgfpathlineto{\pgfqpoint{1.693750in}{3.247630in}}%
\pgfpathlineto{\pgfqpoint{1.556250in}{3.247630in}}%
\pgfpathlineto{\pgfqpoint{1.556250in}{1.867618in}}%
\pgfpathclose%
\pgfusepath{stroke,fill}%
\end{pgfscope}%
\begin{pgfscope}%
\pgfpathrectangle{\pgfqpoint{0.800000in}{0.720000in}}{\pgfqpoint{4.400000in}{4.800000in}}%
\pgfusepath{clip}%
\pgfsetbuttcap%
\pgfsetmiterjoin%
\definecolor{currentfill}{rgb}{0.121569,0.466667,0.705882}%
\pgfsetfillcolor{currentfill}%
\pgfsetlinewidth{1.003750pt}%
\definecolor{currentstroke}{rgb}{0.000000,0.000000,0.000000}%
\pgfsetstrokecolor{currentstroke}%
\pgfsetdash{}{0pt}%
\pgfpathmoveto{\pgfqpoint{1.831250in}{1.935387in}}%
\pgfpathlineto{\pgfqpoint{1.968750in}{1.935387in}}%
\pgfpathlineto{\pgfqpoint{1.968750in}{3.263401in}}%
\pgfpathlineto{\pgfqpoint{1.831250in}{3.263401in}}%
\pgfpathlineto{\pgfqpoint{1.831250in}{1.935387in}}%
\pgfpathclose%
\pgfusepath{stroke,fill}%
\end{pgfscope}%
\begin{pgfscope}%
\pgfpathrectangle{\pgfqpoint{0.800000in}{0.720000in}}{\pgfqpoint{4.400000in}{4.800000in}}%
\pgfusepath{clip}%
\pgfsetbuttcap%
\pgfsetmiterjoin%
\definecolor{currentfill}{rgb}{0.121569,0.466667,0.705882}%
\pgfsetfillcolor{currentfill}%
\pgfsetlinewidth{1.003750pt}%
\definecolor{currentstroke}{rgb}{0.000000,0.000000,0.000000}%
\pgfsetstrokecolor{currentstroke}%
\pgfsetdash{}{0pt}%
\pgfpathmoveto{\pgfqpoint{2.106250in}{2.042205in}}%
\pgfpathlineto{\pgfqpoint{2.243750in}{2.042205in}}%
\pgfpathlineto{\pgfqpoint{2.243750in}{3.178667in}}%
\pgfpathlineto{\pgfqpoint{2.106250in}{3.178667in}}%
\pgfpathlineto{\pgfqpoint{2.106250in}{2.042205in}}%
\pgfpathclose%
\pgfusepath{stroke,fill}%
\end{pgfscope}%
\begin{pgfscope}%
\pgfpathrectangle{\pgfqpoint{0.800000in}{0.720000in}}{\pgfqpoint{4.400000in}{4.800000in}}%
\pgfusepath{clip}%
\pgfsetbuttcap%
\pgfsetmiterjoin%
\definecolor{currentfill}{rgb}{0.121569,0.466667,0.705882}%
\pgfsetfillcolor{currentfill}%
\pgfsetlinewidth{1.003750pt}%
\definecolor{currentstroke}{rgb}{0.000000,0.000000,0.000000}%
\pgfsetstrokecolor{currentstroke}%
\pgfsetdash{}{0pt}%
\pgfpathmoveto{\pgfqpoint{2.381250in}{2.037654in}}%
\pgfpathlineto{\pgfqpoint{2.518750in}{2.037654in}}%
\pgfpathlineto{\pgfqpoint{2.518750in}{3.036130in}}%
\pgfpathlineto{\pgfqpoint{2.381250in}{3.036130in}}%
\pgfpathlineto{\pgfqpoint{2.381250in}{2.037654in}}%
\pgfpathclose%
\pgfusepath{stroke,fill}%
\end{pgfscope}%
\begin{pgfscope}%
\pgfpathrectangle{\pgfqpoint{0.800000in}{0.720000in}}{\pgfqpoint{4.400000in}{4.800000in}}%
\pgfusepath{clip}%
\pgfsetbuttcap%
\pgfsetmiterjoin%
\definecolor{currentfill}{rgb}{0.121569,0.466667,0.705882}%
\pgfsetfillcolor{currentfill}%
\pgfsetlinewidth{1.003750pt}%
\definecolor{currentstroke}{rgb}{0.000000,0.000000,0.000000}%
\pgfsetstrokecolor{currentstroke}%
\pgfsetdash{}{0pt}%
\pgfpathmoveto{\pgfqpoint{2.656250in}{2.045823in}}%
\pgfpathlineto{\pgfqpoint{2.793750in}{2.045823in}}%
\pgfpathlineto{\pgfqpoint{2.793750in}{3.074083in}}%
\pgfpathlineto{\pgfqpoint{2.656250in}{3.074083in}}%
\pgfpathlineto{\pgfqpoint{2.656250in}{2.045823in}}%
\pgfpathclose%
\pgfusepath{stroke,fill}%
\end{pgfscope}%
\begin{pgfscope}%
\pgfpathrectangle{\pgfqpoint{0.800000in}{0.720000in}}{\pgfqpoint{4.400000in}{4.800000in}}%
\pgfusepath{clip}%
\pgfsetbuttcap%
\pgfsetmiterjoin%
\definecolor{currentfill}{rgb}{0.121569,0.466667,0.705882}%
\pgfsetfillcolor{currentfill}%
\pgfsetlinewidth{1.003750pt}%
\definecolor{currentstroke}{rgb}{0.000000,0.000000,0.000000}%
\pgfsetstrokecolor{currentstroke}%
\pgfsetdash{}{0pt}%
\pgfpathmoveto{\pgfqpoint{2.931250in}{2.063607in}}%
\pgfpathlineto{\pgfqpoint{3.068750in}{2.063607in}}%
\pgfpathlineto{\pgfqpoint{3.068750in}{2.960546in}}%
\pgfpathlineto{\pgfqpoint{2.931250in}{2.960546in}}%
\pgfpathlineto{\pgfqpoint{2.931250in}{2.063607in}}%
\pgfpathclose%
\pgfusepath{stroke,fill}%
\end{pgfscope}%
\begin{pgfscope}%
\pgfpathrectangle{\pgfqpoint{0.800000in}{0.720000in}}{\pgfqpoint{4.400000in}{4.800000in}}%
\pgfusepath{clip}%
\pgfsetbuttcap%
\pgfsetmiterjoin%
\definecolor{currentfill}{rgb}{0.121569,0.466667,0.705882}%
\pgfsetfillcolor{currentfill}%
\pgfsetlinewidth{1.003750pt}%
\definecolor{currentstroke}{rgb}{0.000000,0.000000,0.000000}%
\pgfsetstrokecolor{currentstroke}%
\pgfsetdash{}{0pt}%
\pgfpathmoveto{\pgfqpoint{3.206250in}{2.067746in}}%
\pgfpathlineto{\pgfqpoint{3.343750in}{2.067746in}}%
\pgfpathlineto{\pgfqpoint{3.343750in}{2.838840in}}%
\pgfpathlineto{\pgfqpoint{3.206250in}{2.838840in}}%
\pgfpathlineto{\pgfqpoint{3.206250in}{2.067746in}}%
\pgfpathclose%
\pgfusepath{stroke,fill}%
\end{pgfscope}%
\begin{pgfscope}%
\pgfpathrectangle{\pgfqpoint{0.800000in}{0.720000in}}{\pgfqpoint{4.400000in}{4.800000in}}%
\pgfusepath{clip}%
\pgfsetbuttcap%
\pgfsetmiterjoin%
\definecolor{currentfill}{rgb}{0.121569,0.466667,0.705882}%
\pgfsetfillcolor{currentfill}%
\pgfsetlinewidth{1.003750pt}%
\definecolor{currentstroke}{rgb}{0.000000,0.000000,0.000000}%
\pgfsetstrokecolor{currentstroke}%
\pgfsetdash{}{0pt}%
\pgfpathmoveto{\pgfqpoint{3.481250in}{1.955702in}}%
\pgfpathlineto{\pgfqpoint{3.618750in}{1.955702in}}%
\pgfpathlineto{\pgfqpoint{3.618750in}{3.103282in}}%
\pgfpathlineto{\pgfqpoint{3.481250in}{3.103282in}}%
\pgfpathlineto{\pgfqpoint{3.481250in}{1.955702in}}%
\pgfpathclose%
\pgfusepath{stroke,fill}%
\end{pgfscope}%
\begin{pgfscope}%
\pgfpathrectangle{\pgfqpoint{0.800000in}{0.720000in}}{\pgfqpoint{4.400000in}{4.800000in}}%
\pgfusepath{clip}%
\pgfsetbuttcap%
\pgfsetmiterjoin%
\definecolor{currentfill}{rgb}{0.121569,0.466667,0.705882}%
\pgfsetfillcolor{currentfill}%
\pgfsetlinewidth{1.003750pt}%
\definecolor{currentstroke}{rgb}{0.000000,0.000000,0.000000}%
\pgfsetstrokecolor{currentstroke}%
\pgfsetdash{}{0pt}%
\pgfpathmoveto{\pgfqpoint{3.756250in}{1.910933in}}%
\pgfpathlineto{\pgfqpoint{3.893750in}{1.910933in}}%
\pgfpathlineto{\pgfqpoint{3.893750in}{2.383939in}}%
\pgfpathlineto{\pgfqpoint{3.756250in}{2.383939in}}%
\pgfpathlineto{\pgfqpoint{3.756250in}{1.910933in}}%
\pgfpathclose%
\pgfusepath{stroke,fill}%
\end{pgfscope}%
\begin{pgfscope}%
\pgfpathrectangle{\pgfqpoint{0.800000in}{0.720000in}}{\pgfqpoint{4.400000in}{4.800000in}}%
\pgfusepath{clip}%
\pgfsetbuttcap%
\pgfsetmiterjoin%
\definecolor{currentfill}{rgb}{0.121569,0.466667,0.705882}%
\pgfsetfillcolor{currentfill}%
\pgfsetlinewidth{1.003750pt}%
\definecolor{currentstroke}{rgb}{0.000000,0.000000,0.000000}%
\pgfsetstrokecolor{currentstroke}%
\pgfsetdash{}{0pt}%
\pgfpathmoveto{\pgfqpoint{4.031250in}{2.065336in}}%
\pgfpathlineto{\pgfqpoint{4.168750in}{2.065336in}}%
\pgfpathlineto{\pgfqpoint{4.168750in}{2.549325in}}%
\pgfpathlineto{\pgfqpoint{4.031250in}{2.549325in}}%
\pgfpathlineto{\pgfqpoint{4.031250in}{2.065336in}}%
\pgfpathclose%
\pgfusepath{stroke,fill}%
\end{pgfscope}%
\begin{pgfscope}%
\pgfpathrectangle{\pgfqpoint{0.800000in}{0.720000in}}{\pgfqpoint{4.400000in}{4.800000in}}%
\pgfusepath{clip}%
\pgfsetbuttcap%
\pgfsetmiterjoin%
\definecolor{currentfill}{rgb}{0.121569,0.466667,0.705882}%
\pgfsetfillcolor{currentfill}%
\pgfsetlinewidth{1.003750pt}%
\definecolor{currentstroke}{rgb}{0.000000,0.000000,0.000000}%
\pgfsetstrokecolor{currentstroke}%
\pgfsetdash{}{0pt}%
\pgfpathmoveto{\pgfqpoint{4.306250in}{2.612017in}}%
\pgfpathlineto{\pgfqpoint{4.443750in}{2.612017in}}%
\pgfpathlineto{\pgfqpoint{4.443750in}{2.612017in}}%
\pgfpathlineto{\pgfqpoint{4.306250in}{2.612017in}}%
\pgfpathlineto{\pgfqpoint{4.306250in}{2.612017in}}%
\pgfpathclose%
\pgfusepath{stroke,fill}%
\end{pgfscope}%
\begin{pgfscope}%
\pgfpathrectangle{\pgfqpoint{0.800000in}{0.720000in}}{\pgfqpoint{4.400000in}{4.800000in}}%
\pgfusepath{clip}%
\pgfsetbuttcap%
\pgfsetmiterjoin%
\definecolor{currentfill}{rgb}{0.121569,0.466667,0.705882}%
\pgfsetfillcolor{currentfill}%
\pgfsetlinewidth{1.003750pt}%
\definecolor{currentstroke}{rgb}{0.000000,0.000000,0.000000}%
\pgfsetstrokecolor{currentstroke}%
\pgfsetdash{}{0pt}%
\pgfpathmoveto{\pgfqpoint{4.581250in}{3.219781in}}%
\pgfpathlineto{\pgfqpoint{4.718750in}{3.219781in}}%
\pgfpathlineto{\pgfqpoint{4.718750in}{3.219781in}}%
\pgfpathlineto{\pgfqpoint{4.581250in}{3.219781in}}%
\pgfpathlineto{\pgfqpoint{4.581250in}{3.219781in}}%
\pgfpathclose%
\pgfusepath{stroke,fill}%
\end{pgfscope}%
\begin{pgfscope}%
\pgfpathrectangle{\pgfqpoint{0.800000in}{0.720000in}}{\pgfqpoint{4.400000in}{4.800000in}}%
\pgfusepath{clip}%
\pgfsetbuttcap%
\pgfsetmiterjoin%
\definecolor{currentfill}{rgb}{0.121569,0.466667,0.705882}%
\pgfsetfillcolor{currentfill}%
\pgfsetlinewidth{1.003750pt}%
\definecolor{currentstroke}{rgb}{0.000000,0.000000,0.000000}%
\pgfsetstrokecolor{currentstroke}%
\pgfsetdash{}{0pt}%
\pgfpathmoveto{\pgfqpoint{4.856250in}{4.311097in}}%
\pgfpathlineto{\pgfqpoint{4.993750in}{4.311097in}}%
\pgfpathlineto{\pgfqpoint{4.993750in}{4.311097in}}%
\pgfpathlineto{\pgfqpoint{4.856250in}{4.311097in}}%
\pgfpathlineto{\pgfqpoint{4.856250in}{4.311097in}}%
\pgfpathclose%
\pgfusepath{stroke,fill}%
\end{pgfscope}%
\begin{pgfscope}%
\pgfpathrectangle{\pgfqpoint{0.800000in}{0.720000in}}{\pgfqpoint{4.400000in}{4.800000in}}%
\pgfusepath{clip}%
\pgfsetrectcap%
\pgfsetroundjoin%
\pgfsetlinewidth{1.003750pt}%
\definecolor{currentstroke}{rgb}{1.000000,0.498039,0.054902}%
\pgfsetstrokecolor{currentstroke}%
\pgfsetdash{}{0pt}%
\pgfpathmoveto{\pgfqpoint{1.006250in}{1.495020in}}%
\pgfpathlineto{\pgfqpoint{1.143750in}{1.495020in}}%
\pgfusepath{stroke}%
\end{pgfscope}%
\begin{pgfscope}%
\pgfpathrectangle{\pgfqpoint{0.800000in}{0.720000in}}{\pgfqpoint{4.400000in}{4.800000in}}%
\pgfusepath{clip}%
\pgfsetrectcap%
\pgfsetroundjoin%
\pgfsetlinewidth{1.003750pt}%
\definecolor{currentstroke}{rgb}{1.000000,0.498039,0.054902}%
\pgfsetstrokecolor{currentstroke}%
\pgfsetdash{}{0pt}%
\pgfpathmoveto{\pgfqpoint{1.281250in}{2.119255in}}%
\pgfpathlineto{\pgfqpoint{1.418750in}{2.119255in}}%
\pgfusepath{stroke}%
\end{pgfscope}%
\begin{pgfscope}%
\pgfpathrectangle{\pgfqpoint{0.800000in}{0.720000in}}{\pgfqpoint{4.400000in}{4.800000in}}%
\pgfusepath{clip}%
\pgfsetrectcap%
\pgfsetroundjoin%
\pgfsetlinewidth{1.003750pt}%
\definecolor{currentstroke}{rgb}{1.000000,0.498039,0.054902}%
\pgfsetstrokecolor{currentstroke}%
\pgfsetdash{}{0pt}%
\pgfpathmoveto{\pgfqpoint{1.556250in}{2.372613in}}%
\pgfpathlineto{\pgfqpoint{1.693750in}{2.372613in}}%
\pgfusepath{stroke}%
\end{pgfscope}%
\begin{pgfscope}%
\pgfpathrectangle{\pgfqpoint{0.800000in}{0.720000in}}{\pgfqpoint{4.400000in}{4.800000in}}%
\pgfusepath{clip}%
\pgfsetrectcap%
\pgfsetroundjoin%
\pgfsetlinewidth{1.003750pt}%
\definecolor{currentstroke}{rgb}{1.000000,0.498039,0.054902}%
\pgfsetstrokecolor{currentstroke}%
\pgfsetdash{}{0pt}%
\pgfpathmoveto{\pgfqpoint{1.831250in}{2.402056in}}%
\pgfpathlineto{\pgfqpoint{1.968750in}{2.402056in}}%
\pgfusepath{stroke}%
\end{pgfscope}%
\begin{pgfscope}%
\pgfpathrectangle{\pgfqpoint{0.800000in}{0.720000in}}{\pgfqpoint{4.400000in}{4.800000in}}%
\pgfusepath{clip}%
\pgfsetrectcap%
\pgfsetroundjoin%
\pgfsetlinewidth{1.003750pt}%
\definecolor{currentstroke}{rgb}{1.000000,0.498039,0.054902}%
\pgfsetstrokecolor{currentstroke}%
\pgfsetdash{}{0pt}%
\pgfpathmoveto{\pgfqpoint{2.106250in}{2.475098in}}%
\pgfpathlineto{\pgfqpoint{2.243750in}{2.475098in}}%
\pgfusepath{stroke}%
\end{pgfscope}%
\begin{pgfscope}%
\pgfpathrectangle{\pgfqpoint{0.800000in}{0.720000in}}{\pgfqpoint{4.400000in}{4.800000in}}%
\pgfusepath{clip}%
\pgfsetrectcap%
\pgfsetroundjoin%
\pgfsetlinewidth{1.003750pt}%
\definecolor{currentstroke}{rgb}{1.000000,0.498039,0.054902}%
\pgfsetstrokecolor{currentstroke}%
\pgfsetdash{}{0pt}%
\pgfpathmoveto{\pgfqpoint{2.381250in}{2.412981in}}%
\pgfpathlineto{\pgfqpoint{2.518750in}{2.412981in}}%
\pgfusepath{stroke}%
\end{pgfscope}%
\begin{pgfscope}%
\pgfpathrectangle{\pgfqpoint{0.800000in}{0.720000in}}{\pgfqpoint{4.400000in}{4.800000in}}%
\pgfusepath{clip}%
\pgfsetrectcap%
\pgfsetroundjoin%
\pgfsetlinewidth{1.003750pt}%
\definecolor{currentstroke}{rgb}{1.000000,0.498039,0.054902}%
\pgfsetstrokecolor{currentstroke}%
\pgfsetdash{}{0pt}%
\pgfpathmoveto{\pgfqpoint{2.656250in}{2.424576in}}%
\pgfpathlineto{\pgfqpoint{2.793750in}{2.424576in}}%
\pgfusepath{stroke}%
\end{pgfscope}%
\begin{pgfscope}%
\pgfpathrectangle{\pgfqpoint{0.800000in}{0.720000in}}{\pgfqpoint{4.400000in}{4.800000in}}%
\pgfusepath{clip}%
\pgfsetrectcap%
\pgfsetroundjoin%
\pgfsetlinewidth{1.003750pt}%
\definecolor{currentstroke}{rgb}{1.000000,0.498039,0.054902}%
\pgfsetstrokecolor{currentstroke}%
\pgfsetdash{}{0pt}%
\pgfpathmoveto{\pgfqpoint{2.931250in}{2.419029in}}%
\pgfpathlineto{\pgfqpoint{3.068750in}{2.419029in}}%
\pgfusepath{stroke}%
\end{pgfscope}%
\begin{pgfscope}%
\pgfpathrectangle{\pgfqpoint{0.800000in}{0.720000in}}{\pgfqpoint{4.400000in}{4.800000in}}%
\pgfusepath{clip}%
\pgfsetrectcap%
\pgfsetroundjoin%
\pgfsetlinewidth{1.003750pt}%
\definecolor{currentstroke}{rgb}{1.000000,0.498039,0.054902}%
\pgfsetstrokecolor{currentstroke}%
\pgfsetdash{}{0pt}%
\pgfpathmoveto{\pgfqpoint{3.206250in}{2.391766in}}%
\pgfpathlineto{\pgfqpoint{3.343750in}{2.391766in}}%
\pgfusepath{stroke}%
\end{pgfscope}%
\begin{pgfscope}%
\pgfpathrectangle{\pgfqpoint{0.800000in}{0.720000in}}{\pgfqpoint{4.400000in}{4.800000in}}%
\pgfusepath{clip}%
\pgfsetrectcap%
\pgfsetroundjoin%
\pgfsetlinewidth{1.003750pt}%
\definecolor{currentstroke}{rgb}{1.000000,0.498039,0.054902}%
\pgfsetstrokecolor{currentstroke}%
\pgfsetdash{}{0pt}%
\pgfpathmoveto{\pgfqpoint{3.481250in}{2.204143in}}%
\pgfpathlineto{\pgfqpoint{3.618750in}{2.204143in}}%
\pgfusepath{stroke}%
\end{pgfscope}%
\begin{pgfscope}%
\pgfpathrectangle{\pgfqpoint{0.800000in}{0.720000in}}{\pgfqpoint{4.400000in}{4.800000in}}%
\pgfusepath{clip}%
\pgfsetrectcap%
\pgfsetroundjoin%
\pgfsetlinewidth{1.003750pt}%
\definecolor{currentstroke}{rgb}{1.000000,0.498039,0.054902}%
\pgfsetstrokecolor{currentstroke}%
\pgfsetdash{}{0pt}%
\pgfpathmoveto{\pgfqpoint{3.756250in}{2.226955in}}%
\pgfpathlineto{\pgfqpoint{3.893750in}{2.226955in}}%
\pgfusepath{stroke}%
\end{pgfscope}%
\begin{pgfscope}%
\pgfpathrectangle{\pgfqpoint{0.800000in}{0.720000in}}{\pgfqpoint{4.400000in}{4.800000in}}%
\pgfusepath{clip}%
\pgfsetrectcap%
\pgfsetroundjoin%
\pgfsetlinewidth{1.003750pt}%
\definecolor{currentstroke}{rgb}{1.000000,0.498039,0.054902}%
\pgfsetstrokecolor{currentstroke}%
\pgfsetdash{}{0pt}%
\pgfpathmoveto{\pgfqpoint{4.031250in}{2.319927in}}%
\pgfpathlineto{\pgfqpoint{4.168750in}{2.319927in}}%
\pgfusepath{stroke}%
\end{pgfscope}%
\begin{pgfscope}%
\pgfpathrectangle{\pgfqpoint{0.800000in}{0.720000in}}{\pgfqpoint{4.400000in}{4.800000in}}%
\pgfusepath{clip}%
\pgfsetrectcap%
\pgfsetroundjoin%
\pgfsetlinewidth{1.003750pt}%
\definecolor{currentstroke}{rgb}{1.000000,0.498039,0.054902}%
\pgfsetstrokecolor{currentstroke}%
\pgfsetdash{}{0pt}%
\pgfpathmoveto{\pgfqpoint{4.306250in}{2.612017in}}%
\pgfpathlineto{\pgfqpoint{4.443750in}{2.612017in}}%
\pgfusepath{stroke}%
\end{pgfscope}%
\begin{pgfscope}%
\pgfpathrectangle{\pgfqpoint{0.800000in}{0.720000in}}{\pgfqpoint{4.400000in}{4.800000in}}%
\pgfusepath{clip}%
\pgfsetrectcap%
\pgfsetroundjoin%
\pgfsetlinewidth{1.003750pt}%
\definecolor{currentstroke}{rgb}{1.000000,0.498039,0.054902}%
\pgfsetstrokecolor{currentstroke}%
\pgfsetdash{}{0pt}%
\pgfpathmoveto{\pgfqpoint{4.581250in}{3.219781in}}%
\pgfpathlineto{\pgfqpoint{4.718750in}{3.219781in}}%
\pgfusepath{stroke}%
\end{pgfscope}%
\begin{pgfscope}%
\pgfpathrectangle{\pgfqpoint{0.800000in}{0.720000in}}{\pgfqpoint{4.400000in}{4.800000in}}%
\pgfusepath{clip}%
\pgfsetrectcap%
\pgfsetroundjoin%
\pgfsetlinewidth{1.003750pt}%
\definecolor{currentstroke}{rgb}{1.000000,0.498039,0.054902}%
\pgfsetstrokecolor{currentstroke}%
\pgfsetdash{}{0pt}%
\pgfpathmoveto{\pgfqpoint{4.856250in}{4.311097in}}%
\pgfpathlineto{\pgfqpoint{4.993750in}{4.311097in}}%
\pgfusepath{stroke}%
\end{pgfscope}%
\begin{pgfscope}%
\pgfsetrectcap%
\pgfsetmiterjoin%
\pgfsetlinewidth{0.803000pt}%
\definecolor{currentstroke}{rgb}{0.000000,0.000000,0.000000}%
\pgfsetstrokecolor{currentstroke}%
\pgfsetdash{}{0pt}%
\pgfpathmoveto{\pgfqpoint{0.800000in}{0.720000in}}%
\pgfpathlineto{\pgfqpoint{0.800000in}{5.520000in}}%
\pgfusepath{stroke}%
\end{pgfscope}%
\begin{pgfscope}%
\pgfsetrectcap%
\pgfsetmiterjoin%
\pgfsetlinewidth{0.803000pt}%
\definecolor{currentstroke}{rgb}{0.000000,0.000000,0.000000}%
\pgfsetstrokecolor{currentstroke}%
\pgfsetdash{}{0pt}%
\pgfpathmoveto{\pgfqpoint{5.200000in}{0.720000in}}%
\pgfpathlineto{\pgfqpoint{5.200000in}{5.520000in}}%
\pgfusepath{stroke}%
\end{pgfscope}%
\begin{pgfscope}%
\pgfsetrectcap%
\pgfsetmiterjoin%
\pgfsetlinewidth{0.803000pt}%
\definecolor{currentstroke}{rgb}{0.000000,0.000000,0.000000}%
\pgfsetstrokecolor{currentstroke}%
\pgfsetdash{}{0pt}%
\pgfpathmoveto{\pgfqpoint{0.800000in}{0.720000in}}%
\pgfpathlineto{\pgfqpoint{5.200000in}{0.720000in}}%
\pgfusepath{stroke}%
\end{pgfscope}%
\begin{pgfscope}%
\pgfsetrectcap%
\pgfsetmiterjoin%
\pgfsetlinewidth{0.803000pt}%
\definecolor{currentstroke}{rgb}{0.000000,0.000000,0.000000}%
\pgfsetstrokecolor{currentstroke}%
\pgfsetdash{}{0pt}%
\pgfpathmoveto{\pgfqpoint{0.800000in}{5.520000in}}%
\pgfpathlineto{\pgfqpoint{5.200000in}{5.520000in}}%
\pgfusepath{stroke}%
\end{pgfscope}%
\begin{pgfscope}%
\pgfsetbuttcap%
\pgfsetmiterjoin%
\definecolor{currentfill}{rgb}{1.000000,1.000000,1.000000}%
\pgfsetfillcolor{currentfill}%
\pgfsetfillopacity{0.800000}%
\pgfsetlinewidth{1.003750pt}%
\definecolor{currentstroke}{rgb}{0.800000,0.800000,0.800000}%
\pgfsetstrokecolor{currentstroke}%
\pgfsetstrokeopacity{0.800000}%
\pgfsetdash{}{0pt}%
\pgfpathmoveto{\pgfqpoint{3.366065in}{4.902821in}}%
\pgfpathlineto{\pgfqpoint{5.005556in}{4.902821in}}%
\pgfpathquadraticcurveto{\pgfqpoint{5.061111in}{4.902821in}}{\pgfqpoint{5.061111in}{4.958377in}}%
\pgfpathlineto{\pgfqpoint{5.061111in}{5.325556in}}%
\pgfpathquadraticcurveto{\pgfqpoint{5.061111in}{5.381111in}}{\pgfqpoint{5.005556in}{5.381111in}}%
\pgfpathlineto{\pgfqpoint{3.366065in}{5.381111in}}%
\pgfpathquadraticcurveto{\pgfqpoint{3.310509in}{5.381111in}}{\pgfqpoint{3.310509in}{5.325556in}}%
\pgfpathlineto{\pgfqpoint{3.310509in}{4.958377in}}%
\pgfpathquadraticcurveto{\pgfqpoint{3.310509in}{4.902821in}}{\pgfqpoint{3.366065in}{4.902821in}}%
\pgfpathclose%
\pgfusepath{stroke,fill}%
\end{pgfscope}%
\begin{pgfscope}%
\pgfsetrectcap%
\pgfsetroundjoin%
\pgfsetlinewidth{2.007500pt}%
\definecolor{currentstroke}{rgb}{0.121569,0.466667,0.705882}%
\pgfsetstrokecolor{currentstroke}%
\pgfsetdash{}{0pt}%
\pgfpathmoveto{\pgfqpoint{3.421620in}{5.167184in}}%
\pgfpathlineto{\pgfqpoint{3.977176in}{5.167184in}}%
\pgfusepath{stroke}%
\end{pgfscope}%
\begin{pgfscope}%
\definecolor{textcolor}{rgb}{0.000000,0.000000,0.000000}%
\pgfsetstrokecolor{textcolor}%
\pgfsetfillcolor{textcolor}%
\pgftext[x=4.199398in,y=5.069962in,left,base]{\color{textcolor}\sffamily\fontsize{20.000000}{24.000000}\selectfont W-dist}%
\end{pgfscope}%
\begin{pgfscope}%
\pgfsetbuttcap%
\pgfsetmiterjoin%
\definecolor{currentfill}{rgb}{1.000000,1.000000,1.000000}%
\pgfsetfillcolor{currentfill}%
\pgfsetlinewidth{0.000000pt}%
\definecolor{currentstroke}{rgb}{0.000000,0.000000,0.000000}%
\pgfsetstrokecolor{currentstroke}%
\pgfsetstrokeopacity{0.000000}%
\pgfsetdash{}{0pt}%
\pgfpathmoveto{\pgfqpoint{5.200000in}{0.720000in}}%
\pgfpathlineto{\pgfqpoint{7.840000in}{0.720000in}}%
\pgfpathlineto{\pgfqpoint{7.840000in}{5.520000in}}%
\pgfpathlineto{\pgfqpoint{5.200000in}{5.520000in}}%
\pgfpathclose%
\pgfusepath{fill}%
\end{pgfscope}%
\begin{pgfscope}%
\pgfpathrectangle{\pgfqpoint{5.200000in}{0.720000in}}{\pgfqpoint{2.640000in}{4.800000in}}%
\pgfusepath{clip}%
\pgfsetbuttcap%
\pgfsetmiterjoin%
\definecolor{currentfill}{rgb}{0.121569,0.466667,0.705882}%
\pgfsetfillcolor{currentfill}%
\pgfsetlinewidth{0.000000pt}%
\definecolor{currentstroke}{rgb}{0.000000,0.000000,0.000000}%
\pgfsetstrokecolor{currentstroke}%
\pgfsetstrokeopacity{0.000000}%
\pgfsetdash{}{0pt}%
\pgfpathmoveto{\pgfqpoint{5.200000in}{0.720000in}}%
\pgfpathlineto{\pgfqpoint{5.412793in}{0.720000in}}%
\pgfpathlineto{\pgfqpoint{5.412793in}{0.884336in}}%
\pgfpathlineto{\pgfqpoint{5.200000in}{0.884336in}}%
\pgfpathclose%
\pgfusepath{fill}%
\end{pgfscope}%
\begin{pgfscope}%
\pgfpathrectangle{\pgfqpoint{5.200000in}{0.720000in}}{\pgfqpoint{2.640000in}{4.800000in}}%
\pgfusepath{clip}%
\pgfsetbuttcap%
\pgfsetmiterjoin%
\definecolor{currentfill}{rgb}{0.121569,0.466667,0.705882}%
\pgfsetfillcolor{currentfill}%
\pgfsetlinewidth{0.000000pt}%
\definecolor{currentstroke}{rgb}{0.000000,0.000000,0.000000}%
\pgfsetstrokecolor{currentstroke}%
\pgfsetstrokeopacity{0.000000}%
\pgfsetdash{}{0pt}%
\pgfpathmoveto{\pgfqpoint{5.200000in}{0.884336in}}%
\pgfpathlineto{\pgfqpoint{5.420484in}{0.884336in}}%
\pgfpathlineto{\pgfqpoint{5.420484in}{1.048672in}}%
\pgfpathlineto{\pgfqpoint{5.200000in}{1.048672in}}%
\pgfpathclose%
\pgfusepath{fill}%
\end{pgfscope}%
\begin{pgfscope}%
\pgfpathrectangle{\pgfqpoint{5.200000in}{0.720000in}}{\pgfqpoint{2.640000in}{4.800000in}}%
\pgfusepath{clip}%
\pgfsetbuttcap%
\pgfsetmiterjoin%
\definecolor{currentfill}{rgb}{0.121569,0.466667,0.705882}%
\pgfsetfillcolor{currentfill}%
\pgfsetlinewidth{0.000000pt}%
\definecolor{currentstroke}{rgb}{0.000000,0.000000,0.000000}%
\pgfsetstrokecolor{currentstroke}%
\pgfsetstrokeopacity{0.000000}%
\pgfsetdash{}{0pt}%
\pgfpathmoveto{\pgfqpoint{5.200000in}{1.048672in}}%
\pgfpathlineto{\pgfqpoint{5.471759in}{1.048672in}}%
\pgfpathlineto{\pgfqpoint{5.471759in}{1.213008in}}%
\pgfpathlineto{\pgfqpoint{5.200000in}{1.213008in}}%
\pgfpathclose%
\pgfusepath{fill}%
\end{pgfscope}%
\begin{pgfscope}%
\pgfpathrectangle{\pgfqpoint{5.200000in}{0.720000in}}{\pgfqpoint{2.640000in}{4.800000in}}%
\pgfusepath{clip}%
\pgfsetbuttcap%
\pgfsetmiterjoin%
\definecolor{currentfill}{rgb}{0.121569,0.466667,0.705882}%
\pgfsetfillcolor{currentfill}%
\pgfsetlinewidth{0.000000pt}%
\definecolor{currentstroke}{rgb}{0.000000,0.000000,0.000000}%
\pgfsetstrokecolor{currentstroke}%
\pgfsetstrokeopacity{0.000000}%
\pgfsetdash{}{0pt}%
\pgfpathmoveto{\pgfqpoint{5.200000in}{1.213008in}}%
\pgfpathlineto{\pgfqpoint{5.774284in}{1.213008in}}%
\pgfpathlineto{\pgfqpoint{5.774284in}{1.377343in}}%
\pgfpathlineto{\pgfqpoint{5.200000in}{1.377343in}}%
\pgfpathclose%
\pgfusepath{fill}%
\end{pgfscope}%
\begin{pgfscope}%
\pgfpathrectangle{\pgfqpoint{5.200000in}{0.720000in}}{\pgfqpoint{2.640000in}{4.800000in}}%
\pgfusepath{clip}%
\pgfsetbuttcap%
\pgfsetmiterjoin%
\definecolor{currentfill}{rgb}{0.121569,0.466667,0.705882}%
\pgfsetfillcolor{currentfill}%
\pgfsetlinewidth{0.000000pt}%
\definecolor{currentstroke}{rgb}{0.000000,0.000000,0.000000}%
\pgfsetstrokecolor{currentstroke}%
\pgfsetstrokeopacity{0.000000}%
\pgfsetdash{}{0pt}%
\pgfpathmoveto{\pgfqpoint{5.200000in}{1.377343in}}%
\pgfpathlineto{\pgfqpoint{6.412662in}{1.377343in}}%
\pgfpathlineto{\pgfqpoint{6.412662in}{1.541679in}}%
\pgfpathlineto{\pgfqpoint{5.200000in}{1.541679in}}%
\pgfpathclose%
\pgfusepath{fill}%
\end{pgfscope}%
\begin{pgfscope}%
\pgfpathrectangle{\pgfqpoint{5.200000in}{0.720000in}}{\pgfqpoint{2.640000in}{4.800000in}}%
\pgfusepath{clip}%
\pgfsetbuttcap%
\pgfsetmiterjoin%
\definecolor{currentfill}{rgb}{0.121569,0.466667,0.705882}%
\pgfsetfillcolor{currentfill}%
\pgfsetlinewidth{0.000000pt}%
\definecolor{currentstroke}{rgb}{0.000000,0.000000,0.000000}%
\pgfsetstrokecolor{currentstroke}%
\pgfsetstrokeopacity{0.000000}%
\pgfsetdash{}{0pt}%
\pgfpathmoveto{\pgfqpoint{5.200000in}{1.541679in}}%
\pgfpathlineto{\pgfqpoint{7.279215in}{1.541679in}}%
\pgfpathlineto{\pgfqpoint{7.279215in}{1.706015in}}%
\pgfpathlineto{\pgfqpoint{5.200000in}{1.706015in}}%
\pgfpathclose%
\pgfusepath{fill}%
\end{pgfscope}%
\begin{pgfscope}%
\pgfpathrectangle{\pgfqpoint{5.200000in}{0.720000in}}{\pgfqpoint{2.640000in}{4.800000in}}%
\pgfusepath{clip}%
\pgfsetbuttcap%
\pgfsetmiterjoin%
\definecolor{currentfill}{rgb}{0.121569,0.466667,0.705882}%
\pgfsetfillcolor{currentfill}%
\pgfsetlinewidth{0.000000pt}%
\definecolor{currentstroke}{rgb}{0.000000,0.000000,0.000000}%
\pgfsetstrokecolor{currentstroke}%
\pgfsetstrokeopacity{0.000000}%
\pgfsetdash{}{0pt}%
\pgfpathmoveto{\pgfqpoint{5.200000in}{1.706015in}}%
\pgfpathlineto{\pgfqpoint{7.417658in}{1.706015in}}%
\pgfpathlineto{\pgfqpoint{7.417658in}{1.870351in}}%
\pgfpathlineto{\pgfqpoint{5.200000in}{1.870351in}}%
\pgfpathclose%
\pgfusepath{fill}%
\end{pgfscope}%
\begin{pgfscope}%
\pgfpathrectangle{\pgfqpoint{5.200000in}{0.720000in}}{\pgfqpoint{2.640000in}{4.800000in}}%
\pgfusepath{clip}%
\pgfsetbuttcap%
\pgfsetmiterjoin%
\definecolor{currentfill}{rgb}{0.121569,0.466667,0.705882}%
\pgfsetfillcolor{currentfill}%
\pgfsetlinewidth{0.000000pt}%
\definecolor{currentstroke}{rgb}{0.000000,0.000000,0.000000}%
\pgfsetstrokecolor{currentstroke}%
\pgfsetstrokeopacity{0.000000}%
\pgfsetdash{}{0pt}%
\pgfpathmoveto{\pgfqpoint{5.200000in}{1.870351in}}%
\pgfpathlineto{\pgfqpoint{7.594558in}{1.870351in}}%
\pgfpathlineto{\pgfqpoint{7.594558in}{2.034687in}}%
\pgfpathlineto{\pgfqpoint{5.200000in}{2.034687in}}%
\pgfpathclose%
\pgfusepath{fill}%
\end{pgfscope}%
\begin{pgfscope}%
\pgfpathrectangle{\pgfqpoint{5.200000in}{0.720000in}}{\pgfqpoint{2.640000in}{4.800000in}}%
\pgfusepath{clip}%
\pgfsetbuttcap%
\pgfsetmiterjoin%
\definecolor{currentfill}{rgb}{0.121569,0.466667,0.705882}%
\pgfsetfillcolor{currentfill}%
\pgfsetlinewidth{0.000000pt}%
\definecolor{currentstroke}{rgb}{0.000000,0.000000,0.000000}%
\pgfsetstrokecolor{currentstroke}%
\pgfsetstrokeopacity{0.000000}%
\pgfsetdash{}{0pt}%
\pgfpathmoveto{\pgfqpoint{5.200000in}{2.034687in}}%
\pgfpathlineto{\pgfqpoint{7.440732in}{2.034687in}}%
\pgfpathlineto{\pgfqpoint{7.440732in}{2.199023in}}%
\pgfpathlineto{\pgfqpoint{5.200000in}{2.199023in}}%
\pgfpathclose%
\pgfusepath{fill}%
\end{pgfscope}%
\begin{pgfscope}%
\pgfpathrectangle{\pgfqpoint{5.200000in}{0.720000in}}{\pgfqpoint{2.640000in}{4.800000in}}%
\pgfusepath{clip}%
\pgfsetbuttcap%
\pgfsetmiterjoin%
\definecolor{currentfill}{rgb}{0.121569,0.466667,0.705882}%
\pgfsetfillcolor{currentfill}%
\pgfsetlinewidth{0.000000pt}%
\definecolor{currentstroke}{rgb}{0.000000,0.000000,0.000000}%
\pgfsetstrokecolor{currentstroke}%
\pgfsetstrokeopacity{0.000000}%
\pgfsetdash{}{0pt}%
\pgfpathmoveto{\pgfqpoint{5.200000in}{2.199023in}}%
\pgfpathlineto{\pgfqpoint{7.251013in}{2.199023in}}%
\pgfpathlineto{\pgfqpoint{7.251013in}{2.363358in}}%
\pgfpathlineto{\pgfqpoint{5.200000in}{2.363358in}}%
\pgfpathclose%
\pgfusepath{fill}%
\end{pgfscope}%
\begin{pgfscope}%
\pgfpathrectangle{\pgfqpoint{5.200000in}{0.720000in}}{\pgfqpoint{2.640000in}{4.800000in}}%
\pgfusepath{clip}%
\pgfsetbuttcap%
\pgfsetmiterjoin%
\definecolor{currentfill}{rgb}{0.121569,0.466667,0.705882}%
\pgfsetfillcolor{currentfill}%
\pgfsetlinewidth{0.000000pt}%
\definecolor{currentstroke}{rgb}{0.000000,0.000000,0.000000}%
\pgfsetstrokecolor{currentstroke}%
\pgfsetstrokeopacity{0.000000}%
\pgfsetdash{}{0pt}%
\pgfpathmoveto{\pgfqpoint{5.200000in}{2.363358in}}%
\pgfpathlineto{\pgfqpoint{7.038221in}{2.363358in}}%
\pgfpathlineto{\pgfqpoint{7.038221in}{2.527694in}}%
\pgfpathlineto{\pgfqpoint{5.200000in}{2.527694in}}%
\pgfpathclose%
\pgfusepath{fill}%
\end{pgfscope}%
\begin{pgfscope}%
\pgfpathrectangle{\pgfqpoint{5.200000in}{0.720000in}}{\pgfqpoint{2.640000in}{4.800000in}}%
\pgfusepath{clip}%
\pgfsetbuttcap%
\pgfsetmiterjoin%
\definecolor{currentfill}{rgb}{0.121569,0.466667,0.705882}%
\pgfsetfillcolor{currentfill}%
\pgfsetlinewidth{0.000000pt}%
\definecolor{currentstroke}{rgb}{0.000000,0.000000,0.000000}%
\pgfsetstrokecolor{currentstroke}%
\pgfsetstrokeopacity{0.000000}%
\pgfsetdash{}{0pt}%
\pgfpathmoveto{\pgfqpoint{5.200000in}{2.527694in}}%
\pgfpathlineto{\pgfqpoint{6.728005in}{2.527694in}}%
\pgfpathlineto{\pgfqpoint{6.728005in}{2.692030in}}%
\pgfpathlineto{\pgfqpoint{5.200000in}{2.692030in}}%
\pgfpathclose%
\pgfusepath{fill}%
\end{pgfscope}%
\begin{pgfscope}%
\pgfpathrectangle{\pgfqpoint{5.200000in}{0.720000in}}{\pgfqpoint{2.640000in}{4.800000in}}%
\pgfusepath{clip}%
\pgfsetbuttcap%
\pgfsetmiterjoin%
\definecolor{currentfill}{rgb}{0.121569,0.466667,0.705882}%
\pgfsetfillcolor{currentfill}%
\pgfsetlinewidth{0.000000pt}%
\definecolor{currentstroke}{rgb}{0.000000,0.000000,0.000000}%
\pgfsetstrokecolor{currentstroke}%
\pgfsetstrokeopacity{0.000000}%
\pgfsetdash{}{0pt}%
\pgfpathmoveto{\pgfqpoint{5.200000in}{2.692030in}}%
\pgfpathlineto{\pgfqpoint{6.374205in}{2.692030in}}%
\pgfpathlineto{\pgfqpoint{6.374205in}{2.856366in}}%
\pgfpathlineto{\pgfqpoint{5.200000in}{2.856366in}}%
\pgfpathclose%
\pgfusepath{fill}%
\end{pgfscope}%
\begin{pgfscope}%
\pgfpathrectangle{\pgfqpoint{5.200000in}{0.720000in}}{\pgfqpoint{2.640000in}{4.800000in}}%
\pgfusepath{clip}%
\pgfsetbuttcap%
\pgfsetmiterjoin%
\definecolor{currentfill}{rgb}{0.121569,0.466667,0.705882}%
\pgfsetfillcolor{currentfill}%
\pgfsetlinewidth{0.000000pt}%
\definecolor{currentstroke}{rgb}{0.000000,0.000000,0.000000}%
\pgfsetstrokecolor{currentstroke}%
\pgfsetstrokeopacity{0.000000}%
\pgfsetdash{}{0pt}%
\pgfpathmoveto{\pgfqpoint{5.200000in}{2.856366in}}%
\pgfpathlineto{\pgfqpoint{6.163976in}{2.856366in}}%
\pgfpathlineto{\pgfqpoint{6.163976in}{3.020702in}}%
\pgfpathlineto{\pgfqpoint{5.200000in}{3.020702in}}%
\pgfpathclose%
\pgfusepath{fill}%
\end{pgfscope}%
\begin{pgfscope}%
\pgfpathrectangle{\pgfqpoint{5.200000in}{0.720000in}}{\pgfqpoint{2.640000in}{4.800000in}}%
\pgfusepath{clip}%
\pgfsetbuttcap%
\pgfsetmiterjoin%
\definecolor{currentfill}{rgb}{0.121569,0.466667,0.705882}%
\pgfsetfillcolor{currentfill}%
\pgfsetlinewidth{0.000000pt}%
\definecolor{currentstroke}{rgb}{0.000000,0.000000,0.000000}%
\pgfsetstrokecolor{currentstroke}%
\pgfsetstrokeopacity{0.000000}%
\pgfsetdash{}{0pt}%
\pgfpathmoveto{\pgfqpoint{5.200000in}{3.020702in}}%
\pgfpathlineto{\pgfqpoint{6.022969in}{3.020702in}}%
\pgfpathlineto{\pgfqpoint{6.022969in}{3.185038in}}%
\pgfpathlineto{\pgfqpoint{5.200000in}{3.185038in}}%
\pgfpathclose%
\pgfusepath{fill}%
\end{pgfscope}%
\begin{pgfscope}%
\pgfpathrectangle{\pgfqpoint{5.200000in}{0.720000in}}{\pgfqpoint{2.640000in}{4.800000in}}%
\pgfusepath{clip}%
\pgfsetbuttcap%
\pgfsetmiterjoin%
\definecolor{currentfill}{rgb}{0.121569,0.466667,0.705882}%
\pgfsetfillcolor{currentfill}%
\pgfsetlinewidth{0.000000pt}%
\definecolor{currentstroke}{rgb}{0.000000,0.000000,0.000000}%
\pgfsetstrokecolor{currentstroke}%
\pgfsetstrokeopacity{0.000000}%
\pgfsetdash{}{0pt}%
\pgfpathmoveto{\pgfqpoint{5.200000in}{3.185038in}}%
\pgfpathlineto{\pgfqpoint{5.969130in}{3.185038in}}%
\pgfpathlineto{\pgfqpoint{5.969130in}{3.349373in}}%
\pgfpathlineto{\pgfqpoint{5.200000in}{3.349373in}}%
\pgfpathclose%
\pgfusepath{fill}%
\end{pgfscope}%
\begin{pgfscope}%
\pgfpathrectangle{\pgfqpoint{5.200000in}{0.720000in}}{\pgfqpoint{2.640000in}{4.800000in}}%
\pgfusepath{clip}%
\pgfsetbuttcap%
\pgfsetmiterjoin%
\definecolor{currentfill}{rgb}{0.121569,0.466667,0.705882}%
\pgfsetfillcolor{currentfill}%
\pgfsetlinewidth{0.000000pt}%
\definecolor{currentstroke}{rgb}{0.000000,0.000000,0.000000}%
\pgfsetstrokecolor{currentstroke}%
\pgfsetstrokeopacity{0.000000}%
\pgfsetdash{}{0pt}%
\pgfpathmoveto{\pgfqpoint{5.200000in}{3.349373in}}%
\pgfpathlineto{\pgfqpoint{5.858888in}{3.349373in}}%
\pgfpathlineto{\pgfqpoint{5.858888in}{3.513709in}}%
\pgfpathlineto{\pgfqpoint{5.200000in}{3.513709in}}%
\pgfpathclose%
\pgfusepath{fill}%
\end{pgfscope}%
\begin{pgfscope}%
\pgfpathrectangle{\pgfqpoint{5.200000in}{0.720000in}}{\pgfqpoint{2.640000in}{4.800000in}}%
\pgfusepath{clip}%
\pgfsetbuttcap%
\pgfsetmiterjoin%
\definecolor{currentfill}{rgb}{0.121569,0.466667,0.705882}%
\pgfsetfillcolor{currentfill}%
\pgfsetlinewidth{0.000000pt}%
\definecolor{currentstroke}{rgb}{0.000000,0.000000,0.000000}%
\pgfsetstrokecolor{currentstroke}%
\pgfsetstrokeopacity{0.000000}%
\pgfsetdash{}{0pt}%
\pgfpathmoveto{\pgfqpoint{5.200000in}{3.513709in}}%
\pgfpathlineto{\pgfqpoint{5.681988in}{3.513709in}}%
\pgfpathlineto{\pgfqpoint{5.681988in}{3.678045in}}%
\pgfpathlineto{\pgfqpoint{5.200000in}{3.678045in}}%
\pgfpathclose%
\pgfusepath{fill}%
\end{pgfscope}%
\begin{pgfscope}%
\pgfpathrectangle{\pgfqpoint{5.200000in}{0.720000in}}{\pgfqpoint{2.640000in}{4.800000in}}%
\pgfusepath{clip}%
\pgfsetbuttcap%
\pgfsetmiterjoin%
\definecolor{currentfill}{rgb}{0.121569,0.466667,0.705882}%
\pgfsetfillcolor{currentfill}%
\pgfsetlinewidth{0.000000pt}%
\definecolor{currentstroke}{rgb}{0.000000,0.000000,0.000000}%
\pgfsetstrokecolor{currentstroke}%
\pgfsetstrokeopacity{0.000000}%
\pgfsetdash{}{0pt}%
\pgfpathmoveto{\pgfqpoint{5.200000in}{3.678045in}}%
\pgfpathlineto{\pgfqpoint{5.635840in}{3.678045in}}%
\pgfpathlineto{\pgfqpoint{5.635840in}{3.842381in}}%
\pgfpathlineto{\pgfqpoint{5.200000in}{3.842381in}}%
\pgfpathclose%
\pgfusepath{fill}%
\end{pgfscope}%
\begin{pgfscope}%
\pgfpathrectangle{\pgfqpoint{5.200000in}{0.720000in}}{\pgfqpoint{2.640000in}{4.800000in}}%
\pgfusepath{clip}%
\pgfsetbuttcap%
\pgfsetmiterjoin%
\definecolor{currentfill}{rgb}{0.121569,0.466667,0.705882}%
\pgfsetfillcolor{currentfill}%
\pgfsetlinewidth{0.000000pt}%
\definecolor{currentstroke}{rgb}{0.000000,0.000000,0.000000}%
\pgfsetstrokecolor{currentstroke}%
\pgfsetstrokeopacity{0.000000}%
\pgfsetdash{}{0pt}%
\pgfpathmoveto{\pgfqpoint{5.200000in}{3.842381in}}%
\pgfpathlineto{\pgfqpoint{5.535853in}{3.842381in}}%
\pgfpathlineto{\pgfqpoint{5.535853in}{4.006717in}}%
\pgfpathlineto{\pgfqpoint{5.200000in}{4.006717in}}%
\pgfpathclose%
\pgfusepath{fill}%
\end{pgfscope}%
\begin{pgfscope}%
\pgfpathrectangle{\pgfqpoint{5.200000in}{0.720000in}}{\pgfqpoint{2.640000in}{4.800000in}}%
\pgfusepath{clip}%
\pgfsetbuttcap%
\pgfsetmiterjoin%
\definecolor{currentfill}{rgb}{0.121569,0.466667,0.705882}%
\pgfsetfillcolor{currentfill}%
\pgfsetlinewidth{0.000000pt}%
\definecolor{currentstroke}{rgb}{0.000000,0.000000,0.000000}%
\pgfsetstrokecolor{currentstroke}%
\pgfsetstrokeopacity{0.000000}%
\pgfsetdash{}{0pt}%
\pgfpathmoveto{\pgfqpoint{5.200000in}{4.006717in}}%
\pgfpathlineto{\pgfqpoint{5.487142in}{4.006717in}}%
\pgfpathlineto{\pgfqpoint{5.487142in}{4.171053in}}%
\pgfpathlineto{\pgfqpoint{5.200000in}{4.171053in}}%
\pgfpathclose%
\pgfusepath{fill}%
\end{pgfscope}%
\begin{pgfscope}%
\pgfpathrectangle{\pgfqpoint{5.200000in}{0.720000in}}{\pgfqpoint{2.640000in}{4.800000in}}%
\pgfusepath{clip}%
\pgfsetbuttcap%
\pgfsetmiterjoin%
\definecolor{currentfill}{rgb}{0.121569,0.466667,0.705882}%
\pgfsetfillcolor{currentfill}%
\pgfsetlinewidth{0.000000pt}%
\definecolor{currentstroke}{rgb}{0.000000,0.000000,0.000000}%
\pgfsetstrokecolor{currentstroke}%
\pgfsetstrokeopacity{0.000000}%
\pgfsetdash{}{0pt}%
\pgfpathmoveto{\pgfqpoint{5.200000in}{4.171053in}}%
\pgfpathlineto{\pgfqpoint{5.487142in}{4.171053in}}%
\pgfpathlineto{\pgfqpoint{5.487142in}{4.335388in}}%
\pgfpathlineto{\pgfqpoint{5.200000in}{4.335388in}}%
\pgfpathclose%
\pgfusepath{fill}%
\end{pgfscope}%
\begin{pgfscope}%
\pgfpathrectangle{\pgfqpoint{5.200000in}{0.720000in}}{\pgfqpoint{2.640000in}{4.800000in}}%
\pgfusepath{clip}%
\pgfsetbuttcap%
\pgfsetmiterjoin%
\definecolor{currentfill}{rgb}{0.121569,0.466667,0.705882}%
\pgfsetfillcolor{currentfill}%
\pgfsetlinewidth{0.000000pt}%
\definecolor{currentstroke}{rgb}{0.000000,0.000000,0.000000}%
\pgfsetstrokecolor{currentstroke}%
\pgfsetstrokeopacity{0.000000}%
\pgfsetdash{}{0pt}%
\pgfpathmoveto{\pgfqpoint{5.200000in}{4.335388in}}%
\pgfpathlineto{\pgfqpoint{5.438430in}{4.335388in}}%
\pgfpathlineto{\pgfqpoint{5.438430in}{4.499724in}}%
\pgfpathlineto{\pgfqpoint{5.200000in}{4.499724in}}%
\pgfpathclose%
\pgfusepath{fill}%
\end{pgfscope}%
\begin{pgfscope}%
\pgfpathrectangle{\pgfqpoint{5.200000in}{0.720000in}}{\pgfqpoint{2.640000in}{4.800000in}}%
\pgfusepath{clip}%
\pgfsetbuttcap%
\pgfsetmiterjoin%
\definecolor{currentfill}{rgb}{0.121569,0.466667,0.705882}%
\pgfsetfillcolor{currentfill}%
\pgfsetlinewidth{0.000000pt}%
\definecolor{currentstroke}{rgb}{0.000000,0.000000,0.000000}%
\pgfsetstrokecolor{currentstroke}%
\pgfsetstrokeopacity{0.000000}%
\pgfsetdash{}{0pt}%
\pgfpathmoveto{\pgfqpoint{5.200000in}{4.499724in}}%
\pgfpathlineto{\pgfqpoint{5.448685in}{4.499724in}}%
\pgfpathlineto{\pgfqpoint{5.448685in}{4.664060in}}%
\pgfpathlineto{\pgfqpoint{5.200000in}{4.664060in}}%
\pgfpathclose%
\pgfusepath{fill}%
\end{pgfscope}%
\begin{pgfscope}%
\pgfpathrectangle{\pgfqpoint{5.200000in}{0.720000in}}{\pgfqpoint{2.640000in}{4.800000in}}%
\pgfusepath{clip}%
\pgfsetbuttcap%
\pgfsetmiterjoin%
\definecolor{currentfill}{rgb}{0.121569,0.466667,0.705882}%
\pgfsetfillcolor{currentfill}%
\pgfsetlinewidth{0.000000pt}%
\definecolor{currentstroke}{rgb}{0.000000,0.000000,0.000000}%
\pgfsetstrokecolor{currentstroke}%
\pgfsetstrokeopacity{0.000000}%
\pgfsetdash{}{0pt}%
\pgfpathmoveto{\pgfqpoint{5.200000in}{4.664060in}}%
\pgfpathlineto{\pgfqpoint{5.369209in}{4.664060in}}%
\pgfpathlineto{\pgfqpoint{5.369209in}{4.828396in}}%
\pgfpathlineto{\pgfqpoint{5.200000in}{4.828396in}}%
\pgfpathclose%
\pgfusepath{fill}%
\end{pgfscope}%
\begin{pgfscope}%
\pgfpathrectangle{\pgfqpoint{5.200000in}{0.720000in}}{\pgfqpoint{2.640000in}{4.800000in}}%
\pgfusepath{clip}%
\pgfsetbuttcap%
\pgfsetmiterjoin%
\definecolor{currentfill}{rgb}{0.121569,0.466667,0.705882}%
\pgfsetfillcolor{currentfill}%
\pgfsetlinewidth{0.000000pt}%
\definecolor{currentstroke}{rgb}{0.000000,0.000000,0.000000}%
\pgfsetstrokecolor{currentstroke}%
\pgfsetstrokeopacity{0.000000}%
\pgfsetdash{}{0pt}%
\pgfpathmoveto{\pgfqpoint{5.200000in}{4.828396in}}%
\pgfpathlineto{\pgfqpoint{5.397410in}{4.828396in}}%
\pgfpathlineto{\pgfqpoint{5.397410in}{4.992732in}}%
\pgfpathlineto{\pgfqpoint{5.200000in}{4.992732in}}%
\pgfpathclose%
\pgfusepath{fill}%
\end{pgfscope}%
\begin{pgfscope}%
\pgfpathrectangle{\pgfqpoint{5.200000in}{0.720000in}}{\pgfqpoint{2.640000in}{4.800000in}}%
\pgfusepath{clip}%
\pgfsetbuttcap%
\pgfsetmiterjoin%
\definecolor{currentfill}{rgb}{0.121569,0.466667,0.705882}%
\pgfsetfillcolor{currentfill}%
\pgfsetlinewidth{0.000000pt}%
\definecolor{currentstroke}{rgb}{0.000000,0.000000,0.000000}%
\pgfsetstrokecolor{currentstroke}%
\pgfsetstrokeopacity{0.000000}%
\pgfsetdash{}{0pt}%
\pgfpathmoveto{\pgfqpoint{5.200000in}{4.992732in}}%
\pgfpathlineto{\pgfqpoint{5.335880in}{4.992732in}}%
\pgfpathlineto{\pgfqpoint{5.335880in}{5.157068in}}%
\pgfpathlineto{\pgfqpoint{5.200000in}{5.157068in}}%
\pgfpathclose%
\pgfusepath{fill}%
\end{pgfscope}%
\begin{pgfscope}%
\pgfpathrectangle{\pgfqpoint{5.200000in}{0.720000in}}{\pgfqpoint{2.640000in}{4.800000in}}%
\pgfusepath{clip}%
\pgfsetbuttcap%
\pgfsetmiterjoin%
\definecolor{currentfill}{rgb}{0.121569,0.466667,0.705882}%
\pgfsetfillcolor{currentfill}%
\pgfsetlinewidth{0.000000pt}%
\definecolor{currentstroke}{rgb}{0.000000,0.000000,0.000000}%
\pgfsetstrokecolor{currentstroke}%
\pgfsetstrokeopacity{0.000000}%
\pgfsetdash{}{0pt}%
\pgfpathmoveto{\pgfqpoint{5.200000in}{5.157068in}}%
\pgfpathlineto{\pgfqpoint{5.323061in}{5.157068in}}%
\pgfpathlineto{\pgfqpoint{5.323061in}{5.321403in}}%
\pgfpathlineto{\pgfqpoint{5.200000in}{5.321403in}}%
\pgfpathclose%
\pgfusepath{fill}%
\end{pgfscope}%
\begin{pgfscope}%
\pgfpathrectangle{\pgfqpoint{5.200000in}{0.720000in}}{\pgfqpoint{2.640000in}{4.800000in}}%
\pgfusepath{clip}%
\pgfsetbuttcap%
\pgfsetmiterjoin%
\definecolor{currentfill}{rgb}{0.121569,0.466667,0.705882}%
\pgfsetfillcolor{currentfill}%
\pgfsetlinewidth{0.000000pt}%
\definecolor{currentstroke}{rgb}{0.000000,0.000000,0.000000}%
\pgfsetstrokecolor{currentstroke}%
\pgfsetstrokeopacity{0.000000}%
\pgfsetdash{}{0pt}%
\pgfpathmoveto{\pgfqpoint{5.200000in}{5.321403in}}%
\pgfpathlineto{\pgfqpoint{5.302551in}{5.321403in}}%
\pgfpathlineto{\pgfqpoint{5.302551in}{5.485739in}}%
\pgfpathlineto{\pgfqpoint{5.200000in}{5.485739in}}%
\pgfpathclose%
\pgfusepath{fill}%
\end{pgfscope}%
\begin{pgfscope}%
\pgfpathrectangle{\pgfqpoint{5.200000in}{0.720000in}}{\pgfqpoint{2.640000in}{4.800000in}}%
\pgfusepath{clip}%
\pgfsetbuttcap%
\pgfsetmiterjoin%
\definecolor{currentfill}{rgb}{0.121569,0.466667,0.705882}%
\pgfsetfillcolor{currentfill}%
\pgfsetlinewidth{0.000000pt}%
\definecolor{currentstroke}{rgb}{0.000000,0.000000,0.000000}%
\pgfsetstrokecolor{currentstroke}%
\pgfsetstrokeopacity{0.000000}%
\pgfsetdash{}{0pt}%
\pgfpathmoveto{\pgfqpoint{5.200000in}{5.485739in}}%
\pgfpathlineto{\pgfqpoint{5.299987in}{5.485739in}}%
\pgfpathlineto{\pgfqpoint{5.299987in}{5.650075in}}%
\pgfpathlineto{\pgfqpoint{5.200000in}{5.650075in}}%
\pgfpathclose%
\pgfusepath{fill}%
\end{pgfscope}%
\begin{pgfscope}%
\pgfpathrectangle{\pgfqpoint{5.200000in}{0.720000in}}{\pgfqpoint{2.640000in}{4.800000in}}%
\pgfusepath{clip}%
\pgfsetbuttcap%
\pgfsetmiterjoin%
\definecolor{currentfill}{rgb}{0.121569,0.466667,0.705882}%
\pgfsetfillcolor{currentfill}%
\pgfsetlinewidth{0.000000pt}%
\definecolor{currentstroke}{rgb}{0.000000,0.000000,0.000000}%
\pgfsetstrokecolor{currentstroke}%
\pgfsetstrokeopacity{0.000000}%
\pgfsetdash{}{0pt}%
\pgfpathmoveto{\pgfqpoint{5.200000in}{5.650075in}}%
\pgfpathlineto{\pgfqpoint{5.302551in}{5.650075in}}%
\pgfpathlineto{\pgfqpoint{5.302551in}{5.814411in}}%
\pgfpathlineto{\pgfqpoint{5.200000in}{5.814411in}}%
\pgfpathclose%
\pgfusepath{fill}%
\end{pgfscope}%
\begin{pgfscope}%
\pgfpathrectangle{\pgfqpoint{5.200000in}{0.720000in}}{\pgfqpoint{2.640000in}{4.800000in}}%
\pgfusepath{clip}%
\pgfsetbuttcap%
\pgfsetmiterjoin%
\definecolor{currentfill}{rgb}{0.121569,0.466667,0.705882}%
\pgfsetfillcolor{currentfill}%
\pgfsetlinewidth{0.000000pt}%
\definecolor{currentstroke}{rgb}{0.000000,0.000000,0.000000}%
\pgfsetstrokecolor{currentstroke}%
\pgfsetstrokeopacity{0.000000}%
\pgfsetdash{}{0pt}%
\pgfpathmoveto{\pgfqpoint{5.200000in}{5.814411in}}%
\pgfpathlineto{\pgfqpoint{5.287168in}{5.814411in}}%
\pgfpathlineto{\pgfqpoint{5.287168in}{5.978747in}}%
\pgfpathlineto{\pgfqpoint{5.200000in}{5.978747in}}%
\pgfpathclose%
\pgfusepath{fill}%
\end{pgfscope}%
\begin{pgfscope}%
\pgfpathrectangle{\pgfqpoint{5.200000in}{0.720000in}}{\pgfqpoint{2.640000in}{4.800000in}}%
\pgfusepath{clip}%
\pgfsetbuttcap%
\pgfsetmiterjoin%
\definecolor{currentfill}{rgb}{0.121569,0.466667,0.705882}%
\pgfsetfillcolor{currentfill}%
\pgfsetlinewidth{0.000000pt}%
\definecolor{currentstroke}{rgb}{0.000000,0.000000,0.000000}%
\pgfsetstrokecolor{currentstroke}%
\pgfsetstrokeopacity{0.000000}%
\pgfsetdash{}{0pt}%
\pgfpathmoveto{\pgfqpoint{5.200000in}{5.978747in}}%
\pgfpathlineto{\pgfqpoint{5.284604in}{5.978747in}}%
\pgfpathlineto{\pgfqpoint{5.284604in}{6.143083in}}%
\pgfpathlineto{\pgfqpoint{5.200000in}{6.143083in}}%
\pgfpathclose%
\pgfusepath{fill}%
\end{pgfscope}%
\begin{pgfscope}%
\pgfpathrectangle{\pgfqpoint{5.200000in}{0.720000in}}{\pgfqpoint{2.640000in}{4.800000in}}%
\pgfusepath{clip}%
\pgfsetbuttcap%
\pgfsetmiterjoin%
\definecolor{currentfill}{rgb}{0.121569,0.466667,0.705882}%
\pgfsetfillcolor{currentfill}%
\pgfsetlinewidth{0.000000pt}%
\definecolor{currentstroke}{rgb}{0.000000,0.000000,0.000000}%
\pgfsetstrokecolor{currentstroke}%
\pgfsetstrokeopacity{0.000000}%
\pgfsetdash{}{0pt}%
\pgfpathmoveto{\pgfqpoint{5.200000in}{6.143083in}}%
\pgfpathlineto{\pgfqpoint{5.261530in}{6.143083in}}%
\pgfpathlineto{\pgfqpoint{5.261530in}{6.307418in}}%
\pgfpathlineto{\pgfqpoint{5.200000in}{6.307418in}}%
\pgfpathclose%
\pgfusepath{fill}%
\end{pgfscope}%
\begin{pgfscope}%
\pgfpathrectangle{\pgfqpoint{5.200000in}{0.720000in}}{\pgfqpoint{2.640000in}{4.800000in}}%
\pgfusepath{clip}%
\pgfsetbuttcap%
\pgfsetmiterjoin%
\definecolor{currentfill}{rgb}{0.121569,0.466667,0.705882}%
\pgfsetfillcolor{currentfill}%
\pgfsetlinewidth{0.000000pt}%
\definecolor{currentstroke}{rgb}{0.000000,0.000000,0.000000}%
\pgfsetstrokecolor{currentstroke}%
\pgfsetstrokeopacity{0.000000}%
\pgfsetdash{}{0pt}%
\pgfpathmoveto{\pgfqpoint{5.200000in}{6.307418in}}%
\pgfpathlineto{\pgfqpoint{5.279477in}{6.307418in}}%
\pgfpathlineto{\pgfqpoint{5.279477in}{6.471754in}}%
\pgfpathlineto{\pgfqpoint{5.200000in}{6.471754in}}%
\pgfpathclose%
\pgfusepath{fill}%
\end{pgfscope}%
\begin{pgfscope}%
\pgfpathrectangle{\pgfqpoint{5.200000in}{0.720000in}}{\pgfqpoint{2.640000in}{4.800000in}}%
\pgfusepath{clip}%
\pgfsetbuttcap%
\pgfsetmiterjoin%
\definecolor{currentfill}{rgb}{0.121569,0.466667,0.705882}%
\pgfsetfillcolor{currentfill}%
\pgfsetlinewidth{0.000000pt}%
\definecolor{currentstroke}{rgb}{0.000000,0.000000,0.000000}%
\pgfsetstrokecolor{currentstroke}%
\pgfsetstrokeopacity{0.000000}%
\pgfsetdash{}{0pt}%
\pgfpathmoveto{\pgfqpoint{5.200000in}{6.471754in}}%
\pgfpathlineto{\pgfqpoint{5.246148in}{6.471754in}}%
\pgfpathlineto{\pgfqpoint{5.246148in}{6.636090in}}%
\pgfpathlineto{\pgfqpoint{5.200000in}{6.636090in}}%
\pgfpathclose%
\pgfusepath{fill}%
\end{pgfscope}%
\begin{pgfscope}%
\pgfpathrectangle{\pgfqpoint{5.200000in}{0.720000in}}{\pgfqpoint{2.640000in}{4.800000in}}%
\pgfusepath{clip}%
\pgfsetbuttcap%
\pgfsetmiterjoin%
\definecolor{currentfill}{rgb}{0.121569,0.466667,0.705882}%
\pgfsetfillcolor{currentfill}%
\pgfsetlinewidth{0.000000pt}%
\definecolor{currentstroke}{rgb}{0.000000,0.000000,0.000000}%
\pgfsetstrokecolor{currentstroke}%
\pgfsetstrokeopacity{0.000000}%
\pgfsetdash{}{0pt}%
\pgfpathmoveto{\pgfqpoint{5.200000in}{6.636090in}}%
\pgfpathlineto{\pgfqpoint{5.251275in}{6.636090in}}%
\pgfpathlineto{\pgfqpoint{5.251275in}{6.800426in}}%
\pgfpathlineto{\pgfqpoint{5.200000in}{6.800426in}}%
\pgfpathclose%
\pgfusepath{fill}%
\end{pgfscope}%
\begin{pgfscope}%
\pgfpathrectangle{\pgfqpoint{5.200000in}{0.720000in}}{\pgfqpoint{2.640000in}{4.800000in}}%
\pgfusepath{clip}%
\pgfsetbuttcap%
\pgfsetmiterjoin%
\definecolor{currentfill}{rgb}{0.121569,0.466667,0.705882}%
\pgfsetfillcolor{currentfill}%
\pgfsetlinewidth{0.000000pt}%
\definecolor{currentstroke}{rgb}{0.000000,0.000000,0.000000}%
\pgfsetstrokecolor{currentstroke}%
\pgfsetstrokeopacity{0.000000}%
\pgfsetdash{}{0pt}%
\pgfpathmoveto{\pgfqpoint{5.200000in}{6.800426in}}%
\pgfpathlineto{\pgfqpoint{5.251275in}{6.800426in}}%
\pgfpathlineto{\pgfqpoint{5.251275in}{6.964762in}}%
\pgfpathlineto{\pgfqpoint{5.200000in}{6.964762in}}%
\pgfpathclose%
\pgfusepath{fill}%
\end{pgfscope}%
\begin{pgfscope}%
\pgfpathrectangle{\pgfqpoint{5.200000in}{0.720000in}}{\pgfqpoint{2.640000in}{4.800000in}}%
\pgfusepath{clip}%
\pgfsetbuttcap%
\pgfsetmiterjoin%
\definecolor{currentfill}{rgb}{0.121569,0.466667,0.705882}%
\pgfsetfillcolor{currentfill}%
\pgfsetlinewidth{0.000000pt}%
\definecolor{currentstroke}{rgb}{0.000000,0.000000,0.000000}%
\pgfsetstrokecolor{currentstroke}%
\pgfsetstrokeopacity{0.000000}%
\pgfsetdash{}{0pt}%
\pgfpathmoveto{\pgfqpoint{5.200000in}{6.964762in}}%
\pgfpathlineto{\pgfqpoint{5.233329in}{6.964762in}}%
\pgfpathlineto{\pgfqpoint{5.233329in}{7.129098in}}%
\pgfpathlineto{\pgfqpoint{5.200000in}{7.129098in}}%
\pgfpathclose%
\pgfusepath{fill}%
\end{pgfscope}%
\begin{pgfscope}%
\pgfpathrectangle{\pgfqpoint{5.200000in}{0.720000in}}{\pgfqpoint{2.640000in}{4.800000in}}%
\pgfusepath{clip}%
\pgfsetbuttcap%
\pgfsetmiterjoin%
\definecolor{currentfill}{rgb}{0.121569,0.466667,0.705882}%
\pgfsetfillcolor{currentfill}%
\pgfsetlinewidth{0.000000pt}%
\definecolor{currentstroke}{rgb}{0.000000,0.000000,0.000000}%
\pgfsetstrokecolor{currentstroke}%
\pgfsetstrokeopacity{0.000000}%
\pgfsetdash{}{0pt}%
\pgfpathmoveto{\pgfqpoint{5.200000in}{7.129098in}}%
\pgfpathlineto{\pgfqpoint{5.246148in}{7.129098in}}%
\pgfpathlineto{\pgfqpoint{5.246148in}{7.293433in}}%
\pgfpathlineto{\pgfqpoint{5.200000in}{7.293433in}}%
\pgfpathclose%
\pgfusepath{fill}%
\end{pgfscope}%
\begin{pgfscope}%
\pgfpathrectangle{\pgfqpoint{5.200000in}{0.720000in}}{\pgfqpoint{2.640000in}{4.800000in}}%
\pgfusepath{clip}%
\pgfsetbuttcap%
\pgfsetmiterjoin%
\definecolor{currentfill}{rgb}{0.121569,0.466667,0.705882}%
\pgfsetfillcolor{currentfill}%
\pgfsetlinewidth{0.000000pt}%
\definecolor{currentstroke}{rgb}{0.000000,0.000000,0.000000}%
\pgfsetstrokecolor{currentstroke}%
\pgfsetstrokeopacity{0.000000}%
\pgfsetdash{}{0pt}%
\pgfpathmoveto{\pgfqpoint{5.200000in}{7.293433in}}%
\pgfpathlineto{\pgfqpoint{5.220510in}{7.293433in}}%
\pgfpathlineto{\pgfqpoint{5.220510in}{7.457769in}}%
\pgfpathlineto{\pgfqpoint{5.200000in}{7.457769in}}%
\pgfpathclose%
\pgfusepath{fill}%
\end{pgfscope}%
\begin{pgfscope}%
\pgfpathrectangle{\pgfqpoint{5.200000in}{0.720000in}}{\pgfqpoint{2.640000in}{4.800000in}}%
\pgfusepath{clip}%
\pgfsetbuttcap%
\pgfsetmiterjoin%
\definecolor{currentfill}{rgb}{0.121569,0.466667,0.705882}%
\pgfsetfillcolor{currentfill}%
\pgfsetlinewidth{0.000000pt}%
\definecolor{currentstroke}{rgb}{0.000000,0.000000,0.000000}%
\pgfsetstrokecolor{currentstroke}%
\pgfsetstrokeopacity{0.000000}%
\pgfsetdash{}{0pt}%
\pgfpathmoveto{\pgfqpoint{5.200000in}{7.457769in}}%
\pgfpathlineto{\pgfqpoint{5.217946in}{7.457769in}}%
\pgfpathlineto{\pgfqpoint{5.217946in}{7.622105in}}%
\pgfpathlineto{\pgfqpoint{5.200000in}{7.622105in}}%
\pgfpathclose%
\pgfusepath{fill}%
\end{pgfscope}%
\begin{pgfscope}%
\pgfpathrectangle{\pgfqpoint{5.200000in}{0.720000in}}{\pgfqpoint{2.640000in}{4.800000in}}%
\pgfusepath{clip}%
\pgfsetbuttcap%
\pgfsetmiterjoin%
\definecolor{currentfill}{rgb}{0.121569,0.466667,0.705882}%
\pgfsetfillcolor{currentfill}%
\pgfsetlinewidth{0.000000pt}%
\definecolor{currentstroke}{rgb}{0.000000,0.000000,0.000000}%
\pgfsetstrokecolor{currentstroke}%
\pgfsetstrokeopacity{0.000000}%
\pgfsetdash{}{0pt}%
\pgfpathmoveto{\pgfqpoint{5.200000in}{7.622105in}}%
\pgfpathlineto{\pgfqpoint{5.220510in}{7.622105in}}%
\pgfpathlineto{\pgfqpoint{5.220510in}{7.786441in}}%
\pgfpathlineto{\pgfqpoint{5.200000in}{7.786441in}}%
\pgfpathclose%
\pgfusepath{fill}%
\end{pgfscope}%
\begin{pgfscope}%
\pgfpathrectangle{\pgfqpoint{5.200000in}{0.720000in}}{\pgfqpoint{2.640000in}{4.800000in}}%
\pgfusepath{clip}%
\pgfsetbuttcap%
\pgfsetmiterjoin%
\definecolor{currentfill}{rgb}{0.121569,0.466667,0.705882}%
\pgfsetfillcolor{currentfill}%
\pgfsetlinewidth{0.000000pt}%
\definecolor{currentstroke}{rgb}{0.000000,0.000000,0.000000}%
\pgfsetstrokecolor{currentstroke}%
\pgfsetstrokeopacity{0.000000}%
\pgfsetdash{}{0pt}%
\pgfpathmoveto{\pgfqpoint{5.200000in}{7.786441in}}%
\pgfpathlineto{\pgfqpoint{5.223074in}{7.786441in}}%
\pgfpathlineto{\pgfqpoint{5.223074in}{7.950777in}}%
\pgfpathlineto{\pgfqpoint{5.200000in}{7.950777in}}%
\pgfpathclose%
\pgfusepath{fill}%
\end{pgfscope}%
\begin{pgfscope}%
\pgfpathrectangle{\pgfqpoint{5.200000in}{0.720000in}}{\pgfqpoint{2.640000in}{4.800000in}}%
\pgfusepath{clip}%
\pgfsetbuttcap%
\pgfsetmiterjoin%
\definecolor{currentfill}{rgb}{0.121569,0.466667,0.705882}%
\pgfsetfillcolor{currentfill}%
\pgfsetlinewidth{0.000000pt}%
\definecolor{currentstroke}{rgb}{0.000000,0.000000,0.000000}%
\pgfsetstrokecolor{currentstroke}%
\pgfsetstrokeopacity{0.000000}%
\pgfsetdash{}{0pt}%
\pgfpathmoveto{\pgfqpoint{5.200000in}{7.950777in}}%
\pgfpathlineto{\pgfqpoint{5.230765in}{7.950777in}}%
\pgfpathlineto{\pgfqpoint{5.230765in}{8.115113in}}%
\pgfpathlineto{\pgfqpoint{5.200000in}{8.115113in}}%
\pgfpathclose%
\pgfusepath{fill}%
\end{pgfscope}%
\begin{pgfscope}%
\pgfpathrectangle{\pgfqpoint{5.200000in}{0.720000in}}{\pgfqpoint{2.640000in}{4.800000in}}%
\pgfusepath{clip}%
\pgfsetbuttcap%
\pgfsetmiterjoin%
\definecolor{currentfill}{rgb}{0.121569,0.466667,0.705882}%
\pgfsetfillcolor{currentfill}%
\pgfsetlinewidth{0.000000pt}%
\definecolor{currentstroke}{rgb}{0.000000,0.000000,0.000000}%
\pgfsetstrokecolor{currentstroke}%
\pgfsetstrokeopacity{0.000000}%
\pgfsetdash{}{0pt}%
\pgfpathmoveto{\pgfqpoint{5.200000in}{8.115113in}}%
\pgfpathlineto{\pgfqpoint{5.220510in}{8.115113in}}%
\pgfpathlineto{\pgfqpoint{5.220510in}{8.279448in}}%
\pgfpathlineto{\pgfqpoint{5.200000in}{8.279448in}}%
\pgfpathclose%
\pgfusepath{fill}%
\end{pgfscope}%
\begin{pgfscope}%
\pgfpathrectangle{\pgfqpoint{5.200000in}{0.720000in}}{\pgfqpoint{2.640000in}{4.800000in}}%
\pgfusepath{clip}%
\pgfsetbuttcap%
\pgfsetmiterjoin%
\definecolor{currentfill}{rgb}{0.121569,0.466667,0.705882}%
\pgfsetfillcolor{currentfill}%
\pgfsetlinewidth{0.000000pt}%
\definecolor{currentstroke}{rgb}{0.000000,0.000000,0.000000}%
\pgfsetstrokecolor{currentstroke}%
\pgfsetstrokeopacity{0.000000}%
\pgfsetdash{}{0pt}%
\pgfpathmoveto{\pgfqpoint{5.200000in}{8.279448in}}%
\pgfpathlineto{\pgfqpoint{5.220510in}{8.279448in}}%
\pgfpathlineto{\pgfqpoint{5.220510in}{8.443784in}}%
\pgfpathlineto{\pgfqpoint{5.200000in}{8.443784in}}%
\pgfpathclose%
\pgfusepath{fill}%
\end{pgfscope}%
\begin{pgfscope}%
\pgfpathrectangle{\pgfqpoint{5.200000in}{0.720000in}}{\pgfqpoint{2.640000in}{4.800000in}}%
\pgfusepath{clip}%
\pgfsetbuttcap%
\pgfsetmiterjoin%
\definecolor{currentfill}{rgb}{0.121569,0.466667,0.705882}%
\pgfsetfillcolor{currentfill}%
\pgfsetlinewidth{0.000000pt}%
\definecolor{currentstroke}{rgb}{0.000000,0.000000,0.000000}%
\pgfsetstrokecolor{currentstroke}%
\pgfsetstrokeopacity{0.000000}%
\pgfsetdash{}{0pt}%
\pgfpathmoveto{\pgfqpoint{5.200000in}{8.443784in}}%
\pgfpathlineto{\pgfqpoint{5.210255in}{8.443784in}}%
\pgfpathlineto{\pgfqpoint{5.210255in}{8.608120in}}%
\pgfpathlineto{\pgfqpoint{5.200000in}{8.608120in}}%
\pgfpathclose%
\pgfusepath{fill}%
\end{pgfscope}%
\begin{pgfscope}%
\pgfpathrectangle{\pgfqpoint{5.200000in}{0.720000in}}{\pgfqpoint{2.640000in}{4.800000in}}%
\pgfusepath{clip}%
\pgfsetbuttcap%
\pgfsetmiterjoin%
\definecolor{currentfill}{rgb}{0.121569,0.466667,0.705882}%
\pgfsetfillcolor{currentfill}%
\pgfsetlinewidth{0.000000pt}%
\definecolor{currentstroke}{rgb}{0.000000,0.000000,0.000000}%
\pgfsetstrokecolor{currentstroke}%
\pgfsetstrokeopacity{0.000000}%
\pgfsetdash{}{0pt}%
\pgfpathmoveto{\pgfqpoint{5.200000in}{8.608120in}}%
\pgfpathlineto{\pgfqpoint{5.228201in}{8.608120in}}%
\pgfpathlineto{\pgfqpoint{5.228201in}{8.772456in}}%
\pgfpathlineto{\pgfqpoint{5.200000in}{8.772456in}}%
\pgfpathclose%
\pgfusepath{fill}%
\end{pgfscope}%
\begin{pgfscope}%
\pgfpathrectangle{\pgfqpoint{5.200000in}{0.720000in}}{\pgfqpoint{2.640000in}{4.800000in}}%
\pgfusepath{clip}%
\pgfsetbuttcap%
\pgfsetmiterjoin%
\definecolor{currentfill}{rgb}{0.121569,0.466667,0.705882}%
\pgfsetfillcolor{currentfill}%
\pgfsetlinewidth{0.000000pt}%
\definecolor{currentstroke}{rgb}{0.000000,0.000000,0.000000}%
\pgfsetstrokecolor{currentstroke}%
\pgfsetstrokeopacity{0.000000}%
\pgfsetdash{}{0pt}%
\pgfpathmoveto{\pgfqpoint{5.200000in}{8.772456in}}%
\pgfpathlineto{\pgfqpoint{5.212819in}{8.772456in}}%
\pgfpathlineto{\pgfqpoint{5.212819in}{8.936792in}}%
\pgfpathlineto{\pgfqpoint{5.200000in}{8.936792in}}%
\pgfpathclose%
\pgfusepath{fill}%
\end{pgfscope}%
\begin{pgfscope}%
\pgfpathrectangle{\pgfqpoint{5.200000in}{0.720000in}}{\pgfqpoint{2.640000in}{4.800000in}}%
\pgfusepath{clip}%
\pgfsetbuttcap%
\pgfsetmiterjoin%
\definecolor{currentfill}{rgb}{0.121569,0.466667,0.705882}%
\pgfsetfillcolor{currentfill}%
\pgfsetlinewidth{0.000000pt}%
\definecolor{currentstroke}{rgb}{0.000000,0.000000,0.000000}%
\pgfsetstrokecolor{currentstroke}%
\pgfsetstrokeopacity{0.000000}%
\pgfsetdash{}{0pt}%
\pgfpathmoveto{\pgfqpoint{5.200000in}{8.936792in}}%
\pgfpathlineto{\pgfqpoint{5.215383in}{8.936792in}}%
\pgfpathlineto{\pgfqpoint{5.215383in}{9.101128in}}%
\pgfpathlineto{\pgfqpoint{5.200000in}{9.101128in}}%
\pgfpathclose%
\pgfusepath{fill}%
\end{pgfscope}%
\begin{pgfscope}%
\pgfpathrectangle{\pgfqpoint{5.200000in}{0.720000in}}{\pgfqpoint{2.640000in}{4.800000in}}%
\pgfusepath{clip}%
\pgfsetbuttcap%
\pgfsetmiterjoin%
\definecolor{currentfill}{rgb}{0.121569,0.466667,0.705882}%
\pgfsetfillcolor{currentfill}%
\pgfsetlinewidth{0.000000pt}%
\definecolor{currentstroke}{rgb}{0.000000,0.000000,0.000000}%
\pgfsetstrokecolor{currentstroke}%
\pgfsetstrokeopacity{0.000000}%
\pgfsetdash{}{0pt}%
\pgfpathmoveto{\pgfqpoint{5.200000in}{9.101128in}}%
\pgfpathlineto{\pgfqpoint{5.217946in}{9.101128in}}%
\pgfpathlineto{\pgfqpoint{5.217946in}{9.265463in}}%
\pgfpathlineto{\pgfqpoint{5.200000in}{9.265463in}}%
\pgfpathclose%
\pgfusepath{fill}%
\end{pgfscope}%
\begin{pgfscope}%
\pgfpathrectangle{\pgfqpoint{5.200000in}{0.720000in}}{\pgfqpoint{2.640000in}{4.800000in}}%
\pgfusepath{clip}%
\pgfsetbuttcap%
\pgfsetmiterjoin%
\definecolor{currentfill}{rgb}{0.121569,0.466667,0.705882}%
\pgfsetfillcolor{currentfill}%
\pgfsetlinewidth{0.000000pt}%
\definecolor{currentstroke}{rgb}{0.000000,0.000000,0.000000}%
\pgfsetstrokecolor{currentstroke}%
\pgfsetstrokeopacity{0.000000}%
\pgfsetdash{}{0pt}%
\pgfpathmoveto{\pgfqpoint{5.200000in}{9.265463in}}%
\pgfpathlineto{\pgfqpoint{5.217946in}{9.265463in}}%
\pgfpathlineto{\pgfqpoint{5.217946in}{9.429799in}}%
\pgfpathlineto{\pgfqpoint{5.200000in}{9.429799in}}%
\pgfpathclose%
\pgfusepath{fill}%
\end{pgfscope}%
\begin{pgfscope}%
\pgfpathrectangle{\pgfqpoint{5.200000in}{0.720000in}}{\pgfqpoint{2.640000in}{4.800000in}}%
\pgfusepath{clip}%
\pgfsetbuttcap%
\pgfsetmiterjoin%
\definecolor{currentfill}{rgb}{0.121569,0.466667,0.705882}%
\pgfsetfillcolor{currentfill}%
\pgfsetlinewidth{0.000000pt}%
\definecolor{currentstroke}{rgb}{0.000000,0.000000,0.000000}%
\pgfsetstrokecolor{currentstroke}%
\pgfsetstrokeopacity{0.000000}%
\pgfsetdash{}{0pt}%
\pgfpathmoveto{\pgfqpoint{5.200000in}{9.429799in}}%
\pgfpathlineto{\pgfqpoint{5.217946in}{9.429799in}}%
\pgfpathlineto{\pgfqpoint{5.217946in}{9.594135in}}%
\pgfpathlineto{\pgfqpoint{5.200000in}{9.594135in}}%
\pgfpathclose%
\pgfusepath{fill}%
\end{pgfscope}%
\begin{pgfscope}%
\pgfpathrectangle{\pgfqpoint{5.200000in}{0.720000in}}{\pgfqpoint{2.640000in}{4.800000in}}%
\pgfusepath{clip}%
\pgfsetbuttcap%
\pgfsetmiterjoin%
\definecolor{currentfill}{rgb}{0.121569,0.466667,0.705882}%
\pgfsetfillcolor{currentfill}%
\pgfsetlinewidth{0.000000pt}%
\definecolor{currentstroke}{rgb}{0.000000,0.000000,0.000000}%
\pgfsetstrokecolor{currentstroke}%
\pgfsetstrokeopacity{0.000000}%
\pgfsetdash{}{0pt}%
\pgfpathmoveto{\pgfqpoint{5.200000in}{9.594135in}}%
\pgfpathlineto{\pgfqpoint{5.212819in}{9.594135in}}%
\pgfpathlineto{\pgfqpoint{5.212819in}{9.758471in}}%
\pgfpathlineto{\pgfqpoint{5.200000in}{9.758471in}}%
\pgfpathclose%
\pgfusepath{fill}%
\end{pgfscope}%
\begin{pgfscope}%
\pgfpathrectangle{\pgfqpoint{5.200000in}{0.720000in}}{\pgfqpoint{2.640000in}{4.800000in}}%
\pgfusepath{clip}%
\pgfsetbuttcap%
\pgfsetmiterjoin%
\definecolor{currentfill}{rgb}{0.121569,0.466667,0.705882}%
\pgfsetfillcolor{currentfill}%
\pgfsetlinewidth{0.000000pt}%
\definecolor{currentstroke}{rgb}{0.000000,0.000000,0.000000}%
\pgfsetstrokecolor{currentstroke}%
\pgfsetstrokeopacity{0.000000}%
\pgfsetdash{}{0pt}%
\pgfpathmoveto{\pgfqpoint{5.200000in}{9.758471in}}%
\pgfpathlineto{\pgfqpoint{5.215383in}{9.758471in}}%
\pgfpathlineto{\pgfqpoint{5.215383in}{9.922807in}}%
\pgfpathlineto{\pgfqpoint{5.200000in}{9.922807in}}%
\pgfpathclose%
\pgfusepath{fill}%
\end{pgfscope}%
\begin{pgfscope}%
\pgfpathrectangle{\pgfqpoint{5.200000in}{0.720000in}}{\pgfqpoint{2.640000in}{4.800000in}}%
\pgfusepath{clip}%
\pgfsetbuttcap%
\pgfsetmiterjoin%
\definecolor{currentfill}{rgb}{0.121569,0.466667,0.705882}%
\pgfsetfillcolor{currentfill}%
\pgfsetlinewidth{0.000000pt}%
\definecolor{currentstroke}{rgb}{0.000000,0.000000,0.000000}%
\pgfsetstrokecolor{currentstroke}%
\pgfsetstrokeopacity{0.000000}%
\pgfsetdash{}{0pt}%
\pgfpathmoveto{\pgfqpoint{5.200000in}{9.922807in}}%
\pgfpathlineto{\pgfqpoint{5.212819in}{9.922807in}}%
\pgfpathlineto{\pgfqpoint{5.212819in}{10.087143in}}%
\pgfpathlineto{\pgfqpoint{5.200000in}{10.087143in}}%
\pgfpathclose%
\pgfusepath{fill}%
\end{pgfscope}%
\begin{pgfscope}%
\pgfpathrectangle{\pgfqpoint{5.200000in}{0.720000in}}{\pgfqpoint{2.640000in}{4.800000in}}%
\pgfusepath{clip}%
\pgfsetbuttcap%
\pgfsetmiterjoin%
\definecolor{currentfill}{rgb}{0.121569,0.466667,0.705882}%
\pgfsetfillcolor{currentfill}%
\pgfsetlinewidth{0.000000pt}%
\definecolor{currentstroke}{rgb}{0.000000,0.000000,0.000000}%
\pgfsetstrokecolor{currentstroke}%
\pgfsetstrokeopacity{0.000000}%
\pgfsetdash{}{0pt}%
\pgfpathmoveto{\pgfqpoint{5.200000in}{10.087143in}}%
\pgfpathlineto{\pgfqpoint{5.202564in}{10.087143in}}%
\pgfpathlineto{\pgfqpoint{5.202564in}{10.251478in}}%
\pgfpathlineto{\pgfqpoint{5.200000in}{10.251478in}}%
\pgfpathclose%
\pgfusepath{fill}%
\end{pgfscope}%
\begin{pgfscope}%
\pgfpathrectangle{\pgfqpoint{5.200000in}{0.720000in}}{\pgfqpoint{2.640000in}{4.800000in}}%
\pgfusepath{clip}%
\pgfsetbuttcap%
\pgfsetmiterjoin%
\definecolor{currentfill}{rgb}{0.121569,0.466667,0.705882}%
\pgfsetfillcolor{currentfill}%
\pgfsetlinewidth{0.000000pt}%
\definecolor{currentstroke}{rgb}{0.000000,0.000000,0.000000}%
\pgfsetstrokecolor{currentstroke}%
\pgfsetstrokeopacity{0.000000}%
\pgfsetdash{}{0pt}%
\pgfpathmoveto{\pgfqpoint{5.200000in}{10.251478in}}%
\pgfpathlineto{\pgfqpoint{5.215383in}{10.251478in}}%
\pgfpathlineto{\pgfqpoint{5.215383in}{10.415814in}}%
\pgfpathlineto{\pgfqpoint{5.200000in}{10.415814in}}%
\pgfpathclose%
\pgfusepath{fill}%
\end{pgfscope}%
\begin{pgfscope}%
\pgfpathrectangle{\pgfqpoint{5.200000in}{0.720000in}}{\pgfqpoint{2.640000in}{4.800000in}}%
\pgfusepath{clip}%
\pgfsetbuttcap%
\pgfsetmiterjoin%
\definecolor{currentfill}{rgb}{0.121569,0.466667,0.705882}%
\pgfsetfillcolor{currentfill}%
\pgfsetlinewidth{0.000000pt}%
\definecolor{currentstroke}{rgb}{0.000000,0.000000,0.000000}%
\pgfsetstrokecolor{currentstroke}%
\pgfsetstrokeopacity{0.000000}%
\pgfsetdash{}{0pt}%
\pgfpathmoveto{\pgfqpoint{5.200000in}{10.415814in}}%
\pgfpathlineto{\pgfqpoint{5.210255in}{10.415814in}}%
\pgfpathlineto{\pgfqpoint{5.210255in}{10.580150in}}%
\pgfpathlineto{\pgfqpoint{5.200000in}{10.580150in}}%
\pgfpathclose%
\pgfusepath{fill}%
\end{pgfscope}%
\begin{pgfscope}%
\pgfpathrectangle{\pgfqpoint{5.200000in}{0.720000in}}{\pgfqpoint{2.640000in}{4.800000in}}%
\pgfusepath{clip}%
\pgfsetbuttcap%
\pgfsetmiterjoin%
\definecolor{currentfill}{rgb}{0.121569,0.466667,0.705882}%
\pgfsetfillcolor{currentfill}%
\pgfsetlinewidth{0.000000pt}%
\definecolor{currentstroke}{rgb}{0.000000,0.000000,0.000000}%
\pgfsetstrokecolor{currentstroke}%
\pgfsetstrokeopacity{0.000000}%
\pgfsetdash{}{0pt}%
\pgfpathmoveto{\pgfqpoint{5.200000in}{10.580150in}}%
\pgfpathlineto{\pgfqpoint{5.210255in}{10.580150in}}%
\pgfpathlineto{\pgfqpoint{5.210255in}{10.744486in}}%
\pgfpathlineto{\pgfqpoint{5.200000in}{10.744486in}}%
\pgfpathclose%
\pgfusepath{fill}%
\end{pgfscope}%
\begin{pgfscope}%
\pgfpathrectangle{\pgfqpoint{5.200000in}{0.720000in}}{\pgfqpoint{2.640000in}{4.800000in}}%
\pgfusepath{clip}%
\pgfsetbuttcap%
\pgfsetmiterjoin%
\definecolor{currentfill}{rgb}{0.121569,0.466667,0.705882}%
\pgfsetfillcolor{currentfill}%
\pgfsetlinewidth{0.000000pt}%
\definecolor{currentstroke}{rgb}{0.000000,0.000000,0.000000}%
\pgfsetstrokecolor{currentstroke}%
\pgfsetstrokeopacity{0.000000}%
\pgfsetdash{}{0pt}%
\pgfpathmoveto{\pgfqpoint{5.200000in}{10.744486in}}%
\pgfpathlineto{\pgfqpoint{5.212819in}{10.744486in}}%
\pgfpathlineto{\pgfqpoint{5.212819in}{10.908822in}}%
\pgfpathlineto{\pgfqpoint{5.200000in}{10.908822in}}%
\pgfpathclose%
\pgfusepath{fill}%
\end{pgfscope}%
\begin{pgfscope}%
\pgfpathrectangle{\pgfqpoint{5.200000in}{0.720000in}}{\pgfqpoint{2.640000in}{4.800000in}}%
\pgfusepath{clip}%
\pgfsetbuttcap%
\pgfsetmiterjoin%
\definecolor{currentfill}{rgb}{0.121569,0.466667,0.705882}%
\pgfsetfillcolor{currentfill}%
\pgfsetlinewidth{0.000000pt}%
\definecolor{currentstroke}{rgb}{0.000000,0.000000,0.000000}%
\pgfsetstrokecolor{currentstroke}%
\pgfsetstrokeopacity{0.000000}%
\pgfsetdash{}{0pt}%
\pgfpathmoveto{\pgfqpoint{5.200000in}{10.908822in}}%
\pgfpathlineto{\pgfqpoint{5.212819in}{10.908822in}}%
\pgfpathlineto{\pgfqpoint{5.212819in}{11.073158in}}%
\pgfpathlineto{\pgfqpoint{5.200000in}{11.073158in}}%
\pgfpathclose%
\pgfusepath{fill}%
\end{pgfscope}%
\begin{pgfscope}%
\pgfpathrectangle{\pgfqpoint{5.200000in}{0.720000in}}{\pgfqpoint{2.640000in}{4.800000in}}%
\pgfusepath{clip}%
\pgfsetbuttcap%
\pgfsetmiterjoin%
\definecolor{currentfill}{rgb}{0.121569,0.466667,0.705882}%
\pgfsetfillcolor{currentfill}%
\pgfsetlinewidth{0.000000pt}%
\definecolor{currentstroke}{rgb}{0.000000,0.000000,0.000000}%
\pgfsetstrokecolor{currentstroke}%
\pgfsetstrokeopacity{0.000000}%
\pgfsetdash{}{0pt}%
\pgfpathmoveto{\pgfqpoint{5.200000in}{11.073158in}}%
\pgfpathlineto{\pgfqpoint{5.205128in}{11.073158in}}%
\pgfpathlineto{\pgfqpoint{5.205128in}{11.237493in}}%
\pgfpathlineto{\pgfqpoint{5.200000in}{11.237493in}}%
\pgfpathclose%
\pgfusepath{fill}%
\end{pgfscope}%
\begin{pgfscope}%
\pgfpathrectangle{\pgfqpoint{5.200000in}{0.720000in}}{\pgfqpoint{2.640000in}{4.800000in}}%
\pgfusepath{clip}%
\pgfsetbuttcap%
\pgfsetmiterjoin%
\definecolor{currentfill}{rgb}{0.121569,0.466667,0.705882}%
\pgfsetfillcolor{currentfill}%
\pgfsetlinewidth{0.000000pt}%
\definecolor{currentstroke}{rgb}{0.000000,0.000000,0.000000}%
\pgfsetstrokecolor{currentstroke}%
\pgfsetstrokeopacity{0.000000}%
\pgfsetdash{}{0pt}%
\pgfpathmoveto{\pgfqpoint{5.200000in}{11.237493in}}%
\pgfpathlineto{\pgfqpoint{5.212819in}{11.237493in}}%
\pgfpathlineto{\pgfqpoint{5.212819in}{11.401829in}}%
\pgfpathlineto{\pgfqpoint{5.200000in}{11.401829in}}%
\pgfpathclose%
\pgfusepath{fill}%
\end{pgfscope}%
\begin{pgfscope}%
\pgfpathrectangle{\pgfqpoint{5.200000in}{0.720000in}}{\pgfqpoint{2.640000in}{4.800000in}}%
\pgfusepath{clip}%
\pgfsetbuttcap%
\pgfsetmiterjoin%
\definecolor{currentfill}{rgb}{0.121569,0.466667,0.705882}%
\pgfsetfillcolor{currentfill}%
\pgfsetlinewidth{0.000000pt}%
\definecolor{currentstroke}{rgb}{0.000000,0.000000,0.000000}%
\pgfsetstrokecolor{currentstroke}%
\pgfsetstrokeopacity{0.000000}%
\pgfsetdash{}{0pt}%
\pgfpathmoveto{\pgfqpoint{5.200000in}{11.401829in}}%
\pgfpathlineto{\pgfqpoint{5.200000in}{11.401829in}}%
\pgfpathlineto{\pgfqpoint{5.200000in}{11.566165in}}%
\pgfpathlineto{\pgfqpoint{5.200000in}{11.566165in}}%
\pgfpathclose%
\pgfusepath{fill}%
\end{pgfscope}%
\begin{pgfscope}%
\pgfpathrectangle{\pgfqpoint{5.200000in}{0.720000in}}{\pgfqpoint{2.640000in}{4.800000in}}%
\pgfusepath{clip}%
\pgfsetbuttcap%
\pgfsetmiterjoin%
\definecolor{currentfill}{rgb}{0.121569,0.466667,0.705882}%
\pgfsetfillcolor{currentfill}%
\pgfsetlinewidth{0.000000pt}%
\definecolor{currentstroke}{rgb}{0.000000,0.000000,0.000000}%
\pgfsetstrokecolor{currentstroke}%
\pgfsetstrokeopacity{0.000000}%
\pgfsetdash{}{0pt}%
\pgfpathmoveto{\pgfqpoint{5.200000in}{11.566165in}}%
\pgfpathlineto{\pgfqpoint{5.210255in}{11.566165in}}%
\pgfpathlineto{\pgfqpoint{5.210255in}{11.730501in}}%
\pgfpathlineto{\pgfqpoint{5.200000in}{11.730501in}}%
\pgfpathclose%
\pgfusepath{fill}%
\end{pgfscope}%
\begin{pgfscope}%
\pgfpathrectangle{\pgfqpoint{5.200000in}{0.720000in}}{\pgfqpoint{2.640000in}{4.800000in}}%
\pgfusepath{clip}%
\pgfsetbuttcap%
\pgfsetmiterjoin%
\definecolor{currentfill}{rgb}{0.121569,0.466667,0.705882}%
\pgfsetfillcolor{currentfill}%
\pgfsetlinewidth{0.000000pt}%
\definecolor{currentstroke}{rgb}{0.000000,0.000000,0.000000}%
\pgfsetstrokecolor{currentstroke}%
\pgfsetstrokeopacity{0.000000}%
\pgfsetdash{}{0pt}%
\pgfpathmoveto{\pgfqpoint{5.200000in}{11.730501in}}%
\pgfpathlineto{\pgfqpoint{5.202564in}{11.730501in}}%
\pgfpathlineto{\pgfqpoint{5.202564in}{11.894837in}}%
\pgfpathlineto{\pgfqpoint{5.200000in}{11.894837in}}%
\pgfpathclose%
\pgfusepath{fill}%
\end{pgfscope}%
\begin{pgfscope}%
\pgfpathrectangle{\pgfqpoint{5.200000in}{0.720000in}}{\pgfqpoint{2.640000in}{4.800000in}}%
\pgfusepath{clip}%
\pgfsetbuttcap%
\pgfsetmiterjoin%
\definecolor{currentfill}{rgb}{0.121569,0.466667,0.705882}%
\pgfsetfillcolor{currentfill}%
\pgfsetlinewidth{0.000000pt}%
\definecolor{currentstroke}{rgb}{0.000000,0.000000,0.000000}%
\pgfsetstrokecolor{currentstroke}%
\pgfsetstrokeopacity{0.000000}%
\pgfsetdash{}{0pt}%
\pgfpathmoveto{\pgfqpoint{5.200000in}{11.894837in}}%
\pgfpathlineto{\pgfqpoint{5.205128in}{11.894837in}}%
\pgfpathlineto{\pgfqpoint{5.205128in}{12.059173in}}%
\pgfpathlineto{\pgfqpoint{5.200000in}{12.059173in}}%
\pgfpathclose%
\pgfusepath{fill}%
\end{pgfscope}%
\begin{pgfscope}%
\pgfpathrectangle{\pgfqpoint{5.200000in}{0.720000in}}{\pgfqpoint{2.640000in}{4.800000in}}%
\pgfusepath{clip}%
\pgfsetbuttcap%
\pgfsetmiterjoin%
\definecolor{currentfill}{rgb}{0.121569,0.466667,0.705882}%
\pgfsetfillcolor{currentfill}%
\pgfsetlinewidth{0.000000pt}%
\definecolor{currentstroke}{rgb}{0.000000,0.000000,0.000000}%
\pgfsetstrokecolor{currentstroke}%
\pgfsetstrokeopacity{0.000000}%
\pgfsetdash{}{0pt}%
\pgfpathmoveto{\pgfqpoint{5.200000in}{12.059173in}}%
\pgfpathlineto{\pgfqpoint{5.202564in}{12.059173in}}%
\pgfpathlineto{\pgfqpoint{5.202564in}{12.223508in}}%
\pgfpathlineto{\pgfqpoint{5.200000in}{12.223508in}}%
\pgfpathclose%
\pgfusepath{fill}%
\end{pgfscope}%
\begin{pgfscope}%
\pgfpathrectangle{\pgfqpoint{5.200000in}{0.720000in}}{\pgfqpoint{2.640000in}{4.800000in}}%
\pgfusepath{clip}%
\pgfsetbuttcap%
\pgfsetmiterjoin%
\definecolor{currentfill}{rgb}{0.121569,0.466667,0.705882}%
\pgfsetfillcolor{currentfill}%
\pgfsetlinewidth{0.000000pt}%
\definecolor{currentstroke}{rgb}{0.000000,0.000000,0.000000}%
\pgfsetstrokecolor{currentstroke}%
\pgfsetstrokeopacity{0.000000}%
\pgfsetdash{}{0pt}%
\pgfpathmoveto{\pgfqpoint{5.200000in}{12.223508in}}%
\pgfpathlineto{\pgfqpoint{5.202564in}{12.223508in}}%
\pgfpathlineto{\pgfqpoint{5.202564in}{12.387844in}}%
\pgfpathlineto{\pgfqpoint{5.200000in}{12.387844in}}%
\pgfpathclose%
\pgfusepath{fill}%
\end{pgfscope}%
\begin{pgfscope}%
\pgfpathrectangle{\pgfqpoint{5.200000in}{0.720000in}}{\pgfqpoint{2.640000in}{4.800000in}}%
\pgfusepath{clip}%
\pgfsetbuttcap%
\pgfsetmiterjoin%
\definecolor{currentfill}{rgb}{0.121569,0.466667,0.705882}%
\pgfsetfillcolor{currentfill}%
\pgfsetlinewidth{0.000000pt}%
\definecolor{currentstroke}{rgb}{0.000000,0.000000,0.000000}%
\pgfsetstrokecolor{currentstroke}%
\pgfsetstrokeopacity{0.000000}%
\pgfsetdash{}{0pt}%
\pgfpathmoveto{\pgfqpoint{5.200000in}{12.387844in}}%
\pgfpathlineto{\pgfqpoint{5.200000in}{12.387844in}}%
\pgfpathlineto{\pgfqpoint{5.200000in}{12.552180in}}%
\pgfpathlineto{\pgfqpoint{5.200000in}{12.552180in}}%
\pgfpathclose%
\pgfusepath{fill}%
\end{pgfscope}%
\begin{pgfscope}%
\pgfpathrectangle{\pgfqpoint{5.200000in}{0.720000in}}{\pgfqpoint{2.640000in}{4.800000in}}%
\pgfusepath{clip}%
\pgfsetbuttcap%
\pgfsetmiterjoin%
\definecolor{currentfill}{rgb}{0.121569,0.466667,0.705882}%
\pgfsetfillcolor{currentfill}%
\pgfsetlinewidth{0.000000pt}%
\definecolor{currentstroke}{rgb}{0.000000,0.000000,0.000000}%
\pgfsetstrokecolor{currentstroke}%
\pgfsetstrokeopacity{0.000000}%
\pgfsetdash{}{0pt}%
\pgfpathmoveto{\pgfqpoint{5.200000in}{12.552180in}}%
\pgfpathlineto{\pgfqpoint{5.210255in}{12.552180in}}%
\pgfpathlineto{\pgfqpoint{5.210255in}{12.716516in}}%
\pgfpathlineto{\pgfqpoint{5.200000in}{12.716516in}}%
\pgfpathclose%
\pgfusepath{fill}%
\end{pgfscope}%
\begin{pgfscope}%
\pgfpathrectangle{\pgfqpoint{5.200000in}{0.720000in}}{\pgfqpoint{2.640000in}{4.800000in}}%
\pgfusepath{clip}%
\pgfsetbuttcap%
\pgfsetmiterjoin%
\definecolor{currentfill}{rgb}{0.121569,0.466667,0.705882}%
\pgfsetfillcolor{currentfill}%
\pgfsetlinewidth{0.000000pt}%
\definecolor{currentstroke}{rgb}{0.000000,0.000000,0.000000}%
\pgfsetstrokecolor{currentstroke}%
\pgfsetstrokeopacity{0.000000}%
\pgfsetdash{}{0pt}%
\pgfpathmoveto{\pgfqpoint{5.200000in}{12.716516in}}%
\pgfpathlineto{\pgfqpoint{5.207691in}{12.716516in}}%
\pgfpathlineto{\pgfqpoint{5.207691in}{12.880852in}}%
\pgfpathlineto{\pgfqpoint{5.200000in}{12.880852in}}%
\pgfpathclose%
\pgfusepath{fill}%
\end{pgfscope}%
\begin{pgfscope}%
\pgfpathrectangle{\pgfqpoint{5.200000in}{0.720000in}}{\pgfqpoint{2.640000in}{4.800000in}}%
\pgfusepath{clip}%
\pgfsetbuttcap%
\pgfsetmiterjoin%
\definecolor{currentfill}{rgb}{0.121569,0.466667,0.705882}%
\pgfsetfillcolor{currentfill}%
\pgfsetlinewidth{0.000000pt}%
\definecolor{currentstroke}{rgb}{0.000000,0.000000,0.000000}%
\pgfsetstrokecolor{currentstroke}%
\pgfsetstrokeopacity{0.000000}%
\pgfsetdash{}{0pt}%
\pgfpathmoveto{\pgfqpoint{5.200000in}{12.880852in}}%
\pgfpathlineto{\pgfqpoint{5.202564in}{12.880852in}}%
\pgfpathlineto{\pgfqpoint{5.202564in}{13.045188in}}%
\pgfpathlineto{\pgfqpoint{5.200000in}{13.045188in}}%
\pgfpathclose%
\pgfusepath{fill}%
\end{pgfscope}%
\begin{pgfscope}%
\pgfpathrectangle{\pgfqpoint{5.200000in}{0.720000in}}{\pgfqpoint{2.640000in}{4.800000in}}%
\pgfusepath{clip}%
\pgfsetbuttcap%
\pgfsetmiterjoin%
\definecolor{currentfill}{rgb}{0.121569,0.466667,0.705882}%
\pgfsetfillcolor{currentfill}%
\pgfsetlinewidth{0.000000pt}%
\definecolor{currentstroke}{rgb}{0.000000,0.000000,0.000000}%
\pgfsetstrokecolor{currentstroke}%
\pgfsetstrokeopacity{0.000000}%
\pgfsetdash{}{0pt}%
\pgfpathmoveto{\pgfqpoint{5.200000in}{13.045188in}}%
\pgfpathlineto{\pgfqpoint{5.202564in}{13.045188in}}%
\pgfpathlineto{\pgfqpoint{5.202564in}{13.209523in}}%
\pgfpathlineto{\pgfqpoint{5.200000in}{13.209523in}}%
\pgfpathclose%
\pgfusepath{fill}%
\end{pgfscope}%
\begin{pgfscope}%
\pgfpathrectangle{\pgfqpoint{5.200000in}{0.720000in}}{\pgfqpoint{2.640000in}{4.800000in}}%
\pgfusepath{clip}%
\pgfsetbuttcap%
\pgfsetmiterjoin%
\definecolor{currentfill}{rgb}{0.121569,0.466667,0.705882}%
\pgfsetfillcolor{currentfill}%
\pgfsetlinewidth{0.000000pt}%
\definecolor{currentstroke}{rgb}{0.000000,0.000000,0.000000}%
\pgfsetstrokecolor{currentstroke}%
\pgfsetstrokeopacity{0.000000}%
\pgfsetdash{}{0pt}%
\pgfpathmoveto{\pgfqpoint{5.200000in}{13.209523in}}%
\pgfpathlineto{\pgfqpoint{5.207691in}{13.209523in}}%
\pgfpathlineto{\pgfqpoint{5.207691in}{13.373859in}}%
\pgfpathlineto{\pgfqpoint{5.200000in}{13.373859in}}%
\pgfpathclose%
\pgfusepath{fill}%
\end{pgfscope}%
\begin{pgfscope}%
\pgfpathrectangle{\pgfqpoint{5.200000in}{0.720000in}}{\pgfqpoint{2.640000in}{4.800000in}}%
\pgfusepath{clip}%
\pgfsetbuttcap%
\pgfsetmiterjoin%
\definecolor{currentfill}{rgb}{0.121569,0.466667,0.705882}%
\pgfsetfillcolor{currentfill}%
\pgfsetlinewidth{0.000000pt}%
\definecolor{currentstroke}{rgb}{0.000000,0.000000,0.000000}%
\pgfsetstrokecolor{currentstroke}%
\pgfsetstrokeopacity{0.000000}%
\pgfsetdash{}{0pt}%
\pgfpathmoveto{\pgfqpoint{5.200000in}{13.373859in}}%
\pgfpathlineto{\pgfqpoint{5.202564in}{13.373859in}}%
\pgfpathlineto{\pgfqpoint{5.202564in}{13.538195in}}%
\pgfpathlineto{\pgfqpoint{5.200000in}{13.538195in}}%
\pgfpathclose%
\pgfusepath{fill}%
\end{pgfscope}%
\begin{pgfscope}%
\pgfpathrectangle{\pgfqpoint{5.200000in}{0.720000in}}{\pgfqpoint{2.640000in}{4.800000in}}%
\pgfusepath{clip}%
\pgfsetbuttcap%
\pgfsetmiterjoin%
\definecolor{currentfill}{rgb}{0.121569,0.466667,0.705882}%
\pgfsetfillcolor{currentfill}%
\pgfsetlinewidth{0.000000pt}%
\definecolor{currentstroke}{rgb}{0.000000,0.000000,0.000000}%
\pgfsetstrokecolor{currentstroke}%
\pgfsetstrokeopacity{0.000000}%
\pgfsetdash{}{0pt}%
\pgfpathmoveto{\pgfqpoint{5.200000in}{13.538195in}}%
\pgfpathlineto{\pgfqpoint{5.202564in}{13.538195in}}%
\pgfpathlineto{\pgfqpoint{5.202564in}{13.702531in}}%
\pgfpathlineto{\pgfqpoint{5.200000in}{13.702531in}}%
\pgfpathclose%
\pgfusepath{fill}%
\end{pgfscope}%
\begin{pgfscope}%
\pgfpathrectangle{\pgfqpoint{5.200000in}{0.720000in}}{\pgfqpoint{2.640000in}{4.800000in}}%
\pgfusepath{clip}%
\pgfsetbuttcap%
\pgfsetmiterjoin%
\definecolor{currentfill}{rgb}{0.121569,0.466667,0.705882}%
\pgfsetfillcolor{currentfill}%
\pgfsetlinewidth{0.000000pt}%
\definecolor{currentstroke}{rgb}{0.000000,0.000000,0.000000}%
\pgfsetstrokecolor{currentstroke}%
\pgfsetstrokeopacity{0.000000}%
\pgfsetdash{}{0pt}%
\pgfpathmoveto{\pgfqpoint{5.200000in}{13.702531in}}%
\pgfpathlineto{\pgfqpoint{5.207691in}{13.702531in}}%
\pgfpathlineto{\pgfqpoint{5.207691in}{13.866867in}}%
\pgfpathlineto{\pgfqpoint{5.200000in}{13.866867in}}%
\pgfpathclose%
\pgfusepath{fill}%
\end{pgfscope}%
\begin{pgfscope}%
\pgfpathrectangle{\pgfqpoint{5.200000in}{0.720000in}}{\pgfqpoint{2.640000in}{4.800000in}}%
\pgfusepath{clip}%
\pgfsetbuttcap%
\pgfsetmiterjoin%
\definecolor{currentfill}{rgb}{0.121569,0.466667,0.705882}%
\pgfsetfillcolor{currentfill}%
\pgfsetlinewidth{0.000000pt}%
\definecolor{currentstroke}{rgb}{0.000000,0.000000,0.000000}%
\pgfsetstrokecolor{currentstroke}%
\pgfsetstrokeopacity{0.000000}%
\pgfsetdash{}{0pt}%
\pgfpathmoveto{\pgfqpoint{5.200000in}{13.866867in}}%
\pgfpathlineto{\pgfqpoint{5.202564in}{13.866867in}}%
\pgfpathlineto{\pgfqpoint{5.202564in}{14.031203in}}%
\pgfpathlineto{\pgfqpoint{5.200000in}{14.031203in}}%
\pgfpathclose%
\pgfusepath{fill}%
\end{pgfscope}%
\begin{pgfscope}%
\pgfpathrectangle{\pgfqpoint{5.200000in}{0.720000in}}{\pgfqpoint{2.640000in}{4.800000in}}%
\pgfusepath{clip}%
\pgfsetbuttcap%
\pgfsetmiterjoin%
\definecolor{currentfill}{rgb}{0.121569,0.466667,0.705882}%
\pgfsetfillcolor{currentfill}%
\pgfsetlinewidth{0.000000pt}%
\definecolor{currentstroke}{rgb}{0.000000,0.000000,0.000000}%
\pgfsetstrokecolor{currentstroke}%
\pgfsetstrokeopacity{0.000000}%
\pgfsetdash{}{0pt}%
\pgfpathmoveto{\pgfqpoint{5.200000in}{14.031203in}}%
\pgfpathlineto{\pgfqpoint{5.207691in}{14.031203in}}%
\pgfpathlineto{\pgfqpoint{5.207691in}{14.195538in}}%
\pgfpathlineto{\pgfqpoint{5.200000in}{14.195538in}}%
\pgfpathclose%
\pgfusepath{fill}%
\end{pgfscope}%
\begin{pgfscope}%
\pgfpathrectangle{\pgfqpoint{5.200000in}{0.720000in}}{\pgfqpoint{2.640000in}{4.800000in}}%
\pgfusepath{clip}%
\pgfsetbuttcap%
\pgfsetmiterjoin%
\definecolor{currentfill}{rgb}{0.121569,0.466667,0.705882}%
\pgfsetfillcolor{currentfill}%
\pgfsetlinewidth{0.000000pt}%
\definecolor{currentstroke}{rgb}{0.000000,0.000000,0.000000}%
\pgfsetstrokecolor{currentstroke}%
\pgfsetstrokeopacity{0.000000}%
\pgfsetdash{}{0pt}%
\pgfpathmoveto{\pgfqpoint{5.200000in}{14.195538in}}%
\pgfpathlineto{\pgfqpoint{5.200000in}{14.195538in}}%
\pgfpathlineto{\pgfqpoint{5.200000in}{14.359874in}}%
\pgfpathlineto{\pgfqpoint{5.200000in}{14.359874in}}%
\pgfpathclose%
\pgfusepath{fill}%
\end{pgfscope}%
\begin{pgfscope}%
\pgfpathrectangle{\pgfqpoint{5.200000in}{0.720000in}}{\pgfqpoint{2.640000in}{4.800000in}}%
\pgfusepath{clip}%
\pgfsetbuttcap%
\pgfsetmiterjoin%
\definecolor{currentfill}{rgb}{0.121569,0.466667,0.705882}%
\pgfsetfillcolor{currentfill}%
\pgfsetlinewidth{0.000000pt}%
\definecolor{currentstroke}{rgb}{0.000000,0.000000,0.000000}%
\pgfsetstrokecolor{currentstroke}%
\pgfsetstrokeopacity{0.000000}%
\pgfsetdash{}{0pt}%
\pgfpathmoveto{\pgfqpoint{5.200000in}{14.359874in}}%
\pgfpathlineto{\pgfqpoint{5.202564in}{14.359874in}}%
\pgfpathlineto{\pgfqpoint{5.202564in}{14.524210in}}%
\pgfpathlineto{\pgfqpoint{5.200000in}{14.524210in}}%
\pgfpathclose%
\pgfusepath{fill}%
\end{pgfscope}%
\begin{pgfscope}%
\pgfpathrectangle{\pgfqpoint{5.200000in}{0.720000in}}{\pgfqpoint{2.640000in}{4.800000in}}%
\pgfusepath{clip}%
\pgfsetbuttcap%
\pgfsetmiterjoin%
\definecolor{currentfill}{rgb}{0.121569,0.466667,0.705882}%
\pgfsetfillcolor{currentfill}%
\pgfsetlinewidth{0.000000pt}%
\definecolor{currentstroke}{rgb}{0.000000,0.000000,0.000000}%
\pgfsetstrokecolor{currentstroke}%
\pgfsetstrokeopacity{0.000000}%
\pgfsetdash{}{0pt}%
\pgfpathmoveto{\pgfqpoint{5.200000in}{14.524210in}}%
\pgfpathlineto{\pgfqpoint{5.205128in}{14.524210in}}%
\pgfpathlineto{\pgfqpoint{5.205128in}{14.688546in}}%
\pgfpathlineto{\pgfqpoint{5.200000in}{14.688546in}}%
\pgfpathclose%
\pgfusepath{fill}%
\end{pgfscope}%
\begin{pgfscope}%
\pgfpathrectangle{\pgfqpoint{5.200000in}{0.720000in}}{\pgfqpoint{2.640000in}{4.800000in}}%
\pgfusepath{clip}%
\pgfsetbuttcap%
\pgfsetmiterjoin%
\definecolor{currentfill}{rgb}{0.121569,0.466667,0.705882}%
\pgfsetfillcolor{currentfill}%
\pgfsetlinewidth{0.000000pt}%
\definecolor{currentstroke}{rgb}{0.000000,0.000000,0.000000}%
\pgfsetstrokecolor{currentstroke}%
\pgfsetstrokeopacity{0.000000}%
\pgfsetdash{}{0pt}%
\pgfpathmoveto{\pgfqpoint{5.200000in}{14.688546in}}%
\pgfpathlineto{\pgfqpoint{5.205128in}{14.688546in}}%
\pgfpathlineto{\pgfqpoint{5.205128in}{14.852882in}}%
\pgfpathlineto{\pgfqpoint{5.200000in}{14.852882in}}%
\pgfpathclose%
\pgfusepath{fill}%
\end{pgfscope}%
\begin{pgfscope}%
\pgfpathrectangle{\pgfqpoint{5.200000in}{0.720000in}}{\pgfqpoint{2.640000in}{4.800000in}}%
\pgfusepath{clip}%
\pgfsetbuttcap%
\pgfsetmiterjoin%
\definecolor{currentfill}{rgb}{0.121569,0.466667,0.705882}%
\pgfsetfillcolor{currentfill}%
\pgfsetlinewidth{0.000000pt}%
\definecolor{currentstroke}{rgb}{0.000000,0.000000,0.000000}%
\pgfsetstrokecolor{currentstroke}%
\pgfsetstrokeopacity{0.000000}%
\pgfsetdash{}{0pt}%
\pgfpathmoveto{\pgfqpoint{5.200000in}{14.852882in}}%
\pgfpathlineto{\pgfqpoint{5.205128in}{14.852882in}}%
\pgfpathlineto{\pgfqpoint{5.205128in}{15.017218in}}%
\pgfpathlineto{\pgfqpoint{5.200000in}{15.017218in}}%
\pgfpathclose%
\pgfusepath{fill}%
\end{pgfscope}%
\begin{pgfscope}%
\pgfpathrectangle{\pgfqpoint{5.200000in}{0.720000in}}{\pgfqpoint{2.640000in}{4.800000in}}%
\pgfusepath{clip}%
\pgfsetbuttcap%
\pgfsetmiterjoin%
\definecolor{currentfill}{rgb}{0.121569,0.466667,0.705882}%
\pgfsetfillcolor{currentfill}%
\pgfsetlinewidth{0.000000pt}%
\definecolor{currentstroke}{rgb}{0.000000,0.000000,0.000000}%
\pgfsetstrokecolor{currentstroke}%
\pgfsetstrokeopacity{0.000000}%
\pgfsetdash{}{0pt}%
\pgfpathmoveto{\pgfqpoint{5.200000in}{15.017218in}}%
\pgfpathlineto{\pgfqpoint{5.202564in}{15.017218in}}%
\pgfpathlineto{\pgfqpoint{5.202564in}{15.181553in}}%
\pgfpathlineto{\pgfqpoint{5.200000in}{15.181553in}}%
\pgfpathclose%
\pgfusepath{fill}%
\end{pgfscope}%
\begin{pgfscope}%
\pgfpathrectangle{\pgfqpoint{5.200000in}{0.720000in}}{\pgfqpoint{2.640000in}{4.800000in}}%
\pgfusepath{clip}%
\pgfsetbuttcap%
\pgfsetmiterjoin%
\definecolor{currentfill}{rgb}{0.121569,0.466667,0.705882}%
\pgfsetfillcolor{currentfill}%
\pgfsetlinewidth{0.000000pt}%
\definecolor{currentstroke}{rgb}{0.000000,0.000000,0.000000}%
\pgfsetstrokecolor{currentstroke}%
\pgfsetstrokeopacity{0.000000}%
\pgfsetdash{}{0pt}%
\pgfpathmoveto{\pgfqpoint{5.200000in}{15.181553in}}%
\pgfpathlineto{\pgfqpoint{5.205128in}{15.181553in}}%
\pgfpathlineto{\pgfqpoint{5.205128in}{15.345889in}}%
\pgfpathlineto{\pgfqpoint{5.200000in}{15.345889in}}%
\pgfpathclose%
\pgfusepath{fill}%
\end{pgfscope}%
\begin{pgfscope}%
\pgfpathrectangle{\pgfqpoint{5.200000in}{0.720000in}}{\pgfqpoint{2.640000in}{4.800000in}}%
\pgfusepath{clip}%
\pgfsetbuttcap%
\pgfsetmiterjoin%
\definecolor{currentfill}{rgb}{0.121569,0.466667,0.705882}%
\pgfsetfillcolor{currentfill}%
\pgfsetlinewidth{0.000000pt}%
\definecolor{currentstroke}{rgb}{0.000000,0.000000,0.000000}%
\pgfsetstrokecolor{currentstroke}%
\pgfsetstrokeopacity{0.000000}%
\pgfsetdash{}{0pt}%
\pgfpathmoveto{\pgfqpoint{5.200000in}{15.345889in}}%
\pgfpathlineto{\pgfqpoint{5.205128in}{15.345889in}}%
\pgfpathlineto{\pgfqpoint{5.205128in}{15.510225in}}%
\pgfpathlineto{\pgfqpoint{5.200000in}{15.510225in}}%
\pgfpathclose%
\pgfusepath{fill}%
\end{pgfscope}%
\begin{pgfscope}%
\pgfpathrectangle{\pgfqpoint{5.200000in}{0.720000in}}{\pgfqpoint{2.640000in}{4.800000in}}%
\pgfusepath{clip}%
\pgfsetbuttcap%
\pgfsetmiterjoin%
\definecolor{currentfill}{rgb}{0.121569,0.466667,0.705882}%
\pgfsetfillcolor{currentfill}%
\pgfsetlinewidth{0.000000pt}%
\definecolor{currentstroke}{rgb}{0.000000,0.000000,0.000000}%
\pgfsetstrokecolor{currentstroke}%
\pgfsetstrokeopacity{0.000000}%
\pgfsetdash{}{0pt}%
\pgfpathmoveto{\pgfqpoint{5.200000in}{15.510225in}}%
\pgfpathlineto{\pgfqpoint{5.202564in}{15.510225in}}%
\pgfpathlineto{\pgfqpoint{5.202564in}{15.674561in}}%
\pgfpathlineto{\pgfqpoint{5.200000in}{15.674561in}}%
\pgfpathclose%
\pgfusepath{fill}%
\end{pgfscope}%
\begin{pgfscope}%
\pgfpathrectangle{\pgfqpoint{5.200000in}{0.720000in}}{\pgfqpoint{2.640000in}{4.800000in}}%
\pgfusepath{clip}%
\pgfsetbuttcap%
\pgfsetmiterjoin%
\definecolor{currentfill}{rgb}{0.121569,0.466667,0.705882}%
\pgfsetfillcolor{currentfill}%
\pgfsetlinewidth{0.000000pt}%
\definecolor{currentstroke}{rgb}{0.000000,0.000000,0.000000}%
\pgfsetstrokecolor{currentstroke}%
\pgfsetstrokeopacity{0.000000}%
\pgfsetdash{}{0pt}%
\pgfpathmoveto{\pgfqpoint{5.200000in}{15.674561in}}%
\pgfpathlineto{\pgfqpoint{5.202564in}{15.674561in}}%
\pgfpathlineto{\pgfqpoint{5.202564in}{15.838897in}}%
\pgfpathlineto{\pgfqpoint{5.200000in}{15.838897in}}%
\pgfpathclose%
\pgfusepath{fill}%
\end{pgfscope}%
\begin{pgfscope}%
\definecolor{textcolor}{rgb}{0.000000,0.000000,0.000000}%
\pgfsetstrokecolor{textcolor}%
\pgfsetfillcolor{textcolor}%
\pgftext[x=6.520000in,y=0.664444in,,top]{\color{textcolor}\sffamily\fontsize{20.000000}{24.000000}\selectfont Normalized Count}%
\end{pgfscope}%
\begin{pgfscope}%
\pgfsetrectcap%
\pgfsetmiterjoin%
\pgfsetlinewidth{0.803000pt}%
\definecolor{currentstroke}{rgb}{0.000000,0.000000,0.000000}%
\pgfsetstrokecolor{currentstroke}%
\pgfsetdash{}{0pt}%
\pgfpathmoveto{\pgfqpoint{5.200000in}{0.720000in}}%
\pgfpathlineto{\pgfqpoint{5.200000in}{5.520000in}}%
\pgfusepath{stroke}%
\end{pgfscope}%
\begin{pgfscope}%
\pgfsetrectcap%
\pgfsetmiterjoin%
\pgfsetlinewidth{0.803000pt}%
\definecolor{currentstroke}{rgb}{0.000000,0.000000,0.000000}%
\pgfsetstrokecolor{currentstroke}%
\pgfsetdash{}{0pt}%
\pgfpathmoveto{\pgfqpoint{7.840000in}{0.720000in}}%
\pgfpathlineto{\pgfqpoint{7.840000in}{5.520000in}}%
\pgfusepath{stroke}%
\end{pgfscope}%
\begin{pgfscope}%
\pgfsetrectcap%
\pgfsetmiterjoin%
\pgfsetlinewidth{0.803000pt}%
\definecolor{currentstroke}{rgb}{0.000000,0.000000,0.000000}%
\pgfsetstrokecolor{currentstroke}%
\pgfsetdash{}{0pt}%
\pgfpathmoveto{\pgfqpoint{5.200000in}{0.720000in}}%
\pgfpathlineto{\pgfqpoint{7.840000in}{0.720000in}}%
\pgfusepath{stroke}%
\end{pgfscope}%
\begin{pgfscope}%
\pgfsetrectcap%
\pgfsetmiterjoin%
\pgfsetlinewidth{0.803000pt}%
\definecolor{currentstroke}{rgb}{0.000000,0.000000,0.000000}%
\pgfsetstrokecolor{currentstroke}%
\pgfsetdash{}{0pt}%
\pgfpathmoveto{\pgfqpoint{5.200000in}{5.520000in}}%
\pgfpathlineto{\pgfqpoint{7.840000in}{5.520000in}}%
\pgfusepath{stroke}%
\end{pgfscope}%
\end{pgfpicture}%
\makeatother%
\endgroup%
}
    \caption{\label{fig:fitting-npe} $D_\mathrm{w}$ histogram and its distributions conditioned \\ on $N_{\mathrm{PE}}$, errorbar explained in figure~\ref{fig:cnn-performance}.}
  \end{subfigure}
  \begin{subfigure}{.5\textwidth}
    \centering
    \resizebox{\textwidth}{!}{%% Creator: Matplotlib, PGF backend
%%
%% To include the figure in your LaTeX document, write
%%   \input{<filename>.pgf}
%%
%% Make sure the required packages are loaded in your preamble
%%   \usepackage{pgf}
%%
%% Also ensure that all the required font packages are loaded; for instance,
%% the lmodern package is sometimes necessary when using math font.
%%   \usepackage{lmodern}
%%
%% Figures using additional raster images can only be included by \input if
%% they are in the same directory as the main LaTeX file. For loading figures
%% from other directories you can use the `import` package
%%   \usepackage{import}
%%
%% and then include the figures with
%%   \import{<path to file>}{<filename>.pgf}
%%
%% Matplotlib used the following preamble
%%   \usepackage[detect-all,locale=DE]{siunitx}
%%
\begingroup%
\makeatletter%
\begin{pgfpicture}%
\pgfpathrectangle{\pgfpointorigin}{\pgfqpoint{8.000000in}{6.000000in}}%
\pgfusepath{use as bounding box, clip}%
\begin{pgfscope}%
\pgfsetbuttcap%
\pgfsetmiterjoin%
\definecolor{currentfill}{rgb}{1.000000,1.000000,1.000000}%
\pgfsetfillcolor{currentfill}%
\pgfsetlinewidth{0.000000pt}%
\definecolor{currentstroke}{rgb}{1.000000,1.000000,1.000000}%
\pgfsetstrokecolor{currentstroke}%
\pgfsetdash{}{0pt}%
\pgfpathmoveto{\pgfqpoint{0.000000in}{0.000000in}}%
\pgfpathlineto{\pgfqpoint{8.000000in}{0.000000in}}%
\pgfpathlineto{\pgfqpoint{8.000000in}{6.000000in}}%
\pgfpathlineto{\pgfqpoint{0.000000in}{6.000000in}}%
\pgfpathlineto{\pgfqpoint{0.000000in}{0.000000in}}%
\pgfpathclose%
\pgfusepath{fill}%
\end{pgfscope}%
\begin{pgfscope}%
\pgfsetbuttcap%
\pgfsetmiterjoin%
\definecolor{currentfill}{rgb}{1.000000,1.000000,1.000000}%
\pgfsetfillcolor{currentfill}%
\pgfsetlinewidth{0.000000pt}%
\definecolor{currentstroke}{rgb}{0.000000,0.000000,0.000000}%
\pgfsetstrokecolor{currentstroke}%
\pgfsetstrokeopacity{0.000000}%
\pgfsetdash{}{0pt}%
\pgfpathmoveto{\pgfqpoint{1.000000in}{0.720000in}}%
\pgfpathlineto{\pgfqpoint{7.200000in}{0.720000in}}%
\pgfpathlineto{\pgfqpoint{7.200000in}{5.340000in}}%
\pgfpathlineto{\pgfqpoint{1.000000in}{5.340000in}}%
\pgfpathlineto{\pgfqpoint{1.000000in}{0.720000in}}%
\pgfpathclose%
\pgfusepath{fill}%
\end{pgfscope}%
\begin{pgfscope}%
\pgfsetbuttcap%
\pgfsetroundjoin%
\definecolor{currentfill}{rgb}{0.000000,0.000000,0.000000}%
\pgfsetfillcolor{currentfill}%
\pgfsetlinewidth{0.803000pt}%
\definecolor{currentstroke}{rgb}{0.000000,0.000000,0.000000}%
\pgfsetstrokecolor{currentstroke}%
\pgfsetdash{}{0pt}%
\pgfsys@defobject{currentmarker}{\pgfqpoint{0.000000in}{-0.048611in}}{\pgfqpoint{0.000000in}{0.000000in}}{%
\pgfpathmoveto{\pgfqpoint{0.000000in}{0.000000in}}%
\pgfpathlineto{\pgfqpoint{0.000000in}{-0.048611in}}%
\pgfusepath{stroke,fill}%
}%
\begin{pgfscope}%
\pgfsys@transformshift{1.310000in}{0.720000in}%
\pgfsys@useobject{currentmarker}{}%
\end{pgfscope}%
\end{pgfscope}%
\begin{pgfscope}%
\definecolor{textcolor}{rgb}{0.000000,0.000000,0.000000}%
\pgfsetstrokecolor{textcolor}%
\pgfsetfillcolor{textcolor}%
\pgftext[x=1.310000in,y=0.622778in,,top]{\color{textcolor}\sffamily\fontsize{20.000000}{24.000000}\selectfont \(\displaystyle {450}\)}%
\end{pgfscope}%
\begin{pgfscope}%
\pgfsetbuttcap%
\pgfsetroundjoin%
\definecolor{currentfill}{rgb}{0.000000,0.000000,0.000000}%
\pgfsetfillcolor{currentfill}%
\pgfsetlinewidth{0.803000pt}%
\definecolor{currentstroke}{rgb}{0.000000,0.000000,0.000000}%
\pgfsetstrokecolor{currentstroke}%
\pgfsetdash{}{0pt}%
\pgfsys@defobject{currentmarker}{\pgfqpoint{0.000000in}{-0.048611in}}{\pgfqpoint{0.000000in}{0.000000in}}{%
\pgfpathmoveto{\pgfqpoint{0.000000in}{0.000000in}}%
\pgfpathlineto{\pgfqpoint{0.000000in}{-0.048611in}}%
\pgfusepath{stroke,fill}%
}%
\begin{pgfscope}%
\pgfsys@transformshift{2.860000in}{0.720000in}%
\pgfsys@useobject{currentmarker}{}%
\end{pgfscope}%
\end{pgfscope}%
\begin{pgfscope}%
\definecolor{textcolor}{rgb}{0.000000,0.000000,0.000000}%
\pgfsetstrokecolor{textcolor}%
\pgfsetfillcolor{textcolor}%
\pgftext[x=2.860000in,y=0.622778in,,top]{\color{textcolor}\sffamily\fontsize{20.000000}{24.000000}\selectfont \(\displaystyle {500}\)}%
\end{pgfscope}%
\begin{pgfscope}%
\pgfsetbuttcap%
\pgfsetroundjoin%
\definecolor{currentfill}{rgb}{0.000000,0.000000,0.000000}%
\pgfsetfillcolor{currentfill}%
\pgfsetlinewidth{0.803000pt}%
\definecolor{currentstroke}{rgb}{0.000000,0.000000,0.000000}%
\pgfsetstrokecolor{currentstroke}%
\pgfsetdash{}{0pt}%
\pgfsys@defobject{currentmarker}{\pgfqpoint{0.000000in}{-0.048611in}}{\pgfqpoint{0.000000in}{0.000000in}}{%
\pgfpathmoveto{\pgfqpoint{0.000000in}{0.000000in}}%
\pgfpathlineto{\pgfqpoint{0.000000in}{-0.048611in}}%
\pgfusepath{stroke,fill}%
}%
\begin{pgfscope}%
\pgfsys@transformshift{4.410000in}{0.720000in}%
\pgfsys@useobject{currentmarker}{}%
\end{pgfscope}%
\end{pgfscope}%
\begin{pgfscope}%
\definecolor{textcolor}{rgb}{0.000000,0.000000,0.000000}%
\pgfsetstrokecolor{textcolor}%
\pgfsetfillcolor{textcolor}%
\pgftext[x=4.410000in,y=0.622778in,,top]{\color{textcolor}\sffamily\fontsize{20.000000}{24.000000}\selectfont \(\displaystyle {550}\)}%
\end{pgfscope}%
\begin{pgfscope}%
\pgfsetbuttcap%
\pgfsetroundjoin%
\definecolor{currentfill}{rgb}{0.000000,0.000000,0.000000}%
\pgfsetfillcolor{currentfill}%
\pgfsetlinewidth{0.803000pt}%
\definecolor{currentstroke}{rgb}{0.000000,0.000000,0.000000}%
\pgfsetstrokecolor{currentstroke}%
\pgfsetdash{}{0pt}%
\pgfsys@defobject{currentmarker}{\pgfqpoint{0.000000in}{-0.048611in}}{\pgfqpoint{0.000000in}{0.000000in}}{%
\pgfpathmoveto{\pgfqpoint{0.000000in}{0.000000in}}%
\pgfpathlineto{\pgfqpoint{0.000000in}{-0.048611in}}%
\pgfusepath{stroke,fill}%
}%
\begin{pgfscope}%
\pgfsys@transformshift{5.960000in}{0.720000in}%
\pgfsys@useobject{currentmarker}{}%
\end{pgfscope}%
\end{pgfscope}%
\begin{pgfscope}%
\definecolor{textcolor}{rgb}{0.000000,0.000000,0.000000}%
\pgfsetstrokecolor{textcolor}%
\pgfsetfillcolor{textcolor}%
\pgftext[x=5.960000in,y=0.622778in,,top]{\color{textcolor}\sffamily\fontsize{20.000000}{24.000000}\selectfont \(\displaystyle {600}\)}%
\end{pgfscope}%
\begin{pgfscope}%
\definecolor{textcolor}{rgb}{0.000000,0.000000,0.000000}%
\pgfsetstrokecolor{textcolor}%
\pgfsetfillcolor{textcolor}%
\pgftext[x=4.100000in,y=0.311155in,,top]{\color{textcolor}\sffamily\fontsize{20.000000}{24.000000}\selectfont \(\displaystyle \mathrm{t}/\si{ns}\)}%
\end{pgfscope}%
\begin{pgfscope}%
\pgfsetbuttcap%
\pgfsetroundjoin%
\definecolor{currentfill}{rgb}{0.000000,0.000000,0.000000}%
\pgfsetfillcolor{currentfill}%
\pgfsetlinewidth{0.803000pt}%
\definecolor{currentstroke}{rgb}{0.000000,0.000000,0.000000}%
\pgfsetstrokecolor{currentstroke}%
\pgfsetdash{}{0pt}%
\pgfsys@defobject{currentmarker}{\pgfqpoint{-0.048611in}{0.000000in}}{\pgfqpoint{-0.000000in}{0.000000in}}{%
\pgfpathmoveto{\pgfqpoint{-0.000000in}{0.000000in}}%
\pgfpathlineto{\pgfqpoint{-0.048611in}{0.000000in}}%
\pgfusepath{stroke,fill}%
}%
\begin{pgfscope}%
\pgfsys@transformshift{1.000000in}{1.097665in}%
\pgfsys@useobject{currentmarker}{}%
\end{pgfscope}%
\end{pgfscope}%
\begin{pgfscope}%
\definecolor{textcolor}{rgb}{0.000000,0.000000,0.000000}%
\pgfsetstrokecolor{textcolor}%
\pgfsetfillcolor{textcolor}%
\pgftext[x=0.770670in, y=0.997646in, left, base]{\color{textcolor}\sffamily\fontsize{20.000000}{24.000000}\selectfont \(\displaystyle {0}\)}%
\end{pgfscope}%
\begin{pgfscope}%
\pgfsetbuttcap%
\pgfsetroundjoin%
\definecolor{currentfill}{rgb}{0.000000,0.000000,0.000000}%
\pgfsetfillcolor{currentfill}%
\pgfsetlinewidth{0.803000pt}%
\definecolor{currentstroke}{rgb}{0.000000,0.000000,0.000000}%
\pgfsetstrokecolor{currentstroke}%
\pgfsetdash{}{0pt}%
\pgfsys@defobject{currentmarker}{\pgfqpoint{-0.048611in}{0.000000in}}{\pgfqpoint{-0.000000in}{0.000000in}}{%
\pgfpathmoveto{\pgfqpoint{-0.000000in}{0.000000in}}%
\pgfpathlineto{\pgfqpoint{-0.048611in}{0.000000in}}%
\pgfusepath{stroke,fill}%
}%
\begin{pgfscope}%
\pgfsys@transformshift{1.000000in}{1.966843in}%
\pgfsys@useobject{currentmarker}{}%
\end{pgfscope}%
\end{pgfscope}%
\begin{pgfscope}%
\definecolor{textcolor}{rgb}{0.000000,0.000000,0.000000}%
\pgfsetstrokecolor{textcolor}%
\pgfsetfillcolor{textcolor}%
\pgftext[x=0.638563in, y=1.866823in, left, base]{\color{textcolor}\sffamily\fontsize{20.000000}{24.000000}\selectfont \(\displaystyle {10}\)}%
\end{pgfscope}%
\begin{pgfscope}%
\pgfsetbuttcap%
\pgfsetroundjoin%
\definecolor{currentfill}{rgb}{0.000000,0.000000,0.000000}%
\pgfsetfillcolor{currentfill}%
\pgfsetlinewidth{0.803000pt}%
\definecolor{currentstroke}{rgb}{0.000000,0.000000,0.000000}%
\pgfsetstrokecolor{currentstroke}%
\pgfsetdash{}{0pt}%
\pgfsys@defobject{currentmarker}{\pgfqpoint{-0.048611in}{0.000000in}}{\pgfqpoint{-0.000000in}{0.000000in}}{%
\pgfpathmoveto{\pgfqpoint{-0.000000in}{0.000000in}}%
\pgfpathlineto{\pgfqpoint{-0.048611in}{0.000000in}}%
\pgfusepath{stroke,fill}%
}%
\begin{pgfscope}%
\pgfsys@transformshift{1.000000in}{2.836020in}%
\pgfsys@useobject{currentmarker}{}%
\end{pgfscope}%
\end{pgfscope}%
\begin{pgfscope}%
\definecolor{textcolor}{rgb}{0.000000,0.000000,0.000000}%
\pgfsetstrokecolor{textcolor}%
\pgfsetfillcolor{textcolor}%
\pgftext[x=0.638563in, y=2.736001in, left, base]{\color{textcolor}\sffamily\fontsize{20.000000}{24.000000}\selectfont \(\displaystyle {20}\)}%
\end{pgfscope}%
\begin{pgfscope}%
\pgfsetbuttcap%
\pgfsetroundjoin%
\definecolor{currentfill}{rgb}{0.000000,0.000000,0.000000}%
\pgfsetfillcolor{currentfill}%
\pgfsetlinewidth{0.803000pt}%
\definecolor{currentstroke}{rgb}{0.000000,0.000000,0.000000}%
\pgfsetstrokecolor{currentstroke}%
\pgfsetdash{}{0pt}%
\pgfsys@defobject{currentmarker}{\pgfqpoint{-0.048611in}{0.000000in}}{\pgfqpoint{-0.000000in}{0.000000in}}{%
\pgfpathmoveto{\pgfqpoint{-0.000000in}{0.000000in}}%
\pgfpathlineto{\pgfqpoint{-0.048611in}{0.000000in}}%
\pgfusepath{stroke,fill}%
}%
\begin{pgfscope}%
\pgfsys@transformshift{1.000000in}{3.705197in}%
\pgfsys@useobject{currentmarker}{}%
\end{pgfscope}%
\end{pgfscope}%
\begin{pgfscope}%
\definecolor{textcolor}{rgb}{0.000000,0.000000,0.000000}%
\pgfsetstrokecolor{textcolor}%
\pgfsetfillcolor{textcolor}%
\pgftext[x=0.638563in, y=3.605178in, left, base]{\color{textcolor}\sffamily\fontsize{20.000000}{24.000000}\selectfont \(\displaystyle {30}\)}%
\end{pgfscope}%
\begin{pgfscope}%
\pgfsetbuttcap%
\pgfsetroundjoin%
\definecolor{currentfill}{rgb}{0.000000,0.000000,0.000000}%
\pgfsetfillcolor{currentfill}%
\pgfsetlinewidth{0.803000pt}%
\definecolor{currentstroke}{rgb}{0.000000,0.000000,0.000000}%
\pgfsetstrokecolor{currentstroke}%
\pgfsetdash{}{0pt}%
\pgfsys@defobject{currentmarker}{\pgfqpoint{-0.048611in}{0.000000in}}{\pgfqpoint{-0.000000in}{0.000000in}}{%
\pgfpathmoveto{\pgfqpoint{-0.000000in}{0.000000in}}%
\pgfpathlineto{\pgfqpoint{-0.048611in}{0.000000in}}%
\pgfusepath{stroke,fill}%
}%
\begin{pgfscope}%
\pgfsys@transformshift{1.000000in}{4.574374in}%
\pgfsys@useobject{currentmarker}{}%
\end{pgfscope}%
\end{pgfscope}%
\begin{pgfscope}%
\definecolor{textcolor}{rgb}{0.000000,0.000000,0.000000}%
\pgfsetstrokecolor{textcolor}%
\pgfsetfillcolor{textcolor}%
\pgftext[x=0.638563in, y=4.474355in, left, base]{\color{textcolor}\sffamily\fontsize{20.000000}{24.000000}\selectfont \(\displaystyle {40}\)}%
\end{pgfscope}%
\begin{pgfscope}%
\definecolor{textcolor}{rgb}{0.000000,0.000000,0.000000}%
\pgfsetstrokecolor{textcolor}%
\pgfsetfillcolor{textcolor}%
\pgftext[x=0.583007in,y=3.030000in,,bottom,rotate=90.000000]{\color{textcolor}\sffamily\fontsize{20.000000}{24.000000}\selectfont \(\displaystyle \mathrm{Voltage}/\si{mV}\)}%
\end{pgfscope}%
\begin{pgfscope}%
\pgfpathrectangle{\pgfqpoint{1.000000in}{0.720000in}}{\pgfqpoint{6.200000in}{4.620000in}}%
\pgfusepath{clip}%
\pgfsetrectcap%
\pgfsetroundjoin%
\pgfsetlinewidth{2.007500pt}%
\definecolor{currentstroke}{rgb}{0.121569,0.466667,0.705882}%
\pgfsetstrokecolor{currentstroke}%
\pgfsetdash{}{0pt}%
\pgfpathmoveto{\pgfqpoint{0.990000in}{1.151960in}}%
\pgfpathlineto{\pgfqpoint{1.000000in}{1.186125in}}%
\pgfpathlineto{\pgfqpoint{1.031000in}{1.062899in}}%
\pgfpathlineto{\pgfqpoint{1.062000in}{1.135415in}}%
\pgfpathlineto{\pgfqpoint{1.093000in}{1.123369in}}%
\pgfpathlineto{\pgfqpoint{1.124000in}{1.109740in}}%
\pgfpathlineto{\pgfqpoint{1.155000in}{1.195842in}}%
\pgfpathlineto{\pgfqpoint{1.186000in}{1.120047in}}%
\pgfpathlineto{\pgfqpoint{1.217000in}{1.059476in}}%
\pgfpathlineto{\pgfqpoint{1.248000in}{1.184397in}}%
\pgfpathlineto{\pgfqpoint{1.279000in}{1.120885in}}%
\pgfpathlineto{\pgfqpoint{1.310000in}{1.046167in}}%
\pgfpathlineto{\pgfqpoint{1.341000in}{1.140849in}}%
\pgfpathlineto{\pgfqpoint{1.372000in}{1.082852in}}%
\pgfpathlineto{\pgfqpoint{1.403000in}{1.288953in}}%
\pgfpathlineto{\pgfqpoint{1.434000in}{1.042171in}}%
\pgfpathlineto{\pgfqpoint{1.465000in}{0.987254in}}%
\pgfpathlineto{\pgfqpoint{1.496000in}{1.062103in}}%
\pgfpathlineto{\pgfqpoint{1.527000in}{1.012187in}}%
\pgfpathlineto{\pgfqpoint{1.558000in}{0.988212in}}%
\pgfpathlineto{\pgfqpoint{1.589000in}{1.089773in}}%
\pgfpathlineto{\pgfqpoint{1.620000in}{1.168761in}}%
\pgfpathlineto{\pgfqpoint{1.651000in}{1.147787in}}%
\pgfpathlineto{\pgfqpoint{1.682000in}{1.249036in}}%
\pgfpathlineto{\pgfqpoint{1.713000in}{1.172349in}}%
\pgfpathlineto{\pgfqpoint{1.744000in}{1.194647in}}%
\pgfpathlineto{\pgfqpoint{1.775000in}{1.207704in}}%
\pgfpathlineto{\pgfqpoint{1.806000in}{1.178131in}}%
\pgfpathlineto{\pgfqpoint{1.837000in}{1.080698in}}%
\pgfpathlineto{\pgfqpoint{1.868000in}{0.991728in}}%
\pgfpathlineto{\pgfqpoint{1.899000in}{1.115273in}}%
\pgfpathlineto{\pgfqpoint{1.930000in}{1.071294in}}%
\pgfpathlineto{\pgfqpoint{1.961000in}{1.174026in}}%
\pgfpathlineto{\pgfqpoint{1.992000in}{1.168862in}}%
\pgfpathlineto{\pgfqpoint{2.023000in}{1.365018in}}%
\pgfpathlineto{\pgfqpoint{2.054000in}{1.690713in}}%
\pgfpathlineto{\pgfqpoint{2.085000in}{2.067470in}}%
\pgfpathlineto{\pgfqpoint{2.116000in}{2.724481in}}%
\pgfpathlineto{\pgfqpoint{2.147000in}{2.890566in}}%
\pgfpathlineto{\pgfqpoint{2.178000in}{3.262227in}}%
\pgfpathlineto{\pgfqpoint{2.209000in}{3.379761in}}%
\pgfpathlineto{\pgfqpoint{2.240000in}{3.163011in}}%
\pgfpathlineto{\pgfqpoint{2.271000in}{3.147558in}}%
\pgfpathlineto{\pgfqpoint{2.302000in}{3.074916in}}%
\pgfpathlineto{\pgfqpoint{2.333000in}{2.621453in}}%
\pgfpathlineto{\pgfqpoint{2.364000in}{2.464692in}}%
\pgfpathlineto{\pgfqpoint{2.395000in}{2.454597in}}%
\pgfpathlineto{\pgfqpoint{2.426000in}{2.189955in}}%
\pgfpathlineto{\pgfqpoint{2.457000in}{1.844485in}}%
\pgfpathlineto{\pgfqpoint{2.488000in}{1.762348in}}%
\pgfpathlineto{\pgfqpoint{2.519000in}{1.687919in}}%
\pgfpathlineto{\pgfqpoint{2.550000in}{1.647781in}}%
\pgfpathlineto{\pgfqpoint{2.581000in}{1.846398in}}%
\pgfpathlineto{\pgfqpoint{2.612000in}{2.370379in}}%
\pgfpathlineto{\pgfqpoint{2.643000in}{2.590079in}}%
\pgfpathlineto{\pgfqpoint{2.674000in}{2.950153in}}%
\pgfpathlineto{\pgfqpoint{2.705000in}{3.031888in}}%
\pgfpathlineto{\pgfqpoint{2.736000in}{2.928036in}}%
\pgfpathlineto{\pgfqpoint{2.767000in}{2.779330in}}%
\pgfpathlineto{\pgfqpoint{2.798000in}{2.649887in}}%
\pgfpathlineto{\pgfqpoint{2.829000in}{2.498836in}}%
\pgfpathlineto{\pgfqpoint{2.860000in}{2.391904in}}%
\pgfpathlineto{\pgfqpoint{2.891000in}{2.136122in}}%
\pgfpathlineto{\pgfqpoint{2.922000in}{2.178534in}}%
\pgfpathlineto{\pgfqpoint{2.953000in}{1.995645in}}%
\pgfpathlineto{\pgfqpoint{2.984000in}{2.470471in}}%
\pgfpathlineto{\pgfqpoint{3.015000in}{2.476422in}}%
\pgfpathlineto{\pgfqpoint{3.046000in}{2.769870in}}%
\pgfpathlineto{\pgfqpoint{3.077000in}{2.652978in}}%
\pgfpathlineto{\pgfqpoint{3.108000in}{2.673169in}}%
\pgfpathlineto{\pgfqpoint{3.139000in}{2.601055in}}%
\pgfpathlineto{\pgfqpoint{3.170000in}{2.362125in}}%
\pgfpathlineto{\pgfqpoint{3.201000in}{2.259230in}}%
\pgfpathlineto{\pgfqpoint{3.232000in}{2.011916in}}%
\pgfpathlineto{\pgfqpoint{3.263000in}{1.881546in}}%
\pgfpathlineto{\pgfqpoint{3.294000in}{1.739309in}}%
\pgfpathlineto{\pgfqpoint{3.325000in}{1.540859in}}%
\pgfpathlineto{\pgfqpoint{3.356000in}{1.651436in}}%
\pgfpathlineto{\pgfqpoint{3.387000in}{1.664749in}}%
\pgfpathlineto{\pgfqpoint{3.418000in}{1.457141in}}%
\pgfpathlineto{\pgfqpoint{3.449000in}{1.274001in}}%
\pgfpathlineto{\pgfqpoint{3.480000in}{1.231827in}}%
\pgfpathlineto{\pgfqpoint{3.511000in}{1.424988in}}%
\pgfpathlineto{\pgfqpoint{3.542000in}{1.215802in}}%
\pgfpathlineto{\pgfqpoint{3.573000in}{1.195673in}}%
\pgfpathlineto{\pgfqpoint{3.604000in}{1.310742in}}%
\pgfpathlineto{\pgfqpoint{3.635000in}{1.157800in}}%
\pgfpathlineto{\pgfqpoint{3.666000in}{1.118229in}}%
\pgfpathlineto{\pgfqpoint{3.697000in}{1.179889in}}%
\pgfpathlineto{\pgfqpoint{3.728000in}{1.112307in}}%
\pgfpathlineto{\pgfqpoint{3.759000in}{1.060445in}}%
\pgfpathlineto{\pgfqpoint{3.790000in}{1.300916in}}%
\pgfpathlineto{\pgfqpoint{3.821000in}{1.389756in}}%
\pgfpathlineto{\pgfqpoint{3.852000in}{1.822560in}}%
\pgfpathlineto{\pgfqpoint{3.883000in}{2.022729in}}%
\pgfpathlineto{\pgfqpoint{3.914000in}{2.442705in}}%
\pgfpathlineto{\pgfqpoint{3.945000in}{2.405788in}}%
\pgfpathlineto{\pgfqpoint{3.976000in}{2.669575in}}%
\pgfpathlineto{\pgfqpoint{4.007000in}{2.551921in}}%
\pgfpathlineto{\pgfqpoint{4.038000in}{2.231146in}}%
\pgfpathlineto{\pgfqpoint{4.069000in}{2.230027in}}%
\pgfpathlineto{\pgfqpoint{4.100000in}{2.116019in}}%
\pgfpathlineto{\pgfqpoint{4.131000in}{1.921564in}}%
\pgfpathlineto{\pgfqpoint{4.162000in}{1.758902in}}%
\pgfpathlineto{\pgfqpoint{4.193000in}{1.702715in}}%
\pgfpathlineto{\pgfqpoint{4.224000in}{1.743587in}}%
\pgfpathlineto{\pgfqpoint{4.255000in}{1.684562in}}%
\pgfpathlineto{\pgfqpoint{4.286000in}{1.406640in}}%
\pgfpathlineto{\pgfqpoint{4.317000in}{1.378442in}}%
\pgfpathlineto{\pgfqpoint{4.348000in}{1.327012in}}%
\pgfpathlineto{\pgfqpoint{4.379000in}{1.253234in}}%
\pgfpathlineto{\pgfqpoint{4.410000in}{1.150263in}}%
\pgfpathlineto{\pgfqpoint{4.441000in}{1.308629in}}%
\pgfpathlineto{\pgfqpoint{4.472000in}{1.212851in}}%
\pgfpathlineto{\pgfqpoint{4.534000in}{1.098863in}}%
\pgfpathlineto{\pgfqpoint{4.565000in}{1.038354in}}%
\pgfpathlineto{\pgfqpoint{4.596000in}{1.236816in}}%
\pgfpathlineto{\pgfqpoint{4.627000in}{1.250022in}}%
\pgfpathlineto{\pgfqpoint{4.658000in}{1.088029in}}%
\pgfpathlineto{\pgfqpoint{4.689000in}{1.200525in}}%
\pgfpathlineto{\pgfqpoint{4.720000in}{0.960533in}}%
\pgfpathlineto{\pgfqpoint{4.751000in}{1.091762in}}%
\pgfpathlineto{\pgfqpoint{4.782000in}{1.112842in}}%
\pgfpathlineto{\pgfqpoint{4.813000in}{0.999892in}}%
\pgfpathlineto{\pgfqpoint{4.844000in}{1.053161in}}%
\pgfpathlineto{\pgfqpoint{4.875000in}{0.846655in}}%
\pgfpathlineto{\pgfqpoint{4.906000in}{1.001855in}}%
\pgfpathlineto{\pgfqpoint{4.937000in}{1.025292in}}%
\pgfpathlineto{\pgfqpoint{4.968000in}{0.934878in}}%
\pgfpathlineto{\pgfqpoint{4.999000in}{1.254352in}}%
\pgfpathlineto{\pgfqpoint{5.030000in}{1.148710in}}%
\pgfpathlineto{\pgfqpoint{5.061000in}{1.092573in}}%
\pgfpathlineto{\pgfqpoint{5.092000in}{1.165490in}}%
\pgfpathlineto{\pgfqpoint{5.123000in}{0.965994in}}%
\pgfpathlineto{\pgfqpoint{5.154000in}{1.044865in}}%
\pgfpathlineto{\pgfqpoint{5.185000in}{0.998876in}}%
\pgfpathlineto{\pgfqpoint{5.216000in}{0.931155in}}%
\pgfpathlineto{\pgfqpoint{5.247000in}{1.089159in}}%
\pgfpathlineto{\pgfqpoint{5.278000in}{1.109680in}}%
\pgfpathlineto{\pgfqpoint{5.309000in}{1.177106in}}%
\pgfpathlineto{\pgfqpoint{5.371000in}{1.078157in}}%
\pgfpathlineto{\pgfqpoint{5.402000in}{1.145350in}}%
\pgfpathlineto{\pgfqpoint{5.433000in}{1.166896in}}%
\pgfpathlineto{\pgfqpoint{5.464000in}{1.042336in}}%
\pgfpathlineto{\pgfqpoint{5.495000in}{1.015832in}}%
\pgfpathlineto{\pgfqpoint{5.526000in}{1.043377in}}%
\pgfpathlineto{\pgfqpoint{5.557000in}{1.072463in}}%
\pgfpathlineto{\pgfqpoint{5.588000in}{1.008951in}}%
\pgfpathlineto{\pgfqpoint{5.619000in}{0.921030in}}%
\pgfpathlineto{\pgfqpoint{5.650000in}{1.050411in}}%
\pgfpathlineto{\pgfqpoint{5.681000in}{1.054108in}}%
\pgfpathlineto{\pgfqpoint{5.712000in}{1.075896in}}%
\pgfpathlineto{\pgfqpoint{5.743000in}{1.115058in}}%
\pgfpathlineto{\pgfqpoint{5.774000in}{1.226689in}}%
\pgfpathlineto{\pgfqpoint{5.805000in}{1.088322in}}%
\pgfpathlineto{\pgfqpoint{5.836000in}{1.030756in}}%
\pgfpathlineto{\pgfqpoint{5.867000in}{1.055714in}}%
\pgfpathlineto{\pgfqpoint{5.898000in}{1.199851in}}%
\pgfpathlineto{\pgfqpoint{5.929000in}{1.190594in}}%
\pgfpathlineto{\pgfqpoint{5.960000in}{1.048324in}}%
\pgfpathlineto{\pgfqpoint{5.991000in}{1.019408in}}%
\pgfpathlineto{\pgfqpoint{6.022000in}{1.070103in}}%
\pgfpathlineto{\pgfqpoint{6.053000in}{1.000761in}}%
\pgfpathlineto{\pgfqpoint{6.084000in}{1.022737in}}%
\pgfpathlineto{\pgfqpoint{6.115000in}{1.080017in}}%
\pgfpathlineto{\pgfqpoint{6.146000in}{1.112647in}}%
\pgfpathlineto{\pgfqpoint{6.177000in}{1.064582in}}%
\pgfpathlineto{\pgfqpoint{6.208000in}{1.149532in}}%
\pgfpathlineto{\pgfqpoint{6.239000in}{1.016023in}}%
\pgfpathlineto{\pgfqpoint{6.270000in}{1.173698in}}%
\pgfpathlineto{\pgfqpoint{6.301000in}{0.980804in}}%
\pgfpathlineto{\pgfqpoint{6.332000in}{1.036856in}}%
\pgfpathlineto{\pgfqpoint{6.363000in}{1.054359in}}%
\pgfpathlineto{\pgfqpoint{6.394000in}{1.048730in}}%
\pgfpathlineto{\pgfqpoint{6.425000in}{1.232806in}}%
\pgfpathlineto{\pgfqpoint{6.456000in}{1.111368in}}%
\pgfpathlineto{\pgfqpoint{6.487000in}{1.186175in}}%
\pgfpathlineto{\pgfqpoint{6.518000in}{1.142683in}}%
\pgfpathlineto{\pgfqpoint{6.549000in}{1.149602in}}%
\pgfpathlineto{\pgfqpoint{6.580000in}{1.106709in}}%
\pgfpathlineto{\pgfqpoint{6.611000in}{1.023128in}}%
\pgfpathlineto{\pgfqpoint{6.642000in}{1.152338in}}%
\pgfpathlineto{\pgfqpoint{6.673000in}{1.064955in}}%
\pgfpathlineto{\pgfqpoint{6.704000in}{0.950912in}}%
\pgfpathlineto{\pgfqpoint{6.735000in}{1.095036in}}%
\pgfpathlineto{\pgfqpoint{6.766000in}{1.179211in}}%
\pgfpathlineto{\pgfqpoint{6.797000in}{1.152225in}}%
\pgfpathlineto{\pgfqpoint{6.828000in}{1.143464in}}%
\pgfpathlineto{\pgfqpoint{6.859000in}{0.942275in}}%
\pgfpathlineto{\pgfqpoint{6.890000in}{1.107313in}}%
\pgfpathlineto{\pgfqpoint{6.921000in}{0.889117in}}%
\pgfpathlineto{\pgfqpoint{6.952000in}{1.107291in}}%
\pgfpathlineto{\pgfqpoint{6.983000in}{0.990080in}}%
\pgfpathlineto{\pgfqpoint{7.014000in}{1.088183in}}%
\pgfpathlineto{\pgfqpoint{7.045000in}{1.113167in}}%
\pgfpathlineto{\pgfqpoint{7.076000in}{1.033104in}}%
\pgfpathlineto{\pgfqpoint{7.107000in}{1.022218in}}%
\pgfpathlineto{\pgfqpoint{7.138000in}{1.154930in}}%
\pgfpathlineto{\pgfqpoint{7.169000in}{1.256429in}}%
\pgfpathlineto{\pgfqpoint{7.200000in}{1.153926in}}%
\pgfpathlineto{\pgfqpoint{7.210000in}{1.093785in}}%
\pgfpathlineto{\pgfqpoint{7.210000in}{1.093785in}}%
\pgfusepath{stroke}%
\end{pgfscope}%
\begin{pgfscope}%
\pgfsetrectcap%
\pgfsetmiterjoin%
\pgfsetlinewidth{0.803000pt}%
\definecolor{currentstroke}{rgb}{0.000000,0.000000,0.000000}%
\pgfsetstrokecolor{currentstroke}%
\pgfsetdash{}{0pt}%
\pgfpathmoveto{\pgfqpoint{1.000000in}{0.720000in}}%
\pgfpathlineto{\pgfqpoint{1.000000in}{5.340000in}}%
\pgfusepath{stroke}%
\end{pgfscope}%
\begin{pgfscope}%
\pgfsetrectcap%
\pgfsetmiterjoin%
\pgfsetlinewidth{0.803000pt}%
\definecolor{currentstroke}{rgb}{0.000000,0.000000,0.000000}%
\pgfsetstrokecolor{currentstroke}%
\pgfsetdash{}{0pt}%
\pgfpathmoveto{\pgfqpoint{7.200000in}{0.720000in}}%
\pgfpathlineto{\pgfqpoint{7.200000in}{5.340000in}}%
\pgfusepath{stroke}%
\end{pgfscope}%
\begin{pgfscope}%
\pgfsetrectcap%
\pgfsetmiterjoin%
\pgfsetlinewidth{0.803000pt}%
\definecolor{currentstroke}{rgb}{0.000000,0.000000,0.000000}%
\pgfsetstrokecolor{currentstroke}%
\pgfsetdash{}{0pt}%
\pgfpathmoveto{\pgfqpoint{1.000000in}{0.720000in}}%
\pgfpathlineto{\pgfqpoint{7.200000in}{0.720000in}}%
\pgfusepath{stroke}%
\end{pgfscope}%
\begin{pgfscope}%
\pgfsetrectcap%
\pgfsetmiterjoin%
\pgfsetlinewidth{0.803000pt}%
\definecolor{currentstroke}{rgb}{0.000000,0.000000,0.000000}%
\pgfsetstrokecolor{currentstroke}%
\pgfsetdash{}{0pt}%
\pgfpathmoveto{\pgfqpoint{1.000000in}{5.340000in}}%
\pgfpathlineto{\pgfqpoint{7.200000in}{5.340000in}}%
\pgfusepath{stroke}%
\end{pgfscope}%
\begin{pgfscope}%
\pgfsetbuttcap%
\pgfsetroundjoin%
\definecolor{currentfill}{rgb}{0.000000,0.000000,0.000000}%
\pgfsetfillcolor{currentfill}%
\pgfsetlinewidth{0.803000pt}%
\definecolor{currentstroke}{rgb}{0.000000,0.000000,0.000000}%
\pgfsetstrokecolor{currentstroke}%
\pgfsetdash{}{0pt}%
\pgfsys@defobject{currentmarker}{\pgfqpoint{0.000000in}{0.000000in}}{\pgfqpoint{0.048611in}{0.000000in}}{%
\pgfpathmoveto{\pgfqpoint{0.000000in}{0.000000in}}%
\pgfpathlineto{\pgfqpoint{0.048611in}{0.000000in}}%
\pgfusepath{stroke,fill}%
}%
\begin{pgfscope}%
\pgfsys@transformshift{7.200000in}{1.097665in}%
\pgfsys@useobject{currentmarker}{}%
\end{pgfscope}%
\end{pgfscope}%
\begin{pgfscope}%
\definecolor{textcolor}{rgb}{0.000000,0.000000,0.000000}%
\pgfsetstrokecolor{textcolor}%
\pgfsetfillcolor{textcolor}%
\pgftext[x=7.297222in, y=0.997646in, left, base]{\color{textcolor}\sffamily\fontsize{20.000000}{24.000000}\selectfont 0.0}%
\end{pgfscope}%
\begin{pgfscope}%
\pgfsetbuttcap%
\pgfsetroundjoin%
\definecolor{currentfill}{rgb}{0.000000,0.000000,0.000000}%
\pgfsetfillcolor{currentfill}%
\pgfsetlinewidth{0.803000pt}%
\definecolor{currentstroke}{rgb}{0.000000,0.000000,0.000000}%
\pgfsetstrokecolor{currentstroke}%
\pgfsetdash{}{0pt}%
\pgfsys@defobject{currentmarker}{\pgfqpoint{0.000000in}{0.000000in}}{\pgfqpoint{0.048611in}{0.000000in}}{%
\pgfpathmoveto{\pgfqpoint{0.000000in}{0.000000in}}%
\pgfpathlineto{\pgfqpoint{0.048611in}{0.000000in}}%
\pgfusepath{stroke,fill}%
}%
\begin{pgfscope}%
\pgfsys@transformshift{7.200000in}{1.721538in}%
\pgfsys@useobject{currentmarker}{}%
\end{pgfscope}%
\end{pgfscope}%
\begin{pgfscope}%
\definecolor{textcolor}{rgb}{0.000000,0.000000,0.000000}%
\pgfsetstrokecolor{textcolor}%
\pgfsetfillcolor{textcolor}%
\pgftext[x=7.297222in, y=1.621519in, left, base]{\color{textcolor}\sffamily\fontsize{20.000000}{24.000000}\selectfont 0.2}%
\end{pgfscope}%
\begin{pgfscope}%
\pgfsetbuttcap%
\pgfsetroundjoin%
\definecolor{currentfill}{rgb}{0.000000,0.000000,0.000000}%
\pgfsetfillcolor{currentfill}%
\pgfsetlinewidth{0.803000pt}%
\definecolor{currentstroke}{rgb}{0.000000,0.000000,0.000000}%
\pgfsetstrokecolor{currentstroke}%
\pgfsetdash{}{0pt}%
\pgfsys@defobject{currentmarker}{\pgfqpoint{0.000000in}{0.000000in}}{\pgfqpoint{0.048611in}{0.000000in}}{%
\pgfpathmoveto{\pgfqpoint{0.000000in}{0.000000in}}%
\pgfpathlineto{\pgfqpoint{0.048611in}{0.000000in}}%
\pgfusepath{stroke,fill}%
}%
\begin{pgfscope}%
\pgfsys@transformshift{7.200000in}{2.345411in}%
\pgfsys@useobject{currentmarker}{}%
\end{pgfscope}%
\end{pgfscope}%
\begin{pgfscope}%
\definecolor{textcolor}{rgb}{0.000000,0.000000,0.000000}%
\pgfsetstrokecolor{textcolor}%
\pgfsetfillcolor{textcolor}%
\pgftext[x=7.297222in, y=2.245392in, left, base]{\color{textcolor}\sffamily\fontsize{20.000000}{24.000000}\selectfont 0.5}%
\end{pgfscope}%
\begin{pgfscope}%
\pgfsetbuttcap%
\pgfsetroundjoin%
\definecolor{currentfill}{rgb}{0.000000,0.000000,0.000000}%
\pgfsetfillcolor{currentfill}%
\pgfsetlinewidth{0.803000pt}%
\definecolor{currentstroke}{rgb}{0.000000,0.000000,0.000000}%
\pgfsetstrokecolor{currentstroke}%
\pgfsetdash{}{0pt}%
\pgfsys@defobject{currentmarker}{\pgfqpoint{0.000000in}{0.000000in}}{\pgfqpoint{0.048611in}{0.000000in}}{%
\pgfpathmoveto{\pgfqpoint{0.000000in}{0.000000in}}%
\pgfpathlineto{\pgfqpoint{0.048611in}{0.000000in}}%
\pgfusepath{stroke,fill}%
}%
\begin{pgfscope}%
\pgfsys@transformshift{7.200000in}{2.969284in}%
\pgfsys@useobject{currentmarker}{}%
\end{pgfscope}%
\end{pgfscope}%
\begin{pgfscope}%
\definecolor{textcolor}{rgb}{0.000000,0.000000,0.000000}%
\pgfsetstrokecolor{textcolor}%
\pgfsetfillcolor{textcolor}%
\pgftext[x=7.297222in, y=2.869264in, left, base]{\color{textcolor}\sffamily\fontsize{20.000000}{24.000000}\selectfont 0.8}%
\end{pgfscope}%
\begin{pgfscope}%
\pgfsetbuttcap%
\pgfsetroundjoin%
\definecolor{currentfill}{rgb}{0.000000,0.000000,0.000000}%
\pgfsetfillcolor{currentfill}%
\pgfsetlinewidth{0.803000pt}%
\definecolor{currentstroke}{rgb}{0.000000,0.000000,0.000000}%
\pgfsetstrokecolor{currentstroke}%
\pgfsetdash{}{0pt}%
\pgfsys@defobject{currentmarker}{\pgfqpoint{0.000000in}{0.000000in}}{\pgfqpoint{0.048611in}{0.000000in}}{%
\pgfpathmoveto{\pgfqpoint{0.000000in}{0.000000in}}%
\pgfpathlineto{\pgfqpoint{0.048611in}{0.000000in}}%
\pgfusepath{stroke,fill}%
}%
\begin{pgfscope}%
\pgfsys@transformshift{7.200000in}{3.593156in}%
\pgfsys@useobject{currentmarker}{}%
\end{pgfscope}%
\end{pgfscope}%
\begin{pgfscope}%
\definecolor{textcolor}{rgb}{0.000000,0.000000,0.000000}%
\pgfsetstrokecolor{textcolor}%
\pgfsetfillcolor{textcolor}%
\pgftext[x=7.297222in, y=3.493137in, left, base]{\color{textcolor}\sffamily\fontsize{20.000000}{24.000000}\selectfont 1.0}%
\end{pgfscope}%
\begin{pgfscope}%
\pgfsetbuttcap%
\pgfsetroundjoin%
\definecolor{currentfill}{rgb}{0.000000,0.000000,0.000000}%
\pgfsetfillcolor{currentfill}%
\pgfsetlinewidth{0.803000pt}%
\definecolor{currentstroke}{rgb}{0.000000,0.000000,0.000000}%
\pgfsetstrokecolor{currentstroke}%
\pgfsetdash{}{0pt}%
\pgfsys@defobject{currentmarker}{\pgfqpoint{0.000000in}{0.000000in}}{\pgfqpoint{0.048611in}{0.000000in}}{%
\pgfpathmoveto{\pgfqpoint{0.000000in}{0.000000in}}%
\pgfpathlineto{\pgfqpoint{0.048611in}{0.000000in}}%
\pgfusepath{stroke,fill}%
}%
\begin{pgfscope}%
\pgfsys@transformshift{7.200000in}{4.217029in}%
\pgfsys@useobject{currentmarker}{}%
\end{pgfscope}%
\end{pgfscope}%
\begin{pgfscope}%
\definecolor{textcolor}{rgb}{0.000000,0.000000,0.000000}%
\pgfsetstrokecolor{textcolor}%
\pgfsetfillcolor{textcolor}%
\pgftext[x=7.297222in, y=4.117010in, left, base]{\color{textcolor}\sffamily\fontsize{20.000000}{24.000000}\selectfont 1.2}%
\end{pgfscope}%
\begin{pgfscope}%
\pgfsetbuttcap%
\pgfsetroundjoin%
\definecolor{currentfill}{rgb}{0.000000,0.000000,0.000000}%
\pgfsetfillcolor{currentfill}%
\pgfsetlinewidth{0.803000pt}%
\definecolor{currentstroke}{rgb}{0.000000,0.000000,0.000000}%
\pgfsetstrokecolor{currentstroke}%
\pgfsetdash{}{0pt}%
\pgfsys@defobject{currentmarker}{\pgfqpoint{0.000000in}{0.000000in}}{\pgfqpoint{0.048611in}{0.000000in}}{%
\pgfpathmoveto{\pgfqpoint{0.000000in}{0.000000in}}%
\pgfpathlineto{\pgfqpoint{0.048611in}{0.000000in}}%
\pgfusepath{stroke,fill}%
}%
\begin{pgfscope}%
\pgfsys@transformshift{7.200000in}{4.840902in}%
\pgfsys@useobject{currentmarker}{}%
\end{pgfscope}%
\end{pgfscope}%
\begin{pgfscope}%
\definecolor{textcolor}{rgb}{0.000000,0.000000,0.000000}%
\pgfsetstrokecolor{textcolor}%
\pgfsetfillcolor{textcolor}%
\pgftext[x=7.297222in, y=4.740883in, left, base]{\color{textcolor}\sffamily\fontsize{20.000000}{24.000000}\selectfont 1.5}%
\end{pgfscope}%
\begin{pgfscope}%
\definecolor{textcolor}{rgb}{0.000000,0.000000,0.000000}%
\pgfsetstrokecolor{textcolor}%
\pgfsetfillcolor{textcolor}%
\pgftext[x=7.698906in,y=3.030000in,,top,rotate=90.000000]{\color{textcolor}\sffamily\fontsize{20.000000}{24.000000}\selectfont \(\displaystyle \mathrm{Charge}\)}%
\end{pgfscope}%
\begin{pgfscope}%
\pgfpathrectangle{\pgfqpoint{1.000000in}{0.720000in}}{\pgfqpoint{6.200000in}{4.620000in}}%
\pgfusepath{clip}%
\pgfsetbuttcap%
\pgfsetroundjoin%
\pgfsetlinewidth{0.501875pt}%
\definecolor{currentstroke}{rgb}{1.000000,0.000000,0.000000}%
\pgfsetstrokecolor{currentstroke}%
\pgfsetdash{}{0pt}%
\pgfpathmoveto{\pgfqpoint{1.837000in}{1.097665in}}%
\pgfpathlineto{\pgfqpoint{1.837000in}{1.305055in}}%
\pgfusepath{stroke}%
\end{pgfscope}%
\begin{pgfscope}%
\pgfpathrectangle{\pgfqpoint{1.000000in}{0.720000in}}{\pgfqpoint{6.200000in}{4.620000in}}%
\pgfusepath{clip}%
\pgfsetbuttcap%
\pgfsetroundjoin%
\pgfsetlinewidth{0.501875pt}%
\definecolor{currentstroke}{rgb}{1.000000,0.000000,0.000000}%
\pgfsetstrokecolor{currentstroke}%
\pgfsetdash{}{0pt}%
\pgfpathmoveto{\pgfqpoint{1.868000in}{1.097665in}}%
\pgfpathlineto{\pgfqpoint{1.868000in}{1.098469in}}%
\pgfusepath{stroke}%
\end{pgfscope}%
\begin{pgfscope}%
\pgfpathrectangle{\pgfqpoint{1.000000in}{0.720000in}}{\pgfqpoint{6.200000in}{4.620000in}}%
\pgfusepath{clip}%
\pgfsetbuttcap%
\pgfsetroundjoin%
\pgfsetlinewidth{0.501875pt}%
\definecolor{currentstroke}{rgb}{1.000000,0.000000,0.000000}%
\pgfsetstrokecolor{currentstroke}%
\pgfsetdash{}{0pt}%
\pgfpathmoveto{\pgfqpoint{1.930000in}{1.097665in}}%
\pgfpathlineto{\pgfqpoint{1.930000in}{3.537629in}}%
\pgfusepath{stroke}%
\end{pgfscope}%
\begin{pgfscope}%
\pgfpathrectangle{\pgfqpoint{1.000000in}{0.720000in}}{\pgfqpoint{6.200000in}{4.620000in}}%
\pgfusepath{clip}%
\pgfsetbuttcap%
\pgfsetroundjoin%
\pgfsetlinewidth{0.501875pt}%
\definecolor{currentstroke}{rgb}{1.000000,0.000000,0.000000}%
\pgfsetstrokecolor{currentstroke}%
\pgfsetdash{}{0pt}%
\pgfpathmoveto{\pgfqpoint{1.961000in}{1.097665in}}%
\pgfpathlineto{\pgfqpoint{1.961000in}{1.136993in}}%
\pgfusepath{stroke}%
\end{pgfscope}%
\begin{pgfscope}%
\pgfpathrectangle{\pgfqpoint{1.000000in}{0.720000in}}{\pgfqpoint{6.200000in}{4.620000in}}%
\pgfusepath{clip}%
\pgfsetbuttcap%
\pgfsetroundjoin%
\pgfsetlinewidth{0.501875pt}%
\definecolor{currentstroke}{rgb}{1.000000,0.000000,0.000000}%
\pgfsetstrokecolor{currentstroke}%
\pgfsetdash{}{0pt}%
\pgfpathmoveto{\pgfqpoint{1.992000in}{1.097665in}}%
\pgfpathlineto{\pgfqpoint{1.992000in}{3.140324in}}%
\pgfusepath{stroke}%
\end{pgfscope}%
\begin{pgfscope}%
\pgfpathrectangle{\pgfqpoint{1.000000in}{0.720000in}}{\pgfqpoint{6.200000in}{4.620000in}}%
\pgfusepath{clip}%
\pgfsetbuttcap%
\pgfsetroundjoin%
\pgfsetlinewidth{0.501875pt}%
\definecolor{currentstroke}{rgb}{1.000000,0.000000,0.000000}%
\pgfsetstrokecolor{currentstroke}%
\pgfsetdash{}{0pt}%
\pgfpathmoveto{\pgfqpoint{2.457000in}{1.097665in}}%
\pgfpathlineto{\pgfqpoint{2.457000in}{2.581638in}}%
\pgfusepath{stroke}%
\end{pgfscope}%
\begin{pgfscope}%
\pgfpathrectangle{\pgfqpoint{1.000000in}{0.720000in}}{\pgfqpoint{6.200000in}{4.620000in}}%
\pgfusepath{clip}%
\pgfsetbuttcap%
\pgfsetroundjoin%
\pgfsetlinewidth{0.501875pt}%
\definecolor{currentstroke}{rgb}{1.000000,0.000000,0.000000}%
\pgfsetstrokecolor{currentstroke}%
\pgfsetdash{}{0pt}%
\pgfpathmoveto{\pgfqpoint{2.488000in}{1.097665in}}%
\pgfpathlineto{\pgfqpoint{2.488000in}{3.079287in}}%
\pgfusepath{stroke}%
\end{pgfscope}%
\begin{pgfscope}%
\pgfpathrectangle{\pgfqpoint{1.000000in}{0.720000in}}{\pgfqpoint{6.200000in}{4.620000in}}%
\pgfusepath{clip}%
\pgfsetbuttcap%
\pgfsetroundjoin%
\pgfsetlinewidth{0.501875pt}%
\definecolor{currentstroke}{rgb}{1.000000,0.000000,0.000000}%
\pgfsetstrokecolor{currentstroke}%
\pgfsetdash{}{0pt}%
\pgfpathmoveto{\pgfqpoint{2.829000in}{1.097665in}}%
\pgfpathlineto{\pgfqpoint{2.829000in}{2.328706in}}%
\pgfusepath{stroke}%
\end{pgfscope}%
\begin{pgfscope}%
\pgfpathrectangle{\pgfqpoint{1.000000in}{0.720000in}}{\pgfqpoint{6.200000in}{4.620000in}}%
\pgfusepath{clip}%
\pgfsetbuttcap%
\pgfsetroundjoin%
\pgfsetlinewidth{0.501875pt}%
\definecolor{currentstroke}{rgb}{1.000000,0.000000,0.000000}%
\pgfsetstrokecolor{currentstroke}%
\pgfsetdash{}{0pt}%
\pgfpathmoveto{\pgfqpoint{2.860000in}{1.097665in}}%
\pgfpathlineto{\pgfqpoint{2.860000in}{2.372143in}}%
\pgfusepath{stroke}%
\end{pgfscope}%
\begin{pgfscope}%
\pgfpathrectangle{\pgfqpoint{1.000000in}{0.720000in}}{\pgfqpoint{6.200000in}{4.620000in}}%
\pgfusepath{clip}%
\pgfsetbuttcap%
\pgfsetroundjoin%
\pgfsetlinewidth{0.501875pt}%
\definecolor{currentstroke}{rgb}{1.000000,0.000000,0.000000}%
\pgfsetstrokecolor{currentstroke}%
\pgfsetdash{}{0pt}%
\pgfpathmoveto{\pgfqpoint{3.697000in}{1.097665in}}%
\pgfpathlineto{\pgfqpoint{3.697000in}{1.912800in}}%
\pgfusepath{stroke}%
\end{pgfscope}%
\begin{pgfscope}%
\pgfpathrectangle{\pgfqpoint{1.000000in}{0.720000in}}{\pgfqpoint{6.200000in}{4.620000in}}%
\pgfusepath{clip}%
\pgfsetbuttcap%
\pgfsetroundjoin%
\pgfsetlinewidth{0.501875pt}%
\definecolor{currentstroke}{rgb}{1.000000,0.000000,0.000000}%
\pgfsetstrokecolor{currentstroke}%
\pgfsetdash{}{0pt}%
\pgfpathmoveto{\pgfqpoint{3.728000in}{1.097665in}}%
\pgfpathlineto{\pgfqpoint{3.728000in}{3.184841in}}%
\pgfusepath{stroke}%
\end{pgfscope}%
\begin{pgfscope}%
\pgfsetrectcap%
\pgfsetmiterjoin%
\pgfsetlinewidth{0.803000pt}%
\definecolor{currentstroke}{rgb}{0.000000,0.000000,0.000000}%
\pgfsetstrokecolor{currentstroke}%
\pgfsetdash{}{0pt}%
\pgfpathmoveto{\pgfqpoint{1.000000in}{0.720000in}}%
\pgfpathlineto{\pgfqpoint{1.000000in}{5.340000in}}%
\pgfusepath{stroke}%
\end{pgfscope}%
\begin{pgfscope}%
\pgfsetrectcap%
\pgfsetmiterjoin%
\pgfsetlinewidth{0.803000pt}%
\definecolor{currentstroke}{rgb}{0.000000,0.000000,0.000000}%
\pgfsetstrokecolor{currentstroke}%
\pgfsetdash{}{0pt}%
\pgfpathmoveto{\pgfqpoint{7.200000in}{0.720000in}}%
\pgfpathlineto{\pgfqpoint{7.200000in}{5.340000in}}%
\pgfusepath{stroke}%
\end{pgfscope}%
\begin{pgfscope}%
\pgfsetrectcap%
\pgfsetmiterjoin%
\pgfsetlinewidth{0.803000pt}%
\definecolor{currentstroke}{rgb}{0.000000,0.000000,0.000000}%
\pgfsetstrokecolor{currentstroke}%
\pgfsetdash{}{0pt}%
\pgfpathmoveto{\pgfqpoint{1.000000in}{0.720000in}}%
\pgfpathlineto{\pgfqpoint{7.200000in}{0.720000in}}%
\pgfusepath{stroke}%
\end{pgfscope}%
\begin{pgfscope}%
\pgfsetrectcap%
\pgfsetmiterjoin%
\pgfsetlinewidth{0.803000pt}%
\definecolor{currentstroke}{rgb}{0.000000,0.000000,0.000000}%
\pgfsetstrokecolor{currentstroke}%
\pgfsetdash{}{0pt}%
\pgfpathmoveto{\pgfqpoint{1.000000in}{5.340000in}}%
\pgfpathlineto{\pgfqpoint{7.200000in}{5.340000in}}%
\pgfusepath{stroke}%
\end{pgfscope}%
\begin{pgfscope}%
\pgfsetbuttcap%
\pgfsetmiterjoin%
\definecolor{currentfill}{rgb}{1.000000,1.000000,1.000000}%
\pgfsetfillcolor{currentfill}%
\pgfsetfillopacity{0.800000}%
\pgfsetlinewidth{1.003750pt}%
\definecolor{currentstroke}{rgb}{0.800000,0.800000,0.800000}%
\pgfsetstrokecolor{currentstroke}%
\pgfsetstrokeopacity{0.800000}%
\pgfsetdash{}{0pt}%
\pgfpathmoveto{\pgfqpoint{4.976872in}{4.327865in}}%
\pgfpathlineto{\pgfqpoint{7.005556in}{4.327865in}}%
\pgfpathquadraticcurveto{\pgfqpoint{7.061111in}{4.327865in}}{\pgfqpoint{7.061111in}{4.383420in}}%
\pgfpathlineto{\pgfqpoint{7.061111in}{5.145556in}}%
\pgfpathquadraticcurveto{\pgfqpoint{7.061111in}{5.201111in}}{\pgfqpoint{7.005556in}{5.201111in}}%
\pgfpathlineto{\pgfqpoint{4.976872in}{5.201111in}}%
\pgfpathquadraticcurveto{\pgfqpoint{4.921317in}{5.201111in}}{\pgfqpoint{4.921317in}{5.145556in}}%
\pgfpathlineto{\pgfqpoint{4.921317in}{4.383420in}}%
\pgfpathquadraticcurveto{\pgfqpoint{4.921317in}{4.327865in}}{\pgfqpoint{4.976872in}{4.327865in}}%
\pgfpathlineto{\pgfqpoint{4.976872in}{4.327865in}}%
\pgfpathclose%
\pgfusepath{stroke,fill}%
\end{pgfscope}%
\begin{pgfscope}%
\pgfsetrectcap%
\pgfsetroundjoin%
\pgfsetlinewidth{2.007500pt}%
\definecolor{currentstroke}{rgb}{0.121569,0.466667,0.705882}%
\pgfsetstrokecolor{currentstroke}%
\pgfsetdash{}{0pt}%
\pgfpathmoveto{\pgfqpoint{5.032428in}{4.987184in}}%
\pgfpathlineto{\pgfqpoint{5.310206in}{4.987184in}}%
\pgfpathlineto{\pgfqpoint{5.587983in}{4.987184in}}%
\pgfusepath{stroke}%
\end{pgfscope}%
\begin{pgfscope}%
\definecolor{textcolor}{rgb}{0.000000,0.000000,0.000000}%
\pgfsetstrokecolor{textcolor}%
\pgfsetfillcolor{textcolor}%
\pgftext[x=5.810206in,y=4.889962in,left,base]{\color{textcolor}\sffamily\fontsize{20.000000}{24.000000}\selectfont Waveform}%
\end{pgfscope}%
\begin{pgfscope}%
\pgfsetbuttcap%
\pgfsetroundjoin%
\pgfsetlinewidth{0.501875pt}%
\definecolor{currentstroke}{rgb}{1.000000,0.000000,0.000000}%
\pgfsetstrokecolor{currentstroke}%
\pgfsetdash{}{0pt}%
\pgfpathmoveto{\pgfqpoint{5.032428in}{4.592227in}}%
\pgfpathlineto{\pgfqpoint{5.587983in}{4.592227in}}%
\pgfusepath{stroke}%
\end{pgfscope}%
\begin{pgfscope}%
\definecolor{textcolor}{rgb}{0.000000,0.000000,0.000000}%
\pgfsetstrokecolor{textcolor}%
\pgfsetfillcolor{textcolor}%
\pgftext[x=5.810206in,y=4.495005in,left,base]{\color{textcolor}\sffamily\fontsize{20.000000}{24.000000}\selectfont Charge}%
\end{pgfscope}%
\end{pgfpicture}%
\makeatother%
\endgroup%
}
    \caption{\label{fig:fitting}An example giving \\ $\hat{t}_\mathrm{KL} - t_0=\SI{2.73}{ns}$, $\mathrm{RSS}=\SI{6.49}{mV^2}$,$D_\mathrm{w}=\SI{0.91}{ns}$.}
  \end{subfigure}
  \caption{\label{fig:dcf}Demonstration of direct charge fitting with $\num[retain-unity-mantissa=false]{1e4}$ waveforms in~\subref{fig:fitting-npe} and one waveform in~\subref{fig:fitting} sampled from the same setup as figure~\ref{fig:method}.  Direct charge fitting shows a better performance than LucyDDM in figure~\ref{fig:lucy} and a comparable $D_\mathrm{w}$ to CNN in figure~\ref{fig:cnn}.}
\end{figure}

The sparsity of $q'_i$ is evident in figure~\ref{fig:fitting}.  However, the majority of the $\hat{q}_i$ are less than 1.  This feature motivates us to incorporate prior knowledge of $q'_i$ towards a more dedicated model than directly fitting charges.


\subsubsection{Hamilton Monte Carlo}
\label{subsec:mcmc}
Chaining the $q'_i$ distribution~(section~\ref{subsec:spe}), the charge fitting eq.~\eqref{eq:gd-q} and the light curve eq.~\eqref{eq:time-pro}, we arrive at a hierarchical Bayesian model,
\begin{equation}
  \begin{aligned}
    t_{0} &\sim \mathcal{U}(0, \overline{t_0}) \\
    \mu_j &= \mu \int_{t'_j-\frac{\Delta t'}{2}}^{t'_j+\frac{\Delta t'}{2}} \phi(t' - t_0)\mathrm{d}t' \approx \mu\phi(t'_j - t_0)\Delta{t'} \\
    z_j &\sim \mathcal{B}(\mu_j) \\
    q'_{j,0}&=0\\
    q'_{j,1}& \sim \Gamma(k=1/0.4^2, \theta=0.4^2)\\
    q'_j &= q'_{j,z_j}\\
    w'(t) & = \sum_{j=1}^{N_\mathrm{s}}q'_jV_\mathrm{PE}(t-t'_j)\\
    w(t) &\sim \mathcal{N}(w'(t), \Var[\epsilon])
  \end{aligned}
  \label{eq:mixnormal}
\end{equation}
where $\mathcal{U}$, $\mathcal{B}$ and $\Gamma$ stand for uniform, Bernoulli and gamma distributions, $\overline{t_0}$ is an upper bound of $t_0$, and $q'_j$ is a mixture of 0 (no PE) and normally distributed $q'_{j,1}$ (1 PE). When the expectation $\mu_j$ of a PE hitting $(t'_{j} - \frac{\Delta t'}{2}, t'_{j} + \frac{\Delta t'}{2})$ is small enough, that 0-1 approximation is valid.  The inferred waveform $w'(t)$ differs from observable $w(t)$ by a white noise $\epsilon(t) \sim \mathcal{N}(0, \Var[\epsilon])$ motivated by eq.~\eqref{eq:1}.  When an indicator $z_j=0$, it turns off $q'_j$, reducing the number of parameters by one.  That is how eq.~\eqref{eq:mixnormal} achieves sparsity.

We generate posterior samples of $t_0$ and $\bm{q'}$ by Hamiltonian Monte Carlo~(HMC)~\cite{neal_mcmc_2012}, a variant of Markov chain Monte Carlo suitable for high-dimensional problems. Construct $\hat{t}$ and $\hat{q}_j$ as the mean estimators of posterior samples $t_0$ and $q'_j$ at $z_j=1$.  Unlike the $\hat{t}_\mathrm{KL}$ discussed in section~\ref{sec:pseudo}, $\hat{t}_0$ is a direct Bayesian estimator from eq.~\eqref{eq:mixnormal}.  We construct $\hat{\phi}(t)$ by eq.~\eqref{eq:gd-phi} and $\hat{w}(t)$ by $\hat{\phi} \otimes V_\mathrm{PE}$. RSS and $D_\mathrm{w}$ are then calculated according to eqs.~\eqref{eq:rss} and \eqref{eq:numerical}.

\begin{figure}[H]
  \begin{subfigure}{.5\textwidth}
    \centering
    \resizebox{\textwidth}{!}{%% Creator: Matplotlib, PGF backend
%%
%% To include the figure in your LaTeX document, write
%%   \input{<filename>.pgf}
%%
%% Make sure the required packages are loaded in your preamble
%%   \usepackage{pgf}
%%
%% and, on pdftex
%%   \usepackage[utf8]{inputenc}\DeclareUnicodeCharacter{2212}{-}
%%
%% or, on luatex and xetex
%%   \usepackage{unicode-math}
%%
%% Figures using additional raster images can only be included by \input if
%% they are in the same directory as the main LaTeX file. For loading figures
%% from other directories you can use the `import` package
%%   \usepackage{import}
%%
%% and then include the figures with
%%   \import{<path to file>}{<filename>.pgf}
%%
%% Matplotlib used the following preamble
%%   \usepackage[detect-all,locale=DE]{siunitx}
%%
\begingroup%
\makeatletter%
\begin{pgfpicture}%
\pgfpathrectangle{\pgfpointorigin}{\pgfqpoint{8.000000in}{6.000000in}}%
\pgfusepath{use as bounding box, clip}%
\begin{pgfscope}%
\pgfsetbuttcap%
\pgfsetmiterjoin%
\definecolor{currentfill}{rgb}{1.000000,1.000000,1.000000}%
\pgfsetfillcolor{currentfill}%
\pgfsetlinewidth{0.000000pt}%
\definecolor{currentstroke}{rgb}{1.000000,1.000000,1.000000}%
\pgfsetstrokecolor{currentstroke}%
\pgfsetdash{}{0pt}%
\pgfpathmoveto{\pgfqpoint{0.000000in}{0.000000in}}%
\pgfpathlineto{\pgfqpoint{8.000000in}{0.000000in}}%
\pgfpathlineto{\pgfqpoint{8.000000in}{6.000000in}}%
\pgfpathlineto{\pgfqpoint{0.000000in}{6.000000in}}%
\pgfpathclose%
\pgfusepath{fill}%
\end{pgfscope}%
\begin{pgfscope}%
\pgfsetbuttcap%
\pgfsetmiterjoin%
\definecolor{currentfill}{rgb}{1.000000,1.000000,1.000000}%
\pgfsetfillcolor{currentfill}%
\pgfsetlinewidth{0.000000pt}%
\definecolor{currentstroke}{rgb}{0.000000,0.000000,0.000000}%
\pgfsetstrokecolor{currentstroke}%
\pgfsetstrokeopacity{0.000000}%
\pgfsetdash{}{0pt}%
\pgfpathmoveto{\pgfqpoint{1.000000in}{0.720000in}}%
\pgfpathlineto{\pgfqpoint{5.800000in}{0.720000in}}%
\pgfpathlineto{\pgfqpoint{5.800000in}{5.340000in}}%
\pgfpathlineto{\pgfqpoint{1.000000in}{5.340000in}}%
\pgfpathclose%
\pgfusepath{fill}%
\end{pgfscope}%
\begin{pgfscope}%
\pgfsetbuttcap%
\pgfsetroundjoin%
\definecolor{currentfill}{rgb}{0.000000,0.000000,0.000000}%
\pgfsetfillcolor{currentfill}%
\pgfsetlinewidth{0.803000pt}%
\definecolor{currentstroke}{rgb}{0.000000,0.000000,0.000000}%
\pgfsetstrokecolor{currentstroke}%
\pgfsetdash{}{0pt}%
\pgfsys@defobject{currentmarker}{\pgfqpoint{0.000000in}{-0.048611in}}{\pgfqpoint{0.000000in}{0.000000in}}{%
\pgfpathmoveto{\pgfqpoint{0.000000in}{0.000000in}}%
\pgfpathlineto{\pgfqpoint{0.000000in}{-0.048611in}}%
\pgfusepath{stroke,fill}%
}%
\begin{pgfscope}%
\pgfsys@transformshift{1.300000in}{0.720000in}%
\pgfsys@useobject{currentmarker}{}%
\end{pgfscope}%
\end{pgfscope}%
\begin{pgfscope}%
\definecolor{textcolor}{rgb}{0.000000,0.000000,0.000000}%
\pgfsetstrokecolor{textcolor}%
\pgfsetfillcolor{textcolor}%
\pgftext[x=1.300000in,y=0.622778in,,top]{\color{textcolor}\sffamily\fontsize{20.000000}{24.000000}\selectfont 1}%
\end{pgfscope}%
\begin{pgfscope}%
\pgfsetbuttcap%
\pgfsetroundjoin%
\definecolor{currentfill}{rgb}{0.000000,0.000000,0.000000}%
\pgfsetfillcolor{currentfill}%
\pgfsetlinewidth{0.803000pt}%
\definecolor{currentstroke}{rgb}{0.000000,0.000000,0.000000}%
\pgfsetstrokecolor{currentstroke}%
\pgfsetdash{}{0pt}%
\pgfsys@defobject{currentmarker}{\pgfqpoint{0.000000in}{-0.048611in}}{\pgfqpoint{0.000000in}{0.000000in}}{%
\pgfpathmoveto{\pgfqpoint{0.000000in}{0.000000in}}%
\pgfpathlineto{\pgfqpoint{0.000000in}{-0.048611in}}%
\pgfusepath{stroke,fill}%
}%
\begin{pgfscope}%
\pgfsys@transformshift{1.900000in}{0.720000in}%
\pgfsys@useobject{currentmarker}{}%
\end{pgfscope}%
\end{pgfscope}%
\begin{pgfscope}%
\definecolor{textcolor}{rgb}{0.000000,0.000000,0.000000}%
\pgfsetstrokecolor{textcolor}%
\pgfsetfillcolor{textcolor}%
\pgftext[x=1.900000in,y=0.622778in,,top]{\color{textcolor}\sffamily\fontsize{20.000000}{24.000000}\selectfont 3}%
\end{pgfscope}%
\begin{pgfscope}%
\pgfsetbuttcap%
\pgfsetroundjoin%
\definecolor{currentfill}{rgb}{0.000000,0.000000,0.000000}%
\pgfsetfillcolor{currentfill}%
\pgfsetlinewidth{0.803000pt}%
\definecolor{currentstroke}{rgb}{0.000000,0.000000,0.000000}%
\pgfsetstrokecolor{currentstroke}%
\pgfsetdash{}{0pt}%
\pgfsys@defobject{currentmarker}{\pgfqpoint{0.000000in}{-0.048611in}}{\pgfqpoint{0.000000in}{0.000000in}}{%
\pgfpathmoveto{\pgfqpoint{0.000000in}{0.000000in}}%
\pgfpathlineto{\pgfqpoint{0.000000in}{-0.048611in}}%
\pgfusepath{stroke,fill}%
}%
\begin{pgfscope}%
\pgfsys@transformshift{2.500000in}{0.720000in}%
\pgfsys@useobject{currentmarker}{}%
\end{pgfscope}%
\end{pgfscope}%
\begin{pgfscope}%
\definecolor{textcolor}{rgb}{0.000000,0.000000,0.000000}%
\pgfsetstrokecolor{textcolor}%
\pgfsetfillcolor{textcolor}%
\pgftext[x=2.500000in,y=0.622778in,,top]{\color{textcolor}\sffamily\fontsize{20.000000}{24.000000}\selectfont 5}%
\end{pgfscope}%
\begin{pgfscope}%
\pgfsetbuttcap%
\pgfsetroundjoin%
\definecolor{currentfill}{rgb}{0.000000,0.000000,0.000000}%
\pgfsetfillcolor{currentfill}%
\pgfsetlinewidth{0.803000pt}%
\definecolor{currentstroke}{rgb}{0.000000,0.000000,0.000000}%
\pgfsetstrokecolor{currentstroke}%
\pgfsetdash{}{0pt}%
\pgfsys@defobject{currentmarker}{\pgfqpoint{0.000000in}{-0.048611in}}{\pgfqpoint{0.000000in}{0.000000in}}{%
\pgfpathmoveto{\pgfqpoint{0.000000in}{0.000000in}}%
\pgfpathlineto{\pgfqpoint{0.000000in}{-0.048611in}}%
\pgfusepath{stroke,fill}%
}%
\begin{pgfscope}%
\pgfsys@transformshift{3.100000in}{0.720000in}%
\pgfsys@useobject{currentmarker}{}%
\end{pgfscope}%
\end{pgfscope}%
\begin{pgfscope}%
\definecolor{textcolor}{rgb}{0.000000,0.000000,0.000000}%
\pgfsetstrokecolor{textcolor}%
\pgfsetfillcolor{textcolor}%
\pgftext[x=3.100000in,y=0.622778in,,top]{\color{textcolor}\sffamily\fontsize{20.000000}{24.000000}\selectfont 7}%
\end{pgfscope}%
\begin{pgfscope}%
\pgfsetbuttcap%
\pgfsetroundjoin%
\definecolor{currentfill}{rgb}{0.000000,0.000000,0.000000}%
\pgfsetfillcolor{currentfill}%
\pgfsetlinewidth{0.803000pt}%
\definecolor{currentstroke}{rgb}{0.000000,0.000000,0.000000}%
\pgfsetstrokecolor{currentstroke}%
\pgfsetdash{}{0pt}%
\pgfsys@defobject{currentmarker}{\pgfqpoint{0.000000in}{-0.048611in}}{\pgfqpoint{0.000000in}{0.000000in}}{%
\pgfpathmoveto{\pgfqpoint{0.000000in}{0.000000in}}%
\pgfpathlineto{\pgfqpoint{0.000000in}{-0.048611in}}%
\pgfusepath{stroke,fill}%
}%
\begin{pgfscope}%
\pgfsys@transformshift{3.700000in}{0.720000in}%
\pgfsys@useobject{currentmarker}{}%
\end{pgfscope}%
\end{pgfscope}%
\begin{pgfscope}%
\definecolor{textcolor}{rgb}{0.000000,0.000000,0.000000}%
\pgfsetstrokecolor{textcolor}%
\pgfsetfillcolor{textcolor}%
\pgftext[x=3.700000in,y=0.622778in,,top]{\color{textcolor}\sffamily\fontsize{20.000000}{24.000000}\selectfont 9}%
\end{pgfscope}%
\begin{pgfscope}%
\pgfsetbuttcap%
\pgfsetroundjoin%
\definecolor{currentfill}{rgb}{0.000000,0.000000,0.000000}%
\pgfsetfillcolor{currentfill}%
\pgfsetlinewidth{0.803000pt}%
\definecolor{currentstroke}{rgb}{0.000000,0.000000,0.000000}%
\pgfsetstrokecolor{currentstroke}%
\pgfsetdash{}{0pt}%
\pgfsys@defobject{currentmarker}{\pgfqpoint{0.000000in}{-0.048611in}}{\pgfqpoint{0.000000in}{0.000000in}}{%
\pgfpathmoveto{\pgfqpoint{0.000000in}{0.000000in}}%
\pgfpathlineto{\pgfqpoint{0.000000in}{-0.048611in}}%
\pgfusepath{stroke,fill}%
}%
\begin{pgfscope}%
\pgfsys@transformshift{4.300000in}{0.720000in}%
\pgfsys@useobject{currentmarker}{}%
\end{pgfscope}%
\end{pgfscope}%
\begin{pgfscope}%
\definecolor{textcolor}{rgb}{0.000000,0.000000,0.000000}%
\pgfsetstrokecolor{textcolor}%
\pgfsetfillcolor{textcolor}%
\pgftext[x=4.300000in,y=0.622778in,,top]{\color{textcolor}\sffamily\fontsize{20.000000}{24.000000}\selectfont 11}%
\end{pgfscope}%
\begin{pgfscope}%
\pgfsetbuttcap%
\pgfsetroundjoin%
\definecolor{currentfill}{rgb}{0.000000,0.000000,0.000000}%
\pgfsetfillcolor{currentfill}%
\pgfsetlinewidth{0.803000pt}%
\definecolor{currentstroke}{rgb}{0.000000,0.000000,0.000000}%
\pgfsetstrokecolor{currentstroke}%
\pgfsetdash{}{0pt}%
\pgfsys@defobject{currentmarker}{\pgfqpoint{0.000000in}{-0.048611in}}{\pgfqpoint{0.000000in}{0.000000in}}{%
\pgfpathmoveto{\pgfqpoint{0.000000in}{0.000000in}}%
\pgfpathlineto{\pgfqpoint{0.000000in}{-0.048611in}}%
\pgfusepath{stroke,fill}%
}%
\begin{pgfscope}%
\pgfsys@transformshift{4.900000in}{0.720000in}%
\pgfsys@useobject{currentmarker}{}%
\end{pgfscope}%
\end{pgfscope}%
\begin{pgfscope}%
\definecolor{textcolor}{rgb}{0.000000,0.000000,0.000000}%
\pgfsetstrokecolor{textcolor}%
\pgfsetfillcolor{textcolor}%
\pgftext[x=4.900000in,y=0.622778in,,top]{\color{textcolor}\sffamily\fontsize{20.000000}{24.000000}\selectfont 13}%
\end{pgfscope}%
\begin{pgfscope}%
\pgfsetbuttcap%
\pgfsetroundjoin%
\definecolor{currentfill}{rgb}{0.000000,0.000000,0.000000}%
\pgfsetfillcolor{currentfill}%
\pgfsetlinewidth{0.803000pt}%
\definecolor{currentstroke}{rgb}{0.000000,0.000000,0.000000}%
\pgfsetstrokecolor{currentstroke}%
\pgfsetdash{}{0pt}%
\pgfsys@defobject{currentmarker}{\pgfqpoint{0.000000in}{-0.048611in}}{\pgfqpoint{0.000000in}{0.000000in}}{%
\pgfpathmoveto{\pgfqpoint{0.000000in}{0.000000in}}%
\pgfpathlineto{\pgfqpoint{0.000000in}{-0.048611in}}%
\pgfusepath{stroke,fill}%
}%
\begin{pgfscope}%
\pgfsys@transformshift{5.500000in}{0.720000in}%
\pgfsys@useobject{currentmarker}{}%
\end{pgfscope}%
\end{pgfscope}%
\begin{pgfscope}%
\definecolor{textcolor}{rgb}{0.000000,0.000000,0.000000}%
\pgfsetstrokecolor{textcolor}%
\pgfsetfillcolor{textcolor}%
\pgftext[x=5.500000in,y=0.622778in,,top]{\color{textcolor}\sffamily\fontsize{20.000000}{24.000000}\selectfont 15}%
\end{pgfscope}%
\begin{pgfscope}%
\definecolor{textcolor}{rgb}{0.000000,0.000000,0.000000}%
\pgfsetstrokecolor{textcolor}%
\pgfsetfillcolor{textcolor}%
\pgftext[x=3.400000in,y=0.311155in,,top]{\color{textcolor}\sffamily\fontsize{20.000000}{24.000000}\selectfont \(\displaystyle N_{\mathrm{PE}}\)}%
\end{pgfscope}%
\begin{pgfscope}%
\pgfsetbuttcap%
\pgfsetroundjoin%
\definecolor{currentfill}{rgb}{0.000000,0.000000,0.000000}%
\pgfsetfillcolor{currentfill}%
\pgfsetlinewidth{0.803000pt}%
\definecolor{currentstroke}{rgb}{0.000000,0.000000,0.000000}%
\pgfsetstrokecolor{currentstroke}%
\pgfsetdash{}{0pt}%
\pgfsys@defobject{currentmarker}{\pgfqpoint{-0.048611in}{0.000000in}}{\pgfqpoint{-0.000000in}{0.000000in}}{%
\pgfpathmoveto{\pgfqpoint{-0.000000in}{0.000000in}}%
\pgfpathlineto{\pgfqpoint{-0.048611in}{0.000000in}}%
\pgfusepath{stroke,fill}%
}%
\begin{pgfscope}%
\pgfsys@transformshift{1.000000in}{0.720000in}%
\pgfsys@useobject{currentmarker}{}%
\end{pgfscope}%
\end{pgfscope}%
\begin{pgfscope}%
\definecolor{textcolor}{rgb}{0.000000,0.000000,0.000000}%
\pgfsetstrokecolor{textcolor}%
\pgfsetfillcolor{textcolor}%
\pgftext[x=0.560215in, y=0.619981in, left, base]{\color{textcolor}\sffamily\fontsize{20.000000}{24.000000}\selectfont \(\displaystyle {0.0}\)}%
\end{pgfscope}%
\begin{pgfscope}%
\pgfsetbuttcap%
\pgfsetroundjoin%
\definecolor{currentfill}{rgb}{0.000000,0.000000,0.000000}%
\pgfsetfillcolor{currentfill}%
\pgfsetlinewidth{0.803000pt}%
\definecolor{currentstroke}{rgb}{0.000000,0.000000,0.000000}%
\pgfsetstrokecolor{currentstroke}%
\pgfsetdash{}{0pt}%
\pgfsys@defobject{currentmarker}{\pgfqpoint{-0.048611in}{0.000000in}}{\pgfqpoint{-0.000000in}{0.000000in}}{%
\pgfpathmoveto{\pgfqpoint{-0.000000in}{0.000000in}}%
\pgfpathlineto{\pgfqpoint{-0.048611in}{0.000000in}}%
\pgfusepath{stroke,fill}%
}%
\begin{pgfscope}%
\pgfsys@transformshift{1.000000in}{1.297742in}%
\pgfsys@useobject{currentmarker}{}%
\end{pgfscope}%
\end{pgfscope}%
\begin{pgfscope}%
\definecolor{textcolor}{rgb}{0.000000,0.000000,0.000000}%
\pgfsetstrokecolor{textcolor}%
\pgfsetfillcolor{textcolor}%
\pgftext[x=0.560215in, y=1.197723in, left, base]{\color{textcolor}\sffamily\fontsize{20.000000}{24.000000}\selectfont \(\displaystyle {0.2}\)}%
\end{pgfscope}%
\begin{pgfscope}%
\pgfsetbuttcap%
\pgfsetroundjoin%
\definecolor{currentfill}{rgb}{0.000000,0.000000,0.000000}%
\pgfsetfillcolor{currentfill}%
\pgfsetlinewidth{0.803000pt}%
\definecolor{currentstroke}{rgb}{0.000000,0.000000,0.000000}%
\pgfsetstrokecolor{currentstroke}%
\pgfsetdash{}{0pt}%
\pgfsys@defobject{currentmarker}{\pgfqpoint{-0.048611in}{0.000000in}}{\pgfqpoint{-0.000000in}{0.000000in}}{%
\pgfpathmoveto{\pgfqpoint{-0.000000in}{0.000000in}}%
\pgfpathlineto{\pgfqpoint{-0.048611in}{0.000000in}}%
\pgfusepath{stroke,fill}%
}%
\begin{pgfscope}%
\pgfsys@transformshift{1.000000in}{1.875484in}%
\pgfsys@useobject{currentmarker}{}%
\end{pgfscope}%
\end{pgfscope}%
\begin{pgfscope}%
\definecolor{textcolor}{rgb}{0.000000,0.000000,0.000000}%
\pgfsetstrokecolor{textcolor}%
\pgfsetfillcolor{textcolor}%
\pgftext[x=0.560215in, y=1.775465in, left, base]{\color{textcolor}\sffamily\fontsize{20.000000}{24.000000}\selectfont \(\displaystyle {0.4}\)}%
\end{pgfscope}%
\begin{pgfscope}%
\pgfsetbuttcap%
\pgfsetroundjoin%
\definecolor{currentfill}{rgb}{0.000000,0.000000,0.000000}%
\pgfsetfillcolor{currentfill}%
\pgfsetlinewidth{0.803000pt}%
\definecolor{currentstroke}{rgb}{0.000000,0.000000,0.000000}%
\pgfsetstrokecolor{currentstroke}%
\pgfsetdash{}{0pt}%
\pgfsys@defobject{currentmarker}{\pgfqpoint{-0.048611in}{0.000000in}}{\pgfqpoint{-0.000000in}{0.000000in}}{%
\pgfpathmoveto{\pgfqpoint{-0.000000in}{0.000000in}}%
\pgfpathlineto{\pgfqpoint{-0.048611in}{0.000000in}}%
\pgfusepath{stroke,fill}%
}%
\begin{pgfscope}%
\pgfsys@transformshift{1.000000in}{2.453226in}%
\pgfsys@useobject{currentmarker}{}%
\end{pgfscope}%
\end{pgfscope}%
\begin{pgfscope}%
\definecolor{textcolor}{rgb}{0.000000,0.000000,0.000000}%
\pgfsetstrokecolor{textcolor}%
\pgfsetfillcolor{textcolor}%
\pgftext[x=0.560215in, y=2.353207in, left, base]{\color{textcolor}\sffamily\fontsize{20.000000}{24.000000}\selectfont \(\displaystyle {0.6}\)}%
\end{pgfscope}%
\begin{pgfscope}%
\pgfsetbuttcap%
\pgfsetroundjoin%
\definecolor{currentfill}{rgb}{0.000000,0.000000,0.000000}%
\pgfsetfillcolor{currentfill}%
\pgfsetlinewidth{0.803000pt}%
\definecolor{currentstroke}{rgb}{0.000000,0.000000,0.000000}%
\pgfsetstrokecolor{currentstroke}%
\pgfsetdash{}{0pt}%
\pgfsys@defobject{currentmarker}{\pgfqpoint{-0.048611in}{0.000000in}}{\pgfqpoint{-0.000000in}{0.000000in}}{%
\pgfpathmoveto{\pgfqpoint{-0.000000in}{0.000000in}}%
\pgfpathlineto{\pgfqpoint{-0.048611in}{0.000000in}}%
\pgfusepath{stroke,fill}%
}%
\begin{pgfscope}%
\pgfsys@transformshift{1.000000in}{3.030968in}%
\pgfsys@useobject{currentmarker}{}%
\end{pgfscope}%
\end{pgfscope}%
\begin{pgfscope}%
\definecolor{textcolor}{rgb}{0.000000,0.000000,0.000000}%
\pgfsetstrokecolor{textcolor}%
\pgfsetfillcolor{textcolor}%
\pgftext[x=0.560215in, y=2.930949in, left, base]{\color{textcolor}\sffamily\fontsize{20.000000}{24.000000}\selectfont \(\displaystyle {0.8}\)}%
\end{pgfscope}%
\begin{pgfscope}%
\pgfsetbuttcap%
\pgfsetroundjoin%
\definecolor{currentfill}{rgb}{0.000000,0.000000,0.000000}%
\pgfsetfillcolor{currentfill}%
\pgfsetlinewidth{0.803000pt}%
\definecolor{currentstroke}{rgb}{0.000000,0.000000,0.000000}%
\pgfsetstrokecolor{currentstroke}%
\pgfsetdash{}{0pt}%
\pgfsys@defobject{currentmarker}{\pgfqpoint{-0.048611in}{0.000000in}}{\pgfqpoint{-0.000000in}{0.000000in}}{%
\pgfpathmoveto{\pgfqpoint{-0.000000in}{0.000000in}}%
\pgfpathlineto{\pgfqpoint{-0.048611in}{0.000000in}}%
\pgfusepath{stroke,fill}%
}%
\begin{pgfscope}%
\pgfsys@transformshift{1.000000in}{3.608710in}%
\pgfsys@useobject{currentmarker}{}%
\end{pgfscope}%
\end{pgfscope}%
\begin{pgfscope}%
\definecolor{textcolor}{rgb}{0.000000,0.000000,0.000000}%
\pgfsetstrokecolor{textcolor}%
\pgfsetfillcolor{textcolor}%
\pgftext[x=0.560215in, y=3.508691in, left, base]{\color{textcolor}\sffamily\fontsize{20.000000}{24.000000}\selectfont \(\displaystyle {1.0}\)}%
\end{pgfscope}%
\begin{pgfscope}%
\pgfsetbuttcap%
\pgfsetroundjoin%
\definecolor{currentfill}{rgb}{0.000000,0.000000,0.000000}%
\pgfsetfillcolor{currentfill}%
\pgfsetlinewidth{0.803000pt}%
\definecolor{currentstroke}{rgb}{0.000000,0.000000,0.000000}%
\pgfsetstrokecolor{currentstroke}%
\pgfsetdash{}{0pt}%
\pgfsys@defobject{currentmarker}{\pgfqpoint{-0.048611in}{0.000000in}}{\pgfqpoint{-0.000000in}{0.000000in}}{%
\pgfpathmoveto{\pgfqpoint{-0.000000in}{0.000000in}}%
\pgfpathlineto{\pgfqpoint{-0.048611in}{0.000000in}}%
\pgfusepath{stroke,fill}%
}%
\begin{pgfscope}%
\pgfsys@transformshift{1.000000in}{4.186452in}%
\pgfsys@useobject{currentmarker}{}%
\end{pgfscope}%
\end{pgfscope}%
\begin{pgfscope}%
\definecolor{textcolor}{rgb}{0.000000,0.000000,0.000000}%
\pgfsetstrokecolor{textcolor}%
\pgfsetfillcolor{textcolor}%
\pgftext[x=0.560215in, y=4.086432in, left, base]{\color{textcolor}\sffamily\fontsize{20.000000}{24.000000}\selectfont \(\displaystyle {1.2}\)}%
\end{pgfscope}%
\begin{pgfscope}%
\pgfsetbuttcap%
\pgfsetroundjoin%
\definecolor{currentfill}{rgb}{0.000000,0.000000,0.000000}%
\pgfsetfillcolor{currentfill}%
\pgfsetlinewidth{0.803000pt}%
\definecolor{currentstroke}{rgb}{0.000000,0.000000,0.000000}%
\pgfsetstrokecolor{currentstroke}%
\pgfsetdash{}{0pt}%
\pgfsys@defobject{currentmarker}{\pgfqpoint{-0.048611in}{0.000000in}}{\pgfqpoint{-0.000000in}{0.000000in}}{%
\pgfpathmoveto{\pgfqpoint{-0.000000in}{0.000000in}}%
\pgfpathlineto{\pgfqpoint{-0.048611in}{0.000000in}}%
\pgfusepath{stroke,fill}%
}%
\begin{pgfscope}%
\pgfsys@transformshift{1.000000in}{4.764194in}%
\pgfsys@useobject{currentmarker}{}%
\end{pgfscope}%
\end{pgfscope}%
\begin{pgfscope}%
\definecolor{textcolor}{rgb}{0.000000,0.000000,0.000000}%
\pgfsetstrokecolor{textcolor}%
\pgfsetfillcolor{textcolor}%
\pgftext[x=0.560215in, y=4.664174in, left, base]{\color{textcolor}\sffamily\fontsize{20.000000}{24.000000}\selectfont \(\displaystyle {1.4}\)}%
\end{pgfscope}%
\begin{pgfscope}%
\definecolor{textcolor}{rgb}{0.000000,0.000000,0.000000}%
\pgfsetstrokecolor{textcolor}%
\pgfsetfillcolor{textcolor}%
\pgftext[x=0.504660in,y=3.030000in,,bottom,rotate=90.000000]{\color{textcolor}\sffamily\fontsize{20.000000}{24.000000}\selectfont \(\displaystyle \mathrm{Wasserstein\ Distance}/\si{ns}\)}%
\end{pgfscope}%
\begin{pgfscope}%
\pgfpathrectangle{\pgfqpoint{1.000000in}{0.720000in}}{\pgfqpoint{4.800000in}{4.620000in}}%
\pgfusepath{clip}%
\pgfsetrectcap%
\pgfsetroundjoin%
\pgfsetlinewidth{1.003750pt}%
\definecolor{currentstroke}{rgb}{0.000000,0.000000,0.000000}%
\pgfsetstrokecolor{currentstroke}%
\pgfsetdash{}{0pt}%
\pgfpathmoveto{\pgfqpoint{1.300000in}{1.361115in}}%
\pgfpathlineto{\pgfqpoint{1.300000in}{0.720863in}}%
\pgfusepath{stroke}%
\end{pgfscope}%
\begin{pgfscope}%
\pgfpathrectangle{\pgfqpoint{1.000000in}{0.720000in}}{\pgfqpoint{4.800000in}{4.620000in}}%
\pgfusepath{clip}%
\pgfsetrectcap%
\pgfsetroundjoin%
\pgfsetlinewidth{1.003750pt}%
\definecolor{currentstroke}{rgb}{0.000000,0.000000,0.000000}%
\pgfsetstrokecolor{currentstroke}%
\pgfsetdash{}{0pt}%
\pgfpathmoveto{\pgfqpoint{1.300000in}{2.939637in}}%
\pgfpathlineto{\pgfqpoint{1.300000in}{5.120000in}}%
\pgfusepath{stroke}%
\end{pgfscope}%
\begin{pgfscope}%
\pgfpathrectangle{\pgfqpoint{1.000000in}{0.720000in}}{\pgfqpoint{4.800000in}{4.620000in}}%
\pgfusepath{clip}%
\pgfsetrectcap%
\pgfsetroundjoin%
\pgfsetlinewidth{1.003750pt}%
\definecolor{currentstroke}{rgb}{0.000000,0.000000,0.000000}%
\pgfsetstrokecolor{currentstroke}%
\pgfsetdash{}{0pt}%
\pgfpathmoveto{\pgfqpoint{1.262500in}{0.720863in}}%
\pgfpathlineto{\pgfqpoint{1.337500in}{0.720863in}}%
\pgfusepath{stroke}%
\end{pgfscope}%
\begin{pgfscope}%
\pgfpathrectangle{\pgfqpoint{1.000000in}{0.720000in}}{\pgfqpoint{4.800000in}{4.620000in}}%
\pgfusepath{clip}%
\pgfsetrectcap%
\pgfsetroundjoin%
\pgfsetlinewidth{1.003750pt}%
\definecolor{currentstroke}{rgb}{0.000000,0.000000,0.000000}%
\pgfsetstrokecolor{currentstroke}%
\pgfsetdash{}{0pt}%
\pgfpathmoveto{\pgfqpoint{1.262500in}{5.120000in}}%
\pgfpathlineto{\pgfqpoint{1.337500in}{5.120000in}}%
\pgfusepath{stroke}%
\end{pgfscope}%
\begin{pgfscope}%
\pgfpathrectangle{\pgfqpoint{1.000000in}{0.720000in}}{\pgfqpoint{4.800000in}{4.620000in}}%
\pgfusepath{clip}%
\pgfsetrectcap%
\pgfsetroundjoin%
\pgfsetlinewidth{1.003750pt}%
\definecolor{currentstroke}{rgb}{0.000000,0.000000,0.000000}%
\pgfsetstrokecolor{currentstroke}%
\pgfsetdash{}{0pt}%
\pgfpathmoveto{\pgfqpoint{1.600000in}{2.052272in}}%
\pgfpathlineto{\pgfqpoint{1.600000in}{0.784704in}}%
\pgfusepath{stroke}%
\end{pgfscope}%
\begin{pgfscope}%
\pgfpathrectangle{\pgfqpoint{1.000000in}{0.720000in}}{\pgfqpoint{4.800000in}{4.620000in}}%
\pgfusepath{clip}%
\pgfsetrectcap%
\pgfsetroundjoin%
\pgfsetlinewidth{1.003750pt}%
\definecolor{currentstroke}{rgb}{0.000000,0.000000,0.000000}%
\pgfsetstrokecolor{currentstroke}%
\pgfsetdash{}{0pt}%
\pgfpathmoveto{\pgfqpoint{1.600000in}{3.271276in}}%
\pgfpathlineto{\pgfqpoint{1.600000in}{5.088521in}}%
\pgfusepath{stroke}%
\end{pgfscope}%
\begin{pgfscope}%
\pgfpathrectangle{\pgfqpoint{1.000000in}{0.720000in}}{\pgfqpoint{4.800000in}{4.620000in}}%
\pgfusepath{clip}%
\pgfsetrectcap%
\pgfsetroundjoin%
\pgfsetlinewidth{1.003750pt}%
\definecolor{currentstroke}{rgb}{0.000000,0.000000,0.000000}%
\pgfsetstrokecolor{currentstroke}%
\pgfsetdash{}{0pt}%
\pgfpathmoveto{\pgfqpoint{1.562500in}{0.784704in}}%
\pgfpathlineto{\pgfqpoint{1.637500in}{0.784704in}}%
\pgfusepath{stroke}%
\end{pgfscope}%
\begin{pgfscope}%
\pgfpathrectangle{\pgfqpoint{1.000000in}{0.720000in}}{\pgfqpoint{4.800000in}{4.620000in}}%
\pgfusepath{clip}%
\pgfsetrectcap%
\pgfsetroundjoin%
\pgfsetlinewidth{1.003750pt}%
\definecolor{currentstroke}{rgb}{0.000000,0.000000,0.000000}%
\pgfsetstrokecolor{currentstroke}%
\pgfsetdash{}{0pt}%
\pgfpathmoveto{\pgfqpoint{1.562500in}{5.088521in}}%
\pgfpathlineto{\pgfqpoint{1.637500in}{5.088521in}}%
\pgfusepath{stroke}%
\end{pgfscope}%
\begin{pgfscope}%
\pgfpathrectangle{\pgfqpoint{1.000000in}{0.720000in}}{\pgfqpoint{4.800000in}{4.620000in}}%
\pgfusepath{clip}%
\pgfsetrectcap%
\pgfsetroundjoin%
\pgfsetlinewidth{1.003750pt}%
\definecolor{currentstroke}{rgb}{0.000000,0.000000,0.000000}%
\pgfsetstrokecolor{currentstroke}%
\pgfsetdash{}{0pt}%
\pgfpathmoveto{\pgfqpoint{1.900000in}{2.206069in}}%
\pgfpathlineto{\pgfqpoint{1.900000in}{1.049658in}}%
\pgfusepath{stroke}%
\end{pgfscope}%
\begin{pgfscope}%
\pgfpathrectangle{\pgfqpoint{1.000000in}{0.720000in}}{\pgfqpoint{4.800000in}{4.620000in}}%
\pgfusepath{clip}%
\pgfsetrectcap%
\pgfsetroundjoin%
\pgfsetlinewidth{1.003750pt}%
\definecolor{currentstroke}{rgb}{0.000000,0.000000,0.000000}%
\pgfsetstrokecolor{currentstroke}%
\pgfsetdash{}{0pt}%
\pgfpathmoveto{\pgfqpoint{1.900000in}{3.253534in}}%
\pgfpathlineto{\pgfqpoint{1.900000in}{4.816906in}}%
\pgfusepath{stroke}%
\end{pgfscope}%
\begin{pgfscope}%
\pgfpathrectangle{\pgfqpoint{1.000000in}{0.720000in}}{\pgfqpoint{4.800000in}{4.620000in}}%
\pgfusepath{clip}%
\pgfsetrectcap%
\pgfsetroundjoin%
\pgfsetlinewidth{1.003750pt}%
\definecolor{currentstroke}{rgb}{0.000000,0.000000,0.000000}%
\pgfsetstrokecolor{currentstroke}%
\pgfsetdash{}{0pt}%
\pgfpathmoveto{\pgfqpoint{1.862500in}{1.049658in}}%
\pgfpathlineto{\pgfqpoint{1.937500in}{1.049658in}}%
\pgfusepath{stroke}%
\end{pgfscope}%
\begin{pgfscope}%
\pgfpathrectangle{\pgfqpoint{1.000000in}{0.720000in}}{\pgfqpoint{4.800000in}{4.620000in}}%
\pgfusepath{clip}%
\pgfsetrectcap%
\pgfsetroundjoin%
\pgfsetlinewidth{1.003750pt}%
\definecolor{currentstroke}{rgb}{0.000000,0.000000,0.000000}%
\pgfsetstrokecolor{currentstroke}%
\pgfsetdash{}{0pt}%
\pgfpathmoveto{\pgfqpoint{1.862500in}{4.816906in}}%
\pgfpathlineto{\pgfqpoint{1.937500in}{4.816906in}}%
\pgfusepath{stroke}%
\end{pgfscope}%
\begin{pgfscope}%
\pgfpathrectangle{\pgfqpoint{1.000000in}{0.720000in}}{\pgfqpoint{4.800000in}{4.620000in}}%
\pgfusepath{clip}%
\pgfsetrectcap%
\pgfsetroundjoin%
\pgfsetlinewidth{1.003750pt}%
\definecolor{currentstroke}{rgb}{0.000000,0.000000,0.000000}%
\pgfsetstrokecolor{currentstroke}%
\pgfsetdash{}{0pt}%
\pgfpathmoveto{\pgfqpoint{2.200000in}{2.351348in}}%
\pgfpathlineto{\pgfqpoint{2.200000in}{1.236710in}}%
\pgfusepath{stroke}%
\end{pgfscope}%
\begin{pgfscope}%
\pgfpathrectangle{\pgfqpoint{1.000000in}{0.720000in}}{\pgfqpoint{4.800000in}{4.620000in}}%
\pgfusepath{clip}%
\pgfsetrectcap%
\pgfsetroundjoin%
\pgfsetlinewidth{1.003750pt}%
\definecolor{currentstroke}{rgb}{0.000000,0.000000,0.000000}%
\pgfsetstrokecolor{currentstroke}%
\pgfsetdash{}{0pt}%
\pgfpathmoveto{\pgfqpoint{2.200000in}{3.269474in}}%
\pgfpathlineto{\pgfqpoint{2.200000in}{4.640379in}}%
\pgfusepath{stroke}%
\end{pgfscope}%
\begin{pgfscope}%
\pgfpathrectangle{\pgfqpoint{1.000000in}{0.720000in}}{\pgfqpoint{4.800000in}{4.620000in}}%
\pgfusepath{clip}%
\pgfsetrectcap%
\pgfsetroundjoin%
\pgfsetlinewidth{1.003750pt}%
\definecolor{currentstroke}{rgb}{0.000000,0.000000,0.000000}%
\pgfsetstrokecolor{currentstroke}%
\pgfsetdash{}{0pt}%
\pgfpathmoveto{\pgfqpoint{2.162500in}{1.236710in}}%
\pgfpathlineto{\pgfqpoint{2.237500in}{1.236710in}}%
\pgfusepath{stroke}%
\end{pgfscope}%
\begin{pgfscope}%
\pgfpathrectangle{\pgfqpoint{1.000000in}{0.720000in}}{\pgfqpoint{4.800000in}{4.620000in}}%
\pgfusepath{clip}%
\pgfsetrectcap%
\pgfsetroundjoin%
\pgfsetlinewidth{1.003750pt}%
\definecolor{currentstroke}{rgb}{0.000000,0.000000,0.000000}%
\pgfsetstrokecolor{currentstroke}%
\pgfsetdash{}{0pt}%
\pgfpathmoveto{\pgfqpoint{2.162500in}{4.640379in}}%
\pgfpathlineto{\pgfqpoint{2.237500in}{4.640379in}}%
\pgfusepath{stroke}%
\end{pgfscope}%
\begin{pgfscope}%
\pgfpathrectangle{\pgfqpoint{1.000000in}{0.720000in}}{\pgfqpoint{4.800000in}{4.620000in}}%
\pgfusepath{clip}%
\pgfsetrectcap%
\pgfsetroundjoin%
\pgfsetlinewidth{1.003750pt}%
\definecolor{currentstroke}{rgb}{0.000000,0.000000,0.000000}%
\pgfsetstrokecolor{currentstroke}%
\pgfsetdash{}{0pt}%
\pgfpathmoveto{\pgfqpoint{2.500000in}{2.422057in}}%
\pgfpathlineto{\pgfqpoint{2.500000in}{1.439243in}}%
\pgfusepath{stroke}%
\end{pgfscope}%
\begin{pgfscope}%
\pgfpathrectangle{\pgfqpoint{1.000000in}{0.720000in}}{\pgfqpoint{4.800000in}{4.620000in}}%
\pgfusepath{clip}%
\pgfsetrectcap%
\pgfsetroundjoin%
\pgfsetlinewidth{1.003750pt}%
\definecolor{currentstroke}{rgb}{0.000000,0.000000,0.000000}%
\pgfsetstrokecolor{currentstroke}%
\pgfsetdash{}{0pt}%
\pgfpathmoveto{\pgfqpoint{2.500000in}{3.278357in}}%
\pgfpathlineto{\pgfqpoint{2.500000in}{4.539014in}}%
\pgfusepath{stroke}%
\end{pgfscope}%
\begin{pgfscope}%
\pgfpathrectangle{\pgfqpoint{1.000000in}{0.720000in}}{\pgfqpoint{4.800000in}{4.620000in}}%
\pgfusepath{clip}%
\pgfsetrectcap%
\pgfsetroundjoin%
\pgfsetlinewidth{1.003750pt}%
\definecolor{currentstroke}{rgb}{0.000000,0.000000,0.000000}%
\pgfsetstrokecolor{currentstroke}%
\pgfsetdash{}{0pt}%
\pgfpathmoveto{\pgfqpoint{2.462500in}{1.439243in}}%
\pgfpathlineto{\pgfqpoint{2.537500in}{1.439243in}}%
\pgfusepath{stroke}%
\end{pgfscope}%
\begin{pgfscope}%
\pgfpathrectangle{\pgfqpoint{1.000000in}{0.720000in}}{\pgfqpoint{4.800000in}{4.620000in}}%
\pgfusepath{clip}%
\pgfsetrectcap%
\pgfsetroundjoin%
\pgfsetlinewidth{1.003750pt}%
\definecolor{currentstroke}{rgb}{0.000000,0.000000,0.000000}%
\pgfsetstrokecolor{currentstroke}%
\pgfsetdash{}{0pt}%
\pgfpathmoveto{\pgfqpoint{2.462500in}{4.539014in}}%
\pgfpathlineto{\pgfqpoint{2.537500in}{4.539014in}}%
\pgfusepath{stroke}%
\end{pgfscope}%
\begin{pgfscope}%
\pgfpathrectangle{\pgfqpoint{1.000000in}{0.720000in}}{\pgfqpoint{4.800000in}{4.620000in}}%
\pgfusepath{clip}%
\pgfsetrectcap%
\pgfsetroundjoin%
\pgfsetlinewidth{1.003750pt}%
\definecolor{currentstroke}{rgb}{0.000000,0.000000,0.000000}%
\pgfsetstrokecolor{currentstroke}%
\pgfsetdash{}{0pt}%
\pgfpathmoveto{\pgfqpoint{2.800000in}{2.520089in}}%
\pgfpathlineto{\pgfqpoint{2.800000in}{1.443585in}}%
\pgfusepath{stroke}%
\end{pgfscope}%
\begin{pgfscope}%
\pgfpathrectangle{\pgfqpoint{1.000000in}{0.720000in}}{\pgfqpoint{4.800000in}{4.620000in}}%
\pgfusepath{clip}%
\pgfsetrectcap%
\pgfsetroundjoin%
\pgfsetlinewidth{1.003750pt}%
\definecolor{currentstroke}{rgb}{0.000000,0.000000,0.000000}%
\pgfsetstrokecolor{currentstroke}%
\pgfsetdash{}{0pt}%
\pgfpathmoveto{\pgfqpoint{2.800000in}{3.273162in}}%
\pgfpathlineto{\pgfqpoint{2.800000in}{4.366947in}}%
\pgfusepath{stroke}%
\end{pgfscope}%
\begin{pgfscope}%
\pgfpathrectangle{\pgfqpoint{1.000000in}{0.720000in}}{\pgfqpoint{4.800000in}{4.620000in}}%
\pgfusepath{clip}%
\pgfsetrectcap%
\pgfsetroundjoin%
\pgfsetlinewidth{1.003750pt}%
\definecolor{currentstroke}{rgb}{0.000000,0.000000,0.000000}%
\pgfsetstrokecolor{currentstroke}%
\pgfsetdash{}{0pt}%
\pgfpathmoveto{\pgfqpoint{2.762500in}{1.443585in}}%
\pgfpathlineto{\pgfqpoint{2.837500in}{1.443585in}}%
\pgfusepath{stroke}%
\end{pgfscope}%
\begin{pgfscope}%
\pgfpathrectangle{\pgfqpoint{1.000000in}{0.720000in}}{\pgfqpoint{4.800000in}{4.620000in}}%
\pgfusepath{clip}%
\pgfsetrectcap%
\pgfsetroundjoin%
\pgfsetlinewidth{1.003750pt}%
\definecolor{currentstroke}{rgb}{0.000000,0.000000,0.000000}%
\pgfsetstrokecolor{currentstroke}%
\pgfsetdash{}{0pt}%
\pgfpathmoveto{\pgfqpoint{2.762500in}{4.366947in}}%
\pgfpathlineto{\pgfqpoint{2.837500in}{4.366947in}}%
\pgfusepath{stroke}%
\end{pgfscope}%
\begin{pgfscope}%
\pgfpathrectangle{\pgfqpoint{1.000000in}{0.720000in}}{\pgfqpoint{4.800000in}{4.620000in}}%
\pgfusepath{clip}%
\pgfsetrectcap%
\pgfsetroundjoin%
\pgfsetlinewidth{1.003750pt}%
\definecolor{currentstroke}{rgb}{0.000000,0.000000,0.000000}%
\pgfsetstrokecolor{currentstroke}%
\pgfsetdash{}{0pt}%
\pgfpathmoveto{\pgfqpoint{3.100000in}{2.476837in}}%
\pgfpathlineto{\pgfqpoint{3.100000in}{1.548505in}}%
\pgfusepath{stroke}%
\end{pgfscope}%
\begin{pgfscope}%
\pgfpathrectangle{\pgfqpoint{1.000000in}{0.720000in}}{\pgfqpoint{4.800000in}{4.620000in}}%
\pgfusepath{clip}%
\pgfsetrectcap%
\pgfsetroundjoin%
\pgfsetlinewidth{1.003750pt}%
\definecolor{currentstroke}{rgb}{0.000000,0.000000,0.000000}%
\pgfsetstrokecolor{currentstroke}%
\pgfsetdash{}{0pt}%
\pgfpathmoveto{\pgfqpoint{3.100000in}{3.215822in}}%
\pgfpathlineto{\pgfqpoint{3.100000in}{4.226881in}}%
\pgfusepath{stroke}%
\end{pgfscope}%
\begin{pgfscope}%
\pgfpathrectangle{\pgfqpoint{1.000000in}{0.720000in}}{\pgfqpoint{4.800000in}{4.620000in}}%
\pgfusepath{clip}%
\pgfsetrectcap%
\pgfsetroundjoin%
\pgfsetlinewidth{1.003750pt}%
\definecolor{currentstroke}{rgb}{0.000000,0.000000,0.000000}%
\pgfsetstrokecolor{currentstroke}%
\pgfsetdash{}{0pt}%
\pgfpathmoveto{\pgfqpoint{3.062500in}{1.548505in}}%
\pgfpathlineto{\pgfqpoint{3.137500in}{1.548505in}}%
\pgfusepath{stroke}%
\end{pgfscope}%
\begin{pgfscope}%
\pgfpathrectangle{\pgfqpoint{1.000000in}{0.720000in}}{\pgfqpoint{4.800000in}{4.620000in}}%
\pgfusepath{clip}%
\pgfsetrectcap%
\pgfsetroundjoin%
\pgfsetlinewidth{1.003750pt}%
\definecolor{currentstroke}{rgb}{0.000000,0.000000,0.000000}%
\pgfsetstrokecolor{currentstroke}%
\pgfsetdash{}{0pt}%
\pgfpathmoveto{\pgfqpoint{3.062500in}{4.226881in}}%
\pgfpathlineto{\pgfqpoint{3.137500in}{4.226881in}}%
\pgfusepath{stroke}%
\end{pgfscope}%
\begin{pgfscope}%
\pgfpathrectangle{\pgfqpoint{1.000000in}{0.720000in}}{\pgfqpoint{4.800000in}{4.620000in}}%
\pgfusepath{clip}%
\pgfsetrectcap%
\pgfsetroundjoin%
\pgfsetlinewidth{1.003750pt}%
\definecolor{currentstroke}{rgb}{0.000000,0.000000,0.000000}%
\pgfsetstrokecolor{currentstroke}%
\pgfsetdash{}{0pt}%
\pgfpathmoveto{\pgfqpoint{3.400000in}{2.525338in}}%
\pgfpathlineto{\pgfqpoint{3.400000in}{1.628284in}}%
\pgfusepath{stroke}%
\end{pgfscope}%
\begin{pgfscope}%
\pgfpathrectangle{\pgfqpoint{1.000000in}{0.720000in}}{\pgfqpoint{4.800000in}{4.620000in}}%
\pgfusepath{clip}%
\pgfsetrectcap%
\pgfsetroundjoin%
\pgfsetlinewidth{1.003750pt}%
\definecolor{currentstroke}{rgb}{0.000000,0.000000,0.000000}%
\pgfsetstrokecolor{currentstroke}%
\pgfsetdash{}{0pt}%
\pgfpathmoveto{\pgfqpoint{3.400000in}{3.159684in}}%
\pgfpathlineto{\pgfqpoint{3.400000in}{4.076750in}}%
\pgfusepath{stroke}%
\end{pgfscope}%
\begin{pgfscope}%
\pgfpathrectangle{\pgfqpoint{1.000000in}{0.720000in}}{\pgfqpoint{4.800000in}{4.620000in}}%
\pgfusepath{clip}%
\pgfsetrectcap%
\pgfsetroundjoin%
\pgfsetlinewidth{1.003750pt}%
\definecolor{currentstroke}{rgb}{0.000000,0.000000,0.000000}%
\pgfsetstrokecolor{currentstroke}%
\pgfsetdash{}{0pt}%
\pgfpathmoveto{\pgfqpoint{3.362500in}{1.628284in}}%
\pgfpathlineto{\pgfqpoint{3.437500in}{1.628284in}}%
\pgfusepath{stroke}%
\end{pgfscope}%
\begin{pgfscope}%
\pgfpathrectangle{\pgfqpoint{1.000000in}{0.720000in}}{\pgfqpoint{4.800000in}{4.620000in}}%
\pgfusepath{clip}%
\pgfsetrectcap%
\pgfsetroundjoin%
\pgfsetlinewidth{1.003750pt}%
\definecolor{currentstroke}{rgb}{0.000000,0.000000,0.000000}%
\pgfsetstrokecolor{currentstroke}%
\pgfsetdash{}{0pt}%
\pgfpathmoveto{\pgfqpoint{3.362500in}{4.076750in}}%
\pgfpathlineto{\pgfqpoint{3.437500in}{4.076750in}}%
\pgfusepath{stroke}%
\end{pgfscope}%
\begin{pgfscope}%
\pgfpathrectangle{\pgfqpoint{1.000000in}{0.720000in}}{\pgfqpoint{4.800000in}{4.620000in}}%
\pgfusepath{clip}%
\pgfsetrectcap%
\pgfsetroundjoin%
\pgfsetlinewidth{1.003750pt}%
\definecolor{currentstroke}{rgb}{0.000000,0.000000,0.000000}%
\pgfsetstrokecolor{currentstroke}%
\pgfsetdash{}{0pt}%
\pgfpathmoveto{\pgfqpoint{3.700000in}{2.519957in}}%
\pgfpathlineto{\pgfqpoint{3.700000in}{1.742665in}}%
\pgfusepath{stroke}%
\end{pgfscope}%
\begin{pgfscope}%
\pgfpathrectangle{\pgfqpoint{1.000000in}{0.720000in}}{\pgfqpoint{4.800000in}{4.620000in}}%
\pgfusepath{clip}%
\pgfsetrectcap%
\pgfsetroundjoin%
\pgfsetlinewidth{1.003750pt}%
\definecolor{currentstroke}{rgb}{0.000000,0.000000,0.000000}%
\pgfsetstrokecolor{currentstroke}%
\pgfsetdash{}{0pt}%
\pgfpathmoveto{\pgfqpoint{3.700000in}{3.213477in}}%
\pgfpathlineto{\pgfqpoint{3.700000in}{4.226135in}}%
\pgfusepath{stroke}%
\end{pgfscope}%
\begin{pgfscope}%
\pgfpathrectangle{\pgfqpoint{1.000000in}{0.720000in}}{\pgfqpoint{4.800000in}{4.620000in}}%
\pgfusepath{clip}%
\pgfsetrectcap%
\pgfsetroundjoin%
\pgfsetlinewidth{1.003750pt}%
\definecolor{currentstroke}{rgb}{0.000000,0.000000,0.000000}%
\pgfsetstrokecolor{currentstroke}%
\pgfsetdash{}{0pt}%
\pgfpathmoveto{\pgfqpoint{3.662500in}{1.742665in}}%
\pgfpathlineto{\pgfqpoint{3.737500in}{1.742665in}}%
\pgfusepath{stroke}%
\end{pgfscope}%
\begin{pgfscope}%
\pgfpathrectangle{\pgfqpoint{1.000000in}{0.720000in}}{\pgfqpoint{4.800000in}{4.620000in}}%
\pgfusepath{clip}%
\pgfsetrectcap%
\pgfsetroundjoin%
\pgfsetlinewidth{1.003750pt}%
\definecolor{currentstroke}{rgb}{0.000000,0.000000,0.000000}%
\pgfsetstrokecolor{currentstroke}%
\pgfsetdash{}{0pt}%
\pgfpathmoveto{\pgfqpoint{3.662500in}{4.226135in}}%
\pgfpathlineto{\pgfqpoint{3.737500in}{4.226135in}}%
\pgfusepath{stroke}%
\end{pgfscope}%
\begin{pgfscope}%
\pgfpathrectangle{\pgfqpoint{1.000000in}{0.720000in}}{\pgfqpoint{4.800000in}{4.620000in}}%
\pgfusepath{clip}%
\pgfsetrectcap%
\pgfsetroundjoin%
\pgfsetlinewidth{1.003750pt}%
\definecolor{currentstroke}{rgb}{0.000000,0.000000,0.000000}%
\pgfsetstrokecolor{currentstroke}%
\pgfsetdash{}{0pt}%
\pgfpathmoveto{\pgfqpoint{4.000000in}{2.625229in}}%
\pgfpathlineto{\pgfqpoint{4.000000in}{2.082967in}}%
\pgfusepath{stroke}%
\end{pgfscope}%
\begin{pgfscope}%
\pgfpathrectangle{\pgfqpoint{1.000000in}{0.720000in}}{\pgfqpoint{4.800000in}{4.620000in}}%
\pgfusepath{clip}%
\pgfsetrectcap%
\pgfsetroundjoin%
\pgfsetlinewidth{1.003750pt}%
\definecolor{currentstroke}{rgb}{0.000000,0.000000,0.000000}%
\pgfsetstrokecolor{currentstroke}%
\pgfsetdash{}{0pt}%
\pgfpathmoveto{\pgfqpoint{4.000000in}{3.162845in}}%
\pgfpathlineto{\pgfqpoint{4.000000in}{3.626430in}}%
\pgfusepath{stroke}%
\end{pgfscope}%
\begin{pgfscope}%
\pgfpathrectangle{\pgfqpoint{1.000000in}{0.720000in}}{\pgfqpoint{4.800000in}{4.620000in}}%
\pgfusepath{clip}%
\pgfsetrectcap%
\pgfsetroundjoin%
\pgfsetlinewidth{1.003750pt}%
\definecolor{currentstroke}{rgb}{0.000000,0.000000,0.000000}%
\pgfsetstrokecolor{currentstroke}%
\pgfsetdash{}{0pt}%
\pgfpathmoveto{\pgfqpoint{3.962500in}{2.082967in}}%
\pgfpathlineto{\pgfqpoint{4.037500in}{2.082967in}}%
\pgfusepath{stroke}%
\end{pgfscope}%
\begin{pgfscope}%
\pgfpathrectangle{\pgfqpoint{1.000000in}{0.720000in}}{\pgfqpoint{4.800000in}{4.620000in}}%
\pgfusepath{clip}%
\pgfsetrectcap%
\pgfsetroundjoin%
\pgfsetlinewidth{1.003750pt}%
\definecolor{currentstroke}{rgb}{0.000000,0.000000,0.000000}%
\pgfsetstrokecolor{currentstroke}%
\pgfsetdash{}{0pt}%
\pgfpathmoveto{\pgfqpoint{3.962500in}{3.626430in}}%
\pgfpathlineto{\pgfqpoint{4.037500in}{3.626430in}}%
\pgfusepath{stroke}%
\end{pgfscope}%
\begin{pgfscope}%
\pgfpathrectangle{\pgfqpoint{1.000000in}{0.720000in}}{\pgfqpoint{4.800000in}{4.620000in}}%
\pgfusepath{clip}%
\pgfsetrectcap%
\pgfsetroundjoin%
\pgfsetlinewidth{1.003750pt}%
\definecolor{currentstroke}{rgb}{0.000000,0.000000,0.000000}%
\pgfsetstrokecolor{currentstroke}%
\pgfsetdash{}{0pt}%
\pgfpathmoveto{\pgfqpoint{4.300000in}{2.406026in}}%
\pgfpathlineto{\pgfqpoint{4.300000in}{2.150228in}}%
\pgfusepath{stroke}%
\end{pgfscope}%
\begin{pgfscope}%
\pgfpathrectangle{\pgfqpoint{1.000000in}{0.720000in}}{\pgfqpoint{4.800000in}{4.620000in}}%
\pgfusepath{clip}%
\pgfsetrectcap%
\pgfsetroundjoin%
\pgfsetlinewidth{1.003750pt}%
\definecolor{currentstroke}{rgb}{0.000000,0.000000,0.000000}%
\pgfsetstrokecolor{currentstroke}%
\pgfsetdash{}{0pt}%
\pgfpathmoveto{\pgfqpoint{4.300000in}{2.831177in}}%
\pgfpathlineto{\pgfqpoint{4.300000in}{3.406266in}}%
\pgfusepath{stroke}%
\end{pgfscope}%
\begin{pgfscope}%
\pgfpathrectangle{\pgfqpoint{1.000000in}{0.720000in}}{\pgfqpoint{4.800000in}{4.620000in}}%
\pgfusepath{clip}%
\pgfsetrectcap%
\pgfsetroundjoin%
\pgfsetlinewidth{1.003750pt}%
\definecolor{currentstroke}{rgb}{0.000000,0.000000,0.000000}%
\pgfsetstrokecolor{currentstroke}%
\pgfsetdash{}{0pt}%
\pgfpathmoveto{\pgfqpoint{4.262500in}{2.150228in}}%
\pgfpathlineto{\pgfqpoint{4.337500in}{2.150228in}}%
\pgfusepath{stroke}%
\end{pgfscope}%
\begin{pgfscope}%
\pgfpathrectangle{\pgfqpoint{1.000000in}{0.720000in}}{\pgfqpoint{4.800000in}{4.620000in}}%
\pgfusepath{clip}%
\pgfsetrectcap%
\pgfsetroundjoin%
\pgfsetlinewidth{1.003750pt}%
\definecolor{currentstroke}{rgb}{0.000000,0.000000,0.000000}%
\pgfsetstrokecolor{currentstroke}%
\pgfsetdash{}{0pt}%
\pgfpathmoveto{\pgfqpoint{4.262500in}{3.406266in}}%
\pgfpathlineto{\pgfqpoint{4.337500in}{3.406266in}}%
\pgfusepath{stroke}%
\end{pgfscope}%
\begin{pgfscope}%
\pgfpathrectangle{\pgfqpoint{1.000000in}{0.720000in}}{\pgfqpoint{4.800000in}{4.620000in}}%
\pgfusepath{clip}%
\pgfsetrectcap%
\pgfsetroundjoin%
\pgfsetlinewidth{1.003750pt}%
\definecolor{currentstroke}{rgb}{0.000000,0.000000,0.000000}%
\pgfsetstrokecolor{currentstroke}%
\pgfsetdash{}{0pt}%
\pgfpathmoveto{\pgfqpoint{4.600000in}{2.990152in}}%
\pgfpathlineto{\pgfqpoint{4.600000in}{2.716960in}}%
\pgfusepath{stroke}%
\end{pgfscope}%
\begin{pgfscope}%
\pgfpathrectangle{\pgfqpoint{1.000000in}{0.720000in}}{\pgfqpoint{4.800000in}{4.620000in}}%
\pgfusepath{clip}%
\pgfsetrectcap%
\pgfsetroundjoin%
\pgfsetlinewidth{1.003750pt}%
\definecolor{currentstroke}{rgb}{0.000000,0.000000,0.000000}%
\pgfsetstrokecolor{currentstroke}%
\pgfsetdash{}{0pt}%
\pgfpathmoveto{\pgfqpoint{4.600000in}{3.539648in}}%
\pgfpathlineto{\pgfqpoint{4.600000in}{3.998836in}}%
\pgfusepath{stroke}%
\end{pgfscope}%
\begin{pgfscope}%
\pgfpathrectangle{\pgfqpoint{1.000000in}{0.720000in}}{\pgfqpoint{4.800000in}{4.620000in}}%
\pgfusepath{clip}%
\pgfsetrectcap%
\pgfsetroundjoin%
\pgfsetlinewidth{1.003750pt}%
\definecolor{currentstroke}{rgb}{0.000000,0.000000,0.000000}%
\pgfsetstrokecolor{currentstroke}%
\pgfsetdash{}{0pt}%
\pgfpathmoveto{\pgfqpoint{4.562500in}{2.716960in}}%
\pgfpathlineto{\pgfqpoint{4.637500in}{2.716960in}}%
\pgfusepath{stroke}%
\end{pgfscope}%
\begin{pgfscope}%
\pgfpathrectangle{\pgfqpoint{1.000000in}{0.720000in}}{\pgfqpoint{4.800000in}{4.620000in}}%
\pgfusepath{clip}%
\pgfsetrectcap%
\pgfsetroundjoin%
\pgfsetlinewidth{1.003750pt}%
\definecolor{currentstroke}{rgb}{0.000000,0.000000,0.000000}%
\pgfsetstrokecolor{currentstroke}%
\pgfsetdash{}{0pt}%
\pgfpathmoveto{\pgfqpoint{4.562500in}{3.998836in}}%
\pgfpathlineto{\pgfqpoint{4.637500in}{3.998836in}}%
\pgfusepath{stroke}%
\end{pgfscope}%
\begin{pgfscope}%
\pgfpathrectangle{\pgfqpoint{1.000000in}{0.720000in}}{\pgfqpoint{4.800000in}{4.620000in}}%
\pgfusepath{clip}%
\pgfsetrectcap%
\pgfsetroundjoin%
\pgfsetlinewidth{1.003750pt}%
\definecolor{currentstroke}{rgb}{0.000000,0.000000,0.000000}%
\pgfsetstrokecolor{currentstroke}%
\pgfsetdash{}{0pt}%
\pgfpathmoveto{\pgfqpoint{4.900000in}{2.252842in}}%
\pgfpathlineto{\pgfqpoint{4.900000in}{2.252842in}}%
\pgfusepath{stroke}%
\end{pgfscope}%
\begin{pgfscope}%
\pgfpathrectangle{\pgfqpoint{1.000000in}{0.720000in}}{\pgfqpoint{4.800000in}{4.620000in}}%
\pgfusepath{clip}%
\pgfsetrectcap%
\pgfsetroundjoin%
\pgfsetlinewidth{1.003750pt}%
\definecolor{currentstroke}{rgb}{0.000000,0.000000,0.000000}%
\pgfsetstrokecolor{currentstroke}%
\pgfsetdash{}{0pt}%
\pgfpathmoveto{\pgfqpoint{4.900000in}{2.252842in}}%
\pgfpathlineto{\pgfqpoint{4.900000in}{2.252842in}}%
\pgfusepath{stroke}%
\end{pgfscope}%
\begin{pgfscope}%
\pgfpathrectangle{\pgfqpoint{1.000000in}{0.720000in}}{\pgfqpoint{4.800000in}{4.620000in}}%
\pgfusepath{clip}%
\pgfsetrectcap%
\pgfsetroundjoin%
\pgfsetlinewidth{1.003750pt}%
\definecolor{currentstroke}{rgb}{0.000000,0.000000,0.000000}%
\pgfsetstrokecolor{currentstroke}%
\pgfsetdash{}{0pt}%
\pgfpathmoveto{\pgfqpoint{4.862500in}{2.252842in}}%
\pgfpathlineto{\pgfqpoint{4.937500in}{2.252842in}}%
\pgfusepath{stroke}%
\end{pgfscope}%
\begin{pgfscope}%
\pgfpathrectangle{\pgfqpoint{1.000000in}{0.720000in}}{\pgfqpoint{4.800000in}{4.620000in}}%
\pgfusepath{clip}%
\pgfsetrectcap%
\pgfsetroundjoin%
\pgfsetlinewidth{1.003750pt}%
\definecolor{currentstroke}{rgb}{0.000000,0.000000,0.000000}%
\pgfsetstrokecolor{currentstroke}%
\pgfsetdash{}{0pt}%
\pgfpathmoveto{\pgfqpoint{4.862500in}{2.252842in}}%
\pgfpathlineto{\pgfqpoint{4.937500in}{2.252842in}}%
\pgfusepath{stroke}%
\end{pgfscope}%
\begin{pgfscope}%
\pgfpathrectangle{\pgfqpoint{1.000000in}{0.720000in}}{\pgfqpoint{4.800000in}{4.620000in}}%
\pgfusepath{clip}%
\pgfsetrectcap%
\pgfsetroundjoin%
\pgfsetlinewidth{1.003750pt}%
\definecolor{currentstroke}{rgb}{0.000000,0.000000,0.000000}%
\pgfsetstrokecolor{currentstroke}%
\pgfsetdash{}{0pt}%
\pgfpathmoveto{\pgfqpoint{5.200000in}{2.849472in}}%
\pgfpathlineto{\pgfqpoint{5.200000in}{2.849472in}}%
\pgfusepath{stroke}%
\end{pgfscope}%
\begin{pgfscope}%
\pgfpathrectangle{\pgfqpoint{1.000000in}{0.720000in}}{\pgfqpoint{4.800000in}{4.620000in}}%
\pgfusepath{clip}%
\pgfsetrectcap%
\pgfsetroundjoin%
\pgfsetlinewidth{1.003750pt}%
\definecolor{currentstroke}{rgb}{0.000000,0.000000,0.000000}%
\pgfsetstrokecolor{currentstroke}%
\pgfsetdash{}{0pt}%
\pgfpathmoveto{\pgfqpoint{5.200000in}{2.849472in}}%
\pgfpathlineto{\pgfqpoint{5.200000in}{2.849472in}}%
\pgfusepath{stroke}%
\end{pgfscope}%
\begin{pgfscope}%
\pgfpathrectangle{\pgfqpoint{1.000000in}{0.720000in}}{\pgfqpoint{4.800000in}{4.620000in}}%
\pgfusepath{clip}%
\pgfsetrectcap%
\pgfsetroundjoin%
\pgfsetlinewidth{1.003750pt}%
\definecolor{currentstroke}{rgb}{0.000000,0.000000,0.000000}%
\pgfsetstrokecolor{currentstroke}%
\pgfsetdash{}{0pt}%
\pgfpathmoveto{\pgfqpoint{5.162500in}{2.849472in}}%
\pgfpathlineto{\pgfqpoint{5.237500in}{2.849472in}}%
\pgfusepath{stroke}%
\end{pgfscope}%
\begin{pgfscope}%
\pgfpathrectangle{\pgfqpoint{1.000000in}{0.720000in}}{\pgfqpoint{4.800000in}{4.620000in}}%
\pgfusepath{clip}%
\pgfsetrectcap%
\pgfsetroundjoin%
\pgfsetlinewidth{1.003750pt}%
\definecolor{currentstroke}{rgb}{0.000000,0.000000,0.000000}%
\pgfsetstrokecolor{currentstroke}%
\pgfsetdash{}{0pt}%
\pgfpathmoveto{\pgfqpoint{5.162500in}{2.849472in}}%
\pgfpathlineto{\pgfqpoint{5.237500in}{2.849472in}}%
\pgfusepath{stroke}%
\end{pgfscope}%
\begin{pgfscope}%
\pgfpathrectangle{\pgfqpoint{1.000000in}{0.720000in}}{\pgfqpoint{4.800000in}{4.620000in}}%
\pgfusepath{clip}%
\pgfsetrectcap%
\pgfsetroundjoin%
\pgfsetlinewidth{1.003750pt}%
\definecolor{currentstroke}{rgb}{0.000000,0.000000,0.000000}%
\pgfsetstrokecolor{currentstroke}%
\pgfsetdash{}{0pt}%
\pgfpathmoveto{\pgfqpoint{5.500000in}{2.810787in}}%
\pgfpathlineto{\pgfqpoint{5.500000in}{2.810787in}}%
\pgfusepath{stroke}%
\end{pgfscope}%
\begin{pgfscope}%
\pgfpathrectangle{\pgfqpoint{1.000000in}{0.720000in}}{\pgfqpoint{4.800000in}{4.620000in}}%
\pgfusepath{clip}%
\pgfsetrectcap%
\pgfsetroundjoin%
\pgfsetlinewidth{1.003750pt}%
\definecolor{currentstroke}{rgb}{0.000000,0.000000,0.000000}%
\pgfsetstrokecolor{currentstroke}%
\pgfsetdash{}{0pt}%
\pgfpathmoveto{\pgfqpoint{5.500000in}{2.810787in}}%
\pgfpathlineto{\pgfqpoint{5.500000in}{2.810787in}}%
\pgfusepath{stroke}%
\end{pgfscope}%
\begin{pgfscope}%
\pgfpathrectangle{\pgfqpoint{1.000000in}{0.720000in}}{\pgfqpoint{4.800000in}{4.620000in}}%
\pgfusepath{clip}%
\pgfsetrectcap%
\pgfsetroundjoin%
\pgfsetlinewidth{1.003750pt}%
\definecolor{currentstroke}{rgb}{0.000000,0.000000,0.000000}%
\pgfsetstrokecolor{currentstroke}%
\pgfsetdash{}{0pt}%
\pgfpathmoveto{\pgfqpoint{5.462500in}{2.810787in}}%
\pgfpathlineto{\pgfqpoint{5.537500in}{2.810787in}}%
\pgfusepath{stroke}%
\end{pgfscope}%
\begin{pgfscope}%
\pgfpathrectangle{\pgfqpoint{1.000000in}{0.720000in}}{\pgfqpoint{4.800000in}{4.620000in}}%
\pgfusepath{clip}%
\pgfsetrectcap%
\pgfsetroundjoin%
\pgfsetlinewidth{1.003750pt}%
\definecolor{currentstroke}{rgb}{0.000000,0.000000,0.000000}%
\pgfsetstrokecolor{currentstroke}%
\pgfsetdash{}{0pt}%
\pgfpathmoveto{\pgfqpoint{5.462500in}{2.810787in}}%
\pgfpathlineto{\pgfqpoint{5.537500in}{2.810787in}}%
\pgfusepath{stroke}%
\end{pgfscope}%
\begin{pgfscope}%
\pgfpathrectangle{\pgfqpoint{1.000000in}{0.720000in}}{\pgfqpoint{4.800000in}{4.620000in}}%
\pgfusepath{clip}%
\pgfsetrectcap%
\pgfsetroundjoin%
\pgfsetlinewidth{2.007500pt}%
\definecolor{currentstroke}{rgb}{0.121569,0.466667,0.705882}%
\pgfsetstrokecolor{currentstroke}%
\pgfsetdash{}{0pt}%
\pgfpathmoveto{\pgfqpoint{1.300000in}{2.059647in}}%
\pgfpathlineto{\pgfqpoint{1.600000in}{2.626181in}}%
\pgfpathlineto{\pgfqpoint{1.900000in}{2.703584in}}%
\pgfpathlineto{\pgfqpoint{2.200000in}{2.776231in}}%
\pgfpathlineto{\pgfqpoint{2.500000in}{2.844559in}}%
\pgfpathlineto{\pgfqpoint{2.800000in}{2.867053in}}%
\pgfpathlineto{\pgfqpoint{3.100000in}{2.800935in}}%
\pgfpathlineto{\pgfqpoint{3.400000in}{2.823390in}}%
\pgfpathlineto{\pgfqpoint{3.700000in}{2.785253in}}%
\pgfpathlineto{\pgfqpoint{4.000000in}{2.797547in}}%
\pgfpathlineto{\pgfqpoint{4.300000in}{2.763553in}}%
\pgfpathlineto{\pgfqpoint{4.600000in}{3.309428in}}%
\pgfpathlineto{\pgfqpoint{4.900000in}{2.252842in}}%
\pgfpathlineto{\pgfqpoint{5.200000in}{2.849472in}}%
\pgfpathlineto{\pgfqpoint{5.500000in}{2.810787in}}%
\pgfusepath{stroke}%
\end{pgfscope}%
\begin{pgfscope}%
\pgfpathrectangle{\pgfqpoint{1.000000in}{0.720000in}}{\pgfqpoint{4.800000in}{4.620000in}}%
\pgfusepath{clip}%
\pgfsetbuttcap%
\pgfsetmiterjoin%
\definecolor{currentfill}{rgb}{0.121569,0.466667,0.705882}%
\pgfsetfillcolor{currentfill}%
\pgfsetlinewidth{1.003750pt}%
\definecolor{currentstroke}{rgb}{0.000000,0.000000,0.000000}%
\pgfsetstrokecolor{currentstroke}%
\pgfsetdash{}{0pt}%
\pgfpathmoveto{\pgfqpoint{1.225000in}{1.361115in}}%
\pgfpathlineto{\pgfqpoint{1.375000in}{1.361115in}}%
\pgfpathlineto{\pgfqpoint{1.375000in}{2.939637in}}%
\pgfpathlineto{\pgfqpoint{1.225000in}{2.939637in}}%
\pgfpathlineto{\pgfqpoint{1.225000in}{1.361115in}}%
\pgfpathclose%
\pgfusepath{stroke,fill}%
\end{pgfscope}%
\begin{pgfscope}%
\pgfpathrectangle{\pgfqpoint{1.000000in}{0.720000in}}{\pgfqpoint{4.800000in}{4.620000in}}%
\pgfusepath{clip}%
\pgfsetbuttcap%
\pgfsetmiterjoin%
\definecolor{currentfill}{rgb}{0.121569,0.466667,0.705882}%
\pgfsetfillcolor{currentfill}%
\pgfsetlinewidth{1.003750pt}%
\definecolor{currentstroke}{rgb}{0.000000,0.000000,0.000000}%
\pgfsetstrokecolor{currentstroke}%
\pgfsetdash{}{0pt}%
\pgfpathmoveto{\pgfqpoint{1.525000in}{2.052272in}}%
\pgfpathlineto{\pgfqpoint{1.675000in}{2.052272in}}%
\pgfpathlineto{\pgfqpoint{1.675000in}{3.271276in}}%
\pgfpathlineto{\pgfqpoint{1.525000in}{3.271276in}}%
\pgfpathlineto{\pgfqpoint{1.525000in}{2.052272in}}%
\pgfpathclose%
\pgfusepath{stroke,fill}%
\end{pgfscope}%
\begin{pgfscope}%
\pgfpathrectangle{\pgfqpoint{1.000000in}{0.720000in}}{\pgfqpoint{4.800000in}{4.620000in}}%
\pgfusepath{clip}%
\pgfsetbuttcap%
\pgfsetmiterjoin%
\definecolor{currentfill}{rgb}{0.121569,0.466667,0.705882}%
\pgfsetfillcolor{currentfill}%
\pgfsetlinewidth{1.003750pt}%
\definecolor{currentstroke}{rgb}{0.000000,0.000000,0.000000}%
\pgfsetstrokecolor{currentstroke}%
\pgfsetdash{}{0pt}%
\pgfpathmoveto{\pgfqpoint{1.825000in}{2.206069in}}%
\pgfpathlineto{\pgfqpoint{1.975000in}{2.206069in}}%
\pgfpathlineto{\pgfqpoint{1.975000in}{3.253534in}}%
\pgfpathlineto{\pgfqpoint{1.825000in}{3.253534in}}%
\pgfpathlineto{\pgfqpoint{1.825000in}{2.206069in}}%
\pgfpathclose%
\pgfusepath{stroke,fill}%
\end{pgfscope}%
\begin{pgfscope}%
\pgfpathrectangle{\pgfqpoint{1.000000in}{0.720000in}}{\pgfqpoint{4.800000in}{4.620000in}}%
\pgfusepath{clip}%
\pgfsetbuttcap%
\pgfsetmiterjoin%
\definecolor{currentfill}{rgb}{0.121569,0.466667,0.705882}%
\pgfsetfillcolor{currentfill}%
\pgfsetlinewidth{1.003750pt}%
\definecolor{currentstroke}{rgb}{0.000000,0.000000,0.000000}%
\pgfsetstrokecolor{currentstroke}%
\pgfsetdash{}{0pt}%
\pgfpathmoveto{\pgfqpoint{2.125000in}{2.351348in}}%
\pgfpathlineto{\pgfqpoint{2.275000in}{2.351348in}}%
\pgfpathlineto{\pgfqpoint{2.275000in}{3.269474in}}%
\pgfpathlineto{\pgfqpoint{2.125000in}{3.269474in}}%
\pgfpathlineto{\pgfqpoint{2.125000in}{2.351348in}}%
\pgfpathclose%
\pgfusepath{stroke,fill}%
\end{pgfscope}%
\begin{pgfscope}%
\pgfpathrectangle{\pgfqpoint{1.000000in}{0.720000in}}{\pgfqpoint{4.800000in}{4.620000in}}%
\pgfusepath{clip}%
\pgfsetbuttcap%
\pgfsetmiterjoin%
\definecolor{currentfill}{rgb}{0.121569,0.466667,0.705882}%
\pgfsetfillcolor{currentfill}%
\pgfsetlinewidth{1.003750pt}%
\definecolor{currentstroke}{rgb}{0.000000,0.000000,0.000000}%
\pgfsetstrokecolor{currentstroke}%
\pgfsetdash{}{0pt}%
\pgfpathmoveto{\pgfqpoint{2.425000in}{2.422057in}}%
\pgfpathlineto{\pgfqpoint{2.575000in}{2.422057in}}%
\pgfpathlineto{\pgfqpoint{2.575000in}{3.278357in}}%
\pgfpathlineto{\pgfqpoint{2.425000in}{3.278357in}}%
\pgfpathlineto{\pgfqpoint{2.425000in}{2.422057in}}%
\pgfpathclose%
\pgfusepath{stroke,fill}%
\end{pgfscope}%
\begin{pgfscope}%
\pgfpathrectangle{\pgfqpoint{1.000000in}{0.720000in}}{\pgfqpoint{4.800000in}{4.620000in}}%
\pgfusepath{clip}%
\pgfsetbuttcap%
\pgfsetmiterjoin%
\definecolor{currentfill}{rgb}{0.121569,0.466667,0.705882}%
\pgfsetfillcolor{currentfill}%
\pgfsetlinewidth{1.003750pt}%
\definecolor{currentstroke}{rgb}{0.000000,0.000000,0.000000}%
\pgfsetstrokecolor{currentstroke}%
\pgfsetdash{}{0pt}%
\pgfpathmoveto{\pgfqpoint{2.725000in}{2.520089in}}%
\pgfpathlineto{\pgfqpoint{2.875000in}{2.520089in}}%
\pgfpathlineto{\pgfqpoint{2.875000in}{3.273162in}}%
\pgfpathlineto{\pgfqpoint{2.725000in}{3.273162in}}%
\pgfpathlineto{\pgfqpoint{2.725000in}{2.520089in}}%
\pgfpathclose%
\pgfusepath{stroke,fill}%
\end{pgfscope}%
\begin{pgfscope}%
\pgfpathrectangle{\pgfqpoint{1.000000in}{0.720000in}}{\pgfqpoint{4.800000in}{4.620000in}}%
\pgfusepath{clip}%
\pgfsetbuttcap%
\pgfsetmiterjoin%
\definecolor{currentfill}{rgb}{0.121569,0.466667,0.705882}%
\pgfsetfillcolor{currentfill}%
\pgfsetlinewidth{1.003750pt}%
\definecolor{currentstroke}{rgb}{0.000000,0.000000,0.000000}%
\pgfsetstrokecolor{currentstroke}%
\pgfsetdash{}{0pt}%
\pgfpathmoveto{\pgfqpoint{3.025000in}{2.476837in}}%
\pgfpathlineto{\pgfqpoint{3.175000in}{2.476837in}}%
\pgfpathlineto{\pgfqpoint{3.175000in}{3.215822in}}%
\pgfpathlineto{\pgfqpoint{3.025000in}{3.215822in}}%
\pgfpathlineto{\pgfqpoint{3.025000in}{2.476837in}}%
\pgfpathclose%
\pgfusepath{stroke,fill}%
\end{pgfscope}%
\begin{pgfscope}%
\pgfpathrectangle{\pgfqpoint{1.000000in}{0.720000in}}{\pgfqpoint{4.800000in}{4.620000in}}%
\pgfusepath{clip}%
\pgfsetbuttcap%
\pgfsetmiterjoin%
\definecolor{currentfill}{rgb}{0.121569,0.466667,0.705882}%
\pgfsetfillcolor{currentfill}%
\pgfsetlinewidth{1.003750pt}%
\definecolor{currentstroke}{rgb}{0.000000,0.000000,0.000000}%
\pgfsetstrokecolor{currentstroke}%
\pgfsetdash{}{0pt}%
\pgfpathmoveto{\pgfqpoint{3.325000in}{2.525338in}}%
\pgfpathlineto{\pgfqpoint{3.475000in}{2.525338in}}%
\pgfpathlineto{\pgfqpoint{3.475000in}{3.159684in}}%
\pgfpathlineto{\pgfqpoint{3.325000in}{3.159684in}}%
\pgfpathlineto{\pgfqpoint{3.325000in}{2.525338in}}%
\pgfpathclose%
\pgfusepath{stroke,fill}%
\end{pgfscope}%
\begin{pgfscope}%
\pgfpathrectangle{\pgfqpoint{1.000000in}{0.720000in}}{\pgfqpoint{4.800000in}{4.620000in}}%
\pgfusepath{clip}%
\pgfsetbuttcap%
\pgfsetmiterjoin%
\definecolor{currentfill}{rgb}{0.121569,0.466667,0.705882}%
\pgfsetfillcolor{currentfill}%
\pgfsetlinewidth{1.003750pt}%
\definecolor{currentstroke}{rgb}{0.000000,0.000000,0.000000}%
\pgfsetstrokecolor{currentstroke}%
\pgfsetdash{}{0pt}%
\pgfpathmoveto{\pgfqpoint{3.625000in}{2.519957in}}%
\pgfpathlineto{\pgfqpoint{3.775000in}{2.519957in}}%
\pgfpathlineto{\pgfqpoint{3.775000in}{3.213477in}}%
\pgfpathlineto{\pgfqpoint{3.625000in}{3.213477in}}%
\pgfpathlineto{\pgfqpoint{3.625000in}{2.519957in}}%
\pgfpathclose%
\pgfusepath{stroke,fill}%
\end{pgfscope}%
\begin{pgfscope}%
\pgfpathrectangle{\pgfqpoint{1.000000in}{0.720000in}}{\pgfqpoint{4.800000in}{4.620000in}}%
\pgfusepath{clip}%
\pgfsetbuttcap%
\pgfsetmiterjoin%
\definecolor{currentfill}{rgb}{0.121569,0.466667,0.705882}%
\pgfsetfillcolor{currentfill}%
\pgfsetlinewidth{1.003750pt}%
\definecolor{currentstroke}{rgb}{0.000000,0.000000,0.000000}%
\pgfsetstrokecolor{currentstroke}%
\pgfsetdash{}{0pt}%
\pgfpathmoveto{\pgfqpoint{3.925000in}{2.625229in}}%
\pgfpathlineto{\pgfqpoint{4.075000in}{2.625229in}}%
\pgfpathlineto{\pgfqpoint{4.075000in}{3.162845in}}%
\pgfpathlineto{\pgfqpoint{3.925000in}{3.162845in}}%
\pgfpathlineto{\pgfqpoint{3.925000in}{2.625229in}}%
\pgfpathclose%
\pgfusepath{stroke,fill}%
\end{pgfscope}%
\begin{pgfscope}%
\pgfpathrectangle{\pgfqpoint{1.000000in}{0.720000in}}{\pgfqpoint{4.800000in}{4.620000in}}%
\pgfusepath{clip}%
\pgfsetbuttcap%
\pgfsetmiterjoin%
\definecolor{currentfill}{rgb}{0.121569,0.466667,0.705882}%
\pgfsetfillcolor{currentfill}%
\pgfsetlinewidth{1.003750pt}%
\definecolor{currentstroke}{rgb}{0.000000,0.000000,0.000000}%
\pgfsetstrokecolor{currentstroke}%
\pgfsetdash{}{0pt}%
\pgfpathmoveto{\pgfqpoint{4.225000in}{2.406026in}}%
\pgfpathlineto{\pgfqpoint{4.375000in}{2.406026in}}%
\pgfpathlineto{\pgfqpoint{4.375000in}{2.831177in}}%
\pgfpathlineto{\pgfqpoint{4.225000in}{2.831177in}}%
\pgfpathlineto{\pgfqpoint{4.225000in}{2.406026in}}%
\pgfpathclose%
\pgfusepath{stroke,fill}%
\end{pgfscope}%
\begin{pgfscope}%
\pgfpathrectangle{\pgfqpoint{1.000000in}{0.720000in}}{\pgfqpoint{4.800000in}{4.620000in}}%
\pgfusepath{clip}%
\pgfsetbuttcap%
\pgfsetmiterjoin%
\definecolor{currentfill}{rgb}{0.121569,0.466667,0.705882}%
\pgfsetfillcolor{currentfill}%
\pgfsetlinewidth{1.003750pt}%
\definecolor{currentstroke}{rgb}{0.000000,0.000000,0.000000}%
\pgfsetstrokecolor{currentstroke}%
\pgfsetdash{}{0pt}%
\pgfpathmoveto{\pgfqpoint{4.525000in}{2.990152in}}%
\pgfpathlineto{\pgfqpoint{4.675000in}{2.990152in}}%
\pgfpathlineto{\pgfqpoint{4.675000in}{3.539648in}}%
\pgfpathlineto{\pgfqpoint{4.525000in}{3.539648in}}%
\pgfpathlineto{\pgfqpoint{4.525000in}{2.990152in}}%
\pgfpathclose%
\pgfusepath{stroke,fill}%
\end{pgfscope}%
\begin{pgfscope}%
\pgfpathrectangle{\pgfqpoint{1.000000in}{0.720000in}}{\pgfqpoint{4.800000in}{4.620000in}}%
\pgfusepath{clip}%
\pgfsetbuttcap%
\pgfsetmiterjoin%
\definecolor{currentfill}{rgb}{0.121569,0.466667,0.705882}%
\pgfsetfillcolor{currentfill}%
\pgfsetlinewidth{1.003750pt}%
\definecolor{currentstroke}{rgb}{0.000000,0.000000,0.000000}%
\pgfsetstrokecolor{currentstroke}%
\pgfsetdash{}{0pt}%
\pgfpathmoveto{\pgfqpoint{4.825000in}{2.252842in}}%
\pgfpathlineto{\pgfqpoint{4.975000in}{2.252842in}}%
\pgfpathlineto{\pgfqpoint{4.975000in}{2.252842in}}%
\pgfpathlineto{\pgfqpoint{4.825000in}{2.252842in}}%
\pgfpathlineto{\pgfqpoint{4.825000in}{2.252842in}}%
\pgfpathclose%
\pgfusepath{stroke,fill}%
\end{pgfscope}%
\begin{pgfscope}%
\pgfpathrectangle{\pgfqpoint{1.000000in}{0.720000in}}{\pgfqpoint{4.800000in}{4.620000in}}%
\pgfusepath{clip}%
\pgfsetbuttcap%
\pgfsetmiterjoin%
\definecolor{currentfill}{rgb}{0.121569,0.466667,0.705882}%
\pgfsetfillcolor{currentfill}%
\pgfsetlinewidth{1.003750pt}%
\definecolor{currentstroke}{rgb}{0.000000,0.000000,0.000000}%
\pgfsetstrokecolor{currentstroke}%
\pgfsetdash{}{0pt}%
\pgfpathmoveto{\pgfqpoint{5.125000in}{2.849472in}}%
\pgfpathlineto{\pgfqpoint{5.275000in}{2.849472in}}%
\pgfpathlineto{\pgfqpoint{5.275000in}{2.849472in}}%
\pgfpathlineto{\pgfqpoint{5.125000in}{2.849472in}}%
\pgfpathlineto{\pgfqpoint{5.125000in}{2.849472in}}%
\pgfpathclose%
\pgfusepath{stroke,fill}%
\end{pgfscope}%
\begin{pgfscope}%
\pgfpathrectangle{\pgfqpoint{1.000000in}{0.720000in}}{\pgfqpoint{4.800000in}{4.620000in}}%
\pgfusepath{clip}%
\pgfsetbuttcap%
\pgfsetmiterjoin%
\definecolor{currentfill}{rgb}{0.121569,0.466667,0.705882}%
\pgfsetfillcolor{currentfill}%
\pgfsetlinewidth{1.003750pt}%
\definecolor{currentstroke}{rgb}{0.000000,0.000000,0.000000}%
\pgfsetstrokecolor{currentstroke}%
\pgfsetdash{}{0pt}%
\pgfpathmoveto{\pgfqpoint{5.425000in}{2.810787in}}%
\pgfpathlineto{\pgfqpoint{5.575000in}{2.810787in}}%
\pgfpathlineto{\pgfqpoint{5.575000in}{2.810787in}}%
\pgfpathlineto{\pgfqpoint{5.425000in}{2.810787in}}%
\pgfpathlineto{\pgfqpoint{5.425000in}{2.810787in}}%
\pgfpathclose%
\pgfusepath{stroke,fill}%
\end{pgfscope}%
\begin{pgfscope}%
\pgfpathrectangle{\pgfqpoint{1.000000in}{0.720000in}}{\pgfqpoint{4.800000in}{4.620000in}}%
\pgfusepath{clip}%
\pgfsetrectcap%
\pgfsetroundjoin%
\pgfsetlinewidth{1.003750pt}%
\definecolor{currentstroke}{rgb}{1.000000,0.498039,0.054902}%
\pgfsetstrokecolor{currentstroke}%
\pgfsetdash{}{0pt}%
\pgfpathmoveto{\pgfqpoint{1.225000in}{2.059647in}}%
\pgfpathlineto{\pgfqpoint{1.375000in}{2.059647in}}%
\pgfusepath{stroke}%
\end{pgfscope}%
\begin{pgfscope}%
\pgfpathrectangle{\pgfqpoint{1.000000in}{0.720000in}}{\pgfqpoint{4.800000in}{4.620000in}}%
\pgfusepath{clip}%
\pgfsetrectcap%
\pgfsetroundjoin%
\pgfsetlinewidth{1.003750pt}%
\definecolor{currentstroke}{rgb}{1.000000,0.498039,0.054902}%
\pgfsetstrokecolor{currentstroke}%
\pgfsetdash{}{0pt}%
\pgfpathmoveto{\pgfqpoint{1.525000in}{2.626181in}}%
\pgfpathlineto{\pgfqpoint{1.675000in}{2.626181in}}%
\pgfusepath{stroke}%
\end{pgfscope}%
\begin{pgfscope}%
\pgfpathrectangle{\pgfqpoint{1.000000in}{0.720000in}}{\pgfqpoint{4.800000in}{4.620000in}}%
\pgfusepath{clip}%
\pgfsetrectcap%
\pgfsetroundjoin%
\pgfsetlinewidth{1.003750pt}%
\definecolor{currentstroke}{rgb}{1.000000,0.498039,0.054902}%
\pgfsetstrokecolor{currentstroke}%
\pgfsetdash{}{0pt}%
\pgfpathmoveto{\pgfqpoint{1.825000in}{2.703584in}}%
\pgfpathlineto{\pgfqpoint{1.975000in}{2.703584in}}%
\pgfusepath{stroke}%
\end{pgfscope}%
\begin{pgfscope}%
\pgfpathrectangle{\pgfqpoint{1.000000in}{0.720000in}}{\pgfqpoint{4.800000in}{4.620000in}}%
\pgfusepath{clip}%
\pgfsetrectcap%
\pgfsetroundjoin%
\pgfsetlinewidth{1.003750pt}%
\definecolor{currentstroke}{rgb}{1.000000,0.498039,0.054902}%
\pgfsetstrokecolor{currentstroke}%
\pgfsetdash{}{0pt}%
\pgfpathmoveto{\pgfqpoint{2.125000in}{2.776231in}}%
\pgfpathlineto{\pgfqpoint{2.275000in}{2.776231in}}%
\pgfusepath{stroke}%
\end{pgfscope}%
\begin{pgfscope}%
\pgfpathrectangle{\pgfqpoint{1.000000in}{0.720000in}}{\pgfqpoint{4.800000in}{4.620000in}}%
\pgfusepath{clip}%
\pgfsetrectcap%
\pgfsetroundjoin%
\pgfsetlinewidth{1.003750pt}%
\definecolor{currentstroke}{rgb}{1.000000,0.498039,0.054902}%
\pgfsetstrokecolor{currentstroke}%
\pgfsetdash{}{0pt}%
\pgfpathmoveto{\pgfqpoint{2.425000in}{2.844559in}}%
\pgfpathlineto{\pgfqpoint{2.575000in}{2.844559in}}%
\pgfusepath{stroke}%
\end{pgfscope}%
\begin{pgfscope}%
\pgfpathrectangle{\pgfqpoint{1.000000in}{0.720000in}}{\pgfqpoint{4.800000in}{4.620000in}}%
\pgfusepath{clip}%
\pgfsetrectcap%
\pgfsetroundjoin%
\pgfsetlinewidth{1.003750pt}%
\definecolor{currentstroke}{rgb}{1.000000,0.498039,0.054902}%
\pgfsetstrokecolor{currentstroke}%
\pgfsetdash{}{0pt}%
\pgfpathmoveto{\pgfqpoint{2.725000in}{2.867053in}}%
\pgfpathlineto{\pgfqpoint{2.875000in}{2.867053in}}%
\pgfusepath{stroke}%
\end{pgfscope}%
\begin{pgfscope}%
\pgfpathrectangle{\pgfqpoint{1.000000in}{0.720000in}}{\pgfqpoint{4.800000in}{4.620000in}}%
\pgfusepath{clip}%
\pgfsetrectcap%
\pgfsetroundjoin%
\pgfsetlinewidth{1.003750pt}%
\definecolor{currentstroke}{rgb}{1.000000,0.498039,0.054902}%
\pgfsetstrokecolor{currentstroke}%
\pgfsetdash{}{0pt}%
\pgfpathmoveto{\pgfqpoint{3.025000in}{2.800935in}}%
\pgfpathlineto{\pgfqpoint{3.175000in}{2.800935in}}%
\pgfusepath{stroke}%
\end{pgfscope}%
\begin{pgfscope}%
\pgfpathrectangle{\pgfqpoint{1.000000in}{0.720000in}}{\pgfqpoint{4.800000in}{4.620000in}}%
\pgfusepath{clip}%
\pgfsetrectcap%
\pgfsetroundjoin%
\pgfsetlinewidth{1.003750pt}%
\definecolor{currentstroke}{rgb}{1.000000,0.498039,0.054902}%
\pgfsetstrokecolor{currentstroke}%
\pgfsetdash{}{0pt}%
\pgfpathmoveto{\pgfqpoint{3.325000in}{2.823390in}}%
\pgfpathlineto{\pgfqpoint{3.475000in}{2.823390in}}%
\pgfusepath{stroke}%
\end{pgfscope}%
\begin{pgfscope}%
\pgfpathrectangle{\pgfqpoint{1.000000in}{0.720000in}}{\pgfqpoint{4.800000in}{4.620000in}}%
\pgfusepath{clip}%
\pgfsetrectcap%
\pgfsetroundjoin%
\pgfsetlinewidth{1.003750pt}%
\definecolor{currentstroke}{rgb}{1.000000,0.498039,0.054902}%
\pgfsetstrokecolor{currentstroke}%
\pgfsetdash{}{0pt}%
\pgfpathmoveto{\pgfqpoint{3.625000in}{2.785253in}}%
\pgfpathlineto{\pgfqpoint{3.775000in}{2.785253in}}%
\pgfusepath{stroke}%
\end{pgfscope}%
\begin{pgfscope}%
\pgfpathrectangle{\pgfqpoint{1.000000in}{0.720000in}}{\pgfqpoint{4.800000in}{4.620000in}}%
\pgfusepath{clip}%
\pgfsetrectcap%
\pgfsetroundjoin%
\pgfsetlinewidth{1.003750pt}%
\definecolor{currentstroke}{rgb}{1.000000,0.498039,0.054902}%
\pgfsetstrokecolor{currentstroke}%
\pgfsetdash{}{0pt}%
\pgfpathmoveto{\pgfqpoint{3.925000in}{2.797547in}}%
\pgfpathlineto{\pgfqpoint{4.075000in}{2.797547in}}%
\pgfusepath{stroke}%
\end{pgfscope}%
\begin{pgfscope}%
\pgfpathrectangle{\pgfqpoint{1.000000in}{0.720000in}}{\pgfqpoint{4.800000in}{4.620000in}}%
\pgfusepath{clip}%
\pgfsetrectcap%
\pgfsetroundjoin%
\pgfsetlinewidth{1.003750pt}%
\definecolor{currentstroke}{rgb}{1.000000,0.498039,0.054902}%
\pgfsetstrokecolor{currentstroke}%
\pgfsetdash{}{0pt}%
\pgfpathmoveto{\pgfqpoint{4.225000in}{2.763553in}}%
\pgfpathlineto{\pgfqpoint{4.375000in}{2.763553in}}%
\pgfusepath{stroke}%
\end{pgfscope}%
\begin{pgfscope}%
\pgfpathrectangle{\pgfqpoint{1.000000in}{0.720000in}}{\pgfqpoint{4.800000in}{4.620000in}}%
\pgfusepath{clip}%
\pgfsetrectcap%
\pgfsetroundjoin%
\pgfsetlinewidth{1.003750pt}%
\definecolor{currentstroke}{rgb}{1.000000,0.498039,0.054902}%
\pgfsetstrokecolor{currentstroke}%
\pgfsetdash{}{0pt}%
\pgfpathmoveto{\pgfqpoint{4.525000in}{3.309428in}}%
\pgfpathlineto{\pgfqpoint{4.675000in}{3.309428in}}%
\pgfusepath{stroke}%
\end{pgfscope}%
\begin{pgfscope}%
\pgfpathrectangle{\pgfqpoint{1.000000in}{0.720000in}}{\pgfqpoint{4.800000in}{4.620000in}}%
\pgfusepath{clip}%
\pgfsetrectcap%
\pgfsetroundjoin%
\pgfsetlinewidth{1.003750pt}%
\definecolor{currentstroke}{rgb}{1.000000,0.498039,0.054902}%
\pgfsetstrokecolor{currentstroke}%
\pgfsetdash{}{0pt}%
\pgfpathmoveto{\pgfqpoint{4.825000in}{2.252842in}}%
\pgfpathlineto{\pgfqpoint{4.975000in}{2.252842in}}%
\pgfusepath{stroke}%
\end{pgfscope}%
\begin{pgfscope}%
\pgfpathrectangle{\pgfqpoint{1.000000in}{0.720000in}}{\pgfqpoint{4.800000in}{4.620000in}}%
\pgfusepath{clip}%
\pgfsetrectcap%
\pgfsetroundjoin%
\pgfsetlinewidth{1.003750pt}%
\definecolor{currentstroke}{rgb}{1.000000,0.498039,0.054902}%
\pgfsetstrokecolor{currentstroke}%
\pgfsetdash{}{0pt}%
\pgfpathmoveto{\pgfqpoint{5.125000in}{2.849472in}}%
\pgfpathlineto{\pgfqpoint{5.275000in}{2.849472in}}%
\pgfusepath{stroke}%
\end{pgfscope}%
\begin{pgfscope}%
\pgfpathrectangle{\pgfqpoint{1.000000in}{0.720000in}}{\pgfqpoint{4.800000in}{4.620000in}}%
\pgfusepath{clip}%
\pgfsetrectcap%
\pgfsetroundjoin%
\pgfsetlinewidth{1.003750pt}%
\definecolor{currentstroke}{rgb}{1.000000,0.498039,0.054902}%
\pgfsetstrokecolor{currentstroke}%
\pgfsetdash{}{0pt}%
\pgfpathmoveto{\pgfqpoint{5.425000in}{2.810787in}}%
\pgfpathlineto{\pgfqpoint{5.575000in}{2.810787in}}%
\pgfusepath{stroke}%
\end{pgfscope}%
\begin{pgfscope}%
\pgfsetrectcap%
\pgfsetmiterjoin%
\pgfsetlinewidth{0.803000pt}%
\definecolor{currentstroke}{rgb}{0.000000,0.000000,0.000000}%
\pgfsetstrokecolor{currentstroke}%
\pgfsetdash{}{0pt}%
\pgfpathmoveto{\pgfqpoint{1.000000in}{0.720000in}}%
\pgfpathlineto{\pgfqpoint{1.000000in}{5.340000in}}%
\pgfusepath{stroke}%
\end{pgfscope}%
\begin{pgfscope}%
\pgfsetrectcap%
\pgfsetmiterjoin%
\pgfsetlinewidth{0.803000pt}%
\definecolor{currentstroke}{rgb}{0.000000,0.000000,0.000000}%
\pgfsetstrokecolor{currentstroke}%
\pgfsetdash{}{0pt}%
\pgfpathmoveto{\pgfqpoint{5.800000in}{0.720000in}}%
\pgfpathlineto{\pgfqpoint{5.800000in}{5.340000in}}%
\pgfusepath{stroke}%
\end{pgfscope}%
\begin{pgfscope}%
\pgfsetrectcap%
\pgfsetmiterjoin%
\pgfsetlinewidth{0.803000pt}%
\definecolor{currentstroke}{rgb}{0.000000,0.000000,0.000000}%
\pgfsetstrokecolor{currentstroke}%
\pgfsetdash{}{0pt}%
\pgfpathmoveto{\pgfqpoint{1.000000in}{0.720000in}}%
\pgfpathlineto{\pgfqpoint{5.800000in}{0.720000in}}%
\pgfusepath{stroke}%
\end{pgfscope}%
\begin{pgfscope}%
\pgfsetrectcap%
\pgfsetmiterjoin%
\pgfsetlinewidth{0.803000pt}%
\definecolor{currentstroke}{rgb}{0.000000,0.000000,0.000000}%
\pgfsetstrokecolor{currentstroke}%
\pgfsetdash{}{0pt}%
\pgfpathmoveto{\pgfqpoint{1.000000in}{5.340000in}}%
\pgfpathlineto{\pgfqpoint{5.800000in}{5.340000in}}%
\pgfusepath{stroke}%
\end{pgfscope}%
\begin{pgfscope}%
\pgfsetbuttcap%
\pgfsetmiterjoin%
\definecolor{currentfill}{rgb}{1.000000,1.000000,1.000000}%
\pgfsetfillcolor{currentfill}%
\pgfsetfillopacity{0.800000}%
\pgfsetlinewidth{1.003750pt}%
\definecolor{currentstroke}{rgb}{0.800000,0.800000,0.800000}%
\pgfsetstrokecolor{currentstroke}%
\pgfsetstrokeopacity{0.800000}%
\pgfsetdash{}{0pt}%
\pgfpathmoveto{\pgfqpoint{4.330170in}{4.722821in}}%
\pgfpathlineto{\pgfqpoint{5.605556in}{4.722821in}}%
\pgfpathquadraticcurveto{\pgfqpoint{5.661111in}{4.722821in}}{\pgfqpoint{5.661111in}{4.778377in}}%
\pgfpathlineto{\pgfqpoint{5.661111in}{5.145556in}}%
\pgfpathquadraticcurveto{\pgfqpoint{5.661111in}{5.201111in}}{\pgfqpoint{5.605556in}{5.201111in}}%
\pgfpathlineto{\pgfqpoint{4.330170in}{5.201111in}}%
\pgfpathquadraticcurveto{\pgfqpoint{4.274615in}{5.201111in}}{\pgfqpoint{4.274615in}{5.145556in}}%
\pgfpathlineto{\pgfqpoint{4.274615in}{4.778377in}}%
\pgfpathquadraticcurveto{\pgfqpoint{4.274615in}{4.722821in}}{\pgfqpoint{4.330170in}{4.722821in}}%
\pgfpathclose%
\pgfusepath{stroke,fill}%
\end{pgfscope}%
\begin{pgfscope}%
\pgfsetrectcap%
\pgfsetroundjoin%
\pgfsetlinewidth{2.007500pt}%
\definecolor{currentstroke}{rgb}{0.121569,0.466667,0.705882}%
\pgfsetstrokecolor{currentstroke}%
\pgfsetdash{}{0pt}%
\pgfpathmoveto{\pgfqpoint{4.385726in}{4.987184in}}%
\pgfpathlineto{\pgfqpoint{4.941281in}{4.987184in}}%
\pgfusepath{stroke}%
\end{pgfscope}%
\begin{pgfscope}%
\definecolor{textcolor}{rgb}{0.000000,0.000000,0.000000}%
\pgfsetstrokecolor{textcolor}%
\pgfsetfillcolor{textcolor}%
\pgftext[x=5.163504in,y=4.889962in,left,base]{\color{textcolor}\sffamily\fontsize{20.000000}{24.000000}\selectfont \(\displaystyle D_w\)}%
\end{pgfscope}%
\begin{pgfscope}%
\pgfsetbuttcap%
\pgfsetmiterjoin%
\definecolor{currentfill}{rgb}{1.000000,1.000000,1.000000}%
\pgfsetfillcolor{currentfill}%
\pgfsetlinewidth{0.000000pt}%
\definecolor{currentstroke}{rgb}{0.000000,0.000000,0.000000}%
\pgfsetstrokecolor{currentstroke}%
\pgfsetstrokeopacity{0.000000}%
\pgfsetdash{}{0pt}%
\pgfpathmoveto{\pgfqpoint{5.800000in}{0.720000in}}%
\pgfpathlineto{\pgfqpoint{7.200000in}{0.720000in}}%
\pgfpathlineto{\pgfqpoint{7.200000in}{5.340000in}}%
\pgfpathlineto{\pgfqpoint{5.800000in}{5.340000in}}%
\pgfpathclose%
\pgfusepath{fill}%
\end{pgfscope}%
\begin{pgfscope}%
\pgfpathrectangle{\pgfqpoint{5.800000in}{0.720000in}}{\pgfqpoint{1.400000in}{4.620000in}}%
\pgfusepath{clip}%
\pgfsetbuttcap%
\pgfsetmiterjoin%
\definecolor{currentfill}{rgb}{0.121569,0.466667,0.705882}%
\pgfsetfillcolor{currentfill}%
\pgfsetlinewidth{0.000000pt}%
\definecolor{currentstroke}{rgb}{0.000000,0.000000,0.000000}%
\pgfsetstrokecolor{currentstroke}%
\pgfsetstrokeopacity{0.000000}%
\pgfsetdash{}{0pt}%
\pgfpathmoveto{\pgfqpoint{5.800000in}{0.720000in}}%
\pgfpathlineto{\pgfqpoint{5.871339in}{0.720000in}}%
\pgfpathlineto{\pgfqpoint{5.871339in}{0.822837in}}%
\pgfpathlineto{\pgfqpoint{5.800000in}{0.822837in}}%
\pgfpathclose%
\pgfusepath{fill}%
\end{pgfscope}%
\begin{pgfscope}%
\pgfpathrectangle{\pgfqpoint{5.800000in}{0.720000in}}{\pgfqpoint{1.400000in}{4.620000in}}%
\pgfusepath{clip}%
\pgfsetbuttcap%
\pgfsetmiterjoin%
\definecolor{currentfill}{rgb}{0.121569,0.466667,0.705882}%
\pgfsetfillcolor{currentfill}%
\pgfsetlinewidth{0.000000pt}%
\definecolor{currentstroke}{rgb}{0.000000,0.000000,0.000000}%
\pgfsetstrokecolor{currentstroke}%
\pgfsetstrokeopacity{0.000000}%
\pgfsetdash{}{0pt}%
\pgfpathmoveto{\pgfqpoint{5.800000in}{0.822837in}}%
\pgfpathlineto{\pgfqpoint{5.855033in}{0.822837in}}%
\pgfpathlineto{\pgfqpoint{5.855033in}{0.925674in}}%
\pgfpathlineto{\pgfqpoint{5.800000in}{0.925674in}}%
\pgfpathclose%
\pgfusepath{fill}%
\end{pgfscope}%
\begin{pgfscope}%
\pgfpathrectangle{\pgfqpoint{5.800000in}{0.720000in}}{\pgfqpoint{1.400000in}{4.620000in}}%
\pgfusepath{clip}%
\pgfsetbuttcap%
\pgfsetmiterjoin%
\definecolor{currentfill}{rgb}{0.121569,0.466667,0.705882}%
\pgfsetfillcolor{currentfill}%
\pgfsetlinewidth{0.000000pt}%
\definecolor{currentstroke}{rgb}{0.000000,0.000000,0.000000}%
\pgfsetstrokecolor{currentstroke}%
\pgfsetstrokeopacity{0.000000}%
\pgfsetdash{}{0pt}%
\pgfpathmoveto{\pgfqpoint{5.800000in}{0.925674in}}%
\pgfpathlineto{\pgfqpoint{5.871339in}{0.925674in}}%
\pgfpathlineto{\pgfqpoint{5.871339in}{1.028511in}}%
\pgfpathlineto{\pgfqpoint{5.800000in}{1.028511in}}%
\pgfpathclose%
\pgfusepath{fill}%
\end{pgfscope}%
\begin{pgfscope}%
\pgfpathrectangle{\pgfqpoint{5.800000in}{0.720000in}}{\pgfqpoint{1.400000in}{4.620000in}}%
\pgfusepath{clip}%
\pgfsetbuttcap%
\pgfsetmiterjoin%
\definecolor{currentfill}{rgb}{0.121569,0.466667,0.705882}%
\pgfsetfillcolor{currentfill}%
\pgfsetlinewidth{0.000000pt}%
\definecolor{currentstroke}{rgb}{0.000000,0.000000,0.000000}%
\pgfsetstrokecolor{currentstroke}%
\pgfsetstrokeopacity{0.000000}%
\pgfsetdash{}{0pt}%
\pgfpathmoveto{\pgfqpoint{5.800000in}{1.028511in}}%
\pgfpathlineto{\pgfqpoint{5.875416in}{1.028511in}}%
\pgfpathlineto{\pgfqpoint{5.875416in}{1.131348in}}%
\pgfpathlineto{\pgfqpoint{5.800000in}{1.131348in}}%
\pgfpathclose%
\pgfusepath{fill}%
\end{pgfscope}%
\begin{pgfscope}%
\pgfpathrectangle{\pgfqpoint{5.800000in}{0.720000in}}{\pgfqpoint{1.400000in}{4.620000in}}%
\pgfusepath{clip}%
\pgfsetbuttcap%
\pgfsetmiterjoin%
\definecolor{currentfill}{rgb}{0.121569,0.466667,0.705882}%
\pgfsetfillcolor{currentfill}%
\pgfsetlinewidth{0.000000pt}%
\definecolor{currentstroke}{rgb}{0.000000,0.000000,0.000000}%
\pgfsetstrokecolor{currentstroke}%
\pgfsetstrokeopacity{0.000000}%
\pgfsetdash{}{0pt}%
\pgfpathmoveto{\pgfqpoint{5.800000in}{1.131348in}}%
\pgfpathlineto{\pgfqpoint{5.903952in}{1.131348in}}%
\pgfpathlineto{\pgfqpoint{5.903952in}{1.234185in}}%
\pgfpathlineto{\pgfqpoint{5.800000in}{1.234185in}}%
\pgfpathclose%
\pgfusepath{fill}%
\end{pgfscope}%
\begin{pgfscope}%
\pgfpathrectangle{\pgfqpoint{5.800000in}{0.720000in}}{\pgfqpoint{1.400000in}{4.620000in}}%
\pgfusepath{clip}%
\pgfsetbuttcap%
\pgfsetmiterjoin%
\definecolor{currentfill}{rgb}{0.121569,0.466667,0.705882}%
\pgfsetfillcolor{currentfill}%
\pgfsetlinewidth{0.000000pt}%
\definecolor{currentstroke}{rgb}{0.000000,0.000000,0.000000}%
\pgfsetstrokecolor{currentstroke}%
\pgfsetstrokeopacity{0.000000}%
\pgfsetdash{}{0pt}%
\pgfpathmoveto{\pgfqpoint{5.800000in}{1.234185in}}%
\pgfpathlineto{\pgfqpoint{5.920258in}{1.234185in}}%
\pgfpathlineto{\pgfqpoint{5.920258in}{1.337022in}}%
\pgfpathlineto{\pgfqpoint{5.800000in}{1.337022in}}%
\pgfpathclose%
\pgfusepath{fill}%
\end{pgfscope}%
\begin{pgfscope}%
\pgfpathrectangle{\pgfqpoint{5.800000in}{0.720000in}}{\pgfqpoint{1.400000in}{4.620000in}}%
\pgfusepath{clip}%
\pgfsetbuttcap%
\pgfsetmiterjoin%
\definecolor{currentfill}{rgb}{0.121569,0.466667,0.705882}%
\pgfsetfillcolor{currentfill}%
\pgfsetlinewidth{0.000000pt}%
\definecolor{currentstroke}{rgb}{0.000000,0.000000,0.000000}%
\pgfsetstrokecolor{currentstroke}%
\pgfsetstrokeopacity{0.000000}%
\pgfsetdash{}{0pt}%
\pgfpathmoveto{\pgfqpoint{5.800000in}{1.337022in}}%
\pgfpathlineto{\pgfqpoint{5.942679in}{1.337022in}}%
\pgfpathlineto{\pgfqpoint{5.942679in}{1.439859in}}%
\pgfpathlineto{\pgfqpoint{5.800000in}{1.439859in}}%
\pgfpathclose%
\pgfusepath{fill}%
\end{pgfscope}%
\begin{pgfscope}%
\pgfpathrectangle{\pgfqpoint{5.800000in}{0.720000in}}{\pgfqpoint{1.400000in}{4.620000in}}%
\pgfusepath{clip}%
\pgfsetbuttcap%
\pgfsetmiterjoin%
\definecolor{currentfill}{rgb}{0.121569,0.466667,0.705882}%
\pgfsetfillcolor{currentfill}%
\pgfsetlinewidth{0.000000pt}%
\definecolor{currentstroke}{rgb}{0.000000,0.000000,0.000000}%
\pgfsetstrokecolor{currentstroke}%
\pgfsetstrokeopacity{0.000000}%
\pgfsetdash{}{0pt}%
\pgfpathmoveto{\pgfqpoint{5.800000in}{1.439859in}}%
\pgfpathlineto{\pgfqpoint{5.975291in}{1.439859in}}%
\pgfpathlineto{\pgfqpoint{5.975291in}{1.542697in}}%
\pgfpathlineto{\pgfqpoint{5.800000in}{1.542697in}}%
\pgfpathclose%
\pgfusepath{fill}%
\end{pgfscope}%
\begin{pgfscope}%
\pgfpathrectangle{\pgfqpoint{5.800000in}{0.720000in}}{\pgfqpoint{1.400000in}{4.620000in}}%
\pgfusepath{clip}%
\pgfsetbuttcap%
\pgfsetmiterjoin%
\definecolor{currentfill}{rgb}{0.121569,0.466667,0.705882}%
\pgfsetfillcolor{currentfill}%
\pgfsetlinewidth{0.000000pt}%
\definecolor{currentstroke}{rgb}{0.000000,0.000000,0.000000}%
\pgfsetstrokecolor{currentstroke}%
\pgfsetstrokeopacity{0.000000}%
\pgfsetdash{}{0pt}%
\pgfpathmoveto{\pgfqpoint{5.800000in}{1.542697in}}%
\pgfpathlineto{\pgfqpoint{6.022171in}{1.542697in}}%
\pgfpathlineto{\pgfqpoint{6.022171in}{1.645534in}}%
\pgfpathlineto{\pgfqpoint{5.800000in}{1.645534in}}%
\pgfpathclose%
\pgfusepath{fill}%
\end{pgfscope}%
\begin{pgfscope}%
\pgfpathrectangle{\pgfqpoint{5.800000in}{0.720000in}}{\pgfqpoint{1.400000in}{4.620000in}}%
\pgfusepath{clip}%
\pgfsetbuttcap%
\pgfsetmiterjoin%
\definecolor{currentfill}{rgb}{0.121569,0.466667,0.705882}%
\pgfsetfillcolor{currentfill}%
\pgfsetlinewidth{0.000000pt}%
\definecolor{currentstroke}{rgb}{0.000000,0.000000,0.000000}%
\pgfsetstrokecolor{currentstroke}%
\pgfsetstrokeopacity{0.000000}%
\pgfsetdash{}{0pt}%
\pgfpathmoveto{\pgfqpoint{5.800000in}{1.645534in}}%
\pgfpathlineto{\pgfqpoint{6.191348in}{1.645534in}}%
\pgfpathlineto{\pgfqpoint{6.191348in}{1.748371in}}%
\pgfpathlineto{\pgfqpoint{5.800000in}{1.748371in}}%
\pgfpathclose%
\pgfusepath{fill}%
\end{pgfscope}%
\begin{pgfscope}%
\pgfpathrectangle{\pgfqpoint{5.800000in}{0.720000in}}{\pgfqpoint{1.400000in}{4.620000in}}%
\pgfusepath{clip}%
\pgfsetbuttcap%
\pgfsetmiterjoin%
\definecolor{currentfill}{rgb}{0.121569,0.466667,0.705882}%
\pgfsetfillcolor{currentfill}%
\pgfsetlinewidth{0.000000pt}%
\definecolor{currentstroke}{rgb}{0.000000,0.000000,0.000000}%
\pgfsetstrokecolor{currentstroke}%
\pgfsetstrokeopacity{0.000000}%
\pgfsetdash{}{0pt}%
\pgfpathmoveto{\pgfqpoint{5.800000in}{1.748371in}}%
\pgfpathlineto{\pgfqpoint{6.270840in}{1.748371in}}%
\pgfpathlineto{\pgfqpoint{6.270840in}{1.851208in}}%
\pgfpathlineto{\pgfqpoint{5.800000in}{1.851208in}}%
\pgfpathclose%
\pgfusepath{fill}%
\end{pgfscope}%
\begin{pgfscope}%
\pgfpathrectangle{\pgfqpoint{5.800000in}{0.720000in}}{\pgfqpoint{1.400000in}{4.620000in}}%
\pgfusepath{clip}%
\pgfsetbuttcap%
\pgfsetmiterjoin%
\definecolor{currentfill}{rgb}{0.121569,0.466667,0.705882}%
\pgfsetfillcolor{currentfill}%
\pgfsetlinewidth{0.000000pt}%
\definecolor{currentstroke}{rgb}{0.000000,0.000000,0.000000}%
\pgfsetstrokecolor{currentstroke}%
\pgfsetstrokeopacity{0.000000}%
\pgfsetdash{}{0pt}%
\pgfpathmoveto{\pgfqpoint{5.800000in}{1.851208in}}%
\pgfpathlineto{\pgfqpoint{6.378868in}{1.851208in}}%
\pgfpathlineto{\pgfqpoint{6.378868in}{1.954045in}}%
\pgfpathlineto{\pgfqpoint{5.800000in}{1.954045in}}%
\pgfpathclose%
\pgfusepath{fill}%
\end{pgfscope}%
\begin{pgfscope}%
\pgfpathrectangle{\pgfqpoint{5.800000in}{0.720000in}}{\pgfqpoint{1.400000in}{4.620000in}}%
\pgfusepath{clip}%
\pgfsetbuttcap%
\pgfsetmiterjoin%
\definecolor{currentfill}{rgb}{0.121569,0.466667,0.705882}%
\pgfsetfillcolor{currentfill}%
\pgfsetlinewidth{0.000000pt}%
\definecolor{currentstroke}{rgb}{0.000000,0.000000,0.000000}%
\pgfsetstrokecolor{currentstroke}%
\pgfsetstrokeopacity{0.000000}%
\pgfsetdash{}{0pt}%
\pgfpathmoveto{\pgfqpoint{5.800000in}{1.954045in}}%
\pgfpathlineto{\pgfqpoint{6.495050in}{1.954045in}}%
\pgfpathlineto{\pgfqpoint{6.495050in}{2.056882in}}%
\pgfpathlineto{\pgfqpoint{5.800000in}{2.056882in}}%
\pgfpathclose%
\pgfusepath{fill}%
\end{pgfscope}%
\begin{pgfscope}%
\pgfpathrectangle{\pgfqpoint{5.800000in}{0.720000in}}{\pgfqpoint{1.400000in}{4.620000in}}%
\pgfusepath{clip}%
\pgfsetbuttcap%
\pgfsetmiterjoin%
\definecolor{currentfill}{rgb}{0.121569,0.466667,0.705882}%
\pgfsetfillcolor{currentfill}%
\pgfsetlinewidth{0.000000pt}%
\definecolor{currentstroke}{rgb}{0.000000,0.000000,0.000000}%
\pgfsetstrokecolor{currentstroke}%
\pgfsetstrokeopacity{0.000000}%
\pgfsetdash{}{0pt}%
\pgfpathmoveto{\pgfqpoint{5.800000in}{2.056882in}}%
\pgfpathlineto{\pgfqpoint{6.704991in}{2.056882in}}%
\pgfpathlineto{\pgfqpoint{6.704991in}{2.159719in}}%
\pgfpathlineto{\pgfqpoint{5.800000in}{2.159719in}}%
\pgfpathclose%
\pgfusepath{fill}%
\end{pgfscope}%
\begin{pgfscope}%
\pgfpathrectangle{\pgfqpoint{5.800000in}{0.720000in}}{\pgfqpoint{1.400000in}{4.620000in}}%
\pgfusepath{clip}%
\pgfsetbuttcap%
\pgfsetmiterjoin%
\definecolor{currentfill}{rgb}{0.121569,0.466667,0.705882}%
\pgfsetfillcolor{currentfill}%
\pgfsetlinewidth{0.000000pt}%
\definecolor{currentstroke}{rgb}{0.000000,0.000000,0.000000}%
\pgfsetstrokecolor{currentstroke}%
\pgfsetstrokeopacity{0.000000}%
\pgfsetdash{}{0pt}%
\pgfpathmoveto{\pgfqpoint{5.800000in}{2.159719in}}%
\pgfpathlineto{\pgfqpoint{6.739642in}{2.159719in}}%
\pgfpathlineto{\pgfqpoint{6.739642in}{2.262556in}}%
\pgfpathlineto{\pgfqpoint{5.800000in}{2.262556in}}%
\pgfpathclose%
\pgfusepath{fill}%
\end{pgfscope}%
\begin{pgfscope}%
\pgfpathrectangle{\pgfqpoint{5.800000in}{0.720000in}}{\pgfqpoint{1.400000in}{4.620000in}}%
\pgfusepath{clip}%
\pgfsetbuttcap%
\pgfsetmiterjoin%
\definecolor{currentfill}{rgb}{0.121569,0.466667,0.705882}%
\pgfsetfillcolor{currentfill}%
\pgfsetlinewidth{0.000000pt}%
\definecolor{currentstroke}{rgb}{0.000000,0.000000,0.000000}%
\pgfsetstrokecolor{currentstroke}%
\pgfsetstrokeopacity{0.000000}%
\pgfsetdash{}{0pt}%
\pgfpathmoveto{\pgfqpoint{5.800000in}{2.262556in}}%
\pgfpathlineto{\pgfqpoint{6.735565in}{2.262556in}}%
\pgfpathlineto{\pgfqpoint{6.735565in}{2.365393in}}%
\pgfpathlineto{\pgfqpoint{5.800000in}{2.365393in}}%
\pgfpathclose%
\pgfusepath{fill}%
\end{pgfscope}%
\begin{pgfscope}%
\pgfpathrectangle{\pgfqpoint{5.800000in}{0.720000in}}{\pgfqpoint{1.400000in}{4.620000in}}%
\pgfusepath{clip}%
\pgfsetbuttcap%
\pgfsetmiterjoin%
\definecolor{currentfill}{rgb}{0.121569,0.466667,0.705882}%
\pgfsetfillcolor{currentfill}%
\pgfsetlinewidth{0.000000pt}%
\definecolor{currentstroke}{rgb}{0.000000,0.000000,0.000000}%
\pgfsetstrokecolor{currentstroke}%
\pgfsetstrokeopacity{0.000000}%
\pgfsetdash{}{0pt}%
\pgfpathmoveto{\pgfqpoint{5.800000in}{2.365393in}}%
\pgfpathlineto{\pgfqpoint{6.847670in}{2.365393in}}%
\pgfpathlineto{\pgfqpoint{6.847670in}{2.468230in}}%
\pgfpathlineto{\pgfqpoint{5.800000in}{2.468230in}}%
\pgfpathclose%
\pgfusepath{fill}%
\end{pgfscope}%
\begin{pgfscope}%
\pgfpathrectangle{\pgfqpoint{5.800000in}{0.720000in}}{\pgfqpoint{1.400000in}{4.620000in}}%
\pgfusepath{clip}%
\pgfsetbuttcap%
\pgfsetmiterjoin%
\definecolor{currentfill}{rgb}{0.121569,0.466667,0.705882}%
\pgfsetfillcolor{currentfill}%
\pgfsetlinewidth{0.000000pt}%
\definecolor{currentstroke}{rgb}{0.000000,0.000000,0.000000}%
\pgfsetstrokecolor{currentstroke}%
\pgfsetstrokeopacity{0.000000}%
\pgfsetdash{}{0pt}%
\pgfpathmoveto{\pgfqpoint{5.800000in}{2.468230in}}%
\pgfpathlineto{\pgfqpoint{6.957737in}{2.468230in}}%
\pgfpathlineto{\pgfqpoint{6.957737in}{2.571067in}}%
\pgfpathlineto{\pgfqpoint{5.800000in}{2.571067in}}%
\pgfpathclose%
\pgfusepath{fill}%
\end{pgfscope}%
\begin{pgfscope}%
\pgfpathrectangle{\pgfqpoint{5.800000in}{0.720000in}}{\pgfqpoint{1.400000in}{4.620000in}}%
\pgfusepath{clip}%
\pgfsetbuttcap%
\pgfsetmiterjoin%
\definecolor{currentfill}{rgb}{0.121569,0.466667,0.705882}%
\pgfsetfillcolor{currentfill}%
\pgfsetlinewidth{0.000000pt}%
\definecolor{currentstroke}{rgb}{0.000000,0.000000,0.000000}%
\pgfsetstrokecolor{currentstroke}%
\pgfsetstrokeopacity{0.000000}%
\pgfsetdash{}{0pt}%
\pgfpathmoveto{\pgfqpoint{5.800000in}{2.571067in}}%
\pgfpathlineto{\pgfqpoint{7.069841in}{2.571067in}}%
\pgfpathlineto{\pgfqpoint{7.069841in}{2.673904in}}%
\pgfpathlineto{\pgfqpoint{5.800000in}{2.673904in}}%
\pgfpathclose%
\pgfusepath{fill}%
\end{pgfscope}%
\begin{pgfscope}%
\pgfpathrectangle{\pgfqpoint{5.800000in}{0.720000in}}{\pgfqpoint{1.400000in}{4.620000in}}%
\pgfusepath{clip}%
\pgfsetbuttcap%
\pgfsetmiterjoin%
\definecolor{currentfill}{rgb}{0.121569,0.466667,0.705882}%
\pgfsetfillcolor{currentfill}%
\pgfsetlinewidth{0.000000pt}%
\definecolor{currentstroke}{rgb}{0.000000,0.000000,0.000000}%
\pgfsetstrokecolor{currentstroke}%
\pgfsetstrokeopacity{0.000000}%
\pgfsetdash{}{0pt}%
\pgfpathmoveto{\pgfqpoint{5.800000in}{2.673904in}}%
\pgfpathlineto{\pgfqpoint{6.902703in}{2.673904in}}%
\pgfpathlineto{\pgfqpoint{6.902703in}{2.776741in}}%
\pgfpathlineto{\pgfqpoint{5.800000in}{2.776741in}}%
\pgfpathclose%
\pgfusepath{fill}%
\end{pgfscope}%
\begin{pgfscope}%
\pgfpathrectangle{\pgfqpoint{5.800000in}{0.720000in}}{\pgfqpoint{1.400000in}{4.620000in}}%
\pgfusepath{clip}%
\pgfsetbuttcap%
\pgfsetmiterjoin%
\definecolor{currentfill}{rgb}{0.121569,0.466667,0.705882}%
\pgfsetfillcolor{currentfill}%
\pgfsetlinewidth{0.000000pt}%
\definecolor{currentstroke}{rgb}{0.000000,0.000000,0.000000}%
\pgfsetstrokecolor{currentstroke}%
\pgfsetstrokeopacity{0.000000}%
\pgfsetdash{}{0pt}%
\pgfpathmoveto{\pgfqpoint{5.800000in}{2.776741in}}%
\pgfpathlineto{\pgfqpoint{6.900665in}{2.776741in}}%
\pgfpathlineto{\pgfqpoint{6.900665in}{2.879578in}}%
\pgfpathlineto{\pgfqpoint{5.800000in}{2.879578in}}%
\pgfpathclose%
\pgfusepath{fill}%
\end{pgfscope}%
\begin{pgfscope}%
\pgfpathrectangle{\pgfqpoint{5.800000in}{0.720000in}}{\pgfqpoint{1.400000in}{4.620000in}}%
\pgfusepath{clip}%
\pgfsetbuttcap%
\pgfsetmiterjoin%
\definecolor{currentfill}{rgb}{0.121569,0.466667,0.705882}%
\pgfsetfillcolor{currentfill}%
\pgfsetlinewidth{0.000000pt}%
\definecolor{currentstroke}{rgb}{0.000000,0.000000,0.000000}%
\pgfsetstrokecolor{currentstroke}%
\pgfsetstrokeopacity{0.000000}%
\pgfsetdash{}{0pt}%
\pgfpathmoveto{\pgfqpoint{5.800000in}{2.879578in}}%
\pgfpathlineto{\pgfqpoint{6.935316in}{2.879578in}}%
\pgfpathlineto{\pgfqpoint{6.935316in}{2.982415in}}%
\pgfpathlineto{\pgfqpoint{5.800000in}{2.982415in}}%
\pgfpathclose%
\pgfusepath{fill}%
\end{pgfscope}%
\begin{pgfscope}%
\pgfpathrectangle{\pgfqpoint{5.800000in}{0.720000in}}{\pgfqpoint{1.400000in}{4.620000in}}%
\pgfusepath{clip}%
\pgfsetbuttcap%
\pgfsetmiterjoin%
\definecolor{currentfill}{rgb}{0.121569,0.466667,0.705882}%
\pgfsetfillcolor{currentfill}%
\pgfsetlinewidth{0.000000pt}%
\definecolor{currentstroke}{rgb}{0.000000,0.000000,0.000000}%
\pgfsetstrokecolor{currentstroke}%
\pgfsetstrokeopacity{0.000000}%
\pgfsetdash{}{0pt}%
\pgfpathmoveto{\pgfqpoint{5.800000in}{2.982415in}}%
\pgfpathlineto{\pgfqpoint{6.798752in}{2.982415in}}%
\pgfpathlineto{\pgfqpoint{6.798752in}{3.085253in}}%
\pgfpathlineto{\pgfqpoint{5.800000in}{3.085253in}}%
\pgfpathclose%
\pgfusepath{fill}%
\end{pgfscope}%
\begin{pgfscope}%
\pgfpathrectangle{\pgfqpoint{5.800000in}{0.720000in}}{\pgfqpoint{1.400000in}{4.620000in}}%
\pgfusepath{clip}%
\pgfsetbuttcap%
\pgfsetmiterjoin%
\definecolor{currentfill}{rgb}{0.121569,0.466667,0.705882}%
\pgfsetfillcolor{currentfill}%
\pgfsetlinewidth{0.000000pt}%
\definecolor{currentstroke}{rgb}{0.000000,0.000000,0.000000}%
\pgfsetstrokecolor{currentstroke}%
\pgfsetstrokeopacity{0.000000}%
\pgfsetdash{}{0pt}%
\pgfpathmoveto{\pgfqpoint{5.800000in}{3.085253in}}%
\pgfpathlineto{\pgfqpoint{6.696838in}{3.085253in}}%
\pgfpathlineto{\pgfqpoint{6.696838in}{3.188090in}}%
\pgfpathlineto{\pgfqpoint{5.800000in}{3.188090in}}%
\pgfpathclose%
\pgfusepath{fill}%
\end{pgfscope}%
\begin{pgfscope}%
\pgfpathrectangle{\pgfqpoint{5.800000in}{0.720000in}}{\pgfqpoint{1.400000in}{4.620000in}}%
\pgfusepath{clip}%
\pgfsetbuttcap%
\pgfsetmiterjoin%
\definecolor{currentfill}{rgb}{0.121569,0.466667,0.705882}%
\pgfsetfillcolor{currentfill}%
\pgfsetlinewidth{0.000000pt}%
\definecolor{currentstroke}{rgb}{0.000000,0.000000,0.000000}%
\pgfsetstrokecolor{currentstroke}%
\pgfsetstrokeopacity{0.000000}%
\pgfsetdash{}{0pt}%
\pgfpathmoveto{\pgfqpoint{5.800000in}{3.188090in}}%
\pgfpathlineto{\pgfqpoint{6.692762in}{3.188090in}}%
\pgfpathlineto{\pgfqpoint{6.692762in}{3.290927in}}%
\pgfpathlineto{\pgfqpoint{5.800000in}{3.290927in}}%
\pgfpathclose%
\pgfusepath{fill}%
\end{pgfscope}%
\begin{pgfscope}%
\pgfpathrectangle{\pgfqpoint{5.800000in}{0.720000in}}{\pgfqpoint{1.400000in}{4.620000in}}%
\pgfusepath{clip}%
\pgfsetbuttcap%
\pgfsetmiterjoin%
\definecolor{currentfill}{rgb}{0.121569,0.466667,0.705882}%
\pgfsetfillcolor{currentfill}%
\pgfsetlinewidth{0.000000pt}%
\definecolor{currentstroke}{rgb}{0.000000,0.000000,0.000000}%
\pgfsetstrokecolor{currentstroke}%
\pgfsetstrokeopacity{0.000000}%
\pgfsetdash{}{0pt}%
\pgfpathmoveto{\pgfqpoint{5.800000in}{3.290927in}}%
\pgfpathlineto{\pgfqpoint{6.480782in}{3.290927in}}%
\pgfpathlineto{\pgfqpoint{6.480782in}{3.393764in}}%
\pgfpathlineto{\pgfqpoint{5.800000in}{3.393764in}}%
\pgfpathclose%
\pgfusepath{fill}%
\end{pgfscope}%
\begin{pgfscope}%
\pgfpathrectangle{\pgfqpoint{5.800000in}{0.720000in}}{\pgfqpoint{1.400000in}{4.620000in}}%
\pgfusepath{clip}%
\pgfsetbuttcap%
\pgfsetmiterjoin%
\definecolor{currentfill}{rgb}{0.121569,0.466667,0.705882}%
\pgfsetfillcolor{currentfill}%
\pgfsetlinewidth{0.000000pt}%
\definecolor{currentstroke}{rgb}{0.000000,0.000000,0.000000}%
\pgfsetstrokecolor{currentstroke}%
\pgfsetstrokeopacity{0.000000}%
\pgfsetdash{}{0pt}%
\pgfpathmoveto{\pgfqpoint{5.800000in}{3.393764in}}%
\pgfpathlineto{\pgfqpoint{6.429825in}{3.393764in}}%
\pgfpathlineto{\pgfqpoint{6.429825in}{3.496601in}}%
\pgfpathlineto{\pgfqpoint{5.800000in}{3.496601in}}%
\pgfpathclose%
\pgfusepath{fill}%
\end{pgfscope}%
\begin{pgfscope}%
\pgfpathrectangle{\pgfqpoint{5.800000in}{0.720000in}}{\pgfqpoint{1.400000in}{4.620000in}}%
\pgfusepath{clip}%
\pgfsetbuttcap%
\pgfsetmiterjoin%
\definecolor{currentfill}{rgb}{0.121569,0.466667,0.705882}%
\pgfsetfillcolor{currentfill}%
\pgfsetlinewidth{0.000000pt}%
\definecolor{currentstroke}{rgb}{0.000000,0.000000,0.000000}%
\pgfsetstrokecolor{currentstroke}%
\pgfsetstrokeopacity{0.000000}%
\pgfsetdash{}{0pt}%
\pgfpathmoveto{\pgfqpoint{5.800000in}{3.496601in}}%
\pgfpathlineto{\pgfqpoint{6.380907in}{3.496601in}}%
\pgfpathlineto{\pgfqpoint{6.380907in}{3.599438in}}%
\pgfpathlineto{\pgfqpoint{5.800000in}{3.599438in}}%
\pgfpathclose%
\pgfusepath{fill}%
\end{pgfscope}%
\begin{pgfscope}%
\pgfpathrectangle{\pgfqpoint{5.800000in}{0.720000in}}{\pgfqpoint{1.400000in}{4.620000in}}%
\pgfusepath{clip}%
\pgfsetbuttcap%
\pgfsetmiterjoin%
\definecolor{currentfill}{rgb}{0.121569,0.466667,0.705882}%
\pgfsetfillcolor{currentfill}%
\pgfsetlinewidth{0.000000pt}%
\definecolor{currentstroke}{rgb}{0.000000,0.000000,0.000000}%
\pgfsetstrokecolor{currentstroke}%
\pgfsetstrokeopacity{0.000000}%
\pgfsetdash{}{0pt}%
\pgfpathmoveto{\pgfqpoint{5.800000in}{3.599438in}}%
\pgfpathlineto{\pgfqpoint{6.329950in}{3.599438in}}%
\pgfpathlineto{\pgfqpoint{6.329950in}{3.702275in}}%
\pgfpathlineto{\pgfqpoint{5.800000in}{3.702275in}}%
\pgfpathclose%
\pgfusepath{fill}%
\end{pgfscope}%
\begin{pgfscope}%
\pgfpathrectangle{\pgfqpoint{5.800000in}{0.720000in}}{\pgfqpoint{1.400000in}{4.620000in}}%
\pgfusepath{clip}%
\pgfsetbuttcap%
\pgfsetmiterjoin%
\definecolor{currentfill}{rgb}{0.121569,0.466667,0.705882}%
\pgfsetfillcolor{currentfill}%
\pgfsetlinewidth{0.000000pt}%
\definecolor{currentstroke}{rgb}{0.000000,0.000000,0.000000}%
\pgfsetstrokecolor{currentstroke}%
\pgfsetstrokeopacity{0.000000}%
\pgfsetdash{}{0pt}%
\pgfpathmoveto{\pgfqpoint{5.800000in}{3.702275in}}%
\pgfpathlineto{\pgfqpoint{6.189309in}{3.702275in}}%
\pgfpathlineto{\pgfqpoint{6.189309in}{3.805112in}}%
\pgfpathlineto{\pgfqpoint{5.800000in}{3.805112in}}%
\pgfpathclose%
\pgfusepath{fill}%
\end{pgfscope}%
\begin{pgfscope}%
\pgfpathrectangle{\pgfqpoint{5.800000in}{0.720000in}}{\pgfqpoint{1.400000in}{4.620000in}}%
\pgfusepath{clip}%
\pgfsetbuttcap%
\pgfsetmiterjoin%
\definecolor{currentfill}{rgb}{0.121569,0.466667,0.705882}%
\pgfsetfillcolor{currentfill}%
\pgfsetlinewidth{0.000000pt}%
\definecolor{currentstroke}{rgb}{0.000000,0.000000,0.000000}%
\pgfsetstrokecolor{currentstroke}%
\pgfsetstrokeopacity{0.000000}%
\pgfsetdash{}{0pt}%
\pgfpathmoveto{\pgfqpoint{5.800000in}{3.805112in}}%
\pgfpathlineto{\pgfqpoint{6.097587in}{3.805112in}}%
\pgfpathlineto{\pgfqpoint{6.097587in}{3.907949in}}%
\pgfpathlineto{\pgfqpoint{5.800000in}{3.907949in}}%
\pgfpathclose%
\pgfusepath{fill}%
\end{pgfscope}%
\begin{pgfscope}%
\pgfpathrectangle{\pgfqpoint{5.800000in}{0.720000in}}{\pgfqpoint{1.400000in}{4.620000in}}%
\pgfusepath{clip}%
\pgfsetbuttcap%
\pgfsetmiterjoin%
\definecolor{currentfill}{rgb}{0.121569,0.466667,0.705882}%
\pgfsetfillcolor{currentfill}%
\pgfsetlinewidth{0.000000pt}%
\definecolor{currentstroke}{rgb}{0.000000,0.000000,0.000000}%
\pgfsetstrokecolor{currentstroke}%
\pgfsetstrokeopacity{0.000000}%
\pgfsetdash{}{0pt}%
\pgfpathmoveto{\pgfqpoint{5.800000in}{3.907949in}}%
\pgfpathlineto{\pgfqpoint{6.042554in}{3.907949in}}%
\pgfpathlineto{\pgfqpoint{6.042554in}{4.010786in}}%
\pgfpathlineto{\pgfqpoint{5.800000in}{4.010786in}}%
\pgfpathclose%
\pgfusepath{fill}%
\end{pgfscope}%
\begin{pgfscope}%
\pgfpathrectangle{\pgfqpoint{5.800000in}{0.720000in}}{\pgfqpoint{1.400000in}{4.620000in}}%
\pgfusepath{clip}%
\pgfsetbuttcap%
\pgfsetmiterjoin%
\definecolor{currentfill}{rgb}{0.121569,0.466667,0.705882}%
\pgfsetfillcolor{currentfill}%
\pgfsetlinewidth{0.000000pt}%
\definecolor{currentstroke}{rgb}{0.000000,0.000000,0.000000}%
\pgfsetstrokecolor{currentstroke}%
\pgfsetstrokeopacity{0.000000}%
\pgfsetdash{}{0pt}%
\pgfpathmoveto{\pgfqpoint{5.800000in}{4.010786in}}%
\pgfpathlineto{\pgfqpoint{6.026248in}{4.010786in}}%
\pgfpathlineto{\pgfqpoint{6.026248in}{4.113623in}}%
\pgfpathlineto{\pgfqpoint{5.800000in}{4.113623in}}%
\pgfpathclose%
\pgfusepath{fill}%
\end{pgfscope}%
\begin{pgfscope}%
\pgfpathrectangle{\pgfqpoint{5.800000in}{0.720000in}}{\pgfqpoint{1.400000in}{4.620000in}}%
\pgfusepath{clip}%
\pgfsetbuttcap%
\pgfsetmiterjoin%
\definecolor{currentfill}{rgb}{0.121569,0.466667,0.705882}%
\pgfsetfillcolor{currentfill}%
\pgfsetlinewidth{0.000000pt}%
\definecolor{currentstroke}{rgb}{0.000000,0.000000,0.000000}%
\pgfsetstrokecolor{currentstroke}%
\pgfsetstrokeopacity{0.000000}%
\pgfsetdash{}{0pt}%
\pgfpathmoveto{\pgfqpoint{5.800000in}{4.113623in}}%
\pgfpathlineto{\pgfqpoint{5.958985in}{4.113623in}}%
\pgfpathlineto{\pgfqpoint{5.958985in}{4.216460in}}%
\pgfpathlineto{\pgfqpoint{5.800000in}{4.216460in}}%
\pgfpathclose%
\pgfusepath{fill}%
\end{pgfscope}%
\begin{pgfscope}%
\pgfpathrectangle{\pgfqpoint{5.800000in}{0.720000in}}{\pgfqpoint{1.400000in}{4.620000in}}%
\pgfusepath{clip}%
\pgfsetbuttcap%
\pgfsetmiterjoin%
\definecolor{currentfill}{rgb}{0.121569,0.466667,0.705882}%
\pgfsetfillcolor{currentfill}%
\pgfsetlinewidth{0.000000pt}%
\definecolor{currentstroke}{rgb}{0.000000,0.000000,0.000000}%
\pgfsetstrokecolor{currentstroke}%
\pgfsetstrokeopacity{0.000000}%
\pgfsetdash{}{0pt}%
\pgfpathmoveto{\pgfqpoint{5.800000in}{4.216460in}}%
\pgfpathlineto{\pgfqpoint{5.958985in}{4.216460in}}%
\pgfpathlineto{\pgfqpoint{5.958985in}{4.319297in}}%
\pgfpathlineto{\pgfqpoint{5.800000in}{4.319297in}}%
\pgfpathclose%
\pgfusepath{fill}%
\end{pgfscope}%
\begin{pgfscope}%
\pgfpathrectangle{\pgfqpoint{5.800000in}{0.720000in}}{\pgfqpoint{1.400000in}{4.620000in}}%
\pgfusepath{clip}%
\pgfsetbuttcap%
\pgfsetmiterjoin%
\definecolor{currentfill}{rgb}{0.121569,0.466667,0.705882}%
\pgfsetfillcolor{currentfill}%
\pgfsetlinewidth{0.000000pt}%
\definecolor{currentstroke}{rgb}{0.000000,0.000000,0.000000}%
\pgfsetstrokecolor{currentstroke}%
\pgfsetstrokeopacity{0.000000}%
\pgfsetdash{}{0pt}%
\pgfpathmoveto{\pgfqpoint{5.800000in}{4.319297in}}%
\pgfpathlineto{\pgfqpoint{5.930449in}{4.319297in}}%
\pgfpathlineto{\pgfqpoint{5.930449in}{4.422134in}}%
\pgfpathlineto{\pgfqpoint{5.800000in}{4.422134in}}%
\pgfpathclose%
\pgfusepath{fill}%
\end{pgfscope}%
\begin{pgfscope}%
\pgfpathrectangle{\pgfqpoint{5.800000in}{0.720000in}}{\pgfqpoint{1.400000in}{4.620000in}}%
\pgfusepath{clip}%
\pgfsetbuttcap%
\pgfsetmiterjoin%
\definecolor{currentfill}{rgb}{0.121569,0.466667,0.705882}%
\pgfsetfillcolor{currentfill}%
\pgfsetlinewidth{0.000000pt}%
\definecolor{currentstroke}{rgb}{0.000000,0.000000,0.000000}%
\pgfsetstrokecolor{currentstroke}%
\pgfsetstrokeopacity{0.000000}%
\pgfsetdash{}{0pt}%
\pgfpathmoveto{\pgfqpoint{5.800000in}{4.422134in}}%
\pgfpathlineto{\pgfqpoint{5.910066in}{4.422134in}}%
\pgfpathlineto{\pgfqpoint{5.910066in}{4.524972in}}%
\pgfpathlineto{\pgfqpoint{5.800000in}{4.524972in}}%
\pgfpathclose%
\pgfusepath{fill}%
\end{pgfscope}%
\begin{pgfscope}%
\pgfpathrectangle{\pgfqpoint{5.800000in}{0.720000in}}{\pgfqpoint{1.400000in}{4.620000in}}%
\pgfusepath{clip}%
\pgfsetbuttcap%
\pgfsetmiterjoin%
\definecolor{currentfill}{rgb}{0.121569,0.466667,0.705882}%
\pgfsetfillcolor{currentfill}%
\pgfsetlinewidth{0.000000pt}%
\definecolor{currentstroke}{rgb}{0.000000,0.000000,0.000000}%
\pgfsetstrokecolor{currentstroke}%
\pgfsetstrokeopacity{0.000000}%
\pgfsetdash{}{0pt}%
\pgfpathmoveto{\pgfqpoint{5.800000in}{4.524972in}}%
\pgfpathlineto{\pgfqpoint{5.885607in}{4.524972in}}%
\pgfpathlineto{\pgfqpoint{5.885607in}{4.627809in}}%
\pgfpathlineto{\pgfqpoint{5.800000in}{4.627809in}}%
\pgfpathclose%
\pgfusepath{fill}%
\end{pgfscope}%
\begin{pgfscope}%
\pgfpathrectangle{\pgfqpoint{5.800000in}{0.720000in}}{\pgfqpoint{1.400000in}{4.620000in}}%
\pgfusepath{clip}%
\pgfsetbuttcap%
\pgfsetmiterjoin%
\definecolor{currentfill}{rgb}{0.121569,0.466667,0.705882}%
\pgfsetfillcolor{currentfill}%
\pgfsetlinewidth{0.000000pt}%
\definecolor{currentstroke}{rgb}{0.000000,0.000000,0.000000}%
\pgfsetstrokecolor{currentstroke}%
\pgfsetstrokeopacity{0.000000}%
\pgfsetdash{}{0pt}%
\pgfpathmoveto{\pgfqpoint{5.800000in}{4.627809in}}%
\pgfpathlineto{\pgfqpoint{5.881531in}{4.627809in}}%
\pgfpathlineto{\pgfqpoint{5.881531in}{4.730646in}}%
\pgfpathlineto{\pgfqpoint{5.800000in}{4.730646in}}%
\pgfpathclose%
\pgfusepath{fill}%
\end{pgfscope}%
\begin{pgfscope}%
\pgfpathrectangle{\pgfqpoint{5.800000in}{0.720000in}}{\pgfqpoint{1.400000in}{4.620000in}}%
\pgfusepath{clip}%
\pgfsetbuttcap%
\pgfsetmiterjoin%
\definecolor{currentfill}{rgb}{0.121569,0.466667,0.705882}%
\pgfsetfillcolor{currentfill}%
\pgfsetlinewidth{0.000000pt}%
\definecolor{currentstroke}{rgb}{0.000000,0.000000,0.000000}%
\pgfsetstrokecolor{currentstroke}%
\pgfsetstrokeopacity{0.000000}%
\pgfsetdash{}{0pt}%
\pgfpathmoveto{\pgfqpoint{5.800000in}{4.730646in}}%
\pgfpathlineto{\pgfqpoint{5.877454in}{4.730646in}}%
\pgfpathlineto{\pgfqpoint{5.877454in}{4.833483in}}%
\pgfpathlineto{\pgfqpoint{5.800000in}{4.833483in}}%
\pgfpathclose%
\pgfusepath{fill}%
\end{pgfscope}%
\begin{pgfscope}%
\pgfpathrectangle{\pgfqpoint{5.800000in}{0.720000in}}{\pgfqpoint{1.400000in}{4.620000in}}%
\pgfusepath{clip}%
\pgfsetbuttcap%
\pgfsetmiterjoin%
\definecolor{currentfill}{rgb}{0.121569,0.466667,0.705882}%
\pgfsetfillcolor{currentfill}%
\pgfsetlinewidth{0.000000pt}%
\definecolor{currentstroke}{rgb}{0.000000,0.000000,0.000000}%
\pgfsetstrokecolor{currentstroke}%
\pgfsetstrokeopacity{0.000000}%
\pgfsetdash{}{0pt}%
\pgfpathmoveto{\pgfqpoint{5.800000in}{4.833483in}}%
\pgfpathlineto{\pgfqpoint{5.863186in}{4.833483in}}%
\pgfpathlineto{\pgfqpoint{5.863186in}{4.936320in}}%
\pgfpathlineto{\pgfqpoint{5.800000in}{4.936320in}}%
\pgfpathclose%
\pgfusepath{fill}%
\end{pgfscope}%
\begin{pgfscope}%
\pgfpathrectangle{\pgfqpoint{5.800000in}{0.720000in}}{\pgfqpoint{1.400000in}{4.620000in}}%
\pgfusepath{clip}%
\pgfsetbuttcap%
\pgfsetmiterjoin%
\definecolor{currentfill}{rgb}{0.121569,0.466667,0.705882}%
\pgfsetfillcolor{currentfill}%
\pgfsetlinewidth{0.000000pt}%
\definecolor{currentstroke}{rgb}{0.000000,0.000000,0.000000}%
\pgfsetstrokecolor{currentstroke}%
\pgfsetstrokeopacity{0.000000}%
\pgfsetdash{}{0pt}%
\pgfpathmoveto{\pgfqpoint{5.800000in}{4.936320in}}%
\pgfpathlineto{\pgfqpoint{5.855033in}{4.936320in}}%
\pgfpathlineto{\pgfqpoint{5.855033in}{5.039157in}}%
\pgfpathlineto{\pgfqpoint{5.800000in}{5.039157in}}%
\pgfpathclose%
\pgfusepath{fill}%
\end{pgfscope}%
\begin{pgfscope}%
\pgfpathrectangle{\pgfqpoint{5.800000in}{0.720000in}}{\pgfqpoint{1.400000in}{4.620000in}}%
\pgfusepath{clip}%
\pgfsetbuttcap%
\pgfsetmiterjoin%
\definecolor{currentfill}{rgb}{0.121569,0.466667,0.705882}%
\pgfsetfillcolor{currentfill}%
\pgfsetlinewidth{0.000000pt}%
\definecolor{currentstroke}{rgb}{0.000000,0.000000,0.000000}%
\pgfsetstrokecolor{currentstroke}%
\pgfsetstrokeopacity{0.000000}%
\pgfsetdash{}{0pt}%
\pgfpathmoveto{\pgfqpoint{5.800000in}{5.039157in}}%
\pgfpathlineto{\pgfqpoint{5.859110in}{5.039157in}}%
\pgfpathlineto{\pgfqpoint{5.859110in}{5.141994in}}%
\pgfpathlineto{\pgfqpoint{5.800000in}{5.141994in}}%
\pgfpathclose%
\pgfusepath{fill}%
\end{pgfscope}%
\begin{pgfscope}%
\pgfpathrectangle{\pgfqpoint{5.800000in}{0.720000in}}{\pgfqpoint{1.400000in}{4.620000in}}%
\pgfusepath{clip}%
\pgfsetbuttcap%
\pgfsetmiterjoin%
\definecolor{currentfill}{rgb}{0.121569,0.466667,0.705882}%
\pgfsetfillcolor{currentfill}%
\pgfsetlinewidth{0.000000pt}%
\definecolor{currentstroke}{rgb}{0.000000,0.000000,0.000000}%
\pgfsetstrokecolor{currentstroke}%
\pgfsetstrokeopacity{0.000000}%
\pgfsetdash{}{0pt}%
\pgfpathmoveto{\pgfqpoint{5.800000in}{5.141994in}}%
\pgfpathlineto{\pgfqpoint{5.824459in}{5.141994in}}%
\pgfpathlineto{\pgfqpoint{5.824459in}{5.244831in}}%
\pgfpathlineto{\pgfqpoint{5.800000in}{5.244831in}}%
\pgfpathclose%
\pgfusepath{fill}%
\end{pgfscope}%
\begin{pgfscope}%
\pgfpathrectangle{\pgfqpoint{5.800000in}{0.720000in}}{\pgfqpoint{1.400000in}{4.620000in}}%
\pgfusepath{clip}%
\pgfsetbuttcap%
\pgfsetmiterjoin%
\definecolor{currentfill}{rgb}{0.121569,0.466667,0.705882}%
\pgfsetfillcolor{currentfill}%
\pgfsetlinewidth{0.000000pt}%
\definecolor{currentstroke}{rgb}{0.000000,0.000000,0.000000}%
\pgfsetstrokecolor{currentstroke}%
\pgfsetstrokeopacity{0.000000}%
\pgfsetdash{}{0pt}%
\pgfpathmoveto{\pgfqpoint{5.800000in}{5.244831in}}%
\pgfpathlineto{\pgfqpoint{5.814268in}{5.244831in}}%
\pgfpathlineto{\pgfqpoint{5.814268in}{5.347668in}}%
\pgfpathlineto{\pgfqpoint{5.800000in}{5.347668in}}%
\pgfpathclose%
\pgfusepath{fill}%
\end{pgfscope}%
\begin{pgfscope}%
\pgfpathrectangle{\pgfqpoint{5.800000in}{0.720000in}}{\pgfqpoint{1.400000in}{4.620000in}}%
\pgfusepath{clip}%
\pgfsetbuttcap%
\pgfsetmiterjoin%
\definecolor{currentfill}{rgb}{0.121569,0.466667,0.705882}%
\pgfsetfillcolor{currentfill}%
\pgfsetlinewidth{0.000000pt}%
\definecolor{currentstroke}{rgb}{0.000000,0.000000,0.000000}%
\pgfsetstrokecolor{currentstroke}%
\pgfsetstrokeopacity{0.000000}%
\pgfsetdash{}{0pt}%
\pgfpathmoveto{\pgfqpoint{5.800000in}{5.347668in}}%
\pgfpathlineto{\pgfqpoint{5.812230in}{5.347668in}}%
\pgfpathlineto{\pgfqpoint{5.812230in}{5.450505in}}%
\pgfpathlineto{\pgfqpoint{5.800000in}{5.450505in}}%
\pgfpathclose%
\pgfusepath{fill}%
\end{pgfscope}%
\begin{pgfscope}%
\pgfpathrectangle{\pgfqpoint{5.800000in}{0.720000in}}{\pgfqpoint{1.400000in}{4.620000in}}%
\pgfusepath{clip}%
\pgfsetbuttcap%
\pgfsetmiterjoin%
\definecolor{currentfill}{rgb}{0.121569,0.466667,0.705882}%
\pgfsetfillcolor{currentfill}%
\pgfsetlinewidth{0.000000pt}%
\definecolor{currentstroke}{rgb}{0.000000,0.000000,0.000000}%
\pgfsetstrokecolor{currentstroke}%
\pgfsetstrokeopacity{0.000000}%
\pgfsetdash{}{0pt}%
\pgfpathmoveto{\pgfqpoint{5.800000in}{5.450505in}}%
\pgfpathlineto{\pgfqpoint{5.818344in}{5.450505in}}%
\pgfpathlineto{\pgfqpoint{5.818344in}{5.553342in}}%
\pgfpathlineto{\pgfqpoint{5.800000in}{5.553342in}}%
\pgfpathclose%
\pgfusepath{fill}%
\end{pgfscope}%
\begin{pgfscope}%
\pgfpathrectangle{\pgfqpoint{5.800000in}{0.720000in}}{\pgfqpoint{1.400000in}{4.620000in}}%
\pgfusepath{clip}%
\pgfsetbuttcap%
\pgfsetmiterjoin%
\definecolor{currentfill}{rgb}{0.121569,0.466667,0.705882}%
\pgfsetfillcolor{currentfill}%
\pgfsetlinewidth{0.000000pt}%
\definecolor{currentstroke}{rgb}{0.000000,0.000000,0.000000}%
\pgfsetstrokecolor{currentstroke}%
\pgfsetstrokeopacity{0.000000}%
\pgfsetdash{}{0pt}%
\pgfpathmoveto{\pgfqpoint{5.800000in}{5.553342in}}%
\pgfpathlineto{\pgfqpoint{5.814268in}{5.553342in}}%
\pgfpathlineto{\pgfqpoint{5.814268in}{5.656179in}}%
\pgfpathlineto{\pgfqpoint{5.800000in}{5.656179in}}%
\pgfpathclose%
\pgfusepath{fill}%
\end{pgfscope}%
\begin{pgfscope}%
\pgfpathrectangle{\pgfqpoint{5.800000in}{0.720000in}}{\pgfqpoint{1.400000in}{4.620000in}}%
\pgfusepath{clip}%
\pgfsetbuttcap%
\pgfsetmiterjoin%
\definecolor{currentfill}{rgb}{0.121569,0.466667,0.705882}%
\pgfsetfillcolor{currentfill}%
\pgfsetlinewidth{0.000000pt}%
\definecolor{currentstroke}{rgb}{0.000000,0.000000,0.000000}%
\pgfsetstrokecolor{currentstroke}%
\pgfsetstrokeopacity{0.000000}%
\pgfsetdash{}{0pt}%
\pgfpathmoveto{\pgfqpoint{5.800000in}{5.656179in}}%
\pgfpathlineto{\pgfqpoint{5.820383in}{5.656179in}}%
\pgfpathlineto{\pgfqpoint{5.820383in}{5.759016in}}%
\pgfpathlineto{\pgfqpoint{5.800000in}{5.759016in}}%
\pgfpathclose%
\pgfusepath{fill}%
\end{pgfscope}%
\begin{pgfscope}%
\pgfpathrectangle{\pgfqpoint{5.800000in}{0.720000in}}{\pgfqpoint{1.400000in}{4.620000in}}%
\pgfusepath{clip}%
\pgfsetbuttcap%
\pgfsetmiterjoin%
\definecolor{currentfill}{rgb}{0.121569,0.466667,0.705882}%
\pgfsetfillcolor{currentfill}%
\pgfsetlinewidth{0.000000pt}%
\definecolor{currentstroke}{rgb}{0.000000,0.000000,0.000000}%
\pgfsetstrokecolor{currentstroke}%
\pgfsetstrokeopacity{0.000000}%
\pgfsetdash{}{0pt}%
\pgfpathmoveto{\pgfqpoint{5.800000in}{5.759016in}}%
\pgfpathlineto{\pgfqpoint{5.808153in}{5.759016in}}%
\pgfpathlineto{\pgfqpoint{5.808153in}{5.861853in}}%
\pgfpathlineto{\pgfqpoint{5.800000in}{5.861853in}}%
\pgfpathclose%
\pgfusepath{fill}%
\end{pgfscope}%
\begin{pgfscope}%
\pgfpathrectangle{\pgfqpoint{5.800000in}{0.720000in}}{\pgfqpoint{1.400000in}{4.620000in}}%
\pgfusepath{clip}%
\pgfsetbuttcap%
\pgfsetmiterjoin%
\definecolor{currentfill}{rgb}{0.121569,0.466667,0.705882}%
\pgfsetfillcolor{currentfill}%
\pgfsetlinewidth{0.000000pt}%
\definecolor{currentstroke}{rgb}{0.000000,0.000000,0.000000}%
\pgfsetstrokecolor{currentstroke}%
\pgfsetstrokeopacity{0.000000}%
\pgfsetdash{}{0pt}%
\pgfpathmoveto{\pgfqpoint{5.800000in}{5.861853in}}%
\pgfpathlineto{\pgfqpoint{5.804077in}{5.861853in}}%
\pgfpathlineto{\pgfqpoint{5.804077in}{5.964690in}}%
\pgfpathlineto{\pgfqpoint{5.800000in}{5.964690in}}%
\pgfpathclose%
\pgfusepath{fill}%
\end{pgfscope}%
\begin{pgfscope}%
\pgfpathrectangle{\pgfqpoint{5.800000in}{0.720000in}}{\pgfqpoint{1.400000in}{4.620000in}}%
\pgfusepath{clip}%
\pgfsetbuttcap%
\pgfsetmiterjoin%
\definecolor{currentfill}{rgb}{0.121569,0.466667,0.705882}%
\pgfsetfillcolor{currentfill}%
\pgfsetlinewidth{0.000000pt}%
\definecolor{currentstroke}{rgb}{0.000000,0.000000,0.000000}%
\pgfsetstrokecolor{currentstroke}%
\pgfsetstrokeopacity{0.000000}%
\pgfsetdash{}{0pt}%
\pgfpathmoveto{\pgfqpoint{5.800000in}{5.964690in}}%
\pgfpathlineto{\pgfqpoint{5.806115in}{5.964690in}}%
\pgfpathlineto{\pgfqpoint{5.806115in}{6.067528in}}%
\pgfpathlineto{\pgfqpoint{5.800000in}{6.067528in}}%
\pgfpathclose%
\pgfusepath{fill}%
\end{pgfscope}%
\begin{pgfscope}%
\pgfpathrectangle{\pgfqpoint{5.800000in}{0.720000in}}{\pgfqpoint{1.400000in}{4.620000in}}%
\pgfusepath{clip}%
\pgfsetbuttcap%
\pgfsetmiterjoin%
\definecolor{currentfill}{rgb}{0.121569,0.466667,0.705882}%
\pgfsetfillcolor{currentfill}%
\pgfsetlinewidth{0.000000pt}%
\definecolor{currentstroke}{rgb}{0.000000,0.000000,0.000000}%
\pgfsetstrokecolor{currentstroke}%
\pgfsetstrokeopacity{0.000000}%
\pgfsetdash{}{0pt}%
\pgfpathmoveto{\pgfqpoint{5.800000in}{6.067528in}}%
\pgfpathlineto{\pgfqpoint{5.804077in}{6.067528in}}%
\pgfpathlineto{\pgfqpoint{5.804077in}{6.170365in}}%
\pgfpathlineto{\pgfqpoint{5.800000in}{6.170365in}}%
\pgfpathclose%
\pgfusepath{fill}%
\end{pgfscope}%
\begin{pgfscope}%
\pgfpathrectangle{\pgfqpoint{5.800000in}{0.720000in}}{\pgfqpoint{1.400000in}{4.620000in}}%
\pgfusepath{clip}%
\pgfsetbuttcap%
\pgfsetmiterjoin%
\definecolor{currentfill}{rgb}{0.121569,0.466667,0.705882}%
\pgfsetfillcolor{currentfill}%
\pgfsetlinewidth{0.000000pt}%
\definecolor{currentstroke}{rgb}{0.000000,0.000000,0.000000}%
\pgfsetstrokecolor{currentstroke}%
\pgfsetstrokeopacity{0.000000}%
\pgfsetdash{}{0pt}%
\pgfpathmoveto{\pgfqpoint{5.800000in}{6.170365in}}%
\pgfpathlineto{\pgfqpoint{5.806115in}{6.170365in}}%
\pgfpathlineto{\pgfqpoint{5.806115in}{6.273202in}}%
\pgfpathlineto{\pgfqpoint{5.800000in}{6.273202in}}%
\pgfpathclose%
\pgfusepath{fill}%
\end{pgfscope}%
\begin{pgfscope}%
\pgfpathrectangle{\pgfqpoint{5.800000in}{0.720000in}}{\pgfqpoint{1.400000in}{4.620000in}}%
\pgfusepath{clip}%
\pgfsetbuttcap%
\pgfsetmiterjoin%
\definecolor{currentfill}{rgb}{0.121569,0.466667,0.705882}%
\pgfsetfillcolor{currentfill}%
\pgfsetlinewidth{0.000000pt}%
\definecolor{currentstroke}{rgb}{0.000000,0.000000,0.000000}%
\pgfsetstrokecolor{currentstroke}%
\pgfsetstrokeopacity{0.000000}%
\pgfsetdash{}{0pt}%
\pgfpathmoveto{\pgfqpoint{5.800000in}{6.273202in}}%
\pgfpathlineto{\pgfqpoint{5.802038in}{6.273202in}}%
\pgfpathlineto{\pgfqpoint{5.802038in}{6.376039in}}%
\pgfpathlineto{\pgfqpoint{5.800000in}{6.376039in}}%
\pgfpathclose%
\pgfusepath{fill}%
\end{pgfscope}%
\begin{pgfscope}%
\pgfpathrectangle{\pgfqpoint{5.800000in}{0.720000in}}{\pgfqpoint{1.400000in}{4.620000in}}%
\pgfusepath{clip}%
\pgfsetbuttcap%
\pgfsetmiterjoin%
\definecolor{currentfill}{rgb}{0.121569,0.466667,0.705882}%
\pgfsetfillcolor{currentfill}%
\pgfsetlinewidth{0.000000pt}%
\definecolor{currentstroke}{rgb}{0.000000,0.000000,0.000000}%
\pgfsetstrokecolor{currentstroke}%
\pgfsetstrokeopacity{0.000000}%
\pgfsetdash{}{0pt}%
\pgfpathmoveto{\pgfqpoint{5.800000in}{6.376039in}}%
\pgfpathlineto{\pgfqpoint{5.804077in}{6.376039in}}%
\pgfpathlineto{\pgfqpoint{5.804077in}{6.478876in}}%
\pgfpathlineto{\pgfqpoint{5.800000in}{6.478876in}}%
\pgfpathclose%
\pgfusepath{fill}%
\end{pgfscope}%
\begin{pgfscope}%
\pgfpathrectangle{\pgfqpoint{5.800000in}{0.720000in}}{\pgfqpoint{1.400000in}{4.620000in}}%
\pgfusepath{clip}%
\pgfsetbuttcap%
\pgfsetmiterjoin%
\definecolor{currentfill}{rgb}{0.121569,0.466667,0.705882}%
\pgfsetfillcolor{currentfill}%
\pgfsetlinewidth{0.000000pt}%
\definecolor{currentstroke}{rgb}{0.000000,0.000000,0.000000}%
\pgfsetstrokecolor{currentstroke}%
\pgfsetstrokeopacity{0.000000}%
\pgfsetdash{}{0pt}%
\pgfpathmoveto{\pgfqpoint{5.800000in}{6.478876in}}%
\pgfpathlineto{\pgfqpoint{5.802038in}{6.478876in}}%
\pgfpathlineto{\pgfqpoint{5.802038in}{6.581713in}}%
\pgfpathlineto{\pgfqpoint{5.800000in}{6.581713in}}%
\pgfpathclose%
\pgfusepath{fill}%
\end{pgfscope}%
\begin{pgfscope}%
\pgfpathrectangle{\pgfqpoint{5.800000in}{0.720000in}}{\pgfqpoint{1.400000in}{4.620000in}}%
\pgfusepath{clip}%
\pgfsetbuttcap%
\pgfsetmiterjoin%
\definecolor{currentfill}{rgb}{0.121569,0.466667,0.705882}%
\pgfsetfillcolor{currentfill}%
\pgfsetlinewidth{0.000000pt}%
\definecolor{currentstroke}{rgb}{0.000000,0.000000,0.000000}%
\pgfsetstrokecolor{currentstroke}%
\pgfsetstrokeopacity{0.000000}%
\pgfsetdash{}{0pt}%
\pgfpathmoveto{\pgfqpoint{5.800000in}{6.581713in}}%
\pgfpathlineto{\pgfqpoint{5.808153in}{6.581713in}}%
\pgfpathlineto{\pgfqpoint{5.808153in}{6.684550in}}%
\pgfpathlineto{\pgfqpoint{5.800000in}{6.684550in}}%
\pgfpathclose%
\pgfusepath{fill}%
\end{pgfscope}%
\begin{pgfscope}%
\pgfpathrectangle{\pgfqpoint{5.800000in}{0.720000in}}{\pgfqpoint{1.400000in}{4.620000in}}%
\pgfusepath{clip}%
\pgfsetbuttcap%
\pgfsetmiterjoin%
\definecolor{currentfill}{rgb}{0.121569,0.466667,0.705882}%
\pgfsetfillcolor{currentfill}%
\pgfsetlinewidth{0.000000pt}%
\definecolor{currentstroke}{rgb}{0.000000,0.000000,0.000000}%
\pgfsetstrokecolor{currentstroke}%
\pgfsetstrokeopacity{0.000000}%
\pgfsetdash{}{0pt}%
\pgfpathmoveto{\pgfqpoint{5.800000in}{6.684550in}}%
\pgfpathlineto{\pgfqpoint{5.802038in}{6.684550in}}%
\pgfpathlineto{\pgfqpoint{5.802038in}{6.787387in}}%
\pgfpathlineto{\pgfqpoint{5.800000in}{6.787387in}}%
\pgfpathclose%
\pgfusepath{fill}%
\end{pgfscope}%
\begin{pgfscope}%
\pgfpathrectangle{\pgfqpoint{5.800000in}{0.720000in}}{\pgfqpoint{1.400000in}{4.620000in}}%
\pgfusepath{clip}%
\pgfsetbuttcap%
\pgfsetmiterjoin%
\definecolor{currentfill}{rgb}{0.121569,0.466667,0.705882}%
\pgfsetfillcolor{currentfill}%
\pgfsetlinewidth{0.000000pt}%
\definecolor{currentstroke}{rgb}{0.000000,0.000000,0.000000}%
\pgfsetstrokecolor{currentstroke}%
\pgfsetstrokeopacity{0.000000}%
\pgfsetdash{}{0pt}%
\pgfpathmoveto{\pgfqpoint{5.800000in}{6.787387in}}%
\pgfpathlineto{\pgfqpoint{5.802038in}{6.787387in}}%
\pgfpathlineto{\pgfqpoint{5.802038in}{6.890224in}}%
\pgfpathlineto{\pgfqpoint{5.800000in}{6.890224in}}%
\pgfpathclose%
\pgfusepath{fill}%
\end{pgfscope}%
\begin{pgfscope}%
\pgfpathrectangle{\pgfqpoint{5.800000in}{0.720000in}}{\pgfqpoint{1.400000in}{4.620000in}}%
\pgfusepath{clip}%
\pgfsetbuttcap%
\pgfsetmiterjoin%
\definecolor{currentfill}{rgb}{0.121569,0.466667,0.705882}%
\pgfsetfillcolor{currentfill}%
\pgfsetlinewidth{0.000000pt}%
\definecolor{currentstroke}{rgb}{0.000000,0.000000,0.000000}%
\pgfsetstrokecolor{currentstroke}%
\pgfsetstrokeopacity{0.000000}%
\pgfsetdash{}{0pt}%
\pgfpathmoveto{\pgfqpoint{5.800000in}{6.890224in}}%
\pgfpathlineto{\pgfqpoint{5.806115in}{6.890224in}}%
\pgfpathlineto{\pgfqpoint{5.806115in}{6.993061in}}%
\pgfpathlineto{\pgfqpoint{5.800000in}{6.993061in}}%
\pgfpathclose%
\pgfusepath{fill}%
\end{pgfscope}%
\begin{pgfscope}%
\pgfpathrectangle{\pgfqpoint{5.800000in}{0.720000in}}{\pgfqpoint{1.400000in}{4.620000in}}%
\pgfusepath{clip}%
\pgfsetbuttcap%
\pgfsetmiterjoin%
\definecolor{currentfill}{rgb}{0.121569,0.466667,0.705882}%
\pgfsetfillcolor{currentfill}%
\pgfsetlinewidth{0.000000pt}%
\definecolor{currentstroke}{rgb}{0.000000,0.000000,0.000000}%
\pgfsetstrokecolor{currentstroke}%
\pgfsetstrokeopacity{0.000000}%
\pgfsetdash{}{0pt}%
\pgfpathmoveto{\pgfqpoint{5.800000in}{6.993061in}}%
\pgfpathlineto{\pgfqpoint{5.802038in}{6.993061in}}%
\pgfpathlineto{\pgfqpoint{5.802038in}{7.095898in}}%
\pgfpathlineto{\pgfqpoint{5.800000in}{7.095898in}}%
\pgfpathclose%
\pgfusepath{fill}%
\end{pgfscope}%
\begin{pgfscope}%
\pgfpathrectangle{\pgfqpoint{5.800000in}{0.720000in}}{\pgfqpoint{1.400000in}{4.620000in}}%
\pgfusepath{clip}%
\pgfsetbuttcap%
\pgfsetmiterjoin%
\definecolor{currentfill}{rgb}{0.121569,0.466667,0.705882}%
\pgfsetfillcolor{currentfill}%
\pgfsetlinewidth{0.000000pt}%
\definecolor{currentstroke}{rgb}{0.000000,0.000000,0.000000}%
\pgfsetstrokecolor{currentstroke}%
\pgfsetstrokeopacity{0.000000}%
\pgfsetdash{}{0pt}%
\pgfpathmoveto{\pgfqpoint{5.800000in}{7.095898in}}%
\pgfpathlineto{\pgfqpoint{5.806115in}{7.095898in}}%
\pgfpathlineto{\pgfqpoint{5.806115in}{7.198735in}}%
\pgfpathlineto{\pgfqpoint{5.800000in}{7.198735in}}%
\pgfpathclose%
\pgfusepath{fill}%
\end{pgfscope}%
\begin{pgfscope}%
\pgfpathrectangle{\pgfqpoint{5.800000in}{0.720000in}}{\pgfqpoint{1.400000in}{4.620000in}}%
\pgfusepath{clip}%
\pgfsetbuttcap%
\pgfsetmiterjoin%
\definecolor{currentfill}{rgb}{0.121569,0.466667,0.705882}%
\pgfsetfillcolor{currentfill}%
\pgfsetlinewidth{0.000000pt}%
\definecolor{currentstroke}{rgb}{0.000000,0.000000,0.000000}%
\pgfsetstrokecolor{currentstroke}%
\pgfsetstrokeopacity{0.000000}%
\pgfsetdash{}{0pt}%
\pgfpathmoveto{\pgfqpoint{5.800000in}{7.198735in}}%
\pgfpathlineto{\pgfqpoint{5.804077in}{7.198735in}}%
\pgfpathlineto{\pgfqpoint{5.804077in}{7.301572in}}%
\pgfpathlineto{\pgfqpoint{5.800000in}{7.301572in}}%
\pgfpathclose%
\pgfusepath{fill}%
\end{pgfscope}%
\begin{pgfscope}%
\pgfpathrectangle{\pgfqpoint{5.800000in}{0.720000in}}{\pgfqpoint{1.400000in}{4.620000in}}%
\pgfusepath{clip}%
\pgfsetbuttcap%
\pgfsetmiterjoin%
\definecolor{currentfill}{rgb}{0.121569,0.466667,0.705882}%
\pgfsetfillcolor{currentfill}%
\pgfsetlinewidth{0.000000pt}%
\definecolor{currentstroke}{rgb}{0.000000,0.000000,0.000000}%
\pgfsetstrokecolor{currentstroke}%
\pgfsetstrokeopacity{0.000000}%
\pgfsetdash{}{0pt}%
\pgfpathmoveto{\pgfqpoint{5.800000in}{7.301572in}}%
\pgfpathlineto{\pgfqpoint{5.804077in}{7.301572in}}%
\pgfpathlineto{\pgfqpoint{5.804077in}{7.404409in}}%
\pgfpathlineto{\pgfqpoint{5.800000in}{7.404409in}}%
\pgfpathclose%
\pgfusepath{fill}%
\end{pgfscope}%
\begin{pgfscope}%
\pgfpathrectangle{\pgfqpoint{5.800000in}{0.720000in}}{\pgfqpoint{1.400000in}{4.620000in}}%
\pgfusepath{clip}%
\pgfsetbuttcap%
\pgfsetmiterjoin%
\definecolor{currentfill}{rgb}{0.121569,0.466667,0.705882}%
\pgfsetfillcolor{currentfill}%
\pgfsetlinewidth{0.000000pt}%
\definecolor{currentstroke}{rgb}{0.000000,0.000000,0.000000}%
\pgfsetstrokecolor{currentstroke}%
\pgfsetstrokeopacity{0.000000}%
\pgfsetdash{}{0pt}%
\pgfpathmoveto{\pgfqpoint{5.800000in}{7.404409in}}%
\pgfpathlineto{\pgfqpoint{5.800000in}{7.404409in}}%
\pgfpathlineto{\pgfqpoint{5.800000in}{7.507246in}}%
\pgfpathlineto{\pgfqpoint{5.800000in}{7.507246in}}%
\pgfpathclose%
\pgfusepath{fill}%
\end{pgfscope}%
\begin{pgfscope}%
\pgfpathrectangle{\pgfqpoint{5.800000in}{0.720000in}}{\pgfqpoint{1.400000in}{4.620000in}}%
\pgfusepath{clip}%
\pgfsetbuttcap%
\pgfsetmiterjoin%
\definecolor{currentfill}{rgb}{0.121569,0.466667,0.705882}%
\pgfsetfillcolor{currentfill}%
\pgfsetlinewidth{0.000000pt}%
\definecolor{currentstroke}{rgb}{0.000000,0.000000,0.000000}%
\pgfsetstrokecolor{currentstroke}%
\pgfsetstrokeopacity{0.000000}%
\pgfsetdash{}{0pt}%
\pgfpathmoveto{\pgfqpoint{5.800000in}{7.507246in}}%
\pgfpathlineto{\pgfqpoint{5.800000in}{7.507246in}}%
\pgfpathlineto{\pgfqpoint{5.800000in}{7.610084in}}%
\pgfpathlineto{\pgfqpoint{5.800000in}{7.610084in}}%
\pgfpathclose%
\pgfusepath{fill}%
\end{pgfscope}%
\begin{pgfscope}%
\pgfpathrectangle{\pgfqpoint{5.800000in}{0.720000in}}{\pgfqpoint{1.400000in}{4.620000in}}%
\pgfusepath{clip}%
\pgfsetbuttcap%
\pgfsetmiterjoin%
\definecolor{currentfill}{rgb}{0.121569,0.466667,0.705882}%
\pgfsetfillcolor{currentfill}%
\pgfsetlinewidth{0.000000pt}%
\definecolor{currentstroke}{rgb}{0.000000,0.000000,0.000000}%
\pgfsetstrokecolor{currentstroke}%
\pgfsetstrokeopacity{0.000000}%
\pgfsetdash{}{0pt}%
\pgfpathmoveto{\pgfqpoint{5.800000in}{7.610084in}}%
\pgfpathlineto{\pgfqpoint{5.800000in}{7.610084in}}%
\pgfpathlineto{\pgfqpoint{5.800000in}{7.712921in}}%
\pgfpathlineto{\pgfqpoint{5.800000in}{7.712921in}}%
\pgfpathclose%
\pgfusepath{fill}%
\end{pgfscope}%
\begin{pgfscope}%
\pgfpathrectangle{\pgfqpoint{5.800000in}{0.720000in}}{\pgfqpoint{1.400000in}{4.620000in}}%
\pgfusepath{clip}%
\pgfsetbuttcap%
\pgfsetmiterjoin%
\definecolor{currentfill}{rgb}{0.121569,0.466667,0.705882}%
\pgfsetfillcolor{currentfill}%
\pgfsetlinewidth{0.000000pt}%
\definecolor{currentstroke}{rgb}{0.000000,0.000000,0.000000}%
\pgfsetstrokecolor{currentstroke}%
\pgfsetstrokeopacity{0.000000}%
\pgfsetdash{}{0pt}%
\pgfpathmoveto{\pgfqpoint{5.800000in}{7.712921in}}%
\pgfpathlineto{\pgfqpoint{5.804077in}{7.712921in}}%
\pgfpathlineto{\pgfqpoint{5.804077in}{7.815758in}}%
\pgfpathlineto{\pgfqpoint{5.800000in}{7.815758in}}%
\pgfpathclose%
\pgfusepath{fill}%
\end{pgfscope}%
\begin{pgfscope}%
\pgfpathrectangle{\pgfqpoint{5.800000in}{0.720000in}}{\pgfqpoint{1.400000in}{4.620000in}}%
\pgfusepath{clip}%
\pgfsetbuttcap%
\pgfsetmiterjoin%
\definecolor{currentfill}{rgb}{0.121569,0.466667,0.705882}%
\pgfsetfillcolor{currentfill}%
\pgfsetlinewidth{0.000000pt}%
\definecolor{currentstroke}{rgb}{0.000000,0.000000,0.000000}%
\pgfsetstrokecolor{currentstroke}%
\pgfsetstrokeopacity{0.000000}%
\pgfsetdash{}{0pt}%
\pgfpathmoveto{\pgfqpoint{5.800000in}{7.815758in}}%
\pgfpathlineto{\pgfqpoint{5.802038in}{7.815758in}}%
\pgfpathlineto{\pgfqpoint{5.802038in}{7.918595in}}%
\pgfpathlineto{\pgfqpoint{5.800000in}{7.918595in}}%
\pgfpathclose%
\pgfusepath{fill}%
\end{pgfscope}%
\begin{pgfscope}%
\pgfpathrectangle{\pgfqpoint{5.800000in}{0.720000in}}{\pgfqpoint{1.400000in}{4.620000in}}%
\pgfusepath{clip}%
\pgfsetbuttcap%
\pgfsetmiterjoin%
\definecolor{currentfill}{rgb}{0.121569,0.466667,0.705882}%
\pgfsetfillcolor{currentfill}%
\pgfsetlinewidth{0.000000pt}%
\definecolor{currentstroke}{rgb}{0.000000,0.000000,0.000000}%
\pgfsetstrokecolor{currentstroke}%
\pgfsetstrokeopacity{0.000000}%
\pgfsetdash{}{0pt}%
\pgfpathmoveto{\pgfqpoint{5.800000in}{7.918595in}}%
\pgfpathlineto{\pgfqpoint{5.804077in}{7.918595in}}%
\pgfpathlineto{\pgfqpoint{5.804077in}{8.021432in}}%
\pgfpathlineto{\pgfqpoint{5.800000in}{8.021432in}}%
\pgfpathclose%
\pgfusepath{fill}%
\end{pgfscope}%
\begin{pgfscope}%
\pgfpathrectangle{\pgfqpoint{5.800000in}{0.720000in}}{\pgfqpoint{1.400000in}{4.620000in}}%
\pgfusepath{clip}%
\pgfsetbuttcap%
\pgfsetmiterjoin%
\definecolor{currentfill}{rgb}{0.121569,0.466667,0.705882}%
\pgfsetfillcolor{currentfill}%
\pgfsetlinewidth{0.000000pt}%
\definecolor{currentstroke}{rgb}{0.000000,0.000000,0.000000}%
\pgfsetstrokecolor{currentstroke}%
\pgfsetstrokeopacity{0.000000}%
\pgfsetdash{}{0pt}%
\pgfpathmoveto{\pgfqpoint{5.800000in}{8.021432in}}%
\pgfpathlineto{\pgfqpoint{5.800000in}{8.021432in}}%
\pgfpathlineto{\pgfqpoint{5.800000in}{8.124269in}}%
\pgfpathlineto{\pgfqpoint{5.800000in}{8.124269in}}%
\pgfpathclose%
\pgfusepath{fill}%
\end{pgfscope}%
\begin{pgfscope}%
\pgfpathrectangle{\pgfqpoint{5.800000in}{0.720000in}}{\pgfqpoint{1.400000in}{4.620000in}}%
\pgfusepath{clip}%
\pgfsetbuttcap%
\pgfsetmiterjoin%
\definecolor{currentfill}{rgb}{0.121569,0.466667,0.705882}%
\pgfsetfillcolor{currentfill}%
\pgfsetlinewidth{0.000000pt}%
\definecolor{currentstroke}{rgb}{0.000000,0.000000,0.000000}%
\pgfsetstrokecolor{currentstroke}%
\pgfsetstrokeopacity{0.000000}%
\pgfsetdash{}{0pt}%
\pgfpathmoveto{\pgfqpoint{5.800000in}{8.124269in}}%
\pgfpathlineto{\pgfqpoint{5.800000in}{8.124269in}}%
\pgfpathlineto{\pgfqpoint{5.800000in}{8.227106in}}%
\pgfpathlineto{\pgfqpoint{5.800000in}{8.227106in}}%
\pgfpathclose%
\pgfusepath{fill}%
\end{pgfscope}%
\begin{pgfscope}%
\pgfpathrectangle{\pgfqpoint{5.800000in}{0.720000in}}{\pgfqpoint{1.400000in}{4.620000in}}%
\pgfusepath{clip}%
\pgfsetbuttcap%
\pgfsetmiterjoin%
\definecolor{currentfill}{rgb}{0.121569,0.466667,0.705882}%
\pgfsetfillcolor{currentfill}%
\pgfsetlinewidth{0.000000pt}%
\definecolor{currentstroke}{rgb}{0.000000,0.000000,0.000000}%
\pgfsetstrokecolor{currentstroke}%
\pgfsetstrokeopacity{0.000000}%
\pgfsetdash{}{0pt}%
\pgfpathmoveto{\pgfqpoint{5.800000in}{8.227106in}}%
\pgfpathlineto{\pgfqpoint{5.800000in}{8.227106in}}%
\pgfpathlineto{\pgfqpoint{5.800000in}{8.329943in}}%
\pgfpathlineto{\pgfqpoint{5.800000in}{8.329943in}}%
\pgfpathclose%
\pgfusepath{fill}%
\end{pgfscope}%
\begin{pgfscope}%
\pgfpathrectangle{\pgfqpoint{5.800000in}{0.720000in}}{\pgfqpoint{1.400000in}{4.620000in}}%
\pgfusepath{clip}%
\pgfsetbuttcap%
\pgfsetmiterjoin%
\definecolor{currentfill}{rgb}{0.121569,0.466667,0.705882}%
\pgfsetfillcolor{currentfill}%
\pgfsetlinewidth{0.000000pt}%
\definecolor{currentstroke}{rgb}{0.000000,0.000000,0.000000}%
\pgfsetstrokecolor{currentstroke}%
\pgfsetstrokeopacity{0.000000}%
\pgfsetdash{}{0pt}%
\pgfpathmoveto{\pgfqpoint{5.800000in}{8.329943in}}%
\pgfpathlineto{\pgfqpoint{5.800000in}{8.329943in}}%
\pgfpathlineto{\pgfqpoint{5.800000in}{8.432780in}}%
\pgfpathlineto{\pgfqpoint{5.800000in}{8.432780in}}%
\pgfpathclose%
\pgfusepath{fill}%
\end{pgfscope}%
\begin{pgfscope}%
\pgfpathrectangle{\pgfqpoint{5.800000in}{0.720000in}}{\pgfqpoint{1.400000in}{4.620000in}}%
\pgfusepath{clip}%
\pgfsetbuttcap%
\pgfsetmiterjoin%
\definecolor{currentfill}{rgb}{0.121569,0.466667,0.705882}%
\pgfsetfillcolor{currentfill}%
\pgfsetlinewidth{0.000000pt}%
\definecolor{currentstroke}{rgb}{0.000000,0.000000,0.000000}%
\pgfsetstrokecolor{currentstroke}%
\pgfsetstrokeopacity{0.000000}%
\pgfsetdash{}{0pt}%
\pgfpathmoveto{\pgfqpoint{5.800000in}{8.432780in}}%
\pgfpathlineto{\pgfqpoint{5.802038in}{8.432780in}}%
\pgfpathlineto{\pgfqpoint{5.802038in}{8.535617in}}%
\pgfpathlineto{\pgfqpoint{5.800000in}{8.535617in}}%
\pgfpathclose%
\pgfusepath{fill}%
\end{pgfscope}%
\begin{pgfscope}%
\pgfpathrectangle{\pgfqpoint{5.800000in}{0.720000in}}{\pgfqpoint{1.400000in}{4.620000in}}%
\pgfusepath{clip}%
\pgfsetbuttcap%
\pgfsetmiterjoin%
\definecolor{currentfill}{rgb}{0.121569,0.466667,0.705882}%
\pgfsetfillcolor{currentfill}%
\pgfsetlinewidth{0.000000pt}%
\definecolor{currentstroke}{rgb}{0.000000,0.000000,0.000000}%
\pgfsetstrokecolor{currentstroke}%
\pgfsetstrokeopacity{0.000000}%
\pgfsetdash{}{0pt}%
\pgfpathmoveto{\pgfqpoint{5.800000in}{8.535617in}}%
\pgfpathlineto{\pgfqpoint{5.800000in}{8.535617in}}%
\pgfpathlineto{\pgfqpoint{5.800000in}{8.638454in}}%
\pgfpathlineto{\pgfqpoint{5.800000in}{8.638454in}}%
\pgfpathclose%
\pgfusepath{fill}%
\end{pgfscope}%
\begin{pgfscope}%
\pgfpathrectangle{\pgfqpoint{5.800000in}{0.720000in}}{\pgfqpoint{1.400000in}{4.620000in}}%
\pgfusepath{clip}%
\pgfsetbuttcap%
\pgfsetmiterjoin%
\definecolor{currentfill}{rgb}{0.121569,0.466667,0.705882}%
\pgfsetfillcolor{currentfill}%
\pgfsetlinewidth{0.000000pt}%
\definecolor{currentstroke}{rgb}{0.000000,0.000000,0.000000}%
\pgfsetstrokecolor{currentstroke}%
\pgfsetstrokeopacity{0.000000}%
\pgfsetdash{}{0pt}%
\pgfpathmoveto{\pgfqpoint{5.800000in}{8.638454in}}%
\pgfpathlineto{\pgfqpoint{5.800000in}{8.638454in}}%
\pgfpathlineto{\pgfqpoint{5.800000in}{8.741291in}}%
\pgfpathlineto{\pgfqpoint{5.800000in}{8.741291in}}%
\pgfpathclose%
\pgfusepath{fill}%
\end{pgfscope}%
\begin{pgfscope}%
\pgfpathrectangle{\pgfqpoint{5.800000in}{0.720000in}}{\pgfqpoint{1.400000in}{4.620000in}}%
\pgfusepath{clip}%
\pgfsetbuttcap%
\pgfsetmiterjoin%
\definecolor{currentfill}{rgb}{0.121569,0.466667,0.705882}%
\pgfsetfillcolor{currentfill}%
\pgfsetlinewidth{0.000000pt}%
\definecolor{currentstroke}{rgb}{0.000000,0.000000,0.000000}%
\pgfsetstrokecolor{currentstroke}%
\pgfsetstrokeopacity{0.000000}%
\pgfsetdash{}{0pt}%
\pgfpathmoveto{\pgfqpoint{5.800000in}{8.741291in}}%
\pgfpathlineto{\pgfqpoint{5.800000in}{8.741291in}}%
\pgfpathlineto{\pgfqpoint{5.800000in}{8.844128in}}%
\pgfpathlineto{\pgfqpoint{5.800000in}{8.844128in}}%
\pgfpathclose%
\pgfusepath{fill}%
\end{pgfscope}%
\begin{pgfscope}%
\pgfpathrectangle{\pgfqpoint{5.800000in}{0.720000in}}{\pgfqpoint{1.400000in}{4.620000in}}%
\pgfusepath{clip}%
\pgfsetbuttcap%
\pgfsetmiterjoin%
\definecolor{currentfill}{rgb}{0.121569,0.466667,0.705882}%
\pgfsetfillcolor{currentfill}%
\pgfsetlinewidth{0.000000pt}%
\definecolor{currentstroke}{rgb}{0.000000,0.000000,0.000000}%
\pgfsetstrokecolor{currentstroke}%
\pgfsetstrokeopacity{0.000000}%
\pgfsetdash{}{0pt}%
\pgfpathmoveto{\pgfqpoint{5.800000in}{8.844128in}}%
\pgfpathlineto{\pgfqpoint{5.804077in}{8.844128in}}%
\pgfpathlineto{\pgfqpoint{5.804077in}{8.946965in}}%
\pgfpathlineto{\pgfqpoint{5.800000in}{8.946965in}}%
\pgfpathclose%
\pgfusepath{fill}%
\end{pgfscope}%
\begin{pgfscope}%
\pgfpathrectangle{\pgfqpoint{5.800000in}{0.720000in}}{\pgfqpoint{1.400000in}{4.620000in}}%
\pgfusepath{clip}%
\pgfsetbuttcap%
\pgfsetmiterjoin%
\definecolor{currentfill}{rgb}{0.121569,0.466667,0.705882}%
\pgfsetfillcolor{currentfill}%
\pgfsetlinewidth{0.000000pt}%
\definecolor{currentstroke}{rgb}{0.000000,0.000000,0.000000}%
\pgfsetstrokecolor{currentstroke}%
\pgfsetstrokeopacity{0.000000}%
\pgfsetdash{}{0pt}%
\pgfpathmoveto{\pgfqpoint{5.800000in}{8.946965in}}%
\pgfpathlineto{\pgfqpoint{5.800000in}{8.946965in}}%
\pgfpathlineto{\pgfqpoint{5.800000in}{9.049802in}}%
\pgfpathlineto{\pgfqpoint{5.800000in}{9.049802in}}%
\pgfpathclose%
\pgfusepath{fill}%
\end{pgfscope}%
\begin{pgfscope}%
\pgfpathrectangle{\pgfqpoint{5.800000in}{0.720000in}}{\pgfqpoint{1.400000in}{4.620000in}}%
\pgfusepath{clip}%
\pgfsetbuttcap%
\pgfsetmiterjoin%
\definecolor{currentfill}{rgb}{0.121569,0.466667,0.705882}%
\pgfsetfillcolor{currentfill}%
\pgfsetlinewidth{0.000000pt}%
\definecolor{currentstroke}{rgb}{0.000000,0.000000,0.000000}%
\pgfsetstrokecolor{currentstroke}%
\pgfsetstrokeopacity{0.000000}%
\pgfsetdash{}{0pt}%
\pgfpathmoveto{\pgfqpoint{5.800000in}{9.049802in}}%
\pgfpathlineto{\pgfqpoint{5.802038in}{9.049802in}}%
\pgfpathlineto{\pgfqpoint{5.802038in}{9.152640in}}%
\pgfpathlineto{\pgfqpoint{5.800000in}{9.152640in}}%
\pgfpathclose%
\pgfusepath{fill}%
\end{pgfscope}%
\begin{pgfscope}%
\pgfpathrectangle{\pgfqpoint{5.800000in}{0.720000in}}{\pgfqpoint{1.400000in}{4.620000in}}%
\pgfusepath{clip}%
\pgfsetbuttcap%
\pgfsetmiterjoin%
\definecolor{currentfill}{rgb}{0.121569,0.466667,0.705882}%
\pgfsetfillcolor{currentfill}%
\pgfsetlinewidth{0.000000pt}%
\definecolor{currentstroke}{rgb}{0.000000,0.000000,0.000000}%
\pgfsetstrokecolor{currentstroke}%
\pgfsetstrokeopacity{0.000000}%
\pgfsetdash{}{0pt}%
\pgfpathmoveto{\pgfqpoint{5.800000in}{9.152640in}}%
\pgfpathlineto{\pgfqpoint{5.802038in}{9.152640in}}%
\pgfpathlineto{\pgfqpoint{5.802038in}{9.255477in}}%
\pgfpathlineto{\pgfqpoint{5.800000in}{9.255477in}}%
\pgfpathclose%
\pgfusepath{fill}%
\end{pgfscope}%
\begin{pgfscope}%
\pgfpathrectangle{\pgfqpoint{5.800000in}{0.720000in}}{\pgfqpoint{1.400000in}{4.620000in}}%
\pgfusepath{clip}%
\pgfsetbuttcap%
\pgfsetmiterjoin%
\definecolor{currentfill}{rgb}{0.121569,0.466667,0.705882}%
\pgfsetfillcolor{currentfill}%
\pgfsetlinewidth{0.000000pt}%
\definecolor{currentstroke}{rgb}{0.000000,0.000000,0.000000}%
\pgfsetstrokecolor{currentstroke}%
\pgfsetstrokeopacity{0.000000}%
\pgfsetdash{}{0pt}%
\pgfpathmoveto{\pgfqpoint{5.800000in}{9.255477in}}%
\pgfpathlineto{\pgfqpoint{5.800000in}{9.255477in}}%
\pgfpathlineto{\pgfqpoint{5.800000in}{9.358314in}}%
\pgfpathlineto{\pgfqpoint{5.800000in}{9.358314in}}%
\pgfpathclose%
\pgfusepath{fill}%
\end{pgfscope}%
\begin{pgfscope}%
\pgfpathrectangle{\pgfqpoint{5.800000in}{0.720000in}}{\pgfqpoint{1.400000in}{4.620000in}}%
\pgfusepath{clip}%
\pgfsetbuttcap%
\pgfsetmiterjoin%
\definecolor{currentfill}{rgb}{0.121569,0.466667,0.705882}%
\pgfsetfillcolor{currentfill}%
\pgfsetlinewidth{0.000000pt}%
\definecolor{currentstroke}{rgb}{0.000000,0.000000,0.000000}%
\pgfsetstrokecolor{currentstroke}%
\pgfsetstrokeopacity{0.000000}%
\pgfsetdash{}{0pt}%
\pgfpathmoveto{\pgfqpoint{5.800000in}{9.358314in}}%
\pgfpathlineto{\pgfqpoint{5.800000in}{9.358314in}}%
\pgfpathlineto{\pgfqpoint{5.800000in}{9.461151in}}%
\pgfpathlineto{\pgfqpoint{5.800000in}{9.461151in}}%
\pgfpathclose%
\pgfusepath{fill}%
\end{pgfscope}%
\begin{pgfscope}%
\pgfpathrectangle{\pgfqpoint{5.800000in}{0.720000in}}{\pgfqpoint{1.400000in}{4.620000in}}%
\pgfusepath{clip}%
\pgfsetbuttcap%
\pgfsetmiterjoin%
\definecolor{currentfill}{rgb}{0.121569,0.466667,0.705882}%
\pgfsetfillcolor{currentfill}%
\pgfsetlinewidth{0.000000pt}%
\definecolor{currentstroke}{rgb}{0.000000,0.000000,0.000000}%
\pgfsetstrokecolor{currentstroke}%
\pgfsetstrokeopacity{0.000000}%
\pgfsetdash{}{0pt}%
\pgfpathmoveto{\pgfqpoint{5.800000in}{9.461151in}}%
\pgfpathlineto{\pgfqpoint{5.800000in}{9.461151in}}%
\pgfpathlineto{\pgfqpoint{5.800000in}{9.563988in}}%
\pgfpathlineto{\pgfqpoint{5.800000in}{9.563988in}}%
\pgfpathclose%
\pgfusepath{fill}%
\end{pgfscope}%
\begin{pgfscope}%
\pgfpathrectangle{\pgfqpoint{5.800000in}{0.720000in}}{\pgfqpoint{1.400000in}{4.620000in}}%
\pgfusepath{clip}%
\pgfsetbuttcap%
\pgfsetmiterjoin%
\definecolor{currentfill}{rgb}{0.121569,0.466667,0.705882}%
\pgfsetfillcolor{currentfill}%
\pgfsetlinewidth{0.000000pt}%
\definecolor{currentstroke}{rgb}{0.000000,0.000000,0.000000}%
\pgfsetstrokecolor{currentstroke}%
\pgfsetstrokeopacity{0.000000}%
\pgfsetdash{}{0pt}%
\pgfpathmoveto{\pgfqpoint{5.800000in}{9.563988in}}%
\pgfpathlineto{\pgfqpoint{5.800000in}{9.563988in}}%
\pgfpathlineto{\pgfqpoint{5.800000in}{9.666825in}}%
\pgfpathlineto{\pgfqpoint{5.800000in}{9.666825in}}%
\pgfpathclose%
\pgfusepath{fill}%
\end{pgfscope}%
\begin{pgfscope}%
\pgfpathrectangle{\pgfqpoint{5.800000in}{0.720000in}}{\pgfqpoint{1.400000in}{4.620000in}}%
\pgfusepath{clip}%
\pgfsetbuttcap%
\pgfsetmiterjoin%
\definecolor{currentfill}{rgb}{0.121569,0.466667,0.705882}%
\pgfsetfillcolor{currentfill}%
\pgfsetlinewidth{0.000000pt}%
\definecolor{currentstroke}{rgb}{0.000000,0.000000,0.000000}%
\pgfsetstrokecolor{currentstroke}%
\pgfsetstrokeopacity{0.000000}%
\pgfsetdash{}{0pt}%
\pgfpathmoveto{\pgfqpoint{5.800000in}{9.666825in}}%
\pgfpathlineto{\pgfqpoint{5.802038in}{9.666825in}}%
\pgfpathlineto{\pgfqpoint{5.802038in}{9.769662in}}%
\pgfpathlineto{\pgfqpoint{5.800000in}{9.769662in}}%
\pgfpathclose%
\pgfusepath{fill}%
\end{pgfscope}%
\begin{pgfscope}%
\pgfpathrectangle{\pgfqpoint{5.800000in}{0.720000in}}{\pgfqpoint{1.400000in}{4.620000in}}%
\pgfusepath{clip}%
\pgfsetbuttcap%
\pgfsetmiterjoin%
\definecolor{currentfill}{rgb}{0.121569,0.466667,0.705882}%
\pgfsetfillcolor{currentfill}%
\pgfsetlinewidth{0.000000pt}%
\definecolor{currentstroke}{rgb}{0.000000,0.000000,0.000000}%
\pgfsetstrokecolor{currentstroke}%
\pgfsetstrokeopacity{0.000000}%
\pgfsetdash{}{0pt}%
\pgfpathmoveto{\pgfqpoint{5.800000in}{9.769662in}}%
\pgfpathlineto{\pgfqpoint{5.800000in}{9.769662in}}%
\pgfpathlineto{\pgfqpoint{5.800000in}{9.872499in}}%
\pgfpathlineto{\pgfqpoint{5.800000in}{9.872499in}}%
\pgfpathclose%
\pgfusepath{fill}%
\end{pgfscope}%
\begin{pgfscope}%
\pgfpathrectangle{\pgfqpoint{5.800000in}{0.720000in}}{\pgfqpoint{1.400000in}{4.620000in}}%
\pgfusepath{clip}%
\pgfsetbuttcap%
\pgfsetmiterjoin%
\definecolor{currentfill}{rgb}{0.121569,0.466667,0.705882}%
\pgfsetfillcolor{currentfill}%
\pgfsetlinewidth{0.000000pt}%
\definecolor{currentstroke}{rgb}{0.000000,0.000000,0.000000}%
\pgfsetstrokecolor{currentstroke}%
\pgfsetstrokeopacity{0.000000}%
\pgfsetdash{}{0pt}%
\pgfpathmoveto{\pgfqpoint{5.800000in}{9.872499in}}%
\pgfpathlineto{\pgfqpoint{5.800000in}{9.872499in}}%
\pgfpathlineto{\pgfqpoint{5.800000in}{9.975336in}}%
\pgfpathlineto{\pgfqpoint{5.800000in}{9.975336in}}%
\pgfpathclose%
\pgfusepath{fill}%
\end{pgfscope}%
\begin{pgfscope}%
\pgfpathrectangle{\pgfqpoint{5.800000in}{0.720000in}}{\pgfqpoint{1.400000in}{4.620000in}}%
\pgfusepath{clip}%
\pgfsetbuttcap%
\pgfsetmiterjoin%
\definecolor{currentfill}{rgb}{0.121569,0.466667,0.705882}%
\pgfsetfillcolor{currentfill}%
\pgfsetlinewidth{0.000000pt}%
\definecolor{currentstroke}{rgb}{0.000000,0.000000,0.000000}%
\pgfsetstrokecolor{currentstroke}%
\pgfsetstrokeopacity{0.000000}%
\pgfsetdash{}{0pt}%
\pgfpathmoveto{\pgfqpoint{5.800000in}{9.975336in}}%
\pgfpathlineto{\pgfqpoint{5.800000in}{9.975336in}}%
\pgfpathlineto{\pgfqpoint{5.800000in}{10.078173in}}%
\pgfpathlineto{\pgfqpoint{5.800000in}{10.078173in}}%
\pgfpathclose%
\pgfusepath{fill}%
\end{pgfscope}%
\begin{pgfscope}%
\pgfpathrectangle{\pgfqpoint{5.800000in}{0.720000in}}{\pgfqpoint{1.400000in}{4.620000in}}%
\pgfusepath{clip}%
\pgfsetbuttcap%
\pgfsetmiterjoin%
\definecolor{currentfill}{rgb}{0.121569,0.466667,0.705882}%
\pgfsetfillcolor{currentfill}%
\pgfsetlinewidth{0.000000pt}%
\definecolor{currentstroke}{rgb}{0.000000,0.000000,0.000000}%
\pgfsetstrokecolor{currentstroke}%
\pgfsetstrokeopacity{0.000000}%
\pgfsetdash{}{0pt}%
\pgfpathmoveto{\pgfqpoint{5.800000in}{10.078173in}}%
\pgfpathlineto{\pgfqpoint{5.802038in}{10.078173in}}%
\pgfpathlineto{\pgfqpoint{5.802038in}{10.181010in}}%
\pgfpathlineto{\pgfqpoint{5.800000in}{10.181010in}}%
\pgfpathclose%
\pgfusepath{fill}%
\end{pgfscope}%
\begin{pgfscope}%
\pgfpathrectangle{\pgfqpoint{5.800000in}{0.720000in}}{\pgfqpoint{1.400000in}{4.620000in}}%
\pgfusepath{clip}%
\pgfsetbuttcap%
\pgfsetmiterjoin%
\definecolor{currentfill}{rgb}{0.121569,0.466667,0.705882}%
\pgfsetfillcolor{currentfill}%
\pgfsetlinewidth{0.000000pt}%
\definecolor{currentstroke}{rgb}{0.000000,0.000000,0.000000}%
\pgfsetstrokecolor{currentstroke}%
\pgfsetstrokeopacity{0.000000}%
\pgfsetdash{}{0pt}%
\pgfpathmoveto{\pgfqpoint{5.800000in}{10.181010in}}%
\pgfpathlineto{\pgfqpoint{5.800000in}{10.181010in}}%
\pgfpathlineto{\pgfqpoint{5.800000in}{10.283847in}}%
\pgfpathlineto{\pgfqpoint{5.800000in}{10.283847in}}%
\pgfpathclose%
\pgfusepath{fill}%
\end{pgfscope}%
\begin{pgfscope}%
\pgfpathrectangle{\pgfqpoint{5.800000in}{0.720000in}}{\pgfqpoint{1.400000in}{4.620000in}}%
\pgfusepath{clip}%
\pgfsetbuttcap%
\pgfsetmiterjoin%
\definecolor{currentfill}{rgb}{0.121569,0.466667,0.705882}%
\pgfsetfillcolor{currentfill}%
\pgfsetlinewidth{0.000000pt}%
\definecolor{currentstroke}{rgb}{0.000000,0.000000,0.000000}%
\pgfsetstrokecolor{currentstroke}%
\pgfsetstrokeopacity{0.000000}%
\pgfsetdash{}{0pt}%
\pgfpathmoveto{\pgfqpoint{5.800000in}{10.283847in}}%
\pgfpathlineto{\pgfqpoint{5.800000in}{10.283847in}}%
\pgfpathlineto{\pgfqpoint{5.800000in}{10.386684in}}%
\pgfpathlineto{\pgfqpoint{5.800000in}{10.386684in}}%
\pgfpathclose%
\pgfusepath{fill}%
\end{pgfscope}%
\begin{pgfscope}%
\pgfpathrectangle{\pgfqpoint{5.800000in}{0.720000in}}{\pgfqpoint{1.400000in}{4.620000in}}%
\pgfusepath{clip}%
\pgfsetbuttcap%
\pgfsetmiterjoin%
\definecolor{currentfill}{rgb}{0.121569,0.466667,0.705882}%
\pgfsetfillcolor{currentfill}%
\pgfsetlinewidth{0.000000pt}%
\definecolor{currentstroke}{rgb}{0.000000,0.000000,0.000000}%
\pgfsetstrokecolor{currentstroke}%
\pgfsetstrokeopacity{0.000000}%
\pgfsetdash{}{0pt}%
\pgfpathmoveto{\pgfqpoint{5.800000in}{10.386684in}}%
\pgfpathlineto{\pgfqpoint{5.802038in}{10.386684in}}%
\pgfpathlineto{\pgfqpoint{5.802038in}{10.489521in}}%
\pgfpathlineto{\pgfqpoint{5.800000in}{10.489521in}}%
\pgfpathclose%
\pgfusepath{fill}%
\end{pgfscope}%
\begin{pgfscope}%
\pgfpathrectangle{\pgfqpoint{5.800000in}{0.720000in}}{\pgfqpoint{1.400000in}{4.620000in}}%
\pgfusepath{clip}%
\pgfsetbuttcap%
\pgfsetmiterjoin%
\definecolor{currentfill}{rgb}{0.121569,0.466667,0.705882}%
\pgfsetfillcolor{currentfill}%
\pgfsetlinewidth{0.000000pt}%
\definecolor{currentstroke}{rgb}{0.000000,0.000000,0.000000}%
\pgfsetstrokecolor{currentstroke}%
\pgfsetstrokeopacity{0.000000}%
\pgfsetdash{}{0pt}%
\pgfpathmoveto{\pgfqpoint{5.800000in}{10.489521in}}%
\pgfpathlineto{\pgfqpoint{5.800000in}{10.489521in}}%
\pgfpathlineto{\pgfqpoint{5.800000in}{10.592359in}}%
\pgfpathlineto{\pgfqpoint{5.800000in}{10.592359in}}%
\pgfpathclose%
\pgfusepath{fill}%
\end{pgfscope}%
\begin{pgfscope}%
\pgfpathrectangle{\pgfqpoint{5.800000in}{0.720000in}}{\pgfqpoint{1.400000in}{4.620000in}}%
\pgfusepath{clip}%
\pgfsetbuttcap%
\pgfsetmiterjoin%
\definecolor{currentfill}{rgb}{0.121569,0.466667,0.705882}%
\pgfsetfillcolor{currentfill}%
\pgfsetlinewidth{0.000000pt}%
\definecolor{currentstroke}{rgb}{0.000000,0.000000,0.000000}%
\pgfsetstrokecolor{currentstroke}%
\pgfsetstrokeopacity{0.000000}%
\pgfsetdash{}{0pt}%
\pgfpathmoveto{\pgfqpoint{5.800000in}{10.592359in}}%
\pgfpathlineto{\pgfqpoint{5.800000in}{10.592359in}}%
\pgfpathlineto{\pgfqpoint{5.800000in}{10.695196in}}%
\pgfpathlineto{\pgfqpoint{5.800000in}{10.695196in}}%
\pgfpathclose%
\pgfusepath{fill}%
\end{pgfscope}%
\begin{pgfscope}%
\pgfpathrectangle{\pgfqpoint{5.800000in}{0.720000in}}{\pgfqpoint{1.400000in}{4.620000in}}%
\pgfusepath{clip}%
\pgfsetbuttcap%
\pgfsetmiterjoin%
\definecolor{currentfill}{rgb}{0.121569,0.466667,0.705882}%
\pgfsetfillcolor{currentfill}%
\pgfsetlinewidth{0.000000pt}%
\definecolor{currentstroke}{rgb}{0.000000,0.000000,0.000000}%
\pgfsetstrokecolor{currentstroke}%
\pgfsetstrokeopacity{0.000000}%
\pgfsetdash{}{0pt}%
\pgfpathmoveto{\pgfqpoint{5.800000in}{10.695196in}}%
\pgfpathlineto{\pgfqpoint{5.800000in}{10.695196in}}%
\pgfpathlineto{\pgfqpoint{5.800000in}{10.798033in}}%
\pgfpathlineto{\pgfqpoint{5.800000in}{10.798033in}}%
\pgfpathclose%
\pgfusepath{fill}%
\end{pgfscope}%
\begin{pgfscope}%
\pgfpathrectangle{\pgfqpoint{5.800000in}{0.720000in}}{\pgfqpoint{1.400000in}{4.620000in}}%
\pgfusepath{clip}%
\pgfsetbuttcap%
\pgfsetmiterjoin%
\definecolor{currentfill}{rgb}{0.121569,0.466667,0.705882}%
\pgfsetfillcolor{currentfill}%
\pgfsetlinewidth{0.000000pt}%
\definecolor{currentstroke}{rgb}{0.000000,0.000000,0.000000}%
\pgfsetstrokecolor{currentstroke}%
\pgfsetstrokeopacity{0.000000}%
\pgfsetdash{}{0pt}%
\pgfpathmoveto{\pgfqpoint{5.800000in}{10.798033in}}%
\pgfpathlineto{\pgfqpoint{5.800000in}{10.798033in}}%
\pgfpathlineto{\pgfqpoint{5.800000in}{10.900870in}}%
\pgfpathlineto{\pgfqpoint{5.800000in}{10.900870in}}%
\pgfpathclose%
\pgfusepath{fill}%
\end{pgfscope}%
\begin{pgfscope}%
\pgfpathrectangle{\pgfqpoint{5.800000in}{0.720000in}}{\pgfqpoint{1.400000in}{4.620000in}}%
\pgfusepath{clip}%
\pgfsetbuttcap%
\pgfsetmiterjoin%
\definecolor{currentfill}{rgb}{0.121569,0.466667,0.705882}%
\pgfsetfillcolor{currentfill}%
\pgfsetlinewidth{0.000000pt}%
\definecolor{currentstroke}{rgb}{0.000000,0.000000,0.000000}%
\pgfsetstrokecolor{currentstroke}%
\pgfsetstrokeopacity{0.000000}%
\pgfsetdash{}{0pt}%
\pgfpathmoveto{\pgfqpoint{5.800000in}{10.900870in}}%
\pgfpathlineto{\pgfqpoint{5.800000in}{10.900870in}}%
\pgfpathlineto{\pgfqpoint{5.800000in}{11.003707in}}%
\pgfpathlineto{\pgfqpoint{5.800000in}{11.003707in}}%
\pgfpathclose%
\pgfusepath{fill}%
\end{pgfscope}%
\begin{pgfscope}%
\pgfpathrectangle{\pgfqpoint{5.800000in}{0.720000in}}{\pgfqpoint{1.400000in}{4.620000in}}%
\pgfusepath{clip}%
\pgfsetbuttcap%
\pgfsetmiterjoin%
\definecolor{currentfill}{rgb}{0.121569,0.466667,0.705882}%
\pgfsetfillcolor{currentfill}%
\pgfsetlinewidth{0.000000pt}%
\definecolor{currentstroke}{rgb}{0.000000,0.000000,0.000000}%
\pgfsetstrokecolor{currentstroke}%
\pgfsetstrokeopacity{0.000000}%
\pgfsetdash{}{0pt}%
\pgfpathmoveto{\pgfqpoint{5.800000in}{11.003707in}}%
\pgfpathlineto{\pgfqpoint{5.800000in}{11.003707in}}%
\pgfpathlineto{\pgfqpoint{5.800000in}{11.106544in}}%
\pgfpathlineto{\pgfqpoint{5.800000in}{11.106544in}}%
\pgfpathclose%
\pgfusepath{fill}%
\end{pgfscope}%
\begin{pgfscope}%
\pgfpathrectangle{\pgfqpoint{5.800000in}{0.720000in}}{\pgfqpoint{1.400000in}{4.620000in}}%
\pgfusepath{clip}%
\pgfsetbuttcap%
\pgfsetmiterjoin%
\definecolor{currentfill}{rgb}{0.121569,0.466667,0.705882}%
\pgfsetfillcolor{currentfill}%
\pgfsetlinewidth{0.000000pt}%
\definecolor{currentstroke}{rgb}{0.000000,0.000000,0.000000}%
\pgfsetstrokecolor{currentstroke}%
\pgfsetstrokeopacity{0.000000}%
\pgfsetdash{}{0pt}%
\pgfpathmoveto{\pgfqpoint{5.800000in}{11.106544in}}%
\pgfpathlineto{\pgfqpoint{5.800000in}{11.106544in}}%
\pgfpathlineto{\pgfqpoint{5.800000in}{11.209381in}}%
\pgfpathlineto{\pgfqpoint{5.800000in}{11.209381in}}%
\pgfpathclose%
\pgfusepath{fill}%
\end{pgfscope}%
\begin{pgfscope}%
\pgfpathrectangle{\pgfqpoint{5.800000in}{0.720000in}}{\pgfqpoint{1.400000in}{4.620000in}}%
\pgfusepath{clip}%
\pgfsetbuttcap%
\pgfsetmiterjoin%
\definecolor{currentfill}{rgb}{0.121569,0.466667,0.705882}%
\pgfsetfillcolor{currentfill}%
\pgfsetlinewidth{0.000000pt}%
\definecolor{currentstroke}{rgb}{0.000000,0.000000,0.000000}%
\pgfsetstrokecolor{currentstroke}%
\pgfsetstrokeopacity{0.000000}%
\pgfsetdash{}{0pt}%
\pgfpathmoveto{\pgfqpoint{5.800000in}{11.209381in}}%
\pgfpathlineto{\pgfqpoint{5.800000in}{11.209381in}}%
\pgfpathlineto{\pgfqpoint{5.800000in}{11.312218in}}%
\pgfpathlineto{\pgfqpoint{5.800000in}{11.312218in}}%
\pgfpathclose%
\pgfusepath{fill}%
\end{pgfscope}%
\begin{pgfscope}%
\pgfpathrectangle{\pgfqpoint{5.800000in}{0.720000in}}{\pgfqpoint{1.400000in}{4.620000in}}%
\pgfusepath{clip}%
\pgfsetbuttcap%
\pgfsetmiterjoin%
\definecolor{currentfill}{rgb}{0.121569,0.466667,0.705882}%
\pgfsetfillcolor{currentfill}%
\pgfsetlinewidth{0.000000pt}%
\definecolor{currentstroke}{rgb}{0.000000,0.000000,0.000000}%
\pgfsetstrokecolor{currentstroke}%
\pgfsetstrokeopacity{0.000000}%
\pgfsetdash{}{0pt}%
\pgfpathmoveto{\pgfqpoint{5.800000in}{11.312218in}}%
\pgfpathlineto{\pgfqpoint{5.800000in}{11.312218in}}%
\pgfpathlineto{\pgfqpoint{5.800000in}{11.415055in}}%
\pgfpathlineto{\pgfqpoint{5.800000in}{11.415055in}}%
\pgfpathclose%
\pgfusepath{fill}%
\end{pgfscope}%
\begin{pgfscope}%
\pgfpathrectangle{\pgfqpoint{5.800000in}{0.720000in}}{\pgfqpoint{1.400000in}{4.620000in}}%
\pgfusepath{clip}%
\pgfsetbuttcap%
\pgfsetmiterjoin%
\definecolor{currentfill}{rgb}{0.121569,0.466667,0.705882}%
\pgfsetfillcolor{currentfill}%
\pgfsetlinewidth{0.000000pt}%
\definecolor{currentstroke}{rgb}{0.000000,0.000000,0.000000}%
\pgfsetstrokecolor{currentstroke}%
\pgfsetstrokeopacity{0.000000}%
\pgfsetdash{}{0pt}%
\pgfpathmoveto{\pgfqpoint{5.800000in}{11.415055in}}%
\pgfpathlineto{\pgfqpoint{5.800000in}{11.415055in}}%
\pgfpathlineto{\pgfqpoint{5.800000in}{11.517892in}}%
\pgfpathlineto{\pgfqpoint{5.800000in}{11.517892in}}%
\pgfpathclose%
\pgfusepath{fill}%
\end{pgfscope}%
\begin{pgfscope}%
\pgfpathrectangle{\pgfqpoint{5.800000in}{0.720000in}}{\pgfqpoint{1.400000in}{4.620000in}}%
\pgfusepath{clip}%
\pgfsetbuttcap%
\pgfsetmiterjoin%
\definecolor{currentfill}{rgb}{0.121569,0.466667,0.705882}%
\pgfsetfillcolor{currentfill}%
\pgfsetlinewidth{0.000000pt}%
\definecolor{currentstroke}{rgb}{0.000000,0.000000,0.000000}%
\pgfsetstrokecolor{currentstroke}%
\pgfsetstrokeopacity{0.000000}%
\pgfsetdash{}{0pt}%
\pgfpathmoveto{\pgfqpoint{5.800000in}{11.517892in}}%
\pgfpathlineto{\pgfqpoint{5.800000in}{11.517892in}}%
\pgfpathlineto{\pgfqpoint{5.800000in}{11.620729in}}%
\pgfpathlineto{\pgfqpoint{5.800000in}{11.620729in}}%
\pgfpathclose%
\pgfusepath{fill}%
\end{pgfscope}%
\begin{pgfscope}%
\pgfpathrectangle{\pgfqpoint{5.800000in}{0.720000in}}{\pgfqpoint{1.400000in}{4.620000in}}%
\pgfusepath{clip}%
\pgfsetbuttcap%
\pgfsetmiterjoin%
\definecolor{currentfill}{rgb}{0.121569,0.466667,0.705882}%
\pgfsetfillcolor{currentfill}%
\pgfsetlinewidth{0.000000pt}%
\definecolor{currentstroke}{rgb}{0.000000,0.000000,0.000000}%
\pgfsetstrokecolor{currentstroke}%
\pgfsetstrokeopacity{0.000000}%
\pgfsetdash{}{0pt}%
\pgfpathmoveto{\pgfqpoint{5.800000in}{11.620729in}}%
\pgfpathlineto{\pgfqpoint{5.800000in}{11.620729in}}%
\pgfpathlineto{\pgfqpoint{5.800000in}{11.723566in}}%
\pgfpathlineto{\pgfqpoint{5.800000in}{11.723566in}}%
\pgfpathclose%
\pgfusepath{fill}%
\end{pgfscope}%
\begin{pgfscope}%
\pgfpathrectangle{\pgfqpoint{5.800000in}{0.720000in}}{\pgfqpoint{1.400000in}{4.620000in}}%
\pgfusepath{clip}%
\pgfsetbuttcap%
\pgfsetmiterjoin%
\definecolor{currentfill}{rgb}{0.121569,0.466667,0.705882}%
\pgfsetfillcolor{currentfill}%
\pgfsetlinewidth{0.000000pt}%
\definecolor{currentstroke}{rgb}{0.000000,0.000000,0.000000}%
\pgfsetstrokecolor{currentstroke}%
\pgfsetstrokeopacity{0.000000}%
\pgfsetdash{}{0pt}%
\pgfpathmoveto{\pgfqpoint{5.800000in}{11.723566in}}%
\pgfpathlineto{\pgfqpoint{5.800000in}{11.723566in}}%
\pgfpathlineto{\pgfqpoint{5.800000in}{11.826403in}}%
\pgfpathlineto{\pgfqpoint{5.800000in}{11.826403in}}%
\pgfpathclose%
\pgfusepath{fill}%
\end{pgfscope}%
\begin{pgfscope}%
\pgfpathrectangle{\pgfqpoint{5.800000in}{0.720000in}}{\pgfqpoint{1.400000in}{4.620000in}}%
\pgfusepath{clip}%
\pgfsetbuttcap%
\pgfsetmiterjoin%
\definecolor{currentfill}{rgb}{0.121569,0.466667,0.705882}%
\pgfsetfillcolor{currentfill}%
\pgfsetlinewidth{0.000000pt}%
\definecolor{currentstroke}{rgb}{0.000000,0.000000,0.000000}%
\pgfsetstrokecolor{currentstroke}%
\pgfsetstrokeopacity{0.000000}%
\pgfsetdash{}{0pt}%
\pgfpathmoveto{\pgfqpoint{5.800000in}{11.826403in}}%
\pgfpathlineto{\pgfqpoint{5.800000in}{11.826403in}}%
\pgfpathlineto{\pgfqpoint{5.800000in}{11.929240in}}%
\pgfpathlineto{\pgfqpoint{5.800000in}{11.929240in}}%
\pgfpathclose%
\pgfusepath{fill}%
\end{pgfscope}%
\begin{pgfscope}%
\pgfpathrectangle{\pgfqpoint{5.800000in}{0.720000in}}{\pgfqpoint{1.400000in}{4.620000in}}%
\pgfusepath{clip}%
\pgfsetbuttcap%
\pgfsetmiterjoin%
\definecolor{currentfill}{rgb}{0.121569,0.466667,0.705882}%
\pgfsetfillcolor{currentfill}%
\pgfsetlinewidth{0.000000pt}%
\definecolor{currentstroke}{rgb}{0.000000,0.000000,0.000000}%
\pgfsetstrokecolor{currentstroke}%
\pgfsetstrokeopacity{0.000000}%
\pgfsetdash{}{0pt}%
\pgfpathmoveto{\pgfqpoint{5.800000in}{11.929240in}}%
\pgfpathlineto{\pgfqpoint{5.800000in}{11.929240in}}%
\pgfpathlineto{\pgfqpoint{5.800000in}{12.032077in}}%
\pgfpathlineto{\pgfqpoint{5.800000in}{12.032077in}}%
\pgfpathclose%
\pgfusepath{fill}%
\end{pgfscope}%
\begin{pgfscope}%
\pgfpathrectangle{\pgfqpoint{5.800000in}{0.720000in}}{\pgfqpoint{1.400000in}{4.620000in}}%
\pgfusepath{clip}%
\pgfsetbuttcap%
\pgfsetmiterjoin%
\definecolor{currentfill}{rgb}{0.121569,0.466667,0.705882}%
\pgfsetfillcolor{currentfill}%
\pgfsetlinewidth{0.000000pt}%
\definecolor{currentstroke}{rgb}{0.000000,0.000000,0.000000}%
\pgfsetstrokecolor{currentstroke}%
\pgfsetstrokeopacity{0.000000}%
\pgfsetdash{}{0pt}%
\pgfpathmoveto{\pgfqpoint{5.800000in}{12.032077in}}%
\pgfpathlineto{\pgfqpoint{5.800000in}{12.032077in}}%
\pgfpathlineto{\pgfqpoint{5.800000in}{12.134915in}}%
\pgfpathlineto{\pgfqpoint{5.800000in}{12.134915in}}%
\pgfpathclose%
\pgfusepath{fill}%
\end{pgfscope}%
\begin{pgfscope}%
\pgfpathrectangle{\pgfqpoint{5.800000in}{0.720000in}}{\pgfqpoint{1.400000in}{4.620000in}}%
\pgfusepath{clip}%
\pgfsetbuttcap%
\pgfsetmiterjoin%
\definecolor{currentfill}{rgb}{0.121569,0.466667,0.705882}%
\pgfsetfillcolor{currentfill}%
\pgfsetlinewidth{0.000000pt}%
\definecolor{currentstroke}{rgb}{0.000000,0.000000,0.000000}%
\pgfsetstrokecolor{currentstroke}%
\pgfsetstrokeopacity{0.000000}%
\pgfsetdash{}{0pt}%
\pgfpathmoveto{\pgfqpoint{5.800000in}{12.134915in}}%
\pgfpathlineto{\pgfqpoint{5.800000in}{12.134915in}}%
\pgfpathlineto{\pgfqpoint{5.800000in}{12.237752in}}%
\pgfpathlineto{\pgfqpoint{5.800000in}{12.237752in}}%
\pgfpathclose%
\pgfusepath{fill}%
\end{pgfscope}%
\begin{pgfscope}%
\pgfpathrectangle{\pgfqpoint{5.800000in}{0.720000in}}{\pgfqpoint{1.400000in}{4.620000in}}%
\pgfusepath{clip}%
\pgfsetbuttcap%
\pgfsetmiterjoin%
\definecolor{currentfill}{rgb}{0.121569,0.466667,0.705882}%
\pgfsetfillcolor{currentfill}%
\pgfsetlinewidth{0.000000pt}%
\definecolor{currentstroke}{rgb}{0.000000,0.000000,0.000000}%
\pgfsetstrokecolor{currentstroke}%
\pgfsetstrokeopacity{0.000000}%
\pgfsetdash{}{0pt}%
\pgfpathmoveto{\pgfqpoint{5.800000in}{12.237752in}}%
\pgfpathlineto{\pgfqpoint{5.802038in}{12.237752in}}%
\pgfpathlineto{\pgfqpoint{5.802038in}{12.340589in}}%
\pgfpathlineto{\pgfqpoint{5.800000in}{12.340589in}}%
\pgfpathclose%
\pgfusepath{fill}%
\end{pgfscope}%
\begin{pgfscope}%
\pgfpathrectangle{\pgfqpoint{5.800000in}{0.720000in}}{\pgfqpoint{1.400000in}{4.620000in}}%
\pgfusepath{clip}%
\pgfsetbuttcap%
\pgfsetmiterjoin%
\definecolor{currentfill}{rgb}{0.121569,0.466667,0.705882}%
\pgfsetfillcolor{currentfill}%
\pgfsetlinewidth{0.000000pt}%
\definecolor{currentstroke}{rgb}{0.000000,0.000000,0.000000}%
\pgfsetstrokecolor{currentstroke}%
\pgfsetstrokeopacity{0.000000}%
\pgfsetdash{}{0pt}%
\pgfpathmoveto{\pgfqpoint{5.800000in}{12.340589in}}%
\pgfpathlineto{\pgfqpoint{5.800000in}{12.340589in}}%
\pgfpathlineto{\pgfqpoint{5.800000in}{12.443426in}}%
\pgfpathlineto{\pgfqpoint{5.800000in}{12.443426in}}%
\pgfpathclose%
\pgfusepath{fill}%
\end{pgfscope}%
\begin{pgfscope}%
\pgfpathrectangle{\pgfqpoint{5.800000in}{0.720000in}}{\pgfqpoint{1.400000in}{4.620000in}}%
\pgfusepath{clip}%
\pgfsetbuttcap%
\pgfsetmiterjoin%
\definecolor{currentfill}{rgb}{0.121569,0.466667,0.705882}%
\pgfsetfillcolor{currentfill}%
\pgfsetlinewidth{0.000000pt}%
\definecolor{currentstroke}{rgb}{0.000000,0.000000,0.000000}%
\pgfsetstrokecolor{currentstroke}%
\pgfsetstrokeopacity{0.000000}%
\pgfsetdash{}{0pt}%
\pgfpathmoveto{\pgfqpoint{5.800000in}{12.443426in}}%
\pgfpathlineto{\pgfqpoint{5.800000in}{12.443426in}}%
\pgfpathlineto{\pgfqpoint{5.800000in}{12.546263in}}%
\pgfpathlineto{\pgfqpoint{5.800000in}{12.546263in}}%
\pgfpathclose%
\pgfusepath{fill}%
\end{pgfscope}%
\begin{pgfscope}%
\pgfpathrectangle{\pgfqpoint{5.800000in}{0.720000in}}{\pgfqpoint{1.400000in}{4.620000in}}%
\pgfusepath{clip}%
\pgfsetbuttcap%
\pgfsetmiterjoin%
\definecolor{currentfill}{rgb}{0.121569,0.466667,0.705882}%
\pgfsetfillcolor{currentfill}%
\pgfsetlinewidth{0.000000pt}%
\definecolor{currentstroke}{rgb}{0.000000,0.000000,0.000000}%
\pgfsetstrokecolor{currentstroke}%
\pgfsetstrokeopacity{0.000000}%
\pgfsetdash{}{0pt}%
\pgfpathmoveto{\pgfqpoint{5.800000in}{12.546263in}}%
\pgfpathlineto{\pgfqpoint{5.800000in}{12.546263in}}%
\pgfpathlineto{\pgfqpoint{5.800000in}{12.649100in}}%
\pgfpathlineto{\pgfqpoint{5.800000in}{12.649100in}}%
\pgfpathclose%
\pgfusepath{fill}%
\end{pgfscope}%
\begin{pgfscope}%
\pgfpathrectangle{\pgfqpoint{5.800000in}{0.720000in}}{\pgfqpoint{1.400000in}{4.620000in}}%
\pgfusepath{clip}%
\pgfsetbuttcap%
\pgfsetmiterjoin%
\definecolor{currentfill}{rgb}{0.121569,0.466667,0.705882}%
\pgfsetfillcolor{currentfill}%
\pgfsetlinewidth{0.000000pt}%
\definecolor{currentstroke}{rgb}{0.000000,0.000000,0.000000}%
\pgfsetstrokecolor{currentstroke}%
\pgfsetstrokeopacity{0.000000}%
\pgfsetdash{}{0pt}%
\pgfpathmoveto{\pgfqpoint{5.800000in}{12.649100in}}%
\pgfpathlineto{\pgfqpoint{5.800000in}{12.649100in}}%
\pgfpathlineto{\pgfqpoint{5.800000in}{12.751937in}}%
\pgfpathlineto{\pgfqpoint{5.800000in}{12.751937in}}%
\pgfpathclose%
\pgfusepath{fill}%
\end{pgfscope}%
\begin{pgfscope}%
\pgfpathrectangle{\pgfqpoint{5.800000in}{0.720000in}}{\pgfqpoint{1.400000in}{4.620000in}}%
\pgfusepath{clip}%
\pgfsetbuttcap%
\pgfsetmiterjoin%
\definecolor{currentfill}{rgb}{0.121569,0.466667,0.705882}%
\pgfsetfillcolor{currentfill}%
\pgfsetlinewidth{0.000000pt}%
\definecolor{currentstroke}{rgb}{0.000000,0.000000,0.000000}%
\pgfsetstrokecolor{currentstroke}%
\pgfsetstrokeopacity{0.000000}%
\pgfsetdash{}{0pt}%
\pgfpathmoveto{\pgfqpoint{5.800000in}{12.751937in}}%
\pgfpathlineto{\pgfqpoint{5.800000in}{12.751937in}}%
\pgfpathlineto{\pgfqpoint{5.800000in}{12.854774in}}%
\pgfpathlineto{\pgfqpoint{5.800000in}{12.854774in}}%
\pgfpathclose%
\pgfusepath{fill}%
\end{pgfscope}%
\begin{pgfscope}%
\pgfpathrectangle{\pgfqpoint{5.800000in}{0.720000in}}{\pgfqpoint{1.400000in}{4.620000in}}%
\pgfusepath{clip}%
\pgfsetbuttcap%
\pgfsetmiterjoin%
\definecolor{currentfill}{rgb}{0.121569,0.466667,0.705882}%
\pgfsetfillcolor{currentfill}%
\pgfsetlinewidth{0.000000pt}%
\definecolor{currentstroke}{rgb}{0.000000,0.000000,0.000000}%
\pgfsetstrokecolor{currentstroke}%
\pgfsetstrokeopacity{0.000000}%
\pgfsetdash{}{0pt}%
\pgfpathmoveto{\pgfqpoint{5.800000in}{12.854774in}}%
\pgfpathlineto{\pgfqpoint{5.800000in}{12.854774in}}%
\pgfpathlineto{\pgfqpoint{5.800000in}{12.957611in}}%
\pgfpathlineto{\pgfqpoint{5.800000in}{12.957611in}}%
\pgfpathclose%
\pgfusepath{fill}%
\end{pgfscope}%
\begin{pgfscope}%
\pgfpathrectangle{\pgfqpoint{5.800000in}{0.720000in}}{\pgfqpoint{1.400000in}{4.620000in}}%
\pgfusepath{clip}%
\pgfsetbuttcap%
\pgfsetmiterjoin%
\definecolor{currentfill}{rgb}{0.121569,0.466667,0.705882}%
\pgfsetfillcolor{currentfill}%
\pgfsetlinewidth{0.000000pt}%
\definecolor{currentstroke}{rgb}{0.000000,0.000000,0.000000}%
\pgfsetstrokecolor{currentstroke}%
\pgfsetstrokeopacity{0.000000}%
\pgfsetdash{}{0pt}%
\pgfpathmoveto{\pgfqpoint{5.800000in}{12.957611in}}%
\pgfpathlineto{\pgfqpoint{5.800000in}{12.957611in}}%
\pgfpathlineto{\pgfqpoint{5.800000in}{13.060448in}}%
\pgfpathlineto{\pgfqpoint{5.800000in}{13.060448in}}%
\pgfpathclose%
\pgfusepath{fill}%
\end{pgfscope}%
\begin{pgfscope}%
\pgfpathrectangle{\pgfqpoint{5.800000in}{0.720000in}}{\pgfqpoint{1.400000in}{4.620000in}}%
\pgfusepath{clip}%
\pgfsetbuttcap%
\pgfsetmiterjoin%
\definecolor{currentfill}{rgb}{0.121569,0.466667,0.705882}%
\pgfsetfillcolor{currentfill}%
\pgfsetlinewidth{0.000000pt}%
\definecolor{currentstroke}{rgb}{0.000000,0.000000,0.000000}%
\pgfsetstrokecolor{currentstroke}%
\pgfsetstrokeopacity{0.000000}%
\pgfsetdash{}{0pt}%
\pgfpathmoveto{\pgfqpoint{5.800000in}{13.060448in}}%
\pgfpathlineto{\pgfqpoint{5.800000in}{13.060448in}}%
\pgfpathlineto{\pgfqpoint{5.800000in}{13.163285in}}%
\pgfpathlineto{\pgfqpoint{5.800000in}{13.163285in}}%
\pgfpathclose%
\pgfusepath{fill}%
\end{pgfscope}%
\begin{pgfscope}%
\pgfpathrectangle{\pgfqpoint{5.800000in}{0.720000in}}{\pgfqpoint{1.400000in}{4.620000in}}%
\pgfusepath{clip}%
\pgfsetbuttcap%
\pgfsetmiterjoin%
\definecolor{currentfill}{rgb}{0.121569,0.466667,0.705882}%
\pgfsetfillcolor{currentfill}%
\pgfsetlinewidth{0.000000pt}%
\definecolor{currentstroke}{rgb}{0.000000,0.000000,0.000000}%
\pgfsetstrokecolor{currentstroke}%
\pgfsetstrokeopacity{0.000000}%
\pgfsetdash{}{0pt}%
\pgfpathmoveto{\pgfqpoint{5.800000in}{13.163285in}}%
\pgfpathlineto{\pgfqpoint{5.800000in}{13.163285in}}%
\pgfpathlineto{\pgfqpoint{5.800000in}{13.266122in}}%
\pgfpathlineto{\pgfqpoint{5.800000in}{13.266122in}}%
\pgfpathclose%
\pgfusepath{fill}%
\end{pgfscope}%
\begin{pgfscope}%
\pgfpathrectangle{\pgfqpoint{5.800000in}{0.720000in}}{\pgfqpoint{1.400000in}{4.620000in}}%
\pgfusepath{clip}%
\pgfsetbuttcap%
\pgfsetmiterjoin%
\definecolor{currentfill}{rgb}{0.121569,0.466667,0.705882}%
\pgfsetfillcolor{currentfill}%
\pgfsetlinewidth{0.000000pt}%
\definecolor{currentstroke}{rgb}{0.000000,0.000000,0.000000}%
\pgfsetstrokecolor{currentstroke}%
\pgfsetstrokeopacity{0.000000}%
\pgfsetdash{}{0pt}%
\pgfpathmoveto{\pgfqpoint{5.800000in}{13.266122in}}%
\pgfpathlineto{\pgfqpoint{5.800000in}{13.266122in}}%
\pgfpathlineto{\pgfqpoint{5.800000in}{13.368959in}}%
\pgfpathlineto{\pgfqpoint{5.800000in}{13.368959in}}%
\pgfpathclose%
\pgfusepath{fill}%
\end{pgfscope}%
\begin{pgfscope}%
\pgfpathrectangle{\pgfqpoint{5.800000in}{0.720000in}}{\pgfqpoint{1.400000in}{4.620000in}}%
\pgfusepath{clip}%
\pgfsetbuttcap%
\pgfsetmiterjoin%
\definecolor{currentfill}{rgb}{0.121569,0.466667,0.705882}%
\pgfsetfillcolor{currentfill}%
\pgfsetlinewidth{0.000000pt}%
\definecolor{currentstroke}{rgb}{0.000000,0.000000,0.000000}%
\pgfsetstrokecolor{currentstroke}%
\pgfsetstrokeopacity{0.000000}%
\pgfsetdash{}{0pt}%
\pgfpathmoveto{\pgfqpoint{5.800000in}{13.368959in}}%
\pgfpathlineto{\pgfqpoint{5.800000in}{13.368959in}}%
\pgfpathlineto{\pgfqpoint{5.800000in}{13.471796in}}%
\pgfpathlineto{\pgfqpoint{5.800000in}{13.471796in}}%
\pgfpathclose%
\pgfusepath{fill}%
\end{pgfscope}%
\begin{pgfscope}%
\pgfpathrectangle{\pgfqpoint{5.800000in}{0.720000in}}{\pgfqpoint{1.400000in}{4.620000in}}%
\pgfusepath{clip}%
\pgfsetbuttcap%
\pgfsetmiterjoin%
\definecolor{currentfill}{rgb}{0.121569,0.466667,0.705882}%
\pgfsetfillcolor{currentfill}%
\pgfsetlinewidth{0.000000pt}%
\definecolor{currentstroke}{rgb}{0.000000,0.000000,0.000000}%
\pgfsetstrokecolor{currentstroke}%
\pgfsetstrokeopacity{0.000000}%
\pgfsetdash{}{0pt}%
\pgfpathmoveto{\pgfqpoint{5.800000in}{13.471796in}}%
\pgfpathlineto{\pgfqpoint{5.800000in}{13.471796in}}%
\pgfpathlineto{\pgfqpoint{5.800000in}{13.574633in}}%
\pgfpathlineto{\pgfqpoint{5.800000in}{13.574633in}}%
\pgfpathclose%
\pgfusepath{fill}%
\end{pgfscope}%
\begin{pgfscope}%
\pgfpathrectangle{\pgfqpoint{5.800000in}{0.720000in}}{\pgfqpoint{1.400000in}{4.620000in}}%
\pgfusepath{clip}%
\pgfsetbuttcap%
\pgfsetmiterjoin%
\definecolor{currentfill}{rgb}{0.121569,0.466667,0.705882}%
\pgfsetfillcolor{currentfill}%
\pgfsetlinewidth{0.000000pt}%
\definecolor{currentstroke}{rgb}{0.000000,0.000000,0.000000}%
\pgfsetstrokecolor{currentstroke}%
\pgfsetstrokeopacity{0.000000}%
\pgfsetdash{}{0pt}%
\pgfpathmoveto{\pgfqpoint{5.800000in}{13.574633in}}%
\pgfpathlineto{\pgfqpoint{5.802038in}{13.574633in}}%
\pgfpathlineto{\pgfqpoint{5.802038in}{13.677471in}}%
\pgfpathlineto{\pgfqpoint{5.800000in}{13.677471in}}%
\pgfpathclose%
\pgfusepath{fill}%
\end{pgfscope}%
\begin{pgfscope}%
\pgfpathrectangle{\pgfqpoint{5.800000in}{0.720000in}}{\pgfqpoint{1.400000in}{4.620000in}}%
\pgfusepath{clip}%
\pgfsetbuttcap%
\pgfsetmiterjoin%
\definecolor{currentfill}{rgb}{0.121569,0.466667,0.705882}%
\pgfsetfillcolor{currentfill}%
\pgfsetlinewidth{0.000000pt}%
\definecolor{currentstroke}{rgb}{0.000000,0.000000,0.000000}%
\pgfsetstrokecolor{currentstroke}%
\pgfsetstrokeopacity{0.000000}%
\pgfsetdash{}{0pt}%
\pgfpathmoveto{\pgfqpoint{5.800000in}{13.677471in}}%
\pgfpathlineto{\pgfqpoint{5.800000in}{13.677471in}}%
\pgfpathlineto{\pgfqpoint{5.800000in}{13.780308in}}%
\pgfpathlineto{\pgfqpoint{5.800000in}{13.780308in}}%
\pgfpathclose%
\pgfusepath{fill}%
\end{pgfscope}%
\begin{pgfscope}%
\pgfpathrectangle{\pgfqpoint{5.800000in}{0.720000in}}{\pgfqpoint{1.400000in}{4.620000in}}%
\pgfusepath{clip}%
\pgfsetbuttcap%
\pgfsetmiterjoin%
\definecolor{currentfill}{rgb}{0.121569,0.466667,0.705882}%
\pgfsetfillcolor{currentfill}%
\pgfsetlinewidth{0.000000pt}%
\definecolor{currentstroke}{rgb}{0.000000,0.000000,0.000000}%
\pgfsetstrokecolor{currentstroke}%
\pgfsetstrokeopacity{0.000000}%
\pgfsetdash{}{0pt}%
\pgfpathmoveto{\pgfqpoint{5.800000in}{13.780308in}}%
\pgfpathlineto{\pgfqpoint{5.800000in}{13.780308in}}%
\pgfpathlineto{\pgfqpoint{5.800000in}{13.883145in}}%
\pgfpathlineto{\pgfqpoint{5.800000in}{13.883145in}}%
\pgfpathclose%
\pgfusepath{fill}%
\end{pgfscope}%
\begin{pgfscope}%
\pgfpathrectangle{\pgfqpoint{5.800000in}{0.720000in}}{\pgfqpoint{1.400000in}{4.620000in}}%
\pgfusepath{clip}%
\pgfsetbuttcap%
\pgfsetmiterjoin%
\definecolor{currentfill}{rgb}{0.121569,0.466667,0.705882}%
\pgfsetfillcolor{currentfill}%
\pgfsetlinewidth{0.000000pt}%
\definecolor{currentstroke}{rgb}{0.000000,0.000000,0.000000}%
\pgfsetstrokecolor{currentstroke}%
\pgfsetstrokeopacity{0.000000}%
\pgfsetdash{}{0pt}%
\pgfpathmoveto{\pgfqpoint{5.800000in}{13.883145in}}%
\pgfpathlineto{\pgfqpoint{5.800000in}{13.883145in}}%
\pgfpathlineto{\pgfqpoint{5.800000in}{13.985982in}}%
\pgfpathlineto{\pgfqpoint{5.800000in}{13.985982in}}%
\pgfpathclose%
\pgfusepath{fill}%
\end{pgfscope}%
\begin{pgfscope}%
\pgfpathrectangle{\pgfqpoint{5.800000in}{0.720000in}}{\pgfqpoint{1.400000in}{4.620000in}}%
\pgfusepath{clip}%
\pgfsetbuttcap%
\pgfsetmiterjoin%
\definecolor{currentfill}{rgb}{0.121569,0.466667,0.705882}%
\pgfsetfillcolor{currentfill}%
\pgfsetlinewidth{0.000000pt}%
\definecolor{currentstroke}{rgb}{0.000000,0.000000,0.000000}%
\pgfsetstrokecolor{currentstroke}%
\pgfsetstrokeopacity{0.000000}%
\pgfsetdash{}{0pt}%
\pgfpathmoveto{\pgfqpoint{5.800000in}{13.985982in}}%
\pgfpathlineto{\pgfqpoint{5.800000in}{13.985982in}}%
\pgfpathlineto{\pgfqpoint{5.800000in}{14.088819in}}%
\pgfpathlineto{\pgfqpoint{5.800000in}{14.088819in}}%
\pgfpathclose%
\pgfusepath{fill}%
\end{pgfscope}%
\begin{pgfscope}%
\pgfpathrectangle{\pgfqpoint{5.800000in}{0.720000in}}{\pgfqpoint{1.400000in}{4.620000in}}%
\pgfusepath{clip}%
\pgfsetbuttcap%
\pgfsetmiterjoin%
\definecolor{currentfill}{rgb}{0.121569,0.466667,0.705882}%
\pgfsetfillcolor{currentfill}%
\pgfsetlinewidth{0.000000pt}%
\definecolor{currentstroke}{rgb}{0.000000,0.000000,0.000000}%
\pgfsetstrokecolor{currentstroke}%
\pgfsetstrokeopacity{0.000000}%
\pgfsetdash{}{0pt}%
\pgfpathmoveto{\pgfqpoint{5.800000in}{14.088819in}}%
\pgfpathlineto{\pgfqpoint{5.800000in}{14.088819in}}%
\pgfpathlineto{\pgfqpoint{5.800000in}{14.191656in}}%
\pgfpathlineto{\pgfqpoint{5.800000in}{14.191656in}}%
\pgfpathclose%
\pgfusepath{fill}%
\end{pgfscope}%
\begin{pgfscope}%
\pgfpathrectangle{\pgfqpoint{5.800000in}{0.720000in}}{\pgfqpoint{1.400000in}{4.620000in}}%
\pgfusepath{clip}%
\pgfsetbuttcap%
\pgfsetmiterjoin%
\definecolor{currentfill}{rgb}{0.121569,0.466667,0.705882}%
\pgfsetfillcolor{currentfill}%
\pgfsetlinewidth{0.000000pt}%
\definecolor{currentstroke}{rgb}{0.000000,0.000000,0.000000}%
\pgfsetstrokecolor{currentstroke}%
\pgfsetstrokeopacity{0.000000}%
\pgfsetdash{}{0pt}%
\pgfpathmoveto{\pgfqpoint{5.800000in}{14.191656in}}%
\pgfpathlineto{\pgfqpoint{5.800000in}{14.191656in}}%
\pgfpathlineto{\pgfqpoint{5.800000in}{14.294493in}}%
\pgfpathlineto{\pgfqpoint{5.800000in}{14.294493in}}%
\pgfpathclose%
\pgfusepath{fill}%
\end{pgfscope}%
\begin{pgfscope}%
\pgfpathrectangle{\pgfqpoint{5.800000in}{0.720000in}}{\pgfqpoint{1.400000in}{4.620000in}}%
\pgfusepath{clip}%
\pgfsetbuttcap%
\pgfsetmiterjoin%
\definecolor{currentfill}{rgb}{0.121569,0.466667,0.705882}%
\pgfsetfillcolor{currentfill}%
\pgfsetlinewidth{0.000000pt}%
\definecolor{currentstroke}{rgb}{0.000000,0.000000,0.000000}%
\pgfsetstrokecolor{currentstroke}%
\pgfsetstrokeopacity{0.000000}%
\pgfsetdash{}{0pt}%
\pgfpathmoveto{\pgfqpoint{5.800000in}{14.294493in}}%
\pgfpathlineto{\pgfqpoint{5.800000in}{14.294493in}}%
\pgfpathlineto{\pgfqpoint{5.800000in}{14.397330in}}%
\pgfpathlineto{\pgfqpoint{5.800000in}{14.397330in}}%
\pgfpathclose%
\pgfusepath{fill}%
\end{pgfscope}%
\begin{pgfscope}%
\pgfpathrectangle{\pgfqpoint{5.800000in}{0.720000in}}{\pgfqpoint{1.400000in}{4.620000in}}%
\pgfusepath{clip}%
\pgfsetbuttcap%
\pgfsetmiterjoin%
\definecolor{currentfill}{rgb}{0.121569,0.466667,0.705882}%
\pgfsetfillcolor{currentfill}%
\pgfsetlinewidth{0.000000pt}%
\definecolor{currentstroke}{rgb}{0.000000,0.000000,0.000000}%
\pgfsetstrokecolor{currentstroke}%
\pgfsetstrokeopacity{0.000000}%
\pgfsetdash{}{0pt}%
\pgfpathmoveto{\pgfqpoint{5.800000in}{14.397330in}}%
\pgfpathlineto{\pgfqpoint{5.802038in}{14.397330in}}%
\pgfpathlineto{\pgfqpoint{5.802038in}{14.500167in}}%
\pgfpathlineto{\pgfqpoint{5.800000in}{14.500167in}}%
\pgfpathclose%
\pgfusepath{fill}%
\end{pgfscope}%
\begin{pgfscope}%
\pgfpathrectangle{\pgfqpoint{5.800000in}{0.720000in}}{\pgfqpoint{1.400000in}{4.620000in}}%
\pgfusepath{clip}%
\pgfsetbuttcap%
\pgfsetmiterjoin%
\definecolor{currentfill}{rgb}{0.121569,0.466667,0.705882}%
\pgfsetfillcolor{currentfill}%
\pgfsetlinewidth{0.000000pt}%
\definecolor{currentstroke}{rgb}{0.000000,0.000000,0.000000}%
\pgfsetstrokecolor{currentstroke}%
\pgfsetstrokeopacity{0.000000}%
\pgfsetdash{}{0pt}%
\pgfpathmoveto{\pgfqpoint{5.800000in}{14.500167in}}%
\pgfpathlineto{\pgfqpoint{5.800000in}{14.500167in}}%
\pgfpathlineto{\pgfqpoint{5.800000in}{14.603004in}}%
\pgfpathlineto{\pgfqpoint{5.800000in}{14.603004in}}%
\pgfpathclose%
\pgfusepath{fill}%
\end{pgfscope}%
\begin{pgfscope}%
\pgfpathrectangle{\pgfqpoint{5.800000in}{0.720000in}}{\pgfqpoint{1.400000in}{4.620000in}}%
\pgfusepath{clip}%
\pgfsetbuttcap%
\pgfsetmiterjoin%
\definecolor{currentfill}{rgb}{0.121569,0.466667,0.705882}%
\pgfsetfillcolor{currentfill}%
\pgfsetlinewidth{0.000000pt}%
\definecolor{currentstroke}{rgb}{0.000000,0.000000,0.000000}%
\pgfsetstrokecolor{currentstroke}%
\pgfsetstrokeopacity{0.000000}%
\pgfsetdash{}{0pt}%
\pgfpathmoveto{\pgfqpoint{5.800000in}{14.603004in}}%
\pgfpathlineto{\pgfqpoint{5.800000in}{14.603004in}}%
\pgfpathlineto{\pgfqpoint{5.800000in}{14.705841in}}%
\pgfpathlineto{\pgfqpoint{5.800000in}{14.705841in}}%
\pgfpathclose%
\pgfusepath{fill}%
\end{pgfscope}%
\begin{pgfscope}%
\pgfpathrectangle{\pgfqpoint{5.800000in}{0.720000in}}{\pgfqpoint{1.400000in}{4.620000in}}%
\pgfusepath{clip}%
\pgfsetbuttcap%
\pgfsetmiterjoin%
\definecolor{currentfill}{rgb}{0.121569,0.466667,0.705882}%
\pgfsetfillcolor{currentfill}%
\pgfsetlinewidth{0.000000pt}%
\definecolor{currentstroke}{rgb}{0.000000,0.000000,0.000000}%
\pgfsetstrokecolor{currentstroke}%
\pgfsetstrokeopacity{0.000000}%
\pgfsetdash{}{0pt}%
\pgfpathmoveto{\pgfqpoint{5.800000in}{14.705841in}}%
\pgfpathlineto{\pgfqpoint{5.800000in}{14.705841in}}%
\pgfpathlineto{\pgfqpoint{5.800000in}{14.808678in}}%
\pgfpathlineto{\pgfqpoint{5.800000in}{14.808678in}}%
\pgfpathclose%
\pgfusepath{fill}%
\end{pgfscope}%
\begin{pgfscope}%
\pgfpathrectangle{\pgfqpoint{5.800000in}{0.720000in}}{\pgfqpoint{1.400000in}{4.620000in}}%
\pgfusepath{clip}%
\pgfsetbuttcap%
\pgfsetmiterjoin%
\definecolor{currentfill}{rgb}{0.121569,0.466667,0.705882}%
\pgfsetfillcolor{currentfill}%
\pgfsetlinewidth{0.000000pt}%
\definecolor{currentstroke}{rgb}{0.000000,0.000000,0.000000}%
\pgfsetstrokecolor{currentstroke}%
\pgfsetstrokeopacity{0.000000}%
\pgfsetdash{}{0pt}%
\pgfpathmoveto{\pgfqpoint{5.800000in}{14.808678in}}%
\pgfpathlineto{\pgfqpoint{5.800000in}{14.808678in}}%
\pgfpathlineto{\pgfqpoint{5.800000in}{14.911515in}}%
\pgfpathlineto{\pgfqpoint{5.800000in}{14.911515in}}%
\pgfpathclose%
\pgfusepath{fill}%
\end{pgfscope}%
\begin{pgfscope}%
\pgfpathrectangle{\pgfqpoint{5.800000in}{0.720000in}}{\pgfqpoint{1.400000in}{4.620000in}}%
\pgfusepath{clip}%
\pgfsetbuttcap%
\pgfsetmiterjoin%
\definecolor{currentfill}{rgb}{0.121569,0.466667,0.705882}%
\pgfsetfillcolor{currentfill}%
\pgfsetlinewidth{0.000000pt}%
\definecolor{currentstroke}{rgb}{0.000000,0.000000,0.000000}%
\pgfsetstrokecolor{currentstroke}%
\pgfsetstrokeopacity{0.000000}%
\pgfsetdash{}{0pt}%
\pgfpathmoveto{\pgfqpoint{5.800000in}{14.911515in}}%
\pgfpathlineto{\pgfqpoint{5.800000in}{14.911515in}}%
\pgfpathlineto{\pgfqpoint{5.800000in}{15.014352in}}%
\pgfpathlineto{\pgfqpoint{5.800000in}{15.014352in}}%
\pgfpathclose%
\pgfusepath{fill}%
\end{pgfscope}%
\begin{pgfscope}%
\pgfpathrectangle{\pgfqpoint{5.800000in}{0.720000in}}{\pgfqpoint{1.400000in}{4.620000in}}%
\pgfusepath{clip}%
\pgfsetbuttcap%
\pgfsetmiterjoin%
\definecolor{currentfill}{rgb}{0.121569,0.466667,0.705882}%
\pgfsetfillcolor{currentfill}%
\pgfsetlinewidth{0.000000pt}%
\definecolor{currentstroke}{rgb}{0.000000,0.000000,0.000000}%
\pgfsetstrokecolor{currentstroke}%
\pgfsetstrokeopacity{0.000000}%
\pgfsetdash{}{0pt}%
\pgfpathmoveto{\pgfqpoint{5.800000in}{15.014352in}}%
\pgfpathlineto{\pgfqpoint{5.800000in}{15.014352in}}%
\pgfpathlineto{\pgfqpoint{5.800000in}{15.117189in}}%
\pgfpathlineto{\pgfqpoint{5.800000in}{15.117189in}}%
\pgfpathclose%
\pgfusepath{fill}%
\end{pgfscope}%
\begin{pgfscope}%
\pgfpathrectangle{\pgfqpoint{5.800000in}{0.720000in}}{\pgfqpoint{1.400000in}{4.620000in}}%
\pgfusepath{clip}%
\pgfsetbuttcap%
\pgfsetmiterjoin%
\definecolor{currentfill}{rgb}{0.121569,0.466667,0.705882}%
\pgfsetfillcolor{currentfill}%
\pgfsetlinewidth{0.000000pt}%
\definecolor{currentstroke}{rgb}{0.000000,0.000000,0.000000}%
\pgfsetstrokecolor{currentstroke}%
\pgfsetstrokeopacity{0.000000}%
\pgfsetdash{}{0pt}%
\pgfpathmoveto{\pgfqpoint{5.800000in}{15.117189in}}%
\pgfpathlineto{\pgfqpoint{5.800000in}{15.117189in}}%
\pgfpathlineto{\pgfqpoint{5.800000in}{15.220027in}}%
\pgfpathlineto{\pgfqpoint{5.800000in}{15.220027in}}%
\pgfpathclose%
\pgfusepath{fill}%
\end{pgfscope}%
\begin{pgfscope}%
\pgfpathrectangle{\pgfqpoint{5.800000in}{0.720000in}}{\pgfqpoint{1.400000in}{4.620000in}}%
\pgfusepath{clip}%
\pgfsetbuttcap%
\pgfsetmiterjoin%
\definecolor{currentfill}{rgb}{0.121569,0.466667,0.705882}%
\pgfsetfillcolor{currentfill}%
\pgfsetlinewidth{0.000000pt}%
\definecolor{currentstroke}{rgb}{0.000000,0.000000,0.000000}%
\pgfsetstrokecolor{currentstroke}%
\pgfsetstrokeopacity{0.000000}%
\pgfsetdash{}{0pt}%
\pgfpathmoveto{\pgfqpoint{5.800000in}{15.220027in}}%
\pgfpathlineto{\pgfqpoint{5.800000in}{15.220027in}}%
\pgfpathlineto{\pgfqpoint{5.800000in}{15.322864in}}%
\pgfpathlineto{\pgfqpoint{5.800000in}{15.322864in}}%
\pgfpathclose%
\pgfusepath{fill}%
\end{pgfscope}%
\begin{pgfscope}%
\pgfpathrectangle{\pgfqpoint{5.800000in}{0.720000in}}{\pgfqpoint{1.400000in}{4.620000in}}%
\pgfusepath{clip}%
\pgfsetbuttcap%
\pgfsetmiterjoin%
\definecolor{currentfill}{rgb}{0.121569,0.466667,0.705882}%
\pgfsetfillcolor{currentfill}%
\pgfsetlinewidth{0.000000pt}%
\definecolor{currentstroke}{rgb}{0.000000,0.000000,0.000000}%
\pgfsetstrokecolor{currentstroke}%
\pgfsetstrokeopacity{0.000000}%
\pgfsetdash{}{0pt}%
\pgfpathmoveto{\pgfqpoint{5.800000in}{15.322864in}}%
\pgfpathlineto{\pgfqpoint{5.800000in}{15.322864in}}%
\pgfpathlineto{\pgfqpoint{5.800000in}{15.425701in}}%
\pgfpathlineto{\pgfqpoint{5.800000in}{15.425701in}}%
\pgfpathclose%
\pgfusepath{fill}%
\end{pgfscope}%
\begin{pgfscope}%
\pgfpathrectangle{\pgfqpoint{5.800000in}{0.720000in}}{\pgfqpoint{1.400000in}{4.620000in}}%
\pgfusepath{clip}%
\pgfsetbuttcap%
\pgfsetmiterjoin%
\definecolor{currentfill}{rgb}{0.121569,0.466667,0.705882}%
\pgfsetfillcolor{currentfill}%
\pgfsetlinewidth{0.000000pt}%
\definecolor{currentstroke}{rgb}{0.000000,0.000000,0.000000}%
\pgfsetstrokecolor{currentstroke}%
\pgfsetstrokeopacity{0.000000}%
\pgfsetdash{}{0pt}%
\pgfpathmoveto{\pgfqpoint{5.800000in}{15.425701in}}%
\pgfpathlineto{\pgfqpoint{5.800000in}{15.425701in}}%
\pgfpathlineto{\pgfqpoint{5.800000in}{15.528538in}}%
\pgfpathlineto{\pgfqpoint{5.800000in}{15.528538in}}%
\pgfpathclose%
\pgfusepath{fill}%
\end{pgfscope}%
\begin{pgfscope}%
\pgfpathrectangle{\pgfqpoint{5.800000in}{0.720000in}}{\pgfqpoint{1.400000in}{4.620000in}}%
\pgfusepath{clip}%
\pgfsetbuttcap%
\pgfsetmiterjoin%
\definecolor{currentfill}{rgb}{0.121569,0.466667,0.705882}%
\pgfsetfillcolor{currentfill}%
\pgfsetlinewidth{0.000000pt}%
\definecolor{currentstroke}{rgb}{0.000000,0.000000,0.000000}%
\pgfsetstrokecolor{currentstroke}%
\pgfsetstrokeopacity{0.000000}%
\pgfsetdash{}{0pt}%
\pgfpathmoveto{\pgfqpoint{5.800000in}{15.528538in}}%
\pgfpathlineto{\pgfqpoint{5.800000in}{15.528538in}}%
\pgfpathlineto{\pgfqpoint{5.800000in}{15.631375in}}%
\pgfpathlineto{\pgfqpoint{5.800000in}{15.631375in}}%
\pgfpathclose%
\pgfusepath{fill}%
\end{pgfscope}%
\begin{pgfscope}%
\pgfpathrectangle{\pgfqpoint{5.800000in}{0.720000in}}{\pgfqpoint{1.400000in}{4.620000in}}%
\pgfusepath{clip}%
\pgfsetbuttcap%
\pgfsetmiterjoin%
\definecolor{currentfill}{rgb}{0.121569,0.466667,0.705882}%
\pgfsetfillcolor{currentfill}%
\pgfsetlinewidth{0.000000pt}%
\definecolor{currentstroke}{rgb}{0.000000,0.000000,0.000000}%
\pgfsetstrokecolor{currentstroke}%
\pgfsetstrokeopacity{0.000000}%
\pgfsetdash{}{0pt}%
\pgfpathmoveto{\pgfqpoint{5.800000in}{15.631375in}}%
\pgfpathlineto{\pgfqpoint{5.800000in}{15.631375in}}%
\pgfpathlineto{\pgfqpoint{5.800000in}{15.734212in}}%
\pgfpathlineto{\pgfqpoint{5.800000in}{15.734212in}}%
\pgfpathclose%
\pgfusepath{fill}%
\end{pgfscope}%
\begin{pgfscope}%
\pgfpathrectangle{\pgfqpoint{5.800000in}{0.720000in}}{\pgfqpoint{1.400000in}{4.620000in}}%
\pgfusepath{clip}%
\pgfsetbuttcap%
\pgfsetmiterjoin%
\definecolor{currentfill}{rgb}{0.121569,0.466667,0.705882}%
\pgfsetfillcolor{currentfill}%
\pgfsetlinewidth{0.000000pt}%
\definecolor{currentstroke}{rgb}{0.000000,0.000000,0.000000}%
\pgfsetstrokecolor{currentstroke}%
\pgfsetstrokeopacity{0.000000}%
\pgfsetdash{}{0pt}%
\pgfpathmoveto{\pgfqpoint{5.800000in}{15.734212in}}%
\pgfpathlineto{\pgfqpoint{5.802038in}{15.734212in}}%
\pgfpathlineto{\pgfqpoint{5.802038in}{15.837049in}}%
\pgfpathlineto{\pgfqpoint{5.800000in}{15.837049in}}%
\pgfpathclose%
\pgfusepath{fill}%
\end{pgfscope}%
\begin{pgfscope}%
\pgfpathrectangle{\pgfqpoint{5.800000in}{0.720000in}}{\pgfqpoint{1.400000in}{4.620000in}}%
\pgfusepath{clip}%
\pgfsetbuttcap%
\pgfsetmiterjoin%
\definecolor{currentfill}{rgb}{0.121569,0.466667,0.705882}%
\pgfsetfillcolor{currentfill}%
\pgfsetlinewidth{0.000000pt}%
\definecolor{currentstroke}{rgb}{0.000000,0.000000,0.000000}%
\pgfsetstrokecolor{currentstroke}%
\pgfsetstrokeopacity{0.000000}%
\pgfsetdash{}{0pt}%
\pgfpathmoveto{\pgfqpoint{5.800000in}{15.837049in}}%
\pgfpathlineto{\pgfqpoint{5.800000in}{15.837049in}}%
\pgfpathlineto{\pgfqpoint{5.800000in}{15.939886in}}%
\pgfpathlineto{\pgfqpoint{5.800000in}{15.939886in}}%
\pgfpathclose%
\pgfusepath{fill}%
\end{pgfscope}%
\begin{pgfscope}%
\pgfpathrectangle{\pgfqpoint{5.800000in}{0.720000in}}{\pgfqpoint{1.400000in}{4.620000in}}%
\pgfusepath{clip}%
\pgfsetbuttcap%
\pgfsetmiterjoin%
\definecolor{currentfill}{rgb}{0.121569,0.466667,0.705882}%
\pgfsetfillcolor{currentfill}%
\pgfsetlinewidth{0.000000pt}%
\definecolor{currentstroke}{rgb}{0.000000,0.000000,0.000000}%
\pgfsetstrokecolor{currentstroke}%
\pgfsetstrokeopacity{0.000000}%
\pgfsetdash{}{0pt}%
\pgfpathmoveto{\pgfqpoint{5.800000in}{15.939886in}}%
\pgfpathlineto{\pgfqpoint{5.800000in}{15.939886in}}%
\pgfpathlineto{\pgfqpoint{5.800000in}{16.042723in}}%
\pgfpathlineto{\pgfqpoint{5.800000in}{16.042723in}}%
\pgfpathclose%
\pgfusepath{fill}%
\end{pgfscope}%
\begin{pgfscope}%
\pgfpathrectangle{\pgfqpoint{5.800000in}{0.720000in}}{\pgfqpoint{1.400000in}{4.620000in}}%
\pgfusepath{clip}%
\pgfsetbuttcap%
\pgfsetmiterjoin%
\definecolor{currentfill}{rgb}{0.121569,0.466667,0.705882}%
\pgfsetfillcolor{currentfill}%
\pgfsetlinewidth{0.000000pt}%
\definecolor{currentstroke}{rgb}{0.000000,0.000000,0.000000}%
\pgfsetstrokecolor{currentstroke}%
\pgfsetstrokeopacity{0.000000}%
\pgfsetdash{}{0pt}%
\pgfpathmoveto{\pgfqpoint{5.800000in}{16.042723in}}%
\pgfpathlineto{\pgfqpoint{5.800000in}{16.042723in}}%
\pgfpathlineto{\pgfqpoint{5.800000in}{16.145560in}}%
\pgfpathlineto{\pgfqpoint{5.800000in}{16.145560in}}%
\pgfpathclose%
\pgfusepath{fill}%
\end{pgfscope}%
\begin{pgfscope}%
\pgfpathrectangle{\pgfqpoint{5.800000in}{0.720000in}}{\pgfqpoint{1.400000in}{4.620000in}}%
\pgfusepath{clip}%
\pgfsetbuttcap%
\pgfsetmiterjoin%
\definecolor{currentfill}{rgb}{0.121569,0.466667,0.705882}%
\pgfsetfillcolor{currentfill}%
\pgfsetlinewidth{0.000000pt}%
\definecolor{currentstroke}{rgb}{0.000000,0.000000,0.000000}%
\pgfsetstrokecolor{currentstroke}%
\pgfsetstrokeopacity{0.000000}%
\pgfsetdash{}{0pt}%
\pgfpathmoveto{\pgfqpoint{5.800000in}{16.145560in}}%
\pgfpathlineto{\pgfqpoint{5.800000in}{16.145560in}}%
\pgfpathlineto{\pgfqpoint{5.800000in}{16.248397in}}%
\pgfpathlineto{\pgfqpoint{5.800000in}{16.248397in}}%
\pgfpathclose%
\pgfusepath{fill}%
\end{pgfscope}%
\begin{pgfscope}%
\pgfpathrectangle{\pgfqpoint{5.800000in}{0.720000in}}{\pgfqpoint{1.400000in}{4.620000in}}%
\pgfusepath{clip}%
\pgfsetbuttcap%
\pgfsetmiterjoin%
\definecolor{currentfill}{rgb}{0.121569,0.466667,0.705882}%
\pgfsetfillcolor{currentfill}%
\pgfsetlinewidth{0.000000pt}%
\definecolor{currentstroke}{rgb}{0.000000,0.000000,0.000000}%
\pgfsetstrokecolor{currentstroke}%
\pgfsetstrokeopacity{0.000000}%
\pgfsetdash{}{0pt}%
\pgfpathmoveto{\pgfqpoint{5.800000in}{16.248397in}}%
\pgfpathlineto{\pgfqpoint{5.800000in}{16.248397in}}%
\pgfpathlineto{\pgfqpoint{5.800000in}{16.351234in}}%
\pgfpathlineto{\pgfqpoint{5.800000in}{16.351234in}}%
\pgfpathclose%
\pgfusepath{fill}%
\end{pgfscope}%
\begin{pgfscope}%
\pgfpathrectangle{\pgfqpoint{5.800000in}{0.720000in}}{\pgfqpoint{1.400000in}{4.620000in}}%
\pgfusepath{clip}%
\pgfsetbuttcap%
\pgfsetmiterjoin%
\definecolor{currentfill}{rgb}{0.121569,0.466667,0.705882}%
\pgfsetfillcolor{currentfill}%
\pgfsetlinewidth{0.000000pt}%
\definecolor{currentstroke}{rgb}{0.000000,0.000000,0.000000}%
\pgfsetstrokecolor{currentstroke}%
\pgfsetstrokeopacity{0.000000}%
\pgfsetdash{}{0pt}%
\pgfpathmoveto{\pgfqpoint{5.800000in}{16.351234in}}%
\pgfpathlineto{\pgfqpoint{5.800000in}{16.351234in}}%
\pgfpathlineto{\pgfqpoint{5.800000in}{16.454071in}}%
\pgfpathlineto{\pgfqpoint{5.800000in}{16.454071in}}%
\pgfpathclose%
\pgfusepath{fill}%
\end{pgfscope}%
\begin{pgfscope}%
\pgfpathrectangle{\pgfqpoint{5.800000in}{0.720000in}}{\pgfqpoint{1.400000in}{4.620000in}}%
\pgfusepath{clip}%
\pgfsetbuttcap%
\pgfsetmiterjoin%
\definecolor{currentfill}{rgb}{0.121569,0.466667,0.705882}%
\pgfsetfillcolor{currentfill}%
\pgfsetlinewidth{0.000000pt}%
\definecolor{currentstroke}{rgb}{0.000000,0.000000,0.000000}%
\pgfsetstrokecolor{currentstroke}%
\pgfsetstrokeopacity{0.000000}%
\pgfsetdash{}{0pt}%
\pgfpathmoveto{\pgfqpoint{5.800000in}{16.454071in}}%
\pgfpathlineto{\pgfqpoint{5.800000in}{16.454071in}}%
\pgfpathlineto{\pgfqpoint{5.800000in}{16.556908in}}%
\pgfpathlineto{\pgfqpoint{5.800000in}{16.556908in}}%
\pgfpathclose%
\pgfusepath{fill}%
\end{pgfscope}%
\begin{pgfscope}%
\pgfpathrectangle{\pgfqpoint{5.800000in}{0.720000in}}{\pgfqpoint{1.400000in}{4.620000in}}%
\pgfusepath{clip}%
\pgfsetbuttcap%
\pgfsetmiterjoin%
\definecolor{currentfill}{rgb}{0.121569,0.466667,0.705882}%
\pgfsetfillcolor{currentfill}%
\pgfsetlinewidth{0.000000pt}%
\definecolor{currentstroke}{rgb}{0.000000,0.000000,0.000000}%
\pgfsetstrokecolor{currentstroke}%
\pgfsetstrokeopacity{0.000000}%
\pgfsetdash{}{0pt}%
\pgfpathmoveto{\pgfqpoint{5.800000in}{16.556908in}}%
\pgfpathlineto{\pgfqpoint{5.800000in}{16.556908in}}%
\pgfpathlineto{\pgfqpoint{5.800000in}{16.659746in}}%
\pgfpathlineto{\pgfqpoint{5.800000in}{16.659746in}}%
\pgfpathclose%
\pgfusepath{fill}%
\end{pgfscope}%
\begin{pgfscope}%
\pgfpathrectangle{\pgfqpoint{5.800000in}{0.720000in}}{\pgfqpoint{1.400000in}{4.620000in}}%
\pgfusepath{clip}%
\pgfsetbuttcap%
\pgfsetmiterjoin%
\definecolor{currentfill}{rgb}{0.121569,0.466667,0.705882}%
\pgfsetfillcolor{currentfill}%
\pgfsetlinewidth{0.000000pt}%
\definecolor{currentstroke}{rgb}{0.000000,0.000000,0.000000}%
\pgfsetstrokecolor{currentstroke}%
\pgfsetstrokeopacity{0.000000}%
\pgfsetdash{}{0pt}%
\pgfpathmoveto{\pgfqpoint{5.800000in}{16.659746in}}%
\pgfpathlineto{\pgfqpoint{5.800000in}{16.659746in}}%
\pgfpathlineto{\pgfqpoint{5.800000in}{16.762583in}}%
\pgfpathlineto{\pgfqpoint{5.800000in}{16.762583in}}%
\pgfpathclose%
\pgfusepath{fill}%
\end{pgfscope}%
\begin{pgfscope}%
\pgfpathrectangle{\pgfqpoint{5.800000in}{0.720000in}}{\pgfqpoint{1.400000in}{4.620000in}}%
\pgfusepath{clip}%
\pgfsetbuttcap%
\pgfsetmiterjoin%
\definecolor{currentfill}{rgb}{0.121569,0.466667,0.705882}%
\pgfsetfillcolor{currentfill}%
\pgfsetlinewidth{0.000000pt}%
\definecolor{currentstroke}{rgb}{0.000000,0.000000,0.000000}%
\pgfsetstrokecolor{currentstroke}%
\pgfsetstrokeopacity{0.000000}%
\pgfsetdash{}{0pt}%
\pgfpathmoveto{\pgfqpoint{5.800000in}{16.762583in}}%
\pgfpathlineto{\pgfqpoint{5.800000in}{16.762583in}}%
\pgfpathlineto{\pgfqpoint{5.800000in}{16.865420in}}%
\pgfpathlineto{\pgfqpoint{5.800000in}{16.865420in}}%
\pgfpathclose%
\pgfusepath{fill}%
\end{pgfscope}%
\begin{pgfscope}%
\pgfpathrectangle{\pgfqpoint{5.800000in}{0.720000in}}{\pgfqpoint{1.400000in}{4.620000in}}%
\pgfusepath{clip}%
\pgfsetbuttcap%
\pgfsetmiterjoin%
\definecolor{currentfill}{rgb}{0.121569,0.466667,0.705882}%
\pgfsetfillcolor{currentfill}%
\pgfsetlinewidth{0.000000pt}%
\definecolor{currentstroke}{rgb}{0.000000,0.000000,0.000000}%
\pgfsetstrokecolor{currentstroke}%
\pgfsetstrokeopacity{0.000000}%
\pgfsetdash{}{0pt}%
\pgfpathmoveto{\pgfqpoint{5.800000in}{16.865420in}}%
\pgfpathlineto{\pgfqpoint{5.800000in}{16.865420in}}%
\pgfpathlineto{\pgfqpoint{5.800000in}{16.968257in}}%
\pgfpathlineto{\pgfqpoint{5.800000in}{16.968257in}}%
\pgfpathclose%
\pgfusepath{fill}%
\end{pgfscope}%
\begin{pgfscope}%
\pgfpathrectangle{\pgfqpoint{5.800000in}{0.720000in}}{\pgfqpoint{1.400000in}{4.620000in}}%
\pgfusepath{clip}%
\pgfsetbuttcap%
\pgfsetmiterjoin%
\definecolor{currentfill}{rgb}{0.121569,0.466667,0.705882}%
\pgfsetfillcolor{currentfill}%
\pgfsetlinewidth{0.000000pt}%
\definecolor{currentstroke}{rgb}{0.000000,0.000000,0.000000}%
\pgfsetstrokecolor{currentstroke}%
\pgfsetstrokeopacity{0.000000}%
\pgfsetdash{}{0pt}%
\pgfpathmoveto{\pgfqpoint{5.800000in}{16.968257in}}%
\pgfpathlineto{\pgfqpoint{5.800000in}{16.968257in}}%
\pgfpathlineto{\pgfqpoint{5.800000in}{17.071094in}}%
\pgfpathlineto{\pgfqpoint{5.800000in}{17.071094in}}%
\pgfpathclose%
\pgfusepath{fill}%
\end{pgfscope}%
\begin{pgfscope}%
\pgfpathrectangle{\pgfqpoint{5.800000in}{0.720000in}}{\pgfqpoint{1.400000in}{4.620000in}}%
\pgfusepath{clip}%
\pgfsetbuttcap%
\pgfsetmiterjoin%
\definecolor{currentfill}{rgb}{0.121569,0.466667,0.705882}%
\pgfsetfillcolor{currentfill}%
\pgfsetlinewidth{0.000000pt}%
\definecolor{currentstroke}{rgb}{0.000000,0.000000,0.000000}%
\pgfsetstrokecolor{currentstroke}%
\pgfsetstrokeopacity{0.000000}%
\pgfsetdash{}{0pt}%
\pgfpathmoveto{\pgfqpoint{5.800000in}{17.071094in}}%
\pgfpathlineto{\pgfqpoint{5.800000in}{17.071094in}}%
\pgfpathlineto{\pgfqpoint{5.800000in}{17.173931in}}%
\pgfpathlineto{\pgfqpoint{5.800000in}{17.173931in}}%
\pgfpathclose%
\pgfusepath{fill}%
\end{pgfscope}%
\begin{pgfscope}%
\pgfpathrectangle{\pgfqpoint{5.800000in}{0.720000in}}{\pgfqpoint{1.400000in}{4.620000in}}%
\pgfusepath{clip}%
\pgfsetbuttcap%
\pgfsetmiterjoin%
\definecolor{currentfill}{rgb}{0.121569,0.466667,0.705882}%
\pgfsetfillcolor{currentfill}%
\pgfsetlinewidth{0.000000pt}%
\definecolor{currentstroke}{rgb}{0.000000,0.000000,0.000000}%
\pgfsetstrokecolor{currentstroke}%
\pgfsetstrokeopacity{0.000000}%
\pgfsetdash{}{0pt}%
\pgfpathmoveto{\pgfqpoint{5.800000in}{17.173931in}}%
\pgfpathlineto{\pgfqpoint{5.802038in}{17.173931in}}%
\pgfpathlineto{\pgfqpoint{5.802038in}{17.276768in}}%
\pgfpathlineto{\pgfqpoint{5.800000in}{17.276768in}}%
\pgfpathclose%
\pgfusepath{fill}%
\end{pgfscope}%
\begin{pgfscope}%
\pgfpathrectangle{\pgfqpoint{5.800000in}{0.720000in}}{\pgfqpoint{1.400000in}{4.620000in}}%
\pgfusepath{clip}%
\pgfsetbuttcap%
\pgfsetmiterjoin%
\definecolor{currentfill}{rgb}{0.121569,0.466667,0.705882}%
\pgfsetfillcolor{currentfill}%
\pgfsetlinewidth{0.000000pt}%
\definecolor{currentstroke}{rgb}{0.000000,0.000000,0.000000}%
\pgfsetstrokecolor{currentstroke}%
\pgfsetstrokeopacity{0.000000}%
\pgfsetdash{}{0pt}%
\pgfpathmoveto{\pgfqpoint{5.800000in}{17.276768in}}%
\pgfpathlineto{\pgfqpoint{5.800000in}{17.276768in}}%
\pgfpathlineto{\pgfqpoint{5.800000in}{17.379605in}}%
\pgfpathlineto{\pgfqpoint{5.800000in}{17.379605in}}%
\pgfpathclose%
\pgfusepath{fill}%
\end{pgfscope}%
\begin{pgfscope}%
\pgfpathrectangle{\pgfqpoint{5.800000in}{0.720000in}}{\pgfqpoint{1.400000in}{4.620000in}}%
\pgfusepath{clip}%
\pgfsetbuttcap%
\pgfsetmiterjoin%
\definecolor{currentfill}{rgb}{0.121569,0.466667,0.705882}%
\pgfsetfillcolor{currentfill}%
\pgfsetlinewidth{0.000000pt}%
\definecolor{currentstroke}{rgb}{0.000000,0.000000,0.000000}%
\pgfsetstrokecolor{currentstroke}%
\pgfsetstrokeopacity{0.000000}%
\pgfsetdash{}{0pt}%
\pgfpathmoveto{\pgfqpoint{5.800000in}{17.379605in}}%
\pgfpathlineto{\pgfqpoint{5.800000in}{17.379605in}}%
\pgfpathlineto{\pgfqpoint{5.800000in}{17.482442in}}%
\pgfpathlineto{\pgfqpoint{5.800000in}{17.482442in}}%
\pgfpathclose%
\pgfusepath{fill}%
\end{pgfscope}%
\begin{pgfscope}%
\pgfpathrectangle{\pgfqpoint{5.800000in}{0.720000in}}{\pgfqpoint{1.400000in}{4.620000in}}%
\pgfusepath{clip}%
\pgfsetbuttcap%
\pgfsetmiterjoin%
\definecolor{currentfill}{rgb}{0.121569,0.466667,0.705882}%
\pgfsetfillcolor{currentfill}%
\pgfsetlinewidth{0.000000pt}%
\definecolor{currentstroke}{rgb}{0.000000,0.000000,0.000000}%
\pgfsetstrokecolor{currentstroke}%
\pgfsetstrokeopacity{0.000000}%
\pgfsetdash{}{0pt}%
\pgfpathmoveto{\pgfqpoint{5.800000in}{17.482442in}}%
\pgfpathlineto{\pgfqpoint{5.800000in}{17.482442in}}%
\pgfpathlineto{\pgfqpoint{5.800000in}{17.585279in}}%
\pgfpathlineto{\pgfqpoint{5.800000in}{17.585279in}}%
\pgfpathclose%
\pgfusepath{fill}%
\end{pgfscope}%
\begin{pgfscope}%
\pgfpathrectangle{\pgfqpoint{5.800000in}{0.720000in}}{\pgfqpoint{1.400000in}{4.620000in}}%
\pgfusepath{clip}%
\pgfsetbuttcap%
\pgfsetmiterjoin%
\definecolor{currentfill}{rgb}{0.121569,0.466667,0.705882}%
\pgfsetfillcolor{currentfill}%
\pgfsetlinewidth{0.000000pt}%
\definecolor{currentstroke}{rgb}{0.000000,0.000000,0.000000}%
\pgfsetstrokecolor{currentstroke}%
\pgfsetstrokeopacity{0.000000}%
\pgfsetdash{}{0pt}%
\pgfpathmoveto{\pgfqpoint{5.800000in}{17.585279in}}%
\pgfpathlineto{\pgfqpoint{5.800000in}{17.585279in}}%
\pgfpathlineto{\pgfqpoint{5.800000in}{17.688116in}}%
\pgfpathlineto{\pgfqpoint{5.800000in}{17.688116in}}%
\pgfpathclose%
\pgfusepath{fill}%
\end{pgfscope}%
\begin{pgfscope}%
\pgfpathrectangle{\pgfqpoint{5.800000in}{0.720000in}}{\pgfqpoint{1.400000in}{4.620000in}}%
\pgfusepath{clip}%
\pgfsetbuttcap%
\pgfsetmiterjoin%
\definecolor{currentfill}{rgb}{0.121569,0.466667,0.705882}%
\pgfsetfillcolor{currentfill}%
\pgfsetlinewidth{0.000000pt}%
\definecolor{currentstroke}{rgb}{0.000000,0.000000,0.000000}%
\pgfsetstrokecolor{currentstroke}%
\pgfsetstrokeopacity{0.000000}%
\pgfsetdash{}{0pt}%
\pgfpathmoveto{\pgfqpoint{5.800000in}{17.688116in}}%
\pgfpathlineto{\pgfqpoint{5.800000in}{17.688116in}}%
\pgfpathlineto{\pgfqpoint{5.800000in}{17.790953in}}%
\pgfpathlineto{\pgfqpoint{5.800000in}{17.790953in}}%
\pgfpathclose%
\pgfusepath{fill}%
\end{pgfscope}%
\begin{pgfscope}%
\pgfpathrectangle{\pgfqpoint{5.800000in}{0.720000in}}{\pgfqpoint{1.400000in}{4.620000in}}%
\pgfusepath{clip}%
\pgfsetbuttcap%
\pgfsetmiterjoin%
\definecolor{currentfill}{rgb}{0.121569,0.466667,0.705882}%
\pgfsetfillcolor{currentfill}%
\pgfsetlinewidth{0.000000pt}%
\definecolor{currentstroke}{rgb}{0.000000,0.000000,0.000000}%
\pgfsetstrokecolor{currentstroke}%
\pgfsetstrokeopacity{0.000000}%
\pgfsetdash{}{0pt}%
\pgfpathmoveto{\pgfqpoint{5.800000in}{17.790953in}}%
\pgfpathlineto{\pgfqpoint{5.800000in}{17.790953in}}%
\pgfpathlineto{\pgfqpoint{5.800000in}{17.893790in}}%
\pgfpathlineto{\pgfqpoint{5.800000in}{17.893790in}}%
\pgfpathclose%
\pgfusepath{fill}%
\end{pgfscope}%
\begin{pgfscope}%
\pgfpathrectangle{\pgfqpoint{5.800000in}{0.720000in}}{\pgfqpoint{1.400000in}{4.620000in}}%
\pgfusepath{clip}%
\pgfsetbuttcap%
\pgfsetmiterjoin%
\definecolor{currentfill}{rgb}{0.121569,0.466667,0.705882}%
\pgfsetfillcolor{currentfill}%
\pgfsetlinewidth{0.000000pt}%
\definecolor{currentstroke}{rgb}{0.000000,0.000000,0.000000}%
\pgfsetstrokecolor{currentstroke}%
\pgfsetstrokeopacity{0.000000}%
\pgfsetdash{}{0pt}%
\pgfpathmoveto{\pgfqpoint{5.800000in}{17.893790in}}%
\pgfpathlineto{\pgfqpoint{5.800000in}{17.893790in}}%
\pgfpathlineto{\pgfqpoint{5.800000in}{17.996627in}}%
\pgfpathlineto{\pgfqpoint{5.800000in}{17.996627in}}%
\pgfpathclose%
\pgfusepath{fill}%
\end{pgfscope}%
\begin{pgfscope}%
\pgfpathrectangle{\pgfqpoint{5.800000in}{0.720000in}}{\pgfqpoint{1.400000in}{4.620000in}}%
\pgfusepath{clip}%
\pgfsetbuttcap%
\pgfsetmiterjoin%
\definecolor{currentfill}{rgb}{0.121569,0.466667,0.705882}%
\pgfsetfillcolor{currentfill}%
\pgfsetlinewidth{0.000000pt}%
\definecolor{currentstroke}{rgb}{0.000000,0.000000,0.000000}%
\pgfsetstrokecolor{currentstroke}%
\pgfsetstrokeopacity{0.000000}%
\pgfsetdash{}{0pt}%
\pgfpathmoveto{\pgfqpoint{5.800000in}{17.996627in}}%
\pgfpathlineto{\pgfqpoint{5.800000in}{17.996627in}}%
\pgfpathlineto{\pgfqpoint{5.800000in}{18.099464in}}%
\pgfpathlineto{\pgfqpoint{5.800000in}{18.099464in}}%
\pgfpathclose%
\pgfusepath{fill}%
\end{pgfscope}%
\begin{pgfscope}%
\pgfpathrectangle{\pgfqpoint{5.800000in}{0.720000in}}{\pgfqpoint{1.400000in}{4.620000in}}%
\pgfusepath{clip}%
\pgfsetbuttcap%
\pgfsetmiterjoin%
\definecolor{currentfill}{rgb}{0.121569,0.466667,0.705882}%
\pgfsetfillcolor{currentfill}%
\pgfsetlinewidth{0.000000pt}%
\definecolor{currentstroke}{rgb}{0.000000,0.000000,0.000000}%
\pgfsetstrokecolor{currentstroke}%
\pgfsetstrokeopacity{0.000000}%
\pgfsetdash{}{0pt}%
\pgfpathmoveto{\pgfqpoint{5.800000in}{18.099464in}}%
\pgfpathlineto{\pgfqpoint{5.800000in}{18.099464in}}%
\pgfpathlineto{\pgfqpoint{5.800000in}{18.202302in}}%
\pgfpathlineto{\pgfqpoint{5.800000in}{18.202302in}}%
\pgfpathclose%
\pgfusepath{fill}%
\end{pgfscope}%
\begin{pgfscope}%
\pgfpathrectangle{\pgfqpoint{5.800000in}{0.720000in}}{\pgfqpoint{1.400000in}{4.620000in}}%
\pgfusepath{clip}%
\pgfsetbuttcap%
\pgfsetmiterjoin%
\definecolor{currentfill}{rgb}{0.121569,0.466667,0.705882}%
\pgfsetfillcolor{currentfill}%
\pgfsetlinewidth{0.000000pt}%
\definecolor{currentstroke}{rgb}{0.000000,0.000000,0.000000}%
\pgfsetstrokecolor{currentstroke}%
\pgfsetstrokeopacity{0.000000}%
\pgfsetdash{}{0pt}%
\pgfpathmoveto{\pgfqpoint{5.800000in}{18.202302in}}%
\pgfpathlineto{\pgfqpoint{5.800000in}{18.202302in}}%
\pgfpathlineto{\pgfqpoint{5.800000in}{18.305139in}}%
\pgfpathlineto{\pgfqpoint{5.800000in}{18.305139in}}%
\pgfpathclose%
\pgfusepath{fill}%
\end{pgfscope}%
\begin{pgfscope}%
\pgfpathrectangle{\pgfqpoint{5.800000in}{0.720000in}}{\pgfqpoint{1.400000in}{4.620000in}}%
\pgfusepath{clip}%
\pgfsetbuttcap%
\pgfsetmiterjoin%
\definecolor{currentfill}{rgb}{0.121569,0.466667,0.705882}%
\pgfsetfillcolor{currentfill}%
\pgfsetlinewidth{0.000000pt}%
\definecolor{currentstroke}{rgb}{0.000000,0.000000,0.000000}%
\pgfsetstrokecolor{currentstroke}%
\pgfsetstrokeopacity{0.000000}%
\pgfsetdash{}{0pt}%
\pgfpathmoveto{\pgfqpoint{5.800000in}{18.305139in}}%
\pgfpathlineto{\pgfqpoint{5.800000in}{18.305139in}}%
\pgfpathlineto{\pgfqpoint{5.800000in}{18.407976in}}%
\pgfpathlineto{\pgfqpoint{5.800000in}{18.407976in}}%
\pgfpathclose%
\pgfusepath{fill}%
\end{pgfscope}%
\begin{pgfscope}%
\pgfpathrectangle{\pgfqpoint{5.800000in}{0.720000in}}{\pgfqpoint{1.400000in}{4.620000in}}%
\pgfusepath{clip}%
\pgfsetbuttcap%
\pgfsetmiterjoin%
\definecolor{currentfill}{rgb}{0.121569,0.466667,0.705882}%
\pgfsetfillcolor{currentfill}%
\pgfsetlinewidth{0.000000pt}%
\definecolor{currentstroke}{rgb}{0.000000,0.000000,0.000000}%
\pgfsetstrokecolor{currentstroke}%
\pgfsetstrokeopacity{0.000000}%
\pgfsetdash{}{0pt}%
\pgfpathmoveto{\pgfqpoint{5.800000in}{18.407976in}}%
\pgfpathlineto{\pgfqpoint{5.800000in}{18.407976in}}%
\pgfpathlineto{\pgfqpoint{5.800000in}{18.510813in}}%
\pgfpathlineto{\pgfqpoint{5.800000in}{18.510813in}}%
\pgfpathclose%
\pgfusepath{fill}%
\end{pgfscope}%
\begin{pgfscope}%
\pgfpathrectangle{\pgfqpoint{5.800000in}{0.720000in}}{\pgfqpoint{1.400000in}{4.620000in}}%
\pgfusepath{clip}%
\pgfsetbuttcap%
\pgfsetmiterjoin%
\definecolor{currentfill}{rgb}{0.121569,0.466667,0.705882}%
\pgfsetfillcolor{currentfill}%
\pgfsetlinewidth{0.000000pt}%
\definecolor{currentstroke}{rgb}{0.000000,0.000000,0.000000}%
\pgfsetstrokecolor{currentstroke}%
\pgfsetstrokeopacity{0.000000}%
\pgfsetdash{}{0pt}%
\pgfpathmoveto{\pgfqpoint{5.800000in}{18.510813in}}%
\pgfpathlineto{\pgfqpoint{5.800000in}{18.510813in}}%
\pgfpathlineto{\pgfqpoint{5.800000in}{18.613650in}}%
\pgfpathlineto{\pgfqpoint{5.800000in}{18.613650in}}%
\pgfpathclose%
\pgfusepath{fill}%
\end{pgfscope}%
\begin{pgfscope}%
\pgfpathrectangle{\pgfqpoint{5.800000in}{0.720000in}}{\pgfqpoint{1.400000in}{4.620000in}}%
\pgfusepath{clip}%
\pgfsetbuttcap%
\pgfsetmiterjoin%
\definecolor{currentfill}{rgb}{0.121569,0.466667,0.705882}%
\pgfsetfillcolor{currentfill}%
\pgfsetlinewidth{0.000000pt}%
\definecolor{currentstroke}{rgb}{0.000000,0.000000,0.000000}%
\pgfsetstrokecolor{currentstroke}%
\pgfsetstrokeopacity{0.000000}%
\pgfsetdash{}{0pt}%
\pgfpathmoveto{\pgfqpoint{5.800000in}{18.613650in}}%
\pgfpathlineto{\pgfqpoint{5.800000in}{18.613650in}}%
\pgfpathlineto{\pgfqpoint{5.800000in}{18.716487in}}%
\pgfpathlineto{\pgfqpoint{5.800000in}{18.716487in}}%
\pgfpathclose%
\pgfusepath{fill}%
\end{pgfscope}%
\begin{pgfscope}%
\pgfpathrectangle{\pgfqpoint{5.800000in}{0.720000in}}{\pgfqpoint{1.400000in}{4.620000in}}%
\pgfusepath{clip}%
\pgfsetbuttcap%
\pgfsetmiterjoin%
\definecolor{currentfill}{rgb}{0.121569,0.466667,0.705882}%
\pgfsetfillcolor{currentfill}%
\pgfsetlinewidth{0.000000pt}%
\definecolor{currentstroke}{rgb}{0.000000,0.000000,0.000000}%
\pgfsetstrokecolor{currentstroke}%
\pgfsetstrokeopacity{0.000000}%
\pgfsetdash{}{0pt}%
\pgfpathmoveto{\pgfqpoint{5.800000in}{18.716487in}}%
\pgfpathlineto{\pgfqpoint{5.800000in}{18.716487in}}%
\pgfpathlineto{\pgfqpoint{5.800000in}{18.819324in}}%
\pgfpathlineto{\pgfqpoint{5.800000in}{18.819324in}}%
\pgfpathclose%
\pgfusepath{fill}%
\end{pgfscope}%
\begin{pgfscope}%
\pgfpathrectangle{\pgfqpoint{5.800000in}{0.720000in}}{\pgfqpoint{1.400000in}{4.620000in}}%
\pgfusepath{clip}%
\pgfsetbuttcap%
\pgfsetmiterjoin%
\definecolor{currentfill}{rgb}{0.121569,0.466667,0.705882}%
\pgfsetfillcolor{currentfill}%
\pgfsetlinewidth{0.000000pt}%
\definecolor{currentstroke}{rgb}{0.000000,0.000000,0.000000}%
\pgfsetstrokecolor{currentstroke}%
\pgfsetstrokeopacity{0.000000}%
\pgfsetdash{}{0pt}%
\pgfpathmoveto{\pgfqpoint{5.800000in}{18.819324in}}%
\pgfpathlineto{\pgfqpoint{5.800000in}{18.819324in}}%
\pgfpathlineto{\pgfqpoint{5.800000in}{18.922161in}}%
\pgfpathlineto{\pgfqpoint{5.800000in}{18.922161in}}%
\pgfpathclose%
\pgfusepath{fill}%
\end{pgfscope}%
\begin{pgfscope}%
\pgfpathrectangle{\pgfqpoint{5.800000in}{0.720000in}}{\pgfqpoint{1.400000in}{4.620000in}}%
\pgfusepath{clip}%
\pgfsetbuttcap%
\pgfsetmiterjoin%
\definecolor{currentfill}{rgb}{0.121569,0.466667,0.705882}%
\pgfsetfillcolor{currentfill}%
\pgfsetlinewidth{0.000000pt}%
\definecolor{currentstroke}{rgb}{0.000000,0.000000,0.000000}%
\pgfsetstrokecolor{currentstroke}%
\pgfsetstrokeopacity{0.000000}%
\pgfsetdash{}{0pt}%
\pgfpathmoveto{\pgfqpoint{5.800000in}{18.922161in}}%
\pgfpathlineto{\pgfqpoint{5.800000in}{18.922161in}}%
\pgfpathlineto{\pgfqpoint{5.800000in}{19.024998in}}%
\pgfpathlineto{\pgfqpoint{5.800000in}{19.024998in}}%
\pgfpathclose%
\pgfusepath{fill}%
\end{pgfscope}%
\begin{pgfscope}%
\pgfpathrectangle{\pgfqpoint{5.800000in}{0.720000in}}{\pgfqpoint{1.400000in}{4.620000in}}%
\pgfusepath{clip}%
\pgfsetbuttcap%
\pgfsetmiterjoin%
\definecolor{currentfill}{rgb}{0.121569,0.466667,0.705882}%
\pgfsetfillcolor{currentfill}%
\pgfsetlinewidth{0.000000pt}%
\definecolor{currentstroke}{rgb}{0.000000,0.000000,0.000000}%
\pgfsetstrokecolor{currentstroke}%
\pgfsetstrokeopacity{0.000000}%
\pgfsetdash{}{0pt}%
\pgfpathmoveto{\pgfqpoint{5.800000in}{19.024998in}}%
\pgfpathlineto{\pgfqpoint{5.800000in}{19.024998in}}%
\pgfpathlineto{\pgfqpoint{5.800000in}{19.127835in}}%
\pgfpathlineto{\pgfqpoint{5.800000in}{19.127835in}}%
\pgfpathclose%
\pgfusepath{fill}%
\end{pgfscope}%
\begin{pgfscope}%
\pgfpathrectangle{\pgfqpoint{5.800000in}{0.720000in}}{\pgfqpoint{1.400000in}{4.620000in}}%
\pgfusepath{clip}%
\pgfsetbuttcap%
\pgfsetmiterjoin%
\definecolor{currentfill}{rgb}{0.121569,0.466667,0.705882}%
\pgfsetfillcolor{currentfill}%
\pgfsetlinewidth{0.000000pt}%
\definecolor{currentstroke}{rgb}{0.000000,0.000000,0.000000}%
\pgfsetstrokecolor{currentstroke}%
\pgfsetstrokeopacity{0.000000}%
\pgfsetdash{}{0pt}%
\pgfpathmoveto{\pgfqpoint{5.800000in}{19.127835in}}%
\pgfpathlineto{\pgfqpoint{5.800000in}{19.127835in}}%
\pgfpathlineto{\pgfqpoint{5.800000in}{19.230672in}}%
\pgfpathlineto{\pgfqpoint{5.800000in}{19.230672in}}%
\pgfpathclose%
\pgfusepath{fill}%
\end{pgfscope}%
\begin{pgfscope}%
\pgfpathrectangle{\pgfqpoint{5.800000in}{0.720000in}}{\pgfqpoint{1.400000in}{4.620000in}}%
\pgfusepath{clip}%
\pgfsetbuttcap%
\pgfsetmiterjoin%
\definecolor{currentfill}{rgb}{0.121569,0.466667,0.705882}%
\pgfsetfillcolor{currentfill}%
\pgfsetlinewidth{0.000000pt}%
\definecolor{currentstroke}{rgb}{0.000000,0.000000,0.000000}%
\pgfsetstrokecolor{currentstroke}%
\pgfsetstrokeopacity{0.000000}%
\pgfsetdash{}{0pt}%
\pgfpathmoveto{\pgfqpoint{5.800000in}{19.230672in}}%
\pgfpathlineto{\pgfqpoint{5.800000in}{19.230672in}}%
\pgfpathlineto{\pgfqpoint{5.800000in}{19.333509in}}%
\pgfpathlineto{\pgfqpoint{5.800000in}{19.333509in}}%
\pgfpathclose%
\pgfusepath{fill}%
\end{pgfscope}%
\begin{pgfscope}%
\pgfpathrectangle{\pgfqpoint{5.800000in}{0.720000in}}{\pgfqpoint{1.400000in}{4.620000in}}%
\pgfusepath{clip}%
\pgfsetbuttcap%
\pgfsetmiterjoin%
\definecolor{currentfill}{rgb}{0.121569,0.466667,0.705882}%
\pgfsetfillcolor{currentfill}%
\pgfsetlinewidth{0.000000pt}%
\definecolor{currentstroke}{rgb}{0.000000,0.000000,0.000000}%
\pgfsetstrokecolor{currentstroke}%
\pgfsetstrokeopacity{0.000000}%
\pgfsetdash{}{0pt}%
\pgfpathmoveto{\pgfqpoint{5.800000in}{19.333509in}}%
\pgfpathlineto{\pgfqpoint{5.800000in}{19.333509in}}%
\pgfpathlineto{\pgfqpoint{5.800000in}{19.436346in}}%
\pgfpathlineto{\pgfqpoint{5.800000in}{19.436346in}}%
\pgfpathclose%
\pgfusepath{fill}%
\end{pgfscope}%
\begin{pgfscope}%
\pgfpathrectangle{\pgfqpoint{5.800000in}{0.720000in}}{\pgfqpoint{1.400000in}{4.620000in}}%
\pgfusepath{clip}%
\pgfsetbuttcap%
\pgfsetmiterjoin%
\definecolor{currentfill}{rgb}{0.121569,0.466667,0.705882}%
\pgfsetfillcolor{currentfill}%
\pgfsetlinewidth{0.000000pt}%
\definecolor{currentstroke}{rgb}{0.000000,0.000000,0.000000}%
\pgfsetstrokecolor{currentstroke}%
\pgfsetstrokeopacity{0.000000}%
\pgfsetdash{}{0pt}%
\pgfpathmoveto{\pgfqpoint{5.800000in}{19.436346in}}%
\pgfpathlineto{\pgfqpoint{5.800000in}{19.436346in}}%
\pgfpathlineto{\pgfqpoint{5.800000in}{19.539183in}}%
\pgfpathlineto{\pgfqpoint{5.800000in}{19.539183in}}%
\pgfpathclose%
\pgfusepath{fill}%
\end{pgfscope}%
\begin{pgfscope}%
\pgfpathrectangle{\pgfqpoint{5.800000in}{0.720000in}}{\pgfqpoint{1.400000in}{4.620000in}}%
\pgfusepath{clip}%
\pgfsetbuttcap%
\pgfsetmiterjoin%
\definecolor{currentfill}{rgb}{0.121569,0.466667,0.705882}%
\pgfsetfillcolor{currentfill}%
\pgfsetlinewidth{0.000000pt}%
\definecolor{currentstroke}{rgb}{0.000000,0.000000,0.000000}%
\pgfsetstrokecolor{currentstroke}%
\pgfsetstrokeopacity{0.000000}%
\pgfsetdash{}{0pt}%
\pgfpathmoveto{\pgfqpoint{5.800000in}{19.539183in}}%
\pgfpathlineto{\pgfqpoint{5.800000in}{19.539183in}}%
\pgfpathlineto{\pgfqpoint{5.800000in}{19.642020in}}%
\pgfpathlineto{\pgfqpoint{5.800000in}{19.642020in}}%
\pgfpathclose%
\pgfusepath{fill}%
\end{pgfscope}%
\begin{pgfscope}%
\pgfpathrectangle{\pgfqpoint{5.800000in}{0.720000in}}{\pgfqpoint{1.400000in}{4.620000in}}%
\pgfusepath{clip}%
\pgfsetbuttcap%
\pgfsetmiterjoin%
\definecolor{currentfill}{rgb}{0.121569,0.466667,0.705882}%
\pgfsetfillcolor{currentfill}%
\pgfsetlinewidth{0.000000pt}%
\definecolor{currentstroke}{rgb}{0.000000,0.000000,0.000000}%
\pgfsetstrokecolor{currentstroke}%
\pgfsetstrokeopacity{0.000000}%
\pgfsetdash{}{0pt}%
\pgfpathmoveto{\pgfqpoint{5.800000in}{19.642020in}}%
\pgfpathlineto{\pgfqpoint{5.800000in}{19.642020in}}%
\pgfpathlineto{\pgfqpoint{5.800000in}{19.744858in}}%
\pgfpathlineto{\pgfqpoint{5.800000in}{19.744858in}}%
\pgfpathclose%
\pgfusepath{fill}%
\end{pgfscope}%
\begin{pgfscope}%
\pgfpathrectangle{\pgfqpoint{5.800000in}{0.720000in}}{\pgfqpoint{1.400000in}{4.620000in}}%
\pgfusepath{clip}%
\pgfsetbuttcap%
\pgfsetmiterjoin%
\definecolor{currentfill}{rgb}{0.121569,0.466667,0.705882}%
\pgfsetfillcolor{currentfill}%
\pgfsetlinewidth{0.000000pt}%
\definecolor{currentstroke}{rgb}{0.000000,0.000000,0.000000}%
\pgfsetstrokecolor{currentstroke}%
\pgfsetstrokeopacity{0.000000}%
\pgfsetdash{}{0pt}%
\pgfpathmoveto{\pgfqpoint{5.800000in}{19.744858in}}%
\pgfpathlineto{\pgfqpoint{5.800000in}{19.744858in}}%
\pgfpathlineto{\pgfqpoint{5.800000in}{19.847695in}}%
\pgfpathlineto{\pgfqpoint{5.800000in}{19.847695in}}%
\pgfpathclose%
\pgfusepath{fill}%
\end{pgfscope}%
\begin{pgfscope}%
\pgfpathrectangle{\pgfqpoint{5.800000in}{0.720000in}}{\pgfqpoint{1.400000in}{4.620000in}}%
\pgfusepath{clip}%
\pgfsetbuttcap%
\pgfsetmiterjoin%
\definecolor{currentfill}{rgb}{0.121569,0.466667,0.705882}%
\pgfsetfillcolor{currentfill}%
\pgfsetlinewidth{0.000000pt}%
\definecolor{currentstroke}{rgb}{0.000000,0.000000,0.000000}%
\pgfsetstrokecolor{currentstroke}%
\pgfsetstrokeopacity{0.000000}%
\pgfsetdash{}{0pt}%
\pgfpathmoveto{\pgfqpoint{5.800000in}{19.847695in}}%
\pgfpathlineto{\pgfqpoint{5.800000in}{19.847695in}}%
\pgfpathlineto{\pgfqpoint{5.800000in}{19.950532in}}%
\pgfpathlineto{\pgfqpoint{5.800000in}{19.950532in}}%
\pgfpathclose%
\pgfusepath{fill}%
\end{pgfscope}%
\begin{pgfscope}%
\pgfpathrectangle{\pgfqpoint{5.800000in}{0.720000in}}{\pgfqpoint{1.400000in}{4.620000in}}%
\pgfusepath{clip}%
\pgfsetbuttcap%
\pgfsetmiterjoin%
\definecolor{currentfill}{rgb}{0.121569,0.466667,0.705882}%
\pgfsetfillcolor{currentfill}%
\pgfsetlinewidth{0.000000pt}%
\definecolor{currentstroke}{rgb}{0.000000,0.000000,0.000000}%
\pgfsetstrokecolor{currentstroke}%
\pgfsetstrokeopacity{0.000000}%
\pgfsetdash{}{0pt}%
\pgfpathmoveto{\pgfqpoint{5.800000in}{19.950532in}}%
\pgfpathlineto{\pgfqpoint{5.800000in}{19.950532in}}%
\pgfpathlineto{\pgfqpoint{5.800000in}{20.053369in}}%
\pgfpathlineto{\pgfqpoint{5.800000in}{20.053369in}}%
\pgfpathclose%
\pgfusepath{fill}%
\end{pgfscope}%
\begin{pgfscope}%
\pgfpathrectangle{\pgfqpoint{5.800000in}{0.720000in}}{\pgfqpoint{1.400000in}{4.620000in}}%
\pgfusepath{clip}%
\pgfsetbuttcap%
\pgfsetmiterjoin%
\definecolor{currentfill}{rgb}{0.121569,0.466667,0.705882}%
\pgfsetfillcolor{currentfill}%
\pgfsetlinewidth{0.000000pt}%
\definecolor{currentstroke}{rgb}{0.000000,0.000000,0.000000}%
\pgfsetstrokecolor{currentstroke}%
\pgfsetstrokeopacity{0.000000}%
\pgfsetdash{}{0pt}%
\pgfpathmoveto{\pgfqpoint{5.800000in}{20.053369in}}%
\pgfpathlineto{\pgfqpoint{5.800000in}{20.053369in}}%
\pgfpathlineto{\pgfqpoint{5.800000in}{20.156206in}}%
\pgfpathlineto{\pgfqpoint{5.800000in}{20.156206in}}%
\pgfpathclose%
\pgfusepath{fill}%
\end{pgfscope}%
\begin{pgfscope}%
\pgfpathrectangle{\pgfqpoint{5.800000in}{0.720000in}}{\pgfqpoint{1.400000in}{4.620000in}}%
\pgfusepath{clip}%
\pgfsetbuttcap%
\pgfsetmiterjoin%
\definecolor{currentfill}{rgb}{0.121569,0.466667,0.705882}%
\pgfsetfillcolor{currentfill}%
\pgfsetlinewidth{0.000000pt}%
\definecolor{currentstroke}{rgb}{0.000000,0.000000,0.000000}%
\pgfsetstrokecolor{currentstroke}%
\pgfsetstrokeopacity{0.000000}%
\pgfsetdash{}{0pt}%
\pgfpathmoveto{\pgfqpoint{5.800000in}{20.156206in}}%
\pgfpathlineto{\pgfqpoint{5.800000in}{20.156206in}}%
\pgfpathlineto{\pgfqpoint{5.800000in}{20.259043in}}%
\pgfpathlineto{\pgfqpoint{5.800000in}{20.259043in}}%
\pgfpathclose%
\pgfusepath{fill}%
\end{pgfscope}%
\begin{pgfscope}%
\pgfpathrectangle{\pgfqpoint{5.800000in}{0.720000in}}{\pgfqpoint{1.400000in}{4.620000in}}%
\pgfusepath{clip}%
\pgfsetbuttcap%
\pgfsetmiterjoin%
\definecolor{currentfill}{rgb}{0.121569,0.466667,0.705882}%
\pgfsetfillcolor{currentfill}%
\pgfsetlinewidth{0.000000pt}%
\definecolor{currentstroke}{rgb}{0.000000,0.000000,0.000000}%
\pgfsetstrokecolor{currentstroke}%
\pgfsetstrokeopacity{0.000000}%
\pgfsetdash{}{0pt}%
\pgfpathmoveto{\pgfqpoint{5.800000in}{20.259043in}}%
\pgfpathlineto{\pgfqpoint{5.800000in}{20.259043in}}%
\pgfpathlineto{\pgfqpoint{5.800000in}{20.361880in}}%
\pgfpathlineto{\pgfqpoint{5.800000in}{20.361880in}}%
\pgfpathclose%
\pgfusepath{fill}%
\end{pgfscope}%
\begin{pgfscope}%
\pgfpathrectangle{\pgfqpoint{5.800000in}{0.720000in}}{\pgfqpoint{1.400000in}{4.620000in}}%
\pgfusepath{clip}%
\pgfsetbuttcap%
\pgfsetmiterjoin%
\definecolor{currentfill}{rgb}{0.121569,0.466667,0.705882}%
\pgfsetfillcolor{currentfill}%
\pgfsetlinewidth{0.000000pt}%
\definecolor{currentstroke}{rgb}{0.000000,0.000000,0.000000}%
\pgfsetstrokecolor{currentstroke}%
\pgfsetstrokeopacity{0.000000}%
\pgfsetdash{}{0pt}%
\pgfpathmoveto{\pgfqpoint{5.800000in}{20.361880in}}%
\pgfpathlineto{\pgfqpoint{5.800000in}{20.361880in}}%
\pgfpathlineto{\pgfqpoint{5.800000in}{20.464717in}}%
\pgfpathlineto{\pgfqpoint{5.800000in}{20.464717in}}%
\pgfpathclose%
\pgfusepath{fill}%
\end{pgfscope}%
\begin{pgfscope}%
\pgfpathrectangle{\pgfqpoint{5.800000in}{0.720000in}}{\pgfqpoint{1.400000in}{4.620000in}}%
\pgfusepath{clip}%
\pgfsetbuttcap%
\pgfsetmiterjoin%
\definecolor{currentfill}{rgb}{0.121569,0.466667,0.705882}%
\pgfsetfillcolor{currentfill}%
\pgfsetlinewidth{0.000000pt}%
\definecolor{currentstroke}{rgb}{0.000000,0.000000,0.000000}%
\pgfsetstrokecolor{currentstroke}%
\pgfsetstrokeopacity{0.000000}%
\pgfsetdash{}{0pt}%
\pgfpathmoveto{\pgfqpoint{5.800000in}{20.464717in}}%
\pgfpathlineto{\pgfqpoint{5.800000in}{20.464717in}}%
\pgfpathlineto{\pgfqpoint{5.800000in}{20.567554in}}%
\pgfpathlineto{\pgfqpoint{5.800000in}{20.567554in}}%
\pgfpathclose%
\pgfusepath{fill}%
\end{pgfscope}%
\begin{pgfscope}%
\pgfpathrectangle{\pgfqpoint{5.800000in}{0.720000in}}{\pgfqpoint{1.400000in}{4.620000in}}%
\pgfusepath{clip}%
\pgfsetbuttcap%
\pgfsetmiterjoin%
\definecolor{currentfill}{rgb}{0.121569,0.466667,0.705882}%
\pgfsetfillcolor{currentfill}%
\pgfsetlinewidth{0.000000pt}%
\definecolor{currentstroke}{rgb}{0.000000,0.000000,0.000000}%
\pgfsetstrokecolor{currentstroke}%
\pgfsetstrokeopacity{0.000000}%
\pgfsetdash{}{0pt}%
\pgfpathmoveto{\pgfqpoint{5.800000in}{20.567554in}}%
\pgfpathlineto{\pgfqpoint{5.800000in}{20.567554in}}%
\pgfpathlineto{\pgfqpoint{5.800000in}{20.670391in}}%
\pgfpathlineto{\pgfqpoint{5.800000in}{20.670391in}}%
\pgfpathclose%
\pgfusepath{fill}%
\end{pgfscope}%
\begin{pgfscope}%
\pgfpathrectangle{\pgfqpoint{5.800000in}{0.720000in}}{\pgfqpoint{1.400000in}{4.620000in}}%
\pgfusepath{clip}%
\pgfsetbuttcap%
\pgfsetmiterjoin%
\definecolor{currentfill}{rgb}{0.121569,0.466667,0.705882}%
\pgfsetfillcolor{currentfill}%
\pgfsetlinewidth{0.000000pt}%
\definecolor{currentstroke}{rgb}{0.000000,0.000000,0.000000}%
\pgfsetstrokecolor{currentstroke}%
\pgfsetstrokeopacity{0.000000}%
\pgfsetdash{}{0pt}%
\pgfpathmoveto{\pgfqpoint{5.800000in}{20.670391in}}%
\pgfpathlineto{\pgfqpoint{5.802038in}{20.670391in}}%
\pgfpathlineto{\pgfqpoint{5.802038in}{20.773228in}}%
\pgfpathlineto{\pgfqpoint{5.800000in}{20.773228in}}%
\pgfpathclose%
\pgfusepath{fill}%
\end{pgfscope}%
\begin{pgfscope}%
\pgfpathrectangle{\pgfqpoint{5.800000in}{0.720000in}}{\pgfqpoint{1.400000in}{4.620000in}}%
\pgfusepath{clip}%
\pgfsetbuttcap%
\pgfsetmiterjoin%
\definecolor{currentfill}{rgb}{0.121569,0.466667,0.705882}%
\pgfsetfillcolor{currentfill}%
\pgfsetlinewidth{0.000000pt}%
\definecolor{currentstroke}{rgb}{0.000000,0.000000,0.000000}%
\pgfsetstrokecolor{currentstroke}%
\pgfsetstrokeopacity{0.000000}%
\pgfsetdash{}{0pt}%
\pgfpathmoveto{\pgfqpoint{5.800000in}{20.773228in}}%
\pgfpathlineto{\pgfqpoint{5.800000in}{20.773228in}}%
\pgfpathlineto{\pgfqpoint{5.800000in}{20.876065in}}%
\pgfpathlineto{\pgfqpoint{5.800000in}{20.876065in}}%
\pgfpathclose%
\pgfusepath{fill}%
\end{pgfscope}%
\begin{pgfscope}%
\pgfpathrectangle{\pgfqpoint{5.800000in}{0.720000in}}{\pgfqpoint{1.400000in}{4.620000in}}%
\pgfusepath{clip}%
\pgfsetbuttcap%
\pgfsetmiterjoin%
\definecolor{currentfill}{rgb}{0.121569,0.466667,0.705882}%
\pgfsetfillcolor{currentfill}%
\pgfsetlinewidth{0.000000pt}%
\definecolor{currentstroke}{rgb}{0.000000,0.000000,0.000000}%
\pgfsetstrokecolor{currentstroke}%
\pgfsetstrokeopacity{0.000000}%
\pgfsetdash{}{0pt}%
\pgfpathmoveto{\pgfqpoint{5.800000in}{20.876065in}}%
\pgfpathlineto{\pgfqpoint{5.800000in}{20.876065in}}%
\pgfpathlineto{\pgfqpoint{5.800000in}{20.978902in}}%
\pgfpathlineto{\pgfqpoint{5.800000in}{20.978902in}}%
\pgfpathclose%
\pgfusepath{fill}%
\end{pgfscope}%
\begin{pgfscope}%
\pgfpathrectangle{\pgfqpoint{5.800000in}{0.720000in}}{\pgfqpoint{1.400000in}{4.620000in}}%
\pgfusepath{clip}%
\pgfsetbuttcap%
\pgfsetmiterjoin%
\definecolor{currentfill}{rgb}{0.121569,0.466667,0.705882}%
\pgfsetfillcolor{currentfill}%
\pgfsetlinewidth{0.000000pt}%
\definecolor{currentstroke}{rgb}{0.000000,0.000000,0.000000}%
\pgfsetstrokecolor{currentstroke}%
\pgfsetstrokeopacity{0.000000}%
\pgfsetdash{}{0pt}%
\pgfpathmoveto{\pgfqpoint{5.800000in}{20.978902in}}%
\pgfpathlineto{\pgfqpoint{5.800000in}{20.978902in}}%
\pgfpathlineto{\pgfqpoint{5.800000in}{21.081739in}}%
\pgfpathlineto{\pgfqpoint{5.800000in}{21.081739in}}%
\pgfpathclose%
\pgfusepath{fill}%
\end{pgfscope}%
\begin{pgfscope}%
\pgfpathrectangle{\pgfqpoint{5.800000in}{0.720000in}}{\pgfqpoint{1.400000in}{4.620000in}}%
\pgfusepath{clip}%
\pgfsetbuttcap%
\pgfsetmiterjoin%
\definecolor{currentfill}{rgb}{0.121569,0.466667,0.705882}%
\pgfsetfillcolor{currentfill}%
\pgfsetlinewidth{0.000000pt}%
\definecolor{currentstroke}{rgb}{0.000000,0.000000,0.000000}%
\pgfsetstrokecolor{currentstroke}%
\pgfsetstrokeopacity{0.000000}%
\pgfsetdash{}{0pt}%
\pgfpathmoveto{\pgfqpoint{5.800000in}{21.081739in}}%
\pgfpathlineto{\pgfqpoint{5.800000in}{21.081739in}}%
\pgfpathlineto{\pgfqpoint{5.800000in}{21.184576in}}%
\pgfpathlineto{\pgfqpoint{5.800000in}{21.184576in}}%
\pgfpathclose%
\pgfusepath{fill}%
\end{pgfscope}%
\begin{pgfscope}%
\pgfpathrectangle{\pgfqpoint{5.800000in}{0.720000in}}{\pgfqpoint{1.400000in}{4.620000in}}%
\pgfusepath{clip}%
\pgfsetbuttcap%
\pgfsetmiterjoin%
\definecolor{currentfill}{rgb}{0.121569,0.466667,0.705882}%
\pgfsetfillcolor{currentfill}%
\pgfsetlinewidth{0.000000pt}%
\definecolor{currentstroke}{rgb}{0.000000,0.000000,0.000000}%
\pgfsetstrokecolor{currentstroke}%
\pgfsetstrokeopacity{0.000000}%
\pgfsetdash{}{0pt}%
\pgfpathmoveto{\pgfqpoint{5.800000in}{21.184576in}}%
\pgfpathlineto{\pgfqpoint{5.800000in}{21.184576in}}%
\pgfpathlineto{\pgfqpoint{5.800000in}{21.287414in}}%
\pgfpathlineto{\pgfqpoint{5.800000in}{21.287414in}}%
\pgfpathclose%
\pgfusepath{fill}%
\end{pgfscope}%
\begin{pgfscope}%
\pgfpathrectangle{\pgfqpoint{5.800000in}{0.720000in}}{\pgfqpoint{1.400000in}{4.620000in}}%
\pgfusepath{clip}%
\pgfsetbuttcap%
\pgfsetmiterjoin%
\definecolor{currentfill}{rgb}{0.121569,0.466667,0.705882}%
\pgfsetfillcolor{currentfill}%
\pgfsetlinewidth{0.000000pt}%
\definecolor{currentstroke}{rgb}{0.000000,0.000000,0.000000}%
\pgfsetstrokecolor{currentstroke}%
\pgfsetstrokeopacity{0.000000}%
\pgfsetdash{}{0pt}%
\pgfpathmoveto{\pgfqpoint{5.800000in}{21.287414in}}%
\pgfpathlineto{\pgfqpoint{5.800000in}{21.287414in}}%
\pgfpathlineto{\pgfqpoint{5.800000in}{21.390251in}}%
\pgfpathlineto{\pgfqpoint{5.800000in}{21.390251in}}%
\pgfpathclose%
\pgfusepath{fill}%
\end{pgfscope}%
\begin{pgfscope}%
\pgfpathrectangle{\pgfqpoint{5.800000in}{0.720000in}}{\pgfqpoint{1.400000in}{4.620000in}}%
\pgfusepath{clip}%
\pgfsetbuttcap%
\pgfsetmiterjoin%
\definecolor{currentfill}{rgb}{0.121569,0.466667,0.705882}%
\pgfsetfillcolor{currentfill}%
\pgfsetlinewidth{0.000000pt}%
\definecolor{currentstroke}{rgb}{0.000000,0.000000,0.000000}%
\pgfsetstrokecolor{currentstroke}%
\pgfsetstrokeopacity{0.000000}%
\pgfsetdash{}{0pt}%
\pgfpathmoveto{\pgfqpoint{5.800000in}{21.390251in}}%
\pgfpathlineto{\pgfqpoint{5.800000in}{21.390251in}}%
\pgfpathlineto{\pgfqpoint{5.800000in}{21.493088in}}%
\pgfpathlineto{\pgfqpoint{5.800000in}{21.493088in}}%
\pgfpathclose%
\pgfusepath{fill}%
\end{pgfscope}%
\begin{pgfscope}%
\pgfpathrectangle{\pgfqpoint{5.800000in}{0.720000in}}{\pgfqpoint{1.400000in}{4.620000in}}%
\pgfusepath{clip}%
\pgfsetbuttcap%
\pgfsetmiterjoin%
\definecolor{currentfill}{rgb}{0.121569,0.466667,0.705882}%
\pgfsetfillcolor{currentfill}%
\pgfsetlinewidth{0.000000pt}%
\definecolor{currentstroke}{rgb}{0.000000,0.000000,0.000000}%
\pgfsetstrokecolor{currentstroke}%
\pgfsetstrokeopacity{0.000000}%
\pgfsetdash{}{0pt}%
\pgfpathmoveto{\pgfqpoint{5.800000in}{21.493088in}}%
\pgfpathlineto{\pgfqpoint{5.800000in}{21.493088in}}%
\pgfpathlineto{\pgfqpoint{5.800000in}{21.595925in}}%
\pgfpathlineto{\pgfqpoint{5.800000in}{21.595925in}}%
\pgfpathclose%
\pgfusepath{fill}%
\end{pgfscope}%
\begin{pgfscope}%
\pgfpathrectangle{\pgfqpoint{5.800000in}{0.720000in}}{\pgfqpoint{1.400000in}{4.620000in}}%
\pgfusepath{clip}%
\pgfsetbuttcap%
\pgfsetmiterjoin%
\definecolor{currentfill}{rgb}{0.121569,0.466667,0.705882}%
\pgfsetfillcolor{currentfill}%
\pgfsetlinewidth{0.000000pt}%
\definecolor{currentstroke}{rgb}{0.000000,0.000000,0.000000}%
\pgfsetstrokecolor{currentstroke}%
\pgfsetstrokeopacity{0.000000}%
\pgfsetdash{}{0pt}%
\pgfpathmoveto{\pgfqpoint{5.800000in}{21.595925in}}%
\pgfpathlineto{\pgfqpoint{5.800000in}{21.595925in}}%
\pgfpathlineto{\pgfqpoint{5.800000in}{21.698762in}}%
\pgfpathlineto{\pgfqpoint{5.800000in}{21.698762in}}%
\pgfpathclose%
\pgfusepath{fill}%
\end{pgfscope}%
\begin{pgfscope}%
\pgfpathrectangle{\pgfqpoint{5.800000in}{0.720000in}}{\pgfqpoint{1.400000in}{4.620000in}}%
\pgfusepath{clip}%
\pgfsetbuttcap%
\pgfsetmiterjoin%
\definecolor{currentfill}{rgb}{0.121569,0.466667,0.705882}%
\pgfsetfillcolor{currentfill}%
\pgfsetlinewidth{0.000000pt}%
\definecolor{currentstroke}{rgb}{0.000000,0.000000,0.000000}%
\pgfsetstrokecolor{currentstroke}%
\pgfsetstrokeopacity{0.000000}%
\pgfsetdash{}{0pt}%
\pgfpathmoveto{\pgfqpoint{5.800000in}{21.698762in}}%
\pgfpathlineto{\pgfqpoint{5.800000in}{21.698762in}}%
\pgfpathlineto{\pgfqpoint{5.800000in}{21.801599in}}%
\pgfpathlineto{\pgfqpoint{5.800000in}{21.801599in}}%
\pgfpathclose%
\pgfusepath{fill}%
\end{pgfscope}%
\begin{pgfscope}%
\pgfpathrectangle{\pgfqpoint{5.800000in}{0.720000in}}{\pgfqpoint{1.400000in}{4.620000in}}%
\pgfusepath{clip}%
\pgfsetbuttcap%
\pgfsetmiterjoin%
\definecolor{currentfill}{rgb}{0.121569,0.466667,0.705882}%
\pgfsetfillcolor{currentfill}%
\pgfsetlinewidth{0.000000pt}%
\definecolor{currentstroke}{rgb}{0.000000,0.000000,0.000000}%
\pgfsetstrokecolor{currentstroke}%
\pgfsetstrokeopacity{0.000000}%
\pgfsetdash{}{0pt}%
\pgfpathmoveto{\pgfqpoint{5.800000in}{21.801599in}}%
\pgfpathlineto{\pgfqpoint{5.800000in}{21.801599in}}%
\pgfpathlineto{\pgfqpoint{5.800000in}{21.904436in}}%
\pgfpathlineto{\pgfqpoint{5.800000in}{21.904436in}}%
\pgfpathclose%
\pgfusepath{fill}%
\end{pgfscope}%
\begin{pgfscope}%
\pgfpathrectangle{\pgfqpoint{5.800000in}{0.720000in}}{\pgfqpoint{1.400000in}{4.620000in}}%
\pgfusepath{clip}%
\pgfsetbuttcap%
\pgfsetmiterjoin%
\definecolor{currentfill}{rgb}{0.121569,0.466667,0.705882}%
\pgfsetfillcolor{currentfill}%
\pgfsetlinewidth{0.000000pt}%
\definecolor{currentstroke}{rgb}{0.000000,0.000000,0.000000}%
\pgfsetstrokecolor{currentstroke}%
\pgfsetstrokeopacity{0.000000}%
\pgfsetdash{}{0pt}%
\pgfpathmoveto{\pgfqpoint{5.800000in}{21.904436in}}%
\pgfpathlineto{\pgfqpoint{5.800000in}{21.904436in}}%
\pgfpathlineto{\pgfqpoint{5.800000in}{22.007273in}}%
\pgfpathlineto{\pgfqpoint{5.800000in}{22.007273in}}%
\pgfpathclose%
\pgfusepath{fill}%
\end{pgfscope}%
\begin{pgfscope}%
\pgfpathrectangle{\pgfqpoint{5.800000in}{0.720000in}}{\pgfqpoint{1.400000in}{4.620000in}}%
\pgfusepath{clip}%
\pgfsetbuttcap%
\pgfsetmiterjoin%
\definecolor{currentfill}{rgb}{0.121569,0.466667,0.705882}%
\pgfsetfillcolor{currentfill}%
\pgfsetlinewidth{0.000000pt}%
\definecolor{currentstroke}{rgb}{0.000000,0.000000,0.000000}%
\pgfsetstrokecolor{currentstroke}%
\pgfsetstrokeopacity{0.000000}%
\pgfsetdash{}{0pt}%
\pgfpathmoveto{\pgfqpoint{5.800000in}{22.007273in}}%
\pgfpathlineto{\pgfqpoint{5.800000in}{22.007273in}}%
\pgfpathlineto{\pgfqpoint{5.800000in}{22.110110in}}%
\pgfpathlineto{\pgfqpoint{5.800000in}{22.110110in}}%
\pgfpathclose%
\pgfusepath{fill}%
\end{pgfscope}%
\begin{pgfscope}%
\pgfpathrectangle{\pgfqpoint{5.800000in}{0.720000in}}{\pgfqpoint{1.400000in}{4.620000in}}%
\pgfusepath{clip}%
\pgfsetbuttcap%
\pgfsetmiterjoin%
\definecolor{currentfill}{rgb}{0.121569,0.466667,0.705882}%
\pgfsetfillcolor{currentfill}%
\pgfsetlinewidth{0.000000pt}%
\definecolor{currentstroke}{rgb}{0.000000,0.000000,0.000000}%
\pgfsetstrokecolor{currentstroke}%
\pgfsetstrokeopacity{0.000000}%
\pgfsetdash{}{0pt}%
\pgfpathmoveto{\pgfqpoint{5.800000in}{22.110110in}}%
\pgfpathlineto{\pgfqpoint{5.800000in}{22.110110in}}%
\pgfpathlineto{\pgfqpoint{5.800000in}{22.212947in}}%
\pgfpathlineto{\pgfqpoint{5.800000in}{22.212947in}}%
\pgfpathclose%
\pgfusepath{fill}%
\end{pgfscope}%
\begin{pgfscope}%
\pgfpathrectangle{\pgfqpoint{5.800000in}{0.720000in}}{\pgfqpoint{1.400000in}{4.620000in}}%
\pgfusepath{clip}%
\pgfsetbuttcap%
\pgfsetmiterjoin%
\definecolor{currentfill}{rgb}{0.121569,0.466667,0.705882}%
\pgfsetfillcolor{currentfill}%
\pgfsetlinewidth{0.000000pt}%
\definecolor{currentstroke}{rgb}{0.000000,0.000000,0.000000}%
\pgfsetstrokecolor{currentstroke}%
\pgfsetstrokeopacity{0.000000}%
\pgfsetdash{}{0pt}%
\pgfpathmoveto{\pgfqpoint{5.800000in}{22.212947in}}%
\pgfpathlineto{\pgfqpoint{5.800000in}{22.212947in}}%
\pgfpathlineto{\pgfqpoint{5.800000in}{22.315784in}}%
\pgfpathlineto{\pgfqpoint{5.800000in}{22.315784in}}%
\pgfpathclose%
\pgfusepath{fill}%
\end{pgfscope}%
\begin{pgfscope}%
\pgfpathrectangle{\pgfqpoint{5.800000in}{0.720000in}}{\pgfqpoint{1.400000in}{4.620000in}}%
\pgfusepath{clip}%
\pgfsetbuttcap%
\pgfsetmiterjoin%
\definecolor{currentfill}{rgb}{0.121569,0.466667,0.705882}%
\pgfsetfillcolor{currentfill}%
\pgfsetlinewidth{0.000000pt}%
\definecolor{currentstroke}{rgb}{0.000000,0.000000,0.000000}%
\pgfsetstrokecolor{currentstroke}%
\pgfsetstrokeopacity{0.000000}%
\pgfsetdash{}{0pt}%
\pgfpathmoveto{\pgfqpoint{5.800000in}{22.315784in}}%
\pgfpathlineto{\pgfqpoint{5.800000in}{22.315784in}}%
\pgfpathlineto{\pgfqpoint{5.800000in}{22.418621in}}%
\pgfpathlineto{\pgfqpoint{5.800000in}{22.418621in}}%
\pgfpathclose%
\pgfusepath{fill}%
\end{pgfscope}%
\begin{pgfscope}%
\pgfpathrectangle{\pgfqpoint{5.800000in}{0.720000in}}{\pgfqpoint{1.400000in}{4.620000in}}%
\pgfusepath{clip}%
\pgfsetbuttcap%
\pgfsetmiterjoin%
\definecolor{currentfill}{rgb}{0.121569,0.466667,0.705882}%
\pgfsetfillcolor{currentfill}%
\pgfsetlinewidth{0.000000pt}%
\definecolor{currentstroke}{rgb}{0.000000,0.000000,0.000000}%
\pgfsetstrokecolor{currentstroke}%
\pgfsetstrokeopacity{0.000000}%
\pgfsetdash{}{0pt}%
\pgfpathmoveto{\pgfqpoint{5.800000in}{22.418621in}}%
\pgfpathlineto{\pgfqpoint{5.800000in}{22.418621in}}%
\pgfpathlineto{\pgfqpoint{5.800000in}{22.521458in}}%
\pgfpathlineto{\pgfqpoint{5.800000in}{22.521458in}}%
\pgfpathclose%
\pgfusepath{fill}%
\end{pgfscope}%
\begin{pgfscope}%
\pgfpathrectangle{\pgfqpoint{5.800000in}{0.720000in}}{\pgfqpoint{1.400000in}{4.620000in}}%
\pgfusepath{clip}%
\pgfsetbuttcap%
\pgfsetmiterjoin%
\definecolor{currentfill}{rgb}{0.121569,0.466667,0.705882}%
\pgfsetfillcolor{currentfill}%
\pgfsetlinewidth{0.000000pt}%
\definecolor{currentstroke}{rgb}{0.000000,0.000000,0.000000}%
\pgfsetstrokecolor{currentstroke}%
\pgfsetstrokeopacity{0.000000}%
\pgfsetdash{}{0pt}%
\pgfpathmoveto{\pgfqpoint{5.800000in}{22.521458in}}%
\pgfpathlineto{\pgfqpoint{5.800000in}{22.521458in}}%
\pgfpathlineto{\pgfqpoint{5.800000in}{22.624295in}}%
\pgfpathlineto{\pgfqpoint{5.800000in}{22.624295in}}%
\pgfpathclose%
\pgfusepath{fill}%
\end{pgfscope}%
\begin{pgfscope}%
\pgfpathrectangle{\pgfqpoint{5.800000in}{0.720000in}}{\pgfqpoint{1.400000in}{4.620000in}}%
\pgfusepath{clip}%
\pgfsetbuttcap%
\pgfsetmiterjoin%
\definecolor{currentfill}{rgb}{0.121569,0.466667,0.705882}%
\pgfsetfillcolor{currentfill}%
\pgfsetlinewidth{0.000000pt}%
\definecolor{currentstroke}{rgb}{0.000000,0.000000,0.000000}%
\pgfsetstrokecolor{currentstroke}%
\pgfsetstrokeopacity{0.000000}%
\pgfsetdash{}{0pt}%
\pgfpathmoveto{\pgfqpoint{5.800000in}{22.624295in}}%
\pgfpathlineto{\pgfqpoint{5.800000in}{22.624295in}}%
\pgfpathlineto{\pgfqpoint{5.800000in}{22.727133in}}%
\pgfpathlineto{\pgfqpoint{5.800000in}{22.727133in}}%
\pgfpathclose%
\pgfusepath{fill}%
\end{pgfscope}%
\begin{pgfscope}%
\pgfpathrectangle{\pgfqpoint{5.800000in}{0.720000in}}{\pgfqpoint{1.400000in}{4.620000in}}%
\pgfusepath{clip}%
\pgfsetbuttcap%
\pgfsetmiterjoin%
\definecolor{currentfill}{rgb}{0.121569,0.466667,0.705882}%
\pgfsetfillcolor{currentfill}%
\pgfsetlinewidth{0.000000pt}%
\definecolor{currentstroke}{rgb}{0.000000,0.000000,0.000000}%
\pgfsetstrokecolor{currentstroke}%
\pgfsetstrokeopacity{0.000000}%
\pgfsetdash{}{0pt}%
\pgfpathmoveto{\pgfqpoint{5.800000in}{22.727133in}}%
\pgfpathlineto{\pgfqpoint{5.800000in}{22.727133in}}%
\pgfpathlineto{\pgfqpoint{5.800000in}{22.829970in}}%
\pgfpathlineto{\pgfqpoint{5.800000in}{22.829970in}}%
\pgfpathclose%
\pgfusepath{fill}%
\end{pgfscope}%
\begin{pgfscope}%
\pgfpathrectangle{\pgfqpoint{5.800000in}{0.720000in}}{\pgfqpoint{1.400000in}{4.620000in}}%
\pgfusepath{clip}%
\pgfsetbuttcap%
\pgfsetmiterjoin%
\definecolor{currentfill}{rgb}{0.121569,0.466667,0.705882}%
\pgfsetfillcolor{currentfill}%
\pgfsetlinewidth{0.000000pt}%
\definecolor{currentstroke}{rgb}{0.000000,0.000000,0.000000}%
\pgfsetstrokecolor{currentstroke}%
\pgfsetstrokeopacity{0.000000}%
\pgfsetdash{}{0pt}%
\pgfpathmoveto{\pgfqpoint{5.800000in}{22.829970in}}%
\pgfpathlineto{\pgfqpoint{5.800000in}{22.829970in}}%
\pgfpathlineto{\pgfqpoint{5.800000in}{22.932807in}}%
\pgfpathlineto{\pgfqpoint{5.800000in}{22.932807in}}%
\pgfpathclose%
\pgfusepath{fill}%
\end{pgfscope}%
\begin{pgfscope}%
\pgfpathrectangle{\pgfqpoint{5.800000in}{0.720000in}}{\pgfqpoint{1.400000in}{4.620000in}}%
\pgfusepath{clip}%
\pgfsetbuttcap%
\pgfsetmiterjoin%
\definecolor{currentfill}{rgb}{0.121569,0.466667,0.705882}%
\pgfsetfillcolor{currentfill}%
\pgfsetlinewidth{0.000000pt}%
\definecolor{currentstroke}{rgb}{0.000000,0.000000,0.000000}%
\pgfsetstrokecolor{currentstroke}%
\pgfsetstrokeopacity{0.000000}%
\pgfsetdash{}{0pt}%
\pgfpathmoveto{\pgfqpoint{5.800000in}{22.932807in}}%
\pgfpathlineto{\pgfqpoint{5.800000in}{22.932807in}}%
\pgfpathlineto{\pgfqpoint{5.800000in}{23.035644in}}%
\pgfpathlineto{\pgfqpoint{5.800000in}{23.035644in}}%
\pgfpathclose%
\pgfusepath{fill}%
\end{pgfscope}%
\begin{pgfscope}%
\pgfpathrectangle{\pgfqpoint{5.800000in}{0.720000in}}{\pgfqpoint{1.400000in}{4.620000in}}%
\pgfusepath{clip}%
\pgfsetbuttcap%
\pgfsetmiterjoin%
\definecolor{currentfill}{rgb}{0.121569,0.466667,0.705882}%
\pgfsetfillcolor{currentfill}%
\pgfsetlinewidth{0.000000pt}%
\definecolor{currentstroke}{rgb}{0.000000,0.000000,0.000000}%
\pgfsetstrokecolor{currentstroke}%
\pgfsetstrokeopacity{0.000000}%
\pgfsetdash{}{0pt}%
\pgfpathmoveto{\pgfqpoint{5.800000in}{23.035644in}}%
\pgfpathlineto{\pgfqpoint{5.800000in}{23.035644in}}%
\pgfpathlineto{\pgfqpoint{5.800000in}{23.138481in}}%
\pgfpathlineto{\pgfqpoint{5.800000in}{23.138481in}}%
\pgfpathclose%
\pgfusepath{fill}%
\end{pgfscope}%
\begin{pgfscope}%
\pgfpathrectangle{\pgfqpoint{5.800000in}{0.720000in}}{\pgfqpoint{1.400000in}{4.620000in}}%
\pgfusepath{clip}%
\pgfsetbuttcap%
\pgfsetmiterjoin%
\definecolor{currentfill}{rgb}{0.121569,0.466667,0.705882}%
\pgfsetfillcolor{currentfill}%
\pgfsetlinewidth{0.000000pt}%
\definecolor{currentstroke}{rgb}{0.000000,0.000000,0.000000}%
\pgfsetstrokecolor{currentstroke}%
\pgfsetstrokeopacity{0.000000}%
\pgfsetdash{}{0pt}%
\pgfpathmoveto{\pgfqpoint{5.800000in}{23.138481in}}%
\pgfpathlineto{\pgfqpoint{5.800000in}{23.138481in}}%
\pgfpathlineto{\pgfqpoint{5.800000in}{23.241318in}}%
\pgfpathlineto{\pgfqpoint{5.800000in}{23.241318in}}%
\pgfpathclose%
\pgfusepath{fill}%
\end{pgfscope}%
\begin{pgfscope}%
\pgfpathrectangle{\pgfqpoint{5.800000in}{0.720000in}}{\pgfqpoint{1.400000in}{4.620000in}}%
\pgfusepath{clip}%
\pgfsetbuttcap%
\pgfsetmiterjoin%
\definecolor{currentfill}{rgb}{0.121569,0.466667,0.705882}%
\pgfsetfillcolor{currentfill}%
\pgfsetlinewidth{0.000000pt}%
\definecolor{currentstroke}{rgb}{0.000000,0.000000,0.000000}%
\pgfsetstrokecolor{currentstroke}%
\pgfsetstrokeopacity{0.000000}%
\pgfsetdash{}{0pt}%
\pgfpathmoveto{\pgfqpoint{5.800000in}{23.241318in}}%
\pgfpathlineto{\pgfqpoint{5.800000in}{23.241318in}}%
\pgfpathlineto{\pgfqpoint{5.800000in}{23.344155in}}%
\pgfpathlineto{\pgfqpoint{5.800000in}{23.344155in}}%
\pgfpathclose%
\pgfusepath{fill}%
\end{pgfscope}%
\begin{pgfscope}%
\pgfpathrectangle{\pgfqpoint{5.800000in}{0.720000in}}{\pgfqpoint{1.400000in}{4.620000in}}%
\pgfusepath{clip}%
\pgfsetbuttcap%
\pgfsetmiterjoin%
\definecolor{currentfill}{rgb}{0.121569,0.466667,0.705882}%
\pgfsetfillcolor{currentfill}%
\pgfsetlinewidth{0.000000pt}%
\definecolor{currentstroke}{rgb}{0.000000,0.000000,0.000000}%
\pgfsetstrokecolor{currentstroke}%
\pgfsetstrokeopacity{0.000000}%
\pgfsetdash{}{0pt}%
\pgfpathmoveto{\pgfqpoint{5.800000in}{23.344155in}}%
\pgfpathlineto{\pgfqpoint{5.800000in}{23.344155in}}%
\pgfpathlineto{\pgfqpoint{5.800000in}{23.446992in}}%
\pgfpathlineto{\pgfqpoint{5.800000in}{23.446992in}}%
\pgfpathclose%
\pgfusepath{fill}%
\end{pgfscope}%
\begin{pgfscope}%
\pgfpathrectangle{\pgfqpoint{5.800000in}{0.720000in}}{\pgfqpoint{1.400000in}{4.620000in}}%
\pgfusepath{clip}%
\pgfsetbuttcap%
\pgfsetmiterjoin%
\definecolor{currentfill}{rgb}{0.121569,0.466667,0.705882}%
\pgfsetfillcolor{currentfill}%
\pgfsetlinewidth{0.000000pt}%
\definecolor{currentstroke}{rgb}{0.000000,0.000000,0.000000}%
\pgfsetstrokecolor{currentstroke}%
\pgfsetstrokeopacity{0.000000}%
\pgfsetdash{}{0pt}%
\pgfpathmoveto{\pgfqpoint{5.800000in}{23.446992in}}%
\pgfpathlineto{\pgfqpoint{5.800000in}{23.446992in}}%
\pgfpathlineto{\pgfqpoint{5.800000in}{23.549829in}}%
\pgfpathlineto{\pgfqpoint{5.800000in}{23.549829in}}%
\pgfpathclose%
\pgfusepath{fill}%
\end{pgfscope}%
\begin{pgfscope}%
\pgfpathrectangle{\pgfqpoint{5.800000in}{0.720000in}}{\pgfqpoint{1.400000in}{4.620000in}}%
\pgfusepath{clip}%
\pgfsetbuttcap%
\pgfsetmiterjoin%
\definecolor{currentfill}{rgb}{0.121569,0.466667,0.705882}%
\pgfsetfillcolor{currentfill}%
\pgfsetlinewidth{0.000000pt}%
\definecolor{currentstroke}{rgb}{0.000000,0.000000,0.000000}%
\pgfsetstrokecolor{currentstroke}%
\pgfsetstrokeopacity{0.000000}%
\pgfsetdash{}{0pt}%
\pgfpathmoveto{\pgfqpoint{5.800000in}{23.549829in}}%
\pgfpathlineto{\pgfqpoint{5.800000in}{23.549829in}}%
\pgfpathlineto{\pgfqpoint{5.800000in}{23.652666in}}%
\pgfpathlineto{\pgfqpoint{5.800000in}{23.652666in}}%
\pgfpathclose%
\pgfusepath{fill}%
\end{pgfscope}%
\begin{pgfscope}%
\pgfpathrectangle{\pgfqpoint{5.800000in}{0.720000in}}{\pgfqpoint{1.400000in}{4.620000in}}%
\pgfusepath{clip}%
\pgfsetbuttcap%
\pgfsetmiterjoin%
\definecolor{currentfill}{rgb}{0.121569,0.466667,0.705882}%
\pgfsetfillcolor{currentfill}%
\pgfsetlinewidth{0.000000pt}%
\definecolor{currentstroke}{rgb}{0.000000,0.000000,0.000000}%
\pgfsetstrokecolor{currentstroke}%
\pgfsetstrokeopacity{0.000000}%
\pgfsetdash{}{0pt}%
\pgfpathmoveto{\pgfqpoint{5.800000in}{23.652666in}}%
\pgfpathlineto{\pgfqpoint{5.800000in}{23.652666in}}%
\pgfpathlineto{\pgfqpoint{5.800000in}{23.755503in}}%
\pgfpathlineto{\pgfqpoint{5.800000in}{23.755503in}}%
\pgfpathclose%
\pgfusepath{fill}%
\end{pgfscope}%
\begin{pgfscope}%
\pgfpathrectangle{\pgfqpoint{5.800000in}{0.720000in}}{\pgfqpoint{1.400000in}{4.620000in}}%
\pgfusepath{clip}%
\pgfsetbuttcap%
\pgfsetmiterjoin%
\definecolor{currentfill}{rgb}{0.121569,0.466667,0.705882}%
\pgfsetfillcolor{currentfill}%
\pgfsetlinewidth{0.000000pt}%
\definecolor{currentstroke}{rgb}{0.000000,0.000000,0.000000}%
\pgfsetstrokecolor{currentstroke}%
\pgfsetstrokeopacity{0.000000}%
\pgfsetdash{}{0pt}%
\pgfpathmoveto{\pgfqpoint{5.800000in}{23.755503in}}%
\pgfpathlineto{\pgfqpoint{5.800000in}{23.755503in}}%
\pgfpathlineto{\pgfqpoint{5.800000in}{23.858340in}}%
\pgfpathlineto{\pgfqpoint{5.800000in}{23.858340in}}%
\pgfpathclose%
\pgfusepath{fill}%
\end{pgfscope}%
\begin{pgfscope}%
\pgfpathrectangle{\pgfqpoint{5.800000in}{0.720000in}}{\pgfqpoint{1.400000in}{4.620000in}}%
\pgfusepath{clip}%
\pgfsetbuttcap%
\pgfsetmiterjoin%
\definecolor{currentfill}{rgb}{0.121569,0.466667,0.705882}%
\pgfsetfillcolor{currentfill}%
\pgfsetlinewidth{0.000000pt}%
\definecolor{currentstroke}{rgb}{0.000000,0.000000,0.000000}%
\pgfsetstrokecolor{currentstroke}%
\pgfsetstrokeopacity{0.000000}%
\pgfsetdash{}{0pt}%
\pgfpathmoveto{\pgfqpoint{5.800000in}{23.858340in}}%
\pgfpathlineto{\pgfqpoint{5.800000in}{23.858340in}}%
\pgfpathlineto{\pgfqpoint{5.800000in}{23.961177in}}%
\pgfpathlineto{\pgfqpoint{5.800000in}{23.961177in}}%
\pgfpathclose%
\pgfusepath{fill}%
\end{pgfscope}%
\begin{pgfscope}%
\pgfpathrectangle{\pgfqpoint{5.800000in}{0.720000in}}{\pgfqpoint{1.400000in}{4.620000in}}%
\pgfusepath{clip}%
\pgfsetbuttcap%
\pgfsetmiterjoin%
\definecolor{currentfill}{rgb}{0.121569,0.466667,0.705882}%
\pgfsetfillcolor{currentfill}%
\pgfsetlinewidth{0.000000pt}%
\definecolor{currentstroke}{rgb}{0.000000,0.000000,0.000000}%
\pgfsetstrokecolor{currentstroke}%
\pgfsetstrokeopacity{0.000000}%
\pgfsetdash{}{0pt}%
\pgfpathmoveto{\pgfqpoint{5.800000in}{23.961177in}}%
\pgfpathlineto{\pgfqpoint{5.800000in}{23.961177in}}%
\pgfpathlineto{\pgfqpoint{5.800000in}{24.064014in}}%
\pgfpathlineto{\pgfqpoint{5.800000in}{24.064014in}}%
\pgfpathclose%
\pgfusepath{fill}%
\end{pgfscope}%
\begin{pgfscope}%
\pgfpathrectangle{\pgfqpoint{5.800000in}{0.720000in}}{\pgfqpoint{1.400000in}{4.620000in}}%
\pgfusepath{clip}%
\pgfsetbuttcap%
\pgfsetmiterjoin%
\definecolor{currentfill}{rgb}{0.121569,0.466667,0.705882}%
\pgfsetfillcolor{currentfill}%
\pgfsetlinewidth{0.000000pt}%
\definecolor{currentstroke}{rgb}{0.000000,0.000000,0.000000}%
\pgfsetstrokecolor{currentstroke}%
\pgfsetstrokeopacity{0.000000}%
\pgfsetdash{}{0pt}%
\pgfpathmoveto{\pgfqpoint{5.800000in}{24.064014in}}%
\pgfpathlineto{\pgfqpoint{5.800000in}{24.064014in}}%
\pgfpathlineto{\pgfqpoint{5.800000in}{24.166851in}}%
\pgfpathlineto{\pgfqpoint{5.800000in}{24.166851in}}%
\pgfpathclose%
\pgfusepath{fill}%
\end{pgfscope}%
\begin{pgfscope}%
\pgfpathrectangle{\pgfqpoint{5.800000in}{0.720000in}}{\pgfqpoint{1.400000in}{4.620000in}}%
\pgfusepath{clip}%
\pgfsetbuttcap%
\pgfsetmiterjoin%
\definecolor{currentfill}{rgb}{0.121569,0.466667,0.705882}%
\pgfsetfillcolor{currentfill}%
\pgfsetlinewidth{0.000000pt}%
\definecolor{currentstroke}{rgb}{0.000000,0.000000,0.000000}%
\pgfsetstrokecolor{currentstroke}%
\pgfsetstrokeopacity{0.000000}%
\pgfsetdash{}{0pt}%
\pgfpathmoveto{\pgfqpoint{5.800000in}{24.166851in}}%
\pgfpathlineto{\pgfqpoint{5.800000in}{24.166851in}}%
\pgfpathlineto{\pgfqpoint{5.800000in}{24.269689in}}%
\pgfpathlineto{\pgfqpoint{5.800000in}{24.269689in}}%
\pgfpathclose%
\pgfusepath{fill}%
\end{pgfscope}%
\begin{pgfscope}%
\pgfpathrectangle{\pgfqpoint{5.800000in}{0.720000in}}{\pgfqpoint{1.400000in}{4.620000in}}%
\pgfusepath{clip}%
\pgfsetbuttcap%
\pgfsetmiterjoin%
\definecolor{currentfill}{rgb}{0.121569,0.466667,0.705882}%
\pgfsetfillcolor{currentfill}%
\pgfsetlinewidth{0.000000pt}%
\definecolor{currentstroke}{rgb}{0.000000,0.000000,0.000000}%
\pgfsetstrokecolor{currentstroke}%
\pgfsetstrokeopacity{0.000000}%
\pgfsetdash{}{0pt}%
\pgfpathmoveto{\pgfqpoint{5.800000in}{24.269689in}}%
\pgfpathlineto{\pgfqpoint{5.800000in}{24.269689in}}%
\pgfpathlineto{\pgfqpoint{5.800000in}{24.372526in}}%
\pgfpathlineto{\pgfqpoint{5.800000in}{24.372526in}}%
\pgfpathclose%
\pgfusepath{fill}%
\end{pgfscope}%
\begin{pgfscope}%
\pgfpathrectangle{\pgfqpoint{5.800000in}{0.720000in}}{\pgfqpoint{1.400000in}{4.620000in}}%
\pgfusepath{clip}%
\pgfsetbuttcap%
\pgfsetmiterjoin%
\definecolor{currentfill}{rgb}{0.121569,0.466667,0.705882}%
\pgfsetfillcolor{currentfill}%
\pgfsetlinewidth{0.000000pt}%
\definecolor{currentstroke}{rgb}{0.000000,0.000000,0.000000}%
\pgfsetstrokecolor{currentstroke}%
\pgfsetstrokeopacity{0.000000}%
\pgfsetdash{}{0pt}%
\pgfpathmoveto{\pgfqpoint{5.800000in}{24.372526in}}%
\pgfpathlineto{\pgfqpoint{5.800000in}{24.372526in}}%
\pgfpathlineto{\pgfqpoint{5.800000in}{24.475363in}}%
\pgfpathlineto{\pgfqpoint{5.800000in}{24.475363in}}%
\pgfpathclose%
\pgfusepath{fill}%
\end{pgfscope}%
\begin{pgfscope}%
\pgfpathrectangle{\pgfqpoint{5.800000in}{0.720000in}}{\pgfqpoint{1.400000in}{4.620000in}}%
\pgfusepath{clip}%
\pgfsetbuttcap%
\pgfsetmiterjoin%
\definecolor{currentfill}{rgb}{0.121569,0.466667,0.705882}%
\pgfsetfillcolor{currentfill}%
\pgfsetlinewidth{0.000000pt}%
\definecolor{currentstroke}{rgb}{0.000000,0.000000,0.000000}%
\pgfsetstrokecolor{currentstroke}%
\pgfsetstrokeopacity{0.000000}%
\pgfsetdash{}{0pt}%
\pgfpathmoveto{\pgfqpoint{5.800000in}{24.475363in}}%
\pgfpathlineto{\pgfqpoint{5.800000in}{24.475363in}}%
\pgfpathlineto{\pgfqpoint{5.800000in}{24.578200in}}%
\pgfpathlineto{\pgfqpoint{5.800000in}{24.578200in}}%
\pgfpathclose%
\pgfusepath{fill}%
\end{pgfscope}%
\begin{pgfscope}%
\pgfpathrectangle{\pgfqpoint{5.800000in}{0.720000in}}{\pgfqpoint{1.400000in}{4.620000in}}%
\pgfusepath{clip}%
\pgfsetbuttcap%
\pgfsetmiterjoin%
\definecolor{currentfill}{rgb}{0.121569,0.466667,0.705882}%
\pgfsetfillcolor{currentfill}%
\pgfsetlinewidth{0.000000pt}%
\definecolor{currentstroke}{rgb}{0.000000,0.000000,0.000000}%
\pgfsetstrokecolor{currentstroke}%
\pgfsetstrokeopacity{0.000000}%
\pgfsetdash{}{0pt}%
\pgfpathmoveto{\pgfqpoint{5.800000in}{24.578200in}}%
\pgfpathlineto{\pgfqpoint{5.800000in}{24.578200in}}%
\pgfpathlineto{\pgfqpoint{5.800000in}{24.681037in}}%
\pgfpathlineto{\pgfqpoint{5.800000in}{24.681037in}}%
\pgfpathclose%
\pgfusepath{fill}%
\end{pgfscope}%
\begin{pgfscope}%
\pgfpathrectangle{\pgfqpoint{5.800000in}{0.720000in}}{\pgfqpoint{1.400000in}{4.620000in}}%
\pgfusepath{clip}%
\pgfsetbuttcap%
\pgfsetmiterjoin%
\definecolor{currentfill}{rgb}{0.121569,0.466667,0.705882}%
\pgfsetfillcolor{currentfill}%
\pgfsetlinewidth{0.000000pt}%
\definecolor{currentstroke}{rgb}{0.000000,0.000000,0.000000}%
\pgfsetstrokecolor{currentstroke}%
\pgfsetstrokeopacity{0.000000}%
\pgfsetdash{}{0pt}%
\pgfpathmoveto{\pgfqpoint{5.800000in}{24.681037in}}%
\pgfpathlineto{\pgfqpoint{5.800000in}{24.681037in}}%
\pgfpathlineto{\pgfqpoint{5.800000in}{24.783874in}}%
\pgfpathlineto{\pgfqpoint{5.800000in}{24.783874in}}%
\pgfpathclose%
\pgfusepath{fill}%
\end{pgfscope}%
\begin{pgfscope}%
\pgfpathrectangle{\pgfqpoint{5.800000in}{0.720000in}}{\pgfqpoint{1.400000in}{4.620000in}}%
\pgfusepath{clip}%
\pgfsetbuttcap%
\pgfsetmiterjoin%
\definecolor{currentfill}{rgb}{0.121569,0.466667,0.705882}%
\pgfsetfillcolor{currentfill}%
\pgfsetlinewidth{0.000000pt}%
\definecolor{currentstroke}{rgb}{0.000000,0.000000,0.000000}%
\pgfsetstrokecolor{currentstroke}%
\pgfsetstrokeopacity{0.000000}%
\pgfsetdash{}{0pt}%
\pgfpathmoveto{\pgfqpoint{5.800000in}{24.783874in}}%
\pgfpathlineto{\pgfqpoint{5.800000in}{24.783874in}}%
\pgfpathlineto{\pgfqpoint{5.800000in}{24.886711in}}%
\pgfpathlineto{\pgfqpoint{5.800000in}{24.886711in}}%
\pgfpathclose%
\pgfusepath{fill}%
\end{pgfscope}%
\begin{pgfscope}%
\pgfpathrectangle{\pgfqpoint{5.800000in}{0.720000in}}{\pgfqpoint{1.400000in}{4.620000in}}%
\pgfusepath{clip}%
\pgfsetbuttcap%
\pgfsetmiterjoin%
\definecolor{currentfill}{rgb}{0.121569,0.466667,0.705882}%
\pgfsetfillcolor{currentfill}%
\pgfsetlinewidth{0.000000pt}%
\definecolor{currentstroke}{rgb}{0.000000,0.000000,0.000000}%
\pgfsetstrokecolor{currentstroke}%
\pgfsetstrokeopacity{0.000000}%
\pgfsetdash{}{0pt}%
\pgfpathmoveto{\pgfqpoint{5.800000in}{24.886711in}}%
\pgfpathlineto{\pgfqpoint{5.800000in}{24.886711in}}%
\pgfpathlineto{\pgfqpoint{5.800000in}{24.989548in}}%
\pgfpathlineto{\pgfqpoint{5.800000in}{24.989548in}}%
\pgfpathclose%
\pgfusepath{fill}%
\end{pgfscope}%
\begin{pgfscope}%
\pgfpathrectangle{\pgfqpoint{5.800000in}{0.720000in}}{\pgfqpoint{1.400000in}{4.620000in}}%
\pgfusepath{clip}%
\pgfsetbuttcap%
\pgfsetmiterjoin%
\definecolor{currentfill}{rgb}{0.121569,0.466667,0.705882}%
\pgfsetfillcolor{currentfill}%
\pgfsetlinewidth{0.000000pt}%
\definecolor{currentstroke}{rgb}{0.000000,0.000000,0.000000}%
\pgfsetstrokecolor{currentstroke}%
\pgfsetstrokeopacity{0.000000}%
\pgfsetdash{}{0pt}%
\pgfpathmoveto{\pgfqpoint{5.800000in}{24.989548in}}%
\pgfpathlineto{\pgfqpoint{5.800000in}{24.989548in}}%
\pgfpathlineto{\pgfqpoint{5.800000in}{25.092385in}}%
\pgfpathlineto{\pgfqpoint{5.800000in}{25.092385in}}%
\pgfpathclose%
\pgfusepath{fill}%
\end{pgfscope}%
\begin{pgfscope}%
\pgfpathrectangle{\pgfqpoint{5.800000in}{0.720000in}}{\pgfqpoint{1.400000in}{4.620000in}}%
\pgfusepath{clip}%
\pgfsetbuttcap%
\pgfsetmiterjoin%
\definecolor{currentfill}{rgb}{0.121569,0.466667,0.705882}%
\pgfsetfillcolor{currentfill}%
\pgfsetlinewidth{0.000000pt}%
\definecolor{currentstroke}{rgb}{0.000000,0.000000,0.000000}%
\pgfsetstrokecolor{currentstroke}%
\pgfsetstrokeopacity{0.000000}%
\pgfsetdash{}{0pt}%
\pgfpathmoveto{\pgfqpoint{5.800000in}{25.092385in}}%
\pgfpathlineto{\pgfqpoint{5.800000in}{25.092385in}}%
\pgfpathlineto{\pgfqpoint{5.800000in}{25.195222in}}%
\pgfpathlineto{\pgfqpoint{5.800000in}{25.195222in}}%
\pgfpathclose%
\pgfusepath{fill}%
\end{pgfscope}%
\begin{pgfscope}%
\pgfpathrectangle{\pgfqpoint{5.800000in}{0.720000in}}{\pgfqpoint{1.400000in}{4.620000in}}%
\pgfusepath{clip}%
\pgfsetbuttcap%
\pgfsetmiterjoin%
\definecolor{currentfill}{rgb}{0.121569,0.466667,0.705882}%
\pgfsetfillcolor{currentfill}%
\pgfsetlinewidth{0.000000pt}%
\definecolor{currentstroke}{rgb}{0.000000,0.000000,0.000000}%
\pgfsetstrokecolor{currentstroke}%
\pgfsetstrokeopacity{0.000000}%
\pgfsetdash{}{0pt}%
\pgfpathmoveto{\pgfqpoint{5.800000in}{25.195222in}}%
\pgfpathlineto{\pgfqpoint{5.800000in}{25.195222in}}%
\pgfpathlineto{\pgfqpoint{5.800000in}{25.298059in}}%
\pgfpathlineto{\pgfqpoint{5.800000in}{25.298059in}}%
\pgfpathclose%
\pgfusepath{fill}%
\end{pgfscope}%
\begin{pgfscope}%
\pgfpathrectangle{\pgfqpoint{5.800000in}{0.720000in}}{\pgfqpoint{1.400000in}{4.620000in}}%
\pgfusepath{clip}%
\pgfsetbuttcap%
\pgfsetmiterjoin%
\definecolor{currentfill}{rgb}{0.121569,0.466667,0.705882}%
\pgfsetfillcolor{currentfill}%
\pgfsetlinewidth{0.000000pt}%
\definecolor{currentstroke}{rgb}{0.000000,0.000000,0.000000}%
\pgfsetstrokecolor{currentstroke}%
\pgfsetstrokeopacity{0.000000}%
\pgfsetdash{}{0pt}%
\pgfpathmoveto{\pgfqpoint{5.800000in}{25.298059in}}%
\pgfpathlineto{\pgfqpoint{5.800000in}{25.298059in}}%
\pgfpathlineto{\pgfqpoint{5.800000in}{25.400896in}}%
\pgfpathlineto{\pgfqpoint{5.800000in}{25.400896in}}%
\pgfpathclose%
\pgfusepath{fill}%
\end{pgfscope}%
\begin{pgfscope}%
\pgfpathrectangle{\pgfqpoint{5.800000in}{0.720000in}}{\pgfqpoint{1.400000in}{4.620000in}}%
\pgfusepath{clip}%
\pgfsetbuttcap%
\pgfsetmiterjoin%
\definecolor{currentfill}{rgb}{0.121569,0.466667,0.705882}%
\pgfsetfillcolor{currentfill}%
\pgfsetlinewidth{0.000000pt}%
\definecolor{currentstroke}{rgb}{0.000000,0.000000,0.000000}%
\pgfsetstrokecolor{currentstroke}%
\pgfsetstrokeopacity{0.000000}%
\pgfsetdash{}{0pt}%
\pgfpathmoveto{\pgfqpoint{5.800000in}{25.400896in}}%
\pgfpathlineto{\pgfqpoint{5.800000in}{25.400896in}}%
\pgfpathlineto{\pgfqpoint{5.800000in}{25.503733in}}%
\pgfpathlineto{\pgfqpoint{5.800000in}{25.503733in}}%
\pgfpathclose%
\pgfusepath{fill}%
\end{pgfscope}%
\begin{pgfscope}%
\pgfpathrectangle{\pgfqpoint{5.800000in}{0.720000in}}{\pgfqpoint{1.400000in}{4.620000in}}%
\pgfusepath{clip}%
\pgfsetbuttcap%
\pgfsetmiterjoin%
\definecolor{currentfill}{rgb}{0.121569,0.466667,0.705882}%
\pgfsetfillcolor{currentfill}%
\pgfsetlinewidth{0.000000pt}%
\definecolor{currentstroke}{rgb}{0.000000,0.000000,0.000000}%
\pgfsetstrokecolor{currentstroke}%
\pgfsetstrokeopacity{0.000000}%
\pgfsetdash{}{0pt}%
\pgfpathmoveto{\pgfqpoint{5.800000in}{25.503733in}}%
\pgfpathlineto{\pgfqpoint{5.800000in}{25.503733in}}%
\pgfpathlineto{\pgfqpoint{5.800000in}{25.606570in}}%
\pgfpathlineto{\pgfqpoint{5.800000in}{25.606570in}}%
\pgfpathclose%
\pgfusepath{fill}%
\end{pgfscope}%
\begin{pgfscope}%
\pgfpathrectangle{\pgfqpoint{5.800000in}{0.720000in}}{\pgfqpoint{1.400000in}{4.620000in}}%
\pgfusepath{clip}%
\pgfsetbuttcap%
\pgfsetmiterjoin%
\definecolor{currentfill}{rgb}{0.121569,0.466667,0.705882}%
\pgfsetfillcolor{currentfill}%
\pgfsetlinewidth{0.000000pt}%
\definecolor{currentstroke}{rgb}{0.000000,0.000000,0.000000}%
\pgfsetstrokecolor{currentstroke}%
\pgfsetstrokeopacity{0.000000}%
\pgfsetdash{}{0pt}%
\pgfpathmoveto{\pgfqpoint{5.800000in}{25.606570in}}%
\pgfpathlineto{\pgfqpoint{5.800000in}{25.606570in}}%
\pgfpathlineto{\pgfqpoint{5.800000in}{25.709407in}}%
\pgfpathlineto{\pgfqpoint{5.800000in}{25.709407in}}%
\pgfpathclose%
\pgfusepath{fill}%
\end{pgfscope}%
\begin{pgfscope}%
\pgfpathrectangle{\pgfqpoint{5.800000in}{0.720000in}}{\pgfqpoint{1.400000in}{4.620000in}}%
\pgfusepath{clip}%
\pgfsetbuttcap%
\pgfsetmiterjoin%
\definecolor{currentfill}{rgb}{0.121569,0.466667,0.705882}%
\pgfsetfillcolor{currentfill}%
\pgfsetlinewidth{0.000000pt}%
\definecolor{currentstroke}{rgb}{0.000000,0.000000,0.000000}%
\pgfsetstrokecolor{currentstroke}%
\pgfsetstrokeopacity{0.000000}%
\pgfsetdash{}{0pt}%
\pgfpathmoveto{\pgfqpoint{5.800000in}{25.709407in}}%
\pgfpathlineto{\pgfqpoint{5.800000in}{25.709407in}}%
\pgfpathlineto{\pgfqpoint{5.800000in}{25.812245in}}%
\pgfpathlineto{\pgfqpoint{5.800000in}{25.812245in}}%
\pgfpathclose%
\pgfusepath{fill}%
\end{pgfscope}%
\begin{pgfscope}%
\pgfpathrectangle{\pgfqpoint{5.800000in}{0.720000in}}{\pgfqpoint{1.400000in}{4.620000in}}%
\pgfusepath{clip}%
\pgfsetbuttcap%
\pgfsetmiterjoin%
\definecolor{currentfill}{rgb}{0.121569,0.466667,0.705882}%
\pgfsetfillcolor{currentfill}%
\pgfsetlinewidth{0.000000pt}%
\definecolor{currentstroke}{rgb}{0.000000,0.000000,0.000000}%
\pgfsetstrokecolor{currentstroke}%
\pgfsetstrokeopacity{0.000000}%
\pgfsetdash{}{0pt}%
\pgfpathmoveto{\pgfqpoint{5.800000in}{25.812245in}}%
\pgfpathlineto{\pgfqpoint{5.800000in}{25.812245in}}%
\pgfpathlineto{\pgfqpoint{5.800000in}{25.915082in}}%
\pgfpathlineto{\pgfqpoint{5.800000in}{25.915082in}}%
\pgfpathclose%
\pgfusepath{fill}%
\end{pgfscope}%
\begin{pgfscope}%
\pgfpathrectangle{\pgfqpoint{5.800000in}{0.720000in}}{\pgfqpoint{1.400000in}{4.620000in}}%
\pgfusepath{clip}%
\pgfsetbuttcap%
\pgfsetmiterjoin%
\definecolor{currentfill}{rgb}{0.121569,0.466667,0.705882}%
\pgfsetfillcolor{currentfill}%
\pgfsetlinewidth{0.000000pt}%
\definecolor{currentstroke}{rgb}{0.000000,0.000000,0.000000}%
\pgfsetstrokecolor{currentstroke}%
\pgfsetstrokeopacity{0.000000}%
\pgfsetdash{}{0pt}%
\pgfpathmoveto{\pgfqpoint{5.800000in}{25.915082in}}%
\pgfpathlineto{\pgfqpoint{5.800000in}{25.915082in}}%
\pgfpathlineto{\pgfqpoint{5.800000in}{26.017919in}}%
\pgfpathlineto{\pgfqpoint{5.800000in}{26.017919in}}%
\pgfpathclose%
\pgfusepath{fill}%
\end{pgfscope}%
\begin{pgfscope}%
\pgfpathrectangle{\pgfqpoint{5.800000in}{0.720000in}}{\pgfqpoint{1.400000in}{4.620000in}}%
\pgfusepath{clip}%
\pgfsetbuttcap%
\pgfsetmiterjoin%
\definecolor{currentfill}{rgb}{0.121569,0.466667,0.705882}%
\pgfsetfillcolor{currentfill}%
\pgfsetlinewidth{0.000000pt}%
\definecolor{currentstroke}{rgb}{0.000000,0.000000,0.000000}%
\pgfsetstrokecolor{currentstroke}%
\pgfsetstrokeopacity{0.000000}%
\pgfsetdash{}{0pt}%
\pgfpathmoveto{\pgfqpoint{5.800000in}{26.017919in}}%
\pgfpathlineto{\pgfqpoint{5.800000in}{26.017919in}}%
\pgfpathlineto{\pgfqpoint{5.800000in}{26.120756in}}%
\pgfpathlineto{\pgfqpoint{5.800000in}{26.120756in}}%
\pgfpathclose%
\pgfusepath{fill}%
\end{pgfscope}%
\begin{pgfscope}%
\pgfpathrectangle{\pgfqpoint{5.800000in}{0.720000in}}{\pgfqpoint{1.400000in}{4.620000in}}%
\pgfusepath{clip}%
\pgfsetbuttcap%
\pgfsetmiterjoin%
\definecolor{currentfill}{rgb}{0.121569,0.466667,0.705882}%
\pgfsetfillcolor{currentfill}%
\pgfsetlinewidth{0.000000pt}%
\definecolor{currentstroke}{rgb}{0.000000,0.000000,0.000000}%
\pgfsetstrokecolor{currentstroke}%
\pgfsetstrokeopacity{0.000000}%
\pgfsetdash{}{0pt}%
\pgfpathmoveto{\pgfqpoint{5.800000in}{26.120756in}}%
\pgfpathlineto{\pgfqpoint{5.802038in}{26.120756in}}%
\pgfpathlineto{\pgfqpoint{5.802038in}{26.223593in}}%
\pgfpathlineto{\pgfqpoint{5.800000in}{26.223593in}}%
\pgfpathclose%
\pgfusepath{fill}%
\end{pgfscope}%
\begin{pgfscope}%
\pgfpathrectangle{\pgfqpoint{5.800000in}{0.720000in}}{\pgfqpoint{1.400000in}{4.620000in}}%
\pgfusepath{clip}%
\pgfsetbuttcap%
\pgfsetmiterjoin%
\definecolor{currentfill}{rgb}{0.121569,0.466667,0.705882}%
\pgfsetfillcolor{currentfill}%
\pgfsetlinewidth{0.000000pt}%
\definecolor{currentstroke}{rgb}{0.000000,0.000000,0.000000}%
\pgfsetstrokecolor{currentstroke}%
\pgfsetstrokeopacity{0.000000}%
\pgfsetdash{}{0pt}%
\pgfpathmoveto{\pgfqpoint{5.800000in}{26.223593in}}%
\pgfpathlineto{\pgfqpoint{5.800000in}{26.223593in}}%
\pgfpathlineto{\pgfqpoint{5.800000in}{26.326430in}}%
\pgfpathlineto{\pgfqpoint{5.800000in}{26.326430in}}%
\pgfpathclose%
\pgfusepath{fill}%
\end{pgfscope}%
\begin{pgfscope}%
\pgfpathrectangle{\pgfqpoint{5.800000in}{0.720000in}}{\pgfqpoint{1.400000in}{4.620000in}}%
\pgfusepath{clip}%
\pgfsetbuttcap%
\pgfsetmiterjoin%
\definecolor{currentfill}{rgb}{0.121569,0.466667,0.705882}%
\pgfsetfillcolor{currentfill}%
\pgfsetlinewidth{0.000000pt}%
\definecolor{currentstroke}{rgb}{0.000000,0.000000,0.000000}%
\pgfsetstrokecolor{currentstroke}%
\pgfsetstrokeopacity{0.000000}%
\pgfsetdash{}{0pt}%
\pgfpathmoveto{\pgfqpoint{5.800000in}{26.326430in}}%
\pgfpathlineto{\pgfqpoint{5.800000in}{26.326430in}}%
\pgfpathlineto{\pgfqpoint{5.800000in}{26.429267in}}%
\pgfpathlineto{\pgfqpoint{5.800000in}{26.429267in}}%
\pgfpathclose%
\pgfusepath{fill}%
\end{pgfscope}%
\begin{pgfscope}%
\pgfpathrectangle{\pgfqpoint{5.800000in}{0.720000in}}{\pgfqpoint{1.400000in}{4.620000in}}%
\pgfusepath{clip}%
\pgfsetbuttcap%
\pgfsetmiterjoin%
\definecolor{currentfill}{rgb}{0.121569,0.466667,0.705882}%
\pgfsetfillcolor{currentfill}%
\pgfsetlinewidth{0.000000pt}%
\definecolor{currentstroke}{rgb}{0.000000,0.000000,0.000000}%
\pgfsetstrokecolor{currentstroke}%
\pgfsetstrokeopacity{0.000000}%
\pgfsetdash{}{0pt}%
\pgfpathmoveto{\pgfqpoint{5.800000in}{26.429267in}}%
\pgfpathlineto{\pgfqpoint{5.800000in}{26.429267in}}%
\pgfpathlineto{\pgfqpoint{5.800000in}{26.532104in}}%
\pgfpathlineto{\pgfqpoint{5.800000in}{26.532104in}}%
\pgfpathclose%
\pgfusepath{fill}%
\end{pgfscope}%
\begin{pgfscope}%
\pgfpathrectangle{\pgfqpoint{5.800000in}{0.720000in}}{\pgfqpoint{1.400000in}{4.620000in}}%
\pgfusepath{clip}%
\pgfsetbuttcap%
\pgfsetmiterjoin%
\definecolor{currentfill}{rgb}{0.121569,0.466667,0.705882}%
\pgfsetfillcolor{currentfill}%
\pgfsetlinewidth{0.000000pt}%
\definecolor{currentstroke}{rgb}{0.000000,0.000000,0.000000}%
\pgfsetstrokecolor{currentstroke}%
\pgfsetstrokeopacity{0.000000}%
\pgfsetdash{}{0pt}%
\pgfpathmoveto{\pgfqpoint{5.800000in}{26.532104in}}%
\pgfpathlineto{\pgfqpoint{5.800000in}{26.532104in}}%
\pgfpathlineto{\pgfqpoint{5.800000in}{26.634941in}}%
\pgfpathlineto{\pgfqpoint{5.800000in}{26.634941in}}%
\pgfpathclose%
\pgfusepath{fill}%
\end{pgfscope}%
\begin{pgfscope}%
\pgfpathrectangle{\pgfqpoint{5.800000in}{0.720000in}}{\pgfqpoint{1.400000in}{4.620000in}}%
\pgfusepath{clip}%
\pgfsetbuttcap%
\pgfsetmiterjoin%
\definecolor{currentfill}{rgb}{0.121569,0.466667,0.705882}%
\pgfsetfillcolor{currentfill}%
\pgfsetlinewidth{0.000000pt}%
\definecolor{currentstroke}{rgb}{0.000000,0.000000,0.000000}%
\pgfsetstrokecolor{currentstroke}%
\pgfsetstrokeopacity{0.000000}%
\pgfsetdash{}{0pt}%
\pgfpathmoveto{\pgfqpoint{5.800000in}{26.634941in}}%
\pgfpathlineto{\pgfqpoint{5.800000in}{26.634941in}}%
\pgfpathlineto{\pgfqpoint{5.800000in}{26.737778in}}%
\pgfpathlineto{\pgfqpoint{5.800000in}{26.737778in}}%
\pgfpathclose%
\pgfusepath{fill}%
\end{pgfscope}%
\begin{pgfscope}%
\pgfpathrectangle{\pgfqpoint{5.800000in}{0.720000in}}{\pgfqpoint{1.400000in}{4.620000in}}%
\pgfusepath{clip}%
\pgfsetbuttcap%
\pgfsetmiterjoin%
\definecolor{currentfill}{rgb}{0.121569,0.466667,0.705882}%
\pgfsetfillcolor{currentfill}%
\pgfsetlinewidth{0.000000pt}%
\definecolor{currentstroke}{rgb}{0.000000,0.000000,0.000000}%
\pgfsetstrokecolor{currentstroke}%
\pgfsetstrokeopacity{0.000000}%
\pgfsetdash{}{0pt}%
\pgfpathmoveto{\pgfqpoint{5.800000in}{26.737778in}}%
\pgfpathlineto{\pgfqpoint{5.800000in}{26.737778in}}%
\pgfpathlineto{\pgfqpoint{5.800000in}{26.840615in}}%
\pgfpathlineto{\pgfqpoint{5.800000in}{26.840615in}}%
\pgfpathclose%
\pgfusepath{fill}%
\end{pgfscope}%
\begin{pgfscope}%
\pgfpathrectangle{\pgfqpoint{5.800000in}{0.720000in}}{\pgfqpoint{1.400000in}{4.620000in}}%
\pgfusepath{clip}%
\pgfsetbuttcap%
\pgfsetmiterjoin%
\definecolor{currentfill}{rgb}{0.121569,0.466667,0.705882}%
\pgfsetfillcolor{currentfill}%
\pgfsetlinewidth{0.000000pt}%
\definecolor{currentstroke}{rgb}{0.000000,0.000000,0.000000}%
\pgfsetstrokecolor{currentstroke}%
\pgfsetstrokeopacity{0.000000}%
\pgfsetdash{}{0pt}%
\pgfpathmoveto{\pgfqpoint{5.800000in}{26.840615in}}%
\pgfpathlineto{\pgfqpoint{5.800000in}{26.840615in}}%
\pgfpathlineto{\pgfqpoint{5.800000in}{26.943452in}}%
\pgfpathlineto{\pgfqpoint{5.800000in}{26.943452in}}%
\pgfpathclose%
\pgfusepath{fill}%
\end{pgfscope}%
\begin{pgfscope}%
\pgfpathrectangle{\pgfqpoint{5.800000in}{0.720000in}}{\pgfqpoint{1.400000in}{4.620000in}}%
\pgfusepath{clip}%
\pgfsetbuttcap%
\pgfsetmiterjoin%
\definecolor{currentfill}{rgb}{0.121569,0.466667,0.705882}%
\pgfsetfillcolor{currentfill}%
\pgfsetlinewidth{0.000000pt}%
\definecolor{currentstroke}{rgb}{0.000000,0.000000,0.000000}%
\pgfsetstrokecolor{currentstroke}%
\pgfsetstrokeopacity{0.000000}%
\pgfsetdash{}{0pt}%
\pgfpathmoveto{\pgfqpoint{5.800000in}{26.943452in}}%
\pgfpathlineto{\pgfqpoint{5.800000in}{26.943452in}}%
\pgfpathlineto{\pgfqpoint{5.800000in}{27.046289in}}%
\pgfpathlineto{\pgfqpoint{5.800000in}{27.046289in}}%
\pgfpathclose%
\pgfusepath{fill}%
\end{pgfscope}%
\begin{pgfscope}%
\pgfpathrectangle{\pgfqpoint{5.800000in}{0.720000in}}{\pgfqpoint{1.400000in}{4.620000in}}%
\pgfusepath{clip}%
\pgfsetbuttcap%
\pgfsetmiterjoin%
\definecolor{currentfill}{rgb}{0.121569,0.466667,0.705882}%
\pgfsetfillcolor{currentfill}%
\pgfsetlinewidth{0.000000pt}%
\definecolor{currentstroke}{rgb}{0.000000,0.000000,0.000000}%
\pgfsetstrokecolor{currentstroke}%
\pgfsetstrokeopacity{0.000000}%
\pgfsetdash{}{0pt}%
\pgfpathmoveto{\pgfqpoint{5.800000in}{27.046289in}}%
\pgfpathlineto{\pgfqpoint{5.800000in}{27.046289in}}%
\pgfpathlineto{\pgfqpoint{5.800000in}{27.149126in}}%
\pgfpathlineto{\pgfqpoint{5.800000in}{27.149126in}}%
\pgfpathclose%
\pgfusepath{fill}%
\end{pgfscope}%
\begin{pgfscope}%
\pgfpathrectangle{\pgfqpoint{5.800000in}{0.720000in}}{\pgfqpoint{1.400000in}{4.620000in}}%
\pgfusepath{clip}%
\pgfsetbuttcap%
\pgfsetmiterjoin%
\definecolor{currentfill}{rgb}{0.121569,0.466667,0.705882}%
\pgfsetfillcolor{currentfill}%
\pgfsetlinewidth{0.000000pt}%
\definecolor{currentstroke}{rgb}{0.000000,0.000000,0.000000}%
\pgfsetstrokecolor{currentstroke}%
\pgfsetstrokeopacity{0.000000}%
\pgfsetdash{}{0pt}%
\pgfpathmoveto{\pgfqpoint{5.800000in}{27.149126in}}%
\pgfpathlineto{\pgfqpoint{5.800000in}{27.149126in}}%
\pgfpathlineto{\pgfqpoint{5.800000in}{27.251963in}}%
\pgfpathlineto{\pgfqpoint{5.800000in}{27.251963in}}%
\pgfpathclose%
\pgfusepath{fill}%
\end{pgfscope}%
\begin{pgfscope}%
\pgfpathrectangle{\pgfqpoint{5.800000in}{0.720000in}}{\pgfqpoint{1.400000in}{4.620000in}}%
\pgfusepath{clip}%
\pgfsetbuttcap%
\pgfsetmiterjoin%
\definecolor{currentfill}{rgb}{0.121569,0.466667,0.705882}%
\pgfsetfillcolor{currentfill}%
\pgfsetlinewidth{0.000000pt}%
\definecolor{currentstroke}{rgb}{0.000000,0.000000,0.000000}%
\pgfsetstrokecolor{currentstroke}%
\pgfsetstrokeopacity{0.000000}%
\pgfsetdash{}{0pt}%
\pgfpathmoveto{\pgfqpoint{5.800000in}{27.251963in}}%
\pgfpathlineto{\pgfqpoint{5.802038in}{27.251963in}}%
\pgfpathlineto{\pgfqpoint{5.802038in}{27.354801in}}%
\pgfpathlineto{\pgfqpoint{5.800000in}{27.354801in}}%
\pgfpathclose%
\pgfusepath{fill}%
\end{pgfscope}%
\begin{pgfscope}%
\pgfpathrectangle{\pgfqpoint{5.800000in}{0.720000in}}{\pgfqpoint{1.400000in}{4.620000in}}%
\pgfusepath{clip}%
\pgfsetbuttcap%
\pgfsetmiterjoin%
\definecolor{currentfill}{rgb}{0.121569,0.466667,0.705882}%
\pgfsetfillcolor{currentfill}%
\pgfsetlinewidth{0.000000pt}%
\definecolor{currentstroke}{rgb}{0.000000,0.000000,0.000000}%
\pgfsetstrokecolor{currentstroke}%
\pgfsetstrokeopacity{0.000000}%
\pgfsetdash{}{0pt}%
\pgfpathmoveto{\pgfqpoint{5.800000in}{27.354801in}}%
\pgfpathlineto{\pgfqpoint{5.800000in}{27.354801in}}%
\pgfpathlineto{\pgfqpoint{5.800000in}{27.457638in}}%
\pgfpathlineto{\pgfqpoint{5.800000in}{27.457638in}}%
\pgfpathclose%
\pgfusepath{fill}%
\end{pgfscope}%
\begin{pgfscope}%
\pgfpathrectangle{\pgfqpoint{5.800000in}{0.720000in}}{\pgfqpoint{1.400000in}{4.620000in}}%
\pgfusepath{clip}%
\pgfsetbuttcap%
\pgfsetmiterjoin%
\definecolor{currentfill}{rgb}{0.121569,0.466667,0.705882}%
\pgfsetfillcolor{currentfill}%
\pgfsetlinewidth{0.000000pt}%
\definecolor{currentstroke}{rgb}{0.000000,0.000000,0.000000}%
\pgfsetstrokecolor{currentstroke}%
\pgfsetstrokeopacity{0.000000}%
\pgfsetdash{}{0pt}%
\pgfpathmoveto{\pgfqpoint{5.800000in}{27.457638in}}%
\pgfpathlineto{\pgfqpoint{5.800000in}{27.457638in}}%
\pgfpathlineto{\pgfqpoint{5.800000in}{27.560475in}}%
\pgfpathlineto{\pgfqpoint{5.800000in}{27.560475in}}%
\pgfpathclose%
\pgfusepath{fill}%
\end{pgfscope}%
\begin{pgfscope}%
\pgfpathrectangle{\pgfqpoint{5.800000in}{0.720000in}}{\pgfqpoint{1.400000in}{4.620000in}}%
\pgfusepath{clip}%
\pgfsetbuttcap%
\pgfsetmiterjoin%
\definecolor{currentfill}{rgb}{0.121569,0.466667,0.705882}%
\pgfsetfillcolor{currentfill}%
\pgfsetlinewidth{0.000000pt}%
\definecolor{currentstroke}{rgb}{0.000000,0.000000,0.000000}%
\pgfsetstrokecolor{currentstroke}%
\pgfsetstrokeopacity{0.000000}%
\pgfsetdash{}{0pt}%
\pgfpathmoveto{\pgfqpoint{5.800000in}{27.560475in}}%
\pgfpathlineto{\pgfqpoint{5.800000in}{27.560475in}}%
\pgfpathlineto{\pgfqpoint{5.800000in}{27.663312in}}%
\pgfpathlineto{\pgfqpoint{5.800000in}{27.663312in}}%
\pgfpathclose%
\pgfusepath{fill}%
\end{pgfscope}%
\begin{pgfscope}%
\pgfpathrectangle{\pgfqpoint{5.800000in}{0.720000in}}{\pgfqpoint{1.400000in}{4.620000in}}%
\pgfusepath{clip}%
\pgfsetbuttcap%
\pgfsetmiterjoin%
\definecolor{currentfill}{rgb}{0.121569,0.466667,0.705882}%
\pgfsetfillcolor{currentfill}%
\pgfsetlinewidth{0.000000pt}%
\definecolor{currentstroke}{rgb}{0.000000,0.000000,0.000000}%
\pgfsetstrokecolor{currentstroke}%
\pgfsetstrokeopacity{0.000000}%
\pgfsetdash{}{0pt}%
\pgfpathmoveto{\pgfqpoint{5.800000in}{27.663312in}}%
\pgfpathlineto{\pgfqpoint{5.800000in}{27.663312in}}%
\pgfpathlineto{\pgfqpoint{5.800000in}{27.766149in}}%
\pgfpathlineto{\pgfqpoint{5.800000in}{27.766149in}}%
\pgfpathclose%
\pgfusepath{fill}%
\end{pgfscope}%
\begin{pgfscope}%
\pgfpathrectangle{\pgfqpoint{5.800000in}{0.720000in}}{\pgfqpoint{1.400000in}{4.620000in}}%
\pgfusepath{clip}%
\pgfsetbuttcap%
\pgfsetmiterjoin%
\definecolor{currentfill}{rgb}{0.121569,0.466667,0.705882}%
\pgfsetfillcolor{currentfill}%
\pgfsetlinewidth{0.000000pt}%
\definecolor{currentstroke}{rgb}{0.000000,0.000000,0.000000}%
\pgfsetstrokecolor{currentstroke}%
\pgfsetstrokeopacity{0.000000}%
\pgfsetdash{}{0pt}%
\pgfpathmoveto{\pgfqpoint{5.800000in}{27.766149in}}%
\pgfpathlineto{\pgfqpoint{5.800000in}{27.766149in}}%
\pgfpathlineto{\pgfqpoint{5.800000in}{27.868986in}}%
\pgfpathlineto{\pgfqpoint{5.800000in}{27.868986in}}%
\pgfpathclose%
\pgfusepath{fill}%
\end{pgfscope}%
\begin{pgfscope}%
\pgfpathrectangle{\pgfqpoint{5.800000in}{0.720000in}}{\pgfqpoint{1.400000in}{4.620000in}}%
\pgfusepath{clip}%
\pgfsetbuttcap%
\pgfsetmiterjoin%
\definecolor{currentfill}{rgb}{0.121569,0.466667,0.705882}%
\pgfsetfillcolor{currentfill}%
\pgfsetlinewidth{0.000000pt}%
\definecolor{currentstroke}{rgb}{0.000000,0.000000,0.000000}%
\pgfsetstrokecolor{currentstroke}%
\pgfsetstrokeopacity{0.000000}%
\pgfsetdash{}{0pt}%
\pgfpathmoveto{\pgfqpoint{5.800000in}{27.868986in}}%
\pgfpathlineto{\pgfqpoint{5.800000in}{27.868986in}}%
\pgfpathlineto{\pgfqpoint{5.800000in}{27.971823in}}%
\pgfpathlineto{\pgfqpoint{5.800000in}{27.971823in}}%
\pgfpathclose%
\pgfusepath{fill}%
\end{pgfscope}%
\begin{pgfscope}%
\pgfpathrectangle{\pgfqpoint{5.800000in}{0.720000in}}{\pgfqpoint{1.400000in}{4.620000in}}%
\pgfusepath{clip}%
\pgfsetbuttcap%
\pgfsetmiterjoin%
\definecolor{currentfill}{rgb}{0.121569,0.466667,0.705882}%
\pgfsetfillcolor{currentfill}%
\pgfsetlinewidth{0.000000pt}%
\definecolor{currentstroke}{rgb}{0.000000,0.000000,0.000000}%
\pgfsetstrokecolor{currentstroke}%
\pgfsetstrokeopacity{0.000000}%
\pgfsetdash{}{0pt}%
\pgfpathmoveto{\pgfqpoint{5.800000in}{27.971823in}}%
\pgfpathlineto{\pgfqpoint{5.800000in}{27.971823in}}%
\pgfpathlineto{\pgfqpoint{5.800000in}{28.074660in}}%
\pgfpathlineto{\pgfqpoint{5.800000in}{28.074660in}}%
\pgfpathclose%
\pgfusepath{fill}%
\end{pgfscope}%
\begin{pgfscope}%
\pgfpathrectangle{\pgfqpoint{5.800000in}{0.720000in}}{\pgfqpoint{1.400000in}{4.620000in}}%
\pgfusepath{clip}%
\pgfsetbuttcap%
\pgfsetmiterjoin%
\definecolor{currentfill}{rgb}{0.121569,0.466667,0.705882}%
\pgfsetfillcolor{currentfill}%
\pgfsetlinewidth{0.000000pt}%
\definecolor{currentstroke}{rgb}{0.000000,0.000000,0.000000}%
\pgfsetstrokecolor{currentstroke}%
\pgfsetstrokeopacity{0.000000}%
\pgfsetdash{}{0pt}%
\pgfpathmoveto{\pgfqpoint{5.800000in}{28.074660in}}%
\pgfpathlineto{\pgfqpoint{5.800000in}{28.074660in}}%
\pgfpathlineto{\pgfqpoint{5.800000in}{28.177497in}}%
\pgfpathlineto{\pgfqpoint{5.800000in}{28.177497in}}%
\pgfpathclose%
\pgfusepath{fill}%
\end{pgfscope}%
\begin{pgfscope}%
\pgfpathrectangle{\pgfqpoint{5.800000in}{0.720000in}}{\pgfqpoint{1.400000in}{4.620000in}}%
\pgfusepath{clip}%
\pgfsetbuttcap%
\pgfsetmiterjoin%
\definecolor{currentfill}{rgb}{0.121569,0.466667,0.705882}%
\pgfsetfillcolor{currentfill}%
\pgfsetlinewidth{0.000000pt}%
\definecolor{currentstroke}{rgb}{0.000000,0.000000,0.000000}%
\pgfsetstrokecolor{currentstroke}%
\pgfsetstrokeopacity{0.000000}%
\pgfsetdash{}{0pt}%
\pgfpathmoveto{\pgfqpoint{5.800000in}{28.177497in}}%
\pgfpathlineto{\pgfqpoint{5.800000in}{28.177497in}}%
\pgfpathlineto{\pgfqpoint{5.800000in}{28.280334in}}%
\pgfpathlineto{\pgfqpoint{5.800000in}{28.280334in}}%
\pgfpathclose%
\pgfusepath{fill}%
\end{pgfscope}%
\begin{pgfscope}%
\pgfpathrectangle{\pgfqpoint{5.800000in}{0.720000in}}{\pgfqpoint{1.400000in}{4.620000in}}%
\pgfusepath{clip}%
\pgfsetbuttcap%
\pgfsetmiterjoin%
\definecolor{currentfill}{rgb}{0.121569,0.466667,0.705882}%
\pgfsetfillcolor{currentfill}%
\pgfsetlinewidth{0.000000pt}%
\definecolor{currentstroke}{rgb}{0.000000,0.000000,0.000000}%
\pgfsetstrokecolor{currentstroke}%
\pgfsetstrokeopacity{0.000000}%
\pgfsetdash{}{0pt}%
\pgfpathmoveto{\pgfqpoint{5.800000in}{28.280334in}}%
\pgfpathlineto{\pgfqpoint{5.800000in}{28.280334in}}%
\pgfpathlineto{\pgfqpoint{5.800000in}{28.383171in}}%
\pgfpathlineto{\pgfqpoint{5.800000in}{28.383171in}}%
\pgfpathclose%
\pgfusepath{fill}%
\end{pgfscope}%
\begin{pgfscope}%
\pgfpathrectangle{\pgfqpoint{5.800000in}{0.720000in}}{\pgfqpoint{1.400000in}{4.620000in}}%
\pgfusepath{clip}%
\pgfsetbuttcap%
\pgfsetmiterjoin%
\definecolor{currentfill}{rgb}{0.121569,0.466667,0.705882}%
\pgfsetfillcolor{currentfill}%
\pgfsetlinewidth{0.000000pt}%
\definecolor{currentstroke}{rgb}{0.000000,0.000000,0.000000}%
\pgfsetstrokecolor{currentstroke}%
\pgfsetstrokeopacity{0.000000}%
\pgfsetdash{}{0pt}%
\pgfpathmoveto{\pgfqpoint{5.800000in}{28.383171in}}%
\pgfpathlineto{\pgfqpoint{5.800000in}{28.383171in}}%
\pgfpathlineto{\pgfqpoint{5.800000in}{28.486008in}}%
\pgfpathlineto{\pgfqpoint{5.800000in}{28.486008in}}%
\pgfpathclose%
\pgfusepath{fill}%
\end{pgfscope}%
\begin{pgfscope}%
\pgfpathrectangle{\pgfqpoint{5.800000in}{0.720000in}}{\pgfqpoint{1.400000in}{4.620000in}}%
\pgfusepath{clip}%
\pgfsetbuttcap%
\pgfsetmiterjoin%
\definecolor{currentfill}{rgb}{0.121569,0.466667,0.705882}%
\pgfsetfillcolor{currentfill}%
\pgfsetlinewidth{0.000000pt}%
\definecolor{currentstroke}{rgb}{0.000000,0.000000,0.000000}%
\pgfsetstrokecolor{currentstroke}%
\pgfsetstrokeopacity{0.000000}%
\pgfsetdash{}{0pt}%
\pgfpathmoveto{\pgfqpoint{5.800000in}{28.486008in}}%
\pgfpathlineto{\pgfqpoint{5.800000in}{28.486008in}}%
\pgfpathlineto{\pgfqpoint{5.800000in}{28.588845in}}%
\pgfpathlineto{\pgfqpoint{5.800000in}{28.588845in}}%
\pgfpathclose%
\pgfusepath{fill}%
\end{pgfscope}%
\begin{pgfscope}%
\pgfpathrectangle{\pgfqpoint{5.800000in}{0.720000in}}{\pgfqpoint{1.400000in}{4.620000in}}%
\pgfusepath{clip}%
\pgfsetbuttcap%
\pgfsetmiterjoin%
\definecolor{currentfill}{rgb}{0.121569,0.466667,0.705882}%
\pgfsetfillcolor{currentfill}%
\pgfsetlinewidth{0.000000pt}%
\definecolor{currentstroke}{rgb}{0.000000,0.000000,0.000000}%
\pgfsetstrokecolor{currentstroke}%
\pgfsetstrokeopacity{0.000000}%
\pgfsetdash{}{0pt}%
\pgfpathmoveto{\pgfqpoint{5.800000in}{28.588845in}}%
\pgfpathlineto{\pgfqpoint{5.800000in}{28.588845in}}%
\pgfpathlineto{\pgfqpoint{5.800000in}{28.691682in}}%
\pgfpathlineto{\pgfqpoint{5.800000in}{28.691682in}}%
\pgfpathclose%
\pgfusepath{fill}%
\end{pgfscope}%
\begin{pgfscope}%
\pgfpathrectangle{\pgfqpoint{5.800000in}{0.720000in}}{\pgfqpoint{1.400000in}{4.620000in}}%
\pgfusepath{clip}%
\pgfsetbuttcap%
\pgfsetmiterjoin%
\definecolor{currentfill}{rgb}{0.121569,0.466667,0.705882}%
\pgfsetfillcolor{currentfill}%
\pgfsetlinewidth{0.000000pt}%
\definecolor{currentstroke}{rgb}{0.000000,0.000000,0.000000}%
\pgfsetstrokecolor{currentstroke}%
\pgfsetstrokeopacity{0.000000}%
\pgfsetdash{}{0pt}%
\pgfpathmoveto{\pgfqpoint{5.800000in}{28.691682in}}%
\pgfpathlineto{\pgfqpoint{5.800000in}{28.691682in}}%
\pgfpathlineto{\pgfqpoint{5.800000in}{28.794520in}}%
\pgfpathlineto{\pgfqpoint{5.800000in}{28.794520in}}%
\pgfpathclose%
\pgfusepath{fill}%
\end{pgfscope}%
\begin{pgfscope}%
\pgfpathrectangle{\pgfqpoint{5.800000in}{0.720000in}}{\pgfqpoint{1.400000in}{4.620000in}}%
\pgfusepath{clip}%
\pgfsetbuttcap%
\pgfsetmiterjoin%
\definecolor{currentfill}{rgb}{0.121569,0.466667,0.705882}%
\pgfsetfillcolor{currentfill}%
\pgfsetlinewidth{0.000000pt}%
\definecolor{currentstroke}{rgb}{0.000000,0.000000,0.000000}%
\pgfsetstrokecolor{currentstroke}%
\pgfsetstrokeopacity{0.000000}%
\pgfsetdash{}{0pt}%
\pgfpathmoveto{\pgfqpoint{5.800000in}{28.794520in}}%
\pgfpathlineto{\pgfqpoint{5.802038in}{28.794520in}}%
\pgfpathlineto{\pgfqpoint{5.802038in}{28.897357in}}%
\pgfpathlineto{\pgfqpoint{5.800000in}{28.897357in}}%
\pgfpathclose%
\pgfusepath{fill}%
\end{pgfscope}%
\begin{pgfscope}%
\pgfpathrectangle{\pgfqpoint{5.800000in}{0.720000in}}{\pgfqpoint{1.400000in}{4.620000in}}%
\pgfusepath{clip}%
\pgfsetbuttcap%
\pgfsetmiterjoin%
\definecolor{currentfill}{rgb}{0.121569,0.466667,0.705882}%
\pgfsetfillcolor{currentfill}%
\pgfsetlinewidth{0.000000pt}%
\definecolor{currentstroke}{rgb}{0.000000,0.000000,0.000000}%
\pgfsetstrokecolor{currentstroke}%
\pgfsetstrokeopacity{0.000000}%
\pgfsetdash{}{0pt}%
\pgfpathmoveto{\pgfqpoint{5.800000in}{28.897357in}}%
\pgfpathlineto{\pgfqpoint{5.800000in}{28.897357in}}%
\pgfpathlineto{\pgfqpoint{5.800000in}{29.000194in}}%
\pgfpathlineto{\pgfqpoint{5.800000in}{29.000194in}}%
\pgfpathclose%
\pgfusepath{fill}%
\end{pgfscope}%
\begin{pgfscope}%
\pgfpathrectangle{\pgfqpoint{5.800000in}{0.720000in}}{\pgfqpoint{1.400000in}{4.620000in}}%
\pgfusepath{clip}%
\pgfsetbuttcap%
\pgfsetmiterjoin%
\definecolor{currentfill}{rgb}{0.121569,0.466667,0.705882}%
\pgfsetfillcolor{currentfill}%
\pgfsetlinewidth{0.000000pt}%
\definecolor{currentstroke}{rgb}{0.000000,0.000000,0.000000}%
\pgfsetstrokecolor{currentstroke}%
\pgfsetstrokeopacity{0.000000}%
\pgfsetdash{}{0pt}%
\pgfpathmoveto{\pgfqpoint{5.800000in}{29.000194in}}%
\pgfpathlineto{\pgfqpoint{5.800000in}{29.000194in}}%
\pgfpathlineto{\pgfqpoint{5.800000in}{29.103031in}}%
\pgfpathlineto{\pgfqpoint{5.800000in}{29.103031in}}%
\pgfpathclose%
\pgfusepath{fill}%
\end{pgfscope}%
\begin{pgfscope}%
\pgfpathrectangle{\pgfqpoint{5.800000in}{0.720000in}}{\pgfqpoint{1.400000in}{4.620000in}}%
\pgfusepath{clip}%
\pgfsetbuttcap%
\pgfsetmiterjoin%
\definecolor{currentfill}{rgb}{0.121569,0.466667,0.705882}%
\pgfsetfillcolor{currentfill}%
\pgfsetlinewidth{0.000000pt}%
\definecolor{currentstroke}{rgb}{0.000000,0.000000,0.000000}%
\pgfsetstrokecolor{currentstroke}%
\pgfsetstrokeopacity{0.000000}%
\pgfsetdash{}{0pt}%
\pgfpathmoveto{\pgfqpoint{5.800000in}{29.103031in}}%
\pgfpathlineto{\pgfqpoint{5.800000in}{29.103031in}}%
\pgfpathlineto{\pgfqpoint{5.800000in}{29.205868in}}%
\pgfpathlineto{\pgfqpoint{5.800000in}{29.205868in}}%
\pgfpathclose%
\pgfusepath{fill}%
\end{pgfscope}%
\begin{pgfscope}%
\pgfpathrectangle{\pgfqpoint{5.800000in}{0.720000in}}{\pgfqpoint{1.400000in}{4.620000in}}%
\pgfusepath{clip}%
\pgfsetbuttcap%
\pgfsetmiterjoin%
\definecolor{currentfill}{rgb}{0.121569,0.466667,0.705882}%
\pgfsetfillcolor{currentfill}%
\pgfsetlinewidth{0.000000pt}%
\definecolor{currentstroke}{rgb}{0.000000,0.000000,0.000000}%
\pgfsetstrokecolor{currentstroke}%
\pgfsetstrokeopacity{0.000000}%
\pgfsetdash{}{0pt}%
\pgfpathmoveto{\pgfqpoint{5.800000in}{29.205868in}}%
\pgfpathlineto{\pgfqpoint{5.800000in}{29.205868in}}%
\pgfpathlineto{\pgfqpoint{5.800000in}{29.308705in}}%
\pgfpathlineto{\pgfqpoint{5.800000in}{29.308705in}}%
\pgfpathclose%
\pgfusepath{fill}%
\end{pgfscope}%
\begin{pgfscope}%
\pgfpathrectangle{\pgfqpoint{5.800000in}{0.720000in}}{\pgfqpoint{1.400000in}{4.620000in}}%
\pgfusepath{clip}%
\pgfsetbuttcap%
\pgfsetmiterjoin%
\definecolor{currentfill}{rgb}{0.121569,0.466667,0.705882}%
\pgfsetfillcolor{currentfill}%
\pgfsetlinewidth{0.000000pt}%
\definecolor{currentstroke}{rgb}{0.000000,0.000000,0.000000}%
\pgfsetstrokecolor{currentstroke}%
\pgfsetstrokeopacity{0.000000}%
\pgfsetdash{}{0pt}%
\pgfpathmoveto{\pgfqpoint{5.800000in}{29.308705in}}%
\pgfpathlineto{\pgfqpoint{5.800000in}{29.308705in}}%
\pgfpathlineto{\pgfqpoint{5.800000in}{29.411542in}}%
\pgfpathlineto{\pgfqpoint{5.800000in}{29.411542in}}%
\pgfpathclose%
\pgfusepath{fill}%
\end{pgfscope}%
\begin{pgfscope}%
\pgfpathrectangle{\pgfqpoint{5.800000in}{0.720000in}}{\pgfqpoint{1.400000in}{4.620000in}}%
\pgfusepath{clip}%
\pgfsetbuttcap%
\pgfsetmiterjoin%
\definecolor{currentfill}{rgb}{0.121569,0.466667,0.705882}%
\pgfsetfillcolor{currentfill}%
\pgfsetlinewidth{0.000000pt}%
\definecolor{currentstroke}{rgb}{0.000000,0.000000,0.000000}%
\pgfsetstrokecolor{currentstroke}%
\pgfsetstrokeopacity{0.000000}%
\pgfsetdash{}{0pt}%
\pgfpathmoveto{\pgfqpoint{5.800000in}{29.411542in}}%
\pgfpathlineto{\pgfqpoint{5.800000in}{29.411542in}}%
\pgfpathlineto{\pgfqpoint{5.800000in}{29.514379in}}%
\pgfpathlineto{\pgfqpoint{5.800000in}{29.514379in}}%
\pgfpathclose%
\pgfusepath{fill}%
\end{pgfscope}%
\begin{pgfscope}%
\pgfpathrectangle{\pgfqpoint{5.800000in}{0.720000in}}{\pgfqpoint{1.400000in}{4.620000in}}%
\pgfusepath{clip}%
\pgfsetbuttcap%
\pgfsetmiterjoin%
\definecolor{currentfill}{rgb}{0.121569,0.466667,0.705882}%
\pgfsetfillcolor{currentfill}%
\pgfsetlinewidth{0.000000pt}%
\definecolor{currentstroke}{rgb}{0.000000,0.000000,0.000000}%
\pgfsetstrokecolor{currentstroke}%
\pgfsetstrokeopacity{0.000000}%
\pgfsetdash{}{0pt}%
\pgfpathmoveto{\pgfqpoint{5.800000in}{29.514379in}}%
\pgfpathlineto{\pgfqpoint{5.800000in}{29.514379in}}%
\pgfpathlineto{\pgfqpoint{5.800000in}{29.617216in}}%
\pgfpathlineto{\pgfqpoint{5.800000in}{29.617216in}}%
\pgfpathclose%
\pgfusepath{fill}%
\end{pgfscope}%
\begin{pgfscope}%
\pgfpathrectangle{\pgfqpoint{5.800000in}{0.720000in}}{\pgfqpoint{1.400000in}{4.620000in}}%
\pgfusepath{clip}%
\pgfsetbuttcap%
\pgfsetmiterjoin%
\definecolor{currentfill}{rgb}{0.121569,0.466667,0.705882}%
\pgfsetfillcolor{currentfill}%
\pgfsetlinewidth{0.000000pt}%
\definecolor{currentstroke}{rgb}{0.000000,0.000000,0.000000}%
\pgfsetstrokecolor{currentstroke}%
\pgfsetstrokeopacity{0.000000}%
\pgfsetdash{}{0pt}%
\pgfpathmoveto{\pgfqpoint{5.800000in}{29.617216in}}%
\pgfpathlineto{\pgfqpoint{5.800000in}{29.617216in}}%
\pgfpathlineto{\pgfqpoint{5.800000in}{29.720053in}}%
\pgfpathlineto{\pgfqpoint{5.800000in}{29.720053in}}%
\pgfpathclose%
\pgfusepath{fill}%
\end{pgfscope}%
\begin{pgfscope}%
\pgfpathrectangle{\pgfqpoint{5.800000in}{0.720000in}}{\pgfqpoint{1.400000in}{4.620000in}}%
\pgfusepath{clip}%
\pgfsetbuttcap%
\pgfsetmiterjoin%
\definecolor{currentfill}{rgb}{0.121569,0.466667,0.705882}%
\pgfsetfillcolor{currentfill}%
\pgfsetlinewidth{0.000000pt}%
\definecolor{currentstroke}{rgb}{0.000000,0.000000,0.000000}%
\pgfsetstrokecolor{currentstroke}%
\pgfsetstrokeopacity{0.000000}%
\pgfsetdash{}{0pt}%
\pgfpathmoveto{\pgfqpoint{5.800000in}{29.720053in}}%
\pgfpathlineto{\pgfqpoint{5.800000in}{29.720053in}}%
\pgfpathlineto{\pgfqpoint{5.800000in}{29.822890in}}%
\pgfpathlineto{\pgfqpoint{5.800000in}{29.822890in}}%
\pgfpathclose%
\pgfusepath{fill}%
\end{pgfscope}%
\begin{pgfscope}%
\pgfpathrectangle{\pgfqpoint{5.800000in}{0.720000in}}{\pgfqpoint{1.400000in}{4.620000in}}%
\pgfusepath{clip}%
\pgfsetbuttcap%
\pgfsetmiterjoin%
\definecolor{currentfill}{rgb}{0.121569,0.466667,0.705882}%
\pgfsetfillcolor{currentfill}%
\pgfsetlinewidth{0.000000pt}%
\definecolor{currentstroke}{rgb}{0.000000,0.000000,0.000000}%
\pgfsetstrokecolor{currentstroke}%
\pgfsetstrokeopacity{0.000000}%
\pgfsetdash{}{0pt}%
\pgfpathmoveto{\pgfqpoint{5.800000in}{29.822890in}}%
\pgfpathlineto{\pgfqpoint{5.800000in}{29.822890in}}%
\pgfpathlineto{\pgfqpoint{5.800000in}{29.925727in}}%
\pgfpathlineto{\pgfqpoint{5.800000in}{29.925727in}}%
\pgfpathclose%
\pgfusepath{fill}%
\end{pgfscope}%
\begin{pgfscope}%
\pgfpathrectangle{\pgfqpoint{5.800000in}{0.720000in}}{\pgfqpoint{1.400000in}{4.620000in}}%
\pgfusepath{clip}%
\pgfsetbuttcap%
\pgfsetmiterjoin%
\definecolor{currentfill}{rgb}{0.121569,0.466667,0.705882}%
\pgfsetfillcolor{currentfill}%
\pgfsetlinewidth{0.000000pt}%
\definecolor{currentstroke}{rgb}{0.000000,0.000000,0.000000}%
\pgfsetstrokecolor{currentstroke}%
\pgfsetstrokeopacity{0.000000}%
\pgfsetdash{}{0pt}%
\pgfpathmoveto{\pgfqpoint{5.800000in}{29.925727in}}%
\pgfpathlineto{\pgfqpoint{5.800000in}{29.925727in}}%
\pgfpathlineto{\pgfqpoint{5.800000in}{30.028564in}}%
\pgfpathlineto{\pgfqpoint{5.800000in}{30.028564in}}%
\pgfpathclose%
\pgfusepath{fill}%
\end{pgfscope}%
\begin{pgfscope}%
\pgfpathrectangle{\pgfqpoint{5.800000in}{0.720000in}}{\pgfqpoint{1.400000in}{4.620000in}}%
\pgfusepath{clip}%
\pgfsetbuttcap%
\pgfsetmiterjoin%
\definecolor{currentfill}{rgb}{0.121569,0.466667,0.705882}%
\pgfsetfillcolor{currentfill}%
\pgfsetlinewidth{0.000000pt}%
\definecolor{currentstroke}{rgb}{0.000000,0.000000,0.000000}%
\pgfsetstrokecolor{currentstroke}%
\pgfsetstrokeopacity{0.000000}%
\pgfsetdash{}{0pt}%
\pgfpathmoveto{\pgfqpoint{5.800000in}{30.028564in}}%
\pgfpathlineto{\pgfqpoint{5.800000in}{30.028564in}}%
\pgfpathlineto{\pgfqpoint{5.800000in}{30.131401in}}%
\pgfpathlineto{\pgfqpoint{5.800000in}{30.131401in}}%
\pgfpathclose%
\pgfusepath{fill}%
\end{pgfscope}%
\begin{pgfscope}%
\pgfpathrectangle{\pgfqpoint{5.800000in}{0.720000in}}{\pgfqpoint{1.400000in}{4.620000in}}%
\pgfusepath{clip}%
\pgfsetbuttcap%
\pgfsetmiterjoin%
\definecolor{currentfill}{rgb}{0.121569,0.466667,0.705882}%
\pgfsetfillcolor{currentfill}%
\pgfsetlinewidth{0.000000pt}%
\definecolor{currentstroke}{rgb}{0.000000,0.000000,0.000000}%
\pgfsetstrokecolor{currentstroke}%
\pgfsetstrokeopacity{0.000000}%
\pgfsetdash{}{0pt}%
\pgfpathmoveto{\pgfqpoint{5.800000in}{30.131401in}}%
\pgfpathlineto{\pgfqpoint{5.800000in}{30.131401in}}%
\pgfpathlineto{\pgfqpoint{5.800000in}{30.234238in}}%
\pgfpathlineto{\pgfqpoint{5.800000in}{30.234238in}}%
\pgfpathclose%
\pgfusepath{fill}%
\end{pgfscope}%
\begin{pgfscope}%
\pgfpathrectangle{\pgfqpoint{5.800000in}{0.720000in}}{\pgfqpoint{1.400000in}{4.620000in}}%
\pgfusepath{clip}%
\pgfsetbuttcap%
\pgfsetmiterjoin%
\definecolor{currentfill}{rgb}{0.121569,0.466667,0.705882}%
\pgfsetfillcolor{currentfill}%
\pgfsetlinewidth{0.000000pt}%
\definecolor{currentstroke}{rgb}{0.000000,0.000000,0.000000}%
\pgfsetstrokecolor{currentstroke}%
\pgfsetstrokeopacity{0.000000}%
\pgfsetdash{}{0pt}%
\pgfpathmoveto{\pgfqpoint{5.800000in}{30.234238in}}%
\pgfpathlineto{\pgfqpoint{5.800000in}{30.234238in}}%
\pgfpathlineto{\pgfqpoint{5.800000in}{30.337076in}}%
\pgfpathlineto{\pgfqpoint{5.800000in}{30.337076in}}%
\pgfpathclose%
\pgfusepath{fill}%
\end{pgfscope}%
\begin{pgfscope}%
\pgfpathrectangle{\pgfqpoint{5.800000in}{0.720000in}}{\pgfqpoint{1.400000in}{4.620000in}}%
\pgfusepath{clip}%
\pgfsetbuttcap%
\pgfsetmiterjoin%
\definecolor{currentfill}{rgb}{0.121569,0.466667,0.705882}%
\pgfsetfillcolor{currentfill}%
\pgfsetlinewidth{0.000000pt}%
\definecolor{currentstroke}{rgb}{0.000000,0.000000,0.000000}%
\pgfsetstrokecolor{currentstroke}%
\pgfsetstrokeopacity{0.000000}%
\pgfsetdash{}{0pt}%
\pgfpathmoveto{\pgfqpoint{5.800000in}{30.337076in}}%
\pgfpathlineto{\pgfqpoint{5.800000in}{30.337076in}}%
\pgfpathlineto{\pgfqpoint{5.800000in}{30.439913in}}%
\pgfpathlineto{\pgfqpoint{5.800000in}{30.439913in}}%
\pgfpathclose%
\pgfusepath{fill}%
\end{pgfscope}%
\begin{pgfscope}%
\pgfpathrectangle{\pgfqpoint{5.800000in}{0.720000in}}{\pgfqpoint{1.400000in}{4.620000in}}%
\pgfusepath{clip}%
\pgfsetbuttcap%
\pgfsetmiterjoin%
\definecolor{currentfill}{rgb}{0.121569,0.466667,0.705882}%
\pgfsetfillcolor{currentfill}%
\pgfsetlinewidth{0.000000pt}%
\definecolor{currentstroke}{rgb}{0.000000,0.000000,0.000000}%
\pgfsetstrokecolor{currentstroke}%
\pgfsetstrokeopacity{0.000000}%
\pgfsetdash{}{0pt}%
\pgfpathmoveto{\pgfqpoint{5.800000in}{30.439913in}}%
\pgfpathlineto{\pgfqpoint{5.800000in}{30.439913in}}%
\pgfpathlineto{\pgfqpoint{5.800000in}{30.542750in}}%
\pgfpathlineto{\pgfqpoint{5.800000in}{30.542750in}}%
\pgfpathclose%
\pgfusepath{fill}%
\end{pgfscope}%
\begin{pgfscope}%
\pgfpathrectangle{\pgfqpoint{5.800000in}{0.720000in}}{\pgfqpoint{1.400000in}{4.620000in}}%
\pgfusepath{clip}%
\pgfsetbuttcap%
\pgfsetmiterjoin%
\definecolor{currentfill}{rgb}{0.121569,0.466667,0.705882}%
\pgfsetfillcolor{currentfill}%
\pgfsetlinewidth{0.000000pt}%
\definecolor{currentstroke}{rgb}{0.000000,0.000000,0.000000}%
\pgfsetstrokecolor{currentstroke}%
\pgfsetstrokeopacity{0.000000}%
\pgfsetdash{}{0pt}%
\pgfpathmoveto{\pgfqpoint{5.800000in}{30.542750in}}%
\pgfpathlineto{\pgfqpoint{5.800000in}{30.542750in}}%
\pgfpathlineto{\pgfqpoint{5.800000in}{30.645587in}}%
\pgfpathlineto{\pgfqpoint{5.800000in}{30.645587in}}%
\pgfpathclose%
\pgfusepath{fill}%
\end{pgfscope}%
\begin{pgfscope}%
\pgfpathrectangle{\pgfqpoint{5.800000in}{0.720000in}}{\pgfqpoint{1.400000in}{4.620000in}}%
\pgfusepath{clip}%
\pgfsetbuttcap%
\pgfsetmiterjoin%
\definecolor{currentfill}{rgb}{0.121569,0.466667,0.705882}%
\pgfsetfillcolor{currentfill}%
\pgfsetlinewidth{0.000000pt}%
\definecolor{currentstroke}{rgb}{0.000000,0.000000,0.000000}%
\pgfsetstrokecolor{currentstroke}%
\pgfsetstrokeopacity{0.000000}%
\pgfsetdash{}{0pt}%
\pgfpathmoveto{\pgfqpoint{5.800000in}{30.645587in}}%
\pgfpathlineto{\pgfqpoint{5.800000in}{30.645587in}}%
\pgfpathlineto{\pgfqpoint{5.800000in}{30.748424in}}%
\pgfpathlineto{\pgfqpoint{5.800000in}{30.748424in}}%
\pgfpathclose%
\pgfusepath{fill}%
\end{pgfscope}%
\begin{pgfscope}%
\pgfpathrectangle{\pgfqpoint{5.800000in}{0.720000in}}{\pgfqpoint{1.400000in}{4.620000in}}%
\pgfusepath{clip}%
\pgfsetbuttcap%
\pgfsetmiterjoin%
\definecolor{currentfill}{rgb}{0.121569,0.466667,0.705882}%
\pgfsetfillcolor{currentfill}%
\pgfsetlinewidth{0.000000pt}%
\definecolor{currentstroke}{rgb}{0.000000,0.000000,0.000000}%
\pgfsetstrokecolor{currentstroke}%
\pgfsetstrokeopacity{0.000000}%
\pgfsetdash{}{0pt}%
\pgfpathmoveto{\pgfqpoint{5.800000in}{30.748424in}}%
\pgfpathlineto{\pgfqpoint{5.800000in}{30.748424in}}%
\pgfpathlineto{\pgfqpoint{5.800000in}{30.851261in}}%
\pgfpathlineto{\pgfqpoint{5.800000in}{30.851261in}}%
\pgfpathclose%
\pgfusepath{fill}%
\end{pgfscope}%
\begin{pgfscope}%
\pgfpathrectangle{\pgfqpoint{5.800000in}{0.720000in}}{\pgfqpoint{1.400000in}{4.620000in}}%
\pgfusepath{clip}%
\pgfsetbuttcap%
\pgfsetmiterjoin%
\definecolor{currentfill}{rgb}{0.121569,0.466667,0.705882}%
\pgfsetfillcolor{currentfill}%
\pgfsetlinewidth{0.000000pt}%
\definecolor{currentstroke}{rgb}{0.000000,0.000000,0.000000}%
\pgfsetstrokecolor{currentstroke}%
\pgfsetstrokeopacity{0.000000}%
\pgfsetdash{}{0pt}%
\pgfpathmoveto{\pgfqpoint{5.800000in}{30.851261in}}%
\pgfpathlineto{\pgfqpoint{5.800000in}{30.851261in}}%
\pgfpathlineto{\pgfqpoint{5.800000in}{30.954098in}}%
\pgfpathlineto{\pgfqpoint{5.800000in}{30.954098in}}%
\pgfpathclose%
\pgfusepath{fill}%
\end{pgfscope}%
\begin{pgfscope}%
\pgfpathrectangle{\pgfqpoint{5.800000in}{0.720000in}}{\pgfqpoint{1.400000in}{4.620000in}}%
\pgfusepath{clip}%
\pgfsetbuttcap%
\pgfsetmiterjoin%
\definecolor{currentfill}{rgb}{0.121569,0.466667,0.705882}%
\pgfsetfillcolor{currentfill}%
\pgfsetlinewidth{0.000000pt}%
\definecolor{currentstroke}{rgb}{0.000000,0.000000,0.000000}%
\pgfsetstrokecolor{currentstroke}%
\pgfsetstrokeopacity{0.000000}%
\pgfsetdash{}{0pt}%
\pgfpathmoveto{\pgfqpoint{5.800000in}{30.954098in}}%
\pgfpathlineto{\pgfqpoint{5.800000in}{30.954098in}}%
\pgfpathlineto{\pgfqpoint{5.800000in}{31.056935in}}%
\pgfpathlineto{\pgfqpoint{5.800000in}{31.056935in}}%
\pgfpathclose%
\pgfusepath{fill}%
\end{pgfscope}%
\begin{pgfscope}%
\pgfpathrectangle{\pgfqpoint{5.800000in}{0.720000in}}{\pgfqpoint{1.400000in}{4.620000in}}%
\pgfusepath{clip}%
\pgfsetbuttcap%
\pgfsetmiterjoin%
\definecolor{currentfill}{rgb}{0.121569,0.466667,0.705882}%
\pgfsetfillcolor{currentfill}%
\pgfsetlinewidth{0.000000pt}%
\definecolor{currentstroke}{rgb}{0.000000,0.000000,0.000000}%
\pgfsetstrokecolor{currentstroke}%
\pgfsetstrokeopacity{0.000000}%
\pgfsetdash{}{0pt}%
\pgfpathmoveto{\pgfqpoint{5.800000in}{31.056935in}}%
\pgfpathlineto{\pgfqpoint{5.800000in}{31.056935in}}%
\pgfpathlineto{\pgfqpoint{5.800000in}{31.159772in}}%
\pgfpathlineto{\pgfqpoint{5.800000in}{31.159772in}}%
\pgfpathclose%
\pgfusepath{fill}%
\end{pgfscope}%
\begin{pgfscope}%
\pgfpathrectangle{\pgfqpoint{5.800000in}{0.720000in}}{\pgfqpoint{1.400000in}{4.620000in}}%
\pgfusepath{clip}%
\pgfsetbuttcap%
\pgfsetmiterjoin%
\definecolor{currentfill}{rgb}{0.121569,0.466667,0.705882}%
\pgfsetfillcolor{currentfill}%
\pgfsetlinewidth{0.000000pt}%
\definecolor{currentstroke}{rgb}{0.000000,0.000000,0.000000}%
\pgfsetstrokecolor{currentstroke}%
\pgfsetstrokeopacity{0.000000}%
\pgfsetdash{}{0pt}%
\pgfpathmoveto{\pgfqpoint{5.800000in}{31.159772in}}%
\pgfpathlineto{\pgfqpoint{5.800000in}{31.159772in}}%
\pgfpathlineto{\pgfqpoint{5.800000in}{31.262609in}}%
\pgfpathlineto{\pgfqpoint{5.800000in}{31.262609in}}%
\pgfpathclose%
\pgfusepath{fill}%
\end{pgfscope}%
\begin{pgfscope}%
\pgfpathrectangle{\pgfqpoint{5.800000in}{0.720000in}}{\pgfqpoint{1.400000in}{4.620000in}}%
\pgfusepath{clip}%
\pgfsetbuttcap%
\pgfsetmiterjoin%
\definecolor{currentfill}{rgb}{0.121569,0.466667,0.705882}%
\pgfsetfillcolor{currentfill}%
\pgfsetlinewidth{0.000000pt}%
\definecolor{currentstroke}{rgb}{0.000000,0.000000,0.000000}%
\pgfsetstrokecolor{currentstroke}%
\pgfsetstrokeopacity{0.000000}%
\pgfsetdash{}{0pt}%
\pgfpathmoveto{\pgfqpoint{5.800000in}{31.262609in}}%
\pgfpathlineto{\pgfqpoint{5.800000in}{31.262609in}}%
\pgfpathlineto{\pgfqpoint{5.800000in}{31.365446in}}%
\pgfpathlineto{\pgfqpoint{5.800000in}{31.365446in}}%
\pgfpathclose%
\pgfusepath{fill}%
\end{pgfscope}%
\begin{pgfscope}%
\pgfpathrectangle{\pgfqpoint{5.800000in}{0.720000in}}{\pgfqpoint{1.400000in}{4.620000in}}%
\pgfusepath{clip}%
\pgfsetbuttcap%
\pgfsetmiterjoin%
\definecolor{currentfill}{rgb}{0.121569,0.466667,0.705882}%
\pgfsetfillcolor{currentfill}%
\pgfsetlinewidth{0.000000pt}%
\definecolor{currentstroke}{rgb}{0.000000,0.000000,0.000000}%
\pgfsetstrokecolor{currentstroke}%
\pgfsetstrokeopacity{0.000000}%
\pgfsetdash{}{0pt}%
\pgfpathmoveto{\pgfqpoint{5.800000in}{31.365446in}}%
\pgfpathlineto{\pgfqpoint{5.802038in}{31.365446in}}%
\pgfpathlineto{\pgfqpoint{5.802038in}{31.468283in}}%
\pgfpathlineto{\pgfqpoint{5.800000in}{31.468283in}}%
\pgfpathclose%
\pgfusepath{fill}%
\end{pgfscope}%
\begin{pgfscope}%
\pgfpathrectangle{\pgfqpoint{5.800000in}{0.720000in}}{\pgfqpoint{1.400000in}{4.620000in}}%
\pgfusepath{clip}%
\pgfsetbuttcap%
\pgfsetmiterjoin%
\definecolor{currentfill}{rgb}{0.121569,0.466667,0.705882}%
\pgfsetfillcolor{currentfill}%
\pgfsetlinewidth{0.000000pt}%
\definecolor{currentstroke}{rgb}{0.000000,0.000000,0.000000}%
\pgfsetstrokecolor{currentstroke}%
\pgfsetstrokeopacity{0.000000}%
\pgfsetdash{}{0pt}%
\pgfpathmoveto{\pgfqpoint{5.800000in}{31.468283in}}%
\pgfpathlineto{\pgfqpoint{5.800000in}{31.468283in}}%
\pgfpathlineto{\pgfqpoint{5.800000in}{31.571120in}}%
\pgfpathlineto{\pgfqpoint{5.800000in}{31.571120in}}%
\pgfpathclose%
\pgfusepath{fill}%
\end{pgfscope}%
\begin{pgfscope}%
\pgfpathrectangle{\pgfqpoint{5.800000in}{0.720000in}}{\pgfqpoint{1.400000in}{4.620000in}}%
\pgfusepath{clip}%
\pgfsetbuttcap%
\pgfsetmiterjoin%
\definecolor{currentfill}{rgb}{0.121569,0.466667,0.705882}%
\pgfsetfillcolor{currentfill}%
\pgfsetlinewidth{0.000000pt}%
\definecolor{currentstroke}{rgb}{0.000000,0.000000,0.000000}%
\pgfsetstrokecolor{currentstroke}%
\pgfsetstrokeopacity{0.000000}%
\pgfsetdash{}{0pt}%
\pgfpathmoveto{\pgfqpoint{5.800000in}{31.571120in}}%
\pgfpathlineto{\pgfqpoint{5.800000in}{31.571120in}}%
\pgfpathlineto{\pgfqpoint{5.800000in}{31.673957in}}%
\pgfpathlineto{\pgfqpoint{5.800000in}{31.673957in}}%
\pgfpathclose%
\pgfusepath{fill}%
\end{pgfscope}%
\begin{pgfscope}%
\pgfpathrectangle{\pgfqpoint{5.800000in}{0.720000in}}{\pgfqpoint{1.400000in}{4.620000in}}%
\pgfusepath{clip}%
\pgfsetbuttcap%
\pgfsetmiterjoin%
\definecolor{currentfill}{rgb}{0.121569,0.466667,0.705882}%
\pgfsetfillcolor{currentfill}%
\pgfsetlinewidth{0.000000pt}%
\definecolor{currentstroke}{rgb}{0.000000,0.000000,0.000000}%
\pgfsetstrokecolor{currentstroke}%
\pgfsetstrokeopacity{0.000000}%
\pgfsetdash{}{0pt}%
\pgfpathmoveto{\pgfqpoint{5.800000in}{31.673957in}}%
\pgfpathlineto{\pgfqpoint{5.800000in}{31.673957in}}%
\pgfpathlineto{\pgfqpoint{5.800000in}{31.776794in}}%
\pgfpathlineto{\pgfqpoint{5.800000in}{31.776794in}}%
\pgfpathclose%
\pgfusepath{fill}%
\end{pgfscope}%
\begin{pgfscope}%
\pgfpathrectangle{\pgfqpoint{5.800000in}{0.720000in}}{\pgfqpoint{1.400000in}{4.620000in}}%
\pgfusepath{clip}%
\pgfsetbuttcap%
\pgfsetmiterjoin%
\definecolor{currentfill}{rgb}{0.121569,0.466667,0.705882}%
\pgfsetfillcolor{currentfill}%
\pgfsetlinewidth{0.000000pt}%
\definecolor{currentstroke}{rgb}{0.000000,0.000000,0.000000}%
\pgfsetstrokecolor{currentstroke}%
\pgfsetstrokeopacity{0.000000}%
\pgfsetdash{}{0pt}%
\pgfpathmoveto{\pgfqpoint{5.800000in}{31.776794in}}%
\pgfpathlineto{\pgfqpoint{5.800000in}{31.776794in}}%
\pgfpathlineto{\pgfqpoint{5.800000in}{31.879632in}}%
\pgfpathlineto{\pgfqpoint{5.800000in}{31.879632in}}%
\pgfpathclose%
\pgfusepath{fill}%
\end{pgfscope}%
\begin{pgfscope}%
\pgfpathrectangle{\pgfqpoint{5.800000in}{0.720000in}}{\pgfqpoint{1.400000in}{4.620000in}}%
\pgfusepath{clip}%
\pgfsetbuttcap%
\pgfsetmiterjoin%
\definecolor{currentfill}{rgb}{0.121569,0.466667,0.705882}%
\pgfsetfillcolor{currentfill}%
\pgfsetlinewidth{0.000000pt}%
\definecolor{currentstroke}{rgb}{0.000000,0.000000,0.000000}%
\pgfsetstrokecolor{currentstroke}%
\pgfsetstrokeopacity{0.000000}%
\pgfsetdash{}{0pt}%
\pgfpathmoveto{\pgfqpoint{5.800000in}{31.879632in}}%
\pgfpathlineto{\pgfqpoint{5.800000in}{31.879632in}}%
\pgfpathlineto{\pgfqpoint{5.800000in}{31.982469in}}%
\pgfpathlineto{\pgfqpoint{5.800000in}{31.982469in}}%
\pgfpathclose%
\pgfusepath{fill}%
\end{pgfscope}%
\begin{pgfscope}%
\pgfpathrectangle{\pgfqpoint{5.800000in}{0.720000in}}{\pgfqpoint{1.400000in}{4.620000in}}%
\pgfusepath{clip}%
\pgfsetbuttcap%
\pgfsetmiterjoin%
\definecolor{currentfill}{rgb}{0.121569,0.466667,0.705882}%
\pgfsetfillcolor{currentfill}%
\pgfsetlinewidth{0.000000pt}%
\definecolor{currentstroke}{rgb}{0.000000,0.000000,0.000000}%
\pgfsetstrokecolor{currentstroke}%
\pgfsetstrokeopacity{0.000000}%
\pgfsetdash{}{0pt}%
\pgfpathmoveto{\pgfqpoint{5.800000in}{31.982469in}}%
\pgfpathlineto{\pgfqpoint{5.800000in}{31.982469in}}%
\pgfpathlineto{\pgfqpoint{5.800000in}{32.085306in}}%
\pgfpathlineto{\pgfqpoint{5.800000in}{32.085306in}}%
\pgfpathclose%
\pgfusepath{fill}%
\end{pgfscope}%
\begin{pgfscope}%
\pgfpathrectangle{\pgfqpoint{5.800000in}{0.720000in}}{\pgfqpoint{1.400000in}{4.620000in}}%
\pgfusepath{clip}%
\pgfsetbuttcap%
\pgfsetmiterjoin%
\definecolor{currentfill}{rgb}{0.121569,0.466667,0.705882}%
\pgfsetfillcolor{currentfill}%
\pgfsetlinewidth{0.000000pt}%
\definecolor{currentstroke}{rgb}{0.000000,0.000000,0.000000}%
\pgfsetstrokecolor{currentstroke}%
\pgfsetstrokeopacity{0.000000}%
\pgfsetdash{}{0pt}%
\pgfpathmoveto{\pgfqpoint{5.800000in}{32.085306in}}%
\pgfpathlineto{\pgfqpoint{5.800000in}{32.085306in}}%
\pgfpathlineto{\pgfqpoint{5.800000in}{32.188143in}}%
\pgfpathlineto{\pgfqpoint{5.800000in}{32.188143in}}%
\pgfpathclose%
\pgfusepath{fill}%
\end{pgfscope}%
\begin{pgfscope}%
\pgfpathrectangle{\pgfqpoint{5.800000in}{0.720000in}}{\pgfqpoint{1.400000in}{4.620000in}}%
\pgfusepath{clip}%
\pgfsetbuttcap%
\pgfsetmiterjoin%
\definecolor{currentfill}{rgb}{0.121569,0.466667,0.705882}%
\pgfsetfillcolor{currentfill}%
\pgfsetlinewidth{0.000000pt}%
\definecolor{currentstroke}{rgb}{0.000000,0.000000,0.000000}%
\pgfsetstrokecolor{currentstroke}%
\pgfsetstrokeopacity{0.000000}%
\pgfsetdash{}{0pt}%
\pgfpathmoveto{\pgfqpoint{5.800000in}{32.188143in}}%
\pgfpathlineto{\pgfqpoint{5.800000in}{32.188143in}}%
\pgfpathlineto{\pgfqpoint{5.800000in}{32.290980in}}%
\pgfpathlineto{\pgfqpoint{5.800000in}{32.290980in}}%
\pgfpathclose%
\pgfusepath{fill}%
\end{pgfscope}%
\begin{pgfscope}%
\pgfpathrectangle{\pgfqpoint{5.800000in}{0.720000in}}{\pgfqpoint{1.400000in}{4.620000in}}%
\pgfusepath{clip}%
\pgfsetbuttcap%
\pgfsetmiterjoin%
\definecolor{currentfill}{rgb}{0.121569,0.466667,0.705882}%
\pgfsetfillcolor{currentfill}%
\pgfsetlinewidth{0.000000pt}%
\definecolor{currentstroke}{rgb}{0.000000,0.000000,0.000000}%
\pgfsetstrokecolor{currentstroke}%
\pgfsetstrokeopacity{0.000000}%
\pgfsetdash{}{0pt}%
\pgfpathmoveto{\pgfqpoint{5.800000in}{32.290980in}}%
\pgfpathlineto{\pgfqpoint{5.800000in}{32.290980in}}%
\pgfpathlineto{\pgfqpoint{5.800000in}{32.393817in}}%
\pgfpathlineto{\pgfqpoint{5.800000in}{32.393817in}}%
\pgfpathclose%
\pgfusepath{fill}%
\end{pgfscope}%
\begin{pgfscope}%
\pgfpathrectangle{\pgfqpoint{5.800000in}{0.720000in}}{\pgfqpoint{1.400000in}{4.620000in}}%
\pgfusepath{clip}%
\pgfsetbuttcap%
\pgfsetmiterjoin%
\definecolor{currentfill}{rgb}{0.121569,0.466667,0.705882}%
\pgfsetfillcolor{currentfill}%
\pgfsetlinewidth{0.000000pt}%
\definecolor{currentstroke}{rgb}{0.000000,0.000000,0.000000}%
\pgfsetstrokecolor{currentstroke}%
\pgfsetstrokeopacity{0.000000}%
\pgfsetdash{}{0pt}%
\pgfpathmoveto{\pgfqpoint{5.800000in}{32.393817in}}%
\pgfpathlineto{\pgfqpoint{5.800000in}{32.393817in}}%
\pgfpathlineto{\pgfqpoint{5.800000in}{32.496654in}}%
\pgfpathlineto{\pgfqpoint{5.800000in}{32.496654in}}%
\pgfpathclose%
\pgfusepath{fill}%
\end{pgfscope}%
\begin{pgfscope}%
\pgfpathrectangle{\pgfqpoint{5.800000in}{0.720000in}}{\pgfqpoint{1.400000in}{4.620000in}}%
\pgfusepath{clip}%
\pgfsetbuttcap%
\pgfsetmiterjoin%
\definecolor{currentfill}{rgb}{0.121569,0.466667,0.705882}%
\pgfsetfillcolor{currentfill}%
\pgfsetlinewidth{0.000000pt}%
\definecolor{currentstroke}{rgb}{0.000000,0.000000,0.000000}%
\pgfsetstrokecolor{currentstroke}%
\pgfsetstrokeopacity{0.000000}%
\pgfsetdash{}{0pt}%
\pgfpathmoveto{\pgfqpoint{5.800000in}{32.496654in}}%
\pgfpathlineto{\pgfqpoint{5.800000in}{32.496654in}}%
\pgfpathlineto{\pgfqpoint{5.800000in}{32.599491in}}%
\pgfpathlineto{\pgfqpoint{5.800000in}{32.599491in}}%
\pgfpathclose%
\pgfusepath{fill}%
\end{pgfscope}%
\begin{pgfscope}%
\pgfpathrectangle{\pgfqpoint{5.800000in}{0.720000in}}{\pgfqpoint{1.400000in}{4.620000in}}%
\pgfusepath{clip}%
\pgfsetbuttcap%
\pgfsetmiterjoin%
\definecolor{currentfill}{rgb}{0.121569,0.466667,0.705882}%
\pgfsetfillcolor{currentfill}%
\pgfsetlinewidth{0.000000pt}%
\definecolor{currentstroke}{rgb}{0.000000,0.000000,0.000000}%
\pgfsetstrokecolor{currentstroke}%
\pgfsetstrokeopacity{0.000000}%
\pgfsetdash{}{0pt}%
\pgfpathmoveto{\pgfqpoint{5.800000in}{32.599491in}}%
\pgfpathlineto{\pgfqpoint{5.800000in}{32.599491in}}%
\pgfpathlineto{\pgfqpoint{5.800000in}{32.702328in}}%
\pgfpathlineto{\pgfqpoint{5.800000in}{32.702328in}}%
\pgfpathclose%
\pgfusepath{fill}%
\end{pgfscope}%
\begin{pgfscope}%
\pgfpathrectangle{\pgfqpoint{5.800000in}{0.720000in}}{\pgfqpoint{1.400000in}{4.620000in}}%
\pgfusepath{clip}%
\pgfsetbuttcap%
\pgfsetmiterjoin%
\definecolor{currentfill}{rgb}{0.121569,0.466667,0.705882}%
\pgfsetfillcolor{currentfill}%
\pgfsetlinewidth{0.000000pt}%
\definecolor{currentstroke}{rgb}{0.000000,0.000000,0.000000}%
\pgfsetstrokecolor{currentstroke}%
\pgfsetstrokeopacity{0.000000}%
\pgfsetdash{}{0pt}%
\pgfpathmoveto{\pgfqpoint{5.800000in}{32.702328in}}%
\pgfpathlineto{\pgfqpoint{5.800000in}{32.702328in}}%
\pgfpathlineto{\pgfqpoint{5.800000in}{32.805165in}}%
\pgfpathlineto{\pgfqpoint{5.800000in}{32.805165in}}%
\pgfpathclose%
\pgfusepath{fill}%
\end{pgfscope}%
\begin{pgfscope}%
\pgfpathrectangle{\pgfqpoint{5.800000in}{0.720000in}}{\pgfqpoint{1.400000in}{4.620000in}}%
\pgfusepath{clip}%
\pgfsetbuttcap%
\pgfsetmiterjoin%
\definecolor{currentfill}{rgb}{0.121569,0.466667,0.705882}%
\pgfsetfillcolor{currentfill}%
\pgfsetlinewidth{0.000000pt}%
\definecolor{currentstroke}{rgb}{0.000000,0.000000,0.000000}%
\pgfsetstrokecolor{currentstroke}%
\pgfsetstrokeopacity{0.000000}%
\pgfsetdash{}{0pt}%
\pgfpathmoveto{\pgfqpoint{5.800000in}{32.805165in}}%
\pgfpathlineto{\pgfqpoint{5.800000in}{32.805165in}}%
\pgfpathlineto{\pgfqpoint{5.800000in}{32.908002in}}%
\pgfpathlineto{\pgfqpoint{5.800000in}{32.908002in}}%
\pgfpathclose%
\pgfusepath{fill}%
\end{pgfscope}%
\begin{pgfscope}%
\pgfpathrectangle{\pgfqpoint{5.800000in}{0.720000in}}{\pgfqpoint{1.400000in}{4.620000in}}%
\pgfusepath{clip}%
\pgfsetbuttcap%
\pgfsetmiterjoin%
\definecolor{currentfill}{rgb}{0.121569,0.466667,0.705882}%
\pgfsetfillcolor{currentfill}%
\pgfsetlinewidth{0.000000pt}%
\definecolor{currentstroke}{rgb}{0.000000,0.000000,0.000000}%
\pgfsetstrokecolor{currentstroke}%
\pgfsetstrokeopacity{0.000000}%
\pgfsetdash{}{0pt}%
\pgfpathmoveto{\pgfqpoint{5.800000in}{32.908002in}}%
\pgfpathlineto{\pgfqpoint{5.800000in}{32.908002in}}%
\pgfpathlineto{\pgfqpoint{5.800000in}{33.010839in}}%
\pgfpathlineto{\pgfqpoint{5.800000in}{33.010839in}}%
\pgfpathclose%
\pgfusepath{fill}%
\end{pgfscope}%
\begin{pgfscope}%
\pgfpathrectangle{\pgfqpoint{5.800000in}{0.720000in}}{\pgfqpoint{1.400000in}{4.620000in}}%
\pgfusepath{clip}%
\pgfsetbuttcap%
\pgfsetmiterjoin%
\definecolor{currentfill}{rgb}{0.121569,0.466667,0.705882}%
\pgfsetfillcolor{currentfill}%
\pgfsetlinewidth{0.000000pt}%
\definecolor{currentstroke}{rgb}{0.000000,0.000000,0.000000}%
\pgfsetstrokecolor{currentstroke}%
\pgfsetstrokeopacity{0.000000}%
\pgfsetdash{}{0pt}%
\pgfpathmoveto{\pgfqpoint{5.800000in}{33.010839in}}%
\pgfpathlineto{\pgfqpoint{5.800000in}{33.010839in}}%
\pgfpathlineto{\pgfqpoint{5.800000in}{33.113676in}}%
\pgfpathlineto{\pgfqpoint{5.800000in}{33.113676in}}%
\pgfpathclose%
\pgfusepath{fill}%
\end{pgfscope}%
\begin{pgfscope}%
\pgfpathrectangle{\pgfqpoint{5.800000in}{0.720000in}}{\pgfqpoint{1.400000in}{4.620000in}}%
\pgfusepath{clip}%
\pgfsetbuttcap%
\pgfsetmiterjoin%
\definecolor{currentfill}{rgb}{0.121569,0.466667,0.705882}%
\pgfsetfillcolor{currentfill}%
\pgfsetlinewidth{0.000000pt}%
\definecolor{currentstroke}{rgb}{0.000000,0.000000,0.000000}%
\pgfsetstrokecolor{currentstroke}%
\pgfsetstrokeopacity{0.000000}%
\pgfsetdash{}{0pt}%
\pgfpathmoveto{\pgfqpoint{5.800000in}{33.113676in}}%
\pgfpathlineto{\pgfqpoint{5.800000in}{33.113676in}}%
\pgfpathlineto{\pgfqpoint{5.800000in}{33.216513in}}%
\pgfpathlineto{\pgfqpoint{5.800000in}{33.216513in}}%
\pgfpathclose%
\pgfusepath{fill}%
\end{pgfscope}%
\begin{pgfscope}%
\pgfpathrectangle{\pgfqpoint{5.800000in}{0.720000in}}{\pgfqpoint{1.400000in}{4.620000in}}%
\pgfusepath{clip}%
\pgfsetbuttcap%
\pgfsetmiterjoin%
\definecolor{currentfill}{rgb}{0.121569,0.466667,0.705882}%
\pgfsetfillcolor{currentfill}%
\pgfsetlinewidth{0.000000pt}%
\definecolor{currentstroke}{rgb}{0.000000,0.000000,0.000000}%
\pgfsetstrokecolor{currentstroke}%
\pgfsetstrokeopacity{0.000000}%
\pgfsetdash{}{0pt}%
\pgfpathmoveto{\pgfqpoint{5.800000in}{33.216513in}}%
\pgfpathlineto{\pgfqpoint{5.800000in}{33.216513in}}%
\pgfpathlineto{\pgfqpoint{5.800000in}{33.319350in}}%
\pgfpathlineto{\pgfqpoint{5.800000in}{33.319350in}}%
\pgfpathclose%
\pgfusepath{fill}%
\end{pgfscope}%
\begin{pgfscope}%
\pgfpathrectangle{\pgfqpoint{5.800000in}{0.720000in}}{\pgfqpoint{1.400000in}{4.620000in}}%
\pgfusepath{clip}%
\pgfsetbuttcap%
\pgfsetmiterjoin%
\definecolor{currentfill}{rgb}{0.121569,0.466667,0.705882}%
\pgfsetfillcolor{currentfill}%
\pgfsetlinewidth{0.000000pt}%
\definecolor{currentstroke}{rgb}{0.000000,0.000000,0.000000}%
\pgfsetstrokecolor{currentstroke}%
\pgfsetstrokeopacity{0.000000}%
\pgfsetdash{}{0pt}%
\pgfpathmoveto{\pgfqpoint{5.800000in}{33.319350in}}%
\pgfpathlineto{\pgfqpoint{5.800000in}{33.319350in}}%
\pgfpathlineto{\pgfqpoint{5.800000in}{33.422188in}}%
\pgfpathlineto{\pgfqpoint{5.800000in}{33.422188in}}%
\pgfpathclose%
\pgfusepath{fill}%
\end{pgfscope}%
\begin{pgfscope}%
\pgfpathrectangle{\pgfqpoint{5.800000in}{0.720000in}}{\pgfqpoint{1.400000in}{4.620000in}}%
\pgfusepath{clip}%
\pgfsetbuttcap%
\pgfsetmiterjoin%
\definecolor{currentfill}{rgb}{0.121569,0.466667,0.705882}%
\pgfsetfillcolor{currentfill}%
\pgfsetlinewidth{0.000000pt}%
\definecolor{currentstroke}{rgb}{0.000000,0.000000,0.000000}%
\pgfsetstrokecolor{currentstroke}%
\pgfsetstrokeopacity{0.000000}%
\pgfsetdash{}{0pt}%
\pgfpathmoveto{\pgfqpoint{5.800000in}{33.422188in}}%
\pgfpathlineto{\pgfqpoint{5.800000in}{33.422188in}}%
\pgfpathlineto{\pgfqpoint{5.800000in}{33.525025in}}%
\pgfpathlineto{\pgfqpoint{5.800000in}{33.525025in}}%
\pgfpathclose%
\pgfusepath{fill}%
\end{pgfscope}%
\begin{pgfscope}%
\pgfpathrectangle{\pgfqpoint{5.800000in}{0.720000in}}{\pgfqpoint{1.400000in}{4.620000in}}%
\pgfusepath{clip}%
\pgfsetbuttcap%
\pgfsetmiterjoin%
\definecolor{currentfill}{rgb}{0.121569,0.466667,0.705882}%
\pgfsetfillcolor{currentfill}%
\pgfsetlinewidth{0.000000pt}%
\definecolor{currentstroke}{rgb}{0.000000,0.000000,0.000000}%
\pgfsetstrokecolor{currentstroke}%
\pgfsetstrokeopacity{0.000000}%
\pgfsetdash{}{0pt}%
\pgfpathmoveto{\pgfqpoint{5.800000in}{33.525025in}}%
\pgfpathlineto{\pgfqpoint{5.800000in}{33.525025in}}%
\pgfpathlineto{\pgfqpoint{5.800000in}{33.627862in}}%
\pgfpathlineto{\pgfqpoint{5.800000in}{33.627862in}}%
\pgfpathclose%
\pgfusepath{fill}%
\end{pgfscope}%
\begin{pgfscope}%
\pgfpathrectangle{\pgfqpoint{5.800000in}{0.720000in}}{\pgfqpoint{1.400000in}{4.620000in}}%
\pgfusepath{clip}%
\pgfsetbuttcap%
\pgfsetmiterjoin%
\definecolor{currentfill}{rgb}{0.121569,0.466667,0.705882}%
\pgfsetfillcolor{currentfill}%
\pgfsetlinewidth{0.000000pt}%
\definecolor{currentstroke}{rgb}{0.000000,0.000000,0.000000}%
\pgfsetstrokecolor{currentstroke}%
\pgfsetstrokeopacity{0.000000}%
\pgfsetdash{}{0pt}%
\pgfpathmoveto{\pgfqpoint{5.800000in}{33.627862in}}%
\pgfpathlineto{\pgfqpoint{5.800000in}{33.627862in}}%
\pgfpathlineto{\pgfqpoint{5.800000in}{33.730699in}}%
\pgfpathlineto{\pgfqpoint{5.800000in}{33.730699in}}%
\pgfpathclose%
\pgfusepath{fill}%
\end{pgfscope}%
\begin{pgfscope}%
\pgfpathrectangle{\pgfqpoint{5.800000in}{0.720000in}}{\pgfqpoint{1.400000in}{4.620000in}}%
\pgfusepath{clip}%
\pgfsetbuttcap%
\pgfsetmiterjoin%
\definecolor{currentfill}{rgb}{0.121569,0.466667,0.705882}%
\pgfsetfillcolor{currentfill}%
\pgfsetlinewidth{0.000000pt}%
\definecolor{currentstroke}{rgb}{0.000000,0.000000,0.000000}%
\pgfsetstrokecolor{currentstroke}%
\pgfsetstrokeopacity{0.000000}%
\pgfsetdash{}{0pt}%
\pgfpathmoveto{\pgfqpoint{5.800000in}{33.730699in}}%
\pgfpathlineto{\pgfqpoint{5.800000in}{33.730699in}}%
\pgfpathlineto{\pgfqpoint{5.800000in}{33.833536in}}%
\pgfpathlineto{\pgfqpoint{5.800000in}{33.833536in}}%
\pgfpathclose%
\pgfusepath{fill}%
\end{pgfscope}%
\begin{pgfscope}%
\pgfpathrectangle{\pgfqpoint{5.800000in}{0.720000in}}{\pgfqpoint{1.400000in}{4.620000in}}%
\pgfusepath{clip}%
\pgfsetbuttcap%
\pgfsetmiterjoin%
\definecolor{currentfill}{rgb}{0.121569,0.466667,0.705882}%
\pgfsetfillcolor{currentfill}%
\pgfsetlinewidth{0.000000pt}%
\definecolor{currentstroke}{rgb}{0.000000,0.000000,0.000000}%
\pgfsetstrokecolor{currentstroke}%
\pgfsetstrokeopacity{0.000000}%
\pgfsetdash{}{0pt}%
\pgfpathmoveto{\pgfqpoint{5.800000in}{33.833536in}}%
\pgfpathlineto{\pgfqpoint{5.800000in}{33.833536in}}%
\pgfpathlineto{\pgfqpoint{5.800000in}{33.936373in}}%
\pgfpathlineto{\pgfqpoint{5.800000in}{33.936373in}}%
\pgfpathclose%
\pgfusepath{fill}%
\end{pgfscope}%
\begin{pgfscope}%
\pgfpathrectangle{\pgfqpoint{5.800000in}{0.720000in}}{\pgfqpoint{1.400000in}{4.620000in}}%
\pgfusepath{clip}%
\pgfsetbuttcap%
\pgfsetmiterjoin%
\definecolor{currentfill}{rgb}{0.121569,0.466667,0.705882}%
\pgfsetfillcolor{currentfill}%
\pgfsetlinewidth{0.000000pt}%
\definecolor{currentstroke}{rgb}{0.000000,0.000000,0.000000}%
\pgfsetstrokecolor{currentstroke}%
\pgfsetstrokeopacity{0.000000}%
\pgfsetdash{}{0pt}%
\pgfpathmoveto{\pgfqpoint{5.800000in}{33.936373in}}%
\pgfpathlineto{\pgfqpoint{5.800000in}{33.936373in}}%
\pgfpathlineto{\pgfqpoint{5.800000in}{34.039210in}}%
\pgfpathlineto{\pgfqpoint{5.800000in}{34.039210in}}%
\pgfpathclose%
\pgfusepath{fill}%
\end{pgfscope}%
\begin{pgfscope}%
\pgfpathrectangle{\pgfqpoint{5.800000in}{0.720000in}}{\pgfqpoint{1.400000in}{4.620000in}}%
\pgfusepath{clip}%
\pgfsetbuttcap%
\pgfsetmiterjoin%
\definecolor{currentfill}{rgb}{0.121569,0.466667,0.705882}%
\pgfsetfillcolor{currentfill}%
\pgfsetlinewidth{0.000000pt}%
\definecolor{currentstroke}{rgb}{0.000000,0.000000,0.000000}%
\pgfsetstrokecolor{currentstroke}%
\pgfsetstrokeopacity{0.000000}%
\pgfsetdash{}{0pt}%
\pgfpathmoveto{\pgfqpoint{5.800000in}{34.039210in}}%
\pgfpathlineto{\pgfqpoint{5.800000in}{34.039210in}}%
\pgfpathlineto{\pgfqpoint{5.800000in}{34.142047in}}%
\pgfpathlineto{\pgfqpoint{5.800000in}{34.142047in}}%
\pgfpathclose%
\pgfusepath{fill}%
\end{pgfscope}%
\begin{pgfscope}%
\pgfpathrectangle{\pgfqpoint{5.800000in}{0.720000in}}{\pgfqpoint{1.400000in}{4.620000in}}%
\pgfusepath{clip}%
\pgfsetbuttcap%
\pgfsetmiterjoin%
\definecolor{currentfill}{rgb}{0.121569,0.466667,0.705882}%
\pgfsetfillcolor{currentfill}%
\pgfsetlinewidth{0.000000pt}%
\definecolor{currentstroke}{rgb}{0.000000,0.000000,0.000000}%
\pgfsetstrokecolor{currentstroke}%
\pgfsetstrokeopacity{0.000000}%
\pgfsetdash{}{0pt}%
\pgfpathmoveto{\pgfqpoint{5.800000in}{34.142047in}}%
\pgfpathlineto{\pgfqpoint{5.800000in}{34.142047in}}%
\pgfpathlineto{\pgfqpoint{5.800000in}{34.244884in}}%
\pgfpathlineto{\pgfqpoint{5.800000in}{34.244884in}}%
\pgfpathclose%
\pgfusepath{fill}%
\end{pgfscope}%
\begin{pgfscope}%
\pgfpathrectangle{\pgfqpoint{5.800000in}{0.720000in}}{\pgfqpoint{1.400000in}{4.620000in}}%
\pgfusepath{clip}%
\pgfsetbuttcap%
\pgfsetmiterjoin%
\definecolor{currentfill}{rgb}{0.121569,0.466667,0.705882}%
\pgfsetfillcolor{currentfill}%
\pgfsetlinewidth{0.000000pt}%
\definecolor{currentstroke}{rgb}{0.000000,0.000000,0.000000}%
\pgfsetstrokecolor{currentstroke}%
\pgfsetstrokeopacity{0.000000}%
\pgfsetdash{}{0pt}%
\pgfpathmoveto{\pgfqpoint{5.800000in}{34.244884in}}%
\pgfpathlineto{\pgfqpoint{5.800000in}{34.244884in}}%
\pgfpathlineto{\pgfqpoint{5.800000in}{34.347721in}}%
\pgfpathlineto{\pgfqpoint{5.800000in}{34.347721in}}%
\pgfpathclose%
\pgfusepath{fill}%
\end{pgfscope}%
\begin{pgfscope}%
\pgfpathrectangle{\pgfqpoint{5.800000in}{0.720000in}}{\pgfqpoint{1.400000in}{4.620000in}}%
\pgfusepath{clip}%
\pgfsetbuttcap%
\pgfsetmiterjoin%
\definecolor{currentfill}{rgb}{0.121569,0.466667,0.705882}%
\pgfsetfillcolor{currentfill}%
\pgfsetlinewidth{0.000000pt}%
\definecolor{currentstroke}{rgb}{0.000000,0.000000,0.000000}%
\pgfsetstrokecolor{currentstroke}%
\pgfsetstrokeopacity{0.000000}%
\pgfsetdash{}{0pt}%
\pgfpathmoveto{\pgfqpoint{5.800000in}{34.347721in}}%
\pgfpathlineto{\pgfqpoint{5.800000in}{34.347721in}}%
\pgfpathlineto{\pgfqpoint{5.800000in}{34.450558in}}%
\pgfpathlineto{\pgfqpoint{5.800000in}{34.450558in}}%
\pgfpathclose%
\pgfusepath{fill}%
\end{pgfscope}%
\begin{pgfscope}%
\pgfpathrectangle{\pgfqpoint{5.800000in}{0.720000in}}{\pgfqpoint{1.400000in}{4.620000in}}%
\pgfusepath{clip}%
\pgfsetbuttcap%
\pgfsetmiterjoin%
\definecolor{currentfill}{rgb}{0.121569,0.466667,0.705882}%
\pgfsetfillcolor{currentfill}%
\pgfsetlinewidth{0.000000pt}%
\definecolor{currentstroke}{rgb}{0.000000,0.000000,0.000000}%
\pgfsetstrokecolor{currentstroke}%
\pgfsetstrokeopacity{0.000000}%
\pgfsetdash{}{0pt}%
\pgfpathmoveto{\pgfqpoint{5.800000in}{34.450558in}}%
\pgfpathlineto{\pgfqpoint{5.800000in}{34.450558in}}%
\pgfpathlineto{\pgfqpoint{5.800000in}{34.553395in}}%
\pgfpathlineto{\pgfqpoint{5.800000in}{34.553395in}}%
\pgfpathclose%
\pgfusepath{fill}%
\end{pgfscope}%
\begin{pgfscope}%
\pgfpathrectangle{\pgfqpoint{5.800000in}{0.720000in}}{\pgfqpoint{1.400000in}{4.620000in}}%
\pgfusepath{clip}%
\pgfsetbuttcap%
\pgfsetmiterjoin%
\definecolor{currentfill}{rgb}{0.121569,0.466667,0.705882}%
\pgfsetfillcolor{currentfill}%
\pgfsetlinewidth{0.000000pt}%
\definecolor{currentstroke}{rgb}{0.000000,0.000000,0.000000}%
\pgfsetstrokecolor{currentstroke}%
\pgfsetstrokeopacity{0.000000}%
\pgfsetdash{}{0pt}%
\pgfpathmoveto{\pgfqpoint{5.800000in}{34.553395in}}%
\pgfpathlineto{\pgfqpoint{5.800000in}{34.553395in}}%
\pgfpathlineto{\pgfqpoint{5.800000in}{34.656232in}}%
\pgfpathlineto{\pgfqpoint{5.800000in}{34.656232in}}%
\pgfpathclose%
\pgfusepath{fill}%
\end{pgfscope}%
\begin{pgfscope}%
\pgfpathrectangle{\pgfqpoint{5.800000in}{0.720000in}}{\pgfqpoint{1.400000in}{4.620000in}}%
\pgfusepath{clip}%
\pgfsetbuttcap%
\pgfsetmiterjoin%
\definecolor{currentfill}{rgb}{0.121569,0.466667,0.705882}%
\pgfsetfillcolor{currentfill}%
\pgfsetlinewidth{0.000000pt}%
\definecolor{currentstroke}{rgb}{0.000000,0.000000,0.000000}%
\pgfsetstrokecolor{currentstroke}%
\pgfsetstrokeopacity{0.000000}%
\pgfsetdash{}{0pt}%
\pgfpathmoveto{\pgfqpoint{5.800000in}{34.656232in}}%
\pgfpathlineto{\pgfqpoint{5.800000in}{34.656232in}}%
\pgfpathlineto{\pgfqpoint{5.800000in}{34.759069in}}%
\pgfpathlineto{\pgfqpoint{5.800000in}{34.759069in}}%
\pgfpathclose%
\pgfusepath{fill}%
\end{pgfscope}%
\begin{pgfscope}%
\pgfpathrectangle{\pgfqpoint{5.800000in}{0.720000in}}{\pgfqpoint{1.400000in}{4.620000in}}%
\pgfusepath{clip}%
\pgfsetbuttcap%
\pgfsetmiterjoin%
\definecolor{currentfill}{rgb}{0.121569,0.466667,0.705882}%
\pgfsetfillcolor{currentfill}%
\pgfsetlinewidth{0.000000pt}%
\definecolor{currentstroke}{rgb}{0.000000,0.000000,0.000000}%
\pgfsetstrokecolor{currentstroke}%
\pgfsetstrokeopacity{0.000000}%
\pgfsetdash{}{0pt}%
\pgfpathmoveto{\pgfqpoint{5.800000in}{34.759069in}}%
\pgfpathlineto{\pgfqpoint{5.800000in}{34.759069in}}%
\pgfpathlineto{\pgfqpoint{5.800000in}{34.861907in}}%
\pgfpathlineto{\pgfqpoint{5.800000in}{34.861907in}}%
\pgfpathclose%
\pgfusepath{fill}%
\end{pgfscope}%
\begin{pgfscope}%
\pgfpathrectangle{\pgfqpoint{5.800000in}{0.720000in}}{\pgfqpoint{1.400000in}{4.620000in}}%
\pgfusepath{clip}%
\pgfsetbuttcap%
\pgfsetmiterjoin%
\definecolor{currentfill}{rgb}{0.121569,0.466667,0.705882}%
\pgfsetfillcolor{currentfill}%
\pgfsetlinewidth{0.000000pt}%
\definecolor{currentstroke}{rgb}{0.000000,0.000000,0.000000}%
\pgfsetstrokecolor{currentstroke}%
\pgfsetstrokeopacity{0.000000}%
\pgfsetdash{}{0pt}%
\pgfpathmoveto{\pgfqpoint{5.800000in}{34.861907in}}%
\pgfpathlineto{\pgfqpoint{5.800000in}{34.861907in}}%
\pgfpathlineto{\pgfqpoint{5.800000in}{34.964744in}}%
\pgfpathlineto{\pgfqpoint{5.800000in}{34.964744in}}%
\pgfpathclose%
\pgfusepath{fill}%
\end{pgfscope}%
\begin{pgfscope}%
\pgfpathrectangle{\pgfqpoint{5.800000in}{0.720000in}}{\pgfqpoint{1.400000in}{4.620000in}}%
\pgfusepath{clip}%
\pgfsetbuttcap%
\pgfsetmiterjoin%
\definecolor{currentfill}{rgb}{0.121569,0.466667,0.705882}%
\pgfsetfillcolor{currentfill}%
\pgfsetlinewidth{0.000000pt}%
\definecolor{currentstroke}{rgb}{0.000000,0.000000,0.000000}%
\pgfsetstrokecolor{currentstroke}%
\pgfsetstrokeopacity{0.000000}%
\pgfsetdash{}{0pt}%
\pgfpathmoveto{\pgfqpoint{5.800000in}{34.964744in}}%
\pgfpathlineto{\pgfqpoint{5.800000in}{34.964744in}}%
\pgfpathlineto{\pgfqpoint{5.800000in}{35.067581in}}%
\pgfpathlineto{\pgfqpoint{5.800000in}{35.067581in}}%
\pgfpathclose%
\pgfusepath{fill}%
\end{pgfscope}%
\begin{pgfscope}%
\pgfpathrectangle{\pgfqpoint{5.800000in}{0.720000in}}{\pgfqpoint{1.400000in}{4.620000in}}%
\pgfusepath{clip}%
\pgfsetbuttcap%
\pgfsetmiterjoin%
\definecolor{currentfill}{rgb}{0.121569,0.466667,0.705882}%
\pgfsetfillcolor{currentfill}%
\pgfsetlinewidth{0.000000pt}%
\definecolor{currentstroke}{rgb}{0.000000,0.000000,0.000000}%
\pgfsetstrokecolor{currentstroke}%
\pgfsetstrokeopacity{0.000000}%
\pgfsetdash{}{0pt}%
\pgfpathmoveto{\pgfqpoint{5.800000in}{35.067581in}}%
\pgfpathlineto{\pgfqpoint{5.800000in}{35.067581in}}%
\pgfpathlineto{\pgfqpoint{5.800000in}{35.170418in}}%
\pgfpathlineto{\pgfqpoint{5.800000in}{35.170418in}}%
\pgfpathclose%
\pgfusepath{fill}%
\end{pgfscope}%
\begin{pgfscope}%
\pgfpathrectangle{\pgfqpoint{5.800000in}{0.720000in}}{\pgfqpoint{1.400000in}{4.620000in}}%
\pgfusepath{clip}%
\pgfsetbuttcap%
\pgfsetmiterjoin%
\definecolor{currentfill}{rgb}{0.121569,0.466667,0.705882}%
\pgfsetfillcolor{currentfill}%
\pgfsetlinewidth{0.000000pt}%
\definecolor{currentstroke}{rgb}{0.000000,0.000000,0.000000}%
\pgfsetstrokecolor{currentstroke}%
\pgfsetstrokeopacity{0.000000}%
\pgfsetdash{}{0pt}%
\pgfpathmoveto{\pgfqpoint{5.800000in}{35.170418in}}%
\pgfpathlineto{\pgfqpoint{5.800000in}{35.170418in}}%
\pgfpathlineto{\pgfqpoint{5.800000in}{35.273255in}}%
\pgfpathlineto{\pgfqpoint{5.800000in}{35.273255in}}%
\pgfpathclose%
\pgfusepath{fill}%
\end{pgfscope}%
\begin{pgfscope}%
\pgfpathrectangle{\pgfqpoint{5.800000in}{0.720000in}}{\pgfqpoint{1.400000in}{4.620000in}}%
\pgfusepath{clip}%
\pgfsetbuttcap%
\pgfsetmiterjoin%
\definecolor{currentfill}{rgb}{0.121569,0.466667,0.705882}%
\pgfsetfillcolor{currentfill}%
\pgfsetlinewidth{0.000000pt}%
\definecolor{currentstroke}{rgb}{0.000000,0.000000,0.000000}%
\pgfsetstrokecolor{currentstroke}%
\pgfsetstrokeopacity{0.000000}%
\pgfsetdash{}{0pt}%
\pgfpathmoveto{\pgfqpoint{5.800000in}{35.273255in}}%
\pgfpathlineto{\pgfqpoint{5.800000in}{35.273255in}}%
\pgfpathlineto{\pgfqpoint{5.800000in}{35.376092in}}%
\pgfpathlineto{\pgfqpoint{5.800000in}{35.376092in}}%
\pgfpathclose%
\pgfusepath{fill}%
\end{pgfscope}%
\begin{pgfscope}%
\pgfpathrectangle{\pgfqpoint{5.800000in}{0.720000in}}{\pgfqpoint{1.400000in}{4.620000in}}%
\pgfusepath{clip}%
\pgfsetbuttcap%
\pgfsetmiterjoin%
\definecolor{currentfill}{rgb}{0.121569,0.466667,0.705882}%
\pgfsetfillcolor{currentfill}%
\pgfsetlinewidth{0.000000pt}%
\definecolor{currentstroke}{rgb}{0.000000,0.000000,0.000000}%
\pgfsetstrokecolor{currentstroke}%
\pgfsetstrokeopacity{0.000000}%
\pgfsetdash{}{0pt}%
\pgfpathmoveto{\pgfqpoint{5.800000in}{35.376092in}}%
\pgfpathlineto{\pgfqpoint{5.800000in}{35.376092in}}%
\pgfpathlineto{\pgfqpoint{5.800000in}{35.478929in}}%
\pgfpathlineto{\pgfqpoint{5.800000in}{35.478929in}}%
\pgfpathclose%
\pgfusepath{fill}%
\end{pgfscope}%
\begin{pgfscope}%
\pgfpathrectangle{\pgfqpoint{5.800000in}{0.720000in}}{\pgfqpoint{1.400000in}{4.620000in}}%
\pgfusepath{clip}%
\pgfsetbuttcap%
\pgfsetmiterjoin%
\definecolor{currentfill}{rgb}{0.121569,0.466667,0.705882}%
\pgfsetfillcolor{currentfill}%
\pgfsetlinewidth{0.000000pt}%
\definecolor{currentstroke}{rgb}{0.000000,0.000000,0.000000}%
\pgfsetstrokecolor{currentstroke}%
\pgfsetstrokeopacity{0.000000}%
\pgfsetdash{}{0pt}%
\pgfpathmoveto{\pgfqpoint{5.800000in}{35.478929in}}%
\pgfpathlineto{\pgfqpoint{5.800000in}{35.478929in}}%
\pgfpathlineto{\pgfqpoint{5.800000in}{35.581766in}}%
\pgfpathlineto{\pgfqpoint{5.800000in}{35.581766in}}%
\pgfpathclose%
\pgfusepath{fill}%
\end{pgfscope}%
\begin{pgfscope}%
\pgfpathrectangle{\pgfqpoint{5.800000in}{0.720000in}}{\pgfqpoint{1.400000in}{4.620000in}}%
\pgfusepath{clip}%
\pgfsetbuttcap%
\pgfsetmiterjoin%
\definecolor{currentfill}{rgb}{0.121569,0.466667,0.705882}%
\pgfsetfillcolor{currentfill}%
\pgfsetlinewidth{0.000000pt}%
\definecolor{currentstroke}{rgb}{0.000000,0.000000,0.000000}%
\pgfsetstrokecolor{currentstroke}%
\pgfsetstrokeopacity{0.000000}%
\pgfsetdash{}{0pt}%
\pgfpathmoveto{\pgfqpoint{5.800000in}{35.581766in}}%
\pgfpathlineto{\pgfqpoint{5.800000in}{35.581766in}}%
\pgfpathlineto{\pgfqpoint{5.800000in}{35.684603in}}%
\pgfpathlineto{\pgfqpoint{5.800000in}{35.684603in}}%
\pgfpathclose%
\pgfusepath{fill}%
\end{pgfscope}%
\begin{pgfscope}%
\pgfpathrectangle{\pgfqpoint{5.800000in}{0.720000in}}{\pgfqpoint{1.400000in}{4.620000in}}%
\pgfusepath{clip}%
\pgfsetbuttcap%
\pgfsetmiterjoin%
\definecolor{currentfill}{rgb}{0.121569,0.466667,0.705882}%
\pgfsetfillcolor{currentfill}%
\pgfsetlinewidth{0.000000pt}%
\definecolor{currentstroke}{rgb}{0.000000,0.000000,0.000000}%
\pgfsetstrokecolor{currentstroke}%
\pgfsetstrokeopacity{0.000000}%
\pgfsetdash{}{0pt}%
\pgfpathmoveto{\pgfqpoint{5.800000in}{35.684603in}}%
\pgfpathlineto{\pgfqpoint{5.800000in}{35.684603in}}%
\pgfpathlineto{\pgfqpoint{5.800000in}{35.787440in}}%
\pgfpathlineto{\pgfqpoint{5.800000in}{35.787440in}}%
\pgfpathclose%
\pgfusepath{fill}%
\end{pgfscope}%
\begin{pgfscope}%
\pgfpathrectangle{\pgfqpoint{5.800000in}{0.720000in}}{\pgfqpoint{1.400000in}{4.620000in}}%
\pgfusepath{clip}%
\pgfsetbuttcap%
\pgfsetmiterjoin%
\definecolor{currentfill}{rgb}{0.121569,0.466667,0.705882}%
\pgfsetfillcolor{currentfill}%
\pgfsetlinewidth{0.000000pt}%
\definecolor{currentstroke}{rgb}{0.000000,0.000000,0.000000}%
\pgfsetstrokecolor{currentstroke}%
\pgfsetstrokeopacity{0.000000}%
\pgfsetdash{}{0pt}%
\pgfpathmoveto{\pgfqpoint{5.800000in}{35.787440in}}%
\pgfpathlineto{\pgfqpoint{5.800000in}{35.787440in}}%
\pgfpathlineto{\pgfqpoint{5.800000in}{35.890277in}}%
\pgfpathlineto{\pgfqpoint{5.800000in}{35.890277in}}%
\pgfpathclose%
\pgfusepath{fill}%
\end{pgfscope}%
\begin{pgfscope}%
\pgfpathrectangle{\pgfqpoint{5.800000in}{0.720000in}}{\pgfqpoint{1.400000in}{4.620000in}}%
\pgfusepath{clip}%
\pgfsetbuttcap%
\pgfsetmiterjoin%
\definecolor{currentfill}{rgb}{0.121569,0.466667,0.705882}%
\pgfsetfillcolor{currentfill}%
\pgfsetlinewidth{0.000000pt}%
\definecolor{currentstroke}{rgb}{0.000000,0.000000,0.000000}%
\pgfsetstrokecolor{currentstroke}%
\pgfsetstrokeopacity{0.000000}%
\pgfsetdash{}{0pt}%
\pgfpathmoveto{\pgfqpoint{5.800000in}{35.890277in}}%
\pgfpathlineto{\pgfqpoint{5.800000in}{35.890277in}}%
\pgfpathlineto{\pgfqpoint{5.800000in}{35.993114in}}%
\pgfpathlineto{\pgfqpoint{5.800000in}{35.993114in}}%
\pgfpathclose%
\pgfusepath{fill}%
\end{pgfscope}%
\begin{pgfscope}%
\pgfpathrectangle{\pgfqpoint{5.800000in}{0.720000in}}{\pgfqpoint{1.400000in}{4.620000in}}%
\pgfusepath{clip}%
\pgfsetbuttcap%
\pgfsetmiterjoin%
\definecolor{currentfill}{rgb}{0.121569,0.466667,0.705882}%
\pgfsetfillcolor{currentfill}%
\pgfsetlinewidth{0.000000pt}%
\definecolor{currentstroke}{rgb}{0.000000,0.000000,0.000000}%
\pgfsetstrokecolor{currentstroke}%
\pgfsetstrokeopacity{0.000000}%
\pgfsetdash{}{0pt}%
\pgfpathmoveto{\pgfqpoint{5.800000in}{35.993114in}}%
\pgfpathlineto{\pgfqpoint{5.800000in}{35.993114in}}%
\pgfpathlineto{\pgfqpoint{5.800000in}{36.095951in}}%
\pgfpathlineto{\pgfqpoint{5.800000in}{36.095951in}}%
\pgfpathclose%
\pgfusepath{fill}%
\end{pgfscope}%
\begin{pgfscope}%
\pgfpathrectangle{\pgfqpoint{5.800000in}{0.720000in}}{\pgfqpoint{1.400000in}{4.620000in}}%
\pgfusepath{clip}%
\pgfsetbuttcap%
\pgfsetmiterjoin%
\definecolor{currentfill}{rgb}{0.121569,0.466667,0.705882}%
\pgfsetfillcolor{currentfill}%
\pgfsetlinewidth{0.000000pt}%
\definecolor{currentstroke}{rgb}{0.000000,0.000000,0.000000}%
\pgfsetstrokecolor{currentstroke}%
\pgfsetstrokeopacity{0.000000}%
\pgfsetdash{}{0pt}%
\pgfpathmoveto{\pgfqpoint{5.800000in}{36.095951in}}%
\pgfpathlineto{\pgfqpoint{5.800000in}{36.095951in}}%
\pgfpathlineto{\pgfqpoint{5.800000in}{36.198788in}}%
\pgfpathlineto{\pgfqpoint{5.800000in}{36.198788in}}%
\pgfpathclose%
\pgfusepath{fill}%
\end{pgfscope}%
\begin{pgfscope}%
\pgfpathrectangle{\pgfqpoint{5.800000in}{0.720000in}}{\pgfqpoint{1.400000in}{4.620000in}}%
\pgfusepath{clip}%
\pgfsetbuttcap%
\pgfsetmiterjoin%
\definecolor{currentfill}{rgb}{0.121569,0.466667,0.705882}%
\pgfsetfillcolor{currentfill}%
\pgfsetlinewidth{0.000000pt}%
\definecolor{currentstroke}{rgb}{0.000000,0.000000,0.000000}%
\pgfsetstrokecolor{currentstroke}%
\pgfsetstrokeopacity{0.000000}%
\pgfsetdash{}{0pt}%
\pgfpathmoveto{\pgfqpoint{5.800000in}{36.198788in}}%
\pgfpathlineto{\pgfqpoint{5.800000in}{36.198788in}}%
\pgfpathlineto{\pgfqpoint{5.800000in}{36.301625in}}%
\pgfpathlineto{\pgfqpoint{5.800000in}{36.301625in}}%
\pgfpathclose%
\pgfusepath{fill}%
\end{pgfscope}%
\begin{pgfscope}%
\pgfpathrectangle{\pgfqpoint{5.800000in}{0.720000in}}{\pgfqpoint{1.400000in}{4.620000in}}%
\pgfusepath{clip}%
\pgfsetbuttcap%
\pgfsetmiterjoin%
\definecolor{currentfill}{rgb}{0.121569,0.466667,0.705882}%
\pgfsetfillcolor{currentfill}%
\pgfsetlinewidth{0.000000pt}%
\definecolor{currentstroke}{rgb}{0.000000,0.000000,0.000000}%
\pgfsetstrokecolor{currentstroke}%
\pgfsetstrokeopacity{0.000000}%
\pgfsetdash{}{0pt}%
\pgfpathmoveto{\pgfqpoint{5.800000in}{36.301625in}}%
\pgfpathlineto{\pgfqpoint{5.800000in}{36.301625in}}%
\pgfpathlineto{\pgfqpoint{5.800000in}{36.404463in}}%
\pgfpathlineto{\pgfqpoint{5.800000in}{36.404463in}}%
\pgfpathclose%
\pgfusepath{fill}%
\end{pgfscope}%
\begin{pgfscope}%
\pgfpathrectangle{\pgfqpoint{5.800000in}{0.720000in}}{\pgfqpoint{1.400000in}{4.620000in}}%
\pgfusepath{clip}%
\pgfsetbuttcap%
\pgfsetmiterjoin%
\definecolor{currentfill}{rgb}{0.121569,0.466667,0.705882}%
\pgfsetfillcolor{currentfill}%
\pgfsetlinewidth{0.000000pt}%
\definecolor{currentstroke}{rgb}{0.000000,0.000000,0.000000}%
\pgfsetstrokecolor{currentstroke}%
\pgfsetstrokeopacity{0.000000}%
\pgfsetdash{}{0pt}%
\pgfpathmoveto{\pgfqpoint{5.800000in}{36.404463in}}%
\pgfpathlineto{\pgfqpoint{5.800000in}{36.404463in}}%
\pgfpathlineto{\pgfqpoint{5.800000in}{36.507300in}}%
\pgfpathlineto{\pgfqpoint{5.800000in}{36.507300in}}%
\pgfpathclose%
\pgfusepath{fill}%
\end{pgfscope}%
\begin{pgfscope}%
\pgfpathrectangle{\pgfqpoint{5.800000in}{0.720000in}}{\pgfqpoint{1.400000in}{4.620000in}}%
\pgfusepath{clip}%
\pgfsetbuttcap%
\pgfsetmiterjoin%
\definecolor{currentfill}{rgb}{0.121569,0.466667,0.705882}%
\pgfsetfillcolor{currentfill}%
\pgfsetlinewidth{0.000000pt}%
\definecolor{currentstroke}{rgb}{0.000000,0.000000,0.000000}%
\pgfsetstrokecolor{currentstroke}%
\pgfsetstrokeopacity{0.000000}%
\pgfsetdash{}{0pt}%
\pgfpathmoveto{\pgfqpoint{5.800000in}{36.507300in}}%
\pgfpathlineto{\pgfqpoint{5.800000in}{36.507300in}}%
\pgfpathlineto{\pgfqpoint{5.800000in}{36.610137in}}%
\pgfpathlineto{\pgfqpoint{5.800000in}{36.610137in}}%
\pgfpathclose%
\pgfusepath{fill}%
\end{pgfscope}%
\begin{pgfscope}%
\pgfpathrectangle{\pgfqpoint{5.800000in}{0.720000in}}{\pgfqpoint{1.400000in}{4.620000in}}%
\pgfusepath{clip}%
\pgfsetbuttcap%
\pgfsetmiterjoin%
\definecolor{currentfill}{rgb}{0.121569,0.466667,0.705882}%
\pgfsetfillcolor{currentfill}%
\pgfsetlinewidth{0.000000pt}%
\definecolor{currentstroke}{rgb}{0.000000,0.000000,0.000000}%
\pgfsetstrokecolor{currentstroke}%
\pgfsetstrokeopacity{0.000000}%
\pgfsetdash{}{0pt}%
\pgfpathmoveto{\pgfqpoint{5.800000in}{36.610137in}}%
\pgfpathlineto{\pgfqpoint{5.800000in}{36.610137in}}%
\pgfpathlineto{\pgfqpoint{5.800000in}{36.712974in}}%
\pgfpathlineto{\pgfqpoint{5.800000in}{36.712974in}}%
\pgfpathclose%
\pgfusepath{fill}%
\end{pgfscope}%
\begin{pgfscope}%
\pgfpathrectangle{\pgfqpoint{5.800000in}{0.720000in}}{\pgfqpoint{1.400000in}{4.620000in}}%
\pgfusepath{clip}%
\pgfsetbuttcap%
\pgfsetmiterjoin%
\definecolor{currentfill}{rgb}{0.121569,0.466667,0.705882}%
\pgfsetfillcolor{currentfill}%
\pgfsetlinewidth{0.000000pt}%
\definecolor{currentstroke}{rgb}{0.000000,0.000000,0.000000}%
\pgfsetstrokecolor{currentstroke}%
\pgfsetstrokeopacity{0.000000}%
\pgfsetdash{}{0pt}%
\pgfpathmoveto{\pgfqpoint{5.800000in}{36.712974in}}%
\pgfpathlineto{\pgfqpoint{5.800000in}{36.712974in}}%
\pgfpathlineto{\pgfqpoint{5.800000in}{36.815811in}}%
\pgfpathlineto{\pgfqpoint{5.800000in}{36.815811in}}%
\pgfpathclose%
\pgfusepath{fill}%
\end{pgfscope}%
\begin{pgfscope}%
\pgfpathrectangle{\pgfqpoint{5.800000in}{0.720000in}}{\pgfqpoint{1.400000in}{4.620000in}}%
\pgfusepath{clip}%
\pgfsetbuttcap%
\pgfsetmiterjoin%
\definecolor{currentfill}{rgb}{0.121569,0.466667,0.705882}%
\pgfsetfillcolor{currentfill}%
\pgfsetlinewidth{0.000000pt}%
\definecolor{currentstroke}{rgb}{0.000000,0.000000,0.000000}%
\pgfsetstrokecolor{currentstroke}%
\pgfsetstrokeopacity{0.000000}%
\pgfsetdash{}{0pt}%
\pgfpathmoveto{\pgfqpoint{5.800000in}{36.815811in}}%
\pgfpathlineto{\pgfqpoint{5.800000in}{36.815811in}}%
\pgfpathlineto{\pgfqpoint{5.800000in}{36.918648in}}%
\pgfpathlineto{\pgfqpoint{5.800000in}{36.918648in}}%
\pgfpathclose%
\pgfusepath{fill}%
\end{pgfscope}%
\begin{pgfscope}%
\pgfpathrectangle{\pgfqpoint{5.800000in}{0.720000in}}{\pgfqpoint{1.400000in}{4.620000in}}%
\pgfusepath{clip}%
\pgfsetbuttcap%
\pgfsetmiterjoin%
\definecolor{currentfill}{rgb}{0.121569,0.466667,0.705882}%
\pgfsetfillcolor{currentfill}%
\pgfsetlinewidth{0.000000pt}%
\definecolor{currentstroke}{rgb}{0.000000,0.000000,0.000000}%
\pgfsetstrokecolor{currentstroke}%
\pgfsetstrokeopacity{0.000000}%
\pgfsetdash{}{0pt}%
\pgfpathmoveto{\pgfqpoint{5.800000in}{36.918648in}}%
\pgfpathlineto{\pgfqpoint{5.800000in}{36.918648in}}%
\pgfpathlineto{\pgfqpoint{5.800000in}{37.021485in}}%
\pgfpathlineto{\pgfqpoint{5.800000in}{37.021485in}}%
\pgfpathclose%
\pgfusepath{fill}%
\end{pgfscope}%
\begin{pgfscope}%
\pgfpathrectangle{\pgfqpoint{5.800000in}{0.720000in}}{\pgfqpoint{1.400000in}{4.620000in}}%
\pgfusepath{clip}%
\pgfsetbuttcap%
\pgfsetmiterjoin%
\definecolor{currentfill}{rgb}{0.121569,0.466667,0.705882}%
\pgfsetfillcolor{currentfill}%
\pgfsetlinewidth{0.000000pt}%
\definecolor{currentstroke}{rgb}{0.000000,0.000000,0.000000}%
\pgfsetstrokecolor{currentstroke}%
\pgfsetstrokeopacity{0.000000}%
\pgfsetdash{}{0pt}%
\pgfpathmoveto{\pgfqpoint{5.800000in}{37.021485in}}%
\pgfpathlineto{\pgfqpoint{5.800000in}{37.021485in}}%
\pgfpathlineto{\pgfqpoint{5.800000in}{37.124322in}}%
\pgfpathlineto{\pgfqpoint{5.800000in}{37.124322in}}%
\pgfpathclose%
\pgfusepath{fill}%
\end{pgfscope}%
\begin{pgfscope}%
\pgfpathrectangle{\pgfqpoint{5.800000in}{0.720000in}}{\pgfqpoint{1.400000in}{4.620000in}}%
\pgfusepath{clip}%
\pgfsetbuttcap%
\pgfsetmiterjoin%
\definecolor{currentfill}{rgb}{0.121569,0.466667,0.705882}%
\pgfsetfillcolor{currentfill}%
\pgfsetlinewidth{0.000000pt}%
\definecolor{currentstroke}{rgb}{0.000000,0.000000,0.000000}%
\pgfsetstrokecolor{currentstroke}%
\pgfsetstrokeopacity{0.000000}%
\pgfsetdash{}{0pt}%
\pgfpathmoveto{\pgfqpoint{5.800000in}{37.124322in}}%
\pgfpathlineto{\pgfqpoint{5.800000in}{37.124322in}}%
\pgfpathlineto{\pgfqpoint{5.800000in}{37.227159in}}%
\pgfpathlineto{\pgfqpoint{5.800000in}{37.227159in}}%
\pgfpathclose%
\pgfusepath{fill}%
\end{pgfscope}%
\begin{pgfscope}%
\pgfpathrectangle{\pgfqpoint{5.800000in}{0.720000in}}{\pgfqpoint{1.400000in}{4.620000in}}%
\pgfusepath{clip}%
\pgfsetbuttcap%
\pgfsetmiterjoin%
\definecolor{currentfill}{rgb}{0.121569,0.466667,0.705882}%
\pgfsetfillcolor{currentfill}%
\pgfsetlinewidth{0.000000pt}%
\definecolor{currentstroke}{rgb}{0.000000,0.000000,0.000000}%
\pgfsetstrokecolor{currentstroke}%
\pgfsetstrokeopacity{0.000000}%
\pgfsetdash{}{0pt}%
\pgfpathmoveto{\pgfqpoint{5.800000in}{37.227159in}}%
\pgfpathlineto{\pgfqpoint{5.800000in}{37.227159in}}%
\pgfpathlineto{\pgfqpoint{5.800000in}{37.329996in}}%
\pgfpathlineto{\pgfqpoint{5.800000in}{37.329996in}}%
\pgfpathclose%
\pgfusepath{fill}%
\end{pgfscope}%
\begin{pgfscope}%
\pgfpathrectangle{\pgfqpoint{5.800000in}{0.720000in}}{\pgfqpoint{1.400000in}{4.620000in}}%
\pgfusepath{clip}%
\pgfsetbuttcap%
\pgfsetmiterjoin%
\definecolor{currentfill}{rgb}{0.121569,0.466667,0.705882}%
\pgfsetfillcolor{currentfill}%
\pgfsetlinewidth{0.000000pt}%
\definecolor{currentstroke}{rgb}{0.000000,0.000000,0.000000}%
\pgfsetstrokecolor{currentstroke}%
\pgfsetstrokeopacity{0.000000}%
\pgfsetdash{}{0pt}%
\pgfpathmoveto{\pgfqpoint{5.800000in}{37.329996in}}%
\pgfpathlineto{\pgfqpoint{5.800000in}{37.329996in}}%
\pgfpathlineto{\pgfqpoint{5.800000in}{37.432833in}}%
\pgfpathlineto{\pgfqpoint{5.800000in}{37.432833in}}%
\pgfpathclose%
\pgfusepath{fill}%
\end{pgfscope}%
\begin{pgfscope}%
\pgfpathrectangle{\pgfqpoint{5.800000in}{0.720000in}}{\pgfqpoint{1.400000in}{4.620000in}}%
\pgfusepath{clip}%
\pgfsetbuttcap%
\pgfsetmiterjoin%
\definecolor{currentfill}{rgb}{0.121569,0.466667,0.705882}%
\pgfsetfillcolor{currentfill}%
\pgfsetlinewidth{0.000000pt}%
\definecolor{currentstroke}{rgb}{0.000000,0.000000,0.000000}%
\pgfsetstrokecolor{currentstroke}%
\pgfsetstrokeopacity{0.000000}%
\pgfsetdash{}{0pt}%
\pgfpathmoveto{\pgfqpoint{5.800000in}{37.432833in}}%
\pgfpathlineto{\pgfqpoint{5.800000in}{37.432833in}}%
\pgfpathlineto{\pgfqpoint{5.800000in}{37.535670in}}%
\pgfpathlineto{\pgfqpoint{5.800000in}{37.535670in}}%
\pgfpathclose%
\pgfusepath{fill}%
\end{pgfscope}%
\begin{pgfscope}%
\pgfpathrectangle{\pgfqpoint{5.800000in}{0.720000in}}{\pgfqpoint{1.400000in}{4.620000in}}%
\pgfusepath{clip}%
\pgfsetbuttcap%
\pgfsetmiterjoin%
\definecolor{currentfill}{rgb}{0.121569,0.466667,0.705882}%
\pgfsetfillcolor{currentfill}%
\pgfsetlinewidth{0.000000pt}%
\definecolor{currentstroke}{rgb}{0.000000,0.000000,0.000000}%
\pgfsetstrokecolor{currentstroke}%
\pgfsetstrokeopacity{0.000000}%
\pgfsetdash{}{0pt}%
\pgfpathmoveto{\pgfqpoint{5.800000in}{37.535670in}}%
\pgfpathlineto{\pgfqpoint{5.800000in}{37.535670in}}%
\pgfpathlineto{\pgfqpoint{5.800000in}{37.638507in}}%
\pgfpathlineto{\pgfqpoint{5.800000in}{37.638507in}}%
\pgfpathclose%
\pgfusepath{fill}%
\end{pgfscope}%
\begin{pgfscope}%
\pgfpathrectangle{\pgfqpoint{5.800000in}{0.720000in}}{\pgfqpoint{1.400000in}{4.620000in}}%
\pgfusepath{clip}%
\pgfsetbuttcap%
\pgfsetmiterjoin%
\definecolor{currentfill}{rgb}{0.121569,0.466667,0.705882}%
\pgfsetfillcolor{currentfill}%
\pgfsetlinewidth{0.000000pt}%
\definecolor{currentstroke}{rgb}{0.000000,0.000000,0.000000}%
\pgfsetstrokecolor{currentstroke}%
\pgfsetstrokeopacity{0.000000}%
\pgfsetdash{}{0pt}%
\pgfpathmoveto{\pgfqpoint{5.800000in}{37.638507in}}%
\pgfpathlineto{\pgfqpoint{5.800000in}{37.638507in}}%
\pgfpathlineto{\pgfqpoint{5.800000in}{37.741344in}}%
\pgfpathlineto{\pgfqpoint{5.800000in}{37.741344in}}%
\pgfpathclose%
\pgfusepath{fill}%
\end{pgfscope}%
\begin{pgfscope}%
\pgfpathrectangle{\pgfqpoint{5.800000in}{0.720000in}}{\pgfqpoint{1.400000in}{4.620000in}}%
\pgfusepath{clip}%
\pgfsetbuttcap%
\pgfsetmiterjoin%
\definecolor{currentfill}{rgb}{0.121569,0.466667,0.705882}%
\pgfsetfillcolor{currentfill}%
\pgfsetlinewidth{0.000000pt}%
\definecolor{currentstroke}{rgb}{0.000000,0.000000,0.000000}%
\pgfsetstrokecolor{currentstroke}%
\pgfsetstrokeopacity{0.000000}%
\pgfsetdash{}{0pt}%
\pgfpathmoveto{\pgfqpoint{5.800000in}{37.741344in}}%
\pgfpathlineto{\pgfqpoint{5.800000in}{37.741344in}}%
\pgfpathlineto{\pgfqpoint{5.800000in}{37.844181in}}%
\pgfpathlineto{\pgfqpoint{5.800000in}{37.844181in}}%
\pgfpathclose%
\pgfusepath{fill}%
\end{pgfscope}%
\begin{pgfscope}%
\pgfpathrectangle{\pgfqpoint{5.800000in}{0.720000in}}{\pgfqpoint{1.400000in}{4.620000in}}%
\pgfusepath{clip}%
\pgfsetbuttcap%
\pgfsetmiterjoin%
\definecolor{currentfill}{rgb}{0.121569,0.466667,0.705882}%
\pgfsetfillcolor{currentfill}%
\pgfsetlinewidth{0.000000pt}%
\definecolor{currentstroke}{rgb}{0.000000,0.000000,0.000000}%
\pgfsetstrokecolor{currentstroke}%
\pgfsetstrokeopacity{0.000000}%
\pgfsetdash{}{0pt}%
\pgfpathmoveto{\pgfqpoint{5.800000in}{37.844181in}}%
\pgfpathlineto{\pgfqpoint{5.800000in}{37.844181in}}%
\pgfpathlineto{\pgfqpoint{5.800000in}{37.947019in}}%
\pgfpathlineto{\pgfqpoint{5.800000in}{37.947019in}}%
\pgfpathclose%
\pgfusepath{fill}%
\end{pgfscope}%
\begin{pgfscope}%
\pgfpathrectangle{\pgfqpoint{5.800000in}{0.720000in}}{\pgfqpoint{1.400000in}{4.620000in}}%
\pgfusepath{clip}%
\pgfsetbuttcap%
\pgfsetmiterjoin%
\definecolor{currentfill}{rgb}{0.121569,0.466667,0.705882}%
\pgfsetfillcolor{currentfill}%
\pgfsetlinewidth{0.000000pt}%
\definecolor{currentstroke}{rgb}{0.000000,0.000000,0.000000}%
\pgfsetstrokecolor{currentstroke}%
\pgfsetstrokeopacity{0.000000}%
\pgfsetdash{}{0pt}%
\pgfpathmoveto{\pgfqpoint{5.800000in}{37.947019in}}%
\pgfpathlineto{\pgfqpoint{5.800000in}{37.947019in}}%
\pgfpathlineto{\pgfqpoint{5.800000in}{38.049856in}}%
\pgfpathlineto{\pgfqpoint{5.800000in}{38.049856in}}%
\pgfpathclose%
\pgfusepath{fill}%
\end{pgfscope}%
\begin{pgfscope}%
\pgfpathrectangle{\pgfqpoint{5.800000in}{0.720000in}}{\pgfqpoint{1.400000in}{4.620000in}}%
\pgfusepath{clip}%
\pgfsetbuttcap%
\pgfsetmiterjoin%
\definecolor{currentfill}{rgb}{0.121569,0.466667,0.705882}%
\pgfsetfillcolor{currentfill}%
\pgfsetlinewidth{0.000000pt}%
\definecolor{currentstroke}{rgb}{0.000000,0.000000,0.000000}%
\pgfsetstrokecolor{currentstroke}%
\pgfsetstrokeopacity{0.000000}%
\pgfsetdash{}{0pt}%
\pgfpathmoveto{\pgfqpoint{5.800000in}{38.049856in}}%
\pgfpathlineto{\pgfqpoint{5.800000in}{38.049856in}}%
\pgfpathlineto{\pgfqpoint{5.800000in}{38.152693in}}%
\pgfpathlineto{\pgfqpoint{5.800000in}{38.152693in}}%
\pgfpathclose%
\pgfusepath{fill}%
\end{pgfscope}%
\begin{pgfscope}%
\pgfpathrectangle{\pgfqpoint{5.800000in}{0.720000in}}{\pgfqpoint{1.400000in}{4.620000in}}%
\pgfusepath{clip}%
\pgfsetbuttcap%
\pgfsetmiterjoin%
\definecolor{currentfill}{rgb}{0.121569,0.466667,0.705882}%
\pgfsetfillcolor{currentfill}%
\pgfsetlinewidth{0.000000pt}%
\definecolor{currentstroke}{rgb}{0.000000,0.000000,0.000000}%
\pgfsetstrokecolor{currentstroke}%
\pgfsetstrokeopacity{0.000000}%
\pgfsetdash{}{0pt}%
\pgfpathmoveto{\pgfqpoint{5.800000in}{38.152693in}}%
\pgfpathlineto{\pgfqpoint{5.800000in}{38.152693in}}%
\pgfpathlineto{\pgfqpoint{5.800000in}{38.255530in}}%
\pgfpathlineto{\pgfqpoint{5.800000in}{38.255530in}}%
\pgfpathclose%
\pgfusepath{fill}%
\end{pgfscope}%
\begin{pgfscope}%
\pgfpathrectangle{\pgfqpoint{5.800000in}{0.720000in}}{\pgfqpoint{1.400000in}{4.620000in}}%
\pgfusepath{clip}%
\pgfsetbuttcap%
\pgfsetmiterjoin%
\definecolor{currentfill}{rgb}{0.121569,0.466667,0.705882}%
\pgfsetfillcolor{currentfill}%
\pgfsetlinewidth{0.000000pt}%
\definecolor{currentstroke}{rgb}{0.000000,0.000000,0.000000}%
\pgfsetstrokecolor{currentstroke}%
\pgfsetstrokeopacity{0.000000}%
\pgfsetdash{}{0pt}%
\pgfpathmoveto{\pgfqpoint{5.800000in}{38.255530in}}%
\pgfpathlineto{\pgfqpoint{5.800000in}{38.255530in}}%
\pgfpathlineto{\pgfqpoint{5.800000in}{38.358367in}}%
\pgfpathlineto{\pgfqpoint{5.800000in}{38.358367in}}%
\pgfpathclose%
\pgfusepath{fill}%
\end{pgfscope}%
\begin{pgfscope}%
\pgfpathrectangle{\pgfqpoint{5.800000in}{0.720000in}}{\pgfqpoint{1.400000in}{4.620000in}}%
\pgfusepath{clip}%
\pgfsetbuttcap%
\pgfsetmiterjoin%
\definecolor{currentfill}{rgb}{0.121569,0.466667,0.705882}%
\pgfsetfillcolor{currentfill}%
\pgfsetlinewidth{0.000000pt}%
\definecolor{currentstroke}{rgb}{0.000000,0.000000,0.000000}%
\pgfsetstrokecolor{currentstroke}%
\pgfsetstrokeopacity{0.000000}%
\pgfsetdash{}{0pt}%
\pgfpathmoveto{\pgfqpoint{5.800000in}{38.358367in}}%
\pgfpathlineto{\pgfqpoint{5.800000in}{38.358367in}}%
\pgfpathlineto{\pgfqpoint{5.800000in}{38.461204in}}%
\pgfpathlineto{\pgfqpoint{5.800000in}{38.461204in}}%
\pgfpathclose%
\pgfusepath{fill}%
\end{pgfscope}%
\begin{pgfscope}%
\pgfpathrectangle{\pgfqpoint{5.800000in}{0.720000in}}{\pgfqpoint{1.400000in}{4.620000in}}%
\pgfusepath{clip}%
\pgfsetbuttcap%
\pgfsetmiterjoin%
\definecolor{currentfill}{rgb}{0.121569,0.466667,0.705882}%
\pgfsetfillcolor{currentfill}%
\pgfsetlinewidth{0.000000pt}%
\definecolor{currentstroke}{rgb}{0.000000,0.000000,0.000000}%
\pgfsetstrokecolor{currentstroke}%
\pgfsetstrokeopacity{0.000000}%
\pgfsetdash{}{0pt}%
\pgfpathmoveto{\pgfqpoint{5.800000in}{38.461204in}}%
\pgfpathlineto{\pgfqpoint{5.800000in}{38.461204in}}%
\pgfpathlineto{\pgfqpoint{5.800000in}{38.564041in}}%
\pgfpathlineto{\pgfqpoint{5.800000in}{38.564041in}}%
\pgfpathclose%
\pgfusepath{fill}%
\end{pgfscope}%
\begin{pgfscope}%
\pgfpathrectangle{\pgfqpoint{5.800000in}{0.720000in}}{\pgfqpoint{1.400000in}{4.620000in}}%
\pgfusepath{clip}%
\pgfsetbuttcap%
\pgfsetmiterjoin%
\definecolor{currentfill}{rgb}{0.121569,0.466667,0.705882}%
\pgfsetfillcolor{currentfill}%
\pgfsetlinewidth{0.000000pt}%
\definecolor{currentstroke}{rgb}{0.000000,0.000000,0.000000}%
\pgfsetstrokecolor{currentstroke}%
\pgfsetstrokeopacity{0.000000}%
\pgfsetdash{}{0pt}%
\pgfpathmoveto{\pgfqpoint{5.800000in}{38.564041in}}%
\pgfpathlineto{\pgfqpoint{5.800000in}{38.564041in}}%
\pgfpathlineto{\pgfqpoint{5.800000in}{38.666878in}}%
\pgfpathlineto{\pgfqpoint{5.800000in}{38.666878in}}%
\pgfpathclose%
\pgfusepath{fill}%
\end{pgfscope}%
\begin{pgfscope}%
\pgfpathrectangle{\pgfqpoint{5.800000in}{0.720000in}}{\pgfqpoint{1.400000in}{4.620000in}}%
\pgfusepath{clip}%
\pgfsetbuttcap%
\pgfsetmiterjoin%
\definecolor{currentfill}{rgb}{0.121569,0.466667,0.705882}%
\pgfsetfillcolor{currentfill}%
\pgfsetlinewidth{0.000000pt}%
\definecolor{currentstroke}{rgb}{0.000000,0.000000,0.000000}%
\pgfsetstrokecolor{currentstroke}%
\pgfsetstrokeopacity{0.000000}%
\pgfsetdash{}{0pt}%
\pgfpathmoveto{\pgfqpoint{5.800000in}{38.666878in}}%
\pgfpathlineto{\pgfqpoint{5.800000in}{38.666878in}}%
\pgfpathlineto{\pgfqpoint{5.800000in}{38.769715in}}%
\pgfpathlineto{\pgfqpoint{5.800000in}{38.769715in}}%
\pgfpathclose%
\pgfusepath{fill}%
\end{pgfscope}%
\begin{pgfscope}%
\pgfpathrectangle{\pgfqpoint{5.800000in}{0.720000in}}{\pgfqpoint{1.400000in}{4.620000in}}%
\pgfusepath{clip}%
\pgfsetbuttcap%
\pgfsetmiterjoin%
\definecolor{currentfill}{rgb}{0.121569,0.466667,0.705882}%
\pgfsetfillcolor{currentfill}%
\pgfsetlinewidth{0.000000pt}%
\definecolor{currentstroke}{rgb}{0.000000,0.000000,0.000000}%
\pgfsetstrokecolor{currentstroke}%
\pgfsetstrokeopacity{0.000000}%
\pgfsetdash{}{0pt}%
\pgfpathmoveto{\pgfqpoint{5.800000in}{38.769715in}}%
\pgfpathlineto{\pgfqpoint{5.800000in}{38.769715in}}%
\pgfpathlineto{\pgfqpoint{5.800000in}{38.872552in}}%
\pgfpathlineto{\pgfqpoint{5.800000in}{38.872552in}}%
\pgfpathclose%
\pgfusepath{fill}%
\end{pgfscope}%
\begin{pgfscope}%
\pgfpathrectangle{\pgfqpoint{5.800000in}{0.720000in}}{\pgfqpoint{1.400000in}{4.620000in}}%
\pgfusepath{clip}%
\pgfsetbuttcap%
\pgfsetmiterjoin%
\definecolor{currentfill}{rgb}{0.121569,0.466667,0.705882}%
\pgfsetfillcolor{currentfill}%
\pgfsetlinewidth{0.000000pt}%
\definecolor{currentstroke}{rgb}{0.000000,0.000000,0.000000}%
\pgfsetstrokecolor{currentstroke}%
\pgfsetstrokeopacity{0.000000}%
\pgfsetdash{}{0pt}%
\pgfpathmoveto{\pgfqpoint{5.800000in}{38.872552in}}%
\pgfpathlineto{\pgfqpoint{5.800000in}{38.872552in}}%
\pgfpathlineto{\pgfqpoint{5.800000in}{38.975389in}}%
\pgfpathlineto{\pgfqpoint{5.800000in}{38.975389in}}%
\pgfpathclose%
\pgfusepath{fill}%
\end{pgfscope}%
\begin{pgfscope}%
\pgfpathrectangle{\pgfqpoint{5.800000in}{0.720000in}}{\pgfqpoint{1.400000in}{4.620000in}}%
\pgfusepath{clip}%
\pgfsetbuttcap%
\pgfsetmiterjoin%
\definecolor{currentfill}{rgb}{0.121569,0.466667,0.705882}%
\pgfsetfillcolor{currentfill}%
\pgfsetlinewidth{0.000000pt}%
\definecolor{currentstroke}{rgb}{0.000000,0.000000,0.000000}%
\pgfsetstrokecolor{currentstroke}%
\pgfsetstrokeopacity{0.000000}%
\pgfsetdash{}{0pt}%
\pgfpathmoveto{\pgfqpoint{5.800000in}{38.975389in}}%
\pgfpathlineto{\pgfqpoint{5.800000in}{38.975389in}}%
\pgfpathlineto{\pgfqpoint{5.800000in}{39.078226in}}%
\pgfpathlineto{\pgfqpoint{5.800000in}{39.078226in}}%
\pgfpathclose%
\pgfusepath{fill}%
\end{pgfscope}%
\begin{pgfscope}%
\pgfpathrectangle{\pgfqpoint{5.800000in}{0.720000in}}{\pgfqpoint{1.400000in}{4.620000in}}%
\pgfusepath{clip}%
\pgfsetbuttcap%
\pgfsetmiterjoin%
\definecolor{currentfill}{rgb}{0.121569,0.466667,0.705882}%
\pgfsetfillcolor{currentfill}%
\pgfsetlinewidth{0.000000pt}%
\definecolor{currentstroke}{rgb}{0.000000,0.000000,0.000000}%
\pgfsetstrokecolor{currentstroke}%
\pgfsetstrokeopacity{0.000000}%
\pgfsetdash{}{0pt}%
\pgfpathmoveto{\pgfqpoint{5.800000in}{39.078226in}}%
\pgfpathlineto{\pgfqpoint{5.800000in}{39.078226in}}%
\pgfpathlineto{\pgfqpoint{5.800000in}{39.181063in}}%
\pgfpathlineto{\pgfqpoint{5.800000in}{39.181063in}}%
\pgfpathclose%
\pgfusepath{fill}%
\end{pgfscope}%
\begin{pgfscope}%
\pgfpathrectangle{\pgfqpoint{5.800000in}{0.720000in}}{\pgfqpoint{1.400000in}{4.620000in}}%
\pgfusepath{clip}%
\pgfsetbuttcap%
\pgfsetmiterjoin%
\definecolor{currentfill}{rgb}{0.121569,0.466667,0.705882}%
\pgfsetfillcolor{currentfill}%
\pgfsetlinewidth{0.000000pt}%
\definecolor{currentstroke}{rgb}{0.000000,0.000000,0.000000}%
\pgfsetstrokecolor{currentstroke}%
\pgfsetstrokeopacity{0.000000}%
\pgfsetdash{}{0pt}%
\pgfpathmoveto{\pgfqpoint{5.800000in}{39.181063in}}%
\pgfpathlineto{\pgfqpoint{5.800000in}{39.181063in}}%
\pgfpathlineto{\pgfqpoint{5.800000in}{39.283900in}}%
\pgfpathlineto{\pgfqpoint{5.800000in}{39.283900in}}%
\pgfpathclose%
\pgfusepath{fill}%
\end{pgfscope}%
\begin{pgfscope}%
\pgfpathrectangle{\pgfqpoint{5.800000in}{0.720000in}}{\pgfqpoint{1.400000in}{4.620000in}}%
\pgfusepath{clip}%
\pgfsetbuttcap%
\pgfsetmiterjoin%
\definecolor{currentfill}{rgb}{0.121569,0.466667,0.705882}%
\pgfsetfillcolor{currentfill}%
\pgfsetlinewidth{0.000000pt}%
\definecolor{currentstroke}{rgb}{0.000000,0.000000,0.000000}%
\pgfsetstrokecolor{currentstroke}%
\pgfsetstrokeopacity{0.000000}%
\pgfsetdash{}{0pt}%
\pgfpathmoveto{\pgfqpoint{5.800000in}{39.283900in}}%
\pgfpathlineto{\pgfqpoint{5.800000in}{39.283900in}}%
\pgfpathlineto{\pgfqpoint{5.800000in}{39.386737in}}%
\pgfpathlineto{\pgfqpoint{5.800000in}{39.386737in}}%
\pgfpathclose%
\pgfusepath{fill}%
\end{pgfscope}%
\begin{pgfscope}%
\pgfpathrectangle{\pgfqpoint{5.800000in}{0.720000in}}{\pgfqpoint{1.400000in}{4.620000in}}%
\pgfusepath{clip}%
\pgfsetbuttcap%
\pgfsetmiterjoin%
\definecolor{currentfill}{rgb}{0.121569,0.466667,0.705882}%
\pgfsetfillcolor{currentfill}%
\pgfsetlinewidth{0.000000pt}%
\definecolor{currentstroke}{rgb}{0.000000,0.000000,0.000000}%
\pgfsetstrokecolor{currentstroke}%
\pgfsetstrokeopacity{0.000000}%
\pgfsetdash{}{0pt}%
\pgfpathmoveto{\pgfqpoint{5.800000in}{39.386737in}}%
\pgfpathlineto{\pgfqpoint{5.800000in}{39.386737in}}%
\pgfpathlineto{\pgfqpoint{5.800000in}{39.489575in}}%
\pgfpathlineto{\pgfqpoint{5.800000in}{39.489575in}}%
\pgfpathclose%
\pgfusepath{fill}%
\end{pgfscope}%
\begin{pgfscope}%
\pgfpathrectangle{\pgfqpoint{5.800000in}{0.720000in}}{\pgfqpoint{1.400000in}{4.620000in}}%
\pgfusepath{clip}%
\pgfsetbuttcap%
\pgfsetmiterjoin%
\definecolor{currentfill}{rgb}{0.121569,0.466667,0.705882}%
\pgfsetfillcolor{currentfill}%
\pgfsetlinewidth{0.000000pt}%
\definecolor{currentstroke}{rgb}{0.000000,0.000000,0.000000}%
\pgfsetstrokecolor{currentstroke}%
\pgfsetstrokeopacity{0.000000}%
\pgfsetdash{}{0pt}%
\pgfpathmoveto{\pgfqpoint{5.800000in}{39.489575in}}%
\pgfpathlineto{\pgfqpoint{5.800000in}{39.489575in}}%
\pgfpathlineto{\pgfqpoint{5.800000in}{39.592412in}}%
\pgfpathlineto{\pgfqpoint{5.800000in}{39.592412in}}%
\pgfpathclose%
\pgfusepath{fill}%
\end{pgfscope}%
\begin{pgfscope}%
\pgfpathrectangle{\pgfqpoint{5.800000in}{0.720000in}}{\pgfqpoint{1.400000in}{4.620000in}}%
\pgfusepath{clip}%
\pgfsetbuttcap%
\pgfsetmiterjoin%
\definecolor{currentfill}{rgb}{0.121569,0.466667,0.705882}%
\pgfsetfillcolor{currentfill}%
\pgfsetlinewidth{0.000000pt}%
\definecolor{currentstroke}{rgb}{0.000000,0.000000,0.000000}%
\pgfsetstrokecolor{currentstroke}%
\pgfsetstrokeopacity{0.000000}%
\pgfsetdash{}{0pt}%
\pgfpathmoveto{\pgfqpoint{5.800000in}{39.592412in}}%
\pgfpathlineto{\pgfqpoint{5.800000in}{39.592412in}}%
\pgfpathlineto{\pgfqpoint{5.800000in}{39.695249in}}%
\pgfpathlineto{\pgfqpoint{5.800000in}{39.695249in}}%
\pgfpathclose%
\pgfusepath{fill}%
\end{pgfscope}%
\begin{pgfscope}%
\pgfpathrectangle{\pgfqpoint{5.800000in}{0.720000in}}{\pgfqpoint{1.400000in}{4.620000in}}%
\pgfusepath{clip}%
\pgfsetbuttcap%
\pgfsetmiterjoin%
\definecolor{currentfill}{rgb}{0.121569,0.466667,0.705882}%
\pgfsetfillcolor{currentfill}%
\pgfsetlinewidth{0.000000pt}%
\definecolor{currentstroke}{rgb}{0.000000,0.000000,0.000000}%
\pgfsetstrokecolor{currentstroke}%
\pgfsetstrokeopacity{0.000000}%
\pgfsetdash{}{0pt}%
\pgfpathmoveto{\pgfqpoint{5.800000in}{39.695249in}}%
\pgfpathlineto{\pgfqpoint{5.800000in}{39.695249in}}%
\pgfpathlineto{\pgfqpoint{5.800000in}{39.798086in}}%
\pgfpathlineto{\pgfqpoint{5.800000in}{39.798086in}}%
\pgfpathclose%
\pgfusepath{fill}%
\end{pgfscope}%
\begin{pgfscope}%
\pgfpathrectangle{\pgfqpoint{5.800000in}{0.720000in}}{\pgfqpoint{1.400000in}{4.620000in}}%
\pgfusepath{clip}%
\pgfsetbuttcap%
\pgfsetmiterjoin%
\definecolor{currentfill}{rgb}{0.121569,0.466667,0.705882}%
\pgfsetfillcolor{currentfill}%
\pgfsetlinewidth{0.000000pt}%
\definecolor{currentstroke}{rgb}{0.000000,0.000000,0.000000}%
\pgfsetstrokecolor{currentstroke}%
\pgfsetstrokeopacity{0.000000}%
\pgfsetdash{}{0pt}%
\pgfpathmoveto{\pgfqpoint{5.800000in}{39.798086in}}%
\pgfpathlineto{\pgfqpoint{5.800000in}{39.798086in}}%
\pgfpathlineto{\pgfqpoint{5.800000in}{39.900923in}}%
\pgfpathlineto{\pgfqpoint{5.800000in}{39.900923in}}%
\pgfpathclose%
\pgfusepath{fill}%
\end{pgfscope}%
\begin{pgfscope}%
\pgfpathrectangle{\pgfqpoint{5.800000in}{0.720000in}}{\pgfqpoint{1.400000in}{4.620000in}}%
\pgfusepath{clip}%
\pgfsetbuttcap%
\pgfsetmiterjoin%
\definecolor{currentfill}{rgb}{0.121569,0.466667,0.705882}%
\pgfsetfillcolor{currentfill}%
\pgfsetlinewidth{0.000000pt}%
\definecolor{currentstroke}{rgb}{0.000000,0.000000,0.000000}%
\pgfsetstrokecolor{currentstroke}%
\pgfsetstrokeopacity{0.000000}%
\pgfsetdash{}{0pt}%
\pgfpathmoveto{\pgfqpoint{5.800000in}{39.900923in}}%
\pgfpathlineto{\pgfqpoint{5.800000in}{39.900923in}}%
\pgfpathlineto{\pgfqpoint{5.800000in}{40.003760in}}%
\pgfpathlineto{\pgfqpoint{5.800000in}{40.003760in}}%
\pgfpathclose%
\pgfusepath{fill}%
\end{pgfscope}%
\begin{pgfscope}%
\pgfpathrectangle{\pgfqpoint{5.800000in}{0.720000in}}{\pgfqpoint{1.400000in}{4.620000in}}%
\pgfusepath{clip}%
\pgfsetbuttcap%
\pgfsetmiterjoin%
\definecolor{currentfill}{rgb}{0.121569,0.466667,0.705882}%
\pgfsetfillcolor{currentfill}%
\pgfsetlinewidth{0.000000pt}%
\definecolor{currentstroke}{rgb}{0.000000,0.000000,0.000000}%
\pgfsetstrokecolor{currentstroke}%
\pgfsetstrokeopacity{0.000000}%
\pgfsetdash{}{0pt}%
\pgfpathmoveto{\pgfqpoint{5.800000in}{40.003760in}}%
\pgfpathlineto{\pgfqpoint{5.800000in}{40.003760in}}%
\pgfpathlineto{\pgfqpoint{5.800000in}{40.106597in}}%
\pgfpathlineto{\pgfqpoint{5.800000in}{40.106597in}}%
\pgfpathclose%
\pgfusepath{fill}%
\end{pgfscope}%
\begin{pgfscope}%
\pgfpathrectangle{\pgfqpoint{5.800000in}{0.720000in}}{\pgfqpoint{1.400000in}{4.620000in}}%
\pgfusepath{clip}%
\pgfsetbuttcap%
\pgfsetmiterjoin%
\definecolor{currentfill}{rgb}{0.121569,0.466667,0.705882}%
\pgfsetfillcolor{currentfill}%
\pgfsetlinewidth{0.000000pt}%
\definecolor{currentstroke}{rgb}{0.000000,0.000000,0.000000}%
\pgfsetstrokecolor{currentstroke}%
\pgfsetstrokeopacity{0.000000}%
\pgfsetdash{}{0pt}%
\pgfpathmoveto{\pgfqpoint{5.800000in}{40.106597in}}%
\pgfpathlineto{\pgfqpoint{5.800000in}{40.106597in}}%
\pgfpathlineto{\pgfqpoint{5.800000in}{40.209434in}}%
\pgfpathlineto{\pgfqpoint{5.800000in}{40.209434in}}%
\pgfpathclose%
\pgfusepath{fill}%
\end{pgfscope}%
\begin{pgfscope}%
\pgfpathrectangle{\pgfqpoint{5.800000in}{0.720000in}}{\pgfqpoint{1.400000in}{4.620000in}}%
\pgfusepath{clip}%
\pgfsetbuttcap%
\pgfsetmiterjoin%
\definecolor{currentfill}{rgb}{0.121569,0.466667,0.705882}%
\pgfsetfillcolor{currentfill}%
\pgfsetlinewidth{0.000000pt}%
\definecolor{currentstroke}{rgb}{0.000000,0.000000,0.000000}%
\pgfsetstrokecolor{currentstroke}%
\pgfsetstrokeopacity{0.000000}%
\pgfsetdash{}{0pt}%
\pgfpathmoveto{\pgfqpoint{5.800000in}{40.209434in}}%
\pgfpathlineto{\pgfqpoint{5.800000in}{40.209434in}}%
\pgfpathlineto{\pgfqpoint{5.800000in}{40.312271in}}%
\pgfpathlineto{\pgfqpoint{5.800000in}{40.312271in}}%
\pgfpathclose%
\pgfusepath{fill}%
\end{pgfscope}%
\begin{pgfscope}%
\pgfpathrectangle{\pgfqpoint{5.800000in}{0.720000in}}{\pgfqpoint{1.400000in}{4.620000in}}%
\pgfusepath{clip}%
\pgfsetbuttcap%
\pgfsetmiterjoin%
\definecolor{currentfill}{rgb}{0.121569,0.466667,0.705882}%
\pgfsetfillcolor{currentfill}%
\pgfsetlinewidth{0.000000pt}%
\definecolor{currentstroke}{rgb}{0.000000,0.000000,0.000000}%
\pgfsetstrokecolor{currentstroke}%
\pgfsetstrokeopacity{0.000000}%
\pgfsetdash{}{0pt}%
\pgfpathmoveto{\pgfqpoint{5.800000in}{40.312271in}}%
\pgfpathlineto{\pgfqpoint{5.800000in}{40.312271in}}%
\pgfpathlineto{\pgfqpoint{5.800000in}{40.415108in}}%
\pgfpathlineto{\pgfqpoint{5.800000in}{40.415108in}}%
\pgfpathclose%
\pgfusepath{fill}%
\end{pgfscope}%
\begin{pgfscope}%
\pgfpathrectangle{\pgfqpoint{5.800000in}{0.720000in}}{\pgfqpoint{1.400000in}{4.620000in}}%
\pgfusepath{clip}%
\pgfsetbuttcap%
\pgfsetmiterjoin%
\definecolor{currentfill}{rgb}{0.121569,0.466667,0.705882}%
\pgfsetfillcolor{currentfill}%
\pgfsetlinewidth{0.000000pt}%
\definecolor{currentstroke}{rgb}{0.000000,0.000000,0.000000}%
\pgfsetstrokecolor{currentstroke}%
\pgfsetstrokeopacity{0.000000}%
\pgfsetdash{}{0pt}%
\pgfpathmoveto{\pgfqpoint{5.800000in}{40.415108in}}%
\pgfpathlineto{\pgfqpoint{5.800000in}{40.415108in}}%
\pgfpathlineto{\pgfqpoint{5.800000in}{40.517945in}}%
\pgfpathlineto{\pgfqpoint{5.800000in}{40.517945in}}%
\pgfpathclose%
\pgfusepath{fill}%
\end{pgfscope}%
\begin{pgfscope}%
\pgfpathrectangle{\pgfqpoint{5.800000in}{0.720000in}}{\pgfqpoint{1.400000in}{4.620000in}}%
\pgfusepath{clip}%
\pgfsetbuttcap%
\pgfsetmiterjoin%
\definecolor{currentfill}{rgb}{0.121569,0.466667,0.705882}%
\pgfsetfillcolor{currentfill}%
\pgfsetlinewidth{0.000000pt}%
\definecolor{currentstroke}{rgb}{0.000000,0.000000,0.000000}%
\pgfsetstrokecolor{currentstroke}%
\pgfsetstrokeopacity{0.000000}%
\pgfsetdash{}{0pt}%
\pgfpathmoveto{\pgfqpoint{5.800000in}{40.517945in}}%
\pgfpathlineto{\pgfqpoint{5.802038in}{40.517945in}}%
\pgfpathlineto{\pgfqpoint{5.802038in}{40.620782in}}%
\pgfpathlineto{\pgfqpoint{5.800000in}{40.620782in}}%
\pgfpathclose%
\pgfusepath{fill}%
\end{pgfscope}%
\begin{pgfscope}%
\pgfpathrectangle{\pgfqpoint{5.800000in}{0.720000in}}{\pgfqpoint{1.400000in}{4.620000in}}%
\pgfusepath{clip}%
\pgfsetbuttcap%
\pgfsetmiterjoin%
\definecolor{currentfill}{rgb}{0.121569,0.466667,0.705882}%
\pgfsetfillcolor{currentfill}%
\pgfsetlinewidth{0.000000pt}%
\definecolor{currentstroke}{rgb}{0.000000,0.000000,0.000000}%
\pgfsetstrokecolor{currentstroke}%
\pgfsetstrokeopacity{0.000000}%
\pgfsetdash{}{0pt}%
\pgfpathmoveto{\pgfqpoint{5.800000in}{40.620782in}}%
\pgfpathlineto{\pgfqpoint{5.800000in}{40.620782in}}%
\pgfpathlineto{\pgfqpoint{5.800000in}{40.723619in}}%
\pgfpathlineto{\pgfqpoint{5.800000in}{40.723619in}}%
\pgfpathclose%
\pgfusepath{fill}%
\end{pgfscope}%
\begin{pgfscope}%
\pgfpathrectangle{\pgfqpoint{5.800000in}{0.720000in}}{\pgfqpoint{1.400000in}{4.620000in}}%
\pgfusepath{clip}%
\pgfsetbuttcap%
\pgfsetmiterjoin%
\definecolor{currentfill}{rgb}{0.121569,0.466667,0.705882}%
\pgfsetfillcolor{currentfill}%
\pgfsetlinewidth{0.000000pt}%
\definecolor{currentstroke}{rgb}{0.000000,0.000000,0.000000}%
\pgfsetstrokecolor{currentstroke}%
\pgfsetstrokeopacity{0.000000}%
\pgfsetdash{}{0pt}%
\pgfpathmoveto{\pgfqpoint{5.800000in}{40.723619in}}%
\pgfpathlineto{\pgfqpoint{5.800000in}{40.723619in}}%
\pgfpathlineto{\pgfqpoint{5.800000in}{40.826456in}}%
\pgfpathlineto{\pgfqpoint{5.800000in}{40.826456in}}%
\pgfpathclose%
\pgfusepath{fill}%
\end{pgfscope}%
\begin{pgfscope}%
\pgfpathrectangle{\pgfqpoint{5.800000in}{0.720000in}}{\pgfqpoint{1.400000in}{4.620000in}}%
\pgfusepath{clip}%
\pgfsetbuttcap%
\pgfsetmiterjoin%
\definecolor{currentfill}{rgb}{0.121569,0.466667,0.705882}%
\pgfsetfillcolor{currentfill}%
\pgfsetlinewidth{0.000000pt}%
\definecolor{currentstroke}{rgb}{0.000000,0.000000,0.000000}%
\pgfsetstrokecolor{currentstroke}%
\pgfsetstrokeopacity{0.000000}%
\pgfsetdash{}{0pt}%
\pgfpathmoveto{\pgfqpoint{5.800000in}{40.826456in}}%
\pgfpathlineto{\pgfqpoint{5.800000in}{40.826456in}}%
\pgfpathlineto{\pgfqpoint{5.800000in}{40.929294in}}%
\pgfpathlineto{\pgfqpoint{5.800000in}{40.929294in}}%
\pgfpathclose%
\pgfusepath{fill}%
\end{pgfscope}%
\begin{pgfscope}%
\pgfpathrectangle{\pgfqpoint{5.800000in}{0.720000in}}{\pgfqpoint{1.400000in}{4.620000in}}%
\pgfusepath{clip}%
\pgfsetbuttcap%
\pgfsetmiterjoin%
\definecolor{currentfill}{rgb}{0.121569,0.466667,0.705882}%
\pgfsetfillcolor{currentfill}%
\pgfsetlinewidth{0.000000pt}%
\definecolor{currentstroke}{rgb}{0.000000,0.000000,0.000000}%
\pgfsetstrokecolor{currentstroke}%
\pgfsetstrokeopacity{0.000000}%
\pgfsetdash{}{0pt}%
\pgfpathmoveto{\pgfqpoint{5.800000in}{40.929294in}}%
\pgfpathlineto{\pgfqpoint{5.800000in}{40.929294in}}%
\pgfpathlineto{\pgfqpoint{5.800000in}{41.032131in}}%
\pgfpathlineto{\pgfqpoint{5.800000in}{41.032131in}}%
\pgfpathclose%
\pgfusepath{fill}%
\end{pgfscope}%
\begin{pgfscope}%
\pgfpathrectangle{\pgfqpoint{5.800000in}{0.720000in}}{\pgfqpoint{1.400000in}{4.620000in}}%
\pgfusepath{clip}%
\pgfsetbuttcap%
\pgfsetmiterjoin%
\definecolor{currentfill}{rgb}{0.121569,0.466667,0.705882}%
\pgfsetfillcolor{currentfill}%
\pgfsetlinewidth{0.000000pt}%
\definecolor{currentstroke}{rgb}{0.000000,0.000000,0.000000}%
\pgfsetstrokecolor{currentstroke}%
\pgfsetstrokeopacity{0.000000}%
\pgfsetdash{}{0pt}%
\pgfpathmoveto{\pgfqpoint{5.800000in}{41.032131in}}%
\pgfpathlineto{\pgfqpoint{5.800000in}{41.032131in}}%
\pgfpathlineto{\pgfqpoint{5.800000in}{41.134968in}}%
\pgfpathlineto{\pgfqpoint{5.800000in}{41.134968in}}%
\pgfpathclose%
\pgfusepath{fill}%
\end{pgfscope}%
\begin{pgfscope}%
\pgfpathrectangle{\pgfqpoint{5.800000in}{0.720000in}}{\pgfqpoint{1.400000in}{4.620000in}}%
\pgfusepath{clip}%
\pgfsetbuttcap%
\pgfsetmiterjoin%
\definecolor{currentfill}{rgb}{0.121569,0.466667,0.705882}%
\pgfsetfillcolor{currentfill}%
\pgfsetlinewidth{0.000000pt}%
\definecolor{currentstroke}{rgb}{0.000000,0.000000,0.000000}%
\pgfsetstrokecolor{currentstroke}%
\pgfsetstrokeopacity{0.000000}%
\pgfsetdash{}{0pt}%
\pgfpathmoveto{\pgfqpoint{5.800000in}{41.134968in}}%
\pgfpathlineto{\pgfqpoint{5.800000in}{41.134968in}}%
\pgfpathlineto{\pgfqpoint{5.800000in}{41.237805in}}%
\pgfpathlineto{\pgfqpoint{5.800000in}{41.237805in}}%
\pgfpathclose%
\pgfusepath{fill}%
\end{pgfscope}%
\begin{pgfscope}%
\pgfpathrectangle{\pgfqpoint{5.800000in}{0.720000in}}{\pgfqpoint{1.400000in}{4.620000in}}%
\pgfusepath{clip}%
\pgfsetbuttcap%
\pgfsetmiterjoin%
\definecolor{currentfill}{rgb}{0.121569,0.466667,0.705882}%
\pgfsetfillcolor{currentfill}%
\pgfsetlinewidth{0.000000pt}%
\definecolor{currentstroke}{rgb}{0.000000,0.000000,0.000000}%
\pgfsetstrokecolor{currentstroke}%
\pgfsetstrokeopacity{0.000000}%
\pgfsetdash{}{0pt}%
\pgfpathmoveto{\pgfqpoint{5.800000in}{41.237805in}}%
\pgfpathlineto{\pgfqpoint{5.800000in}{41.237805in}}%
\pgfpathlineto{\pgfqpoint{5.800000in}{41.340642in}}%
\pgfpathlineto{\pgfqpoint{5.800000in}{41.340642in}}%
\pgfpathclose%
\pgfusepath{fill}%
\end{pgfscope}%
\begin{pgfscope}%
\pgfpathrectangle{\pgfqpoint{5.800000in}{0.720000in}}{\pgfqpoint{1.400000in}{4.620000in}}%
\pgfusepath{clip}%
\pgfsetbuttcap%
\pgfsetmiterjoin%
\definecolor{currentfill}{rgb}{0.121569,0.466667,0.705882}%
\pgfsetfillcolor{currentfill}%
\pgfsetlinewidth{0.000000pt}%
\definecolor{currentstroke}{rgb}{0.000000,0.000000,0.000000}%
\pgfsetstrokecolor{currentstroke}%
\pgfsetstrokeopacity{0.000000}%
\pgfsetdash{}{0pt}%
\pgfpathmoveto{\pgfqpoint{5.800000in}{41.340642in}}%
\pgfpathlineto{\pgfqpoint{5.800000in}{41.340642in}}%
\pgfpathlineto{\pgfqpoint{5.800000in}{41.443479in}}%
\pgfpathlineto{\pgfqpoint{5.800000in}{41.443479in}}%
\pgfpathclose%
\pgfusepath{fill}%
\end{pgfscope}%
\begin{pgfscope}%
\pgfpathrectangle{\pgfqpoint{5.800000in}{0.720000in}}{\pgfqpoint{1.400000in}{4.620000in}}%
\pgfusepath{clip}%
\pgfsetbuttcap%
\pgfsetmiterjoin%
\definecolor{currentfill}{rgb}{0.121569,0.466667,0.705882}%
\pgfsetfillcolor{currentfill}%
\pgfsetlinewidth{0.000000pt}%
\definecolor{currentstroke}{rgb}{0.000000,0.000000,0.000000}%
\pgfsetstrokecolor{currentstroke}%
\pgfsetstrokeopacity{0.000000}%
\pgfsetdash{}{0pt}%
\pgfpathmoveto{\pgfqpoint{5.800000in}{41.443479in}}%
\pgfpathlineto{\pgfqpoint{5.800000in}{41.443479in}}%
\pgfpathlineto{\pgfqpoint{5.800000in}{41.546316in}}%
\pgfpathlineto{\pgfqpoint{5.800000in}{41.546316in}}%
\pgfpathclose%
\pgfusepath{fill}%
\end{pgfscope}%
\begin{pgfscope}%
\pgfpathrectangle{\pgfqpoint{5.800000in}{0.720000in}}{\pgfqpoint{1.400000in}{4.620000in}}%
\pgfusepath{clip}%
\pgfsetbuttcap%
\pgfsetmiterjoin%
\definecolor{currentfill}{rgb}{0.121569,0.466667,0.705882}%
\pgfsetfillcolor{currentfill}%
\pgfsetlinewidth{0.000000pt}%
\definecolor{currentstroke}{rgb}{0.000000,0.000000,0.000000}%
\pgfsetstrokecolor{currentstroke}%
\pgfsetstrokeopacity{0.000000}%
\pgfsetdash{}{0pt}%
\pgfpathmoveto{\pgfqpoint{5.800000in}{41.546316in}}%
\pgfpathlineto{\pgfqpoint{5.800000in}{41.546316in}}%
\pgfpathlineto{\pgfqpoint{5.800000in}{41.649153in}}%
\pgfpathlineto{\pgfqpoint{5.800000in}{41.649153in}}%
\pgfpathclose%
\pgfusepath{fill}%
\end{pgfscope}%
\begin{pgfscope}%
\pgfpathrectangle{\pgfqpoint{5.800000in}{0.720000in}}{\pgfqpoint{1.400000in}{4.620000in}}%
\pgfusepath{clip}%
\pgfsetbuttcap%
\pgfsetmiterjoin%
\definecolor{currentfill}{rgb}{0.121569,0.466667,0.705882}%
\pgfsetfillcolor{currentfill}%
\pgfsetlinewidth{0.000000pt}%
\definecolor{currentstroke}{rgb}{0.000000,0.000000,0.000000}%
\pgfsetstrokecolor{currentstroke}%
\pgfsetstrokeopacity{0.000000}%
\pgfsetdash{}{0pt}%
\pgfpathmoveto{\pgfqpoint{5.800000in}{41.649153in}}%
\pgfpathlineto{\pgfqpoint{5.800000in}{41.649153in}}%
\pgfpathlineto{\pgfqpoint{5.800000in}{41.751990in}}%
\pgfpathlineto{\pgfqpoint{5.800000in}{41.751990in}}%
\pgfpathclose%
\pgfusepath{fill}%
\end{pgfscope}%
\begin{pgfscope}%
\pgfpathrectangle{\pgfqpoint{5.800000in}{0.720000in}}{\pgfqpoint{1.400000in}{4.620000in}}%
\pgfusepath{clip}%
\pgfsetbuttcap%
\pgfsetmiterjoin%
\definecolor{currentfill}{rgb}{0.121569,0.466667,0.705882}%
\pgfsetfillcolor{currentfill}%
\pgfsetlinewidth{0.000000pt}%
\definecolor{currentstroke}{rgb}{0.000000,0.000000,0.000000}%
\pgfsetstrokecolor{currentstroke}%
\pgfsetstrokeopacity{0.000000}%
\pgfsetdash{}{0pt}%
\pgfpathmoveto{\pgfqpoint{5.800000in}{41.751990in}}%
\pgfpathlineto{\pgfqpoint{5.800000in}{41.751990in}}%
\pgfpathlineto{\pgfqpoint{5.800000in}{41.854827in}}%
\pgfpathlineto{\pgfqpoint{5.800000in}{41.854827in}}%
\pgfpathclose%
\pgfusepath{fill}%
\end{pgfscope}%
\begin{pgfscope}%
\pgfpathrectangle{\pgfqpoint{5.800000in}{0.720000in}}{\pgfqpoint{1.400000in}{4.620000in}}%
\pgfusepath{clip}%
\pgfsetbuttcap%
\pgfsetmiterjoin%
\definecolor{currentfill}{rgb}{0.121569,0.466667,0.705882}%
\pgfsetfillcolor{currentfill}%
\pgfsetlinewidth{0.000000pt}%
\definecolor{currentstroke}{rgb}{0.000000,0.000000,0.000000}%
\pgfsetstrokecolor{currentstroke}%
\pgfsetstrokeopacity{0.000000}%
\pgfsetdash{}{0pt}%
\pgfpathmoveto{\pgfqpoint{5.800000in}{41.854827in}}%
\pgfpathlineto{\pgfqpoint{5.800000in}{41.854827in}}%
\pgfpathlineto{\pgfqpoint{5.800000in}{41.957664in}}%
\pgfpathlineto{\pgfqpoint{5.800000in}{41.957664in}}%
\pgfpathclose%
\pgfusepath{fill}%
\end{pgfscope}%
\begin{pgfscope}%
\pgfpathrectangle{\pgfqpoint{5.800000in}{0.720000in}}{\pgfqpoint{1.400000in}{4.620000in}}%
\pgfusepath{clip}%
\pgfsetbuttcap%
\pgfsetmiterjoin%
\definecolor{currentfill}{rgb}{0.121569,0.466667,0.705882}%
\pgfsetfillcolor{currentfill}%
\pgfsetlinewidth{0.000000pt}%
\definecolor{currentstroke}{rgb}{0.000000,0.000000,0.000000}%
\pgfsetstrokecolor{currentstroke}%
\pgfsetstrokeopacity{0.000000}%
\pgfsetdash{}{0pt}%
\pgfpathmoveto{\pgfqpoint{5.800000in}{41.957664in}}%
\pgfpathlineto{\pgfqpoint{5.800000in}{41.957664in}}%
\pgfpathlineto{\pgfqpoint{5.800000in}{42.060501in}}%
\pgfpathlineto{\pgfqpoint{5.800000in}{42.060501in}}%
\pgfpathclose%
\pgfusepath{fill}%
\end{pgfscope}%
\begin{pgfscope}%
\pgfpathrectangle{\pgfqpoint{5.800000in}{0.720000in}}{\pgfqpoint{1.400000in}{4.620000in}}%
\pgfusepath{clip}%
\pgfsetbuttcap%
\pgfsetmiterjoin%
\definecolor{currentfill}{rgb}{0.121569,0.466667,0.705882}%
\pgfsetfillcolor{currentfill}%
\pgfsetlinewidth{0.000000pt}%
\definecolor{currentstroke}{rgb}{0.000000,0.000000,0.000000}%
\pgfsetstrokecolor{currentstroke}%
\pgfsetstrokeopacity{0.000000}%
\pgfsetdash{}{0pt}%
\pgfpathmoveto{\pgfqpoint{5.800000in}{42.060501in}}%
\pgfpathlineto{\pgfqpoint{5.800000in}{42.060501in}}%
\pgfpathlineto{\pgfqpoint{5.800000in}{42.163338in}}%
\pgfpathlineto{\pgfqpoint{5.800000in}{42.163338in}}%
\pgfpathclose%
\pgfusepath{fill}%
\end{pgfscope}%
\begin{pgfscope}%
\pgfpathrectangle{\pgfqpoint{5.800000in}{0.720000in}}{\pgfqpoint{1.400000in}{4.620000in}}%
\pgfusepath{clip}%
\pgfsetbuttcap%
\pgfsetmiterjoin%
\definecolor{currentfill}{rgb}{0.121569,0.466667,0.705882}%
\pgfsetfillcolor{currentfill}%
\pgfsetlinewidth{0.000000pt}%
\definecolor{currentstroke}{rgb}{0.000000,0.000000,0.000000}%
\pgfsetstrokecolor{currentstroke}%
\pgfsetstrokeopacity{0.000000}%
\pgfsetdash{}{0pt}%
\pgfpathmoveto{\pgfqpoint{5.800000in}{42.163338in}}%
\pgfpathlineto{\pgfqpoint{5.800000in}{42.163338in}}%
\pgfpathlineto{\pgfqpoint{5.800000in}{42.266175in}}%
\pgfpathlineto{\pgfqpoint{5.800000in}{42.266175in}}%
\pgfpathclose%
\pgfusepath{fill}%
\end{pgfscope}%
\begin{pgfscope}%
\pgfpathrectangle{\pgfqpoint{5.800000in}{0.720000in}}{\pgfqpoint{1.400000in}{4.620000in}}%
\pgfusepath{clip}%
\pgfsetbuttcap%
\pgfsetmiterjoin%
\definecolor{currentfill}{rgb}{0.121569,0.466667,0.705882}%
\pgfsetfillcolor{currentfill}%
\pgfsetlinewidth{0.000000pt}%
\definecolor{currentstroke}{rgb}{0.000000,0.000000,0.000000}%
\pgfsetstrokecolor{currentstroke}%
\pgfsetstrokeopacity{0.000000}%
\pgfsetdash{}{0pt}%
\pgfpathmoveto{\pgfqpoint{5.800000in}{42.266175in}}%
\pgfpathlineto{\pgfqpoint{5.800000in}{42.266175in}}%
\pgfpathlineto{\pgfqpoint{5.800000in}{42.369012in}}%
\pgfpathlineto{\pgfqpoint{5.800000in}{42.369012in}}%
\pgfpathclose%
\pgfusepath{fill}%
\end{pgfscope}%
\begin{pgfscope}%
\pgfpathrectangle{\pgfqpoint{5.800000in}{0.720000in}}{\pgfqpoint{1.400000in}{4.620000in}}%
\pgfusepath{clip}%
\pgfsetbuttcap%
\pgfsetmiterjoin%
\definecolor{currentfill}{rgb}{0.121569,0.466667,0.705882}%
\pgfsetfillcolor{currentfill}%
\pgfsetlinewidth{0.000000pt}%
\definecolor{currentstroke}{rgb}{0.000000,0.000000,0.000000}%
\pgfsetstrokecolor{currentstroke}%
\pgfsetstrokeopacity{0.000000}%
\pgfsetdash{}{0pt}%
\pgfpathmoveto{\pgfqpoint{5.800000in}{42.369012in}}%
\pgfpathlineto{\pgfqpoint{5.800000in}{42.369012in}}%
\pgfpathlineto{\pgfqpoint{5.800000in}{42.471850in}}%
\pgfpathlineto{\pgfqpoint{5.800000in}{42.471850in}}%
\pgfpathclose%
\pgfusepath{fill}%
\end{pgfscope}%
\begin{pgfscope}%
\pgfpathrectangle{\pgfqpoint{5.800000in}{0.720000in}}{\pgfqpoint{1.400000in}{4.620000in}}%
\pgfusepath{clip}%
\pgfsetbuttcap%
\pgfsetmiterjoin%
\definecolor{currentfill}{rgb}{0.121569,0.466667,0.705882}%
\pgfsetfillcolor{currentfill}%
\pgfsetlinewidth{0.000000pt}%
\definecolor{currentstroke}{rgb}{0.000000,0.000000,0.000000}%
\pgfsetstrokecolor{currentstroke}%
\pgfsetstrokeopacity{0.000000}%
\pgfsetdash{}{0pt}%
\pgfpathmoveto{\pgfqpoint{5.800000in}{42.471850in}}%
\pgfpathlineto{\pgfqpoint{5.800000in}{42.471850in}}%
\pgfpathlineto{\pgfqpoint{5.800000in}{42.574687in}}%
\pgfpathlineto{\pgfqpoint{5.800000in}{42.574687in}}%
\pgfpathclose%
\pgfusepath{fill}%
\end{pgfscope}%
\begin{pgfscope}%
\pgfpathrectangle{\pgfqpoint{5.800000in}{0.720000in}}{\pgfqpoint{1.400000in}{4.620000in}}%
\pgfusepath{clip}%
\pgfsetbuttcap%
\pgfsetmiterjoin%
\definecolor{currentfill}{rgb}{0.121569,0.466667,0.705882}%
\pgfsetfillcolor{currentfill}%
\pgfsetlinewidth{0.000000pt}%
\definecolor{currentstroke}{rgb}{0.000000,0.000000,0.000000}%
\pgfsetstrokecolor{currentstroke}%
\pgfsetstrokeopacity{0.000000}%
\pgfsetdash{}{0pt}%
\pgfpathmoveto{\pgfqpoint{5.800000in}{42.574687in}}%
\pgfpathlineto{\pgfqpoint{5.800000in}{42.574687in}}%
\pgfpathlineto{\pgfqpoint{5.800000in}{42.677524in}}%
\pgfpathlineto{\pgfqpoint{5.800000in}{42.677524in}}%
\pgfpathclose%
\pgfusepath{fill}%
\end{pgfscope}%
\begin{pgfscope}%
\pgfpathrectangle{\pgfqpoint{5.800000in}{0.720000in}}{\pgfqpoint{1.400000in}{4.620000in}}%
\pgfusepath{clip}%
\pgfsetbuttcap%
\pgfsetmiterjoin%
\definecolor{currentfill}{rgb}{0.121569,0.466667,0.705882}%
\pgfsetfillcolor{currentfill}%
\pgfsetlinewidth{0.000000pt}%
\definecolor{currentstroke}{rgb}{0.000000,0.000000,0.000000}%
\pgfsetstrokecolor{currentstroke}%
\pgfsetstrokeopacity{0.000000}%
\pgfsetdash{}{0pt}%
\pgfpathmoveto{\pgfqpoint{5.800000in}{42.677524in}}%
\pgfpathlineto{\pgfqpoint{5.800000in}{42.677524in}}%
\pgfpathlineto{\pgfqpoint{5.800000in}{42.780361in}}%
\pgfpathlineto{\pgfqpoint{5.800000in}{42.780361in}}%
\pgfpathclose%
\pgfusepath{fill}%
\end{pgfscope}%
\begin{pgfscope}%
\pgfpathrectangle{\pgfqpoint{5.800000in}{0.720000in}}{\pgfqpoint{1.400000in}{4.620000in}}%
\pgfusepath{clip}%
\pgfsetbuttcap%
\pgfsetmiterjoin%
\definecolor{currentfill}{rgb}{0.121569,0.466667,0.705882}%
\pgfsetfillcolor{currentfill}%
\pgfsetlinewidth{0.000000pt}%
\definecolor{currentstroke}{rgb}{0.000000,0.000000,0.000000}%
\pgfsetstrokecolor{currentstroke}%
\pgfsetstrokeopacity{0.000000}%
\pgfsetdash{}{0pt}%
\pgfpathmoveto{\pgfqpoint{5.800000in}{42.780361in}}%
\pgfpathlineto{\pgfqpoint{5.800000in}{42.780361in}}%
\pgfpathlineto{\pgfqpoint{5.800000in}{42.883198in}}%
\pgfpathlineto{\pgfqpoint{5.800000in}{42.883198in}}%
\pgfpathclose%
\pgfusepath{fill}%
\end{pgfscope}%
\begin{pgfscope}%
\pgfpathrectangle{\pgfqpoint{5.800000in}{0.720000in}}{\pgfqpoint{1.400000in}{4.620000in}}%
\pgfusepath{clip}%
\pgfsetbuttcap%
\pgfsetmiterjoin%
\definecolor{currentfill}{rgb}{0.121569,0.466667,0.705882}%
\pgfsetfillcolor{currentfill}%
\pgfsetlinewidth{0.000000pt}%
\definecolor{currentstroke}{rgb}{0.000000,0.000000,0.000000}%
\pgfsetstrokecolor{currentstroke}%
\pgfsetstrokeopacity{0.000000}%
\pgfsetdash{}{0pt}%
\pgfpathmoveto{\pgfqpoint{5.800000in}{42.883198in}}%
\pgfpathlineto{\pgfqpoint{5.800000in}{42.883198in}}%
\pgfpathlineto{\pgfqpoint{5.800000in}{42.986035in}}%
\pgfpathlineto{\pgfqpoint{5.800000in}{42.986035in}}%
\pgfpathclose%
\pgfusepath{fill}%
\end{pgfscope}%
\begin{pgfscope}%
\pgfpathrectangle{\pgfqpoint{5.800000in}{0.720000in}}{\pgfqpoint{1.400000in}{4.620000in}}%
\pgfusepath{clip}%
\pgfsetbuttcap%
\pgfsetmiterjoin%
\definecolor{currentfill}{rgb}{0.121569,0.466667,0.705882}%
\pgfsetfillcolor{currentfill}%
\pgfsetlinewidth{0.000000pt}%
\definecolor{currentstroke}{rgb}{0.000000,0.000000,0.000000}%
\pgfsetstrokecolor{currentstroke}%
\pgfsetstrokeopacity{0.000000}%
\pgfsetdash{}{0pt}%
\pgfpathmoveto{\pgfqpoint{5.800000in}{42.986035in}}%
\pgfpathlineto{\pgfqpoint{5.800000in}{42.986035in}}%
\pgfpathlineto{\pgfqpoint{5.800000in}{43.088872in}}%
\pgfpathlineto{\pgfqpoint{5.800000in}{43.088872in}}%
\pgfpathclose%
\pgfusepath{fill}%
\end{pgfscope}%
\begin{pgfscope}%
\pgfpathrectangle{\pgfqpoint{5.800000in}{0.720000in}}{\pgfqpoint{1.400000in}{4.620000in}}%
\pgfusepath{clip}%
\pgfsetbuttcap%
\pgfsetmiterjoin%
\definecolor{currentfill}{rgb}{0.121569,0.466667,0.705882}%
\pgfsetfillcolor{currentfill}%
\pgfsetlinewidth{0.000000pt}%
\definecolor{currentstroke}{rgb}{0.000000,0.000000,0.000000}%
\pgfsetstrokecolor{currentstroke}%
\pgfsetstrokeopacity{0.000000}%
\pgfsetdash{}{0pt}%
\pgfpathmoveto{\pgfqpoint{5.800000in}{43.088872in}}%
\pgfpathlineto{\pgfqpoint{5.800000in}{43.088872in}}%
\pgfpathlineto{\pgfqpoint{5.800000in}{43.191709in}}%
\pgfpathlineto{\pgfqpoint{5.800000in}{43.191709in}}%
\pgfpathclose%
\pgfusepath{fill}%
\end{pgfscope}%
\begin{pgfscope}%
\pgfpathrectangle{\pgfqpoint{5.800000in}{0.720000in}}{\pgfqpoint{1.400000in}{4.620000in}}%
\pgfusepath{clip}%
\pgfsetbuttcap%
\pgfsetmiterjoin%
\definecolor{currentfill}{rgb}{0.121569,0.466667,0.705882}%
\pgfsetfillcolor{currentfill}%
\pgfsetlinewidth{0.000000pt}%
\definecolor{currentstroke}{rgb}{0.000000,0.000000,0.000000}%
\pgfsetstrokecolor{currentstroke}%
\pgfsetstrokeopacity{0.000000}%
\pgfsetdash{}{0pt}%
\pgfpathmoveto{\pgfqpoint{5.800000in}{43.191709in}}%
\pgfpathlineto{\pgfqpoint{5.800000in}{43.191709in}}%
\pgfpathlineto{\pgfqpoint{5.800000in}{43.294546in}}%
\pgfpathlineto{\pgfqpoint{5.800000in}{43.294546in}}%
\pgfpathclose%
\pgfusepath{fill}%
\end{pgfscope}%
\begin{pgfscope}%
\pgfpathrectangle{\pgfqpoint{5.800000in}{0.720000in}}{\pgfqpoint{1.400000in}{4.620000in}}%
\pgfusepath{clip}%
\pgfsetbuttcap%
\pgfsetmiterjoin%
\definecolor{currentfill}{rgb}{0.121569,0.466667,0.705882}%
\pgfsetfillcolor{currentfill}%
\pgfsetlinewidth{0.000000pt}%
\definecolor{currentstroke}{rgb}{0.000000,0.000000,0.000000}%
\pgfsetstrokecolor{currentstroke}%
\pgfsetstrokeopacity{0.000000}%
\pgfsetdash{}{0pt}%
\pgfpathmoveto{\pgfqpoint{5.800000in}{43.294546in}}%
\pgfpathlineto{\pgfqpoint{5.800000in}{43.294546in}}%
\pgfpathlineto{\pgfqpoint{5.800000in}{43.397383in}}%
\pgfpathlineto{\pgfqpoint{5.800000in}{43.397383in}}%
\pgfpathclose%
\pgfusepath{fill}%
\end{pgfscope}%
\begin{pgfscope}%
\pgfpathrectangle{\pgfqpoint{5.800000in}{0.720000in}}{\pgfqpoint{1.400000in}{4.620000in}}%
\pgfusepath{clip}%
\pgfsetbuttcap%
\pgfsetmiterjoin%
\definecolor{currentfill}{rgb}{0.121569,0.466667,0.705882}%
\pgfsetfillcolor{currentfill}%
\pgfsetlinewidth{0.000000pt}%
\definecolor{currentstroke}{rgb}{0.000000,0.000000,0.000000}%
\pgfsetstrokecolor{currentstroke}%
\pgfsetstrokeopacity{0.000000}%
\pgfsetdash{}{0pt}%
\pgfpathmoveto{\pgfqpoint{5.800000in}{43.397383in}}%
\pgfpathlineto{\pgfqpoint{5.800000in}{43.397383in}}%
\pgfpathlineto{\pgfqpoint{5.800000in}{43.500220in}}%
\pgfpathlineto{\pgfqpoint{5.800000in}{43.500220in}}%
\pgfpathclose%
\pgfusepath{fill}%
\end{pgfscope}%
\begin{pgfscope}%
\pgfpathrectangle{\pgfqpoint{5.800000in}{0.720000in}}{\pgfqpoint{1.400000in}{4.620000in}}%
\pgfusepath{clip}%
\pgfsetbuttcap%
\pgfsetmiterjoin%
\definecolor{currentfill}{rgb}{0.121569,0.466667,0.705882}%
\pgfsetfillcolor{currentfill}%
\pgfsetlinewidth{0.000000pt}%
\definecolor{currentstroke}{rgb}{0.000000,0.000000,0.000000}%
\pgfsetstrokecolor{currentstroke}%
\pgfsetstrokeopacity{0.000000}%
\pgfsetdash{}{0pt}%
\pgfpathmoveto{\pgfqpoint{5.800000in}{43.500220in}}%
\pgfpathlineto{\pgfqpoint{5.800000in}{43.500220in}}%
\pgfpathlineto{\pgfqpoint{5.800000in}{43.603057in}}%
\pgfpathlineto{\pgfqpoint{5.800000in}{43.603057in}}%
\pgfpathclose%
\pgfusepath{fill}%
\end{pgfscope}%
\begin{pgfscope}%
\pgfpathrectangle{\pgfqpoint{5.800000in}{0.720000in}}{\pgfqpoint{1.400000in}{4.620000in}}%
\pgfusepath{clip}%
\pgfsetbuttcap%
\pgfsetmiterjoin%
\definecolor{currentfill}{rgb}{0.121569,0.466667,0.705882}%
\pgfsetfillcolor{currentfill}%
\pgfsetlinewidth{0.000000pt}%
\definecolor{currentstroke}{rgb}{0.000000,0.000000,0.000000}%
\pgfsetstrokecolor{currentstroke}%
\pgfsetstrokeopacity{0.000000}%
\pgfsetdash{}{0pt}%
\pgfpathmoveto{\pgfqpoint{5.800000in}{43.603057in}}%
\pgfpathlineto{\pgfqpoint{5.800000in}{43.603057in}}%
\pgfpathlineto{\pgfqpoint{5.800000in}{43.705894in}}%
\pgfpathlineto{\pgfqpoint{5.800000in}{43.705894in}}%
\pgfpathclose%
\pgfusepath{fill}%
\end{pgfscope}%
\begin{pgfscope}%
\pgfpathrectangle{\pgfqpoint{5.800000in}{0.720000in}}{\pgfqpoint{1.400000in}{4.620000in}}%
\pgfusepath{clip}%
\pgfsetbuttcap%
\pgfsetmiterjoin%
\definecolor{currentfill}{rgb}{0.121569,0.466667,0.705882}%
\pgfsetfillcolor{currentfill}%
\pgfsetlinewidth{0.000000pt}%
\definecolor{currentstroke}{rgb}{0.000000,0.000000,0.000000}%
\pgfsetstrokecolor{currentstroke}%
\pgfsetstrokeopacity{0.000000}%
\pgfsetdash{}{0pt}%
\pgfpathmoveto{\pgfqpoint{5.800000in}{43.705894in}}%
\pgfpathlineto{\pgfqpoint{5.800000in}{43.705894in}}%
\pgfpathlineto{\pgfqpoint{5.800000in}{43.808731in}}%
\pgfpathlineto{\pgfqpoint{5.800000in}{43.808731in}}%
\pgfpathclose%
\pgfusepath{fill}%
\end{pgfscope}%
\begin{pgfscope}%
\pgfpathrectangle{\pgfqpoint{5.800000in}{0.720000in}}{\pgfqpoint{1.400000in}{4.620000in}}%
\pgfusepath{clip}%
\pgfsetbuttcap%
\pgfsetmiterjoin%
\definecolor{currentfill}{rgb}{0.121569,0.466667,0.705882}%
\pgfsetfillcolor{currentfill}%
\pgfsetlinewidth{0.000000pt}%
\definecolor{currentstroke}{rgb}{0.000000,0.000000,0.000000}%
\pgfsetstrokecolor{currentstroke}%
\pgfsetstrokeopacity{0.000000}%
\pgfsetdash{}{0pt}%
\pgfpathmoveto{\pgfqpoint{5.800000in}{43.808731in}}%
\pgfpathlineto{\pgfqpoint{5.800000in}{43.808731in}}%
\pgfpathlineto{\pgfqpoint{5.800000in}{43.911568in}}%
\pgfpathlineto{\pgfqpoint{5.800000in}{43.911568in}}%
\pgfpathclose%
\pgfusepath{fill}%
\end{pgfscope}%
\begin{pgfscope}%
\pgfpathrectangle{\pgfqpoint{5.800000in}{0.720000in}}{\pgfqpoint{1.400000in}{4.620000in}}%
\pgfusepath{clip}%
\pgfsetbuttcap%
\pgfsetmiterjoin%
\definecolor{currentfill}{rgb}{0.121569,0.466667,0.705882}%
\pgfsetfillcolor{currentfill}%
\pgfsetlinewidth{0.000000pt}%
\definecolor{currentstroke}{rgb}{0.000000,0.000000,0.000000}%
\pgfsetstrokecolor{currentstroke}%
\pgfsetstrokeopacity{0.000000}%
\pgfsetdash{}{0pt}%
\pgfpathmoveto{\pgfqpoint{5.800000in}{43.911568in}}%
\pgfpathlineto{\pgfqpoint{5.800000in}{43.911568in}}%
\pgfpathlineto{\pgfqpoint{5.800000in}{44.014406in}}%
\pgfpathlineto{\pgfqpoint{5.800000in}{44.014406in}}%
\pgfpathclose%
\pgfusepath{fill}%
\end{pgfscope}%
\begin{pgfscope}%
\pgfpathrectangle{\pgfqpoint{5.800000in}{0.720000in}}{\pgfqpoint{1.400000in}{4.620000in}}%
\pgfusepath{clip}%
\pgfsetbuttcap%
\pgfsetmiterjoin%
\definecolor{currentfill}{rgb}{0.121569,0.466667,0.705882}%
\pgfsetfillcolor{currentfill}%
\pgfsetlinewidth{0.000000pt}%
\definecolor{currentstroke}{rgb}{0.000000,0.000000,0.000000}%
\pgfsetstrokecolor{currentstroke}%
\pgfsetstrokeopacity{0.000000}%
\pgfsetdash{}{0pt}%
\pgfpathmoveto{\pgfqpoint{5.800000in}{44.014406in}}%
\pgfpathlineto{\pgfqpoint{5.800000in}{44.014406in}}%
\pgfpathlineto{\pgfqpoint{5.800000in}{44.117243in}}%
\pgfpathlineto{\pgfqpoint{5.800000in}{44.117243in}}%
\pgfpathclose%
\pgfusepath{fill}%
\end{pgfscope}%
\begin{pgfscope}%
\pgfpathrectangle{\pgfqpoint{5.800000in}{0.720000in}}{\pgfqpoint{1.400000in}{4.620000in}}%
\pgfusepath{clip}%
\pgfsetbuttcap%
\pgfsetmiterjoin%
\definecolor{currentfill}{rgb}{0.121569,0.466667,0.705882}%
\pgfsetfillcolor{currentfill}%
\pgfsetlinewidth{0.000000pt}%
\definecolor{currentstroke}{rgb}{0.000000,0.000000,0.000000}%
\pgfsetstrokecolor{currentstroke}%
\pgfsetstrokeopacity{0.000000}%
\pgfsetdash{}{0pt}%
\pgfpathmoveto{\pgfqpoint{5.800000in}{44.117243in}}%
\pgfpathlineto{\pgfqpoint{5.800000in}{44.117243in}}%
\pgfpathlineto{\pgfqpoint{5.800000in}{44.220080in}}%
\pgfpathlineto{\pgfqpoint{5.800000in}{44.220080in}}%
\pgfpathclose%
\pgfusepath{fill}%
\end{pgfscope}%
\begin{pgfscope}%
\pgfpathrectangle{\pgfqpoint{5.800000in}{0.720000in}}{\pgfqpoint{1.400000in}{4.620000in}}%
\pgfusepath{clip}%
\pgfsetbuttcap%
\pgfsetmiterjoin%
\definecolor{currentfill}{rgb}{0.121569,0.466667,0.705882}%
\pgfsetfillcolor{currentfill}%
\pgfsetlinewidth{0.000000pt}%
\definecolor{currentstroke}{rgb}{0.000000,0.000000,0.000000}%
\pgfsetstrokecolor{currentstroke}%
\pgfsetstrokeopacity{0.000000}%
\pgfsetdash{}{0pt}%
\pgfpathmoveto{\pgfqpoint{5.800000in}{44.220080in}}%
\pgfpathlineto{\pgfqpoint{5.800000in}{44.220080in}}%
\pgfpathlineto{\pgfqpoint{5.800000in}{44.322917in}}%
\pgfpathlineto{\pgfqpoint{5.800000in}{44.322917in}}%
\pgfpathclose%
\pgfusepath{fill}%
\end{pgfscope}%
\begin{pgfscope}%
\pgfpathrectangle{\pgfqpoint{5.800000in}{0.720000in}}{\pgfqpoint{1.400000in}{4.620000in}}%
\pgfusepath{clip}%
\pgfsetbuttcap%
\pgfsetmiterjoin%
\definecolor{currentfill}{rgb}{0.121569,0.466667,0.705882}%
\pgfsetfillcolor{currentfill}%
\pgfsetlinewidth{0.000000pt}%
\definecolor{currentstroke}{rgb}{0.000000,0.000000,0.000000}%
\pgfsetstrokecolor{currentstroke}%
\pgfsetstrokeopacity{0.000000}%
\pgfsetdash{}{0pt}%
\pgfpathmoveto{\pgfqpoint{5.800000in}{44.322917in}}%
\pgfpathlineto{\pgfqpoint{5.800000in}{44.322917in}}%
\pgfpathlineto{\pgfqpoint{5.800000in}{44.425754in}}%
\pgfpathlineto{\pgfqpoint{5.800000in}{44.425754in}}%
\pgfpathclose%
\pgfusepath{fill}%
\end{pgfscope}%
\begin{pgfscope}%
\pgfpathrectangle{\pgfqpoint{5.800000in}{0.720000in}}{\pgfqpoint{1.400000in}{4.620000in}}%
\pgfusepath{clip}%
\pgfsetbuttcap%
\pgfsetmiterjoin%
\definecolor{currentfill}{rgb}{0.121569,0.466667,0.705882}%
\pgfsetfillcolor{currentfill}%
\pgfsetlinewidth{0.000000pt}%
\definecolor{currentstroke}{rgb}{0.000000,0.000000,0.000000}%
\pgfsetstrokecolor{currentstroke}%
\pgfsetstrokeopacity{0.000000}%
\pgfsetdash{}{0pt}%
\pgfpathmoveto{\pgfqpoint{5.800000in}{44.425754in}}%
\pgfpathlineto{\pgfqpoint{5.800000in}{44.425754in}}%
\pgfpathlineto{\pgfqpoint{5.800000in}{44.528591in}}%
\pgfpathlineto{\pgfqpoint{5.800000in}{44.528591in}}%
\pgfpathclose%
\pgfusepath{fill}%
\end{pgfscope}%
\begin{pgfscope}%
\pgfpathrectangle{\pgfqpoint{5.800000in}{0.720000in}}{\pgfqpoint{1.400000in}{4.620000in}}%
\pgfusepath{clip}%
\pgfsetbuttcap%
\pgfsetmiterjoin%
\definecolor{currentfill}{rgb}{0.121569,0.466667,0.705882}%
\pgfsetfillcolor{currentfill}%
\pgfsetlinewidth{0.000000pt}%
\definecolor{currentstroke}{rgb}{0.000000,0.000000,0.000000}%
\pgfsetstrokecolor{currentstroke}%
\pgfsetstrokeopacity{0.000000}%
\pgfsetdash{}{0pt}%
\pgfpathmoveto{\pgfqpoint{5.800000in}{44.528591in}}%
\pgfpathlineto{\pgfqpoint{5.800000in}{44.528591in}}%
\pgfpathlineto{\pgfqpoint{5.800000in}{44.631428in}}%
\pgfpathlineto{\pgfqpoint{5.800000in}{44.631428in}}%
\pgfpathclose%
\pgfusepath{fill}%
\end{pgfscope}%
\begin{pgfscope}%
\pgfpathrectangle{\pgfqpoint{5.800000in}{0.720000in}}{\pgfqpoint{1.400000in}{4.620000in}}%
\pgfusepath{clip}%
\pgfsetbuttcap%
\pgfsetmiterjoin%
\definecolor{currentfill}{rgb}{0.121569,0.466667,0.705882}%
\pgfsetfillcolor{currentfill}%
\pgfsetlinewidth{0.000000pt}%
\definecolor{currentstroke}{rgb}{0.000000,0.000000,0.000000}%
\pgfsetstrokecolor{currentstroke}%
\pgfsetstrokeopacity{0.000000}%
\pgfsetdash{}{0pt}%
\pgfpathmoveto{\pgfqpoint{5.800000in}{44.631428in}}%
\pgfpathlineto{\pgfqpoint{5.800000in}{44.631428in}}%
\pgfpathlineto{\pgfqpoint{5.800000in}{44.734265in}}%
\pgfpathlineto{\pgfqpoint{5.800000in}{44.734265in}}%
\pgfpathclose%
\pgfusepath{fill}%
\end{pgfscope}%
\begin{pgfscope}%
\pgfpathrectangle{\pgfqpoint{5.800000in}{0.720000in}}{\pgfqpoint{1.400000in}{4.620000in}}%
\pgfusepath{clip}%
\pgfsetbuttcap%
\pgfsetmiterjoin%
\definecolor{currentfill}{rgb}{0.121569,0.466667,0.705882}%
\pgfsetfillcolor{currentfill}%
\pgfsetlinewidth{0.000000pt}%
\definecolor{currentstroke}{rgb}{0.000000,0.000000,0.000000}%
\pgfsetstrokecolor{currentstroke}%
\pgfsetstrokeopacity{0.000000}%
\pgfsetdash{}{0pt}%
\pgfpathmoveto{\pgfqpoint{5.800000in}{44.734265in}}%
\pgfpathlineto{\pgfqpoint{5.800000in}{44.734265in}}%
\pgfpathlineto{\pgfqpoint{5.800000in}{44.837102in}}%
\pgfpathlineto{\pgfqpoint{5.800000in}{44.837102in}}%
\pgfpathclose%
\pgfusepath{fill}%
\end{pgfscope}%
\begin{pgfscope}%
\pgfpathrectangle{\pgfqpoint{5.800000in}{0.720000in}}{\pgfqpoint{1.400000in}{4.620000in}}%
\pgfusepath{clip}%
\pgfsetbuttcap%
\pgfsetmiterjoin%
\definecolor{currentfill}{rgb}{0.121569,0.466667,0.705882}%
\pgfsetfillcolor{currentfill}%
\pgfsetlinewidth{0.000000pt}%
\definecolor{currentstroke}{rgb}{0.000000,0.000000,0.000000}%
\pgfsetstrokecolor{currentstroke}%
\pgfsetstrokeopacity{0.000000}%
\pgfsetdash{}{0pt}%
\pgfpathmoveto{\pgfqpoint{5.800000in}{44.837102in}}%
\pgfpathlineto{\pgfqpoint{5.800000in}{44.837102in}}%
\pgfpathlineto{\pgfqpoint{5.800000in}{44.939939in}}%
\pgfpathlineto{\pgfqpoint{5.800000in}{44.939939in}}%
\pgfpathclose%
\pgfusepath{fill}%
\end{pgfscope}%
\begin{pgfscope}%
\pgfpathrectangle{\pgfqpoint{5.800000in}{0.720000in}}{\pgfqpoint{1.400000in}{4.620000in}}%
\pgfusepath{clip}%
\pgfsetbuttcap%
\pgfsetmiterjoin%
\definecolor{currentfill}{rgb}{0.121569,0.466667,0.705882}%
\pgfsetfillcolor{currentfill}%
\pgfsetlinewidth{0.000000pt}%
\definecolor{currentstroke}{rgb}{0.000000,0.000000,0.000000}%
\pgfsetstrokecolor{currentstroke}%
\pgfsetstrokeopacity{0.000000}%
\pgfsetdash{}{0pt}%
\pgfpathmoveto{\pgfqpoint{5.800000in}{44.939939in}}%
\pgfpathlineto{\pgfqpoint{5.800000in}{44.939939in}}%
\pgfpathlineto{\pgfqpoint{5.800000in}{45.042776in}}%
\pgfpathlineto{\pgfqpoint{5.800000in}{45.042776in}}%
\pgfpathclose%
\pgfusepath{fill}%
\end{pgfscope}%
\begin{pgfscope}%
\pgfpathrectangle{\pgfqpoint{5.800000in}{0.720000in}}{\pgfqpoint{1.400000in}{4.620000in}}%
\pgfusepath{clip}%
\pgfsetbuttcap%
\pgfsetmiterjoin%
\definecolor{currentfill}{rgb}{0.121569,0.466667,0.705882}%
\pgfsetfillcolor{currentfill}%
\pgfsetlinewidth{0.000000pt}%
\definecolor{currentstroke}{rgb}{0.000000,0.000000,0.000000}%
\pgfsetstrokecolor{currentstroke}%
\pgfsetstrokeopacity{0.000000}%
\pgfsetdash{}{0pt}%
\pgfpathmoveto{\pgfqpoint{5.800000in}{45.042776in}}%
\pgfpathlineto{\pgfqpoint{5.800000in}{45.042776in}}%
\pgfpathlineto{\pgfqpoint{5.800000in}{45.145613in}}%
\pgfpathlineto{\pgfqpoint{5.800000in}{45.145613in}}%
\pgfpathclose%
\pgfusepath{fill}%
\end{pgfscope}%
\begin{pgfscope}%
\pgfpathrectangle{\pgfqpoint{5.800000in}{0.720000in}}{\pgfqpoint{1.400000in}{4.620000in}}%
\pgfusepath{clip}%
\pgfsetbuttcap%
\pgfsetmiterjoin%
\definecolor{currentfill}{rgb}{0.121569,0.466667,0.705882}%
\pgfsetfillcolor{currentfill}%
\pgfsetlinewidth{0.000000pt}%
\definecolor{currentstroke}{rgb}{0.000000,0.000000,0.000000}%
\pgfsetstrokecolor{currentstroke}%
\pgfsetstrokeopacity{0.000000}%
\pgfsetdash{}{0pt}%
\pgfpathmoveto{\pgfqpoint{5.800000in}{45.145613in}}%
\pgfpathlineto{\pgfqpoint{5.800000in}{45.145613in}}%
\pgfpathlineto{\pgfqpoint{5.800000in}{45.248450in}}%
\pgfpathlineto{\pgfqpoint{5.800000in}{45.248450in}}%
\pgfpathclose%
\pgfusepath{fill}%
\end{pgfscope}%
\begin{pgfscope}%
\pgfpathrectangle{\pgfqpoint{5.800000in}{0.720000in}}{\pgfqpoint{1.400000in}{4.620000in}}%
\pgfusepath{clip}%
\pgfsetbuttcap%
\pgfsetmiterjoin%
\definecolor{currentfill}{rgb}{0.121569,0.466667,0.705882}%
\pgfsetfillcolor{currentfill}%
\pgfsetlinewidth{0.000000pt}%
\definecolor{currentstroke}{rgb}{0.000000,0.000000,0.000000}%
\pgfsetstrokecolor{currentstroke}%
\pgfsetstrokeopacity{0.000000}%
\pgfsetdash{}{0pt}%
\pgfpathmoveto{\pgfqpoint{5.800000in}{45.248450in}}%
\pgfpathlineto{\pgfqpoint{5.800000in}{45.248450in}}%
\pgfpathlineto{\pgfqpoint{5.800000in}{45.351287in}}%
\pgfpathlineto{\pgfqpoint{5.800000in}{45.351287in}}%
\pgfpathclose%
\pgfusepath{fill}%
\end{pgfscope}%
\begin{pgfscope}%
\pgfpathrectangle{\pgfqpoint{5.800000in}{0.720000in}}{\pgfqpoint{1.400000in}{4.620000in}}%
\pgfusepath{clip}%
\pgfsetbuttcap%
\pgfsetmiterjoin%
\definecolor{currentfill}{rgb}{0.121569,0.466667,0.705882}%
\pgfsetfillcolor{currentfill}%
\pgfsetlinewidth{0.000000pt}%
\definecolor{currentstroke}{rgb}{0.000000,0.000000,0.000000}%
\pgfsetstrokecolor{currentstroke}%
\pgfsetstrokeopacity{0.000000}%
\pgfsetdash{}{0pt}%
\pgfpathmoveto{\pgfqpoint{5.800000in}{45.351287in}}%
\pgfpathlineto{\pgfqpoint{5.800000in}{45.351287in}}%
\pgfpathlineto{\pgfqpoint{5.800000in}{45.454124in}}%
\pgfpathlineto{\pgfqpoint{5.800000in}{45.454124in}}%
\pgfpathclose%
\pgfusepath{fill}%
\end{pgfscope}%
\begin{pgfscope}%
\pgfpathrectangle{\pgfqpoint{5.800000in}{0.720000in}}{\pgfqpoint{1.400000in}{4.620000in}}%
\pgfusepath{clip}%
\pgfsetbuttcap%
\pgfsetmiterjoin%
\definecolor{currentfill}{rgb}{0.121569,0.466667,0.705882}%
\pgfsetfillcolor{currentfill}%
\pgfsetlinewidth{0.000000pt}%
\definecolor{currentstroke}{rgb}{0.000000,0.000000,0.000000}%
\pgfsetstrokecolor{currentstroke}%
\pgfsetstrokeopacity{0.000000}%
\pgfsetdash{}{0pt}%
\pgfpathmoveto{\pgfqpoint{5.800000in}{45.454124in}}%
\pgfpathlineto{\pgfqpoint{5.800000in}{45.454124in}}%
\pgfpathlineto{\pgfqpoint{5.800000in}{45.556962in}}%
\pgfpathlineto{\pgfqpoint{5.800000in}{45.556962in}}%
\pgfpathclose%
\pgfusepath{fill}%
\end{pgfscope}%
\begin{pgfscope}%
\pgfpathrectangle{\pgfqpoint{5.800000in}{0.720000in}}{\pgfqpoint{1.400000in}{4.620000in}}%
\pgfusepath{clip}%
\pgfsetbuttcap%
\pgfsetmiterjoin%
\definecolor{currentfill}{rgb}{0.121569,0.466667,0.705882}%
\pgfsetfillcolor{currentfill}%
\pgfsetlinewidth{0.000000pt}%
\definecolor{currentstroke}{rgb}{0.000000,0.000000,0.000000}%
\pgfsetstrokecolor{currentstroke}%
\pgfsetstrokeopacity{0.000000}%
\pgfsetdash{}{0pt}%
\pgfpathmoveto{\pgfqpoint{5.800000in}{45.556962in}}%
\pgfpathlineto{\pgfqpoint{5.800000in}{45.556962in}}%
\pgfpathlineto{\pgfqpoint{5.800000in}{45.659799in}}%
\pgfpathlineto{\pgfqpoint{5.800000in}{45.659799in}}%
\pgfpathclose%
\pgfusepath{fill}%
\end{pgfscope}%
\begin{pgfscope}%
\pgfpathrectangle{\pgfqpoint{5.800000in}{0.720000in}}{\pgfqpoint{1.400000in}{4.620000in}}%
\pgfusepath{clip}%
\pgfsetbuttcap%
\pgfsetmiterjoin%
\definecolor{currentfill}{rgb}{0.121569,0.466667,0.705882}%
\pgfsetfillcolor{currentfill}%
\pgfsetlinewidth{0.000000pt}%
\definecolor{currentstroke}{rgb}{0.000000,0.000000,0.000000}%
\pgfsetstrokecolor{currentstroke}%
\pgfsetstrokeopacity{0.000000}%
\pgfsetdash{}{0pt}%
\pgfpathmoveto{\pgfqpoint{5.800000in}{45.659799in}}%
\pgfpathlineto{\pgfqpoint{5.800000in}{45.659799in}}%
\pgfpathlineto{\pgfqpoint{5.800000in}{45.762636in}}%
\pgfpathlineto{\pgfqpoint{5.800000in}{45.762636in}}%
\pgfpathclose%
\pgfusepath{fill}%
\end{pgfscope}%
\begin{pgfscope}%
\pgfpathrectangle{\pgfqpoint{5.800000in}{0.720000in}}{\pgfqpoint{1.400000in}{4.620000in}}%
\pgfusepath{clip}%
\pgfsetbuttcap%
\pgfsetmiterjoin%
\definecolor{currentfill}{rgb}{0.121569,0.466667,0.705882}%
\pgfsetfillcolor{currentfill}%
\pgfsetlinewidth{0.000000pt}%
\definecolor{currentstroke}{rgb}{0.000000,0.000000,0.000000}%
\pgfsetstrokecolor{currentstroke}%
\pgfsetstrokeopacity{0.000000}%
\pgfsetdash{}{0pt}%
\pgfpathmoveto{\pgfqpoint{5.800000in}{45.762636in}}%
\pgfpathlineto{\pgfqpoint{5.800000in}{45.762636in}}%
\pgfpathlineto{\pgfqpoint{5.800000in}{45.865473in}}%
\pgfpathlineto{\pgfqpoint{5.800000in}{45.865473in}}%
\pgfpathclose%
\pgfusepath{fill}%
\end{pgfscope}%
\begin{pgfscope}%
\pgfpathrectangle{\pgfqpoint{5.800000in}{0.720000in}}{\pgfqpoint{1.400000in}{4.620000in}}%
\pgfusepath{clip}%
\pgfsetbuttcap%
\pgfsetmiterjoin%
\definecolor{currentfill}{rgb}{0.121569,0.466667,0.705882}%
\pgfsetfillcolor{currentfill}%
\pgfsetlinewidth{0.000000pt}%
\definecolor{currentstroke}{rgb}{0.000000,0.000000,0.000000}%
\pgfsetstrokecolor{currentstroke}%
\pgfsetstrokeopacity{0.000000}%
\pgfsetdash{}{0pt}%
\pgfpathmoveto{\pgfqpoint{5.800000in}{45.865473in}}%
\pgfpathlineto{\pgfqpoint{5.800000in}{45.865473in}}%
\pgfpathlineto{\pgfqpoint{5.800000in}{45.968310in}}%
\pgfpathlineto{\pgfqpoint{5.800000in}{45.968310in}}%
\pgfpathclose%
\pgfusepath{fill}%
\end{pgfscope}%
\begin{pgfscope}%
\pgfpathrectangle{\pgfqpoint{5.800000in}{0.720000in}}{\pgfqpoint{1.400000in}{4.620000in}}%
\pgfusepath{clip}%
\pgfsetbuttcap%
\pgfsetmiterjoin%
\definecolor{currentfill}{rgb}{0.121569,0.466667,0.705882}%
\pgfsetfillcolor{currentfill}%
\pgfsetlinewidth{0.000000pt}%
\definecolor{currentstroke}{rgb}{0.000000,0.000000,0.000000}%
\pgfsetstrokecolor{currentstroke}%
\pgfsetstrokeopacity{0.000000}%
\pgfsetdash{}{0pt}%
\pgfpathmoveto{\pgfqpoint{5.800000in}{45.968310in}}%
\pgfpathlineto{\pgfqpoint{5.800000in}{45.968310in}}%
\pgfpathlineto{\pgfqpoint{5.800000in}{46.071147in}}%
\pgfpathlineto{\pgfqpoint{5.800000in}{46.071147in}}%
\pgfpathclose%
\pgfusepath{fill}%
\end{pgfscope}%
\begin{pgfscope}%
\pgfpathrectangle{\pgfqpoint{5.800000in}{0.720000in}}{\pgfqpoint{1.400000in}{4.620000in}}%
\pgfusepath{clip}%
\pgfsetbuttcap%
\pgfsetmiterjoin%
\definecolor{currentfill}{rgb}{0.121569,0.466667,0.705882}%
\pgfsetfillcolor{currentfill}%
\pgfsetlinewidth{0.000000pt}%
\definecolor{currentstroke}{rgb}{0.000000,0.000000,0.000000}%
\pgfsetstrokecolor{currentstroke}%
\pgfsetstrokeopacity{0.000000}%
\pgfsetdash{}{0pt}%
\pgfpathmoveto{\pgfqpoint{5.800000in}{46.071147in}}%
\pgfpathlineto{\pgfqpoint{5.800000in}{46.071147in}}%
\pgfpathlineto{\pgfqpoint{5.800000in}{46.173984in}}%
\pgfpathlineto{\pgfqpoint{5.800000in}{46.173984in}}%
\pgfpathclose%
\pgfusepath{fill}%
\end{pgfscope}%
\begin{pgfscope}%
\pgfpathrectangle{\pgfqpoint{5.800000in}{0.720000in}}{\pgfqpoint{1.400000in}{4.620000in}}%
\pgfusepath{clip}%
\pgfsetbuttcap%
\pgfsetmiterjoin%
\definecolor{currentfill}{rgb}{0.121569,0.466667,0.705882}%
\pgfsetfillcolor{currentfill}%
\pgfsetlinewidth{0.000000pt}%
\definecolor{currentstroke}{rgb}{0.000000,0.000000,0.000000}%
\pgfsetstrokecolor{currentstroke}%
\pgfsetstrokeopacity{0.000000}%
\pgfsetdash{}{0pt}%
\pgfpathmoveto{\pgfqpoint{5.800000in}{46.173984in}}%
\pgfpathlineto{\pgfqpoint{5.800000in}{46.173984in}}%
\pgfpathlineto{\pgfqpoint{5.800000in}{46.276821in}}%
\pgfpathlineto{\pgfqpoint{5.800000in}{46.276821in}}%
\pgfpathclose%
\pgfusepath{fill}%
\end{pgfscope}%
\begin{pgfscope}%
\pgfpathrectangle{\pgfqpoint{5.800000in}{0.720000in}}{\pgfqpoint{1.400000in}{4.620000in}}%
\pgfusepath{clip}%
\pgfsetbuttcap%
\pgfsetmiterjoin%
\definecolor{currentfill}{rgb}{0.121569,0.466667,0.705882}%
\pgfsetfillcolor{currentfill}%
\pgfsetlinewidth{0.000000pt}%
\definecolor{currentstroke}{rgb}{0.000000,0.000000,0.000000}%
\pgfsetstrokecolor{currentstroke}%
\pgfsetstrokeopacity{0.000000}%
\pgfsetdash{}{0pt}%
\pgfpathmoveto{\pgfqpoint{5.800000in}{46.276821in}}%
\pgfpathlineto{\pgfqpoint{5.800000in}{46.276821in}}%
\pgfpathlineto{\pgfqpoint{5.800000in}{46.379658in}}%
\pgfpathlineto{\pgfqpoint{5.800000in}{46.379658in}}%
\pgfpathclose%
\pgfusepath{fill}%
\end{pgfscope}%
\begin{pgfscope}%
\pgfpathrectangle{\pgfqpoint{5.800000in}{0.720000in}}{\pgfqpoint{1.400000in}{4.620000in}}%
\pgfusepath{clip}%
\pgfsetbuttcap%
\pgfsetmiterjoin%
\definecolor{currentfill}{rgb}{0.121569,0.466667,0.705882}%
\pgfsetfillcolor{currentfill}%
\pgfsetlinewidth{0.000000pt}%
\definecolor{currentstroke}{rgb}{0.000000,0.000000,0.000000}%
\pgfsetstrokecolor{currentstroke}%
\pgfsetstrokeopacity{0.000000}%
\pgfsetdash{}{0pt}%
\pgfpathmoveto{\pgfqpoint{5.800000in}{46.379658in}}%
\pgfpathlineto{\pgfqpoint{5.800000in}{46.379658in}}%
\pgfpathlineto{\pgfqpoint{5.800000in}{46.482495in}}%
\pgfpathlineto{\pgfqpoint{5.800000in}{46.482495in}}%
\pgfpathclose%
\pgfusepath{fill}%
\end{pgfscope}%
\begin{pgfscope}%
\pgfpathrectangle{\pgfqpoint{5.800000in}{0.720000in}}{\pgfqpoint{1.400000in}{4.620000in}}%
\pgfusepath{clip}%
\pgfsetbuttcap%
\pgfsetmiterjoin%
\definecolor{currentfill}{rgb}{0.121569,0.466667,0.705882}%
\pgfsetfillcolor{currentfill}%
\pgfsetlinewidth{0.000000pt}%
\definecolor{currentstroke}{rgb}{0.000000,0.000000,0.000000}%
\pgfsetstrokecolor{currentstroke}%
\pgfsetstrokeopacity{0.000000}%
\pgfsetdash{}{0pt}%
\pgfpathmoveto{\pgfqpoint{5.800000in}{46.482495in}}%
\pgfpathlineto{\pgfqpoint{5.800000in}{46.482495in}}%
\pgfpathlineto{\pgfqpoint{5.800000in}{46.585332in}}%
\pgfpathlineto{\pgfqpoint{5.800000in}{46.585332in}}%
\pgfpathclose%
\pgfusepath{fill}%
\end{pgfscope}%
\begin{pgfscope}%
\pgfpathrectangle{\pgfqpoint{5.800000in}{0.720000in}}{\pgfqpoint{1.400000in}{4.620000in}}%
\pgfusepath{clip}%
\pgfsetbuttcap%
\pgfsetmiterjoin%
\definecolor{currentfill}{rgb}{0.121569,0.466667,0.705882}%
\pgfsetfillcolor{currentfill}%
\pgfsetlinewidth{0.000000pt}%
\definecolor{currentstroke}{rgb}{0.000000,0.000000,0.000000}%
\pgfsetstrokecolor{currentstroke}%
\pgfsetstrokeopacity{0.000000}%
\pgfsetdash{}{0pt}%
\pgfpathmoveto{\pgfqpoint{5.800000in}{46.585332in}}%
\pgfpathlineto{\pgfqpoint{5.800000in}{46.585332in}}%
\pgfpathlineto{\pgfqpoint{5.800000in}{46.688169in}}%
\pgfpathlineto{\pgfqpoint{5.800000in}{46.688169in}}%
\pgfpathclose%
\pgfusepath{fill}%
\end{pgfscope}%
\begin{pgfscope}%
\pgfpathrectangle{\pgfqpoint{5.800000in}{0.720000in}}{\pgfqpoint{1.400000in}{4.620000in}}%
\pgfusepath{clip}%
\pgfsetbuttcap%
\pgfsetmiterjoin%
\definecolor{currentfill}{rgb}{0.121569,0.466667,0.705882}%
\pgfsetfillcolor{currentfill}%
\pgfsetlinewidth{0.000000pt}%
\definecolor{currentstroke}{rgb}{0.000000,0.000000,0.000000}%
\pgfsetstrokecolor{currentstroke}%
\pgfsetstrokeopacity{0.000000}%
\pgfsetdash{}{0pt}%
\pgfpathmoveto{\pgfqpoint{5.800000in}{46.688169in}}%
\pgfpathlineto{\pgfqpoint{5.800000in}{46.688169in}}%
\pgfpathlineto{\pgfqpoint{5.800000in}{46.791006in}}%
\pgfpathlineto{\pgfqpoint{5.800000in}{46.791006in}}%
\pgfpathclose%
\pgfusepath{fill}%
\end{pgfscope}%
\begin{pgfscope}%
\pgfpathrectangle{\pgfqpoint{5.800000in}{0.720000in}}{\pgfqpoint{1.400000in}{4.620000in}}%
\pgfusepath{clip}%
\pgfsetbuttcap%
\pgfsetmiterjoin%
\definecolor{currentfill}{rgb}{0.121569,0.466667,0.705882}%
\pgfsetfillcolor{currentfill}%
\pgfsetlinewidth{0.000000pt}%
\definecolor{currentstroke}{rgb}{0.000000,0.000000,0.000000}%
\pgfsetstrokecolor{currentstroke}%
\pgfsetstrokeopacity{0.000000}%
\pgfsetdash{}{0pt}%
\pgfpathmoveto{\pgfqpoint{5.800000in}{46.791006in}}%
\pgfpathlineto{\pgfqpoint{5.800000in}{46.791006in}}%
\pgfpathlineto{\pgfqpoint{5.800000in}{46.893843in}}%
\pgfpathlineto{\pgfqpoint{5.800000in}{46.893843in}}%
\pgfpathclose%
\pgfusepath{fill}%
\end{pgfscope}%
\begin{pgfscope}%
\pgfpathrectangle{\pgfqpoint{5.800000in}{0.720000in}}{\pgfqpoint{1.400000in}{4.620000in}}%
\pgfusepath{clip}%
\pgfsetbuttcap%
\pgfsetmiterjoin%
\definecolor{currentfill}{rgb}{0.121569,0.466667,0.705882}%
\pgfsetfillcolor{currentfill}%
\pgfsetlinewidth{0.000000pt}%
\definecolor{currentstroke}{rgb}{0.000000,0.000000,0.000000}%
\pgfsetstrokecolor{currentstroke}%
\pgfsetstrokeopacity{0.000000}%
\pgfsetdash{}{0pt}%
\pgfpathmoveto{\pgfqpoint{5.800000in}{46.893843in}}%
\pgfpathlineto{\pgfqpoint{5.800000in}{46.893843in}}%
\pgfpathlineto{\pgfqpoint{5.800000in}{46.996680in}}%
\pgfpathlineto{\pgfqpoint{5.800000in}{46.996680in}}%
\pgfpathclose%
\pgfusepath{fill}%
\end{pgfscope}%
\begin{pgfscope}%
\pgfpathrectangle{\pgfqpoint{5.800000in}{0.720000in}}{\pgfqpoint{1.400000in}{4.620000in}}%
\pgfusepath{clip}%
\pgfsetbuttcap%
\pgfsetmiterjoin%
\definecolor{currentfill}{rgb}{0.121569,0.466667,0.705882}%
\pgfsetfillcolor{currentfill}%
\pgfsetlinewidth{0.000000pt}%
\definecolor{currentstroke}{rgb}{0.000000,0.000000,0.000000}%
\pgfsetstrokecolor{currentstroke}%
\pgfsetstrokeopacity{0.000000}%
\pgfsetdash{}{0pt}%
\pgfpathmoveto{\pgfqpoint{5.800000in}{46.996680in}}%
\pgfpathlineto{\pgfqpoint{5.800000in}{46.996680in}}%
\pgfpathlineto{\pgfqpoint{5.800000in}{47.099518in}}%
\pgfpathlineto{\pgfqpoint{5.800000in}{47.099518in}}%
\pgfpathclose%
\pgfusepath{fill}%
\end{pgfscope}%
\begin{pgfscope}%
\pgfpathrectangle{\pgfqpoint{5.800000in}{0.720000in}}{\pgfqpoint{1.400000in}{4.620000in}}%
\pgfusepath{clip}%
\pgfsetbuttcap%
\pgfsetmiterjoin%
\definecolor{currentfill}{rgb}{0.121569,0.466667,0.705882}%
\pgfsetfillcolor{currentfill}%
\pgfsetlinewidth{0.000000pt}%
\definecolor{currentstroke}{rgb}{0.000000,0.000000,0.000000}%
\pgfsetstrokecolor{currentstroke}%
\pgfsetstrokeopacity{0.000000}%
\pgfsetdash{}{0pt}%
\pgfpathmoveto{\pgfqpoint{5.800000in}{47.099518in}}%
\pgfpathlineto{\pgfqpoint{5.800000in}{47.099518in}}%
\pgfpathlineto{\pgfqpoint{5.800000in}{47.202355in}}%
\pgfpathlineto{\pgfqpoint{5.800000in}{47.202355in}}%
\pgfpathclose%
\pgfusepath{fill}%
\end{pgfscope}%
\begin{pgfscope}%
\pgfpathrectangle{\pgfqpoint{5.800000in}{0.720000in}}{\pgfqpoint{1.400000in}{4.620000in}}%
\pgfusepath{clip}%
\pgfsetbuttcap%
\pgfsetmiterjoin%
\definecolor{currentfill}{rgb}{0.121569,0.466667,0.705882}%
\pgfsetfillcolor{currentfill}%
\pgfsetlinewidth{0.000000pt}%
\definecolor{currentstroke}{rgb}{0.000000,0.000000,0.000000}%
\pgfsetstrokecolor{currentstroke}%
\pgfsetstrokeopacity{0.000000}%
\pgfsetdash{}{0pt}%
\pgfpathmoveto{\pgfqpoint{5.800000in}{47.202355in}}%
\pgfpathlineto{\pgfqpoint{5.800000in}{47.202355in}}%
\pgfpathlineto{\pgfqpoint{5.800000in}{47.305192in}}%
\pgfpathlineto{\pgfqpoint{5.800000in}{47.305192in}}%
\pgfpathclose%
\pgfusepath{fill}%
\end{pgfscope}%
\begin{pgfscope}%
\pgfpathrectangle{\pgfqpoint{5.800000in}{0.720000in}}{\pgfqpoint{1.400000in}{4.620000in}}%
\pgfusepath{clip}%
\pgfsetbuttcap%
\pgfsetmiterjoin%
\definecolor{currentfill}{rgb}{0.121569,0.466667,0.705882}%
\pgfsetfillcolor{currentfill}%
\pgfsetlinewidth{0.000000pt}%
\definecolor{currentstroke}{rgb}{0.000000,0.000000,0.000000}%
\pgfsetstrokecolor{currentstroke}%
\pgfsetstrokeopacity{0.000000}%
\pgfsetdash{}{0pt}%
\pgfpathmoveto{\pgfqpoint{5.800000in}{47.305192in}}%
\pgfpathlineto{\pgfqpoint{5.800000in}{47.305192in}}%
\pgfpathlineto{\pgfqpoint{5.800000in}{47.408029in}}%
\pgfpathlineto{\pgfqpoint{5.800000in}{47.408029in}}%
\pgfpathclose%
\pgfusepath{fill}%
\end{pgfscope}%
\begin{pgfscope}%
\pgfpathrectangle{\pgfqpoint{5.800000in}{0.720000in}}{\pgfqpoint{1.400000in}{4.620000in}}%
\pgfusepath{clip}%
\pgfsetbuttcap%
\pgfsetmiterjoin%
\definecolor{currentfill}{rgb}{0.121569,0.466667,0.705882}%
\pgfsetfillcolor{currentfill}%
\pgfsetlinewidth{0.000000pt}%
\definecolor{currentstroke}{rgb}{0.000000,0.000000,0.000000}%
\pgfsetstrokecolor{currentstroke}%
\pgfsetstrokeopacity{0.000000}%
\pgfsetdash{}{0pt}%
\pgfpathmoveto{\pgfqpoint{5.800000in}{47.408029in}}%
\pgfpathlineto{\pgfqpoint{5.800000in}{47.408029in}}%
\pgfpathlineto{\pgfqpoint{5.800000in}{47.510866in}}%
\pgfpathlineto{\pgfqpoint{5.800000in}{47.510866in}}%
\pgfpathclose%
\pgfusepath{fill}%
\end{pgfscope}%
\begin{pgfscope}%
\pgfpathrectangle{\pgfqpoint{5.800000in}{0.720000in}}{\pgfqpoint{1.400000in}{4.620000in}}%
\pgfusepath{clip}%
\pgfsetbuttcap%
\pgfsetmiterjoin%
\definecolor{currentfill}{rgb}{0.121569,0.466667,0.705882}%
\pgfsetfillcolor{currentfill}%
\pgfsetlinewidth{0.000000pt}%
\definecolor{currentstroke}{rgb}{0.000000,0.000000,0.000000}%
\pgfsetstrokecolor{currentstroke}%
\pgfsetstrokeopacity{0.000000}%
\pgfsetdash{}{0pt}%
\pgfpathmoveto{\pgfqpoint{5.800000in}{47.510866in}}%
\pgfpathlineto{\pgfqpoint{5.800000in}{47.510866in}}%
\pgfpathlineto{\pgfqpoint{5.800000in}{47.613703in}}%
\pgfpathlineto{\pgfqpoint{5.800000in}{47.613703in}}%
\pgfpathclose%
\pgfusepath{fill}%
\end{pgfscope}%
\begin{pgfscope}%
\pgfpathrectangle{\pgfqpoint{5.800000in}{0.720000in}}{\pgfqpoint{1.400000in}{4.620000in}}%
\pgfusepath{clip}%
\pgfsetbuttcap%
\pgfsetmiterjoin%
\definecolor{currentfill}{rgb}{0.121569,0.466667,0.705882}%
\pgfsetfillcolor{currentfill}%
\pgfsetlinewidth{0.000000pt}%
\definecolor{currentstroke}{rgb}{0.000000,0.000000,0.000000}%
\pgfsetstrokecolor{currentstroke}%
\pgfsetstrokeopacity{0.000000}%
\pgfsetdash{}{0pt}%
\pgfpathmoveto{\pgfqpoint{5.800000in}{47.613703in}}%
\pgfpathlineto{\pgfqpoint{5.800000in}{47.613703in}}%
\pgfpathlineto{\pgfqpoint{5.800000in}{47.716540in}}%
\pgfpathlineto{\pgfqpoint{5.800000in}{47.716540in}}%
\pgfpathclose%
\pgfusepath{fill}%
\end{pgfscope}%
\begin{pgfscope}%
\pgfpathrectangle{\pgfqpoint{5.800000in}{0.720000in}}{\pgfqpoint{1.400000in}{4.620000in}}%
\pgfusepath{clip}%
\pgfsetbuttcap%
\pgfsetmiterjoin%
\definecolor{currentfill}{rgb}{0.121569,0.466667,0.705882}%
\pgfsetfillcolor{currentfill}%
\pgfsetlinewidth{0.000000pt}%
\definecolor{currentstroke}{rgb}{0.000000,0.000000,0.000000}%
\pgfsetstrokecolor{currentstroke}%
\pgfsetstrokeopacity{0.000000}%
\pgfsetdash{}{0pt}%
\pgfpathmoveto{\pgfqpoint{5.800000in}{47.716540in}}%
\pgfpathlineto{\pgfqpoint{5.800000in}{47.716540in}}%
\pgfpathlineto{\pgfqpoint{5.800000in}{47.819377in}}%
\pgfpathlineto{\pgfqpoint{5.800000in}{47.819377in}}%
\pgfpathclose%
\pgfusepath{fill}%
\end{pgfscope}%
\begin{pgfscope}%
\pgfpathrectangle{\pgfqpoint{5.800000in}{0.720000in}}{\pgfqpoint{1.400000in}{4.620000in}}%
\pgfusepath{clip}%
\pgfsetbuttcap%
\pgfsetmiterjoin%
\definecolor{currentfill}{rgb}{0.121569,0.466667,0.705882}%
\pgfsetfillcolor{currentfill}%
\pgfsetlinewidth{0.000000pt}%
\definecolor{currentstroke}{rgb}{0.000000,0.000000,0.000000}%
\pgfsetstrokecolor{currentstroke}%
\pgfsetstrokeopacity{0.000000}%
\pgfsetdash{}{0pt}%
\pgfpathmoveto{\pgfqpoint{5.800000in}{47.819377in}}%
\pgfpathlineto{\pgfqpoint{5.800000in}{47.819377in}}%
\pgfpathlineto{\pgfqpoint{5.800000in}{47.922214in}}%
\pgfpathlineto{\pgfqpoint{5.800000in}{47.922214in}}%
\pgfpathclose%
\pgfusepath{fill}%
\end{pgfscope}%
\begin{pgfscope}%
\pgfpathrectangle{\pgfqpoint{5.800000in}{0.720000in}}{\pgfqpoint{1.400000in}{4.620000in}}%
\pgfusepath{clip}%
\pgfsetbuttcap%
\pgfsetmiterjoin%
\definecolor{currentfill}{rgb}{0.121569,0.466667,0.705882}%
\pgfsetfillcolor{currentfill}%
\pgfsetlinewidth{0.000000pt}%
\definecolor{currentstroke}{rgb}{0.000000,0.000000,0.000000}%
\pgfsetstrokecolor{currentstroke}%
\pgfsetstrokeopacity{0.000000}%
\pgfsetdash{}{0pt}%
\pgfpathmoveto{\pgfqpoint{5.800000in}{47.922214in}}%
\pgfpathlineto{\pgfqpoint{5.800000in}{47.922214in}}%
\pgfpathlineto{\pgfqpoint{5.800000in}{48.025051in}}%
\pgfpathlineto{\pgfqpoint{5.800000in}{48.025051in}}%
\pgfpathclose%
\pgfusepath{fill}%
\end{pgfscope}%
\begin{pgfscope}%
\pgfpathrectangle{\pgfqpoint{5.800000in}{0.720000in}}{\pgfqpoint{1.400000in}{4.620000in}}%
\pgfusepath{clip}%
\pgfsetbuttcap%
\pgfsetmiterjoin%
\definecolor{currentfill}{rgb}{0.121569,0.466667,0.705882}%
\pgfsetfillcolor{currentfill}%
\pgfsetlinewidth{0.000000pt}%
\definecolor{currentstroke}{rgb}{0.000000,0.000000,0.000000}%
\pgfsetstrokecolor{currentstroke}%
\pgfsetstrokeopacity{0.000000}%
\pgfsetdash{}{0pt}%
\pgfpathmoveto{\pgfqpoint{5.800000in}{48.025051in}}%
\pgfpathlineto{\pgfqpoint{5.800000in}{48.025051in}}%
\pgfpathlineto{\pgfqpoint{5.800000in}{48.127888in}}%
\pgfpathlineto{\pgfqpoint{5.800000in}{48.127888in}}%
\pgfpathclose%
\pgfusepath{fill}%
\end{pgfscope}%
\begin{pgfscope}%
\pgfpathrectangle{\pgfqpoint{5.800000in}{0.720000in}}{\pgfqpoint{1.400000in}{4.620000in}}%
\pgfusepath{clip}%
\pgfsetbuttcap%
\pgfsetmiterjoin%
\definecolor{currentfill}{rgb}{0.121569,0.466667,0.705882}%
\pgfsetfillcolor{currentfill}%
\pgfsetlinewidth{0.000000pt}%
\definecolor{currentstroke}{rgb}{0.000000,0.000000,0.000000}%
\pgfsetstrokecolor{currentstroke}%
\pgfsetstrokeopacity{0.000000}%
\pgfsetdash{}{0pt}%
\pgfpathmoveto{\pgfqpoint{5.800000in}{48.127888in}}%
\pgfpathlineto{\pgfqpoint{5.800000in}{48.127888in}}%
\pgfpathlineto{\pgfqpoint{5.800000in}{48.230725in}}%
\pgfpathlineto{\pgfqpoint{5.800000in}{48.230725in}}%
\pgfpathclose%
\pgfusepath{fill}%
\end{pgfscope}%
\begin{pgfscope}%
\pgfpathrectangle{\pgfqpoint{5.800000in}{0.720000in}}{\pgfqpoint{1.400000in}{4.620000in}}%
\pgfusepath{clip}%
\pgfsetbuttcap%
\pgfsetmiterjoin%
\definecolor{currentfill}{rgb}{0.121569,0.466667,0.705882}%
\pgfsetfillcolor{currentfill}%
\pgfsetlinewidth{0.000000pt}%
\definecolor{currentstroke}{rgb}{0.000000,0.000000,0.000000}%
\pgfsetstrokecolor{currentstroke}%
\pgfsetstrokeopacity{0.000000}%
\pgfsetdash{}{0pt}%
\pgfpathmoveto{\pgfqpoint{5.800000in}{48.230725in}}%
\pgfpathlineto{\pgfqpoint{5.800000in}{48.230725in}}%
\pgfpathlineto{\pgfqpoint{5.800000in}{48.333562in}}%
\pgfpathlineto{\pgfqpoint{5.800000in}{48.333562in}}%
\pgfpathclose%
\pgfusepath{fill}%
\end{pgfscope}%
\begin{pgfscope}%
\pgfpathrectangle{\pgfqpoint{5.800000in}{0.720000in}}{\pgfqpoint{1.400000in}{4.620000in}}%
\pgfusepath{clip}%
\pgfsetbuttcap%
\pgfsetmiterjoin%
\definecolor{currentfill}{rgb}{0.121569,0.466667,0.705882}%
\pgfsetfillcolor{currentfill}%
\pgfsetlinewidth{0.000000pt}%
\definecolor{currentstroke}{rgb}{0.000000,0.000000,0.000000}%
\pgfsetstrokecolor{currentstroke}%
\pgfsetstrokeopacity{0.000000}%
\pgfsetdash{}{0pt}%
\pgfpathmoveto{\pgfqpoint{5.800000in}{48.333562in}}%
\pgfpathlineto{\pgfqpoint{5.800000in}{48.333562in}}%
\pgfpathlineto{\pgfqpoint{5.800000in}{48.436399in}}%
\pgfpathlineto{\pgfqpoint{5.800000in}{48.436399in}}%
\pgfpathclose%
\pgfusepath{fill}%
\end{pgfscope}%
\begin{pgfscope}%
\pgfpathrectangle{\pgfqpoint{5.800000in}{0.720000in}}{\pgfqpoint{1.400000in}{4.620000in}}%
\pgfusepath{clip}%
\pgfsetbuttcap%
\pgfsetmiterjoin%
\definecolor{currentfill}{rgb}{0.121569,0.466667,0.705882}%
\pgfsetfillcolor{currentfill}%
\pgfsetlinewidth{0.000000pt}%
\definecolor{currentstroke}{rgb}{0.000000,0.000000,0.000000}%
\pgfsetstrokecolor{currentstroke}%
\pgfsetstrokeopacity{0.000000}%
\pgfsetdash{}{0pt}%
\pgfpathmoveto{\pgfqpoint{5.800000in}{48.436399in}}%
\pgfpathlineto{\pgfqpoint{5.800000in}{48.436399in}}%
\pgfpathlineto{\pgfqpoint{5.800000in}{48.539237in}}%
\pgfpathlineto{\pgfqpoint{5.800000in}{48.539237in}}%
\pgfpathclose%
\pgfusepath{fill}%
\end{pgfscope}%
\begin{pgfscope}%
\pgfpathrectangle{\pgfqpoint{5.800000in}{0.720000in}}{\pgfqpoint{1.400000in}{4.620000in}}%
\pgfusepath{clip}%
\pgfsetbuttcap%
\pgfsetmiterjoin%
\definecolor{currentfill}{rgb}{0.121569,0.466667,0.705882}%
\pgfsetfillcolor{currentfill}%
\pgfsetlinewidth{0.000000pt}%
\definecolor{currentstroke}{rgb}{0.000000,0.000000,0.000000}%
\pgfsetstrokecolor{currentstroke}%
\pgfsetstrokeopacity{0.000000}%
\pgfsetdash{}{0pt}%
\pgfpathmoveto{\pgfqpoint{5.800000in}{48.539237in}}%
\pgfpathlineto{\pgfqpoint{5.800000in}{48.539237in}}%
\pgfpathlineto{\pgfqpoint{5.800000in}{48.642074in}}%
\pgfpathlineto{\pgfqpoint{5.800000in}{48.642074in}}%
\pgfpathclose%
\pgfusepath{fill}%
\end{pgfscope}%
\begin{pgfscope}%
\pgfpathrectangle{\pgfqpoint{5.800000in}{0.720000in}}{\pgfqpoint{1.400000in}{4.620000in}}%
\pgfusepath{clip}%
\pgfsetbuttcap%
\pgfsetmiterjoin%
\definecolor{currentfill}{rgb}{0.121569,0.466667,0.705882}%
\pgfsetfillcolor{currentfill}%
\pgfsetlinewidth{0.000000pt}%
\definecolor{currentstroke}{rgb}{0.000000,0.000000,0.000000}%
\pgfsetstrokecolor{currentstroke}%
\pgfsetstrokeopacity{0.000000}%
\pgfsetdash{}{0pt}%
\pgfpathmoveto{\pgfqpoint{5.800000in}{48.642074in}}%
\pgfpathlineto{\pgfqpoint{5.800000in}{48.642074in}}%
\pgfpathlineto{\pgfqpoint{5.800000in}{48.744911in}}%
\pgfpathlineto{\pgfqpoint{5.800000in}{48.744911in}}%
\pgfpathclose%
\pgfusepath{fill}%
\end{pgfscope}%
\begin{pgfscope}%
\pgfpathrectangle{\pgfqpoint{5.800000in}{0.720000in}}{\pgfqpoint{1.400000in}{4.620000in}}%
\pgfusepath{clip}%
\pgfsetbuttcap%
\pgfsetmiterjoin%
\definecolor{currentfill}{rgb}{0.121569,0.466667,0.705882}%
\pgfsetfillcolor{currentfill}%
\pgfsetlinewidth{0.000000pt}%
\definecolor{currentstroke}{rgb}{0.000000,0.000000,0.000000}%
\pgfsetstrokecolor{currentstroke}%
\pgfsetstrokeopacity{0.000000}%
\pgfsetdash{}{0pt}%
\pgfpathmoveto{\pgfqpoint{5.800000in}{48.744911in}}%
\pgfpathlineto{\pgfqpoint{5.800000in}{48.744911in}}%
\pgfpathlineto{\pgfqpoint{5.800000in}{48.847748in}}%
\pgfpathlineto{\pgfqpoint{5.800000in}{48.847748in}}%
\pgfpathclose%
\pgfusepath{fill}%
\end{pgfscope}%
\begin{pgfscope}%
\pgfpathrectangle{\pgfqpoint{5.800000in}{0.720000in}}{\pgfqpoint{1.400000in}{4.620000in}}%
\pgfusepath{clip}%
\pgfsetbuttcap%
\pgfsetmiterjoin%
\definecolor{currentfill}{rgb}{0.121569,0.466667,0.705882}%
\pgfsetfillcolor{currentfill}%
\pgfsetlinewidth{0.000000pt}%
\definecolor{currentstroke}{rgb}{0.000000,0.000000,0.000000}%
\pgfsetstrokecolor{currentstroke}%
\pgfsetstrokeopacity{0.000000}%
\pgfsetdash{}{0pt}%
\pgfpathmoveto{\pgfqpoint{5.800000in}{48.847748in}}%
\pgfpathlineto{\pgfqpoint{5.800000in}{48.847748in}}%
\pgfpathlineto{\pgfqpoint{5.800000in}{48.950585in}}%
\pgfpathlineto{\pgfqpoint{5.800000in}{48.950585in}}%
\pgfpathclose%
\pgfusepath{fill}%
\end{pgfscope}%
\begin{pgfscope}%
\pgfpathrectangle{\pgfqpoint{5.800000in}{0.720000in}}{\pgfqpoint{1.400000in}{4.620000in}}%
\pgfusepath{clip}%
\pgfsetbuttcap%
\pgfsetmiterjoin%
\definecolor{currentfill}{rgb}{0.121569,0.466667,0.705882}%
\pgfsetfillcolor{currentfill}%
\pgfsetlinewidth{0.000000pt}%
\definecolor{currentstroke}{rgb}{0.000000,0.000000,0.000000}%
\pgfsetstrokecolor{currentstroke}%
\pgfsetstrokeopacity{0.000000}%
\pgfsetdash{}{0pt}%
\pgfpathmoveto{\pgfqpoint{5.800000in}{48.950585in}}%
\pgfpathlineto{\pgfqpoint{5.800000in}{48.950585in}}%
\pgfpathlineto{\pgfqpoint{5.800000in}{49.053422in}}%
\pgfpathlineto{\pgfqpoint{5.800000in}{49.053422in}}%
\pgfpathclose%
\pgfusepath{fill}%
\end{pgfscope}%
\begin{pgfscope}%
\pgfpathrectangle{\pgfqpoint{5.800000in}{0.720000in}}{\pgfqpoint{1.400000in}{4.620000in}}%
\pgfusepath{clip}%
\pgfsetbuttcap%
\pgfsetmiterjoin%
\definecolor{currentfill}{rgb}{0.121569,0.466667,0.705882}%
\pgfsetfillcolor{currentfill}%
\pgfsetlinewidth{0.000000pt}%
\definecolor{currentstroke}{rgb}{0.000000,0.000000,0.000000}%
\pgfsetstrokecolor{currentstroke}%
\pgfsetstrokeopacity{0.000000}%
\pgfsetdash{}{0pt}%
\pgfpathmoveto{\pgfqpoint{5.800000in}{49.053422in}}%
\pgfpathlineto{\pgfqpoint{5.800000in}{49.053422in}}%
\pgfpathlineto{\pgfqpoint{5.800000in}{49.156259in}}%
\pgfpathlineto{\pgfqpoint{5.800000in}{49.156259in}}%
\pgfpathclose%
\pgfusepath{fill}%
\end{pgfscope}%
\begin{pgfscope}%
\pgfpathrectangle{\pgfqpoint{5.800000in}{0.720000in}}{\pgfqpoint{1.400000in}{4.620000in}}%
\pgfusepath{clip}%
\pgfsetbuttcap%
\pgfsetmiterjoin%
\definecolor{currentfill}{rgb}{0.121569,0.466667,0.705882}%
\pgfsetfillcolor{currentfill}%
\pgfsetlinewidth{0.000000pt}%
\definecolor{currentstroke}{rgb}{0.000000,0.000000,0.000000}%
\pgfsetstrokecolor{currentstroke}%
\pgfsetstrokeopacity{0.000000}%
\pgfsetdash{}{0pt}%
\pgfpathmoveto{\pgfqpoint{5.800000in}{49.156259in}}%
\pgfpathlineto{\pgfqpoint{5.800000in}{49.156259in}}%
\pgfpathlineto{\pgfqpoint{5.800000in}{49.259096in}}%
\pgfpathlineto{\pgfqpoint{5.800000in}{49.259096in}}%
\pgfpathclose%
\pgfusepath{fill}%
\end{pgfscope}%
\begin{pgfscope}%
\pgfpathrectangle{\pgfqpoint{5.800000in}{0.720000in}}{\pgfqpoint{1.400000in}{4.620000in}}%
\pgfusepath{clip}%
\pgfsetbuttcap%
\pgfsetmiterjoin%
\definecolor{currentfill}{rgb}{0.121569,0.466667,0.705882}%
\pgfsetfillcolor{currentfill}%
\pgfsetlinewidth{0.000000pt}%
\definecolor{currentstroke}{rgb}{0.000000,0.000000,0.000000}%
\pgfsetstrokecolor{currentstroke}%
\pgfsetstrokeopacity{0.000000}%
\pgfsetdash{}{0pt}%
\pgfpathmoveto{\pgfqpoint{5.800000in}{49.259096in}}%
\pgfpathlineto{\pgfqpoint{5.800000in}{49.259096in}}%
\pgfpathlineto{\pgfqpoint{5.800000in}{49.361933in}}%
\pgfpathlineto{\pgfqpoint{5.800000in}{49.361933in}}%
\pgfpathclose%
\pgfusepath{fill}%
\end{pgfscope}%
\begin{pgfscope}%
\pgfpathrectangle{\pgfqpoint{5.800000in}{0.720000in}}{\pgfqpoint{1.400000in}{4.620000in}}%
\pgfusepath{clip}%
\pgfsetbuttcap%
\pgfsetmiterjoin%
\definecolor{currentfill}{rgb}{0.121569,0.466667,0.705882}%
\pgfsetfillcolor{currentfill}%
\pgfsetlinewidth{0.000000pt}%
\definecolor{currentstroke}{rgb}{0.000000,0.000000,0.000000}%
\pgfsetstrokecolor{currentstroke}%
\pgfsetstrokeopacity{0.000000}%
\pgfsetdash{}{0pt}%
\pgfpathmoveto{\pgfqpoint{5.800000in}{49.361933in}}%
\pgfpathlineto{\pgfqpoint{5.800000in}{49.361933in}}%
\pgfpathlineto{\pgfqpoint{5.800000in}{49.464770in}}%
\pgfpathlineto{\pgfqpoint{5.800000in}{49.464770in}}%
\pgfpathclose%
\pgfusepath{fill}%
\end{pgfscope}%
\begin{pgfscope}%
\pgfpathrectangle{\pgfqpoint{5.800000in}{0.720000in}}{\pgfqpoint{1.400000in}{4.620000in}}%
\pgfusepath{clip}%
\pgfsetbuttcap%
\pgfsetmiterjoin%
\definecolor{currentfill}{rgb}{0.121569,0.466667,0.705882}%
\pgfsetfillcolor{currentfill}%
\pgfsetlinewidth{0.000000pt}%
\definecolor{currentstroke}{rgb}{0.000000,0.000000,0.000000}%
\pgfsetstrokecolor{currentstroke}%
\pgfsetstrokeopacity{0.000000}%
\pgfsetdash{}{0pt}%
\pgfpathmoveto{\pgfqpoint{5.800000in}{49.464770in}}%
\pgfpathlineto{\pgfqpoint{5.800000in}{49.464770in}}%
\pgfpathlineto{\pgfqpoint{5.800000in}{49.567607in}}%
\pgfpathlineto{\pgfqpoint{5.800000in}{49.567607in}}%
\pgfpathclose%
\pgfusepath{fill}%
\end{pgfscope}%
\begin{pgfscope}%
\pgfpathrectangle{\pgfqpoint{5.800000in}{0.720000in}}{\pgfqpoint{1.400000in}{4.620000in}}%
\pgfusepath{clip}%
\pgfsetbuttcap%
\pgfsetmiterjoin%
\definecolor{currentfill}{rgb}{0.121569,0.466667,0.705882}%
\pgfsetfillcolor{currentfill}%
\pgfsetlinewidth{0.000000pt}%
\definecolor{currentstroke}{rgb}{0.000000,0.000000,0.000000}%
\pgfsetstrokecolor{currentstroke}%
\pgfsetstrokeopacity{0.000000}%
\pgfsetdash{}{0pt}%
\pgfpathmoveto{\pgfqpoint{5.800000in}{49.567607in}}%
\pgfpathlineto{\pgfqpoint{5.800000in}{49.567607in}}%
\pgfpathlineto{\pgfqpoint{5.800000in}{49.670444in}}%
\pgfpathlineto{\pgfqpoint{5.800000in}{49.670444in}}%
\pgfpathclose%
\pgfusepath{fill}%
\end{pgfscope}%
\begin{pgfscope}%
\pgfpathrectangle{\pgfqpoint{5.800000in}{0.720000in}}{\pgfqpoint{1.400000in}{4.620000in}}%
\pgfusepath{clip}%
\pgfsetbuttcap%
\pgfsetmiterjoin%
\definecolor{currentfill}{rgb}{0.121569,0.466667,0.705882}%
\pgfsetfillcolor{currentfill}%
\pgfsetlinewidth{0.000000pt}%
\definecolor{currentstroke}{rgb}{0.000000,0.000000,0.000000}%
\pgfsetstrokecolor{currentstroke}%
\pgfsetstrokeopacity{0.000000}%
\pgfsetdash{}{0pt}%
\pgfpathmoveto{\pgfqpoint{5.800000in}{49.670444in}}%
\pgfpathlineto{\pgfqpoint{5.800000in}{49.670444in}}%
\pgfpathlineto{\pgfqpoint{5.800000in}{49.773281in}}%
\pgfpathlineto{\pgfqpoint{5.800000in}{49.773281in}}%
\pgfpathclose%
\pgfusepath{fill}%
\end{pgfscope}%
\begin{pgfscope}%
\pgfpathrectangle{\pgfqpoint{5.800000in}{0.720000in}}{\pgfqpoint{1.400000in}{4.620000in}}%
\pgfusepath{clip}%
\pgfsetbuttcap%
\pgfsetmiterjoin%
\definecolor{currentfill}{rgb}{0.121569,0.466667,0.705882}%
\pgfsetfillcolor{currentfill}%
\pgfsetlinewidth{0.000000pt}%
\definecolor{currentstroke}{rgb}{0.000000,0.000000,0.000000}%
\pgfsetstrokecolor{currentstroke}%
\pgfsetstrokeopacity{0.000000}%
\pgfsetdash{}{0pt}%
\pgfpathmoveto{\pgfqpoint{5.800000in}{49.773281in}}%
\pgfpathlineto{\pgfqpoint{5.800000in}{49.773281in}}%
\pgfpathlineto{\pgfqpoint{5.800000in}{49.876118in}}%
\pgfpathlineto{\pgfqpoint{5.800000in}{49.876118in}}%
\pgfpathclose%
\pgfusepath{fill}%
\end{pgfscope}%
\begin{pgfscope}%
\pgfpathrectangle{\pgfqpoint{5.800000in}{0.720000in}}{\pgfqpoint{1.400000in}{4.620000in}}%
\pgfusepath{clip}%
\pgfsetbuttcap%
\pgfsetmiterjoin%
\definecolor{currentfill}{rgb}{0.121569,0.466667,0.705882}%
\pgfsetfillcolor{currentfill}%
\pgfsetlinewidth{0.000000pt}%
\definecolor{currentstroke}{rgb}{0.000000,0.000000,0.000000}%
\pgfsetstrokecolor{currentstroke}%
\pgfsetstrokeopacity{0.000000}%
\pgfsetdash{}{0pt}%
\pgfpathmoveto{\pgfqpoint{5.800000in}{49.876118in}}%
\pgfpathlineto{\pgfqpoint{5.800000in}{49.876118in}}%
\pgfpathlineto{\pgfqpoint{5.800000in}{49.978955in}}%
\pgfpathlineto{\pgfqpoint{5.800000in}{49.978955in}}%
\pgfpathclose%
\pgfusepath{fill}%
\end{pgfscope}%
\begin{pgfscope}%
\pgfpathrectangle{\pgfqpoint{5.800000in}{0.720000in}}{\pgfqpoint{1.400000in}{4.620000in}}%
\pgfusepath{clip}%
\pgfsetbuttcap%
\pgfsetmiterjoin%
\definecolor{currentfill}{rgb}{0.121569,0.466667,0.705882}%
\pgfsetfillcolor{currentfill}%
\pgfsetlinewidth{0.000000pt}%
\definecolor{currentstroke}{rgb}{0.000000,0.000000,0.000000}%
\pgfsetstrokecolor{currentstroke}%
\pgfsetstrokeopacity{0.000000}%
\pgfsetdash{}{0pt}%
\pgfpathmoveto{\pgfqpoint{5.800000in}{49.978955in}}%
\pgfpathlineto{\pgfqpoint{5.800000in}{49.978955in}}%
\pgfpathlineto{\pgfqpoint{5.800000in}{50.081793in}}%
\pgfpathlineto{\pgfqpoint{5.800000in}{50.081793in}}%
\pgfpathclose%
\pgfusepath{fill}%
\end{pgfscope}%
\begin{pgfscope}%
\pgfpathrectangle{\pgfqpoint{5.800000in}{0.720000in}}{\pgfqpoint{1.400000in}{4.620000in}}%
\pgfusepath{clip}%
\pgfsetbuttcap%
\pgfsetmiterjoin%
\definecolor{currentfill}{rgb}{0.121569,0.466667,0.705882}%
\pgfsetfillcolor{currentfill}%
\pgfsetlinewidth{0.000000pt}%
\definecolor{currentstroke}{rgb}{0.000000,0.000000,0.000000}%
\pgfsetstrokecolor{currentstroke}%
\pgfsetstrokeopacity{0.000000}%
\pgfsetdash{}{0pt}%
\pgfpathmoveto{\pgfqpoint{5.800000in}{50.081793in}}%
\pgfpathlineto{\pgfqpoint{5.800000in}{50.081793in}}%
\pgfpathlineto{\pgfqpoint{5.800000in}{50.184630in}}%
\pgfpathlineto{\pgfqpoint{5.800000in}{50.184630in}}%
\pgfpathclose%
\pgfusepath{fill}%
\end{pgfscope}%
\begin{pgfscope}%
\pgfpathrectangle{\pgfqpoint{5.800000in}{0.720000in}}{\pgfqpoint{1.400000in}{4.620000in}}%
\pgfusepath{clip}%
\pgfsetbuttcap%
\pgfsetmiterjoin%
\definecolor{currentfill}{rgb}{0.121569,0.466667,0.705882}%
\pgfsetfillcolor{currentfill}%
\pgfsetlinewidth{0.000000pt}%
\definecolor{currentstroke}{rgb}{0.000000,0.000000,0.000000}%
\pgfsetstrokecolor{currentstroke}%
\pgfsetstrokeopacity{0.000000}%
\pgfsetdash{}{0pt}%
\pgfpathmoveto{\pgfqpoint{5.800000in}{50.184630in}}%
\pgfpathlineto{\pgfqpoint{5.800000in}{50.184630in}}%
\pgfpathlineto{\pgfqpoint{5.800000in}{50.287467in}}%
\pgfpathlineto{\pgfqpoint{5.800000in}{50.287467in}}%
\pgfpathclose%
\pgfusepath{fill}%
\end{pgfscope}%
\begin{pgfscope}%
\pgfpathrectangle{\pgfqpoint{5.800000in}{0.720000in}}{\pgfqpoint{1.400000in}{4.620000in}}%
\pgfusepath{clip}%
\pgfsetbuttcap%
\pgfsetmiterjoin%
\definecolor{currentfill}{rgb}{0.121569,0.466667,0.705882}%
\pgfsetfillcolor{currentfill}%
\pgfsetlinewidth{0.000000pt}%
\definecolor{currentstroke}{rgb}{0.000000,0.000000,0.000000}%
\pgfsetstrokecolor{currentstroke}%
\pgfsetstrokeopacity{0.000000}%
\pgfsetdash{}{0pt}%
\pgfpathmoveto{\pgfqpoint{5.800000in}{50.287467in}}%
\pgfpathlineto{\pgfqpoint{5.800000in}{50.287467in}}%
\pgfpathlineto{\pgfqpoint{5.800000in}{50.390304in}}%
\pgfpathlineto{\pgfqpoint{5.800000in}{50.390304in}}%
\pgfpathclose%
\pgfusepath{fill}%
\end{pgfscope}%
\begin{pgfscope}%
\pgfpathrectangle{\pgfqpoint{5.800000in}{0.720000in}}{\pgfqpoint{1.400000in}{4.620000in}}%
\pgfusepath{clip}%
\pgfsetbuttcap%
\pgfsetmiterjoin%
\definecolor{currentfill}{rgb}{0.121569,0.466667,0.705882}%
\pgfsetfillcolor{currentfill}%
\pgfsetlinewidth{0.000000pt}%
\definecolor{currentstroke}{rgb}{0.000000,0.000000,0.000000}%
\pgfsetstrokecolor{currentstroke}%
\pgfsetstrokeopacity{0.000000}%
\pgfsetdash{}{0pt}%
\pgfpathmoveto{\pgfqpoint{5.800000in}{50.390304in}}%
\pgfpathlineto{\pgfqpoint{5.800000in}{50.390304in}}%
\pgfpathlineto{\pgfqpoint{5.800000in}{50.493141in}}%
\pgfpathlineto{\pgfqpoint{5.800000in}{50.493141in}}%
\pgfpathclose%
\pgfusepath{fill}%
\end{pgfscope}%
\begin{pgfscope}%
\pgfpathrectangle{\pgfqpoint{5.800000in}{0.720000in}}{\pgfqpoint{1.400000in}{4.620000in}}%
\pgfusepath{clip}%
\pgfsetbuttcap%
\pgfsetmiterjoin%
\definecolor{currentfill}{rgb}{0.121569,0.466667,0.705882}%
\pgfsetfillcolor{currentfill}%
\pgfsetlinewidth{0.000000pt}%
\definecolor{currentstroke}{rgb}{0.000000,0.000000,0.000000}%
\pgfsetstrokecolor{currentstroke}%
\pgfsetstrokeopacity{0.000000}%
\pgfsetdash{}{0pt}%
\pgfpathmoveto{\pgfqpoint{5.800000in}{50.493141in}}%
\pgfpathlineto{\pgfqpoint{5.800000in}{50.493141in}}%
\pgfpathlineto{\pgfqpoint{5.800000in}{50.595978in}}%
\pgfpathlineto{\pgfqpoint{5.800000in}{50.595978in}}%
\pgfpathclose%
\pgfusepath{fill}%
\end{pgfscope}%
\begin{pgfscope}%
\pgfpathrectangle{\pgfqpoint{5.800000in}{0.720000in}}{\pgfqpoint{1.400000in}{4.620000in}}%
\pgfusepath{clip}%
\pgfsetbuttcap%
\pgfsetmiterjoin%
\definecolor{currentfill}{rgb}{0.121569,0.466667,0.705882}%
\pgfsetfillcolor{currentfill}%
\pgfsetlinewidth{0.000000pt}%
\definecolor{currentstroke}{rgb}{0.000000,0.000000,0.000000}%
\pgfsetstrokecolor{currentstroke}%
\pgfsetstrokeopacity{0.000000}%
\pgfsetdash{}{0pt}%
\pgfpathmoveto{\pgfqpoint{5.800000in}{50.595978in}}%
\pgfpathlineto{\pgfqpoint{5.800000in}{50.595978in}}%
\pgfpathlineto{\pgfqpoint{5.800000in}{50.698815in}}%
\pgfpathlineto{\pgfqpoint{5.800000in}{50.698815in}}%
\pgfpathclose%
\pgfusepath{fill}%
\end{pgfscope}%
\begin{pgfscope}%
\pgfpathrectangle{\pgfqpoint{5.800000in}{0.720000in}}{\pgfqpoint{1.400000in}{4.620000in}}%
\pgfusepath{clip}%
\pgfsetbuttcap%
\pgfsetmiterjoin%
\definecolor{currentfill}{rgb}{0.121569,0.466667,0.705882}%
\pgfsetfillcolor{currentfill}%
\pgfsetlinewidth{0.000000pt}%
\definecolor{currentstroke}{rgb}{0.000000,0.000000,0.000000}%
\pgfsetstrokecolor{currentstroke}%
\pgfsetstrokeopacity{0.000000}%
\pgfsetdash{}{0pt}%
\pgfpathmoveto{\pgfqpoint{5.800000in}{50.698815in}}%
\pgfpathlineto{\pgfqpoint{5.800000in}{50.698815in}}%
\pgfpathlineto{\pgfqpoint{5.800000in}{50.801652in}}%
\pgfpathlineto{\pgfqpoint{5.800000in}{50.801652in}}%
\pgfpathclose%
\pgfusepath{fill}%
\end{pgfscope}%
\begin{pgfscope}%
\pgfpathrectangle{\pgfqpoint{5.800000in}{0.720000in}}{\pgfqpoint{1.400000in}{4.620000in}}%
\pgfusepath{clip}%
\pgfsetbuttcap%
\pgfsetmiterjoin%
\definecolor{currentfill}{rgb}{0.121569,0.466667,0.705882}%
\pgfsetfillcolor{currentfill}%
\pgfsetlinewidth{0.000000pt}%
\definecolor{currentstroke}{rgb}{0.000000,0.000000,0.000000}%
\pgfsetstrokecolor{currentstroke}%
\pgfsetstrokeopacity{0.000000}%
\pgfsetdash{}{0pt}%
\pgfpathmoveto{\pgfqpoint{5.800000in}{50.801652in}}%
\pgfpathlineto{\pgfqpoint{5.800000in}{50.801652in}}%
\pgfpathlineto{\pgfqpoint{5.800000in}{50.904489in}}%
\pgfpathlineto{\pgfqpoint{5.800000in}{50.904489in}}%
\pgfpathclose%
\pgfusepath{fill}%
\end{pgfscope}%
\begin{pgfscope}%
\pgfpathrectangle{\pgfqpoint{5.800000in}{0.720000in}}{\pgfqpoint{1.400000in}{4.620000in}}%
\pgfusepath{clip}%
\pgfsetbuttcap%
\pgfsetmiterjoin%
\definecolor{currentfill}{rgb}{0.121569,0.466667,0.705882}%
\pgfsetfillcolor{currentfill}%
\pgfsetlinewidth{0.000000pt}%
\definecolor{currentstroke}{rgb}{0.000000,0.000000,0.000000}%
\pgfsetstrokecolor{currentstroke}%
\pgfsetstrokeopacity{0.000000}%
\pgfsetdash{}{0pt}%
\pgfpathmoveto{\pgfqpoint{5.800000in}{50.904489in}}%
\pgfpathlineto{\pgfqpoint{5.800000in}{50.904489in}}%
\pgfpathlineto{\pgfqpoint{5.800000in}{51.007326in}}%
\pgfpathlineto{\pgfqpoint{5.800000in}{51.007326in}}%
\pgfpathclose%
\pgfusepath{fill}%
\end{pgfscope}%
\begin{pgfscope}%
\pgfpathrectangle{\pgfqpoint{5.800000in}{0.720000in}}{\pgfqpoint{1.400000in}{4.620000in}}%
\pgfusepath{clip}%
\pgfsetbuttcap%
\pgfsetmiterjoin%
\definecolor{currentfill}{rgb}{0.121569,0.466667,0.705882}%
\pgfsetfillcolor{currentfill}%
\pgfsetlinewidth{0.000000pt}%
\definecolor{currentstroke}{rgb}{0.000000,0.000000,0.000000}%
\pgfsetstrokecolor{currentstroke}%
\pgfsetstrokeopacity{0.000000}%
\pgfsetdash{}{0pt}%
\pgfpathmoveto{\pgfqpoint{5.800000in}{51.007326in}}%
\pgfpathlineto{\pgfqpoint{5.800000in}{51.007326in}}%
\pgfpathlineto{\pgfqpoint{5.800000in}{51.110163in}}%
\pgfpathlineto{\pgfqpoint{5.800000in}{51.110163in}}%
\pgfpathclose%
\pgfusepath{fill}%
\end{pgfscope}%
\begin{pgfscope}%
\pgfpathrectangle{\pgfqpoint{5.800000in}{0.720000in}}{\pgfqpoint{1.400000in}{4.620000in}}%
\pgfusepath{clip}%
\pgfsetbuttcap%
\pgfsetmiterjoin%
\definecolor{currentfill}{rgb}{0.121569,0.466667,0.705882}%
\pgfsetfillcolor{currentfill}%
\pgfsetlinewidth{0.000000pt}%
\definecolor{currentstroke}{rgb}{0.000000,0.000000,0.000000}%
\pgfsetstrokecolor{currentstroke}%
\pgfsetstrokeopacity{0.000000}%
\pgfsetdash{}{0pt}%
\pgfpathmoveto{\pgfqpoint{5.800000in}{51.110163in}}%
\pgfpathlineto{\pgfqpoint{5.800000in}{51.110163in}}%
\pgfpathlineto{\pgfqpoint{5.800000in}{51.213000in}}%
\pgfpathlineto{\pgfqpoint{5.800000in}{51.213000in}}%
\pgfpathclose%
\pgfusepath{fill}%
\end{pgfscope}%
\begin{pgfscope}%
\pgfpathrectangle{\pgfqpoint{5.800000in}{0.720000in}}{\pgfqpoint{1.400000in}{4.620000in}}%
\pgfusepath{clip}%
\pgfsetbuttcap%
\pgfsetmiterjoin%
\definecolor{currentfill}{rgb}{0.121569,0.466667,0.705882}%
\pgfsetfillcolor{currentfill}%
\pgfsetlinewidth{0.000000pt}%
\definecolor{currentstroke}{rgb}{0.000000,0.000000,0.000000}%
\pgfsetstrokecolor{currentstroke}%
\pgfsetstrokeopacity{0.000000}%
\pgfsetdash{}{0pt}%
\pgfpathmoveto{\pgfqpoint{5.800000in}{51.213000in}}%
\pgfpathlineto{\pgfqpoint{5.800000in}{51.213000in}}%
\pgfpathlineto{\pgfqpoint{5.800000in}{51.315837in}}%
\pgfpathlineto{\pgfqpoint{5.800000in}{51.315837in}}%
\pgfpathclose%
\pgfusepath{fill}%
\end{pgfscope}%
\begin{pgfscope}%
\pgfpathrectangle{\pgfqpoint{5.800000in}{0.720000in}}{\pgfqpoint{1.400000in}{4.620000in}}%
\pgfusepath{clip}%
\pgfsetbuttcap%
\pgfsetmiterjoin%
\definecolor{currentfill}{rgb}{0.121569,0.466667,0.705882}%
\pgfsetfillcolor{currentfill}%
\pgfsetlinewidth{0.000000pt}%
\definecolor{currentstroke}{rgb}{0.000000,0.000000,0.000000}%
\pgfsetstrokecolor{currentstroke}%
\pgfsetstrokeopacity{0.000000}%
\pgfsetdash{}{0pt}%
\pgfpathmoveto{\pgfqpoint{5.800000in}{51.315837in}}%
\pgfpathlineto{\pgfqpoint{5.800000in}{51.315837in}}%
\pgfpathlineto{\pgfqpoint{5.800000in}{51.418674in}}%
\pgfpathlineto{\pgfqpoint{5.800000in}{51.418674in}}%
\pgfpathclose%
\pgfusepath{fill}%
\end{pgfscope}%
\begin{pgfscope}%
\pgfpathrectangle{\pgfqpoint{5.800000in}{0.720000in}}{\pgfqpoint{1.400000in}{4.620000in}}%
\pgfusepath{clip}%
\pgfsetbuttcap%
\pgfsetmiterjoin%
\definecolor{currentfill}{rgb}{0.121569,0.466667,0.705882}%
\pgfsetfillcolor{currentfill}%
\pgfsetlinewidth{0.000000pt}%
\definecolor{currentstroke}{rgb}{0.000000,0.000000,0.000000}%
\pgfsetstrokecolor{currentstroke}%
\pgfsetstrokeopacity{0.000000}%
\pgfsetdash{}{0pt}%
\pgfpathmoveto{\pgfqpoint{5.800000in}{51.418674in}}%
\pgfpathlineto{\pgfqpoint{5.800000in}{51.418674in}}%
\pgfpathlineto{\pgfqpoint{5.800000in}{51.521511in}}%
\pgfpathlineto{\pgfqpoint{5.800000in}{51.521511in}}%
\pgfpathclose%
\pgfusepath{fill}%
\end{pgfscope}%
\begin{pgfscope}%
\pgfpathrectangle{\pgfqpoint{5.800000in}{0.720000in}}{\pgfqpoint{1.400000in}{4.620000in}}%
\pgfusepath{clip}%
\pgfsetbuttcap%
\pgfsetmiterjoin%
\definecolor{currentfill}{rgb}{0.121569,0.466667,0.705882}%
\pgfsetfillcolor{currentfill}%
\pgfsetlinewidth{0.000000pt}%
\definecolor{currentstroke}{rgb}{0.000000,0.000000,0.000000}%
\pgfsetstrokecolor{currentstroke}%
\pgfsetstrokeopacity{0.000000}%
\pgfsetdash{}{0pt}%
\pgfpathmoveto{\pgfqpoint{5.800000in}{51.521511in}}%
\pgfpathlineto{\pgfqpoint{5.800000in}{51.521511in}}%
\pgfpathlineto{\pgfqpoint{5.800000in}{51.624349in}}%
\pgfpathlineto{\pgfqpoint{5.800000in}{51.624349in}}%
\pgfpathclose%
\pgfusepath{fill}%
\end{pgfscope}%
\begin{pgfscope}%
\pgfpathrectangle{\pgfqpoint{5.800000in}{0.720000in}}{\pgfqpoint{1.400000in}{4.620000in}}%
\pgfusepath{clip}%
\pgfsetbuttcap%
\pgfsetmiterjoin%
\definecolor{currentfill}{rgb}{0.121569,0.466667,0.705882}%
\pgfsetfillcolor{currentfill}%
\pgfsetlinewidth{0.000000pt}%
\definecolor{currentstroke}{rgb}{0.000000,0.000000,0.000000}%
\pgfsetstrokecolor{currentstroke}%
\pgfsetstrokeopacity{0.000000}%
\pgfsetdash{}{0pt}%
\pgfpathmoveto{\pgfqpoint{5.800000in}{51.624349in}}%
\pgfpathlineto{\pgfqpoint{5.800000in}{51.624349in}}%
\pgfpathlineto{\pgfqpoint{5.800000in}{51.727186in}}%
\pgfpathlineto{\pgfqpoint{5.800000in}{51.727186in}}%
\pgfpathclose%
\pgfusepath{fill}%
\end{pgfscope}%
\begin{pgfscope}%
\pgfpathrectangle{\pgfqpoint{5.800000in}{0.720000in}}{\pgfqpoint{1.400000in}{4.620000in}}%
\pgfusepath{clip}%
\pgfsetbuttcap%
\pgfsetmiterjoin%
\definecolor{currentfill}{rgb}{0.121569,0.466667,0.705882}%
\pgfsetfillcolor{currentfill}%
\pgfsetlinewidth{0.000000pt}%
\definecolor{currentstroke}{rgb}{0.000000,0.000000,0.000000}%
\pgfsetstrokecolor{currentstroke}%
\pgfsetstrokeopacity{0.000000}%
\pgfsetdash{}{0pt}%
\pgfpathmoveto{\pgfqpoint{5.800000in}{51.727186in}}%
\pgfpathlineto{\pgfqpoint{5.800000in}{51.727186in}}%
\pgfpathlineto{\pgfqpoint{5.800000in}{51.830023in}}%
\pgfpathlineto{\pgfqpoint{5.800000in}{51.830023in}}%
\pgfpathclose%
\pgfusepath{fill}%
\end{pgfscope}%
\begin{pgfscope}%
\pgfpathrectangle{\pgfqpoint{5.800000in}{0.720000in}}{\pgfqpoint{1.400000in}{4.620000in}}%
\pgfusepath{clip}%
\pgfsetbuttcap%
\pgfsetmiterjoin%
\definecolor{currentfill}{rgb}{0.121569,0.466667,0.705882}%
\pgfsetfillcolor{currentfill}%
\pgfsetlinewidth{0.000000pt}%
\definecolor{currentstroke}{rgb}{0.000000,0.000000,0.000000}%
\pgfsetstrokecolor{currentstroke}%
\pgfsetstrokeopacity{0.000000}%
\pgfsetdash{}{0pt}%
\pgfpathmoveto{\pgfqpoint{5.800000in}{51.830023in}}%
\pgfpathlineto{\pgfqpoint{5.800000in}{51.830023in}}%
\pgfpathlineto{\pgfqpoint{5.800000in}{51.932860in}}%
\pgfpathlineto{\pgfqpoint{5.800000in}{51.932860in}}%
\pgfpathclose%
\pgfusepath{fill}%
\end{pgfscope}%
\begin{pgfscope}%
\pgfpathrectangle{\pgfqpoint{5.800000in}{0.720000in}}{\pgfqpoint{1.400000in}{4.620000in}}%
\pgfusepath{clip}%
\pgfsetbuttcap%
\pgfsetmiterjoin%
\definecolor{currentfill}{rgb}{0.121569,0.466667,0.705882}%
\pgfsetfillcolor{currentfill}%
\pgfsetlinewidth{0.000000pt}%
\definecolor{currentstroke}{rgb}{0.000000,0.000000,0.000000}%
\pgfsetstrokecolor{currentstroke}%
\pgfsetstrokeopacity{0.000000}%
\pgfsetdash{}{0pt}%
\pgfpathmoveto{\pgfqpoint{5.800000in}{51.932860in}}%
\pgfpathlineto{\pgfqpoint{5.800000in}{51.932860in}}%
\pgfpathlineto{\pgfqpoint{5.800000in}{52.035697in}}%
\pgfpathlineto{\pgfqpoint{5.800000in}{52.035697in}}%
\pgfpathclose%
\pgfusepath{fill}%
\end{pgfscope}%
\begin{pgfscope}%
\pgfpathrectangle{\pgfqpoint{5.800000in}{0.720000in}}{\pgfqpoint{1.400000in}{4.620000in}}%
\pgfusepath{clip}%
\pgfsetbuttcap%
\pgfsetmiterjoin%
\definecolor{currentfill}{rgb}{0.121569,0.466667,0.705882}%
\pgfsetfillcolor{currentfill}%
\pgfsetlinewidth{0.000000pt}%
\definecolor{currentstroke}{rgb}{0.000000,0.000000,0.000000}%
\pgfsetstrokecolor{currentstroke}%
\pgfsetstrokeopacity{0.000000}%
\pgfsetdash{}{0pt}%
\pgfpathmoveto{\pgfqpoint{5.800000in}{52.035697in}}%
\pgfpathlineto{\pgfqpoint{5.800000in}{52.035697in}}%
\pgfpathlineto{\pgfqpoint{5.800000in}{52.138534in}}%
\pgfpathlineto{\pgfqpoint{5.800000in}{52.138534in}}%
\pgfpathclose%
\pgfusepath{fill}%
\end{pgfscope}%
\begin{pgfscope}%
\pgfpathrectangle{\pgfqpoint{5.800000in}{0.720000in}}{\pgfqpoint{1.400000in}{4.620000in}}%
\pgfusepath{clip}%
\pgfsetbuttcap%
\pgfsetmiterjoin%
\definecolor{currentfill}{rgb}{0.121569,0.466667,0.705882}%
\pgfsetfillcolor{currentfill}%
\pgfsetlinewidth{0.000000pt}%
\definecolor{currentstroke}{rgb}{0.000000,0.000000,0.000000}%
\pgfsetstrokecolor{currentstroke}%
\pgfsetstrokeopacity{0.000000}%
\pgfsetdash{}{0pt}%
\pgfpathmoveto{\pgfqpoint{5.800000in}{52.138534in}}%
\pgfpathlineto{\pgfqpoint{5.800000in}{52.138534in}}%
\pgfpathlineto{\pgfqpoint{5.800000in}{52.241371in}}%
\pgfpathlineto{\pgfqpoint{5.800000in}{52.241371in}}%
\pgfpathclose%
\pgfusepath{fill}%
\end{pgfscope}%
\begin{pgfscope}%
\pgfpathrectangle{\pgfqpoint{5.800000in}{0.720000in}}{\pgfqpoint{1.400000in}{4.620000in}}%
\pgfusepath{clip}%
\pgfsetbuttcap%
\pgfsetmiterjoin%
\definecolor{currentfill}{rgb}{0.121569,0.466667,0.705882}%
\pgfsetfillcolor{currentfill}%
\pgfsetlinewidth{0.000000pt}%
\definecolor{currentstroke}{rgb}{0.000000,0.000000,0.000000}%
\pgfsetstrokecolor{currentstroke}%
\pgfsetstrokeopacity{0.000000}%
\pgfsetdash{}{0pt}%
\pgfpathmoveto{\pgfqpoint{5.800000in}{52.241371in}}%
\pgfpathlineto{\pgfqpoint{5.800000in}{52.241371in}}%
\pgfpathlineto{\pgfqpoint{5.800000in}{52.344208in}}%
\pgfpathlineto{\pgfqpoint{5.800000in}{52.344208in}}%
\pgfpathclose%
\pgfusepath{fill}%
\end{pgfscope}%
\begin{pgfscope}%
\pgfpathrectangle{\pgfqpoint{5.800000in}{0.720000in}}{\pgfqpoint{1.400000in}{4.620000in}}%
\pgfusepath{clip}%
\pgfsetbuttcap%
\pgfsetmiterjoin%
\definecolor{currentfill}{rgb}{0.121569,0.466667,0.705882}%
\pgfsetfillcolor{currentfill}%
\pgfsetlinewidth{0.000000pt}%
\definecolor{currentstroke}{rgb}{0.000000,0.000000,0.000000}%
\pgfsetstrokecolor{currentstroke}%
\pgfsetstrokeopacity{0.000000}%
\pgfsetdash{}{0pt}%
\pgfpathmoveto{\pgfqpoint{5.800000in}{52.344208in}}%
\pgfpathlineto{\pgfqpoint{5.800000in}{52.344208in}}%
\pgfpathlineto{\pgfqpoint{5.800000in}{52.447045in}}%
\pgfpathlineto{\pgfqpoint{5.800000in}{52.447045in}}%
\pgfpathclose%
\pgfusepath{fill}%
\end{pgfscope}%
\begin{pgfscope}%
\pgfpathrectangle{\pgfqpoint{5.800000in}{0.720000in}}{\pgfqpoint{1.400000in}{4.620000in}}%
\pgfusepath{clip}%
\pgfsetbuttcap%
\pgfsetmiterjoin%
\definecolor{currentfill}{rgb}{0.121569,0.466667,0.705882}%
\pgfsetfillcolor{currentfill}%
\pgfsetlinewidth{0.000000pt}%
\definecolor{currentstroke}{rgb}{0.000000,0.000000,0.000000}%
\pgfsetstrokecolor{currentstroke}%
\pgfsetstrokeopacity{0.000000}%
\pgfsetdash{}{0pt}%
\pgfpathmoveto{\pgfqpoint{5.800000in}{52.447045in}}%
\pgfpathlineto{\pgfqpoint{5.800000in}{52.447045in}}%
\pgfpathlineto{\pgfqpoint{5.800000in}{52.549882in}}%
\pgfpathlineto{\pgfqpoint{5.800000in}{52.549882in}}%
\pgfpathclose%
\pgfusepath{fill}%
\end{pgfscope}%
\begin{pgfscope}%
\pgfpathrectangle{\pgfqpoint{5.800000in}{0.720000in}}{\pgfqpoint{1.400000in}{4.620000in}}%
\pgfusepath{clip}%
\pgfsetbuttcap%
\pgfsetmiterjoin%
\definecolor{currentfill}{rgb}{0.121569,0.466667,0.705882}%
\pgfsetfillcolor{currentfill}%
\pgfsetlinewidth{0.000000pt}%
\definecolor{currentstroke}{rgb}{0.000000,0.000000,0.000000}%
\pgfsetstrokecolor{currentstroke}%
\pgfsetstrokeopacity{0.000000}%
\pgfsetdash{}{0pt}%
\pgfpathmoveto{\pgfqpoint{5.800000in}{52.549882in}}%
\pgfpathlineto{\pgfqpoint{5.800000in}{52.549882in}}%
\pgfpathlineto{\pgfqpoint{5.800000in}{52.652719in}}%
\pgfpathlineto{\pgfqpoint{5.800000in}{52.652719in}}%
\pgfpathclose%
\pgfusepath{fill}%
\end{pgfscope}%
\begin{pgfscope}%
\pgfpathrectangle{\pgfqpoint{5.800000in}{0.720000in}}{\pgfqpoint{1.400000in}{4.620000in}}%
\pgfusepath{clip}%
\pgfsetbuttcap%
\pgfsetmiterjoin%
\definecolor{currentfill}{rgb}{0.121569,0.466667,0.705882}%
\pgfsetfillcolor{currentfill}%
\pgfsetlinewidth{0.000000pt}%
\definecolor{currentstroke}{rgb}{0.000000,0.000000,0.000000}%
\pgfsetstrokecolor{currentstroke}%
\pgfsetstrokeopacity{0.000000}%
\pgfsetdash{}{0pt}%
\pgfpathmoveto{\pgfqpoint{5.800000in}{52.652719in}}%
\pgfpathlineto{\pgfqpoint{5.800000in}{52.652719in}}%
\pgfpathlineto{\pgfqpoint{5.800000in}{52.755556in}}%
\pgfpathlineto{\pgfqpoint{5.800000in}{52.755556in}}%
\pgfpathclose%
\pgfusepath{fill}%
\end{pgfscope}%
\begin{pgfscope}%
\pgfpathrectangle{\pgfqpoint{5.800000in}{0.720000in}}{\pgfqpoint{1.400000in}{4.620000in}}%
\pgfusepath{clip}%
\pgfsetbuttcap%
\pgfsetmiterjoin%
\definecolor{currentfill}{rgb}{0.121569,0.466667,0.705882}%
\pgfsetfillcolor{currentfill}%
\pgfsetlinewidth{0.000000pt}%
\definecolor{currentstroke}{rgb}{0.000000,0.000000,0.000000}%
\pgfsetstrokecolor{currentstroke}%
\pgfsetstrokeopacity{0.000000}%
\pgfsetdash{}{0pt}%
\pgfpathmoveto{\pgfqpoint{5.800000in}{52.755556in}}%
\pgfpathlineto{\pgfqpoint{5.800000in}{52.755556in}}%
\pgfpathlineto{\pgfqpoint{5.800000in}{52.858393in}}%
\pgfpathlineto{\pgfqpoint{5.800000in}{52.858393in}}%
\pgfpathclose%
\pgfusepath{fill}%
\end{pgfscope}%
\begin{pgfscope}%
\pgfpathrectangle{\pgfqpoint{5.800000in}{0.720000in}}{\pgfqpoint{1.400000in}{4.620000in}}%
\pgfusepath{clip}%
\pgfsetbuttcap%
\pgfsetmiterjoin%
\definecolor{currentfill}{rgb}{0.121569,0.466667,0.705882}%
\pgfsetfillcolor{currentfill}%
\pgfsetlinewidth{0.000000pt}%
\definecolor{currentstroke}{rgb}{0.000000,0.000000,0.000000}%
\pgfsetstrokecolor{currentstroke}%
\pgfsetstrokeopacity{0.000000}%
\pgfsetdash{}{0pt}%
\pgfpathmoveto{\pgfqpoint{5.800000in}{52.858393in}}%
\pgfpathlineto{\pgfqpoint{5.800000in}{52.858393in}}%
\pgfpathlineto{\pgfqpoint{5.800000in}{52.961230in}}%
\pgfpathlineto{\pgfqpoint{5.800000in}{52.961230in}}%
\pgfpathclose%
\pgfusepath{fill}%
\end{pgfscope}%
\begin{pgfscope}%
\pgfpathrectangle{\pgfqpoint{5.800000in}{0.720000in}}{\pgfqpoint{1.400000in}{4.620000in}}%
\pgfusepath{clip}%
\pgfsetbuttcap%
\pgfsetmiterjoin%
\definecolor{currentfill}{rgb}{0.121569,0.466667,0.705882}%
\pgfsetfillcolor{currentfill}%
\pgfsetlinewidth{0.000000pt}%
\definecolor{currentstroke}{rgb}{0.000000,0.000000,0.000000}%
\pgfsetstrokecolor{currentstroke}%
\pgfsetstrokeopacity{0.000000}%
\pgfsetdash{}{0pt}%
\pgfpathmoveto{\pgfqpoint{5.800000in}{52.961230in}}%
\pgfpathlineto{\pgfqpoint{5.800000in}{52.961230in}}%
\pgfpathlineto{\pgfqpoint{5.800000in}{53.064067in}}%
\pgfpathlineto{\pgfqpoint{5.800000in}{53.064067in}}%
\pgfpathclose%
\pgfusepath{fill}%
\end{pgfscope}%
\begin{pgfscope}%
\pgfpathrectangle{\pgfqpoint{5.800000in}{0.720000in}}{\pgfqpoint{1.400000in}{4.620000in}}%
\pgfusepath{clip}%
\pgfsetbuttcap%
\pgfsetmiterjoin%
\definecolor{currentfill}{rgb}{0.121569,0.466667,0.705882}%
\pgfsetfillcolor{currentfill}%
\pgfsetlinewidth{0.000000pt}%
\definecolor{currentstroke}{rgb}{0.000000,0.000000,0.000000}%
\pgfsetstrokecolor{currentstroke}%
\pgfsetstrokeopacity{0.000000}%
\pgfsetdash{}{0pt}%
\pgfpathmoveto{\pgfqpoint{5.800000in}{53.064067in}}%
\pgfpathlineto{\pgfqpoint{5.800000in}{53.064067in}}%
\pgfpathlineto{\pgfqpoint{5.800000in}{53.166905in}}%
\pgfpathlineto{\pgfqpoint{5.800000in}{53.166905in}}%
\pgfpathclose%
\pgfusepath{fill}%
\end{pgfscope}%
\begin{pgfscope}%
\pgfpathrectangle{\pgfqpoint{5.800000in}{0.720000in}}{\pgfqpoint{1.400000in}{4.620000in}}%
\pgfusepath{clip}%
\pgfsetbuttcap%
\pgfsetmiterjoin%
\definecolor{currentfill}{rgb}{0.121569,0.466667,0.705882}%
\pgfsetfillcolor{currentfill}%
\pgfsetlinewidth{0.000000pt}%
\definecolor{currentstroke}{rgb}{0.000000,0.000000,0.000000}%
\pgfsetstrokecolor{currentstroke}%
\pgfsetstrokeopacity{0.000000}%
\pgfsetdash{}{0pt}%
\pgfpathmoveto{\pgfqpoint{5.800000in}{53.166905in}}%
\pgfpathlineto{\pgfqpoint{5.800000in}{53.166905in}}%
\pgfpathlineto{\pgfqpoint{5.800000in}{53.269742in}}%
\pgfpathlineto{\pgfqpoint{5.800000in}{53.269742in}}%
\pgfpathclose%
\pgfusepath{fill}%
\end{pgfscope}%
\begin{pgfscope}%
\pgfpathrectangle{\pgfqpoint{5.800000in}{0.720000in}}{\pgfqpoint{1.400000in}{4.620000in}}%
\pgfusepath{clip}%
\pgfsetbuttcap%
\pgfsetmiterjoin%
\definecolor{currentfill}{rgb}{0.121569,0.466667,0.705882}%
\pgfsetfillcolor{currentfill}%
\pgfsetlinewidth{0.000000pt}%
\definecolor{currentstroke}{rgb}{0.000000,0.000000,0.000000}%
\pgfsetstrokecolor{currentstroke}%
\pgfsetstrokeopacity{0.000000}%
\pgfsetdash{}{0pt}%
\pgfpathmoveto{\pgfqpoint{5.800000in}{53.269742in}}%
\pgfpathlineto{\pgfqpoint{5.800000in}{53.269742in}}%
\pgfpathlineto{\pgfqpoint{5.800000in}{53.372579in}}%
\pgfpathlineto{\pgfqpoint{5.800000in}{53.372579in}}%
\pgfpathclose%
\pgfusepath{fill}%
\end{pgfscope}%
\begin{pgfscope}%
\pgfpathrectangle{\pgfqpoint{5.800000in}{0.720000in}}{\pgfqpoint{1.400000in}{4.620000in}}%
\pgfusepath{clip}%
\pgfsetbuttcap%
\pgfsetmiterjoin%
\definecolor{currentfill}{rgb}{0.121569,0.466667,0.705882}%
\pgfsetfillcolor{currentfill}%
\pgfsetlinewidth{0.000000pt}%
\definecolor{currentstroke}{rgb}{0.000000,0.000000,0.000000}%
\pgfsetstrokecolor{currentstroke}%
\pgfsetstrokeopacity{0.000000}%
\pgfsetdash{}{0pt}%
\pgfpathmoveto{\pgfqpoint{5.800000in}{53.372579in}}%
\pgfpathlineto{\pgfqpoint{5.800000in}{53.372579in}}%
\pgfpathlineto{\pgfqpoint{5.800000in}{53.475416in}}%
\pgfpathlineto{\pgfqpoint{5.800000in}{53.475416in}}%
\pgfpathclose%
\pgfusepath{fill}%
\end{pgfscope}%
\begin{pgfscope}%
\pgfpathrectangle{\pgfqpoint{5.800000in}{0.720000in}}{\pgfqpoint{1.400000in}{4.620000in}}%
\pgfusepath{clip}%
\pgfsetbuttcap%
\pgfsetmiterjoin%
\definecolor{currentfill}{rgb}{0.121569,0.466667,0.705882}%
\pgfsetfillcolor{currentfill}%
\pgfsetlinewidth{0.000000pt}%
\definecolor{currentstroke}{rgb}{0.000000,0.000000,0.000000}%
\pgfsetstrokecolor{currentstroke}%
\pgfsetstrokeopacity{0.000000}%
\pgfsetdash{}{0pt}%
\pgfpathmoveto{\pgfqpoint{5.800000in}{53.475416in}}%
\pgfpathlineto{\pgfqpoint{5.800000in}{53.475416in}}%
\pgfpathlineto{\pgfqpoint{5.800000in}{53.578253in}}%
\pgfpathlineto{\pgfqpoint{5.800000in}{53.578253in}}%
\pgfpathclose%
\pgfusepath{fill}%
\end{pgfscope}%
\begin{pgfscope}%
\pgfpathrectangle{\pgfqpoint{5.800000in}{0.720000in}}{\pgfqpoint{1.400000in}{4.620000in}}%
\pgfusepath{clip}%
\pgfsetbuttcap%
\pgfsetmiterjoin%
\definecolor{currentfill}{rgb}{0.121569,0.466667,0.705882}%
\pgfsetfillcolor{currentfill}%
\pgfsetlinewidth{0.000000pt}%
\definecolor{currentstroke}{rgb}{0.000000,0.000000,0.000000}%
\pgfsetstrokecolor{currentstroke}%
\pgfsetstrokeopacity{0.000000}%
\pgfsetdash{}{0pt}%
\pgfpathmoveto{\pgfqpoint{5.800000in}{53.578253in}}%
\pgfpathlineto{\pgfqpoint{5.800000in}{53.578253in}}%
\pgfpathlineto{\pgfqpoint{5.800000in}{53.681090in}}%
\pgfpathlineto{\pgfqpoint{5.800000in}{53.681090in}}%
\pgfpathclose%
\pgfusepath{fill}%
\end{pgfscope}%
\begin{pgfscope}%
\pgfpathrectangle{\pgfqpoint{5.800000in}{0.720000in}}{\pgfqpoint{1.400000in}{4.620000in}}%
\pgfusepath{clip}%
\pgfsetbuttcap%
\pgfsetmiterjoin%
\definecolor{currentfill}{rgb}{0.121569,0.466667,0.705882}%
\pgfsetfillcolor{currentfill}%
\pgfsetlinewidth{0.000000pt}%
\definecolor{currentstroke}{rgb}{0.000000,0.000000,0.000000}%
\pgfsetstrokecolor{currentstroke}%
\pgfsetstrokeopacity{0.000000}%
\pgfsetdash{}{0pt}%
\pgfpathmoveto{\pgfqpoint{5.800000in}{53.681090in}}%
\pgfpathlineto{\pgfqpoint{5.800000in}{53.681090in}}%
\pgfpathlineto{\pgfqpoint{5.800000in}{53.783927in}}%
\pgfpathlineto{\pgfqpoint{5.800000in}{53.783927in}}%
\pgfpathclose%
\pgfusepath{fill}%
\end{pgfscope}%
\begin{pgfscope}%
\pgfpathrectangle{\pgfqpoint{5.800000in}{0.720000in}}{\pgfqpoint{1.400000in}{4.620000in}}%
\pgfusepath{clip}%
\pgfsetbuttcap%
\pgfsetmiterjoin%
\definecolor{currentfill}{rgb}{0.121569,0.466667,0.705882}%
\pgfsetfillcolor{currentfill}%
\pgfsetlinewidth{0.000000pt}%
\definecolor{currentstroke}{rgb}{0.000000,0.000000,0.000000}%
\pgfsetstrokecolor{currentstroke}%
\pgfsetstrokeopacity{0.000000}%
\pgfsetdash{}{0pt}%
\pgfpathmoveto{\pgfqpoint{5.800000in}{53.783927in}}%
\pgfpathlineto{\pgfqpoint{5.800000in}{53.783927in}}%
\pgfpathlineto{\pgfqpoint{5.800000in}{53.886764in}}%
\pgfpathlineto{\pgfqpoint{5.800000in}{53.886764in}}%
\pgfpathclose%
\pgfusepath{fill}%
\end{pgfscope}%
\begin{pgfscope}%
\pgfpathrectangle{\pgfqpoint{5.800000in}{0.720000in}}{\pgfqpoint{1.400000in}{4.620000in}}%
\pgfusepath{clip}%
\pgfsetbuttcap%
\pgfsetmiterjoin%
\definecolor{currentfill}{rgb}{0.121569,0.466667,0.705882}%
\pgfsetfillcolor{currentfill}%
\pgfsetlinewidth{0.000000pt}%
\definecolor{currentstroke}{rgb}{0.000000,0.000000,0.000000}%
\pgfsetstrokecolor{currentstroke}%
\pgfsetstrokeopacity{0.000000}%
\pgfsetdash{}{0pt}%
\pgfpathmoveto{\pgfqpoint{5.800000in}{53.886764in}}%
\pgfpathlineto{\pgfqpoint{5.800000in}{53.886764in}}%
\pgfpathlineto{\pgfqpoint{5.800000in}{53.989601in}}%
\pgfpathlineto{\pgfqpoint{5.800000in}{53.989601in}}%
\pgfpathclose%
\pgfusepath{fill}%
\end{pgfscope}%
\begin{pgfscope}%
\pgfpathrectangle{\pgfqpoint{5.800000in}{0.720000in}}{\pgfqpoint{1.400000in}{4.620000in}}%
\pgfusepath{clip}%
\pgfsetbuttcap%
\pgfsetmiterjoin%
\definecolor{currentfill}{rgb}{0.121569,0.466667,0.705882}%
\pgfsetfillcolor{currentfill}%
\pgfsetlinewidth{0.000000pt}%
\definecolor{currentstroke}{rgb}{0.000000,0.000000,0.000000}%
\pgfsetstrokecolor{currentstroke}%
\pgfsetstrokeopacity{0.000000}%
\pgfsetdash{}{0pt}%
\pgfpathmoveto{\pgfqpoint{5.800000in}{53.989601in}}%
\pgfpathlineto{\pgfqpoint{5.800000in}{53.989601in}}%
\pgfpathlineto{\pgfqpoint{5.800000in}{54.092438in}}%
\pgfpathlineto{\pgfqpoint{5.800000in}{54.092438in}}%
\pgfpathclose%
\pgfusepath{fill}%
\end{pgfscope}%
\begin{pgfscope}%
\pgfpathrectangle{\pgfqpoint{5.800000in}{0.720000in}}{\pgfqpoint{1.400000in}{4.620000in}}%
\pgfusepath{clip}%
\pgfsetbuttcap%
\pgfsetmiterjoin%
\definecolor{currentfill}{rgb}{0.121569,0.466667,0.705882}%
\pgfsetfillcolor{currentfill}%
\pgfsetlinewidth{0.000000pt}%
\definecolor{currentstroke}{rgb}{0.000000,0.000000,0.000000}%
\pgfsetstrokecolor{currentstroke}%
\pgfsetstrokeopacity{0.000000}%
\pgfsetdash{}{0pt}%
\pgfpathmoveto{\pgfqpoint{5.800000in}{54.092438in}}%
\pgfpathlineto{\pgfqpoint{5.800000in}{54.092438in}}%
\pgfpathlineto{\pgfqpoint{5.800000in}{54.195275in}}%
\pgfpathlineto{\pgfqpoint{5.800000in}{54.195275in}}%
\pgfpathclose%
\pgfusepath{fill}%
\end{pgfscope}%
\begin{pgfscope}%
\pgfpathrectangle{\pgfqpoint{5.800000in}{0.720000in}}{\pgfqpoint{1.400000in}{4.620000in}}%
\pgfusepath{clip}%
\pgfsetbuttcap%
\pgfsetmiterjoin%
\definecolor{currentfill}{rgb}{0.121569,0.466667,0.705882}%
\pgfsetfillcolor{currentfill}%
\pgfsetlinewidth{0.000000pt}%
\definecolor{currentstroke}{rgb}{0.000000,0.000000,0.000000}%
\pgfsetstrokecolor{currentstroke}%
\pgfsetstrokeopacity{0.000000}%
\pgfsetdash{}{0pt}%
\pgfpathmoveto{\pgfqpoint{5.800000in}{54.195275in}}%
\pgfpathlineto{\pgfqpoint{5.800000in}{54.195275in}}%
\pgfpathlineto{\pgfqpoint{5.800000in}{54.298112in}}%
\pgfpathlineto{\pgfqpoint{5.800000in}{54.298112in}}%
\pgfpathclose%
\pgfusepath{fill}%
\end{pgfscope}%
\begin{pgfscope}%
\pgfpathrectangle{\pgfqpoint{5.800000in}{0.720000in}}{\pgfqpoint{1.400000in}{4.620000in}}%
\pgfusepath{clip}%
\pgfsetbuttcap%
\pgfsetmiterjoin%
\definecolor{currentfill}{rgb}{0.121569,0.466667,0.705882}%
\pgfsetfillcolor{currentfill}%
\pgfsetlinewidth{0.000000pt}%
\definecolor{currentstroke}{rgb}{0.000000,0.000000,0.000000}%
\pgfsetstrokecolor{currentstroke}%
\pgfsetstrokeopacity{0.000000}%
\pgfsetdash{}{0pt}%
\pgfpathmoveto{\pgfqpoint{5.800000in}{54.298112in}}%
\pgfpathlineto{\pgfqpoint{5.800000in}{54.298112in}}%
\pgfpathlineto{\pgfqpoint{5.800000in}{54.400949in}}%
\pgfpathlineto{\pgfqpoint{5.800000in}{54.400949in}}%
\pgfpathclose%
\pgfusepath{fill}%
\end{pgfscope}%
\begin{pgfscope}%
\pgfpathrectangle{\pgfqpoint{5.800000in}{0.720000in}}{\pgfqpoint{1.400000in}{4.620000in}}%
\pgfusepath{clip}%
\pgfsetbuttcap%
\pgfsetmiterjoin%
\definecolor{currentfill}{rgb}{0.121569,0.466667,0.705882}%
\pgfsetfillcolor{currentfill}%
\pgfsetlinewidth{0.000000pt}%
\definecolor{currentstroke}{rgb}{0.000000,0.000000,0.000000}%
\pgfsetstrokecolor{currentstroke}%
\pgfsetstrokeopacity{0.000000}%
\pgfsetdash{}{0pt}%
\pgfpathmoveto{\pgfqpoint{5.800000in}{54.400949in}}%
\pgfpathlineto{\pgfqpoint{5.800000in}{54.400949in}}%
\pgfpathlineto{\pgfqpoint{5.800000in}{54.503786in}}%
\pgfpathlineto{\pgfqpoint{5.800000in}{54.503786in}}%
\pgfpathclose%
\pgfusepath{fill}%
\end{pgfscope}%
\begin{pgfscope}%
\pgfpathrectangle{\pgfqpoint{5.800000in}{0.720000in}}{\pgfqpoint{1.400000in}{4.620000in}}%
\pgfusepath{clip}%
\pgfsetbuttcap%
\pgfsetmiterjoin%
\definecolor{currentfill}{rgb}{0.121569,0.466667,0.705882}%
\pgfsetfillcolor{currentfill}%
\pgfsetlinewidth{0.000000pt}%
\definecolor{currentstroke}{rgb}{0.000000,0.000000,0.000000}%
\pgfsetstrokecolor{currentstroke}%
\pgfsetstrokeopacity{0.000000}%
\pgfsetdash{}{0pt}%
\pgfpathmoveto{\pgfqpoint{5.800000in}{54.503786in}}%
\pgfpathlineto{\pgfqpoint{5.800000in}{54.503786in}}%
\pgfpathlineto{\pgfqpoint{5.800000in}{54.606624in}}%
\pgfpathlineto{\pgfqpoint{5.800000in}{54.606624in}}%
\pgfpathclose%
\pgfusepath{fill}%
\end{pgfscope}%
\begin{pgfscope}%
\pgfpathrectangle{\pgfqpoint{5.800000in}{0.720000in}}{\pgfqpoint{1.400000in}{4.620000in}}%
\pgfusepath{clip}%
\pgfsetbuttcap%
\pgfsetmiterjoin%
\definecolor{currentfill}{rgb}{0.121569,0.466667,0.705882}%
\pgfsetfillcolor{currentfill}%
\pgfsetlinewidth{0.000000pt}%
\definecolor{currentstroke}{rgb}{0.000000,0.000000,0.000000}%
\pgfsetstrokecolor{currentstroke}%
\pgfsetstrokeopacity{0.000000}%
\pgfsetdash{}{0pt}%
\pgfpathmoveto{\pgfqpoint{5.800000in}{54.606624in}}%
\pgfpathlineto{\pgfqpoint{5.800000in}{54.606624in}}%
\pgfpathlineto{\pgfqpoint{5.800000in}{54.709461in}}%
\pgfpathlineto{\pgfqpoint{5.800000in}{54.709461in}}%
\pgfpathclose%
\pgfusepath{fill}%
\end{pgfscope}%
\begin{pgfscope}%
\pgfpathrectangle{\pgfqpoint{5.800000in}{0.720000in}}{\pgfqpoint{1.400000in}{4.620000in}}%
\pgfusepath{clip}%
\pgfsetbuttcap%
\pgfsetmiterjoin%
\definecolor{currentfill}{rgb}{0.121569,0.466667,0.705882}%
\pgfsetfillcolor{currentfill}%
\pgfsetlinewidth{0.000000pt}%
\definecolor{currentstroke}{rgb}{0.000000,0.000000,0.000000}%
\pgfsetstrokecolor{currentstroke}%
\pgfsetstrokeopacity{0.000000}%
\pgfsetdash{}{0pt}%
\pgfpathmoveto{\pgfqpoint{5.800000in}{54.709461in}}%
\pgfpathlineto{\pgfqpoint{5.800000in}{54.709461in}}%
\pgfpathlineto{\pgfqpoint{5.800000in}{54.812298in}}%
\pgfpathlineto{\pgfqpoint{5.800000in}{54.812298in}}%
\pgfpathclose%
\pgfusepath{fill}%
\end{pgfscope}%
\begin{pgfscope}%
\pgfpathrectangle{\pgfqpoint{5.800000in}{0.720000in}}{\pgfqpoint{1.400000in}{4.620000in}}%
\pgfusepath{clip}%
\pgfsetbuttcap%
\pgfsetmiterjoin%
\definecolor{currentfill}{rgb}{0.121569,0.466667,0.705882}%
\pgfsetfillcolor{currentfill}%
\pgfsetlinewidth{0.000000pt}%
\definecolor{currentstroke}{rgb}{0.000000,0.000000,0.000000}%
\pgfsetstrokecolor{currentstroke}%
\pgfsetstrokeopacity{0.000000}%
\pgfsetdash{}{0pt}%
\pgfpathmoveto{\pgfqpoint{5.800000in}{54.812298in}}%
\pgfpathlineto{\pgfqpoint{5.800000in}{54.812298in}}%
\pgfpathlineto{\pgfqpoint{5.800000in}{54.915135in}}%
\pgfpathlineto{\pgfqpoint{5.800000in}{54.915135in}}%
\pgfpathclose%
\pgfusepath{fill}%
\end{pgfscope}%
\begin{pgfscope}%
\pgfpathrectangle{\pgfqpoint{5.800000in}{0.720000in}}{\pgfqpoint{1.400000in}{4.620000in}}%
\pgfusepath{clip}%
\pgfsetbuttcap%
\pgfsetmiterjoin%
\definecolor{currentfill}{rgb}{0.121569,0.466667,0.705882}%
\pgfsetfillcolor{currentfill}%
\pgfsetlinewidth{0.000000pt}%
\definecolor{currentstroke}{rgb}{0.000000,0.000000,0.000000}%
\pgfsetstrokecolor{currentstroke}%
\pgfsetstrokeopacity{0.000000}%
\pgfsetdash{}{0pt}%
\pgfpathmoveto{\pgfqpoint{5.800000in}{54.915135in}}%
\pgfpathlineto{\pgfqpoint{5.800000in}{54.915135in}}%
\pgfpathlineto{\pgfqpoint{5.800000in}{55.017972in}}%
\pgfpathlineto{\pgfqpoint{5.800000in}{55.017972in}}%
\pgfpathclose%
\pgfusepath{fill}%
\end{pgfscope}%
\begin{pgfscope}%
\pgfpathrectangle{\pgfqpoint{5.800000in}{0.720000in}}{\pgfqpoint{1.400000in}{4.620000in}}%
\pgfusepath{clip}%
\pgfsetbuttcap%
\pgfsetmiterjoin%
\definecolor{currentfill}{rgb}{0.121569,0.466667,0.705882}%
\pgfsetfillcolor{currentfill}%
\pgfsetlinewidth{0.000000pt}%
\definecolor{currentstroke}{rgb}{0.000000,0.000000,0.000000}%
\pgfsetstrokecolor{currentstroke}%
\pgfsetstrokeopacity{0.000000}%
\pgfsetdash{}{0pt}%
\pgfpathmoveto{\pgfqpoint{5.800000in}{55.017972in}}%
\pgfpathlineto{\pgfqpoint{5.800000in}{55.017972in}}%
\pgfpathlineto{\pgfqpoint{5.800000in}{55.120809in}}%
\pgfpathlineto{\pgfqpoint{5.800000in}{55.120809in}}%
\pgfpathclose%
\pgfusepath{fill}%
\end{pgfscope}%
\begin{pgfscope}%
\pgfpathrectangle{\pgfqpoint{5.800000in}{0.720000in}}{\pgfqpoint{1.400000in}{4.620000in}}%
\pgfusepath{clip}%
\pgfsetbuttcap%
\pgfsetmiterjoin%
\definecolor{currentfill}{rgb}{0.121569,0.466667,0.705882}%
\pgfsetfillcolor{currentfill}%
\pgfsetlinewidth{0.000000pt}%
\definecolor{currentstroke}{rgb}{0.000000,0.000000,0.000000}%
\pgfsetstrokecolor{currentstroke}%
\pgfsetstrokeopacity{0.000000}%
\pgfsetdash{}{0pt}%
\pgfpathmoveto{\pgfqpoint{5.800000in}{55.120809in}}%
\pgfpathlineto{\pgfqpoint{5.800000in}{55.120809in}}%
\pgfpathlineto{\pgfqpoint{5.800000in}{55.223646in}}%
\pgfpathlineto{\pgfqpoint{5.800000in}{55.223646in}}%
\pgfpathclose%
\pgfusepath{fill}%
\end{pgfscope}%
\begin{pgfscope}%
\pgfpathrectangle{\pgfqpoint{5.800000in}{0.720000in}}{\pgfqpoint{1.400000in}{4.620000in}}%
\pgfusepath{clip}%
\pgfsetbuttcap%
\pgfsetmiterjoin%
\definecolor{currentfill}{rgb}{0.121569,0.466667,0.705882}%
\pgfsetfillcolor{currentfill}%
\pgfsetlinewidth{0.000000pt}%
\definecolor{currentstroke}{rgb}{0.000000,0.000000,0.000000}%
\pgfsetstrokecolor{currentstroke}%
\pgfsetstrokeopacity{0.000000}%
\pgfsetdash{}{0pt}%
\pgfpathmoveto{\pgfqpoint{5.800000in}{55.223646in}}%
\pgfpathlineto{\pgfqpoint{5.800000in}{55.223646in}}%
\pgfpathlineto{\pgfqpoint{5.800000in}{55.326483in}}%
\pgfpathlineto{\pgfqpoint{5.800000in}{55.326483in}}%
\pgfpathclose%
\pgfusepath{fill}%
\end{pgfscope}%
\begin{pgfscope}%
\pgfpathrectangle{\pgfqpoint{5.800000in}{0.720000in}}{\pgfqpoint{1.400000in}{4.620000in}}%
\pgfusepath{clip}%
\pgfsetbuttcap%
\pgfsetmiterjoin%
\definecolor{currentfill}{rgb}{0.121569,0.466667,0.705882}%
\pgfsetfillcolor{currentfill}%
\pgfsetlinewidth{0.000000pt}%
\definecolor{currentstroke}{rgb}{0.000000,0.000000,0.000000}%
\pgfsetstrokecolor{currentstroke}%
\pgfsetstrokeopacity{0.000000}%
\pgfsetdash{}{0pt}%
\pgfpathmoveto{\pgfqpoint{5.800000in}{55.326483in}}%
\pgfpathlineto{\pgfqpoint{5.800000in}{55.326483in}}%
\pgfpathlineto{\pgfqpoint{5.800000in}{55.429320in}}%
\pgfpathlineto{\pgfqpoint{5.800000in}{55.429320in}}%
\pgfpathclose%
\pgfusepath{fill}%
\end{pgfscope}%
\begin{pgfscope}%
\pgfpathrectangle{\pgfqpoint{5.800000in}{0.720000in}}{\pgfqpoint{1.400000in}{4.620000in}}%
\pgfusepath{clip}%
\pgfsetbuttcap%
\pgfsetmiterjoin%
\definecolor{currentfill}{rgb}{0.121569,0.466667,0.705882}%
\pgfsetfillcolor{currentfill}%
\pgfsetlinewidth{0.000000pt}%
\definecolor{currentstroke}{rgb}{0.000000,0.000000,0.000000}%
\pgfsetstrokecolor{currentstroke}%
\pgfsetstrokeopacity{0.000000}%
\pgfsetdash{}{0pt}%
\pgfpathmoveto{\pgfqpoint{5.800000in}{55.429320in}}%
\pgfpathlineto{\pgfqpoint{5.800000in}{55.429320in}}%
\pgfpathlineto{\pgfqpoint{5.800000in}{55.532157in}}%
\pgfpathlineto{\pgfqpoint{5.800000in}{55.532157in}}%
\pgfpathclose%
\pgfusepath{fill}%
\end{pgfscope}%
\begin{pgfscope}%
\pgfpathrectangle{\pgfqpoint{5.800000in}{0.720000in}}{\pgfqpoint{1.400000in}{4.620000in}}%
\pgfusepath{clip}%
\pgfsetbuttcap%
\pgfsetmiterjoin%
\definecolor{currentfill}{rgb}{0.121569,0.466667,0.705882}%
\pgfsetfillcolor{currentfill}%
\pgfsetlinewidth{0.000000pt}%
\definecolor{currentstroke}{rgb}{0.000000,0.000000,0.000000}%
\pgfsetstrokecolor{currentstroke}%
\pgfsetstrokeopacity{0.000000}%
\pgfsetdash{}{0pt}%
\pgfpathmoveto{\pgfqpoint{5.800000in}{55.532157in}}%
\pgfpathlineto{\pgfqpoint{5.800000in}{55.532157in}}%
\pgfpathlineto{\pgfqpoint{5.800000in}{55.634994in}}%
\pgfpathlineto{\pgfqpoint{5.800000in}{55.634994in}}%
\pgfpathclose%
\pgfusepath{fill}%
\end{pgfscope}%
\begin{pgfscope}%
\pgfpathrectangle{\pgfqpoint{5.800000in}{0.720000in}}{\pgfqpoint{1.400000in}{4.620000in}}%
\pgfusepath{clip}%
\pgfsetbuttcap%
\pgfsetmiterjoin%
\definecolor{currentfill}{rgb}{0.121569,0.466667,0.705882}%
\pgfsetfillcolor{currentfill}%
\pgfsetlinewidth{0.000000pt}%
\definecolor{currentstroke}{rgb}{0.000000,0.000000,0.000000}%
\pgfsetstrokecolor{currentstroke}%
\pgfsetstrokeopacity{0.000000}%
\pgfsetdash{}{0pt}%
\pgfpathmoveto{\pgfqpoint{5.800000in}{55.634994in}}%
\pgfpathlineto{\pgfqpoint{5.800000in}{55.634994in}}%
\pgfpathlineto{\pgfqpoint{5.800000in}{55.737831in}}%
\pgfpathlineto{\pgfqpoint{5.800000in}{55.737831in}}%
\pgfpathclose%
\pgfusepath{fill}%
\end{pgfscope}%
\begin{pgfscope}%
\pgfpathrectangle{\pgfqpoint{5.800000in}{0.720000in}}{\pgfqpoint{1.400000in}{4.620000in}}%
\pgfusepath{clip}%
\pgfsetbuttcap%
\pgfsetmiterjoin%
\definecolor{currentfill}{rgb}{0.121569,0.466667,0.705882}%
\pgfsetfillcolor{currentfill}%
\pgfsetlinewidth{0.000000pt}%
\definecolor{currentstroke}{rgb}{0.000000,0.000000,0.000000}%
\pgfsetstrokecolor{currentstroke}%
\pgfsetstrokeopacity{0.000000}%
\pgfsetdash{}{0pt}%
\pgfpathmoveto{\pgfqpoint{5.800000in}{55.737831in}}%
\pgfpathlineto{\pgfqpoint{5.800000in}{55.737831in}}%
\pgfpathlineto{\pgfqpoint{5.800000in}{55.840668in}}%
\pgfpathlineto{\pgfqpoint{5.800000in}{55.840668in}}%
\pgfpathclose%
\pgfusepath{fill}%
\end{pgfscope}%
\begin{pgfscope}%
\pgfpathrectangle{\pgfqpoint{5.800000in}{0.720000in}}{\pgfqpoint{1.400000in}{4.620000in}}%
\pgfusepath{clip}%
\pgfsetbuttcap%
\pgfsetmiterjoin%
\definecolor{currentfill}{rgb}{0.121569,0.466667,0.705882}%
\pgfsetfillcolor{currentfill}%
\pgfsetlinewidth{0.000000pt}%
\definecolor{currentstroke}{rgb}{0.000000,0.000000,0.000000}%
\pgfsetstrokecolor{currentstroke}%
\pgfsetstrokeopacity{0.000000}%
\pgfsetdash{}{0pt}%
\pgfpathmoveto{\pgfqpoint{5.800000in}{55.840668in}}%
\pgfpathlineto{\pgfqpoint{5.800000in}{55.840668in}}%
\pgfpathlineto{\pgfqpoint{5.800000in}{55.943505in}}%
\pgfpathlineto{\pgfqpoint{5.800000in}{55.943505in}}%
\pgfpathclose%
\pgfusepath{fill}%
\end{pgfscope}%
\begin{pgfscope}%
\pgfpathrectangle{\pgfqpoint{5.800000in}{0.720000in}}{\pgfqpoint{1.400000in}{4.620000in}}%
\pgfusepath{clip}%
\pgfsetbuttcap%
\pgfsetmiterjoin%
\definecolor{currentfill}{rgb}{0.121569,0.466667,0.705882}%
\pgfsetfillcolor{currentfill}%
\pgfsetlinewidth{0.000000pt}%
\definecolor{currentstroke}{rgb}{0.000000,0.000000,0.000000}%
\pgfsetstrokecolor{currentstroke}%
\pgfsetstrokeopacity{0.000000}%
\pgfsetdash{}{0pt}%
\pgfpathmoveto{\pgfqpoint{5.800000in}{55.943505in}}%
\pgfpathlineto{\pgfqpoint{5.800000in}{55.943505in}}%
\pgfpathlineto{\pgfqpoint{5.800000in}{56.046342in}}%
\pgfpathlineto{\pgfqpoint{5.800000in}{56.046342in}}%
\pgfpathclose%
\pgfusepath{fill}%
\end{pgfscope}%
\begin{pgfscope}%
\pgfpathrectangle{\pgfqpoint{5.800000in}{0.720000in}}{\pgfqpoint{1.400000in}{4.620000in}}%
\pgfusepath{clip}%
\pgfsetbuttcap%
\pgfsetmiterjoin%
\definecolor{currentfill}{rgb}{0.121569,0.466667,0.705882}%
\pgfsetfillcolor{currentfill}%
\pgfsetlinewidth{0.000000pt}%
\definecolor{currentstroke}{rgb}{0.000000,0.000000,0.000000}%
\pgfsetstrokecolor{currentstroke}%
\pgfsetstrokeopacity{0.000000}%
\pgfsetdash{}{0pt}%
\pgfpathmoveto{\pgfqpoint{5.800000in}{56.046342in}}%
\pgfpathlineto{\pgfqpoint{5.800000in}{56.046342in}}%
\pgfpathlineto{\pgfqpoint{5.800000in}{56.149180in}}%
\pgfpathlineto{\pgfqpoint{5.800000in}{56.149180in}}%
\pgfpathclose%
\pgfusepath{fill}%
\end{pgfscope}%
\begin{pgfscope}%
\pgfpathrectangle{\pgfqpoint{5.800000in}{0.720000in}}{\pgfqpoint{1.400000in}{4.620000in}}%
\pgfusepath{clip}%
\pgfsetbuttcap%
\pgfsetmiterjoin%
\definecolor{currentfill}{rgb}{0.121569,0.466667,0.705882}%
\pgfsetfillcolor{currentfill}%
\pgfsetlinewidth{0.000000pt}%
\definecolor{currentstroke}{rgb}{0.000000,0.000000,0.000000}%
\pgfsetstrokecolor{currentstroke}%
\pgfsetstrokeopacity{0.000000}%
\pgfsetdash{}{0pt}%
\pgfpathmoveto{\pgfqpoint{5.800000in}{56.149180in}}%
\pgfpathlineto{\pgfqpoint{5.800000in}{56.149180in}}%
\pgfpathlineto{\pgfqpoint{5.800000in}{56.252017in}}%
\pgfpathlineto{\pgfqpoint{5.800000in}{56.252017in}}%
\pgfpathclose%
\pgfusepath{fill}%
\end{pgfscope}%
\begin{pgfscope}%
\pgfpathrectangle{\pgfqpoint{5.800000in}{0.720000in}}{\pgfqpoint{1.400000in}{4.620000in}}%
\pgfusepath{clip}%
\pgfsetbuttcap%
\pgfsetmiterjoin%
\definecolor{currentfill}{rgb}{0.121569,0.466667,0.705882}%
\pgfsetfillcolor{currentfill}%
\pgfsetlinewidth{0.000000pt}%
\definecolor{currentstroke}{rgb}{0.000000,0.000000,0.000000}%
\pgfsetstrokecolor{currentstroke}%
\pgfsetstrokeopacity{0.000000}%
\pgfsetdash{}{0pt}%
\pgfpathmoveto{\pgfqpoint{5.800000in}{56.252017in}}%
\pgfpathlineto{\pgfqpoint{5.800000in}{56.252017in}}%
\pgfpathlineto{\pgfqpoint{5.800000in}{56.354854in}}%
\pgfpathlineto{\pgfqpoint{5.800000in}{56.354854in}}%
\pgfpathclose%
\pgfusepath{fill}%
\end{pgfscope}%
\begin{pgfscope}%
\pgfpathrectangle{\pgfqpoint{5.800000in}{0.720000in}}{\pgfqpoint{1.400000in}{4.620000in}}%
\pgfusepath{clip}%
\pgfsetbuttcap%
\pgfsetmiterjoin%
\definecolor{currentfill}{rgb}{0.121569,0.466667,0.705882}%
\pgfsetfillcolor{currentfill}%
\pgfsetlinewidth{0.000000pt}%
\definecolor{currentstroke}{rgb}{0.000000,0.000000,0.000000}%
\pgfsetstrokecolor{currentstroke}%
\pgfsetstrokeopacity{0.000000}%
\pgfsetdash{}{0pt}%
\pgfpathmoveto{\pgfqpoint{5.800000in}{56.354854in}}%
\pgfpathlineto{\pgfqpoint{5.800000in}{56.354854in}}%
\pgfpathlineto{\pgfqpoint{5.800000in}{56.457691in}}%
\pgfpathlineto{\pgfqpoint{5.800000in}{56.457691in}}%
\pgfpathclose%
\pgfusepath{fill}%
\end{pgfscope}%
\begin{pgfscope}%
\pgfpathrectangle{\pgfqpoint{5.800000in}{0.720000in}}{\pgfqpoint{1.400000in}{4.620000in}}%
\pgfusepath{clip}%
\pgfsetbuttcap%
\pgfsetmiterjoin%
\definecolor{currentfill}{rgb}{0.121569,0.466667,0.705882}%
\pgfsetfillcolor{currentfill}%
\pgfsetlinewidth{0.000000pt}%
\definecolor{currentstroke}{rgb}{0.000000,0.000000,0.000000}%
\pgfsetstrokecolor{currentstroke}%
\pgfsetstrokeopacity{0.000000}%
\pgfsetdash{}{0pt}%
\pgfpathmoveto{\pgfqpoint{5.800000in}{56.457691in}}%
\pgfpathlineto{\pgfqpoint{5.800000in}{56.457691in}}%
\pgfpathlineto{\pgfqpoint{5.800000in}{56.560528in}}%
\pgfpathlineto{\pgfqpoint{5.800000in}{56.560528in}}%
\pgfpathclose%
\pgfusepath{fill}%
\end{pgfscope}%
\begin{pgfscope}%
\pgfpathrectangle{\pgfqpoint{5.800000in}{0.720000in}}{\pgfqpoint{1.400000in}{4.620000in}}%
\pgfusepath{clip}%
\pgfsetbuttcap%
\pgfsetmiterjoin%
\definecolor{currentfill}{rgb}{0.121569,0.466667,0.705882}%
\pgfsetfillcolor{currentfill}%
\pgfsetlinewidth{0.000000pt}%
\definecolor{currentstroke}{rgb}{0.000000,0.000000,0.000000}%
\pgfsetstrokecolor{currentstroke}%
\pgfsetstrokeopacity{0.000000}%
\pgfsetdash{}{0pt}%
\pgfpathmoveto{\pgfqpoint{5.800000in}{56.560528in}}%
\pgfpathlineto{\pgfqpoint{5.800000in}{56.560528in}}%
\pgfpathlineto{\pgfqpoint{5.800000in}{56.663365in}}%
\pgfpathlineto{\pgfqpoint{5.800000in}{56.663365in}}%
\pgfpathclose%
\pgfusepath{fill}%
\end{pgfscope}%
\begin{pgfscope}%
\pgfpathrectangle{\pgfqpoint{5.800000in}{0.720000in}}{\pgfqpoint{1.400000in}{4.620000in}}%
\pgfusepath{clip}%
\pgfsetbuttcap%
\pgfsetmiterjoin%
\definecolor{currentfill}{rgb}{0.121569,0.466667,0.705882}%
\pgfsetfillcolor{currentfill}%
\pgfsetlinewidth{0.000000pt}%
\definecolor{currentstroke}{rgb}{0.000000,0.000000,0.000000}%
\pgfsetstrokecolor{currentstroke}%
\pgfsetstrokeopacity{0.000000}%
\pgfsetdash{}{0pt}%
\pgfpathmoveto{\pgfqpoint{5.800000in}{56.663365in}}%
\pgfpathlineto{\pgfqpoint{5.800000in}{56.663365in}}%
\pgfpathlineto{\pgfqpoint{5.800000in}{56.766202in}}%
\pgfpathlineto{\pgfqpoint{5.800000in}{56.766202in}}%
\pgfpathclose%
\pgfusepath{fill}%
\end{pgfscope}%
\begin{pgfscope}%
\pgfpathrectangle{\pgfqpoint{5.800000in}{0.720000in}}{\pgfqpoint{1.400000in}{4.620000in}}%
\pgfusepath{clip}%
\pgfsetbuttcap%
\pgfsetmiterjoin%
\definecolor{currentfill}{rgb}{0.121569,0.466667,0.705882}%
\pgfsetfillcolor{currentfill}%
\pgfsetlinewidth{0.000000pt}%
\definecolor{currentstroke}{rgb}{0.000000,0.000000,0.000000}%
\pgfsetstrokecolor{currentstroke}%
\pgfsetstrokeopacity{0.000000}%
\pgfsetdash{}{0pt}%
\pgfpathmoveto{\pgfqpoint{5.800000in}{56.766202in}}%
\pgfpathlineto{\pgfqpoint{5.800000in}{56.766202in}}%
\pgfpathlineto{\pgfqpoint{5.800000in}{56.869039in}}%
\pgfpathlineto{\pgfqpoint{5.800000in}{56.869039in}}%
\pgfpathclose%
\pgfusepath{fill}%
\end{pgfscope}%
\begin{pgfscope}%
\pgfpathrectangle{\pgfqpoint{5.800000in}{0.720000in}}{\pgfqpoint{1.400000in}{4.620000in}}%
\pgfusepath{clip}%
\pgfsetbuttcap%
\pgfsetmiterjoin%
\definecolor{currentfill}{rgb}{0.121569,0.466667,0.705882}%
\pgfsetfillcolor{currentfill}%
\pgfsetlinewidth{0.000000pt}%
\definecolor{currentstroke}{rgb}{0.000000,0.000000,0.000000}%
\pgfsetstrokecolor{currentstroke}%
\pgfsetstrokeopacity{0.000000}%
\pgfsetdash{}{0pt}%
\pgfpathmoveto{\pgfqpoint{5.800000in}{56.869039in}}%
\pgfpathlineto{\pgfqpoint{5.800000in}{56.869039in}}%
\pgfpathlineto{\pgfqpoint{5.800000in}{56.971876in}}%
\pgfpathlineto{\pgfqpoint{5.800000in}{56.971876in}}%
\pgfpathclose%
\pgfusepath{fill}%
\end{pgfscope}%
\begin{pgfscope}%
\pgfpathrectangle{\pgfqpoint{5.800000in}{0.720000in}}{\pgfqpoint{1.400000in}{4.620000in}}%
\pgfusepath{clip}%
\pgfsetbuttcap%
\pgfsetmiterjoin%
\definecolor{currentfill}{rgb}{0.121569,0.466667,0.705882}%
\pgfsetfillcolor{currentfill}%
\pgfsetlinewidth{0.000000pt}%
\definecolor{currentstroke}{rgb}{0.000000,0.000000,0.000000}%
\pgfsetstrokecolor{currentstroke}%
\pgfsetstrokeopacity{0.000000}%
\pgfsetdash{}{0pt}%
\pgfpathmoveto{\pgfqpoint{5.800000in}{56.971876in}}%
\pgfpathlineto{\pgfqpoint{5.800000in}{56.971876in}}%
\pgfpathlineto{\pgfqpoint{5.800000in}{57.074713in}}%
\pgfpathlineto{\pgfqpoint{5.800000in}{57.074713in}}%
\pgfpathclose%
\pgfusepath{fill}%
\end{pgfscope}%
\begin{pgfscope}%
\pgfpathrectangle{\pgfqpoint{5.800000in}{0.720000in}}{\pgfqpoint{1.400000in}{4.620000in}}%
\pgfusepath{clip}%
\pgfsetbuttcap%
\pgfsetmiterjoin%
\definecolor{currentfill}{rgb}{0.121569,0.466667,0.705882}%
\pgfsetfillcolor{currentfill}%
\pgfsetlinewidth{0.000000pt}%
\definecolor{currentstroke}{rgb}{0.000000,0.000000,0.000000}%
\pgfsetstrokecolor{currentstroke}%
\pgfsetstrokeopacity{0.000000}%
\pgfsetdash{}{0pt}%
\pgfpathmoveto{\pgfqpoint{5.800000in}{57.074713in}}%
\pgfpathlineto{\pgfqpoint{5.800000in}{57.074713in}}%
\pgfpathlineto{\pgfqpoint{5.800000in}{57.177550in}}%
\pgfpathlineto{\pgfqpoint{5.800000in}{57.177550in}}%
\pgfpathclose%
\pgfusepath{fill}%
\end{pgfscope}%
\begin{pgfscope}%
\pgfpathrectangle{\pgfqpoint{5.800000in}{0.720000in}}{\pgfqpoint{1.400000in}{4.620000in}}%
\pgfusepath{clip}%
\pgfsetbuttcap%
\pgfsetmiterjoin%
\definecolor{currentfill}{rgb}{0.121569,0.466667,0.705882}%
\pgfsetfillcolor{currentfill}%
\pgfsetlinewidth{0.000000pt}%
\definecolor{currentstroke}{rgb}{0.000000,0.000000,0.000000}%
\pgfsetstrokecolor{currentstroke}%
\pgfsetstrokeopacity{0.000000}%
\pgfsetdash{}{0pt}%
\pgfpathmoveto{\pgfqpoint{5.800000in}{57.177550in}}%
\pgfpathlineto{\pgfqpoint{5.800000in}{57.177550in}}%
\pgfpathlineto{\pgfqpoint{5.800000in}{57.280387in}}%
\pgfpathlineto{\pgfqpoint{5.800000in}{57.280387in}}%
\pgfpathclose%
\pgfusepath{fill}%
\end{pgfscope}%
\begin{pgfscope}%
\pgfpathrectangle{\pgfqpoint{5.800000in}{0.720000in}}{\pgfqpoint{1.400000in}{4.620000in}}%
\pgfusepath{clip}%
\pgfsetbuttcap%
\pgfsetmiterjoin%
\definecolor{currentfill}{rgb}{0.121569,0.466667,0.705882}%
\pgfsetfillcolor{currentfill}%
\pgfsetlinewidth{0.000000pt}%
\definecolor{currentstroke}{rgb}{0.000000,0.000000,0.000000}%
\pgfsetstrokecolor{currentstroke}%
\pgfsetstrokeopacity{0.000000}%
\pgfsetdash{}{0pt}%
\pgfpathmoveto{\pgfqpoint{5.800000in}{57.280387in}}%
\pgfpathlineto{\pgfqpoint{5.800000in}{57.280387in}}%
\pgfpathlineto{\pgfqpoint{5.800000in}{57.383224in}}%
\pgfpathlineto{\pgfqpoint{5.800000in}{57.383224in}}%
\pgfpathclose%
\pgfusepath{fill}%
\end{pgfscope}%
\begin{pgfscope}%
\pgfpathrectangle{\pgfqpoint{5.800000in}{0.720000in}}{\pgfqpoint{1.400000in}{4.620000in}}%
\pgfusepath{clip}%
\pgfsetbuttcap%
\pgfsetmiterjoin%
\definecolor{currentfill}{rgb}{0.121569,0.466667,0.705882}%
\pgfsetfillcolor{currentfill}%
\pgfsetlinewidth{0.000000pt}%
\definecolor{currentstroke}{rgb}{0.000000,0.000000,0.000000}%
\pgfsetstrokecolor{currentstroke}%
\pgfsetstrokeopacity{0.000000}%
\pgfsetdash{}{0pt}%
\pgfpathmoveto{\pgfqpoint{5.800000in}{57.383224in}}%
\pgfpathlineto{\pgfqpoint{5.800000in}{57.383224in}}%
\pgfpathlineto{\pgfqpoint{5.800000in}{57.486061in}}%
\pgfpathlineto{\pgfqpoint{5.800000in}{57.486061in}}%
\pgfpathclose%
\pgfusepath{fill}%
\end{pgfscope}%
\begin{pgfscope}%
\pgfpathrectangle{\pgfqpoint{5.800000in}{0.720000in}}{\pgfqpoint{1.400000in}{4.620000in}}%
\pgfusepath{clip}%
\pgfsetbuttcap%
\pgfsetmiterjoin%
\definecolor{currentfill}{rgb}{0.121569,0.466667,0.705882}%
\pgfsetfillcolor{currentfill}%
\pgfsetlinewidth{0.000000pt}%
\definecolor{currentstroke}{rgb}{0.000000,0.000000,0.000000}%
\pgfsetstrokecolor{currentstroke}%
\pgfsetstrokeopacity{0.000000}%
\pgfsetdash{}{0pt}%
\pgfpathmoveto{\pgfqpoint{5.800000in}{57.486061in}}%
\pgfpathlineto{\pgfqpoint{5.800000in}{57.486061in}}%
\pgfpathlineto{\pgfqpoint{5.800000in}{57.588898in}}%
\pgfpathlineto{\pgfqpoint{5.800000in}{57.588898in}}%
\pgfpathclose%
\pgfusepath{fill}%
\end{pgfscope}%
\begin{pgfscope}%
\pgfpathrectangle{\pgfqpoint{5.800000in}{0.720000in}}{\pgfqpoint{1.400000in}{4.620000in}}%
\pgfusepath{clip}%
\pgfsetbuttcap%
\pgfsetmiterjoin%
\definecolor{currentfill}{rgb}{0.121569,0.466667,0.705882}%
\pgfsetfillcolor{currentfill}%
\pgfsetlinewidth{0.000000pt}%
\definecolor{currentstroke}{rgb}{0.000000,0.000000,0.000000}%
\pgfsetstrokecolor{currentstroke}%
\pgfsetstrokeopacity{0.000000}%
\pgfsetdash{}{0pt}%
\pgfpathmoveto{\pgfqpoint{5.800000in}{57.588898in}}%
\pgfpathlineto{\pgfqpoint{5.800000in}{57.588898in}}%
\pgfpathlineto{\pgfqpoint{5.800000in}{57.691736in}}%
\pgfpathlineto{\pgfqpoint{5.800000in}{57.691736in}}%
\pgfpathclose%
\pgfusepath{fill}%
\end{pgfscope}%
\begin{pgfscope}%
\pgfpathrectangle{\pgfqpoint{5.800000in}{0.720000in}}{\pgfqpoint{1.400000in}{4.620000in}}%
\pgfusepath{clip}%
\pgfsetbuttcap%
\pgfsetmiterjoin%
\definecolor{currentfill}{rgb}{0.121569,0.466667,0.705882}%
\pgfsetfillcolor{currentfill}%
\pgfsetlinewidth{0.000000pt}%
\definecolor{currentstroke}{rgb}{0.000000,0.000000,0.000000}%
\pgfsetstrokecolor{currentstroke}%
\pgfsetstrokeopacity{0.000000}%
\pgfsetdash{}{0pt}%
\pgfpathmoveto{\pgfqpoint{5.800000in}{57.691736in}}%
\pgfpathlineto{\pgfqpoint{5.800000in}{57.691736in}}%
\pgfpathlineto{\pgfqpoint{5.800000in}{57.794573in}}%
\pgfpathlineto{\pgfqpoint{5.800000in}{57.794573in}}%
\pgfpathclose%
\pgfusepath{fill}%
\end{pgfscope}%
\begin{pgfscope}%
\pgfpathrectangle{\pgfqpoint{5.800000in}{0.720000in}}{\pgfqpoint{1.400000in}{4.620000in}}%
\pgfusepath{clip}%
\pgfsetbuttcap%
\pgfsetmiterjoin%
\definecolor{currentfill}{rgb}{0.121569,0.466667,0.705882}%
\pgfsetfillcolor{currentfill}%
\pgfsetlinewidth{0.000000pt}%
\definecolor{currentstroke}{rgb}{0.000000,0.000000,0.000000}%
\pgfsetstrokecolor{currentstroke}%
\pgfsetstrokeopacity{0.000000}%
\pgfsetdash{}{0pt}%
\pgfpathmoveto{\pgfqpoint{5.800000in}{57.794573in}}%
\pgfpathlineto{\pgfqpoint{5.800000in}{57.794573in}}%
\pgfpathlineto{\pgfqpoint{5.800000in}{57.897410in}}%
\pgfpathlineto{\pgfqpoint{5.800000in}{57.897410in}}%
\pgfpathclose%
\pgfusepath{fill}%
\end{pgfscope}%
\begin{pgfscope}%
\pgfpathrectangle{\pgfqpoint{5.800000in}{0.720000in}}{\pgfqpoint{1.400000in}{4.620000in}}%
\pgfusepath{clip}%
\pgfsetbuttcap%
\pgfsetmiterjoin%
\definecolor{currentfill}{rgb}{0.121569,0.466667,0.705882}%
\pgfsetfillcolor{currentfill}%
\pgfsetlinewidth{0.000000pt}%
\definecolor{currentstroke}{rgb}{0.000000,0.000000,0.000000}%
\pgfsetstrokecolor{currentstroke}%
\pgfsetstrokeopacity{0.000000}%
\pgfsetdash{}{0pt}%
\pgfpathmoveto{\pgfqpoint{5.800000in}{57.897410in}}%
\pgfpathlineto{\pgfqpoint{5.800000in}{57.897410in}}%
\pgfpathlineto{\pgfqpoint{5.800000in}{58.000247in}}%
\pgfpathlineto{\pgfqpoint{5.800000in}{58.000247in}}%
\pgfpathclose%
\pgfusepath{fill}%
\end{pgfscope}%
\begin{pgfscope}%
\pgfpathrectangle{\pgfqpoint{5.800000in}{0.720000in}}{\pgfqpoint{1.400000in}{4.620000in}}%
\pgfusepath{clip}%
\pgfsetbuttcap%
\pgfsetmiterjoin%
\definecolor{currentfill}{rgb}{0.121569,0.466667,0.705882}%
\pgfsetfillcolor{currentfill}%
\pgfsetlinewidth{0.000000pt}%
\definecolor{currentstroke}{rgb}{0.000000,0.000000,0.000000}%
\pgfsetstrokecolor{currentstroke}%
\pgfsetstrokeopacity{0.000000}%
\pgfsetdash{}{0pt}%
\pgfpathmoveto{\pgfqpoint{5.800000in}{58.000247in}}%
\pgfpathlineto{\pgfqpoint{5.800000in}{58.000247in}}%
\pgfpathlineto{\pgfqpoint{5.800000in}{58.103084in}}%
\pgfpathlineto{\pgfqpoint{5.800000in}{58.103084in}}%
\pgfpathclose%
\pgfusepath{fill}%
\end{pgfscope}%
\begin{pgfscope}%
\pgfpathrectangle{\pgfqpoint{5.800000in}{0.720000in}}{\pgfqpoint{1.400000in}{4.620000in}}%
\pgfusepath{clip}%
\pgfsetbuttcap%
\pgfsetmiterjoin%
\definecolor{currentfill}{rgb}{0.121569,0.466667,0.705882}%
\pgfsetfillcolor{currentfill}%
\pgfsetlinewidth{0.000000pt}%
\definecolor{currentstroke}{rgb}{0.000000,0.000000,0.000000}%
\pgfsetstrokecolor{currentstroke}%
\pgfsetstrokeopacity{0.000000}%
\pgfsetdash{}{0pt}%
\pgfpathmoveto{\pgfqpoint{5.800000in}{58.103084in}}%
\pgfpathlineto{\pgfqpoint{5.800000in}{58.103084in}}%
\pgfpathlineto{\pgfqpoint{5.800000in}{58.205921in}}%
\pgfpathlineto{\pgfqpoint{5.800000in}{58.205921in}}%
\pgfpathclose%
\pgfusepath{fill}%
\end{pgfscope}%
\begin{pgfscope}%
\pgfpathrectangle{\pgfqpoint{5.800000in}{0.720000in}}{\pgfqpoint{1.400000in}{4.620000in}}%
\pgfusepath{clip}%
\pgfsetbuttcap%
\pgfsetmiterjoin%
\definecolor{currentfill}{rgb}{0.121569,0.466667,0.705882}%
\pgfsetfillcolor{currentfill}%
\pgfsetlinewidth{0.000000pt}%
\definecolor{currentstroke}{rgb}{0.000000,0.000000,0.000000}%
\pgfsetstrokecolor{currentstroke}%
\pgfsetstrokeopacity{0.000000}%
\pgfsetdash{}{0pt}%
\pgfpathmoveto{\pgfqpoint{5.800000in}{58.205921in}}%
\pgfpathlineto{\pgfqpoint{5.800000in}{58.205921in}}%
\pgfpathlineto{\pgfqpoint{5.800000in}{58.308758in}}%
\pgfpathlineto{\pgfqpoint{5.800000in}{58.308758in}}%
\pgfpathclose%
\pgfusepath{fill}%
\end{pgfscope}%
\begin{pgfscope}%
\pgfpathrectangle{\pgfqpoint{5.800000in}{0.720000in}}{\pgfqpoint{1.400000in}{4.620000in}}%
\pgfusepath{clip}%
\pgfsetbuttcap%
\pgfsetmiterjoin%
\definecolor{currentfill}{rgb}{0.121569,0.466667,0.705882}%
\pgfsetfillcolor{currentfill}%
\pgfsetlinewidth{0.000000pt}%
\definecolor{currentstroke}{rgb}{0.000000,0.000000,0.000000}%
\pgfsetstrokecolor{currentstroke}%
\pgfsetstrokeopacity{0.000000}%
\pgfsetdash{}{0pt}%
\pgfpathmoveto{\pgfqpoint{5.800000in}{58.308758in}}%
\pgfpathlineto{\pgfqpoint{5.800000in}{58.308758in}}%
\pgfpathlineto{\pgfqpoint{5.800000in}{58.411595in}}%
\pgfpathlineto{\pgfqpoint{5.800000in}{58.411595in}}%
\pgfpathclose%
\pgfusepath{fill}%
\end{pgfscope}%
\begin{pgfscope}%
\pgfpathrectangle{\pgfqpoint{5.800000in}{0.720000in}}{\pgfqpoint{1.400000in}{4.620000in}}%
\pgfusepath{clip}%
\pgfsetbuttcap%
\pgfsetmiterjoin%
\definecolor{currentfill}{rgb}{0.121569,0.466667,0.705882}%
\pgfsetfillcolor{currentfill}%
\pgfsetlinewidth{0.000000pt}%
\definecolor{currentstroke}{rgb}{0.000000,0.000000,0.000000}%
\pgfsetstrokecolor{currentstroke}%
\pgfsetstrokeopacity{0.000000}%
\pgfsetdash{}{0pt}%
\pgfpathmoveto{\pgfqpoint{5.800000in}{58.411595in}}%
\pgfpathlineto{\pgfqpoint{5.800000in}{58.411595in}}%
\pgfpathlineto{\pgfqpoint{5.800000in}{58.514432in}}%
\pgfpathlineto{\pgfqpoint{5.800000in}{58.514432in}}%
\pgfpathclose%
\pgfusepath{fill}%
\end{pgfscope}%
\begin{pgfscope}%
\pgfpathrectangle{\pgfqpoint{5.800000in}{0.720000in}}{\pgfqpoint{1.400000in}{4.620000in}}%
\pgfusepath{clip}%
\pgfsetbuttcap%
\pgfsetmiterjoin%
\definecolor{currentfill}{rgb}{0.121569,0.466667,0.705882}%
\pgfsetfillcolor{currentfill}%
\pgfsetlinewidth{0.000000pt}%
\definecolor{currentstroke}{rgb}{0.000000,0.000000,0.000000}%
\pgfsetstrokecolor{currentstroke}%
\pgfsetstrokeopacity{0.000000}%
\pgfsetdash{}{0pt}%
\pgfpathmoveto{\pgfqpoint{5.800000in}{58.514432in}}%
\pgfpathlineto{\pgfqpoint{5.800000in}{58.514432in}}%
\pgfpathlineto{\pgfqpoint{5.800000in}{58.617269in}}%
\pgfpathlineto{\pgfqpoint{5.800000in}{58.617269in}}%
\pgfpathclose%
\pgfusepath{fill}%
\end{pgfscope}%
\begin{pgfscope}%
\pgfpathrectangle{\pgfqpoint{5.800000in}{0.720000in}}{\pgfqpoint{1.400000in}{4.620000in}}%
\pgfusepath{clip}%
\pgfsetbuttcap%
\pgfsetmiterjoin%
\definecolor{currentfill}{rgb}{0.121569,0.466667,0.705882}%
\pgfsetfillcolor{currentfill}%
\pgfsetlinewidth{0.000000pt}%
\definecolor{currentstroke}{rgb}{0.000000,0.000000,0.000000}%
\pgfsetstrokecolor{currentstroke}%
\pgfsetstrokeopacity{0.000000}%
\pgfsetdash{}{0pt}%
\pgfpathmoveto{\pgfqpoint{5.800000in}{58.617269in}}%
\pgfpathlineto{\pgfqpoint{5.800000in}{58.617269in}}%
\pgfpathlineto{\pgfqpoint{5.800000in}{58.720106in}}%
\pgfpathlineto{\pgfqpoint{5.800000in}{58.720106in}}%
\pgfpathclose%
\pgfusepath{fill}%
\end{pgfscope}%
\begin{pgfscope}%
\pgfpathrectangle{\pgfqpoint{5.800000in}{0.720000in}}{\pgfqpoint{1.400000in}{4.620000in}}%
\pgfusepath{clip}%
\pgfsetbuttcap%
\pgfsetmiterjoin%
\definecolor{currentfill}{rgb}{0.121569,0.466667,0.705882}%
\pgfsetfillcolor{currentfill}%
\pgfsetlinewidth{0.000000pt}%
\definecolor{currentstroke}{rgb}{0.000000,0.000000,0.000000}%
\pgfsetstrokecolor{currentstroke}%
\pgfsetstrokeopacity{0.000000}%
\pgfsetdash{}{0pt}%
\pgfpathmoveto{\pgfqpoint{5.800000in}{58.720106in}}%
\pgfpathlineto{\pgfqpoint{5.800000in}{58.720106in}}%
\pgfpathlineto{\pgfqpoint{5.800000in}{58.822943in}}%
\pgfpathlineto{\pgfqpoint{5.800000in}{58.822943in}}%
\pgfpathclose%
\pgfusepath{fill}%
\end{pgfscope}%
\begin{pgfscope}%
\pgfpathrectangle{\pgfqpoint{5.800000in}{0.720000in}}{\pgfqpoint{1.400000in}{4.620000in}}%
\pgfusepath{clip}%
\pgfsetbuttcap%
\pgfsetmiterjoin%
\definecolor{currentfill}{rgb}{0.121569,0.466667,0.705882}%
\pgfsetfillcolor{currentfill}%
\pgfsetlinewidth{0.000000pt}%
\definecolor{currentstroke}{rgb}{0.000000,0.000000,0.000000}%
\pgfsetstrokecolor{currentstroke}%
\pgfsetstrokeopacity{0.000000}%
\pgfsetdash{}{0pt}%
\pgfpathmoveto{\pgfqpoint{5.800000in}{58.822943in}}%
\pgfpathlineto{\pgfqpoint{5.800000in}{58.822943in}}%
\pgfpathlineto{\pgfqpoint{5.800000in}{58.925780in}}%
\pgfpathlineto{\pgfqpoint{5.800000in}{58.925780in}}%
\pgfpathclose%
\pgfusepath{fill}%
\end{pgfscope}%
\begin{pgfscope}%
\pgfpathrectangle{\pgfqpoint{5.800000in}{0.720000in}}{\pgfqpoint{1.400000in}{4.620000in}}%
\pgfusepath{clip}%
\pgfsetbuttcap%
\pgfsetmiterjoin%
\definecolor{currentfill}{rgb}{0.121569,0.466667,0.705882}%
\pgfsetfillcolor{currentfill}%
\pgfsetlinewidth{0.000000pt}%
\definecolor{currentstroke}{rgb}{0.000000,0.000000,0.000000}%
\pgfsetstrokecolor{currentstroke}%
\pgfsetstrokeopacity{0.000000}%
\pgfsetdash{}{0pt}%
\pgfpathmoveto{\pgfqpoint{5.800000in}{58.925780in}}%
\pgfpathlineto{\pgfqpoint{5.800000in}{58.925780in}}%
\pgfpathlineto{\pgfqpoint{5.800000in}{59.028617in}}%
\pgfpathlineto{\pgfqpoint{5.800000in}{59.028617in}}%
\pgfpathclose%
\pgfusepath{fill}%
\end{pgfscope}%
\begin{pgfscope}%
\pgfpathrectangle{\pgfqpoint{5.800000in}{0.720000in}}{\pgfqpoint{1.400000in}{4.620000in}}%
\pgfusepath{clip}%
\pgfsetbuttcap%
\pgfsetmiterjoin%
\definecolor{currentfill}{rgb}{0.121569,0.466667,0.705882}%
\pgfsetfillcolor{currentfill}%
\pgfsetlinewidth{0.000000pt}%
\definecolor{currentstroke}{rgb}{0.000000,0.000000,0.000000}%
\pgfsetstrokecolor{currentstroke}%
\pgfsetstrokeopacity{0.000000}%
\pgfsetdash{}{0pt}%
\pgfpathmoveto{\pgfqpoint{5.800000in}{59.028617in}}%
\pgfpathlineto{\pgfqpoint{5.800000in}{59.028617in}}%
\pgfpathlineto{\pgfqpoint{5.800000in}{59.131454in}}%
\pgfpathlineto{\pgfqpoint{5.800000in}{59.131454in}}%
\pgfpathclose%
\pgfusepath{fill}%
\end{pgfscope}%
\begin{pgfscope}%
\pgfpathrectangle{\pgfqpoint{5.800000in}{0.720000in}}{\pgfqpoint{1.400000in}{4.620000in}}%
\pgfusepath{clip}%
\pgfsetbuttcap%
\pgfsetmiterjoin%
\definecolor{currentfill}{rgb}{0.121569,0.466667,0.705882}%
\pgfsetfillcolor{currentfill}%
\pgfsetlinewidth{0.000000pt}%
\definecolor{currentstroke}{rgb}{0.000000,0.000000,0.000000}%
\pgfsetstrokecolor{currentstroke}%
\pgfsetstrokeopacity{0.000000}%
\pgfsetdash{}{0pt}%
\pgfpathmoveto{\pgfqpoint{5.800000in}{59.131454in}}%
\pgfpathlineto{\pgfqpoint{5.800000in}{59.131454in}}%
\pgfpathlineto{\pgfqpoint{5.800000in}{59.234292in}}%
\pgfpathlineto{\pgfqpoint{5.800000in}{59.234292in}}%
\pgfpathclose%
\pgfusepath{fill}%
\end{pgfscope}%
\begin{pgfscope}%
\pgfpathrectangle{\pgfqpoint{5.800000in}{0.720000in}}{\pgfqpoint{1.400000in}{4.620000in}}%
\pgfusepath{clip}%
\pgfsetbuttcap%
\pgfsetmiterjoin%
\definecolor{currentfill}{rgb}{0.121569,0.466667,0.705882}%
\pgfsetfillcolor{currentfill}%
\pgfsetlinewidth{0.000000pt}%
\definecolor{currentstroke}{rgb}{0.000000,0.000000,0.000000}%
\pgfsetstrokecolor{currentstroke}%
\pgfsetstrokeopacity{0.000000}%
\pgfsetdash{}{0pt}%
\pgfpathmoveto{\pgfqpoint{5.800000in}{59.234292in}}%
\pgfpathlineto{\pgfqpoint{5.800000in}{59.234292in}}%
\pgfpathlineto{\pgfqpoint{5.800000in}{59.337129in}}%
\pgfpathlineto{\pgfqpoint{5.800000in}{59.337129in}}%
\pgfpathclose%
\pgfusepath{fill}%
\end{pgfscope}%
\begin{pgfscope}%
\pgfpathrectangle{\pgfqpoint{5.800000in}{0.720000in}}{\pgfqpoint{1.400000in}{4.620000in}}%
\pgfusepath{clip}%
\pgfsetbuttcap%
\pgfsetmiterjoin%
\definecolor{currentfill}{rgb}{0.121569,0.466667,0.705882}%
\pgfsetfillcolor{currentfill}%
\pgfsetlinewidth{0.000000pt}%
\definecolor{currentstroke}{rgb}{0.000000,0.000000,0.000000}%
\pgfsetstrokecolor{currentstroke}%
\pgfsetstrokeopacity{0.000000}%
\pgfsetdash{}{0pt}%
\pgfpathmoveto{\pgfqpoint{5.800000in}{59.337129in}}%
\pgfpathlineto{\pgfqpoint{5.800000in}{59.337129in}}%
\pgfpathlineto{\pgfqpoint{5.800000in}{59.439966in}}%
\pgfpathlineto{\pgfqpoint{5.800000in}{59.439966in}}%
\pgfpathclose%
\pgfusepath{fill}%
\end{pgfscope}%
\begin{pgfscope}%
\pgfpathrectangle{\pgfqpoint{5.800000in}{0.720000in}}{\pgfqpoint{1.400000in}{4.620000in}}%
\pgfusepath{clip}%
\pgfsetbuttcap%
\pgfsetmiterjoin%
\definecolor{currentfill}{rgb}{0.121569,0.466667,0.705882}%
\pgfsetfillcolor{currentfill}%
\pgfsetlinewidth{0.000000pt}%
\definecolor{currentstroke}{rgb}{0.000000,0.000000,0.000000}%
\pgfsetstrokecolor{currentstroke}%
\pgfsetstrokeopacity{0.000000}%
\pgfsetdash{}{0pt}%
\pgfpathmoveto{\pgfqpoint{5.800000in}{59.439966in}}%
\pgfpathlineto{\pgfqpoint{5.800000in}{59.439966in}}%
\pgfpathlineto{\pgfqpoint{5.800000in}{59.542803in}}%
\pgfpathlineto{\pgfqpoint{5.800000in}{59.542803in}}%
\pgfpathclose%
\pgfusepath{fill}%
\end{pgfscope}%
\begin{pgfscope}%
\pgfpathrectangle{\pgfqpoint{5.800000in}{0.720000in}}{\pgfqpoint{1.400000in}{4.620000in}}%
\pgfusepath{clip}%
\pgfsetbuttcap%
\pgfsetmiterjoin%
\definecolor{currentfill}{rgb}{0.121569,0.466667,0.705882}%
\pgfsetfillcolor{currentfill}%
\pgfsetlinewidth{0.000000pt}%
\definecolor{currentstroke}{rgb}{0.000000,0.000000,0.000000}%
\pgfsetstrokecolor{currentstroke}%
\pgfsetstrokeopacity{0.000000}%
\pgfsetdash{}{0pt}%
\pgfpathmoveto{\pgfqpoint{5.800000in}{59.542803in}}%
\pgfpathlineto{\pgfqpoint{5.800000in}{59.542803in}}%
\pgfpathlineto{\pgfqpoint{5.800000in}{59.645640in}}%
\pgfpathlineto{\pgfqpoint{5.800000in}{59.645640in}}%
\pgfpathclose%
\pgfusepath{fill}%
\end{pgfscope}%
\begin{pgfscope}%
\pgfpathrectangle{\pgfqpoint{5.800000in}{0.720000in}}{\pgfqpoint{1.400000in}{4.620000in}}%
\pgfusepath{clip}%
\pgfsetbuttcap%
\pgfsetmiterjoin%
\definecolor{currentfill}{rgb}{0.121569,0.466667,0.705882}%
\pgfsetfillcolor{currentfill}%
\pgfsetlinewidth{0.000000pt}%
\definecolor{currentstroke}{rgb}{0.000000,0.000000,0.000000}%
\pgfsetstrokecolor{currentstroke}%
\pgfsetstrokeopacity{0.000000}%
\pgfsetdash{}{0pt}%
\pgfpathmoveto{\pgfqpoint{5.800000in}{59.645640in}}%
\pgfpathlineto{\pgfqpoint{5.800000in}{59.645640in}}%
\pgfpathlineto{\pgfqpoint{5.800000in}{59.748477in}}%
\pgfpathlineto{\pgfqpoint{5.800000in}{59.748477in}}%
\pgfpathclose%
\pgfusepath{fill}%
\end{pgfscope}%
\begin{pgfscope}%
\pgfpathrectangle{\pgfqpoint{5.800000in}{0.720000in}}{\pgfqpoint{1.400000in}{4.620000in}}%
\pgfusepath{clip}%
\pgfsetbuttcap%
\pgfsetmiterjoin%
\definecolor{currentfill}{rgb}{0.121569,0.466667,0.705882}%
\pgfsetfillcolor{currentfill}%
\pgfsetlinewidth{0.000000pt}%
\definecolor{currentstroke}{rgb}{0.000000,0.000000,0.000000}%
\pgfsetstrokecolor{currentstroke}%
\pgfsetstrokeopacity{0.000000}%
\pgfsetdash{}{0pt}%
\pgfpathmoveto{\pgfqpoint{5.800000in}{59.748477in}}%
\pgfpathlineto{\pgfqpoint{5.800000in}{59.748477in}}%
\pgfpathlineto{\pgfqpoint{5.800000in}{59.851314in}}%
\pgfpathlineto{\pgfqpoint{5.800000in}{59.851314in}}%
\pgfpathclose%
\pgfusepath{fill}%
\end{pgfscope}%
\begin{pgfscope}%
\pgfpathrectangle{\pgfqpoint{5.800000in}{0.720000in}}{\pgfqpoint{1.400000in}{4.620000in}}%
\pgfusepath{clip}%
\pgfsetbuttcap%
\pgfsetmiterjoin%
\definecolor{currentfill}{rgb}{0.121569,0.466667,0.705882}%
\pgfsetfillcolor{currentfill}%
\pgfsetlinewidth{0.000000pt}%
\definecolor{currentstroke}{rgb}{0.000000,0.000000,0.000000}%
\pgfsetstrokecolor{currentstroke}%
\pgfsetstrokeopacity{0.000000}%
\pgfsetdash{}{0pt}%
\pgfpathmoveto{\pgfqpoint{5.800000in}{59.851314in}}%
\pgfpathlineto{\pgfqpoint{5.800000in}{59.851314in}}%
\pgfpathlineto{\pgfqpoint{5.800000in}{59.954151in}}%
\pgfpathlineto{\pgfqpoint{5.800000in}{59.954151in}}%
\pgfpathclose%
\pgfusepath{fill}%
\end{pgfscope}%
\begin{pgfscope}%
\pgfpathrectangle{\pgfqpoint{5.800000in}{0.720000in}}{\pgfqpoint{1.400000in}{4.620000in}}%
\pgfusepath{clip}%
\pgfsetbuttcap%
\pgfsetmiterjoin%
\definecolor{currentfill}{rgb}{0.121569,0.466667,0.705882}%
\pgfsetfillcolor{currentfill}%
\pgfsetlinewidth{0.000000pt}%
\definecolor{currentstroke}{rgb}{0.000000,0.000000,0.000000}%
\pgfsetstrokecolor{currentstroke}%
\pgfsetstrokeopacity{0.000000}%
\pgfsetdash{}{0pt}%
\pgfpathmoveto{\pgfqpoint{5.800000in}{59.954151in}}%
\pgfpathlineto{\pgfqpoint{5.800000in}{59.954151in}}%
\pgfpathlineto{\pgfqpoint{5.800000in}{60.056988in}}%
\pgfpathlineto{\pgfqpoint{5.800000in}{60.056988in}}%
\pgfpathclose%
\pgfusepath{fill}%
\end{pgfscope}%
\begin{pgfscope}%
\pgfpathrectangle{\pgfqpoint{5.800000in}{0.720000in}}{\pgfqpoint{1.400000in}{4.620000in}}%
\pgfusepath{clip}%
\pgfsetbuttcap%
\pgfsetmiterjoin%
\definecolor{currentfill}{rgb}{0.121569,0.466667,0.705882}%
\pgfsetfillcolor{currentfill}%
\pgfsetlinewidth{0.000000pt}%
\definecolor{currentstroke}{rgb}{0.000000,0.000000,0.000000}%
\pgfsetstrokecolor{currentstroke}%
\pgfsetstrokeopacity{0.000000}%
\pgfsetdash{}{0pt}%
\pgfpathmoveto{\pgfqpoint{5.800000in}{60.056988in}}%
\pgfpathlineto{\pgfqpoint{5.800000in}{60.056988in}}%
\pgfpathlineto{\pgfqpoint{5.800000in}{60.159825in}}%
\pgfpathlineto{\pgfqpoint{5.800000in}{60.159825in}}%
\pgfpathclose%
\pgfusepath{fill}%
\end{pgfscope}%
\begin{pgfscope}%
\pgfpathrectangle{\pgfqpoint{5.800000in}{0.720000in}}{\pgfqpoint{1.400000in}{4.620000in}}%
\pgfusepath{clip}%
\pgfsetbuttcap%
\pgfsetmiterjoin%
\definecolor{currentfill}{rgb}{0.121569,0.466667,0.705882}%
\pgfsetfillcolor{currentfill}%
\pgfsetlinewidth{0.000000pt}%
\definecolor{currentstroke}{rgb}{0.000000,0.000000,0.000000}%
\pgfsetstrokecolor{currentstroke}%
\pgfsetstrokeopacity{0.000000}%
\pgfsetdash{}{0pt}%
\pgfpathmoveto{\pgfqpoint{5.800000in}{60.159825in}}%
\pgfpathlineto{\pgfqpoint{5.800000in}{60.159825in}}%
\pgfpathlineto{\pgfqpoint{5.800000in}{60.262662in}}%
\pgfpathlineto{\pgfqpoint{5.800000in}{60.262662in}}%
\pgfpathclose%
\pgfusepath{fill}%
\end{pgfscope}%
\begin{pgfscope}%
\pgfpathrectangle{\pgfqpoint{5.800000in}{0.720000in}}{\pgfqpoint{1.400000in}{4.620000in}}%
\pgfusepath{clip}%
\pgfsetbuttcap%
\pgfsetmiterjoin%
\definecolor{currentfill}{rgb}{0.121569,0.466667,0.705882}%
\pgfsetfillcolor{currentfill}%
\pgfsetlinewidth{0.000000pt}%
\definecolor{currentstroke}{rgb}{0.000000,0.000000,0.000000}%
\pgfsetstrokecolor{currentstroke}%
\pgfsetstrokeopacity{0.000000}%
\pgfsetdash{}{0pt}%
\pgfpathmoveto{\pgfqpoint{5.800000in}{60.262662in}}%
\pgfpathlineto{\pgfqpoint{5.800000in}{60.262662in}}%
\pgfpathlineto{\pgfqpoint{5.800000in}{60.365499in}}%
\pgfpathlineto{\pgfqpoint{5.800000in}{60.365499in}}%
\pgfpathclose%
\pgfusepath{fill}%
\end{pgfscope}%
\begin{pgfscope}%
\pgfpathrectangle{\pgfqpoint{5.800000in}{0.720000in}}{\pgfqpoint{1.400000in}{4.620000in}}%
\pgfusepath{clip}%
\pgfsetbuttcap%
\pgfsetmiterjoin%
\definecolor{currentfill}{rgb}{0.121569,0.466667,0.705882}%
\pgfsetfillcolor{currentfill}%
\pgfsetlinewidth{0.000000pt}%
\definecolor{currentstroke}{rgb}{0.000000,0.000000,0.000000}%
\pgfsetstrokecolor{currentstroke}%
\pgfsetstrokeopacity{0.000000}%
\pgfsetdash{}{0pt}%
\pgfpathmoveto{\pgfqpoint{5.800000in}{60.365499in}}%
\pgfpathlineto{\pgfqpoint{5.800000in}{60.365499in}}%
\pgfpathlineto{\pgfqpoint{5.800000in}{60.468336in}}%
\pgfpathlineto{\pgfqpoint{5.800000in}{60.468336in}}%
\pgfpathclose%
\pgfusepath{fill}%
\end{pgfscope}%
\begin{pgfscope}%
\pgfpathrectangle{\pgfqpoint{5.800000in}{0.720000in}}{\pgfqpoint{1.400000in}{4.620000in}}%
\pgfusepath{clip}%
\pgfsetbuttcap%
\pgfsetmiterjoin%
\definecolor{currentfill}{rgb}{0.121569,0.466667,0.705882}%
\pgfsetfillcolor{currentfill}%
\pgfsetlinewidth{0.000000pt}%
\definecolor{currentstroke}{rgb}{0.000000,0.000000,0.000000}%
\pgfsetstrokecolor{currentstroke}%
\pgfsetstrokeopacity{0.000000}%
\pgfsetdash{}{0pt}%
\pgfpathmoveto{\pgfqpoint{5.800000in}{60.468336in}}%
\pgfpathlineto{\pgfqpoint{5.800000in}{60.468336in}}%
\pgfpathlineto{\pgfqpoint{5.800000in}{60.571173in}}%
\pgfpathlineto{\pgfqpoint{5.800000in}{60.571173in}}%
\pgfpathclose%
\pgfusepath{fill}%
\end{pgfscope}%
\begin{pgfscope}%
\pgfpathrectangle{\pgfqpoint{5.800000in}{0.720000in}}{\pgfqpoint{1.400000in}{4.620000in}}%
\pgfusepath{clip}%
\pgfsetbuttcap%
\pgfsetmiterjoin%
\definecolor{currentfill}{rgb}{0.121569,0.466667,0.705882}%
\pgfsetfillcolor{currentfill}%
\pgfsetlinewidth{0.000000pt}%
\definecolor{currentstroke}{rgb}{0.000000,0.000000,0.000000}%
\pgfsetstrokecolor{currentstroke}%
\pgfsetstrokeopacity{0.000000}%
\pgfsetdash{}{0pt}%
\pgfpathmoveto{\pgfqpoint{5.800000in}{60.571173in}}%
\pgfpathlineto{\pgfqpoint{5.800000in}{60.571173in}}%
\pgfpathlineto{\pgfqpoint{5.800000in}{60.674011in}}%
\pgfpathlineto{\pgfqpoint{5.800000in}{60.674011in}}%
\pgfpathclose%
\pgfusepath{fill}%
\end{pgfscope}%
\begin{pgfscope}%
\pgfpathrectangle{\pgfqpoint{5.800000in}{0.720000in}}{\pgfqpoint{1.400000in}{4.620000in}}%
\pgfusepath{clip}%
\pgfsetbuttcap%
\pgfsetmiterjoin%
\definecolor{currentfill}{rgb}{0.121569,0.466667,0.705882}%
\pgfsetfillcolor{currentfill}%
\pgfsetlinewidth{0.000000pt}%
\definecolor{currentstroke}{rgb}{0.000000,0.000000,0.000000}%
\pgfsetstrokecolor{currentstroke}%
\pgfsetstrokeopacity{0.000000}%
\pgfsetdash{}{0pt}%
\pgfpathmoveto{\pgfqpoint{5.800000in}{60.674011in}}%
\pgfpathlineto{\pgfqpoint{5.800000in}{60.674011in}}%
\pgfpathlineto{\pgfqpoint{5.800000in}{60.776848in}}%
\pgfpathlineto{\pgfqpoint{5.800000in}{60.776848in}}%
\pgfpathclose%
\pgfusepath{fill}%
\end{pgfscope}%
\begin{pgfscope}%
\pgfpathrectangle{\pgfqpoint{5.800000in}{0.720000in}}{\pgfqpoint{1.400000in}{4.620000in}}%
\pgfusepath{clip}%
\pgfsetbuttcap%
\pgfsetmiterjoin%
\definecolor{currentfill}{rgb}{0.121569,0.466667,0.705882}%
\pgfsetfillcolor{currentfill}%
\pgfsetlinewidth{0.000000pt}%
\definecolor{currentstroke}{rgb}{0.000000,0.000000,0.000000}%
\pgfsetstrokecolor{currentstroke}%
\pgfsetstrokeopacity{0.000000}%
\pgfsetdash{}{0pt}%
\pgfpathmoveto{\pgfqpoint{5.800000in}{60.776848in}}%
\pgfpathlineto{\pgfqpoint{5.800000in}{60.776848in}}%
\pgfpathlineto{\pgfqpoint{5.800000in}{60.879685in}}%
\pgfpathlineto{\pgfqpoint{5.800000in}{60.879685in}}%
\pgfpathclose%
\pgfusepath{fill}%
\end{pgfscope}%
\begin{pgfscope}%
\pgfpathrectangle{\pgfqpoint{5.800000in}{0.720000in}}{\pgfqpoint{1.400000in}{4.620000in}}%
\pgfusepath{clip}%
\pgfsetbuttcap%
\pgfsetmiterjoin%
\definecolor{currentfill}{rgb}{0.121569,0.466667,0.705882}%
\pgfsetfillcolor{currentfill}%
\pgfsetlinewidth{0.000000pt}%
\definecolor{currentstroke}{rgb}{0.000000,0.000000,0.000000}%
\pgfsetstrokecolor{currentstroke}%
\pgfsetstrokeopacity{0.000000}%
\pgfsetdash{}{0pt}%
\pgfpathmoveto{\pgfqpoint{5.800000in}{60.879685in}}%
\pgfpathlineto{\pgfqpoint{5.800000in}{60.879685in}}%
\pgfpathlineto{\pgfqpoint{5.800000in}{60.982522in}}%
\pgfpathlineto{\pgfqpoint{5.800000in}{60.982522in}}%
\pgfpathclose%
\pgfusepath{fill}%
\end{pgfscope}%
\begin{pgfscope}%
\pgfpathrectangle{\pgfqpoint{5.800000in}{0.720000in}}{\pgfqpoint{1.400000in}{4.620000in}}%
\pgfusepath{clip}%
\pgfsetbuttcap%
\pgfsetmiterjoin%
\definecolor{currentfill}{rgb}{0.121569,0.466667,0.705882}%
\pgfsetfillcolor{currentfill}%
\pgfsetlinewidth{0.000000pt}%
\definecolor{currentstroke}{rgb}{0.000000,0.000000,0.000000}%
\pgfsetstrokecolor{currentstroke}%
\pgfsetstrokeopacity{0.000000}%
\pgfsetdash{}{0pt}%
\pgfpathmoveto{\pgfqpoint{5.800000in}{60.982522in}}%
\pgfpathlineto{\pgfqpoint{5.800000in}{60.982522in}}%
\pgfpathlineto{\pgfqpoint{5.800000in}{61.085359in}}%
\pgfpathlineto{\pgfqpoint{5.800000in}{61.085359in}}%
\pgfpathclose%
\pgfusepath{fill}%
\end{pgfscope}%
\begin{pgfscope}%
\pgfpathrectangle{\pgfqpoint{5.800000in}{0.720000in}}{\pgfqpoint{1.400000in}{4.620000in}}%
\pgfusepath{clip}%
\pgfsetbuttcap%
\pgfsetmiterjoin%
\definecolor{currentfill}{rgb}{0.121569,0.466667,0.705882}%
\pgfsetfillcolor{currentfill}%
\pgfsetlinewidth{0.000000pt}%
\definecolor{currentstroke}{rgb}{0.000000,0.000000,0.000000}%
\pgfsetstrokecolor{currentstroke}%
\pgfsetstrokeopacity{0.000000}%
\pgfsetdash{}{0pt}%
\pgfpathmoveto{\pgfqpoint{5.800000in}{61.085359in}}%
\pgfpathlineto{\pgfqpoint{5.800000in}{61.085359in}}%
\pgfpathlineto{\pgfqpoint{5.800000in}{61.188196in}}%
\pgfpathlineto{\pgfqpoint{5.800000in}{61.188196in}}%
\pgfpathclose%
\pgfusepath{fill}%
\end{pgfscope}%
\begin{pgfscope}%
\pgfpathrectangle{\pgfqpoint{5.800000in}{0.720000in}}{\pgfqpoint{1.400000in}{4.620000in}}%
\pgfusepath{clip}%
\pgfsetbuttcap%
\pgfsetmiterjoin%
\definecolor{currentfill}{rgb}{0.121569,0.466667,0.705882}%
\pgfsetfillcolor{currentfill}%
\pgfsetlinewidth{0.000000pt}%
\definecolor{currentstroke}{rgb}{0.000000,0.000000,0.000000}%
\pgfsetstrokecolor{currentstroke}%
\pgfsetstrokeopacity{0.000000}%
\pgfsetdash{}{0pt}%
\pgfpathmoveto{\pgfqpoint{5.800000in}{61.188196in}}%
\pgfpathlineto{\pgfqpoint{5.800000in}{61.188196in}}%
\pgfpathlineto{\pgfqpoint{5.800000in}{61.291033in}}%
\pgfpathlineto{\pgfqpoint{5.800000in}{61.291033in}}%
\pgfpathclose%
\pgfusepath{fill}%
\end{pgfscope}%
\begin{pgfscope}%
\pgfpathrectangle{\pgfqpoint{5.800000in}{0.720000in}}{\pgfqpoint{1.400000in}{4.620000in}}%
\pgfusepath{clip}%
\pgfsetbuttcap%
\pgfsetmiterjoin%
\definecolor{currentfill}{rgb}{0.121569,0.466667,0.705882}%
\pgfsetfillcolor{currentfill}%
\pgfsetlinewidth{0.000000pt}%
\definecolor{currentstroke}{rgb}{0.000000,0.000000,0.000000}%
\pgfsetstrokecolor{currentstroke}%
\pgfsetstrokeopacity{0.000000}%
\pgfsetdash{}{0pt}%
\pgfpathmoveto{\pgfqpoint{5.800000in}{61.291033in}}%
\pgfpathlineto{\pgfqpoint{5.800000in}{61.291033in}}%
\pgfpathlineto{\pgfqpoint{5.800000in}{61.393870in}}%
\pgfpathlineto{\pgfqpoint{5.800000in}{61.393870in}}%
\pgfpathclose%
\pgfusepath{fill}%
\end{pgfscope}%
\begin{pgfscope}%
\pgfpathrectangle{\pgfqpoint{5.800000in}{0.720000in}}{\pgfqpoint{1.400000in}{4.620000in}}%
\pgfusepath{clip}%
\pgfsetbuttcap%
\pgfsetmiterjoin%
\definecolor{currentfill}{rgb}{0.121569,0.466667,0.705882}%
\pgfsetfillcolor{currentfill}%
\pgfsetlinewidth{0.000000pt}%
\definecolor{currentstroke}{rgb}{0.000000,0.000000,0.000000}%
\pgfsetstrokecolor{currentstroke}%
\pgfsetstrokeopacity{0.000000}%
\pgfsetdash{}{0pt}%
\pgfpathmoveto{\pgfqpoint{5.800000in}{61.393870in}}%
\pgfpathlineto{\pgfqpoint{5.800000in}{61.393870in}}%
\pgfpathlineto{\pgfqpoint{5.800000in}{61.496707in}}%
\pgfpathlineto{\pgfqpoint{5.800000in}{61.496707in}}%
\pgfpathclose%
\pgfusepath{fill}%
\end{pgfscope}%
\begin{pgfscope}%
\pgfpathrectangle{\pgfqpoint{5.800000in}{0.720000in}}{\pgfqpoint{1.400000in}{4.620000in}}%
\pgfusepath{clip}%
\pgfsetbuttcap%
\pgfsetmiterjoin%
\definecolor{currentfill}{rgb}{0.121569,0.466667,0.705882}%
\pgfsetfillcolor{currentfill}%
\pgfsetlinewidth{0.000000pt}%
\definecolor{currentstroke}{rgb}{0.000000,0.000000,0.000000}%
\pgfsetstrokecolor{currentstroke}%
\pgfsetstrokeopacity{0.000000}%
\pgfsetdash{}{0pt}%
\pgfpathmoveto{\pgfqpoint{5.800000in}{61.496707in}}%
\pgfpathlineto{\pgfqpoint{5.800000in}{61.496707in}}%
\pgfpathlineto{\pgfqpoint{5.800000in}{61.599544in}}%
\pgfpathlineto{\pgfqpoint{5.800000in}{61.599544in}}%
\pgfpathclose%
\pgfusepath{fill}%
\end{pgfscope}%
\begin{pgfscope}%
\pgfpathrectangle{\pgfqpoint{5.800000in}{0.720000in}}{\pgfqpoint{1.400000in}{4.620000in}}%
\pgfusepath{clip}%
\pgfsetbuttcap%
\pgfsetmiterjoin%
\definecolor{currentfill}{rgb}{0.121569,0.466667,0.705882}%
\pgfsetfillcolor{currentfill}%
\pgfsetlinewidth{0.000000pt}%
\definecolor{currentstroke}{rgb}{0.000000,0.000000,0.000000}%
\pgfsetstrokecolor{currentstroke}%
\pgfsetstrokeopacity{0.000000}%
\pgfsetdash{}{0pt}%
\pgfpathmoveto{\pgfqpoint{5.800000in}{61.599544in}}%
\pgfpathlineto{\pgfqpoint{5.800000in}{61.599544in}}%
\pgfpathlineto{\pgfqpoint{5.800000in}{61.702381in}}%
\pgfpathlineto{\pgfqpoint{5.800000in}{61.702381in}}%
\pgfpathclose%
\pgfusepath{fill}%
\end{pgfscope}%
\begin{pgfscope}%
\pgfpathrectangle{\pgfqpoint{5.800000in}{0.720000in}}{\pgfqpoint{1.400000in}{4.620000in}}%
\pgfusepath{clip}%
\pgfsetbuttcap%
\pgfsetmiterjoin%
\definecolor{currentfill}{rgb}{0.121569,0.466667,0.705882}%
\pgfsetfillcolor{currentfill}%
\pgfsetlinewidth{0.000000pt}%
\definecolor{currentstroke}{rgb}{0.000000,0.000000,0.000000}%
\pgfsetstrokecolor{currentstroke}%
\pgfsetstrokeopacity{0.000000}%
\pgfsetdash{}{0pt}%
\pgfpathmoveto{\pgfqpoint{5.800000in}{61.702381in}}%
\pgfpathlineto{\pgfqpoint{5.800000in}{61.702381in}}%
\pgfpathlineto{\pgfqpoint{5.800000in}{61.805218in}}%
\pgfpathlineto{\pgfqpoint{5.800000in}{61.805218in}}%
\pgfpathclose%
\pgfusepath{fill}%
\end{pgfscope}%
\begin{pgfscope}%
\pgfpathrectangle{\pgfqpoint{5.800000in}{0.720000in}}{\pgfqpoint{1.400000in}{4.620000in}}%
\pgfusepath{clip}%
\pgfsetbuttcap%
\pgfsetmiterjoin%
\definecolor{currentfill}{rgb}{0.121569,0.466667,0.705882}%
\pgfsetfillcolor{currentfill}%
\pgfsetlinewidth{0.000000pt}%
\definecolor{currentstroke}{rgb}{0.000000,0.000000,0.000000}%
\pgfsetstrokecolor{currentstroke}%
\pgfsetstrokeopacity{0.000000}%
\pgfsetdash{}{0pt}%
\pgfpathmoveto{\pgfqpoint{5.800000in}{61.805218in}}%
\pgfpathlineto{\pgfqpoint{5.800000in}{61.805218in}}%
\pgfpathlineto{\pgfqpoint{5.800000in}{61.908055in}}%
\pgfpathlineto{\pgfqpoint{5.800000in}{61.908055in}}%
\pgfpathclose%
\pgfusepath{fill}%
\end{pgfscope}%
\begin{pgfscope}%
\pgfpathrectangle{\pgfqpoint{5.800000in}{0.720000in}}{\pgfqpoint{1.400000in}{4.620000in}}%
\pgfusepath{clip}%
\pgfsetbuttcap%
\pgfsetmiterjoin%
\definecolor{currentfill}{rgb}{0.121569,0.466667,0.705882}%
\pgfsetfillcolor{currentfill}%
\pgfsetlinewidth{0.000000pt}%
\definecolor{currentstroke}{rgb}{0.000000,0.000000,0.000000}%
\pgfsetstrokecolor{currentstroke}%
\pgfsetstrokeopacity{0.000000}%
\pgfsetdash{}{0pt}%
\pgfpathmoveto{\pgfqpoint{5.800000in}{61.908055in}}%
\pgfpathlineto{\pgfqpoint{5.800000in}{61.908055in}}%
\pgfpathlineto{\pgfqpoint{5.800000in}{62.010892in}}%
\pgfpathlineto{\pgfqpoint{5.800000in}{62.010892in}}%
\pgfpathclose%
\pgfusepath{fill}%
\end{pgfscope}%
\begin{pgfscope}%
\pgfpathrectangle{\pgfqpoint{5.800000in}{0.720000in}}{\pgfqpoint{1.400000in}{4.620000in}}%
\pgfusepath{clip}%
\pgfsetbuttcap%
\pgfsetmiterjoin%
\definecolor{currentfill}{rgb}{0.121569,0.466667,0.705882}%
\pgfsetfillcolor{currentfill}%
\pgfsetlinewidth{0.000000pt}%
\definecolor{currentstroke}{rgb}{0.000000,0.000000,0.000000}%
\pgfsetstrokecolor{currentstroke}%
\pgfsetstrokeopacity{0.000000}%
\pgfsetdash{}{0pt}%
\pgfpathmoveto{\pgfqpoint{5.800000in}{62.010892in}}%
\pgfpathlineto{\pgfqpoint{5.800000in}{62.010892in}}%
\pgfpathlineto{\pgfqpoint{5.800000in}{62.113729in}}%
\pgfpathlineto{\pgfqpoint{5.800000in}{62.113729in}}%
\pgfpathclose%
\pgfusepath{fill}%
\end{pgfscope}%
\begin{pgfscope}%
\pgfpathrectangle{\pgfqpoint{5.800000in}{0.720000in}}{\pgfqpoint{1.400000in}{4.620000in}}%
\pgfusepath{clip}%
\pgfsetbuttcap%
\pgfsetmiterjoin%
\definecolor{currentfill}{rgb}{0.121569,0.466667,0.705882}%
\pgfsetfillcolor{currentfill}%
\pgfsetlinewidth{0.000000pt}%
\definecolor{currentstroke}{rgb}{0.000000,0.000000,0.000000}%
\pgfsetstrokecolor{currentstroke}%
\pgfsetstrokeopacity{0.000000}%
\pgfsetdash{}{0pt}%
\pgfpathmoveto{\pgfqpoint{5.800000in}{62.113729in}}%
\pgfpathlineto{\pgfqpoint{5.800000in}{62.113729in}}%
\pgfpathlineto{\pgfqpoint{5.800000in}{62.216567in}}%
\pgfpathlineto{\pgfqpoint{5.800000in}{62.216567in}}%
\pgfpathclose%
\pgfusepath{fill}%
\end{pgfscope}%
\begin{pgfscope}%
\pgfpathrectangle{\pgfqpoint{5.800000in}{0.720000in}}{\pgfqpoint{1.400000in}{4.620000in}}%
\pgfusepath{clip}%
\pgfsetbuttcap%
\pgfsetmiterjoin%
\definecolor{currentfill}{rgb}{0.121569,0.466667,0.705882}%
\pgfsetfillcolor{currentfill}%
\pgfsetlinewidth{0.000000pt}%
\definecolor{currentstroke}{rgb}{0.000000,0.000000,0.000000}%
\pgfsetstrokecolor{currentstroke}%
\pgfsetstrokeopacity{0.000000}%
\pgfsetdash{}{0pt}%
\pgfpathmoveto{\pgfqpoint{5.800000in}{62.216567in}}%
\pgfpathlineto{\pgfqpoint{5.800000in}{62.216567in}}%
\pgfpathlineto{\pgfqpoint{5.800000in}{62.319404in}}%
\pgfpathlineto{\pgfqpoint{5.800000in}{62.319404in}}%
\pgfpathclose%
\pgfusepath{fill}%
\end{pgfscope}%
\begin{pgfscope}%
\pgfpathrectangle{\pgfqpoint{5.800000in}{0.720000in}}{\pgfqpoint{1.400000in}{4.620000in}}%
\pgfusepath{clip}%
\pgfsetbuttcap%
\pgfsetmiterjoin%
\definecolor{currentfill}{rgb}{0.121569,0.466667,0.705882}%
\pgfsetfillcolor{currentfill}%
\pgfsetlinewidth{0.000000pt}%
\definecolor{currentstroke}{rgb}{0.000000,0.000000,0.000000}%
\pgfsetstrokecolor{currentstroke}%
\pgfsetstrokeopacity{0.000000}%
\pgfsetdash{}{0pt}%
\pgfpathmoveto{\pgfqpoint{5.800000in}{62.319404in}}%
\pgfpathlineto{\pgfqpoint{5.800000in}{62.319404in}}%
\pgfpathlineto{\pgfqpoint{5.800000in}{62.422241in}}%
\pgfpathlineto{\pgfqpoint{5.800000in}{62.422241in}}%
\pgfpathclose%
\pgfusepath{fill}%
\end{pgfscope}%
\begin{pgfscope}%
\pgfpathrectangle{\pgfqpoint{5.800000in}{0.720000in}}{\pgfqpoint{1.400000in}{4.620000in}}%
\pgfusepath{clip}%
\pgfsetbuttcap%
\pgfsetmiterjoin%
\definecolor{currentfill}{rgb}{0.121569,0.466667,0.705882}%
\pgfsetfillcolor{currentfill}%
\pgfsetlinewidth{0.000000pt}%
\definecolor{currentstroke}{rgb}{0.000000,0.000000,0.000000}%
\pgfsetstrokecolor{currentstroke}%
\pgfsetstrokeopacity{0.000000}%
\pgfsetdash{}{0pt}%
\pgfpathmoveto{\pgfqpoint{5.800000in}{62.422241in}}%
\pgfpathlineto{\pgfqpoint{5.800000in}{62.422241in}}%
\pgfpathlineto{\pgfqpoint{5.800000in}{62.525078in}}%
\pgfpathlineto{\pgfqpoint{5.800000in}{62.525078in}}%
\pgfpathclose%
\pgfusepath{fill}%
\end{pgfscope}%
\begin{pgfscope}%
\pgfpathrectangle{\pgfqpoint{5.800000in}{0.720000in}}{\pgfqpoint{1.400000in}{4.620000in}}%
\pgfusepath{clip}%
\pgfsetbuttcap%
\pgfsetmiterjoin%
\definecolor{currentfill}{rgb}{0.121569,0.466667,0.705882}%
\pgfsetfillcolor{currentfill}%
\pgfsetlinewidth{0.000000pt}%
\definecolor{currentstroke}{rgb}{0.000000,0.000000,0.000000}%
\pgfsetstrokecolor{currentstroke}%
\pgfsetstrokeopacity{0.000000}%
\pgfsetdash{}{0pt}%
\pgfpathmoveto{\pgfqpoint{5.800000in}{62.525078in}}%
\pgfpathlineto{\pgfqpoint{5.800000in}{62.525078in}}%
\pgfpathlineto{\pgfqpoint{5.800000in}{62.627915in}}%
\pgfpathlineto{\pgfqpoint{5.800000in}{62.627915in}}%
\pgfpathclose%
\pgfusepath{fill}%
\end{pgfscope}%
\begin{pgfscope}%
\pgfpathrectangle{\pgfqpoint{5.800000in}{0.720000in}}{\pgfqpoint{1.400000in}{4.620000in}}%
\pgfusepath{clip}%
\pgfsetbuttcap%
\pgfsetmiterjoin%
\definecolor{currentfill}{rgb}{0.121569,0.466667,0.705882}%
\pgfsetfillcolor{currentfill}%
\pgfsetlinewidth{0.000000pt}%
\definecolor{currentstroke}{rgb}{0.000000,0.000000,0.000000}%
\pgfsetstrokecolor{currentstroke}%
\pgfsetstrokeopacity{0.000000}%
\pgfsetdash{}{0pt}%
\pgfpathmoveto{\pgfqpoint{5.800000in}{62.627915in}}%
\pgfpathlineto{\pgfqpoint{5.800000in}{62.627915in}}%
\pgfpathlineto{\pgfqpoint{5.800000in}{62.730752in}}%
\pgfpathlineto{\pgfqpoint{5.800000in}{62.730752in}}%
\pgfpathclose%
\pgfusepath{fill}%
\end{pgfscope}%
\begin{pgfscope}%
\pgfpathrectangle{\pgfqpoint{5.800000in}{0.720000in}}{\pgfqpoint{1.400000in}{4.620000in}}%
\pgfusepath{clip}%
\pgfsetbuttcap%
\pgfsetmiterjoin%
\definecolor{currentfill}{rgb}{0.121569,0.466667,0.705882}%
\pgfsetfillcolor{currentfill}%
\pgfsetlinewidth{0.000000pt}%
\definecolor{currentstroke}{rgb}{0.000000,0.000000,0.000000}%
\pgfsetstrokecolor{currentstroke}%
\pgfsetstrokeopacity{0.000000}%
\pgfsetdash{}{0pt}%
\pgfpathmoveto{\pgfqpoint{5.800000in}{62.730752in}}%
\pgfpathlineto{\pgfqpoint{5.800000in}{62.730752in}}%
\pgfpathlineto{\pgfqpoint{5.800000in}{62.833589in}}%
\pgfpathlineto{\pgfqpoint{5.800000in}{62.833589in}}%
\pgfpathclose%
\pgfusepath{fill}%
\end{pgfscope}%
\begin{pgfscope}%
\pgfpathrectangle{\pgfqpoint{5.800000in}{0.720000in}}{\pgfqpoint{1.400000in}{4.620000in}}%
\pgfusepath{clip}%
\pgfsetbuttcap%
\pgfsetmiterjoin%
\definecolor{currentfill}{rgb}{0.121569,0.466667,0.705882}%
\pgfsetfillcolor{currentfill}%
\pgfsetlinewidth{0.000000pt}%
\definecolor{currentstroke}{rgb}{0.000000,0.000000,0.000000}%
\pgfsetstrokecolor{currentstroke}%
\pgfsetstrokeopacity{0.000000}%
\pgfsetdash{}{0pt}%
\pgfpathmoveto{\pgfqpoint{5.800000in}{62.833589in}}%
\pgfpathlineto{\pgfqpoint{5.800000in}{62.833589in}}%
\pgfpathlineto{\pgfqpoint{5.800000in}{62.936426in}}%
\pgfpathlineto{\pgfqpoint{5.800000in}{62.936426in}}%
\pgfpathclose%
\pgfusepath{fill}%
\end{pgfscope}%
\begin{pgfscope}%
\pgfpathrectangle{\pgfqpoint{5.800000in}{0.720000in}}{\pgfqpoint{1.400000in}{4.620000in}}%
\pgfusepath{clip}%
\pgfsetbuttcap%
\pgfsetmiterjoin%
\definecolor{currentfill}{rgb}{0.121569,0.466667,0.705882}%
\pgfsetfillcolor{currentfill}%
\pgfsetlinewidth{0.000000pt}%
\definecolor{currentstroke}{rgb}{0.000000,0.000000,0.000000}%
\pgfsetstrokecolor{currentstroke}%
\pgfsetstrokeopacity{0.000000}%
\pgfsetdash{}{0pt}%
\pgfpathmoveto{\pgfqpoint{5.800000in}{62.936426in}}%
\pgfpathlineto{\pgfqpoint{5.800000in}{62.936426in}}%
\pgfpathlineto{\pgfqpoint{5.800000in}{63.039263in}}%
\pgfpathlineto{\pgfqpoint{5.800000in}{63.039263in}}%
\pgfpathclose%
\pgfusepath{fill}%
\end{pgfscope}%
\begin{pgfscope}%
\pgfpathrectangle{\pgfqpoint{5.800000in}{0.720000in}}{\pgfqpoint{1.400000in}{4.620000in}}%
\pgfusepath{clip}%
\pgfsetbuttcap%
\pgfsetmiterjoin%
\definecolor{currentfill}{rgb}{0.121569,0.466667,0.705882}%
\pgfsetfillcolor{currentfill}%
\pgfsetlinewidth{0.000000pt}%
\definecolor{currentstroke}{rgb}{0.000000,0.000000,0.000000}%
\pgfsetstrokecolor{currentstroke}%
\pgfsetstrokeopacity{0.000000}%
\pgfsetdash{}{0pt}%
\pgfpathmoveto{\pgfqpoint{5.800000in}{63.039263in}}%
\pgfpathlineto{\pgfqpoint{5.800000in}{63.039263in}}%
\pgfpathlineto{\pgfqpoint{5.800000in}{63.142100in}}%
\pgfpathlineto{\pgfqpoint{5.800000in}{63.142100in}}%
\pgfpathclose%
\pgfusepath{fill}%
\end{pgfscope}%
\begin{pgfscope}%
\pgfpathrectangle{\pgfqpoint{5.800000in}{0.720000in}}{\pgfqpoint{1.400000in}{4.620000in}}%
\pgfusepath{clip}%
\pgfsetbuttcap%
\pgfsetmiterjoin%
\definecolor{currentfill}{rgb}{0.121569,0.466667,0.705882}%
\pgfsetfillcolor{currentfill}%
\pgfsetlinewidth{0.000000pt}%
\definecolor{currentstroke}{rgb}{0.000000,0.000000,0.000000}%
\pgfsetstrokecolor{currentstroke}%
\pgfsetstrokeopacity{0.000000}%
\pgfsetdash{}{0pt}%
\pgfpathmoveto{\pgfqpoint{5.800000in}{63.142100in}}%
\pgfpathlineto{\pgfqpoint{5.800000in}{63.142100in}}%
\pgfpathlineto{\pgfqpoint{5.800000in}{63.244937in}}%
\pgfpathlineto{\pgfqpoint{5.800000in}{63.244937in}}%
\pgfpathclose%
\pgfusepath{fill}%
\end{pgfscope}%
\begin{pgfscope}%
\pgfpathrectangle{\pgfqpoint{5.800000in}{0.720000in}}{\pgfqpoint{1.400000in}{4.620000in}}%
\pgfusepath{clip}%
\pgfsetbuttcap%
\pgfsetmiterjoin%
\definecolor{currentfill}{rgb}{0.121569,0.466667,0.705882}%
\pgfsetfillcolor{currentfill}%
\pgfsetlinewidth{0.000000pt}%
\definecolor{currentstroke}{rgb}{0.000000,0.000000,0.000000}%
\pgfsetstrokecolor{currentstroke}%
\pgfsetstrokeopacity{0.000000}%
\pgfsetdash{}{0pt}%
\pgfpathmoveto{\pgfqpoint{5.800000in}{63.244937in}}%
\pgfpathlineto{\pgfqpoint{5.800000in}{63.244937in}}%
\pgfpathlineto{\pgfqpoint{5.800000in}{63.347774in}}%
\pgfpathlineto{\pgfqpoint{5.800000in}{63.347774in}}%
\pgfpathclose%
\pgfusepath{fill}%
\end{pgfscope}%
\begin{pgfscope}%
\pgfpathrectangle{\pgfqpoint{5.800000in}{0.720000in}}{\pgfqpoint{1.400000in}{4.620000in}}%
\pgfusepath{clip}%
\pgfsetbuttcap%
\pgfsetmiterjoin%
\definecolor{currentfill}{rgb}{0.121569,0.466667,0.705882}%
\pgfsetfillcolor{currentfill}%
\pgfsetlinewidth{0.000000pt}%
\definecolor{currentstroke}{rgb}{0.000000,0.000000,0.000000}%
\pgfsetstrokecolor{currentstroke}%
\pgfsetstrokeopacity{0.000000}%
\pgfsetdash{}{0pt}%
\pgfpathmoveto{\pgfqpoint{5.800000in}{63.347774in}}%
\pgfpathlineto{\pgfqpoint{5.800000in}{63.347774in}}%
\pgfpathlineto{\pgfqpoint{5.800000in}{63.450611in}}%
\pgfpathlineto{\pgfqpoint{5.800000in}{63.450611in}}%
\pgfpathclose%
\pgfusepath{fill}%
\end{pgfscope}%
\begin{pgfscope}%
\pgfpathrectangle{\pgfqpoint{5.800000in}{0.720000in}}{\pgfqpoint{1.400000in}{4.620000in}}%
\pgfusepath{clip}%
\pgfsetbuttcap%
\pgfsetmiterjoin%
\definecolor{currentfill}{rgb}{0.121569,0.466667,0.705882}%
\pgfsetfillcolor{currentfill}%
\pgfsetlinewidth{0.000000pt}%
\definecolor{currentstroke}{rgb}{0.000000,0.000000,0.000000}%
\pgfsetstrokecolor{currentstroke}%
\pgfsetstrokeopacity{0.000000}%
\pgfsetdash{}{0pt}%
\pgfpathmoveto{\pgfqpoint{5.800000in}{63.450611in}}%
\pgfpathlineto{\pgfqpoint{5.800000in}{63.450611in}}%
\pgfpathlineto{\pgfqpoint{5.800000in}{63.553448in}}%
\pgfpathlineto{\pgfqpoint{5.800000in}{63.553448in}}%
\pgfpathclose%
\pgfusepath{fill}%
\end{pgfscope}%
\begin{pgfscope}%
\pgfpathrectangle{\pgfqpoint{5.800000in}{0.720000in}}{\pgfqpoint{1.400000in}{4.620000in}}%
\pgfusepath{clip}%
\pgfsetbuttcap%
\pgfsetmiterjoin%
\definecolor{currentfill}{rgb}{0.121569,0.466667,0.705882}%
\pgfsetfillcolor{currentfill}%
\pgfsetlinewidth{0.000000pt}%
\definecolor{currentstroke}{rgb}{0.000000,0.000000,0.000000}%
\pgfsetstrokecolor{currentstroke}%
\pgfsetstrokeopacity{0.000000}%
\pgfsetdash{}{0pt}%
\pgfpathmoveto{\pgfqpoint{5.800000in}{63.553448in}}%
\pgfpathlineto{\pgfqpoint{5.800000in}{63.553448in}}%
\pgfpathlineto{\pgfqpoint{5.800000in}{63.656285in}}%
\pgfpathlineto{\pgfqpoint{5.800000in}{63.656285in}}%
\pgfpathclose%
\pgfusepath{fill}%
\end{pgfscope}%
\begin{pgfscope}%
\pgfpathrectangle{\pgfqpoint{5.800000in}{0.720000in}}{\pgfqpoint{1.400000in}{4.620000in}}%
\pgfusepath{clip}%
\pgfsetbuttcap%
\pgfsetmiterjoin%
\definecolor{currentfill}{rgb}{0.121569,0.466667,0.705882}%
\pgfsetfillcolor{currentfill}%
\pgfsetlinewidth{0.000000pt}%
\definecolor{currentstroke}{rgb}{0.000000,0.000000,0.000000}%
\pgfsetstrokecolor{currentstroke}%
\pgfsetstrokeopacity{0.000000}%
\pgfsetdash{}{0pt}%
\pgfpathmoveto{\pgfqpoint{5.800000in}{63.656285in}}%
\pgfpathlineto{\pgfqpoint{5.800000in}{63.656285in}}%
\pgfpathlineto{\pgfqpoint{5.800000in}{63.759123in}}%
\pgfpathlineto{\pgfqpoint{5.800000in}{63.759123in}}%
\pgfpathclose%
\pgfusepath{fill}%
\end{pgfscope}%
\begin{pgfscope}%
\pgfpathrectangle{\pgfqpoint{5.800000in}{0.720000in}}{\pgfqpoint{1.400000in}{4.620000in}}%
\pgfusepath{clip}%
\pgfsetbuttcap%
\pgfsetmiterjoin%
\definecolor{currentfill}{rgb}{0.121569,0.466667,0.705882}%
\pgfsetfillcolor{currentfill}%
\pgfsetlinewidth{0.000000pt}%
\definecolor{currentstroke}{rgb}{0.000000,0.000000,0.000000}%
\pgfsetstrokecolor{currentstroke}%
\pgfsetstrokeopacity{0.000000}%
\pgfsetdash{}{0pt}%
\pgfpathmoveto{\pgfqpoint{5.800000in}{63.759123in}}%
\pgfpathlineto{\pgfqpoint{5.800000in}{63.759123in}}%
\pgfpathlineto{\pgfqpoint{5.800000in}{63.861960in}}%
\pgfpathlineto{\pgfqpoint{5.800000in}{63.861960in}}%
\pgfpathclose%
\pgfusepath{fill}%
\end{pgfscope}%
\begin{pgfscope}%
\pgfpathrectangle{\pgfqpoint{5.800000in}{0.720000in}}{\pgfqpoint{1.400000in}{4.620000in}}%
\pgfusepath{clip}%
\pgfsetbuttcap%
\pgfsetmiterjoin%
\definecolor{currentfill}{rgb}{0.121569,0.466667,0.705882}%
\pgfsetfillcolor{currentfill}%
\pgfsetlinewidth{0.000000pt}%
\definecolor{currentstroke}{rgb}{0.000000,0.000000,0.000000}%
\pgfsetstrokecolor{currentstroke}%
\pgfsetstrokeopacity{0.000000}%
\pgfsetdash{}{0pt}%
\pgfpathmoveto{\pgfqpoint{5.800000in}{63.861960in}}%
\pgfpathlineto{\pgfqpoint{5.800000in}{63.861960in}}%
\pgfpathlineto{\pgfqpoint{5.800000in}{63.964797in}}%
\pgfpathlineto{\pgfqpoint{5.800000in}{63.964797in}}%
\pgfpathclose%
\pgfusepath{fill}%
\end{pgfscope}%
\begin{pgfscope}%
\pgfpathrectangle{\pgfqpoint{5.800000in}{0.720000in}}{\pgfqpoint{1.400000in}{4.620000in}}%
\pgfusepath{clip}%
\pgfsetbuttcap%
\pgfsetmiterjoin%
\definecolor{currentfill}{rgb}{0.121569,0.466667,0.705882}%
\pgfsetfillcolor{currentfill}%
\pgfsetlinewidth{0.000000pt}%
\definecolor{currentstroke}{rgb}{0.000000,0.000000,0.000000}%
\pgfsetstrokecolor{currentstroke}%
\pgfsetstrokeopacity{0.000000}%
\pgfsetdash{}{0pt}%
\pgfpathmoveto{\pgfqpoint{5.800000in}{63.964797in}}%
\pgfpathlineto{\pgfqpoint{5.800000in}{63.964797in}}%
\pgfpathlineto{\pgfqpoint{5.800000in}{64.067634in}}%
\pgfpathlineto{\pgfqpoint{5.800000in}{64.067634in}}%
\pgfpathclose%
\pgfusepath{fill}%
\end{pgfscope}%
\begin{pgfscope}%
\pgfpathrectangle{\pgfqpoint{5.800000in}{0.720000in}}{\pgfqpoint{1.400000in}{4.620000in}}%
\pgfusepath{clip}%
\pgfsetbuttcap%
\pgfsetmiterjoin%
\definecolor{currentfill}{rgb}{0.121569,0.466667,0.705882}%
\pgfsetfillcolor{currentfill}%
\pgfsetlinewidth{0.000000pt}%
\definecolor{currentstroke}{rgb}{0.000000,0.000000,0.000000}%
\pgfsetstrokecolor{currentstroke}%
\pgfsetstrokeopacity{0.000000}%
\pgfsetdash{}{0pt}%
\pgfpathmoveto{\pgfqpoint{5.800000in}{64.067634in}}%
\pgfpathlineto{\pgfqpoint{5.800000in}{64.067634in}}%
\pgfpathlineto{\pgfqpoint{5.800000in}{64.170471in}}%
\pgfpathlineto{\pgfqpoint{5.800000in}{64.170471in}}%
\pgfpathclose%
\pgfusepath{fill}%
\end{pgfscope}%
\begin{pgfscope}%
\pgfpathrectangle{\pgfqpoint{5.800000in}{0.720000in}}{\pgfqpoint{1.400000in}{4.620000in}}%
\pgfusepath{clip}%
\pgfsetbuttcap%
\pgfsetmiterjoin%
\definecolor{currentfill}{rgb}{0.121569,0.466667,0.705882}%
\pgfsetfillcolor{currentfill}%
\pgfsetlinewidth{0.000000pt}%
\definecolor{currentstroke}{rgb}{0.000000,0.000000,0.000000}%
\pgfsetstrokecolor{currentstroke}%
\pgfsetstrokeopacity{0.000000}%
\pgfsetdash{}{0pt}%
\pgfpathmoveto{\pgfqpoint{5.800000in}{64.170471in}}%
\pgfpathlineto{\pgfqpoint{5.800000in}{64.170471in}}%
\pgfpathlineto{\pgfqpoint{5.800000in}{64.273308in}}%
\pgfpathlineto{\pgfqpoint{5.800000in}{64.273308in}}%
\pgfpathclose%
\pgfusepath{fill}%
\end{pgfscope}%
\begin{pgfscope}%
\pgfpathrectangle{\pgfqpoint{5.800000in}{0.720000in}}{\pgfqpoint{1.400000in}{4.620000in}}%
\pgfusepath{clip}%
\pgfsetbuttcap%
\pgfsetmiterjoin%
\definecolor{currentfill}{rgb}{0.121569,0.466667,0.705882}%
\pgfsetfillcolor{currentfill}%
\pgfsetlinewidth{0.000000pt}%
\definecolor{currentstroke}{rgb}{0.000000,0.000000,0.000000}%
\pgfsetstrokecolor{currentstroke}%
\pgfsetstrokeopacity{0.000000}%
\pgfsetdash{}{0pt}%
\pgfpathmoveto{\pgfqpoint{5.800000in}{64.273308in}}%
\pgfpathlineto{\pgfqpoint{5.800000in}{64.273308in}}%
\pgfpathlineto{\pgfqpoint{5.800000in}{64.376145in}}%
\pgfpathlineto{\pgfqpoint{5.800000in}{64.376145in}}%
\pgfpathclose%
\pgfusepath{fill}%
\end{pgfscope}%
\begin{pgfscope}%
\pgfpathrectangle{\pgfqpoint{5.800000in}{0.720000in}}{\pgfqpoint{1.400000in}{4.620000in}}%
\pgfusepath{clip}%
\pgfsetbuttcap%
\pgfsetmiterjoin%
\definecolor{currentfill}{rgb}{0.121569,0.466667,0.705882}%
\pgfsetfillcolor{currentfill}%
\pgfsetlinewidth{0.000000pt}%
\definecolor{currentstroke}{rgb}{0.000000,0.000000,0.000000}%
\pgfsetstrokecolor{currentstroke}%
\pgfsetstrokeopacity{0.000000}%
\pgfsetdash{}{0pt}%
\pgfpathmoveto{\pgfqpoint{5.800000in}{64.376145in}}%
\pgfpathlineto{\pgfqpoint{5.800000in}{64.376145in}}%
\pgfpathlineto{\pgfqpoint{5.800000in}{64.478982in}}%
\pgfpathlineto{\pgfqpoint{5.800000in}{64.478982in}}%
\pgfpathclose%
\pgfusepath{fill}%
\end{pgfscope}%
\begin{pgfscope}%
\pgfpathrectangle{\pgfqpoint{5.800000in}{0.720000in}}{\pgfqpoint{1.400000in}{4.620000in}}%
\pgfusepath{clip}%
\pgfsetbuttcap%
\pgfsetmiterjoin%
\definecolor{currentfill}{rgb}{0.121569,0.466667,0.705882}%
\pgfsetfillcolor{currentfill}%
\pgfsetlinewidth{0.000000pt}%
\definecolor{currentstroke}{rgb}{0.000000,0.000000,0.000000}%
\pgfsetstrokecolor{currentstroke}%
\pgfsetstrokeopacity{0.000000}%
\pgfsetdash{}{0pt}%
\pgfpathmoveto{\pgfqpoint{5.800000in}{64.478982in}}%
\pgfpathlineto{\pgfqpoint{5.800000in}{64.478982in}}%
\pgfpathlineto{\pgfqpoint{5.800000in}{64.581819in}}%
\pgfpathlineto{\pgfqpoint{5.800000in}{64.581819in}}%
\pgfpathclose%
\pgfusepath{fill}%
\end{pgfscope}%
\begin{pgfscope}%
\pgfpathrectangle{\pgfqpoint{5.800000in}{0.720000in}}{\pgfqpoint{1.400000in}{4.620000in}}%
\pgfusepath{clip}%
\pgfsetbuttcap%
\pgfsetmiterjoin%
\definecolor{currentfill}{rgb}{0.121569,0.466667,0.705882}%
\pgfsetfillcolor{currentfill}%
\pgfsetlinewidth{0.000000pt}%
\definecolor{currentstroke}{rgb}{0.000000,0.000000,0.000000}%
\pgfsetstrokecolor{currentstroke}%
\pgfsetstrokeopacity{0.000000}%
\pgfsetdash{}{0pt}%
\pgfpathmoveto{\pgfqpoint{5.800000in}{64.581819in}}%
\pgfpathlineto{\pgfqpoint{5.800000in}{64.581819in}}%
\pgfpathlineto{\pgfqpoint{5.800000in}{64.684656in}}%
\pgfpathlineto{\pgfqpoint{5.800000in}{64.684656in}}%
\pgfpathclose%
\pgfusepath{fill}%
\end{pgfscope}%
\begin{pgfscope}%
\pgfpathrectangle{\pgfqpoint{5.800000in}{0.720000in}}{\pgfqpoint{1.400000in}{4.620000in}}%
\pgfusepath{clip}%
\pgfsetbuttcap%
\pgfsetmiterjoin%
\definecolor{currentfill}{rgb}{0.121569,0.466667,0.705882}%
\pgfsetfillcolor{currentfill}%
\pgfsetlinewidth{0.000000pt}%
\definecolor{currentstroke}{rgb}{0.000000,0.000000,0.000000}%
\pgfsetstrokecolor{currentstroke}%
\pgfsetstrokeopacity{0.000000}%
\pgfsetdash{}{0pt}%
\pgfpathmoveto{\pgfqpoint{5.800000in}{64.684656in}}%
\pgfpathlineto{\pgfqpoint{5.800000in}{64.684656in}}%
\pgfpathlineto{\pgfqpoint{5.800000in}{64.787493in}}%
\pgfpathlineto{\pgfqpoint{5.800000in}{64.787493in}}%
\pgfpathclose%
\pgfusepath{fill}%
\end{pgfscope}%
\begin{pgfscope}%
\pgfpathrectangle{\pgfqpoint{5.800000in}{0.720000in}}{\pgfqpoint{1.400000in}{4.620000in}}%
\pgfusepath{clip}%
\pgfsetbuttcap%
\pgfsetmiterjoin%
\definecolor{currentfill}{rgb}{0.121569,0.466667,0.705882}%
\pgfsetfillcolor{currentfill}%
\pgfsetlinewidth{0.000000pt}%
\definecolor{currentstroke}{rgb}{0.000000,0.000000,0.000000}%
\pgfsetstrokecolor{currentstroke}%
\pgfsetstrokeopacity{0.000000}%
\pgfsetdash{}{0pt}%
\pgfpathmoveto{\pgfqpoint{5.800000in}{64.787493in}}%
\pgfpathlineto{\pgfqpoint{5.800000in}{64.787493in}}%
\pgfpathlineto{\pgfqpoint{5.800000in}{64.890330in}}%
\pgfpathlineto{\pgfqpoint{5.800000in}{64.890330in}}%
\pgfpathclose%
\pgfusepath{fill}%
\end{pgfscope}%
\begin{pgfscope}%
\pgfpathrectangle{\pgfqpoint{5.800000in}{0.720000in}}{\pgfqpoint{1.400000in}{4.620000in}}%
\pgfusepath{clip}%
\pgfsetbuttcap%
\pgfsetmiterjoin%
\definecolor{currentfill}{rgb}{0.121569,0.466667,0.705882}%
\pgfsetfillcolor{currentfill}%
\pgfsetlinewidth{0.000000pt}%
\definecolor{currentstroke}{rgb}{0.000000,0.000000,0.000000}%
\pgfsetstrokecolor{currentstroke}%
\pgfsetstrokeopacity{0.000000}%
\pgfsetdash{}{0pt}%
\pgfpathmoveto{\pgfqpoint{5.800000in}{64.890330in}}%
\pgfpathlineto{\pgfqpoint{5.800000in}{64.890330in}}%
\pgfpathlineto{\pgfqpoint{5.800000in}{64.993167in}}%
\pgfpathlineto{\pgfqpoint{5.800000in}{64.993167in}}%
\pgfpathclose%
\pgfusepath{fill}%
\end{pgfscope}%
\begin{pgfscope}%
\pgfpathrectangle{\pgfqpoint{5.800000in}{0.720000in}}{\pgfqpoint{1.400000in}{4.620000in}}%
\pgfusepath{clip}%
\pgfsetbuttcap%
\pgfsetmiterjoin%
\definecolor{currentfill}{rgb}{0.121569,0.466667,0.705882}%
\pgfsetfillcolor{currentfill}%
\pgfsetlinewidth{0.000000pt}%
\definecolor{currentstroke}{rgb}{0.000000,0.000000,0.000000}%
\pgfsetstrokecolor{currentstroke}%
\pgfsetstrokeopacity{0.000000}%
\pgfsetdash{}{0pt}%
\pgfpathmoveto{\pgfqpoint{5.800000in}{64.993167in}}%
\pgfpathlineto{\pgfqpoint{5.800000in}{64.993167in}}%
\pgfpathlineto{\pgfqpoint{5.800000in}{65.096004in}}%
\pgfpathlineto{\pgfqpoint{5.800000in}{65.096004in}}%
\pgfpathclose%
\pgfusepath{fill}%
\end{pgfscope}%
\begin{pgfscope}%
\pgfpathrectangle{\pgfqpoint{5.800000in}{0.720000in}}{\pgfqpoint{1.400000in}{4.620000in}}%
\pgfusepath{clip}%
\pgfsetbuttcap%
\pgfsetmiterjoin%
\definecolor{currentfill}{rgb}{0.121569,0.466667,0.705882}%
\pgfsetfillcolor{currentfill}%
\pgfsetlinewidth{0.000000pt}%
\definecolor{currentstroke}{rgb}{0.000000,0.000000,0.000000}%
\pgfsetstrokecolor{currentstroke}%
\pgfsetstrokeopacity{0.000000}%
\pgfsetdash{}{0pt}%
\pgfpathmoveto{\pgfqpoint{5.800000in}{65.096004in}}%
\pgfpathlineto{\pgfqpoint{5.800000in}{65.096004in}}%
\pgfpathlineto{\pgfqpoint{5.800000in}{65.198841in}}%
\pgfpathlineto{\pgfqpoint{5.800000in}{65.198841in}}%
\pgfpathclose%
\pgfusepath{fill}%
\end{pgfscope}%
\begin{pgfscope}%
\pgfpathrectangle{\pgfqpoint{5.800000in}{0.720000in}}{\pgfqpoint{1.400000in}{4.620000in}}%
\pgfusepath{clip}%
\pgfsetbuttcap%
\pgfsetmiterjoin%
\definecolor{currentfill}{rgb}{0.121569,0.466667,0.705882}%
\pgfsetfillcolor{currentfill}%
\pgfsetlinewidth{0.000000pt}%
\definecolor{currentstroke}{rgb}{0.000000,0.000000,0.000000}%
\pgfsetstrokecolor{currentstroke}%
\pgfsetstrokeopacity{0.000000}%
\pgfsetdash{}{0pt}%
\pgfpathmoveto{\pgfqpoint{5.800000in}{65.198841in}}%
\pgfpathlineto{\pgfqpoint{5.800000in}{65.198841in}}%
\pgfpathlineto{\pgfqpoint{5.800000in}{65.301679in}}%
\pgfpathlineto{\pgfqpoint{5.800000in}{65.301679in}}%
\pgfpathclose%
\pgfusepath{fill}%
\end{pgfscope}%
\begin{pgfscope}%
\pgfpathrectangle{\pgfqpoint{5.800000in}{0.720000in}}{\pgfqpoint{1.400000in}{4.620000in}}%
\pgfusepath{clip}%
\pgfsetbuttcap%
\pgfsetmiterjoin%
\definecolor{currentfill}{rgb}{0.121569,0.466667,0.705882}%
\pgfsetfillcolor{currentfill}%
\pgfsetlinewidth{0.000000pt}%
\definecolor{currentstroke}{rgb}{0.000000,0.000000,0.000000}%
\pgfsetstrokecolor{currentstroke}%
\pgfsetstrokeopacity{0.000000}%
\pgfsetdash{}{0pt}%
\pgfpathmoveto{\pgfqpoint{5.800000in}{65.301679in}}%
\pgfpathlineto{\pgfqpoint{5.800000in}{65.301679in}}%
\pgfpathlineto{\pgfqpoint{5.800000in}{65.404516in}}%
\pgfpathlineto{\pgfqpoint{5.800000in}{65.404516in}}%
\pgfpathclose%
\pgfusepath{fill}%
\end{pgfscope}%
\begin{pgfscope}%
\pgfpathrectangle{\pgfqpoint{5.800000in}{0.720000in}}{\pgfqpoint{1.400000in}{4.620000in}}%
\pgfusepath{clip}%
\pgfsetbuttcap%
\pgfsetmiterjoin%
\definecolor{currentfill}{rgb}{0.121569,0.466667,0.705882}%
\pgfsetfillcolor{currentfill}%
\pgfsetlinewidth{0.000000pt}%
\definecolor{currentstroke}{rgb}{0.000000,0.000000,0.000000}%
\pgfsetstrokecolor{currentstroke}%
\pgfsetstrokeopacity{0.000000}%
\pgfsetdash{}{0pt}%
\pgfpathmoveto{\pgfqpoint{5.800000in}{65.404516in}}%
\pgfpathlineto{\pgfqpoint{5.800000in}{65.404516in}}%
\pgfpathlineto{\pgfqpoint{5.800000in}{65.507353in}}%
\pgfpathlineto{\pgfqpoint{5.800000in}{65.507353in}}%
\pgfpathclose%
\pgfusepath{fill}%
\end{pgfscope}%
\begin{pgfscope}%
\pgfpathrectangle{\pgfqpoint{5.800000in}{0.720000in}}{\pgfqpoint{1.400000in}{4.620000in}}%
\pgfusepath{clip}%
\pgfsetbuttcap%
\pgfsetmiterjoin%
\definecolor{currentfill}{rgb}{0.121569,0.466667,0.705882}%
\pgfsetfillcolor{currentfill}%
\pgfsetlinewidth{0.000000pt}%
\definecolor{currentstroke}{rgb}{0.000000,0.000000,0.000000}%
\pgfsetstrokecolor{currentstroke}%
\pgfsetstrokeopacity{0.000000}%
\pgfsetdash{}{0pt}%
\pgfpathmoveto{\pgfqpoint{5.800000in}{65.507353in}}%
\pgfpathlineto{\pgfqpoint{5.800000in}{65.507353in}}%
\pgfpathlineto{\pgfqpoint{5.800000in}{65.610190in}}%
\pgfpathlineto{\pgfqpoint{5.800000in}{65.610190in}}%
\pgfpathclose%
\pgfusepath{fill}%
\end{pgfscope}%
\begin{pgfscope}%
\pgfpathrectangle{\pgfqpoint{5.800000in}{0.720000in}}{\pgfqpoint{1.400000in}{4.620000in}}%
\pgfusepath{clip}%
\pgfsetbuttcap%
\pgfsetmiterjoin%
\definecolor{currentfill}{rgb}{0.121569,0.466667,0.705882}%
\pgfsetfillcolor{currentfill}%
\pgfsetlinewidth{0.000000pt}%
\definecolor{currentstroke}{rgb}{0.000000,0.000000,0.000000}%
\pgfsetstrokecolor{currentstroke}%
\pgfsetstrokeopacity{0.000000}%
\pgfsetdash{}{0pt}%
\pgfpathmoveto{\pgfqpoint{5.800000in}{65.610190in}}%
\pgfpathlineto{\pgfqpoint{5.800000in}{65.610190in}}%
\pgfpathlineto{\pgfqpoint{5.800000in}{65.713027in}}%
\pgfpathlineto{\pgfqpoint{5.800000in}{65.713027in}}%
\pgfpathclose%
\pgfusepath{fill}%
\end{pgfscope}%
\begin{pgfscope}%
\pgfpathrectangle{\pgfqpoint{5.800000in}{0.720000in}}{\pgfqpoint{1.400000in}{4.620000in}}%
\pgfusepath{clip}%
\pgfsetbuttcap%
\pgfsetmiterjoin%
\definecolor{currentfill}{rgb}{0.121569,0.466667,0.705882}%
\pgfsetfillcolor{currentfill}%
\pgfsetlinewidth{0.000000pt}%
\definecolor{currentstroke}{rgb}{0.000000,0.000000,0.000000}%
\pgfsetstrokecolor{currentstroke}%
\pgfsetstrokeopacity{0.000000}%
\pgfsetdash{}{0pt}%
\pgfpathmoveto{\pgfqpoint{5.800000in}{65.713027in}}%
\pgfpathlineto{\pgfqpoint{5.800000in}{65.713027in}}%
\pgfpathlineto{\pgfqpoint{5.800000in}{65.815864in}}%
\pgfpathlineto{\pgfqpoint{5.800000in}{65.815864in}}%
\pgfpathclose%
\pgfusepath{fill}%
\end{pgfscope}%
\begin{pgfscope}%
\pgfpathrectangle{\pgfqpoint{5.800000in}{0.720000in}}{\pgfqpoint{1.400000in}{4.620000in}}%
\pgfusepath{clip}%
\pgfsetbuttcap%
\pgfsetmiterjoin%
\definecolor{currentfill}{rgb}{0.121569,0.466667,0.705882}%
\pgfsetfillcolor{currentfill}%
\pgfsetlinewidth{0.000000pt}%
\definecolor{currentstroke}{rgb}{0.000000,0.000000,0.000000}%
\pgfsetstrokecolor{currentstroke}%
\pgfsetstrokeopacity{0.000000}%
\pgfsetdash{}{0pt}%
\pgfpathmoveto{\pgfqpoint{5.800000in}{65.815864in}}%
\pgfpathlineto{\pgfqpoint{5.800000in}{65.815864in}}%
\pgfpathlineto{\pgfqpoint{5.800000in}{65.918701in}}%
\pgfpathlineto{\pgfqpoint{5.800000in}{65.918701in}}%
\pgfpathclose%
\pgfusepath{fill}%
\end{pgfscope}%
\begin{pgfscope}%
\pgfpathrectangle{\pgfqpoint{5.800000in}{0.720000in}}{\pgfqpoint{1.400000in}{4.620000in}}%
\pgfusepath{clip}%
\pgfsetbuttcap%
\pgfsetmiterjoin%
\definecolor{currentfill}{rgb}{0.121569,0.466667,0.705882}%
\pgfsetfillcolor{currentfill}%
\pgfsetlinewidth{0.000000pt}%
\definecolor{currentstroke}{rgb}{0.000000,0.000000,0.000000}%
\pgfsetstrokecolor{currentstroke}%
\pgfsetstrokeopacity{0.000000}%
\pgfsetdash{}{0pt}%
\pgfpathmoveto{\pgfqpoint{5.800000in}{65.918701in}}%
\pgfpathlineto{\pgfqpoint{5.800000in}{65.918701in}}%
\pgfpathlineto{\pgfqpoint{5.800000in}{66.021538in}}%
\pgfpathlineto{\pgfqpoint{5.800000in}{66.021538in}}%
\pgfpathclose%
\pgfusepath{fill}%
\end{pgfscope}%
\begin{pgfscope}%
\pgfpathrectangle{\pgfqpoint{5.800000in}{0.720000in}}{\pgfqpoint{1.400000in}{4.620000in}}%
\pgfusepath{clip}%
\pgfsetbuttcap%
\pgfsetmiterjoin%
\definecolor{currentfill}{rgb}{0.121569,0.466667,0.705882}%
\pgfsetfillcolor{currentfill}%
\pgfsetlinewidth{0.000000pt}%
\definecolor{currentstroke}{rgb}{0.000000,0.000000,0.000000}%
\pgfsetstrokecolor{currentstroke}%
\pgfsetstrokeopacity{0.000000}%
\pgfsetdash{}{0pt}%
\pgfpathmoveto{\pgfqpoint{5.800000in}{66.021538in}}%
\pgfpathlineto{\pgfqpoint{5.800000in}{66.021538in}}%
\pgfpathlineto{\pgfqpoint{5.800000in}{66.124375in}}%
\pgfpathlineto{\pgfqpoint{5.800000in}{66.124375in}}%
\pgfpathclose%
\pgfusepath{fill}%
\end{pgfscope}%
\begin{pgfscope}%
\pgfpathrectangle{\pgfqpoint{5.800000in}{0.720000in}}{\pgfqpoint{1.400000in}{4.620000in}}%
\pgfusepath{clip}%
\pgfsetbuttcap%
\pgfsetmiterjoin%
\definecolor{currentfill}{rgb}{0.121569,0.466667,0.705882}%
\pgfsetfillcolor{currentfill}%
\pgfsetlinewidth{0.000000pt}%
\definecolor{currentstroke}{rgb}{0.000000,0.000000,0.000000}%
\pgfsetstrokecolor{currentstroke}%
\pgfsetstrokeopacity{0.000000}%
\pgfsetdash{}{0pt}%
\pgfpathmoveto{\pgfqpoint{5.800000in}{66.124375in}}%
\pgfpathlineto{\pgfqpoint{5.800000in}{66.124375in}}%
\pgfpathlineto{\pgfqpoint{5.800000in}{66.227212in}}%
\pgfpathlineto{\pgfqpoint{5.800000in}{66.227212in}}%
\pgfpathclose%
\pgfusepath{fill}%
\end{pgfscope}%
\begin{pgfscope}%
\pgfpathrectangle{\pgfqpoint{5.800000in}{0.720000in}}{\pgfqpoint{1.400000in}{4.620000in}}%
\pgfusepath{clip}%
\pgfsetbuttcap%
\pgfsetmiterjoin%
\definecolor{currentfill}{rgb}{0.121569,0.466667,0.705882}%
\pgfsetfillcolor{currentfill}%
\pgfsetlinewidth{0.000000pt}%
\definecolor{currentstroke}{rgb}{0.000000,0.000000,0.000000}%
\pgfsetstrokecolor{currentstroke}%
\pgfsetstrokeopacity{0.000000}%
\pgfsetdash{}{0pt}%
\pgfpathmoveto{\pgfqpoint{5.800000in}{66.227212in}}%
\pgfpathlineto{\pgfqpoint{5.800000in}{66.227212in}}%
\pgfpathlineto{\pgfqpoint{5.800000in}{66.330049in}}%
\pgfpathlineto{\pgfqpoint{5.800000in}{66.330049in}}%
\pgfpathclose%
\pgfusepath{fill}%
\end{pgfscope}%
\begin{pgfscope}%
\pgfpathrectangle{\pgfqpoint{5.800000in}{0.720000in}}{\pgfqpoint{1.400000in}{4.620000in}}%
\pgfusepath{clip}%
\pgfsetbuttcap%
\pgfsetmiterjoin%
\definecolor{currentfill}{rgb}{0.121569,0.466667,0.705882}%
\pgfsetfillcolor{currentfill}%
\pgfsetlinewidth{0.000000pt}%
\definecolor{currentstroke}{rgb}{0.000000,0.000000,0.000000}%
\pgfsetstrokecolor{currentstroke}%
\pgfsetstrokeopacity{0.000000}%
\pgfsetdash{}{0pt}%
\pgfpathmoveto{\pgfqpoint{5.800000in}{66.330049in}}%
\pgfpathlineto{\pgfqpoint{5.800000in}{66.330049in}}%
\pgfpathlineto{\pgfqpoint{5.800000in}{66.432886in}}%
\pgfpathlineto{\pgfqpoint{5.800000in}{66.432886in}}%
\pgfpathclose%
\pgfusepath{fill}%
\end{pgfscope}%
\begin{pgfscope}%
\pgfpathrectangle{\pgfqpoint{5.800000in}{0.720000in}}{\pgfqpoint{1.400000in}{4.620000in}}%
\pgfusepath{clip}%
\pgfsetbuttcap%
\pgfsetmiterjoin%
\definecolor{currentfill}{rgb}{0.121569,0.466667,0.705882}%
\pgfsetfillcolor{currentfill}%
\pgfsetlinewidth{0.000000pt}%
\definecolor{currentstroke}{rgb}{0.000000,0.000000,0.000000}%
\pgfsetstrokecolor{currentstroke}%
\pgfsetstrokeopacity{0.000000}%
\pgfsetdash{}{0pt}%
\pgfpathmoveto{\pgfqpoint{5.800000in}{66.432886in}}%
\pgfpathlineto{\pgfqpoint{5.800000in}{66.432886in}}%
\pgfpathlineto{\pgfqpoint{5.800000in}{66.535723in}}%
\pgfpathlineto{\pgfqpoint{5.800000in}{66.535723in}}%
\pgfpathclose%
\pgfusepath{fill}%
\end{pgfscope}%
\begin{pgfscope}%
\pgfpathrectangle{\pgfqpoint{5.800000in}{0.720000in}}{\pgfqpoint{1.400000in}{4.620000in}}%
\pgfusepath{clip}%
\pgfsetbuttcap%
\pgfsetmiterjoin%
\definecolor{currentfill}{rgb}{0.121569,0.466667,0.705882}%
\pgfsetfillcolor{currentfill}%
\pgfsetlinewidth{0.000000pt}%
\definecolor{currentstroke}{rgb}{0.000000,0.000000,0.000000}%
\pgfsetstrokecolor{currentstroke}%
\pgfsetstrokeopacity{0.000000}%
\pgfsetdash{}{0pt}%
\pgfpathmoveto{\pgfqpoint{5.800000in}{66.535723in}}%
\pgfpathlineto{\pgfqpoint{5.800000in}{66.535723in}}%
\pgfpathlineto{\pgfqpoint{5.800000in}{66.638560in}}%
\pgfpathlineto{\pgfqpoint{5.800000in}{66.638560in}}%
\pgfpathclose%
\pgfusepath{fill}%
\end{pgfscope}%
\begin{pgfscope}%
\pgfpathrectangle{\pgfqpoint{5.800000in}{0.720000in}}{\pgfqpoint{1.400000in}{4.620000in}}%
\pgfusepath{clip}%
\pgfsetbuttcap%
\pgfsetmiterjoin%
\definecolor{currentfill}{rgb}{0.121569,0.466667,0.705882}%
\pgfsetfillcolor{currentfill}%
\pgfsetlinewidth{0.000000pt}%
\definecolor{currentstroke}{rgb}{0.000000,0.000000,0.000000}%
\pgfsetstrokecolor{currentstroke}%
\pgfsetstrokeopacity{0.000000}%
\pgfsetdash{}{0pt}%
\pgfpathmoveto{\pgfqpoint{5.800000in}{66.638560in}}%
\pgfpathlineto{\pgfqpoint{5.800000in}{66.638560in}}%
\pgfpathlineto{\pgfqpoint{5.800000in}{66.741398in}}%
\pgfpathlineto{\pgfqpoint{5.800000in}{66.741398in}}%
\pgfpathclose%
\pgfusepath{fill}%
\end{pgfscope}%
\begin{pgfscope}%
\pgfpathrectangle{\pgfqpoint{5.800000in}{0.720000in}}{\pgfqpoint{1.400000in}{4.620000in}}%
\pgfusepath{clip}%
\pgfsetbuttcap%
\pgfsetmiterjoin%
\definecolor{currentfill}{rgb}{0.121569,0.466667,0.705882}%
\pgfsetfillcolor{currentfill}%
\pgfsetlinewidth{0.000000pt}%
\definecolor{currentstroke}{rgb}{0.000000,0.000000,0.000000}%
\pgfsetstrokecolor{currentstroke}%
\pgfsetstrokeopacity{0.000000}%
\pgfsetdash{}{0pt}%
\pgfpathmoveto{\pgfqpoint{5.800000in}{66.741398in}}%
\pgfpathlineto{\pgfqpoint{5.800000in}{66.741398in}}%
\pgfpathlineto{\pgfqpoint{5.800000in}{66.844235in}}%
\pgfpathlineto{\pgfqpoint{5.800000in}{66.844235in}}%
\pgfpathclose%
\pgfusepath{fill}%
\end{pgfscope}%
\begin{pgfscope}%
\pgfpathrectangle{\pgfqpoint{5.800000in}{0.720000in}}{\pgfqpoint{1.400000in}{4.620000in}}%
\pgfusepath{clip}%
\pgfsetbuttcap%
\pgfsetmiterjoin%
\definecolor{currentfill}{rgb}{0.121569,0.466667,0.705882}%
\pgfsetfillcolor{currentfill}%
\pgfsetlinewidth{0.000000pt}%
\definecolor{currentstroke}{rgb}{0.000000,0.000000,0.000000}%
\pgfsetstrokecolor{currentstroke}%
\pgfsetstrokeopacity{0.000000}%
\pgfsetdash{}{0pt}%
\pgfpathmoveto{\pgfqpoint{5.800000in}{66.844235in}}%
\pgfpathlineto{\pgfqpoint{5.800000in}{66.844235in}}%
\pgfpathlineto{\pgfqpoint{5.800000in}{66.947072in}}%
\pgfpathlineto{\pgfqpoint{5.800000in}{66.947072in}}%
\pgfpathclose%
\pgfusepath{fill}%
\end{pgfscope}%
\begin{pgfscope}%
\pgfpathrectangle{\pgfqpoint{5.800000in}{0.720000in}}{\pgfqpoint{1.400000in}{4.620000in}}%
\pgfusepath{clip}%
\pgfsetbuttcap%
\pgfsetmiterjoin%
\definecolor{currentfill}{rgb}{0.121569,0.466667,0.705882}%
\pgfsetfillcolor{currentfill}%
\pgfsetlinewidth{0.000000pt}%
\definecolor{currentstroke}{rgb}{0.000000,0.000000,0.000000}%
\pgfsetstrokecolor{currentstroke}%
\pgfsetstrokeopacity{0.000000}%
\pgfsetdash{}{0pt}%
\pgfpathmoveto{\pgfqpoint{5.800000in}{66.947072in}}%
\pgfpathlineto{\pgfqpoint{5.800000in}{66.947072in}}%
\pgfpathlineto{\pgfqpoint{5.800000in}{67.049909in}}%
\pgfpathlineto{\pgfqpoint{5.800000in}{67.049909in}}%
\pgfpathclose%
\pgfusepath{fill}%
\end{pgfscope}%
\begin{pgfscope}%
\pgfpathrectangle{\pgfqpoint{5.800000in}{0.720000in}}{\pgfqpoint{1.400000in}{4.620000in}}%
\pgfusepath{clip}%
\pgfsetbuttcap%
\pgfsetmiterjoin%
\definecolor{currentfill}{rgb}{0.121569,0.466667,0.705882}%
\pgfsetfillcolor{currentfill}%
\pgfsetlinewidth{0.000000pt}%
\definecolor{currentstroke}{rgb}{0.000000,0.000000,0.000000}%
\pgfsetstrokecolor{currentstroke}%
\pgfsetstrokeopacity{0.000000}%
\pgfsetdash{}{0pt}%
\pgfpathmoveto{\pgfqpoint{5.800000in}{67.049909in}}%
\pgfpathlineto{\pgfqpoint{5.800000in}{67.049909in}}%
\pgfpathlineto{\pgfqpoint{5.800000in}{67.152746in}}%
\pgfpathlineto{\pgfqpoint{5.800000in}{67.152746in}}%
\pgfpathclose%
\pgfusepath{fill}%
\end{pgfscope}%
\begin{pgfscope}%
\pgfpathrectangle{\pgfqpoint{5.800000in}{0.720000in}}{\pgfqpoint{1.400000in}{4.620000in}}%
\pgfusepath{clip}%
\pgfsetbuttcap%
\pgfsetmiterjoin%
\definecolor{currentfill}{rgb}{0.121569,0.466667,0.705882}%
\pgfsetfillcolor{currentfill}%
\pgfsetlinewidth{0.000000pt}%
\definecolor{currentstroke}{rgb}{0.000000,0.000000,0.000000}%
\pgfsetstrokecolor{currentstroke}%
\pgfsetstrokeopacity{0.000000}%
\pgfsetdash{}{0pt}%
\pgfpathmoveto{\pgfqpoint{5.800000in}{67.152746in}}%
\pgfpathlineto{\pgfqpoint{5.800000in}{67.152746in}}%
\pgfpathlineto{\pgfqpoint{5.800000in}{67.255583in}}%
\pgfpathlineto{\pgfqpoint{5.800000in}{67.255583in}}%
\pgfpathclose%
\pgfusepath{fill}%
\end{pgfscope}%
\begin{pgfscope}%
\pgfpathrectangle{\pgfqpoint{5.800000in}{0.720000in}}{\pgfqpoint{1.400000in}{4.620000in}}%
\pgfusepath{clip}%
\pgfsetbuttcap%
\pgfsetmiterjoin%
\definecolor{currentfill}{rgb}{0.121569,0.466667,0.705882}%
\pgfsetfillcolor{currentfill}%
\pgfsetlinewidth{0.000000pt}%
\definecolor{currentstroke}{rgb}{0.000000,0.000000,0.000000}%
\pgfsetstrokecolor{currentstroke}%
\pgfsetstrokeopacity{0.000000}%
\pgfsetdash{}{0pt}%
\pgfpathmoveto{\pgfqpoint{5.800000in}{67.255583in}}%
\pgfpathlineto{\pgfqpoint{5.800000in}{67.255583in}}%
\pgfpathlineto{\pgfqpoint{5.800000in}{67.358420in}}%
\pgfpathlineto{\pgfqpoint{5.800000in}{67.358420in}}%
\pgfpathclose%
\pgfusepath{fill}%
\end{pgfscope}%
\begin{pgfscope}%
\pgfpathrectangle{\pgfqpoint{5.800000in}{0.720000in}}{\pgfqpoint{1.400000in}{4.620000in}}%
\pgfusepath{clip}%
\pgfsetbuttcap%
\pgfsetmiterjoin%
\definecolor{currentfill}{rgb}{0.121569,0.466667,0.705882}%
\pgfsetfillcolor{currentfill}%
\pgfsetlinewidth{0.000000pt}%
\definecolor{currentstroke}{rgb}{0.000000,0.000000,0.000000}%
\pgfsetstrokecolor{currentstroke}%
\pgfsetstrokeopacity{0.000000}%
\pgfsetdash{}{0pt}%
\pgfpathmoveto{\pgfqpoint{5.800000in}{67.358420in}}%
\pgfpathlineto{\pgfqpoint{5.800000in}{67.358420in}}%
\pgfpathlineto{\pgfqpoint{5.800000in}{67.461257in}}%
\pgfpathlineto{\pgfqpoint{5.800000in}{67.461257in}}%
\pgfpathclose%
\pgfusepath{fill}%
\end{pgfscope}%
\begin{pgfscope}%
\pgfpathrectangle{\pgfqpoint{5.800000in}{0.720000in}}{\pgfqpoint{1.400000in}{4.620000in}}%
\pgfusepath{clip}%
\pgfsetbuttcap%
\pgfsetmiterjoin%
\definecolor{currentfill}{rgb}{0.121569,0.466667,0.705882}%
\pgfsetfillcolor{currentfill}%
\pgfsetlinewidth{0.000000pt}%
\definecolor{currentstroke}{rgb}{0.000000,0.000000,0.000000}%
\pgfsetstrokecolor{currentstroke}%
\pgfsetstrokeopacity{0.000000}%
\pgfsetdash{}{0pt}%
\pgfpathmoveto{\pgfqpoint{5.800000in}{67.461257in}}%
\pgfpathlineto{\pgfqpoint{5.800000in}{67.461257in}}%
\pgfpathlineto{\pgfqpoint{5.800000in}{67.564094in}}%
\pgfpathlineto{\pgfqpoint{5.800000in}{67.564094in}}%
\pgfpathclose%
\pgfusepath{fill}%
\end{pgfscope}%
\begin{pgfscope}%
\pgfpathrectangle{\pgfqpoint{5.800000in}{0.720000in}}{\pgfqpoint{1.400000in}{4.620000in}}%
\pgfusepath{clip}%
\pgfsetbuttcap%
\pgfsetmiterjoin%
\definecolor{currentfill}{rgb}{0.121569,0.466667,0.705882}%
\pgfsetfillcolor{currentfill}%
\pgfsetlinewidth{0.000000pt}%
\definecolor{currentstroke}{rgb}{0.000000,0.000000,0.000000}%
\pgfsetstrokecolor{currentstroke}%
\pgfsetstrokeopacity{0.000000}%
\pgfsetdash{}{0pt}%
\pgfpathmoveto{\pgfqpoint{5.800000in}{67.564094in}}%
\pgfpathlineto{\pgfqpoint{5.800000in}{67.564094in}}%
\pgfpathlineto{\pgfqpoint{5.800000in}{67.666931in}}%
\pgfpathlineto{\pgfqpoint{5.800000in}{67.666931in}}%
\pgfpathclose%
\pgfusepath{fill}%
\end{pgfscope}%
\begin{pgfscope}%
\pgfpathrectangle{\pgfqpoint{5.800000in}{0.720000in}}{\pgfqpoint{1.400000in}{4.620000in}}%
\pgfusepath{clip}%
\pgfsetbuttcap%
\pgfsetmiterjoin%
\definecolor{currentfill}{rgb}{0.121569,0.466667,0.705882}%
\pgfsetfillcolor{currentfill}%
\pgfsetlinewidth{0.000000pt}%
\definecolor{currentstroke}{rgb}{0.000000,0.000000,0.000000}%
\pgfsetstrokecolor{currentstroke}%
\pgfsetstrokeopacity{0.000000}%
\pgfsetdash{}{0pt}%
\pgfpathmoveto{\pgfqpoint{5.800000in}{67.666931in}}%
\pgfpathlineto{\pgfqpoint{5.800000in}{67.666931in}}%
\pgfpathlineto{\pgfqpoint{5.800000in}{67.769768in}}%
\pgfpathlineto{\pgfqpoint{5.800000in}{67.769768in}}%
\pgfpathclose%
\pgfusepath{fill}%
\end{pgfscope}%
\begin{pgfscope}%
\pgfpathrectangle{\pgfqpoint{5.800000in}{0.720000in}}{\pgfqpoint{1.400000in}{4.620000in}}%
\pgfusepath{clip}%
\pgfsetbuttcap%
\pgfsetmiterjoin%
\definecolor{currentfill}{rgb}{0.121569,0.466667,0.705882}%
\pgfsetfillcolor{currentfill}%
\pgfsetlinewidth{0.000000pt}%
\definecolor{currentstroke}{rgb}{0.000000,0.000000,0.000000}%
\pgfsetstrokecolor{currentstroke}%
\pgfsetstrokeopacity{0.000000}%
\pgfsetdash{}{0pt}%
\pgfpathmoveto{\pgfqpoint{5.800000in}{67.769768in}}%
\pgfpathlineto{\pgfqpoint{5.800000in}{67.769768in}}%
\pgfpathlineto{\pgfqpoint{5.800000in}{67.872605in}}%
\pgfpathlineto{\pgfqpoint{5.800000in}{67.872605in}}%
\pgfpathclose%
\pgfusepath{fill}%
\end{pgfscope}%
\begin{pgfscope}%
\pgfpathrectangle{\pgfqpoint{5.800000in}{0.720000in}}{\pgfqpoint{1.400000in}{4.620000in}}%
\pgfusepath{clip}%
\pgfsetbuttcap%
\pgfsetmiterjoin%
\definecolor{currentfill}{rgb}{0.121569,0.466667,0.705882}%
\pgfsetfillcolor{currentfill}%
\pgfsetlinewidth{0.000000pt}%
\definecolor{currentstroke}{rgb}{0.000000,0.000000,0.000000}%
\pgfsetstrokecolor{currentstroke}%
\pgfsetstrokeopacity{0.000000}%
\pgfsetdash{}{0pt}%
\pgfpathmoveto{\pgfqpoint{5.800000in}{67.872605in}}%
\pgfpathlineto{\pgfqpoint{5.800000in}{67.872605in}}%
\pgfpathlineto{\pgfqpoint{5.800000in}{67.975442in}}%
\pgfpathlineto{\pgfqpoint{5.800000in}{67.975442in}}%
\pgfpathclose%
\pgfusepath{fill}%
\end{pgfscope}%
\begin{pgfscope}%
\pgfpathrectangle{\pgfqpoint{5.800000in}{0.720000in}}{\pgfqpoint{1.400000in}{4.620000in}}%
\pgfusepath{clip}%
\pgfsetbuttcap%
\pgfsetmiterjoin%
\definecolor{currentfill}{rgb}{0.121569,0.466667,0.705882}%
\pgfsetfillcolor{currentfill}%
\pgfsetlinewidth{0.000000pt}%
\definecolor{currentstroke}{rgb}{0.000000,0.000000,0.000000}%
\pgfsetstrokecolor{currentstroke}%
\pgfsetstrokeopacity{0.000000}%
\pgfsetdash{}{0pt}%
\pgfpathmoveto{\pgfqpoint{5.800000in}{67.975442in}}%
\pgfpathlineto{\pgfqpoint{5.800000in}{67.975442in}}%
\pgfpathlineto{\pgfqpoint{5.800000in}{68.078279in}}%
\pgfpathlineto{\pgfqpoint{5.800000in}{68.078279in}}%
\pgfpathclose%
\pgfusepath{fill}%
\end{pgfscope}%
\begin{pgfscope}%
\pgfpathrectangle{\pgfqpoint{5.800000in}{0.720000in}}{\pgfqpoint{1.400000in}{4.620000in}}%
\pgfusepath{clip}%
\pgfsetbuttcap%
\pgfsetmiterjoin%
\definecolor{currentfill}{rgb}{0.121569,0.466667,0.705882}%
\pgfsetfillcolor{currentfill}%
\pgfsetlinewidth{0.000000pt}%
\definecolor{currentstroke}{rgb}{0.000000,0.000000,0.000000}%
\pgfsetstrokecolor{currentstroke}%
\pgfsetstrokeopacity{0.000000}%
\pgfsetdash{}{0pt}%
\pgfpathmoveto{\pgfqpoint{5.800000in}{68.078279in}}%
\pgfpathlineto{\pgfqpoint{5.800000in}{68.078279in}}%
\pgfpathlineto{\pgfqpoint{5.800000in}{68.181116in}}%
\pgfpathlineto{\pgfqpoint{5.800000in}{68.181116in}}%
\pgfpathclose%
\pgfusepath{fill}%
\end{pgfscope}%
\begin{pgfscope}%
\pgfpathrectangle{\pgfqpoint{5.800000in}{0.720000in}}{\pgfqpoint{1.400000in}{4.620000in}}%
\pgfusepath{clip}%
\pgfsetbuttcap%
\pgfsetmiterjoin%
\definecolor{currentfill}{rgb}{0.121569,0.466667,0.705882}%
\pgfsetfillcolor{currentfill}%
\pgfsetlinewidth{0.000000pt}%
\definecolor{currentstroke}{rgb}{0.000000,0.000000,0.000000}%
\pgfsetstrokecolor{currentstroke}%
\pgfsetstrokeopacity{0.000000}%
\pgfsetdash{}{0pt}%
\pgfpathmoveto{\pgfqpoint{5.800000in}{68.181116in}}%
\pgfpathlineto{\pgfqpoint{5.800000in}{68.181116in}}%
\pgfpathlineto{\pgfqpoint{5.800000in}{68.283954in}}%
\pgfpathlineto{\pgfqpoint{5.800000in}{68.283954in}}%
\pgfpathclose%
\pgfusepath{fill}%
\end{pgfscope}%
\begin{pgfscope}%
\pgfpathrectangle{\pgfqpoint{5.800000in}{0.720000in}}{\pgfqpoint{1.400000in}{4.620000in}}%
\pgfusepath{clip}%
\pgfsetbuttcap%
\pgfsetmiterjoin%
\definecolor{currentfill}{rgb}{0.121569,0.466667,0.705882}%
\pgfsetfillcolor{currentfill}%
\pgfsetlinewidth{0.000000pt}%
\definecolor{currentstroke}{rgb}{0.000000,0.000000,0.000000}%
\pgfsetstrokecolor{currentstroke}%
\pgfsetstrokeopacity{0.000000}%
\pgfsetdash{}{0pt}%
\pgfpathmoveto{\pgfqpoint{5.800000in}{68.283954in}}%
\pgfpathlineto{\pgfqpoint{5.800000in}{68.283954in}}%
\pgfpathlineto{\pgfqpoint{5.800000in}{68.386791in}}%
\pgfpathlineto{\pgfqpoint{5.800000in}{68.386791in}}%
\pgfpathclose%
\pgfusepath{fill}%
\end{pgfscope}%
\begin{pgfscope}%
\pgfpathrectangle{\pgfqpoint{5.800000in}{0.720000in}}{\pgfqpoint{1.400000in}{4.620000in}}%
\pgfusepath{clip}%
\pgfsetbuttcap%
\pgfsetmiterjoin%
\definecolor{currentfill}{rgb}{0.121569,0.466667,0.705882}%
\pgfsetfillcolor{currentfill}%
\pgfsetlinewidth{0.000000pt}%
\definecolor{currentstroke}{rgb}{0.000000,0.000000,0.000000}%
\pgfsetstrokecolor{currentstroke}%
\pgfsetstrokeopacity{0.000000}%
\pgfsetdash{}{0pt}%
\pgfpathmoveto{\pgfqpoint{5.800000in}{68.386791in}}%
\pgfpathlineto{\pgfqpoint{5.800000in}{68.386791in}}%
\pgfpathlineto{\pgfqpoint{5.800000in}{68.489628in}}%
\pgfpathlineto{\pgfqpoint{5.800000in}{68.489628in}}%
\pgfpathclose%
\pgfusepath{fill}%
\end{pgfscope}%
\begin{pgfscope}%
\pgfpathrectangle{\pgfqpoint{5.800000in}{0.720000in}}{\pgfqpoint{1.400000in}{4.620000in}}%
\pgfusepath{clip}%
\pgfsetbuttcap%
\pgfsetmiterjoin%
\definecolor{currentfill}{rgb}{0.121569,0.466667,0.705882}%
\pgfsetfillcolor{currentfill}%
\pgfsetlinewidth{0.000000pt}%
\definecolor{currentstroke}{rgb}{0.000000,0.000000,0.000000}%
\pgfsetstrokecolor{currentstroke}%
\pgfsetstrokeopacity{0.000000}%
\pgfsetdash{}{0pt}%
\pgfpathmoveto{\pgfqpoint{5.800000in}{68.489628in}}%
\pgfpathlineto{\pgfqpoint{5.800000in}{68.489628in}}%
\pgfpathlineto{\pgfqpoint{5.800000in}{68.592465in}}%
\pgfpathlineto{\pgfqpoint{5.800000in}{68.592465in}}%
\pgfpathclose%
\pgfusepath{fill}%
\end{pgfscope}%
\begin{pgfscope}%
\pgfpathrectangle{\pgfqpoint{5.800000in}{0.720000in}}{\pgfqpoint{1.400000in}{4.620000in}}%
\pgfusepath{clip}%
\pgfsetbuttcap%
\pgfsetmiterjoin%
\definecolor{currentfill}{rgb}{0.121569,0.466667,0.705882}%
\pgfsetfillcolor{currentfill}%
\pgfsetlinewidth{0.000000pt}%
\definecolor{currentstroke}{rgb}{0.000000,0.000000,0.000000}%
\pgfsetstrokecolor{currentstroke}%
\pgfsetstrokeopacity{0.000000}%
\pgfsetdash{}{0pt}%
\pgfpathmoveto{\pgfqpoint{5.800000in}{68.592465in}}%
\pgfpathlineto{\pgfqpoint{5.800000in}{68.592465in}}%
\pgfpathlineto{\pgfqpoint{5.800000in}{68.695302in}}%
\pgfpathlineto{\pgfqpoint{5.800000in}{68.695302in}}%
\pgfpathclose%
\pgfusepath{fill}%
\end{pgfscope}%
\begin{pgfscope}%
\pgfpathrectangle{\pgfqpoint{5.800000in}{0.720000in}}{\pgfqpoint{1.400000in}{4.620000in}}%
\pgfusepath{clip}%
\pgfsetbuttcap%
\pgfsetmiterjoin%
\definecolor{currentfill}{rgb}{0.121569,0.466667,0.705882}%
\pgfsetfillcolor{currentfill}%
\pgfsetlinewidth{0.000000pt}%
\definecolor{currentstroke}{rgb}{0.000000,0.000000,0.000000}%
\pgfsetstrokecolor{currentstroke}%
\pgfsetstrokeopacity{0.000000}%
\pgfsetdash{}{0pt}%
\pgfpathmoveto{\pgfqpoint{5.800000in}{68.695302in}}%
\pgfpathlineto{\pgfqpoint{5.800000in}{68.695302in}}%
\pgfpathlineto{\pgfqpoint{5.800000in}{68.798139in}}%
\pgfpathlineto{\pgfqpoint{5.800000in}{68.798139in}}%
\pgfpathclose%
\pgfusepath{fill}%
\end{pgfscope}%
\begin{pgfscope}%
\pgfpathrectangle{\pgfqpoint{5.800000in}{0.720000in}}{\pgfqpoint{1.400000in}{4.620000in}}%
\pgfusepath{clip}%
\pgfsetbuttcap%
\pgfsetmiterjoin%
\definecolor{currentfill}{rgb}{0.121569,0.466667,0.705882}%
\pgfsetfillcolor{currentfill}%
\pgfsetlinewidth{0.000000pt}%
\definecolor{currentstroke}{rgb}{0.000000,0.000000,0.000000}%
\pgfsetstrokecolor{currentstroke}%
\pgfsetstrokeopacity{0.000000}%
\pgfsetdash{}{0pt}%
\pgfpathmoveto{\pgfqpoint{5.800000in}{68.798139in}}%
\pgfpathlineto{\pgfqpoint{5.800000in}{68.798139in}}%
\pgfpathlineto{\pgfqpoint{5.800000in}{68.900976in}}%
\pgfpathlineto{\pgfqpoint{5.800000in}{68.900976in}}%
\pgfpathclose%
\pgfusepath{fill}%
\end{pgfscope}%
\begin{pgfscope}%
\pgfpathrectangle{\pgfqpoint{5.800000in}{0.720000in}}{\pgfqpoint{1.400000in}{4.620000in}}%
\pgfusepath{clip}%
\pgfsetbuttcap%
\pgfsetmiterjoin%
\definecolor{currentfill}{rgb}{0.121569,0.466667,0.705882}%
\pgfsetfillcolor{currentfill}%
\pgfsetlinewidth{0.000000pt}%
\definecolor{currentstroke}{rgb}{0.000000,0.000000,0.000000}%
\pgfsetstrokecolor{currentstroke}%
\pgfsetstrokeopacity{0.000000}%
\pgfsetdash{}{0pt}%
\pgfpathmoveto{\pgfqpoint{5.800000in}{68.900976in}}%
\pgfpathlineto{\pgfqpoint{5.800000in}{68.900976in}}%
\pgfpathlineto{\pgfqpoint{5.800000in}{69.003813in}}%
\pgfpathlineto{\pgfqpoint{5.800000in}{69.003813in}}%
\pgfpathclose%
\pgfusepath{fill}%
\end{pgfscope}%
\begin{pgfscope}%
\pgfpathrectangle{\pgfqpoint{5.800000in}{0.720000in}}{\pgfqpoint{1.400000in}{4.620000in}}%
\pgfusepath{clip}%
\pgfsetbuttcap%
\pgfsetmiterjoin%
\definecolor{currentfill}{rgb}{0.121569,0.466667,0.705882}%
\pgfsetfillcolor{currentfill}%
\pgfsetlinewidth{0.000000pt}%
\definecolor{currentstroke}{rgb}{0.000000,0.000000,0.000000}%
\pgfsetstrokecolor{currentstroke}%
\pgfsetstrokeopacity{0.000000}%
\pgfsetdash{}{0pt}%
\pgfpathmoveto{\pgfqpoint{5.800000in}{69.003813in}}%
\pgfpathlineto{\pgfqpoint{5.800000in}{69.003813in}}%
\pgfpathlineto{\pgfqpoint{5.800000in}{69.106650in}}%
\pgfpathlineto{\pgfqpoint{5.800000in}{69.106650in}}%
\pgfpathclose%
\pgfusepath{fill}%
\end{pgfscope}%
\begin{pgfscope}%
\pgfpathrectangle{\pgfqpoint{5.800000in}{0.720000in}}{\pgfqpoint{1.400000in}{4.620000in}}%
\pgfusepath{clip}%
\pgfsetbuttcap%
\pgfsetmiterjoin%
\definecolor{currentfill}{rgb}{0.121569,0.466667,0.705882}%
\pgfsetfillcolor{currentfill}%
\pgfsetlinewidth{0.000000pt}%
\definecolor{currentstroke}{rgb}{0.000000,0.000000,0.000000}%
\pgfsetstrokecolor{currentstroke}%
\pgfsetstrokeopacity{0.000000}%
\pgfsetdash{}{0pt}%
\pgfpathmoveto{\pgfqpoint{5.800000in}{69.106650in}}%
\pgfpathlineto{\pgfqpoint{5.800000in}{69.106650in}}%
\pgfpathlineto{\pgfqpoint{5.800000in}{69.209487in}}%
\pgfpathlineto{\pgfqpoint{5.800000in}{69.209487in}}%
\pgfpathclose%
\pgfusepath{fill}%
\end{pgfscope}%
\begin{pgfscope}%
\pgfpathrectangle{\pgfqpoint{5.800000in}{0.720000in}}{\pgfqpoint{1.400000in}{4.620000in}}%
\pgfusepath{clip}%
\pgfsetbuttcap%
\pgfsetmiterjoin%
\definecolor{currentfill}{rgb}{0.121569,0.466667,0.705882}%
\pgfsetfillcolor{currentfill}%
\pgfsetlinewidth{0.000000pt}%
\definecolor{currentstroke}{rgb}{0.000000,0.000000,0.000000}%
\pgfsetstrokecolor{currentstroke}%
\pgfsetstrokeopacity{0.000000}%
\pgfsetdash{}{0pt}%
\pgfpathmoveto{\pgfqpoint{5.800000in}{69.209487in}}%
\pgfpathlineto{\pgfqpoint{5.800000in}{69.209487in}}%
\pgfpathlineto{\pgfqpoint{5.800000in}{69.312324in}}%
\pgfpathlineto{\pgfqpoint{5.800000in}{69.312324in}}%
\pgfpathclose%
\pgfusepath{fill}%
\end{pgfscope}%
\begin{pgfscope}%
\pgfpathrectangle{\pgfqpoint{5.800000in}{0.720000in}}{\pgfqpoint{1.400000in}{4.620000in}}%
\pgfusepath{clip}%
\pgfsetbuttcap%
\pgfsetmiterjoin%
\definecolor{currentfill}{rgb}{0.121569,0.466667,0.705882}%
\pgfsetfillcolor{currentfill}%
\pgfsetlinewidth{0.000000pt}%
\definecolor{currentstroke}{rgb}{0.000000,0.000000,0.000000}%
\pgfsetstrokecolor{currentstroke}%
\pgfsetstrokeopacity{0.000000}%
\pgfsetdash{}{0pt}%
\pgfpathmoveto{\pgfqpoint{5.800000in}{69.312324in}}%
\pgfpathlineto{\pgfqpoint{5.800000in}{69.312324in}}%
\pgfpathlineto{\pgfqpoint{5.800000in}{69.415161in}}%
\pgfpathlineto{\pgfqpoint{5.800000in}{69.415161in}}%
\pgfpathclose%
\pgfusepath{fill}%
\end{pgfscope}%
\begin{pgfscope}%
\pgfpathrectangle{\pgfqpoint{5.800000in}{0.720000in}}{\pgfqpoint{1.400000in}{4.620000in}}%
\pgfusepath{clip}%
\pgfsetbuttcap%
\pgfsetmiterjoin%
\definecolor{currentfill}{rgb}{0.121569,0.466667,0.705882}%
\pgfsetfillcolor{currentfill}%
\pgfsetlinewidth{0.000000pt}%
\definecolor{currentstroke}{rgb}{0.000000,0.000000,0.000000}%
\pgfsetstrokecolor{currentstroke}%
\pgfsetstrokeopacity{0.000000}%
\pgfsetdash{}{0pt}%
\pgfpathmoveto{\pgfqpoint{5.800000in}{69.415161in}}%
\pgfpathlineto{\pgfqpoint{5.800000in}{69.415161in}}%
\pgfpathlineto{\pgfqpoint{5.800000in}{69.517998in}}%
\pgfpathlineto{\pgfqpoint{5.800000in}{69.517998in}}%
\pgfpathclose%
\pgfusepath{fill}%
\end{pgfscope}%
\begin{pgfscope}%
\pgfpathrectangle{\pgfqpoint{5.800000in}{0.720000in}}{\pgfqpoint{1.400000in}{4.620000in}}%
\pgfusepath{clip}%
\pgfsetbuttcap%
\pgfsetmiterjoin%
\definecolor{currentfill}{rgb}{0.121569,0.466667,0.705882}%
\pgfsetfillcolor{currentfill}%
\pgfsetlinewidth{0.000000pt}%
\definecolor{currentstroke}{rgb}{0.000000,0.000000,0.000000}%
\pgfsetstrokecolor{currentstroke}%
\pgfsetstrokeopacity{0.000000}%
\pgfsetdash{}{0pt}%
\pgfpathmoveto{\pgfqpoint{5.800000in}{69.517998in}}%
\pgfpathlineto{\pgfqpoint{5.800000in}{69.517998in}}%
\pgfpathlineto{\pgfqpoint{5.800000in}{69.620835in}}%
\pgfpathlineto{\pgfqpoint{5.800000in}{69.620835in}}%
\pgfpathclose%
\pgfusepath{fill}%
\end{pgfscope}%
\begin{pgfscope}%
\pgfpathrectangle{\pgfqpoint{5.800000in}{0.720000in}}{\pgfqpoint{1.400000in}{4.620000in}}%
\pgfusepath{clip}%
\pgfsetbuttcap%
\pgfsetmiterjoin%
\definecolor{currentfill}{rgb}{0.121569,0.466667,0.705882}%
\pgfsetfillcolor{currentfill}%
\pgfsetlinewidth{0.000000pt}%
\definecolor{currentstroke}{rgb}{0.000000,0.000000,0.000000}%
\pgfsetstrokecolor{currentstroke}%
\pgfsetstrokeopacity{0.000000}%
\pgfsetdash{}{0pt}%
\pgfpathmoveto{\pgfqpoint{5.800000in}{69.620835in}}%
\pgfpathlineto{\pgfqpoint{5.800000in}{69.620835in}}%
\pgfpathlineto{\pgfqpoint{5.800000in}{69.723672in}}%
\pgfpathlineto{\pgfqpoint{5.800000in}{69.723672in}}%
\pgfpathclose%
\pgfusepath{fill}%
\end{pgfscope}%
\begin{pgfscope}%
\pgfpathrectangle{\pgfqpoint{5.800000in}{0.720000in}}{\pgfqpoint{1.400000in}{4.620000in}}%
\pgfusepath{clip}%
\pgfsetbuttcap%
\pgfsetmiterjoin%
\definecolor{currentfill}{rgb}{0.121569,0.466667,0.705882}%
\pgfsetfillcolor{currentfill}%
\pgfsetlinewidth{0.000000pt}%
\definecolor{currentstroke}{rgb}{0.000000,0.000000,0.000000}%
\pgfsetstrokecolor{currentstroke}%
\pgfsetstrokeopacity{0.000000}%
\pgfsetdash{}{0pt}%
\pgfpathmoveto{\pgfqpoint{5.800000in}{69.723672in}}%
\pgfpathlineto{\pgfqpoint{5.800000in}{69.723672in}}%
\pgfpathlineto{\pgfqpoint{5.800000in}{69.826510in}}%
\pgfpathlineto{\pgfqpoint{5.800000in}{69.826510in}}%
\pgfpathclose%
\pgfusepath{fill}%
\end{pgfscope}%
\begin{pgfscope}%
\pgfpathrectangle{\pgfqpoint{5.800000in}{0.720000in}}{\pgfqpoint{1.400000in}{4.620000in}}%
\pgfusepath{clip}%
\pgfsetbuttcap%
\pgfsetmiterjoin%
\definecolor{currentfill}{rgb}{0.121569,0.466667,0.705882}%
\pgfsetfillcolor{currentfill}%
\pgfsetlinewidth{0.000000pt}%
\definecolor{currentstroke}{rgb}{0.000000,0.000000,0.000000}%
\pgfsetstrokecolor{currentstroke}%
\pgfsetstrokeopacity{0.000000}%
\pgfsetdash{}{0pt}%
\pgfpathmoveto{\pgfqpoint{5.800000in}{69.826510in}}%
\pgfpathlineto{\pgfqpoint{5.800000in}{69.826510in}}%
\pgfpathlineto{\pgfqpoint{5.800000in}{69.929347in}}%
\pgfpathlineto{\pgfqpoint{5.800000in}{69.929347in}}%
\pgfpathclose%
\pgfusepath{fill}%
\end{pgfscope}%
\begin{pgfscope}%
\pgfpathrectangle{\pgfqpoint{5.800000in}{0.720000in}}{\pgfqpoint{1.400000in}{4.620000in}}%
\pgfusepath{clip}%
\pgfsetbuttcap%
\pgfsetmiterjoin%
\definecolor{currentfill}{rgb}{0.121569,0.466667,0.705882}%
\pgfsetfillcolor{currentfill}%
\pgfsetlinewidth{0.000000pt}%
\definecolor{currentstroke}{rgb}{0.000000,0.000000,0.000000}%
\pgfsetstrokecolor{currentstroke}%
\pgfsetstrokeopacity{0.000000}%
\pgfsetdash{}{0pt}%
\pgfpathmoveto{\pgfqpoint{5.800000in}{69.929347in}}%
\pgfpathlineto{\pgfqpoint{5.800000in}{69.929347in}}%
\pgfpathlineto{\pgfqpoint{5.800000in}{70.032184in}}%
\pgfpathlineto{\pgfqpoint{5.800000in}{70.032184in}}%
\pgfpathclose%
\pgfusepath{fill}%
\end{pgfscope}%
\begin{pgfscope}%
\pgfpathrectangle{\pgfqpoint{5.800000in}{0.720000in}}{\pgfqpoint{1.400000in}{4.620000in}}%
\pgfusepath{clip}%
\pgfsetbuttcap%
\pgfsetmiterjoin%
\definecolor{currentfill}{rgb}{0.121569,0.466667,0.705882}%
\pgfsetfillcolor{currentfill}%
\pgfsetlinewidth{0.000000pt}%
\definecolor{currentstroke}{rgb}{0.000000,0.000000,0.000000}%
\pgfsetstrokecolor{currentstroke}%
\pgfsetstrokeopacity{0.000000}%
\pgfsetdash{}{0pt}%
\pgfpathmoveto{\pgfqpoint{5.800000in}{70.032184in}}%
\pgfpathlineto{\pgfqpoint{5.800000in}{70.032184in}}%
\pgfpathlineto{\pgfqpoint{5.800000in}{70.135021in}}%
\pgfpathlineto{\pgfqpoint{5.800000in}{70.135021in}}%
\pgfpathclose%
\pgfusepath{fill}%
\end{pgfscope}%
\begin{pgfscope}%
\pgfpathrectangle{\pgfqpoint{5.800000in}{0.720000in}}{\pgfqpoint{1.400000in}{4.620000in}}%
\pgfusepath{clip}%
\pgfsetbuttcap%
\pgfsetmiterjoin%
\definecolor{currentfill}{rgb}{0.121569,0.466667,0.705882}%
\pgfsetfillcolor{currentfill}%
\pgfsetlinewidth{0.000000pt}%
\definecolor{currentstroke}{rgb}{0.000000,0.000000,0.000000}%
\pgfsetstrokecolor{currentstroke}%
\pgfsetstrokeopacity{0.000000}%
\pgfsetdash{}{0pt}%
\pgfpathmoveto{\pgfqpoint{5.800000in}{70.135021in}}%
\pgfpathlineto{\pgfqpoint{5.800000in}{70.135021in}}%
\pgfpathlineto{\pgfqpoint{5.800000in}{70.237858in}}%
\pgfpathlineto{\pgfqpoint{5.800000in}{70.237858in}}%
\pgfpathclose%
\pgfusepath{fill}%
\end{pgfscope}%
\begin{pgfscope}%
\pgfpathrectangle{\pgfqpoint{5.800000in}{0.720000in}}{\pgfqpoint{1.400000in}{4.620000in}}%
\pgfusepath{clip}%
\pgfsetbuttcap%
\pgfsetmiterjoin%
\definecolor{currentfill}{rgb}{0.121569,0.466667,0.705882}%
\pgfsetfillcolor{currentfill}%
\pgfsetlinewidth{0.000000pt}%
\definecolor{currentstroke}{rgb}{0.000000,0.000000,0.000000}%
\pgfsetstrokecolor{currentstroke}%
\pgfsetstrokeopacity{0.000000}%
\pgfsetdash{}{0pt}%
\pgfpathmoveto{\pgfqpoint{5.800000in}{70.237858in}}%
\pgfpathlineto{\pgfqpoint{5.800000in}{70.237858in}}%
\pgfpathlineto{\pgfqpoint{5.800000in}{70.340695in}}%
\pgfpathlineto{\pgfqpoint{5.800000in}{70.340695in}}%
\pgfpathclose%
\pgfusepath{fill}%
\end{pgfscope}%
\begin{pgfscope}%
\pgfpathrectangle{\pgfqpoint{5.800000in}{0.720000in}}{\pgfqpoint{1.400000in}{4.620000in}}%
\pgfusepath{clip}%
\pgfsetbuttcap%
\pgfsetmiterjoin%
\definecolor{currentfill}{rgb}{0.121569,0.466667,0.705882}%
\pgfsetfillcolor{currentfill}%
\pgfsetlinewidth{0.000000pt}%
\definecolor{currentstroke}{rgb}{0.000000,0.000000,0.000000}%
\pgfsetstrokecolor{currentstroke}%
\pgfsetstrokeopacity{0.000000}%
\pgfsetdash{}{0pt}%
\pgfpathmoveto{\pgfqpoint{5.800000in}{70.340695in}}%
\pgfpathlineto{\pgfqpoint{5.800000in}{70.340695in}}%
\pgfpathlineto{\pgfqpoint{5.800000in}{70.443532in}}%
\pgfpathlineto{\pgfqpoint{5.800000in}{70.443532in}}%
\pgfpathclose%
\pgfusepath{fill}%
\end{pgfscope}%
\begin{pgfscope}%
\pgfpathrectangle{\pgfqpoint{5.800000in}{0.720000in}}{\pgfqpoint{1.400000in}{4.620000in}}%
\pgfusepath{clip}%
\pgfsetbuttcap%
\pgfsetmiterjoin%
\definecolor{currentfill}{rgb}{0.121569,0.466667,0.705882}%
\pgfsetfillcolor{currentfill}%
\pgfsetlinewidth{0.000000pt}%
\definecolor{currentstroke}{rgb}{0.000000,0.000000,0.000000}%
\pgfsetstrokecolor{currentstroke}%
\pgfsetstrokeopacity{0.000000}%
\pgfsetdash{}{0pt}%
\pgfpathmoveto{\pgfqpoint{5.800000in}{70.443532in}}%
\pgfpathlineto{\pgfqpoint{5.800000in}{70.443532in}}%
\pgfpathlineto{\pgfqpoint{5.800000in}{70.546369in}}%
\pgfpathlineto{\pgfqpoint{5.800000in}{70.546369in}}%
\pgfpathclose%
\pgfusepath{fill}%
\end{pgfscope}%
\begin{pgfscope}%
\pgfpathrectangle{\pgfqpoint{5.800000in}{0.720000in}}{\pgfqpoint{1.400000in}{4.620000in}}%
\pgfusepath{clip}%
\pgfsetbuttcap%
\pgfsetmiterjoin%
\definecolor{currentfill}{rgb}{0.121569,0.466667,0.705882}%
\pgfsetfillcolor{currentfill}%
\pgfsetlinewidth{0.000000pt}%
\definecolor{currentstroke}{rgb}{0.000000,0.000000,0.000000}%
\pgfsetstrokecolor{currentstroke}%
\pgfsetstrokeopacity{0.000000}%
\pgfsetdash{}{0pt}%
\pgfpathmoveto{\pgfqpoint{5.800000in}{70.546369in}}%
\pgfpathlineto{\pgfqpoint{5.800000in}{70.546369in}}%
\pgfpathlineto{\pgfqpoint{5.800000in}{70.649206in}}%
\pgfpathlineto{\pgfqpoint{5.800000in}{70.649206in}}%
\pgfpathclose%
\pgfusepath{fill}%
\end{pgfscope}%
\begin{pgfscope}%
\pgfpathrectangle{\pgfqpoint{5.800000in}{0.720000in}}{\pgfqpoint{1.400000in}{4.620000in}}%
\pgfusepath{clip}%
\pgfsetbuttcap%
\pgfsetmiterjoin%
\definecolor{currentfill}{rgb}{0.121569,0.466667,0.705882}%
\pgfsetfillcolor{currentfill}%
\pgfsetlinewidth{0.000000pt}%
\definecolor{currentstroke}{rgb}{0.000000,0.000000,0.000000}%
\pgfsetstrokecolor{currentstroke}%
\pgfsetstrokeopacity{0.000000}%
\pgfsetdash{}{0pt}%
\pgfpathmoveto{\pgfqpoint{5.800000in}{70.649206in}}%
\pgfpathlineto{\pgfqpoint{5.800000in}{70.649206in}}%
\pgfpathlineto{\pgfqpoint{5.800000in}{70.752043in}}%
\pgfpathlineto{\pgfqpoint{5.800000in}{70.752043in}}%
\pgfpathclose%
\pgfusepath{fill}%
\end{pgfscope}%
\begin{pgfscope}%
\pgfpathrectangle{\pgfqpoint{5.800000in}{0.720000in}}{\pgfqpoint{1.400000in}{4.620000in}}%
\pgfusepath{clip}%
\pgfsetbuttcap%
\pgfsetmiterjoin%
\definecolor{currentfill}{rgb}{0.121569,0.466667,0.705882}%
\pgfsetfillcolor{currentfill}%
\pgfsetlinewidth{0.000000pt}%
\definecolor{currentstroke}{rgb}{0.000000,0.000000,0.000000}%
\pgfsetstrokecolor{currentstroke}%
\pgfsetstrokeopacity{0.000000}%
\pgfsetdash{}{0pt}%
\pgfpathmoveto{\pgfqpoint{5.800000in}{70.752043in}}%
\pgfpathlineto{\pgfqpoint{5.800000in}{70.752043in}}%
\pgfpathlineto{\pgfqpoint{5.800000in}{70.854880in}}%
\pgfpathlineto{\pgfqpoint{5.800000in}{70.854880in}}%
\pgfpathclose%
\pgfusepath{fill}%
\end{pgfscope}%
\begin{pgfscope}%
\pgfpathrectangle{\pgfqpoint{5.800000in}{0.720000in}}{\pgfqpoint{1.400000in}{4.620000in}}%
\pgfusepath{clip}%
\pgfsetbuttcap%
\pgfsetmiterjoin%
\definecolor{currentfill}{rgb}{0.121569,0.466667,0.705882}%
\pgfsetfillcolor{currentfill}%
\pgfsetlinewidth{0.000000pt}%
\definecolor{currentstroke}{rgb}{0.000000,0.000000,0.000000}%
\pgfsetstrokecolor{currentstroke}%
\pgfsetstrokeopacity{0.000000}%
\pgfsetdash{}{0pt}%
\pgfpathmoveto{\pgfqpoint{5.800000in}{70.854880in}}%
\pgfpathlineto{\pgfqpoint{5.800000in}{70.854880in}}%
\pgfpathlineto{\pgfqpoint{5.800000in}{70.957717in}}%
\pgfpathlineto{\pgfqpoint{5.800000in}{70.957717in}}%
\pgfpathclose%
\pgfusepath{fill}%
\end{pgfscope}%
\begin{pgfscope}%
\pgfpathrectangle{\pgfqpoint{5.800000in}{0.720000in}}{\pgfqpoint{1.400000in}{4.620000in}}%
\pgfusepath{clip}%
\pgfsetbuttcap%
\pgfsetmiterjoin%
\definecolor{currentfill}{rgb}{0.121569,0.466667,0.705882}%
\pgfsetfillcolor{currentfill}%
\pgfsetlinewidth{0.000000pt}%
\definecolor{currentstroke}{rgb}{0.000000,0.000000,0.000000}%
\pgfsetstrokecolor{currentstroke}%
\pgfsetstrokeopacity{0.000000}%
\pgfsetdash{}{0pt}%
\pgfpathmoveto{\pgfqpoint{5.800000in}{70.957717in}}%
\pgfpathlineto{\pgfqpoint{5.800000in}{70.957717in}}%
\pgfpathlineto{\pgfqpoint{5.800000in}{71.060554in}}%
\pgfpathlineto{\pgfqpoint{5.800000in}{71.060554in}}%
\pgfpathclose%
\pgfusepath{fill}%
\end{pgfscope}%
\begin{pgfscope}%
\pgfpathrectangle{\pgfqpoint{5.800000in}{0.720000in}}{\pgfqpoint{1.400000in}{4.620000in}}%
\pgfusepath{clip}%
\pgfsetbuttcap%
\pgfsetmiterjoin%
\definecolor{currentfill}{rgb}{0.121569,0.466667,0.705882}%
\pgfsetfillcolor{currentfill}%
\pgfsetlinewidth{0.000000pt}%
\definecolor{currentstroke}{rgb}{0.000000,0.000000,0.000000}%
\pgfsetstrokecolor{currentstroke}%
\pgfsetstrokeopacity{0.000000}%
\pgfsetdash{}{0pt}%
\pgfpathmoveto{\pgfqpoint{5.800000in}{71.060554in}}%
\pgfpathlineto{\pgfqpoint{5.800000in}{71.060554in}}%
\pgfpathlineto{\pgfqpoint{5.800000in}{71.163391in}}%
\pgfpathlineto{\pgfqpoint{5.800000in}{71.163391in}}%
\pgfpathclose%
\pgfusepath{fill}%
\end{pgfscope}%
\begin{pgfscope}%
\pgfpathrectangle{\pgfqpoint{5.800000in}{0.720000in}}{\pgfqpoint{1.400000in}{4.620000in}}%
\pgfusepath{clip}%
\pgfsetbuttcap%
\pgfsetmiterjoin%
\definecolor{currentfill}{rgb}{0.121569,0.466667,0.705882}%
\pgfsetfillcolor{currentfill}%
\pgfsetlinewidth{0.000000pt}%
\definecolor{currentstroke}{rgb}{0.000000,0.000000,0.000000}%
\pgfsetstrokecolor{currentstroke}%
\pgfsetstrokeopacity{0.000000}%
\pgfsetdash{}{0pt}%
\pgfpathmoveto{\pgfqpoint{5.800000in}{71.163391in}}%
\pgfpathlineto{\pgfqpoint{5.800000in}{71.163391in}}%
\pgfpathlineto{\pgfqpoint{5.800000in}{71.266228in}}%
\pgfpathlineto{\pgfqpoint{5.800000in}{71.266228in}}%
\pgfpathclose%
\pgfusepath{fill}%
\end{pgfscope}%
\begin{pgfscope}%
\pgfpathrectangle{\pgfqpoint{5.800000in}{0.720000in}}{\pgfqpoint{1.400000in}{4.620000in}}%
\pgfusepath{clip}%
\pgfsetbuttcap%
\pgfsetmiterjoin%
\definecolor{currentfill}{rgb}{0.121569,0.466667,0.705882}%
\pgfsetfillcolor{currentfill}%
\pgfsetlinewidth{0.000000pt}%
\definecolor{currentstroke}{rgb}{0.000000,0.000000,0.000000}%
\pgfsetstrokecolor{currentstroke}%
\pgfsetstrokeopacity{0.000000}%
\pgfsetdash{}{0pt}%
\pgfpathmoveto{\pgfqpoint{5.800000in}{71.266228in}}%
\pgfpathlineto{\pgfqpoint{5.800000in}{71.266228in}}%
\pgfpathlineto{\pgfqpoint{5.800000in}{71.369066in}}%
\pgfpathlineto{\pgfqpoint{5.800000in}{71.369066in}}%
\pgfpathclose%
\pgfusepath{fill}%
\end{pgfscope}%
\begin{pgfscope}%
\pgfpathrectangle{\pgfqpoint{5.800000in}{0.720000in}}{\pgfqpoint{1.400000in}{4.620000in}}%
\pgfusepath{clip}%
\pgfsetbuttcap%
\pgfsetmiterjoin%
\definecolor{currentfill}{rgb}{0.121569,0.466667,0.705882}%
\pgfsetfillcolor{currentfill}%
\pgfsetlinewidth{0.000000pt}%
\definecolor{currentstroke}{rgb}{0.000000,0.000000,0.000000}%
\pgfsetstrokecolor{currentstroke}%
\pgfsetstrokeopacity{0.000000}%
\pgfsetdash{}{0pt}%
\pgfpathmoveto{\pgfqpoint{5.800000in}{71.369066in}}%
\pgfpathlineto{\pgfqpoint{5.800000in}{71.369066in}}%
\pgfpathlineto{\pgfqpoint{5.800000in}{71.471903in}}%
\pgfpathlineto{\pgfqpoint{5.800000in}{71.471903in}}%
\pgfpathclose%
\pgfusepath{fill}%
\end{pgfscope}%
\begin{pgfscope}%
\pgfpathrectangle{\pgfqpoint{5.800000in}{0.720000in}}{\pgfqpoint{1.400000in}{4.620000in}}%
\pgfusepath{clip}%
\pgfsetbuttcap%
\pgfsetmiterjoin%
\definecolor{currentfill}{rgb}{0.121569,0.466667,0.705882}%
\pgfsetfillcolor{currentfill}%
\pgfsetlinewidth{0.000000pt}%
\definecolor{currentstroke}{rgb}{0.000000,0.000000,0.000000}%
\pgfsetstrokecolor{currentstroke}%
\pgfsetstrokeopacity{0.000000}%
\pgfsetdash{}{0pt}%
\pgfpathmoveto{\pgfqpoint{5.800000in}{71.471903in}}%
\pgfpathlineto{\pgfqpoint{5.800000in}{71.471903in}}%
\pgfpathlineto{\pgfqpoint{5.800000in}{71.574740in}}%
\pgfpathlineto{\pgfqpoint{5.800000in}{71.574740in}}%
\pgfpathclose%
\pgfusepath{fill}%
\end{pgfscope}%
\begin{pgfscope}%
\pgfpathrectangle{\pgfqpoint{5.800000in}{0.720000in}}{\pgfqpoint{1.400000in}{4.620000in}}%
\pgfusepath{clip}%
\pgfsetbuttcap%
\pgfsetmiterjoin%
\definecolor{currentfill}{rgb}{0.121569,0.466667,0.705882}%
\pgfsetfillcolor{currentfill}%
\pgfsetlinewidth{0.000000pt}%
\definecolor{currentstroke}{rgb}{0.000000,0.000000,0.000000}%
\pgfsetstrokecolor{currentstroke}%
\pgfsetstrokeopacity{0.000000}%
\pgfsetdash{}{0pt}%
\pgfpathmoveto{\pgfqpoint{5.800000in}{71.574740in}}%
\pgfpathlineto{\pgfqpoint{5.800000in}{71.574740in}}%
\pgfpathlineto{\pgfqpoint{5.800000in}{71.677577in}}%
\pgfpathlineto{\pgfqpoint{5.800000in}{71.677577in}}%
\pgfpathclose%
\pgfusepath{fill}%
\end{pgfscope}%
\begin{pgfscope}%
\pgfpathrectangle{\pgfqpoint{5.800000in}{0.720000in}}{\pgfqpoint{1.400000in}{4.620000in}}%
\pgfusepath{clip}%
\pgfsetbuttcap%
\pgfsetmiterjoin%
\definecolor{currentfill}{rgb}{0.121569,0.466667,0.705882}%
\pgfsetfillcolor{currentfill}%
\pgfsetlinewidth{0.000000pt}%
\definecolor{currentstroke}{rgb}{0.000000,0.000000,0.000000}%
\pgfsetstrokecolor{currentstroke}%
\pgfsetstrokeopacity{0.000000}%
\pgfsetdash{}{0pt}%
\pgfpathmoveto{\pgfqpoint{5.800000in}{71.677577in}}%
\pgfpathlineto{\pgfqpoint{5.800000in}{71.677577in}}%
\pgfpathlineto{\pgfqpoint{5.800000in}{71.780414in}}%
\pgfpathlineto{\pgfqpoint{5.800000in}{71.780414in}}%
\pgfpathclose%
\pgfusepath{fill}%
\end{pgfscope}%
\begin{pgfscope}%
\pgfpathrectangle{\pgfqpoint{5.800000in}{0.720000in}}{\pgfqpoint{1.400000in}{4.620000in}}%
\pgfusepath{clip}%
\pgfsetbuttcap%
\pgfsetmiterjoin%
\definecolor{currentfill}{rgb}{0.121569,0.466667,0.705882}%
\pgfsetfillcolor{currentfill}%
\pgfsetlinewidth{0.000000pt}%
\definecolor{currentstroke}{rgb}{0.000000,0.000000,0.000000}%
\pgfsetstrokecolor{currentstroke}%
\pgfsetstrokeopacity{0.000000}%
\pgfsetdash{}{0pt}%
\pgfpathmoveto{\pgfqpoint{5.800000in}{71.780414in}}%
\pgfpathlineto{\pgfqpoint{5.800000in}{71.780414in}}%
\pgfpathlineto{\pgfqpoint{5.800000in}{71.883251in}}%
\pgfpathlineto{\pgfqpoint{5.800000in}{71.883251in}}%
\pgfpathclose%
\pgfusepath{fill}%
\end{pgfscope}%
\begin{pgfscope}%
\pgfpathrectangle{\pgfqpoint{5.800000in}{0.720000in}}{\pgfqpoint{1.400000in}{4.620000in}}%
\pgfusepath{clip}%
\pgfsetbuttcap%
\pgfsetmiterjoin%
\definecolor{currentfill}{rgb}{0.121569,0.466667,0.705882}%
\pgfsetfillcolor{currentfill}%
\pgfsetlinewidth{0.000000pt}%
\definecolor{currentstroke}{rgb}{0.000000,0.000000,0.000000}%
\pgfsetstrokecolor{currentstroke}%
\pgfsetstrokeopacity{0.000000}%
\pgfsetdash{}{0pt}%
\pgfpathmoveto{\pgfqpoint{5.800000in}{71.883251in}}%
\pgfpathlineto{\pgfqpoint{5.800000in}{71.883251in}}%
\pgfpathlineto{\pgfqpoint{5.800000in}{71.986088in}}%
\pgfpathlineto{\pgfqpoint{5.800000in}{71.986088in}}%
\pgfpathclose%
\pgfusepath{fill}%
\end{pgfscope}%
\begin{pgfscope}%
\pgfpathrectangle{\pgfqpoint{5.800000in}{0.720000in}}{\pgfqpoint{1.400000in}{4.620000in}}%
\pgfusepath{clip}%
\pgfsetbuttcap%
\pgfsetmiterjoin%
\definecolor{currentfill}{rgb}{0.121569,0.466667,0.705882}%
\pgfsetfillcolor{currentfill}%
\pgfsetlinewidth{0.000000pt}%
\definecolor{currentstroke}{rgb}{0.000000,0.000000,0.000000}%
\pgfsetstrokecolor{currentstroke}%
\pgfsetstrokeopacity{0.000000}%
\pgfsetdash{}{0pt}%
\pgfpathmoveto{\pgfqpoint{5.800000in}{71.986088in}}%
\pgfpathlineto{\pgfqpoint{5.800000in}{71.986088in}}%
\pgfpathlineto{\pgfqpoint{5.800000in}{72.088925in}}%
\pgfpathlineto{\pgfqpoint{5.800000in}{72.088925in}}%
\pgfpathclose%
\pgfusepath{fill}%
\end{pgfscope}%
\begin{pgfscope}%
\pgfpathrectangle{\pgfqpoint{5.800000in}{0.720000in}}{\pgfqpoint{1.400000in}{4.620000in}}%
\pgfusepath{clip}%
\pgfsetbuttcap%
\pgfsetmiterjoin%
\definecolor{currentfill}{rgb}{0.121569,0.466667,0.705882}%
\pgfsetfillcolor{currentfill}%
\pgfsetlinewidth{0.000000pt}%
\definecolor{currentstroke}{rgb}{0.000000,0.000000,0.000000}%
\pgfsetstrokecolor{currentstroke}%
\pgfsetstrokeopacity{0.000000}%
\pgfsetdash{}{0pt}%
\pgfpathmoveto{\pgfqpoint{5.800000in}{72.088925in}}%
\pgfpathlineto{\pgfqpoint{5.800000in}{72.088925in}}%
\pgfpathlineto{\pgfqpoint{5.800000in}{72.191762in}}%
\pgfpathlineto{\pgfqpoint{5.800000in}{72.191762in}}%
\pgfpathclose%
\pgfusepath{fill}%
\end{pgfscope}%
\begin{pgfscope}%
\pgfpathrectangle{\pgfqpoint{5.800000in}{0.720000in}}{\pgfqpoint{1.400000in}{4.620000in}}%
\pgfusepath{clip}%
\pgfsetbuttcap%
\pgfsetmiterjoin%
\definecolor{currentfill}{rgb}{0.121569,0.466667,0.705882}%
\pgfsetfillcolor{currentfill}%
\pgfsetlinewidth{0.000000pt}%
\definecolor{currentstroke}{rgb}{0.000000,0.000000,0.000000}%
\pgfsetstrokecolor{currentstroke}%
\pgfsetstrokeopacity{0.000000}%
\pgfsetdash{}{0pt}%
\pgfpathmoveto{\pgfqpoint{5.800000in}{72.191762in}}%
\pgfpathlineto{\pgfqpoint{5.800000in}{72.191762in}}%
\pgfpathlineto{\pgfqpoint{5.800000in}{72.294599in}}%
\pgfpathlineto{\pgfqpoint{5.800000in}{72.294599in}}%
\pgfpathclose%
\pgfusepath{fill}%
\end{pgfscope}%
\begin{pgfscope}%
\pgfpathrectangle{\pgfqpoint{5.800000in}{0.720000in}}{\pgfqpoint{1.400000in}{4.620000in}}%
\pgfusepath{clip}%
\pgfsetbuttcap%
\pgfsetmiterjoin%
\definecolor{currentfill}{rgb}{0.121569,0.466667,0.705882}%
\pgfsetfillcolor{currentfill}%
\pgfsetlinewidth{0.000000pt}%
\definecolor{currentstroke}{rgb}{0.000000,0.000000,0.000000}%
\pgfsetstrokecolor{currentstroke}%
\pgfsetstrokeopacity{0.000000}%
\pgfsetdash{}{0pt}%
\pgfpathmoveto{\pgfqpoint{5.800000in}{72.294599in}}%
\pgfpathlineto{\pgfqpoint{5.800000in}{72.294599in}}%
\pgfpathlineto{\pgfqpoint{5.800000in}{72.397436in}}%
\pgfpathlineto{\pgfqpoint{5.800000in}{72.397436in}}%
\pgfpathclose%
\pgfusepath{fill}%
\end{pgfscope}%
\begin{pgfscope}%
\pgfpathrectangle{\pgfqpoint{5.800000in}{0.720000in}}{\pgfqpoint{1.400000in}{4.620000in}}%
\pgfusepath{clip}%
\pgfsetbuttcap%
\pgfsetmiterjoin%
\definecolor{currentfill}{rgb}{0.121569,0.466667,0.705882}%
\pgfsetfillcolor{currentfill}%
\pgfsetlinewidth{0.000000pt}%
\definecolor{currentstroke}{rgb}{0.000000,0.000000,0.000000}%
\pgfsetstrokecolor{currentstroke}%
\pgfsetstrokeopacity{0.000000}%
\pgfsetdash{}{0pt}%
\pgfpathmoveto{\pgfqpoint{5.800000in}{72.397436in}}%
\pgfpathlineto{\pgfqpoint{5.800000in}{72.397436in}}%
\pgfpathlineto{\pgfqpoint{5.800000in}{72.500273in}}%
\pgfpathlineto{\pgfqpoint{5.800000in}{72.500273in}}%
\pgfpathclose%
\pgfusepath{fill}%
\end{pgfscope}%
\begin{pgfscope}%
\pgfpathrectangle{\pgfqpoint{5.800000in}{0.720000in}}{\pgfqpoint{1.400000in}{4.620000in}}%
\pgfusepath{clip}%
\pgfsetbuttcap%
\pgfsetmiterjoin%
\definecolor{currentfill}{rgb}{0.121569,0.466667,0.705882}%
\pgfsetfillcolor{currentfill}%
\pgfsetlinewidth{0.000000pt}%
\definecolor{currentstroke}{rgb}{0.000000,0.000000,0.000000}%
\pgfsetstrokecolor{currentstroke}%
\pgfsetstrokeopacity{0.000000}%
\pgfsetdash{}{0pt}%
\pgfpathmoveto{\pgfqpoint{5.800000in}{72.500273in}}%
\pgfpathlineto{\pgfqpoint{5.800000in}{72.500273in}}%
\pgfpathlineto{\pgfqpoint{5.800000in}{72.603110in}}%
\pgfpathlineto{\pgfqpoint{5.800000in}{72.603110in}}%
\pgfpathclose%
\pgfusepath{fill}%
\end{pgfscope}%
\begin{pgfscope}%
\pgfpathrectangle{\pgfqpoint{5.800000in}{0.720000in}}{\pgfqpoint{1.400000in}{4.620000in}}%
\pgfusepath{clip}%
\pgfsetbuttcap%
\pgfsetmiterjoin%
\definecolor{currentfill}{rgb}{0.121569,0.466667,0.705882}%
\pgfsetfillcolor{currentfill}%
\pgfsetlinewidth{0.000000pt}%
\definecolor{currentstroke}{rgb}{0.000000,0.000000,0.000000}%
\pgfsetstrokecolor{currentstroke}%
\pgfsetstrokeopacity{0.000000}%
\pgfsetdash{}{0pt}%
\pgfpathmoveto{\pgfqpoint{5.800000in}{72.603110in}}%
\pgfpathlineto{\pgfqpoint{5.800000in}{72.603110in}}%
\pgfpathlineto{\pgfqpoint{5.800000in}{72.705947in}}%
\pgfpathlineto{\pgfqpoint{5.800000in}{72.705947in}}%
\pgfpathclose%
\pgfusepath{fill}%
\end{pgfscope}%
\begin{pgfscope}%
\pgfpathrectangle{\pgfqpoint{5.800000in}{0.720000in}}{\pgfqpoint{1.400000in}{4.620000in}}%
\pgfusepath{clip}%
\pgfsetbuttcap%
\pgfsetmiterjoin%
\definecolor{currentfill}{rgb}{0.121569,0.466667,0.705882}%
\pgfsetfillcolor{currentfill}%
\pgfsetlinewidth{0.000000pt}%
\definecolor{currentstroke}{rgb}{0.000000,0.000000,0.000000}%
\pgfsetstrokecolor{currentstroke}%
\pgfsetstrokeopacity{0.000000}%
\pgfsetdash{}{0pt}%
\pgfpathmoveto{\pgfqpoint{5.800000in}{72.705947in}}%
\pgfpathlineto{\pgfqpoint{5.800000in}{72.705947in}}%
\pgfpathlineto{\pgfqpoint{5.800000in}{72.808785in}}%
\pgfpathlineto{\pgfqpoint{5.800000in}{72.808785in}}%
\pgfpathclose%
\pgfusepath{fill}%
\end{pgfscope}%
\begin{pgfscope}%
\pgfpathrectangle{\pgfqpoint{5.800000in}{0.720000in}}{\pgfqpoint{1.400000in}{4.620000in}}%
\pgfusepath{clip}%
\pgfsetbuttcap%
\pgfsetmiterjoin%
\definecolor{currentfill}{rgb}{0.121569,0.466667,0.705882}%
\pgfsetfillcolor{currentfill}%
\pgfsetlinewidth{0.000000pt}%
\definecolor{currentstroke}{rgb}{0.000000,0.000000,0.000000}%
\pgfsetstrokecolor{currentstroke}%
\pgfsetstrokeopacity{0.000000}%
\pgfsetdash{}{0pt}%
\pgfpathmoveto{\pgfqpoint{5.800000in}{72.808785in}}%
\pgfpathlineto{\pgfqpoint{5.800000in}{72.808785in}}%
\pgfpathlineto{\pgfqpoint{5.800000in}{72.911622in}}%
\pgfpathlineto{\pgfqpoint{5.800000in}{72.911622in}}%
\pgfpathclose%
\pgfusepath{fill}%
\end{pgfscope}%
\begin{pgfscope}%
\pgfpathrectangle{\pgfqpoint{5.800000in}{0.720000in}}{\pgfqpoint{1.400000in}{4.620000in}}%
\pgfusepath{clip}%
\pgfsetbuttcap%
\pgfsetmiterjoin%
\definecolor{currentfill}{rgb}{0.121569,0.466667,0.705882}%
\pgfsetfillcolor{currentfill}%
\pgfsetlinewidth{0.000000pt}%
\definecolor{currentstroke}{rgb}{0.000000,0.000000,0.000000}%
\pgfsetstrokecolor{currentstroke}%
\pgfsetstrokeopacity{0.000000}%
\pgfsetdash{}{0pt}%
\pgfpathmoveto{\pgfqpoint{5.800000in}{72.911622in}}%
\pgfpathlineto{\pgfqpoint{5.800000in}{72.911622in}}%
\pgfpathlineto{\pgfqpoint{5.800000in}{73.014459in}}%
\pgfpathlineto{\pgfqpoint{5.800000in}{73.014459in}}%
\pgfpathclose%
\pgfusepath{fill}%
\end{pgfscope}%
\begin{pgfscope}%
\pgfpathrectangle{\pgfqpoint{5.800000in}{0.720000in}}{\pgfqpoint{1.400000in}{4.620000in}}%
\pgfusepath{clip}%
\pgfsetbuttcap%
\pgfsetmiterjoin%
\definecolor{currentfill}{rgb}{0.121569,0.466667,0.705882}%
\pgfsetfillcolor{currentfill}%
\pgfsetlinewidth{0.000000pt}%
\definecolor{currentstroke}{rgb}{0.000000,0.000000,0.000000}%
\pgfsetstrokecolor{currentstroke}%
\pgfsetstrokeopacity{0.000000}%
\pgfsetdash{}{0pt}%
\pgfpathmoveto{\pgfqpoint{5.800000in}{73.014459in}}%
\pgfpathlineto{\pgfqpoint{5.800000in}{73.014459in}}%
\pgfpathlineto{\pgfqpoint{5.800000in}{73.117296in}}%
\pgfpathlineto{\pgfqpoint{5.800000in}{73.117296in}}%
\pgfpathclose%
\pgfusepath{fill}%
\end{pgfscope}%
\begin{pgfscope}%
\pgfpathrectangle{\pgfqpoint{5.800000in}{0.720000in}}{\pgfqpoint{1.400000in}{4.620000in}}%
\pgfusepath{clip}%
\pgfsetbuttcap%
\pgfsetmiterjoin%
\definecolor{currentfill}{rgb}{0.121569,0.466667,0.705882}%
\pgfsetfillcolor{currentfill}%
\pgfsetlinewidth{0.000000pt}%
\definecolor{currentstroke}{rgb}{0.000000,0.000000,0.000000}%
\pgfsetstrokecolor{currentstroke}%
\pgfsetstrokeopacity{0.000000}%
\pgfsetdash{}{0pt}%
\pgfpathmoveto{\pgfqpoint{5.800000in}{73.117296in}}%
\pgfpathlineto{\pgfqpoint{5.800000in}{73.117296in}}%
\pgfpathlineto{\pgfqpoint{5.800000in}{73.220133in}}%
\pgfpathlineto{\pgfqpoint{5.800000in}{73.220133in}}%
\pgfpathclose%
\pgfusepath{fill}%
\end{pgfscope}%
\begin{pgfscope}%
\pgfpathrectangle{\pgfqpoint{5.800000in}{0.720000in}}{\pgfqpoint{1.400000in}{4.620000in}}%
\pgfusepath{clip}%
\pgfsetbuttcap%
\pgfsetmiterjoin%
\definecolor{currentfill}{rgb}{0.121569,0.466667,0.705882}%
\pgfsetfillcolor{currentfill}%
\pgfsetlinewidth{0.000000pt}%
\definecolor{currentstroke}{rgb}{0.000000,0.000000,0.000000}%
\pgfsetstrokecolor{currentstroke}%
\pgfsetstrokeopacity{0.000000}%
\pgfsetdash{}{0pt}%
\pgfpathmoveto{\pgfqpoint{5.800000in}{73.220133in}}%
\pgfpathlineto{\pgfqpoint{5.800000in}{73.220133in}}%
\pgfpathlineto{\pgfqpoint{5.800000in}{73.322970in}}%
\pgfpathlineto{\pgfqpoint{5.800000in}{73.322970in}}%
\pgfpathclose%
\pgfusepath{fill}%
\end{pgfscope}%
\begin{pgfscope}%
\pgfpathrectangle{\pgfqpoint{5.800000in}{0.720000in}}{\pgfqpoint{1.400000in}{4.620000in}}%
\pgfusepath{clip}%
\pgfsetbuttcap%
\pgfsetmiterjoin%
\definecolor{currentfill}{rgb}{0.121569,0.466667,0.705882}%
\pgfsetfillcolor{currentfill}%
\pgfsetlinewidth{0.000000pt}%
\definecolor{currentstroke}{rgb}{0.000000,0.000000,0.000000}%
\pgfsetstrokecolor{currentstroke}%
\pgfsetstrokeopacity{0.000000}%
\pgfsetdash{}{0pt}%
\pgfpathmoveto{\pgfqpoint{5.800000in}{73.322970in}}%
\pgfpathlineto{\pgfqpoint{5.800000in}{73.322970in}}%
\pgfpathlineto{\pgfqpoint{5.800000in}{73.425807in}}%
\pgfpathlineto{\pgfqpoint{5.800000in}{73.425807in}}%
\pgfpathclose%
\pgfusepath{fill}%
\end{pgfscope}%
\begin{pgfscope}%
\pgfpathrectangle{\pgfqpoint{5.800000in}{0.720000in}}{\pgfqpoint{1.400000in}{4.620000in}}%
\pgfusepath{clip}%
\pgfsetbuttcap%
\pgfsetmiterjoin%
\definecolor{currentfill}{rgb}{0.121569,0.466667,0.705882}%
\pgfsetfillcolor{currentfill}%
\pgfsetlinewidth{0.000000pt}%
\definecolor{currentstroke}{rgb}{0.000000,0.000000,0.000000}%
\pgfsetstrokecolor{currentstroke}%
\pgfsetstrokeopacity{0.000000}%
\pgfsetdash{}{0pt}%
\pgfpathmoveto{\pgfqpoint{5.800000in}{73.425807in}}%
\pgfpathlineto{\pgfqpoint{5.800000in}{73.425807in}}%
\pgfpathlineto{\pgfqpoint{5.800000in}{73.528644in}}%
\pgfpathlineto{\pgfqpoint{5.800000in}{73.528644in}}%
\pgfpathclose%
\pgfusepath{fill}%
\end{pgfscope}%
\begin{pgfscope}%
\pgfpathrectangle{\pgfqpoint{5.800000in}{0.720000in}}{\pgfqpoint{1.400000in}{4.620000in}}%
\pgfusepath{clip}%
\pgfsetbuttcap%
\pgfsetmiterjoin%
\definecolor{currentfill}{rgb}{0.121569,0.466667,0.705882}%
\pgfsetfillcolor{currentfill}%
\pgfsetlinewidth{0.000000pt}%
\definecolor{currentstroke}{rgb}{0.000000,0.000000,0.000000}%
\pgfsetstrokecolor{currentstroke}%
\pgfsetstrokeopacity{0.000000}%
\pgfsetdash{}{0pt}%
\pgfpathmoveto{\pgfqpoint{5.800000in}{73.528644in}}%
\pgfpathlineto{\pgfqpoint{5.800000in}{73.528644in}}%
\pgfpathlineto{\pgfqpoint{5.800000in}{73.631481in}}%
\pgfpathlineto{\pgfqpoint{5.800000in}{73.631481in}}%
\pgfpathclose%
\pgfusepath{fill}%
\end{pgfscope}%
\begin{pgfscope}%
\pgfpathrectangle{\pgfqpoint{5.800000in}{0.720000in}}{\pgfqpoint{1.400000in}{4.620000in}}%
\pgfusepath{clip}%
\pgfsetbuttcap%
\pgfsetmiterjoin%
\definecolor{currentfill}{rgb}{0.121569,0.466667,0.705882}%
\pgfsetfillcolor{currentfill}%
\pgfsetlinewidth{0.000000pt}%
\definecolor{currentstroke}{rgb}{0.000000,0.000000,0.000000}%
\pgfsetstrokecolor{currentstroke}%
\pgfsetstrokeopacity{0.000000}%
\pgfsetdash{}{0pt}%
\pgfpathmoveto{\pgfqpoint{5.800000in}{73.631481in}}%
\pgfpathlineto{\pgfqpoint{5.802038in}{73.631481in}}%
\pgfpathlineto{\pgfqpoint{5.802038in}{73.734318in}}%
\pgfpathlineto{\pgfqpoint{5.800000in}{73.734318in}}%
\pgfpathclose%
\pgfusepath{fill}%
\end{pgfscope}%
\begin{pgfscope}%
\pgfpathrectangle{\pgfqpoint{5.800000in}{0.720000in}}{\pgfqpoint{1.400000in}{4.620000in}}%
\pgfusepath{clip}%
\pgfsetbuttcap%
\pgfsetmiterjoin%
\definecolor{currentfill}{rgb}{0.121569,0.466667,0.705882}%
\pgfsetfillcolor{currentfill}%
\pgfsetlinewidth{0.000000pt}%
\definecolor{currentstroke}{rgb}{0.000000,0.000000,0.000000}%
\pgfsetstrokecolor{currentstroke}%
\pgfsetstrokeopacity{0.000000}%
\pgfsetdash{}{0pt}%
\pgfpathmoveto{\pgfqpoint{5.800000in}{73.734318in}}%
\pgfpathlineto{\pgfqpoint{5.800000in}{73.734318in}}%
\pgfpathlineto{\pgfqpoint{5.800000in}{73.837155in}}%
\pgfpathlineto{\pgfqpoint{5.800000in}{73.837155in}}%
\pgfpathclose%
\pgfusepath{fill}%
\end{pgfscope}%
\begin{pgfscope}%
\pgfpathrectangle{\pgfqpoint{5.800000in}{0.720000in}}{\pgfqpoint{1.400000in}{4.620000in}}%
\pgfusepath{clip}%
\pgfsetbuttcap%
\pgfsetmiterjoin%
\definecolor{currentfill}{rgb}{0.121569,0.466667,0.705882}%
\pgfsetfillcolor{currentfill}%
\pgfsetlinewidth{0.000000pt}%
\definecolor{currentstroke}{rgb}{0.000000,0.000000,0.000000}%
\pgfsetstrokecolor{currentstroke}%
\pgfsetstrokeopacity{0.000000}%
\pgfsetdash{}{0pt}%
\pgfpathmoveto{\pgfqpoint{5.800000in}{73.837155in}}%
\pgfpathlineto{\pgfqpoint{5.800000in}{73.837155in}}%
\pgfpathlineto{\pgfqpoint{5.800000in}{73.939992in}}%
\pgfpathlineto{\pgfqpoint{5.800000in}{73.939992in}}%
\pgfpathclose%
\pgfusepath{fill}%
\end{pgfscope}%
\begin{pgfscope}%
\pgfpathrectangle{\pgfqpoint{5.800000in}{0.720000in}}{\pgfqpoint{1.400000in}{4.620000in}}%
\pgfusepath{clip}%
\pgfsetbuttcap%
\pgfsetmiterjoin%
\definecolor{currentfill}{rgb}{0.121569,0.466667,0.705882}%
\pgfsetfillcolor{currentfill}%
\pgfsetlinewidth{0.000000pt}%
\definecolor{currentstroke}{rgb}{0.000000,0.000000,0.000000}%
\pgfsetstrokecolor{currentstroke}%
\pgfsetstrokeopacity{0.000000}%
\pgfsetdash{}{0pt}%
\pgfpathmoveto{\pgfqpoint{5.800000in}{73.939992in}}%
\pgfpathlineto{\pgfqpoint{5.800000in}{73.939992in}}%
\pgfpathlineto{\pgfqpoint{5.800000in}{74.042829in}}%
\pgfpathlineto{\pgfqpoint{5.800000in}{74.042829in}}%
\pgfpathclose%
\pgfusepath{fill}%
\end{pgfscope}%
\begin{pgfscope}%
\pgfpathrectangle{\pgfqpoint{5.800000in}{0.720000in}}{\pgfqpoint{1.400000in}{4.620000in}}%
\pgfusepath{clip}%
\pgfsetbuttcap%
\pgfsetmiterjoin%
\definecolor{currentfill}{rgb}{0.121569,0.466667,0.705882}%
\pgfsetfillcolor{currentfill}%
\pgfsetlinewidth{0.000000pt}%
\definecolor{currentstroke}{rgb}{0.000000,0.000000,0.000000}%
\pgfsetstrokecolor{currentstroke}%
\pgfsetstrokeopacity{0.000000}%
\pgfsetdash{}{0pt}%
\pgfpathmoveto{\pgfqpoint{5.800000in}{74.042829in}}%
\pgfpathlineto{\pgfqpoint{5.800000in}{74.042829in}}%
\pgfpathlineto{\pgfqpoint{5.800000in}{74.145666in}}%
\pgfpathlineto{\pgfqpoint{5.800000in}{74.145666in}}%
\pgfpathclose%
\pgfusepath{fill}%
\end{pgfscope}%
\begin{pgfscope}%
\pgfpathrectangle{\pgfqpoint{5.800000in}{0.720000in}}{\pgfqpoint{1.400000in}{4.620000in}}%
\pgfusepath{clip}%
\pgfsetbuttcap%
\pgfsetmiterjoin%
\definecolor{currentfill}{rgb}{0.121569,0.466667,0.705882}%
\pgfsetfillcolor{currentfill}%
\pgfsetlinewidth{0.000000pt}%
\definecolor{currentstroke}{rgb}{0.000000,0.000000,0.000000}%
\pgfsetstrokecolor{currentstroke}%
\pgfsetstrokeopacity{0.000000}%
\pgfsetdash{}{0pt}%
\pgfpathmoveto{\pgfqpoint{5.800000in}{74.145666in}}%
\pgfpathlineto{\pgfqpoint{5.800000in}{74.145666in}}%
\pgfpathlineto{\pgfqpoint{5.800000in}{74.248503in}}%
\pgfpathlineto{\pgfqpoint{5.800000in}{74.248503in}}%
\pgfpathclose%
\pgfusepath{fill}%
\end{pgfscope}%
\begin{pgfscope}%
\pgfpathrectangle{\pgfqpoint{5.800000in}{0.720000in}}{\pgfqpoint{1.400000in}{4.620000in}}%
\pgfusepath{clip}%
\pgfsetbuttcap%
\pgfsetmiterjoin%
\definecolor{currentfill}{rgb}{0.121569,0.466667,0.705882}%
\pgfsetfillcolor{currentfill}%
\pgfsetlinewidth{0.000000pt}%
\definecolor{currentstroke}{rgb}{0.000000,0.000000,0.000000}%
\pgfsetstrokecolor{currentstroke}%
\pgfsetstrokeopacity{0.000000}%
\pgfsetdash{}{0pt}%
\pgfpathmoveto{\pgfqpoint{5.800000in}{74.248503in}}%
\pgfpathlineto{\pgfqpoint{5.800000in}{74.248503in}}%
\pgfpathlineto{\pgfqpoint{5.800000in}{74.351341in}}%
\pgfpathlineto{\pgfqpoint{5.800000in}{74.351341in}}%
\pgfpathclose%
\pgfusepath{fill}%
\end{pgfscope}%
\begin{pgfscope}%
\pgfpathrectangle{\pgfqpoint{5.800000in}{0.720000in}}{\pgfqpoint{1.400000in}{4.620000in}}%
\pgfusepath{clip}%
\pgfsetbuttcap%
\pgfsetmiterjoin%
\definecolor{currentfill}{rgb}{0.121569,0.466667,0.705882}%
\pgfsetfillcolor{currentfill}%
\pgfsetlinewidth{0.000000pt}%
\definecolor{currentstroke}{rgb}{0.000000,0.000000,0.000000}%
\pgfsetstrokecolor{currentstroke}%
\pgfsetstrokeopacity{0.000000}%
\pgfsetdash{}{0pt}%
\pgfpathmoveto{\pgfqpoint{5.800000in}{74.351341in}}%
\pgfpathlineto{\pgfqpoint{5.800000in}{74.351341in}}%
\pgfpathlineto{\pgfqpoint{5.800000in}{74.454178in}}%
\pgfpathlineto{\pgfqpoint{5.800000in}{74.454178in}}%
\pgfpathclose%
\pgfusepath{fill}%
\end{pgfscope}%
\begin{pgfscope}%
\pgfpathrectangle{\pgfqpoint{5.800000in}{0.720000in}}{\pgfqpoint{1.400000in}{4.620000in}}%
\pgfusepath{clip}%
\pgfsetbuttcap%
\pgfsetmiterjoin%
\definecolor{currentfill}{rgb}{0.121569,0.466667,0.705882}%
\pgfsetfillcolor{currentfill}%
\pgfsetlinewidth{0.000000pt}%
\definecolor{currentstroke}{rgb}{0.000000,0.000000,0.000000}%
\pgfsetstrokecolor{currentstroke}%
\pgfsetstrokeopacity{0.000000}%
\pgfsetdash{}{0pt}%
\pgfpathmoveto{\pgfqpoint{5.800000in}{74.454178in}}%
\pgfpathlineto{\pgfqpoint{5.800000in}{74.454178in}}%
\pgfpathlineto{\pgfqpoint{5.800000in}{74.557015in}}%
\pgfpathlineto{\pgfqpoint{5.800000in}{74.557015in}}%
\pgfpathclose%
\pgfusepath{fill}%
\end{pgfscope}%
\begin{pgfscope}%
\pgfpathrectangle{\pgfqpoint{5.800000in}{0.720000in}}{\pgfqpoint{1.400000in}{4.620000in}}%
\pgfusepath{clip}%
\pgfsetbuttcap%
\pgfsetmiterjoin%
\definecolor{currentfill}{rgb}{0.121569,0.466667,0.705882}%
\pgfsetfillcolor{currentfill}%
\pgfsetlinewidth{0.000000pt}%
\definecolor{currentstroke}{rgb}{0.000000,0.000000,0.000000}%
\pgfsetstrokecolor{currentstroke}%
\pgfsetstrokeopacity{0.000000}%
\pgfsetdash{}{0pt}%
\pgfpathmoveto{\pgfqpoint{5.800000in}{74.557015in}}%
\pgfpathlineto{\pgfqpoint{5.800000in}{74.557015in}}%
\pgfpathlineto{\pgfqpoint{5.800000in}{74.659852in}}%
\pgfpathlineto{\pgfqpoint{5.800000in}{74.659852in}}%
\pgfpathclose%
\pgfusepath{fill}%
\end{pgfscope}%
\begin{pgfscope}%
\pgfpathrectangle{\pgfqpoint{5.800000in}{0.720000in}}{\pgfqpoint{1.400000in}{4.620000in}}%
\pgfusepath{clip}%
\pgfsetbuttcap%
\pgfsetmiterjoin%
\definecolor{currentfill}{rgb}{0.121569,0.466667,0.705882}%
\pgfsetfillcolor{currentfill}%
\pgfsetlinewidth{0.000000pt}%
\definecolor{currentstroke}{rgb}{0.000000,0.000000,0.000000}%
\pgfsetstrokecolor{currentstroke}%
\pgfsetstrokeopacity{0.000000}%
\pgfsetdash{}{0pt}%
\pgfpathmoveto{\pgfqpoint{5.800000in}{74.659852in}}%
\pgfpathlineto{\pgfqpoint{5.800000in}{74.659852in}}%
\pgfpathlineto{\pgfqpoint{5.800000in}{74.762689in}}%
\pgfpathlineto{\pgfqpoint{5.800000in}{74.762689in}}%
\pgfpathclose%
\pgfusepath{fill}%
\end{pgfscope}%
\begin{pgfscope}%
\pgfpathrectangle{\pgfqpoint{5.800000in}{0.720000in}}{\pgfqpoint{1.400000in}{4.620000in}}%
\pgfusepath{clip}%
\pgfsetbuttcap%
\pgfsetmiterjoin%
\definecolor{currentfill}{rgb}{0.121569,0.466667,0.705882}%
\pgfsetfillcolor{currentfill}%
\pgfsetlinewidth{0.000000pt}%
\definecolor{currentstroke}{rgb}{0.000000,0.000000,0.000000}%
\pgfsetstrokecolor{currentstroke}%
\pgfsetstrokeopacity{0.000000}%
\pgfsetdash{}{0pt}%
\pgfpathmoveto{\pgfqpoint{5.800000in}{74.762689in}}%
\pgfpathlineto{\pgfqpoint{5.800000in}{74.762689in}}%
\pgfpathlineto{\pgfqpoint{5.800000in}{74.865526in}}%
\pgfpathlineto{\pgfqpoint{5.800000in}{74.865526in}}%
\pgfpathclose%
\pgfusepath{fill}%
\end{pgfscope}%
\begin{pgfscope}%
\pgfpathrectangle{\pgfqpoint{5.800000in}{0.720000in}}{\pgfqpoint{1.400000in}{4.620000in}}%
\pgfusepath{clip}%
\pgfsetbuttcap%
\pgfsetmiterjoin%
\definecolor{currentfill}{rgb}{0.121569,0.466667,0.705882}%
\pgfsetfillcolor{currentfill}%
\pgfsetlinewidth{0.000000pt}%
\definecolor{currentstroke}{rgb}{0.000000,0.000000,0.000000}%
\pgfsetstrokecolor{currentstroke}%
\pgfsetstrokeopacity{0.000000}%
\pgfsetdash{}{0pt}%
\pgfpathmoveto{\pgfqpoint{5.800000in}{74.865526in}}%
\pgfpathlineto{\pgfqpoint{5.800000in}{74.865526in}}%
\pgfpathlineto{\pgfqpoint{5.800000in}{74.968363in}}%
\pgfpathlineto{\pgfqpoint{5.800000in}{74.968363in}}%
\pgfpathclose%
\pgfusepath{fill}%
\end{pgfscope}%
\begin{pgfscope}%
\pgfpathrectangle{\pgfqpoint{5.800000in}{0.720000in}}{\pgfqpoint{1.400000in}{4.620000in}}%
\pgfusepath{clip}%
\pgfsetbuttcap%
\pgfsetmiterjoin%
\definecolor{currentfill}{rgb}{0.121569,0.466667,0.705882}%
\pgfsetfillcolor{currentfill}%
\pgfsetlinewidth{0.000000pt}%
\definecolor{currentstroke}{rgb}{0.000000,0.000000,0.000000}%
\pgfsetstrokecolor{currentstroke}%
\pgfsetstrokeopacity{0.000000}%
\pgfsetdash{}{0pt}%
\pgfpathmoveto{\pgfqpoint{5.800000in}{74.968363in}}%
\pgfpathlineto{\pgfqpoint{5.800000in}{74.968363in}}%
\pgfpathlineto{\pgfqpoint{5.800000in}{75.071200in}}%
\pgfpathlineto{\pgfqpoint{5.800000in}{75.071200in}}%
\pgfpathclose%
\pgfusepath{fill}%
\end{pgfscope}%
\begin{pgfscope}%
\pgfpathrectangle{\pgfqpoint{5.800000in}{0.720000in}}{\pgfqpoint{1.400000in}{4.620000in}}%
\pgfusepath{clip}%
\pgfsetbuttcap%
\pgfsetmiterjoin%
\definecolor{currentfill}{rgb}{0.121569,0.466667,0.705882}%
\pgfsetfillcolor{currentfill}%
\pgfsetlinewidth{0.000000pt}%
\definecolor{currentstroke}{rgb}{0.000000,0.000000,0.000000}%
\pgfsetstrokecolor{currentstroke}%
\pgfsetstrokeopacity{0.000000}%
\pgfsetdash{}{0pt}%
\pgfpathmoveto{\pgfqpoint{5.800000in}{75.071200in}}%
\pgfpathlineto{\pgfqpoint{5.800000in}{75.071200in}}%
\pgfpathlineto{\pgfqpoint{5.800000in}{75.174037in}}%
\pgfpathlineto{\pgfqpoint{5.800000in}{75.174037in}}%
\pgfpathclose%
\pgfusepath{fill}%
\end{pgfscope}%
\begin{pgfscope}%
\pgfpathrectangle{\pgfqpoint{5.800000in}{0.720000in}}{\pgfqpoint{1.400000in}{4.620000in}}%
\pgfusepath{clip}%
\pgfsetbuttcap%
\pgfsetmiterjoin%
\definecolor{currentfill}{rgb}{0.121569,0.466667,0.705882}%
\pgfsetfillcolor{currentfill}%
\pgfsetlinewidth{0.000000pt}%
\definecolor{currentstroke}{rgb}{0.000000,0.000000,0.000000}%
\pgfsetstrokecolor{currentstroke}%
\pgfsetstrokeopacity{0.000000}%
\pgfsetdash{}{0pt}%
\pgfpathmoveto{\pgfqpoint{5.800000in}{75.174037in}}%
\pgfpathlineto{\pgfqpoint{5.800000in}{75.174037in}}%
\pgfpathlineto{\pgfqpoint{5.800000in}{75.276874in}}%
\pgfpathlineto{\pgfqpoint{5.800000in}{75.276874in}}%
\pgfpathclose%
\pgfusepath{fill}%
\end{pgfscope}%
\begin{pgfscope}%
\pgfpathrectangle{\pgfqpoint{5.800000in}{0.720000in}}{\pgfqpoint{1.400000in}{4.620000in}}%
\pgfusepath{clip}%
\pgfsetbuttcap%
\pgfsetmiterjoin%
\definecolor{currentfill}{rgb}{0.121569,0.466667,0.705882}%
\pgfsetfillcolor{currentfill}%
\pgfsetlinewidth{0.000000pt}%
\definecolor{currentstroke}{rgb}{0.000000,0.000000,0.000000}%
\pgfsetstrokecolor{currentstroke}%
\pgfsetstrokeopacity{0.000000}%
\pgfsetdash{}{0pt}%
\pgfpathmoveto{\pgfqpoint{5.800000in}{75.276874in}}%
\pgfpathlineto{\pgfqpoint{5.800000in}{75.276874in}}%
\pgfpathlineto{\pgfqpoint{5.800000in}{75.379711in}}%
\pgfpathlineto{\pgfqpoint{5.800000in}{75.379711in}}%
\pgfpathclose%
\pgfusepath{fill}%
\end{pgfscope}%
\begin{pgfscope}%
\pgfpathrectangle{\pgfqpoint{5.800000in}{0.720000in}}{\pgfqpoint{1.400000in}{4.620000in}}%
\pgfusepath{clip}%
\pgfsetbuttcap%
\pgfsetmiterjoin%
\definecolor{currentfill}{rgb}{0.121569,0.466667,0.705882}%
\pgfsetfillcolor{currentfill}%
\pgfsetlinewidth{0.000000pt}%
\definecolor{currentstroke}{rgb}{0.000000,0.000000,0.000000}%
\pgfsetstrokecolor{currentstroke}%
\pgfsetstrokeopacity{0.000000}%
\pgfsetdash{}{0pt}%
\pgfpathmoveto{\pgfqpoint{5.800000in}{75.379711in}}%
\pgfpathlineto{\pgfqpoint{5.800000in}{75.379711in}}%
\pgfpathlineto{\pgfqpoint{5.800000in}{75.482548in}}%
\pgfpathlineto{\pgfqpoint{5.800000in}{75.482548in}}%
\pgfpathclose%
\pgfusepath{fill}%
\end{pgfscope}%
\begin{pgfscope}%
\pgfpathrectangle{\pgfqpoint{5.800000in}{0.720000in}}{\pgfqpoint{1.400000in}{4.620000in}}%
\pgfusepath{clip}%
\pgfsetbuttcap%
\pgfsetmiterjoin%
\definecolor{currentfill}{rgb}{0.121569,0.466667,0.705882}%
\pgfsetfillcolor{currentfill}%
\pgfsetlinewidth{0.000000pt}%
\definecolor{currentstroke}{rgb}{0.000000,0.000000,0.000000}%
\pgfsetstrokecolor{currentstroke}%
\pgfsetstrokeopacity{0.000000}%
\pgfsetdash{}{0pt}%
\pgfpathmoveto{\pgfqpoint{5.800000in}{75.482548in}}%
\pgfpathlineto{\pgfqpoint{5.800000in}{75.482548in}}%
\pgfpathlineto{\pgfqpoint{5.800000in}{75.585385in}}%
\pgfpathlineto{\pgfqpoint{5.800000in}{75.585385in}}%
\pgfpathclose%
\pgfusepath{fill}%
\end{pgfscope}%
\begin{pgfscope}%
\pgfpathrectangle{\pgfqpoint{5.800000in}{0.720000in}}{\pgfqpoint{1.400000in}{4.620000in}}%
\pgfusepath{clip}%
\pgfsetbuttcap%
\pgfsetmiterjoin%
\definecolor{currentfill}{rgb}{0.121569,0.466667,0.705882}%
\pgfsetfillcolor{currentfill}%
\pgfsetlinewidth{0.000000pt}%
\definecolor{currentstroke}{rgb}{0.000000,0.000000,0.000000}%
\pgfsetstrokecolor{currentstroke}%
\pgfsetstrokeopacity{0.000000}%
\pgfsetdash{}{0pt}%
\pgfpathmoveto{\pgfqpoint{5.800000in}{75.585385in}}%
\pgfpathlineto{\pgfqpoint{5.800000in}{75.585385in}}%
\pgfpathlineto{\pgfqpoint{5.800000in}{75.688222in}}%
\pgfpathlineto{\pgfqpoint{5.800000in}{75.688222in}}%
\pgfpathclose%
\pgfusepath{fill}%
\end{pgfscope}%
\begin{pgfscope}%
\pgfpathrectangle{\pgfqpoint{5.800000in}{0.720000in}}{\pgfqpoint{1.400000in}{4.620000in}}%
\pgfusepath{clip}%
\pgfsetbuttcap%
\pgfsetmiterjoin%
\definecolor{currentfill}{rgb}{0.121569,0.466667,0.705882}%
\pgfsetfillcolor{currentfill}%
\pgfsetlinewidth{0.000000pt}%
\definecolor{currentstroke}{rgb}{0.000000,0.000000,0.000000}%
\pgfsetstrokecolor{currentstroke}%
\pgfsetstrokeopacity{0.000000}%
\pgfsetdash{}{0pt}%
\pgfpathmoveto{\pgfqpoint{5.800000in}{75.688222in}}%
\pgfpathlineto{\pgfqpoint{5.800000in}{75.688222in}}%
\pgfpathlineto{\pgfqpoint{5.800000in}{75.791059in}}%
\pgfpathlineto{\pgfqpoint{5.800000in}{75.791059in}}%
\pgfpathclose%
\pgfusepath{fill}%
\end{pgfscope}%
\begin{pgfscope}%
\pgfpathrectangle{\pgfqpoint{5.800000in}{0.720000in}}{\pgfqpoint{1.400000in}{4.620000in}}%
\pgfusepath{clip}%
\pgfsetbuttcap%
\pgfsetmiterjoin%
\definecolor{currentfill}{rgb}{0.121569,0.466667,0.705882}%
\pgfsetfillcolor{currentfill}%
\pgfsetlinewidth{0.000000pt}%
\definecolor{currentstroke}{rgb}{0.000000,0.000000,0.000000}%
\pgfsetstrokecolor{currentstroke}%
\pgfsetstrokeopacity{0.000000}%
\pgfsetdash{}{0pt}%
\pgfpathmoveto{\pgfqpoint{5.800000in}{75.791059in}}%
\pgfpathlineto{\pgfqpoint{5.800000in}{75.791059in}}%
\pgfpathlineto{\pgfqpoint{5.800000in}{75.893897in}}%
\pgfpathlineto{\pgfqpoint{5.800000in}{75.893897in}}%
\pgfpathclose%
\pgfusepath{fill}%
\end{pgfscope}%
\begin{pgfscope}%
\pgfpathrectangle{\pgfqpoint{5.800000in}{0.720000in}}{\pgfqpoint{1.400000in}{4.620000in}}%
\pgfusepath{clip}%
\pgfsetbuttcap%
\pgfsetmiterjoin%
\definecolor{currentfill}{rgb}{0.121569,0.466667,0.705882}%
\pgfsetfillcolor{currentfill}%
\pgfsetlinewidth{0.000000pt}%
\definecolor{currentstroke}{rgb}{0.000000,0.000000,0.000000}%
\pgfsetstrokecolor{currentstroke}%
\pgfsetstrokeopacity{0.000000}%
\pgfsetdash{}{0pt}%
\pgfpathmoveto{\pgfqpoint{5.800000in}{75.893897in}}%
\pgfpathlineto{\pgfqpoint{5.800000in}{75.893897in}}%
\pgfpathlineto{\pgfqpoint{5.800000in}{75.996734in}}%
\pgfpathlineto{\pgfqpoint{5.800000in}{75.996734in}}%
\pgfpathclose%
\pgfusepath{fill}%
\end{pgfscope}%
\begin{pgfscope}%
\pgfpathrectangle{\pgfqpoint{5.800000in}{0.720000in}}{\pgfqpoint{1.400000in}{4.620000in}}%
\pgfusepath{clip}%
\pgfsetbuttcap%
\pgfsetmiterjoin%
\definecolor{currentfill}{rgb}{0.121569,0.466667,0.705882}%
\pgfsetfillcolor{currentfill}%
\pgfsetlinewidth{0.000000pt}%
\definecolor{currentstroke}{rgb}{0.000000,0.000000,0.000000}%
\pgfsetstrokecolor{currentstroke}%
\pgfsetstrokeopacity{0.000000}%
\pgfsetdash{}{0pt}%
\pgfpathmoveto{\pgfqpoint{5.800000in}{75.996734in}}%
\pgfpathlineto{\pgfqpoint{5.800000in}{75.996734in}}%
\pgfpathlineto{\pgfqpoint{5.800000in}{76.099571in}}%
\pgfpathlineto{\pgfqpoint{5.800000in}{76.099571in}}%
\pgfpathclose%
\pgfusepath{fill}%
\end{pgfscope}%
\begin{pgfscope}%
\pgfpathrectangle{\pgfqpoint{5.800000in}{0.720000in}}{\pgfqpoint{1.400000in}{4.620000in}}%
\pgfusepath{clip}%
\pgfsetbuttcap%
\pgfsetmiterjoin%
\definecolor{currentfill}{rgb}{0.121569,0.466667,0.705882}%
\pgfsetfillcolor{currentfill}%
\pgfsetlinewidth{0.000000pt}%
\definecolor{currentstroke}{rgb}{0.000000,0.000000,0.000000}%
\pgfsetstrokecolor{currentstroke}%
\pgfsetstrokeopacity{0.000000}%
\pgfsetdash{}{0pt}%
\pgfpathmoveto{\pgfqpoint{5.800000in}{76.099571in}}%
\pgfpathlineto{\pgfqpoint{5.800000in}{76.099571in}}%
\pgfpathlineto{\pgfqpoint{5.800000in}{76.202408in}}%
\pgfpathlineto{\pgfqpoint{5.800000in}{76.202408in}}%
\pgfpathclose%
\pgfusepath{fill}%
\end{pgfscope}%
\begin{pgfscope}%
\pgfpathrectangle{\pgfqpoint{5.800000in}{0.720000in}}{\pgfqpoint{1.400000in}{4.620000in}}%
\pgfusepath{clip}%
\pgfsetbuttcap%
\pgfsetmiterjoin%
\definecolor{currentfill}{rgb}{0.121569,0.466667,0.705882}%
\pgfsetfillcolor{currentfill}%
\pgfsetlinewidth{0.000000pt}%
\definecolor{currentstroke}{rgb}{0.000000,0.000000,0.000000}%
\pgfsetstrokecolor{currentstroke}%
\pgfsetstrokeopacity{0.000000}%
\pgfsetdash{}{0pt}%
\pgfpathmoveto{\pgfqpoint{5.800000in}{76.202408in}}%
\pgfpathlineto{\pgfqpoint{5.800000in}{76.202408in}}%
\pgfpathlineto{\pgfqpoint{5.800000in}{76.305245in}}%
\pgfpathlineto{\pgfqpoint{5.800000in}{76.305245in}}%
\pgfpathclose%
\pgfusepath{fill}%
\end{pgfscope}%
\begin{pgfscope}%
\pgfpathrectangle{\pgfqpoint{5.800000in}{0.720000in}}{\pgfqpoint{1.400000in}{4.620000in}}%
\pgfusepath{clip}%
\pgfsetbuttcap%
\pgfsetmiterjoin%
\definecolor{currentfill}{rgb}{0.121569,0.466667,0.705882}%
\pgfsetfillcolor{currentfill}%
\pgfsetlinewidth{0.000000pt}%
\definecolor{currentstroke}{rgb}{0.000000,0.000000,0.000000}%
\pgfsetstrokecolor{currentstroke}%
\pgfsetstrokeopacity{0.000000}%
\pgfsetdash{}{0pt}%
\pgfpathmoveto{\pgfqpoint{5.800000in}{76.305245in}}%
\pgfpathlineto{\pgfqpoint{5.800000in}{76.305245in}}%
\pgfpathlineto{\pgfqpoint{5.800000in}{76.408082in}}%
\pgfpathlineto{\pgfqpoint{5.800000in}{76.408082in}}%
\pgfpathclose%
\pgfusepath{fill}%
\end{pgfscope}%
\begin{pgfscope}%
\pgfpathrectangle{\pgfqpoint{5.800000in}{0.720000in}}{\pgfqpoint{1.400000in}{4.620000in}}%
\pgfusepath{clip}%
\pgfsetbuttcap%
\pgfsetmiterjoin%
\definecolor{currentfill}{rgb}{0.121569,0.466667,0.705882}%
\pgfsetfillcolor{currentfill}%
\pgfsetlinewidth{0.000000pt}%
\definecolor{currentstroke}{rgb}{0.000000,0.000000,0.000000}%
\pgfsetstrokecolor{currentstroke}%
\pgfsetstrokeopacity{0.000000}%
\pgfsetdash{}{0pt}%
\pgfpathmoveto{\pgfqpoint{5.800000in}{76.408082in}}%
\pgfpathlineto{\pgfqpoint{5.800000in}{76.408082in}}%
\pgfpathlineto{\pgfqpoint{5.800000in}{76.510919in}}%
\pgfpathlineto{\pgfqpoint{5.800000in}{76.510919in}}%
\pgfpathclose%
\pgfusepath{fill}%
\end{pgfscope}%
\begin{pgfscope}%
\pgfpathrectangle{\pgfqpoint{5.800000in}{0.720000in}}{\pgfqpoint{1.400000in}{4.620000in}}%
\pgfusepath{clip}%
\pgfsetbuttcap%
\pgfsetmiterjoin%
\definecolor{currentfill}{rgb}{0.121569,0.466667,0.705882}%
\pgfsetfillcolor{currentfill}%
\pgfsetlinewidth{0.000000pt}%
\definecolor{currentstroke}{rgb}{0.000000,0.000000,0.000000}%
\pgfsetstrokecolor{currentstroke}%
\pgfsetstrokeopacity{0.000000}%
\pgfsetdash{}{0pt}%
\pgfpathmoveto{\pgfqpoint{5.800000in}{76.510919in}}%
\pgfpathlineto{\pgfqpoint{5.800000in}{76.510919in}}%
\pgfpathlineto{\pgfqpoint{5.800000in}{76.613756in}}%
\pgfpathlineto{\pgfqpoint{5.800000in}{76.613756in}}%
\pgfpathclose%
\pgfusepath{fill}%
\end{pgfscope}%
\begin{pgfscope}%
\pgfpathrectangle{\pgfqpoint{5.800000in}{0.720000in}}{\pgfqpoint{1.400000in}{4.620000in}}%
\pgfusepath{clip}%
\pgfsetbuttcap%
\pgfsetmiterjoin%
\definecolor{currentfill}{rgb}{0.121569,0.466667,0.705882}%
\pgfsetfillcolor{currentfill}%
\pgfsetlinewidth{0.000000pt}%
\definecolor{currentstroke}{rgb}{0.000000,0.000000,0.000000}%
\pgfsetstrokecolor{currentstroke}%
\pgfsetstrokeopacity{0.000000}%
\pgfsetdash{}{0pt}%
\pgfpathmoveto{\pgfqpoint{5.800000in}{76.613756in}}%
\pgfpathlineto{\pgfqpoint{5.800000in}{76.613756in}}%
\pgfpathlineto{\pgfqpoint{5.800000in}{76.716593in}}%
\pgfpathlineto{\pgfqpoint{5.800000in}{76.716593in}}%
\pgfpathclose%
\pgfusepath{fill}%
\end{pgfscope}%
\begin{pgfscope}%
\pgfpathrectangle{\pgfqpoint{5.800000in}{0.720000in}}{\pgfqpoint{1.400000in}{4.620000in}}%
\pgfusepath{clip}%
\pgfsetbuttcap%
\pgfsetmiterjoin%
\definecolor{currentfill}{rgb}{0.121569,0.466667,0.705882}%
\pgfsetfillcolor{currentfill}%
\pgfsetlinewidth{0.000000pt}%
\definecolor{currentstroke}{rgb}{0.000000,0.000000,0.000000}%
\pgfsetstrokecolor{currentstroke}%
\pgfsetstrokeopacity{0.000000}%
\pgfsetdash{}{0pt}%
\pgfpathmoveto{\pgfqpoint{5.800000in}{76.716593in}}%
\pgfpathlineto{\pgfqpoint{5.800000in}{76.716593in}}%
\pgfpathlineto{\pgfqpoint{5.800000in}{76.819430in}}%
\pgfpathlineto{\pgfqpoint{5.800000in}{76.819430in}}%
\pgfpathclose%
\pgfusepath{fill}%
\end{pgfscope}%
\begin{pgfscope}%
\pgfpathrectangle{\pgfqpoint{5.800000in}{0.720000in}}{\pgfqpoint{1.400000in}{4.620000in}}%
\pgfusepath{clip}%
\pgfsetbuttcap%
\pgfsetmiterjoin%
\definecolor{currentfill}{rgb}{0.121569,0.466667,0.705882}%
\pgfsetfillcolor{currentfill}%
\pgfsetlinewidth{0.000000pt}%
\definecolor{currentstroke}{rgb}{0.000000,0.000000,0.000000}%
\pgfsetstrokecolor{currentstroke}%
\pgfsetstrokeopacity{0.000000}%
\pgfsetdash{}{0pt}%
\pgfpathmoveto{\pgfqpoint{5.800000in}{76.819430in}}%
\pgfpathlineto{\pgfqpoint{5.800000in}{76.819430in}}%
\pgfpathlineto{\pgfqpoint{5.800000in}{76.922267in}}%
\pgfpathlineto{\pgfqpoint{5.800000in}{76.922267in}}%
\pgfpathclose%
\pgfusepath{fill}%
\end{pgfscope}%
\begin{pgfscope}%
\pgfpathrectangle{\pgfqpoint{5.800000in}{0.720000in}}{\pgfqpoint{1.400000in}{4.620000in}}%
\pgfusepath{clip}%
\pgfsetbuttcap%
\pgfsetmiterjoin%
\definecolor{currentfill}{rgb}{0.121569,0.466667,0.705882}%
\pgfsetfillcolor{currentfill}%
\pgfsetlinewidth{0.000000pt}%
\definecolor{currentstroke}{rgb}{0.000000,0.000000,0.000000}%
\pgfsetstrokecolor{currentstroke}%
\pgfsetstrokeopacity{0.000000}%
\pgfsetdash{}{0pt}%
\pgfpathmoveto{\pgfqpoint{5.800000in}{76.922267in}}%
\pgfpathlineto{\pgfqpoint{5.800000in}{76.922267in}}%
\pgfpathlineto{\pgfqpoint{5.800000in}{77.025104in}}%
\pgfpathlineto{\pgfqpoint{5.800000in}{77.025104in}}%
\pgfpathclose%
\pgfusepath{fill}%
\end{pgfscope}%
\begin{pgfscope}%
\pgfpathrectangle{\pgfqpoint{5.800000in}{0.720000in}}{\pgfqpoint{1.400000in}{4.620000in}}%
\pgfusepath{clip}%
\pgfsetbuttcap%
\pgfsetmiterjoin%
\definecolor{currentfill}{rgb}{0.121569,0.466667,0.705882}%
\pgfsetfillcolor{currentfill}%
\pgfsetlinewidth{0.000000pt}%
\definecolor{currentstroke}{rgb}{0.000000,0.000000,0.000000}%
\pgfsetstrokecolor{currentstroke}%
\pgfsetstrokeopacity{0.000000}%
\pgfsetdash{}{0pt}%
\pgfpathmoveto{\pgfqpoint{5.800000in}{77.025104in}}%
\pgfpathlineto{\pgfqpoint{5.800000in}{77.025104in}}%
\pgfpathlineto{\pgfqpoint{5.800000in}{77.127941in}}%
\pgfpathlineto{\pgfqpoint{5.800000in}{77.127941in}}%
\pgfpathclose%
\pgfusepath{fill}%
\end{pgfscope}%
\begin{pgfscope}%
\pgfpathrectangle{\pgfqpoint{5.800000in}{0.720000in}}{\pgfqpoint{1.400000in}{4.620000in}}%
\pgfusepath{clip}%
\pgfsetbuttcap%
\pgfsetmiterjoin%
\definecolor{currentfill}{rgb}{0.121569,0.466667,0.705882}%
\pgfsetfillcolor{currentfill}%
\pgfsetlinewidth{0.000000pt}%
\definecolor{currentstroke}{rgb}{0.000000,0.000000,0.000000}%
\pgfsetstrokecolor{currentstroke}%
\pgfsetstrokeopacity{0.000000}%
\pgfsetdash{}{0pt}%
\pgfpathmoveto{\pgfqpoint{5.800000in}{77.127941in}}%
\pgfpathlineto{\pgfqpoint{5.800000in}{77.127941in}}%
\pgfpathlineto{\pgfqpoint{5.800000in}{77.230778in}}%
\pgfpathlineto{\pgfqpoint{5.800000in}{77.230778in}}%
\pgfpathclose%
\pgfusepath{fill}%
\end{pgfscope}%
\begin{pgfscope}%
\pgfpathrectangle{\pgfqpoint{5.800000in}{0.720000in}}{\pgfqpoint{1.400000in}{4.620000in}}%
\pgfusepath{clip}%
\pgfsetbuttcap%
\pgfsetmiterjoin%
\definecolor{currentfill}{rgb}{0.121569,0.466667,0.705882}%
\pgfsetfillcolor{currentfill}%
\pgfsetlinewidth{0.000000pt}%
\definecolor{currentstroke}{rgb}{0.000000,0.000000,0.000000}%
\pgfsetstrokecolor{currentstroke}%
\pgfsetstrokeopacity{0.000000}%
\pgfsetdash{}{0pt}%
\pgfpathmoveto{\pgfqpoint{5.800000in}{77.230778in}}%
\pgfpathlineto{\pgfqpoint{5.800000in}{77.230778in}}%
\pgfpathlineto{\pgfqpoint{5.800000in}{77.333615in}}%
\pgfpathlineto{\pgfqpoint{5.800000in}{77.333615in}}%
\pgfpathclose%
\pgfusepath{fill}%
\end{pgfscope}%
\begin{pgfscope}%
\pgfpathrectangle{\pgfqpoint{5.800000in}{0.720000in}}{\pgfqpoint{1.400000in}{4.620000in}}%
\pgfusepath{clip}%
\pgfsetbuttcap%
\pgfsetmiterjoin%
\definecolor{currentfill}{rgb}{0.121569,0.466667,0.705882}%
\pgfsetfillcolor{currentfill}%
\pgfsetlinewidth{0.000000pt}%
\definecolor{currentstroke}{rgb}{0.000000,0.000000,0.000000}%
\pgfsetstrokecolor{currentstroke}%
\pgfsetstrokeopacity{0.000000}%
\pgfsetdash{}{0pt}%
\pgfpathmoveto{\pgfqpoint{5.800000in}{77.333615in}}%
\pgfpathlineto{\pgfqpoint{5.800000in}{77.333615in}}%
\pgfpathlineto{\pgfqpoint{5.800000in}{77.436453in}}%
\pgfpathlineto{\pgfqpoint{5.800000in}{77.436453in}}%
\pgfpathclose%
\pgfusepath{fill}%
\end{pgfscope}%
\begin{pgfscope}%
\pgfpathrectangle{\pgfqpoint{5.800000in}{0.720000in}}{\pgfqpoint{1.400000in}{4.620000in}}%
\pgfusepath{clip}%
\pgfsetbuttcap%
\pgfsetmiterjoin%
\definecolor{currentfill}{rgb}{0.121569,0.466667,0.705882}%
\pgfsetfillcolor{currentfill}%
\pgfsetlinewidth{0.000000pt}%
\definecolor{currentstroke}{rgb}{0.000000,0.000000,0.000000}%
\pgfsetstrokecolor{currentstroke}%
\pgfsetstrokeopacity{0.000000}%
\pgfsetdash{}{0pt}%
\pgfpathmoveto{\pgfqpoint{5.800000in}{77.436453in}}%
\pgfpathlineto{\pgfqpoint{5.800000in}{77.436453in}}%
\pgfpathlineto{\pgfqpoint{5.800000in}{77.539290in}}%
\pgfpathlineto{\pgfqpoint{5.800000in}{77.539290in}}%
\pgfpathclose%
\pgfusepath{fill}%
\end{pgfscope}%
\begin{pgfscope}%
\pgfpathrectangle{\pgfqpoint{5.800000in}{0.720000in}}{\pgfqpoint{1.400000in}{4.620000in}}%
\pgfusepath{clip}%
\pgfsetbuttcap%
\pgfsetmiterjoin%
\definecolor{currentfill}{rgb}{0.121569,0.466667,0.705882}%
\pgfsetfillcolor{currentfill}%
\pgfsetlinewidth{0.000000pt}%
\definecolor{currentstroke}{rgb}{0.000000,0.000000,0.000000}%
\pgfsetstrokecolor{currentstroke}%
\pgfsetstrokeopacity{0.000000}%
\pgfsetdash{}{0pt}%
\pgfpathmoveto{\pgfqpoint{5.800000in}{77.539290in}}%
\pgfpathlineto{\pgfqpoint{5.800000in}{77.539290in}}%
\pgfpathlineto{\pgfqpoint{5.800000in}{77.642127in}}%
\pgfpathlineto{\pgfqpoint{5.800000in}{77.642127in}}%
\pgfpathclose%
\pgfusepath{fill}%
\end{pgfscope}%
\begin{pgfscope}%
\pgfpathrectangle{\pgfqpoint{5.800000in}{0.720000in}}{\pgfqpoint{1.400000in}{4.620000in}}%
\pgfusepath{clip}%
\pgfsetbuttcap%
\pgfsetmiterjoin%
\definecolor{currentfill}{rgb}{0.121569,0.466667,0.705882}%
\pgfsetfillcolor{currentfill}%
\pgfsetlinewidth{0.000000pt}%
\definecolor{currentstroke}{rgb}{0.000000,0.000000,0.000000}%
\pgfsetstrokecolor{currentstroke}%
\pgfsetstrokeopacity{0.000000}%
\pgfsetdash{}{0pt}%
\pgfpathmoveto{\pgfqpoint{5.800000in}{77.642127in}}%
\pgfpathlineto{\pgfqpoint{5.800000in}{77.642127in}}%
\pgfpathlineto{\pgfqpoint{5.800000in}{77.744964in}}%
\pgfpathlineto{\pgfqpoint{5.800000in}{77.744964in}}%
\pgfpathclose%
\pgfusepath{fill}%
\end{pgfscope}%
\begin{pgfscope}%
\pgfpathrectangle{\pgfqpoint{5.800000in}{0.720000in}}{\pgfqpoint{1.400000in}{4.620000in}}%
\pgfusepath{clip}%
\pgfsetbuttcap%
\pgfsetmiterjoin%
\definecolor{currentfill}{rgb}{0.121569,0.466667,0.705882}%
\pgfsetfillcolor{currentfill}%
\pgfsetlinewidth{0.000000pt}%
\definecolor{currentstroke}{rgb}{0.000000,0.000000,0.000000}%
\pgfsetstrokecolor{currentstroke}%
\pgfsetstrokeopacity{0.000000}%
\pgfsetdash{}{0pt}%
\pgfpathmoveto{\pgfqpoint{5.800000in}{77.744964in}}%
\pgfpathlineto{\pgfqpoint{5.800000in}{77.744964in}}%
\pgfpathlineto{\pgfqpoint{5.800000in}{77.847801in}}%
\pgfpathlineto{\pgfqpoint{5.800000in}{77.847801in}}%
\pgfpathclose%
\pgfusepath{fill}%
\end{pgfscope}%
\begin{pgfscope}%
\pgfpathrectangle{\pgfqpoint{5.800000in}{0.720000in}}{\pgfqpoint{1.400000in}{4.620000in}}%
\pgfusepath{clip}%
\pgfsetbuttcap%
\pgfsetmiterjoin%
\definecolor{currentfill}{rgb}{0.121569,0.466667,0.705882}%
\pgfsetfillcolor{currentfill}%
\pgfsetlinewidth{0.000000pt}%
\definecolor{currentstroke}{rgb}{0.000000,0.000000,0.000000}%
\pgfsetstrokecolor{currentstroke}%
\pgfsetstrokeopacity{0.000000}%
\pgfsetdash{}{0pt}%
\pgfpathmoveto{\pgfqpoint{5.800000in}{77.847801in}}%
\pgfpathlineto{\pgfqpoint{5.800000in}{77.847801in}}%
\pgfpathlineto{\pgfqpoint{5.800000in}{77.950638in}}%
\pgfpathlineto{\pgfqpoint{5.800000in}{77.950638in}}%
\pgfpathclose%
\pgfusepath{fill}%
\end{pgfscope}%
\begin{pgfscope}%
\pgfpathrectangle{\pgfqpoint{5.800000in}{0.720000in}}{\pgfqpoint{1.400000in}{4.620000in}}%
\pgfusepath{clip}%
\pgfsetbuttcap%
\pgfsetmiterjoin%
\definecolor{currentfill}{rgb}{0.121569,0.466667,0.705882}%
\pgfsetfillcolor{currentfill}%
\pgfsetlinewidth{0.000000pt}%
\definecolor{currentstroke}{rgb}{0.000000,0.000000,0.000000}%
\pgfsetstrokecolor{currentstroke}%
\pgfsetstrokeopacity{0.000000}%
\pgfsetdash{}{0pt}%
\pgfpathmoveto{\pgfqpoint{5.800000in}{77.950638in}}%
\pgfpathlineto{\pgfqpoint{5.800000in}{77.950638in}}%
\pgfpathlineto{\pgfqpoint{5.800000in}{78.053475in}}%
\pgfpathlineto{\pgfqpoint{5.800000in}{78.053475in}}%
\pgfpathclose%
\pgfusepath{fill}%
\end{pgfscope}%
\begin{pgfscope}%
\pgfpathrectangle{\pgfqpoint{5.800000in}{0.720000in}}{\pgfqpoint{1.400000in}{4.620000in}}%
\pgfusepath{clip}%
\pgfsetbuttcap%
\pgfsetmiterjoin%
\definecolor{currentfill}{rgb}{0.121569,0.466667,0.705882}%
\pgfsetfillcolor{currentfill}%
\pgfsetlinewidth{0.000000pt}%
\definecolor{currentstroke}{rgb}{0.000000,0.000000,0.000000}%
\pgfsetstrokecolor{currentstroke}%
\pgfsetstrokeopacity{0.000000}%
\pgfsetdash{}{0pt}%
\pgfpathmoveto{\pgfqpoint{5.800000in}{78.053475in}}%
\pgfpathlineto{\pgfqpoint{5.800000in}{78.053475in}}%
\pgfpathlineto{\pgfqpoint{5.800000in}{78.156312in}}%
\pgfpathlineto{\pgfqpoint{5.800000in}{78.156312in}}%
\pgfpathclose%
\pgfusepath{fill}%
\end{pgfscope}%
\begin{pgfscope}%
\pgfpathrectangle{\pgfqpoint{5.800000in}{0.720000in}}{\pgfqpoint{1.400000in}{4.620000in}}%
\pgfusepath{clip}%
\pgfsetbuttcap%
\pgfsetmiterjoin%
\definecolor{currentfill}{rgb}{0.121569,0.466667,0.705882}%
\pgfsetfillcolor{currentfill}%
\pgfsetlinewidth{0.000000pt}%
\definecolor{currentstroke}{rgb}{0.000000,0.000000,0.000000}%
\pgfsetstrokecolor{currentstroke}%
\pgfsetstrokeopacity{0.000000}%
\pgfsetdash{}{0pt}%
\pgfpathmoveto{\pgfqpoint{5.800000in}{78.156312in}}%
\pgfpathlineto{\pgfqpoint{5.800000in}{78.156312in}}%
\pgfpathlineto{\pgfqpoint{5.800000in}{78.259149in}}%
\pgfpathlineto{\pgfqpoint{5.800000in}{78.259149in}}%
\pgfpathclose%
\pgfusepath{fill}%
\end{pgfscope}%
\begin{pgfscope}%
\pgfpathrectangle{\pgfqpoint{5.800000in}{0.720000in}}{\pgfqpoint{1.400000in}{4.620000in}}%
\pgfusepath{clip}%
\pgfsetbuttcap%
\pgfsetmiterjoin%
\definecolor{currentfill}{rgb}{0.121569,0.466667,0.705882}%
\pgfsetfillcolor{currentfill}%
\pgfsetlinewidth{0.000000pt}%
\definecolor{currentstroke}{rgb}{0.000000,0.000000,0.000000}%
\pgfsetstrokecolor{currentstroke}%
\pgfsetstrokeopacity{0.000000}%
\pgfsetdash{}{0pt}%
\pgfpathmoveto{\pgfqpoint{5.800000in}{78.259149in}}%
\pgfpathlineto{\pgfqpoint{5.800000in}{78.259149in}}%
\pgfpathlineto{\pgfqpoint{5.800000in}{78.361986in}}%
\pgfpathlineto{\pgfqpoint{5.800000in}{78.361986in}}%
\pgfpathclose%
\pgfusepath{fill}%
\end{pgfscope}%
\begin{pgfscope}%
\pgfpathrectangle{\pgfqpoint{5.800000in}{0.720000in}}{\pgfqpoint{1.400000in}{4.620000in}}%
\pgfusepath{clip}%
\pgfsetbuttcap%
\pgfsetmiterjoin%
\definecolor{currentfill}{rgb}{0.121569,0.466667,0.705882}%
\pgfsetfillcolor{currentfill}%
\pgfsetlinewidth{0.000000pt}%
\definecolor{currentstroke}{rgb}{0.000000,0.000000,0.000000}%
\pgfsetstrokecolor{currentstroke}%
\pgfsetstrokeopacity{0.000000}%
\pgfsetdash{}{0pt}%
\pgfpathmoveto{\pgfqpoint{5.800000in}{78.361986in}}%
\pgfpathlineto{\pgfqpoint{5.800000in}{78.361986in}}%
\pgfpathlineto{\pgfqpoint{5.800000in}{78.464823in}}%
\pgfpathlineto{\pgfqpoint{5.800000in}{78.464823in}}%
\pgfpathclose%
\pgfusepath{fill}%
\end{pgfscope}%
\begin{pgfscope}%
\pgfpathrectangle{\pgfqpoint{5.800000in}{0.720000in}}{\pgfqpoint{1.400000in}{4.620000in}}%
\pgfusepath{clip}%
\pgfsetbuttcap%
\pgfsetmiterjoin%
\definecolor{currentfill}{rgb}{0.121569,0.466667,0.705882}%
\pgfsetfillcolor{currentfill}%
\pgfsetlinewidth{0.000000pt}%
\definecolor{currentstroke}{rgb}{0.000000,0.000000,0.000000}%
\pgfsetstrokecolor{currentstroke}%
\pgfsetstrokeopacity{0.000000}%
\pgfsetdash{}{0pt}%
\pgfpathmoveto{\pgfqpoint{5.800000in}{78.464823in}}%
\pgfpathlineto{\pgfqpoint{5.800000in}{78.464823in}}%
\pgfpathlineto{\pgfqpoint{5.800000in}{78.567660in}}%
\pgfpathlineto{\pgfqpoint{5.800000in}{78.567660in}}%
\pgfpathclose%
\pgfusepath{fill}%
\end{pgfscope}%
\begin{pgfscope}%
\pgfpathrectangle{\pgfqpoint{5.800000in}{0.720000in}}{\pgfqpoint{1.400000in}{4.620000in}}%
\pgfusepath{clip}%
\pgfsetbuttcap%
\pgfsetmiterjoin%
\definecolor{currentfill}{rgb}{0.121569,0.466667,0.705882}%
\pgfsetfillcolor{currentfill}%
\pgfsetlinewidth{0.000000pt}%
\definecolor{currentstroke}{rgb}{0.000000,0.000000,0.000000}%
\pgfsetstrokecolor{currentstroke}%
\pgfsetstrokeopacity{0.000000}%
\pgfsetdash{}{0pt}%
\pgfpathmoveto{\pgfqpoint{5.800000in}{78.567660in}}%
\pgfpathlineto{\pgfqpoint{5.800000in}{78.567660in}}%
\pgfpathlineto{\pgfqpoint{5.800000in}{78.670497in}}%
\pgfpathlineto{\pgfqpoint{5.800000in}{78.670497in}}%
\pgfpathclose%
\pgfusepath{fill}%
\end{pgfscope}%
\begin{pgfscope}%
\pgfpathrectangle{\pgfqpoint{5.800000in}{0.720000in}}{\pgfqpoint{1.400000in}{4.620000in}}%
\pgfusepath{clip}%
\pgfsetbuttcap%
\pgfsetmiterjoin%
\definecolor{currentfill}{rgb}{0.121569,0.466667,0.705882}%
\pgfsetfillcolor{currentfill}%
\pgfsetlinewidth{0.000000pt}%
\definecolor{currentstroke}{rgb}{0.000000,0.000000,0.000000}%
\pgfsetstrokecolor{currentstroke}%
\pgfsetstrokeopacity{0.000000}%
\pgfsetdash{}{0pt}%
\pgfpathmoveto{\pgfqpoint{5.800000in}{78.670497in}}%
\pgfpathlineto{\pgfqpoint{5.800000in}{78.670497in}}%
\pgfpathlineto{\pgfqpoint{5.800000in}{78.773334in}}%
\pgfpathlineto{\pgfqpoint{5.800000in}{78.773334in}}%
\pgfpathclose%
\pgfusepath{fill}%
\end{pgfscope}%
\begin{pgfscope}%
\pgfpathrectangle{\pgfqpoint{5.800000in}{0.720000in}}{\pgfqpoint{1.400000in}{4.620000in}}%
\pgfusepath{clip}%
\pgfsetbuttcap%
\pgfsetmiterjoin%
\definecolor{currentfill}{rgb}{0.121569,0.466667,0.705882}%
\pgfsetfillcolor{currentfill}%
\pgfsetlinewidth{0.000000pt}%
\definecolor{currentstroke}{rgb}{0.000000,0.000000,0.000000}%
\pgfsetstrokecolor{currentstroke}%
\pgfsetstrokeopacity{0.000000}%
\pgfsetdash{}{0pt}%
\pgfpathmoveto{\pgfqpoint{5.800000in}{78.773334in}}%
\pgfpathlineto{\pgfqpoint{5.800000in}{78.773334in}}%
\pgfpathlineto{\pgfqpoint{5.800000in}{78.876172in}}%
\pgfpathlineto{\pgfqpoint{5.800000in}{78.876172in}}%
\pgfpathclose%
\pgfusepath{fill}%
\end{pgfscope}%
\begin{pgfscope}%
\pgfpathrectangle{\pgfqpoint{5.800000in}{0.720000in}}{\pgfqpoint{1.400000in}{4.620000in}}%
\pgfusepath{clip}%
\pgfsetbuttcap%
\pgfsetmiterjoin%
\definecolor{currentfill}{rgb}{0.121569,0.466667,0.705882}%
\pgfsetfillcolor{currentfill}%
\pgfsetlinewidth{0.000000pt}%
\definecolor{currentstroke}{rgb}{0.000000,0.000000,0.000000}%
\pgfsetstrokecolor{currentstroke}%
\pgfsetstrokeopacity{0.000000}%
\pgfsetdash{}{0pt}%
\pgfpathmoveto{\pgfqpoint{5.800000in}{78.876172in}}%
\pgfpathlineto{\pgfqpoint{5.800000in}{78.876172in}}%
\pgfpathlineto{\pgfqpoint{5.800000in}{78.979009in}}%
\pgfpathlineto{\pgfqpoint{5.800000in}{78.979009in}}%
\pgfpathclose%
\pgfusepath{fill}%
\end{pgfscope}%
\begin{pgfscope}%
\pgfpathrectangle{\pgfqpoint{5.800000in}{0.720000in}}{\pgfqpoint{1.400000in}{4.620000in}}%
\pgfusepath{clip}%
\pgfsetbuttcap%
\pgfsetmiterjoin%
\definecolor{currentfill}{rgb}{0.121569,0.466667,0.705882}%
\pgfsetfillcolor{currentfill}%
\pgfsetlinewidth{0.000000pt}%
\definecolor{currentstroke}{rgb}{0.000000,0.000000,0.000000}%
\pgfsetstrokecolor{currentstroke}%
\pgfsetstrokeopacity{0.000000}%
\pgfsetdash{}{0pt}%
\pgfpathmoveto{\pgfqpoint{5.800000in}{78.979009in}}%
\pgfpathlineto{\pgfqpoint{5.800000in}{78.979009in}}%
\pgfpathlineto{\pgfqpoint{5.800000in}{79.081846in}}%
\pgfpathlineto{\pgfqpoint{5.800000in}{79.081846in}}%
\pgfpathclose%
\pgfusepath{fill}%
\end{pgfscope}%
\begin{pgfscope}%
\pgfpathrectangle{\pgfqpoint{5.800000in}{0.720000in}}{\pgfqpoint{1.400000in}{4.620000in}}%
\pgfusepath{clip}%
\pgfsetbuttcap%
\pgfsetmiterjoin%
\definecolor{currentfill}{rgb}{0.121569,0.466667,0.705882}%
\pgfsetfillcolor{currentfill}%
\pgfsetlinewidth{0.000000pt}%
\definecolor{currentstroke}{rgb}{0.000000,0.000000,0.000000}%
\pgfsetstrokecolor{currentstroke}%
\pgfsetstrokeopacity{0.000000}%
\pgfsetdash{}{0pt}%
\pgfpathmoveto{\pgfqpoint{5.800000in}{79.081846in}}%
\pgfpathlineto{\pgfqpoint{5.800000in}{79.081846in}}%
\pgfpathlineto{\pgfqpoint{5.800000in}{79.184683in}}%
\pgfpathlineto{\pgfqpoint{5.800000in}{79.184683in}}%
\pgfpathclose%
\pgfusepath{fill}%
\end{pgfscope}%
\begin{pgfscope}%
\pgfpathrectangle{\pgfqpoint{5.800000in}{0.720000in}}{\pgfqpoint{1.400000in}{4.620000in}}%
\pgfusepath{clip}%
\pgfsetbuttcap%
\pgfsetmiterjoin%
\definecolor{currentfill}{rgb}{0.121569,0.466667,0.705882}%
\pgfsetfillcolor{currentfill}%
\pgfsetlinewidth{0.000000pt}%
\definecolor{currentstroke}{rgb}{0.000000,0.000000,0.000000}%
\pgfsetstrokecolor{currentstroke}%
\pgfsetstrokeopacity{0.000000}%
\pgfsetdash{}{0pt}%
\pgfpathmoveto{\pgfqpoint{5.800000in}{79.184683in}}%
\pgfpathlineto{\pgfqpoint{5.800000in}{79.184683in}}%
\pgfpathlineto{\pgfqpoint{5.800000in}{79.287520in}}%
\pgfpathlineto{\pgfqpoint{5.800000in}{79.287520in}}%
\pgfpathclose%
\pgfusepath{fill}%
\end{pgfscope}%
\begin{pgfscope}%
\pgfpathrectangle{\pgfqpoint{5.800000in}{0.720000in}}{\pgfqpoint{1.400000in}{4.620000in}}%
\pgfusepath{clip}%
\pgfsetbuttcap%
\pgfsetmiterjoin%
\definecolor{currentfill}{rgb}{0.121569,0.466667,0.705882}%
\pgfsetfillcolor{currentfill}%
\pgfsetlinewidth{0.000000pt}%
\definecolor{currentstroke}{rgb}{0.000000,0.000000,0.000000}%
\pgfsetstrokecolor{currentstroke}%
\pgfsetstrokeopacity{0.000000}%
\pgfsetdash{}{0pt}%
\pgfpathmoveto{\pgfqpoint{5.800000in}{79.287520in}}%
\pgfpathlineto{\pgfqpoint{5.800000in}{79.287520in}}%
\pgfpathlineto{\pgfqpoint{5.800000in}{79.390357in}}%
\pgfpathlineto{\pgfqpoint{5.800000in}{79.390357in}}%
\pgfpathclose%
\pgfusepath{fill}%
\end{pgfscope}%
\begin{pgfscope}%
\pgfpathrectangle{\pgfqpoint{5.800000in}{0.720000in}}{\pgfqpoint{1.400000in}{4.620000in}}%
\pgfusepath{clip}%
\pgfsetbuttcap%
\pgfsetmiterjoin%
\definecolor{currentfill}{rgb}{0.121569,0.466667,0.705882}%
\pgfsetfillcolor{currentfill}%
\pgfsetlinewidth{0.000000pt}%
\definecolor{currentstroke}{rgb}{0.000000,0.000000,0.000000}%
\pgfsetstrokecolor{currentstroke}%
\pgfsetstrokeopacity{0.000000}%
\pgfsetdash{}{0pt}%
\pgfpathmoveto{\pgfqpoint{5.800000in}{79.390357in}}%
\pgfpathlineto{\pgfqpoint{5.800000in}{79.390357in}}%
\pgfpathlineto{\pgfqpoint{5.800000in}{79.493194in}}%
\pgfpathlineto{\pgfqpoint{5.800000in}{79.493194in}}%
\pgfpathclose%
\pgfusepath{fill}%
\end{pgfscope}%
\begin{pgfscope}%
\pgfpathrectangle{\pgfqpoint{5.800000in}{0.720000in}}{\pgfqpoint{1.400000in}{4.620000in}}%
\pgfusepath{clip}%
\pgfsetbuttcap%
\pgfsetmiterjoin%
\definecolor{currentfill}{rgb}{0.121569,0.466667,0.705882}%
\pgfsetfillcolor{currentfill}%
\pgfsetlinewidth{0.000000pt}%
\definecolor{currentstroke}{rgb}{0.000000,0.000000,0.000000}%
\pgfsetstrokecolor{currentstroke}%
\pgfsetstrokeopacity{0.000000}%
\pgfsetdash{}{0pt}%
\pgfpathmoveto{\pgfqpoint{5.800000in}{79.493194in}}%
\pgfpathlineto{\pgfqpoint{5.800000in}{79.493194in}}%
\pgfpathlineto{\pgfqpoint{5.800000in}{79.596031in}}%
\pgfpathlineto{\pgfqpoint{5.800000in}{79.596031in}}%
\pgfpathclose%
\pgfusepath{fill}%
\end{pgfscope}%
\begin{pgfscope}%
\pgfpathrectangle{\pgfqpoint{5.800000in}{0.720000in}}{\pgfqpoint{1.400000in}{4.620000in}}%
\pgfusepath{clip}%
\pgfsetbuttcap%
\pgfsetmiterjoin%
\definecolor{currentfill}{rgb}{0.121569,0.466667,0.705882}%
\pgfsetfillcolor{currentfill}%
\pgfsetlinewidth{0.000000pt}%
\definecolor{currentstroke}{rgb}{0.000000,0.000000,0.000000}%
\pgfsetstrokecolor{currentstroke}%
\pgfsetstrokeopacity{0.000000}%
\pgfsetdash{}{0pt}%
\pgfpathmoveto{\pgfqpoint{5.800000in}{79.596031in}}%
\pgfpathlineto{\pgfqpoint{5.800000in}{79.596031in}}%
\pgfpathlineto{\pgfqpoint{5.800000in}{79.698868in}}%
\pgfpathlineto{\pgfqpoint{5.800000in}{79.698868in}}%
\pgfpathclose%
\pgfusepath{fill}%
\end{pgfscope}%
\begin{pgfscope}%
\pgfpathrectangle{\pgfqpoint{5.800000in}{0.720000in}}{\pgfqpoint{1.400000in}{4.620000in}}%
\pgfusepath{clip}%
\pgfsetbuttcap%
\pgfsetmiterjoin%
\definecolor{currentfill}{rgb}{0.121569,0.466667,0.705882}%
\pgfsetfillcolor{currentfill}%
\pgfsetlinewidth{0.000000pt}%
\definecolor{currentstroke}{rgb}{0.000000,0.000000,0.000000}%
\pgfsetstrokecolor{currentstroke}%
\pgfsetstrokeopacity{0.000000}%
\pgfsetdash{}{0pt}%
\pgfpathmoveto{\pgfqpoint{5.800000in}{79.698868in}}%
\pgfpathlineto{\pgfqpoint{5.800000in}{79.698868in}}%
\pgfpathlineto{\pgfqpoint{5.800000in}{79.801705in}}%
\pgfpathlineto{\pgfqpoint{5.800000in}{79.801705in}}%
\pgfpathclose%
\pgfusepath{fill}%
\end{pgfscope}%
\begin{pgfscope}%
\pgfpathrectangle{\pgfqpoint{5.800000in}{0.720000in}}{\pgfqpoint{1.400000in}{4.620000in}}%
\pgfusepath{clip}%
\pgfsetbuttcap%
\pgfsetmiterjoin%
\definecolor{currentfill}{rgb}{0.121569,0.466667,0.705882}%
\pgfsetfillcolor{currentfill}%
\pgfsetlinewidth{0.000000pt}%
\definecolor{currentstroke}{rgb}{0.000000,0.000000,0.000000}%
\pgfsetstrokecolor{currentstroke}%
\pgfsetstrokeopacity{0.000000}%
\pgfsetdash{}{0pt}%
\pgfpathmoveto{\pgfqpoint{5.800000in}{79.801705in}}%
\pgfpathlineto{\pgfqpoint{5.800000in}{79.801705in}}%
\pgfpathlineto{\pgfqpoint{5.800000in}{79.904542in}}%
\pgfpathlineto{\pgfqpoint{5.800000in}{79.904542in}}%
\pgfpathclose%
\pgfusepath{fill}%
\end{pgfscope}%
\begin{pgfscope}%
\pgfpathrectangle{\pgfqpoint{5.800000in}{0.720000in}}{\pgfqpoint{1.400000in}{4.620000in}}%
\pgfusepath{clip}%
\pgfsetbuttcap%
\pgfsetmiterjoin%
\definecolor{currentfill}{rgb}{0.121569,0.466667,0.705882}%
\pgfsetfillcolor{currentfill}%
\pgfsetlinewidth{0.000000pt}%
\definecolor{currentstroke}{rgb}{0.000000,0.000000,0.000000}%
\pgfsetstrokecolor{currentstroke}%
\pgfsetstrokeopacity{0.000000}%
\pgfsetdash{}{0pt}%
\pgfpathmoveto{\pgfqpoint{5.800000in}{79.904542in}}%
\pgfpathlineto{\pgfqpoint{5.800000in}{79.904542in}}%
\pgfpathlineto{\pgfqpoint{5.800000in}{80.007379in}}%
\pgfpathlineto{\pgfqpoint{5.800000in}{80.007379in}}%
\pgfpathclose%
\pgfusepath{fill}%
\end{pgfscope}%
\begin{pgfscope}%
\pgfpathrectangle{\pgfqpoint{5.800000in}{0.720000in}}{\pgfqpoint{1.400000in}{4.620000in}}%
\pgfusepath{clip}%
\pgfsetbuttcap%
\pgfsetmiterjoin%
\definecolor{currentfill}{rgb}{0.121569,0.466667,0.705882}%
\pgfsetfillcolor{currentfill}%
\pgfsetlinewidth{0.000000pt}%
\definecolor{currentstroke}{rgb}{0.000000,0.000000,0.000000}%
\pgfsetstrokecolor{currentstroke}%
\pgfsetstrokeopacity{0.000000}%
\pgfsetdash{}{0pt}%
\pgfpathmoveto{\pgfqpoint{5.800000in}{80.007379in}}%
\pgfpathlineto{\pgfqpoint{5.800000in}{80.007379in}}%
\pgfpathlineto{\pgfqpoint{5.800000in}{80.110216in}}%
\pgfpathlineto{\pgfqpoint{5.800000in}{80.110216in}}%
\pgfpathclose%
\pgfusepath{fill}%
\end{pgfscope}%
\begin{pgfscope}%
\pgfpathrectangle{\pgfqpoint{5.800000in}{0.720000in}}{\pgfqpoint{1.400000in}{4.620000in}}%
\pgfusepath{clip}%
\pgfsetbuttcap%
\pgfsetmiterjoin%
\definecolor{currentfill}{rgb}{0.121569,0.466667,0.705882}%
\pgfsetfillcolor{currentfill}%
\pgfsetlinewidth{0.000000pt}%
\definecolor{currentstroke}{rgb}{0.000000,0.000000,0.000000}%
\pgfsetstrokecolor{currentstroke}%
\pgfsetstrokeopacity{0.000000}%
\pgfsetdash{}{0pt}%
\pgfpathmoveto{\pgfqpoint{5.800000in}{80.110216in}}%
\pgfpathlineto{\pgfqpoint{5.800000in}{80.110216in}}%
\pgfpathlineto{\pgfqpoint{5.800000in}{80.213053in}}%
\pgfpathlineto{\pgfqpoint{5.800000in}{80.213053in}}%
\pgfpathclose%
\pgfusepath{fill}%
\end{pgfscope}%
\begin{pgfscope}%
\pgfpathrectangle{\pgfqpoint{5.800000in}{0.720000in}}{\pgfqpoint{1.400000in}{4.620000in}}%
\pgfusepath{clip}%
\pgfsetbuttcap%
\pgfsetmiterjoin%
\definecolor{currentfill}{rgb}{0.121569,0.466667,0.705882}%
\pgfsetfillcolor{currentfill}%
\pgfsetlinewidth{0.000000pt}%
\definecolor{currentstroke}{rgb}{0.000000,0.000000,0.000000}%
\pgfsetstrokecolor{currentstroke}%
\pgfsetstrokeopacity{0.000000}%
\pgfsetdash{}{0pt}%
\pgfpathmoveto{\pgfqpoint{5.800000in}{80.213053in}}%
\pgfpathlineto{\pgfqpoint{5.800000in}{80.213053in}}%
\pgfpathlineto{\pgfqpoint{5.800000in}{80.315890in}}%
\pgfpathlineto{\pgfqpoint{5.800000in}{80.315890in}}%
\pgfpathclose%
\pgfusepath{fill}%
\end{pgfscope}%
\begin{pgfscope}%
\pgfpathrectangle{\pgfqpoint{5.800000in}{0.720000in}}{\pgfqpoint{1.400000in}{4.620000in}}%
\pgfusepath{clip}%
\pgfsetbuttcap%
\pgfsetmiterjoin%
\definecolor{currentfill}{rgb}{0.121569,0.466667,0.705882}%
\pgfsetfillcolor{currentfill}%
\pgfsetlinewidth{0.000000pt}%
\definecolor{currentstroke}{rgb}{0.000000,0.000000,0.000000}%
\pgfsetstrokecolor{currentstroke}%
\pgfsetstrokeopacity{0.000000}%
\pgfsetdash{}{0pt}%
\pgfpathmoveto{\pgfqpoint{5.800000in}{80.315890in}}%
\pgfpathlineto{\pgfqpoint{5.800000in}{80.315890in}}%
\pgfpathlineto{\pgfqpoint{5.800000in}{80.418728in}}%
\pgfpathlineto{\pgfqpoint{5.800000in}{80.418728in}}%
\pgfpathclose%
\pgfusepath{fill}%
\end{pgfscope}%
\begin{pgfscope}%
\pgfpathrectangle{\pgfqpoint{5.800000in}{0.720000in}}{\pgfqpoint{1.400000in}{4.620000in}}%
\pgfusepath{clip}%
\pgfsetbuttcap%
\pgfsetmiterjoin%
\definecolor{currentfill}{rgb}{0.121569,0.466667,0.705882}%
\pgfsetfillcolor{currentfill}%
\pgfsetlinewidth{0.000000pt}%
\definecolor{currentstroke}{rgb}{0.000000,0.000000,0.000000}%
\pgfsetstrokecolor{currentstroke}%
\pgfsetstrokeopacity{0.000000}%
\pgfsetdash{}{0pt}%
\pgfpathmoveto{\pgfqpoint{5.800000in}{80.418728in}}%
\pgfpathlineto{\pgfqpoint{5.800000in}{80.418728in}}%
\pgfpathlineto{\pgfqpoint{5.800000in}{80.521565in}}%
\pgfpathlineto{\pgfqpoint{5.800000in}{80.521565in}}%
\pgfpathclose%
\pgfusepath{fill}%
\end{pgfscope}%
\begin{pgfscope}%
\pgfpathrectangle{\pgfqpoint{5.800000in}{0.720000in}}{\pgfqpoint{1.400000in}{4.620000in}}%
\pgfusepath{clip}%
\pgfsetbuttcap%
\pgfsetmiterjoin%
\definecolor{currentfill}{rgb}{0.121569,0.466667,0.705882}%
\pgfsetfillcolor{currentfill}%
\pgfsetlinewidth{0.000000pt}%
\definecolor{currentstroke}{rgb}{0.000000,0.000000,0.000000}%
\pgfsetstrokecolor{currentstroke}%
\pgfsetstrokeopacity{0.000000}%
\pgfsetdash{}{0pt}%
\pgfpathmoveto{\pgfqpoint{5.800000in}{80.521565in}}%
\pgfpathlineto{\pgfqpoint{5.800000in}{80.521565in}}%
\pgfpathlineto{\pgfqpoint{5.800000in}{80.624402in}}%
\pgfpathlineto{\pgfqpoint{5.800000in}{80.624402in}}%
\pgfpathclose%
\pgfusepath{fill}%
\end{pgfscope}%
\begin{pgfscope}%
\pgfpathrectangle{\pgfqpoint{5.800000in}{0.720000in}}{\pgfqpoint{1.400000in}{4.620000in}}%
\pgfusepath{clip}%
\pgfsetbuttcap%
\pgfsetmiterjoin%
\definecolor{currentfill}{rgb}{0.121569,0.466667,0.705882}%
\pgfsetfillcolor{currentfill}%
\pgfsetlinewidth{0.000000pt}%
\definecolor{currentstroke}{rgb}{0.000000,0.000000,0.000000}%
\pgfsetstrokecolor{currentstroke}%
\pgfsetstrokeopacity{0.000000}%
\pgfsetdash{}{0pt}%
\pgfpathmoveto{\pgfqpoint{5.800000in}{80.624402in}}%
\pgfpathlineto{\pgfqpoint{5.800000in}{80.624402in}}%
\pgfpathlineto{\pgfqpoint{5.800000in}{80.727239in}}%
\pgfpathlineto{\pgfqpoint{5.800000in}{80.727239in}}%
\pgfpathclose%
\pgfusepath{fill}%
\end{pgfscope}%
\begin{pgfscope}%
\pgfpathrectangle{\pgfqpoint{5.800000in}{0.720000in}}{\pgfqpoint{1.400000in}{4.620000in}}%
\pgfusepath{clip}%
\pgfsetbuttcap%
\pgfsetmiterjoin%
\definecolor{currentfill}{rgb}{0.121569,0.466667,0.705882}%
\pgfsetfillcolor{currentfill}%
\pgfsetlinewidth{0.000000pt}%
\definecolor{currentstroke}{rgb}{0.000000,0.000000,0.000000}%
\pgfsetstrokecolor{currentstroke}%
\pgfsetstrokeopacity{0.000000}%
\pgfsetdash{}{0pt}%
\pgfpathmoveto{\pgfqpoint{5.800000in}{80.727239in}}%
\pgfpathlineto{\pgfqpoint{5.800000in}{80.727239in}}%
\pgfpathlineto{\pgfqpoint{5.800000in}{80.830076in}}%
\pgfpathlineto{\pgfqpoint{5.800000in}{80.830076in}}%
\pgfpathclose%
\pgfusepath{fill}%
\end{pgfscope}%
\begin{pgfscope}%
\pgfpathrectangle{\pgfqpoint{5.800000in}{0.720000in}}{\pgfqpoint{1.400000in}{4.620000in}}%
\pgfusepath{clip}%
\pgfsetbuttcap%
\pgfsetmiterjoin%
\definecolor{currentfill}{rgb}{0.121569,0.466667,0.705882}%
\pgfsetfillcolor{currentfill}%
\pgfsetlinewidth{0.000000pt}%
\definecolor{currentstroke}{rgb}{0.000000,0.000000,0.000000}%
\pgfsetstrokecolor{currentstroke}%
\pgfsetstrokeopacity{0.000000}%
\pgfsetdash{}{0pt}%
\pgfpathmoveto{\pgfqpoint{5.800000in}{80.830076in}}%
\pgfpathlineto{\pgfqpoint{5.800000in}{80.830076in}}%
\pgfpathlineto{\pgfqpoint{5.800000in}{80.932913in}}%
\pgfpathlineto{\pgfqpoint{5.800000in}{80.932913in}}%
\pgfpathclose%
\pgfusepath{fill}%
\end{pgfscope}%
\begin{pgfscope}%
\pgfpathrectangle{\pgfqpoint{5.800000in}{0.720000in}}{\pgfqpoint{1.400000in}{4.620000in}}%
\pgfusepath{clip}%
\pgfsetbuttcap%
\pgfsetmiterjoin%
\definecolor{currentfill}{rgb}{0.121569,0.466667,0.705882}%
\pgfsetfillcolor{currentfill}%
\pgfsetlinewidth{0.000000pt}%
\definecolor{currentstroke}{rgb}{0.000000,0.000000,0.000000}%
\pgfsetstrokecolor{currentstroke}%
\pgfsetstrokeopacity{0.000000}%
\pgfsetdash{}{0pt}%
\pgfpathmoveto{\pgfqpoint{5.800000in}{80.932913in}}%
\pgfpathlineto{\pgfqpoint{5.800000in}{80.932913in}}%
\pgfpathlineto{\pgfqpoint{5.800000in}{81.035750in}}%
\pgfpathlineto{\pgfqpoint{5.800000in}{81.035750in}}%
\pgfpathclose%
\pgfusepath{fill}%
\end{pgfscope}%
\begin{pgfscope}%
\pgfpathrectangle{\pgfqpoint{5.800000in}{0.720000in}}{\pgfqpoint{1.400000in}{4.620000in}}%
\pgfusepath{clip}%
\pgfsetbuttcap%
\pgfsetmiterjoin%
\definecolor{currentfill}{rgb}{0.121569,0.466667,0.705882}%
\pgfsetfillcolor{currentfill}%
\pgfsetlinewidth{0.000000pt}%
\definecolor{currentstroke}{rgb}{0.000000,0.000000,0.000000}%
\pgfsetstrokecolor{currentstroke}%
\pgfsetstrokeopacity{0.000000}%
\pgfsetdash{}{0pt}%
\pgfpathmoveto{\pgfqpoint{5.800000in}{81.035750in}}%
\pgfpathlineto{\pgfqpoint{5.800000in}{81.035750in}}%
\pgfpathlineto{\pgfqpoint{5.800000in}{81.138587in}}%
\pgfpathlineto{\pgfqpoint{5.800000in}{81.138587in}}%
\pgfpathclose%
\pgfusepath{fill}%
\end{pgfscope}%
\begin{pgfscope}%
\pgfpathrectangle{\pgfqpoint{5.800000in}{0.720000in}}{\pgfqpoint{1.400000in}{4.620000in}}%
\pgfusepath{clip}%
\pgfsetbuttcap%
\pgfsetmiterjoin%
\definecolor{currentfill}{rgb}{0.121569,0.466667,0.705882}%
\pgfsetfillcolor{currentfill}%
\pgfsetlinewidth{0.000000pt}%
\definecolor{currentstroke}{rgb}{0.000000,0.000000,0.000000}%
\pgfsetstrokecolor{currentstroke}%
\pgfsetstrokeopacity{0.000000}%
\pgfsetdash{}{0pt}%
\pgfpathmoveto{\pgfqpoint{5.800000in}{81.138587in}}%
\pgfpathlineto{\pgfqpoint{5.800000in}{81.138587in}}%
\pgfpathlineto{\pgfqpoint{5.800000in}{81.241424in}}%
\pgfpathlineto{\pgfqpoint{5.800000in}{81.241424in}}%
\pgfpathclose%
\pgfusepath{fill}%
\end{pgfscope}%
\begin{pgfscope}%
\pgfpathrectangle{\pgfqpoint{5.800000in}{0.720000in}}{\pgfqpoint{1.400000in}{4.620000in}}%
\pgfusepath{clip}%
\pgfsetbuttcap%
\pgfsetmiterjoin%
\definecolor{currentfill}{rgb}{0.121569,0.466667,0.705882}%
\pgfsetfillcolor{currentfill}%
\pgfsetlinewidth{0.000000pt}%
\definecolor{currentstroke}{rgb}{0.000000,0.000000,0.000000}%
\pgfsetstrokecolor{currentstroke}%
\pgfsetstrokeopacity{0.000000}%
\pgfsetdash{}{0pt}%
\pgfpathmoveto{\pgfqpoint{5.800000in}{81.241424in}}%
\pgfpathlineto{\pgfqpoint{5.800000in}{81.241424in}}%
\pgfpathlineto{\pgfqpoint{5.800000in}{81.344261in}}%
\pgfpathlineto{\pgfqpoint{5.800000in}{81.344261in}}%
\pgfpathclose%
\pgfusepath{fill}%
\end{pgfscope}%
\begin{pgfscope}%
\pgfpathrectangle{\pgfqpoint{5.800000in}{0.720000in}}{\pgfqpoint{1.400000in}{4.620000in}}%
\pgfusepath{clip}%
\pgfsetbuttcap%
\pgfsetmiterjoin%
\definecolor{currentfill}{rgb}{0.121569,0.466667,0.705882}%
\pgfsetfillcolor{currentfill}%
\pgfsetlinewidth{0.000000pt}%
\definecolor{currentstroke}{rgb}{0.000000,0.000000,0.000000}%
\pgfsetstrokecolor{currentstroke}%
\pgfsetstrokeopacity{0.000000}%
\pgfsetdash{}{0pt}%
\pgfpathmoveto{\pgfqpoint{5.800000in}{81.344261in}}%
\pgfpathlineto{\pgfqpoint{5.800000in}{81.344261in}}%
\pgfpathlineto{\pgfqpoint{5.800000in}{81.447098in}}%
\pgfpathlineto{\pgfqpoint{5.800000in}{81.447098in}}%
\pgfpathclose%
\pgfusepath{fill}%
\end{pgfscope}%
\begin{pgfscope}%
\pgfpathrectangle{\pgfqpoint{5.800000in}{0.720000in}}{\pgfqpoint{1.400000in}{4.620000in}}%
\pgfusepath{clip}%
\pgfsetbuttcap%
\pgfsetmiterjoin%
\definecolor{currentfill}{rgb}{0.121569,0.466667,0.705882}%
\pgfsetfillcolor{currentfill}%
\pgfsetlinewidth{0.000000pt}%
\definecolor{currentstroke}{rgb}{0.000000,0.000000,0.000000}%
\pgfsetstrokecolor{currentstroke}%
\pgfsetstrokeopacity{0.000000}%
\pgfsetdash{}{0pt}%
\pgfpathmoveto{\pgfqpoint{5.800000in}{81.447098in}}%
\pgfpathlineto{\pgfqpoint{5.800000in}{81.447098in}}%
\pgfpathlineto{\pgfqpoint{5.800000in}{81.549935in}}%
\pgfpathlineto{\pgfqpoint{5.800000in}{81.549935in}}%
\pgfpathclose%
\pgfusepath{fill}%
\end{pgfscope}%
\begin{pgfscope}%
\pgfpathrectangle{\pgfqpoint{5.800000in}{0.720000in}}{\pgfqpoint{1.400000in}{4.620000in}}%
\pgfusepath{clip}%
\pgfsetbuttcap%
\pgfsetmiterjoin%
\definecolor{currentfill}{rgb}{0.121569,0.466667,0.705882}%
\pgfsetfillcolor{currentfill}%
\pgfsetlinewidth{0.000000pt}%
\definecolor{currentstroke}{rgb}{0.000000,0.000000,0.000000}%
\pgfsetstrokecolor{currentstroke}%
\pgfsetstrokeopacity{0.000000}%
\pgfsetdash{}{0pt}%
\pgfpathmoveto{\pgfqpoint{5.800000in}{81.549935in}}%
\pgfpathlineto{\pgfqpoint{5.800000in}{81.549935in}}%
\pgfpathlineto{\pgfqpoint{5.800000in}{81.652772in}}%
\pgfpathlineto{\pgfqpoint{5.800000in}{81.652772in}}%
\pgfpathclose%
\pgfusepath{fill}%
\end{pgfscope}%
\begin{pgfscope}%
\pgfpathrectangle{\pgfqpoint{5.800000in}{0.720000in}}{\pgfqpoint{1.400000in}{4.620000in}}%
\pgfusepath{clip}%
\pgfsetbuttcap%
\pgfsetmiterjoin%
\definecolor{currentfill}{rgb}{0.121569,0.466667,0.705882}%
\pgfsetfillcolor{currentfill}%
\pgfsetlinewidth{0.000000pt}%
\definecolor{currentstroke}{rgb}{0.000000,0.000000,0.000000}%
\pgfsetstrokecolor{currentstroke}%
\pgfsetstrokeopacity{0.000000}%
\pgfsetdash{}{0pt}%
\pgfpathmoveto{\pgfqpoint{5.800000in}{81.652772in}}%
\pgfpathlineto{\pgfqpoint{5.800000in}{81.652772in}}%
\pgfpathlineto{\pgfqpoint{5.800000in}{81.755609in}}%
\pgfpathlineto{\pgfqpoint{5.800000in}{81.755609in}}%
\pgfpathclose%
\pgfusepath{fill}%
\end{pgfscope}%
\begin{pgfscope}%
\pgfpathrectangle{\pgfqpoint{5.800000in}{0.720000in}}{\pgfqpoint{1.400000in}{4.620000in}}%
\pgfusepath{clip}%
\pgfsetbuttcap%
\pgfsetmiterjoin%
\definecolor{currentfill}{rgb}{0.121569,0.466667,0.705882}%
\pgfsetfillcolor{currentfill}%
\pgfsetlinewidth{0.000000pt}%
\definecolor{currentstroke}{rgb}{0.000000,0.000000,0.000000}%
\pgfsetstrokecolor{currentstroke}%
\pgfsetstrokeopacity{0.000000}%
\pgfsetdash{}{0pt}%
\pgfpathmoveto{\pgfqpoint{5.800000in}{81.755609in}}%
\pgfpathlineto{\pgfqpoint{5.800000in}{81.755609in}}%
\pgfpathlineto{\pgfqpoint{5.800000in}{81.858446in}}%
\pgfpathlineto{\pgfqpoint{5.800000in}{81.858446in}}%
\pgfpathclose%
\pgfusepath{fill}%
\end{pgfscope}%
\begin{pgfscope}%
\pgfpathrectangle{\pgfqpoint{5.800000in}{0.720000in}}{\pgfqpoint{1.400000in}{4.620000in}}%
\pgfusepath{clip}%
\pgfsetbuttcap%
\pgfsetmiterjoin%
\definecolor{currentfill}{rgb}{0.121569,0.466667,0.705882}%
\pgfsetfillcolor{currentfill}%
\pgfsetlinewidth{0.000000pt}%
\definecolor{currentstroke}{rgb}{0.000000,0.000000,0.000000}%
\pgfsetstrokecolor{currentstroke}%
\pgfsetstrokeopacity{0.000000}%
\pgfsetdash{}{0pt}%
\pgfpathmoveto{\pgfqpoint{5.800000in}{81.858446in}}%
\pgfpathlineto{\pgfqpoint{5.800000in}{81.858446in}}%
\pgfpathlineto{\pgfqpoint{5.800000in}{81.961284in}}%
\pgfpathlineto{\pgfqpoint{5.800000in}{81.961284in}}%
\pgfpathclose%
\pgfusepath{fill}%
\end{pgfscope}%
\begin{pgfscope}%
\pgfpathrectangle{\pgfqpoint{5.800000in}{0.720000in}}{\pgfqpoint{1.400000in}{4.620000in}}%
\pgfusepath{clip}%
\pgfsetbuttcap%
\pgfsetmiterjoin%
\definecolor{currentfill}{rgb}{0.121569,0.466667,0.705882}%
\pgfsetfillcolor{currentfill}%
\pgfsetlinewidth{0.000000pt}%
\definecolor{currentstroke}{rgb}{0.000000,0.000000,0.000000}%
\pgfsetstrokecolor{currentstroke}%
\pgfsetstrokeopacity{0.000000}%
\pgfsetdash{}{0pt}%
\pgfpathmoveto{\pgfqpoint{5.800000in}{81.961284in}}%
\pgfpathlineto{\pgfqpoint{5.800000in}{81.961284in}}%
\pgfpathlineto{\pgfqpoint{5.800000in}{82.064121in}}%
\pgfpathlineto{\pgfqpoint{5.800000in}{82.064121in}}%
\pgfpathclose%
\pgfusepath{fill}%
\end{pgfscope}%
\begin{pgfscope}%
\pgfpathrectangle{\pgfqpoint{5.800000in}{0.720000in}}{\pgfqpoint{1.400000in}{4.620000in}}%
\pgfusepath{clip}%
\pgfsetbuttcap%
\pgfsetmiterjoin%
\definecolor{currentfill}{rgb}{0.121569,0.466667,0.705882}%
\pgfsetfillcolor{currentfill}%
\pgfsetlinewidth{0.000000pt}%
\definecolor{currentstroke}{rgb}{0.000000,0.000000,0.000000}%
\pgfsetstrokecolor{currentstroke}%
\pgfsetstrokeopacity{0.000000}%
\pgfsetdash{}{0pt}%
\pgfpathmoveto{\pgfqpoint{5.800000in}{82.064121in}}%
\pgfpathlineto{\pgfqpoint{5.800000in}{82.064121in}}%
\pgfpathlineto{\pgfqpoint{5.800000in}{82.166958in}}%
\pgfpathlineto{\pgfqpoint{5.800000in}{82.166958in}}%
\pgfpathclose%
\pgfusepath{fill}%
\end{pgfscope}%
\begin{pgfscope}%
\pgfpathrectangle{\pgfqpoint{5.800000in}{0.720000in}}{\pgfqpoint{1.400000in}{4.620000in}}%
\pgfusepath{clip}%
\pgfsetbuttcap%
\pgfsetmiterjoin%
\definecolor{currentfill}{rgb}{0.121569,0.466667,0.705882}%
\pgfsetfillcolor{currentfill}%
\pgfsetlinewidth{0.000000pt}%
\definecolor{currentstroke}{rgb}{0.000000,0.000000,0.000000}%
\pgfsetstrokecolor{currentstroke}%
\pgfsetstrokeopacity{0.000000}%
\pgfsetdash{}{0pt}%
\pgfpathmoveto{\pgfqpoint{5.800000in}{82.166958in}}%
\pgfpathlineto{\pgfqpoint{5.800000in}{82.166958in}}%
\pgfpathlineto{\pgfqpoint{5.800000in}{82.269795in}}%
\pgfpathlineto{\pgfqpoint{5.800000in}{82.269795in}}%
\pgfpathclose%
\pgfusepath{fill}%
\end{pgfscope}%
\begin{pgfscope}%
\pgfpathrectangle{\pgfqpoint{5.800000in}{0.720000in}}{\pgfqpoint{1.400000in}{4.620000in}}%
\pgfusepath{clip}%
\pgfsetbuttcap%
\pgfsetmiterjoin%
\definecolor{currentfill}{rgb}{0.121569,0.466667,0.705882}%
\pgfsetfillcolor{currentfill}%
\pgfsetlinewidth{0.000000pt}%
\definecolor{currentstroke}{rgb}{0.000000,0.000000,0.000000}%
\pgfsetstrokecolor{currentstroke}%
\pgfsetstrokeopacity{0.000000}%
\pgfsetdash{}{0pt}%
\pgfpathmoveto{\pgfqpoint{5.800000in}{82.269795in}}%
\pgfpathlineto{\pgfqpoint{5.800000in}{82.269795in}}%
\pgfpathlineto{\pgfqpoint{5.800000in}{82.372632in}}%
\pgfpathlineto{\pgfqpoint{5.800000in}{82.372632in}}%
\pgfpathclose%
\pgfusepath{fill}%
\end{pgfscope}%
\begin{pgfscope}%
\pgfpathrectangle{\pgfqpoint{5.800000in}{0.720000in}}{\pgfqpoint{1.400000in}{4.620000in}}%
\pgfusepath{clip}%
\pgfsetbuttcap%
\pgfsetmiterjoin%
\definecolor{currentfill}{rgb}{0.121569,0.466667,0.705882}%
\pgfsetfillcolor{currentfill}%
\pgfsetlinewidth{0.000000pt}%
\definecolor{currentstroke}{rgb}{0.000000,0.000000,0.000000}%
\pgfsetstrokecolor{currentstroke}%
\pgfsetstrokeopacity{0.000000}%
\pgfsetdash{}{0pt}%
\pgfpathmoveto{\pgfqpoint{5.800000in}{82.372632in}}%
\pgfpathlineto{\pgfqpoint{5.800000in}{82.372632in}}%
\pgfpathlineto{\pgfqpoint{5.800000in}{82.475469in}}%
\pgfpathlineto{\pgfqpoint{5.800000in}{82.475469in}}%
\pgfpathclose%
\pgfusepath{fill}%
\end{pgfscope}%
\begin{pgfscope}%
\pgfpathrectangle{\pgfqpoint{5.800000in}{0.720000in}}{\pgfqpoint{1.400000in}{4.620000in}}%
\pgfusepath{clip}%
\pgfsetbuttcap%
\pgfsetmiterjoin%
\definecolor{currentfill}{rgb}{0.121569,0.466667,0.705882}%
\pgfsetfillcolor{currentfill}%
\pgfsetlinewidth{0.000000pt}%
\definecolor{currentstroke}{rgb}{0.000000,0.000000,0.000000}%
\pgfsetstrokecolor{currentstroke}%
\pgfsetstrokeopacity{0.000000}%
\pgfsetdash{}{0pt}%
\pgfpathmoveto{\pgfqpoint{5.800000in}{82.475469in}}%
\pgfpathlineto{\pgfqpoint{5.800000in}{82.475469in}}%
\pgfpathlineto{\pgfqpoint{5.800000in}{82.578306in}}%
\pgfpathlineto{\pgfqpoint{5.800000in}{82.578306in}}%
\pgfpathclose%
\pgfusepath{fill}%
\end{pgfscope}%
\begin{pgfscope}%
\pgfpathrectangle{\pgfqpoint{5.800000in}{0.720000in}}{\pgfqpoint{1.400000in}{4.620000in}}%
\pgfusepath{clip}%
\pgfsetbuttcap%
\pgfsetmiterjoin%
\definecolor{currentfill}{rgb}{0.121569,0.466667,0.705882}%
\pgfsetfillcolor{currentfill}%
\pgfsetlinewidth{0.000000pt}%
\definecolor{currentstroke}{rgb}{0.000000,0.000000,0.000000}%
\pgfsetstrokecolor{currentstroke}%
\pgfsetstrokeopacity{0.000000}%
\pgfsetdash{}{0pt}%
\pgfpathmoveto{\pgfqpoint{5.800000in}{82.578306in}}%
\pgfpathlineto{\pgfqpoint{5.800000in}{82.578306in}}%
\pgfpathlineto{\pgfqpoint{5.800000in}{82.681143in}}%
\pgfpathlineto{\pgfqpoint{5.800000in}{82.681143in}}%
\pgfpathclose%
\pgfusepath{fill}%
\end{pgfscope}%
\begin{pgfscope}%
\pgfpathrectangle{\pgfqpoint{5.800000in}{0.720000in}}{\pgfqpoint{1.400000in}{4.620000in}}%
\pgfusepath{clip}%
\pgfsetbuttcap%
\pgfsetmiterjoin%
\definecolor{currentfill}{rgb}{0.121569,0.466667,0.705882}%
\pgfsetfillcolor{currentfill}%
\pgfsetlinewidth{0.000000pt}%
\definecolor{currentstroke}{rgb}{0.000000,0.000000,0.000000}%
\pgfsetstrokecolor{currentstroke}%
\pgfsetstrokeopacity{0.000000}%
\pgfsetdash{}{0pt}%
\pgfpathmoveto{\pgfqpoint{5.800000in}{82.681143in}}%
\pgfpathlineto{\pgfqpoint{5.800000in}{82.681143in}}%
\pgfpathlineto{\pgfqpoint{5.800000in}{82.783980in}}%
\pgfpathlineto{\pgfqpoint{5.800000in}{82.783980in}}%
\pgfpathclose%
\pgfusepath{fill}%
\end{pgfscope}%
\begin{pgfscope}%
\pgfpathrectangle{\pgfqpoint{5.800000in}{0.720000in}}{\pgfqpoint{1.400000in}{4.620000in}}%
\pgfusepath{clip}%
\pgfsetbuttcap%
\pgfsetmiterjoin%
\definecolor{currentfill}{rgb}{0.121569,0.466667,0.705882}%
\pgfsetfillcolor{currentfill}%
\pgfsetlinewidth{0.000000pt}%
\definecolor{currentstroke}{rgb}{0.000000,0.000000,0.000000}%
\pgfsetstrokecolor{currentstroke}%
\pgfsetstrokeopacity{0.000000}%
\pgfsetdash{}{0pt}%
\pgfpathmoveto{\pgfqpoint{5.800000in}{82.783980in}}%
\pgfpathlineto{\pgfqpoint{5.800000in}{82.783980in}}%
\pgfpathlineto{\pgfqpoint{5.800000in}{82.886817in}}%
\pgfpathlineto{\pgfqpoint{5.800000in}{82.886817in}}%
\pgfpathclose%
\pgfusepath{fill}%
\end{pgfscope}%
\begin{pgfscope}%
\pgfpathrectangle{\pgfqpoint{5.800000in}{0.720000in}}{\pgfqpoint{1.400000in}{4.620000in}}%
\pgfusepath{clip}%
\pgfsetbuttcap%
\pgfsetmiterjoin%
\definecolor{currentfill}{rgb}{0.121569,0.466667,0.705882}%
\pgfsetfillcolor{currentfill}%
\pgfsetlinewidth{0.000000pt}%
\definecolor{currentstroke}{rgb}{0.000000,0.000000,0.000000}%
\pgfsetstrokecolor{currentstroke}%
\pgfsetstrokeopacity{0.000000}%
\pgfsetdash{}{0pt}%
\pgfpathmoveto{\pgfqpoint{5.800000in}{82.886817in}}%
\pgfpathlineto{\pgfqpoint{5.800000in}{82.886817in}}%
\pgfpathlineto{\pgfqpoint{5.800000in}{82.989654in}}%
\pgfpathlineto{\pgfqpoint{5.800000in}{82.989654in}}%
\pgfpathclose%
\pgfusepath{fill}%
\end{pgfscope}%
\begin{pgfscope}%
\pgfpathrectangle{\pgfqpoint{5.800000in}{0.720000in}}{\pgfqpoint{1.400000in}{4.620000in}}%
\pgfusepath{clip}%
\pgfsetbuttcap%
\pgfsetmiterjoin%
\definecolor{currentfill}{rgb}{0.121569,0.466667,0.705882}%
\pgfsetfillcolor{currentfill}%
\pgfsetlinewidth{0.000000pt}%
\definecolor{currentstroke}{rgb}{0.000000,0.000000,0.000000}%
\pgfsetstrokecolor{currentstroke}%
\pgfsetstrokeopacity{0.000000}%
\pgfsetdash{}{0pt}%
\pgfpathmoveto{\pgfqpoint{5.800000in}{82.989654in}}%
\pgfpathlineto{\pgfqpoint{5.800000in}{82.989654in}}%
\pgfpathlineto{\pgfqpoint{5.800000in}{83.092491in}}%
\pgfpathlineto{\pgfqpoint{5.800000in}{83.092491in}}%
\pgfpathclose%
\pgfusepath{fill}%
\end{pgfscope}%
\begin{pgfscope}%
\pgfpathrectangle{\pgfqpoint{5.800000in}{0.720000in}}{\pgfqpoint{1.400000in}{4.620000in}}%
\pgfusepath{clip}%
\pgfsetbuttcap%
\pgfsetmiterjoin%
\definecolor{currentfill}{rgb}{0.121569,0.466667,0.705882}%
\pgfsetfillcolor{currentfill}%
\pgfsetlinewidth{0.000000pt}%
\definecolor{currentstroke}{rgb}{0.000000,0.000000,0.000000}%
\pgfsetstrokecolor{currentstroke}%
\pgfsetstrokeopacity{0.000000}%
\pgfsetdash{}{0pt}%
\pgfpathmoveto{\pgfqpoint{5.800000in}{83.092491in}}%
\pgfpathlineto{\pgfqpoint{5.800000in}{83.092491in}}%
\pgfpathlineto{\pgfqpoint{5.800000in}{83.195328in}}%
\pgfpathlineto{\pgfqpoint{5.800000in}{83.195328in}}%
\pgfpathclose%
\pgfusepath{fill}%
\end{pgfscope}%
\begin{pgfscope}%
\pgfpathrectangle{\pgfqpoint{5.800000in}{0.720000in}}{\pgfqpoint{1.400000in}{4.620000in}}%
\pgfusepath{clip}%
\pgfsetbuttcap%
\pgfsetmiterjoin%
\definecolor{currentfill}{rgb}{0.121569,0.466667,0.705882}%
\pgfsetfillcolor{currentfill}%
\pgfsetlinewidth{0.000000pt}%
\definecolor{currentstroke}{rgb}{0.000000,0.000000,0.000000}%
\pgfsetstrokecolor{currentstroke}%
\pgfsetstrokeopacity{0.000000}%
\pgfsetdash{}{0pt}%
\pgfpathmoveto{\pgfqpoint{5.800000in}{83.195328in}}%
\pgfpathlineto{\pgfqpoint{5.800000in}{83.195328in}}%
\pgfpathlineto{\pgfqpoint{5.800000in}{83.298165in}}%
\pgfpathlineto{\pgfqpoint{5.800000in}{83.298165in}}%
\pgfpathclose%
\pgfusepath{fill}%
\end{pgfscope}%
\begin{pgfscope}%
\pgfpathrectangle{\pgfqpoint{5.800000in}{0.720000in}}{\pgfqpoint{1.400000in}{4.620000in}}%
\pgfusepath{clip}%
\pgfsetbuttcap%
\pgfsetmiterjoin%
\definecolor{currentfill}{rgb}{0.121569,0.466667,0.705882}%
\pgfsetfillcolor{currentfill}%
\pgfsetlinewidth{0.000000pt}%
\definecolor{currentstroke}{rgb}{0.000000,0.000000,0.000000}%
\pgfsetstrokecolor{currentstroke}%
\pgfsetstrokeopacity{0.000000}%
\pgfsetdash{}{0pt}%
\pgfpathmoveto{\pgfqpoint{5.800000in}{83.298165in}}%
\pgfpathlineto{\pgfqpoint{5.800000in}{83.298165in}}%
\pgfpathlineto{\pgfqpoint{5.800000in}{83.401002in}}%
\pgfpathlineto{\pgfqpoint{5.800000in}{83.401002in}}%
\pgfpathclose%
\pgfusepath{fill}%
\end{pgfscope}%
\begin{pgfscope}%
\pgfpathrectangle{\pgfqpoint{5.800000in}{0.720000in}}{\pgfqpoint{1.400000in}{4.620000in}}%
\pgfusepath{clip}%
\pgfsetbuttcap%
\pgfsetmiterjoin%
\definecolor{currentfill}{rgb}{0.121569,0.466667,0.705882}%
\pgfsetfillcolor{currentfill}%
\pgfsetlinewidth{0.000000pt}%
\definecolor{currentstroke}{rgb}{0.000000,0.000000,0.000000}%
\pgfsetstrokecolor{currentstroke}%
\pgfsetstrokeopacity{0.000000}%
\pgfsetdash{}{0pt}%
\pgfpathmoveto{\pgfqpoint{5.800000in}{83.401002in}}%
\pgfpathlineto{\pgfqpoint{5.800000in}{83.401002in}}%
\pgfpathlineto{\pgfqpoint{5.800000in}{83.503840in}}%
\pgfpathlineto{\pgfqpoint{5.800000in}{83.503840in}}%
\pgfpathclose%
\pgfusepath{fill}%
\end{pgfscope}%
\begin{pgfscope}%
\pgfpathrectangle{\pgfqpoint{5.800000in}{0.720000in}}{\pgfqpoint{1.400000in}{4.620000in}}%
\pgfusepath{clip}%
\pgfsetbuttcap%
\pgfsetmiterjoin%
\definecolor{currentfill}{rgb}{0.121569,0.466667,0.705882}%
\pgfsetfillcolor{currentfill}%
\pgfsetlinewidth{0.000000pt}%
\definecolor{currentstroke}{rgb}{0.000000,0.000000,0.000000}%
\pgfsetstrokecolor{currentstroke}%
\pgfsetstrokeopacity{0.000000}%
\pgfsetdash{}{0pt}%
\pgfpathmoveto{\pgfqpoint{5.800000in}{83.503840in}}%
\pgfpathlineto{\pgfqpoint{5.800000in}{83.503840in}}%
\pgfpathlineto{\pgfqpoint{5.800000in}{83.606677in}}%
\pgfpathlineto{\pgfqpoint{5.800000in}{83.606677in}}%
\pgfpathclose%
\pgfusepath{fill}%
\end{pgfscope}%
\begin{pgfscope}%
\pgfpathrectangle{\pgfqpoint{5.800000in}{0.720000in}}{\pgfqpoint{1.400000in}{4.620000in}}%
\pgfusepath{clip}%
\pgfsetbuttcap%
\pgfsetmiterjoin%
\definecolor{currentfill}{rgb}{0.121569,0.466667,0.705882}%
\pgfsetfillcolor{currentfill}%
\pgfsetlinewidth{0.000000pt}%
\definecolor{currentstroke}{rgb}{0.000000,0.000000,0.000000}%
\pgfsetstrokecolor{currentstroke}%
\pgfsetstrokeopacity{0.000000}%
\pgfsetdash{}{0pt}%
\pgfpathmoveto{\pgfqpoint{5.800000in}{83.606677in}}%
\pgfpathlineto{\pgfqpoint{5.800000in}{83.606677in}}%
\pgfpathlineto{\pgfqpoint{5.800000in}{83.709514in}}%
\pgfpathlineto{\pgfqpoint{5.800000in}{83.709514in}}%
\pgfpathclose%
\pgfusepath{fill}%
\end{pgfscope}%
\begin{pgfscope}%
\pgfpathrectangle{\pgfqpoint{5.800000in}{0.720000in}}{\pgfqpoint{1.400000in}{4.620000in}}%
\pgfusepath{clip}%
\pgfsetbuttcap%
\pgfsetmiterjoin%
\definecolor{currentfill}{rgb}{0.121569,0.466667,0.705882}%
\pgfsetfillcolor{currentfill}%
\pgfsetlinewidth{0.000000pt}%
\definecolor{currentstroke}{rgb}{0.000000,0.000000,0.000000}%
\pgfsetstrokecolor{currentstroke}%
\pgfsetstrokeopacity{0.000000}%
\pgfsetdash{}{0pt}%
\pgfpathmoveto{\pgfqpoint{5.800000in}{83.709514in}}%
\pgfpathlineto{\pgfqpoint{5.800000in}{83.709514in}}%
\pgfpathlineto{\pgfqpoint{5.800000in}{83.812351in}}%
\pgfpathlineto{\pgfqpoint{5.800000in}{83.812351in}}%
\pgfpathclose%
\pgfusepath{fill}%
\end{pgfscope}%
\begin{pgfscope}%
\pgfpathrectangle{\pgfqpoint{5.800000in}{0.720000in}}{\pgfqpoint{1.400000in}{4.620000in}}%
\pgfusepath{clip}%
\pgfsetbuttcap%
\pgfsetmiterjoin%
\definecolor{currentfill}{rgb}{0.121569,0.466667,0.705882}%
\pgfsetfillcolor{currentfill}%
\pgfsetlinewidth{0.000000pt}%
\definecolor{currentstroke}{rgb}{0.000000,0.000000,0.000000}%
\pgfsetstrokecolor{currentstroke}%
\pgfsetstrokeopacity{0.000000}%
\pgfsetdash{}{0pt}%
\pgfpathmoveto{\pgfqpoint{5.800000in}{83.812351in}}%
\pgfpathlineto{\pgfqpoint{5.800000in}{83.812351in}}%
\pgfpathlineto{\pgfqpoint{5.800000in}{83.915188in}}%
\pgfpathlineto{\pgfqpoint{5.800000in}{83.915188in}}%
\pgfpathclose%
\pgfusepath{fill}%
\end{pgfscope}%
\begin{pgfscope}%
\pgfpathrectangle{\pgfqpoint{5.800000in}{0.720000in}}{\pgfqpoint{1.400000in}{4.620000in}}%
\pgfusepath{clip}%
\pgfsetbuttcap%
\pgfsetmiterjoin%
\definecolor{currentfill}{rgb}{0.121569,0.466667,0.705882}%
\pgfsetfillcolor{currentfill}%
\pgfsetlinewidth{0.000000pt}%
\definecolor{currentstroke}{rgb}{0.000000,0.000000,0.000000}%
\pgfsetstrokecolor{currentstroke}%
\pgfsetstrokeopacity{0.000000}%
\pgfsetdash{}{0pt}%
\pgfpathmoveto{\pgfqpoint{5.800000in}{83.915188in}}%
\pgfpathlineto{\pgfqpoint{5.800000in}{83.915188in}}%
\pgfpathlineto{\pgfqpoint{5.800000in}{84.018025in}}%
\pgfpathlineto{\pgfqpoint{5.800000in}{84.018025in}}%
\pgfpathclose%
\pgfusepath{fill}%
\end{pgfscope}%
\begin{pgfscope}%
\pgfpathrectangle{\pgfqpoint{5.800000in}{0.720000in}}{\pgfqpoint{1.400000in}{4.620000in}}%
\pgfusepath{clip}%
\pgfsetbuttcap%
\pgfsetmiterjoin%
\definecolor{currentfill}{rgb}{0.121569,0.466667,0.705882}%
\pgfsetfillcolor{currentfill}%
\pgfsetlinewidth{0.000000pt}%
\definecolor{currentstroke}{rgb}{0.000000,0.000000,0.000000}%
\pgfsetstrokecolor{currentstroke}%
\pgfsetstrokeopacity{0.000000}%
\pgfsetdash{}{0pt}%
\pgfpathmoveto{\pgfqpoint{5.800000in}{84.018025in}}%
\pgfpathlineto{\pgfqpoint{5.800000in}{84.018025in}}%
\pgfpathlineto{\pgfqpoint{5.800000in}{84.120862in}}%
\pgfpathlineto{\pgfqpoint{5.800000in}{84.120862in}}%
\pgfpathclose%
\pgfusepath{fill}%
\end{pgfscope}%
\begin{pgfscope}%
\pgfpathrectangle{\pgfqpoint{5.800000in}{0.720000in}}{\pgfqpoint{1.400000in}{4.620000in}}%
\pgfusepath{clip}%
\pgfsetbuttcap%
\pgfsetmiterjoin%
\definecolor{currentfill}{rgb}{0.121569,0.466667,0.705882}%
\pgfsetfillcolor{currentfill}%
\pgfsetlinewidth{0.000000pt}%
\definecolor{currentstroke}{rgb}{0.000000,0.000000,0.000000}%
\pgfsetstrokecolor{currentstroke}%
\pgfsetstrokeopacity{0.000000}%
\pgfsetdash{}{0pt}%
\pgfpathmoveto{\pgfqpoint{5.800000in}{84.120862in}}%
\pgfpathlineto{\pgfqpoint{5.800000in}{84.120862in}}%
\pgfpathlineto{\pgfqpoint{5.800000in}{84.223699in}}%
\pgfpathlineto{\pgfqpoint{5.800000in}{84.223699in}}%
\pgfpathclose%
\pgfusepath{fill}%
\end{pgfscope}%
\begin{pgfscope}%
\pgfpathrectangle{\pgfqpoint{5.800000in}{0.720000in}}{\pgfqpoint{1.400000in}{4.620000in}}%
\pgfusepath{clip}%
\pgfsetbuttcap%
\pgfsetmiterjoin%
\definecolor{currentfill}{rgb}{0.121569,0.466667,0.705882}%
\pgfsetfillcolor{currentfill}%
\pgfsetlinewidth{0.000000pt}%
\definecolor{currentstroke}{rgb}{0.000000,0.000000,0.000000}%
\pgfsetstrokecolor{currentstroke}%
\pgfsetstrokeopacity{0.000000}%
\pgfsetdash{}{0pt}%
\pgfpathmoveto{\pgfqpoint{5.800000in}{84.223699in}}%
\pgfpathlineto{\pgfqpoint{5.800000in}{84.223699in}}%
\pgfpathlineto{\pgfqpoint{5.800000in}{84.326536in}}%
\pgfpathlineto{\pgfqpoint{5.800000in}{84.326536in}}%
\pgfpathclose%
\pgfusepath{fill}%
\end{pgfscope}%
\begin{pgfscope}%
\pgfpathrectangle{\pgfqpoint{5.800000in}{0.720000in}}{\pgfqpoint{1.400000in}{4.620000in}}%
\pgfusepath{clip}%
\pgfsetbuttcap%
\pgfsetmiterjoin%
\definecolor{currentfill}{rgb}{0.121569,0.466667,0.705882}%
\pgfsetfillcolor{currentfill}%
\pgfsetlinewidth{0.000000pt}%
\definecolor{currentstroke}{rgb}{0.000000,0.000000,0.000000}%
\pgfsetstrokecolor{currentstroke}%
\pgfsetstrokeopacity{0.000000}%
\pgfsetdash{}{0pt}%
\pgfpathmoveto{\pgfqpoint{5.800000in}{84.326536in}}%
\pgfpathlineto{\pgfqpoint{5.800000in}{84.326536in}}%
\pgfpathlineto{\pgfqpoint{5.800000in}{84.429373in}}%
\pgfpathlineto{\pgfqpoint{5.800000in}{84.429373in}}%
\pgfpathclose%
\pgfusepath{fill}%
\end{pgfscope}%
\begin{pgfscope}%
\pgfpathrectangle{\pgfqpoint{5.800000in}{0.720000in}}{\pgfqpoint{1.400000in}{4.620000in}}%
\pgfusepath{clip}%
\pgfsetbuttcap%
\pgfsetmiterjoin%
\definecolor{currentfill}{rgb}{0.121569,0.466667,0.705882}%
\pgfsetfillcolor{currentfill}%
\pgfsetlinewidth{0.000000pt}%
\definecolor{currentstroke}{rgb}{0.000000,0.000000,0.000000}%
\pgfsetstrokecolor{currentstroke}%
\pgfsetstrokeopacity{0.000000}%
\pgfsetdash{}{0pt}%
\pgfpathmoveto{\pgfqpoint{5.800000in}{84.429373in}}%
\pgfpathlineto{\pgfqpoint{5.800000in}{84.429373in}}%
\pgfpathlineto{\pgfqpoint{5.800000in}{84.532210in}}%
\pgfpathlineto{\pgfqpoint{5.800000in}{84.532210in}}%
\pgfpathclose%
\pgfusepath{fill}%
\end{pgfscope}%
\begin{pgfscope}%
\pgfpathrectangle{\pgfqpoint{5.800000in}{0.720000in}}{\pgfqpoint{1.400000in}{4.620000in}}%
\pgfusepath{clip}%
\pgfsetbuttcap%
\pgfsetmiterjoin%
\definecolor{currentfill}{rgb}{0.121569,0.466667,0.705882}%
\pgfsetfillcolor{currentfill}%
\pgfsetlinewidth{0.000000pt}%
\definecolor{currentstroke}{rgb}{0.000000,0.000000,0.000000}%
\pgfsetstrokecolor{currentstroke}%
\pgfsetstrokeopacity{0.000000}%
\pgfsetdash{}{0pt}%
\pgfpathmoveto{\pgfqpoint{5.800000in}{84.532210in}}%
\pgfpathlineto{\pgfqpoint{5.800000in}{84.532210in}}%
\pgfpathlineto{\pgfqpoint{5.800000in}{84.635047in}}%
\pgfpathlineto{\pgfqpoint{5.800000in}{84.635047in}}%
\pgfpathclose%
\pgfusepath{fill}%
\end{pgfscope}%
\begin{pgfscope}%
\pgfpathrectangle{\pgfqpoint{5.800000in}{0.720000in}}{\pgfqpoint{1.400000in}{4.620000in}}%
\pgfusepath{clip}%
\pgfsetbuttcap%
\pgfsetmiterjoin%
\definecolor{currentfill}{rgb}{0.121569,0.466667,0.705882}%
\pgfsetfillcolor{currentfill}%
\pgfsetlinewidth{0.000000pt}%
\definecolor{currentstroke}{rgb}{0.000000,0.000000,0.000000}%
\pgfsetstrokecolor{currentstroke}%
\pgfsetstrokeopacity{0.000000}%
\pgfsetdash{}{0pt}%
\pgfpathmoveto{\pgfqpoint{5.800000in}{84.635047in}}%
\pgfpathlineto{\pgfqpoint{5.800000in}{84.635047in}}%
\pgfpathlineto{\pgfqpoint{5.800000in}{84.737884in}}%
\pgfpathlineto{\pgfqpoint{5.800000in}{84.737884in}}%
\pgfpathclose%
\pgfusepath{fill}%
\end{pgfscope}%
\begin{pgfscope}%
\pgfpathrectangle{\pgfqpoint{5.800000in}{0.720000in}}{\pgfqpoint{1.400000in}{4.620000in}}%
\pgfusepath{clip}%
\pgfsetbuttcap%
\pgfsetmiterjoin%
\definecolor{currentfill}{rgb}{0.121569,0.466667,0.705882}%
\pgfsetfillcolor{currentfill}%
\pgfsetlinewidth{0.000000pt}%
\definecolor{currentstroke}{rgb}{0.000000,0.000000,0.000000}%
\pgfsetstrokecolor{currentstroke}%
\pgfsetstrokeopacity{0.000000}%
\pgfsetdash{}{0pt}%
\pgfpathmoveto{\pgfqpoint{5.800000in}{84.737884in}}%
\pgfpathlineto{\pgfqpoint{5.800000in}{84.737884in}}%
\pgfpathlineto{\pgfqpoint{5.800000in}{84.840721in}}%
\pgfpathlineto{\pgfqpoint{5.800000in}{84.840721in}}%
\pgfpathclose%
\pgfusepath{fill}%
\end{pgfscope}%
\begin{pgfscope}%
\pgfpathrectangle{\pgfqpoint{5.800000in}{0.720000in}}{\pgfqpoint{1.400000in}{4.620000in}}%
\pgfusepath{clip}%
\pgfsetbuttcap%
\pgfsetmiterjoin%
\definecolor{currentfill}{rgb}{0.121569,0.466667,0.705882}%
\pgfsetfillcolor{currentfill}%
\pgfsetlinewidth{0.000000pt}%
\definecolor{currentstroke}{rgb}{0.000000,0.000000,0.000000}%
\pgfsetstrokecolor{currentstroke}%
\pgfsetstrokeopacity{0.000000}%
\pgfsetdash{}{0pt}%
\pgfpathmoveto{\pgfqpoint{5.800000in}{84.840721in}}%
\pgfpathlineto{\pgfqpoint{5.800000in}{84.840721in}}%
\pgfpathlineto{\pgfqpoint{5.800000in}{84.943559in}}%
\pgfpathlineto{\pgfqpoint{5.800000in}{84.943559in}}%
\pgfpathclose%
\pgfusepath{fill}%
\end{pgfscope}%
\begin{pgfscope}%
\pgfpathrectangle{\pgfqpoint{5.800000in}{0.720000in}}{\pgfqpoint{1.400000in}{4.620000in}}%
\pgfusepath{clip}%
\pgfsetbuttcap%
\pgfsetmiterjoin%
\definecolor{currentfill}{rgb}{0.121569,0.466667,0.705882}%
\pgfsetfillcolor{currentfill}%
\pgfsetlinewidth{0.000000pt}%
\definecolor{currentstroke}{rgb}{0.000000,0.000000,0.000000}%
\pgfsetstrokecolor{currentstroke}%
\pgfsetstrokeopacity{0.000000}%
\pgfsetdash{}{0pt}%
\pgfpathmoveto{\pgfqpoint{5.800000in}{84.943559in}}%
\pgfpathlineto{\pgfqpoint{5.800000in}{84.943559in}}%
\pgfpathlineto{\pgfqpoint{5.800000in}{85.046396in}}%
\pgfpathlineto{\pgfqpoint{5.800000in}{85.046396in}}%
\pgfpathclose%
\pgfusepath{fill}%
\end{pgfscope}%
\begin{pgfscope}%
\pgfpathrectangle{\pgfqpoint{5.800000in}{0.720000in}}{\pgfqpoint{1.400000in}{4.620000in}}%
\pgfusepath{clip}%
\pgfsetbuttcap%
\pgfsetmiterjoin%
\definecolor{currentfill}{rgb}{0.121569,0.466667,0.705882}%
\pgfsetfillcolor{currentfill}%
\pgfsetlinewidth{0.000000pt}%
\definecolor{currentstroke}{rgb}{0.000000,0.000000,0.000000}%
\pgfsetstrokecolor{currentstroke}%
\pgfsetstrokeopacity{0.000000}%
\pgfsetdash{}{0pt}%
\pgfpathmoveto{\pgfqpoint{5.800000in}{85.046396in}}%
\pgfpathlineto{\pgfqpoint{5.800000in}{85.046396in}}%
\pgfpathlineto{\pgfqpoint{5.800000in}{85.149233in}}%
\pgfpathlineto{\pgfqpoint{5.800000in}{85.149233in}}%
\pgfpathclose%
\pgfusepath{fill}%
\end{pgfscope}%
\begin{pgfscope}%
\pgfpathrectangle{\pgfqpoint{5.800000in}{0.720000in}}{\pgfqpoint{1.400000in}{4.620000in}}%
\pgfusepath{clip}%
\pgfsetbuttcap%
\pgfsetmiterjoin%
\definecolor{currentfill}{rgb}{0.121569,0.466667,0.705882}%
\pgfsetfillcolor{currentfill}%
\pgfsetlinewidth{0.000000pt}%
\definecolor{currentstroke}{rgb}{0.000000,0.000000,0.000000}%
\pgfsetstrokecolor{currentstroke}%
\pgfsetstrokeopacity{0.000000}%
\pgfsetdash{}{0pt}%
\pgfpathmoveto{\pgfqpoint{5.800000in}{85.149233in}}%
\pgfpathlineto{\pgfqpoint{5.800000in}{85.149233in}}%
\pgfpathlineto{\pgfqpoint{5.800000in}{85.252070in}}%
\pgfpathlineto{\pgfqpoint{5.800000in}{85.252070in}}%
\pgfpathclose%
\pgfusepath{fill}%
\end{pgfscope}%
\begin{pgfscope}%
\pgfpathrectangle{\pgfqpoint{5.800000in}{0.720000in}}{\pgfqpoint{1.400000in}{4.620000in}}%
\pgfusepath{clip}%
\pgfsetbuttcap%
\pgfsetmiterjoin%
\definecolor{currentfill}{rgb}{0.121569,0.466667,0.705882}%
\pgfsetfillcolor{currentfill}%
\pgfsetlinewidth{0.000000pt}%
\definecolor{currentstroke}{rgb}{0.000000,0.000000,0.000000}%
\pgfsetstrokecolor{currentstroke}%
\pgfsetstrokeopacity{0.000000}%
\pgfsetdash{}{0pt}%
\pgfpathmoveto{\pgfqpoint{5.800000in}{85.252070in}}%
\pgfpathlineto{\pgfqpoint{5.800000in}{85.252070in}}%
\pgfpathlineto{\pgfqpoint{5.800000in}{85.354907in}}%
\pgfpathlineto{\pgfqpoint{5.800000in}{85.354907in}}%
\pgfpathclose%
\pgfusepath{fill}%
\end{pgfscope}%
\begin{pgfscope}%
\pgfpathrectangle{\pgfqpoint{5.800000in}{0.720000in}}{\pgfqpoint{1.400000in}{4.620000in}}%
\pgfusepath{clip}%
\pgfsetbuttcap%
\pgfsetmiterjoin%
\definecolor{currentfill}{rgb}{0.121569,0.466667,0.705882}%
\pgfsetfillcolor{currentfill}%
\pgfsetlinewidth{0.000000pt}%
\definecolor{currentstroke}{rgb}{0.000000,0.000000,0.000000}%
\pgfsetstrokecolor{currentstroke}%
\pgfsetstrokeopacity{0.000000}%
\pgfsetdash{}{0pt}%
\pgfpathmoveto{\pgfqpoint{5.800000in}{85.354907in}}%
\pgfpathlineto{\pgfqpoint{5.800000in}{85.354907in}}%
\pgfpathlineto{\pgfqpoint{5.800000in}{85.457744in}}%
\pgfpathlineto{\pgfqpoint{5.800000in}{85.457744in}}%
\pgfpathclose%
\pgfusepath{fill}%
\end{pgfscope}%
\begin{pgfscope}%
\pgfpathrectangle{\pgfqpoint{5.800000in}{0.720000in}}{\pgfqpoint{1.400000in}{4.620000in}}%
\pgfusepath{clip}%
\pgfsetbuttcap%
\pgfsetmiterjoin%
\definecolor{currentfill}{rgb}{0.121569,0.466667,0.705882}%
\pgfsetfillcolor{currentfill}%
\pgfsetlinewidth{0.000000pt}%
\definecolor{currentstroke}{rgb}{0.000000,0.000000,0.000000}%
\pgfsetstrokecolor{currentstroke}%
\pgfsetstrokeopacity{0.000000}%
\pgfsetdash{}{0pt}%
\pgfpathmoveto{\pgfqpoint{5.800000in}{85.457744in}}%
\pgfpathlineto{\pgfqpoint{5.800000in}{85.457744in}}%
\pgfpathlineto{\pgfqpoint{5.800000in}{85.560581in}}%
\pgfpathlineto{\pgfqpoint{5.800000in}{85.560581in}}%
\pgfpathclose%
\pgfusepath{fill}%
\end{pgfscope}%
\begin{pgfscope}%
\pgfpathrectangle{\pgfqpoint{5.800000in}{0.720000in}}{\pgfqpoint{1.400000in}{4.620000in}}%
\pgfusepath{clip}%
\pgfsetbuttcap%
\pgfsetmiterjoin%
\definecolor{currentfill}{rgb}{0.121569,0.466667,0.705882}%
\pgfsetfillcolor{currentfill}%
\pgfsetlinewidth{0.000000pt}%
\definecolor{currentstroke}{rgb}{0.000000,0.000000,0.000000}%
\pgfsetstrokecolor{currentstroke}%
\pgfsetstrokeopacity{0.000000}%
\pgfsetdash{}{0pt}%
\pgfpathmoveto{\pgfqpoint{5.800000in}{85.560581in}}%
\pgfpathlineto{\pgfqpoint{5.800000in}{85.560581in}}%
\pgfpathlineto{\pgfqpoint{5.800000in}{85.663418in}}%
\pgfpathlineto{\pgfqpoint{5.800000in}{85.663418in}}%
\pgfpathclose%
\pgfusepath{fill}%
\end{pgfscope}%
\begin{pgfscope}%
\pgfpathrectangle{\pgfqpoint{5.800000in}{0.720000in}}{\pgfqpoint{1.400000in}{4.620000in}}%
\pgfusepath{clip}%
\pgfsetbuttcap%
\pgfsetmiterjoin%
\definecolor{currentfill}{rgb}{0.121569,0.466667,0.705882}%
\pgfsetfillcolor{currentfill}%
\pgfsetlinewidth{0.000000pt}%
\definecolor{currentstroke}{rgb}{0.000000,0.000000,0.000000}%
\pgfsetstrokecolor{currentstroke}%
\pgfsetstrokeopacity{0.000000}%
\pgfsetdash{}{0pt}%
\pgfpathmoveto{\pgfqpoint{5.800000in}{85.663418in}}%
\pgfpathlineto{\pgfqpoint{5.800000in}{85.663418in}}%
\pgfpathlineto{\pgfqpoint{5.800000in}{85.766255in}}%
\pgfpathlineto{\pgfqpoint{5.800000in}{85.766255in}}%
\pgfpathclose%
\pgfusepath{fill}%
\end{pgfscope}%
\begin{pgfscope}%
\pgfpathrectangle{\pgfqpoint{5.800000in}{0.720000in}}{\pgfqpoint{1.400000in}{4.620000in}}%
\pgfusepath{clip}%
\pgfsetbuttcap%
\pgfsetmiterjoin%
\definecolor{currentfill}{rgb}{0.121569,0.466667,0.705882}%
\pgfsetfillcolor{currentfill}%
\pgfsetlinewidth{0.000000pt}%
\definecolor{currentstroke}{rgb}{0.000000,0.000000,0.000000}%
\pgfsetstrokecolor{currentstroke}%
\pgfsetstrokeopacity{0.000000}%
\pgfsetdash{}{0pt}%
\pgfpathmoveto{\pgfqpoint{5.800000in}{85.766255in}}%
\pgfpathlineto{\pgfqpoint{5.800000in}{85.766255in}}%
\pgfpathlineto{\pgfqpoint{5.800000in}{85.869092in}}%
\pgfpathlineto{\pgfqpoint{5.800000in}{85.869092in}}%
\pgfpathclose%
\pgfusepath{fill}%
\end{pgfscope}%
\begin{pgfscope}%
\pgfpathrectangle{\pgfqpoint{5.800000in}{0.720000in}}{\pgfqpoint{1.400000in}{4.620000in}}%
\pgfusepath{clip}%
\pgfsetbuttcap%
\pgfsetmiterjoin%
\definecolor{currentfill}{rgb}{0.121569,0.466667,0.705882}%
\pgfsetfillcolor{currentfill}%
\pgfsetlinewidth{0.000000pt}%
\definecolor{currentstroke}{rgb}{0.000000,0.000000,0.000000}%
\pgfsetstrokecolor{currentstroke}%
\pgfsetstrokeopacity{0.000000}%
\pgfsetdash{}{0pt}%
\pgfpathmoveto{\pgfqpoint{5.800000in}{85.869092in}}%
\pgfpathlineto{\pgfqpoint{5.800000in}{85.869092in}}%
\pgfpathlineto{\pgfqpoint{5.800000in}{85.971929in}}%
\pgfpathlineto{\pgfqpoint{5.800000in}{85.971929in}}%
\pgfpathclose%
\pgfusepath{fill}%
\end{pgfscope}%
\begin{pgfscope}%
\pgfpathrectangle{\pgfqpoint{5.800000in}{0.720000in}}{\pgfqpoint{1.400000in}{4.620000in}}%
\pgfusepath{clip}%
\pgfsetbuttcap%
\pgfsetmiterjoin%
\definecolor{currentfill}{rgb}{0.121569,0.466667,0.705882}%
\pgfsetfillcolor{currentfill}%
\pgfsetlinewidth{0.000000pt}%
\definecolor{currentstroke}{rgb}{0.000000,0.000000,0.000000}%
\pgfsetstrokecolor{currentstroke}%
\pgfsetstrokeopacity{0.000000}%
\pgfsetdash{}{0pt}%
\pgfpathmoveto{\pgfqpoint{5.800000in}{85.971929in}}%
\pgfpathlineto{\pgfqpoint{5.800000in}{85.971929in}}%
\pgfpathlineto{\pgfqpoint{5.800000in}{86.074766in}}%
\pgfpathlineto{\pgfqpoint{5.800000in}{86.074766in}}%
\pgfpathclose%
\pgfusepath{fill}%
\end{pgfscope}%
\begin{pgfscope}%
\pgfpathrectangle{\pgfqpoint{5.800000in}{0.720000in}}{\pgfqpoint{1.400000in}{4.620000in}}%
\pgfusepath{clip}%
\pgfsetbuttcap%
\pgfsetmiterjoin%
\definecolor{currentfill}{rgb}{0.121569,0.466667,0.705882}%
\pgfsetfillcolor{currentfill}%
\pgfsetlinewidth{0.000000pt}%
\definecolor{currentstroke}{rgb}{0.000000,0.000000,0.000000}%
\pgfsetstrokecolor{currentstroke}%
\pgfsetstrokeopacity{0.000000}%
\pgfsetdash{}{0pt}%
\pgfpathmoveto{\pgfqpoint{5.800000in}{86.074766in}}%
\pgfpathlineto{\pgfqpoint{5.800000in}{86.074766in}}%
\pgfpathlineto{\pgfqpoint{5.800000in}{86.177603in}}%
\pgfpathlineto{\pgfqpoint{5.800000in}{86.177603in}}%
\pgfpathclose%
\pgfusepath{fill}%
\end{pgfscope}%
\begin{pgfscope}%
\pgfpathrectangle{\pgfqpoint{5.800000in}{0.720000in}}{\pgfqpoint{1.400000in}{4.620000in}}%
\pgfusepath{clip}%
\pgfsetbuttcap%
\pgfsetmiterjoin%
\definecolor{currentfill}{rgb}{0.121569,0.466667,0.705882}%
\pgfsetfillcolor{currentfill}%
\pgfsetlinewidth{0.000000pt}%
\definecolor{currentstroke}{rgb}{0.000000,0.000000,0.000000}%
\pgfsetstrokecolor{currentstroke}%
\pgfsetstrokeopacity{0.000000}%
\pgfsetdash{}{0pt}%
\pgfpathmoveto{\pgfqpoint{5.800000in}{86.177603in}}%
\pgfpathlineto{\pgfqpoint{5.800000in}{86.177603in}}%
\pgfpathlineto{\pgfqpoint{5.800000in}{86.280440in}}%
\pgfpathlineto{\pgfqpoint{5.800000in}{86.280440in}}%
\pgfpathclose%
\pgfusepath{fill}%
\end{pgfscope}%
\begin{pgfscope}%
\pgfpathrectangle{\pgfqpoint{5.800000in}{0.720000in}}{\pgfqpoint{1.400000in}{4.620000in}}%
\pgfusepath{clip}%
\pgfsetbuttcap%
\pgfsetmiterjoin%
\definecolor{currentfill}{rgb}{0.121569,0.466667,0.705882}%
\pgfsetfillcolor{currentfill}%
\pgfsetlinewidth{0.000000pt}%
\definecolor{currentstroke}{rgb}{0.000000,0.000000,0.000000}%
\pgfsetstrokecolor{currentstroke}%
\pgfsetstrokeopacity{0.000000}%
\pgfsetdash{}{0pt}%
\pgfpathmoveto{\pgfqpoint{5.800000in}{86.280440in}}%
\pgfpathlineto{\pgfqpoint{5.800000in}{86.280440in}}%
\pgfpathlineto{\pgfqpoint{5.800000in}{86.383277in}}%
\pgfpathlineto{\pgfqpoint{5.800000in}{86.383277in}}%
\pgfpathclose%
\pgfusepath{fill}%
\end{pgfscope}%
\begin{pgfscope}%
\pgfpathrectangle{\pgfqpoint{5.800000in}{0.720000in}}{\pgfqpoint{1.400000in}{4.620000in}}%
\pgfusepath{clip}%
\pgfsetbuttcap%
\pgfsetmiterjoin%
\definecolor{currentfill}{rgb}{0.121569,0.466667,0.705882}%
\pgfsetfillcolor{currentfill}%
\pgfsetlinewidth{0.000000pt}%
\definecolor{currentstroke}{rgb}{0.000000,0.000000,0.000000}%
\pgfsetstrokecolor{currentstroke}%
\pgfsetstrokeopacity{0.000000}%
\pgfsetdash{}{0pt}%
\pgfpathmoveto{\pgfqpoint{5.800000in}{86.383277in}}%
\pgfpathlineto{\pgfqpoint{5.800000in}{86.383277in}}%
\pgfpathlineto{\pgfqpoint{5.800000in}{86.486115in}}%
\pgfpathlineto{\pgfqpoint{5.800000in}{86.486115in}}%
\pgfpathclose%
\pgfusepath{fill}%
\end{pgfscope}%
\begin{pgfscope}%
\pgfpathrectangle{\pgfqpoint{5.800000in}{0.720000in}}{\pgfqpoint{1.400000in}{4.620000in}}%
\pgfusepath{clip}%
\pgfsetbuttcap%
\pgfsetmiterjoin%
\definecolor{currentfill}{rgb}{0.121569,0.466667,0.705882}%
\pgfsetfillcolor{currentfill}%
\pgfsetlinewidth{0.000000pt}%
\definecolor{currentstroke}{rgb}{0.000000,0.000000,0.000000}%
\pgfsetstrokecolor{currentstroke}%
\pgfsetstrokeopacity{0.000000}%
\pgfsetdash{}{0pt}%
\pgfpathmoveto{\pgfqpoint{5.800000in}{86.486115in}}%
\pgfpathlineto{\pgfqpoint{5.800000in}{86.486115in}}%
\pgfpathlineto{\pgfqpoint{5.800000in}{86.588952in}}%
\pgfpathlineto{\pgfqpoint{5.800000in}{86.588952in}}%
\pgfpathclose%
\pgfusepath{fill}%
\end{pgfscope}%
\begin{pgfscope}%
\pgfpathrectangle{\pgfqpoint{5.800000in}{0.720000in}}{\pgfqpoint{1.400000in}{4.620000in}}%
\pgfusepath{clip}%
\pgfsetbuttcap%
\pgfsetmiterjoin%
\definecolor{currentfill}{rgb}{0.121569,0.466667,0.705882}%
\pgfsetfillcolor{currentfill}%
\pgfsetlinewidth{0.000000pt}%
\definecolor{currentstroke}{rgb}{0.000000,0.000000,0.000000}%
\pgfsetstrokecolor{currentstroke}%
\pgfsetstrokeopacity{0.000000}%
\pgfsetdash{}{0pt}%
\pgfpathmoveto{\pgfqpoint{5.800000in}{86.588952in}}%
\pgfpathlineto{\pgfqpoint{5.800000in}{86.588952in}}%
\pgfpathlineto{\pgfqpoint{5.800000in}{86.691789in}}%
\pgfpathlineto{\pgfqpoint{5.800000in}{86.691789in}}%
\pgfpathclose%
\pgfusepath{fill}%
\end{pgfscope}%
\begin{pgfscope}%
\pgfpathrectangle{\pgfqpoint{5.800000in}{0.720000in}}{\pgfqpoint{1.400000in}{4.620000in}}%
\pgfusepath{clip}%
\pgfsetbuttcap%
\pgfsetmiterjoin%
\definecolor{currentfill}{rgb}{0.121569,0.466667,0.705882}%
\pgfsetfillcolor{currentfill}%
\pgfsetlinewidth{0.000000pt}%
\definecolor{currentstroke}{rgb}{0.000000,0.000000,0.000000}%
\pgfsetstrokecolor{currentstroke}%
\pgfsetstrokeopacity{0.000000}%
\pgfsetdash{}{0pt}%
\pgfpathmoveto{\pgfqpoint{5.800000in}{86.691789in}}%
\pgfpathlineto{\pgfqpoint{5.800000in}{86.691789in}}%
\pgfpathlineto{\pgfqpoint{5.800000in}{86.794626in}}%
\pgfpathlineto{\pgfqpoint{5.800000in}{86.794626in}}%
\pgfpathclose%
\pgfusepath{fill}%
\end{pgfscope}%
\begin{pgfscope}%
\pgfpathrectangle{\pgfqpoint{5.800000in}{0.720000in}}{\pgfqpoint{1.400000in}{4.620000in}}%
\pgfusepath{clip}%
\pgfsetbuttcap%
\pgfsetmiterjoin%
\definecolor{currentfill}{rgb}{0.121569,0.466667,0.705882}%
\pgfsetfillcolor{currentfill}%
\pgfsetlinewidth{0.000000pt}%
\definecolor{currentstroke}{rgb}{0.000000,0.000000,0.000000}%
\pgfsetstrokecolor{currentstroke}%
\pgfsetstrokeopacity{0.000000}%
\pgfsetdash{}{0pt}%
\pgfpathmoveto{\pgfqpoint{5.800000in}{86.794626in}}%
\pgfpathlineto{\pgfqpoint{5.800000in}{86.794626in}}%
\pgfpathlineto{\pgfqpoint{5.800000in}{86.897463in}}%
\pgfpathlineto{\pgfqpoint{5.800000in}{86.897463in}}%
\pgfpathclose%
\pgfusepath{fill}%
\end{pgfscope}%
\begin{pgfscope}%
\pgfpathrectangle{\pgfqpoint{5.800000in}{0.720000in}}{\pgfqpoint{1.400000in}{4.620000in}}%
\pgfusepath{clip}%
\pgfsetbuttcap%
\pgfsetmiterjoin%
\definecolor{currentfill}{rgb}{0.121569,0.466667,0.705882}%
\pgfsetfillcolor{currentfill}%
\pgfsetlinewidth{0.000000pt}%
\definecolor{currentstroke}{rgb}{0.000000,0.000000,0.000000}%
\pgfsetstrokecolor{currentstroke}%
\pgfsetstrokeopacity{0.000000}%
\pgfsetdash{}{0pt}%
\pgfpathmoveto{\pgfqpoint{5.800000in}{86.897463in}}%
\pgfpathlineto{\pgfqpoint{5.800000in}{86.897463in}}%
\pgfpathlineto{\pgfqpoint{5.800000in}{87.000300in}}%
\pgfpathlineto{\pgfqpoint{5.800000in}{87.000300in}}%
\pgfpathclose%
\pgfusepath{fill}%
\end{pgfscope}%
\begin{pgfscope}%
\pgfpathrectangle{\pgfqpoint{5.800000in}{0.720000in}}{\pgfqpoint{1.400000in}{4.620000in}}%
\pgfusepath{clip}%
\pgfsetbuttcap%
\pgfsetmiterjoin%
\definecolor{currentfill}{rgb}{0.121569,0.466667,0.705882}%
\pgfsetfillcolor{currentfill}%
\pgfsetlinewidth{0.000000pt}%
\definecolor{currentstroke}{rgb}{0.000000,0.000000,0.000000}%
\pgfsetstrokecolor{currentstroke}%
\pgfsetstrokeopacity{0.000000}%
\pgfsetdash{}{0pt}%
\pgfpathmoveto{\pgfqpoint{5.800000in}{87.000300in}}%
\pgfpathlineto{\pgfqpoint{5.800000in}{87.000300in}}%
\pgfpathlineto{\pgfqpoint{5.800000in}{87.103137in}}%
\pgfpathlineto{\pgfqpoint{5.800000in}{87.103137in}}%
\pgfpathclose%
\pgfusepath{fill}%
\end{pgfscope}%
\begin{pgfscope}%
\pgfpathrectangle{\pgfqpoint{5.800000in}{0.720000in}}{\pgfqpoint{1.400000in}{4.620000in}}%
\pgfusepath{clip}%
\pgfsetbuttcap%
\pgfsetmiterjoin%
\definecolor{currentfill}{rgb}{0.121569,0.466667,0.705882}%
\pgfsetfillcolor{currentfill}%
\pgfsetlinewidth{0.000000pt}%
\definecolor{currentstroke}{rgb}{0.000000,0.000000,0.000000}%
\pgfsetstrokecolor{currentstroke}%
\pgfsetstrokeopacity{0.000000}%
\pgfsetdash{}{0pt}%
\pgfpathmoveto{\pgfqpoint{5.800000in}{87.103137in}}%
\pgfpathlineto{\pgfqpoint{5.800000in}{87.103137in}}%
\pgfpathlineto{\pgfqpoint{5.800000in}{87.205974in}}%
\pgfpathlineto{\pgfqpoint{5.800000in}{87.205974in}}%
\pgfpathclose%
\pgfusepath{fill}%
\end{pgfscope}%
\begin{pgfscope}%
\pgfpathrectangle{\pgfqpoint{5.800000in}{0.720000in}}{\pgfqpoint{1.400000in}{4.620000in}}%
\pgfusepath{clip}%
\pgfsetbuttcap%
\pgfsetmiterjoin%
\definecolor{currentfill}{rgb}{0.121569,0.466667,0.705882}%
\pgfsetfillcolor{currentfill}%
\pgfsetlinewidth{0.000000pt}%
\definecolor{currentstroke}{rgb}{0.000000,0.000000,0.000000}%
\pgfsetstrokecolor{currentstroke}%
\pgfsetstrokeopacity{0.000000}%
\pgfsetdash{}{0pt}%
\pgfpathmoveto{\pgfqpoint{5.800000in}{87.205974in}}%
\pgfpathlineto{\pgfqpoint{5.800000in}{87.205974in}}%
\pgfpathlineto{\pgfqpoint{5.800000in}{87.308811in}}%
\pgfpathlineto{\pgfqpoint{5.800000in}{87.308811in}}%
\pgfpathclose%
\pgfusepath{fill}%
\end{pgfscope}%
\begin{pgfscope}%
\pgfpathrectangle{\pgfqpoint{5.800000in}{0.720000in}}{\pgfqpoint{1.400000in}{4.620000in}}%
\pgfusepath{clip}%
\pgfsetbuttcap%
\pgfsetmiterjoin%
\definecolor{currentfill}{rgb}{0.121569,0.466667,0.705882}%
\pgfsetfillcolor{currentfill}%
\pgfsetlinewidth{0.000000pt}%
\definecolor{currentstroke}{rgb}{0.000000,0.000000,0.000000}%
\pgfsetstrokecolor{currentstroke}%
\pgfsetstrokeopacity{0.000000}%
\pgfsetdash{}{0pt}%
\pgfpathmoveto{\pgfqpoint{5.800000in}{87.308811in}}%
\pgfpathlineto{\pgfqpoint{5.800000in}{87.308811in}}%
\pgfpathlineto{\pgfqpoint{5.800000in}{87.411648in}}%
\pgfpathlineto{\pgfqpoint{5.800000in}{87.411648in}}%
\pgfpathclose%
\pgfusepath{fill}%
\end{pgfscope}%
\begin{pgfscope}%
\pgfpathrectangle{\pgfqpoint{5.800000in}{0.720000in}}{\pgfqpoint{1.400000in}{4.620000in}}%
\pgfusepath{clip}%
\pgfsetbuttcap%
\pgfsetmiterjoin%
\definecolor{currentfill}{rgb}{0.121569,0.466667,0.705882}%
\pgfsetfillcolor{currentfill}%
\pgfsetlinewidth{0.000000pt}%
\definecolor{currentstroke}{rgb}{0.000000,0.000000,0.000000}%
\pgfsetstrokecolor{currentstroke}%
\pgfsetstrokeopacity{0.000000}%
\pgfsetdash{}{0pt}%
\pgfpathmoveto{\pgfqpoint{5.800000in}{87.411648in}}%
\pgfpathlineto{\pgfqpoint{5.800000in}{87.411648in}}%
\pgfpathlineto{\pgfqpoint{5.800000in}{87.514485in}}%
\pgfpathlineto{\pgfqpoint{5.800000in}{87.514485in}}%
\pgfpathclose%
\pgfusepath{fill}%
\end{pgfscope}%
\begin{pgfscope}%
\pgfpathrectangle{\pgfqpoint{5.800000in}{0.720000in}}{\pgfqpoint{1.400000in}{4.620000in}}%
\pgfusepath{clip}%
\pgfsetbuttcap%
\pgfsetmiterjoin%
\definecolor{currentfill}{rgb}{0.121569,0.466667,0.705882}%
\pgfsetfillcolor{currentfill}%
\pgfsetlinewidth{0.000000pt}%
\definecolor{currentstroke}{rgb}{0.000000,0.000000,0.000000}%
\pgfsetstrokecolor{currentstroke}%
\pgfsetstrokeopacity{0.000000}%
\pgfsetdash{}{0pt}%
\pgfpathmoveto{\pgfqpoint{5.800000in}{87.514485in}}%
\pgfpathlineto{\pgfqpoint{5.800000in}{87.514485in}}%
\pgfpathlineto{\pgfqpoint{5.800000in}{87.617322in}}%
\pgfpathlineto{\pgfqpoint{5.800000in}{87.617322in}}%
\pgfpathclose%
\pgfusepath{fill}%
\end{pgfscope}%
\begin{pgfscope}%
\pgfpathrectangle{\pgfqpoint{5.800000in}{0.720000in}}{\pgfqpoint{1.400000in}{4.620000in}}%
\pgfusepath{clip}%
\pgfsetbuttcap%
\pgfsetmiterjoin%
\definecolor{currentfill}{rgb}{0.121569,0.466667,0.705882}%
\pgfsetfillcolor{currentfill}%
\pgfsetlinewidth{0.000000pt}%
\definecolor{currentstroke}{rgb}{0.000000,0.000000,0.000000}%
\pgfsetstrokecolor{currentstroke}%
\pgfsetstrokeopacity{0.000000}%
\pgfsetdash{}{0pt}%
\pgfpathmoveto{\pgfqpoint{5.800000in}{87.617322in}}%
\pgfpathlineto{\pgfqpoint{5.800000in}{87.617322in}}%
\pgfpathlineto{\pgfqpoint{5.800000in}{87.720159in}}%
\pgfpathlineto{\pgfqpoint{5.800000in}{87.720159in}}%
\pgfpathclose%
\pgfusepath{fill}%
\end{pgfscope}%
\begin{pgfscope}%
\pgfpathrectangle{\pgfqpoint{5.800000in}{0.720000in}}{\pgfqpoint{1.400000in}{4.620000in}}%
\pgfusepath{clip}%
\pgfsetbuttcap%
\pgfsetmiterjoin%
\definecolor{currentfill}{rgb}{0.121569,0.466667,0.705882}%
\pgfsetfillcolor{currentfill}%
\pgfsetlinewidth{0.000000pt}%
\definecolor{currentstroke}{rgb}{0.000000,0.000000,0.000000}%
\pgfsetstrokecolor{currentstroke}%
\pgfsetstrokeopacity{0.000000}%
\pgfsetdash{}{0pt}%
\pgfpathmoveto{\pgfqpoint{5.800000in}{87.720159in}}%
\pgfpathlineto{\pgfqpoint{5.800000in}{87.720159in}}%
\pgfpathlineto{\pgfqpoint{5.800000in}{87.822996in}}%
\pgfpathlineto{\pgfqpoint{5.800000in}{87.822996in}}%
\pgfpathclose%
\pgfusepath{fill}%
\end{pgfscope}%
\begin{pgfscope}%
\pgfpathrectangle{\pgfqpoint{5.800000in}{0.720000in}}{\pgfqpoint{1.400000in}{4.620000in}}%
\pgfusepath{clip}%
\pgfsetbuttcap%
\pgfsetmiterjoin%
\definecolor{currentfill}{rgb}{0.121569,0.466667,0.705882}%
\pgfsetfillcolor{currentfill}%
\pgfsetlinewidth{0.000000pt}%
\definecolor{currentstroke}{rgb}{0.000000,0.000000,0.000000}%
\pgfsetstrokecolor{currentstroke}%
\pgfsetstrokeopacity{0.000000}%
\pgfsetdash{}{0pt}%
\pgfpathmoveto{\pgfqpoint{5.800000in}{87.822996in}}%
\pgfpathlineto{\pgfqpoint{5.800000in}{87.822996in}}%
\pgfpathlineto{\pgfqpoint{5.800000in}{87.925833in}}%
\pgfpathlineto{\pgfqpoint{5.800000in}{87.925833in}}%
\pgfpathclose%
\pgfusepath{fill}%
\end{pgfscope}%
\begin{pgfscope}%
\pgfpathrectangle{\pgfqpoint{5.800000in}{0.720000in}}{\pgfqpoint{1.400000in}{4.620000in}}%
\pgfusepath{clip}%
\pgfsetbuttcap%
\pgfsetmiterjoin%
\definecolor{currentfill}{rgb}{0.121569,0.466667,0.705882}%
\pgfsetfillcolor{currentfill}%
\pgfsetlinewidth{0.000000pt}%
\definecolor{currentstroke}{rgb}{0.000000,0.000000,0.000000}%
\pgfsetstrokecolor{currentstroke}%
\pgfsetstrokeopacity{0.000000}%
\pgfsetdash{}{0pt}%
\pgfpathmoveto{\pgfqpoint{5.800000in}{87.925833in}}%
\pgfpathlineto{\pgfqpoint{5.800000in}{87.925833in}}%
\pgfpathlineto{\pgfqpoint{5.800000in}{88.028671in}}%
\pgfpathlineto{\pgfqpoint{5.800000in}{88.028671in}}%
\pgfpathclose%
\pgfusepath{fill}%
\end{pgfscope}%
\begin{pgfscope}%
\pgfpathrectangle{\pgfqpoint{5.800000in}{0.720000in}}{\pgfqpoint{1.400000in}{4.620000in}}%
\pgfusepath{clip}%
\pgfsetbuttcap%
\pgfsetmiterjoin%
\definecolor{currentfill}{rgb}{0.121569,0.466667,0.705882}%
\pgfsetfillcolor{currentfill}%
\pgfsetlinewidth{0.000000pt}%
\definecolor{currentstroke}{rgb}{0.000000,0.000000,0.000000}%
\pgfsetstrokecolor{currentstroke}%
\pgfsetstrokeopacity{0.000000}%
\pgfsetdash{}{0pt}%
\pgfpathmoveto{\pgfqpoint{5.800000in}{88.028671in}}%
\pgfpathlineto{\pgfqpoint{5.800000in}{88.028671in}}%
\pgfpathlineto{\pgfqpoint{5.800000in}{88.131508in}}%
\pgfpathlineto{\pgfqpoint{5.800000in}{88.131508in}}%
\pgfpathclose%
\pgfusepath{fill}%
\end{pgfscope}%
\begin{pgfscope}%
\pgfpathrectangle{\pgfqpoint{5.800000in}{0.720000in}}{\pgfqpoint{1.400000in}{4.620000in}}%
\pgfusepath{clip}%
\pgfsetbuttcap%
\pgfsetmiterjoin%
\definecolor{currentfill}{rgb}{0.121569,0.466667,0.705882}%
\pgfsetfillcolor{currentfill}%
\pgfsetlinewidth{0.000000pt}%
\definecolor{currentstroke}{rgb}{0.000000,0.000000,0.000000}%
\pgfsetstrokecolor{currentstroke}%
\pgfsetstrokeopacity{0.000000}%
\pgfsetdash{}{0pt}%
\pgfpathmoveto{\pgfqpoint{5.800000in}{88.131508in}}%
\pgfpathlineto{\pgfqpoint{5.800000in}{88.131508in}}%
\pgfpathlineto{\pgfqpoint{5.800000in}{88.234345in}}%
\pgfpathlineto{\pgfqpoint{5.800000in}{88.234345in}}%
\pgfpathclose%
\pgfusepath{fill}%
\end{pgfscope}%
\begin{pgfscope}%
\pgfpathrectangle{\pgfqpoint{5.800000in}{0.720000in}}{\pgfqpoint{1.400000in}{4.620000in}}%
\pgfusepath{clip}%
\pgfsetbuttcap%
\pgfsetmiterjoin%
\definecolor{currentfill}{rgb}{0.121569,0.466667,0.705882}%
\pgfsetfillcolor{currentfill}%
\pgfsetlinewidth{0.000000pt}%
\definecolor{currentstroke}{rgb}{0.000000,0.000000,0.000000}%
\pgfsetstrokecolor{currentstroke}%
\pgfsetstrokeopacity{0.000000}%
\pgfsetdash{}{0pt}%
\pgfpathmoveto{\pgfqpoint{5.800000in}{88.234345in}}%
\pgfpathlineto{\pgfqpoint{5.800000in}{88.234345in}}%
\pgfpathlineto{\pgfqpoint{5.800000in}{88.337182in}}%
\pgfpathlineto{\pgfqpoint{5.800000in}{88.337182in}}%
\pgfpathclose%
\pgfusepath{fill}%
\end{pgfscope}%
\begin{pgfscope}%
\pgfpathrectangle{\pgfqpoint{5.800000in}{0.720000in}}{\pgfqpoint{1.400000in}{4.620000in}}%
\pgfusepath{clip}%
\pgfsetbuttcap%
\pgfsetmiterjoin%
\definecolor{currentfill}{rgb}{0.121569,0.466667,0.705882}%
\pgfsetfillcolor{currentfill}%
\pgfsetlinewidth{0.000000pt}%
\definecolor{currentstroke}{rgb}{0.000000,0.000000,0.000000}%
\pgfsetstrokecolor{currentstroke}%
\pgfsetstrokeopacity{0.000000}%
\pgfsetdash{}{0pt}%
\pgfpathmoveto{\pgfqpoint{5.800000in}{88.337182in}}%
\pgfpathlineto{\pgfqpoint{5.800000in}{88.337182in}}%
\pgfpathlineto{\pgfqpoint{5.800000in}{88.440019in}}%
\pgfpathlineto{\pgfqpoint{5.800000in}{88.440019in}}%
\pgfpathclose%
\pgfusepath{fill}%
\end{pgfscope}%
\begin{pgfscope}%
\pgfpathrectangle{\pgfqpoint{5.800000in}{0.720000in}}{\pgfqpoint{1.400000in}{4.620000in}}%
\pgfusepath{clip}%
\pgfsetbuttcap%
\pgfsetmiterjoin%
\definecolor{currentfill}{rgb}{0.121569,0.466667,0.705882}%
\pgfsetfillcolor{currentfill}%
\pgfsetlinewidth{0.000000pt}%
\definecolor{currentstroke}{rgb}{0.000000,0.000000,0.000000}%
\pgfsetstrokecolor{currentstroke}%
\pgfsetstrokeopacity{0.000000}%
\pgfsetdash{}{0pt}%
\pgfpathmoveto{\pgfqpoint{5.800000in}{88.440019in}}%
\pgfpathlineto{\pgfqpoint{5.800000in}{88.440019in}}%
\pgfpathlineto{\pgfqpoint{5.800000in}{88.542856in}}%
\pgfpathlineto{\pgfqpoint{5.800000in}{88.542856in}}%
\pgfpathclose%
\pgfusepath{fill}%
\end{pgfscope}%
\begin{pgfscope}%
\pgfpathrectangle{\pgfqpoint{5.800000in}{0.720000in}}{\pgfqpoint{1.400000in}{4.620000in}}%
\pgfusepath{clip}%
\pgfsetbuttcap%
\pgfsetmiterjoin%
\definecolor{currentfill}{rgb}{0.121569,0.466667,0.705882}%
\pgfsetfillcolor{currentfill}%
\pgfsetlinewidth{0.000000pt}%
\definecolor{currentstroke}{rgb}{0.000000,0.000000,0.000000}%
\pgfsetstrokecolor{currentstroke}%
\pgfsetstrokeopacity{0.000000}%
\pgfsetdash{}{0pt}%
\pgfpathmoveto{\pgfqpoint{5.800000in}{88.542856in}}%
\pgfpathlineto{\pgfqpoint{5.800000in}{88.542856in}}%
\pgfpathlineto{\pgfqpoint{5.800000in}{88.645693in}}%
\pgfpathlineto{\pgfqpoint{5.800000in}{88.645693in}}%
\pgfpathclose%
\pgfusepath{fill}%
\end{pgfscope}%
\begin{pgfscope}%
\pgfpathrectangle{\pgfqpoint{5.800000in}{0.720000in}}{\pgfqpoint{1.400000in}{4.620000in}}%
\pgfusepath{clip}%
\pgfsetbuttcap%
\pgfsetmiterjoin%
\definecolor{currentfill}{rgb}{0.121569,0.466667,0.705882}%
\pgfsetfillcolor{currentfill}%
\pgfsetlinewidth{0.000000pt}%
\definecolor{currentstroke}{rgb}{0.000000,0.000000,0.000000}%
\pgfsetstrokecolor{currentstroke}%
\pgfsetstrokeopacity{0.000000}%
\pgfsetdash{}{0pt}%
\pgfpathmoveto{\pgfqpoint{5.800000in}{88.645693in}}%
\pgfpathlineto{\pgfqpoint{5.800000in}{88.645693in}}%
\pgfpathlineto{\pgfqpoint{5.800000in}{88.748530in}}%
\pgfpathlineto{\pgfqpoint{5.800000in}{88.748530in}}%
\pgfpathclose%
\pgfusepath{fill}%
\end{pgfscope}%
\begin{pgfscope}%
\pgfpathrectangle{\pgfqpoint{5.800000in}{0.720000in}}{\pgfqpoint{1.400000in}{4.620000in}}%
\pgfusepath{clip}%
\pgfsetbuttcap%
\pgfsetmiterjoin%
\definecolor{currentfill}{rgb}{0.121569,0.466667,0.705882}%
\pgfsetfillcolor{currentfill}%
\pgfsetlinewidth{0.000000pt}%
\definecolor{currentstroke}{rgb}{0.000000,0.000000,0.000000}%
\pgfsetstrokecolor{currentstroke}%
\pgfsetstrokeopacity{0.000000}%
\pgfsetdash{}{0pt}%
\pgfpathmoveto{\pgfqpoint{5.800000in}{88.748530in}}%
\pgfpathlineto{\pgfqpoint{5.800000in}{88.748530in}}%
\pgfpathlineto{\pgfqpoint{5.800000in}{88.851367in}}%
\pgfpathlineto{\pgfqpoint{5.800000in}{88.851367in}}%
\pgfpathclose%
\pgfusepath{fill}%
\end{pgfscope}%
\begin{pgfscope}%
\pgfpathrectangle{\pgfqpoint{5.800000in}{0.720000in}}{\pgfqpoint{1.400000in}{4.620000in}}%
\pgfusepath{clip}%
\pgfsetbuttcap%
\pgfsetmiterjoin%
\definecolor{currentfill}{rgb}{0.121569,0.466667,0.705882}%
\pgfsetfillcolor{currentfill}%
\pgfsetlinewidth{0.000000pt}%
\definecolor{currentstroke}{rgb}{0.000000,0.000000,0.000000}%
\pgfsetstrokecolor{currentstroke}%
\pgfsetstrokeopacity{0.000000}%
\pgfsetdash{}{0pt}%
\pgfpathmoveto{\pgfqpoint{5.800000in}{88.851367in}}%
\pgfpathlineto{\pgfqpoint{5.800000in}{88.851367in}}%
\pgfpathlineto{\pgfqpoint{5.800000in}{88.954204in}}%
\pgfpathlineto{\pgfqpoint{5.800000in}{88.954204in}}%
\pgfpathclose%
\pgfusepath{fill}%
\end{pgfscope}%
\begin{pgfscope}%
\pgfpathrectangle{\pgfqpoint{5.800000in}{0.720000in}}{\pgfqpoint{1.400000in}{4.620000in}}%
\pgfusepath{clip}%
\pgfsetbuttcap%
\pgfsetmiterjoin%
\definecolor{currentfill}{rgb}{0.121569,0.466667,0.705882}%
\pgfsetfillcolor{currentfill}%
\pgfsetlinewidth{0.000000pt}%
\definecolor{currentstroke}{rgb}{0.000000,0.000000,0.000000}%
\pgfsetstrokecolor{currentstroke}%
\pgfsetstrokeopacity{0.000000}%
\pgfsetdash{}{0pt}%
\pgfpathmoveto{\pgfqpoint{5.800000in}{88.954204in}}%
\pgfpathlineto{\pgfqpoint{5.800000in}{88.954204in}}%
\pgfpathlineto{\pgfqpoint{5.800000in}{89.057041in}}%
\pgfpathlineto{\pgfqpoint{5.800000in}{89.057041in}}%
\pgfpathclose%
\pgfusepath{fill}%
\end{pgfscope}%
\begin{pgfscope}%
\pgfpathrectangle{\pgfqpoint{5.800000in}{0.720000in}}{\pgfqpoint{1.400000in}{4.620000in}}%
\pgfusepath{clip}%
\pgfsetbuttcap%
\pgfsetmiterjoin%
\definecolor{currentfill}{rgb}{0.121569,0.466667,0.705882}%
\pgfsetfillcolor{currentfill}%
\pgfsetlinewidth{0.000000pt}%
\definecolor{currentstroke}{rgb}{0.000000,0.000000,0.000000}%
\pgfsetstrokecolor{currentstroke}%
\pgfsetstrokeopacity{0.000000}%
\pgfsetdash{}{0pt}%
\pgfpathmoveto{\pgfqpoint{5.800000in}{89.057041in}}%
\pgfpathlineto{\pgfqpoint{5.800000in}{89.057041in}}%
\pgfpathlineto{\pgfqpoint{5.800000in}{89.159878in}}%
\pgfpathlineto{\pgfqpoint{5.800000in}{89.159878in}}%
\pgfpathclose%
\pgfusepath{fill}%
\end{pgfscope}%
\begin{pgfscope}%
\pgfpathrectangle{\pgfqpoint{5.800000in}{0.720000in}}{\pgfqpoint{1.400000in}{4.620000in}}%
\pgfusepath{clip}%
\pgfsetbuttcap%
\pgfsetmiterjoin%
\definecolor{currentfill}{rgb}{0.121569,0.466667,0.705882}%
\pgfsetfillcolor{currentfill}%
\pgfsetlinewidth{0.000000pt}%
\definecolor{currentstroke}{rgb}{0.000000,0.000000,0.000000}%
\pgfsetstrokecolor{currentstroke}%
\pgfsetstrokeopacity{0.000000}%
\pgfsetdash{}{0pt}%
\pgfpathmoveto{\pgfqpoint{5.800000in}{89.159878in}}%
\pgfpathlineto{\pgfqpoint{5.800000in}{89.159878in}}%
\pgfpathlineto{\pgfqpoint{5.800000in}{89.262715in}}%
\pgfpathlineto{\pgfqpoint{5.800000in}{89.262715in}}%
\pgfpathclose%
\pgfusepath{fill}%
\end{pgfscope}%
\begin{pgfscope}%
\pgfpathrectangle{\pgfqpoint{5.800000in}{0.720000in}}{\pgfqpoint{1.400000in}{4.620000in}}%
\pgfusepath{clip}%
\pgfsetbuttcap%
\pgfsetmiterjoin%
\definecolor{currentfill}{rgb}{0.121569,0.466667,0.705882}%
\pgfsetfillcolor{currentfill}%
\pgfsetlinewidth{0.000000pt}%
\definecolor{currentstroke}{rgb}{0.000000,0.000000,0.000000}%
\pgfsetstrokecolor{currentstroke}%
\pgfsetstrokeopacity{0.000000}%
\pgfsetdash{}{0pt}%
\pgfpathmoveto{\pgfqpoint{5.800000in}{89.262715in}}%
\pgfpathlineto{\pgfqpoint{5.800000in}{89.262715in}}%
\pgfpathlineto{\pgfqpoint{5.800000in}{89.365552in}}%
\pgfpathlineto{\pgfqpoint{5.800000in}{89.365552in}}%
\pgfpathclose%
\pgfusepath{fill}%
\end{pgfscope}%
\begin{pgfscope}%
\pgfpathrectangle{\pgfqpoint{5.800000in}{0.720000in}}{\pgfqpoint{1.400000in}{4.620000in}}%
\pgfusepath{clip}%
\pgfsetbuttcap%
\pgfsetmiterjoin%
\definecolor{currentfill}{rgb}{0.121569,0.466667,0.705882}%
\pgfsetfillcolor{currentfill}%
\pgfsetlinewidth{0.000000pt}%
\definecolor{currentstroke}{rgb}{0.000000,0.000000,0.000000}%
\pgfsetstrokecolor{currentstroke}%
\pgfsetstrokeopacity{0.000000}%
\pgfsetdash{}{0pt}%
\pgfpathmoveto{\pgfqpoint{5.800000in}{89.365552in}}%
\pgfpathlineto{\pgfqpoint{5.800000in}{89.365552in}}%
\pgfpathlineto{\pgfqpoint{5.800000in}{89.468389in}}%
\pgfpathlineto{\pgfqpoint{5.800000in}{89.468389in}}%
\pgfpathclose%
\pgfusepath{fill}%
\end{pgfscope}%
\begin{pgfscope}%
\pgfpathrectangle{\pgfqpoint{5.800000in}{0.720000in}}{\pgfqpoint{1.400000in}{4.620000in}}%
\pgfusepath{clip}%
\pgfsetbuttcap%
\pgfsetmiterjoin%
\definecolor{currentfill}{rgb}{0.121569,0.466667,0.705882}%
\pgfsetfillcolor{currentfill}%
\pgfsetlinewidth{0.000000pt}%
\definecolor{currentstroke}{rgb}{0.000000,0.000000,0.000000}%
\pgfsetstrokecolor{currentstroke}%
\pgfsetstrokeopacity{0.000000}%
\pgfsetdash{}{0pt}%
\pgfpathmoveto{\pgfqpoint{5.800000in}{89.468389in}}%
\pgfpathlineto{\pgfqpoint{5.800000in}{89.468389in}}%
\pgfpathlineto{\pgfqpoint{5.800000in}{89.571227in}}%
\pgfpathlineto{\pgfqpoint{5.800000in}{89.571227in}}%
\pgfpathclose%
\pgfusepath{fill}%
\end{pgfscope}%
\begin{pgfscope}%
\pgfpathrectangle{\pgfqpoint{5.800000in}{0.720000in}}{\pgfqpoint{1.400000in}{4.620000in}}%
\pgfusepath{clip}%
\pgfsetbuttcap%
\pgfsetmiterjoin%
\definecolor{currentfill}{rgb}{0.121569,0.466667,0.705882}%
\pgfsetfillcolor{currentfill}%
\pgfsetlinewidth{0.000000pt}%
\definecolor{currentstroke}{rgb}{0.000000,0.000000,0.000000}%
\pgfsetstrokecolor{currentstroke}%
\pgfsetstrokeopacity{0.000000}%
\pgfsetdash{}{0pt}%
\pgfpathmoveto{\pgfqpoint{5.800000in}{89.571227in}}%
\pgfpathlineto{\pgfqpoint{5.800000in}{89.571227in}}%
\pgfpathlineto{\pgfqpoint{5.800000in}{89.674064in}}%
\pgfpathlineto{\pgfqpoint{5.800000in}{89.674064in}}%
\pgfpathclose%
\pgfusepath{fill}%
\end{pgfscope}%
\begin{pgfscope}%
\pgfpathrectangle{\pgfqpoint{5.800000in}{0.720000in}}{\pgfqpoint{1.400000in}{4.620000in}}%
\pgfusepath{clip}%
\pgfsetbuttcap%
\pgfsetmiterjoin%
\definecolor{currentfill}{rgb}{0.121569,0.466667,0.705882}%
\pgfsetfillcolor{currentfill}%
\pgfsetlinewidth{0.000000pt}%
\definecolor{currentstroke}{rgb}{0.000000,0.000000,0.000000}%
\pgfsetstrokecolor{currentstroke}%
\pgfsetstrokeopacity{0.000000}%
\pgfsetdash{}{0pt}%
\pgfpathmoveto{\pgfqpoint{5.800000in}{89.674064in}}%
\pgfpathlineto{\pgfqpoint{5.800000in}{89.674064in}}%
\pgfpathlineto{\pgfqpoint{5.800000in}{89.776901in}}%
\pgfpathlineto{\pgfqpoint{5.800000in}{89.776901in}}%
\pgfpathclose%
\pgfusepath{fill}%
\end{pgfscope}%
\begin{pgfscope}%
\pgfpathrectangle{\pgfqpoint{5.800000in}{0.720000in}}{\pgfqpoint{1.400000in}{4.620000in}}%
\pgfusepath{clip}%
\pgfsetbuttcap%
\pgfsetmiterjoin%
\definecolor{currentfill}{rgb}{0.121569,0.466667,0.705882}%
\pgfsetfillcolor{currentfill}%
\pgfsetlinewidth{0.000000pt}%
\definecolor{currentstroke}{rgb}{0.000000,0.000000,0.000000}%
\pgfsetstrokecolor{currentstroke}%
\pgfsetstrokeopacity{0.000000}%
\pgfsetdash{}{0pt}%
\pgfpathmoveto{\pgfqpoint{5.800000in}{89.776901in}}%
\pgfpathlineto{\pgfqpoint{5.800000in}{89.776901in}}%
\pgfpathlineto{\pgfqpoint{5.800000in}{89.879738in}}%
\pgfpathlineto{\pgfqpoint{5.800000in}{89.879738in}}%
\pgfpathclose%
\pgfusepath{fill}%
\end{pgfscope}%
\begin{pgfscope}%
\pgfpathrectangle{\pgfqpoint{5.800000in}{0.720000in}}{\pgfqpoint{1.400000in}{4.620000in}}%
\pgfusepath{clip}%
\pgfsetbuttcap%
\pgfsetmiterjoin%
\definecolor{currentfill}{rgb}{0.121569,0.466667,0.705882}%
\pgfsetfillcolor{currentfill}%
\pgfsetlinewidth{0.000000pt}%
\definecolor{currentstroke}{rgb}{0.000000,0.000000,0.000000}%
\pgfsetstrokecolor{currentstroke}%
\pgfsetstrokeopacity{0.000000}%
\pgfsetdash{}{0pt}%
\pgfpathmoveto{\pgfqpoint{5.800000in}{89.879738in}}%
\pgfpathlineto{\pgfqpoint{5.800000in}{89.879738in}}%
\pgfpathlineto{\pgfqpoint{5.800000in}{89.982575in}}%
\pgfpathlineto{\pgfqpoint{5.800000in}{89.982575in}}%
\pgfpathclose%
\pgfusepath{fill}%
\end{pgfscope}%
\begin{pgfscope}%
\pgfpathrectangle{\pgfqpoint{5.800000in}{0.720000in}}{\pgfqpoint{1.400000in}{4.620000in}}%
\pgfusepath{clip}%
\pgfsetbuttcap%
\pgfsetmiterjoin%
\definecolor{currentfill}{rgb}{0.121569,0.466667,0.705882}%
\pgfsetfillcolor{currentfill}%
\pgfsetlinewidth{0.000000pt}%
\definecolor{currentstroke}{rgb}{0.000000,0.000000,0.000000}%
\pgfsetstrokecolor{currentstroke}%
\pgfsetstrokeopacity{0.000000}%
\pgfsetdash{}{0pt}%
\pgfpathmoveto{\pgfqpoint{5.800000in}{89.982575in}}%
\pgfpathlineto{\pgfqpoint{5.800000in}{89.982575in}}%
\pgfpathlineto{\pgfqpoint{5.800000in}{90.085412in}}%
\pgfpathlineto{\pgfqpoint{5.800000in}{90.085412in}}%
\pgfpathclose%
\pgfusepath{fill}%
\end{pgfscope}%
\begin{pgfscope}%
\pgfpathrectangle{\pgfqpoint{5.800000in}{0.720000in}}{\pgfqpoint{1.400000in}{4.620000in}}%
\pgfusepath{clip}%
\pgfsetbuttcap%
\pgfsetmiterjoin%
\definecolor{currentfill}{rgb}{0.121569,0.466667,0.705882}%
\pgfsetfillcolor{currentfill}%
\pgfsetlinewidth{0.000000pt}%
\definecolor{currentstroke}{rgb}{0.000000,0.000000,0.000000}%
\pgfsetstrokecolor{currentstroke}%
\pgfsetstrokeopacity{0.000000}%
\pgfsetdash{}{0pt}%
\pgfpathmoveto{\pgfqpoint{5.800000in}{90.085412in}}%
\pgfpathlineto{\pgfqpoint{5.800000in}{90.085412in}}%
\pgfpathlineto{\pgfqpoint{5.800000in}{90.188249in}}%
\pgfpathlineto{\pgfqpoint{5.800000in}{90.188249in}}%
\pgfpathclose%
\pgfusepath{fill}%
\end{pgfscope}%
\begin{pgfscope}%
\pgfpathrectangle{\pgfqpoint{5.800000in}{0.720000in}}{\pgfqpoint{1.400000in}{4.620000in}}%
\pgfusepath{clip}%
\pgfsetbuttcap%
\pgfsetmiterjoin%
\definecolor{currentfill}{rgb}{0.121569,0.466667,0.705882}%
\pgfsetfillcolor{currentfill}%
\pgfsetlinewidth{0.000000pt}%
\definecolor{currentstroke}{rgb}{0.000000,0.000000,0.000000}%
\pgfsetstrokecolor{currentstroke}%
\pgfsetstrokeopacity{0.000000}%
\pgfsetdash{}{0pt}%
\pgfpathmoveto{\pgfqpoint{5.800000in}{90.188249in}}%
\pgfpathlineto{\pgfqpoint{5.800000in}{90.188249in}}%
\pgfpathlineto{\pgfqpoint{5.800000in}{90.291086in}}%
\pgfpathlineto{\pgfqpoint{5.800000in}{90.291086in}}%
\pgfpathclose%
\pgfusepath{fill}%
\end{pgfscope}%
\begin{pgfscope}%
\pgfpathrectangle{\pgfqpoint{5.800000in}{0.720000in}}{\pgfqpoint{1.400000in}{4.620000in}}%
\pgfusepath{clip}%
\pgfsetbuttcap%
\pgfsetmiterjoin%
\definecolor{currentfill}{rgb}{0.121569,0.466667,0.705882}%
\pgfsetfillcolor{currentfill}%
\pgfsetlinewidth{0.000000pt}%
\definecolor{currentstroke}{rgb}{0.000000,0.000000,0.000000}%
\pgfsetstrokecolor{currentstroke}%
\pgfsetstrokeopacity{0.000000}%
\pgfsetdash{}{0pt}%
\pgfpathmoveto{\pgfqpoint{5.800000in}{90.291086in}}%
\pgfpathlineto{\pgfqpoint{5.800000in}{90.291086in}}%
\pgfpathlineto{\pgfqpoint{5.800000in}{90.393923in}}%
\pgfpathlineto{\pgfqpoint{5.800000in}{90.393923in}}%
\pgfpathclose%
\pgfusepath{fill}%
\end{pgfscope}%
\begin{pgfscope}%
\pgfpathrectangle{\pgfqpoint{5.800000in}{0.720000in}}{\pgfqpoint{1.400000in}{4.620000in}}%
\pgfusepath{clip}%
\pgfsetbuttcap%
\pgfsetmiterjoin%
\definecolor{currentfill}{rgb}{0.121569,0.466667,0.705882}%
\pgfsetfillcolor{currentfill}%
\pgfsetlinewidth{0.000000pt}%
\definecolor{currentstroke}{rgb}{0.000000,0.000000,0.000000}%
\pgfsetstrokecolor{currentstroke}%
\pgfsetstrokeopacity{0.000000}%
\pgfsetdash{}{0pt}%
\pgfpathmoveto{\pgfqpoint{5.800000in}{90.393923in}}%
\pgfpathlineto{\pgfqpoint{5.800000in}{90.393923in}}%
\pgfpathlineto{\pgfqpoint{5.800000in}{90.496760in}}%
\pgfpathlineto{\pgfqpoint{5.800000in}{90.496760in}}%
\pgfpathclose%
\pgfusepath{fill}%
\end{pgfscope}%
\begin{pgfscope}%
\pgfpathrectangle{\pgfqpoint{5.800000in}{0.720000in}}{\pgfqpoint{1.400000in}{4.620000in}}%
\pgfusepath{clip}%
\pgfsetbuttcap%
\pgfsetmiterjoin%
\definecolor{currentfill}{rgb}{0.121569,0.466667,0.705882}%
\pgfsetfillcolor{currentfill}%
\pgfsetlinewidth{0.000000pt}%
\definecolor{currentstroke}{rgb}{0.000000,0.000000,0.000000}%
\pgfsetstrokecolor{currentstroke}%
\pgfsetstrokeopacity{0.000000}%
\pgfsetdash{}{0pt}%
\pgfpathmoveto{\pgfqpoint{5.800000in}{90.496760in}}%
\pgfpathlineto{\pgfqpoint{5.800000in}{90.496760in}}%
\pgfpathlineto{\pgfqpoint{5.800000in}{90.599597in}}%
\pgfpathlineto{\pgfqpoint{5.800000in}{90.599597in}}%
\pgfpathclose%
\pgfusepath{fill}%
\end{pgfscope}%
\begin{pgfscope}%
\pgfpathrectangle{\pgfqpoint{5.800000in}{0.720000in}}{\pgfqpoint{1.400000in}{4.620000in}}%
\pgfusepath{clip}%
\pgfsetbuttcap%
\pgfsetmiterjoin%
\definecolor{currentfill}{rgb}{0.121569,0.466667,0.705882}%
\pgfsetfillcolor{currentfill}%
\pgfsetlinewidth{0.000000pt}%
\definecolor{currentstroke}{rgb}{0.000000,0.000000,0.000000}%
\pgfsetstrokecolor{currentstroke}%
\pgfsetstrokeopacity{0.000000}%
\pgfsetdash{}{0pt}%
\pgfpathmoveto{\pgfqpoint{5.800000in}{90.599597in}}%
\pgfpathlineto{\pgfqpoint{5.800000in}{90.599597in}}%
\pgfpathlineto{\pgfqpoint{5.800000in}{90.702434in}}%
\pgfpathlineto{\pgfqpoint{5.800000in}{90.702434in}}%
\pgfpathclose%
\pgfusepath{fill}%
\end{pgfscope}%
\begin{pgfscope}%
\pgfpathrectangle{\pgfqpoint{5.800000in}{0.720000in}}{\pgfqpoint{1.400000in}{4.620000in}}%
\pgfusepath{clip}%
\pgfsetbuttcap%
\pgfsetmiterjoin%
\definecolor{currentfill}{rgb}{0.121569,0.466667,0.705882}%
\pgfsetfillcolor{currentfill}%
\pgfsetlinewidth{0.000000pt}%
\definecolor{currentstroke}{rgb}{0.000000,0.000000,0.000000}%
\pgfsetstrokecolor{currentstroke}%
\pgfsetstrokeopacity{0.000000}%
\pgfsetdash{}{0pt}%
\pgfpathmoveto{\pgfqpoint{5.800000in}{90.702434in}}%
\pgfpathlineto{\pgfqpoint{5.800000in}{90.702434in}}%
\pgfpathlineto{\pgfqpoint{5.800000in}{90.805271in}}%
\pgfpathlineto{\pgfqpoint{5.800000in}{90.805271in}}%
\pgfpathclose%
\pgfusepath{fill}%
\end{pgfscope}%
\begin{pgfscope}%
\pgfpathrectangle{\pgfqpoint{5.800000in}{0.720000in}}{\pgfqpoint{1.400000in}{4.620000in}}%
\pgfusepath{clip}%
\pgfsetbuttcap%
\pgfsetmiterjoin%
\definecolor{currentfill}{rgb}{0.121569,0.466667,0.705882}%
\pgfsetfillcolor{currentfill}%
\pgfsetlinewidth{0.000000pt}%
\definecolor{currentstroke}{rgb}{0.000000,0.000000,0.000000}%
\pgfsetstrokecolor{currentstroke}%
\pgfsetstrokeopacity{0.000000}%
\pgfsetdash{}{0pt}%
\pgfpathmoveto{\pgfqpoint{5.800000in}{90.805271in}}%
\pgfpathlineto{\pgfqpoint{5.800000in}{90.805271in}}%
\pgfpathlineto{\pgfqpoint{5.800000in}{90.908108in}}%
\pgfpathlineto{\pgfqpoint{5.800000in}{90.908108in}}%
\pgfpathclose%
\pgfusepath{fill}%
\end{pgfscope}%
\begin{pgfscope}%
\pgfpathrectangle{\pgfqpoint{5.800000in}{0.720000in}}{\pgfqpoint{1.400000in}{4.620000in}}%
\pgfusepath{clip}%
\pgfsetbuttcap%
\pgfsetmiterjoin%
\definecolor{currentfill}{rgb}{0.121569,0.466667,0.705882}%
\pgfsetfillcolor{currentfill}%
\pgfsetlinewidth{0.000000pt}%
\definecolor{currentstroke}{rgb}{0.000000,0.000000,0.000000}%
\pgfsetstrokecolor{currentstroke}%
\pgfsetstrokeopacity{0.000000}%
\pgfsetdash{}{0pt}%
\pgfpathmoveto{\pgfqpoint{5.800000in}{90.908108in}}%
\pgfpathlineto{\pgfqpoint{5.800000in}{90.908108in}}%
\pgfpathlineto{\pgfqpoint{5.800000in}{91.010946in}}%
\pgfpathlineto{\pgfqpoint{5.800000in}{91.010946in}}%
\pgfpathclose%
\pgfusepath{fill}%
\end{pgfscope}%
\begin{pgfscope}%
\pgfpathrectangle{\pgfqpoint{5.800000in}{0.720000in}}{\pgfqpoint{1.400000in}{4.620000in}}%
\pgfusepath{clip}%
\pgfsetbuttcap%
\pgfsetmiterjoin%
\definecolor{currentfill}{rgb}{0.121569,0.466667,0.705882}%
\pgfsetfillcolor{currentfill}%
\pgfsetlinewidth{0.000000pt}%
\definecolor{currentstroke}{rgb}{0.000000,0.000000,0.000000}%
\pgfsetstrokecolor{currentstroke}%
\pgfsetstrokeopacity{0.000000}%
\pgfsetdash{}{0pt}%
\pgfpathmoveto{\pgfqpoint{5.800000in}{91.010946in}}%
\pgfpathlineto{\pgfqpoint{5.800000in}{91.010946in}}%
\pgfpathlineto{\pgfqpoint{5.800000in}{91.113783in}}%
\pgfpathlineto{\pgfqpoint{5.800000in}{91.113783in}}%
\pgfpathclose%
\pgfusepath{fill}%
\end{pgfscope}%
\begin{pgfscope}%
\pgfpathrectangle{\pgfqpoint{5.800000in}{0.720000in}}{\pgfqpoint{1.400000in}{4.620000in}}%
\pgfusepath{clip}%
\pgfsetbuttcap%
\pgfsetmiterjoin%
\definecolor{currentfill}{rgb}{0.121569,0.466667,0.705882}%
\pgfsetfillcolor{currentfill}%
\pgfsetlinewidth{0.000000pt}%
\definecolor{currentstroke}{rgb}{0.000000,0.000000,0.000000}%
\pgfsetstrokecolor{currentstroke}%
\pgfsetstrokeopacity{0.000000}%
\pgfsetdash{}{0pt}%
\pgfpathmoveto{\pgfqpoint{5.800000in}{91.113783in}}%
\pgfpathlineto{\pgfqpoint{5.800000in}{91.113783in}}%
\pgfpathlineto{\pgfqpoint{5.800000in}{91.216620in}}%
\pgfpathlineto{\pgfqpoint{5.800000in}{91.216620in}}%
\pgfpathclose%
\pgfusepath{fill}%
\end{pgfscope}%
\begin{pgfscope}%
\pgfpathrectangle{\pgfqpoint{5.800000in}{0.720000in}}{\pgfqpoint{1.400000in}{4.620000in}}%
\pgfusepath{clip}%
\pgfsetbuttcap%
\pgfsetmiterjoin%
\definecolor{currentfill}{rgb}{0.121569,0.466667,0.705882}%
\pgfsetfillcolor{currentfill}%
\pgfsetlinewidth{0.000000pt}%
\definecolor{currentstroke}{rgb}{0.000000,0.000000,0.000000}%
\pgfsetstrokecolor{currentstroke}%
\pgfsetstrokeopacity{0.000000}%
\pgfsetdash{}{0pt}%
\pgfpathmoveto{\pgfqpoint{5.800000in}{91.216620in}}%
\pgfpathlineto{\pgfqpoint{5.800000in}{91.216620in}}%
\pgfpathlineto{\pgfqpoint{5.800000in}{91.319457in}}%
\pgfpathlineto{\pgfqpoint{5.800000in}{91.319457in}}%
\pgfpathclose%
\pgfusepath{fill}%
\end{pgfscope}%
\begin{pgfscope}%
\pgfpathrectangle{\pgfqpoint{5.800000in}{0.720000in}}{\pgfqpoint{1.400000in}{4.620000in}}%
\pgfusepath{clip}%
\pgfsetbuttcap%
\pgfsetmiterjoin%
\definecolor{currentfill}{rgb}{0.121569,0.466667,0.705882}%
\pgfsetfillcolor{currentfill}%
\pgfsetlinewidth{0.000000pt}%
\definecolor{currentstroke}{rgb}{0.000000,0.000000,0.000000}%
\pgfsetstrokecolor{currentstroke}%
\pgfsetstrokeopacity{0.000000}%
\pgfsetdash{}{0pt}%
\pgfpathmoveto{\pgfqpoint{5.800000in}{91.319457in}}%
\pgfpathlineto{\pgfqpoint{5.800000in}{91.319457in}}%
\pgfpathlineto{\pgfqpoint{5.800000in}{91.422294in}}%
\pgfpathlineto{\pgfqpoint{5.800000in}{91.422294in}}%
\pgfpathclose%
\pgfusepath{fill}%
\end{pgfscope}%
\begin{pgfscope}%
\pgfpathrectangle{\pgfqpoint{5.800000in}{0.720000in}}{\pgfqpoint{1.400000in}{4.620000in}}%
\pgfusepath{clip}%
\pgfsetbuttcap%
\pgfsetmiterjoin%
\definecolor{currentfill}{rgb}{0.121569,0.466667,0.705882}%
\pgfsetfillcolor{currentfill}%
\pgfsetlinewidth{0.000000pt}%
\definecolor{currentstroke}{rgb}{0.000000,0.000000,0.000000}%
\pgfsetstrokecolor{currentstroke}%
\pgfsetstrokeopacity{0.000000}%
\pgfsetdash{}{0pt}%
\pgfpathmoveto{\pgfqpoint{5.800000in}{91.422294in}}%
\pgfpathlineto{\pgfqpoint{5.800000in}{91.422294in}}%
\pgfpathlineto{\pgfqpoint{5.800000in}{91.525131in}}%
\pgfpathlineto{\pgfqpoint{5.800000in}{91.525131in}}%
\pgfpathclose%
\pgfusepath{fill}%
\end{pgfscope}%
\begin{pgfscope}%
\pgfpathrectangle{\pgfqpoint{5.800000in}{0.720000in}}{\pgfqpoint{1.400000in}{4.620000in}}%
\pgfusepath{clip}%
\pgfsetbuttcap%
\pgfsetmiterjoin%
\definecolor{currentfill}{rgb}{0.121569,0.466667,0.705882}%
\pgfsetfillcolor{currentfill}%
\pgfsetlinewidth{0.000000pt}%
\definecolor{currentstroke}{rgb}{0.000000,0.000000,0.000000}%
\pgfsetstrokecolor{currentstroke}%
\pgfsetstrokeopacity{0.000000}%
\pgfsetdash{}{0pt}%
\pgfpathmoveto{\pgfqpoint{5.800000in}{91.525131in}}%
\pgfpathlineto{\pgfqpoint{5.800000in}{91.525131in}}%
\pgfpathlineto{\pgfqpoint{5.800000in}{91.627968in}}%
\pgfpathlineto{\pgfqpoint{5.800000in}{91.627968in}}%
\pgfpathclose%
\pgfusepath{fill}%
\end{pgfscope}%
\begin{pgfscope}%
\pgfpathrectangle{\pgfqpoint{5.800000in}{0.720000in}}{\pgfqpoint{1.400000in}{4.620000in}}%
\pgfusepath{clip}%
\pgfsetbuttcap%
\pgfsetmiterjoin%
\definecolor{currentfill}{rgb}{0.121569,0.466667,0.705882}%
\pgfsetfillcolor{currentfill}%
\pgfsetlinewidth{0.000000pt}%
\definecolor{currentstroke}{rgb}{0.000000,0.000000,0.000000}%
\pgfsetstrokecolor{currentstroke}%
\pgfsetstrokeopacity{0.000000}%
\pgfsetdash{}{0pt}%
\pgfpathmoveto{\pgfqpoint{5.800000in}{91.627968in}}%
\pgfpathlineto{\pgfqpoint{5.800000in}{91.627968in}}%
\pgfpathlineto{\pgfqpoint{5.800000in}{91.730805in}}%
\pgfpathlineto{\pgfqpoint{5.800000in}{91.730805in}}%
\pgfpathclose%
\pgfusepath{fill}%
\end{pgfscope}%
\begin{pgfscope}%
\pgfpathrectangle{\pgfqpoint{5.800000in}{0.720000in}}{\pgfqpoint{1.400000in}{4.620000in}}%
\pgfusepath{clip}%
\pgfsetbuttcap%
\pgfsetmiterjoin%
\definecolor{currentfill}{rgb}{0.121569,0.466667,0.705882}%
\pgfsetfillcolor{currentfill}%
\pgfsetlinewidth{0.000000pt}%
\definecolor{currentstroke}{rgb}{0.000000,0.000000,0.000000}%
\pgfsetstrokecolor{currentstroke}%
\pgfsetstrokeopacity{0.000000}%
\pgfsetdash{}{0pt}%
\pgfpathmoveto{\pgfqpoint{5.800000in}{91.730805in}}%
\pgfpathlineto{\pgfqpoint{5.800000in}{91.730805in}}%
\pgfpathlineto{\pgfqpoint{5.800000in}{91.833642in}}%
\pgfpathlineto{\pgfqpoint{5.800000in}{91.833642in}}%
\pgfpathclose%
\pgfusepath{fill}%
\end{pgfscope}%
\begin{pgfscope}%
\pgfpathrectangle{\pgfqpoint{5.800000in}{0.720000in}}{\pgfqpoint{1.400000in}{4.620000in}}%
\pgfusepath{clip}%
\pgfsetbuttcap%
\pgfsetmiterjoin%
\definecolor{currentfill}{rgb}{0.121569,0.466667,0.705882}%
\pgfsetfillcolor{currentfill}%
\pgfsetlinewidth{0.000000pt}%
\definecolor{currentstroke}{rgb}{0.000000,0.000000,0.000000}%
\pgfsetstrokecolor{currentstroke}%
\pgfsetstrokeopacity{0.000000}%
\pgfsetdash{}{0pt}%
\pgfpathmoveto{\pgfqpoint{5.800000in}{91.833642in}}%
\pgfpathlineto{\pgfqpoint{5.800000in}{91.833642in}}%
\pgfpathlineto{\pgfqpoint{5.800000in}{91.936479in}}%
\pgfpathlineto{\pgfqpoint{5.800000in}{91.936479in}}%
\pgfpathclose%
\pgfusepath{fill}%
\end{pgfscope}%
\begin{pgfscope}%
\pgfpathrectangle{\pgfqpoint{5.800000in}{0.720000in}}{\pgfqpoint{1.400000in}{4.620000in}}%
\pgfusepath{clip}%
\pgfsetbuttcap%
\pgfsetmiterjoin%
\definecolor{currentfill}{rgb}{0.121569,0.466667,0.705882}%
\pgfsetfillcolor{currentfill}%
\pgfsetlinewidth{0.000000pt}%
\definecolor{currentstroke}{rgb}{0.000000,0.000000,0.000000}%
\pgfsetstrokecolor{currentstroke}%
\pgfsetstrokeopacity{0.000000}%
\pgfsetdash{}{0pt}%
\pgfpathmoveto{\pgfqpoint{5.800000in}{91.936479in}}%
\pgfpathlineto{\pgfqpoint{5.800000in}{91.936479in}}%
\pgfpathlineto{\pgfqpoint{5.800000in}{92.039316in}}%
\pgfpathlineto{\pgfqpoint{5.800000in}{92.039316in}}%
\pgfpathclose%
\pgfusepath{fill}%
\end{pgfscope}%
\begin{pgfscope}%
\pgfpathrectangle{\pgfqpoint{5.800000in}{0.720000in}}{\pgfqpoint{1.400000in}{4.620000in}}%
\pgfusepath{clip}%
\pgfsetbuttcap%
\pgfsetmiterjoin%
\definecolor{currentfill}{rgb}{0.121569,0.466667,0.705882}%
\pgfsetfillcolor{currentfill}%
\pgfsetlinewidth{0.000000pt}%
\definecolor{currentstroke}{rgb}{0.000000,0.000000,0.000000}%
\pgfsetstrokecolor{currentstroke}%
\pgfsetstrokeopacity{0.000000}%
\pgfsetdash{}{0pt}%
\pgfpathmoveto{\pgfqpoint{5.800000in}{92.039316in}}%
\pgfpathlineto{\pgfqpoint{5.800000in}{92.039316in}}%
\pgfpathlineto{\pgfqpoint{5.800000in}{92.142153in}}%
\pgfpathlineto{\pgfqpoint{5.800000in}{92.142153in}}%
\pgfpathclose%
\pgfusepath{fill}%
\end{pgfscope}%
\begin{pgfscope}%
\pgfpathrectangle{\pgfqpoint{5.800000in}{0.720000in}}{\pgfqpoint{1.400000in}{4.620000in}}%
\pgfusepath{clip}%
\pgfsetbuttcap%
\pgfsetmiterjoin%
\definecolor{currentfill}{rgb}{0.121569,0.466667,0.705882}%
\pgfsetfillcolor{currentfill}%
\pgfsetlinewidth{0.000000pt}%
\definecolor{currentstroke}{rgb}{0.000000,0.000000,0.000000}%
\pgfsetstrokecolor{currentstroke}%
\pgfsetstrokeopacity{0.000000}%
\pgfsetdash{}{0pt}%
\pgfpathmoveto{\pgfqpoint{5.800000in}{92.142153in}}%
\pgfpathlineto{\pgfqpoint{5.800000in}{92.142153in}}%
\pgfpathlineto{\pgfqpoint{5.800000in}{92.244990in}}%
\pgfpathlineto{\pgfqpoint{5.800000in}{92.244990in}}%
\pgfpathclose%
\pgfusepath{fill}%
\end{pgfscope}%
\begin{pgfscope}%
\pgfpathrectangle{\pgfqpoint{5.800000in}{0.720000in}}{\pgfqpoint{1.400000in}{4.620000in}}%
\pgfusepath{clip}%
\pgfsetbuttcap%
\pgfsetmiterjoin%
\definecolor{currentfill}{rgb}{0.121569,0.466667,0.705882}%
\pgfsetfillcolor{currentfill}%
\pgfsetlinewidth{0.000000pt}%
\definecolor{currentstroke}{rgb}{0.000000,0.000000,0.000000}%
\pgfsetstrokecolor{currentstroke}%
\pgfsetstrokeopacity{0.000000}%
\pgfsetdash{}{0pt}%
\pgfpathmoveto{\pgfqpoint{5.800000in}{92.244990in}}%
\pgfpathlineto{\pgfqpoint{5.800000in}{92.244990in}}%
\pgfpathlineto{\pgfqpoint{5.800000in}{92.347827in}}%
\pgfpathlineto{\pgfqpoint{5.800000in}{92.347827in}}%
\pgfpathclose%
\pgfusepath{fill}%
\end{pgfscope}%
\begin{pgfscope}%
\pgfpathrectangle{\pgfqpoint{5.800000in}{0.720000in}}{\pgfqpoint{1.400000in}{4.620000in}}%
\pgfusepath{clip}%
\pgfsetbuttcap%
\pgfsetmiterjoin%
\definecolor{currentfill}{rgb}{0.121569,0.466667,0.705882}%
\pgfsetfillcolor{currentfill}%
\pgfsetlinewidth{0.000000pt}%
\definecolor{currentstroke}{rgb}{0.000000,0.000000,0.000000}%
\pgfsetstrokecolor{currentstroke}%
\pgfsetstrokeopacity{0.000000}%
\pgfsetdash{}{0pt}%
\pgfpathmoveto{\pgfqpoint{5.800000in}{92.347827in}}%
\pgfpathlineto{\pgfqpoint{5.800000in}{92.347827in}}%
\pgfpathlineto{\pgfqpoint{5.800000in}{92.450664in}}%
\pgfpathlineto{\pgfqpoint{5.800000in}{92.450664in}}%
\pgfpathclose%
\pgfusepath{fill}%
\end{pgfscope}%
\begin{pgfscope}%
\pgfpathrectangle{\pgfqpoint{5.800000in}{0.720000in}}{\pgfqpoint{1.400000in}{4.620000in}}%
\pgfusepath{clip}%
\pgfsetbuttcap%
\pgfsetmiterjoin%
\definecolor{currentfill}{rgb}{0.121569,0.466667,0.705882}%
\pgfsetfillcolor{currentfill}%
\pgfsetlinewidth{0.000000pt}%
\definecolor{currentstroke}{rgb}{0.000000,0.000000,0.000000}%
\pgfsetstrokecolor{currentstroke}%
\pgfsetstrokeopacity{0.000000}%
\pgfsetdash{}{0pt}%
\pgfpathmoveto{\pgfqpoint{5.800000in}{92.450664in}}%
\pgfpathlineto{\pgfqpoint{5.800000in}{92.450664in}}%
\pgfpathlineto{\pgfqpoint{5.800000in}{92.553502in}}%
\pgfpathlineto{\pgfqpoint{5.800000in}{92.553502in}}%
\pgfpathclose%
\pgfusepath{fill}%
\end{pgfscope}%
\begin{pgfscope}%
\pgfpathrectangle{\pgfqpoint{5.800000in}{0.720000in}}{\pgfqpoint{1.400000in}{4.620000in}}%
\pgfusepath{clip}%
\pgfsetbuttcap%
\pgfsetmiterjoin%
\definecolor{currentfill}{rgb}{0.121569,0.466667,0.705882}%
\pgfsetfillcolor{currentfill}%
\pgfsetlinewidth{0.000000pt}%
\definecolor{currentstroke}{rgb}{0.000000,0.000000,0.000000}%
\pgfsetstrokecolor{currentstroke}%
\pgfsetstrokeopacity{0.000000}%
\pgfsetdash{}{0pt}%
\pgfpathmoveto{\pgfqpoint{5.800000in}{92.553502in}}%
\pgfpathlineto{\pgfqpoint{5.800000in}{92.553502in}}%
\pgfpathlineto{\pgfqpoint{5.800000in}{92.656339in}}%
\pgfpathlineto{\pgfqpoint{5.800000in}{92.656339in}}%
\pgfpathclose%
\pgfusepath{fill}%
\end{pgfscope}%
\begin{pgfscope}%
\pgfpathrectangle{\pgfqpoint{5.800000in}{0.720000in}}{\pgfqpoint{1.400000in}{4.620000in}}%
\pgfusepath{clip}%
\pgfsetbuttcap%
\pgfsetmiterjoin%
\definecolor{currentfill}{rgb}{0.121569,0.466667,0.705882}%
\pgfsetfillcolor{currentfill}%
\pgfsetlinewidth{0.000000pt}%
\definecolor{currentstroke}{rgb}{0.000000,0.000000,0.000000}%
\pgfsetstrokecolor{currentstroke}%
\pgfsetstrokeopacity{0.000000}%
\pgfsetdash{}{0pt}%
\pgfpathmoveto{\pgfqpoint{5.800000in}{92.656339in}}%
\pgfpathlineto{\pgfqpoint{5.800000in}{92.656339in}}%
\pgfpathlineto{\pgfqpoint{5.800000in}{92.759176in}}%
\pgfpathlineto{\pgfqpoint{5.800000in}{92.759176in}}%
\pgfpathclose%
\pgfusepath{fill}%
\end{pgfscope}%
\begin{pgfscope}%
\pgfpathrectangle{\pgfqpoint{5.800000in}{0.720000in}}{\pgfqpoint{1.400000in}{4.620000in}}%
\pgfusepath{clip}%
\pgfsetbuttcap%
\pgfsetmiterjoin%
\definecolor{currentfill}{rgb}{0.121569,0.466667,0.705882}%
\pgfsetfillcolor{currentfill}%
\pgfsetlinewidth{0.000000pt}%
\definecolor{currentstroke}{rgb}{0.000000,0.000000,0.000000}%
\pgfsetstrokecolor{currentstroke}%
\pgfsetstrokeopacity{0.000000}%
\pgfsetdash{}{0pt}%
\pgfpathmoveto{\pgfqpoint{5.800000in}{92.759176in}}%
\pgfpathlineto{\pgfqpoint{5.800000in}{92.759176in}}%
\pgfpathlineto{\pgfqpoint{5.800000in}{92.862013in}}%
\pgfpathlineto{\pgfqpoint{5.800000in}{92.862013in}}%
\pgfpathclose%
\pgfusepath{fill}%
\end{pgfscope}%
\begin{pgfscope}%
\pgfpathrectangle{\pgfqpoint{5.800000in}{0.720000in}}{\pgfqpoint{1.400000in}{4.620000in}}%
\pgfusepath{clip}%
\pgfsetbuttcap%
\pgfsetmiterjoin%
\definecolor{currentfill}{rgb}{0.121569,0.466667,0.705882}%
\pgfsetfillcolor{currentfill}%
\pgfsetlinewidth{0.000000pt}%
\definecolor{currentstroke}{rgb}{0.000000,0.000000,0.000000}%
\pgfsetstrokecolor{currentstroke}%
\pgfsetstrokeopacity{0.000000}%
\pgfsetdash{}{0pt}%
\pgfpathmoveto{\pgfqpoint{5.800000in}{92.862013in}}%
\pgfpathlineto{\pgfqpoint{5.800000in}{92.862013in}}%
\pgfpathlineto{\pgfqpoint{5.800000in}{92.964850in}}%
\pgfpathlineto{\pgfqpoint{5.800000in}{92.964850in}}%
\pgfpathclose%
\pgfusepath{fill}%
\end{pgfscope}%
\begin{pgfscope}%
\pgfpathrectangle{\pgfqpoint{5.800000in}{0.720000in}}{\pgfqpoint{1.400000in}{4.620000in}}%
\pgfusepath{clip}%
\pgfsetbuttcap%
\pgfsetmiterjoin%
\definecolor{currentfill}{rgb}{0.121569,0.466667,0.705882}%
\pgfsetfillcolor{currentfill}%
\pgfsetlinewidth{0.000000pt}%
\definecolor{currentstroke}{rgb}{0.000000,0.000000,0.000000}%
\pgfsetstrokecolor{currentstroke}%
\pgfsetstrokeopacity{0.000000}%
\pgfsetdash{}{0pt}%
\pgfpathmoveto{\pgfqpoint{5.800000in}{92.964850in}}%
\pgfpathlineto{\pgfqpoint{5.800000in}{92.964850in}}%
\pgfpathlineto{\pgfqpoint{5.800000in}{93.067687in}}%
\pgfpathlineto{\pgfqpoint{5.800000in}{93.067687in}}%
\pgfpathclose%
\pgfusepath{fill}%
\end{pgfscope}%
\begin{pgfscope}%
\pgfpathrectangle{\pgfqpoint{5.800000in}{0.720000in}}{\pgfqpoint{1.400000in}{4.620000in}}%
\pgfusepath{clip}%
\pgfsetbuttcap%
\pgfsetmiterjoin%
\definecolor{currentfill}{rgb}{0.121569,0.466667,0.705882}%
\pgfsetfillcolor{currentfill}%
\pgfsetlinewidth{0.000000pt}%
\definecolor{currentstroke}{rgb}{0.000000,0.000000,0.000000}%
\pgfsetstrokecolor{currentstroke}%
\pgfsetstrokeopacity{0.000000}%
\pgfsetdash{}{0pt}%
\pgfpathmoveto{\pgfqpoint{5.800000in}{93.067687in}}%
\pgfpathlineto{\pgfqpoint{5.800000in}{93.067687in}}%
\pgfpathlineto{\pgfqpoint{5.800000in}{93.170524in}}%
\pgfpathlineto{\pgfqpoint{5.800000in}{93.170524in}}%
\pgfpathclose%
\pgfusepath{fill}%
\end{pgfscope}%
\begin{pgfscope}%
\pgfpathrectangle{\pgfqpoint{5.800000in}{0.720000in}}{\pgfqpoint{1.400000in}{4.620000in}}%
\pgfusepath{clip}%
\pgfsetbuttcap%
\pgfsetmiterjoin%
\definecolor{currentfill}{rgb}{0.121569,0.466667,0.705882}%
\pgfsetfillcolor{currentfill}%
\pgfsetlinewidth{0.000000pt}%
\definecolor{currentstroke}{rgb}{0.000000,0.000000,0.000000}%
\pgfsetstrokecolor{currentstroke}%
\pgfsetstrokeopacity{0.000000}%
\pgfsetdash{}{0pt}%
\pgfpathmoveto{\pgfqpoint{5.800000in}{93.170524in}}%
\pgfpathlineto{\pgfqpoint{5.800000in}{93.170524in}}%
\pgfpathlineto{\pgfqpoint{5.800000in}{93.273361in}}%
\pgfpathlineto{\pgfqpoint{5.800000in}{93.273361in}}%
\pgfpathclose%
\pgfusepath{fill}%
\end{pgfscope}%
\begin{pgfscope}%
\pgfpathrectangle{\pgfqpoint{5.800000in}{0.720000in}}{\pgfqpoint{1.400000in}{4.620000in}}%
\pgfusepath{clip}%
\pgfsetbuttcap%
\pgfsetmiterjoin%
\definecolor{currentfill}{rgb}{0.121569,0.466667,0.705882}%
\pgfsetfillcolor{currentfill}%
\pgfsetlinewidth{0.000000pt}%
\definecolor{currentstroke}{rgb}{0.000000,0.000000,0.000000}%
\pgfsetstrokecolor{currentstroke}%
\pgfsetstrokeopacity{0.000000}%
\pgfsetdash{}{0pt}%
\pgfpathmoveto{\pgfqpoint{5.800000in}{93.273361in}}%
\pgfpathlineto{\pgfqpoint{5.800000in}{93.273361in}}%
\pgfpathlineto{\pgfqpoint{5.800000in}{93.376198in}}%
\pgfpathlineto{\pgfqpoint{5.800000in}{93.376198in}}%
\pgfpathclose%
\pgfusepath{fill}%
\end{pgfscope}%
\begin{pgfscope}%
\pgfpathrectangle{\pgfqpoint{5.800000in}{0.720000in}}{\pgfqpoint{1.400000in}{4.620000in}}%
\pgfusepath{clip}%
\pgfsetbuttcap%
\pgfsetmiterjoin%
\definecolor{currentfill}{rgb}{0.121569,0.466667,0.705882}%
\pgfsetfillcolor{currentfill}%
\pgfsetlinewidth{0.000000pt}%
\definecolor{currentstroke}{rgb}{0.000000,0.000000,0.000000}%
\pgfsetstrokecolor{currentstroke}%
\pgfsetstrokeopacity{0.000000}%
\pgfsetdash{}{0pt}%
\pgfpathmoveto{\pgfqpoint{5.800000in}{93.376198in}}%
\pgfpathlineto{\pgfqpoint{5.800000in}{93.376198in}}%
\pgfpathlineto{\pgfqpoint{5.800000in}{93.479035in}}%
\pgfpathlineto{\pgfqpoint{5.800000in}{93.479035in}}%
\pgfpathclose%
\pgfusepath{fill}%
\end{pgfscope}%
\begin{pgfscope}%
\pgfpathrectangle{\pgfqpoint{5.800000in}{0.720000in}}{\pgfqpoint{1.400000in}{4.620000in}}%
\pgfusepath{clip}%
\pgfsetbuttcap%
\pgfsetmiterjoin%
\definecolor{currentfill}{rgb}{0.121569,0.466667,0.705882}%
\pgfsetfillcolor{currentfill}%
\pgfsetlinewidth{0.000000pt}%
\definecolor{currentstroke}{rgb}{0.000000,0.000000,0.000000}%
\pgfsetstrokecolor{currentstroke}%
\pgfsetstrokeopacity{0.000000}%
\pgfsetdash{}{0pt}%
\pgfpathmoveto{\pgfqpoint{5.800000in}{93.479035in}}%
\pgfpathlineto{\pgfqpoint{5.800000in}{93.479035in}}%
\pgfpathlineto{\pgfqpoint{5.800000in}{93.581872in}}%
\pgfpathlineto{\pgfqpoint{5.800000in}{93.581872in}}%
\pgfpathclose%
\pgfusepath{fill}%
\end{pgfscope}%
\begin{pgfscope}%
\pgfpathrectangle{\pgfqpoint{5.800000in}{0.720000in}}{\pgfqpoint{1.400000in}{4.620000in}}%
\pgfusepath{clip}%
\pgfsetbuttcap%
\pgfsetmiterjoin%
\definecolor{currentfill}{rgb}{0.121569,0.466667,0.705882}%
\pgfsetfillcolor{currentfill}%
\pgfsetlinewidth{0.000000pt}%
\definecolor{currentstroke}{rgb}{0.000000,0.000000,0.000000}%
\pgfsetstrokecolor{currentstroke}%
\pgfsetstrokeopacity{0.000000}%
\pgfsetdash{}{0pt}%
\pgfpathmoveto{\pgfqpoint{5.800000in}{93.581872in}}%
\pgfpathlineto{\pgfqpoint{5.800000in}{93.581872in}}%
\pgfpathlineto{\pgfqpoint{5.800000in}{93.684709in}}%
\pgfpathlineto{\pgfqpoint{5.800000in}{93.684709in}}%
\pgfpathclose%
\pgfusepath{fill}%
\end{pgfscope}%
\begin{pgfscope}%
\pgfpathrectangle{\pgfqpoint{5.800000in}{0.720000in}}{\pgfqpoint{1.400000in}{4.620000in}}%
\pgfusepath{clip}%
\pgfsetbuttcap%
\pgfsetmiterjoin%
\definecolor{currentfill}{rgb}{0.121569,0.466667,0.705882}%
\pgfsetfillcolor{currentfill}%
\pgfsetlinewidth{0.000000pt}%
\definecolor{currentstroke}{rgb}{0.000000,0.000000,0.000000}%
\pgfsetstrokecolor{currentstroke}%
\pgfsetstrokeopacity{0.000000}%
\pgfsetdash{}{0pt}%
\pgfpathmoveto{\pgfqpoint{5.800000in}{93.684709in}}%
\pgfpathlineto{\pgfqpoint{5.800000in}{93.684709in}}%
\pgfpathlineto{\pgfqpoint{5.800000in}{93.787546in}}%
\pgfpathlineto{\pgfqpoint{5.800000in}{93.787546in}}%
\pgfpathclose%
\pgfusepath{fill}%
\end{pgfscope}%
\begin{pgfscope}%
\pgfpathrectangle{\pgfqpoint{5.800000in}{0.720000in}}{\pgfqpoint{1.400000in}{4.620000in}}%
\pgfusepath{clip}%
\pgfsetbuttcap%
\pgfsetmiterjoin%
\definecolor{currentfill}{rgb}{0.121569,0.466667,0.705882}%
\pgfsetfillcolor{currentfill}%
\pgfsetlinewidth{0.000000pt}%
\definecolor{currentstroke}{rgb}{0.000000,0.000000,0.000000}%
\pgfsetstrokecolor{currentstroke}%
\pgfsetstrokeopacity{0.000000}%
\pgfsetdash{}{0pt}%
\pgfpathmoveto{\pgfqpoint{5.800000in}{93.787546in}}%
\pgfpathlineto{\pgfqpoint{5.800000in}{93.787546in}}%
\pgfpathlineto{\pgfqpoint{5.800000in}{93.890383in}}%
\pgfpathlineto{\pgfqpoint{5.800000in}{93.890383in}}%
\pgfpathclose%
\pgfusepath{fill}%
\end{pgfscope}%
\begin{pgfscope}%
\pgfpathrectangle{\pgfqpoint{5.800000in}{0.720000in}}{\pgfqpoint{1.400000in}{4.620000in}}%
\pgfusepath{clip}%
\pgfsetbuttcap%
\pgfsetmiterjoin%
\definecolor{currentfill}{rgb}{0.121569,0.466667,0.705882}%
\pgfsetfillcolor{currentfill}%
\pgfsetlinewidth{0.000000pt}%
\definecolor{currentstroke}{rgb}{0.000000,0.000000,0.000000}%
\pgfsetstrokecolor{currentstroke}%
\pgfsetstrokeopacity{0.000000}%
\pgfsetdash{}{0pt}%
\pgfpathmoveto{\pgfqpoint{5.800000in}{93.890383in}}%
\pgfpathlineto{\pgfqpoint{5.800000in}{93.890383in}}%
\pgfpathlineto{\pgfqpoint{5.800000in}{93.993220in}}%
\pgfpathlineto{\pgfqpoint{5.800000in}{93.993220in}}%
\pgfpathclose%
\pgfusepath{fill}%
\end{pgfscope}%
\begin{pgfscope}%
\pgfpathrectangle{\pgfqpoint{5.800000in}{0.720000in}}{\pgfqpoint{1.400000in}{4.620000in}}%
\pgfusepath{clip}%
\pgfsetbuttcap%
\pgfsetmiterjoin%
\definecolor{currentfill}{rgb}{0.121569,0.466667,0.705882}%
\pgfsetfillcolor{currentfill}%
\pgfsetlinewidth{0.000000pt}%
\definecolor{currentstroke}{rgb}{0.000000,0.000000,0.000000}%
\pgfsetstrokecolor{currentstroke}%
\pgfsetstrokeopacity{0.000000}%
\pgfsetdash{}{0pt}%
\pgfpathmoveto{\pgfqpoint{5.800000in}{93.993220in}}%
\pgfpathlineto{\pgfqpoint{5.800000in}{93.993220in}}%
\pgfpathlineto{\pgfqpoint{5.800000in}{94.096058in}}%
\pgfpathlineto{\pgfqpoint{5.800000in}{94.096058in}}%
\pgfpathclose%
\pgfusepath{fill}%
\end{pgfscope}%
\begin{pgfscope}%
\pgfpathrectangle{\pgfqpoint{5.800000in}{0.720000in}}{\pgfqpoint{1.400000in}{4.620000in}}%
\pgfusepath{clip}%
\pgfsetbuttcap%
\pgfsetmiterjoin%
\definecolor{currentfill}{rgb}{0.121569,0.466667,0.705882}%
\pgfsetfillcolor{currentfill}%
\pgfsetlinewidth{0.000000pt}%
\definecolor{currentstroke}{rgb}{0.000000,0.000000,0.000000}%
\pgfsetstrokecolor{currentstroke}%
\pgfsetstrokeopacity{0.000000}%
\pgfsetdash{}{0pt}%
\pgfpathmoveto{\pgfqpoint{5.800000in}{94.096058in}}%
\pgfpathlineto{\pgfqpoint{5.800000in}{94.096058in}}%
\pgfpathlineto{\pgfqpoint{5.800000in}{94.198895in}}%
\pgfpathlineto{\pgfqpoint{5.800000in}{94.198895in}}%
\pgfpathclose%
\pgfusepath{fill}%
\end{pgfscope}%
\begin{pgfscope}%
\pgfpathrectangle{\pgfqpoint{5.800000in}{0.720000in}}{\pgfqpoint{1.400000in}{4.620000in}}%
\pgfusepath{clip}%
\pgfsetbuttcap%
\pgfsetmiterjoin%
\definecolor{currentfill}{rgb}{0.121569,0.466667,0.705882}%
\pgfsetfillcolor{currentfill}%
\pgfsetlinewidth{0.000000pt}%
\definecolor{currentstroke}{rgb}{0.000000,0.000000,0.000000}%
\pgfsetstrokecolor{currentstroke}%
\pgfsetstrokeopacity{0.000000}%
\pgfsetdash{}{0pt}%
\pgfpathmoveto{\pgfqpoint{5.800000in}{94.198895in}}%
\pgfpathlineto{\pgfqpoint{5.800000in}{94.198895in}}%
\pgfpathlineto{\pgfqpoint{5.800000in}{94.301732in}}%
\pgfpathlineto{\pgfqpoint{5.800000in}{94.301732in}}%
\pgfpathclose%
\pgfusepath{fill}%
\end{pgfscope}%
\begin{pgfscope}%
\pgfpathrectangle{\pgfqpoint{5.800000in}{0.720000in}}{\pgfqpoint{1.400000in}{4.620000in}}%
\pgfusepath{clip}%
\pgfsetbuttcap%
\pgfsetmiterjoin%
\definecolor{currentfill}{rgb}{0.121569,0.466667,0.705882}%
\pgfsetfillcolor{currentfill}%
\pgfsetlinewidth{0.000000pt}%
\definecolor{currentstroke}{rgb}{0.000000,0.000000,0.000000}%
\pgfsetstrokecolor{currentstroke}%
\pgfsetstrokeopacity{0.000000}%
\pgfsetdash{}{0pt}%
\pgfpathmoveto{\pgfqpoint{5.800000in}{94.301732in}}%
\pgfpathlineto{\pgfqpoint{5.800000in}{94.301732in}}%
\pgfpathlineto{\pgfqpoint{5.800000in}{94.404569in}}%
\pgfpathlineto{\pgfqpoint{5.800000in}{94.404569in}}%
\pgfpathclose%
\pgfusepath{fill}%
\end{pgfscope}%
\begin{pgfscope}%
\pgfpathrectangle{\pgfqpoint{5.800000in}{0.720000in}}{\pgfqpoint{1.400000in}{4.620000in}}%
\pgfusepath{clip}%
\pgfsetbuttcap%
\pgfsetmiterjoin%
\definecolor{currentfill}{rgb}{0.121569,0.466667,0.705882}%
\pgfsetfillcolor{currentfill}%
\pgfsetlinewidth{0.000000pt}%
\definecolor{currentstroke}{rgb}{0.000000,0.000000,0.000000}%
\pgfsetstrokecolor{currentstroke}%
\pgfsetstrokeopacity{0.000000}%
\pgfsetdash{}{0pt}%
\pgfpathmoveto{\pgfqpoint{5.800000in}{94.404569in}}%
\pgfpathlineto{\pgfqpoint{5.800000in}{94.404569in}}%
\pgfpathlineto{\pgfqpoint{5.800000in}{94.507406in}}%
\pgfpathlineto{\pgfqpoint{5.800000in}{94.507406in}}%
\pgfpathclose%
\pgfusepath{fill}%
\end{pgfscope}%
\begin{pgfscope}%
\pgfpathrectangle{\pgfqpoint{5.800000in}{0.720000in}}{\pgfqpoint{1.400000in}{4.620000in}}%
\pgfusepath{clip}%
\pgfsetbuttcap%
\pgfsetmiterjoin%
\definecolor{currentfill}{rgb}{0.121569,0.466667,0.705882}%
\pgfsetfillcolor{currentfill}%
\pgfsetlinewidth{0.000000pt}%
\definecolor{currentstroke}{rgb}{0.000000,0.000000,0.000000}%
\pgfsetstrokecolor{currentstroke}%
\pgfsetstrokeopacity{0.000000}%
\pgfsetdash{}{0pt}%
\pgfpathmoveto{\pgfqpoint{5.800000in}{94.507406in}}%
\pgfpathlineto{\pgfqpoint{5.800000in}{94.507406in}}%
\pgfpathlineto{\pgfqpoint{5.800000in}{94.610243in}}%
\pgfpathlineto{\pgfqpoint{5.800000in}{94.610243in}}%
\pgfpathclose%
\pgfusepath{fill}%
\end{pgfscope}%
\begin{pgfscope}%
\pgfpathrectangle{\pgfqpoint{5.800000in}{0.720000in}}{\pgfqpoint{1.400000in}{4.620000in}}%
\pgfusepath{clip}%
\pgfsetbuttcap%
\pgfsetmiterjoin%
\definecolor{currentfill}{rgb}{0.121569,0.466667,0.705882}%
\pgfsetfillcolor{currentfill}%
\pgfsetlinewidth{0.000000pt}%
\definecolor{currentstroke}{rgb}{0.000000,0.000000,0.000000}%
\pgfsetstrokecolor{currentstroke}%
\pgfsetstrokeopacity{0.000000}%
\pgfsetdash{}{0pt}%
\pgfpathmoveto{\pgfqpoint{5.800000in}{94.610243in}}%
\pgfpathlineto{\pgfqpoint{5.800000in}{94.610243in}}%
\pgfpathlineto{\pgfqpoint{5.800000in}{94.713080in}}%
\pgfpathlineto{\pgfqpoint{5.800000in}{94.713080in}}%
\pgfpathclose%
\pgfusepath{fill}%
\end{pgfscope}%
\begin{pgfscope}%
\pgfpathrectangle{\pgfqpoint{5.800000in}{0.720000in}}{\pgfqpoint{1.400000in}{4.620000in}}%
\pgfusepath{clip}%
\pgfsetbuttcap%
\pgfsetmiterjoin%
\definecolor{currentfill}{rgb}{0.121569,0.466667,0.705882}%
\pgfsetfillcolor{currentfill}%
\pgfsetlinewidth{0.000000pt}%
\definecolor{currentstroke}{rgb}{0.000000,0.000000,0.000000}%
\pgfsetstrokecolor{currentstroke}%
\pgfsetstrokeopacity{0.000000}%
\pgfsetdash{}{0pt}%
\pgfpathmoveto{\pgfqpoint{5.800000in}{94.713080in}}%
\pgfpathlineto{\pgfqpoint{5.800000in}{94.713080in}}%
\pgfpathlineto{\pgfqpoint{5.800000in}{94.815917in}}%
\pgfpathlineto{\pgfqpoint{5.800000in}{94.815917in}}%
\pgfpathclose%
\pgfusepath{fill}%
\end{pgfscope}%
\begin{pgfscope}%
\pgfpathrectangle{\pgfqpoint{5.800000in}{0.720000in}}{\pgfqpoint{1.400000in}{4.620000in}}%
\pgfusepath{clip}%
\pgfsetbuttcap%
\pgfsetmiterjoin%
\definecolor{currentfill}{rgb}{0.121569,0.466667,0.705882}%
\pgfsetfillcolor{currentfill}%
\pgfsetlinewidth{0.000000pt}%
\definecolor{currentstroke}{rgb}{0.000000,0.000000,0.000000}%
\pgfsetstrokecolor{currentstroke}%
\pgfsetstrokeopacity{0.000000}%
\pgfsetdash{}{0pt}%
\pgfpathmoveto{\pgfqpoint{5.800000in}{94.815917in}}%
\pgfpathlineto{\pgfqpoint{5.800000in}{94.815917in}}%
\pgfpathlineto{\pgfqpoint{5.800000in}{94.918754in}}%
\pgfpathlineto{\pgfqpoint{5.800000in}{94.918754in}}%
\pgfpathclose%
\pgfusepath{fill}%
\end{pgfscope}%
\begin{pgfscope}%
\pgfpathrectangle{\pgfqpoint{5.800000in}{0.720000in}}{\pgfqpoint{1.400000in}{4.620000in}}%
\pgfusepath{clip}%
\pgfsetbuttcap%
\pgfsetmiterjoin%
\definecolor{currentfill}{rgb}{0.121569,0.466667,0.705882}%
\pgfsetfillcolor{currentfill}%
\pgfsetlinewidth{0.000000pt}%
\definecolor{currentstroke}{rgb}{0.000000,0.000000,0.000000}%
\pgfsetstrokecolor{currentstroke}%
\pgfsetstrokeopacity{0.000000}%
\pgfsetdash{}{0pt}%
\pgfpathmoveto{\pgfqpoint{5.800000in}{94.918754in}}%
\pgfpathlineto{\pgfqpoint{5.800000in}{94.918754in}}%
\pgfpathlineto{\pgfqpoint{5.800000in}{95.021591in}}%
\pgfpathlineto{\pgfqpoint{5.800000in}{95.021591in}}%
\pgfpathclose%
\pgfusepath{fill}%
\end{pgfscope}%
\begin{pgfscope}%
\pgfpathrectangle{\pgfqpoint{5.800000in}{0.720000in}}{\pgfqpoint{1.400000in}{4.620000in}}%
\pgfusepath{clip}%
\pgfsetbuttcap%
\pgfsetmiterjoin%
\definecolor{currentfill}{rgb}{0.121569,0.466667,0.705882}%
\pgfsetfillcolor{currentfill}%
\pgfsetlinewidth{0.000000pt}%
\definecolor{currentstroke}{rgb}{0.000000,0.000000,0.000000}%
\pgfsetstrokecolor{currentstroke}%
\pgfsetstrokeopacity{0.000000}%
\pgfsetdash{}{0pt}%
\pgfpathmoveto{\pgfqpoint{5.800000in}{95.021591in}}%
\pgfpathlineto{\pgfqpoint{5.800000in}{95.021591in}}%
\pgfpathlineto{\pgfqpoint{5.800000in}{95.124428in}}%
\pgfpathlineto{\pgfqpoint{5.800000in}{95.124428in}}%
\pgfpathclose%
\pgfusepath{fill}%
\end{pgfscope}%
\begin{pgfscope}%
\pgfpathrectangle{\pgfqpoint{5.800000in}{0.720000in}}{\pgfqpoint{1.400000in}{4.620000in}}%
\pgfusepath{clip}%
\pgfsetbuttcap%
\pgfsetmiterjoin%
\definecolor{currentfill}{rgb}{0.121569,0.466667,0.705882}%
\pgfsetfillcolor{currentfill}%
\pgfsetlinewidth{0.000000pt}%
\definecolor{currentstroke}{rgb}{0.000000,0.000000,0.000000}%
\pgfsetstrokecolor{currentstroke}%
\pgfsetstrokeopacity{0.000000}%
\pgfsetdash{}{0pt}%
\pgfpathmoveto{\pgfqpoint{5.800000in}{95.124428in}}%
\pgfpathlineto{\pgfqpoint{5.800000in}{95.124428in}}%
\pgfpathlineto{\pgfqpoint{5.800000in}{95.227265in}}%
\pgfpathlineto{\pgfqpoint{5.800000in}{95.227265in}}%
\pgfpathclose%
\pgfusepath{fill}%
\end{pgfscope}%
\begin{pgfscope}%
\pgfpathrectangle{\pgfqpoint{5.800000in}{0.720000in}}{\pgfqpoint{1.400000in}{4.620000in}}%
\pgfusepath{clip}%
\pgfsetbuttcap%
\pgfsetmiterjoin%
\definecolor{currentfill}{rgb}{0.121569,0.466667,0.705882}%
\pgfsetfillcolor{currentfill}%
\pgfsetlinewidth{0.000000pt}%
\definecolor{currentstroke}{rgb}{0.000000,0.000000,0.000000}%
\pgfsetstrokecolor{currentstroke}%
\pgfsetstrokeopacity{0.000000}%
\pgfsetdash{}{0pt}%
\pgfpathmoveto{\pgfqpoint{5.800000in}{95.227265in}}%
\pgfpathlineto{\pgfqpoint{5.800000in}{95.227265in}}%
\pgfpathlineto{\pgfqpoint{5.800000in}{95.330102in}}%
\pgfpathlineto{\pgfqpoint{5.800000in}{95.330102in}}%
\pgfpathclose%
\pgfusepath{fill}%
\end{pgfscope}%
\begin{pgfscope}%
\pgfpathrectangle{\pgfqpoint{5.800000in}{0.720000in}}{\pgfqpoint{1.400000in}{4.620000in}}%
\pgfusepath{clip}%
\pgfsetbuttcap%
\pgfsetmiterjoin%
\definecolor{currentfill}{rgb}{0.121569,0.466667,0.705882}%
\pgfsetfillcolor{currentfill}%
\pgfsetlinewidth{0.000000pt}%
\definecolor{currentstroke}{rgb}{0.000000,0.000000,0.000000}%
\pgfsetstrokecolor{currentstroke}%
\pgfsetstrokeopacity{0.000000}%
\pgfsetdash{}{0pt}%
\pgfpathmoveto{\pgfqpoint{5.800000in}{95.330102in}}%
\pgfpathlineto{\pgfqpoint{5.800000in}{95.330102in}}%
\pgfpathlineto{\pgfqpoint{5.800000in}{95.432939in}}%
\pgfpathlineto{\pgfqpoint{5.800000in}{95.432939in}}%
\pgfpathclose%
\pgfusepath{fill}%
\end{pgfscope}%
\begin{pgfscope}%
\pgfpathrectangle{\pgfqpoint{5.800000in}{0.720000in}}{\pgfqpoint{1.400000in}{4.620000in}}%
\pgfusepath{clip}%
\pgfsetbuttcap%
\pgfsetmiterjoin%
\definecolor{currentfill}{rgb}{0.121569,0.466667,0.705882}%
\pgfsetfillcolor{currentfill}%
\pgfsetlinewidth{0.000000pt}%
\definecolor{currentstroke}{rgb}{0.000000,0.000000,0.000000}%
\pgfsetstrokecolor{currentstroke}%
\pgfsetstrokeopacity{0.000000}%
\pgfsetdash{}{0pt}%
\pgfpathmoveto{\pgfqpoint{5.800000in}{95.432939in}}%
\pgfpathlineto{\pgfqpoint{5.800000in}{95.432939in}}%
\pgfpathlineto{\pgfqpoint{5.800000in}{95.535776in}}%
\pgfpathlineto{\pgfqpoint{5.800000in}{95.535776in}}%
\pgfpathclose%
\pgfusepath{fill}%
\end{pgfscope}%
\begin{pgfscope}%
\pgfpathrectangle{\pgfqpoint{5.800000in}{0.720000in}}{\pgfqpoint{1.400000in}{4.620000in}}%
\pgfusepath{clip}%
\pgfsetbuttcap%
\pgfsetmiterjoin%
\definecolor{currentfill}{rgb}{0.121569,0.466667,0.705882}%
\pgfsetfillcolor{currentfill}%
\pgfsetlinewidth{0.000000pt}%
\definecolor{currentstroke}{rgb}{0.000000,0.000000,0.000000}%
\pgfsetstrokecolor{currentstroke}%
\pgfsetstrokeopacity{0.000000}%
\pgfsetdash{}{0pt}%
\pgfpathmoveto{\pgfqpoint{5.800000in}{95.535776in}}%
\pgfpathlineto{\pgfqpoint{5.800000in}{95.535776in}}%
\pgfpathlineto{\pgfqpoint{5.800000in}{95.638614in}}%
\pgfpathlineto{\pgfqpoint{5.800000in}{95.638614in}}%
\pgfpathclose%
\pgfusepath{fill}%
\end{pgfscope}%
\begin{pgfscope}%
\pgfpathrectangle{\pgfqpoint{5.800000in}{0.720000in}}{\pgfqpoint{1.400000in}{4.620000in}}%
\pgfusepath{clip}%
\pgfsetbuttcap%
\pgfsetmiterjoin%
\definecolor{currentfill}{rgb}{0.121569,0.466667,0.705882}%
\pgfsetfillcolor{currentfill}%
\pgfsetlinewidth{0.000000pt}%
\definecolor{currentstroke}{rgb}{0.000000,0.000000,0.000000}%
\pgfsetstrokecolor{currentstroke}%
\pgfsetstrokeopacity{0.000000}%
\pgfsetdash{}{0pt}%
\pgfpathmoveto{\pgfqpoint{5.800000in}{95.638614in}}%
\pgfpathlineto{\pgfqpoint{5.800000in}{95.638614in}}%
\pgfpathlineto{\pgfqpoint{5.800000in}{95.741451in}}%
\pgfpathlineto{\pgfqpoint{5.800000in}{95.741451in}}%
\pgfpathclose%
\pgfusepath{fill}%
\end{pgfscope}%
\begin{pgfscope}%
\pgfpathrectangle{\pgfqpoint{5.800000in}{0.720000in}}{\pgfqpoint{1.400000in}{4.620000in}}%
\pgfusepath{clip}%
\pgfsetbuttcap%
\pgfsetmiterjoin%
\definecolor{currentfill}{rgb}{0.121569,0.466667,0.705882}%
\pgfsetfillcolor{currentfill}%
\pgfsetlinewidth{0.000000pt}%
\definecolor{currentstroke}{rgb}{0.000000,0.000000,0.000000}%
\pgfsetstrokecolor{currentstroke}%
\pgfsetstrokeopacity{0.000000}%
\pgfsetdash{}{0pt}%
\pgfpathmoveto{\pgfqpoint{5.800000in}{95.741451in}}%
\pgfpathlineto{\pgfqpoint{5.800000in}{95.741451in}}%
\pgfpathlineto{\pgfqpoint{5.800000in}{95.844288in}}%
\pgfpathlineto{\pgfqpoint{5.800000in}{95.844288in}}%
\pgfpathclose%
\pgfusepath{fill}%
\end{pgfscope}%
\begin{pgfscope}%
\pgfpathrectangle{\pgfqpoint{5.800000in}{0.720000in}}{\pgfqpoint{1.400000in}{4.620000in}}%
\pgfusepath{clip}%
\pgfsetbuttcap%
\pgfsetmiterjoin%
\definecolor{currentfill}{rgb}{0.121569,0.466667,0.705882}%
\pgfsetfillcolor{currentfill}%
\pgfsetlinewidth{0.000000pt}%
\definecolor{currentstroke}{rgb}{0.000000,0.000000,0.000000}%
\pgfsetstrokecolor{currentstroke}%
\pgfsetstrokeopacity{0.000000}%
\pgfsetdash{}{0pt}%
\pgfpathmoveto{\pgfqpoint{5.800000in}{95.844288in}}%
\pgfpathlineto{\pgfqpoint{5.800000in}{95.844288in}}%
\pgfpathlineto{\pgfqpoint{5.800000in}{95.947125in}}%
\pgfpathlineto{\pgfqpoint{5.800000in}{95.947125in}}%
\pgfpathclose%
\pgfusepath{fill}%
\end{pgfscope}%
\begin{pgfscope}%
\pgfpathrectangle{\pgfqpoint{5.800000in}{0.720000in}}{\pgfqpoint{1.400000in}{4.620000in}}%
\pgfusepath{clip}%
\pgfsetbuttcap%
\pgfsetmiterjoin%
\definecolor{currentfill}{rgb}{0.121569,0.466667,0.705882}%
\pgfsetfillcolor{currentfill}%
\pgfsetlinewidth{0.000000pt}%
\definecolor{currentstroke}{rgb}{0.000000,0.000000,0.000000}%
\pgfsetstrokecolor{currentstroke}%
\pgfsetstrokeopacity{0.000000}%
\pgfsetdash{}{0pt}%
\pgfpathmoveto{\pgfqpoint{5.800000in}{95.947125in}}%
\pgfpathlineto{\pgfqpoint{5.800000in}{95.947125in}}%
\pgfpathlineto{\pgfqpoint{5.800000in}{96.049962in}}%
\pgfpathlineto{\pgfqpoint{5.800000in}{96.049962in}}%
\pgfpathclose%
\pgfusepath{fill}%
\end{pgfscope}%
\begin{pgfscope}%
\pgfpathrectangle{\pgfqpoint{5.800000in}{0.720000in}}{\pgfqpoint{1.400000in}{4.620000in}}%
\pgfusepath{clip}%
\pgfsetbuttcap%
\pgfsetmiterjoin%
\definecolor{currentfill}{rgb}{0.121569,0.466667,0.705882}%
\pgfsetfillcolor{currentfill}%
\pgfsetlinewidth{0.000000pt}%
\definecolor{currentstroke}{rgb}{0.000000,0.000000,0.000000}%
\pgfsetstrokecolor{currentstroke}%
\pgfsetstrokeopacity{0.000000}%
\pgfsetdash{}{0pt}%
\pgfpathmoveto{\pgfqpoint{5.800000in}{96.049962in}}%
\pgfpathlineto{\pgfqpoint{5.800000in}{96.049962in}}%
\pgfpathlineto{\pgfqpoint{5.800000in}{96.152799in}}%
\pgfpathlineto{\pgfqpoint{5.800000in}{96.152799in}}%
\pgfpathclose%
\pgfusepath{fill}%
\end{pgfscope}%
\begin{pgfscope}%
\pgfpathrectangle{\pgfqpoint{5.800000in}{0.720000in}}{\pgfqpoint{1.400000in}{4.620000in}}%
\pgfusepath{clip}%
\pgfsetbuttcap%
\pgfsetmiterjoin%
\definecolor{currentfill}{rgb}{0.121569,0.466667,0.705882}%
\pgfsetfillcolor{currentfill}%
\pgfsetlinewidth{0.000000pt}%
\definecolor{currentstroke}{rgb}{0.000000,0.000000,0.000000}%
\pgfsetstrokecolor{currentstroke}%
\pgfsetstrokeopacity{0.000000}%
\pgfsetdash{}{0pt}%
\pgfpathmoveto{\pgfqpoint{5.800000in}{96.152799in}}%
\pgfpathlineto{\pgfqpoint{5.800000in}{96.152799in}}%
\pgfpathlineto{\pgfqpoint{5.800000in}{96.255636in}}%
\pgfpathlineto{\pgfqpoint{5.800000in}{96.255636in}}%
\pgfpathclose%
\pgfusepath{fill}%
\end{pgfscope}%
\begin{pgfscope}%
\pgfpathrectangle{\pgfqpoint{5.800000in}{0.720000in}}{\pgfqpoint{1.400000in}{4.620000in}}%
\pgfusepath{clip}%
\pgfsetbuttcap%
\pgfsetmiterjoin%
\definecolor{currentfill}{rgb}{0.121569,0.466667,0.705882}%
\pgfsetfillcolor{currentfill}%
\pgfsetlinewidth{0.000000pt}%
\definecolor{currentstroke}{rgb}{0.000000,0.000000,0.000000}%
\pgfsetstrokecolor{currentstroke}%
\pgfsetstrokeopacity{0.000000}%
\pgfsetdash{}{0pt}%
\pgfpathmoveto{\pgfqpoint{5.800000in}{96.255636in}}%
\pgfpathlineto{\pgfqpoint{5.800000in}{96.255636in}}%
\pgfpathlineto{\pgfqpoint{5.800000in}{96.358473in}}%
\pgfpathlineto{\pgfqpoint{5.800000in}{96.358473in}}%
\pgfpathclose%
\pgfusepath{fill}%
\end{pgfscope}%
\begin{pgfscope}%
\pgfpathrectangle{\pgfqpoint{5.800000in}{0.720000in}}{\pgfqpoint{1.400000in}{4.620000in}}%
\pgfusepath{clip}%
\pgfsetbuttcap%
\pgfsetmiterjoin%
\definecolor{currentfill}{rgb}{0.121569,0.466667,0.705882}%
\pgfsetfillcolor{currentfill}%
\pgfsetlinewidth{0.000000pt}%
\definecolor{currentstroke}{rgb}{0.000000,0.000000,0.000000}%
\pgfsetstrokecolor{currentstroke}%
\pgfsetstrokeopacity{0.000000}%
\pgfsetdash{}{0pt}%
\pgfpathmoveto{\pgfqpoint{5.800000in}{96.358473in}}%
\pgfpathlineto{\pgfqpoint{5.800000in}{96.358473in}}%
\pgfpathlineto{\pgfqpoint{5.800000in}{96.461310in}}%
\pgfpathlineto{\pgfqpoint{5.800000in}{96.461310in}}%
\pgfpathclose%
\pgfusepath{fill}%
\end{pgfscope}%
\begin{pgfscope}%
\pgfpathrectangle{\pgfqpoint{5.800000in}{0.720000in}}{\pgfqpoint{1.400000in}{4.620000in}}%
\pgfusepath{clip}%
\pgfsetbuttcap%
\pgfsetmiterjoin%
\definecolor{currentfill}{rgb}{0.121569,0.466667,0.705882}%
\pgfsetfillcolor{currentfill}%
\pgfsetlinewidth{0.000000pt}%
\definecolor{currentstroke}{rgb}{0.000000,0.000000,0.000000}%
\pgfsetstrokecolor{currentstroke}%
\pgfsetstrokeopacity{0.000000}%
\pgfsetdash{}{0pt}%
\pgfpathmoveto{\pgfqpoint{5.800000in}{96.461310in}}%
\pgfpathlineto{\pgfqpoint{5.800000in}{96.461310in}}%
\pgfpathlineto{\pgfqpoint{5.800000in}{96.564147in}}%
\pgfpathlineto{\pgfqpoint{5.800000in}{96.564147in}}%
\pgfpathclose%
\pgfusepath{fill}%
\end{pgfscope}%
\begin{pgfscope}%
\pgfpathrectangle{\pgfqpoint{5.800000in}{0.720000in}}{\pgfqpoint{1.400000in}{4.620000in}}%
\pgfusepath{clip}%
\pgfsetbuttcap%
\pgfsetmiterjoin%
\definecolor{currentfill}{rgb}{0.121569,0.466667,0.705882}%
\pgfsetfillcolor{currentfill}%
\pgfsetlinewidth{0.000000pt}%
\definecolor{currentstroke}{rgb}{0.000000,0.000000,0.000000}%
\pgfsetstrokecolor{currentstroke}%
\pgfsetstrokeopacity{0.000000}%
\pgfsetdash{}{0pt}%
\pgfpathmoveto{\pgfqpoint{5.800000in}{96.564147in}}%
\pgfpathlineto{\pgfqpoint{5.800000in}{96.564147in}}%
\pgfpathlineto{\pgfqpoint{5.800000in}{96.666984in}}%
\pgfpathlineto{\pgfqpoint{5.800000in}{96.666984in}}%
\pgfpathclose%
\pgfusepath{fill}%
\end{pgfscope}%
\begin{pgfscope}%
\pgfpathrectangle{\pgfqpoint{5.800000in}{0.720000in}}{\pgfqpoint{1.400000in}{4.620000in}}%
\pgfusepath{clip}%
\pgfsetbuttcap%
\pgfsetmiterjoin%
\definecolor{currentfill}{rgb}{0.121569,0.466667,0.705882}%
\pgfsetfillcolor{currentfill}%
\pgfsetlinewidth{0.000000pt}%
\definecolor{currentstroke}{rgb}{0.000000,0.000000,0.000000}%
\pgfsetstrokecolor{currentstroke}%
\pgfsetstrokeopacity{0.000000}%
\pgfsetdash{}{0pt}%
\pgfpathmoveto{\pgfqpoint{5.800000in}{96.666984in}}%
\pgfpathlineto{\pgfqpoint{5.800000in}{96.666984in}}%
\pgfpathlineto{\pgfqpoint{5.800000in}{96.769821in}}%
\pgfpathlineto{\pgfqpoint{5.800000in}{96.769821in}}%
\pgfpathclose%
\pgfusepath{fill}%
\end{pgfscope}%
\begin{pgfscope}%
\pgfpathrectangle{\pgfqpoint{5.800000in}{0.720000in}}{\pgfqpoint{1.400000in}{4.620000in}}%
\pgfusepath{clip}%
\pgfsetbuttcap%
\pgfsetmiterjoin%
\definecolor{currentfill}{rgb}{0.121569,0.466667,0.705882}%
\pgfsetfillcolor{currentfill}%
\pgfsetlinewidth{0.000000pt}%
\definecolor{currentstroke}{rgb}{0.000000,0.000000,0.000000}%
\pgfsetstrokecolor{currentstroke}%
\pgfsetstrokeopacity{0.000000}%
\pgfsetdash{}{0pt}%
\pgfpathmoveto{\pgfqpoint{5.800000in}{96.769821in}}%
\pgfpathlineto{\pgfqpoint{5.800000in}{96.769821in}}%
\pgfpathlineto{\pgfqpoint{5.800000in}{96.872658in}}%
\pgfpathlineto{\pgfqpoint{5.800000in}{96.872658in}}%
\pgfpathclose%
\pgfusepath{fill}%
\end{pgfscope}%
\begin{pgfscope}%
\pgfpathrectangle{\pgfqpoint{5.800000in}{0.720000in}}{\pgfqpoint{1.400000in}{4.620000in}}%
\pgfusepath{clip}%
\pgfsetbuttcap%
\pgfsetmiterjoin%
\definecolor{currentfill}{rgb}{0.121569,0.466667,0.705882}%
\pgfsetfillcolor{currentfill}%
\pgfsetlinewidth{0.000000pt}%
\definecolor{currentstroke}{rgb}{0.000000,0.000000,0.000000}%
\pgfsetstrokecolor{currentstroke}%
\pgfsetstrokeopacity{0.000000}%
\pgfsetdash{}{0pt}%
\pgfpathmoveto{\pgfqpoint{5.800000in}{96.872658in}}%
\pgfpathlineto{\pgfqpoint{5.800000in}{96.872658in}}%
\pgfpathlineto{\pgfqpoint{5.800000in}{96.975495in}}%
\pgfpathlineto{\pgfqpoint{5.800000in}{96.975495in}}%
\pgfpathclose%
\pgfusepath{fill}%
\end{pgfscope}%
\begin{pgfscope}%
\pgfpathrectangle{\pgfqpoint{5.800000in}{0.720000in}}{\pgfqpoint{1.400000in}{4.620000in}}%
\pgfusepath{clip}%
\pgfsetbuttcap%
\pgfsetmiterjoin%
\definecolor{currentfill}{rgb}{0.121569,0.466667,0.705882}%
\pgfsetfillcolor{currentfill}%
\pgfsetlinewidth{0.000000pt}%
\definecolor{currentstroke}{rgb}{0.000000,0.000000,0.000000}%
\pgfsetstrokecolor{currentstroke}%
\pgfsetstrokeopacity{0.000000}%
\pgfsetdash{}{0pt}%
\pgfpathmoveto{\pgfqpoint{5.800000in}{96.975495in}}%
\pgfpathlineto{\pgfqpoint{5.800000in}{96.975495in}}%
\pgfpathlineto{\pgfqpoint{5.800000in}{97.078333in}}%
\pgfpathlineto{\pgfqpoint{5.800000in}{97.078333in}}%
\pgfpathclose%
\pgfusepath{fill}%
\end{pgfscope}%
\begin{pgfscope}%
\pgfpathrectangle{\pgfqpoint{5.800000in}{0.720000in}}{\pgfqpoint{1.400000in}{4.620000in}}%
\pgfusepath{clip}%
\pgfsetbuttcap%
\pgfsetmiterjoin%
\definecolor{currentfill}{rgb}{0.121569,0.466667,0.705882}%
\pgfsetfillcolor{currentfill}%
\pgfsetlinewidth{0.000000pt}%
\definecolor{currentstroke}{rgb}{0.000000,0.000000,0.000000}%
\pgfsetstrokecolor{currentstroke}%
\pgfsetstrokeopacity{0.000000}%
\pgfsetdash{}{0pt}%
\pgfpathmoveto{\pgfqpoint{5.800000in}{97.078333in}}%
\pgfpathlineto{\pgfqpoint{5.800000in}{97.078333in}}%
\pgfpathlineto{\pgfqpoint{5.800000in}{97.181170in}}%
\pgfpathlineto{\pgfqpoint{5.800000in}{97.181170in}}%
\pgfpathclose%
\pgfusepath{fill}%
\end{pgfscope}%
\begin{pgfscope}%
\pgfpathrectangle{\pgfqpoint{5.800000in}{0.720000in}}{\pgfqpoint{1.400000in}{4.620000in}}%
\pgfusepath{clip}%
\pgfsetbuttcap%
\pgfsetmiterjoin%
\definecolor{currentfill}{rgb}{0.121569,0.466667,0.705882}%
\pgfsetfillcolor{currentfill}%
\pgfsetlinewidth{0.000000pt}%
\definecolor{currentstroke}{rgb}{0.000000,0.000000,0.000000}%
\pgfsetstrokecolor{currentstroke}%
\pgfsetstrokeopacity{0.000000}%
\pgfsetdash{}{0pt}%
\pgfpathmoveto{\pgfqpoint{5.800000in}{97.181170in}}%
\pgfpathlineto{\pgfqpoint{5.800000in}{97.181170in}}%
\pgfpathlineto{\pgfqpoint{5.800000in}{97.284007in}}%
\pgfpathlineto{\pgfqpoint{5.800000in}{97.284007in}}%
\pgfpathclose%
\pgfusepath{fill}%
\end{pgfscope}%
\begin{pgfscope}%
\pgfpathrectangle{\pgfqpoint{5.800000in}{0.720000in}}{\pgfqpoint{1.400000in}{4.620000in}}%
\pgfusepath{clip}%
\pgfsetbuttcap%
\pgfsetmiterjoin%
\definecolor{currentfill}{rgb}{0.121569,0.466667,0.705882}%
\pgfsetfillcolor{currentfill}%
\pgfsetlinewidth{0.000000pt}%
\definecolor{currentstroke}{rgb}{0.000000,0.000000,0.000000}%
\pgfsetstrokecolor{currentstroke}%
\pgfsetstrokeopacity{0.000000}%
\pgfsetdash{}{0pt}%
\pgfpathmoveto{\pgfqpoint{5.800000in}{97.284007in}}%
\pgfpathlineto{\pgfqpoint{5.800000in}{97.284007in}}%
\pgfpathlineto{\pgfqpoint{5.800000in}{97.386844in}}%
\pgfpathlineto{\pgfqpoint{5.800000in}{97.386844in}}%
\pgfpathclose%
\pgfusepath{fill}%
\end{pgfscope}%
\begin{pgfscope}%
\pgfpathrectangle{\pgfqpoint{5.800000in}{0.720000in}}{\pgfqpoint{1.400000in}{4.620000in}}%
\pgfusepath{clip}%
\pgfsetbuttcap%
\pgfsetmiterjoin%
\definecolor{currentfill}{rgb}{0.121569,0.466667,0.705882}%
\pgfsetfillcolor{currentfill}%
\pgfsetlinewidth{0.000000pt}%
\definecolor{currentstroke}{rgb}{0.000000,0.000000,0.000000}%
\pgfsetstrokecolor{currentstroke}%
\pgfsetstrokeopacity{0.000000}%
\pgfsetdash{}{0pt}%
\pgfpathmoveto{\pgfqpoint{5.800000in}{97.386844in}}%
\pgfpathlineto{\pgfqpoint{5.800000in}{97.386844in}}%
\pgfpathlineto{\pgfqpoint{5.800000in}{97.489681in}}%
\pgfpathlineto{\pgfqpoint{5.800000in}{97.489681in}}%
\pgfpathclose%
\pgfusepath{fill}%
\end{pgfscope}%
\begin{pgfscope}%
\pgfpathrectangle{\pgfqpoint{5.800000in}{0.720000in}}{\pgfqpoint{1.400000in}{4.620000in}}%
\pgfusepath{clip}%
\pgfsetbuttcap%
\pgfsetmiterjoin%
\definecolor{currentfill}{rgb}{0.121569,0.466667,0.705882}%
\pgfsetfillcolor{currentfill}%
\pgfsetlinewidth{0.000000pt}%
\definecolor{currentstroke}{rgb}{0.000000,0.000000,0.000000}%
\pgfsetstrokecolor{currentstroke}%
\pgfsetstrokeopacity{0.000000}%
\pgfsetdash{}{0pt}%
\pgfpathmoveto{\pgfqpoint{5.800000in}{97.489681in}}%
\pgfpathlineto{\pgfqpoint{5.800000in}{97.489681in}}%
\pgfpathlineto{\pgfqpoint{5.800000in}{97.592518in}}%
\pgfpathlineto{\pgfqpoint{5.800000in}{97.592518in}}%
\pgfpathclose%
\pgfusepath{fill}%
\end{pgfscope}%
\begin{pgfscope}%
\pgfpathrectangle{\pgfqpoint{5.800000in}{0.720000in}}{\pgfqpoint{1.400000in}{4.620000in}}%
\pgfusepath{clip}%
\pgfsetbuttcap%
\pgfsetmiterjoin%
\definecolor{currentfill}{rgb}{0.121569,0.466667,0.705882}%
\pgfsetfillcolor{currentfill}%
\pgfsetlinewidth{0.000000pt}%
\definecolor{currentstroke}{rgb}{0.000000,0.000000,0.000000}%
\pgfsetstrokecolor{currentstroke}%
\pgfsetstrokeopacity{0.000000}%
\pgfsetdash{}{0pt}%
\pgfpathmoveto{\pgfqpoint{5.800000in}{97.592518in}}%
\pgfpathlineto{\pgfqpoint{5.800000in}{97.592518in}}%
\pgfpathlineto{\pgfqpoint{5.800000in}{97.695355in}}%
\pgfpathlineto{\pgfqpoint{5.800000in}{97.695355in}}%
\pgfpathclose%
\pgfusepath{fill}%
\end{pgfscope}%
\begin{pgfscope}%
\pgfpathrectangle{\pgfqpoint{5.800000in}{0.720000in}}{\pgfqpoint{1.400000in}{4.620000in}}%
\pgfusepath{clip}%
\pgfsetbuttcap%
\pgfsetmiterjoin%
\definecolor{currentfill}{rgb}{0.121569,0.466667,0.705882}%
\pgfsetfillcolor{currentfill}%
\pgfsetlinewidth{0.000000pt}%
\definecolor{currentstroke}{rgb}{0.000000,0.000000,0.000000}%
\pgfsetstrokecolor{currentstroke}%
\pgfsetstrokeopacity{0.000000}%
\pgfsetdash{}{0pt}%
\pgfpathmoveto{\pgfqpoint{5.800000in}{97.695355in}}%
\pgfpathlineto{\pgfqpoint{5.800000in}{97.695355in}}%
\pgfpathlineto{\pgfqpoint{5.800000in}{97.798192in}}%
\pgfpathlineto{\pgfqpoint{5.800000in}{97.798192in}}%
\pgfpathclose%
\pgfusepath{fill}%
\end{pgfscope}%
\begin{pgfscope}%
\pgfpathrectangle{\pgfqpoint{5.800000in}{0.720000in}}{\pgfqpoint{1.400000in}{4.620000in}}%
\pgfusepath{clip}%
\pgfsetbuttcap%
\pgfsetmiterjoin%
\definecolor{currentfill}{rgb}{0.121569,0.466667,0.705882}%
\pgfsetfillcolor{currentfill}%
\pgfsetlinewidth{0.000000pt}%
\definecolor{currentstroke}{rgb}{0.000000,0.000000,0.000000}%
\pgfsetstrokecolor{currentstroke}%
\pgfsetstrokeopacity{0.000000}%
\pgfsetdash{}{0pt}%
\pgfpathmoveto{\pgfqpoint{5.800000in}{97.798192in}}%
\pgfpathlineto{\pgfqpoint{5.800000in}{97.798192in}}%
\pgfpathlineto{\pgfqpoint{5.800000in}{97.901029in}}%
\pgfpathlineto{\pgfqpoint{5.800000in}{97.901029in}}%
\pgfpathclose%
\pgfusepath{fill}%
\end{pgfscope}%
\begin{pgfscope}%
\pgfpathrectangle{\pgfqpoint{5.800000in}{0.720000in}}{\pgfqpoint{1.400000in}{4.620000in}}%
\pgfusepath{clip}%
\pgfsetbuttcap%
\pgfsetmiterjoin%
\definecolor{currentfill}{rgb}{0.121569,0.466667,0.705882}%
\pgfsetfillcolor{currentfill}%
\pgfsetlinewidth{0.000000pt}%
\definecolor{currentstroke}{rgb}{0.000000,0.000000,0.000000}%
\pgfsetstrokecolor{currentstroke}%
\pgfsetstrokeopacity{0.000000}%
\pgfsetdash{}{0pt}%
\pgfpathmoveto{\pgfqpoint{5.800000in}{97.901029in}}%
\pgfpathlineto{\pgfqpoint{5.800000in}{97.901029in}}%
\pgfpathlineto{\pgfqpoint{5.800000in}{98.003866in}}%
\pgfpathlineto{\pgfqpoint{5.800000in}{98.003866in}}%
\pgfpathclose%
\pgfusepath{fill}%
\end{pgfscope}%
\begin{pgfscope}%
\pgfpathrectangle{\pgfqpoint{5.800000in}{0.720000in}}{\pgfqpoint{1.400000in}{4.620000in}}%
\pgfusepath{clip}%
\pgfsetbuttcap%
\pgfsetmiterjoin%
\definecolor{currentfill}{rgb}{0.121569,0.466667,0.705882}%
\pgfsetfillcolor{currentfill}%
\pgfsetlinewidth{0.000000pt}%
\definecolor{currentstroke}{rgb}{0.000000,0.000000,0.000000}%
\pgfsetstrokecolor{currentstroke}%
\pgfsetstrokeopacity{0.000000}%
\pgfsetdash{}{0pt}%
\pgfpathmoveto{\pgfqpoint{5.800000in}{98.003866in}}%
\pgfpathlineto{\pgfqpoint{5.800000in}{98.003866in}}%
\pgfpathlineto{\pgfqpoint{5.800000in}{98.106703in}}%
\pgfpathlineto{\pgfqpoint{5.800000in}{98.106703in}}%
\pgfpathclose%
\pgfusepath{fill}%
\end{pgfscope}%
\begin{pgfscope}%
\pgfpathrectangle{\pgfqpoint{5.800000in}{0.720000in}}{\pgfqpoint{1.400000in}{4.620000in}}%
\pgfusepath{clip}%
\pgfsetbuttcap%
\pgfsetmiterjoin%
\definecolor{currentfill}{rgb}{0.121569,0.466667,0.705882}%
\pgfsetfillcolor{currentfill}%
\pgfsetlinewidth{0.000000pt}%
\definecolor{currentstroke}{rgb}{0.000000,0.000000,0.000000}%
\pgfsetstrokecolor{currentstroke}%
\pgfsetstrokeopacity{0.000000}%
\pgfsetdash{}{0pt}%
\pgfpathmoveto{\pgfqpoint{5.800000in}{98.106703in}}%
\pgfpathlineto{\pgfqpoint{5.800000in}{98.106703in}}%
\pgfpathlineto{\pgfqpoint{5.800000in}{98.209540in}}%
\pgfpathlineto{\pgfqpoint{5.800000in}{98.209540in}}%
\pgfpathclose%
\pgfusepath{fill}%
\end{pgfscope}%
\begin{pgfscope}%
\pgfpathrectangle{\pgfqpoint{5.800000in}{0.720000in}}{\pgfqpoint{1.400000in}{4.620000in}}%
\pgfusepath{clip}%
\pgfsetbuttcap%
\pgfsetmiterjoin%
\definecolor{currentfill}{rgb}{0.121569,0.466667,0.705882}%
\pgfsetfillcolor{currentfill}%
\pgfsetlinewidth{0.000000pt}%
\definecolor{currentstroke}{rgb}{0.000000,0.000000,0.000000}%
\pgfsetstrokecolor{currentstroke}%
\pgfsetstrokeopacity{0.000000}%
\pgfsetdash{}{0pt}%
\pgfpathmoveto{\pgfqpoint{5.800000in}{98.209540in}}%
\pgfpathlineto{\pgfqpoint{5.800000in}{98.209540in}}%
\pgfpathlineto{\pgfqpoint{5.800000in}{98.312377in}}%
\pgfpathlineto{\pgfqpoint{5.800000in}{98.312377in}}%
\pgfpathclose%
\pgfusepath{fill}%
\end{pgfscope}%
\begin{pgfscope}%
\pgfpathrectangle{\pgfqpoint{5.800000in}{0.720000in}}{\pgfqpoint{1.400000in}{4.620000in}}%
\pgfusepath{clip}%
\pgfsetbuttcap%
\pgfsetmiterjoin%
\definecolor{currentfill}{rgb}{0.121569,0.466667,0.705882}%
\pgfsetfillcolor{currentfill}%
\pgfsetlinewidth{0.000000pt}%
\definecolor{currentstroke}{rgb}{0.000000,0.000000,0.000000}%
\pgfsetstrokecolor{currentstroke}%
\pgfsetstrokeopacity{0.000000}%
\pgfsetdash{}{0pt}%
\pgfpathmoveto{\pgfqpoint{5.800000in}{98.312377in}}%
\pgfpathlineto{\pgfqpoint{5.800000in}{98.312377in}}%
\pgfpathlineto{\pgfqpoint{5.800000in}{98.415214in}}%
\pgfpathlineto{\pgfqpoint{5.800000in}{98.415214in}}%
\pgfpathclose%
\pgfusepath{fill}%
\end{pgfscope}%
\begin{pgfscope}%
\pgfpathrectangle{\pgfqpoint{5.800000in}{0.720000in}}{\pgfqpoint{1.400000in}{4.620000in}}%
\pgfusepath{clip}%
\pgfsetbuttcap%
\pgfsetmiterjoin%
\definecolor{currentfill}{rgb}{0.121569,0.466667,0.705882}%
\pgfsetfillcolor{currentfill}%
\pgfsetlinewidth{0.000000pt}%
\definecolor{currentstroke}{rgb}{0.000000,0.000000,0.000000}%
\pgfsetstrokecolor{currentstroke}%
\pgfsetstrokeopacity{0.000000}%
\pgfsetdash{}{0pt}%
\pgfpathmoveto{\pgfqpoint{5.800000in}{98.415214in}}%
\pgfpathlineto{\pgfqpoint{5.800000in}{98.415214in}}%
\pgfpathlineto{\pgfqpoint{5.800000in}{98.518051in}}%
\pgfpathlineto{\pgfqpoint{5.800000in}{98.518051in}}%
\pgfpathclose%
\pgfusepath{fill}%
\end{pgfscope}%
\begin{pgfscope}%
\pgfpathrectangle{\pgfqpoint{5.800000in}{0.720000in}}{\pgfqpoint{1.400000in}{4.620000in}}%
\pgfusepath{clip}%
\pgfsetbuttcap%
\pgfsetmiterjoin%
\definecolor{currentfill}{rgb}{0.121569,0.466667,0.705882}%
\pgfsetfillcolor{currentfill}%
\pgfsetlinewidth{0.000000pt}%
\definecolor{currentstroke}{rgb}{0.000000,0.000000,0.000000}%
\pgfsetstrokecolor{currentstroke}%
\pgfsetstrokeopacity{0.000000}%
\pgfsetdash{}{0pt}%
\pgfpathmoveto{\pgfqpoint{5.800000in}{98.518051in}}%
\pgfpathlineto{\pgfqpoint{5.800000in}{98.518051in}}%
\pgfpathlineto{\pgfqpoint{5.800000in}{98.620889in}}%
\pgfpathlineto{\pgfqpoint{5.800000in}{98.620889in}}%
\pgfpathclose%
\pgfusepath{fill}%
\end{pgfscope}%
\begin{pgfscope}%
\pgfpathrectangle{\pgfqpoint{5.800000in}{0.720000in}}{\pgfqpoint{1.400000in}{4.620000in}}%
\pgfusepath{clip}%
\pgfsetbuttcap%
\pgfsetmiterjoin%
\definecolor{currentfill}{rgb}{0.121569,0.466667,0.705882}%
\pgfsetfillcolor{currentfill}%
\pgfsetlinewidth{0.000000pt}%
\definecolor{currentstroke}{rgb}{0.000000,0.000000,0.000000}%
\pgfsetstrokecolor{currentstroke}%
\pgfsetstrokeopacity{0.000000}%
\pgfsetdash{}{0pt}%
\pgfpathmoveto{\pgfqpoint{5.800000in}{98.620889in}}%
\pgfpathlineto{\pgfqpoint{5.800000in}{98.620889in}}%
\pgfpathlineto{\pgfqpoint{5.800000in}{98.723726in}}%
\pgfpathlineto{\pgfqpoint{5.800000in}{98.723726in}}%
\pgfpathclose%
\pgfusepath{fill}%
\end{pgfscope}%
\begin{pgfscope}%
\pgfpathrectangle{\pgfqpoint{5.800000in}{0.720000in}}{\pgfqpoint{1.400000in}{4.620000in}}%
\pgfusepath{clip}%
\pgfsetbuttcap%
\pgfsetmiterjoin%
\definecolor{currentfill}{rgb}{0.121569,0.466667,0.705882}%
\pgfsetfillcolor{currentfill}%
\pgfsetlinewidth{0.000000pt}%
\definecolor{currentstroke}{rgb}{0.000000,0.000000,0.000000}%
\pgfsetstrokecolor{currentstroke}%
\pgfsetstrokeopacity{0.000000}%
\pgfsetdash{}{0pt}%
\pgfpathmoveto{\pgfqpoint{5.800000in}{98.723726in}}%
\pgfpathlineto{\pgfqpoint{5.800000in}{98.723726in}}%
\pgfpathlineto{\pgfqpoint{5.800000in}{98.826563in}}%
\pgfpathlineto{\pgfqpoint{5.800000in}{98.826563in}}%
\pgfpathclose%
\pgfusepath{fill}%
\end{pgfscope}%
\begin{pgfscope}%
\pgfpathrectangle{\pgfqpoint{5.800000in}{0.720000in}}{\pgfqpoint{1.400000in}{4.620000in}}%
\pgfusepath{clip}%
\pgfsetbuttcap%
\pgfsetmiterjoin%
\definecolor{currentfill}{rgb}{0.121569,0.466667,0.705882}%
\pgfsetfillcolor{currentfill}%
\pgfsetlinewidth{0.000000pt}%
\definecolor{currentstroke}{rgb}{0.000000,0.000000,0.000000}%
\pgfsetstrokecolor{currentstroke}%
\pgfsetstrokeopacity{0.000000}%
\pgfsetdash{}{0pt}%
\pgfpathmoveto{\pgfqpoint{5.800000in}{98.826563in}}%
\pgfpathlineto{\pgfqpoint{5.800000in}{98.826563in}}%
\pgfpathlineto{\pgfqpoint{5.800000in}{98.929400in}}%
\pgfpathlineto{\pgfqpoint{5.800000in}{98.929400in}}%
\pgfpathclose%
\pgfusepath{fill}%
\end{pgfscope}%
\begin{pgfscope}%
\pgfpathrectangle{\pgfqpoint{5.800000in}{0.720000in}}{\pgfqpoint{1.400000in}{4.620000in}}%
\pgfusepath{clip}%
\pgfsetbuttcap%
\pgfsetmiterjoin%
\definecolor{currentfill}{rgb}{0.121569,0.466667,0.705882}%
\pgfsetfillcolor{currentfill}%
\pgfsetlinewidth{0.000000pt}%
\definecolor{currentstroke}{rgb}{0.000000,0.000000,0.000000}%
\pgfsetstrokecolor{currentstroke}%
\pgfsetstrokeopacity{0.000000}%
\pgfsetdash{}{0pt}%
\pgfpathmoveto{\pgfqpoint{5.800000in}{98.929400in}}%
\pgfpathlineto{\pgfqpoint{5.800000in}{98.929400in}}%
\pgfpathlineto{\pgfqpoint{5.800000in}{99.032237in}}%
\pgfpathlineto{\pgfqpoint{5.800000in}{99.032237in}}%
\pgfpathclose%
\pgfusepath{fill}%
\end{pgfscope}%
\begin{pgfscope}%
\pgfpathrectangle{\pgfqpoint{5.800000in}{0.720000in}}{\pgfqpoint{1.400000in}{4.620000in}}%
\pgfusepath{clip}%
\pgfsetbuttcap%
\pgfsetmiterjoin%
\definecolor{currentfill}{rgb}{0.121569,0.466667,0.705882}%
\pgfsetfillcolor{currentfill}%
\pgfsetlinewidth{0.000000pt}%
\definecolor{currentstroke}{rgb}{0.000000,0.000000,0.000000}%
\pgfsetstrokecolor{currentstroke}%
\pgfsetstrokeopacity{0.000000}%
\pgfsetdash{}{0pt}%
\pgfpathmoveto{\pgfqpoint{5.800000in}{99.032237in}}%
\pgfpathlineto{\pgfqpoint{5.800000in}{99.032237in}}%
\pgfpathlineto{\pgfqpoint{5.800000in}{99.135074in}}%
\pgfpathlineto{\pgfqpoint{5.800000in}{99.135074in}}%
\pgfpathclose%
\pgfusepath{fill}%
\end{pgfscope}%
\begin{pgfscope}%
\pgfpathrectangle{\pgfqpoint{5.800000in}{0.720000in}}{\pgfqpoint{1.400000in}{4.620000in}}%
\pgfusepath{clip}%
\pgfsetbuttcap%
\pgfsetmiterjoin%
\definecolor{currentfill}{rgb}{0.121569,0.466667,0.705882}%
\pgfsetfillcolor{currentfill}%
\pgfsetlinewidth{0.000000pt}%
\definecolor{currentstroke}{rgb}{0.000000,0.000000,0.000000}%
\pgfsetstrokecolor{currentstroke}%
\pgfsetstrokeopacity{0.000000}%
\pgfsetdash{}{0pt}%
\pgfpathmoveto{\pgfqpoint{5.800000in}{99.135074in}}%
\pgfpathlineto{\pgfqpoint{5.800000in}{99.135074in}}%
\pgfpathlineto{\pgfqpoint{5.800000in}{99.237911in}}%
\pgfpathlineto{\pgfqpoint{5.800000in}{99.237911in}}%
\pgfpathclose%
\pgfusepath{fill}%
\end{pgfscope}%
\begin{pgfscope}%
\pgfpathrectangle{\pgfqpoint{5.800000in}{0.720000in}}{\pgfqpoint{1.400000in}{4.620000in}}%
\pgfusepath{clip}%
\pgfsetbuttcap%
\pgfsetmiterjoin%
\definecolor{currentfill}{rgb}{0.121569,0.466667,0.705882}%
\pgfsetfillcolor{currentfill}%
\pgfsetlinewidth{0.000000pt}%
\definecolor{currentstroke}{rgb}{0.000000,0.000000,0.000000}%
\pgfsetstrokecolor{currentstroke}%
\pgfsetstrokeopacity{0.000000}%
\pgfsetdash{}{0pt}%
\pgfpathmoveto{\pgfqpoint{5.800000in}{99.237911in}}%
\pgfpathlineto{\pgfqpoint{5.800000in}{99.237911in}}%
\pgfpathlineto{\pgfqpoint{5.800000in}{99.340748in}}%
\pgfpathlineto{\pgfqpoint{5.800000in}{99.340748in}}%
\pgfpathclose%
\pgfusepath{fill}%
\end{pgfscope}%
\begin{pgfscope}%
\pgfpathrectangle{\pgfqpoint{5.800000in}{0.720000in}}{\pgfqpoint{1.400000in}{4.620000in}}%
\pgfusepath{clip}%
\pgfsetbuttcap%
\pgfsetmiterjoin%
\definecolor{currentfill}{rgb}{0.121569,0.466667,0.705882}%
\pgfsetfillcolor{currentfill}%
\pgfsetlinewidth{0.000000pt}%
\definecolor{currentstroke}{rgb}{0.000000,0.000000,0.000000}%
\pgfsetstrokecolor{currentstroke}%
\pgfsetstrokeopacity{0.000000}%
\pgfsetdash{}{0pt}%
\pgfpathmoveto{\pgfqpoint{5.800000in}{99.340748in}}%
\pgfpathlineto{\pgfqpoint{5.800000in}{99.340748in}}%
\pgfpathlineto{\pgfqpoint{5.800000in}{99.443585in}}%
\pgfpathlineto{\pgfqpoint{5.800000in}{99.443585in}}%
\pgfpathclose%
\pgfusepath{fill}%
\end{pgfscope}%
\begin{pgfscope}%
\pgfpathrectangle{\pgfqpoint{5.800000in}{0.720000in}}{\pgfqpoint{1.400000in}{4.620000in}}%
\pgfusepath{clip}%
\pgfsetbuttcap%
\pgfsetmiterjoin%
\definecolor{currentfill}{rgb}{0.121569,0.466667,0.705882}%
\pgfsetfillcolor{currentfill}%
\pgfsetlinewidth{0.000000pt}%
\definecolor{currentstroke}{rgb}{0.000000,0.000000,0.000000}%
\pgfsetstrokecolor{currentstroke}%
\pgfsetstrokeopacity{0.000000}%
\pgfsetdash{}{0pt}%
\pgfpathmoveto{\pgfqpoint{5.800000in}{99.443585in}}%
\pgfpathlineto{\pgfqpoint{5.800000in}{99.443585in}}%
\pgfpathlineto{\pgfqpoint{5.800000in}{99.546422in}}%
\pgfpathlineto{\pgfqpoint{5.800000in}{99.546422in}}%
\pgfpathclose%
\pgfusepath{fill}%
\end{pgfscope}%
\begin{pgfscope}%
\pgfpathrectangle{\pgfqpoint{5.800000in}{0.720000in}}{\pgfqpoint{1.400000in}{4.620000in}}%
\pgfusepath{clip}%
\pgfsetbuttcap%
\pgfsetmiterjoin%
\definecolor{currentfill}{rgb}{0.121569,0.466667,0.705882}%
\pgfsetfillcolor{currentfill}%
\pgfsetlinewidth{0.000000pt}%
\definecolor{currentstroke}{rgb}{0.000000,0.000000,0.000000}%
\pgfsetstrokecolor{currentstroke}%
\pgfsetstrokeopacity{0.000000}%
\pgfsetdash{}{0pt}%
\pgfpathmoveto{\pgfqpoint{5.800000in}{99.546422in}}%
\pgfpathlineto{\pgfqpoint{5.800000in}{99.546422in}}%
\pgfpathlineto{\pgfqpoint{5.800000in}{99.649259in}}%
\pgfpathlineto{\pgfqpoint{5.800000in}{99.649259in}}%
\pgfpathclose%
\pgfusepath{fill}%
\end{pgfscope}%
\begin{pgfscope}%
\pgfpathrectangle{\pgfqpoint{5.800000in}{0.720000in}}{\pgfqpoint{1.400000in}{4.620000in}}%
\pgfusepath{clip}%
\pgfsetbuttcap%
\pgfsetmiterjoin%
\definecolor{currentfill}{rgb}{0.121569,0.466667,0.705882}%
\pgfsetfillcolor{currentfill}%
\pgfsetlinewidth{0.000000pt}%
\definecolor{currentstroke}{rgb}{0.000000,0.000000,0.000000}%
\pgfsetstrokecolor{currentstroke}%
\pgfsetstrokeopacity{0.000000}%
\pgfsetdash{}{0pt}%
\pgfpathmoveto{\pgfqpoint{5.800000in}{99.649259in}}%
\pgfpathlineto{\pgfqpoint{5.800000in}{99.649259in}}%
\pgfpathlineto{\pgfqpoint{5.800000in}{99.752096in}}%
\pgfpathlineto{\pgfqpoint{5.800000in}{99.752096in}}%
\pgfpathclose%
\pgfusepath{fill}%
\end{pgfscope}%
\begin{pgfscope}%
\pgfpathrectangle{\pgfqpoint{5.800000in}{0.720000in}}{\pgfqpoint{1.400000in}{4.620000in}}%
\pgfusepath{clip}%
\pgfsetbuttcap%
\pgfsetmiterjoin%
\definecolor{currentfill}{rgb}{0.121569,0.466667,0.705882}%
\pgfsetfillcolor{currentfill}%
\pgfsetlinewidth{0.000000pt}%
\definecolor{currentstroke}{rgb}{0.000000,0.000000,0.000000}%
\pgfsetstrokecolor{currentstroke}%
\pgfsetstrokeopacity{0.000000}%
\pgfsetdash{}{0pt}%
\pgfpathmoveto{\pgfqpoint{5.800000in}{99.752096in}}%
\pgfpathlineto{\pgfqpoint{5.800000in}{99.752096in}}%
\pgfpathlineto{\pgfqpoint{5.800000in}{99.854933in}}%
\pgfpathlineto{\pgfqpoint{5.800000in}{99.854933in}}%
\pgfpathclose%
\pgfusepath{fill}%
\end{pgfscope}%
\begin{pgfscope}%
\pgfpathrectangle{\pgfqpoint{5.800000in}{0.720000in}}{\pgfqpoint{1.400000in}{4.620000in}}%
\pgfusepath{clip}%
\pgfsetbuttcap%
\pgfsetmiterjoin%
\definecolor{currentfill}{rgb}{0.121569,0.466667,0.705882}%
\pgfsetfillcolor{currentfill}%
\pgfsetlinewidth{0.000000pt}%
\definecolor{currentstroke}{rgb}{0.000000,0.000000,0.000000}%
\pgfsetstrokecolor{currentstroke}%
\pgfsetstrokeopacity{0.000000}%
\pgfsetdash{}{0pt}%
\pgfpathmoveto{\pgfqpoint{5.800000in}{99.854933in}}%
\pgfpathlineto{\pgfqpoint{5.800000in}{99.854933in}}%
\pgfpathlineto{\pgfqpoint{5.800000in}{99.957770in}}%
\pgfpathlineto{\pgfqpoint{5.800000in}{99.957770in}}%
\pgfpathclose%
\pgfusepath{fill}%
\end{pgfscope}%
\begin{pgfscope}%
\pgfpathrectangle{\pgfqpoint{5.800000in}{0.720000in}}{\pgfqpoint{1.400000in}{4.620000in}}%
\pgfusepath{clip}%
\pgfsetbuttcap%
\pgfsetmiterjoin%
\definecolor{currentfill}{rgb}{0.121569,0.466667,0.705882}%
\pgfsetfillcolor{currentfill}%
\pgfsetlinewidth{0.000000pt}%
\definecolor{currentstroke}{rgb}{0.000000,0.000000,0.000000}%
\pgfsetstrokecolor{currentstroke}%
\pgfsetstrokeopacity{0.000000}%
\pgfsetdash{}{0pt}%
\pgfpathmoveto{\pgfqpoint{5.800000in}{99.957770in}}%
\pgfpathlineto{\pgfqpoint{5.800000in}{99.957770in}}%
\pgfpathlineto{\pgfqpoint{5.800000in}{100.060607in}}%
\pgfpathlineto{\pgfqpoint{5.800000in}{100.060607in}}%
\pgfpathclose%
\pgfusepath{fill}%
\end{pgfscope}%
\begin{pgfscope}%
\pgfpathrectangle{\pgfqpoint{5.800000in}{0.720000in}}{\pgfqpoint{1.400000in}{4.620000in}}%
\pgfusepath{clip}%
\pgfsetbuttcap%
\pgfsetmiterjoin%
\definecolor{currentfill}{rgb}{0.121569,0.466667,0.705882}%
\pgfsetfillcolor{currentfill}%
\pgfsetlinewidth{0.000000pt}%
\definecolor{currentstroke}{rgb}{0.000000,0.000000,0.000000}%
\pgfsetstrokecolor{currentstroke}%
\pgfsetstrokeopacity{0.000000}%
\pgfsetdash{}{0pt}%
\pgfpathmoveto{\pgfqpoint{5.800000in}{100.060607in}}%
\pgfpathlineto{\pgfqpoint{5.800000in}{100.060607in}}%
\pgfpathlineto{\pgfqpoint{5.800000in}{100.163445in}}%
\pgfpathlineto{\pgfqpoint{5.800000in}{100.163445in}}%
\pgfpathclose%
\pgfusepath{fill}%
\end{pgfscope}%
\begin{pgfscope}%
\pgfpathrectangle{\pgfqpoint{5.800000in}{0.720000in}}{\pgfqpoint{1.400000in}{4.620000in}}%
\pgfusepath{clip}%
\pgfsetbuttcap%
\pgfsetmiterjoin%
\definecolor{currentfill}{rgb}{0.121569,0.466667,0.705882}%
\pgfsetfillcolor{currentfill}%
\pgfsetlinewidth{0.000000pt}%
\definecolor{currentstroke}{rgb}{0.000000,0.000000,0.000000}%
\pgfsetstrokecolor{currentstroke}%
\pgfsetstrokeopacity{0.000000}%
\pgfsetdash{}{0pt}%
\pgfpathmoveto{\pgfqpoint{5.800000in}{100.163445in}}%
\pgfpathlineto{\pgfqpoint{5.800000in}{100.163445in}}%
\pgfpathlineto{\pgfqpoint{5.800000in}{100.266282in}}%
\pgfpathlineto{\pgfqpoint{5.800000in}{100.266282in}}%
\pgfpathclose%
\pgfusepath{fill}%
\end{pgfscope}%
\begin{pgfscope}%
\pgfpathrectangle{\pgfqpoint{5.800000in}{0.720000in}}{\pgfqpoint{1.400000in}{4.620000in}}%
\pgfusepath{clip}%
\pgfsetbuttcap%
\pgfsetmiterjoin%
\definecolor{currentfill}{rgb}{0.121569,0.466667,0.705882}%
\pgfsetfillcolor{currentfill}%
\pgfsetlinewidth{0.000000pt}%
\definecolor{currentstroke}{rgb}{0.000000,0.000000,0.000000}%
\pgfsetstrokecolor{currentstroke}%
\pgfsetstrokeopacity{0.000000}%
\pgfsetdash{}{0pt}%
\pgfpathmoveto{\pgfqpoint{5.800000in}{100.266282in}}%
\pgfpathlineto{\pgfqpoint{5.800000in}{100.266282in}}%
\pgfpathlineto{\pgfqpoint{5.800000in}{100.369119in}}%
\pgfpathlineto{\pgfqpoint{5.800000in}{100.369119in}}%
\pgfpathclose%
\pgfusepath{fill}%
\end{pgfscope}%
\begin{pgfscope}%
\pgfpathrectangle{\pgfqpoint{5.800000in}{0.720000in}}{\pgfqpoint{1.400000in}{4.620000in}}%
\pgfusepath{clip}%
\pgfsetbuttcap%
\pgfsetmiterjoin%
\definecolor{currentfill}{rgb}{0.121569,0.466667,0.705882}%
\pgfsetfillcolor{currentfill}%
\pgfsetlinewidth{0.000000pt}%
\definecolor{currentstroke}{rgb}{0.000000,0.000000,0.000000}%
\pgfsetstrokecolor{currentstroke}%
\pgfsetstrokeopacity{0.000000}%
\pgfsetdash{}{0pt}%
\pgfpathmoveto{\pgfqpoint{5.800000in}{100.369119in}}%
\pgfpathlineto{\pgfqpoint{5.800000in}{100.369119in}}%
\pgfpathlineto{\pgfqpoint{5.800000in}{100.471956in}}%
\pgfpathlineto{\pgfqpoint{5.800000in}{100.471956in}}%
\pgfpathclose%
\pgfusepath{fill}%
\end{pgfscope}%
\begin{pgfscope}%
\pgfpathrectangle{\pgfqpoint{5.800000in}{0.720000in}}{\pgfqpoint{1.400000in}{4.620000in}}%
\pgfusepath{clip}%
\pgfsetbuttcap%
\pgfsetmiterjoin%
\definecolor{currentfill}{rgb}{0.121569,0.466667,0.705882}%
\pgfsetfillcolor{currentfill}%
\pgfsetlinewidth{0.000000pt}%
\definecolor{currentstroke}{rgb}{0.000000,0.000000,0.000000}%
\pgfsetstrokecolor{currentstroke}%
\pgfsetstrokeopacity{0.000000}%
\pgfsetdash{}{0pt}%
\pgfpathmoveto{\pgfqpoint{5.800000in}{100.471956in}}%
\pgfpathlineto{\pgfqpoint{5.800000in}{100.471956in}}%
\pgfpathlineto{\pgfqpoint{5.800000in}{100.574793in}}%
\pgfpathlineto{\pgfqpoint{5.800000in}{100.574793in}}%
\pgfpathclose%
\pgfusepath{fill}%
\end{pgfscope}%
\begin{pgfscope}%
\pgfpathrectangle{\pgfqpoint{5.800000in}{0.720000in}}{\pgfqpoint{1.400000in}{4.620000in}}%
\pgfusepath{clip}%
\pgfsetbuttcap%
\pgfsetmiterjoin%
\definecolor{currentfill}{rgb}{0.121569,0.466667,0.705882}%
\pgfsetfillcolor{currentfill}%
\pgfsetlinewidth{0.000000pt}%
\definecolor{currentstroke}{rgb}{0.000000,0.000000,0.000000}%
\pgfsetstrokecolor{currentstroke}%
\pgfsetstrokeopacity{0.000000}%
\pgfsetdash{}{0pt}%
\pgfpathmoveto{\pgfqpoint{5.800000in}{100.574793in}}%
\pgfpathlineto{\pgfqpoint{5.800000in}{100.574793in}}%
\pgfpathlineto{\pgfqpoint{5.800000in}{100.677630in}}%
\pgfpathlineto{\pgfqpoint{5.800000in}{100.677630in}}%
\pgfpathclose%
\pgfusepath{fill}%
\end{pgfscope}%
\begin{pgfscope}%
\pgfpathrectangle{\pgfqpoint{5.800000in}{0.720000in}}{\pgfqpoint{1.400000in}{4.620000in}}%
\pgfusepath{clip}%
\pgfsetbuttcap%
\pgfsetmiterjoin%
\definecolor{currentfill}{rgb}{0.121569,0.466667,0.705882}%
\pgfsetfillcolor{currentfill}%
\pgfsetlinewidth{0.000000pt}%
\definecolor{currentstroke}{rgb}{0.000000,0.000000,0.000000}%
\pgfsetstrokecolor{currentstroke}%
\pgfsetstrokeopacity{0.000000}%
\pgfsetdash{}{0pt}%
\pgfpathmoveto{\pgfqpoint{5.800000in}{100.677630in}}%
\pgfpathlineto{\pgfqpoint{5.800000in}{100.677630in}}%
\pgfpathlineto{\pgfqpoint{5.800000in}{100.780467in}}%
\pgfpathlineto{\pgfqpoint{5.800000in}{100.780467in}}%
\pgfpathclose%
\pgfusepath{fill}%
\end{pgfscope}%
\begin{pgfscope}%
\pgfpathrectangle{\pgfqpoint{5.800000in}{0.720000in}}{\pgfqpoint{1.400000in}{4.620000in}}%
\pgfusepath{clip}%
\pgfsetbuttcap%
\pgfsetmiterjoin%
\definecolor{currentfill}{rgb}{0.121569,0.466667,0.705882}%
\pgfsetfillcolor{currentfill}%
\pgfsetlinewidth{0.000000pt}%
\definecolor{currentstroke}{rgb}{0.000000,0.000000,0.000000}%
\pgfsetstrokecolor{currentstroke}%
\pgfsetstrokeopacity{0.000000}%
\pgfsetdash{}{0pt}%
\pgfpathmoveto{\pgfqpoint{5.800000in}{100.780467in}}%
\pgfpathlineto{\pgfqpoint{5.800000in}{100.780467in}}%
\pgfpathlineto{\pgfqpoint{5.800000in}{100.883304in}}%
\pgfpathlineto{\pgfqpoint{5.800000in}{100.883304in}}%
\pgfpathclose%
\pgfusepath{fill}%
\end{pgfscope}%
\begin{pgfscope}%
\pgfpathrectangle{\pgfqpoint{5.800000in}{0.720000in}}{\pgfqpoint{1.400000in}{4.620000in}}%
\pgfusepath{clip}%
\pgfsetbuttcap%
\pgfsetmiterjoin%
\definecolor{currentfill}{rgb}{0.121569,0.466667,0.705882}%
\pgfsetfillcolor{currentfill}%
\pgfsetlinewidth{0.000000pt}%
\definecolor{currentstroke}{rgb}{0.000000,0.000000,0.000000}%
\pgfsetstrokecolor{currentstroke}%
\pgfsetstrokeopacity{0.000000}%
\pgfsetdash{}{0pt}%
\pgfpathmoveto{\pgfqpoint{5.800000in}{100.883304in}}%
\pgfpathlineto{\pgfqpoint{5.800000in}{100.883304in}}%
\pgfpathlineto{\pgfqpoint{5.800000in}{100.986141in}}%
\pgfpathlineto{\pgfqpoint{5.800000in}{100.986141in}}%
\pgfpathclose%
\pgfusepath{fill}%
\end{pgfscope}%
\begin{pgfscope}%
\pgfpathrectangle{\pgfqpoint{5.800000in}{0.720000in}}{\pgfqpoint{1.400000in}{4.620000in}}%
\pgfusepath{clip}%
\pgfsetbuttcap%
\pgfsetmiterjoin%
\definecolor{currentfill}{rgb}{0.121569,0.466667,0.705882}%
\pgfsetfillcolor{currentfill}%
\pgfsetlinewidth{0.000000pt}%
\definecolor{currentstroke}{rgb}{0.000000,0.000000,0.000000}%
\pgfsetstrokecolor{currentstroke}%
\pgfsetstrokeopacity{0.000000}%
\pgfsetdash{}{0pt}%
\pgfpathmoveto{\pgfqpoint{5.800000in}{100.986141in}}%
\pgfpathlineto{\pgfqpoint{5.800000in}{100.986141in}}%
\pgfpathlineto{\pgfqpoint{5.800000in}{101.088978in}}%
\pgfpathlineto{\pgfqpoint{5.800000in}{101.088978in}}%
\pgfpathclose%
\pgfusepath{fill}%
\end{pgfscope}%
\begin{pgfscope}%
\pgfpathrectangle{\pgfqpoint{5.800000in}{0.720000in}}{\pgfqpoint{1.400000in}{4.620000in}}%
\pgfusepath{clip}%
\pgfsetbuttcap%
\pgfsetmiterjoin%
\definecolor{currentfill}{rgb}{0.121569,0.466667,0.705882}%
\pgfsetfillcolor{currentfill}%
\pgfsetlinewidth{0.000000pt}%
\definecolor{currentstroke}{rgb}{0.000000,0.000000,0.000000}%
\pgfsetstrokecolor{currentstroke}%
\pgfsetstrokeopacity{0.000000}%
\pgfsetdash{}{0pt}%
\pgfpathmoveto{\pgfqpoint{5.800000in}{101.088978in}}%
\pgfpathlineto{\pgfqpoint{5.800000in}{101.088978in}}%
\pgfpathlineto{\pgfqpoint{5.800000in}{101.191815in}}%
\pgfpathlineto{\pgfqpoint{5.800000in}{101.191815in}}%
\pgfpathclose%
\pgfusepath{fill}%
\end{pgfscope}%
\begin{pgfscope}%
\pgfpathrectangle{\pgfqpoint{5.800000in}{0.720000in}}{\pgfqpoint{1.400000in}{4.620000in}}%
\pgfusepath{clip}%
\pgfsetbuttcap%
\pgfsetmiterjoin%
\definecolor{currentfill}{rgb}{0.121569,0.466667,0.705882}%
\pgfsetfillcolor{currentfill}%
\pgfsetlinewidth{0.000000pt}%
\definecolor{currentstroke}{rgb}{0.000000,0.000000,0.000000}%
\pgfsetstrokecolor{currentstroke}%
\pgfsetstrokeopacity{0.000000}%
\pgfsetdash{}{0pt}%
\pgfpathmoveto{\pgfqpoint{5.800000in}{101.191815in}}%
\pgfpathlineto{\pgfqpoint{5.800000in}{101.191815in}}%
\pgfpathlineto{\pgfqpoint{5.800000in}{101.294652in}}%
\pgfpathlineto{\pgfqpoint{5.800000in}{101.294652in}}%
\pgfpathclose%
\pgfusepath{fill}%
\end{pgfscope}%
\begin{pgfscope}%
\pgfpathrectangle{\pgfqpoint{5.800000in}{0.720000in}}{\pgfqpoint{1.400000in}{4.620000in}}%
\pgfusepath{clip}%
\pgfsetbuttcap%
\pgfsetmiterjoin%
\definecolor{currentfill}{rgb}{0.121569,0.466667,0.705882}%
\pgfsetfillcolor{currentfill}%
\pgfsetlinewidth{0.000000pt}%
\definecolor{currentstroke}{rgb}{0.000000,0.000000,0.000000}%
\pgfsetstrokecolor{currentstroke}%
\pgfsetstrokeopacity{0.000000}%
\pgfsetdash{}{0pt}%
\pgfpathmoveto{\pgfqpoint{5.800000in}{101.294652in}}%
\pgfpathlineto{\pgfqpoint{5.800000in}{101.294652in}}%
\pgfpathlineto{\pgfqpoint{5.800000in}{101.397489in}}%
\pgfpathlineto{\pgfqpoint{5.800000in}{101.397489in}}%
\pgfpathclose%
\pgfusepath{fill}%
\end{pgfscope}%
\begin{pgfscope}%
\pgfpathrectangle{\pgfqpoint{5.800000in}{0.720000in}}{\pgfqpoint{1.400000in}{4.620000in}}%
\pgfusepath{clip}%
\pgfsetbuttcap%
\pgfsetmiterjoin%
\definecolor{currentfill}{rgb}{0.121569,0.466667,0.705882}%
\pgfsetfillcolor{currentfill}%
\pgfsetlinewidth{0.000000pt}%
\definecolor{currentstroke}{rgb}{0.000000,0.000000,0.000000}%
\pgfsetstrokecolor{currentstroke}%
\pgfsetstrokeopacity{0.000000}%
\pgfsetdash{}{0pt}%
\pgfpathmoveto{\pgfqpoint{5.800000in}{101.397489in}}%
\pgfpathlineto{\pgfqpoint{5.800000in}{101.397489in}}%
\pgfpathlineto{\pgfqpoint{5.800000in}{101.500326in}}%
\pgfpathlineto{\pgfqpoint{5.800000in}{101.500326in}}%
\pgfpathclose%
\pgfusepath{fill}%
\end{pgfscope}%
\begin{pgfscope}%
\pgfpathrectangle{\pgfqpoint{5.800000in}{0.720000in}}{\pgfqpoint{1.400000in}{4.620000in}}%
\pgfusepath{clip}%
\pgfsetbuttcap%
\pgfsetmiterjoin%
\definecolor{currentfill}{rgb}{0.121569,0.466667,0.705882}%
\pgfsetfillcolor{currentfill}%
\pgfsetlinewidth{0.000000pt}%
\definecolor{currentstroke}{rgb}{0.000000,0.000000,0.000000}%
\pgfsetstrokecolor{currentstroke}%
\pgfsetstrokeopacity{0.000000}%
\pgfsetdash{}{0pt}%
\pgfpathmoveto{\pgfqpoint{5.800000in}{101.500326in}}%
\pgfpathlineto{\pgfqpoint{5.800000in}{101.500326in}}%
\pgfpathlineto{\pgfqpoint{5.800000in}{101.603163in}}%
\pgfpathlineto{\pgfqpoint{5.800000in}{101.603163in}}%
\pgfpathclose%
\pgfusepath{fill}%
\end{pgfscope}%
\begin{pgfscope}%
\pgfpathrectangle{\pgfqpoint{5.800000in}{0.720000in}}{\pgfqpoint{1.400000in}{4.620000in}}%
\pgfusepath{clip}%
\pgfsetbuttcap%
\pgfsetmiterjoin%
\definecolor{currentfill}{rgb}{0.121569,0.466667,0.705882}%
\pgfsetfillcolor{currentfill}%
\pgfsetlinewidth{0.000000pt}%
\definecolor{currentstroke}{rgb}{0.000000,0.000000,0.000000}%
\pgfsetstrokecolor{currentstroke}%
\pgfsetstrokeopacity{0.000000}%
\pgfsetdash{}{0pt}%
\pgfpathmoveto{\pgfqpoint{5.800000in}{101.603163in}}%
\pgfpathlineto{\pgfqpoint{5.800000in}{101.603163in}}%
\pgfpathlineto{\pgfqpoint{5.800000in}{101.706001in}}%
\pgfpathlineto{\pgfqpoint{5.800000in}{101.706001in}}%
\pgfpathclose%
\pgfusepath{fill}%
\end{pgfscope}%
\begin{pgfscope}%
\pgfpathrectangle{\pgfqpoint{5.800000in}{0.720000in}}{\pgfqpoint{1.400000in}{4.620000in}}%
\pgfusepath{clip}%
\pgfsetbuttcap%
\pgfsetmiterjoin%
\definecolor{currentfill}{rgb}{0.121569,0.466667,0.705882}%
\pgfsetfillcolor{currentfill}%
\pgfsetlinewidth{0.000000pt}%
\definecolor{currentstroke}{rgb}{0.000000,0.000000,0.000000}%
\pgfsetstrokecolor{currentstroke}%
\pgfsetstrokeopacity{0.000000}%
\pgfsetdash{}{0pt}%
\pgfpathmoveto{\pgfqpoint{5.800000in}{101.706001in}}%
\pgfpathlineto{\pgfqpoint{5.800000in}{101.706001in}}%
\pgfpathlineto{\pgfqpoint{5.800000in}{101.808838in}}%
\pgfpathlineto{\pgfqpoint{5.800000in}{101.808838in}}%
\pgfpathclose%
\pgfusepath{fill}%
\end{pgfscope}%
\begin{pgfscope}%
\pgfpathrectangle{\pgfqpoint{5.800000in}{0.720000in}}{\pgfqpoint{1.400000in}{4.620000in}}%
\pgfusepath{clip}%
\pgfsetbuttcap%
\pgfsetmiterjoin%
\definecolor{currentfill}{rgb}{0.121569,0.466667,0.705882}%
\pgfsetfillcolor{currentfill}%
\pgfsetlinewidth{0.000000pt}%
\definecolor{currentstroke}{rgb}{0.000000,0.000000,0.000000}%
\pgfsetstrokecolor{currentstroke}%
\pgfsetstrokeopacity{0.000000}%
\pgfsetdash{}{0pt}%
\pgfpathmoveto{\pgfqpoint{5.800000in}{101.808838in}}%
\pgfpathlineto{\pgfqpoint{5.800000in}{101.808838in}}%
\pgfpathlineto{\pgfqpoint{5.800000in}{101.911675in}}%
\pgfpathlineto{\pgfqpoint{5.800000in}{101.911675in}}%
\pgfpathclose%
\pgfusepath{fill}%
\end{pgfscope}%
\begin{pgfscope}%
\pgfpathrectangle{\pgfqpoint{5.800000in}{0.720000in}}{\pgfqpoint{1.400000in}{4.620000in}}%
\pgfusepath{clip}%
\pgfsetbuttcap%
\pgfsetmiterjoin%
\definecolor{currentfill}{rgb}{0.121569,0.466667,0.705882}%
\pgfsetfillcolor{currentfill}%
\pgfsetlinewidth{0.000000pt}%
\definecolor{currentstroke}{rgb}{0.000000,0.000000,0.000000}%
\pgfsetstrokecolor{currentstroke}%
\pgfsetstrokeopacity{0.000000}%
\pgfsetdash{}{0pt}%
\pgfpathmoveto{\pgfqpoint{5.800000in}{101.911675in}}%
\pgfpathlineto{\pgfqpoint{5.800000in}{101.911675in}}%
\pgfpathlineto{\pgfqpoint{5.800000in}{102.014512in}}%
\pgfpathlineto{\pgfqpoint{5.800000in}{102.014512in}}%
\pgfpathclose%
\pgfusepath{fill}%
\end{pgfscope}%
\begin{pgfscope}%
\pgfpathrectangle{\pgfqpoint{5.800000in}{0.720000in}}{\pgfqpoint{1.400000in}{4.620000in}}%
\pgfusepath{clip}%
\pgfsetbuttcap%
\pgfsetmiterjoin%
\definecolor{currentfill}{rgb}{0.121569,0.466667,0.705882}%
\pgfsetfillcolor{currentfill}%
\pgfsetlinewidth{0.000000pt}%
\definecolor{currentstroke}{rgb}{0.000000,0.000000,0.000000}%
\pgfsetstrokecolor{currentstroke}%
\pgfsetstrokeopacity{0.000000}%
\pgfsetdash{}{0pt}%
\pgfpathmoveto{\pgfqpoint{5.800000in}{102.014512in}}%
\pgfpathlineto{\pgfqpoint{5.800000in}{102.014512in}}%
\pgfpathlineto{\pgfqpoint{5.800000in}{102.117349in}}%
\pgfpathlineto{\pgfqpoint{5.800000in}{102.117349in}}%
\pgfpathclose%
\pgfusepath{fill}%
\end{pgfscope}%
\begin{pgfscope}%
\pgfpathrectangle{\pgfqpoint{5.800000in}{0.720000in}}{\pgfqpoint{1.400000in}{4.620000in}}%
\pgfusepath{clip}%
\pgfsetbuttcap%
\pgfsetmiterjoin%
\definecolor{currentfill}{rgb}{0.121569,0.466667,0.705882}%
\pgfsetfillcolor{currentfill}%
\pgfsetlinewidth{0.000000pt}%
\definecolor{currentstroke}{rgb}{0.000000,0.000000,0.000000}%
\pgfsetstrokecolor{currentstroke}%
\pgfsetstrokeopacity{0.000000}%
\pgfsetdash{}{0pt}%
\pgfpathmoveto{\pgfqpoint{5.800000in}{102.117349in}}%
\pgfpathlineto{\pgfqpoint{5.800000in}{102.117349in}}%
\pgfpathlineto{\pgfqpoint{5.800000in}{102.220186in}}%
\pgfpathlineto{\pgfqpoint{5.800000in}{102.220186in}}%
\pgfpathclose%
\pgfusepath{fill}%
\end{pgfscope}%
\begin{pgfscope}%
\pgfpathrectangle{\pgfqpoint{5.800000in}{0.720000in}}{\pgfqpoint{1.400000in}{4.620000in}}%
\pgfusepath{clip}%
\pgfsetbuttcap%
\pgfsetmiterjoin%
\definecolor{currentfill}{rgb}{0.121569,0.466667,0.705882}%
\pgfsetfillcolor{currentfill}%
\pgfsetlinewidth{0.000000pt}%
\definecolor{currentstroke}{rgb}{0.000000,0.000000,0.000000}%
\pgfsetstrokecolor{currentstroke}%
\pgfsetstrokeopacity{0.000000}%
\pgfsetdash{}{0pt}%
\pgfpathmoveto{\pgfqpoint{5.800000in}{102.220186in}}%
\pgfpathlineto{\pgfqpoint{5.800000in}{102.220186in}}%
\pgfpathlineto{\pgfqpoint{5.800000in}{102.323023in}}%
\pgfpathlineto{\pgfqpoint{5.800000in}{102.323023in}}%
\pgfpathclose%
\pgfusepath{fill}%
\end{pgfscope}%
\begin{pgfscope}%
\pgfpathrectangle{\pgfqpoint{5.800000in}{0.720000in}}{\pgfqpoint{1.400000in}{4.620000in}}%
\pgfusepath{clip}%
\pgfsetbuttcap%
\pgfsetmiterjoin%
\definecolor{currentfill}{rgb}{0.121569,0.466667,0.705882}%
\pgfsetfillcolor{currentfill}%
\pgfsetlinewidth{0.000000pt}%
\definecolor{currentstroke}{rgb}{0.000000,0.000000,0.000000}%
\pgfsetstrokecolor{currentstroke}%
\pgfsetstrokeopacity{0.000000}%
\pgfsetdash{}{0pt}%
\pgfpathmoveto{\pgfqpoint{5.800000in}{102.323023in}}%
\pgfpathlineto{\pgfqpoint{5.800000in}{102.323023in}}%
\pgfpathlineto{\pgfqpoint{5.800000in}{102.425860in}}%
\pgfpathlineto{\pgfqpoint{5.800000in}{102.425860in}}%
\pgfpathclose%
\pgfusepath{fill}%
\end{pgfscope}%
\begin{pgfscope}%
\pgfpathrectangle{\pgfqpoint{5.800000in}{0.720000in}}{\pgfqpoint{1.400000in}{4.620000in}}%
\pgfusepath{clip}%
\pgfsetbuttcap%
\pgfsetmiterjoin%
\definecolor{currentfill}{rgb}{0.121569,0.466667,0.705882}%
\pgfsetfillcolor{currentfill}%
\pgfsetlinewidth{0.000000pt}%
\definecolor{currentstroke}{rgb}{0.000000,0.000000,0.000000}%
\pgfsetstrokecolor{currentstroke}%
\pgfsetstrokeopacity{0.000000}%
\pgfsetdash{}{0pt}%
\pgfpathmoveto{\pgfqpoint{5.800000in}{102.425860in}}%
\pgfpathlineto{\pgfqpoint{5.800000in}{102.425860in}}%
\pgfpathlineto{\pgfqpoint{5.800000in}{102.528697in}}%
\pgfpathlineto{\pgfqpoint{5.800000in}{102.528697in}}%
\pgfpathclose%
\pgfusepath{fill}%
\end{pgfscope}%
\begin{pgfscope}%
\pgfpathrectangle{\pgfqpoint{5.800000in}{0.720000in}}{\pgfqpoint{1.400000in}{4.620000in}}%
\pgfusepath{clip}%
\pgfsetbuttcap%
\pgfsetmiterjoin%
\definecolor{currentfill}{rgb}{0.121569,0.466667,0.705882}%
\pgfsetfillcolor{currentfill}%
\pgfsetlinewidth{0.000000pt}%
\definecolor{currentstroke}{rgb}{0.000000,0.000000,0.000000}%
\pgfsetstrokecolor{currentstroke}%
\pgfsetstrokeopacity{0.000000}%
\pgfsetdash{}{0pt}%
\pgfpathmoveto{\pgfqpoint{5.800000in}{102.528697in}}%
\pgfpathlineto{\pgfqpoint{5.800000in}{102.528697in}}%
\pgfpathlineto{\pgfqpoint{5.800000in}{102.631534in}}%
\pgfpathlineto{\pgfqpoint{5.800000in}{102.631534in}}%
\pgfpathclose%
\pgfusepath{fill}%
\end{pgfscope}%
\begin{pgfscope}%
\pgfpathrectangle{\pgfqpoint{5.800000in}{0.720000in}}{\pgfqpoint{1.400000in}{4.620000in}}%
\pgfusepath{clip}%
\pgfsetbuttcap%
\pgfsetmiterjoin%
\definecolor{currentfill}{rgb}{0.121569,0.466667,0.705882}%
\pgfsetfillcolor{currentfill}%
\pgfsetlinewidth{0.000000pt}%
\definecolor{currentstroke}{rgb}{0.000000,0.000000,0.000000}%
\pgfsetstrokecolor{currentstroke}%
\pgfsetstrokeopacity{0.000000}%
\pgfsetdash{}{0pt}%
\pgfpathmoveto{\pgfqpoint{5.800000in}{102.631534in}}%
\pgfpathlineto{\pgfqpoint{5.800000in}{102.631534in}}%
\pgfpathlineto{\pgfqpoint{5.800000in}{102.734371in}}%
\pgfpathlineto{\pgfqpoint{5.800000in}{102.734371in}}%
\pgfpathclose%
\pgfusepath{fill}%
\end{pgfscope}%
\begin{pgfscope}%
\pgfpathrectangle{\pgfqpoint{5.800000in}{0.720000in}}{\pgfqpoint{1.400000in}{4.620000in}}%
\pgfusepath{clip}%
\pgfsetbuttcap%
\pgfsetmiterjoin%
\definecolor{currentfill}{rgb}{0.121569,0.466667,0.705882}%
\pgfsetfillcolor{currentfill}%
\pgfsetlinewidth{0.000000pt}%
\definecolor{currentstroke}{rgb}{0.000000,0.000000,0.000000}%
\pgfsetstrokecolor{currentstroke}%
\pgfsetstrokeopacity{0.000000}%
\pgfsetdash{}{0pt}%
\pgfpathmoveto{\pgfqpoint{5.800000in}{102.734371in}}%
\pgfpathlineto{\pgfqpoint{5.800000in}{102.734371in}}%
\pgfpathlineto{\pgfqpoint{5.800000in}{102.837208in}}%
\pgfpathlineto{\pgfqpoint{5.800000in}{102.837208in}}%
\pgfpathclose%
\pgfusepath{fill}%
\end{pgfscope}%
\begin{pgfscope}%
\pgfpathrectangle{\pgfqpoint{5.800000in}{0.720000in}}{\pgfqpoint{1.400000in}{4.620000in}}%
\pgfusepath{clip}%
\pgfsetbuttcap%
\pgfsetmiterjoin%
\definecolor{currentfill}{rgb}{0.121569,0.466667,0.705882}%
\pgfsetfillcolor{currentfill}%
\pgfsetlinewidth{0.000000pt}%
\definecolor{currentstroke}{rgb}{0.000000,0.000000,0.000000}%
\pgfsetstrokecolor{currentstroke}%
\pgfsetstrokeopacity{0.000000}%
\pgfsetdash{}{0pt}%
\pgfpathmoveto{\pgfqpoint{5.800000in}{102.837208in}}%
\pgfpathlineto{\pgfqpoint{5.800000in}{102.837208in}}%
\pgfpathlineto{\pgfqpoint{5.800000in}{102.940045in}}%
\pgfpathlineto{\pgfqpoint{5.800000in}{102.940045in}}%
\pgfpathclose%
\pgfusepath{fill}%
\end{pgfscope}%
\begin{pgfscope}%
\pgfpathrectangle{\pgfqpoint{5.800000in}{0.720000in}}{\pgfqpoint{1.400000in}{4.620000in}}%
\pgfusepath{clip}%
\pgfsetbuttcap%
\pgfsetmiterjoin%
\definecolor{currentfill}{rgb}{0.121569,0.466667,0.705882}%
\pgfsetfillcolor{currentfill}%
\pgfsetlinewidth{0.000000pt}%
\definecolor{currentstroke}{rgb}{0.000000,0.000000,0.000000}%
\pgfsetstrokecolor{currentstroke}%
\pgfsetstrokeopacity{0.000000}%
\pgfsetdash{}{0pt}%
\pgfpathmoveto{\pgfqpoint{5.800000in}{102.940045in}}%
\pgfpathlineto{\pgfqpoint{5.800000in}{102.940045in}}%
\pgfpathlineto{\pgfqpoint{5.800000in}{103.042882in}}%
\pgfpathlineto{\pgfqpoint{5.800000in}{103.042882in}}%
\pgfpathclose%
\pgfusepath{fill}%
\end{pgfscope}%
\begin{pgfscope}%
\pgfpathrectangle{\pgfqpoint{5.800000in}{0.720000in}}{\pgfqpoint{1.400000in}{4.620000in}}%
\pgfusepath{clip}%
\pgfsetbuttcap%
\pgfsetmiterjoin%
\definecolor{currentfill}{rgb}{0.121569,0.466667,0.705882}%
\pgfsetfillcolor{currentfill}%
\pgfsetlinewidth{0.000000pt}%
\definecolor{currentstroke}{rgb}{0.000000,0.000000,0.000000}%
\pgfsetstrokecolor{currentstroke}%
\pgfsetstrokeopacity{0.000000}%
\pgfsetdash{}{0pt}%
\pgfpathmoveto{\pgfqpoint{5.800000in}{103.042882in}}%
\pgfpathlineto{\pgfqpoint{5.800000in}{103.042882in}}%
\pgfpathlineto{\pgfqpoint{5.800000in}{103.145720in}}%
\pgfpathlineto{\pgfqpoint{5.800000in}{103.145720in}}%
\pgfpathclose%
\pgfusepath{fill}%
\end{pgfscope}%
\begin{pgfscope}%
\pgfpathrectangle{\pgfqpoint{5.800000in}{0.720000in}}{\pgfqpoint{1.400000in}{4.620000in}}%
\pgfusepath{clip}%
\pgfsetbuttcap%
\pgfsetmiterjoin%
\definecolor{currentfill}{rgb}{0.121569,0.466667,0.705882}%
\pgfsetfillcolor{currentfill}%
\pgfsetlinewidth{0.000000pt}%
\definecolor{currentstroke}{rgb}{0.000000,0.000000,0.000000}%
\pgfsetstrokecolor{currentstroke}%
\pgfsetstrokeopacity{0.000000}%
\pgfsetdash{}{0pt}%
\pgfpathmoveto{\pgfqpoint{5.800000in}{103.145720in}}%
\pgfpathlineto{\pgfqpoint{5.800000in}{103.145720in}}%
\pgfpathlineto{\pgfqpoint{5.800000in}{103.248557in}}%
\pgfpathlineto{\pgfqpoint{5.800000in}{103.248557in}}%
\pgfpathclose%
\pgfusepath{fill}%
\end{pgfscope}%
\begin{pgfscope}%
\pgfpathrectangle{\pgfqpoint{5.800000in}{0.720000in}}{\pgfqpoint{1.400000in}{4.620000in}}%
\pgfusepath{clip}%
\pgfsetbuttcap%
\pgfsetmiterjoin%
\definecolor{currentfill}{rgb}{0.121569,0.466667,0.705882}%
\pgfsetfillcolor{currentfill}%
\pgfsetlinewidth{0.000000pt}%
\definecolor{currentstroke}{rgb}{0.000000,0.000000,0.000000}%
\pgfsetstrokecolor{currentstroke}%
\pgfsetstrokeopacity{0.000000}%
\pgfsetdash{}{0pt}%
\pgfpathmoveto{\pgfqpoint{5.800000in}{103.248557in}}%
\pgfpathlineto{\pgfqpoint{5.800000in}{103.248557in}}%
\pgfpathlineto{\pgfqpoint{5.800000in}{103.351394in}}%
\pgfpathlineto{\pgfqpoint{5.800000in}{103.351394in}}%
\pgfpathclose%
\pgfusepath{fill}%
\end{pgfscope}%
\begin{pgfscope}%
\pgfpathrectangle{\pgfqpoint{5.800000in}{0.720000in}}{\pgfqpoint{1.400000in}{4.620000in}}%
\pgfusepath{clip}%
\pgfsetbuttcap%
\pgfsetmiterjoin%
\definecolor{currentfill}{rgb}{0.121569,0.466667,0.705882}%
\pgfsetfillcolor{currentfill}%
\pgfsetlinewidth{0.000000pt}%
\definecolor{currentstroke}{rgb}{0.000000,0.000000,0.000000}%
\pgfsetstrokecolor{currentstroke}%
\pgfsetstrokeopacity{0.000000}%
\pgfsetdash{}{0pt}%
\pgfpathmoveto{\pgfqpoint{5.800000in}{103.351394in}}%
\pgfpathlineto{\pgfqpoint{5.800000in}{103.351394in}}%
\pgfpathlineto{\pgfqpoint{5.800000in}{103.454231in}}%
\pgfpathlineto{\pgfqpoint{5.800000in}{103.454231in}}%
\pgfpathclose%
\pgfusepath{fill}%
\end{pgfscope}%
\begin{pgfscope}%
\pgfpathrectangle{\pgfqpoint{5.800000in}{0.720000in}}{\pgfqpoint{1.400000in}{4.620000in}}%
\pgfusepath{clip}%
\pgfsetbuttcap%
\pgfsetmiterjoin%
\definecolor{currentfill}{rgb}{0.121569,0.466667,0.705882}%
\pgfsetfillcolor{currentfill}%
\pgfsetlinewidth{0.000000pt}%
\definecolor{currentstroke}{rgb}{0.000000,0.000000,0.000000}%
\pgfsetstrokecolor{currentstroke}%
\pgfsetstrokeopacity{0.000000}%
\pgfsetdash{}{0pt}%
\pgfpathmoveto{\pgfqpoint{5.800000in}{103.454231in}}%
\pgfpathlineto{\pgfqpoint{5.800000in}{103.454231in}}%
\pgfpathlineto{\pgfqpoint{5.800000in}{103.557068in}}%
\pgfpathlineto{\pgfqpoint{5.800000in}{103.557068in}}%
\pgfpathclose%
\pgfusepath{fill}%
\end{pgfscope}%
\begin{pgfscope}%
\pgfpathrectangle{\pgfqpoint{5.800000in}{0.720000in}}{\pgfqpoint{1.400000in}{4.620000in}}%
\pgfusepath{clip}%
\pgfsetbuttcap%
\pgfsetmiterjoin%
\definecolor{currentfill}{rgb}{0.121569,0.466667,0.705882}%
\pgfsetfillcolor{currentfill}%
\pgfsetlinewidth{0.000000pt}%
\definecolor{currentstroke}{rgb}{0.000000,0.000000,0.000000}%
\pgfsetstrokecolor{currentstroke}%
\pgfsetstrokeopacity{0.000000}%
\pgfsetdash{}{0pt}%
\pgfpathmoveto{\pgfqpoint{5.800000in}{103.557068in}}%
\pgfpathlineto{\pgfqpoint{5.800000in}{103.557068in}}%
\pgfpathlineto{\pgfqpoint{5.800000in}{103.659905in}}%
\pgfpathlineto{\pgfqpoint{5.800000in}{103.659905in}}%
\pgfpathclose%
\pgfusepath{fill}%
\end{pgfscope}%
\begin{pgfscope}%
\pgfpathrectangle{\pgfqpoint{5.800000in}{0.720000in}}{\pgfqpoint{1.400000in}{4.620000in}}%
\pgfusepath{clip}%
\pgfsetbuttcap%
\pgfsetmiterjoin%
\definecolor{currentfill}{rgb}{0.121569,0.466667,0.705882}%
\pgfsetfillcolor{currentfill}%
\pgfsetlinewidth{0.000000pt}%
\definecolor{currentstroke}{rgb}{0.000000,0.000000,0.000000}%
\pgfsetstrokecolor{currentstroke}%
\pgfsetstrokeopacity{0.000000}%
\pgfsetdash{}{0pt}%
\pgfpathmoveto{\pgfqpoint{5.800000in}{103.659905in}}%
\pgfpathlineto{\pgfqpoint{5.800000in}{103.659905in}}%
\pgfpathlineto{\pgfqpoint{5.800000in}{103.762742in}}%
\pgfpathlineto{\pgfqpoint{5.800000in}{103.762742in}}%
\pgfpathclose%
\pgfusepath{fill}%
\end{pgfscope}%
\begin{pgfscope}%
\pgfpathrectangle{\pgfqpoint{5.800000in}{0.720000in}}{\pgfqpoint{1.400000in}{4.620000in}}%
\pgfusepath{clip}%
\pgfsetbuttcap%
\pgfsetmiterjoin%
\definecolor{currentfill}{rgb}{0.121569,0.466667,0.705882}%
\pgfsetfillcolor{currentfill}%
\pgfsetlinewidth{0.000000pt}%
\definecolor{currentstroke}{rgb}{0.000000,0.000000,0.000000}%
\pgfsetstrokecolor{currentstroke}%
\pgfsetstrokeopacity{0.000000}%
\pgfsetdash{}{0pt}%
\pgfpathmoveto{\pgfqpoint{5.800000in}{103.762742in}}%
\pgfpathlineto{\pgfqpoint{5.800000in}{103.762742in}}%
\pgfpathlineto{\pgfqpoint{5.800000in}{103.865579in}}%
\pgfpathlineto{\pgfqpoint{5.800000in}{103.865579in}}%
\pgfpathclose%
\pgfusepath{fill}%
\end{pgfscope}%
\begin{pgfscope}%
\pgfpathrectangle{\pgfqpoint{5.800000in}{0.720000in}}{\pgfqpoint{1.400000in}{4.620000in}}%
\pgfusepath{clip}%
\pgfsetbuttcap%
\pgfsetmiterjoin%
\definecolor{currentfill}{rgb}{0.121569,0.466667,0.705882}%
\pgfsetfillcolor{currentfill}%
\pgfsetlinewidth{0.000000pt}%
\definecolor{currentstroke}{rgb}{0.000000,0.000000,0.000000}%
\pgfsetstrokecolor{currentstroke}%
\pgfsetstrokeopacity{0.000000}%
\pgfsetdash{}{0pt}%
\pgfpathmoveto{\pgfqpoint{5.800000in}{103.865579in}}%
\pgfpathlineto{\pgfqpoint{5.800000in}{103.865579in}}%
\pgfpathlineto{\pgfqpoint{5.800000in}{103.968416in}}%
\pgfpathlineto{\pgfqpoint{5.800000in}{103.968416in}}%
\pgfpathclose%
\pgfusepath{fill}%
\end{pgfscope}%
\begin{pgfscope}%
\pgfpathrectangle{\pgfqpoint{5.800000in}{0.720000in}}{\pgfqpoint{1.400000in}{4.620000in}}%
\pgfusepath{clip}%
\pgfsetbuttcap%
\pgfsetmiterjoin%
\definecolor{currentfill}{rgb}{0.121569,0.466667,0.705882}%
\pgfsetfillcolor{currentfill}%
\pgfsetlinewidth{0.000000pt}%
\definecolor{currentstroke}{rgb}{0.000000,0.000000,0.000000}%
\pgfsetstrokecolor{currentstroke}%
\pgfsetstrokeopacity{0.000000}%
\pgfsetdash{}{0pt}%
\pgfpathmoveto{\pgfqpoint{5.800000in}{103.968416in}}%
\pgfpathlineto{\pgfqpoint{5.800000in}{103.968416in}}%
\pgfpathlineto{\pgfqpoint{5.800000in}{104.071253in}}%
\pgfpathlineto{\pgfqpoint{5.800000in}{104.071253in}}%
\pgfpathclose%
\pgfusepath{fill}%
\end{pgfscope}%
\begin{pgfscope}%
\pgfpathrectangle{\pgfqpoint{5.800000in}{0.720000in}}{\pgfqpoint{1.400000in}{4.620000in}}%
\pgfusepath{clip}%
\pgfsetbuttcap%
\pgfsetmiterjoin%
\definecolor{currentfill}{rgb}{0.121569,0.466667,0.705882}%
\pgfsetfillcolor{currentfill}%
\pgfsetlinewidth{0.000000pt}%
\definecolor{currentstroke}{rgb}{0.000000,0.000000,0.000000}%
\pgfsetstrokecolor{currentstroke}%
\pgfsetstrokeopacity{0.000000}%
\pgfsetdash{}{0pt}%
\pgfpathmoveto{\pgfqpoint{5.800000in}{104.071253in}}%
\pgfpathlineto{\pgfqpoint{5.800000in}{104.071253in}}%
\pgfpathlineto{\pgfqpoint{5.800000in}{104.174090in}}%
\pgfpathlineto{\pgfqpoint{5.800000in}{104.174090in}}%
\pgfpathclose%
\pgfusepath{fill}%
\end{pgfscope}%
\begin{pgfscope}%
\pgfpathrectangle{\pgfqpoint{5.800000in}{0.720000in}}{\pgfqpoint{1.400000in}{4.620000in}}%
\pgfusepath{clip}%
\pgfsetbuttcap%
\pgfsetmiterjoin%
\definecolor{currentfill}{rgb}{0.121569,0.466667,0.705882}%
\pgfsetfillcolor{currentfill}%
\pgfsetlinewidth{0.000000pt}%
\definecolor{currentstroke}{rgb}{0.000000,0.000000,0.000000}%
\pgfsetstrokecolor{currentstroke}%
\pgfsetstrokeopacity{0.000000}%
\pgfsetdash{}{0pt}%
\pgfpathmoveto{\pgfqpoint{5.800000in}{104.174090in}}%
\pgfpathlineto{\pgfqpoint{5.800000in}{104.174090in}}%
\pgfpathlineto{\pgfqpoint{5.800000in}{104.276927in}}%
\pgfpathlineto{\pgfqpoint{5.800000in}{104.276927in}}%
\pgfpathclose%
\pgfusepath{fill}%
\end{pgfscope}%
\begin{pgfscope}%
\pgfpathrectangle{\pgfqpoint{5.800000in}{0.720000in}}{\pgfqpoint{1.400000in}{4.620000in}}%
\pgfusepath{clip}%
\pgfsetbuttcap%
\pgfsetmiterjoin%
\definecolor{currentfill}{rgb}{0.121569,0.466667,0.705882}%
\pgfsetfillcolor{currentfill}%
\pgfsetlinewidth{0.000000pt}%
\definecolor{currentstroke}{rgb}{0.000000,0.000000,0.000000}%
\pgfsetstrokecolor{currentstroke}%
\pgfsetstrokeopacity{0.000000}%
\pgfsetdash{}{0pt}%
\pgfpathmoveto{\pgfqpoint{5.800000in}{104.276927in}}%
\pgfpathlineto{\pgfqpoint{5.800000in}{104.276927in}}%
\pgfpathlineto{\pgfqpoint{5.800000in}{104.379764in}}%
\pgfpathlineto{\pgfqpoint{5.800000in}{104.379764in}}%
\pgfpathclose%
\pgfusepath{fill}%
\end{pgfscope}%
\begin{pgfscope}%
\pgfpathrectangle{\pgfqpoint{5.800000in}{0.720000in}}{\pgfqpoint{1.400000in}{4.620000in}}%
\pgfusepath{clip}%
\pgfsetbuttcap%
\pgfsetmiterjoin%
\definecolor{currentfill}{rgb}{0.121569,0.466667,0.705882}%
\pgfsetfillcolor{currentfill}%
\pgfsetlinewidth{0.000000pt}%
\definecolor{currentstroke}{rgb}{0.000000,0.000000,0.000000}%
\pgfsetstrokecolor{currentstroke}%
\pgfsetstrokeopacity{0.000000}%
\pgfsetdash{}{0pt}%
\pgfpathmoveto{\pgfqpoint{5.800000in}{104.379764in}}%
\pgfpathlineto{\pgfqpoint{5.800000in}{104.379764in}}%
\pgfpathlineto{\pgfqpoint{5.800000in}{104.482601in}}%
\pgfpathlineto{\pgfqpoint{5.800000in}{104.482601in}}%
\pgfpathclose%
\pgfusepath{fill}%
\end{pgfscope}%
\begin{pgfscope}%
\pgfpathrectangle{\pgfqpoint{5.800000in}{0.720000in}}{\pgfqpoint{1.400000in}{4.620000in}}%
\pgfusepath{clip}%
\pgfsetbuttcap%
\pgfsetmiterjoin%
\definecolor{currentfill}{rgb}{0.121569,0.466667,0.705882}%
\pgfsetfillcolor{currentfill}%
\pgfsetlinewidth{0.000000pt}%
\definecolor{currentstroke}{rgb}{0.000000,0.000000,0.000000}%
\pgfsetstrokecolor{currentstroke}%
\pgfsetstrokeopacity{0.000000}%
\pgfsetdash{}{0pt}%
\pgfpathmoveto{\pgfqpoint{5.800000in}{104.482601in}}%
\pgfpathlineto{\pgfqpoint{5.800000in}{104.482601in}}%
\pgfpathlineto{\pgfqpoint{5.800000in}{104.585438in}}%
\pgfpathlineto{\pgfqpoint{5.800000in}{104.585438in}}%
\pgfpathclose%
\pgfusepath{fill}%
\end{pgfscope}%
\begin{pgfscope}%
\pgfpathrectangle{\pgfqpoint{5.800000in}{0.720000in}}{\pgfqpoint{1.400000in}{4.620000in}}%
\pgfusepath{clip}%
\pgfsetbuttcap%
\pgfsetmiterjoin%
\definecolor{currentfill}{rgb}{0.121569,0.466667,0.705882}%
\pgfsetfillcolor{currentfill}%
\pgfsetlinewidth{0.000000pt}%
\definecolor{currentstroke}{rgb}{0.000000,0.000000,0.000000}%
\pgfsetstrokecolor{currentstroke}%
\pgfsetstrokeopacity{0.000000}%
\pgfsetdash{}{0pt}%
\pgfpathmoveto{\pgfqpoint{5.800000in}{104.585438in}}%
\pgfpathlineto{\pgfqpoint{5.800000in}{104.585438in}}%
\pgfpathlineto{\pgfqpoint{5.800000in}{104.688276in}}%
\pgfpathlineto{\pgfqpoint{5.800000in}{104.688276in}}%
\pgfpathclose%
\pgfusepath{fill}%
\end{pgfscope}%
\begin{pgfscope}%
\pgfpathrectangle{\pgfqpoint{5.800000in}{0.720000in}}{\pgfqpoint{1.400000in}{4.620000in}}%
\pgfusepath{clip}%
\pgfsetbuttcap%
\pgfsetmiterjoin%
\definecolor{currentfill}{rgb}{0.121569,0.466667,0.705882}%
\pgfsetfillcolor{currentfill}%
\pgfsetlinewidth{0.000000pt}%
\definecolor{currentstroke}{rgb}{0.000000,0.000000,0.000000}%
\pgfsetstrokecolor{currentstroke}%
\pgfsetstrokeopacity{0.000000}%
\pgfsetdash{}{0pt}%
\pgfpathmoveto{\pgfqpoint{5.800000in}{104.688276in}}%
\pgfpathlineto{\pgfqpoint{5.800000in}{104.688276in}}%
\pgfpathlineto{\pgfqpoint{5.800000in}{104.791113in}}%
\pgfpathlineto{\pgfqpoint{5.800000in}{104.791113in}}%
\pgfpathclose%
\pgfusepath{fill}%
\end{pgfscope}%
\begin{pgfscope}%
\pgfpathrectangle{\pgfqpoint{5.800000in}{0.720000in}}{\pgfqpoint{1.400000in}{4.620000in}}%
\pgfusepath{clip}%
\pgfsetbuttcap%
\pgfsetmiterjoin%
\definecolor{currentfill}{rgb}{0.121569,0.466667,0.705882}%
\pgfsetfillcolor{currentfill}%
\pgfsetlinewidth{0.000000pt}%
\definecolor{currentstroke}{rgb}{0.000000,0.000000,0.000000}%
\pgfsetstrokecolor{currentstroke}%
\pgfsetstrokeopacity{0.000000}%
\pgfsetdash{}{0pt}%
\pgfpathmoveto{\pgfqpoint{5.800000in}{104.791113in}}%
\pgfpathlineto{\pgfqpoint{5.800000in}{104.791113in}}%
\pgfpathlineto{\pgfqpoint{5.800000in}{104.893950in}}%
\pgfpathlineto{\pgfqpoint{5.800000in}{104.893950in}}%
\pgfpathclose%
\pgfusepath{fill}%
\end{pgfscope}%
\begin{pgfscope}%
\pgfpathrectangle{\pgfqpoint{5.800000in}{0.720000in}}{\pgfqpoint{1.400000in}{4.620000in}}%
\pgfusepath{clip}%
\pgfsetbuttcap%
\pgfsetmiterjoin%
\definecolor{currentfill}{rgb}{0.121569,0.466667,0.705882}%
\pgfsetfillcolor{currentfill}%
\pgfsetlinewidth{0.000000pt}%
\definecolor{currentstroke}{rgb}{0.000000,0.000000,0.000000}%
\pgfsetstrokecolor{currentstroke}%
\pgfsetstrokeopacity{0.000000}%
\pgfsetdash{}{0pt}%
\pgfpathmoveto{\pgfqpoint{5.800000in}{104.893950in}}%
\pgfpathlineto{\pgfqpoint{5.800000in}{104.893950in}}%
\pgfpathlineto{\pgfqpoint{5.800000in}{104.996787in}}%
\pgfpathlineto{\pgfqpoint{5.800000in}{104.996787in}}%
\pgfpathclose%
\pgfusepath{fill}%
\end{pgfscope}%
\begin{pgfscope}%
\pgfpathrectangle{\pgfqpoint{5.800000in}{0.720000in}}{\pgfqpoint{1.400000in}{4.620000in}}%
\pgfusepath{clip}%
\pgfsetbuttcap%
\pgfsetmiterjoin%
\definecolor{currentfill}{rgb}{0.121569,0.466667,0.705882}%
\pgfsetfillcolor{currentfill}%
\pgfsetlinewidth{0.000000pt}%
\definecolor{currentstroke}{rgb}{0.000000,0.000000,0.000000}%
\pgfsetstrokecolor{currentstroke}%
\pgfsetstrokeopacity{0.000000}%
\pgfsetdash{}{0pt}%
\pgfpathmoveto{\pgfqpoint{5.800000in}{104.996787in}}%
\pgfpathlineto{\pgfqpoint{5.800000in}{104.996787in}}%
\pgfpathlineto{\pgfqpoint{5.800000in}{105.099624in}}%
\pgfpathlineto{\pgfqpoint{5.800000in}{105.099624in}}%
\pgfpathclose%
\pgfusepath{fill}%
\end{pgfscope}%
\begin{pgfscope}%
\pgfpathrectangle{\pgfqpoint{5.800000in}{0.720000in}}{\pgfqpoint{1.400000in}{4.620000in}}%
\pgfusepath{clip}%
\pgfsetbuttcap%
\pgfsetmiterjoin%
\definecolor{currentfill}{rgb}{0.121569,0.466667,0.705882}%
\pgfsetfillcolor{currentfill}%
\pgfsetlinewidth{0.000000pt}%
\definecolor{currentstroke}{rgb}{0.000000,0.000000,0.000000}%
\pgfsetstrokecolor{currentstroke}%
\pgfsetstrokeopacity{0.000000}%
\pgfsetdash{}{0pt}%
\pgfpathmoveto{\pgfqpoint{5.800000in}{105.099624in}}%
\pgfpathlineto{\pgfqpoint{5.800000in}{105.099624in}}%
\pgfpathlineto{\pgfqpoint{5.800000in}{105.202461in}}%
\pgfpathlineto{\pgfqpoint{5.800000in}{105.202461in}}%
\pgfpathclose%
\pgfusepath{fill}%
\end{pgfscope}%
\begin{pgfscope}%
\pgfpathrectangle{\pgfqpoint{5.800000in}{0.720000in}}{\pgfqpoint{1.400000in}{4.620000in}}%
\pgfusepath{clip}%
\pgfsetbuttcap%
\pgfsetmiterjoin%
\definecolor{currentfill}{rgb}{0.121569,0.466667,0.705882}%
\pgfsetfillcolor{currentfill}%
\pgfsetlinewidth{0.000000pt}%
\definecolor{currentstroke}{rgb}{0.000000,0.000000,0.000000}%
\pgfsetstrokecolor{currentstroke}%
\pgfsetstrokeopacity{0.000000}%
\pgfsetdash{}{0pt}%
\pgfpathmoveto{\pgfqpoint{5.800000in}{105.202461in}}%
\pgfpathlineto{\pgfqpoint{5.800000in}{105.202461in}}%
\pgfpathlineto{\pgfqpoint{5.800000in}{105.305298in}}%
\pgfpathlineto{\pgfqpoint{5.800000in}{105.305298in}}%
\pgfpathclose%
\pgfusepath{fill}%
\end{pgfscope}%
\begin{pgfscope}%
\pgfpathrectangle{\pgfqpoint{5.800000in}{0.720000in}}{\pgfqpoint{1.400000in}{4.620000in}}%
\pgfusepath{clip}%
\pgfsetbuttcap%
\pgfsetmiterjoin%
\definecolor{currentfill}{rgb}{0.121569,0.466667,0.705882}%
\pgfsetfillcolor{currentfill}%
\pgfsetlinewidth{0.000000pt}%
\definecolor{currentstroke}{rgb}{0.000000,0.000000,0.000000}%
\pgfsetstrokecolor{currentstroke}%
\pgfsetstrokeopacity{0.000000}%
\pgfsetdash{}{0pt}%
\pgfpathmoveto{\pgfqpoint{5.800000in}{105.305298in}}%
\pgfpathlineto{\pgfqpoint{5.800000in}{105.305298in}}%
\pgfpathlineto{\pgfqpoint{5.800000in}{105.408135in}}%
\pgfpathlineto{\pgfqpoint{5.800000in}{105.408135in}}%
\pgfpathclose%
\pgfusepath{fill}%
\end{pgfscope}%
\begin{pgfscope}%
\pgfpathrectangle{\pgfqpoint{5.800000in}{0.720000in}}{\pgfqpoint{1.400000in}{4.620000in}}%
\pgfusepath{clip}%
\pgfsetbuttcap%
\pgfsetmiterjoin%
\definecolor{currentfill}{rgb}{0.121569,0.466667,0.705882}%
\pgfsetfillcolor{currentfill}%
\pgfsetlinewidth{0.000000pt}%
\definecolor{currentstroke}{rgb}{0.000000,0.000000,0.000000}%
\pgfsetstrokecolor{currentstroke}%
\pgfsetstrokeopacity{0.000000}%
\pgfsetdash{}{0pt}%
\pgfpathmoveto{\pgfqpoint{5.800000in}{105.408135in}}%
\pgfpathlineto{\pgfqpoint{5.800000in}{105.408135in}}%
\pgfpathlineto{\pgfqpoint{5.800000in}{105.510972in}}%
\pgfpathlineto{\pgfqpoint{5.800000in}{105.510972in}}%
\pgfpathclose%
\pgfusepath{fill}%
\end{pgfscope}%
\begin{pgfscope}%
\pgfpathrectangle{\pgfqpoint{5.800000in}{0.720000in}}{\pgfqpoint{1.400000in}{4.620000in}}%
\pgfusepath{clip}%
\pgfsetbuttcap%
\pgfsetmiterjoin%
\definecolor{currentfill}{rgb}{0.121569,0.466667,0.705882}%
\pgfsetfillcolor{currentfill}%
\pgfsetlinewidth{0.000000pt}%
\definecolor{currentstroke}{rgb}{0.000000,0.000000,0.000000}%
\pgfsetstrokecolor{currentstroke}%
\pgfsetstrokeopacity{0.000000}%
\pgfsetdash{}{0pt}%
\pgfpathmoveto{\pgfqpoint{5.800000in}{105.510972in}}%
\pgfpathlineto{\pgfqpoint{5.800000in}{105.510972in}}%
\pgfpathlineto{\pgfqpoint{5.800000in}{105.613809in}}%
\pgfpathlineto{\pgfqpoint{5.800000in}{105.613809in}}%
\pgfpathclose%
\pgfusepath{fill}%
\end{pgfscope}%
\begin{pgfscope}%
\pgfpathrectangle{\pgfqpoint{5.800000in}{0.720000in}}{\pgfqpoint{1.400000in}{4.620000in}}%
\pgfusepath{clip}%
\pgfsetbuttcap%
\pgfsetmiterjoin%
\definecolor{currentfill}{rgb}{0.121569,0.466667,0.705882}%
\pgfsetfillcolor{currentfill}%
\pgfsetlinewidth{0.000000pt}%
\definecolor{currentstroke}{rgb}{0.000000,0.000000,0.000000}%
\pgfsetstrokecolor{currentstroke}%
\pgfsetstrokeopacity{0.000000}%
\pgfsetdash{}{0pt}%
\pgfpathmoveto{\pgfqpoint{5.800000in}{105.613809in}}%
\pgfpathlineto{\pgfqpoint{5.800000in}{105.613809in}}%
\pgfpathlineto{\pgfqpoint{5.800000in}{105.716646in}}%
\pgfpathlineto{\pgfqpoint{5.800000in}{105.716646in}}%
\pgfpathclose%
\pgfusepath{fill}%
\end{pgfscope}%
\begin{pgfscope}%
\pgfpathrectangle{\pgfqpoint{5.800000in}{0.720000in}}{\pgfqpoint{1.400000in}{4.620000in}}%
\pgfusepath{clip}%
\pgfsetbuttcap%
\pgfsetmiterjoin%
\definecolor{currentfill}{rgb}{0.121569,0.466667,0.705882}%
\pgfsetfillcolor{currentfill}%
\pgfsetlinewidth{0.000000pt}%
\definecolor{currentstroke}{rgb}{0.000000,0.000000,0.000000}%
\pgfsetstrokecolor{currentstroke}%
\pgfsetstrokeopacity{0.000000}%
\pgfsetdash{}{0pt}%
\pgfpathmoveto{\pgfqpoint{5.800000in}{105.716646in}}%
\pgfpathlineto{\pgfqpoint{5.800000in}{105.716646in}}%
\pgfpathlineto{\pgfqpoint{5.800000in}{105.819483in}}%
\pgfpathlineto{\pgfqpoint{5.800000in}{105.819483in}}%
\pgfpathclose%
\pgfusepath{fill}%
\end{pgfscope}%
\begin{pgfscope}%
\pgfpathrectangle{\pgfqpoint{5.800000in}{0.720000in}}{\pgfqpoint{1.400000in}{4.620000in}}%
\pgfusepath{clip}%
\pgfsetbuttcap%
\pgfsetmiterjoin%
\definecolor{currentfill}{rgb}{0.121569,0.466667,0.705882}%
\pgfsetfillcolor{currentfill}%
\pgfsetlinewidth{0.000000pt}%
\definecolor{currentstroke}{rgb}{0.000000,0.000000,0.000000}%
\pgfsetstrokecolor{currentstroke}%
\pgfsetstrokeopacity{0.000000}%
\pgfsetdash{}{0pt}%
\pgfpathmoveto{\pgfqpoint{5.800000in}{105.819483in}}%
\pgfpathlineto{\pgfqpoint{5.800000in}{105.819483in}}%
\pgfpathlineto{\pgfqpoint{5.800000in}{105.922320in}}%
\pgfpathlineto{\pgfqpoint{5.800000in}{105.922320in}}%
\pgfpathclose%
\pgfusepath{fill}%
\end{pgfscope}%
\begin{pgfscope}%
\pgfpathrectangle{\pgfqpoint{5.800000in}{0.720000in}}{\pgfqpoint{1.400000in}{4.620000in}}%
\pgfusepath{clip}%
\pgfsetbuttcap%
\pgfsetmiterjoin%
\definecolor{currentfill}{rgb}{0.121569,0.466667,0.705882}%
\pgfsetfillcolor{currentfill}%
\pgfsetlinewidth{0.000000pt}%
\definecolor{currentstroke}{rgb}{0.000000,0.000000,0.000000}%
\pgfsetstrokecolor{currentstroke}%
\pgfsetstrokeopacity{0.000000}%
\pgfsetdash{}{0pt}%
\pgfpathmoveto{\pgfqpoint{5.800000in}{105.922320in}}%
\pgfpathlineto{\pgfqpoint{5.800000in}{105.922320in}}%
\pgfpathlineto{\pgfqpoint{5.800000in}{106.025157in}}%
\pgfpathlineto{\pgfqpoint{5.800000in}{106.025157in}}%
\pgfpathclose%
\pgfusepath{fill}%
\end{pgfscope}%
\begin{pgfscope}%
\pgfpathrectangle{\pgfqpoint{5.800000in}{0.720000in}}{\pgfqpoint{1.400000in}{4.620000in}}%
\pgfusepath{clip}%
\pgfsetbuttcap%
\pgfsetmiterjoin%
\definecolor{currentfill}{rgb}{0.121569,0.466667,0.705882}%
\pgfsetfillcolor{currentfill}%
\pgfsetlinewidth{0.000000pt}%
\definecolor{currentstroke}{rgb}{0.000000,0.000000,0.000000}%
\pgfsetstrokecolor{currentstroke}%
\pgfsetstrokeopacity{0.000000}%
\pgfsetdash{}{0pt}%
\pgfpathmoveto{\pgfqpoint{5.800000in}{106.025157in}}%
\pgfpathlineto{\pgfqpoint{5.800000in}{106.025157in}}%
\pgfpathlineto{\pgfqpoint{5.800000in}{106.127994in}}%
\pgfpathlineto{\pgfqpoint{5.800000in}{106.127994in}}%
\pgfpathclose%
\pgfusepath{fill}%
\end{pgfscope}%
\begin{pgfscope}%
\pgfpathrectangle{\pgfqpoint{5.800000in}{0.720000in}}{\pgfqpoint{1.400000in}{4.620000in}}%
\pgfusepath{clip}%
\pgfsetbuttcap%
\pgfsetmiterjoin%
\definecolor{currentfill}{rgb}{0.121569,0.466667,0.705882}%
\pgfsetfillcolor{currentfill}%
\pgfsetlinewidth{0.000000pt}%
\definecolor{currentstroke}{rgb}{0.000000,0.000000,0.000000}%
\pgfsetstrokecolor{currentstroke}%
\pgfsetstrokeopacity{0.000000}%
\pgfsetdash{}{0pt}%
\pgfpathmoveto{\pgfqpoint{5.800000in}{106.127994in}}%
\pgfpathlineto{\pgfqpoint{5.800000in}{106.127994in}}%
\pgfpathlineto{\pgfqpoint{5.800000in}{106.230832in}}%
\pgfpathlineto{\pgfqpoint{5.800000in}{106.230832in}}%
\pgfpathclose%
\pgfusepath{fill}%
\end{pgfscope}%
\begin{pgfscope}%
\pgfpathrectangle{\pgfqpoint{5.800000in}{0.720000in}}{\pgfqpoint{1.400000in}{4.620000in}}%
\pgfusepath{clip}%
\pgfsetbuttcap%
\pgfsetmiterjoin%
\definecolor{currentfill}{rgb}{0.121569,0.466667,0.705882}%
\pgfsetfillcolor{currentfill}%
\pgfsetlinewidth{0.000000pt}%
\definecolor{currentstroke}{rgb}{0.000000,0.000000,0.000000}%
\pgfsetstrokecolor{currentstroke}%
\pgfsetstrokeopacity{0.000000}%
\pgfsetdash{}{0pt}%
\pgfpathmoveto{\pgfqpoint{5.800000in}{106.230832in}}%
\pgfpathlineto{\pgfqpoint{5.800000in}{106.230832in}}%
\pgfpathlineto{\pgfqpoint{5.800000in}{106.333669in}}%
\pgfpathlineto{\pgfqpoint{5.800000in}{106.333669in}}%
\pgfpathclose%
\pgfusepath{fill}%
\end{pgfscope}%
\begin{pgfscope}%
\pgfpathrectangle{\pgfqpoint{5.800000in}{0.720000in}}{\pgfqpoint{1.400000in}{4.620000in}}%
\pgfusepath{clip}%
\pgfsetbuttcap%
\pgfsetmiterjoin%
\definecolor{currentfill}{rgb}{0.121569,0.466667,0.705882}%
\pgfsetfillcolor{currentfill}%
\pgfsetlinewidth{0.000000pt}%
\definecolor{currentstroke}{rgb}{0.000000,0.000000,0.000000}%
\pgfsetstrokecolor{currentstroke}%
\pgfsetstrokeopacity{0.000000}%
\pgfsetdash{}{0pt}%
\pgfpathmoveto{\pgfqpoint{5.800000in}{106.333669in}}%
\pgfpathlineto{\pgfqpoint{5.800000in}{106.333669in}}%
\pgfpathlineto{\pgfqpoint{5.800000in}{106.436506in}}%
\pgfpathlineto{\pgfqpoint{5.800000in}{106.436506in}}%
\pgfpathclose%
\pgfusepath{fill}%
\end{pgfscope}%
\begin{pgfscope}%
\pgfpathrectangle{\pgfqpoint{5.800000in}{0.720000in}}{\pgfqpoint{1.400000in}{4.620000in}}%
\pgfusepath{clip}%
\pgfsetbuttcap%
\pgfsetmiterjoin%
\definecolor{currentfill}{rgb}{0.121569,0.466667,0.705882}%
\pgfsetfillcolor{currentfill}%
\pgfsetlinewidth{0.000000pt}%
\definecolor{currentstroke}{rgb}{0.000000,0.000000,0.000000}%
\pgfsetstrokecolor{currentstroke}%
\pgfsetstrokeopacity{0.000000}%
\pgfsetdash{}{0pt}%
\pgfpathmoveto{\pgfqpoint{5.800000in}{106.436506in}}%
\pgfpathlineto{\pgfqpoint{5.800000in}{106.436506in}}%
\pgfpathlineto{\pgfqpoint{5.800000in}{106.539343in}}%
\pgfpathlineto{\pgfqpoint{5.800000in}{106.539343in}}%
\pgfpathclose%
\pgfusepath{fill}%
\end{pgfscope}%
\begin{pgfscope}%
\pgfpathrectangle{\pgfqpoint{5.800000in}{0.720000in}}{\pgfqpoint{1.400000in}{4.620000in}}%
\pgfusepath{clip}%
\pgfsetbuttcap%
\pgfsetmiterjoin%
\definecolor{currentfill}{rgb}{0.121569,0.466667,0.705882}%
\pgfsetfillcolor{currentfill}%
\pgfsetlinewidth{0.000000pt}%
\definecolor{currentstroke}{rgb}{0.000000,0.000000,0.000000}%
\pgfsetstrokecolor{currentstroke}%
\pgfsetstrokeopacity{0.000000}%
\pgfsetdash{}{0pt}%
\pgfpathmoveto{\pgfqpoint{5.800000in}{106.539343in}}%
\pgfpathlineto{\pgfqpoint{5.800000in}{106.539343in}}%
\pgfpathlineto{\pgfqpoint{5.800000in}{106.642180in}}%
\pgfpathlineto{\pgfqpoint{5.800000in}{106.642180in}}%
\pgfpathclose%
\pgfusepath{fill}%
\end{pgfscope}%
\begin{pgfscope}%
\pgfpathrectangle{\pgfqpoint{5.800000in}{0.720000in}}{\pgfqpoint{1.400000in}{4.620000in}}%
\pgfusepath{clip}%
\pgfsetbuttcap%
\pgfsetmiterjoin%
\definecolor{currentfill}{rgb}{0.121569,0.466667,0.705882}%
\pgfsetfillcolor{currentfill}%
\pgfsetlinewidth{0.000000pt}%
\definecolor{currentstroke}{rgb}{0.000000,0.000000,0.000000}%
\pgfsetstrokecolor{currentstroke}%
\pgfsetstrokeopacity{0.000000}%
\pgfsetdash{}{0pt}%
\pgfpathmoveto{\pgfqpoint{5.800000in}{106.642180in}}%
\pgfpathlineto{\pgfqpoint{5.800000in}{106.642180in}}%
\pgfpathlineto{\pgfqpoint{5.800000in}{106.745017in}}%
\pgfpathlineto{\pgfqpoint{5.800000in}{106.745017in}}%
\pgfpathclose%
\pgfusepath{fill}%
\end{pgfscope}%
\begin{pgfscope}%
\pgfpathrectangle{\pgfqpoint{5.800000in}{0.720000in}}{\pgfqpoint{1.400000in}{4.620000in}}%
\pgfusepath{clip}%
\pgfsetbuttcap%
\pgfsetmiterjoin%
\definecolor{currentfill}{rgb}{0.121569,0.466667,0.705882}%
\pgfsetfillcolor{currentfill}%
\pgfsetlinewidth{0.000000pt}%
\definecolor{currentstroke}{rgb}{0.000000,0.000000,0.000000}%
\pgfsetstrokecolor{currentstroke}%
\pgfsetstrokeopacity{0.000000}%
\pgfsetdash{}{0pt}%
\pgfpathmoveto{\pgfqpoint{5.800000in}{106.745017in}}%
\pgfpathlineto{\pgfqpoint{5.800000in}{106.745017in}}%
\pgfpathlineto{\pgfqpoint{5.800000in}{106.847854in}}%
\pgfpathlineto{\pgfqpoint{5.800000in}{106.847854in}}%
\pgfpathclose%
\pgfusepath{fill}%
\end{pgfscope}%
\begin{pgfscope}%
\pgfpathrectangle{\pgfqpoint{5.800000in}{0.720000in}}{\pgfqpoint{1.400000in}{4.620000in}}%
\pgfusepath{clip}%
\pgfsetbuttcap%
\pgfsetmiterjoin%
\definecolor{currentfill}{rgb}{0.121569,0.466667,0.705882}%
\pgfsetfillcolor{currentfill}%
\pgfsetlinewidth{0.000000pt}%
\definecolor{currentstroke}{rgb}{0.000000,0.000000,0.000000}%
\pgfsetstrokecolor{currentstroke}%
\pgfsetstrokeopacity{0.000000}%
\pgfsetdash{}{0pt}%
\pgfpathmoveto{\pgfqpoint{5.800000in}{106.847854in}}%
\pgfpathlineto{\pgfqpoint{5.800000in}{106.847854in}}%
\pgfpathlineto{\pgfqpoint{5.800000in}{106.950691in}}%
\pgfpathlineto{\pgfqpoint{5.800000in}{106.950691in}}%
\pgfpathclose%
\pgfusepath{fill}%
\end{pgfscope}%
\begin{pgfscope}%
\pgfpathrectangle{\pgfqpoint{5.800000in}{0.720000in}}{\pgfqpoint{1.400000in}{4.620000in}}%
\pgfusepath{clip}%
\pgfsetbuttcap%
\pgfsetmiterjoin%
\definecolor{currentfill}{rgb}{0.121569,0.466667,0.705882}%
\pgfsetfillcolor{currentfill}%
\pgfsetlinewidth{0.000000pt}%
\definecolor{currentstroke}{rgb}{0.000000,0.000000,0.000000}%
\pgfsetstrokecolor{currentstroke}%
\pgfsetstrokeopacity{0.000000}%
\pgfsetdash{}{0pt}%
\pgfpathmoveto{\pgfqpoint{5.800000in}{106.950691in}}%
\pgfpathlineto{\pgfqpoint{5.800000in}{106.950691in}}%
\pgfpathlineto{\pgfqpoint{5.800000in}{107.053528in}}%
\pgfpathlineto{\pgfqpoint{5.800000in}{107.053528in}}%
\pgfpathclose%
\pgfusepath{fill}%
\end{pgfscope}%
\begin{pgfscope}%
\pgfpathrectangle{\pgfqpoint{5.800000in}{0.720000in}}{\pgfqpoint{1.400000in}{4.620000in}}%
\pgfusepath{clip}%
\pgfsetbuttcap%
\pgfsetmiterjoin%
\definecolor{currentfill}{rgb}{0.121569,0.466667,0.705882}%
\pgfsetfillcolor{currentfill}%
\pgfsetlinewidth{0.000000pt}%
\definecolor{currentstroke}{rgb}{0.000000,0.000000,0.000000}%
\pgfsetstrokecolor{currentstroke}%
\pgfsetstrokeopacity{0.000000}%
\pgfsetdash{}{0pt}%
\pgfpathmoveto{\pgfqpoint{5.800000in}{107.053528in}}%
\pgfpathlineto{\pgfqpoint{5.800000in}{107.053528in}}%
\pgfpathlineto{\pgfqpoint{5.800000in}{107.156365in}}%
\pgfpathlineto{\pgfqpoint{5.800000in}{107.156365in}}%
\pgfpathclose%
\pgfusepath{fill}%
\end{pgfscope}%
\begin{pgfscope}%
\pgfpathrectangle{\pgfqpoint{5.800000in}{0.720000in}}{\pgfqpoint{1.400000in}{4.620000in}}%
\pgfusepath{clip}%
\pgfsetbuttcap%
\pgfsetmiterjoin%
\definecolor{currentfill}{rgb}{0.121569,0.466667,0.705882}%
\pgfsetfillcolor{currentfill}%
\pgfsetlinewidth{0.000000pt}%
\definecolor{currentstroke}{rgb}{0.000000,0.000000,0.000000}%
\pgfsetstrokecolor{currentstroke}%
\pgfsetstrokeopacity{0.000000}%
\pgfsetdash{}{0pt}%
\pgfpathmoveto{\pgfqpoint{5.800000in}{107.156365in}}%
\pgfpathlineto{\pgfqpoint{5.800000in}{107.156365in}}%
\pgfpathlineto{\pgfqpoint{5.800000in}{107.259202in}}%
\pgfpathlineto{\pgfqpoint{5.800000in}{107.259202in}}%
\pgfpathclose%
\pgfusepath{fill}%
\end{pgfscope}%
\begin{pgfscope}%
\pgfpathrectangle{\pgfqpoint{5.800000in}{0.720000in}}{\pgfqpoint{1.400000in}{4.620000in}}%
\pgfusepath{clip}%
\pgfsetbuttcap%
\pgfsetmiterjoin%
\definecolor{currentfill}{rgb}{0.121569,0.466667,0.705882}%
\pgfsetfillcolor{currentfill}%
\pgfsetlinewidth{0.000000pt}%
\definecolor{currentstroke}{rgb}{0.000000,0.000000,0.000000}%
\pgfsetstrokecolor{currentstroke}%
\pgfsetstrokeopacity{0.000000}%
\pgfsetdash{}{0pt}%
\pgfpathmoveto{\pgfqpoint{5.800000in}{107.259202in}}%
\pgfpathlineto{\pgfqpoint{5.800000in}{107.259202in}}%
\pgfpathlineto{\pgfqpoint{5.800000in}{107.362039in}}%
\pgfpathlineto{\pgfqpoint{5.800000in}{107.362039in}}%
\pgfpathclose%
\pgfusepath{fill}%
\end{pgfscope}%
\begin{pgfscope}%
\pgfpathrectangle{\pgfqpoint{5.800000in}{0.720000in}}{\pgfqpoint{1.400000in}{4.620000in}}%
\pgfusepath{clip}%
\pgfsetbuttcap%
\pgfsetmiterjoin%
\definecolor{currentfill}{rgb}{0.121569,0.466667,0.705882}%
\pgfsetfillcolor{currentfill}%
\pgfsetlinewidth{0.000000pt}%
\definecolor{currentstroke}{rgb}{0.000000,0.000000,0.000000}%
\pgfsetstrokecolor{currentstroke}%
\pgfsetstrokeopacity{0.000000}%
\pgfsetdash{}{0pt}%
\pgfpathmoveto{\pgfqpoint{5.800000in}{107.362039in}}%
\pgfpathlineto{\pgfqpoint{5.800000in}{107.362039in}}%
\pgfpathlineto{\pgfqpoint{5.800000in}{107.464876in}}%
\pgfpathlineto{\pgfqpoint{5.800000in}{107.464876in}}%
\pgfpathclose%
\pgfusepath{fill}%
\end{pgfscope}%
\begin{pgfscope}%
\pgfpathrectangle{\pgfqpoint{5.800000in}{0.720000in}}{\pgfqpoint{1.400000in}{4.620000in}}%
\pgfusepath{clip}%
\pgfsetbuttcap%
\pgfsetmiterjoin%
\definecolor{currentfill}{rgb}{0.121569,0.466667,0.705882}%
\pgfsetfillcolor{currentfill}%
\pgfsetlinewidth{0.000000pt}%
\definecolor{currentstroke}{rgb}{0.000000,0.000000,0.000000}%
\pgfsetstrokecolor{currentstroke}%
\pgfsetstrokeopacity{0.000000}%
\pgfsetdash{}{0pt}%
\pgfpathmoveto{\pgfqpoint{5.800000in}{107.464876in}}%
\pgfpathlineto{\pgfqpoint{5.800000in}{107.464876in}}%
\pgfpathlineto{\pgfqpoint{5.800000in}{107.567713in}}%
\pgfpathlineto{\pgfqpoint{5.800000in}{107.567713in}}%
\pgfpathclose%
\pgfusepath{fill}%
\end{pgfscope}%
\begin{pgfscope}%
\pgfpathrectangle{\pgfqpoint{5.800000in}{0.720000in}}{\pgfqpoint{1.400000in}{4.620000in}}%
\pgfusepath{clip}%
\pgfsetbuttcap%
\pgfsetmiterjoin%
\definecolor{currentfill}{rgb}{0.121569,0.466667,0.705882}%
\pgfsetfillcolor{currentfill}%
\pgfsetlinewidth{0.000000pt}%
\definecolor{currentstroke}{rgb}{0.000000,0.000000,0.000000}%
\pgfsetstrokecolor{currentstroke}%
\pgfsetstrokeopacity{0.000000}%
\pgfsetdash{}{0pt}%
\pgfpathmoveto{\pgfqpoint{5.800000in}{107.567713in}}%
\pgfpathlineto{\pgfqpoint{5.800000in}{107.567713in}}%
\pgfpathlineto{\pgfqpoint{5.800000in}{107.670550in}}%
\pgfpathlineto{\pgfqpoint{5.800000in}{107.670550in}}%
\pgfpathclose%
\pgfusepath{fill}%
\end{pgfscope}%
\begin{pgfscope}%
\pgfpathrectangle{\pgfqpoint{5.800000in}{0.720000in}}{\pgfqpoint{1.400000in}{4.620000in}}%
\pgfusepath{clip}%
\pgfsetbuttcap%
\pgfsetmiterjoin%
\definecolor{currentfill}{rgb}{0.121569,0.466667,0.705882}%
\pgfsetfillcolor{currentfill}%
\pgfsetlinewidth{0.000000pt}%
\definecolor{currentstroke}{rgb}{0.000000,0.000000,0.000000}%
\pgfsetstrokecolor{currentstroke}%
\pgfsetstrokeopacity{0.000000}%
\pgfsetdash{}{0pt}%
\pgfpathmoveto{\pgfqpoint{5.800000in}{107.670550in}}%
\pgfpathlineto{\pgfqpoint{5.800000in}{107.670550in}}%
\pgfpathlineto{\pgfqpoint{5.800000in}{107.773388in}}%
\pgfpathlineto{\pgfqpoint{5.800000in}{107.773388in}}%
\pgfpathclose%
\pgfusepath{fill}%
\end{pgfscope}%
\begin{pgfscope}%
\pgfpathrectangle{\pgfqpoint{5.800000in}{0.720000in}}{\pgfqpoint{1.400000in}{4.620000in}}%
\pgfusepath{clip}%
\pgfsetbuttcap%
\pgfsetmiterjoin%
\definecolor{currentfill}{rgb}{0.121569,0.466667,0.705882}%
\pgfsetfillcolor{currentfill}%
\pgfsetlinewidth{0.000000pt}%
\definecolor{currentstroke}{rgb}{0.000000,0.000000,0.000000}%
\pgfsetstrokecolor{currentstroke}%
\pgfsetstrokeopacity{0.000000}%
\pgfsetdash{}{0pt}%
\pgfpathmoveto{\pgfqpoint{5.800000in}{107.773388in}}%
\pgfpathlineto{\pgfqpoint{5.800000in}{107.773388in}}%
\pgfpathlineto{\pgfqpoint{5.800000in}{107.876225in}}%
\pgfpathlineto{\pgfqpoint{5.800000in}{107.876225in}}%
\pgfpathclose%
\pgfusepath{fill}%
\end{pgfscope}%
\begin{pgfscope}%
\pgfpathrectangle{\pgfqpoint{5.800000in}{0.720000in}}{\pgfqpoint{1.400000in}{4.620000in}}%
\pgfusepath{clip}%
\pgfsetbuttcap%
\pgfsetmiterjoin%
\definecolor{currentfill}{rgb}{0.121569,0.466667,0.705882}%
\pgfsetfillcolor{currentfill}%
\pgfsetlinewidth{0.000000pt}%
\definecolor{currentstroke}{rgb}{0.000000,0.000000,0.000000}%
\pgfsetstrokecolor{currentstroke}%
\pgfsetstrokeopacity{0.000000}%
\pgfsetdash{}{0pt}%
\pgfpathmoveto{\pgfqpoint{5.800000in}{107.876225in}}%
\pgfpathlineto{\pgfqpoint{5.800000in}{107.876225in}}%
\pgfpathlineto{\pgfqpoint{5.800000in}{107.979062in}}%
\pgfpathlineto{\pgfqpoint{5.800000in}{107.979062in}}%
\pgfpathclose%
\pgfusepath{fill}%
\end{pgfscope}%
\begin{pgfscope}%
\pgfpathrectangle{\pgfqpoint{5.800000in}{0.720000in}}{\pgfqpoint{1.400000in}{4.620000in}}%
\pgfusepath{clip}%
\pgfsetbuttcap%
\pgfsetmiterjoin%
\definecolor{currentfill}{rgb}{0.121569,0.466667,0.705882}%
\pgfsetfillcolor{currentfill}%
\pgfsetlinewidth{0.000000pt}%
\definecolor{currentstroke}{rgb}{0.000000,0.000000,0.000000}%
\pgfsetstrokecolor{currentstroke}%
\pgfsetstrokeopacity{0.000000}%
\pgfsetdash{}{0pt}%
\pgfpathmoveto{\pgfqpoint{5.800000in}{107.979062in}}%
\pgfpathlineto{\pgfqpoint{5.800000in}{107.979062in}}%
\pgfpathlineto{\pgfqpoint{5.800000in}{108.081899in}}%
\pgfpathlineto{\pgfqpoint{5.800000in}{108.081899in}}%
\pgfpathclose%
\pgfusepath{fill}%
\end{pgfscope}%
\begin{pgfscope}%
\pgfpathrectangle{\pgfqpoint{5.800000in}{0.720000in}}{\pgfqpoint{1.400000in}{4.620000in}}%
\pgfusepath{clip}%
\pgfsetbuttcap%
\pgfsetmiterjoin%
\definecolor{currentfill}{rgb}{0.121569,0.466667,0.705882}%
\pgfsetfillcolor{currentfill}%
\pgfsetlinewidth{0.000000pt}%
\definecolor{currentstroke}{rgb}{0.000000,0.000000,0.000000}%
\pgfsetstrokecolor{currentstroke}%
\pgfsetstrokeopacity{0.000000}%
\pgfsetdash{}{0pt}%
\pgfpathmoveto{\pgfqpoint{5.800000in}{108.081899in}}%
\pgfpathlineto{\pgfqpoint{5.800000in}{108.081899in}}%
\pgfpathlineto{\pgfqpoint{5.800000in}{108.184736in}}%
\pgfpathlineto{\pgfqpoint{5.800000in}{108.184736in}}%
\pgfpathclose%
\pgfusepath{fill}%
\end{pgfscope}%
\begin{pgfscope}%
\pgfpathrectangle{\pgfqpoint{5.800000in}{0.720000in}}{\pgfqpoint{1.400000in}{4.620000in}}%
\pgfusepath{clip}%
\pgfsetbuttcap%
\pgfsetmiterjoin%
\definecolor{currentfill}{rgb}{0.121569,0.466667,0.705882}%
\pgfsetfillcolor{currentfill}%
\pgfsetlinewidth{0.000000pt}%
\definecolor{currentstroke}{rgb}{0.000000,0.000000,0.000000}%
\pgfsetstrokecolor{currentstroke}%
\pgfsetstrokeopacity{0.000000}%
\pgfsetdash{}{0pt}%
\pgfpathmoveto{\pgfqpoint{5.800000in}{108.184736in}}%
\pgfpathlineto{\pgfqpoint{5.800000in}{108.184736in}}%
\pgfpathlineto{\pgfqpoint{5.800000in}{108.287573in}}%
\pgfpathlineto{\pgfqpoint{5.800000in}{108.287573in}}%
\pgfpathclose%
\pgfusepath{fill}%
\end{pgfscope}%
\begin{pgfscope}%
\pgfpathrectangle{\pgfqpoint{5.800000in}{0.720000in}}{\pgfqpoint{1.400000in}{4.620000in}}%
\pgfusepath{clip}%
\pgfsetbuttcap%
\pgfsetmiterjoin%
\definecolor{currentfill}{rgb}{0.121569,0.466667,0.705882}%
\pgfsetfillcolor{currentfill}%
\pgfsetlinewidth{0.000000pt}%
\definecolor{currentstroke}{rgb}{0.000000,0.000000,0.000000}%
\pgfsetstrokecolor{currentstroke}%
\pgfsetstrokeopacity{0.000000}%
\pgfsetdash{}{0pt}%
\pgfpathmoveto{\pgfqpoint{5.800000in}{108.287573in}}%
\pgfpathlineto{\pgfqpoint{5.800000in}{108.287573in}}%
\pgfpathlineto{\pgfqpoint{5.800000in}{108.390410in}}%
\pgfpathlineto{\pgfqpoint{5.800000in}{108.390410in}}%
\pgfpathclose%
\pgfusepath{fill}%
\end{pgfscope}%
\begin{pgfscope}%
\pgfpathrectangle{\pgfqpoint{5.800000in}{0.720000in}}{\pgfqpoint{1.400000in}{4.620000in}}%
\pgfusepath{clip}%
\pgfsetbuttcap%
\pgfsetmiterjoin%
\definecolor{currentfill}{rgb}{0.121569,0.466667,0.705882}%
\pgfsetfillcolor{currentfill}%
\pgfsetlinewidth{0.000000pt}%
\definecolor{currentstroke}{rgb}{0.000000,0.000000,0.000000}%
\pgfsetstrokecolor{currentstroke}%
\pgfsetstrokeopacity{0.000000}%
\pgfsetdash{}{0pt}%
\pgfpathmoveto{\pgfqpoint{5.800000in}{108.390410in}}%
\pgfpathlineto{\pgfqpoint{5.800000in}{108.390410in}}%
\pgfpathlineto{\pgfqpoint{5.800000in}{108.493247in}}%
\pgfpathlineto{\pgfqpoint{5.800000in}{108.493247in}}%
\pgfpathclose%
\pgfusepath{fill}%
\end{pgfscope}%
\begin{pgfscope}%
\pgfpathrectangle{\pgfqpoint{5.800000in}{0.720000in}}{\pgfqpoint{1.400000in}{4.620000in}}%
\pgfusepath{clip}%
\pgfsetbuttcap%
\pgfsetmiterjoin%
\definecolor{currentfill}{rgb}{0.121569,0.466667,0.705882}%
\pgfsetfillcolor{currentfill}%
\pgfsetlinewidth{0.000000pt}%
\definecolor{currentstroke}{rgb}{0.000000,0.000000,0.000000}%
\pgfsetstrokecolor{currentstroke}%
\pgfsetstrokeopacity{0.000000}%
\pgfsetdash{}{0pt}%
\pgfpathmoveto{\pgfqpoint{5.800000in}{108.493247in}}%
\pgfpathlineto{\pgfqpoint{5.800000in}{108.493247in}}%
\pgfpathlineto{\pgfqpoint{5.800000in}{108.596084in}}%
\pgfpathlineto{\pgfqpoint{5.800000in}{108.596084in}}%
\pgfpathclose%
\pgfusepath{fill}%
\end{pgfscope}%
\begin{pgfscope}%
\pgfpathrectangle{\pgfqpoint{5.800000in}{0.720000in}}{\pgfqpoint{1.400000in}{4.620000in}}%
\pgfusepath{clip}%
\pgfsetbuttcap%
\pgfsetmiterjoin%
\definecolor{currentfill}{rgb}{0.121569,0.466667,0.705882}%
\pgfsetfillcolor{currentfill}%
\pgfsetlinewidth{0.000000pt}%
\definecolor{currentstroke}{rgb}{0.000000,0.000000,0.000000}%
\pgfsetstrokecolor{currentstroke}%
\pgfsetstrokeopacity{0.000000}%
\pgfsetdash{}{0pt}%
\pgfpathmoveto{\pgfqpoint{5.800000in}{108.596084in}}%
\pgfpathlineto{\pgfqpoint{5.800000in}{108.596084in}}%
\pgfpathlineto{\pgfqpoint{5.800000in}{108.698921in}}%
\pgfpathlineto{\pgfqpoint{5.800000in}{108.698921in}}%
\pgfpathclose%
\pgfusepath{fill}%
\end{pgfscope}%
\begin{pgfscope}%
\pgfpathrectangle{\pgfqpoint{5.800000in}{0.720000in}}{\pgfqpoint{1.400000in}{4.620000in}}%
\pgfusepath{clip}%
\pgfsetbuttcap%
\pgfsetmiterjoin%
\definecolor{currentfill}{rgb}{0.121569,0.466667,0.705882}%
\pgfsetfillcolor{currentfill}%
\pgfsetlinewidth{0.000000pt}%
\definecolor{currentstroke}{rgb}{0.000000,0.000000,0.000000}%
\pgfsetstrokecolor{currentstroke}%
\pgfsetstrokeopacity{0.000000}%
\pgfsetdash{}{0pt}%
\pgfpathmoveto{\pgfqpoint{5.800000in}{108.698921in}}%
\pgfpathlineto{\pgfqpoint{5.800000in}{108.698921in}}%
\pgfpathlineto{\pgfqpoint{5.800000in}{108.801758in}}%
\pgfpathlineto{\pgfqpoint{5.800000in}{108.801758in}}%
\pgfpathclose%
\pgfusepath{fill}%
\end{pgfscope}%
\begin{pgfscope}%
\pgfpathrectangle{\pgfqpoint{5.800000in}{0.720000in}}{\pgfqpoint{1.400000in}{4.620000in}}%
\pgfusepath{clip}%
\pgfsetbuttcap%
\pgfsetmiterjoin%
\definecolor{currentfill}{rgb}{0.121569,0.466667,0.705882}%
\pgfsetfillcolor{currentfill}%
\pgfsetlinewidth{0.000000pt}%
\definecolor{currentstroke}{rgb}{0.000000,0.000000,0.000000}%
\pgfsetstrokecolor{currentstroke}%
\pgfsetstrokeopacity{0.000000}%
\pgfsetdash{}{0pt}%
\pgfpathmoveto{\pgfqpoint{5.800000in}{108.801758in}}%
\pgfpathlineto{\pgfqpoint{5.800000in}{108.801758in}}%
\pgfpathlineto{\pgfqpoint{5.800000in}{108.904595in}}%
\pgfpathlineto{\pgfqpoint{5.800000in}{108.904595in}}%
\pgfpathclose%
\pgfusepath{fill}%
\end{pgfscope}%
\begin{pgfscope}%
\pgfpathrectangle{\pgfqpoint{5.800000in}{0.720000in}}{\pgfqpoint{1.400000in}{4.620000in}}%
\pgfusepath{clip}%
\pgfsetbuttcap%
\pgfsetmiterjoin%
\definecolor{currentfill}{rgb}{0.121569,0.466667,0.705882}%
\pgfsetfillcolor{currentfill}%
\pgfsetlinewidth{0.000000pt}%
\definecolor{currentstroke}{rgb}{0.000000,0.000000,0.000000}%
\pgfsetstrokecolor{currentstroke}%
\pgfsetstrokeopacity{0.000000}%
\pgfsetdash{}{0pt}%
\pgfpathmoveto{\pgfqpoint{5.800000in}{108.904595in}}%
\pgfpathlineto{\pgfqpoint{5.800000in}{108.904595in}}%
\pgfpathlineto{\pgfqpoint{5.800000in}{109.007432in}}%
\pgfpathlineto{\pgfqpoint{5.800000in}{109.007432in}}%
\pgfpathclose%
\pgfusepath{fill}%
\end{pgfscope}%
\begin{pgfscope}%
\pgfpathrectangle{\pgfqpoint{5.800000in}{0.720000in}}{\pgfqpoint{1.400000in}{4.620000in}}%
\pgfusepath{clip}%
\pgfsetbuttcap%
\pgfsetmiterjoin%
\definecolor{currentfill}{rgb}{0.121569,0.466667,0.705882}%
\pgfsetfillcolor{currentfill}%
\pgfsetlinewidth{0.000000pt}%
\definecolor{currentstroke}{rgb}{0.000000,0.000000,0.000000}%
\pgfsetstrokecolor{currentstroke}%
\pgfsetstrokeopacity{0.000000}%
\pgfsetdash{}{0pt}%
\pgfpathmoveto{\pgfqpoint{5.800000in}{109.007432in}}%
\pgfpathlineto{\pgfqpoint{5.800000in}{109.007432in}}%
\pgfpathlineto{\pgfqpoint{5.800000in}{109.110269in}}%
\pgfpathlineto{\pgfqpoint{5.800000in}{109.110269in}}%
\pgfpathclose%
\pgfusepath{fill}%
\end{pgfscope}%
\begin{pgfscope}%
\pgfpathrectangle{\pgfqpoint{5.800000in}{0.720000in}}{\pgfqpoint{1.400000in}{4.620000in}}%
\pgfusepath{clip}%
\pgfsetbuttcap%
\pgfsetmiterjoin%
\definecolor{currentfill}{rgb}{0.121569,0.466667,0.705882}%
\pgfsetfillcolor{currentfill}%
\pgfsetlinewidth{0.000000pt}%
\definecolor{currentstroke}{rgb}{0.000000,0.000000,0.000000}%
\pgfsetstrokecolor{currentstroke}%
\pgfsetstrokeopacity{0.000000}%
\pgfsetdash{}{0pt}%
\pgfpathmoveto{\pgfqpoint{5.800000in}{109.110269in}}%
\pgfpathlineto{\pgfqpoint{5.800000in}{109.110269in}}%
\pgfpathlineto{\pgfqpoint{5.800000in}{109.213107in}}%
\pgfpathlineto{\pgfqpoint{5.800000in}{109.213107in}}%
\pgfpathclose%
\pgfusepath{fill}%
\end{pgfscope}%
\begin{pgfscope}%
\pgfpathrectangle{\pgfqpoint{5.800000in}{0.720000in}}{\pgfqpoint{1.400000in}{4.620000in}}%
\pgfusepath{clip}%
\pgfsetbuttcap%
\pgfsetmiterjoin%
\definecolor{currentfill}{rgb}{0.121569,0.466667,0.705882}%
\pgfsetfillcolor{currentfill}%
\pgfsetlinewidth{0.000000pt}%
\definecolor{currentstroke}{rgb}{0.000000,0.000000,0.000000}%
\pgfsetstrokecolor{currentstroke}%
\pgfsetstrokeopacity{0.000000}%
\pgfsetdash{}{0pt}%
\pgfpathmoveto{\pgfqpoint{5.800000in}{109.213107in}}%
\pgfpathlineto{\pgfqpoint{5.800000in}{109.213107in}}%
\pgfpathlineto{\pgfqpoint{5.800000in}{109.315944in}}%
\pgfpathlineto{\pgfqpoint{5.800000in}{109.315944in}}%
\pgfpathclose%
\pgfusepath{fill}%
\end{pgfscope}%
\begin{pgfscope}%
\pgfpathrectangle{\pgfqpoint{5.800000in}{0.720000in}}{\pgfqpoint{1.400000in}{4.620000in}}%
\pgfusepath{clip}%
\pgfsetbuttcap%
\pgfsetmiterjoin%
\definecolor{currentfill}{rgb}{0.121569,0.466667,0.705882}%
\pgfsetfillcolor{currentfill}%
\pgfsetlinewidth{0.000000pt}%
\definecolor{currentstroke}{rgb}{0.000000,0.000000,0.000000}%
\pgfsetstrokecolor{currentstroke}%
\pgfsetstrokeopacity{0.000000}%
\pgfsetdash{}{0pt}%
\pgfpathmoveto{\pgfqpoint{5.800000in}{109.315944in}}%
\pgfpathlineto{\pgfqpoint{5.800000in}{109.315944in}}%
\pgfpathlineto{\pgfqpoint{5.800000in}{109.418781in}}%
\pgfpathlineto{\pgfqpoint{5.800000in}{109.418781in}}%
\pgfpathclose%
\pgfusepath{fill}%
\end{pgfscope}%
\begin{pgfscope}%
\pgfpathrectangle{\pgfqpoint{5.800000in}{0.720000in}}{\pgfqpoint{1.400000in}{4.620000in}}%
\pgfusepath{clip}%
\pgfsetbuttcap%
\pgfsetmiterjoin%
\definecolor{currentfill}{rgb}{0.121569,0.466667,0.705882}%
\pgfsetfillcolor{currentfill}%
\pgfsetlinewidth{0.000000pt}%
\definecolor{currentstroke}{rgb}{0.000000,0.000000,0.000000}%
\pgfsetstrokecolor{currentstroke}%
\pgfsetstrokeopacity{0.000000}%
\pgfsetdash{}{0pt}%
\pgfpathmoveto{\pgfqpoint{5.800000in}{109.418781in}}%
\pgfpathlineto{\pgfqpoint{5.800000in}{109.418781in}}%
\pgfpathlineto{\pgfqpoint{5.800000in}{109.521618in}}%
\pgfpathlineto{\pgfqpoint{5.800000in}{109.521618in}}%
\pgfpathclose%
\pgfusepath{fill}%
\end{pgfscope}%
\begin{pgfscope}%
\pgfpathrectangle{\pgfqpoint{5.800000in}{0.720000in}}{\pgfqpoint{1.400000in}{4.620000in}}%
\pgfusepath{clip}%
\pgfsetbuttcap%
\pgfsetmiterjoin%
\definecolor{currentfill}{rgb}{0.121569,0.466667,0.705882}%
\pgfsetfillcolor{currentfill}%
\pgfsetlinewidth{0.000000pt}%
\definecolor{currentstroke}{rgb}{0.000000,0.000000,0.000000}%
\pgfsetstrokecolor{currentstroke}%
\pgfsetstrokeopacity{0.000000}%
\pgfsetdash{}{0pt}%
\pgfpathmoveto{\pgfqpoint{5.800000in}{109.521618in}}%
\pgfpathlineto{\pgfqpoint{5.800000in}{109.521618in}}%
\pgfpathlineto{\pgfqpoint{5.800000in}{109.624455in}}%
\pgfpathlineto{\pgfqpoint{5.800000in}{109.624455in}}%
\pgfpathclose%
\pgfusepath{fill}%
\end{pgfscope}%
\begin{pgfscope}%
\pgfpathrectangle{\pgfqpoint{5.800000in}{0.720000in}}{\pgfqpoint{1.400000in}{4.620000in}}%
\pgfusepath{clip}%
\pgfsetbuttcap%
\pgfsetmiterjoin%
\definecolor{currentfill}{rgb}{0.121569,0.466667,0.705882}%
\pgfsetfillcolor{currentfill}%
\pgfsetlinewidth{0.000000pt}%
\definecolor{currentstroke}{rgb}{0.000000,0.000000,0.000000}%
\pgfsetstrokecolor{currentstroke}%
\pgfsetstrokeopacity{0.000000}%
\pgfsetdash{}{0pt}%
\pgfpathmoveto{\pgfqpoint{5.800000in}{109.624455in}}%
\pgfpathlineto{\pgfqpoint{5.800000in}{109.624455in}}%
\pgfpathlineto{\pgfqpoint{5.800000in}{109.727292in}}%
\pgfpathlineto{\pgfqpoint{5.800000in}{109.727292in}}%
\pgfpathclose%
\pgfusepath{fill}%
\end{pgfscope}%
\begin{pgfscope}%
\pgfpathrectangle{\pgfqpoint{5.800000in}{0.720000in}}{\pgfqpoint{1.400000in}{4.620000in}}%
\pgfusepath{clip}%
\pgfsetbuttcap%
\pgfsetmiterjoin%
\definecolor{currentfill}{rgb}{0.121569,0.466667,0.705882}%
\pgfsetfillcolor{currentfill}%
\pgfsetlinewidth{0.000000pt}%
\definecolor{currentstroke}{rgb}{0.000000,0.000000,0.000000}%
\pgfsetstrokecolor{currentstroke}%
\pgfsetstrokeopacity{0.000000}%
\pgfsetdash{}{0pt}%
\pgfpathmoveto{\pgfqpoint{5.800000in}{109.727292in}}%
\pgfpathlineto{\pgfqpoint{5.800000in}{109.727292in}}%
\pgfpathlineto{\pgfqpoint{5.800000in}{109.830129in}}%
\pgfpathlineto{\pgfqpoint{5.800000in}{109.830129in}}%
\pgfpathclose%
\pgfusepath{fill}%
\end{pgfscope}%
\begin{pgfscope}%
\pgfpathrectangle{\pgfqpoint{5.800000in}{0.720000in}}{\pgfqpoint{1.400000in}{4.620000in}}%
\pgfusepath{clip}%
\pgfsetbuttcap%
\pgfsetmiterjoin%
\definecolor{currentfill}{rgb}{0.121569,0.466667,0.705882}%
\pgfsetfillcolor{currentfill}%
\pgfsetlinewidth{0.000000pt}%
\definecolor{currentstroke}{rgb}{0.000000,0.000000,0.000000}%
\pgfsetstrokecolor{currentstroke}%
\pgfsetstrokeopacity{0.000000}%
\pgfsetdash{}{0pt}%
\pgfpathmoveto{\pgfqpoint{5.800000in}{109.830129in}}%
\pgfpathlineto{\pgfqpoint{5.800000in}{109.830129in}}%
\pgfpathlineto{\pgfqpoint{5.800000in}{109.932966in}}%
\pgfpathlineto{\pgfqpoint{5.800000in}{109.932966in}}%
\pgfpathclose%
\pgfusepath{fill}%
\end{pgfscope}%
\begin{pgfscope}%
\pgfpathrectangle{\pgfqpoint{5.800000in}{0.720000in}}{\pgfqpoint{1.400000in}{4.620000in}}%
\pgfusepath{clip}%
\pgfsetbuttcap%
\pgfsetmiterjoin%
\definecolor{currentfill}{rgb}{0.121569,0.466667,0.705882}%
\pgfsetfillcolor{currentfill}%
\pgfsetlinewidth{0.000000pt}%
\definecolor{currentstroke}{rgb}{0.000000,0.000000,0.000000}%
\pgfsetstrokecolor{currentstroke}%
\pgfsetstrokeopacity{0.000000}%
\pgfsetdash{}{0pt}%
\pgfpathmoveto{\pgfqpoint{5.800000in}{109.932966in}}%
\pgfpathlineto{\pgfqpoint{5.800000in}{109.932966in}}%
\pgfpathlineto{\pgfqpoint{5.800000in}{110.035803in}}%
\pgfpathlineto{\pgfqpoint{5.800000in}{110.035803in}}%
\pgfpathclose%
\pgfusepath{fill}%
\end{pgfscope}%
\begin{pgfscope}%
\pgfpathrectangle{\pgfqpoint{5.800000in}{0.720000in}}{\pgfqpoint{1.400000in}{4.620000in}}%
\pgfusepath{clip}%
\pgfsetbuttcap%
\pgfsetmiterjoin%
\definecolor{currentfill}{rgb}{0.121569,0.466667,0.705882}%
\pgfsetfillcolor{currentfill}%
\pgfsetlinewidth{0.000000pt}%
\definecolor{currentstroke}{rgb}{0.000000,0.000000,0.000000}%
\pgfsetstrokecolor{currentstroke}%
\pgfsetstrokeopacity{0.000000}%
\pgfsetdash{}{0pt}%
\pgfpathmoveto{\pgfqpoint{5.800000in}{110.035803in}}%
\pgfpathlineto{\pgfqpoint{5.800000in}{110.035803in}}%
\pgfpathlineto{\pgfqpoint{5.800000in}{110.138640in}}%
\pgfpathlineto{\pgfqpoint{5.800000in}{110.138640in}}%
\pgfpathclose%
\pgfusepath{fill}%
\end{pgfscope}%
\begin{pgfscope}%
\pgfpathrectangle{\pgfqpoint{5.800000in}{0.720000in}}{\pgfqpoint{1.400000in}{4.620000in}}%
\pgfusepath{clip}%
\pgfsetbuttcap%
\pgfsetmiterjoin%
\definecolor{currentfill}{rgb}{0.121569,0.466667,0.705882}%
\pgfsetfillcolor{currentfill}%
\pgfsetlinewidth{0.000000pt}%
\definecolor{currentstroke}{rgb}{0.000000,0.000000,0.000000}%
\pgfsetstrokecolor{currentstroke}%
\pgfsetstrokeopacity{0.000000}%
\pgfsetdash{}{0pt}%
\pgfpathmoveto{\pgfqpoint{5.800000in}{110.138640in}}%
\pgfpathlineto{\pgfqpoint{5.800000in}{110.138640in}}%
\pgfpathlineto{\pgfqpoint{5.800000in}{110.241477in}}%
\pgfpathlineto{\pgfqpoint{5.800000in}{110.241477in}}%
\pgfpathclose%
\pgfusepath{fill}%
\end{pgfscope}%
\begin{pgfscope}%
\pgfpathrectangle{\pgfqpoint{5.800000in}{0.720000in}}{\pgfqpoint{1.400000in}{4.620000in}}%
\pgfusepath{clip}%
\pgfsetbuttcap%
\pgfsetmiterjoin%
\definecolor{currentfill}{rgb}{0.121569,0.466667,0.705882}%
\pgfsetfillcolor{currentfill}%
\pgfsetlinewidth{0.000000pt}%
\definecolor{currentstroke}{rgb}{0.000000,0.000000,0.000000}%
\pgfsetstrokecolor{currentstroke}%
\pgfsetstrokeopacity{0.000000}%
\pgfsetdash{}{0pt}%
\pgfpathmoveto{\pgfqpoint{5.800000in}{110.241477in}}%
\pgfpathlineto{\pgfqpoint{5.800000in}{110.241477in}}%
\pgfpathlineto{\pgfqpoint{5.800000in}{110.344314in}}%
\pgfpathlineto{\pgfqpoint{5.800000in}{110.344314in}}%
\pgfpathclose%
\pgfusepath{fill}%
\end{pgfscope}%
\begin{pgfscope}%
\pgfpathrectangle{\pgfqpoint{5.800000in}{0.720000in}}{\pgfqpoint{1.400000in}{4.620000in}}%
\pgfusepath{clip}%
\pgfsetbuttcap%
\pgfsetmiterjoin%
\definecolor{currentfill}{rgb}{0.121569,0.466667,0.705882}%
\pgfsetfillcolor{currentfill}%
\pgfsetlinewidth{0.000000pt}%
\definecolor{currentstroke}{rgb}{0.000000,0.000000,0.000000}%
\pgfsetstrokecolor{currentstroke}%
\pgfsetstrokeopacity{0.000000}%
\pgfsetdash{}{0pt}%
\pgfpathmoveto{\pgfqpoint{5.800000in}{110.344314in}}%
\pgfpathlineto{\pgfqpoint{5.800000in}{110.344314in}}%
\pgfpathlineto{\pgfqpoint{5.800000in}{110.447151in}}%
\pgfpathlineto{\pgfqpoint{5.800000in}{110.447151in}}%
\pgfpathclose%
\pgfusepath{fill}%
\end{pgfscope}%
\begin{pgfscope}%
\pgfpathrectangle{\pgfqpoint{5.800000in}{0.720000in}}{\pgfqpoint{1.400000in}{4.620000in}}%
\pgfusepath{clip}%
\pgfsetbuttcap%
\pgfsetmiterjoin%
\definecolor{currentfill}{rgb}{0.121569,0.466667,0.705882}%
\pgfsetfillcolor{currentfill}%
\pgfsetlinewidth{0.000000pt}%
\definecolor{currentstroke}{rgb}{0.000000,0.000000,0.000000}%
\pgfsetstrokecolor{currentstroke}%
\pgfsetstrokeopacity{0.000000}%
\pgfsetdash{}{0pt}%
\pgfpathmoveto{\pgfqpoint{5.800000in}{110.447151in}}%
\pgfpathlineto{\pgfqpoint{5.800000in}{110.447151in}}%
\pgfpathlineto{\pgfqpoint{5.800000in}{110.549988in}}%
\pgfpathlineto{\pgfqpoint{5.800000in}{110.549988in}}%
\pgfpathclose%
\pgfusepath{fill}%
\end{pgfscope}%
\begin{pgfscope}%
\pgfpathrectangle{\pgfqpoint{5.800000in}{0.720000in}}{\pgfqpoint{1.400000in}{4.620000in}}%
\pgfusepath{clip}%
\pgfsetbuttcap%
\pgfsetmiterjoin%
\definecolor{currentfill}{rgb}{0.121569,0.466667,0.705882}%
\pgfsetfillcolor{currentfill}%
\pgfsetlinewidth{0.000000pt}%
\definecolor{currentstroke}{rgb}{0.000000,0.000000,0.000000}%
\pgfsetstrokecolor{currentstroke}%
\pgfsetstrokeopacity{0.000000}%
\pgfsetdash{}{0pt}%
\pgfpathmoveto{\pgfqpoint{5.800000in}{110.549988in}}%
\pgfpathlineto{\pgfqpoint{5.800000in}{110.549988in}}%
\pgfpathlineto{\pgfqpoint{5.800000in}{110.652825in}}%
\pgfpathlineto{\pgfqpoint{5.800000in}{110.652825in}}%
\pgfpathclose%
\pgfusepath{fill}%
\end{pgfscope}%
\begin{pgfscope}%
\pgfpathrectangle{\pgfqpoint{5.800000in}{0.720000in}}{\pgfqpoint{1.400000in}{4.620000in}}%
\pgfusepath{clip}%
\pgfsetbuttcap%
\pgfsetmiterjoin%
\definecolor{currentfill}{rgb}{0.121569,0.466667,0.705882}%
\pgfsetfillcolor{currentfill}%
\pgfsetlinewidth{0.000000pt}%
\definecolor{currentstroke}{rgb}{0.000000,0.000000,0.000000}%
\pgfsetstrokecolor{currentstroke}%
\pgfsetstrokeopacity{0.000000}%
\pgfsetdash{}{0pt}%
\pgfpathmoveto{\pgfqpoint{5.800000in}{110.652825in}}%
\pgfpathlineto{\pgfqpoint{5.800000in}{110.652825in}}%
\pgfpathlineto{\pgfqpoint{5.800000in}{110.755663in}}%
\pgfpathlineto{\pgfqpoint{5.800000in}{110.755663in}}%
\pgfpathclose%
\pgfusepath{fill}%
\end{pgfscope}%
\begin{pgfscope}%
\pgfpathrectangle{\pgfqpoint{5.800000in}{0.720000in}}{\pgfqpoint{1.400000in}{4.620000in}}%
\pgfusepath{clip}%
\pgfsetbuttcap%
\pgfsetmiterjoin%
\definecolor{currentfill}{rgb}{0.121569,0.466667,0.705882}%
\pgfsetfillcolor{currentfill}%
\pgfsetlinewidth{0.000000pt}%
\definecolor{currentstroke}{rgb}{0.000000,0.000000,0.000000}%
\pgfsetstrokecolor{currentstroke}%
\pgfsetstrokeopacity{0.000000}%
\pgfsetdash{}{0pt}%
\pgfpathmoveto{\pgfqpoint{5.800000in}{110.755663in}}%
\pgfpathlineto{\pgfqpoint{5.800000in}{110.755663in}}%
\pgfpathlineto{\pgfqpoint{5.800000in}{110.858500in}}%
\pgfpathlineto{\pgfqpoint{5.800000in}{110.858500in}}%
\pgfpathclose%
\pgfusepath{fill}%
\end{pgfscope}%
\begin{pgfscope}%
\pgfpathrectangle{\pgfqpoint{5.800000in}{0.720000in}}{\pgfqpoint{1.400000in}{4.620000in}}%
\pgfusepath{clip}%
\pgfsetbuttcap%
\pgfsetmiterjoin%
\definecolor{currentfill}{rgb}{0.121569,0.466667,0.705882}%
\pgfsetfillcolor{currentfill}%
\pgfsetlinewidth{0.000000pt}%
\definecolor{currentstroke}{rgb}{0.000000,0.000000,0.000000}%
\pgfsetstrokecolor{currentstroke}%
\pgfsetstrokeopacity{0.000000}%
\pgfsetdash{}{0pt}%
\pgfpathmoveto{\pgfqpoint{5.800000in}{110.858500in}}%
\pgfpathlineto{\pgfqpoint{5.800000in}{110.858500in}}%
\pgfpathlineto{\pgfqpoint{5.800000in}{110.961337in}}%
\pgfpathlineto{\pgfqpoint{5.800000in}{110.961337in}}%
\pgfpathclose%
\pgfusepath{fill}%
\end{pgfscope}%
\begin{pgfscope}%
\pgfpathrectangle{\pgfqpoint{5.800000in}{0.720000in}}{\pgfqpoint{1.400000in}{4.620000in}}%
\pgfusepath{clip}%
\pgfsetbuttcap%
\pgfsetmiterjoin%
\definecolor{currentfill}{rgb}{0.121569,0.466667,0.705882}%
\pgfsetfillcolor{currentfill}%
\pgfsetlinewidth{0.000000pt}%
\definecolor{currentstroke}{rgb}{0.000000,0.000000,0.000000}%
\pgfsetstrokecolor{currentstroke}%
\pgfsetstrokeopacity{0.000000}%
\pgfsetdash{}{0pt}%
\pgfpathmoveto{\pgfqpoint{5.800000in}{110.961337in}}%
\pgfpathlineto{\pgfqpoint{5.800000in}{110.961337in}}%
\pgfpathlineto{\pgfqpoint{5.800000in}{111.064174in}}%
\pgfpathlineto{\pgfqpoint{5.800000in}{111.064174in}}%
\pgfpathclose%
\pgfusepath{fill}%
\end{pgfscope}%
\begin{pgfscope}%
\pgfpathrectangle{\pgfqpoint{5.800000in}{0.720000in}}{\pgfqpoint{1.400000in}{4.620000in}}%
\pgfusepath{clip}%
\pgfsetbuttcap%
\pgfsetmiterjoin%
\definecolor{currentfill}{rgb}{0.121569,0.466667,0.705882}%
\pgfsetfillcolor{currentfill}%
\pgfsetlinewidth{0.000000pt}%
\definecolor{currentstroke}{rgb}{0.000000,0.000000,0.000000}%
\pgfsetstrokecolor{currentstroke}%
\pgfsetstrokeopacity{0.000000}%
\pgfsetdash{}{0pt}%
\pgfpathmoveto{\pgfqpoint{5.800000in}{111.064174in}}%
\pgfpathlineto{\pgfqpoint{5.800000in}{111.064174in}}%
\pgfpathlineto{\pgfqpoint{5.800000in}{111.167011in}}%
\pgfpathlineto{\pgfqpoint{5.800000in}{111.167011in}}%
\pgfpathclose%
\pgfusepath{fill}%
\end{pgfscope}%
\begin{pgfscope}%
\pgfpathrectangle{\pgfqpoint{5.800000in}{0.720000in}}{\pgfqpoint{1.400000in}{4.620000in}}%
\pgfusepath{clip}%
\pgfsetbuttcap%
\pgfsetmiterjoin%
\definecolor{currentfill}{rgb}{0.121569,0.466667,0.705882}%
\pgfsetfillcolor{currentfill}%
\pgfsetlinewidth{0.000000pt}%
\definecolor{currentstroke}{rgb}{0.000000,0.000000,0.000000}%
\pgfsetstrokecolor{currentstroke}%
\pgfsetstrokeopacity{0.000000}%
\pgfsetdash{}{0pt}%
\pgfpathmoveto{\pgfqpoint{5.800000in}{111.167011in}}%
\pgfpathlineto{\pgfqpoint{5.800000in}{111.167011in}}%
\pgfpathlineto{\pgfqpoint{5.800000in}{111.269848in}}%
\pgfpathlineto{\pgfqpoint{5.800000in}{111.269848in}}%
\pgfpathclose%
\pgfusepath{fill}%
\end{pgfscope}%
\begin{pgfscope}%
\pgfpathrectangle{\pgfqpoint{5.800000in}{0.720000in}}{\pgfqpoint{1.400000in}{4.620000in}}%
\pgfusepath{clip}%
\pgfsetbuttcap%
\pgfsetmiterjoin%
\definecolor{currentfill}{rgb}{0.121569,0.466667,0.705882}%
\pgfsetfillcolor{currentfill}%
\pgfsetlinewidth{0.000000pt}%
\definecolor{currentstroke}{rgb}{0.000000,0.000000,0.000000}%
\pgfsetstrokecolor{currentstroke}%
\pgfsetstrokeopacity{0.000000}%
\pgfsetdash{}{0pt}%
\pgfpathmoveto{\pgfqpoint{5.800000in}{111.269848in}}%
\pgfpathlineto{\pgfqpoint{5.800000in}{111.269848in}}%
\pgfpathlineto{\pgfqpoint{5.800000in}{111.372685in}}%
\pgfpathlineto{\pgfqpoint{5.800000in}{111.372685in}}%
\pgfpathclose%
\pgfusepath{fill}%
\end{pgfscope}%
\begin{pgfscope}%
\pgfpathrectangle{\pgfqpoint{5.800000in}{0.720000in}}{\pgfqpoint{1.400000in}{4.620000in}}%
\pgfusepath{clip}%
\pgfsetbuttcap%
\pgfsetmiterjoin%
\definecolor{currentfill}{rgb}{0.121569,0.466667,0.705882}%
\pgfsetfillcolor{currentfill}%
\pgfsetlinewidth{0.000000pt}%
\definecolor{currentstroke}{rgb}{0.000000,0.000000,0.000000}%
\pgfsetstrokecolor{currentstroke}%
\pgfsetstrokeopacity{0.000000}%
\pgfsetdash{}{0pt}%
\pgfpathmoveto{\pgfqpoint{5.800000in}{111.372685in}}%
\pgfpathlineto{\pgfqpoint{5.800000in}{111.372685in}}%
\pgfpathlineto{\pgfqpoint{5.800000in}{111.475522in}}%
\pgfpathlineto{\pgfqpoint{5.800000in}{111.475522in}}%
\pgfpathclose%
\pgfusepath{fill}%
\end{pgfscope}%
\begin{pgfscope}%
\pgfpathrectangle{\pgfqpoint{5.800000in}{0.720000in}}{\pgfqpoint{1.400000in}{4.620000in}}%
\pgfusepath{clip}%
\pgfsetbuttcap%
\pgfsetmiterjoin%
\definecolor{currentfill}{rgb}{0.121569,0.466667,0.705882}%
\pgfsetfillcolor{currentfill}%
\pgfsetlinewidth{0.000000pt}%
\definecolor{currentstroke}{rgb}{0.000000,0.000000,0.000000}%
\pgfsetstrokecolor{currentstroke}%
\pgfsetstrokeopacity{0.000000}%
\pgfsetdash{}{0pt}%
\pgfpathmoveto{\pgfqpoint{5.800000in}{111.475522in}}%
\pgfpathlineto{\pgfqpoint{5.800000in}{111.475522in}}%
\pgfpathlineto{\pgfqpoint{5.800000in}{111.578359in}}%
\pgfpathlineto{\pgfqpoint{5.800000in}{111.578359in}}%
\pgfpathclose%
\pgfusepath{fill}%
\end{pgfscope}%
\begin{pgfscope}%
\pgfpathrectangle{\pgfqpoint{5.800000in}{0.720000in}}{\pgfqpoint{1.400000in}{4.620000in}}%
\pgfusepath{clip}%
\pgfsetbuttcap%
\pgfsetmiterjoin%
\definecolor{currentfill}{rgb}{0.121569,0.466667,0.705882}%
\pgfsetfillcolor{currentfill}%
\pgfsetlinewidth{0.000000pt}%
\definecolor{currentstroke}{rgb}{0.000000,0.000000,0.000000}%
\pgfsetstrokecolor{currentstroke}%
\pgfsetstrokeopacity{0.000000}%
\pgfsetdash{}{0pt}%
\pgfpathmoveto{\pgfqpoint{5.800000in}{111.578359in}}%
\pgfpathlineto{\pgfqpoint{5.800000in}{111.578359in}}%
\pgfpathlineto{\pgfqpoint{5.800000in}{111.681196in}}%
\pgfpathlineto{\pgfqpoint{5.800000in}{111.681196in}}%
\pgfpathclose%
\pgfusepath{fill}%
\end{pgfscope}%
\begin{pgfscope}%
\pgfpathrectangle{\pgfqpoint{5.800000in}{0.720000in}}{\pgfqpoint{1.400000in}{4.620000in}}%
\pgfusepath{clip}%
\pgfsetbuttcap%
\pgfsetmiterjoin%
\definecolor{currentfill}{rgb}{0.121569,0.466667,0.705882}%
\pgfsetfillcolor{currentfill}%
\pgfsetlinewidth{0.000000pt}%
\definecolor{currentstroke}{rgb}{0.000000,0.000000,0.000000}%
\pgfsetstrokecolor{currentstroke}%
\pgfsetstrokeopacity{0.000000}%
\pgfsetdash{}{0pt}%
\pgfpathmoveto{\pgfqpoint{5.800000in}{111.681196in}}%
\pgfpathlineto{\pgfqpoint{5.800000in}{111.681196in}}%
\pgfpathlineto{\pgfqpoint{5.800000in}{111.784033in}}%
\pgfpathlineto{\pgfqpoint{5.800000in}{111.784033in}}%
\pgfpathclose%
\pgfusepath{fill}%
\end{pgfscope}%
\begin{pgfscope}%
\pgfpathrectangle{\pgfqpoint{5.800000in}{0.720000in}}{\pgfqpoint{1.400000in}{4.620000in}}%
\pgfusepath{clip}%
\pgfsetbuttcap%
\pgfsetmiterjoin%
\definecolor{currentfill}{rgb}{0.121569,0.466667,0.705882}%
\pgfsetfillcolor{currentfill}%
\pgfsetlinewidth{0.000000pt}%
\definecolor{currentstroke}{rgb}{0.000000,0.000000,0.000000}%
\pgfsetstrokecolor{currentstroke}%
\pgfsetstrokeopacity{0.000000}%
\pgfsetdash{}{0pt}%
\pgfpathmoveto{\pgfqpoint{5.800000in}{111.784033in}}%
\pgfpathlineto{\pgfqpoint{5.800000in}{111.784033in}}%
\pgfpathlineto{\pgfqpoint{5.800000in}{111.886870in}}%
\pgfpathlineto{\pgfqpoint{5.800000in}{111.886870in}}%
\pgfpathclose%
\pgfusepath{fill}%
\end{pgfscope}%
\begin{pgfscope}%
\pgfpathrectangle{\pgfqpoint{5.800000in}{0.720000in}}{\pgfqpoint{1.400000in}{4.620000in}}%
\pgfusepath{clip}%
\pgfsetbuttcap%
\pgfsetmiterjoin%
\definecolor{currentfill}{rgb}{0.121569,0.466667,0.705882}%
\pgfsetfillcolor{currentfill}%
\pgfsetlinewidth{0.000000pt}%
\definecolor{currentstroke}{rgb}{0.000000,0.000000,0.000000}%
\pgfsetstrokecolor{currentstroke}%
\pgfsetstrokeopacity{0.000000}%
\pgfsetdash{}{0pt}%
\pgfpathmoveto{\pgfqpoint{5.800000in}{111.886870in}}%
\pgfpathlineto{\pgfqpoint{5.800000in}{111.886870in}}%
\pgfpathlineto{\pgfqpoint{5.800000in}{111.989707in}}%
\pgfpathlineto{\pgfqpoint{5.800000in}{111.989707in}}%
\pgfpathclose%
\pgfusepath{fill}%
\end{pgfscope}%
\begin{pgfscope}%
\pgfpathrectangle{\pgfqpoint{5.800000in}{0.720000in}}{\pgfqpoint{1.400000in}{4.620000in}}%
\pgfusepath{clip}%
\pgfsetbuttcap%
\pgfsetmiterjoin%
\definecolor{currentfill}{rgb}{0.121569,0.466667,0.705882}%
\pgfsetfillcolor{currentfill}%
\pgfsetlinewidth{0.000000pt}%
\definecolor{currentstroke}{rgb}{0.000000,0.000000,0.000000}%
\pgfsetstrokecolor{currentstroke}%
\pgfsetstrokeopacity{0.000000}%
\pgfsetdash{}{0pt}%
\pgfpathmoveto{\pgfqpoint{5.800000in}{111.989707in}}%
\pgfpathlineto{\pgfqpoint{5.800000in}{111.989707in}}%
\pgfpathlineto{\pgfqpoint{5.800000in}{112.092544in}}%
\pgfpathlineto{\pgfqpoint{5.800000in}{112.092544in}}%
\pgfpathclose%
\pgfusepath{fill}%
\end{pgfscope}%
\begin{pgfscope}%
\pgfpathrectangle{\pgfqpoint{5.800000in}{0.720000in}}{\pgfqpoint{1.400000in}{4.620000in}}%
\pgfusepath{clip}%
\pgfsetbuttcap%
\pgfsetmiterjoin%
\definecolor{currentfill}{rgb}{0.121569,0.466667,0.705882}%
\pgfsetfillcolor{currentfill}%
\pgfsetlinewidth{0.000000pt}%
\definecolor{currentstroke}{rgb}{0.000000,0.000000,0.000000}%
\pgfsetstrokecolor{currentstroke}%
\pgfsetstrokeopacity{0.000000}%
\pgfsetdash{}{0pt}%
\pgfpathmoveto{\pgfqpoint{5.800000in}{112.092544in}}%
\pgfpathlineto{\pgfqpoint{5.800000in}{112.092544in}}%
\pgfpathlineto{\pgfqpoint{5.800000in}{112.195381in}}%
\pgfpathlineto{\pgfqpoint{5.800000in}{112.195381in}}%
\pgfpathclose%
\pgfusepath{fill}%
\end{pgfscope}%
\begin{pgfscope}%
\pgfpathrectangle{\pgfqpoint{5.800000in}{0.720000in}}{\pgfqpoint{1.400000in}{4.620000in}}%
\pgfusepath{clip}%
\pgfsetbuttcap%
\pgfsetmiterjoin%
\definecolor{currentfill}{rgb}{0.121569,0.466667,0.705882}%
\pgfsetfillcolor{currentfill}%
\pgfsetlinewidth{0.000000pt}%
\definecolor{currentstroke}{rgb}{0.000000,0.000000,0.000000}%
\pgfsetstrokecolor{currentstroke}%
\pgfsetstrokeopacity{0.000000}%
\pgfsetdash{}{0pt}%
\pgfpathmoveto{\pgfqpoint{5.800000in}{112.195381in}}%
\pgfpathlineto{\pgfqpoint{5.800000in}{112.195381in}}%
\pgfpathlineto{\pgfqpoint{5.800000in}{112.298219in}}%
\pgfpathlineto{\pgfqpoint{5.800000in}{112.298219in}}%
\pgfpathclose%
\pgfusepath{fill}%
\end{pgfscope}%
\begin{pgfscope}%
\pgfpathrectangle{\pgfqpoint{5.800000in}{0.720000in}}{\pgfqpoint{1.400000in}{4.620000in}}%
\pgfusepath{clip}%
\pgfsetbuttcap%
\pgfsetmiterjoin%
\definecolor{currentfill}{rgb}{0.121569,0.466667,0.705882}%
\pgfsetfillcolor{currentfill}%
\pgfsetlinewidth{0.000000pt}%
\definecolor{currentstroke}{rgb}{0.000000,0.000000,0.000000}%
\pgfsetstrokecolor{currentstroke}%
\pgfsetstrokeopacity{0.000000}%
\pgfsetdash{}{0pt}%
\pgfpathmoveto{\pgfqpoint{5.800000in}{112.298219in}}%
\pgfpathlineto{\pgfqpoint{5.800000in}{112.298219in}}%
\pgfpathlineto{\pgfqpoint{5.800000in}{112.401056in}}%
\pgfpathlineto{\pgfqpoint{5.800000in}{112.401056in}}%
\pgfpathclose%
\pgfusepath{fill}%
\end{pgfscope}%
\begin{pgfscope}%
\pgfpathrectangle{\pgfqpoint{5.800000in}{0.720000in}}{\pgfqpoint{1.400000in}{4.620000in}}%
\pgfusepath{clip}%
\pgfsetbuttcap%
\pgfsetmiterjoin%
\definecolor{currentfill}{rgb}{0.121569,0.466667,0.705882}%
\pgfsetfillcolor{currentfill}%
\pgfsetlinewidth{0.000000pt}%
\definecolor{currentstroke}{rgb}{0.000000,0.000000,0.000000}%
\pgfsetstrokecolor{currentstroke}%
\pgfsetstrokeopacity{0.000000}%
\pgfsetdash{}{0pt}%
\pgfpathmoveto{\pgfqpoint{5.800000in}{112.401056in}}%
\pgfpathlineto{\pgfqpoint{5.800000in}{112.401056in}}%
\pgfpathlineto{\pgfqpoint{5.800000in}{112.503893in}}%
\pgfpathlineto{\pgfqpoint{5.800000in}{112.503893in}}%
\pgfpathclose%
\pgfusepath{fill}%
\end{pgfscope}%
\begin{pgfscope}%
\pgfpathrectangle{\pgfqpoint{5.800000in}{0.720000in}}{\pgfqpoint{1.400000in}{4.620000in}}%
\pgfusepath{clip}%
\pgfsetbuttcap%
\pgfsetmiterjoin%
\definecolor{currentfill}{rgb}{0.121569,0.466667,0.705882}%
\pgfsetfillcolor{currentfill}%
\pgfsetlinewidth{0.000000pt}%
\definecolor{currentstroke}{rgb}{0.000000,0.000000,0.000000}%
\pgfsetstrokecolor{currentstroke}%
\pgfsetstrokeopacity{0.000000}%
\pgfsetdash{}{0pt}%
\pgfpathmoveto{\pgfqpoint{5.800000in}{112.503893in}}%
\pgfpathlineto{\pgfqpoint{5.800000in}{112.503893in}}%
\pgfpathlineto{\pgfqpoint{5.800000in}{112.606730in}}%
\pgfpathlineto{\pgfqpoint{5.800000in}{112.606730in}}%
\pgfpathclose%
\pgfusepath{fill}%
\end{pgfscope}%
\begin{pgfscope}%
\pgfpathrectangle{\pgfqpoint{5.800000in}{0.720000in}}{\pgfqpoint{1.400000in}{4.620000in}}%
\pgfusepath{clip}%
\pgfsetbuttcap%
\pgfsetmiterjoin%
\definecolor{currentfill}{rgb}{0.121569,0.466667,0.705882}%
\pgfsetfillcolor{currentfill}%
\pgfsetlinewidth{0.000000pt}%
\definecolor{currentstroke}{rgb}{0.000000,0.000000,0.000000}%
\pgfsetstrokecolor{currentstroke}%
\pgfsetstrokeopacity{0.000000}%
\pgfsetdash{}{0pt}%
\pgfpathmoveto{\pgfqpoint{5.800000in}{112.606730in}}%
\pgfpathlineto{\pgfqpoint{5.800000in}{112.606730in}}%
\pgfpathlineto{\pgfqpoint{5.800000in}{112.709567in}}%
\pgfpathlineto{\pgfqpoint{5.800000in}{112.709567in}}%
\pgfpathclose%
\pgfusepath{fill}%
\end{pgfscope}%
\begin{pgfscope}%
\pgfpathrectangle{\pgfqpoint{5.800000in}{0.720000in}}{\pgfqpoint{1.400000in}{4.620000in}}%
\pgfusepath{clip}%
\pgfsetbuttcap%
\pgfsetmiterjoin%
\definecolor{currentfill}{rgb}{0.121569,0.466667,0.705882}%
\pgfsetfillcolor{currentfill}%
\pgfsetlinewidth{0.000000pt}%
\definecolor{currentstroke}{rgb}{0.000000,0.000000,0.000000}%
\pgfsetstrokecolor{currentstroke}%
\pgfsetstrokeopacity{0.000000}%
\pgfsetdash{}{0pt}%
\pgfpathmoveto{\pgfqpoint{5.800000in}{112.709567in}}%
\pgfpathlineto{\pgfqpoint{5.800000in}{112.709567in}}%
\pgfpathlineto{\pgfqpoint{5.800000in}{112.812404in}}%
\pgfpathlineto{\pgfqpoint{5.800000in}{112.812404in}}%
\pgfpathclose%
\pgfusepath{fill}%
\end{pgfscope}%
\begin{pgfscope}%
\pgfpathrectangle{\pgfqpoint{5.800000in}{0.720000in}}{\pgfqpoint{1.400000in}{4.620000in}}%
\pgfusepath{clip}%
\pgfsetbuttcap%
\pgfsetmiterjoin%
\definecolor{currentfill}{rgb}{0.121569,0.466667,0.705882}%
\pgfsetfillcolor{currentfill}%
\pgfsetlinewidth{0.000000pt}%
\definecolor{currentstroke}{rgb}{0.000000,0.000000,0.000000}%
\pgfsetstrokecolor{currentstroke}%
\pgfsetstrokeopacity{0.000000}%
\pgfsetdash{}{0pt}%
\pgfpathmoveto{\pgfqpoint{5.800000in}{112.812404in}}%
\pgfpathlineto{\pgfqpoint{5.800000in}{112.812404in}}%
\pgfpathlineto{\pgfqpoint{5.800000in}{112.915241in}}%
\pgfpathlineto{\pgfqpoint{5.800000in}{112.915241in}}%
\pgfpathclose%
\pgfusepath{fill}%
\end{pgfscope}%
\begin{pgfscope}%
\pgfpathrectangle{\pgfqpoint{5.800000in}{0.720000in}}{\pgfqpoint{1.400000in}{4.620000in}}%
\pgfusepath{clip}%
\pgfsetbuttcap%
\pgfsetmiterjoin%
\definecolor{currentfill}{rgb}{0.121569,0.466667,0.705882}%
\pgfsetfillcolor{currentfill}%
\pgfsetlinewidth{0.000000pt}%
\definecolor{currentstroke}{rgb}{0.000000,0.000000,0.000000}%
\pgfsetstrokecolor{currentstroke}%
\pgfsetstrokeopacity{0.000000}%
\pgfsetdash{}{0pt}%
\pgfpathmoveto{\pgfqpoint{5.800000in}{112.915241in}}%
\pgfpathlineto{\pgfqpoint{5.800000in}{112.915241in}}%
\pgfpathlineto{\pgfqpoint{5.800000in}{113.018078in}}%
\pgfpathlineto{\pgfqpoint{5.800000in}{113.018078in}}%
\pgfpathclose%
\pgfusepath{fill}%
\end{pgfscope}%
\begin{pgfscope}%
\pgfpathrectangle{\pgfqpoint{5.800000in}{0.720000in}}{\pgfqpoint{1.400000in}{4.620000in}}%
\pgfusepath{clip}%
\pgfsetbuttcap%
\pgfsetmiterjoin%
\definecolor{currentfill}{rgb}{0.121569,0.466667,0.705882}%
\pgfsetfillcolor{currentfill}%
\pgfsetlinewidth{0.000000pt}%
\definecolor{currentstroke}{rgb}{0.000000,0.000000,0.000000}%
\pgfsetstrokecolor{currentstroke}%
\pgfsetstrokeopacity{0.000000}%
\pgfsetdash{}{0pt}%
\pgfpathmoveto{\pgfqpoint{5.800000in}{113.018078in}}%
\pgfpathlineto{\pgfqpoint{5.800000in}{113.018078in}}%
\pgfpathlineto{\pgfqpoint{5.800000in}{113.120915in}}%
\pgfpathlineto{\pgfqpoint{5.800000in}{113.120915in}}%
\pgfpathclose%
\pgfusepath{fill}%
\end{pgfscope}%
\begin{pgfscope}%
\pgfpathrectangle{\pgfqpoint{5.800000in}{0.720000in}}{\pgfqpoint{1.400000in}{4.620000in}}%
\pgfusepath{clip}%
\pgfsetbuttcap%
\pgfsetmiterjoin%
\definecolor{currentfill}{rgb}{0.121569,0.466667,0.705882}%
\pgfsetfillcolor{currentfill}%
\pgfsetlinewidth{0.000000pt}%
\definecolor{currentstroke}{rgb}{0.000000,0.000000,0.000000}%
\pgfsetstrokecolor{currentstroke}%
\pgfsetstrokeopacity{0.000000}%
\pgfsetdash{}{0pt}%
\pgfpathmoveto{\pgfqpoint{5.800000in}{113.120915in}}%
\pgfpathlineto{\pgfqpoint{5.800000in}{113.120915in}}%
\pgfpathlineto{\pgfqpoint{5.800000in}{113.223752in}}%
\pgfpathlineto{\pgfqpoint{5.800000in}{113.223752in}}%
\pgfpathclose%
\pgfusepath{fill}%
\end{pgfscope}%
\begin{pgfscope}%
\pgfpathrectangle{\pgfqpoint{5.800000in}{0.720000in}}{\pgfqpoint{1.400000in}{4.620000in}}%
\pgfusepath{clip}%
\pgfsetbuttcap%
\pgfsetmiterjoin%
\definecolor{currentfill}{rgb}{0.121569,0.466667,0.705882}%
\pgfsetfillcolor{currentfill}%
\pgfsetlinewidth{0.000000pt}%
\definecolor{currentstroke}{rgb}{0.000000,0.000000,0.000000}%
\pgfsetstrokecolor{currentstroke}%
\pgfsetstrokeopacity{0.000000}%
\pgfsetdash{}{0pt}%
\pgfpathmoveto{\pgfqpoint{5.800000in}{113.223752in}}%
\pgfpathlineto{\pgfqpoint{5.800000in}{113.223752in}}%
\pgfpathlineto{\pgfqpoint{5.800000in}{113.326589in}}%
\pgfpathlineto{\pgfqpoint{5.800000in}{113.326589in}}%
\pgfpathclose%
\pgfusepath{fill}%
\end{pgfscope}%
\begin{pgfscope}%
\pgfpathrectangle{\pgfqpoint{5.800000in}{0.720000in}}{\pgfqpoint{1.400000in}{4.620000in}}%
\pgfusepath{clip}%
\pgfsetbuttcap%
\pgfsetmiterjoin%
\definecolor{currentfill}{rgb}{0.121569,0.466667,0.705882}%
\pgfsetfillcolor{currentfill}%
\pgfsetlinewidth{0.000000pt}%
\definecolor{currentstroke}{rgb}{0.000000,0.000000,0.000000}%
\pgfsetstrokecolor{currentstroke}%
\pgfsetstrokeopacity{0.000000}%
\pgfsetdash{}{0pt}%
\pgfpathmoveto{\pgfqpoint{5.800000in}{113.326589in}}%
\pgfpathlineto{\pgfqpoint{5.800000in}{113.326589in}}%
\pgfpathlineto{\pgfqpoint{5.800000in}{113.429426in}}%
\pgfpathlineto{\pgfqpoint{5.800000in}{113.429426in}}%
\pgfpathclose%
\pgfusepath{fill}%
\end{pgfscope}%
\begin{pgfscope}%
\pgfpathrectangle{\pgfqpoint{5.800000in}{0.720000in}}{\pgfqpoint{1.400000in}{4.620000in}}%
\pgfusepath{clip}%
\pgfsetbuttcap%
\pgfsetmiterjoin%
\definecolor{currentfill}{rgb}{0.121569,0.466667,0.705882}%
\pgfsetfillcolor{currentfill}%
\pgfsetlinewidth{0.000000pt}%
\definecolor{currentstroke}{rgb}{0.000000,0.000000,0.000000}%
\pgfsetstrokecolor{currentstroke}%
\pgfsetstrokeopacity{0.000000}%
\pgfsetdash{}{0pt}%
\pgfpathmoveto{\pgfqpoint{5.800000in}{113.429426in}}%
\pgfpathlineto{\pgfqpoint{5.800000in}{113.429426in}}%
\pgfpathlineto{\pgfqpoint{5.800000in}{113.532263in}}%
\pgfpathlineto{\pgfqpoint{5.800000in}{113.532263in}}%
\pgfpathclose%
\pgfusepath{fill}%
\end{pgfscope}%
\begin{pgfscope}%
\pgfpathrectangle{\pgfqpoint{5.800000in}{0.720000in}}{\pgfqpoint{1.400000in}{4.620000in}}%
\pgfusepath{clip}%
\pgfsetbuttcap%
\pgfsetmiterjoin%
\definecolor{currentfill}{rgb}{0.121569,0.466667,0.705882}%
\pgfsetfillcolor{currentfill}%
\pgfsetlinewidth{0.000000pt}%
\definecolor{currentstroke}{rgb}{0.000000,0.000000,0.000000}%
\pgfsetstrokecolor{currentstroke}%
\pgfsetstrokeopacity{0.000000}%
\pgfsetdash{}{0pt}%
\pgfpathmoveto{\pgfqpoint{5.800000in}{113.532263in}}%
\pgfpathlineto{\pgfqpoint{5.800000in}{113.532263in}}%
\pgfpathlineto{\pgfqpoint{5.800000in}{113.635100in}}%
\pgfpathlineto{\pgfqpoint{5.800000in}{113.635100in}}%
\pgfpathclose%
\pgfusepath{fill}%
\end{pgfscope}%
\begin{pgfscope}%
\pgfpathrectangle{\pgfqpoint{5.800000in}{0.720000in}}{\pgfqpoint{1.400000in}{4.620000in}}%
\pgfusepath{clip}%
\pgfsetbuttcap%
\pgfsetmiterjoin%
\definecolor{currentfill}{rgb}{0.121569,0.466667,0.705882}%
\pgfsetfillcolor{currentfill}%
\pgfsetlinewidth{0.000000pt}%
\definecolor{currentstroke}{rgb}{0.000000,0.000000,0.000000}%
\pgfsetstrokecolor{currentstroke}%
\pgfsetstrokeopacity{0.000000}%
\pgfsetdash{}{0pt}%
\pgfpathmoveto{\pgfqpoint{5.800000in}{113.635100in}}%
\pgfpathlineto{\pgfqpoint{5.800000in}{113.635100in}}%
\pgfpathlineto{\pgfqpoint{5.800000in}{113.737937in}}%
\pgfpathlineto{\pgfqpoint{5.800000in}{113.737937in}}%
\pgfpathclose%
\pgfusepath{fill}%
\end{pgfscope}%
\begin{pgfscope}%
\pgfpathrectangle{\pgfqpoint{5.800000in}{0.720000in}}{\pgfqpoint{1.400000in}{4.620000in}}%
\pgfusepath{clip}%
\pgfsetbuttcap%
\pgfsetmiterjoin%
\definecolor{currentfill}{rgb}{0.121569,0.466667,0.705882}%
\pgfsetfillcolor{currentfill}%
\pgfsetlinewidth{0.000000pt}%
\definecolor{currentstroke}{rgb}{0.000000,0.000000,0.000000}%
\pgfsetstrokecolor{currentstroke}%
\pgfsetstrokeopacity{0.000000}%
\pgfsetdash{}{0pt}%
\pgfpathmoveto{\pgfqpoint{5.800000in}{113.737937in}}%
\pgfpathlineto{\pgfqpoint{5.800000in}{113.737937in}}%
\pgfpathlineto{\pgfqpoint{5.800000in}{113.840775in}}%
\pgfpathlineto{\pgfqpoint{5.800000in}{113.840775in}}%
\pgfpathclose%
\pgfusepath{fill}%
\end{pgfscope}%
\begin{pgfscope}%
\pgfpathrectangle{\pgfqpoint{5.800000in}{0.720000in}}{\pgfqpoint{1.400000in}{4.620000in}}%
\pgfusepath{clip}%
\pgfsetbuttcap%
\pgfsetmiterjoin%
\definecolor{currentfill}{rgb}{0.121569,0.466667,0.705882}%
\pgfsetfillcolor{currentfill}%
\pgfsetlinewidth{0.000000pt}%
\definecolor{currentstroke}{rgb}{0.000000,0.000000,0.000000}%
\pgfsetstrokecolor{currentstroke}%
\pgfsetstrokeopacity{0.000000}%
\pgfsetdash{}{0pt}%
\pgfpathmoveto{\pgfqpoint{5.800000in}{113.840775in}}%
\pgfpathlineto{\pgfqpoint{5.800000in}{113.840775in}}%
\pgfpathlineto{\pgfqpoint{5.800000in}{113.943612in}}%
\pgfpathlineto{\pgfqpoint{5.800000in}{113.943612in}}%
\pgfpathclose%
\pgfusepath{fill}%
\end{pgfscope}%
\begin{pgfscope}%
\pgfpathrectangle{\pgfqpoint{5.800000in}{0.720000in}}{\pgfqpoint{1.400000in}{4.620000in}}%
\pgfusepath{clip}%
\pgfsetbuttcap%
\pgfsetmiterjoin%
\definecolor{currentfill}{rgb}{0.121569,0.466667,0.705882}%
\pgfsetfillcolor{currentfill}%
\pgfsetlinewidth{0.000000pt}%
\definecolor{currentstroke}{rgb}{0.000000,0.000000,0.000000}%
\pgfsetstrokecolor{currentstroke}%
\pgfsetstrokeopacity{0.000000}%
\pgfsetdash{}{0pt}%
\pgfpathmoveto{\pgfqpoint{5.800000in}{113.943612in}}%
\pgfpathlineto{\pgfqpoint{5.800000in}{113.943612in}}%
\pgfpathlineto{\pgfqpoint{5.800000in}{114.046449in}}%
\pgfpathlineto{\pgfqpoint{5.800000in}{114.046449in}}%
\pgfpathclose%
\pgfusepath{fill}%
\end{pgfscope}%
\begin{pgfscope}%
\pgfpathrectangle{\pgfqpoint{5.800000in}{0.720000in}}{\pgfqpoint{1.400000in}{4.620000in}}%
\pgfusepath{clip}%
\pgfsetbuttcap%
\pgfsetmiterjoin%
\definecolor{currentfill}{rgb}{0.121569,0.466667,0.705882}%
\pgfsetfillcolor{currentfill}%
\pgfsetlinewidth{0.000000pt}%
\definecolor{currentstroke}{rgb}{0.000000,0.000000,0.000000}%
\pgfsetstrokecolor{currentstroke}%
\pgfsetstrokeopacity{0.000000}%
\pgfsetdash{}{0pt}%
\pgfpathmoveto{\pgfqpoint{5.800000in}{114.046449in}}%
\pgfpathlineto{\pgfqpoint{5.800000in}{114.046449in}}%
\pgfpathlineto{\pgfqpoint{5.800000in}{114.149286in}}%
\pgfpathlineto{\pgfqpoint{5.800000in}{114.149286in}}%
\pgfpathclose%
\pgfusepath{fill}%
\end{pgfscope}%
\begin{pgfscope}%
\pgfpathrectangle{\pgfqpoint{5.800000in}{0.720000in}}{\pgfqpoint{1.400000in}{4.620000in}}%
\pgfusepath{clip}%
\pgfsetbuttcap%
\pgfsetmiterjoin%
\definecolor{currentfill}{rgb}{0.121569,0.466667,0.705882}%
\pgfsetfillcolor{currentfill}%
\pgfsetlinewidth{0.000000pt}%
\definecolor{currentstroke}{rgb}{0.000000,0.000000,0.000000}%
\pgfsetstrokecolor{currentstroke}%
\pgfsetstrokeopacity{0.000000}%
\pgfsetdash{}{0pt}%
\pgfpathmoveto{\pgfqpoint{5.800000in}{114.149286in}}%
\pgfpathlineto{\pgfqpoint{5.800000in}{114.149286in}}%
\pgfpathlineto{\pgfqpoint{5.800000in}{114.252123in}}%
\pgfpathlineto{\pgfqpoint{5.800000in}{114.252123in}}%
\pgfpathclose%
\pgfusepath{fill}%
\end{pgfscope}%
\begin{pgfscope}%
\pgfpathrectangle{\pgfqpoint{5.800000in}{0.720000in}}{\pgfqpoint{1.400000in}{4.620000in}}%
\pgfusepath{clip}%
\pgfsetbuttcap%
\pgfsetmiterjoin%
\definecolor{currentfill}{rgb}{0.121569,0.466667,0.705882}%
\pgfsetfillcolor{currentfill}%
\pgfsetlinewidth{0.000000pt}%
\definecolor{currentstroke}{rgb}{0.000000,0.000000,0.000000}%
\pgfsetstrokecolor{currentstroke}%
\pgfsetstrokeopacity{0.000000}%
\pgfsetdash{}{0pt}%
\pgfpathmoveto{\pgfqpoint{5.800000in}{114.252123in}}%
\pgfpathlineto{\pgfqpoint{5.800000in}{114.252123in}}%
\pgfpathlineto{\pgfqpoint{5.800000in}{114.354960in}}%
\pgfpathlineto{\pgfqpoint{5.800000in}{114.354960in}}%
\pgfpathclose%
\pgfusepath{fill}%
\end{pgfscope}%
\begin{pgfscope}%
\pgfpathrectangle{\pgfqpoint{5.800000in}{0.720000in}}{\pgfqpoint{1.400000in}{4.620000in}}%
\pgfusepath{clip}%
\pgfsetbuttcap%
\pgfsetmiterjoin%
\definecolor{currentfill}{rgb}{0.121569,0.466667,0.705882}%
\pgfsetfillcolor{currentfill}%
\pgfsetlinewidth{0.000000pt}%
\definecolor{currentstroke}{rgb}{0.000000,0.000000,0.000000}%
\pgfsetstrokecolor{currentstroke}%
\pgfsetstrokeopacity{0.000000}%
\pgfsetdash{}{0pt}%
\pgfpathmoveto{\pgfqpoint{5.800000in}{114.354960in}}%
\pgfpathlineto{\pgfqpoint{5.800000in}{114.354960in}}%
\pgfpathlineto{\pgfqpoint{5.800000in}{114.457797in}}%
\pgfpathlineto{\pgfqpoint{5.800000in}{114.457797in}}%
\pgfpathclose%
\pgfusepath{fill}%
\end{pgfscope}%
\begin{pgfscope}%
\pgfpathrectangle{\pgfqpoint{5.800000in}{0.720000in}}{\pgfqpoint{1.400000in}{4.620000in}}%
\pgfusepath{clip}%
\pgfsetbuttcap%
\pgfsetmiterjoin%
\definecolor{currentfill}{rgb}{0.121569,0.466667,0.705882}%
\pgfsetfillcolor{currentfill}%
\pgfsetlinewidth{0.000000pt}%
\definecolor{currentstroke}{rgb}{0.000000,0.000000,0.000000}%
\pgfsetstrokecolor{currentstroke}%
\pgfsetstrokeopacity{0.000000}%
\pgfsetdash{}{0pt}%
\pgfpathmoveto{\pgfqpoint{5.800000in}{114.457797in}}%
\pgfpathlineto{\pgfqpoint{5.800000in}{114.457797in}}%
\pgfpathlineto{\pgfqpoint{5.800000in}{114.560634in}}%
\pgfpathlineto{\pgfqpoint{5.800000in}{114.560634in}}%
\pgfpathclose%
\pgfusepath{fill}%
\end{pgfscope}%
\begin{pgfscope}%
\pgfpathrectangle{\pgfqpoint{5.800000in}{0.720000in}}{\pgfqpoint{1.400000in}{4.620000in}}%
\pgfusepath{clip}%
\pgfsetbuttcap%
\pgfsetmiterjoin%
\definecolor{currentfill}{rgb}{0.121569,0.466667,0.705882}%
\pgfsetfillcolor{currentfill}%
\pgfsetlinewidth{0.000000pt}%
\definecolor{currentstroke}{rgb}{0.000000,0.000000,0.000000}%
\pgfsetstrokecolor{currentstroke}%
\pgfsetstrokeopacity{0.000000}%
\pgfsetdash{}{0pt}%
\pgfpathmoveto{\pgfqpoint{5.800000in}{114.560634in}}%
\pgfpathlineto{\pgfqpoint{5.800000in}{114.560634in}}%
\pgfpathlineto{\pgfqpoint{5.800000in}{114.663471in}}%
\pgfpathlineto{\pgfqpoint{5.800000in}{114.663471in}}%
\pgfpathclose%
\pgfusepath{fill}%
\end{pgfscope}%
\begin{pgfscope}%
\pgfpathrectangle{\pgfqpoint{5.800000in}{0.720000in}}{\pgfqpoint{1.400000in}{4.620000in}}%
\pgfusepath{clip}%
\pgfsetbuttcap%
\pgfsetmiterjoin%
\definecolor{currentfill}{rgb}{0.121569,0.466667,0.705882}%
\pgfsetfillcolor{currentfill}%
\pgfsetlinewidth{0.000000pt}%
\definecolor{currentstroke}{rgb}{0.000000,0.000000,0.000000}%
\pgfsetstrokecolor{currentstroke}%
\pgfsetstrokeopacity{0.000000}%
\pgfsetdash{}{0pt}%
\pgfpathmoveto{\pgfqpoint{5.800000in}{114.663471in}}%
\pgfpathlineto{\pgfqpoint{5.800000in}{114.663471in}}%
\pgfpathlineto{\pgfqpoint{5.800000in}{114.766308in}}%
\pgfpathlineto{\pgfqpoint{5.800000in}{114.766308in}}%
\pgfpathclose%
\pgfusepath{fill}%
\end{pgfscope}%
\begin{pgfscope}%
\pgfpathrectangle{\pgfqpoint{5.800000in}{0.720000in}}{\pgfqpoint{1.400000in}{4.620000in}}%
\pgfusepath{clip}%
\pgfsetbuttcap%
\pgfsetmiterjoin%
\definecolor{currentfill}{rgb}{0.121569,0.466667,0.705882}%
\pgfsetfillcolor{currentfill}%
\pgfsetlinewidth{0.000000pt}%
\definecolor{currentstroke}{rgb}{0.000000,0.000000,0.000000}%
\pgfsetstrokecolor{currentstroke}%
\pgfsetstrokeopacity{0.000000}%
\pgfsetdash{}{0pt}%
\pgfpathmoveto{\pgfqpoint{5.800000in}{114.766308in}}%
\pgfpathlineto{\pgfqpoint{5.800000in}{114.766308in}}%
\pgfpathlineto{\pgfqpoint{5.800000in}{114.869145in}}%
\pgfpathlineto{\pgfqpoint{5.800000in}{114.869145in}}%
\pgfpathclose%
\pgfusepath{fill}%
\end{pgfscope}%
\begin{pgfscope}%
\pgfpathrectangle{\pgfqpoint{5.800000in}{0.720000in}}{\pgfqpoint{1.400000in}{4.620000in}}%
\pgfusepath{clip}%
\pgfsetbuttcap%
\pgfsetmiterjoin%
\definecolor{currentfill}{rgb}{0.121569,0.466667,0.705882}%
\pgfsetfillcolor{currentfill}%
\pgfsetlinewidth{0.000000pt}%
\definecolor{currentstroke}{rgb}{0.000000,0.000000,0.000000}%
\pgfsetstrokecolor{currentstroke}%
\pgfsetstrokeopacity{0.000000}%
\pgfsetdash{}{0pt}%
\pgfpathmoveto{\pgfqpoint{5.800000in}{114.869145in}}%
\pgfpathlineto{\pgfqpoint{5.800000in}{114.869145in}}%
\pgfpathlineto{\pgfqpoint{5.800000in}{114.971982in}}%
\pgfpathlineto{\pgfqpoint{5.800000in}{114.971982in}}%
\pgfpathclose%
\pgfusepath{fill}%
\end{pgfscope}%
\begin{pgfscope}%
\pgfpathrectangle{\pgfqpoint{5.800000in}{0.720000in}}{\pgfqpoint{1.400000in}{4.620000in}}%
\pgfusepath{clip}%
\pgfsetbuttcap%
\pgfsetmiterjoin%
\definecolor{currentfill}{rgb}{0.121569,0.466667,0.705882}%
\pgfsetfillcolor{currentfill}%
\pgfsetlinewidth{0.000000pt}%
\definecolor{currentstroke}{rgb}{0.000000,0.000000,0.000000}%
\pgfsetstrokecolor{currentstroke}%
\pgfsetstrokeopacity{0.000000}%
\pgfsetdash{}{0pt}%
\pgfpathmoveto{\pgfqpoint{5.800000in}{114.971982in}}%
\pgfpathlineto{\pgfqpoint{5.800000in}{114.971982in}}%
\pgfpathlineto{\pgfqpoint{5.800000in}{115.074819in}}%
\pgfpathlineto{\pgfqpoint{5.800000in}{115.074819in}}%
\pgfpathclose%
\pgfusepath{fill}%
\end{pgfscope}%
\begin{pgfscope}%
\pgfpathrectangle{\pgfqpoint{5.800000in}{0.720000in}}{\pgfqpoint{1.400000in}{4.620000in}}%
\pgfusepath{clip}%
\pgfsetbuttcap%
\pgfsetmiterjoin%
\definecolor{currentfill}{rgb}{0.121569,0.466667,0.705882}%
\pgfsetfillcolor{currentfill}%
\pgfsetlinewidth{0.000000pt}%
\definecolor{currentstroke}{rgb}{0.000000,0.000000,0.000000}%
\pgfsetstrokecolor{currentstroke}%
\pgfsetstrokeopacity{0.000000}%
\pgfsetdash{}{0pt}%
\pgfpathmoveto{\pgfqpoint{5.800000in}{115.074819in}}%
\pgfpathlineto{\pgfqpoint{5.800000in}{115.074819in}}%
\pgfpathlineto{\pgfqpoint{5.800000in}{115.177656in}}%
\pgfpathlineto{\pgfqpoint{5.800000in}{115.177656in}}%
\pgfpathclose%
\pgfusepath{fill}%
\end{pgfscope}%
\begin{pgfscope}%
\pgfpathrectangle{\pgfqpoint{5.800000in}{0.720000in}}{\pgfqpoint{1.400000in}{4.620000in}}%
\pgfusepath{clip}%
\pgfsetbuttcap%
\pgfsetmiterjoin%
\definecolor{currentfill}{rgb}{0.121569,0.466667,0.705882}%
\pgfsetfillcolor{currentfill}%
\pgfsetlinewidth{0.000000pt}%
\definecolor{currentstroke}{rgb}{0.000000,0.000000,0.000000}%
\pgfsetstrokecolor{currentstroke}%
\pgfsetstrokeopacity{0.000000}%
\pgfsetdash{}{0pt}%
\pgfpathmoveto{\pgfqpoint{5.800000in}{115.177656in}}%
\pgfpathlineto{\pgfqpoint{5.800000in}{115.177656in}}%
\pgfpathlineto{\pgfqpoint{5.800000in}{115.280494in}}%
\pgfpathlineto{\pgfqpoint{5.800000in}{115.280494in}}%
\pgfpathclose%
\pgfusepath{fill}%
\end{pgfscope}%
\begin{pgfscope}%
\pgfpathrectangle{\pgfqpoint{5.800000in}{0.720000in}}{\pgfqpoint{1.400000in}{4.620000in}}%
\pgfusepath{clip}%
\pgfsetbuttcap%
\pgfsetmiterjoin%
\definecolor{currentfill}{rgb}{0.121569,0.466667,0.705882}%
\pgfsetfillcolor{currentfill}%
\pgfsetlinewidth{0.000000pt}%
\definecolor{currentstroke}{rgb}{0.000000,0.000000,0.000000}%
\pgfsetstrokecolor{currentstroke}%
\pgfsetstrokeopacity{0.000000}%
\pgfsetdash{}{0pt}%
\pgfpathmoveto{\pgfqpoint{5.800000in}{115.280494in}}%
\pgfpathlineto{\pgfqpoint{5.800000in}{115.280494in}}%
\pgfpathlineto{\pgfqpoint{5.800000in}{115.383331in}}%
\pgfpathlineto{\pgfqpoint{5.800000in}{115.383331in}}%
\pgfpathclose%
\pgfusepath{fill}%
\end{pgfscope}%
\begin{pgfscope}%
\pgfpathrectangle{\pgfqpoint{5.800000in}{0.720000in}}{\pgfqpoint{1.400000in}{4.620000in}}%
\pgfusepath{clip}%
\pgfsetbuttcap%
\pgfsetmiterjoin%
\definecolor{currentfill}{rgb}{0.121569,0.466667,0.705882}%
\pgfsetfillcolor{currentfill}%
\pgfsetlinewidth{0.000000pt}%
\definecolor{currentstroke}{rgb}{0.000000,0.000000,0.000000}%
\pgfsetstrokecolor{currentstroke}%
\pgfsetstrokeopacity{0.000000}%
\pgfsetdash{}{0pt}%
\pgfpathmoveto{\pgfqpoint{5.800000in}{115.383331in}}%
\pgfpathlineto{\pgfqpoint{5.800000in}{115.383331in}}%
\pgfpathlineto{\pgfqpoint{5.800000in}{115.486168in}}%
\pgfpathlineto{\pgfqpoint{5.800000in}{115.486168in}}%
\pgfpathclose%
\pgfusepath{fill}%
\end{pgfscope}%
\begin{pgfscope}%
\pgfpathrectangle{\pgfqpoint{5.800000in}{0.720000in}}{\pgfqpoint{1.400000in}{4.620000in}}%
\pgfusepath{clip}%
\pgfsetbuttcap%
\pgfsetmiterjoin%
\definecolor{currentfill}{rgb}{0.121569,0.466667,0.705882}%
\pgfsetfillcolor{currentfill}%
\pgfsetlinewidth{0.000000pt}%
\definecolor{currentstroke}{rgb}{0.000000,0.000000,0.000000}%
\pgfsetstrokecolor{currentstroke}%
\pgfsetstrokeopacity{0.000000}%
\pgfsetdash{}{0pt}%
\pgfpathmoveto{\pgfqpoint{5.800000in}{115.486168in}}%
\pgfpathlineto{\pgfqpoint{5.800000in}{115.486168in}}%
\pgfpathlineto{\pgfqpoint{5.800000in}{115.589005in}}%
\pgfpathlineto{\pgfqpoint{5.800000in}{115.589005in}}%
\pgfpathclose%
\pgfusepath{fill}%
\end{pgfscope}%
\begin{pgfscope}%
\pgfpathrectangle{\pgfqpoint{5.800000in}{0.720000in}}{\pgfqpoint{1.400000in}{4.620000in}}%
\pgfusepath{clip}%
\pgfsetbuttcap%
\pgfsetmiterjoin%
\definecolor{currentfill}{rgb}{0.121569,0.466667,0.705882}%
\pgfsetfillcolor{currentfill}%
\pgfsetlinewidth{0.000000pt}%
\definecolor{currentstroke}{rgb}{0.000000,0.000000,0.000000}%
\pgfsetstrokecolor{currentstroke}%
\pgfsetstrokeopacity{0.000000}%
\pgfsetdash{}{0pt}%
\pgfpathmoveto{\pgfqpoint{5.800000in}{115.589005in}}%
\pgfpathlineto{\pgfqpoint{5.800000in}{115.589005in}}%
\pgfpathlineto{\pgfqpoint{5.800000in}{115.691842in}}%
\pgfpathlineto{\pgfqpoint{5.800000in}{115.691842in}}%
\pgfpathclose%
\pgfusepath{fill}%
\end{pgfscope}%
\begin{pgfscope}%
\pgfpathrectangle{\pgfqpoint{5.800000in}{0.720000in}}{\pgfqpoint{1.400000in}{4.620000in}}%
\pgfusepath{clip}%
\pgfsetbuttcap%
\pgfsetmiterjoin%
\definecolor{currentfill}{rgb}{0.121569,0.466667,0.705882}%
\pgfsetfillcolor{currentfill}%
\pgfsetlinewidth{0.000000pt}%
\definecolor{currentstroke}{rgb}{0.000000,0.000000,0.000000}%
\pgfsetstrokecolor{currentstroke}%
\pgfsetstrokeopacity{0.000000}%
\pgfsetdash{}{0pt}%
\pgfpathmoveto{\pgfqpoint{5.800000in}{115.691842in}}%
\pgfpathlineto{\pgfqpoint{5.800000in}{115.691842in}}%
\pgfpathlineto{\pgfqpoint{5.800000in}{115.794679in}}%
\pgfpathlineto{\pgfqpoint{5.800000in}{115.794679in}}%
\pgfpathclose%
\pgfusepath{fill}%
\end{pgfscope}%
\begin{pgfscope}%
\pgfpathrectangle{\pgfqpoint{5.800000in}{0.720000in}}{\pgfqpoint{1.400000in}{4.620000in}}%
\pgfusepath{clip}%
\pgfsetbuttcap%
\pgfsetmiterjoin%
\definecolor{currentfill}{rgb}{0.121569,0.466667,0.705882}%
\pgfsetfillcolor{currentfill}%
\pgfsetlinewidth{0.000000pt}%
\definecolor{currentstroke}{rgb}{0.000000,0.000000,0.000000}%
\pgfsetstrokecolor{currentstroke}%
\pgfsetstrokeopacity{0.000000}%
\pgfsetdash{}{0pt}%
\pgfpathmoveto{\pgfqpoint{5.800000in}{115.794679in}}%
\pgfpathlineto{\pgfqpoint{5.800000in}{115.794679in}}%
\pgfpathlineto{\pgfqpoint{5.800000in}{115.897516in}}%
\pgfpathlineto{\pgfqpoint{5.800000in}{115.897516in}}%
\pgfpathclose%
\pgfusepath{fill}%
\end{pgfscope}%
\begin{pgfscope}%
\pgfpathrectangle{\pgfqpoint{5.800000in}{0.720000in}}{\pgfqpoint{1.400000in}{4.620000in}}%
\pgfusepath{clip}%
\pgfsetbuttcap%
\pgfsetmiterjoin%
\definecolor{currentfill}{rgb}{0.121569,0.466667,0.705882}%
\pgfsetfillcolor{currentfill}%
\pgfsetlinewidth{0.000000pt}%
\definecolor{currentstroke}{rgb}{0.000000,0.000000,0.000000}%
\pgfsetstrokecolor{currentstroke}%
\pgfsetstrokeopacity{0.000000}%
\pgfsetdash{}{0pt}%
\pgfpathmoveto{\pgfqpoint{5.800000in}{115.897516in}}%
\pgfpathlineto{\pgfqpoint{5.800000in}{115.897516in}}%
\pgfpathlineto{\pgfqpoint{5.800000in}{116.000353in}}%
\pgfpathlineto{\pgfqpoint{5.800000in}{116.000353in}}%
\pgfpathclose%
\pgfusepath{fill}%
\end{pgfscope}%
\begin{pgfscope}%
\pgfpathrectangle{\pgfqpoint{5.800000in}{0.720000in}}{\pgfqpoint{1.400000in}{4.620000in}}%
\pgfusepath{clip}%
\pgfsetbuttcap%
\pgfsetmiterjoin%
\definecolor{currentfill}{rgb}{0.121569,0.466667,0.705882}%
\pgfsetfillcolor{currentfill}%
\pgfsetlinewidth{0.000000pt}%
\definecolor{currentstroke}{rgb}{0.000000,0.000000,0.000000}%
\pgfsetstrokecolor{currentstroke}%
\pgfsetstrokeopacity{0.000000}%
\pgfsetdash{}{0pt}%
\pgfpathmoveto{\pgfqpoint{5.800000in}{116.000353in}}%
\pgfpathlineto{\pgfqpoint{5.800000in}{116.000353in}}%
\pgfpathlineto{\pgfqpoint{5.800000in}{116.103190in}}%
\pgfpathlineto{\pgfqpoint{5.800000in}{116.103190in}}%
\pgfpathclose%
\pgfusepath{fill}%
\end{pgfscope}%
\begin{pgfscope}%
\pgfpathrectangle{\pgfqpoint{5.800000in}{0.720000in}}{\pgfqpoint{1.400000in}{4.620000in}}%
\pgfusepath{clip}%
\pgfsetbuttcap%
\pgfsetmiterjoin%
\definecolor{currentfill}{rgb}{0.121569,0.466667,0.705882}%
\pgfsetfillcolor{currentfill}%
\pgfsetlinewidth{0.000000pt}%
\definecolor{currentstroke}{rgb}{0.000000,0.000000,0.000000}%
\pgfsetstrokecolor{currentstroke}%
\pgfsetstrokeopacity{0.000000}%
\pgfsetdash{}{0pt}%
\pgfpathmoveto{\pgfqpoint{5.800000in}{116.103190in}}%
\pgfpathlineto{\pgfqpoint{5.800000in}{116.103190in}}%
\pgfpathlineto{\pgfqpoint{5.800000in}{116.206027in}}%
\pgfpathlineto{\pgfqpoint{5.800000in}{116.206027in}}%
\pgfpathclose%
\pgfusepath{fill}%
\end{pgfscope}%
\begin{pgfscope}%
\pgfpathrectangle{\pgfqpoint{5.800000in}{0.720000in}}{\pgfqpoint{1.400000in}{4.620000in}}%
\pgfusepath{clip}%
\pgfsetbuttcap%
\pgfsetmiterjoin%
\definecolor{currentfill}{rgb}{0.121569,0.466667,0.705882}%
\pgfsetfillcolor{currentfill}%
\pgfsetlinewidth{0.000000pt}%
\definecolor{currentstroke}{rgb}{0.000000,0.000000,0.000000}%
\pgfsetstrokecolor{currentstroke}%
\pgfsetstrokeopacity{0.000000}%
\pgfsetdash{}{0pt}%
\pgfpathmoveto{\pgfqpoint{5.800000in}{116.206027in}}%
\pgfpathlineto{\pgfqpoint{5.800000in}{116.206027in}}%
\pgfpathlineto{\pgfqpoint{5.800000in}{116.308864in}}%
\pgfpathlineto{\pgfqpoint{5.800000in}{116.308864in}}%
\pgfpathclose%
\pgfusepath{fill}%
\end{pgfscope}%
\begin{pgfscope}%
\pgfpathrectangle{\pgfqpoint{5.800000in}{0.720000in}}{\pgfqpoint{1.400000in}{4.620000in}}%
\pgfusepath{clip}%
\pgfsetbuttcap%
\pgfsetmiterjoin%
\definecolor{currentfill}{rgb}{0.121569,0.466667,0.705882}%
\pgfsetfillcolor{currentfill}%
\pgfsetlinewidth{0.000000pt}%
\definecolor{currentstroke}{rgb}{0.000000,0.000000,0.000000}%
\pgfsetstrokecolor{currentstroke}%
\pgfsetstrokeopacity{0.000000}%
\pgfsetdash{}{0pt}%
\pgfpathmoveto{\pgfqpoint{5.800000in}{116.308864in}}%
\pgfpathlineto{\pgfqpoint{5.800000in}{116.308864in}}%
\pgfpathlineto{\pgfqpoint{5.800000in}{116.411701in}}%
\pgfpathlineto{\pgfqpoint{5.800000in}{116.411701in}}%
\pgfpathclose%
\pgfusepath{fill}%
\end{pgfscope}%
\begin{pgfscope}%
\pgfpathrectangle{\pgfqpoint{5.800000in}{0.720000in}}{\pgfqpoint{1.400000in}{4.620000in}}%
\pgfusepath{clip}%
\pgfsetbuttcap%
\pgfsetmiterjoin%
\definecolor{currentfill}{rgb}{0.121569,0.466667,0.705882}%
\pgfsetfillcolor{currentfill}%
\pgfsetlinewidth{0.000000pt}%
\definecolor{currentstroke}{rgb}{0.000000,0.000000,0.000000}%
\pgfsetstrokecolor{currentstroke}%
\pgfsetstrokeopacity{0.000000}%
\pgfsetdash{}{0pt}%
\pgfpathmoveto{\pgfqpoint{5.800000in}{116.411701in}}%
\pgfpathlineto{\pgfqpoint{5.800000in}{116.411701in}}%
\pgfpathlineto{\pgfqpoint{5.800000in}{116.514538in}}%
\pgfpathlineto{\pgfqpoint{5.800000in}{116.514538in}}%
\pgfpathclose%
\pgfusepath{fill}%
\end{pgfscope}%
\begin{pgfscope}%
\pgfpathrectangle{\pgfqpoint{5.800000in}{0.720000in}}{\pgfqpoint{1.400000in}{4.620000in}}%
\pgfusepath{clip}%
\pgfsetbuttcap%
\pgfsetmiterjoin%
\definecolor{currentfill}{rgb}{0.121569,0.466667,0.705882}%
\pgfsetfillcolor{currentfill}%
\pgfsetlinewidth{0.000000pt}%
\definecolor{currentstroke}{rgb}{0.000000,0.000000,0.000000}%
\pgfsetstrokecolor{currentstroke}%
\pgfsetstrokeopacity{0.000000}%
\pgfsetdash{}{0pt}%
\pgfpathmoveto{\pgfqpoint{5.800000in}{116.514538in}}%
\pgfpathlineto{\pgfqpoint{5.800000in}{116.514538in}}%
\pgfpathlineto{\pgfqpoint{5.800000in}{116.617375in}}%
\pgfpathlineto{\pgfqpoint{5.800000in}{116.617375in}}%
\pgfpathclose%
\pgfusepath{fill}%
\end{pgfscope}%
\begin{pgfscope}%
\pgfpathrectangle{\pgfqpoint{5.800000in}{0.720000in}}{\pgfqpoint{1.400000in}{4.620000in}}%
\pgfusepath{clip}%
\pgfsetbuttcap%
\pgfsetmiterjoin%
\definecolor{currentfill}{rgb}{0.121569,0.466667,0.705882}%
\pgfsetfillcolor{currentfill}%
\pgfsetlinewidth{0.000000pt}%
\definecolor{currentstroke}{rgb}{0.000000,0.000000,0.000000}%
\pgfsetstrokecolor{currentstroke}%
\pgfsetstrokeopacity{0.000000}%
\pgfsetdash{}{0pt}%
\pgfpathmoveto{\pgfqpoint{5.800000in}{116.617375in}}%
\pgfpathlineto{\pgfqpoint{5.800000in}{116.617375in}}%
\pgfpathlineto{\pgfqpoint{5.800000in}{116.720212in}}%
\pgfpathlineto{\pgfqpoint{5.800000in}{116.720212in}}%
\pgfpathclose%
\pgfusepath{fill}%
\end{pgfscope}%
\begin{pgfscope}%
\pgfpathrectangle{\pgfqpoint{5.800000in}{0.720000in}}{\pgfqpoint{1.400000in}{4.620000in}}%
\pgfusepath{clip}%
\pgfsetbuttcap%
\pgfsetmiterjoin%
\definecolor{currentfill}{rgb}{0.121569,0.466667,0.705882}%
\pgfsetfillcolor{currentfill}%
\pgfsetlinewidth{0.000000pt}%
\definecolor{currentstroke}{rgb}{0.000000,0.000000,0.000000}%
\pgfsetstrokecolor{currentstroke}%
\pgfsetstrokeopacity{0.000000}%
\pgfsetdash{}{0pt}%
\pgfpathmoveto{\pgfqpoint{5.800000in}{116.720212in}}%
\pgfpathlineto{\pgfqpoint{5.800000in}{116.720212in}}%
\pgfpathlineto{\pgfqpoint{5.800000in}{116.823050in}}%
\pgfpathlineto{\pgfqpoint{5.800000in}{116.823050in}}%
\pgfpathclose%
\pgfusepath{fill}%
\end{pgfscope}%
\begin{pgfscope}%
\pgfpathrectangle{\pgfqpoint{5.800000in}{0.720000in}}{\pgfqpoint{1.400000in}{4.620000in}}%
\pgfusepath{clip}%
\pgfsetbuttcap%
\pgfsetmiterjoin%
\definecolor{currentfill}{rgb}{0.121569,0.466667,0.705882}%
\pgfsetfillcolor{currentfill}%
\pgfsetlinewidth{0.000000pt}%
\definecolor{currentstroke}{rgb}{0.000000,0.000000,0.000000}%
\pgfsetstrokecolor{currentstroke}%
\pgfsetstrokeopacity{0.000000}%
\pgfsetdash{}{0pt}%
\pgfpathmoveto{\pgfqpoint{5.800000in}{116.823050in}}%
\pgfpathlineto{\pgfqpoint{5.800000in}{116.823050in}}%
\pgfpathlineto{\pgfqpoint{5.800000in}{116.925887in}}%
\pgfpathlineto{\pgfqpoint{5.800000in}{116.925887in}}%
\pgfpathclose%
\pgfusepath{fill}%
\end{pgfscope}%
\begin{pgfscope}%
\pgfpathrectangle{\pgfqpoint{5.800000in}{0.720000in}}{\pgfqpoint{1.400000in}{4.620000in}}%
\pgfusepath{clip}%
\pgfsetbuttcap%
\pgfsetmiterjoin%
\definecolor{currentfill}{rgb}{0.121569,0.466667,0.705882}%
\pgfsetfillcolor{currentfill}%
\pgfsetlinewidth{0.000000pt}%
\definecolor{currentstroke}{rgb}{0.000000,0.000000,0.000000}%
\pgfsetstrokecolor{currentstroke}%
\pgfsetstrokeopacity{0.000000}%
\pgfsetdash{}{0pt}%
\pgfpathmoveto{\pgfqpoint{5.800000in}{116.925887in}}%
\pgfpathlineto{\pgfqpoint{5.800000in}{116.925887in}}%
\pgfpathlineto{\pgfqpoint{5.800000in}{117.028724in}}%
\pgfpathlineto{\pgfqpoint{5.800000in}{117.028724in}}%
\pgfpathclose%
\pgfusepath{fill}%
\end{pgfscope}%
\begin{pgfscope}%
\pgfpathrectangle{\pgfqpoint{5.800000in}{0.720000in}}{\pgfqpoint{1.400000in}{4.620000in}}%
\pgfusepath{clip}%
\pgfsetbuttcap%
\pgfsetmiterjoin%
\definecolor{currentfill}{rgb}{0.121569,0.466667,0.705882}%
\pgfsetfillcolor{currentfill}%
\pgfsetlinewidth{0.000000pt}%
\definecolor{currentstroke}{rgb}{0.000000,0.000000,0.000000}%
\pgfsetstrokecolor{currentstroke}%
\pgfsetstrokeopacity{0.000000}%
\pgfsetdash{}{0pt}%
\pgfpathmoveto{\pgfqpoint{5.800000in}{117.028724in}}%
\pgfpathlineto{\pgfqpoint{5.800000in}{117.028724in}}%
\pgfpathlineto{\pgfqpoint{5.800000in}{117.131561in}}%
\pgfpathlineto{\pgfqpoint{5.800000in}{117.131561in}}%
\pgfpathclose%
\pgfusepath{fill}%
\end{pgfscope}%
\begin{pgfscope}%
\pgfpathrectangle{\pgfqpoint{5.800000in}{0.720000in}}{\pgfqpoint{1.400000in}{4.620000in}}%
\pgfusepath{clip}%
\pgfsetbuttcap%
\pgfsetmiterjoin%
\definecolor{currentfill}{rgb}{0.121569,0.466667,0.705882}%
\pgfsetfillcolor{currentfill}%
\pgfsetlinewidth{0.000000pt}%
\definecolor{currentstroke}{rgb}{0.000000,0.000000,0.000000}%
\pgfsetstrokecolor{currentstroke}%
\pgfsetstrokeopacity{0.000000}%
\pgfsetdash{}{0pt}%
\pgfpathmoveto{\pgfqpoint{5.800000in}{117.131561in}}%
\pgfpathlineto{\pgfqpoint{5.800000in}{117.131561in}}%
\pgfpathlineto{\pgfqpoint{5.800000in}{117.234398in}}%
\pgfpathlineto{\pgfqpoint{5.800000in}{117.234398in}}%
\pgfpathclose%
\pgfusepath{fill}%
\end{pgfscope}%
\begin{pgfscope}%
\pgfpathrectangle{\pgfqpoint{5.800000in}{0.720000in}}{\pgfqpoint{1.400000in}{4.620000in}}%
\pgfusepath{clip}%
\pgfsetbuttcap%
\pgfsetmiterjoin%
\definecolor{currentfill}{rgb}{0.121569,0.466667,0.705882}%
\pgfsetfillcolor{currentfill}%
\pgfsetlinewidth{0.000000pt}%
\definecolor{currentstroke}{rgb}{0.000000,0.000000,0.000000}%
\pgfsetstrokecolor{currentstroke}%
\pgfsetstrokeopacity{0.000000}%
\pgfsetdash{}{0pt}%
\pgfpathmoveto{\pgfqpoint{5.800000in}{117.234398in}}%
\pgfpathlineto{\pgfqpoint{5.800000in}{117.234398in}}%
\pgfpathlineto{\pgfqpoint{5.800000in}{117.337235in}}%
\pgfpathlineto{\pgfqpoint{5.800000in}{117.337235in}}%
\pgfpathclose%
\pgfusepath{fill}%
\end{pgfscope}%
\begin{pgfscope}%
\pgfpathrectangle{\pgfqpoint{5.800000in}{0.720000in}}{\pgfqpoint{1.400000in}{4.620000in}}%
\pgfusepath{clip}%
\pgfsetbuttcap%
\pgfsetmiterjoin%
\definecolor{currentfill}{rgb}{0.121569,0.466667,0.705882}%
\pgfsetfillcolor{currentfill}%
\pgfsetlinewidth{0.000000pt}%
\definecolor{currentstroke}{rgb}{0.000000,0.000000,0.000000}%
\pgfsetstrokecolor{currentstroke}%
\pgfsetstrokeopacity{0.000000}%
\pgfsetdash{}{0pt}%
\pgfpathmoveto{\pgfqpoint{5.800000in}{117.337235in}}%
\pgfpathlineto{\pgfqpoint{5.800000in}{117.337235in}}%
\pgfpathlineto{\pgfqpoint{5.800000in}{117.440072in}}%
\pgfpathlineto{\pgfqpoint{5.800000in}{117.440072in}}%
\pgfpathclose%
\pgfusepath{fill}%
\end{pgfscope}%
\begin{pgfscope}%
\pgfpathrectangle{\pgfqpoint{5.800000in}{0.720000in}}{\pgfqpoint{1.400000in}{4.620000in}}%
\pgfusepath{clip}%
\pgfsetbuttcap%
\pgfsetmiterjoin%
\definecolor{currentfill}{rgb}{0.121569,0.466667,0.705882}%
\pgfsetfillcolor{currentfill}%
\pgfsetlinewidth{0.000000pt}%
\definecolor{currentstroke}{rgb}{0.000000,0.000000,0.000000}%
\pgfsetstrokecolor{currentstroke}%
\pgfsetstrokeopacity{0.000000}%
\pgfsetdash{}{0pt}%
\pgfpathmoveto{\pgfqpoint{5.800000in}{117.440072in}}%
\pgfpathlineto{\pgfqpoint{5.800000in}{117.440072in}}%
\pgfpathlineto{\pgfqpoint{5.800000in}{117.542909in}}%
\pgfpathlineto{\pgfqpoint{5.800000in}{117.542909in}}%
\pgfpathclose%
\pgfusepath{fill}%
\end{pgfscope}%
\begin{pgfscope}%
\pgfpathrectangle{\pgfqpoint{5.800000in}{0.720000in}}{\pgfqpoint{1.400000in}{4.620000in}}%
\pgfusepath{clip}%
\pgfsetbuttcap%
\pgfsetmiterjoin%
\definecolor{currentfill}{rgb}{0.121569,0.466667,0.705882}%
\pgfsetfillcolor{currentfill}%
\pgfsetlinewidth{0.000000pt}%
\definecolor{currentstroke}{rgb}{0.000000,0.000000,0.000000}%
\pgfsetstrokecolor{currentstroke}%
\pgfsetstrokeopacity{0.000000}%
\pgfsetdash{}{0pt}%
\pgfpathmoveto{\pgfqpoint{5.800000in}{117.542909in}}%
\pgfpathlineto{\pgfqpoint{5.800000in}{117.542909in}}%
\pgfpathlineto{\pgfqpoint{5.800000in}{117.645746in}}%
\pgfpathlineto{\pgfqpoint{5.800000in}{117.645746in}}%
\pgfpathclose%
\pgfusepath{fill}%
\end{pgfscope}%
\begin{pgfscope}%
\pgfpathrectangle{\pgfqpoint{5.800000in}{0.720000in}}{\pgfqpoint{1.400000in}{4.620000in}}%
\pgfusepath{clip}%
\pgfsetbuttcap%
\pgfsetmiterjoin%
\definecolor{currentfill}{rgb}{0.121569,0.466667,0.705882}%
\pgfsetfillcolor{currentfill}%
\pgfsetlinewidth{0.000000pt}%
\definecolor{currentstroke}{rgb}{0.000000,0.000000,0.000000}%
\pgfsetstrokecolor{currentstroke}%
\pgfsetstrokeopacity{0.000000}%
\pgfsetdash{}{0pt}%
\pgfpathmoveto{\pgfqpoint{5.800000in}{117.645746in}}%
\pgfpathlineto{\pgfqpoint{5.800000in}{117.645746in}}%
\pgfpathlineto{\pgfqpoint{5.800000in}{117.748583in}}%
\pgfpathlineto{\pgfqpoint{5.800000in}{117.748583in}}%
\pgfpathclose%
\pgfusepath{fill}%
\end{pgfscope}%
\begin{pgfscope}%
\pgfpathrectangle{\pgfqpoint{5.800000in}{0.720000in}}{\pgfqpoint{1.400000in}{4.620000in}}%
\pgfusepath{clip}%
\pgfsetbuttcap%
\pgfsetmiterjoin%
\definecolor{currentfill}{rgb}{0.121569,0.466667,0.705882}%
\pgfsetfillcolor{currentfill}%
\pgfsetlinewidth{0.000000pt}%
\definecolor{currentstroke}{rgb}{0.000000,0.000000,0.000000}%
\pgfsetstrokecolor{currentstroke}%
\pgfsetstrokeopacity{0.000000}%
\pgfsetdash{}{0pt}%
\pgfpathmoveto{\pgfqpoint{5.800000in}{117.748583in}}%
\pgfpathlineto{\pgfqpoint{5.800000in}{117.748583in}}%
\pgfpathlineto{\pgfqpoint{5.800000in}{117.851420in}}%
\pgfpathlineto{\pgfqpoint{5.800000in}{117.851420in}}%
\pgfpathclose%
\pgfusepath{fill}%
\end{pgfscope}%
\begin{pgfscope}%
\pgfpathrectangle{\pgfqpoint{5.800000in}{0.720000in}}{\pgfqpoint{1.400000in}{4.620000in}}%
\pgfusepath{clip}%
\pgfsetbuttcap%
\pgfsetmiterjoin%
\definecolor{currentfill}{rgb}{0.121569,0.466667,0.705882}%
\pgfsetfillcolor{currentfill}%
\pgfsetlinewidth{0.000000pt}%
\definecolor{currentstroke}{rgb}{0.000000,0.000000,0.000000}%
\pgfsetstrokecolor{currentstroke}%
\pgfsetstrokeopacity{0.000000}%
\pgfsetdash{}{0pt}%
\pgfpathmoveto{\pgfqpoint{5.800000in}{117.851420in}}%
\pgfpathlineto{\pgfqpoint{5.800000in}{117.851420in}}%
\pgfpathlineto{\pgfqpoint{5.800000in}{117.954257in}}%
\pgfpathlineto{\pgfqpoint{5.800000in}{117.954257in}}%
\pgfpathclose%
\pgfusepath{fill}%
\end{pgfscope}%
\begin{pgfscope}%
\pgfpathrectangle{\pgfqpoint{5.800000in}{0.720000in}}{\pgfqpoint{1.400000in}{4.620000in}}%
\pgfusepath{clip}%
\pgfsetbuttcap%
\pgfsetmiterjoin%
\definecolor{currentfill}{rgb}{0.121569,0.466667,0.705882}%
\pgfsetfillcolor{currentfill}%
\pgfsetlinewidth{0.000000pt}%
\definecolor{currentstroke}{rgb}{0.000000,0.000000,0.000000}%
\pgfsetstrokecolor{currentstroke}%
\pgfsetstrokeopacity{0.000000}%
\pgfsetdash{}{0pt}%
\pgfpathmoveto{\pgfqpoint{5.800000in}{117.954257in}}%
\pgfpathlineto{\pgfqpoint{5.800000in}{117.954257in}}%
\pgfpathlineto{\pgfqpoint{5.800000in}{118.057094in}}%
\pgfpathlineto{\pgfqpoint{5.800000in}{118.057094in}}%
\pgfpathclose%
\pgfusepath{fill}%
\end{pgfscope}%
\begin{pgfscope}%
\pgfpathrectangle{\pgfqpoint{5.800000in}{0.720000in}}{\pgfqpoint{1.400000in}{4.620000in}}%
\pgfusepath{clip}%
\pgfsetbuttcap%
\pgfsetmiterjoin%
\definecolor{currentfill}{rgb}{0.121569,0.466667,0.705882}%
\pgfsetfillcolor{currentfill}%
\pgfsetlinewidth{0.000000pt}%
\definecolor{currentstroke}{rgb}{0.000000,0.000000,0.000000}%
\pgfsetstrokecolor{currentstroke}%
\pgfsetstrokeopacity{0.000000}%
\pgfsetdash{}{0pt}%
\pgfpathmoveto{\pgfqpoint{5.800000in}{118.057094in}}%
\pgfpathlineto{\pgfqpoint{5.800000in}{118.057094in}}%
\pgfpathlineto{\pgfqpoint{5.800000in}{118.159931in}}%
\pgfpathlineto{\pgfqpoint{5.800000in}{118.159931in}}%
\pgfpathclose%
\pgfusepath{fill}%
\end{pgfscope}%
\begin{pgfscope}%
\pgfpathrectangle{\pgfqpoint{5.800000in}{0.720000in}}{\pgfqpoint{1.400000in}{4.620000in}}%
\pgfusepath{clip}%
\pgfsetbuttcap%
\pgfsetmiterjoin%
\definecolor{currentfill}{rgb}{0.121569,0.466667,0.705882}%
\pgfsetfillcolor{currentfill}%
\pgfsetlinewidth{0.000000pt}%
\definecolor{currentstroke}{rgb}{0.000000,0.000000,0.000000}%
\pgfsetstrokecolor{currentstroke}%
\pgfsetstrokeopacity{0.000000}%
\pgfsetdash{}{0pt}%
\pgfpathmoveto{\pgfqpoint{5.800000in}{118.159931in}}%
\pgfpathlineto{\pgfqpoint{5.800000in}{118.159931in}}%
\pgfpathlineto{\pgfqpoint{5.800000in}{118.262768in}}%
\pgfpathlineto{\pgfqpoint{5.800000in}{118.262768in}}%
\pgfpathclose%
\pgfusepath{fill}%
\end{pgfscope}%
\begin{pgfscope}%
\pgfpathrectangle{\pgfqpoint{5.800000in}{0.720000in}}{\pgfqpoint{1.400000in}{4.620000in}}%
\pgfusepath{clip}%
\pgfsetbuttcap%
\pgfsetmiterjoin%
\definecolor{currentfill}{rgb}{0.121569,0.466667,0.705882}%
\pgfsetfillcolor{currentfill}%
\pgfsetlinewidth{0.000000pt}%
\definecolor{currentstroke}{rgb}{0.000000,0.000000,0.000000}%
\pgfsetstrokecolor{currentstroke}%
\pgfsetstrokeopacity{0.000000}%
\pgfsetdash{}{0pt}%
\pgfpathmoveto{\pgfqpoint{5.800000in}{118.262768in}}%
\pgfpathlineto{\pgfqpoint{5.800000in}{118.262768in}}%
\pgfpathlineto{\pgfqpoint{5.800000in}{118.365606in}}%
\pgfpathlineto{\pgfqpoint{5.800000in}{118.365606in}}%
\pgfpathclose%
\pgfusepath{fill}%
\end{pgfscope}%
\begin{pgfscope}%
\pgfpathrectangle{\pgfqpoint{5.800000in}{0.720000in}}{\pgfqpoint{1.400000in}{4.620000in}}%
\pgfusepath{clip}%
\pgfsetbuttcap%
\pgfsetmiterjoin%
\definecolor{currentfill}{rgb}{0.121569,0.466667,0.705882}%
\pgfsetfillcolor{currentfill}%
\pgfsetlinewidth{0.000000pt}%
\definecolor{currentstroke}{rgb}{0.000000,0.000000,0.000000}%
\pgfsetstrokecolor{currentstroke}%
\pgfsetstrokeopacity{0.000000}%
\pgfsetdash{}{0pt}%
\pgfpathmoveto{\pgfqpoint{5.800000in}{118.365606in}}%
\pgfpathlineto{\pgfqpoint{5.800000in}{118.365606in}}%
\pgfpathlineto{\pgfqpoint{5.800000in}{118.468443in}}%
\pgfpathlineto{\pgfqpoint{5.800000in}{118.468443in}}%
\pgfpathclose%
\pgfusepath{fill}%
\end{pgfscope}%
\begin{pgfscope}%
\pgfpathrectangle{\pgfqpoint{5.800000in}{0.720000in}}{\pgfqpoint{1.400000in}{4.620000in}}%
\pgfusepath{clip}%
\pgfsetbuttcap%
\pgfsetmiterjoin%
\definecolor{currentfill}{rgb}{0.121569,0.466667,0.705882}%
\pgfsetfillcolor{currentfill}%
\pgfsetlinewidth{0.000000pt}%
\definecolor{currentstroke}{rgb}{0.000000,0.000000,0.000000}%
\pgfsetstrokecolor{currentstroke}%
\pgfsetstrokeopacity{0.000000}%
\pgfsetdash{}{0pt}%
\pgfpathmoveto{\pgfqpoint{5.800000in}{118.468443in}}%
\pgfpathlineto{\pgfqpoint{5.800000in}{118.468443in}}%
\pgfpathlineto{\pgfqpoint{5.800000in}{118.571280in}}%
\pgfpathlineto{\pgfqpoint{5.800000in}{118.571280in}}%
\pgfpathclose%
\pgfusepath{fill}%
\end{pgfscope}%
\begin{pgfscope}%
\pgfpathrectangle{\pgfqpoint{5.800000in}{0.720000in}}{\pgfqpoint{1.400000in}{4.620000in}}%
\pgfusepath{clip}%
\pgfsetbuttcap%
\pgfsetmiterjoin%
\definecolor{currentfill}{rgb}{0.121569,0.466667,0.705882}%
\pgfsetfillcolor{currentfill}%
\pgfsetlinewidth{0.000000pt}%
\definecolor{currentstroke}{rgb}{0.000000,0.000000,0.000000}%
\pgfsetstrokecolor{currentstroke}%
\pgfsetstrokeopacity{0.000000}%
\pgfsetdash{}{0pt}%
\pgfpathmoveto{\pgfqpoint{5.800000in}{118.571280in}}%
\pgfpathlineto{\pgfqpoint{5.800000in}{118.571280in}}%
\pgfpathlineto{\pgfqpoint{5.800000in}{118.674117in}}%
\pgfpathlineto{\pgfqpoint{5.800000in}{118.674117in}}%
\pgfpathclose%
\pgfusepath{fill}%
\end{pgfscope}%
\begin{pgfscope}%
\pgfpathrectangle{\pgfqpoint{5.800000in}{0.720000in}}{\pgfqpoint{1.400000in}{4.620000in}}%
\pgfusepath{clip}%
\pgfsetbuttcap%
\pgfsetmiterjoin%
\definecolor{currentfill}{rgb}{0.121569,0.466667,0.705882}%
\pgfsetfillcolor{currentfill}%
\pgfsetlinewidth{0.000000pt}%
\definecolor{currentstroke}{rgb}{0.000000,0.000000,0.000000}%
\pgfsetstrokecolor{currentstroke}%
\pgfsetstrokeopacity{0.000000}%
\pgfsetdash{}{0pt}%
\pgfpathmoveto{\pgfqpoint{5.800000in}{118.674117in}}%
\pgfpathlineto{\pgfqpoint{5.800000in}{118.674117in}}%
\pgfpathlineto{\pgfqpoint{5.800000in}{118.776954in}}%
\pgfpathlineto{\pgfqpoint{5.800000in}{118.776954in}}%
\pgfpathclose%
\pgfusepath{fill}%
\end{pgfscope}%
\begin{pgfscope}%
\pgfpathrectangle{\pgfqpoint{5.800000in}{0.720000in}}{\pgfqpoint{1.400000in}{4.620000in}}%
\pgfusepath{clip}%
\pgfsetbuttcap%
\pgfsetmiterjoin%
\definecolor{currentfill}{rgb}{0.121569,0.466667,0.705882}%
\pgfsetfillcolor{currentfill}%
\pgfsetlinewidth{0.000000pt}%
\definecolor{currentstroke}{rgb}{0.000000,0.000000,0.000000}%
\pgfsetstrokecolor{currentstroke}%
\pgfsetstrokeopacity{0.000000}%
\pgfsetdash{}{0pt}%
\pgfpathmoveto{\pgfqpoint{5.800000in}{118.776954in}}%
\pgfpathlineto{\pgfqpoint{5.800000in}{118.776954in}}%
\pgfpathlineto{\pgfqpoint{5.800000in}{118.879791in}}%
\pgfpathlineto{\pgfqpoint{5.800000in}{118.879791in}}%
\pgfpathclose%
\pgfusepath{fill}%
\end{pgfscope}%
\begin{pgfscope}%
\pgfpathrectangle{\pgfqpoint{5.800000in}{0.720000in}}{\pgfqpoint{1.400000in}{4.620000in}}%
\pgfusepath{clip}%
\pgfsetbuttcap%
\pgfsetmiterjoin%
\definecolor{currentfill}{rgb}{0.121569,0.466667,0.705882}%
\pgfsetfillcolor{currentfill}%
\pgfsetlinewidth{0.000000pt}%
\definecolor{currentstroke}{rgb}{0.000000,0.000000,0.000000}%
\pgfsetstrokecolor{currentstroke}%
\pgfsetstrokeopacity{0.000000}%
\pgfsetdash{}{0pt}%
\pgfpathmoveto{\pgfqpoint{5.800000in}{118.879791in}}%
\pgfpathlineto{\pgfqpoint{5.800000in}{118.879791in}}%
\pgfpathlineto{\pgfqpoint{5.800000in}{118.982628in}}%
\pgfpathlineto{\pgfqpoint{5.800000in}{118.982628in}}%
\pgfpathclose%
\pgfusepath{fill}%
\end{pgfscope}%
\begin{pgfscope}%
\pgfpathrectangle{\pgfqpoint{5.800000in}{0.720000in}}{\pgfqpoint{1.400000in}{4.620000in}}%
\pgfusepath{clip}%
\pgfsetbuttcap%
\pgfsetmiterjoin%
\definecolor{currentfill}{rgb}{0.121569,0.466667,0.705882}%
\pgfsetfillcolor{currentfill}%
\pgfsetlinewidth{0.000000pt}%
\definecolor{currentstroke}{rgb}{0.000000,0.000000,0.000000}%
\pgfsetstrokecolor{currentstroke}%
\pgfsetstrokeopacity{0.000000}%
\pgfsetdash{}{0pt}%
\pgfpathmoveto{\pgfqpoint{5.800000in}{118.982628in}}%
\pgfpathlineto{\pgfqpoint{5.800000in}{118.982628in}}%
\pgfpathlineto{\pgfqpoint{5.800000in}{119.085465in}}%
\pgfpathlineto{\pgfqpoint{5.800000in}{119.085465in}}%
\pgfpathclose%
\pgfusepath{fill}%
\end{pgfscope}%
\begin{pgfscope}%
\pgfpathrectangle{\pgfqpoint{5.800000in}{0.720000in}}{\pgfqpoint{1.400000in}{4.620000in}}%
\pgfusepath{clip}%
\pgfsetbuttcap%
\pgfsetmiterjoin%
\definecolor{currentfill}{rgb}{0.121569,0.466667,0.705882}%
\pgfsetfillcolor{currentfill}%
\pgfsetlinewidth{0.000000pt}%
\definecolor{currentstroke}{rgb}{0.000000,0.000000,0.000000}%
\pgfsetstrokecolor{currentstroke}%
\pgfsetstrokeopacity{0.000000}%
\pgfsetdash{}{0pt}%
\pgfpathmoveto{\pgfqpoint{5.800000in}{119.085465in}}%
\pgfpathlineto{\pgfqpoint{5.800000in}{119.085465in}}%
\pgfpathlineto{\pgfqpoint{5.800000in}{119.188302in}}%
\pgfpathlineto{\pgfqpoint{5.800000in}{119.188302in}}%
\pgfpathclose%
\pgfusepath{fill}%
\end{pgfscope}%
\begin{pgfscope}%
\pgfpathrectangle{\pgfqpoint{5.800000in}{0.720000in}}{\pgfqpoint{1.400000in}{4.620000in}}%
\pgfusepath{clip}%
\pgfsetbuttcap%
\pgfsetmiterjoin%
\definecolor{currentfill}{rgb}{0.121569,0.466667,0.705882}%
\pgfsetfillcolor{currentfill}%
\pgfsetlinewidth{0.000000pt}%
\definecolor{currentstroke}{rgb}{0.000000,0.000000,0.000000}%
\pgfsetstrokecolor{currentstroke}%
\pgfsetstrokeopacity{0.000000}%
\pgfsetdash{}{0pt}%
\pgfpathmoveto{\pgfqpoint{5.800000in}{119.188302in}}%
\pgfpathlineto{\pgfqpoint{5.800000in}{119.188302in}}%
\pgfpathlineto{\pgfqpoint{5.800000in}{119.291139in}}%
\pgfpathlineto{\pgfqpoint{5.800000in}{119.291139in}}%
\pgfpathclose%
\pgfusepath{fill}%
\end{pgfscope}%
\begin{pgfscope}%
\pgfpathrectangle{\pgfqpoint{5.800000in}{0.720000in}}{\pgfqpoint{1.400000in}{4.620000in}}%
\pgfusepath{clip}%
\pgfsetbuttcap%
\pgfsetmiterjoin%
\definecolor{currentfill}{rgb}{0.121569,0.466667,0.705882}%
\pgfsetfillcolor{currentfill}%
\pgfsetlinewidth{0.000000pt}%
\definecolor{currentstroke}{rgb}{0.000000,0.000000,0.000000}%
\pgfsetstrokecolor{currentstroke}%
\pgfsetstrokeopacity{0.000000}%
\pgfsetdash{}{0pt}%
\pgfpathmoveto{\pgfqpoint{5.800000in}{119.291139in}}%
\pgfpathlineto{\pgfqpoint{5.800000in}{119.291139in}}%
\pgfpathlineto{\pgfqpoint{5.800000in}{119.393976in}}%
\pgfpathlineto{\pgfqpoint{5.800000in}{119.393976in}}%
\pgfpathclose%
\pgfusepath{fill}%
\end{pgfscope}%
\begin{pgfscope}%
\pgfpathrectangle{\pgfqpoint{5.800000in}{0.720000in}}{\pgfqpoint{1.400000in}{4.620000in}}%
\pgfusepath{clip}%
\pgfsetbuttcap%
\pgfsetmiterjoin%
\definecolor{currentfill}{rgb}{0.121569,0.466667,0.705882}%
\pgfsetfillcolor{currentfill}%
\pgfsetlinewidth{0.000000pt}%
\definecolor{currentstroke}{rgb}{0.000000,0.000000,0.000000}%
\pgfsetstrokecolor{currentstroke}%
\pgfsetstrokeopacity{0.000000}%
\pgfsetdash{}{0pt}%
\pgfpathmoveto{\pgfqpoint{5.800000in}{119.393976in}}%
\pgfpathlineto{\pgfqpoint{5.800000in}{119.393976in}}%
\pgfpathlineto{\pgfqpoint{5.800000in}{119.496813in}}%
\pgfpathlineto{\pgfqpoint{5.800000in}{119.496813in}}%
\pgfpathclose%
\pgfusepath{fill}%
\end{pgfscope}%
\begin{pgfscope}%
\pgfpathrectangle{\pgfqpoint{5.800000in}{0.720000in}}{\pgfqpoint{1.400000in}{4.620000in}}%
\pgfusepath{clip}%
\pgfsetbuttcap%
\pgfsetmiterjoin%
\definecolor{currentfill}{rgb}{0.121569,0.466667,0.705882}%
\pgfsetfillcolor{currentfill}%
\pgfsetlinewidth{0.000000pt}%
\definecolor{currentstroke}{rgb}{0.000000,0.000000,0.000000}%
\pgfsetstrokecolor{currentstroke}%
\pgfsetstrokeopacity{0.000000}%
\pgfsetdash{}{0pt}%
\pgfpathmoveto{\pgfqpoint{5.800000in}{119.496813in}}%
\pgfpathlineto{\pgfqpoint{5.800000in}{119.496813in}}%
\pgfpathlineto{\pgfqpoint{5.800000in}{119.599650in}}%
\pgfpathlineto{\pgfqpoint{5.800000in}{119.599650in}}%
\pgfpathclose%
\pgfusepath{fill}%
\end{pgfscope}%
\begin{pgfscope}%
\pgfpathrectangle{\pgfqpoint{5.800000in}{0.720000in}}{\pgfqpoint{1.400000in}{4.620000in}}%
\pgfusepath{clip}%
\pgfsetbuttcap%
\pgfsetmiterjoin%
\definecolor{currentfill}{rgb}{0.121569,0.466667,0.705882}%
\pgfsetfillcolor{currentfill}%
\pgfsetlinewidth{0.000000pt}%
\definecolor{currentstroke}{rgb}{0.000000,0.000000,0.000000}%
\pgfsetstrokecolor{currentstroke}%
\pgfsetstrokeopacity{0.000000}%
\pgfsetdash{}{0pt}%
\pgfpathmoveto{\pgfqpoint{5.800000in}{119.599650in}}%
\pgfpathlineto{\pgfqpoint{5.800000in}{119.599650in}}%
\pgfpathlineto{\pgfqpoint{5.800000in}{119.702487in}}%
\pgfpathlineto{\pgfqpoint{5.800000in}{119.702487in}}%
\pgfpathclose%
\pgfusepath{fill}%
\end{pgfscope}%
\begin{pgfscope}%
\pgfpathrectangle{\pgfqpoint{5.800000in}{0.720000in}}{\pgfqpoint{1.400000in}{4.620000in}}%
\pgfusepath{clip}%
\pgfsetbuttcap%
\pgfsetmiterjoin%
\definecolor{currentfill}{rgb}{0.121569,0.466667,0.705882}%
\pgfsetfillcolor{currentfill}%
\pgfsetlinewidth{0.000000pt}%
\definecolor{currentstroke}{rgb}{0.000000,0.000000,0.000000}%
\pgfsetstrokecolor{currentstroke}%
\pgfsetstrokeopacity{0.000000}%
\pgfsetdash{}{0pt}%
\pgfpathmoveto{\pgfqpoint{5.800000in}{119.702487in}}%
\pgfpathlineto{\pgfqpoint{5.800000in}{119.702487in}}%
\pgfpathlineto{\pgfqpoint{5.800000in}{119.805324in}}%
\pgfpathlineto{\pgfqpoint{5.800000in}{119.805324in}}%
\pgfpathclose%
\pgfusepath{fill}%
\end{pgfscope}%
\begin{pgfscope}%
\pgfpathrectangle{\pgfqpoint{5.800000in}{0.720000in}}{\pgfqpoint{1.400000in}{4.620000in}}%
\pgfusepath{clip}%
\pgfsetbuttcap%
\pgfsetmiterjoin%
\definecolor{currentfill}{rgb}{0.121569,0.466667,0.705882}%
\pgfsetfillcolor{currentfill}%
\pgfsetlinewidth{0.000000pt}%
\definecolor{currentstroke}{rgb}{0.000000,0.000000,0.000000}%
\pgfsetstrokecolor{currentstroke}%
\pgfsetstrokeopacity{0.000000}%
\pgfsetdash{}{0pt}%
\pgfpathmoveto{\pgfqpoint{5.800000in}{119.805324in}}%
\pgfpathlineto{\pgfqpoint{5.800000in}{119.805324in}}%
\pgfpathlineto{\pgfqpoint{5.800000in}{119.908162in}}%
\pgfpathlineto{\pgfqpoint{5.800000in}{119.908162in}}%
\pgfpathclose%
\pgfusepath{fill}%
\end{pgfscope}%
\begin{pgfscope}%
\pgfpathrectangle{\pgfqpoint{5.800000in}{0.720000in}}{\pgfqpoint{1.400000in}{4.620000in}}%
\pgfusepath{clip}%
\pgfsetbuttcap%
\pgfsetmiterjoin%
\definecolor{currentfill}{rgb}{0.121569,0.466667,0.705882}%
\pgfsetfillcolor{currentfill}%
\pgfsetlinewidth{0.000000pt}%
\definecolor{currentstroke}{rgb}{0.000000,0.000000,0.000000}%
\pgfsetstrokecolor{currentstroke}%
\pgfsetstrokeopacity{0.000000}%
\pgfsetdash{}{0pt}%
\pgfpathmoveto{\pgfqpoint{5.800000in}{119.908162in}}%
\pgfpathlineto{\pgfqpoint{5.800000in}{119.908162in}}%
\pgfpathlineto{\pgfqpoint{5.800000in}{120.010999in}}%
\pgfpathlineto{\pgfqpoint{5.800000in}{120.010999in}}%
\pgfpathclose%
\pgfusepath{fill}%
\end{pgfscope}%
\begin{pgfscope}%
\pgfpathrectangle{\pgfqpoint{5.800000in}{0.720000in}}{\pgfqpoint{1.400000in}{4.620000in}}%
\pgfusepath{clip}%
\pgfsetbuttcap%
\pgfsetmiterjoin%
\definecolor{currentfill}{rgb}{0.121569,0.466667,0.705882}%
\pgfsetfillcolor{currentfill}%
\pgfsetlinewidth{0.000000pt}%
\definecolor{currentstroke}{rgb}{0.000000,0.000000,0.000000}%
\pgfsetstrokecolor{currentstroke}%
\pgfsetstrokeopacity{0.000000}%
\pgfsetdash{}{0pt}%
\pgfpathmoveto{\pgfqpoint{5.800000in}{120.010999in}}%
\pgfpathlineto{\pgfqpoint{5.800000in}{120.010999in}}%
\pgfpathlineto{\pgfqpoint{5.800000in}{120.113836in}}%
\pgfpathlineto{\pgfqpoint{5.800000in}{120.113836in}}%
\pgfpathclose%
\pgfusepath{fill}%
\end{pgfscope}%
\begin{pgfscope}%
\pgfpathrectangle{\pgfqpoint{5.800000in}{0.720000in}}{\pgfqpoint{1.400000in}{4.620000in}}%
\pgfusepath{clip}%
\pgfsetbuttcap%
\pgfsetmiterjoin%
\definecolor{currentfill}{rgb}{0.121569,0.466667,0.705882}%
\pgfsetfillcolor{currentfill}%
\pgfsetlinewidth{0.000000pt}%
\definecolor{currentstroke}{rgb}{0.000000,0.000000,0.000000}%
\pgfsetstrokecolor{currentstroke}%
\pgfsetstrokeopacity{0.000000}%
\pgfsetdash{}{0pt}%
\pgfpathmoveto{\pgfqpoint{5.800000in}{120.113836in}}%
\pgfpathlineto{\pgfqpoint{5.800000in}{120.113836in}}%
\pgfpathlineto{\pgfqpoint{5.800000in}{120.216673in}}%
\pgfpathlineto{\pgfqpoint{5.800000in}{120.216673in}}%
\pgfpathclose%
\pgfusepath{fill}%
\end{pgfscope}%
\begin{pgfscope}%
\pgfpathrectangle{\pgfqpoint{5.800000in}{0.720000in}}{\pgfqpoint{1.400000in}{4.620000in}}%
\pgfusepath{clip}%
\pgfsetbuttcap%
\pgfsetmiterjoin%
\definecolor{currentfill}{rgb}{0.121569,0.466667,0.705882}%
\pgfsetfillcolor{currentfill}%
\pgfsetlinewidth{0.000000pt}%
\definecolor{currentstroke}{rgb}{0.000000,0.000000,0.000000}%
\pgfsetstrokecolor{currentstroke}%
\pgfsetstrokeopacity{0.000000}%
\pgfsetdash{}{0pt}%
\pgfpathmoveto{\pgfqpoint{5.800000in}{120.216673in}}%
\pgfpathlineto{\pgfqpoint{5.800000in}{120.216673in}}%
\pgfpathlineto{\pgfqpoint{5.800000in}{120.319510in}}%
\pgfpathlineto{\pgfqpoint{5.800000in}{120.319510in}}%
\pgfpathclose%
\pgfusepath{fill}%
\end{pgfscope}%
\begin{pgfscope}%
\pgfpathrectangle{\pgfqpoint{5.800000in}{0.720000in}}{\pgfqpoint{1.400000in}{4.620000in}}%
\pgfusepath{clip}%
\pgfsetbuttcap%
\pgfsetmiterjoin%
\definecolor{currentfill}{rgb}{0.121569,0.466667,0.705882}%
\pgfsetfillcolor{currentfill}%
\pgfsetlinewidth{0.000000pt}%
\definecolor{currentstroke}{rgb}{0.000000,0.000000,0.000000}%
\pgfsetstrokecolor{currentstroke}%
\pgfsetstrokeopacity{0.000000}%
\pgfsetdash{}{0pt}%
\pgfpathmoveto{\pgfqpoint{5.800000in}{120.319510in}}%
\pgfpathlineto{\pgfqpoint{5.800000in}{120.319510in}}%
\pgfpathlineto{\pgfqpoint{5.800000in}{120.422347in}}%
\pgfpathlineto{\pgfqpoint{5.800000in}{120.422347in}}%
\pgfpathclose%
\pgfusepath{fill}%
\end{pgfscope}%
\begin{pgfscope}%
\pgfpathrectangle{\pgfqpoint{5.800000in}{0.720000in}}{\pgfqpoint{1.400000in}{4.620000in}}%
\pgfusepath{clip}%
\pgfsetbuttcap%
\pgfsetmiterjoin%
\definecolor{currentfill}{rgb}{0.121569,0.466667,0.705882}%
\pgfsetfillcolor{currentfill}%
\pgfsetlinewidth{0.000000pt}%
\definecolor{currentstroke}{rgb}{0.000000,0.000000,0.000000}%
\pgfsetstrokecolor{currentstroke}%
\pgfsetstrokeopacity{0.000000}%
\pgfsetdash{}{0pt}%
\pgfpathmoveto{\pgfqpoint{5.800000in}{120.422347in}}%
\pgfpathlineto{\pgfqpoint{5.800000in}{120.422347in}}%
\pgfpathlineto{\pgfqpoint{5.800000in}{120.525184in}}%
\pgfpathlineto{\pgfqpoint{5.800000in}{120.525184in}}%
\pgfpathclose%
\pgfusepath{fill}%
\end{pgfscope}%
\begin{pgfscope}%
\pgfpathrectangle{\pgfqpoint{5.800000in}{0.720000in}}{\pgfqpoint{1.400000in}{4.620000in}}%
\pgfusepath{clip}%
\pgfsetbuttcap%
\pgfsetmiterjoin%
\definecolor{currentfill}{rgb}{0.121569,0.466667,0.705882}%
\pgfsetfillcolor{currentfill}%
\pgfsetlinewidth{0.000000pt}%
\definecolor{currentstroke}{rgb}{0.000000,0.000000,0.000000}%
\pgfsetstrokecolor{currentstroke}%
\pgfsetstrokeopacity{0.000000}%
\pgfsetdash{}{0pt}%
\pgfpathmoveto{\pgfqpoint{5.800000in}{120.525184in}}%
\pgfpathlineto{\pgfqpoint{5.800000in}{120.525184in}}%
\pgfpathlineto{\pgfqpoint{5.800000in}{120.628021in}}%
\pgfpathlineto{\pgfqpoint{5.800000in}{120.628021in}}%
\pgfpathclose%
\pgfusepath{fill}%
\end{pgfscope}%
\begin{pgfscope}%
\pgfpathrectangle{\pgfqpoint{5.800000in}{0.720000in}}{\pgfqpoint{1.400000in}{4.620000in}}%
\pgfusepath{clip}%
\pgfsetbuttcap%
\pgfsetmiterjoin%
\definecolor{currentfill}{rgb}{0.121569,0.466667,0.705882}%
\pgfsetfillcolor{currentfill}%
\pgfsetlinewidth{0.000000pt}%
\definecolor{currentstroke}{rgb}{0.000000,0.000000,0.000000}%
\pgfsetstrokecolor{currentstroke}%
\pgfsetstrokeopacity{0.000000}%
\pgfsetdash{}{0pt}%
\pgfpathmoveto{\pgfqpoint{5.800000in}{120.628021in}}%
\pgfpathlineto{\pgfqpoint{5.800000in}{120.628021in}}%
\pgfpathlineto{\pgfqpoint{5.800000in}{120.730858in}}%
\pgfpathlineto{\pgfqpoint{5.800000in}{120.730858in}}%
\pgfpathclose%
\pgfusepath{fill}%
\end{pgfscope}%
\begin{pgfscope}%
\pgfpathrectangle{\pgfqpoint{5.800000in}{0.720000in}}{\pgfqpoint{1.400000in}{4.620000in}}%
\pgfusepath{clip}%
\pgfsetbuttcap%
\pgfsetmiterjoin%
\definecolor{currentfill}{rgb}{0.121569,0.466667,0.705882}%
\pgfsetfillcolor{currentfill}%
\pgfsetlinewidth{0.000000pt}%
\definecolor{currentstroke}{rgb}{0.000000,0.000000,0.000000}%
\pgfsetstrokecolor{currentstroke}%
\pgfsetstrokeopacity{0.000000}%
\pgfsetdash{}{0pt}%
\pgfpathmoveto{\pgfqpoint{5.800000in}{120.730858in}}%
\pgfpathlineto{\pgfqpoint{5.800000in}{120.730858in}}%
\pgfpathlineto{\pgfqpoint{5.800000in}{120.833695in}}%
\pgfpathlineto{\pgfqpoint{5.800000in}{120.833695in}}%
\pgfpathclose%
\pgfusepath{fill}%
\end{pgfscope}%
\begin{pgfscope}%
\pgfpathrectangle{\pgfqpoint{5.800000in}{0.720000in}}{\pgfqpoint{1.400000in}{4.620000in}}%
\pgfusepath{clip}%
\pgfsetbuttcap%
\pgfsetmiterjoin%
\definecolor{currentfill}{rgb}{0.121569,0.466667,0.705882}%
\pgfsetfillcolor{currentfill}%
\pgfsetlinewidth{0.000000pt}%
\definecolor{currentstroke}{rgb}{0.000000,0.000000,0.000000}%
\pgfsetstrokecolor{currentstroke}%
\pgfsetstrokeopacity{0.000000}%
\pgfsetdash{}{0pt}%
\pgfpathmoveto{\pgfqpoint{5.800000in}{120.833695in}}%
\pgfpathlineto{\pgfqpoint{5.800000in}{120.833695in}}%
\pgfpathlineto{\pgfqpoint{5.800000in}{120.936532in}}%
\pgfpathlineto{\pgfqpoint{5.800000in}{120.936532in}}%
\pgfpathclose%
\pgfusepath{fill}%
\end{pgfscope}%
\begin{pgfscope}%
\pgfpathrectangle{\pgfqpoint{5.800000in}{0.720000in}}{\pgfqpoint{1.400000in}{4.620000in}}%
\pgfusepath{clip}%
\pgfsetbuttcap%
\pgfsetmiterjoin%
\definecolor{currentfill}{rgb}{0.121569,0.466667,0.705882}%
\pgfsetfillcolor{currentfill}%
\pgfsetlinewidth{0.000000pt}%
\definecolor{currentstroke}{rgb}{0.000000,0.000000,0.000000}%
\pgfsetstrokecolor{currentstroke}%
\pgfsetstrokeopacity{0.000000}%
\pgfsetdash{}{0pt}%
\pgfpathmoveto{\pgfqpoint{5.800000in}{120.936532in}}%
\pgfpathlineto{\pgfqpoint{5.800000in}{120.936532in}}%
\pgfpathlineto{\pgfqpoint{5.800000in}{121.039369in}}%
\pgfpathlineto{\pgfqpoint{5.800000in}{121.039369in}}%
\pgfpathclose%
\pgfusepath{fill}%
\end{pgfscope}%
\begin{pgfscope}%
\pgfpathrectangle{\pgfqpoint{5.800000in}{0.720000in}}{\pgfqpoint{1.400000in}{4.620000in}}%
\pgfusepath{clip}%
\pgfsetbuttcap%
\pgfsetmiterjoin%
\definecolor{currentfill}{rgb}{0.121569,0.466667,0.705882}%
\pgfsetfillcolor{currentfill}%
\pgfsetlinewidth{0.000000pt}%
\definecolor{currentstroke}{rgb}{0.000000,0.000000,0.000000}%
\pgfsetstrokecolor{currentstroke}%
\pgfsetstrokeopacity{0.000000}%
\pgfsetdash{}{0pt}%
\pgfpathmoveto{\pgfqpoint{5.800000in}{121.039369in}}%
\pgfpathlineto{\pgfqpoint{5.800000in}{121.039369in}}%
\pgfpathlineto{\pgfqpoint{5.800000in}{121.142206in}}%
\pgfpathlineto{\pgfqpoint{5.800000in}{121.142206in}}%
\pgfpathclose%
\pgfusepath{fill}%
\end{pgfscope}%
\begin{pgfscope}%
\pgfpathrectangle{\pgfqpoint{5.800000in}{0.720000in}}{\pgfqpoint{1.400000in}{4.620000in}}%
\pgfusepath{clip}%
\pgfsetbuttcap%
\pgfsetmiterjoin%
\definecolor{currentfill}{rgb}{0.121569,0.466667,0.705882}%
\pgfsetfillcolor{currentfill}%
\pgfsetlinewidth{0.000000pt}%
\definecolor{currentstroke}{rgb}{0.000000,0.000000,0.000000}%
\pgfsetstrokecolor{currentstroke}%
\pgfsetstrokeopacity{0.000000}%
\pgfsetdash{}{0pt}%
\pgfpathmoveto{\pgfqpoint{5.800000in}{121.142206in}}%
\pgfpathlineto{\pgfqpoint{5.800000in}{121.142206in}}%
\pgfpathlineto{\pgfqpoint{5.800000in}{121.245043in}}%
\pgfpathlineto{\pgfqpoint{5.800000in}{121.245043in}}%
\pgfpathclose%
\pgfusepath{fill}%
\end{pgfscope}%
\begin{pgfscope}%
\pgfpathrectangle{\pgfqpoint{5.800000in}{0.720000in}}{\pgfqpoint{1.400000in}{4.620000in}}%
\pgfusepath{clip}%
\pgfsetbuttcap%
\pgfsetmiterjoin%
\definecolor{currentfill}{rgb}{0.121569,0.466667,0.705882}%
\pgfsetfillcolor{currentfill}%
\pgfsetlinewidth{0.000000pt}%
\definecolor{currentstroke}{rgb}{0.000000,0.000000,0.000000}%
\pgfsetstrokecolor{currentstroke}%
\pgfsetstrokeopacity{0.000000}%
\pgfsetdash{}{0pt}%
\pgfpathmoveto{\pgfqpoint{5.800000in}{121.245043in}}%
\pgfpathlineto{\pgfqpoint{5.800000in}{121.245043in}}%
\pgfpathlineto{\pgfqpoint{5.800000in}{121.347881in}}%
\pgfpathlineto{\pgfqpoint{5.800000in}{121.347881in}}%
\pgfpathclose%
\pgfusepath{fill}%
\end{pgfscope}%
\begin{pgfscope}%
\pgfpathrectangle{\pgfqpoint{5.800000in}{0.720000in}}{\pgfqpoint{1.400000in}{4.620000in}}%
\pgfusepath{clip}%
\pgfsetbuttcap%
\pgfsetmiterjoin%
\definecolor{currentfill}{rgb}{0.121569,0.466667,0.705882}%
\pgfsetfillcolor{currentfill}%
\pgfsetlinewidth{0.000000pt}%
\definecolor{currentstroke}{rgb}{0.000000,0.000000,0.000000}%
\pgfsetstrokecolor{currentstroke}%
\pgfsetstrokeopacity{0.000000}%
\pgfsetdash{}{0pt}%
\pgfpathmoveto{\pgfqpoint{5.800000in}{121.347881in}}%
\pgfpathlineto{\pgfqpoint{5.800000in}{121.347881in}}%
\pgfpathlineto{\pgfqpoint{5.800000in}{121.450718in}}%
\pgfpathlineto{\pgfqpoint{5.800000in}{121.450718in}}%
\pgfpathclose%
\pgfusepath{fill}%
\end{pgfscope}%
\begin{pgfscope}%
\pgfpathrectangle{\pgfqpoint{5.800000in}{0.720000in}}{\pgfqpoint{1.400000in}{4.620000in}}%
\pgfusepath{clip}%
\pgfsetbuttcap%
\pgfsetmiterjoin%
\definecolor{currentfill}{rgb}{0.121569,0.466667,0.705882}%
\pgfsetfillcolor{currentfill}%
\pgfsetlinewidth{0.000000pt}%
\definecolor{currentstroke}{rgb}{0.000000,0.000000,0.000000}%
\pgfsetstrokecolor{currentstroke}%
\pgfsetstrokeopacity{0.000000}%
\pgfsetdash{}{0pt}%
\pgfpathmoveto{\pgfqpoint{5.800000in}{121.450718in}}%
\pgfpathlineto{\pgfqpoint{5.800000in}{121.450718in}}%
\pgfpathlineto{\pgfqpoint{5.800000in}{121.553555in}}%
\pgfpathlineto{\pgfqpoint{5.800000in}{121.553555in}}%
\pgfpathclose%
\pgfusepath{fill}%
\end{pgfscope}%
\begin{pgfscope}%
\pgfpathrectangle{\pgfqpoint{5.800000in}{0.720000in}}{\pgfqpoint{1.400000in}{4.620000in}}%
\pgfusepath{clip}%
\pgfsetbuttcap%
\pgfsetmiterjoin%
\definecolor{currentfill}{rgb}{0.121569,0.466667,0.705882}%
\pgfsetfillcolor{currentfill}%
\pgfsetlinewidth{0.000000pt}%
\definecolor{currentstroke}{rgb}{0.000000,0.000000,0.000000}%
\pgfsetstrokecolor{currentstroke}%
\pgfsetstrokeopacity{0.000000}%
\pgfsetdash{}{0pt}%
\pgfpathmoveto{\pgfqpoint{5.800000in}{121.553555in}}%
\pgfpathlineto{\pgfqpoint{5.800000in}{121.553555in}}%
\pgfpathlineto{\pgfqpoint{5.800000in}{121.656392in}}%
\pgfpathlineto{\pgfqpoint{5.800000in}{121.656392in}}%
\pgfpathclose%
\pgfusepath{fill}%
\end{pgfscope}%
\begin{pgfscope}%
\pgfpathrectangle{\pgfqpoint{5.800000in}{0.720000in}}{\pgfqpoint{1.400000in}{4.620000in}}%
\pgfusepath{clip}%
\pgfsetbuttcap%
\pgfsetmiterjoin%
\definecolor{currentfill}{rgb}{0.121569,0.466667,0.705882}%
\pgfsetfillcolor{currentfill}%
\pgfsetlinewidth{0.000000pt}%
\definecolor{currentstroke}{rgb}{0.000000,0.000000,0.000000}%
\pgfsetstrokecolor{currentstroke}%
\pgfsetstrokeopacity{0.000000}%
\pgfsetdash{}{0pt}%
\pgfpathmoveto{\pgfqpoint{5.800000in}{121.656392in}}%
\pgfpathlineto{\pgfqpoint{5.800000in}{121.656392in}}%
\pgfpathlineto{\pgfqpoint{5.800000in}{121.759229in}}%
\pgfpathlineto{\pgfqpoint{5.800000in}{121.759229in}}%
\pgfpathclose%
\pgfusepath{fill}%
\end{pgfscope}%
\begin{pgfscope}%
\pgfpathrectangle{\pgfqpoint{5.800000in}{0.720000in}}{\pgfqpoint{1.400000in}{4.620000in}}%
\pgfusepath{clip}%
\pgfsetbuttcap%
\pgfsetmiterjoin%
\definecolor{currentfill}{rgb}{0.121569,0.466667,0.705882}%
\pgfsetfillcolor{currentfill}%
\pgfsetlinewidth{0.000000pt}%
\definecolor{currentstroke}{rgb}{0.000000,0.000000,0.000000}%
\pgfsetstrokecolor{currentstroke}%
\pgfsetstrokeopacity{0.000000}%
\pgfsetdash{}{0pt}%
\pgfpathmoveto{\pgfqpoint{5.800000in}{121.759229in}}%
\pgfpathlineto{\pgfqpoint{5.800000in}{121.759229in}}%
\pgfpathlineto{\pgfqpoint{5.800000in}{121.862066in}}%
\pgfpathlineto{\pgfqpoint{5.800000in}{121.862066in}}%
\pgfpathclose%
\pgfusepath{fill}%
\end{pgfscope}%
\begin{pgfscope}%
\pgfpathrectangle{\pgfqpoint{5.800000in}{0.720000in}}{\pgfqpoint{1.400000in}{4.620000in}}%
\pgfusepath{clip}%
\pgfsetbuttcap%
\pgfsetmiterjoin%
\definecolor{currentfill}{rgb}{0.121569,0.466667,0.705882}%
\pgfsetfillcolor{currentfill}%
\pgfsetlinewidth{0.000000pt}%
\definecolor{currentstroke}{rgb}{0.000000,0.000000,0.000000}%
\pgfsetstrokecolor{currentstroke}%
\pgfsetstrokeopacity{0.000000}%
\pgfsetdash{}{0pt}%
\pgfpathmoveto{\pgfqpoint{5.800000in}{121.862066in}}%
\pgfpathlineto{\pgfqpoint{5.800000in}{121.862066in}}%
\pgfpathlineto{\pgfqpoint{5.800000in}{121.964903in}}%
\pgfpathlineto{\pgfqpoint{5.800000in}{121.964903in}}%
\pgfpathclose%
\pgfusepath{fill}%
\end{pgfscope}%
\begin{pgfscope}%
\pgfpathrectangle{\pgfqpoint{5.800000in}{0.720000in}}{\pgfqpoint{1.400000in}{4.620000in}}%
\pgfusepath{clip}%
\pgfsetbuttcap%
\pgfsetmiterjoin%
\definecolor{currentfill}{rgb}{0.121569,0.466667,0.705882}%
\pgfsetfillcolor{currentfill}%
\pgfsetlinewidth{0.000000pt}%
\definecolor{currentstroke}{rgb}{0.000000,0.000000,0.000000}%
\pgfsetstrokecolor{currentstroke}%
\pgfsetstrokeopacity{0.000000}%
\pgfsetdash{}{0pt}%
\pgfpathmoveto{\pgfqpoint{5.800000in}{121.964903in}}%
\pgfpathlineto{\pgfqpoint{5.800000in}{121.964903in}}%
\pgfpathlineto{\pgfqpoint{5.800000in}{122.067740in}}%
\pgfpathlineto{\pgfqpoint{5.800000in}{122.067740in}}%
\pgfpathclose%
\pgfusepath{fill}%
\end{pgfscope}%
\begin{pgfscope}%
\pgfpathrectangle{\pgfqpoint{5.800000in}{0.720000in}}{\pgfqpoint{1.400000in}{4.620000in}}%
\pgfusepath{clip}%
\pgfsetbuttcap%
\pgfsetmiterjoin%
\definecolor{currentfill}{rgb}{0.121569,0.466667,0.705882}%
\pgfsetfillcolor{currentfill}%
\pgfsetlinewidth{0.000000pt}%
\definecolor{currentstroke}{rgb}{0.000000,0.000000,0.000000}%
\pgfsetstrokecolor{currentstroke}%
\pgfsetstrokeopacity{0.000000}%
\pgfsetdash{}{0pt}%
\pgfpathmoveto{\pgfqpoint{5.800000in}{122.067740in}}%
\pgfpathlineto{\pgfqpoint{5.800000in}{122.067740in}}%
\pgfpathlineto{\pgfqpoint{5.800000in}{122.170577in}}%
\pgfpathlineto{\pgfqpoint{5.800000in}{122.170577in}}%
\pgfpathclose%
\pgfusepath{fill}%
\end{pgfscope}%
\begin{pgfscope}%
\pgfpathrectangle{\pgfqpoint{5.800000in}{0.720000in}}{\pgfqpoint{1.400000in}{4.620000in}}%
\pgfusepath{clip}%
\pgfsetbuttcap%
\pgfsetmiterjoin%
\definecolor{currentfill}{rgb}{0.121569,0.466667,0.705882}%
\pgfsetfillcolor{currentfill}%
\pgfsetlinewidth{0.000000pt}%
\definecolor{currentstroke}{rgb}{0.000000,0.000000,0.000000}%
\pgfsetstrokecolor{currentstroke}%
\pgfsetstrokeopacity{0.000000}%
\pgfsetdash{}{0pt}%
\pgfpathmoveto{\pgfqpoint{5.800000in}{122.170577in}}%
\pgfpathlineto{\pgfqpoint{5.800000in}{122.170577in}}%
\pgfpathlineto{\pgfqpoint{5.800000in}{122.273414in}}%
\pgfpathlineto{\pgfqpoint{5.800000in}{122.273414in}}%
\pgfpathclose%
\pgfusepath{fill}%
\end{pgfscope}%
\begin{pgfscope}%
\pgfpathrectangle{\pgfqpoint{5.800000in}{0.720000in}}{\pgfqpoint{1.400000in}{4.620000in}}%
\pgfusepath{clip}%
\pgfsetbuttcap%
\pgfsetmiterjoin%
\definecolor{currentfill}{rgb}{0.121569,0.466667,0.705882}%
\pgfsetfillcolor{currentfill}%
\pgfsetlinewidth{0.000000pt}%
\definecolor{currentstroke}{rgb}{0.000000,0.000000,0.000000}%
\pgfsetstrokecolor{currentstroke}%
\pgfsetstrokeopacity{0.000000}%
\pgfsetdash{}{0pt}%
\pgfpathmoveto{\pgfqpoint{5.800000in}{122.273414in}}%
\pgfpathlineto{\pgfqpoint{5.800000in}{122.273414in}}%
\pgfpathlineto{\pgfqpoint{5.800000in}{122.376251in}}%
\pgfpathlineto{\pgfqpoint{5.800000in}{122.376251in}}%
\pgfpathclose%
\pgfusepath{fill}%
\end{pgfscope}%
\begin{pgfscope}%
\pgfpathrectangle{\pgfqpoint{5.800000in}{0.720000in}}{\pgfqpoint{1.400000in}{4.620000in}}%
\pgfusepath{clip}%
\pgfsetbuttcap%
\pgfsetmiterjoin%
\definecolor{currentfill}{rgb}{0.121569,0.466667,0.705882}%
\pgfsetfillcolor{currentfill}%
\pgfsetlinewidth{0.000000pt}%
\definecolor{currentstroke}{rgb}{0.000000,0.000000,0.000000}%
\pgfsetstrokecolor{currentstroke}%
\pgfsetstrokeopacity{0.000000}%
\pgfsetdash{}{0pt}%
\pgfpathmoveto{\pgfqpoint{5.800000in}{122.376251in}}%
\pgfpathlineto{\pgfqpoint{5.800000in}{122.376251in}}%
\pgfpathlineto{\pgfqpoint{5.800000in}{122.479088in}}%
\pgfpathlineto{\pgfqpoint{5.800000in}{122.479088in}}%
\pgfpathclose%
\pgfusepath{fill}%
\end{pgfscope}%
\begin{pgfscope}%
\pgfpathrectangle{\pgfqpoint{5.800000in}{0.720000in}}{\pgfqpoint{1.400000in}{4.620000in}}%
\pgfusepath{clip}%
\pgfsetbuttcap%
\pgfsetmiterjoin%
\definecolor{currentfill}{rgb}{0.121569,0.466667,0.705882}%
\pgfsetfillcolor{currentfill}%
\pgfsetlinewidth{0.000000pt}%
\definecolor{currentstroke}{rgb}{0.000000,0.000000,0.000000}%
\pgfsetstrokecolor{currentstroke}%
\pgfsetstrokeopacity{0.000000}%
\pgfsetdash{}{0pt}%
\pgfpathmoveto{\pgfqpoint{5.800000in}{122.479088in}}%
\pgfpathlineto{\pgfqpoint{5.800000in}{122.479088in}}%
\pgfpathlineto{\pgfqpoint{5.800000in}{122.581925in}}%
\pgfpathlineto{\pgfqpoint{5.800000in}{122.581925in}}%
\pgfpathclose%
\pgfusepath{fill}%
\end{pgfscope}%
\begin{pgfscope}%
\pgfpathrectangle{\pgfqpoint{5.800000in}{0.720000in}}{\pgfqpoint{1.400000in}{4.620000in}}%
\pgfusepath{clip}%
\pgfsetbuttcap%
\pgfsetmiterjoin%
\definecolor{currentfill}{rgb}{0.121569,0.466667,0.705882}%
\pgfsetfillcolor{currentfill}%
\pgfsetlinewidth{0.000000pt}%
\definecolor{currentstroke}{rgb}{0.000000,0.000000,0.000000}%
\pgfsetstrokecolor{currentstroke}%
\pgfsetstrokeopacity{0.000000}%
\pgfsetdash{}{0pt}%
\pgfpathmoveto{\pgfqpoint{5.800000in}{122.581925in}}%
\pgfpathlineto{\pgfqpoint{5.800000in}{122.581925in}}%
\pgfpathlineto{\pgfqpoint{5.800000in}{122.684762in}}%
\pgfpathlineto{\pgfqpoint{5.800000in}{122.684762in}}%
\pgfpathclose%
\pgfusepath{fill}%
\end{pgfscope}%
\begin{pgfscope}%
\pgfpathrectangle{\pgfqpoint{5.800000in}{0.720000in}}{\pgfqpoint{1.400000in}{4.620000in}}%
\pgfusepath{clip}%
\pgfsetbuttcap%
\pgfsetmiterjoin%
\definecolor{currentfill}{rgb}{0.121569,0.466667,0.705882}%
\pgfsetfillcolor{currentfill}%
\pgfsetlinewidth{0.000000pt}%
\definecolor{currentstroke}{rgb}{0.000000,0.000000,0.000000}%
\pgfsetstrokecolor{currentstroke}%
\pgfsetstrokeopacity{0.000000}%
\pgfsetdash{}{0pt}%
\pgfpathmoveto{\pgfqpoint{5.800000in}{122.684762in}}%
\pgfpathlineto{\pgfqpoint{5.800000in}{122.684762in}}%
\pgfpathlineto{\pgfqpoint{5.800000in}{122.787599in}}%
\pgfpathlineto{\pgfqpoint{5.800000in}{122.787599in}}%
\pgfpathclose%
\pgfusepath{fill}%
\end{pgfscope}%
\begin{pgfscope}%
\pgfpathrectangle{\pgfqpoint{5.800000in}{0.720000in}}{\pgfqpoint{1.400000in}{4.620000in}}%
\pgfusepath{clip}%
\pgfsetbuttcap%
\pgfsetmiterjoin%
\definecolor{currentfill}{rgb}{0.121569,0.466667,0.705882}%
\pgfsetfillcolor{currentfill}%
\pgfsetlinewidth{0.000000pt}%
\definecolor{currentstroke}{rgb}{0.000000,0.000000,0.000000}%
\pgfsetstrokecolor{currentstroke}%
\pgfsetstrokeopacity{0.000000}%
\pgfsetdash{}{0pt}%
\pgfpathmoveto{\pgfqpoint{5.800000in}{122.787599in}}%
\pgfpathlineto{\pgfqpoint{5.800000in}{122.787599in}}%
\pgfpathlineto{\pgfqpoint{5.800000in}{122.890437in}}%
\pgfpathlineto{\pgfqpoint{5.800000in}{122.890437in}}%
\pgfpathclose%
\pgfusepath{fill}%
\end{pgfscope}%
\begin{pgfscope}%
\pgfpathrectangle{\pgfqpoint{5.800000in}{0.720000in}}{\pgfqpoint{1.400000in}{4.620000in}}%
\pgfusepath{clip}%
\pgfsetbuttcap%
\pgfsetmiterjoin%
\definecolor{currentfill}{rgb}{0.121569,0.466667,0.705882}%
\pgfsetfillcolor{currentfill}%
\pgfsetlinewidth{0.000000pt}%
\definecolor{currentstroke}{rgb}{0.000000,0.000000,0.000000}%
\pgfsetstrokecolor{currentstroke}%
\pgfsetstrokeopacity{0.000000}%
\pgfsetdash{}{0pt}%
\pgfpathmoveto{\pgfqpoint{5.800000in}{122.890437in}}%
\pgfpathlineto{\pgfqpoint{5.800000in}{122.890437in}}%
\pgfpathlineto{\pgfqpoint{5.800000in}{122.993274in}}%
\pgfpathlineto{\pgfqpoint{5.800000in}{122.993274in}}%
\pgfpathclose%
\pgfusepath{fill}%
\end{pgfscope}%
\begin{pgfscope}%
\pgfpathrectangle{\pgfqpoint{5.800000in}{0.720000in}}{\pgfqpoint{1.400000in}{4.620000in}}%
\pgfusepath{clip}%
\pgfsetbuttcap%
\pgfsetmiterjoin%
\definecolor{currentfill}{rgb}{0.121569,0.466667,0.705882}%
\pgfsetfillcolor{currentfill}%
\pgfsetlinewidth{0.000000pt}%
\definecolor{currentstroke}{rgb}{0.000000,0.000000,0.000000}%
\pgfsetstrokecolor{currentstroke}%
\pgfsetstrokeopacity{0.000000}%
\pgfsetdash{}{0pt}%
\pgfpathmoveto{\pgfqpoint{5.800000in}{122.993274in}}%
\pgfpathlineto{\pgfqpoint{5.800000in}{122.993274in}}%
\pgfpathlineto{\pgfqpoint{5.800000in}{123.096111in}}%
\pgfpathlineto{\pgfqpoint{5.800000in}{123.096111in}}%
\pgfpathclose%
\pgfusepath{fill}%
\end{pgfscope}%
\begin{pgfscope}%
\pgfpathrectangle{\pgfqpoint{5.800000in}{0.720000in}}{\pgfqpoint{1.400000in}{4.620000in}}%
\pgfusepath{clip}%
\pgfsetbuttcap%
\pgfsetmiterjoin%
\definecolor{currentfill}{rgb}{0.121569,0.466667,0.705882}%
\pgfsetfillcolor{currentfill}%
\pgfsetlinewidth{0.000000pt}%
\definecolor{currentstroke}{rgb}{0.000000,0.000000,0.000000}%
\pgfsetstrokecolor{currentstroke}%
\pgfsetstrokeopacity{0.000000}%
\pgfsetdash{}{0pt}%
\pgfpathmoveto{\pgfqpoint{5.800000in}{123.096111in}}%
\pgfpathlineto{\pgfqpoint{5.800000in}{123.096111in}}%
\pgfpathlineto{\pgfqpoint{5.800000in}{123.198948in}}%
\pgfpathlineto{\pgfqpoint{5.800000in}{123.198948in}}%
\pgfpathclose%
\pgfusepath{fill}%
\end{pgfscope}%
\begin{pgfscope}%
\pgfpathrectangle{\pgfqpoint{5.800000in}{0.720000in}}{\pgfqpoint{1.400000in}{4.620000in}}%
\pgfusepath{clip}%
\pgfsetbuttcap%
\pgfsetmiterjoin%
\definecolor{currentfill}{rgb}{0.121569,0.466667,0.705882}%
\pgfsetfillcolor{currentfill}%
\pgfsetlinewidth{0.000000pt}%
\definecolor{currentstroke}{rgb}{0.000000,0.000000,0.000000}%
\pgfsetstrokecolor{currentstroke}%
\pgfsetstrokeopacity{0.000000}%
\pgfsetdash{}{0pt}%
\pgfpathmoveto{\pgfqpoint{5.800000in}{123.198948in}}%
\pgfpathlineto{\pgfqpoint{5.800000in}{123.198948in}}%
\pgfpathlineto{\pgfqpoint{5.800000in}{123.301785in}}%
\pgfpathlineto{\pgfqpoint{5.800000in}{123.301785in}}%
\pgfpathclose%
\pgfusepath{fill}%
\end{pgfscope}%
\begin{pgfscope}%
\pgfpathrectangle{\pgfqpoint{5.800000in}{0.720000in}}{\pgfqpoint{1.400000in}{4.620000in}}%
\pgfusepath{clip}%
\pgfsetbuttcap%
\pgfsetmiterjoin%
\definecolor{currentfill}{rgb}{0.121569,0.466667,0.705882}%
\pgfsetfillcolor{currentfill}%
\pgfsetlinewidth{0.000000pt}%
\definecolor{currentstroke}{rgb}{0.000000,0.000000,0.000000}%
\pgfsetstrokecolor{currentstroke}%
\pgfsetstrokeopacity{0.000000}%
\pgfsetdash{}{0pt}%
\pgfpathmoveto{\pgfqpoint{5.800000in}{123.301785in}}%
\pgfpathlineto{\pgfqpoint{5.800000in}{123.301785in}}%
\pgfpathlineto{\pgfqpoint{5.800000in}{123.404622in}}%
\pgfpathlineto{\pgfqpoint{5.800000in}{123.404622in}}%
\pgfpathclose%
\pgfusepath{fill}%
\end{pgfscope}%
\begin{pgfscope}%
\pgfpathrectangle{\pgfqpoint{5.800000in}{0.720000in}}{\pgfqpoint{1.400000in}{4.620000in}}%
\pgfusepath{clip}%
\pgfsetbuttcap%
\pgfsetmiterjoin%
\definecolor{currentfill}{rgb}{0.121569,0.466667,0.705882}%
\pgfsetfillcolor{currentfill}%
\pgfsetlinewidth{0.000000pt}%
\definecolor{currentstroke}{rgb}{0.000000,0.000000,0.000000}%
\pgfsetstrokecolor{currentstroke}%
\pgfsetstrokeopacity{0.000000}%
\pgfsetdash{}{0pt}%
\pgfpathmoveto{\pgfqpoint{5.800000in}{123.404622in}}%
\pgfpathlineto{\pgfqpoint{5.800000in}{123.404622in}}%
\pgfpathlineto{\pgfqpoint{5.800000in}{123.507459in}}%
\pgfpathlineto{\pgfqpoint{5.800000in}{123.507459in}}%
\pgfpathclose%
\pgfusepath{fill}%
\end{pgfscope}%
\begin{pgfscope}%
\pgfpathrectangle{\pgfqpoint{5.800000in}{0.720000in}}{\pgfqpoint{1.400000in}{4.620000in}}%
\pgfusepath{clip}%
\pgfsetbuttcap%
\pgfsetmiterjoin%
\definecolor{currentfill}{rgb}{0.121569,0.466667,0.705882}%
\pgfsetfillcolor{currentfill}%
\pgfsetlinewidth{0.000000pt}%
\definecolor{currentstroke}{rgb}{0.000000,0.000000,0.000000}%
\pgfsetstrokecolor{currentstroke}%
\pgfsetstrokeopacity{0.000000}%
\pgfsetdash{}{0pt}%
\pgfpathmoveto{\pgfqpoint{5.800000in}{123.507459in}}%
\pgfpathlineto{\pgfqpoint{5.800000in}{123.507459in}}%
\pgfpathlineto{\pgfqpoint{5.800000in}{123.610296in}}%
\pgfpathlineto{\pgfqpoint{5.800000in}{123.610296in}}%
\pgfpathclose%
\pgfusepath{fill}%
\end{pgfscope}%
\begin{pgfscope}%
\pgfpathrectangle{\pgfqpoint{5.800000in}{0.720000in}}{\pgfqpoint{1.400000in}{4.620000in}}%
\pgfusepath{clip}%
\pgfsetbuttcap%
\pgfsetmiterjoin%
\definecolor{currentfill}{rgb}{0.121569,0.466667,0.705882}%
\pgfsetfillcolor{currentfill}%
\pgfsetlinewidth{0.000000pt}%
\definecolor{currentstroke}{rgb}{0.000000,0.000000,0.000000}%
\pgfsetstrokecolor{currentstroke}%
\pgfsetstrokeopacity{0.000000}%
\pgfsetdash{}{0pt}%
\pgfpathmoveto{\pgfqpoint{5.800000in}{123.610296in}}%
\pgfpathlineto{\pgfqpoint{5.800000in}{123.610296in}}%
\pgfpathlineto{\pgfqpoint{5.800000in}{123.713133in}}%
\pgfpathlineto{\pgfqpoint{5.800000in}{123.713133in}}%
\pgfpathclose%
\pgfusepath{fill}%
\end{pgfscope}%
\begin{pgfscope}%
\pgfpathrectangle{\pgfqpoint{5.800000in}{0.720000in}}{\pgfqpoint{1.400000in}{4.620000in}}%
\pgfusepath{clip}%
\pgfsetbuttcap%
\pgfsetmiterjoin%
\definecolor{currentfill}{rgb}{0.121569,0.466667,0.705882}%
\pgfsetfillcolor{currentfill}%
\pgfsetlinewidth{0.000000pt}%
\definecolor{currentstroke}{rgb}{0.000000,0.000000,0.000000}%
\pgfsetstrokecolor{currentstroke}%
\pgfsetstrokeopacity{0.000000}%
\pgfsetdash{}{0pt}%
\pgfpathmoveto{\pgfqpoint{5.800000in}{123.713133in}}%
\pgfpathlineto{\pgfqpoint{5.800000in}{123.713133in}}%
\pgfpathlineto{\pgfqpoint{5.800000in}{123.815970in}}%
\pgfpathlineto{\pgfqpoint{5.800000in}{123.815970in}}%
\pgfpathclose%
\pgfusepath{fill}%
\end{pgfscope}%
\begin{pgfscope}%
\pgfpathrectangle{\pgfqpoint{5.800000in}{0.720000in}}{\pgfqpoint{1.400000in}{4.620000in}}%
\pgfusepath{clip}%
\pgfsetbuttcap%
\pgfsetmiterjoin%
\definecolor{currentfill}{rgb}{0.121569,0.466667,0.705882}%
\pgfsetfillcolor{currentfill}%
\pgfsetlinewidth{0.000000pt}%
\definecolor{currentstroke}{rgb}{0.000000,0.000000,0.000000}%
\pgfsetstrokecolor{currentstroke}%
\pgfsetstrokeopacity{0.000000}%
\pgfsetdash{}{0pt}%
\pgfpathmoveto{\pgfqpoint{5.800000in}{123.815970in}}%
\pgfpathlineto{\pgfqpoint{5.800000in}{123.815970in}}%
\pgfpathlineto{\pgfqpoint{5.800000in}{123.918807in}}%
\pgfpathlineto{\pgfqpoint{5.800000in}{123.918807in}}%
\pgfpathclose%
\pgfusepath{fill}%
\end{pgfscope}%
\begin{pgfscope}%
\pgfpathrectangle{\pgfqpoint{5.800000in}{0.720000in}}{\pgfqpoint{1.400000in}{4.620000in}}%
\pgfusepath{clip}%
\pgfsetbuttcap%
\pgfsetmiterjoin%
\definecolor{currentfill}{rgb}{0.121569,0.466667,0.705882}%
\pgfsetfillcolor{currentfill}%
\pgfsetlinewidth{0.000000pt}%
\definecolor{currentstroke}{rgb}{0.000000,0.000000,0.000000}%
\pgfsetstrokecolor{currentstroke}%
\pgfsetstrokeopacity{0.000000}%
\pgfsetdash{}{0pt}%
\pgfpathmoveto{\pgfqpoint{5.800000in}{123.918807in}}%
\pgfpathlineto{\pgfqpoint{5.800000in}{123.918807in}}%
\pgfpathlineto{\pgfqpoint{5.800000in}{124.021644in}}%
\pgfpathlineto{\pgfqpoint{5.800000in}{124.021644in}}%
\pgfpathclose%
\pgfusepath{fill}%
\end{pgfscope}%
\begin{pgfscope}%
\pgfpathrectangle{\pgfqpoint{5.800000in}{0.720000in}}{\pgfqpoint{1.400000in}{4.620000in}}%
\pgfusepath{clip}%
\pgfsetbuttcap%
\pgfsetmiterjoin%
\definecolor{currentfill}{rgb}{0.121569,0.466667,0.705882}%
\pgfsetfillcolor{currentfill}%
\pgfsetlinewidth{0.000000pt}%
\definecolor{currentstroke}{rgb}{0.000000,0.000000,0.000000}%
\pgfsetstrokecolor{currentstroke}%
\pgfsetstrokeopacity{0.000000}%
\pgfsetdash{}{0pt}%
\pgfpathmoveto{\pgfqpoint{5.800000in}{124.021644in}}%
\pgfpathlineto{\pgfqpoint{5.800000in}{124.021644in}}%
\pgfpathlineto{\pgfqpoint{5.800000in}{124.124481in}}%
\pgfpathlineto{\pgfqpoint{5.800000in}{124.124481in}}%
\pgfpathclose%
\pgfusepath{fill}%
\end{pgfscope}%
\begin{pgfscope}%
\pgfpathrectangle{\pgfqpoint{5.800000in}{0.720000in}}{\pgfqpoint{1.400000in}{4.620000in}}%
\pgfusepath{clip}%
\pgfsetbuttcap%
\pgfsetmiterjoin%
\definecolor{currentfill}{rgb}{0.121569,0.466667,0.705882}%
\pgfsetfillcolor{currentfill}%
\pgfsetlinewidth{0.000000pt}%
\definecolor{currentstroke}{rgb}{0.000000,0.000000,0.000000}%
\pgfsetstrokecolor{currentstroke}%
\pgfsetstrokeopacity{0.000000}%
\pgfsetdash{}{0pt}%
\pgfpathmoveto{\pgfqpoint{5.800000in}{124.124481in}}%
\pgfpathlineto{\pgfqpoint{5.800000in}{124.124481in}}%
\pgfpathlineto{\pgfqpoint{5.800000in}{124.227318in}}%
\pgfpathlineto{\pgfqpoint{5.800000in}{124.227318in}}%
\pgfpathclose%
\pgfusepath{fill}%
\end{pgfscope}%
\begin{pgfscope}%
\pgfpathrectangle{\pgfqpoint{5.800000in}{0.720000in}}{\pgfqpoint{1.400000in}{4.620000in}}%
\pgfusepath{clip}%
\pgfsetbuttcap%
\pgfsetmiterjoin%
\definecolor{currentfill}{rgb}{0.121569,0.466667,0.705882}%
\pgfsetfillcolor{currentfill}%
\pgfsetlinewidth{0.000000pt}%
\definecolor{currentstroke}{rgb}{0.000000,0.000000,0.000000}%
\pgfsetstrokecolor{currentstroke}%
\pgfsetstrokeopacity{0.000000}%
\pgfsetdash{}{0pt}%
\pgfpathmoveto{\pgfqpoint{5.800000in}{124.227318in}}%
\pgfpathlineto{\pgfqpoint{5.800000in}{124.227318in}}%
\pgfpathlineto{\pgfqpoint{5.800000in}{124.330155in}}%
\pgfpathlineto{\pgfqpoint{5.800000in}{124.330155in}}%
\pgfpathclose%
\pgfusepath{fill}%
\end{pgfscope}%
\begin{pgfscope}%
\pgfpathrectangle{\pgfqpoint{5.800000in}{0.720000in}}{\pgfqpoint{1.400000in}{4.620000in}}%
\pgfusepath{clip}%
\pgfsetbuttcap%
\pgfsetmiterjoin%
\definecolor{currentfill}{rgb}{0.121569,0.466667,0.705882}%
\pgfsetfillcolor{currentfill}%
\pgfsetlinewidth{0.000000pt}%
\definecolor{currentstroke}{rgb}{0.000000,0.000000,0.000000}%
\pgfsetstrokecolor{currentstroke}%
\pgfsetstrokeopacity{0.000000}%
\pgfsetdash{}{0pt}%
\pgfpathmoveto{\pgfqpoint{5.800000in}{124.330155in}}%
\pgfpathlineto{\pgfqpoint{5.800000in}{124.330155in}}%
\pgfpathlineto{\pgfqpoint{5.800000in}{124.432993in}}%
\pgfpathlineto{\pgfqpoint{5.800000in}{124.432993in}}%
\pgfpathclose%
\pgfusepath{fill}%
\end{pgfscope}%
\begin{pgfscope}%
\pgfpathrectangle{\pgfqpoint{5.800000in}{0.720000in}}{\pgfqpoint{1.400000in}{4.620000in}}%
\pgfusepath{clip}%
\pgfsetbuttcap%
\pgfsetmiterjoin%
\definecolor{currentfill}{rgb}{0.121569,0.466667,0.705882}%
\pgfsetfillcolor{currentfill}%
\pgfsetlinewidth{0.000000pt}%
\definecolor{currentstroke}{rgb}{0.000000,0.000000,0.000000}%
\pgfsetstrokecolor{currentstroke}%
\pgfsetstrokeopacity{0.000000}%
\pgfsetdash{}{0pt}%
\pgfpathmoveto{\pgfqpoint{5.800000in}{124.432993in}}%
\pgfpathlineto{\pgfqpoint{5.800000in}{124.432993in}}%
\pgfpathlineto{\pgfqpoint{5.800000in}{124.535830in}}%
\pgfpathlineto{\pgfqpoint{5.800000in}{124.535830in}}%
\pgfpathclose%
\pgfusepath{fill}%
\end{pgfscope}%
\begin{pgfscope}%
\pgfpathrectangle{\pgfqpoint{5.800000in}{0.720000in}}{\pgfqpoint{1.400000in}{4.620000in}}%
\pgfusepath{clip}%
\pgfsetbuttcap%
\pgfsetmiterjoin%
\definecolor{currentfill}{rgb}{0.121569,0.466667,0.705882}%
\pgfsetfillcolor{currentfill}%
\pgfsetlinewidth{0.000000pt}%
\definecolor{currentstroke}{rgb}{0.000000,0.000000,0.000000}%
\pgfsetstrokecolor{currentstroke}%
\pgfsetstrokeopacity{0.000000}%
\pgfsetdash{}{0pt}%
\pgfpathmoveto{\pgfqpoint{5.800000in}{124.535830in}}%
\pgfpathlineto{\pgfqpoint{5.800000in}{124.535830in}}%
\pgfpathlineto{\pgfqpoint{5.800000in}{124.638667in}}%
\pgfpathlineto{\pgfqpoint{5.800000in}{124.638667in}}%
\pgfpathclose%
\pgfusepath{fill}%
\end{pgfscope}%
\begin{pgfscope}%
\pgfpathrectangle{\pgfqpoint{5.800000in}{0.720000in}}{\pgfqpoint{1.400000in}{4.620000in}}%
\pgfusepath{clip}%
\pgfsetbuttcap%
\pgfsetmiterjoin%
\definecolor{currentfill}{rgb}{0.121569,0.466667,0.705882}%
\pgfsetfillcolor{currentfill}%
\pgfsetlinewidth{0.000000pt}%
\definecolor{currentstroke}{rgb}{0.000000,0.000000,0.000000}%
\pgfsetstrokecolor{currentstroke}%
\pgfsetstrokeopacity{0.000000}%
\pgfsetdash{}{0pt}%
\pgfpathmoveto{\pgfqpoint{5.800000in}{124.638667in}}%
\pgfpathlineto{\pgfqpoint{5.800000in}{124.638667in}}%
\pgfpathlineto{\pgfqpoint{5.800000in}{124.741504in}}%
\pgfpathlineto{\pgfqpoint{5.800000in}{124.741504in}}%
\pgfpathclose%
\pgfusepath{fill}%
\end{pgfscope}%
\begin{pgfscope}%
\pgfpathrectangle{\pgfqpoint{5.800000in}{0.720000in}}{\pgfqpoint{1.400000in}{4.620000in}}%
\pgfusepath{clip}%
\pgfsetbuttcap%
\pgfsetmiterjoin%
\definecolor{currentfill}{rgb}{0.121569,0.466667,0.705882}%
\pgfsetfillcolor{currentfill}%
\pgfsetlinewidth{0.000000pt}%
\definecolor{currentstroke}{rgb}{0.000000,0.000000,0.000000}%
\pgfsetstrokecolor{currentstroke}%
\pgfsetstrokeopacity{0.000000}%
\pgfsetdash{}{0pt}%
\pgfpathmoveto{\pgfqpoint{5.800000in}{124.741504in}}%
\pgfpathlineto{\pgfqpoint{5.800000in}{124.741504in}}%
\pgfpathlineto{\pgfqpoint{5.800000in}{124.844341in}}%
\pgfpathlineto{\pgfqpoint{5.800000in}{124.844341in}}%
\pgfpathclose%
\pgfusepath{fill}%
\end{pgfscope}%
\begin{pgfscope}%
\pgfpathrectangle{\pgfqpoint{5.800000in}{0.720000in}}{\pgfqpoint{1.400000in}{4.620000in}}%
\pgfusepath{clip}%
\pgfsetbuttcap%
\pgfsetmiterjoin%
\definecolor{currentfill}{rgb}{0.121569,0.466667,0.705882}%
\pgfsetfillcolor{currentfill}%
\pgfsetlinewidth{0.000000pt}%
\definecolor{currentstroke}{rgb}{0.000000,0.000000,0.000000}%
\pgfsetstrokecolor{currentstroke}%
\pgfsetstrokeopacity{0.000000}%
\pgfsetdash{}{0pt}%
\pgfpathmoveto{\pgfqpoint{5.800000in}{124.844341in}}%
\pgfpathlineto{\pgfqpoint{5.800000in}{124.844341in}}%
\pgfpathlineto{\pgfqpoint{5.800000in}{124.947178in}}%
\pgfpathlineto{\pgfqpoint{5.800000in}{124.947178in}}%
\pgfpathclose%
\pgfusepath{fill}%
\end{pgfscope}%
\begin{pgfscope}%
\pgfpathrectangle{\pgfqpoint{5.800000in}{0.720000in}}{\pgfqpoint{1.400000in}{4.620000in}}%
\pgfusepath{clip}%
\pgfsetbuttcap%
\pgfsetmiterjoin%
\definecolor{currentfill}{rgb}{0.121569,0.466667,0.705882}%
\pgfsetfillcolor{currentfill}%
\pgfsetlinewidth{0.000000pt}%
\definecolor{currentstroke}{rgb}{0.000000,0.000000,0.000000}%
\pgfsetstrokecolor{currentstroke}%
\pgfsetstrokeopacity{0.000000}%
\pgfsetdash{}{0pt}%
\pgfpathmoveto{\pgfqpoint{5.800000in}{124.947178in}}%
\pgfpathlineto{\pgfqpoint{5.800000in}{124.947178in}}%
\pgfpathlineto{\pgfqpoint{5.800000in}{125.050015in}}%
\pgfpathlineto{\pgfqpoint{5.800000in}{125.050015in}}%
\pgfpathclose%
\pgfusepath{fill}%
\end{pgfscope}%
\begin{pgfscope}%
\pgfpathrectangle{\pgfqpoint{5.800000in}{0.720000in}}{\pgfqpoint{1.400000in}{4.620000in}}%
\pgfusepath{clip}%
\pgfsetbuttcap%
\pgfsetmiterjoin%
\definecolor{currentfill}{rgb}{0.121569,0.466667,0.705882}%
\pgfsetfillcolor{currentfill}%
\pgfsetlinewidth{0.000000pt}%
\definecolor{currentstroke}{rgb}{0.000000,0.000000,0.000000}%
\pgfsetstrokecolor{currentstroke}%
\pgfsetstrokeopacity{0.000000}%
\pgfsetdash{}{0pt}%
\pgfpathmoveto{\pgfqpoint{5.800000in}{125.050015in}}%
\pgfpathlineto{\pgfqpoint{5.800000in}{125.050015in}}%
\pgfpathlineto{\pgfqpoint{5.800000in}{125.152852in}}%
\pgfpathlineto{\pgfqpoint{5.800000in}{125.152852in}}%
\pgfpathclose%
\pgfusepath{fill}%
\end{pgfscope}%
\begin{pgfscope}%
\pgfpathrectangle{\pgfqpoint{5.800000in}{0.720000in}}{\pgfqpoint{1.400000in}{4.620000in}}%
\pgfusepath{clip}%
\pgfsetbuttcap%
\pgfsetmiterjoin%
\definecolor{currentfill}{rgb}{0.121569,0.466667,0.705882}%
\pgfsetfillcolor{currentfill}%
\pgfsetlinewidth{0.000000pt}%
\definecolor{currentstroke}{rgb}{0.000000,0.000000,0.000000}%
\pgfsetstrokecolor{currentstroke}%
\pgfsetstrokeopacity{0.000000}%
\pgfsetdash{}{0pt}%
\pgfpathmoveto{\pgfqpoint{5.800000in}{125.152852in}}%
\pgfpathlineto{\pgfqpoint{5.800000in}{125.152852in}}%
\pgfpathlineto{\pgfqpoint{5.800000in}{125.255689in}}%
\pgfpathlineto{\pgfqpoint{5.800000in}{125.255689in}}%
\pgfpathclose%
\pgfusepath{fill}%
\end{pgfscope}%
\begin{pgfscope}%
\pgfpathrectangle{\pgfqpoint{5.800000in}{0.720000in}}{\pgfqpoint{1.400000in}{4.620000in}}%
\pgfusepath{clip}%
\pgfsetbuttcap%
\pgfsetmiterjoin%
\definecolor{currentfill}{rgb}{0.121569,0.466667,0.705882}%
\pgfsetfillcolor{currentfill}%
\pgfsetlinewidth{0.000000pt}%
\definecolor{currentstroke}{rgb}{0.000000,0.000000,0.000000}%
\pgfsetstrokecolor{currentstroke}%
\pgfsetstrokeopacity{0.000000}%
\pgfsetdash{}{0pt}%
\pgfpathmoveto{\pgfqpoint{5.800000in}{125.255689in}}%
\pgfpathlineto{\pgfqpoint{5.800000in}{125.255689in}}%
\pgfpathlineto{\pgfqpoint{5.800000in}{125.358526in}}%
\pgfpathlineto{\pgfqpoint{5.800000in}{125.358526in}}%
\pgfpathclose%
\pgfusepath{fill}%
\end{pgfscope}%
\begin{pgfscope}%
\pgfpathrectangle{\pgfqpoint{5.800000in}{0.720000in}}{\pgfqpoint{1.400000in}{4.620000in}}%
\pgfusepath{clip}%
\pgfsetbuttcap%
\pgfsetmiterjoin%
\definecolor{currentfill}{rgb}{0.121569,0.466667,0.705882}%
\pgfsetfillcolor{currentfill}%
\pgfsetlinewidth{0.000000pt}%
\definecolor{currentstroke}{rgb}{0.000000,0.000000,0.000000}%
\pgfsetstrokecolor{currentstroke}%
\pgfsetstrokeopacity{0.000000}%
\pgfsetdash{}{0pt}%
\pgfpathmoveto{\pgfqpoint{5.800000in}{125.358526in}}%
\pgfpathlineto{\pgfqpoint{5.800000in}{125.358526in}}%
\pgfpathlineto{\pgfqpoint{5.800000in}{125.461363in}}%
\pgfpathlineto{\pgfqpoint{5.800000in}{125.461363in}}%
\pgfpathclose%
\pgfusepath{fill}%
\end{pgfscope}%
\begin{pgfscope}%
\pgfpathrectangle{\pgfqpoint{5.800000in}{0.720000in}}{\pgfqpoint{1.400000in}{4.620000in}}%
\pgfusepath{clip}%
\pgfsetbuttcap%
\pgfsetmiterjoin%
\definecolor{currentfill}{rgb}{0.121569,0.466667,0.705882}%
\pgfsetfillcolor{currentfill}%
\pgfsetlinewidth{0.000000pt}%
\definecolor{currentstroke}{rgb}{0.000000,0.000000,0.000000}%
\pgfsetstrokecolor{currentstroke}%
\pgfsetstrokeopacity{0.000000}%
\pgfsetdash{}{0pt}%
\pgfpathmoveto{\pgfqpoint{5.800000in}{125.461363in}}%
\pgfpathlineto{\pgfqpoint{5.800000in}{125.461363in}}%
\pgfpathlineto{\pgfqpoint{5.800000in}{125.564200in}}%
\pgfpathlineto{\pgfqpoint{5.800000in}{125.564200in}}%
\pgfpathclose%
\pgfusepath{fill}%
\end{pgfscope}%
\begin{pgfscope}%
\pgfpathrectangle{\pgfqpoint{5.800000in}{0.720000in}}{\pgfqpoint{1.400000in}{4.620000in}}%
\pgfusepath{clip}%
\pgfsetbuttcap%
\pgfsetmiterjoin%
\definecolor{currentfill}{rgb}{0.121569,0.466667,0.705882}%
\pgfsetfillcolor{currentfill}%
\pgfsetlinewidth{0.000000pt}%
\definecolor{currentstroke}{rgb}{0.000000,0.000000,0.000000}%
\pgfsetstrokecolor{currentstroke}%
\pgfsetstrokeopacity{0.000000}%
\pgfsetdash{}{0pt}%
\pgfpathmoveto{\pgfqpoint{5.800000in}{125.564200in}}%
\pgfpathlineto{\pgfqpoint{5.800000in}{125.564200in}}%
\pgfpathlineto{\pgfqpoint{5.800000in}{125.667037in}}%
\pgfpathlineto{\pgfqpoint{5.800000in}{125.667037in}}%
\pgfpathclose%
\pgfusepath{fill}%
\end{pgfscope}%
\begin{pgfscope}%
\pgfpathrectangle{\pgfqpoint{5.800000in}{0.720000in}}{\pgfqpoint{1.400000in}{4.620000in}}%
\pgfusepath{clip}%
\pgfsetbuttcap%
\pgfsetmiterjoin%
\definecolor{currentfill}{rgb}{0.121569,0.466667,0.705882}%
\pgfsetfillcolor{currentfill}%
\pgfsetlinewidth{0.000000pt}%
\definecolor{currentstroke}{rgb}{0.000000,0.000000,0.000000}%
\pgfsetstrokecolor{currentstroke}%
\pgfsetstrokeopacity{0.000000}%
\pgfsetdash{}{0pt}%
\pgfpathmoveto{\pgfqpoint{5.800000in}{125.667037in}}%
\pgfpathlineto{\pgfqpoint{5.800000in}{125.667037in}}%
\pgfpathlineto{\pgfqpoint{5.800000in}{125.769874in}}%
\pgfpathlineto{\pgfqpoint{5.800000in}{125.769874in}}%
\pgfpathclose%
\pgfusepath{fill}%
\end{pgfscope}%
\begin{pgfscope}%
\pgfpathrectangle{\pgfqpoint{5.800000in}{0.720000in}}{\pgfqpoint{1.400000in}{4.620000in}}%
\pgfusepath{clip}%
\pgfsetbuttcap%
\pgfsetmiterjoin%
\definecolor{currentfill}{rgb}{0.121569,0.466667,0.705882}%
\pgfsetfillcolor{currentfill}%
\pgfsetlinewidth{0.000000pt}%
\definecolor{currentstroke}{rgb}{0.000000,0.000000,0.000000}%
\pgfsetstrokecolor{currentstroke}%
\pgfsetstrokeopacity{0.000000}%
\pgfsetdash{}{0pt}%
\pgfpathmoveto{\pgfqpoint{5.800000in}{125.769874in}}%
\pgfpathlineto{\pgfqpoint{5.800000in}{125.769874in}}%
\pgfpathlineto{\pgfqpoint{5.800000in}{125.872711in}}%
\pgfpathlineto{\pgfqpoint{5.800000in}{125.872711in}}%
\pgfpathclose%
\pgfusepath{fill}%
\end{pgfscope}%
\begin{pgfscope}%
\pgfpathrectangle{\pgfqpoint{5.800000in}{0.720000in}}{\pgfqpoint{1.400000in}{4.620000in}}%
\pgfusepath{clip}%
\pgfsetbuttcap%
\pgfsetmiterjoin%
\definecolor{currentfill}{rgb}{0.121569,0.466667,0.705882}%
\pgfsetfillcolor{currentfill}%
\pgfsetlinewidth{0.000000pt}%
\definecolor{currentstroke}{rgb}{0.000000,0.000000,0.000000}%
\pgfsetstrokecolor{currentstroke}%
\pgfsetstrokeopacity{0.000000}%
\pgfsetdash{}{0pt}%
\pgfpathmoveto{\pgfqpoint{5.800000in}{125.872711in}}%
\pgfpathlineto{\pgfqpoint{5.800000in}{125.872711in}}%
\pgfpathlineto{\pgfqpoint{5.800000in}{125.975549in}}%
\pgfpathlineto{\pgfqpoint{5.800000in}{125.975549in}}%
\pgfpathclose%
\pgfusepath{fill}%
\end{pgfscope}%
\begin{pgfscope}%
\pgfpathrectangle{\pgfqpoint{5.800000in}{0.720000in}}{\pgfqpoint{1.400000in}{4.620000in}}%
\pgfusepath{clip}%
\pgfsetbuttcap%
\pgfsetmiterjoin%
\definecolor{currentfill}{rgb}{0.121569,0.466667,0.705882}%
\pgfsetfillcolor{currentfill}%
\pgfsetlinewidth{0.000000pt}%
\definecolor{currentstroke}{rgb}{0.000000,0.000000,0.000000}%
\pgfsetstrokecolor{currentstroke}%
\pgfsetstrokeopacity{0.000000}%
\pgfsetdash{}{0pt}%
\pgfpathmoveto{\pgfqpoint{5.800000in}{125.975549in}}%
\pgfpathlineto{\pgfqpoint{5.800000in}{125.975549in}}%
\pgfpathlineto{\pgfqpoint{5.800000in}{126.078386in}}%
\pgfpathlineto{\pgfqpoint{5.800000in}{126.078386in}}%
\pgfpathclose%
\pgfusepath{fill}%
\end{pgfscope}%
\begin{pgfscope}%
\pgfpathrectangle{\pgfqpoint{5.800000in}{0.720000in}}{\pgfqpoint{1.400000in}{4.620000in}}%
\pgfusepath{clip}%
\pgfsetbuttcap%
\pgfsetmiterjoin%
\definecolor{currentfill}{rgb}{0.121569,0.466667,0.705882}%
\pgfsetfillcolor{currentfill}%
\pgfsetlinewidth{0.000000pt}%
\definecolor{currentstroke}{rgb}{0.000000,0.000000,0.000000}%
\pgfsetstrokecolor{currentstroke}%
\pgfsetstrokeopacity{0.000000}%
\pgfsetdash{}{0pt}%
\pgfpathmoveto{\pgfqpoint{5.800000in}{126.078386in}}%
\pgfpathlineto{\pgfqpoint{5.800000in}{126.078386in}}%
\pgfpathlineto{\pgfqpoint{5.800000in}{126.181223in}}%
\pgfpathlineto{\pgfqpoint{5.800000in}{126.181223in}}%
\pgfpathclose%
\pgfusepath{fill}%
\end{pgfscope}%
\begin{pgfscope}%
\pgfpathrectangle{\pgfqpoint{5.800000in}{0.720000in}}{\pgfqpoint{1.400000in}{4.620000in}}%
\pgfusepath{clip}%
\pgfsetbuttcap%
\pgfsetmiterjoin%
\definecolor{currentfill}{rgb}{0.121569,0.466667,0.705882}%
\pgfsetfillcolor{currentfill}%
\pgfsetlinewidth{0.000000pt}%
\definecolor{currentstroke}{rgb}{0.000000,0.000000,0.000000}%
\pgfsetstrokecolor{currentstroke}%
\pgfsetstrokeopacity{0.000000}%
\pgfsetdash{}{0pt}%
\pgfpathmoveto{\pgfqpoint{5.800000in}{126.181223in}}%
\pgfpathlineto{\pgfqpoint{5.800000in}{126.181223in}}%
\pgfpathlineto{\pgfqpoint{5.800000in}{126.284060in}}%
\pgfpathlineto{\pgfqpoint{5.800000in}{126.284060in}}%
\pgfpathclose%
\pgfusepath{fill}%
\end{pgfscope}%
\begin{pgfscope}%
\pgfpathrectangle{\pgfqpoint{5.800000in}{0.720000in}}{\pgfqpoint{1.400000in}{4.620000in}}%
\pgfusepath{clip}%
\pgfsetbuttcap%
\pgfsetmiterjoin%
\definecolor{currentfill}{rgb}{0.121569,0.466667,0.705882}%
\pgfsetfillcolor{currentfill}%
\pgfsetlinewidth{0.000000pt}%
\definecolor{currentstroke}{rgb}{0.000000,0.000000,0.000000}%
\pgfsetstrokecolor{currentstroke}%
\pgfsetstrokeopacity{0.000000}%
\pgfsetdash{}{0pt}%
\pgfpathmoveto{\pgfqpoint{5.800000in}{126.284060in}}%
\pgfpathlineto{\pgfqpoint{5.800000in}{126.284060in}}%
\pgfpathlineto{\pgfqpoint{5.800000in}{126.386897in}}%
\pgfpathlineto{\pgfqpoint{5.800000in}{126.386897in}}%
\pgfpathclose%
\pgfusepath{fill}%
\end{pgfscope}%
\begin{pgfscope}%
\pgfpathrectangle{\pgfqpoint{5.800000in}{0.720000in}}{\pgfqpoint{1.400000in}{4.620000in}}%
\pgfusepath{clip}%
\pgfsetbuttcap%
\pgfsetmiterjoin%
\definecolor{currentfill}{rgb}{0.121569,0.466667,0.705882}%
\pgfsetfillcolor{currentfill}%
\pgfsetlinewidth{0.000000pt}%
\definecolor{currentstroke}{rgb}{0.000000,0.000000,0.000000}%
\pgfsetstrokecolor{currentstroke}%
\pgfsetstrokeopacity{0.000000}%
\pgfsetdash{}{0pt}%
\pgfpathmoveto{\pgfqpoint{5.800000in}{126.386897in}}%
\pgfpathlineto{\pgfqpoint{5.800000in}{126.386897in}}%
\pgfpathlineto{\pgfqpoint{5.800000in}{126.489734in}}%
\pgfpathlineto{\pgfqpoint{5.800000in}{126.489734in}}%
\pgfpathclose%
\pgfusepath{fill}%
\end{pgfscope}%
\begin{pgfscope}%
\pgfpathrectangle{\pgfqpoint{5.800000in}{0.720000in}}{\pgfqpoint{1.400000in}{4.620000in}}%
\pgfusepath{clip}%
\pgfsetbuttcap%
\pgfsetmiterjoin%
\definecolor{currentfill}{rgb}{0.121569,0.466667,0.705882}%
\pgfsetfillcolor{currentfill}%
\pgfsetlinewidth{0.000000pt}%
\definecolor{currentstroke}{rgb}{0.000000,0.000000,0.000000}%
\pgfsetstrokecolor{currentstroke}%
\pgfsetstrokeopacity{0.000000}%
\pgfsetdash{}{0pt}%
\pgfpathmoveto{\pgfqpoint{5.800000in}{126.489734in}}%
\pgfpathlineto{\pgfqpoint{5.800000in}{126.489734in}}%
\pgfpathlineto{\pgfqpoint{5.800000in}{126.592571in}}%
\pgfpathlineto{\pgfqpoint{5.800000in}{126.592571in}}%
\pgfpathclose%
\pgfusepath{fill}%
\end{pgfscope}%
\begin{pgfscope}%
\pgfpathrectangle{\pgfqpoint{5.800000in}{0.720000in}}{\pgfqpoint{1.400000in}{4.620000in}}%
\pgfusepath{clip}%
\pgfsetbuttcap%
\pgfsetmiterjoin%
\definecolor{currentfill}{rgb}{0.121569,0.466667,0.705882}%
\pgfsetfillcolor{currentfill}%
\pgfsetlinewidth{0.000000pt}%
\definecolor{currentstroke}{rgb}{0.000000,0.000000,0.000000}%
\pgfsetstrokecolor{currentstroke}%
\pgfsetstrokeopacity{0.000000}%
\pgfsetdash{}{0pt}%
\pgfpathmoveto{\pgfqpoint{5.800000in}{126.592571in}}%
\pgfpathlineto{\pgfqpoint{5.800000in}{126.592571in}}%
\pgfpathlineto{\pgfqpoint{5.800000in}{126.695408in}}%
\pgfpathlineto{\pgfqpoint{5.800000in}{126.695408in}}%
\pgfpathclose%
\pgfusepath{fill}%
\end{pgfscope}%
\begin{pgfscope}%
\pgfpathrectangle{\pgfqpoint{5.800000in}{0.720000in}}{\pgfqpoint{1.400000in}{4.620000in}}%
\pgfusepath{clip}%
\pgfsetbuttcap%
\pgfsetmiterjoin%
\definecolor{currentfill}{rgb}{0.121569,0.466667,0.705882}%
\pgfsetfillcolor{currentfill}%
\pgfsetlinewidth{0.000000pt}%
\definecolor{currentstroke}{rgb}{0.000000,0.000000,0.000000}%
\pgfsetstrokecolor{currentstroke}%
\pgfsetstrokeopacity{0.000000}%
\pgfsetdash{}{0pt}%
\pgfpathmoveto{\pgfqpoint{5.800000in}{126.695408in}}%
\pgfpathlineto{\pgfqpoint{5.802038in}{126.695408in}}%
\pgfpathlineto{\pgfqpoint{5.802038in}{126.798245in}}%
\pgfpathlineto{\pgfqpoint{5.800000in}{126.798245in}}%
\pgfpathclose%
\pgfusepath{fill}%
\end{pgfscope}%
\begin{pgfscope}%
\pgfpathrectangle{\pgfqpoint{5.800000in}{0.720000in}}{\pgfqpoint{1.400000in}{4.620000in}}%
\pgfusepath{clip}%
\pgfsetbuttcap%
\pgfsetmiterjoin%
\definecolor{currentfill}{rgb}{0.121569,0.466667,0.705882}%
\pgfsetfillcolor{currentfill}%
\pgfsetlinewidth{0.000000pt}%
\definecolor{currentstroke}{rgb}{0.000000,0.000000,0.000000}%
\pgfsetstrokecolor{currentstroke}%
\pgfsetstrokeopacity{0.000000}%
\pgfsetdash{}{0pt}%
\pgfpathmoveto{\pgfqpoint{5.800000in}{126.798245in}}%
\pgfpathlineto{\pgfqpoint{5.800000in}{126.798245in}}%
\pgfpathlineto{\pgfqpoint{5.800000in}{126.901082in}}%
\pgfpathlineto{\pgfqpoint{5.800000in}{126.901082in}}%
\pgfpathclose%
\pgfusepath{fill}%
\end{pgfscope}%
\begin{pgfscope}%
\pgfpathrectangle{\pgfqpoint{5.800000in}{0.720000in}}{\pgfqpoint{1.400000in}{4.620000in}}%
\pgfusepath{clip}%
\pgfsetbuttcap%
\pgfsetmiterjoin%
\definecolor{currentfill}{rgb}{0.121569,0.466667,0.705882}%
\pgfsetfillcolor{currentfill}%
\pgfsetlinewidth{0.000000pt}%
\definecolor{currentstroke}{rgb}{0.000000,0.000000,0.000000}%
\pgfsetstrokecolor{currentstroke}%
\pgfsetstrokeopacity{0.000000}%
\pgfsetdash{}{0pt}%
\pgfpathmoveto{\pgfqpoint{5.800000in}{126.901082in}}%
\pgfpathlineto{\pgfqpoint{5.800000in}{126.901082in}}%
\pgfpathlineto{\pgfqpoint{5.800000in}{127.003919in}}%
\pgfpathlineto{\pgfqpoint{5.800000in}{127.003919in}}%
\pgfpathclose%
\pgfusepath{fill}%
\end{pgfscope}%
\begin{pgfscope}%
\pgfpathrectangle{\pgfqpoint{5.800000in}{0.720000in}}{\pgfqpoint{1.400000in}{4.620000in}}%
\pgfusepath{clip}%
\pgfsetbuttcap%
\pgfsetmiterjoin%
\definecolor{currentfill}{rgb}{0.121569,0.466667,0.705882}%
\pgfsetfillcolor{currentfill}%
\pgfsetlinewidth{0.000000pt}%
\definecolor{currentstroke}{rgb}{0.000000,0.000000,0.000000}%
\pgfsetstrokecolor{currentstroke}%
\pgfsetstrokeopacity{0.000000}%
\pgfsetdash{}{0pt}%
\pgfpathmoveto{\pgfqpoint{5.800000in}{127.003919in}}%
\pgfpathlineto{\pgfqpoint{5.800000in}{127.003919in}}%
\pgfpathlineto{\pgfqpoint{5.800000in}{127.106756in}}%
\pgfpathlineto{\pgfqpoint{5.800000in}{127.106756in}}%
\pgfpathclose%
\pgfusepath{fill}%
\end{pgfscope}%
\begin{pgfscope}%
\pgfpathrectangle{\pgfqpoint{5.800000in}{0.720000in}}{\pgfqpoint{1.400000in}{4.620000in}}%
\pgfusepath{clip}%
\pgfsetbuttcap%
\pgfsetmiterjoin%
\definecolor{currentfill}{rgb}{0.121569,0.466667,0.705882}%
\pgfsetfillcolor{currentfill}%
\pgfsetlinewidth{0.000000pt}%
\definecolor{currentstroke}{rgb}{0.000000,0.000000,0.000000}%
\pgfsetstrokecolor{currentstroke}%
\pgfsetstrokeopacity{0.000000}%
\pgfsetdash{}{0pt}%
\pgfpathmoveto{\pgfqpoint{5.800000in}{127.106756in}}%
\pgfpathlineto{\pgfqpoint{5.800000in}{127.106756in}}%
\pgfpathlineto{\pgfqpoint{5.800000in}{127.209593in}}%
\pgfpathlineto{\pgfqpoint{5.800000in}{127.209593in}}%
\pgfpathclose%
\pgfusepath{fill}%
\end{pgfscope}%
\begin{pgfscope}%
\pgfpathrectangle{\pgfqpoint{5.800000in}{0.720000in}}{\pgfqpoint{1.400000in}{4.620000in}}%
\pgfusepath{clip}%
\pgfsetbuttcap%
\pgfsetmiterjoin%
\definecolor{currentfill}{rgb}{0.121569,0.466667,0.705882}%
\pgfsetfillcolor{currentfill}%
\pgfsetlinewidth{0.000000pt}%
\definecolor{currentstroke}{rgb}{0.000000,0.000000,0.000000}%
\pgfsetstrokecolor{currentstroke}%
\pgfsetstrokeopacity{0.000000}%
\pgfsetdash{}{0pt}%
\pgfpathmoveto{\pgfqpoint{5.800000in}{127.209593in}}%
\pgfpathlineto{\pgfqpoint{5.800000in}{127.209593in}}%
\pgfpathlineto{\pgfqpoint{5.800000in}{127.312430in}}%
\pgfpathlineto{\pgfqpoint{5.800000in}{127.312430in}}%
\pgfpathclose%
\pgfusepath{fill}%
\end{pgfscope}%
\begin{pgfscope}%
\pgfpathrectangle{\pgfqpoint{5.800000in}{0.720000in}}{\pgfqpoint{1.400000in}{4.620000in}}%
\pgfusepath{clip}%
\pgfsetbuttcap%
\pgfsetmiterjoin%
\definecolor{currentfill}{rgb}{0.121569,0.466667,0.705882}%
\pgfsetfillcolor{currentfill}%
\pgfsetlinewidth{0.000000pt}%
\definecolor{currentstroke}{rgb}{0.000000,0.000000,0.000000}%
\pgfsetstrokecolor{currentstroke}%
\pgfsetstrokeopacity{0.000000}%
\pgfsetdash{}{0pt}%
\pgfpathmoveto{\pgfqpoint{5.800000in}{127.312430in}}%
\pgfpathlineto{\pgfqpoint{5.800000in}{127.312430in}}%
\pgfpathlineto{\pgfqpoint{5.800000in}{127.415268in}}%
\pgfpathlineto{\pgfqpoint{5.800000in}{127.415268in}}%
\pgfpathclose%
\pgfusepath{fill}%
\end{pgfscope}%
\begin{pgfscope}%
\pgfpathrectangle{\pgfqpoint{5.800000in}{0.720000in}}{\pgfqpoint{1.400000in}{4.620000in}}%
\pgfusepath{clip}%
\pgfsetbuttcap%
\pgfsetmiterjoin%
\definecolor{currentfill}{rgb}{0.121569,0.466667,0.705882}%
\pgfsetfillcolor{currentfill}%
\pgfsetlinewidth{0.000000pt}%
\definecolor{currentstroke}{rgb}{0.000000,0.000000,0.000000}%
\pgfsetstrokecolor{currentstroke}%
\pgfsetstrokeopacity{0.000000}%
\pgfsetdash{}{0pt}%
\pgfpathmoveto{\pgfqpoint{5.800000in}{127.415268in}}%
\pgfpathlineto{\pgfqpoint{5.800000in}{127.415268in}}%
\pgfpathlineto{\pgfqpoint{5.800000in}{127.518105in}}%
\pgfpathlineto{\pgfqpoint{5.800000in}{127.518105in}}%
\pgfpathclose%
\pgfusepath{fill}%
\end{pgfscope}%
\begin{pgfscope}%
\pgfpathrectangle{\pgfqpoint{5.800000in}{0.720000in}}{\pgfqpoint{1.400000in}{4.620000in}}%
\pgfusepath{clip}%
\pgfsetbuttcap%
\pgfsetmiterjoin%
\definecolor{currentfill}{rgb}{0.121569,0.466667,0.705882}%
\pgfsetfillcolor{currentfill}%
\pgfsetlinewidth{0.000000pt}%
\definecolor{currentstroke}{rgb}{0.000000,0.000000,0.000000}%
\pgfsetstrokecolor{currentstroke}%
\pgfsetstrokeopacity{0.000000}%
\pgfsetdash{}{0pt}%
\pgfpathmoveto{\pgfqpoint{5.800000in}{127.518105in}}%
\pgfpathlineto{\pgfqpoint{5.800000in}{127.518105in}}%
\pgfpathlineto{\pgfqpoint{5.800000in}{127.620942in}}%
\pgfpathlineto{\pgfqpoint{5.800000in}{127.620942in}}%
\pgfpathclose%
\pgfusepath{fill}%
\end{pgfscope}%
\begin{pgfscope}%
\pgfpathrectangle{\pgfqpoint{5.800000in}{0.720000in}}{\pgfqpoint{1.400000in}{4.620000in}}%
\pgfusepath{clip}%
\pgfsetbuttcap%
\pgfsetmiterjoin%
\definecolor{currentfill}{rgb}{0.121569,0.466667,0.705882}%
\pgfsetfillcolor{currentfill}%
\pgfsetlinewidth{0.000000pt}%
\definecolor{currentstroke}{rgb}{0.000000,0.000000,0.000000}%
\pgfsetstrokecolor{currentstroke}%
\pgfsetstrokeopacity{0.000000}%
\pgfsetdash{}{0pt}%
\pgfpathmoveto{\pgfqpoint{5.800000in}{127.620942in}}%
\pgfpathlineto{\pgfqpoint{5.800000in}{127.620942in}}%
\pgfpathlineto{\pgfqpoint{5.800000in}{127.723779in}}%
\pgfpathlineto{\pgfqpoint{5.800000in}{127.723779in}}%
\pgfpathclose%
\pgfusepath{fill}%
\end{pgfscope}%
\begin{pgfscope}%
\pgfpathrectangle{\pgfqpoint{5.800000in}{0.720000in}}{\pgfqpoint{1.400000in}{4.620000in}}%
\pgfusepath{clip}%
\pgfsetbuttcap%
\pgfsetmiterjoin%
\definecolor{currentfill}{rgb}{0.121569,0.466667,0.705882}%
\pgfsetfillcolor{currentfill}%
\pgfsetlinewidth{0.000000pt}%
\definecolor{currentstroke}{rgb}{0.000000,0.000000,0.000000}%
\pgfsetstrokecolor{currentstroke}%
\pgfsetstrokeopacity{0.000000}%
\pgfsetdash{}{0pt}%
\pgfpathmoveto{\pgfqpoint{5.800000in}{127.723779in}}%
\pgfpathlineto{\pgfqpoint{5.800000in}{127.723779in}}%
\pgfpathlineto{\pgfqpoint{5.800000in}{127.826616in}}%
\pgfpathlineto{\pgfqpoint{5.800000in}{127.826616in}}%
\pgfpathclose%
\pgfusepath{fill}%
\end{pgfscope}%
\begin{pgfscope}%
\pgfpathrectangle{\pgfqpoint{5.800000in}{0.720000in}}{\pgfqpoint{1.400000in}{4.620000in}}%
\pgfusepath{clip}%
\pgfsetbuttcap%
\pgfsetmiterjoin%
\definecolor{currentfill}{rgb}{0.121569,0.466667,0.705882}%
\pgfsetfillcolor{currentfill}%
\pgfsetlinewidth{0.000000pt}%
\definecolor{currentstroke}{rgb}{0.000000,0.000000,0.000000}%
\pgfsetstrokecolor{currentstroke}%
\pgfsetstrokeopacity{0.000000}%
\pgfsetdash{}{0pt}%
\pgfpathmoveto{\pgfqpoint{5.800000in}{127.826616in}}%
\pgfpathlineto{\pgfqpoint{5.800000in}{127.826616in}}%
\pgfpathlineto{\pgfqpoint{5.800000in}{127.929453in}}%
\pgfpathlineto{\pgfqpoint{5.800000in}{127.929453in}}%
\pgfpathclose%
\pgfusepath{fill}%
\end{pgfscope}%
\begin{pgfscope}%
\pgfpathrectangle{\pgfqpoint{5.800000in}{0.720000in}}{\pgfqpoint{1.400000in}{4.620000in}}%
\pgfusepath{clip}%
\pgfsetbuttcap%
\pgfsetmiterjoin%
\definecolor{currentfill}{rgb}{0.121569,0.466667,0.705882}%
\pgfsetfillcolor{currentfill}%
\pgfsetlinewidth{0.000000pt}%
\definecolor{currentstroke}{rgb}{0.000000,0.000000,0.000000}%
\pgfsetstrokecolor{currentstroke}%
\pgfsetstrokeopacity{0.000000}%
\pgfsetdash{}{0pt}%
\pgfpathmoveto{\pgfqpoint{5.800000in}{127.929453in}}%
\pgfpathlineto{\pgfqpoint{5.800000in}{127.929453in}}%
\pgfpathlineto{\pgfqpoint{5.800000in}{128.032290in}}%
\pgfpathlineto{\pgfqpoint{5.800000in}{128.032290in}}%
\pgfpathclose%
\pgfusepath{fill}%
\end{pgfscope}%
\begin{pgfscope}%
\pgfpathrectangle{\pgfqpoint{5.800000in}{0.720000in}}{\pgfqpoint{1.400000in}{4.620000in}}%
\pgfusepath{clip}%
\pgfsetbuttcap%
\pgfsetmiterjoin%
\definecolor{currentfill}{rgb}{0.121569,0.466667,0.705882}%
\pgfsetfillcolor{currentfill}%
\pgfsetlinewidth{0.000000pt}%
\definecolor{currentstroke}{rgb}{0.000000,0.000000,0.000000}%
\pgfsetstrokecolor{currentstroke}%
\pgfsetstrokeopacity{0.000000}%
\pgfsetdash{}{0pt}%
\pgfpathmoveto{\pgfqpoint{5.800000in}{128.032290in}}%
\pgfpathlineto{\pgfqpoint{5.800000in}{128.032290in}}%
\pgfpathlineto{\pgfqpoint{5.800000in}{128.135127in}}%
\pgfpathlineto{\pgfqpoint{5.800000in}{128.135127in}}%
\pgfpathclose%
\pgfusepath{fill}%
\end{pgfscope}%
\begin{pgfscope}%
\pgfpathrectangle{\pgfqpoint{5.800000in}{0.720000in}}{\pgfqpoint{1.400000in}{4.620000in}}%
\pgfusepath{clip}%
\pgfsetbuttcap%
\pgfsetmiterjoin%
\definecolor{currentfill}{rgb}{0.121569,0.466667,0.705882}%
\pgfsetfillcolor{currentfill}%
\pgfsetlinewidth{0.000000pt}%
\definecolor{currentstroke}{rgb}{0.000000,0.000000,0.000000}%
\pgfsetstrokecolor{currentstroke}%
\pgfsetstrokeopacity{0.000000}%
\pgfsetdash{}{0pt}%
\pgfpathmoveto{\pgfqpoint{5.800000in}{128.135127in}}%
\pgfpathlineto{\pgfqpoint{5.800000in}{128.135127in}}%
\pgfpathlineto{\pgfqpoint{5.800000in}{128.237964in}}%
\pgfpathlineto{\pgfqpoint{5.800000in}{128.237964in}}%
\pgfpathclose%
\pgfusepath{fill}%
\end{pgfscope}%
\begin{pgfscope}%
\pgfpathrectangle{\pgfqpoint{5.800000in}{0.720000in}}{\pgfqpoint{1.400000in}{4.620000in}}%
\pgfusepath{clip}%
\pgfsetbuttcap%
\pgfsetmiterjoin%
\definecolor{currentfill}{rgb}{0.121569,0.466667,0.705882}%
\pgfsetfillcolor{currentfill}%
\pgfsetlinewidth{0.000000pt}%
\definecolor{currentstroke}{rgb}{0.000000,0.000000,0.000000}%
\pgfsetstrokecolor{currentstroke}%
\pgfsetstrokeopacity{0.000000}%
\pgfsetdash{}{0pt}%
\pgfpathmoveto{\pgfqpoint{5.800000in}{128.237964in}}%
\pgfpathlineto{\pgfqpoint{5.800000in}{128.237964in}}%
\pgfpathlineto{\pgfqpoint{5.800000in}{128.340801in}}%
\pgfpathlineto{\pgfqpoint{5.800000in}{128.340801in}}%
\pgfpathclose%
\pgfusepath{fill}%
\end{pgfscope}%
\begin{pgfscope}%
\pgfpathrectangle{\pgfqpoint{5.800000in}{0.720000in}}{\pgfqpoint{1.400000in}{4.620000in}}%
\pgfusepath{clip}%
\pgfsetbuttcap%
\pgfsetmiterjoin%
\definecolor{currentfill}{rgb}{0.121569,0.466667,0.705882}%
\pgfsetfillcolor{currentfill}%
\pgfsetlinewidth{0.000000pt}%
\definecolor{currentstroke}{rgb}{0.000000,0.000000,0.000000}%
\pgfsetstrokecolor{currentstroke}%
\pgfsetstrokeopacity{0.000000}%
\pgfsetdash{}{0pt}%
\pgfpathmoveto{\pgfqpoint{5.800000in}{128.340801in}}%
\pgfpathlineto{\pgfqpoint{5.800000in}{128.340801in}}%
\pgfpathlineto{\pgfqpoint{5.800000in}{128.443638in}}%
\pgfpathlineto{\pgfqpoint{5.800000in}{128.443638in}}%
\pgfpathclose%
\pgfusepath{fill}%
\end{pgfscope}%
\begin{pgfscope}%
\pgfpathrectangle{\pgfqpoint{5.800000in}{0.720000in}}{\pgfqpoint{1.400000in}{4.620000in}}%
\pgfusepath{clip}%
\pgfsetbuttcap%
\pgfsetmiterjoin%
\definecolor{currentfill}{rgb}{0.121569,0.466667,0.705882}%
\pgfsetfillcolor{currentfill}%
\pgfsetlinewidth{0.000000pt}%
\definecolor{currentstroke}{rgb}{0.000000,0.000000,0.000000}%
\pgfsetstrokecolor{currentstroke}%
\pgfsetstrokeopacity{0.000000}%
\pgfsetdash{}{0pt}%
\pgfpathmoveto{\pgfqpoint{5.800000in}{128.443638in}}%
\pgfpathlineto{\pgfqpoint{5.800000in}{128.443638in}}%
\pgfpathlineto{\pgfqpoint{5.800000in}{128.546475in}}%
\pgfpathlineto{\pgfqpoint{5.800000in}{128.546475in}}%
\pgfpathclose%
\pgfusepath{fill}%
\end{pgfscope}%
\begin{pgfscope}%
\pgfpathrectangle{\pgfqpoint{5.800000in}{0.720000in}}{\pgfqpoint{1.400000in}{4.620000in}}%
\pgfusepath{clip}%
\pgfsetbuttcap%
\pgfsetmiterjoin%
\definecolor{currentfill}{rgb}{0.121569,0.466667,0.705882}%
\pgfsetfillcolor{currentfill}%
\pgfsetlinewidth{0.000000pt}%
\definecolor{currentstroke}{rgb}{0.000000,0.000000,0.000000}%
\pgfsetstrokecolor{currentstroke}%
\pgfsetstrokeopacity{0.000000}%
\pgfsetdash{}{0pt}%
\pgfpathmoveto{\pgfqpoint{5.800000in}{128.546475in}}%
\pgfpathlineto{\pgfqpoint{5.800000in}{128.546475in}}%
\pgfpathlineto{\pgfqpoint{5.800000in}{128.649312in}}%
\pgfpathlineto{\pgfqpoint{5.800000in}{128.649312in}}%
\pgfpathclose%
\pgfusepath{fill}%
\end{pgfscope}%
\begin{pgfscope}%
\pgfpathrectangle{\pgfqpoint{5.800000in}{0.720000in}}{\pgfqpoint{1.400000in}{4.620000in}}%
\pgfusepath{clip}%
\pgfsetbuttcap%
\pgfsetmiterjoin%
\definecolor{currentfill}{rgb}{0.121569,0.466667,0.705882}%
\pgfsetfillcolor{currentfill}%
\pgfsetlinewidth{0.000000pt}%
\definecolor{currentstroke}{rgb}{0.000000,0.000000,0.000000}%
\pgfsetstrokecolor{currentstroke}%
\pgfsetstrokeopacity{0.000000}%
\pgfsetdash{}{0pt}%
\pgfpathmoveto{\pgfqpoint{5.800000in}{128.649312in}}%
\pgfpathlineto{\pgfqpoint{5.800000in}{128.649312in}}%
\pgfpathlineto{\pgfqpoint{5.800000in}{128.752149in}}%
\pgfpathlineto{\pgfqpoint{5.800000in}{128.752149in}}%
\pgfpathclose%
\pgfusepath{fill}%
\end{pgfscope}%
\begin{pgfscope}%
\pgfpathrectangle{\pgfqpoint{5.800000in}{0.720000in}}{\pgfqpoint{1.400000in}{4.620000in}}%
\pgfusepath{clip}%
\pgfsetbuttcap%
\pgfsetmiterjoin%
\definecolor{currentfill}{rgb}{0.121569,0.466667,0.705882}%
\pgfsetfillcolor{currentfill}%
\pgfsetlinewidth{0.000000pt}%
\definecolor{currentstroke}{rgb}{0.000000,0.000000,0.000000}%
\pgfsetstrokecolor{currentstroke}%
\pgfsetstrokeopacity{0.000000}%
\pgfsetdash{}{0pt}%
\pgfpathmoveto{\pgfqpoint{5.800000in}{128.752149in}}%
\pgfpathlineto{\pgfqpoint{5.800000in}{128.752149in}}%
\pgfpathlineto{\pgfqpoint{5.800000in}{128.854986in}}%
\pgfpathlineto{\pgfqpoint{5.800000in}{128.854986in}}%
\pgfpathclose%
\pgfusepath{fill}%
\end{pgfscope}%
\begin{pgfscope}%
\pgfpathrectangle{\pgfqpoint{5.800000in}{0.720000in}}{\pgfqpoint{1.400000in}{4.620000in}}%
\pgfusepath{clip}%
\pgfsetbuttcap%
\pgfsetmiterjoin%
\definecolor{currentfill}{rgb}{0.121569,0.466667,0.705882}%
\pgfsetfillcolor{currentfill}%
\pgfsetlinewidth{0.000000pt}%
\definecolor{currentstroke}{rgb}{0.000000,0.000000,0.000000}%
\pgfsetstrokecolor{currentstroke}%
\pgfsetstrokeopacity{0.000000}%
\pgfsetdash{}{0pt}%
\pgfpathmoveto{\pgfqpoint{5.800000in}{128.854986in}}%
\pgfpathlineto{\pgfqpoint{5.800000in}{128.854986in}}%
\pgfpathlineto{\pgfqpoint{5.800000in}{128.957824in}}%
\pgfpathlineto{\pgfqpoint{5.800000in}{128.957824in}}%
\pgfpathclose%
\pgfusepath{fill}%
\end{pgfscope}%
\begin{pgfscope}%
\pgfpathrectangle{\pgfqpoint{5.800000in}{0.720000in}}{\pgfqpoint{1.400000in}{4.620000in}}%
\pgfusepath{clip}%
\pgfsetbuttcap%
\pgfsetmiterjoin%
\definecolor{currentfill}{rgb}{0.121569,0.466667,0.705882}%
\pgfsetfillcolor{currentfill}%
\pgfsetlinewidth{0.000000pt}%
\definecolor{currentstroke}{rgb}{0.000000,0.000000,0.000000}%
\pgfsetstrokecolor{currentstroke}%
\pgfsetstrokeopacity{0.000000}%
\pgfsetdash{}{0pt}%
\pgfpathmoveto{\pgfqpoint{5.800000in}{128.957824in}}%
\pgfpathlineto{\pgfqpoint{5.800000in}{128.957824in}}%
\pgfpathlineto{\pgfqpoint{5.800000in}{129.060661in}}%
\pgfpathlineto{\pgfqpoint{5.800000in}{129.060661in}}%
\pgfpathclose%
\pgfusepath{fill}%
\end{pgfscope}%
\begin{pgfscope}%
\pgfpathrectangle{\pgfqpoint{5.800000in}{0.720000in}}{\pgfqpoint{1.400000in}{4.620000in}}%
\pgfusepath{clip}%
\pgfsetbuttcap%
\pgfsetmiterjoin%
\definecolor{currentfill}{rgb}{0.121569,0.466667,0.705882}%
\pgfsetfillcolor{currentfill}%
\pgfsetlinewidth{0.000000pt}%
\definecolor{currentstroke}{rgb}{0.000000,0.000000,0.000000}%
\pgfsetstrokecolor{currentstroke}%
\pgfsetstrokeopacity{0.000000}%
\pgfsetdash{}{0pt}%
\pgfpathmoveto{\pgfqpoint{5.800000in}{129.060661in}}%
\pgfpathlineto{\pgfqpoint{5.800000in}{129.060661in}}%
\pgfpathlineto{\pgfqpoint{5.800000in}{129.163498in}}%
\pgfpathlineto{\pgfqpoint{5.800000in}{129.163498in}}%
\pgfpathclose%
\pgfusepath{fill}%
\end{pgfscope}%
\begin{pgfscope}%
\pgfpathrectangle{\pgfqpoint{5.800000in}{0.720000in}}{\pgfqpoint{1.400000in}{4.620000in}}%
\pgfusepath{clip}%
\pgfsetbuttcap%
\pgfsetmiterjoin%
\definecolor{currentfill}{rgb}{0.121569,0.466667,0.705882}%
\pgfsetfillcolor{currentfill}%
\pgfsetlinewidth{0.000000pt}%
\definecolor{currentstroke}{rgb}{0.000000,0.000000,0.000000}%
\pgfsetstrokecolor{currentstroke}%
\pgfsetstrokeopacity{0.000000}%
\pgfsetdash{}{0pt}%
\pgfpathmoveto{\pgfqpoint{5.800000in}{129.163498in}}%
\pgfpathlineto{\pgfqpoint{5.800000in}{129.163498in}}%
\pgfpathlineto{\pgfqpoint{5.800000in}{129.266335in}}%
\pgfpathlineto{\pgfqpoint{5.800000in}{129.266335in}}%
\pgfpathclose%
\pgfusepath{fill}%
\end{pgfscope}%
\begin{pgfscope}%
\pgfpathrectangle{\pgfqpoint{5.800000in}{0.720000in}}{\pgfqpoint{1.400000in}{4.620000in}}%
\pgfusepath{clip}%
\pgfsetbuttcap%
\pgfsetmiterjoin%
\definecolor{currentfill}{rgb}{0.121569,0.466667,0.705882}%
\pgfsetfillcolor{currentfill}%
\pgfsetlinewidth{0.000000pt}%
\definecolor{currentstroke}{rgb}{0.000000,0.000000,0.000000}%
\pgfsetstrokecolor{currentstroke}%
\pgfsetstrokeopacity{0.000000}%
\pgfsetdash{}{0pt}%
\pgfpathmoveto{\pgfqpoint{5.800000in}{129.266335in}}%
\pgfpathlineto{\pgfqpoint{5.800000in}{129.266335in}}%
\pgfpathlineto{\pgfqpoint{5.800000in}{129.369172in}}%
\pgfpathlineto{\pgfqpoint{5.800000in}{129.369172in}}%
\pgfpathclose%
\pgfusepath{fill}%
\end{pgfscope}%
\begin{pgfscope}%
\pgfpathrectangle{\pgfqpoint{5.800000in}{0.720000in}}{\pgfqpoint{1.400000in}{4.620000in}}%
\pgfusepath{clip}%
\pgfsetbuttcap%
\pgfsetmiterjoin%
\definecolor{currentfill}{rgb}{0.121569,0.466667,0.705882}%
\pgfsetfillcolor{currentfill}%
\pgfsetlinewidth{0.000000pt}%
\definecolor{currentstroke}{rgb}{0.000000,0.000000,0.000000}%
\pgfsetstrokecolor{currentstroke}%
\pgfsetstrokeopacity{0.000000}%
\pgfsetdash{}{0pt}%
\pgfpathmoveto{\pgfqpoint{5.800000in}{129.369172in}}%
\pgfpathlineto{\pgfqpoint{5.800000in}{129.369172in}}%
\pgfpathlineto{\pgfqpoint{5.800000in}{129.472009in}}%
\pgfpathlineto{\pgfqpoint{5.800000in}{129.472009in}}%
\pgfpathclose%
\pgfusepath{fill}%
\end{pgfscope}%
\begin{pgfscope}%
\pgfpathrectangle{\pgfqpoint{5.800000in}{0.720000in}}{\pgfqpoint{1.400000in}{4.620000in}}%
\pgfusepath{clip}%
\pgfsetbuttcap%
\pgfsetmiterjoin%
\definecolor{currentfill}{rgb}{0.121569,0.466667,0.705882}%
\pgfsetfillcolor{currentfill}%
\pgfsetlinewidth{0.000000pt}%
\definecolor{currentstroke}{rgb}{0.000000,0.000000,0.000000}%
\pgfsetstrokecolor{currentstroke}%
\pgfsetstrokeopacity{0.000000}%
\pgfsetdash{}{0pt}%
\pgfpathmoveto{\pgfqpoint{5.800000in}{129.472009in}}%
\pgfpathlineto{\pgfqpoint{5.800000in}{129.472009in}}%
\pgfpathlineto{\pgfqpoint{5.800000in}{129.574846in}}%
\pgfpathlineto{\pgfqpoint{5.800000in}{129.574846in}}%
\pgfpathclose%
\pgfusepath{fill}%
\end{pgfscope}%
\begin{pgfscope}%
\pgfpathrectangle{\pgfqpoint{5.800000in}{0.720000in}}{\pgfqpoint{1.400000in}{4.620000in}}%
\pgfusepath{clip}%
\pgfsetbuttcap%
\pgfsetmiterjoin%
\definecolor{currentfill}{rgb}{0.121569,0.466667,0.705882}%
\pgfsetfillcolor{currentfill}%
\pgfsetlinewidth{0.000000pt}%
\definecolor{currentstroke}{rgb}{0.000000,0.000000,0.000000}%
\pgfsetstrokecolor{currentstroke}%
\pgfsetstrokeopacity{0.000000}%
\pgfsetdash{}{0pt}%
\pgfpathmoveto{\pgfqpoint{5.800000in}{129.574846in}}%
\pgfpathlineto{\pgfqpoint{5.800000in}{129.574846in}}%
\pgfpathlineto{\pgfqpoint{5.800000in}{129.677683in}}%
\pgfpathlineto{\pgfqpoint{5.800000in}{129.677683in}}%
\pgfpathclose%
\pgfusepath{fill}%
\end{pgfscope}%
\begin{pgfscope}%
\pgfpathrectangle{\pgfqpoint{5.800000in}{0.720000in}}{\pgfqpoint{1.400000in}{4.620000in}}%
\pgfusepath{clip}%
\pgfsetbuttcap%
\pgfsetmiterjoin%
\definecolor{currentfill}{rgb}{0.121569,0.466667,0.705882}%
\pgfsetfillcolor{currentfill}%
\pgfsetlinewidth{0.000000pt}%
\definecolor{currentstroke}{rgb}{0.000000,0.000000,0.000000}%
\pgfsetstrokecolor{currentstroke}%
\pgfsetstrokeopacity{0.000000}%
\pgfsetdash{}{0pt}%
\pgfpathmoveto{\pgfqpoint{5.800000in}{129.677683in}}%
\pgfpathlineto{\pgfqpoint{5.800000in}{129.677683in}}%
\pgfpathlineto{\pgfqpoint{5.800000in}{129.780520in}}%
\pgfpathlineto{\pgfqpoint{5.800000in}{129.780520in}}%
\pgfpathclose%
\pgfusepath{fill}%
\end{pgfscope}%
\begin{pgfscope}%
\pgfpathrectangle{\pgfqpoint{5.800000in}{0.720000in}}{\pgfqpoint{1.400000in}{4.620000in}}%
\pgfusepath{clip}%
\pgfsetbuttcap%
\pgfsetmiterjoin%
\definecolor{currentfill}{rgb}{0.121569,0.466667,0.705882}%
\pgfsetfillcolor{currentfill}%
\pgfsetlinewidth{0.000000pt}%
\definecolor{currentstroke}{rgb}{0.000000,0.000000,0.000000}%
\pgfsetstrokecolor{currentstroke}%
\pgfsetstrokeopacity{0.000000}%
\pgfsetdash{}{0pt}%
\pgfpathmoveto{\pgfqpoint{5.800000in}{129.780520in}}%
\pgfpathlineto{\pgfqpoint{5.800000in}{129.780520in}}%
\pgfpathlineto{\pgfqpoint{5.800000in}{129.883357in}}%
\pgfpathlineto{\pgfqpoint{5.800000in}{129.883357in}}%
\pgfpathclose%
\pgfusepath{fill}%
\end{pgfscope}%
\begin{pgfscope}%
\pgfpathrectangle{\pgfqpoint{5.800000in}{0.720000in}}{\pgfqpoint{1.400000in}{4.620000in}}%
\pgfusepath{clip}%
\pgfsetbuttcap%
\pgfsetmiterjoin%
\definecolor{currentfill}{rgb}{0.121569,0.466667,0.705882}%
\pgfsetfillcolor{currentfill}%
\pgfsetlinewidth{0.000000pt}%
\definecolor{currentstroke}{rgb}{0.000000,0.000000,0.000000}%
\pgfsetstrokecolor{currentstroke}%
\pgfsetstrokeopacity{0.000000}%
\pgfsetdash{}{0pt}%
\pgfpathmoveto{\pgfqpoint{5.800000in}{129.883357in}}%
\pgfpathlineto{\pgfqpoint{5.800000in}{129.883357in}}%
\pgfpathlineto{\pgfqpoint{5.800000in}{129.986194in}}%
\pgfpathlineto{\pgfqpoint{5.800000in}{129.986194in}}%
\pgfpathclose%
\pgfusepath{fill}%
\end{pgfscope}%
\begin{pgfscope}%
\pgfpathrectangle{\pgfqpoint{5.800000in}{0.720000in}}{\pgfqpoint{1.400000in}{4.620000in}}%
\pgfusepath{clip}%
\pgfsetbuttcap%
\pgfsetmiterjoin%
\definecolor{currentfill}{rgb}{0.121569,0.466667,0.705882}%
\pgfsetfillcolor{currentfill}%
\pgfsetlinewidth{0.000000pt}%
\definecolor{currentstroke}{rgb}{0.000000,0.000000,0.000000}%
\pgfsetstrokecolor{currentstroke}%
\pgfsetstrokeopacity{0.000000}%
\pgfsetdash{}{0pt}%
\pgfpathmoveto{\pgfqpoint{5.800000in}{129.986194in}}%
\pgfpathlineto{\pgfqpoint{5.800000in}{129.986194in}}%
\pgfpathlineto{\pgfqpoint{5.800000in}{130.089031in}}%
\pgfpathlineto{\pgfqpoint{5.800000in}{130.089031in}}%
\pgfpathclose%
\pgfusepath{fill}%
\end{pgfscope}%
\begin{pgfscope}%
\pgfpathrectangle{\pgfqpoint{5.800000in}{0.720000in}}{\pgfqpoint{1.400000in}{4.620000in}}%
\pgfusepath{clip}%
\pgfsetbuttcap%
\pgfsetmiterjoin%
\definecolor{currentfill}{rgb}{0.121569,0.466667,0.705882}%
\pgfsetfillcolor{currentfill}%
\pgfsetlinewidth{0.000000pt}%
\definecolor{currentstroke}{rgb}{0.000000,0.000000,0.000000}%
\pgfsetstrokecolor{currentstroke}%
\pgfsetstrokeopacity{0.000000}%
\pgfsetdash{}{0pt}%
\pgfpathmoveto{\pgfqpoint{5.800000in}{130.089031in}}%
\pgfpathlineto{\pgfqpoint{5.800000in}{130.089031in}}%
\pgfpathlineto{\pgfqpoint{5.800000in}{130.191868in}}%
\pgfpathlineto{\pgfqpoint{5.800000in}{130.191868in}}%
\pgfpathclose%
\pgfusepath{fill}%
\end{pgfscope}%
\begin{pgfscope}%
\pgfpathrectangle{\pgfqpoint{5.800000in}{0.720000in}}{\pgfqpoint{1.400000in}{4.620000in}}%
\pgfusepath{clip}%
\pgfsetbuttcap%
\pgfsetmiterjoin%
\definecolor{currentfill}{rgb}{0.121569,0.466667,0.705882}%
\pgfsetfillcolor{currentfill}%
\pgfsetlinewidth{0.000000pt}%
\definecolor{currentstroke}{rgb}{0.000000,0.000000,0.000000}%
\pgfsetstrokecolor{currentstroke}%
\pgfsetstrokeopacity{0.000000}%
\pgfsetdash{}{0pt}%
\pgfpathmoveto{\pgfqpoint{5.800000in}{130.191868in}}%
\pgfpathlineto{\pgfqpoint{5.800000in}{130.191868in}}%
\pgfpathlineto{\pgfqpoint{5.800000in}{130.294705in}}%
\pgfpathlineto{\pgfqpoint{5.800000in}{130.294705in}}%
\pgfpathclose%
\pgfusepath{fill}%
\end{pgfscope}%
\begin{pgfscope}%
\pgfpathrectangle{\pgfqpoint{5.800000in}{0.720000in}}{\pgfqpoint{1.400000in}{4.620000in}}%
\pgfusepath{clip}%
\pgfsetbuttcap%
\pgfsetmiterjoin%
\definecolor{currentfill}{rgb}{0.121569,0.466667,0.705882}%
\pgfsetfillcolor{currentfill}%
\pgfsetlinewidth{0.000000pt}%
\definecolor{currentstroke}{rgb}{0.000000,0.000000,0.000000}%
\pgfsetstrokecolor{currentstroke}%
\pgfsetstrokeopacity{0.000000}%
\pgfsetdash{}{0pt}%
\pgfpathmoveto{\pgfqpoint{5.800000in}{130.294705in}}%
\pgfpathlineto{\pgfqpoint{5.800000in}{130.294705in}}%
\pgfpathlineto{\pgfqpoint{5.800000in}{130.397542in}}%
\pgfpathlineto{\pgfqpoint{5.800000in}{130.397542in}}%
\pgfpathclose%
\pgfusepath{fill}%
\end{pgfscope}%
\begin{pgfscope}%
\pgfpathrectangle{\pgfqpoint{5.800000in}{0.720000in}}{\pgfqpoint{1.400000in}{4.620000in}}%
\pgfusepath{clip}%
\pgfsetbuttcap%
\pgfsetmiterjoin%
\definecolor{currentfill}{rgb}{0.121569,0.466667,0.705882}%
\pgfsetfillcolor{currentfill}%
\pgfsetlinewidth{0.000000pt}%
\definecolor{currentstroke}{rgb}{0.000000,0.000000,0.000000}%
\pgfsetstrokecolor{currentstroke}%
\pgfsetstrokeopacity{0.000000}%
\pgfsetdash{}{0pt}%
\pgfpathmoveto{\pgfqpoint{5.800000in}{130.397542in}}%
\pgfpathlineto{\pgfqpoint{5.800000in}{130.397542in}}%
\pgfpathlineto{\pgfqpoint{5.800000in}{130.500380in}}%
\pgfpathlineto{\pgfqpoint{5.800000in}{130.500380in}}%
\pgfpathclose%
\pgfusepath{fill}%
\end{pgfscope}%
\begin{pgfscope}%
\pgfpathrectangle{\pgfqpoint{5.800000in}{0.720000in}}{\pgfqpoint{1.400000in}{4.620000in}}%
\pgfusepath{clip}%
\pgfsetbuttcap%
\pgfsetmiterjoin%
\definecolor{currentfill}{rgb}{0.121569,0.466667,0.705882}%
\pgfsetfillcolor{currentfill}%
\pgfsetlinewidth{0.000000pt}%
\definecolor{currentstroke}{rgb}{0.000000,0.000000,0.000000}%
\pgfsetstrokecolor{currentstroke}%
\pgfsetstrokeopacity{0.000000}%
\pgfsetdash{}{0pt}%
\pgfpathmoveto{\pgfqpoint{5.800000in}{130.500380in}}%
\pgfpathlineto{\pgfqpoint{5.800000in}{130.500380in}}%
\pgfpathlineto{\pgfqpoint{5.800000in}{130.603217in}}%
\pgfpathlineto{\pgfqpoint{5.800000in}{130.603217in}}%
\pgfpathclose%
\pgfusepath{fill}%
\end{pgfscope}%
\begin{pgfscope}%
\pgfpathrectangle{\pgfqpoint{5.800000in}{0.720000in}}{\pgfqpoint{1.400000in}{4.620000in}}%
\pgfusepath{clip}%
\pgfsetbuttcap%
\pgfsetmiterjoin%
\definecolor{currentfill}{rgb}{0.121569,0.466667,0.705882}%
\pgfsetfillcolor{currentfill}%
\pgfsetlinewidth{0.000000pt}%
\definecolor{currentstroke}{rgb}{0.000000,0.000000,0.000000}%
\pgfsetstrokecolor{currentstroke}%
\pgfsetstrokeopacity{0.000000}%
\pgfsetdash{}{0pt}%
\pgfpathmoveto{\pgfqpoint{5.800000in}{130.603217in}}%
\pgfpathlineto{\pgfqpoint{5.800000in}{130.603217in}}%
\pgfpathlineto{\pgfqpoint{5.800000in}{130.706054in}}%
\pgfpathlineto{\pgfqpoint{5.800000in}{130.706054in}}%
\pgfpathclose%
\pgfusepath{fill}%
\end{pgfscope}%
\begin{pgfscope}%
\pgfpathrectangle{\pgfqpoint{5.800000in}{0.720000in}}{\pgfqpoint{1.400000in}{4.620000in}}%
\pgfusepath{clip}%
\pgfsetbuttcap%
\pgfsetmiterjoin%
\definecolor{currentfill}{rgb}{0.121569,0.466667,0.705882}%
\pgfsetfillcolor{currentfill}%
\pgfsetlinewidth{0.000000pt}%
\definecolor{currentstroke}{rgb}{0.000000,0.000000,0.000000}%
\pgfsetstrokecolor{currentstroke}%
\pgfsetstrokeopacity{0.000000}%
\pgfsetdash{}{0pt}%
\pgfpathmoveto{\pgfqpoint{5.800000in}{130.706054in}}%
\pgfpathlineto{\pgfqpoint{5.800000in}{130.706054in}}%
\pgfpathlineto{\pgfqpoint{5.800000in}{130.808891in}}%
\pgfpathlineto{\pgfqpoint{5.800000in}{130.808891in}}%
\pgfpathclose%
\pgfusepath{fill}%
\end{pgfscope}%
\begin{pgfscope}%
\pgfpathrectangle{\pgfqpoint{5.800000in}{0.720000in}}{\pgfqpoint{1.400000in}{4.620000in}}%
\pgfusepath{clip}%
\pgfsetbuttcap%
\pgfsetmiterjoin%
\definecolor{currentfill}{rgb}{0.121569,0.466667,0.705882}%
\pgfsetfillcolor{currentfill}%
\pgfsetlinewidth{0.000000pt}%
\definecolor{currentstroke}{rgb}{0.000000,0.000000,0.000000}%
\pgfsetstrokecolor{currentstroke}%
\pgfsetstrokeopacity{0.000000}%
\pgfsetdash{}{0pt}%
\pgfpathmoveto{\pgfqpoint{5.800000in}{130.808891in}}%
\pgfpathlineto{\pgfqpoint{5.800000in}{130.808891in}}%
\pgfpathlineto{\pgfqpoint{5.800000in}{130.911728in}}%
\pgfpathlineto{\pgfqpoint{5.800000in}{130.911728in}}%
\pgfpathclose%
\pgfusepath{fill}%
\end{pgfscope}%
\begin{pgfscope}%
\pgfpathrectangle{\pgfqpoint{5.800000in}{0.720000in}}{\pgfqpoint{1.400000in}{4.620000in}}%
\pgfusepath{clip}%
\pgfsetbuttcap%
\pgfsetmiterjoin%
\definecolor{currentfill}{rgb}{0.121569,0.466667,0.705882}%
\pgfsetfillcolor{currentfill}%
\pgfsetlinewidth{0.000000pt}%
\definecolor{currentstroke}{rgb}{0.000000,0.000000,0.000000}%
\pgfsetstrokecolor{currentstroke}%
\pgfsetstrokeopacity{0.000000}%
\pgfsetdash{}{0pt}%
\pgfpathmoveto{\pgfqpoint{5.800000in}{130.911728in}}%
\pgfpathlineto{\pgfqpoint{5.800000in}{130.911728in}}%
\pgfpathlineto{\pgfqpoint{5.800000in}{131.014565in}}%
\pgfpathlineto{\pgfqpoint{5.800000in}{131.014565in}}%
\pgfpathclose%
\pgfusepath{fill}%
\end{pgfscope}%
\begin{pgfscope}%
\pgfpathrectangle{\pgfqpoint{5.800000in}{0.720000in}}{\pgfqpoint{1.400000in}{4.620000in}}%
\pgfusepath{clip}%
\pgfsetbuttcap%
\pgfsetmiterjoin%
\definecolor{currentfill}{rgb}{0.121569,0.466667,0.705882}%
\pgfsetfillcolor{currentfill}%
\pgfsetlinewidth{0.000000pt}%
\definecolor{currentstroke}{rgb}{0.000000,0.000000,0.000000}%
\pgfsetstrokecolor{currentstroke}%
\pgfsetstrokeopacity{0.000000}%
\pgfsetdash{}{0pt}%
\pgfpathmoveto{\pgfqpoint{5.800000in}{131.014565in}}%
\pgfpathlineto{\pgfqpoint{5.800000in}{131.014565in}}%
\pgfpathlineto{\pgfqpoint{5.800000in}{131.117402in}}%
\pgfpathlineto{\pgfqpoint{5.800000in}{131.117402in}}%
\pgfpathclose%
\pgfusepath{fill}%
\end{pgfscope}%
\begin{pgfscope}%
\pgfpathrectangle{\pgfqpoint{5.800000in}{0.720000in}}{\pgfqpoint{1.400000in}{4.620000in}}%
\pgfusepath{clip}%
\pgfsetbuttcap%
\pgfsetmiterjoin%
\definecolor{currentfill}{rgb}{0.121569,0.466667,0.705882}%
\pgfsetfillcolor{currentfill}%
\pgfsetlinewidth{0.000000pt}%
\definecolor{currentstroke}{rgb}{0.000000,0.000000,0.000000}%
\pgfsetstrokecolor{currentstroke}%
\pgfsetstrokeopacity{0.000000}%
\pgfsetdash{}{0pt}%
\pgfpathmoveto{\pgfqpoint{5.800000in}{131.117402in}}%
\pgfpathlineto{\pgfqpoint{5.800000in}{131.117402in}}%
\pgfpathlineto{\pgfqpoint{5.800000in}{131.220239in}}%
\pgfpathlineto{\pgfqpoint{5.800000in}{131.220239in}}%
\pgfpathclose%
\pgfusepath{fill}%
\end{pgfscope}%
\begin{pgfscope}%
\pgfpathrectangle{\pgfqpoint{5.800000in}{0.720000in}}{\pgfqpoint{1.400000in}{4.620000in}}%
\pgfusepath{clip}%
\pgfsetbuttcap%
\pgfsetmiterjoin%
\definecolor{currentfill}{rgb}{0.121569,0.466667,0.705882}%
\pgfsetfillcolor{currentfill}%
\pgfsetlinewidth{0.000000pt}%
\definecolor{currentstroke}{rgb}{0.000000,0.000000,0.000000}%
\pgfsetstrokecolor{currentstroke}%
\pgfsetstrokeopacity{0.000000}%
\pgfsetdash{}{0pt}%
\pgfpathmoveto{\pgfqpoint{5.800000in}{131.220239in}}%
\pgfpathlineto{\pgfqpoint{5.800000in}{131.220239in}}%
\pgfpathlineto{\pgfqpoint{5.800000in}{131.323076in}}%
\pgfpathlineto{\pgfqpoint{5.800000in}{131.323076in}}%
\pgfpathclose%
\pgfusepath{fill}%
\end{pgfscope}%
\begin{pgfscope}%
\pgfpathrectangle{\pgfqpoint{5.800000in}{0.720000in}}{\pgfqpoint{1.400000in}{4.620000in}}%
\pgfusepath{clip}%
\pgfsetbuttcap%
\pgfsetmiterjoin%
\definecolor{currentfill}{rgb}{0.121569,0.466667,0.705882}%
\pgfsetfillcolor{currentfill}%
\pgfsetlinewidth{0.000000pt}%
\definecolor{currentstroke}{rgb}{0.000000,0.000000,0.000000}%
\pgfsetstrokecolor{currentstroke}%
\pgfsetstrokeopacity{0.000000}%
\pgfsetdash{}{0pt}%
\pgfpathmoveto{\pgfqpoint{5.800000in}{131.323076in}}%
\pgfpathlineto{\pgfqpoint{5.800000in}{131.323076in}}%
\pgfpathlineto{\pgfqpoint{5.800000in}{131.425913in}}%
\pgfpathlineto{\pgfqpoint{5.800000in}{131.425913in}}%
\pgfpathclose%
\pgfusepath{fill}%
\end{pgfscope}%
\begin{pgfscope}%
\pgfpathrectangle{\pgfqpoint{5.800000in}{0.720000in}}{\pgfqpoint{1.400000in}{4.620000in}}%
\pgfusepath{clip}%
\pgfsetbuttcap%
\pgfsetmiterjoin%
\definecolor{currentfill}{rgb}{0.121569,0.466667,0.705882}%
\pgfsetfillcolor{currentfill}%
\pgfsetlinewidth{0.000000pt}%
\definecolor{currentstroke}{rgb}{0.000000,0.000000,0.000000}%
\pgfsetstrokecolor{currentstroke}%
\pgfsetstrokeopacity{0.000000}%
\pgfsetdash{}{0pt}%
\pgfpathmoveto{\pgfqpoint{5.800000in}{131.425913in}}%
\pgfpathlineto{\pgfqpoint{5.800000in}{131.425913in}}%
\pgfpathlineto{\pgfqpoint{5.800000in}{131.528750in}}%
\pgfpathlineto{\pgfqpoint{5.800000in}{131.528750in}}%
\pgfpathclose%
\pgfusepath{fill}%
\end{pgfscope}%
\begin{pgfscope}%
\pgfpathrectangle{\pgfqpoint{5.800000in}{0.720000in}}{\pgfqpoint{1.400000in}{4.620000in}}%
\pgfusepath{clip}%
\pgfsetbuttcap%
\pgfsetmiterjoin%
\definecolor{currentfill}{rgb}{0.121569,0.466667,0.705882}%
\pgfsetfillcolor{currentfill}%
\pgfsetlinewidth{0.000000pt}%
\definecolor{currentstroke}{rgb}{0.000000,0.000000,0.000000}%
\pgfsetstrokecolor{currentstroke}%
\pgfsetstrokeopacity{0.000000}%
\pgfsetdash{}{0pt}%
\pgfpathmoveto{\pgfqpoint{5.800000in}{131.528750in}}%
\pgfpathlineto{\pgfqpoint{5.800000in}{131.528750in}}%
\pgfpathlineto{\pgfqpoint{5.800000in}{131.631587in}}%
\pgfpathlineto{\pgfqpoint{5.800000in}{131.631587in}}%
\pgfpathclose%
\pgfusepath{fill}%
\end{pgfscope}%
\begin{pgfscope}%
\pgfpathrectangle{\pgfqpoint{5.800000in}{0.720000in}}{\pgfqpoint{1.400000in}{4.620000in}}%
\pgfusepath{clip}%
\pgfsetbuttcap%
\pgfsetmiterjoin%
\definecolor{currentfill}{rgb}{0.121569,0.466667,0.705882}%
\pgfsetfillcolor{currentfill}%
\pgfsetlinewidth{0.000000pt}%
\definecolor{currentstroke}{rgb}{0.000000,0.000000,0.000000}%
\pgfsetstrokecolor{currentstroke}%
\pgfsetstrokeopacity{0.000000}%
\pgfsetdash{}{0pt}%
\pgfpathmoveto{\pgfqpoint{5.800000in}{131.631587in}}%
\pgfpathlineto{\pgfqpoint{5.800000in}{131.631587in}}%
\pgfpathlineto{\pgfqpoint{5.800000in}{131.734424in}}%
\pgfpathlineto{\pgfqpoint{5.800000in}{131.734424in}}%
\pgfpathclose%
\pgfusepath{fill}%
\end{pgfscope}%
\begin{pgfscope}%
\pgfpathrectangle{\pgfqpoint{5.800000in}{0.720000in}}{\pgfqpoint{1.400000in}{4.620000in}}%
\pgfusepath{clip}%
\pgfsetbuttcap%
\pgfsetmiterjoin%
\definecolor{currentfill}{rgb}{0.121569,0.466667,0.705882}%
\pgfsetfillcolor{currentfill}%
\pgfsetlinewidth{0.000000pt}%
\definecolor{currentstroke}{rgb}{0.000000,0.000000,0.000000}%
\pgfsetstrokecolor{currentstroke}%
\pgfsetstrokeopacity{0.000000}%
\pgfsetdash{}{0pt}%
\pgfpathmoveto{\pgfqpoint{5.800000in}{131.734424in}}%
\pgfpathlineto{\pgfqpoint{5.800000in}{131.734424in}}%
\pgfpathlineto{\pgfqpoint{5.800000in}{131.837261in}}%
\pgfpathlineto{\pgfqpoint{5.800000in}{131.837261in}}%
\pgfpathclose%
\pgfusepath{fill}%
\end{pgfscope}%
\begin{pgfscope}%
\pgfpathrectangle{\pgfqpoint{5.800000in}{0.720000in}}{\pgfqpoint{1.400000in}{4.620000in}}%
\pgfusepath{clip}%
\pgfsetbuttcap%
\pgfsetmiterjoin%
\definecolor{currentfill}{rgb}{0.121569,0.466667,0.705882}%
\pgfsetfillcolor{currentfill}%
\pgfsetlinewidth{0.000000pt}%
\definecolor{currentstroke}{rgb}{0.000000,0.000000,0.000000}%
\pgfsetstrokecolor{currentstroke}%
\pgfsetstrokeopacity{0.000000}%
\pgfsetdash{}{0pt}%
\pgfpathmoveto{\pgfqpoint{5.800000in}{131.837261in}}%
\pgfpathlineto{\pgfqpoint{5.800000in}{131.837261in}}%
\pgfpathlineto{\pgfqpoint{5.800000in}{131.940098in}}%
\pgfpathlineto{\pgfqpoint{5.800000in}{131.940098in}}%
\pgfpathclose%
\pgfusepath{fill}%
\end{pgfscope}%
\begin{pgfscope}%
\pgfpathrectangle{\pgfqpoint{5.800000in}{0.720000in}}{\pgfqpoint{1.400000in}{4.620000in}}%
\pgfusepath{clip}%
\pgfsetbuttcap%
\pgfsetmiterjoin%
\definecolor{currentfill}{rgb}{0.121569,0.466667,0.705882}%
\pgfsetfillcolor{currentfill}%
\pgfsetlinewidth{0.000000pt}%
\definecolor{currentstroke}{rgb}{0.000000,0.000000,0.000000}%
\pgfsetstrokecolor{currentstroke}%
\pgfsetstrokeopacity{0.000000}%
\pgfsetdash{}{0pt}%
\pgfpathmoveto{\pgfqpoint{5.800000in}{131.940098in}}%
\pgfpathlineto{\pgfqpoint{5.800000in}{131.940098in}}%
\pgfpathlineto{\pgfqpoint{5.800000in}{132.042936in}}%
\pgfpathlineto{\pgfqpoint{5.800000in}{132.042936in}}%
\pgfpathclose%
\pgfusepath{fill}%
\end{pgfscope}%
\begin{pgfscope}%
\pgfpathrectangle{\pgfqpoint{5.800000in}{0.720000in}}{\pgfqpoint{1.400000in}{4.620000in}}%
\pgfusepath{clip}%
\pgfsetbuttcap%
\pgfsetmiterjoin%
\definecolor{currentfill}{rgb}{0.121569,0.466667,0.705882}%
\pgfsetfillcolor{currentfill}%
\pgfsetlinewidth{0.000000pt}%
\definecolor{currentstroke}{rgb}{0.000000,0.000000,0.000000}%
\pgfsetstrokecolor{currentstroke}%
\pgfsetstrokeopacity{0.000000}%
\pgfsetdash{}{0pt}%
\pgfpathmoveto{\pgfqpoint{5.800000in}{132.042936in}}%
\pgfpathlineto{\pgfqpoint{5.800000in}{132.042936in}}%
\pgfpathlineto{\pgfqpoint{5.800000in}{132.145773in}}%
\pgfpathlineto{\pgfqpoint{5.800000in}{132.145773in}}%
\pgfpathclose%
\pgfusepath{fill}%
\end{pgfscope}%
\begin{pgfscope}%
\pgfpathrectangle{\pgfqpoint{5.800000in}{0.720000in}}{\pgfqpoint{1.400000in}{4.620000in}}%
\pgfusepath{clip}%
\pgfsetbuttcap%
\pgfsetmiterjoin%
\definecolor{currentfill}{rgb}{0.121569,0.466667,0.705882}%
\pgfsetfillcolor{currentfill}%
\pgfsetlinewidth{0.000000pt}%
\definecolor{currentstroke}{rgb}{0.000000,0.000000,0.000000}%
\pgfsetstrokecolor{currentstroke}%
\pgfsetstrokeopacity{0.000000}%
\pgfsetdash{}{0pt}%
\pgfpathmoveto{\pgfqpoint{5.800000in}{132.145773in}}%
\pgfpathlineto{\pgfqpoint{5.800000in}{132.145773in}}%
\pgfpathlineto{\pgfqpoint{5.800000in}{132.248610in}}%
\pgfpathlineto{\pgfqpoint{5.800000in}{132.248610in}}%
\pgfpathclose%
\pgfusepath{fill}%
\end{pgfscope}%
\begin{pgfscope}%
\pgfpathrectangle{\pgfqpoint{5.800000in}{0.720000in}}{\pgfqpoint{1.400000in}{4.620000in}}%
\pgfusepath{clip}%
\pgfsetbuttcap%
\pgfsetmiterjoin%
\definecolor{currentfill}{rgb}{0.121569,0.466667,0.705882}%
\pgfsetfillcolor{currentfill}%
\pgfsetlinewidth{0.000000pt}%
\definecolor{currentstroke}{rgb}{0.000000,0.000000,0.000000}%
\pgfsetstrokecolor{currentstroke}%
\pgfsetstrokeopacity{0.000000}%
\pgfsetdash{}{0pt}%
\pgfpathmoveto{\pgfqpoint{5.800000in}{132.248610in}}%
\pgfpathlineto{\pgfqpoint{5.800000in}{132.248610in}}%
\pgfpathlineto{\pgfqpoint{5.800000in}{132.351447in}}%
\pgfpathlineto{\pgfqpoint{5.800000in}{132.351447in}}%
\pgfpathclose%
\pgfusepath{fill}%
\end{pgfscope}%
\begin{pgfscope}%
\pgfpathrectangle{\pgfqpoint{5.800000in}{0.720000in}}{\pgfqpoint{1.400000in}{4.620000in}}%
\pgfusepath{clip}%
\pgfsetbuttcap%
\pgfsetmiterjoin%
\definecolor{currentfill}{rgb}{0.121569,0.466667,0.705882}%
\pgfsetfillcolor{currentfill}%
\pgfsetlinewidth{0.000000pt}%
\definecolor{currentstroke}{rgb}{0.000000,0.000000,0.000000}%
\pgfsetstrokecolor{currentstroke}%
\pgfsetstrokeopacity{0.000000}%
\pgfsetdash{}{0pt}%
\pgfpathmoveto{\pgfqpoint{5.800000in}{132.351447in}}%
\pgfpathlineto{\pgfqpoint{5.800000in}{132.351447in}}%
\pgfpathlineto{\pgfqpoint{5.800000in}{132.454284in}}%
\pgfpathlineto{\pgfqpoint{5.800000in}{132.454284in}}%
\pgfpathclose%
\pgfusepath{fill}%
\end{pgfscope}%
\begin{pgfscope}%
\pgfpathrectangle{\pgfqpoint{5.800000in}{0.720000in}}{\pgfqpoint{1.400000in}{4.620000in}}%
\pgfusepath{clip}%
\pgfsetbuttcap%
\pgfsetmiterjoin%
\definecolor{currentfill}{rgb}{0.121569,0.466667,0.705882}%
\pgfsetfillcolor{currentfill}%
\pgfsetlinewidth{0.000000pt}%
\definecolor{currentstroke}{rgb}{0.000000,0.000000,0.000000}%
\pgfsetstrokecolor{currentstroke}%
\pgfsetstrokeopacity{0.000000}%
\pgfsetdash{}{0pt}%
\pgfpathmoveto{\pgfqpoint{5.800000in}{132.454284in}}%
\pgfpathlineto{\pgfqpoint{5.800000in}{132.454284in}}%
\pgfpathlineto{\pgfqpoint{5.800000in}{132.557121in}}%
\pgfpathlineto{\pgfqpoint{5.800000in}{132.557121in}}%
\pgfpathclose%
\pgfusepath{fill}%
\end{pgfscope}%
\begin{pgfscope}%
\pgfpathrectangle{\pgfqpoint{5.800000in}{0.720000in}}{\pgfqpoint{1.400000in}{4.620000in}}%
\pgfusepath{clip}%
\pgfsetbuttcap%
\pgfsetmiterjoin%
\definecolor{currentfill}{rgb}{0.121569,0.466667,0.705882}%
\pgfsetfillcolor{currentfill}%
\pgfsetlinewidth{0.000000pt}%
\definecolor{currentstroke}{rgb}{0.000000,0.000000,0.000000}%
\pgfsetstrokecolor{currentstroke}%
\pgfsetstrokeopacity{0.000000}%
\pgfsetdash{}{0pt}%
\pgfpathmoveto{\pgfqpoint{5.800000in}{132.557121in}}%
\pgfpathlineto{\pgfqpoint{5.800000in}{132.557121in}}%
\pgfpathlineto{\pgfqpoint{5.800000in}{132.659958in}}%
\pgfpathlineto{\pgfqpoint{5.800000in}{132.659958in}}%
\pgfpathclose%
\pgfusepath{fill}%
\end{pgfscope}%
\begin{pgfscope}%
\pgfpathrectangle{\pgfqpoint{5.800000in}{0.720000in}}{\pgfqpoint{1.400000in}{4.620000in}}%
\pgfusepath{clip}%
\pgfsetbuttcap%
\pgfsetmiterjoin%
\definecolor{currentfill}{rgb}{0.121569,0.466667,0.705882}%
\pgfsetfillcolor{currentfill}%
\pgfsetlinewidth{0.000000pt}%
\definecolor{currentstroke}{rgb}{0.000000,0.000000,0.000000}%
\pgfsetstrokecolor{currentstroke}%
\pgfsetstrokeopacity{0.000000}%
\pgfsetdash{}{0pt}%
\pgfpathmoveto{\pgfqpoint{5.800000in}{132.659958in}}%
\pgfpathlineto{\pgfqpoint{5.800000in}{132.659958in}}%
\pgfpathlineto{\pgfqpoint{5.800000in}{132.762795in}}%
\pgfpathlineto{\pgfqpoint{5.800000in}{132.762795in}}%
\pgfpathclose%
\pgfusepath{fill}%
\end{pgfscope}%
\begin{pgfscope}%
\pgfpathrectangle{\pgfqpoint{5.800000in}{0.720000in}}{\pgfqpoint{1.400000in}{4.620000in}}%
\pgfusepath{clip}%
\pgfsetbuttcap%
\pgfsetmiterjoin%
\definecolor{currentfill}{rgb}{0.121569,0.466667,0.705882}%
\pgfsetfillcolor{currentfill}%
\pgfsetlinewidth{0.000000pt}%
\definecolor{currentstroke}{rgb}{0.000000,0.000000,0.000000}%
\pgfsetstrokecolor{currentstroke}%
\pgfsetstrokeopacity{0.000000}%
\pgfsetdash{}{0pt}%
\pgfpathmoveto{\pgfqpoint{5.800000in}{132.762795in}}%
\pgfpathlineto{\pgfqpoint{5.800000in}{132.762795in}}%
\pgfpathlineto{\pgfqpoint{5.800000in}{132.865632in}}%
\pgfpathlineto{\pgfqpoint{5.800000in}{132.865632in}}%
\pgfpathclose%
\pgfusepath{fill}%
\end{pgfscope}%
\begin{pgfscope}%
\pgfpathrectangle{\pgfqpoint{5.800000in}{0.720000in}}{\pgfqpoint{1.400000in}{4.620000in}}%
\pgfusepath{clip}%
\pgfsetbuttcap%
\pgfsetmiterjoin%
\definecolor{currentfill}{rgb}{0.121569,0.466667,0.705882}%
\pgfsetfillcolor{currentfill}%
\pgfsetlinewidth{0.000000pt}%
\definecolor{currentstroke}{rgb}{0.000000,0.000000,0.000000}%
\pgfsetstrokecolor{currentstroke}%
\pgfsetstrokeopacity{0.000000}%
\pgfsetdash{}{0pt}%
\pgfpathmoveto{\pgfqpoint{5.800000in}{132.865632in}}%
\pgfpathlineto{\pgfqpoint{5.800000in}{132.865632in}}%
\pgfpathlineto{\pgfqpoint{5.800000in}{132.968469in}}%
\pgfpathlineto{\pgfqpoint{5.800000in}{132.968469in}}%
\pgfpathclose%
\pgfusepath{fill}%
\end{pgfscope}%
\begin{pgfscope}%
\pgfpathrectangle{\pgfqpoint{5.800000in}{0.720000in}}{\pgfqpoint{1.400000in}{4.620000in}}%
\pgfusepath{clip}%
\pgfsetbuttcap%
\pgfsetmiterjoin%
\definecolor{currentfill}{rgb}{0.121569,0.466667,0.705882}%
\pgfsetfillcolor{currentfill}%
\pgfsetlinewidth{0.000000pt}%
\definecolor{currentstroke}{rgb}{0.000000,0.000000,0.000000}%
\pgfsetstrokecolor{currentstroke}%
\pgfsetstrokeopacity{0.000000}%
\pgfsetdash{}{0pt}%
\pgfpathmoveto{\pgfqpoint{5.800000in}{132.968469in}}%
\pgfpathlineto{\pgfqpoint{5.800000in}{132.968469in}}%
\pgfpathlineto{\pgfqpoint{5.800000in}{133.071306in}}%
\pgfpathlineto{\pgfqpoint{5.800000in}{133.071306in}}%
\pgfpathclose%
\pgfusepath{fill}%
\end{pgfscope}%
\begin{pgfscope}%
\pgfpathrectangle{\pgfqpoint{5.800000in}{0.720000in}}{\pgfqpoint{1.400000in}{4.620000in}}%
\pgfusepath{clip}%
\pgfsetbuttcap%
\pgfsetmiterjoin%
\definecolor{currentfill}{rgb}{0.121569,0.466667,0.705882}%
\pgfsetfillcolor{currentfill}%
\pgfsetlinewidth{0.000000pt}%
\definecolor{currentstroke}{rgb}{0.000000,0.000000,0.000000}%
\pgfsetstrokecolor{currentstroke}%
\pgfsetstrokeopacity{0.000000}%
\pgfsetdash{}{0pt}%
\pgfpathmoveto{\pgfqpoint{5.800000in}{133.071306in}}%
\pgfpathlineto{\pgfqpoint{5.800000in}{133.071306in}}%
\pgfpathlineto{\pgfqpoint{5.800000in}{133.174143in}}%
\pgfpathlineto{\pgfqpoint{5.800000in}{133.174143in}}%
\pgfpathclose%
\pgfusepath{fill}%
\end{pgfscope}%
\begin{pgfscope}%
\pgfpathrectangle{\pgfqpoint{5.800000in}{0.720000in}}{\pgfqpoint{1.400000in}{4.620000in}}%
\pgfusepath{clip}%
\pgfsetbuttcap%
\pgfsetmiterjoin%
\definecolor{currentfill}{rgb}{0.121569,0.466667,0.705882}%
\pgfsetfillcolor{currentfill}%
\pgfsetlinewidth{0.000000pt}%
\definecolor{currentstroke}{rgb}{0.000000,0.000000,0.000000}%
\pgfsetstrokecolor{currentstroke}%
\pgfsetstrokeopacity{0.000000}%
\pgfsetdash{}{0pt}%
\pgfpathmoveto{\pgfqpoint{5.800000in}{133.174143in}}%
\pgfpathlineto{\pgfqpoint{5.800000in}{133.174143in}}%
\pgfpathlineto{\pgfqpoint{5.800000in}{133.276980in}}%
\pgfpathlineto{\pgfqpoint{5.800000in}{133.276980in}}%
\pgfpathclose%
\pgfusepath{fill}%
\end{pgfscope}%
\begin{pgfscope}%
\pgfpathrectangle{\pgfqpoint{5.800000in}{0.720000in}}{\pgfqpoint{1.400000in}{4.620000in}}%
\pgfusepath{clip}%
\pgfsetbuttcap%
\pgfsetmiterjoin%
\definecolor{currentfill}{rgb}{0.121569,0.466667,0.705882}%
\pgfsetfillcolor{currentfill}%
\pgfsetlinewidth{0.000000pt}%
\definecolor{currentstroke}{rgb}{0.000000,0.000000,0.000000}%
\pgfsetstrokecolor{currentstroke}%
\pgfsetstrokeopacity{0.000000}%
\pgfsetdash{}{0pt}%
\pgfpathmoveto{\pgfqpoint{5.800000in}{133.276980in}}%
\pgfpathlineto{\pgfqpoint{5.800000in}{133.276980in}}%
\pgfpathlineto{\pgfqpoint{5.800000in}{133.379817in}}%
\pgfpathlineto{\pgfqpoint{5.800000in}{133.379817in}}%
\pgfpathclose%
\pgfusepath{fill}%
\end{pgfscope}%
\begin{pgfscope}%
\pgfpathrectangle{\pgfqpoint{5.800000in}{0.720000in}}{\pgfqpoint{1.400000in}{4.620000in}}%
\pgfusepath{clip}%
\pgfsetbuttcap%
\pgfsetmiterjoin%
\definecolor{currentfill}{rgb}{0.121569,0.466667,0.705882}%
\pgfsetfillcolor{currentfill}%
\pgfsetlinewidth{0.000000pt}%
\definecolor{currentstroke}{rgb}{0.000000,0.000000,0.000000}%
\pgfsetstrokecolor{currentstroke}%
\pgfsetstrokeopacity{0.000000}%
\pgfsetdash{}{0pt}%
\pgfpathmoveto{\pgfqpoint{5.800000in}{133.379817in}}%
\pgfpathlineto{\pgfqpoint{5.800000in}{133.379817in}}%
\pgfpathlineto{\pgfqpoint{5.800000in}{133.482655in}}%
\pgfpathlineto{\pgfqpoint{5.800000in}{133.482655in}}%
\pgfpathclose%
\pgfusepath{fill}%
\end{pgfscope}%
\begin{pgfscope}%
\pgfpathrectangle{\pgfqpoint{5.800000in}{0.720000in}}{\pgfqpoint{1.400000in}{4.620000in}}%
\pgfusepath{clip}%
\pgfsetbuttcap%
\pgfsetmiterjoin%
\definecolor{currentfill}{rgb}{0.121569,0.466667,0.705882}%
\pgfsetfillcolor{currentfill}%
\pgfsetlinewidth{0.000000pt}%
\definecolor{currentstroke}{rgb}{0.000000,0.000000,0.000000}%
\pgfsetstrokecolor{currentstroke}%
\pgfsetstrokeopacity{0.000000}%
\pgfsetdash{}{0pt}%
\pgfpathmoveto{\pgfqpoint{5.800000in}{133.482655in}}%
\pgfpathlineto{\pgfqpoint{5.800000in}{133.482655in}}%
\pgfpathlineto{\pgfqpoint{5.800000in}{133.585492in}}%
\pgfpathlineto{\pgfqpoint{5.800000in}{133.585492in}}%
\pgfpathclose%
\pgfusepath{fill}%
\end{pgfscope}%
\begin{pgfscope}%
\pgfpathrectangle{\pgfqpoint{5.800000in}{0.720000in}}{\pgfqpoint{1.400000in}{4.620000in}}%
\pgfusepath{clip}%
\pgfsetbuttcap%
\pgfsetmiterjoin%
\definecolor{currentfill}{rgb}{0.121569,0.466667,0.705882}%
\pgfsetfillcolor{currentfill}%
\pgfsetlinewidth{0.000000pt}%
\definecolor{currentstroke}{rgb}{0.000000,0.000000,0.000000}%
\pgfsetstrokecolor{currentstroke}%
\pgfsetstrokeopacity{0.000000}%
\pgfsetdash{}{0pt}%
\pgfpathmoveto{\pgfqpoint{5.800000in}{133.585492in}}%
\pgfpathlineto{\pgfqpoint{5.800000in}{133.585492in}}%
\pgfpathlineto{\pgfqpoint{5.800000in}{133.688329in}}%
\pgfpathlineto{\pgfqpoint{5.800000in}{133.688329in}}%
\pgfpathclose%
\pgfusepath{fill}%
\end{pgfscope}%
\begin{pgfscope}%
\pgfpathrectangle{\pgfqpoint{5.800000in}{0.720000in}}{\pgfqpoint{1.400000in}{4.620000in}}%
\pgfusepath{clip}%
\pgfsetbuttcap%
\pgfsetmiterjoin%
\definecolor{currentfill}{rgb}{0.121569,0.466667,0.705882}%
\pgfsetfillcolor{currentfill}%
\pgfsetlinewidth{0.000000pt}%
\definecolor{currentstroke}{rgb}{0.000000,0.000000,0.000000}%
\pgfsetstrokecolor{currentstroke}%
\pgfsetstrokeopacity{0.000000}%
\pgfsetdash{}{0pt}%
\pgfpathmoveto{\pgfqpoint{5.800000in}{133.688329in}}%
\pgfpathlineto{\pgfqpoint{5.800000in}{133.688329in}}%
\pgfpathlineto{\pgfqpoint{5.800000in}{133.791166in}}%
\pgfpathlineto{\pgfqpoint{5.800000in}{133.791166in}}%
\pgfpathclose%
\pgfusepath{fill}%
\end{pgfscope}%
\begin{pgfscope}%
\pgfpathrectangle{\pgfqpoint{5.800000in}{0.720000in}}{\pgfqpoint{1.400000in}{4.620000in}}%
\pgfusepath{clip}%
\pgfsetbuttcap%
\pgfsetmiterjoin%
\definecolor{currentfill}{rgb}{0.121569,0.466667,0.705882}%
\pgfsetfillcolor{currentfill}%
\pgfsetlinewidth{0.000000pt}%
\definecolor{currentstroke}{rgb}{0.000000,0.000000,0.000000}%
\pgfsetstrokecolor{currentstroke}%
\pgfsetstrokeopacity{0.000000}%
\pgfsetdash{}{0pt}%
\pgfpathmoveto{\pgfqpoint{5.800000in}{133.791166in}}%
\pgfpathlineto{\pgfqpoint{5.800000in}{133.791166in}}%
\pgfpathlineto{\pgfqpoint{5.800000in}{133.894003in}}%
\pgfpathlineto{\pgfqpoint{5.800000in}{133.894003in}}%
\pgfpathclose%
\pgfusepath{fill}%
\end{pgfscope}%
\begin{pgfscope}%
\pgfpathrectangle{\pgfqpoint{5.800000in}{0.720000in}}{\pgfqpoint{1.400000in}{4.620000in}}%
\pgfusepath{clip}%
\pgfsetbuttcap%
\pgfsetmiterjoin%
\definecolor{currentfill}{rgb}{0.121569,0.466667,0.705882}%
\pgfsetfillcolor{currentfill}%
\pgfsetlinewidth{0.000000pt}%
\definecolor{currentstroke}{rgb}{0.000000,0.000000,0.000000}%
\pgfsetstrokecolor{currentstroke}%
\pgfsetstrokeopacity{0.000000}%
\pgfsetdash{}{0pt}%
\pgfpathmoveto{\pgfqpoint{5.800000in}{133.894003in}}%
\pgfpathlineto{\pgfqpoint{5.800000in}{133.894003in}}%
\pgfpathlineto{\pgfqpoint{5.800000in}{133.996840in}}%
\pgfpathlineto{\pgfqpoint{5.800000in}{133.996840in}}%
\pgfpathclose%
\pgfusepath{fill}%
\end{pgfscope}%
\begin{pgfscope}%
\pgfpathrectangle{\pgfqpoint{5.800000in}{0.720000in}}{\pgfqpoint{1.400000in}{4.620000in}}%
\pgfusepath{clip}%
\pgfsetbuttcap%
\pgfsetmiterjoin%
\definecolor{currentfill}{rgb}{0.121569,0.466667,0.705882}%
\pgfsetfillcolor{currentfill}%
\pgfsetlinewidth{0.000000pt}%
\definecolor{currentstroke}{rgb}{0.000000,0.000000,0.000000}%
\pgfsetstrokecolor{currentstroke}%
\pgfsetstrokeopacity{0.000000}%
\pgfsetdash{}{0pt}%
\pgfpathmoveto{\pgfqpoint{5.800000in}{133.996840in}}%
\pgfpathlineto{\pgfqpoint{5.800000in}{133.996840in}}%
\pgfpathlineto{\pgfqpoint{5.800000in}{134.099677in}}%
\pgfpathlineto{\pgfqpoint{5.800000in}{134.099677in}}%
\pgfpathclose%
\pgfusepath{fill}%
\end{pgfscope}%
\begin{pgfscope}%
\pgfpathrectangle{\pgfqpoint{5.800000in}{0.720000in}}{\pgfqpoint{1.400000in}{4.620000in}}%
\pgfusepath{clip}%
\pgfsetbuttcap%
\pgfsetmiterjoin%
\definecolor{currentfill}{rgb}{0.121569,0.466667,0.705882}%
\pgfsetfillcolor{currentfill}%
\pgfsetlinewidth{0.000000pt}%
\definecolor{currentstroke}{rgb}{0.000000,0.000000,0.000000}%
\pgfsetstrokecolor{currentstroke}%
\pgfsetstrokeopacity{0.000000}%
\pgfsetdash{}{0pt}%
\pgfpathmoveto{\pgfqpoint{5.800000in}{134.099677in}}%
\pgfpathlineto{\pgfqpoint{5.800000in}{134.099677in}}%
\pgfpathlineto{\pgfqpoint{5.800000in}{134.202514in}}%
\pgfpathlineto{\pgfqpoint{5.800000in}{134.202514in}}%
\pgfpathclose%
\pgfusepath{fill}%
\end{pgfscope}%
\begin{pgfscope}%
\pgfpathrectangle{\pgfqpoint{5.800000in}{0.720000in}}{\pgfqpoint{1.400000in}{4.620000in}}%
\pgfusepath{clip}%
\pgfsetbuttcap%
\pgfsetmiterjoin%
\definecolor{currentfill}{rgb}{0.121569,0.466667,0.705882}%
\pgfsetfillcolor{currentfill}%
\pgfsetlinewidth{0.000000pt}%
\definecolor{currentstroke}{rgb}{0.000000,0.000000,0.000000}%
\pgfsetstrokecolor{currentstroke}%
\pgfsetstrokeopacity{0.000000}%
\pgfsetdash{}{0pt}%
\pgfpathmoveto{\pgfqpoint{5.800000in}{134.202514in}}%
\pgfpathlineto{\pgfqpoint{5.800000in}{134.202514in}}%
\pgfpathlineto{\pgfqpoint{5.800000in}{134.305351in}}%
\pgfpathlineto{\pgfqpoint{5.800000in}{134.305351in}}%
\pgfpathclose%
\pgfusepath{fill}%
\end{pgfscope}%
\begin{pgfscope}%
\pgfpathrectangle{\pgfqpoint{5.800000in}{0.720000in}}{\pgfqpoint{1.400000in}{4.620000in}}%
\pgfusepath{clip}%
\pgfsetbuttcap%
\pgfsetmiterjoin%
\definecolor{currentfill}{rgb}{0.121569,0.466667,0.705882}%
\pgfsetfillcolor{currentfill}%
\pgfsetlinewidth{0.000000pt}%
\definecolor{currentstroke}{rgb}{0.000000,0.000000,0.000000}%
\pgfsetstrokecolor{currentstroke}%
\pgfsetstrokeopacity{0.000000}%
\pgfsetdash{}{0pt}%
\pgfpathmoveto{\pgfqpoint{5.800000in}{134.305351in}}%
\pgfpathlineto{\pgfqpoint{5.800000in}{134.305351in}}%
\pgfpathlineto{\pgfqpoint{5.800000in}{134.408188in}}%
\pgfpathlineto{\pgfqpoint{5.800000in}{134.408188in}}%
\pgfpathclose%
\pgfusepath{fill}%
\end{pgfscope}%
\begin{pgfscope}%
\pgfpathrectangle{\pgfqpoint{5.800000in}{0.720000in}}{\pgfqpoint{1.400000in}{4.620000in}}%
\pgfusepath{clip}%
\pgfsetbuttcap%
\pgfsetmiterjoin%
\definecolor{currentfill}{rgb}{0.121569,0.466667,0.705882}%
\pgfsetfillcolor{currentfill}%
\pgfsetlinewidth{0.000000pt}%
\definecolor{currentstroke}{rgb}{0.000000,0.000000,0.000000}%
\pgfsetstrokecolor{currentstroke}%
\pgfsetstrokeopacity{0.000000}%
\pgfsetdash{}{0pt}%
\pgfpathmoveto{\pgfqpoint{5.800000in}{134.408188in}}%
\pgfpathlineto{\pgfqpoint{5.800000in}{134.408188in}}%
\pgfpathlineto{\pgfqpoint{5.800000in}{134.511025in}}%
\pgfpathlineto{\pgfqpoint{5.800000in}{134.511025in}}%
\pgfpathclose%
\pgfusepath{fill}%
\end{pgfscope}%
\begin{pgfscope}%
\pgfpathrectangle{\pgfqpoint{5.800000in}{0.720000in}}{\pgfqpoint{1.400000in}{4.620000in}}%
\pgfusepath{clip}%
\pgfsetbuttcap%
\pgfsetmiterjoin%
\definecolor{currentfill}{rgb}{0.121569,0.466667,0.705882}%
\pgfsetfillcolor{currentfill}%
\pgfsetlinewidth{0.000000pt}%
\definecolor{currentstroke}{rgb}{0.000000,0.000000,0.000000}%
\pgfsetstrokecolor{currentstroke}%
\pgfsetstrokeopacity{0.000000}%
\pgfsetdash{}{0pt}%
\pgfpathmoveto{\pgfqpoint{5.800000in}{134.511025in}}%
\pgfpathlineto{\pgfqpoint{5.800000in}{134.511025in}}%
\pgfpathlineto{\pgfqpoint{5.800000in}{134.613862in}}%
\pgfpathlineto{\pgfqpoint{5.800000in}{134.613862in}}%
\pgfpathclose%
\pgfusepath{fill}%
\end{pgfscope}%
\begin{pgfscope}%
\pgfpathrectangle{\pgfqpoint{5.800000in}{0.720000in}}{\pgfqpoint{1.400000in}{4.620000in}}%
\pgfusepath{clip}%
\pgfsetbuttcap%
\pgfsetmiterjoin%
\definecolor{currentfill}{rgb}{0.121569,0.466667,0.705882}%
\pgfsetfillcolor{currentfill}%
\pgfsetlinewidth{0.000000pt}%
\definecolor{currentstroke}{rgb}{0.000000,0.000000,0.000000}%
\pgfsetstrokecolor{currentstroke}%
\pgfsetstrokeopacity{0.000000}%
\pgfsetdash{}{0pt}%
\pgfpathmoveto{\pgfqpoint{5.800000in}{134.613862in}}%
\pgfpathlineto{\pgfqpoint{5.800000in}{134.613862in}}%
\pgfpathlineto{\pgfqpoint{5.800000in}{134.716699in}}%
\pgfpathlineto{\pgfqpoint{5.800000in}{134.716699in}}%
\pgfpathclose%
\pgfusepath{fill}%
\end{pgfscope}%
\begin{pgfscope}%
\pgfpathrectangle{\pgfqpoint{5.800000in}{0.720000in}}{\pgfqpoint{1.400000in}{4.620000in}}%
\pgfusepath{clip}%
\pgfsetbuttcap%
\pgfsetmiterjoin%
\definecolor{currentfill}{rgb}{0.121569,0.466667,0.705882}%
\pgfsetfillcolor{currentfill}%
\pgfsetlinewidth{0.000000pt}%
\definecolor{currentstroke}{rgb}{0.000000,0.000000,0.000000}%
\pgfsetstrokecolor{currentstroke}%
\pgfsetstrokeopacity{0.000000}%
\pgfsetdash{}{0pt}%
\pgfpathmoveto{\pgfqpoint{5.800000in}{134.716699in}}%
\pgfpathlineto{\pgfqpoint{5.800000in}{134.716699in}}%
\pgfpathlineto{\pgfqpoint{5.800000in}{134.819536in}}%
\pgfpathlineto{\pgfqpoint{5.800000in}{134.819536in}}%
\pgfpathclose%
\pgfusepath{fill}%
\end{pgfscope}%
\begin{pgfscope}%
\pgfpathrectangle{\pgfqpoint{5.800000in}{0.720000in}}{\pgfqpoint{1.400000in}{4.620000in}}%
\pgfusepath{clip}%
\pgfsetbuttcap%
\pgfsetmiterjoin%
\definecolor{currentfill}{rgb}{0.121569,0.466667,0.705882}%
\pgfsetfillcolor{currentfill}%
\pgfsetlinewidth{0.000000pt}%
\definecolor{currentstroke}{rgb}{0.000000,0.000000,0.000000}%
\pgfsetstrokecolor{currentstroke}%
\pgfsetstrokeopacity{0.000000}%
\pgfsetdash{}{0pt}%
\pgfpathmoveto{\pgfqpoint{5.800000in}{134.819536in}}%
\pgfpathlineto{\pgfqpoint{5.800000in}{134.819536in}}%
\pgfpathlineto{\pgfqpoint{5.800000in}{134.922373in}}%
\pgfpathlineto{\pgfqpoint{5.800000in}{134.922373in}}%
\pgfpathclose%
\pgfusepath{fill}%
\end{pgfscope}%
\begin{pgfscope}%
\pgfpathrectangle{\pgfqpoint{5.800000in}{0.720000in}}{\pgfqpoint{1.400000in}{4.620000in}}%
\pgfusepath{clip}%
\pgfsetbuttcap%
\pgfsetmiterjoin%
\definecolor{currentfill}{rgb}{0.121569,0.466667,0.705882}%
\pgfsetfillcolor{currentfill}%
\pgfsetlinewidth{0.000000pt}%
\definecolor{currentstroke}{rgb}{0.000000,0.000000,0.000000}%
\pgfsetstrokecolor{currentstroke}%
\pgfsetstrokeopacity{0.000000}%
\pgfsetdash{}{0pt}%
\pgfpathmoveto{\pgfqpoint{5.800000in}{134.922373in}}%
\pgfpathlineto{\pgfqpoint{5.800000in}{134.922373in}}%
\pgfpathlineto{\pgfqpoint{5.800000in}{135.025211in}}%
\pgfpathlineto{\pgfqpoint{5.800000in}{135.025211in}}%
\pgfpathclose%
\pgfusepath{fill}%
\end{pgfscope}%
\begin{pgfscope}%
\pgfpathrectangle{\pgfqpoint{5.800000in}{0.720000in}}{\pgfqpoint{1.400000in}{4.620000in}}%
\pgfusepath{clip}%
\pgfsetbuttcap%
\pgfsetmiterjoin%
\definecolor{currentfill}{rgb}{0.121569,0.466667,0.705882}%
\pgfsetfillcolor{currentfill}%
\pgfsetlinewidth{0.000000pt}%
\definecolor{currentstroke}{rgb}{0.000000,0.000000,0.000000}%
\pgfsetstrokecolor{currentstroke}%
\pgfsetstrokeopacity{0.000000}%
\pgfsetdash{}{0pt}%
\pgfpathmoveto{\pgfqpoint{5.800000in}{135.025211in}}%
\pgfpathlineto{\pgfqpoint{5.800000in}{135.025211in}}%
\pgfpathlineto{\pgfqpoint{5.800000in}{135.128048in}}%
\pgfpathlineto{\pgfqpoint{5.800000in}{135.128048in}}%
\pgfpathclose%
\pgfusepath{fill}%
\end{pgfscope}%
\begin{pgfscope}%
\pgfpathrectangle{\pgfqpoint{5.800000in}{0.720000in}}{\pgfqpoint{1.400000in}{4.620000in}}%
\pgfusepath{clip}%
\pgfsetbuttcap%
\pgfsetmiterjoin%
\definecolor{currentfill}{rgb}{0.121569,0.466667,0.705882}%
\pgfsetfillcolor{currentfill}%
\pgfsetlinewidth{0.000000pt}%
\definecolor{currentstroke}{rgb}{0.000000,0.000000,0.000000}%
\pgfsetstrokecolor{currentstroke}%
\pgfsetstrokeopacity{0.000000}%
\pgfsetdash{}{0pt}%
\pgfpathmoveto{\pgfqpoint{5.800000in}{135.128048in}}%
\pgfpathlineto{\pgfqpoint{5.800000in}{135.128048in}}%
\pgfpathlineto{\pgfqpoint{5.800000in}{135.230885in}}%
\pgfpathlineto{\pgfqpoint{5.800000in}{135.230885in}}%
\pgfpathclose%
\pgfusepath{fill}%
\end{pgfscope}%
\begin{pgfscope}%
\pgfpathrectangle{\pgfqpoint{5.800000in}{0.720000in}}{\pgfqpoint{1.400000in}{4.620000in}}%
\pgfusepath{clip}%
\pgfsetbuttcap%
\pgfsetmiterjoin%
\definecolor{currentfill}{rgb}{0.121569,0.466667,0.705882}%
\pgfsetfillcolor{currentfill}%
\pgfsetlinewidth{0.000000pt}%
\definecolor{currentstroke}{rgb}{0.000000,0.000000,0.000000}%
\pgfsetstrokecolor{currentstroke}%
\pgfsetstrokeopacity{0.000000}%
\pgfsetdash{}{0pt}%
\pgfpathmoveto{\pgfqpoint{5.800000in}{135.230885in}}%
\pgfpathlineto{\pgfqpoint{5.800000in}{135.230885in}}%
\pgfpathlineto{\pgfqpoint{5.800000in}{135.333722in}}%
\pgfpathlineto{\pgfqpoint{5.800000in}{135.333722in}}%
\pgfpathclose%
\pgfusepath{fill}%
\end{pgfscope}%
\begin{pgfscope}%
\pgfpathrectangle{\pgfqpoint{5.800000in}{0.720000in}}{\pgfqpoint{1.400000in}{4.620000in}}%
\pgfusepath{clip}%
\pgfsetbuttcap%
\pgfsetmiterjoin%
\definecolor{currentfill}{rgb}{0.121569,0.466667,0.705882}%
\pgfsetfillcolor{currentfill}%
\pgfsetlinewidth{0.000000pt}%
\definecolor{currentstroke}{rgb}{0.000000,0.000000,0.000000}%
\pgfsetstrokecolor{currentstroke}%
\pgfsetstrokeopacity{0.000000}%
\pgfsetdash{}{0pt}%
\pgfpathmoveto{\pgfqpoint{5.800000in}{135.333722in}}%
\pgfpathlineto{\pgfqpoint{5.800000in}{135.333722in}}%
\pgfpathlineto{\pgfqpoint{5.800000in}{135.436559in}}%
\pgfpathlineto{\pgfqpoint{5.800000in}{135.436559in}}%
\pgfpathclose%
\pgfusepath{fill}%
\end{pgfscope}%
\begin{pgfscope}%
\pgfpathrectangle{\pgfqpoint{5.800000in}{0.720000in}}{\pgfqpoint{1.400000in}{4.620000in}}%
\pgfusepath{clip}%
\pgfsetbuttcap%
\pgfsetmiterjoin%
\definecolor{currentfill}{rgb}{0.121569,0.466667,0.705882}%
\pgfsetfillcolor{currentfill}%
\pgfsetlinewidth{0.000000pt}%
\definecolor{currentstroke}{rgb}{0.000000,0.000000,0.000000}%
\pgfsetstrokecolor{currentstroke}%
\pgfsetstrokeopacity{0.000000}%
\pgfsetdash{}{0pt}%
\pgfpathmoveto{\pgfqpoint{5.800000in}{135.436559in}}%
\pgfpathlineto{\pgfqpoint{5.800000in}{135.436559in}}%
\pgfpathlineto{\pgfqpoint{5.800000in}{135.539396in}}%
\pgfpathlineto{\pgfqpoint{5.800000in}{135.539396in}}%
\pgfpathclose%
\pgfusepath{fill}%
\end{pgfscope}%
\begin{pgfscope}%
\pgfpathrectangle{\pgfqpoint{5.800000in}{0.720000in}}{\pgfqpoint{1.400000in}{4.620000in}}%
\pgfusepath{clip}%
\pgfsetbuttcap%
\pgfsetmiterjoin%
\definecolor{currentfill}{rgb}{0.121569,0.466667,0.705882}%
\pgfsetfillcolor{currentfill}%
\pgfsetlinewidth{0.000000pt}%
\definecolor{currentstroke}{rgb}{0.000000,0.000000,0.000000}%
\pgfsetstrokecolor{currentstroke}%
\pgfsetstrokeopacity{0.000000}%
\pgfsetdash{}{0pt}%
\pgfpathmoveto{\pgfqpoint{5.800000in}{135.539396in}}%
\pgfpathlineto{\pgfqpoint{5.800000in}{135.539396in}}%
\pgfpathlineto{\pgfqpoint{5.800000in}{135.642233in}}%
\pgfpathlineto{\pgfqpoint{5.800000in}{135.642233in}}%
\pgfpathclose%
\pgfusepath{fill}%
\end{pgfscope}%
\begin{pgfscope}%
\pgfpathrectangle{\pgfqpoint{5.800000in}{0.720000in}}{\pgfqpoint{1.400000in}{4.620000in}}%
\pgfusepath{clip}%
\pgfsetbuttcap%
\pgfsetmiterjoin%
\definecolor{currentfill}{rgb}{0.121569,0.466667,0.705882}%
\pgfsetfillcolor{currentfill}%
\pgfsetlinewidth{0.000000pt}%
\definecolor{currentstroke}{rgb}{0.000000,0.000000,0.000000}%
\pgfsetstrokecolor{currentstroke}%
\pgfsetstrokeopacity{0.000000}%
\pgfsetdash{}{0pt}%
\pgfpathmoveto{\pgfqpoint{5.800000in}{135.642233in}}%
\pgfpathlineto{\pgfqpoint{5.800000in}{135.642233in}}%
\pgfpathlineto{\pgfqpoint{5.800000in}{135.745070in}}%
\pgfpathlineto{\pgfqpoint{5.800000in}{135.745070in}}%
\pgfpathclose%
\pgfusepath{fill}%
\end{pgfscope}%
\begin{pgfscope}%
\pgfpathrectangle{\pgfqpoint{5.800000in}{0.720000in}}{\pgfqpoint{1.400000in}{4.620000in}}%
\pgfusepath{clip}%
\pgfsetbuttcap%
\pgfsetmiterjoin%
\definecolor{currentfill}{rgb}{0.121569,0.466667,0.705882}%
\pgfsetfillcolor{currentfill}%
\pgfsetlinewidth{0.000000pt}%
\definecolor{currentstroke}{rgb}{0.000000,0.000000,0.000000}%
\pgfsetstrokecolor{currentstroke}%
\pgfsetstrokeopacity{0.000000}%
\pgfsetdash{}{0pt}%
\pgfpathmoveto{\pgfqpoint{5.800000in}{135.745070in}}%
\pgfpathlineto{\pgfqpoint{5.800000in}{135.745070in}}%
\pgfpathlineto{\pgfqpoint{5.800000in}{135.847907in}}%
\pgfpathlineto{\pgfqpoint{5.800000in}{135.847907in}}%
\pgfpathclose%
\pgfusepath{fill}%
\end{pgfscope}%
\begin{pgfscope}%
\pgfpathrectangle{\pgfqpoint{5.800000in}{0.720000in}}{\pgfqpoint{1.400000in}{4.620000in}}%
\pgfusepath{clip}%
\pgfsetbuttcap%
\pgfsetmiterjoin%
\definecolor{currentfill}{rgb}{0.121569,0.466667,0.705882}%
\pgfsetfillcolor{currentfill}%
\pgfsetlinewidth{0.000000pt}%
\definecolor{currentstroke}{rgb}{0.000000,0.000000,0.000000}%
\pgfsetstrokecolor{currentstroke}%
\pgfsetstrokeopacity{0.000000}%
\pgfsetdash{}{0pt}%
\pgfpathmoveto{\pgfqpoint{5.800000in}{135.847907in}}%
\pgfpathlineto{\pgfqpoint{5.800000in}{135.847907in}}%
\pgfpathlineto{\pgfqpoint{5.800000in}{135.950744in}}%
\pgfpathlineto{\pgfqpoint{5.800000in}{135.950744in}}%
\pgfpathclose%
\pgfusepath{fill}%
\end{pgfscope}%
\begin{pgfscope}%
\pgfpathrectangle{\pgfqpoint{5.800000in}{0.720000in}}{\pgfqpoint{1.400000in}{4.620000in}}%
\pgfusepath{clip}%
\pgfsetbuttcap%
\pgfsetmiterjoin%
\definecolor{currentfill}{rgb}{0.121569,0.466667,0.705882}%
\pgfsetfillcolor{currentfill}%
\pgfsetlinewidth{0.000000pt}%
\definecolor{currentstroke}{rgb}{0.000000,0.000000,0.000000}%
\pgfsetstrokecolor{currentstroke}%
\pgfsetstrokeopacity{0.000000}%
\pgfsetdash{}{0pt}%
\pgfpathmoveto{\pgfqpoint{5.800000in}{135.950744in}}%
\pgfpathlineto{\pgfqpoint{5.800000in}{135.950744in}}%
\pgfpathlineto{\pgfqpoint{5.800000in}{136.053581in}}%
\pgfpathlineto{\pgfqpoint{5.800000in}{136.053581in}}%
\pgfpathclose%
\pgfusepath{fill}%
\end{pgfscope}%
\begin{pgfscope}%
\pgfpathrectangle{\pgfqpoint{5.800000in}{0.720000in}}{\pgfqpoint{1.400000in}{4.620000in}}%
\pgfusepath{clip}%
\pgfsetbuttcap%
\pgfsetmiterjoin%
\definecolor{currentfill}{rgb}{0.121569,0.466667,0.705882}%
\pgfsetfillcolor{currentfill}%
\pgfsetlinewidth{0.000000pt}%
\definecolor{currentstroke}{rgb}{0.000000,0.000000,0.000000}%
\pgfsetstrokecolor{currentstroke}%
\pgfsetstrokeopacity{0.000000}%
\pgfsetdash{}{0pt}%
\pgfpathmoveto{\pgfqpoint{5.800000in}{136.053581in}}%
\pgfpathlineto{\pgfqpoint{5.800000in}{136.053581in}}%
\pgfpathlineto{\pgfqpoint{5.800000in}{136.156418in}}%
\pgfpathlineto{\pgfqpoint{5.800000in}{136.156418in}}%
\pgfpathclose%
\pgfusepath{fill}%
\end{pgfscope}%
\begin{pgfscope}%
\pgfpathrectangle{\pgfqpoint{5.800000in}{0.720000in}}{\pgfqpoint{1.400000in}{4.620000in}}%
\pgfusepath{clip}%
\pgfsetbuttcap%
\pgfsetmiterjoin%
\definecolor{currentfill}{rgb}{0.121569,0.466667,0.705882}%
\pgfsetfillcolor{currentfill}%
\pgfsetlinewidth{0.000000pt}%
\definecolor{currentstroke}{rgb}{0.000000,0.000000,0.000000}%
\pgfsetstrokecolor{currentstroke}%
\pgfsetstrokeopacity{0.000000}%
\pgfsetdash{}{0pt}%
\pgfpathmoveto{\pgfqpoint{5.800000in}{136.156418in}}%
\pgfpathlineto{\pgfqpoint{5.800000in}{136.156418in}}%
\pgfpathlineto{\pgfqpoint{5.800000in}{136.259255in}}%
\pgfpathlineto{\pgfqpoint{5.800000in}{136.259255in}}%
\pgfpathclose%
\pgfusepath{fill}%
\end{pgfscope}%
\begin{pgfscope}%
\pgfpathrectangle{\pgfqpoint{5.800000in}{0.720000in}}{\pgfqpoint{1.400000in}{4.620000in}}%
\pgfusepath{clip}%
\pgfsetbuttcap%
\pgfsetmiterjoin%
\definecolor{currentfill}{rgb}{0.121569,0.466667,0.705882}%
\pgfsetfillcolor{currentfill}%
\pgfsetlinewidth{0.000000pt}%
\definecolor{currentstroke}{rgb}{0.000000,0.000000,0.000000}%
\pgfsetstrokecolor{currentstroke}%
\pgfsetstrokeopacity{0.000000}%
\pgfsetdash{}{0pt}%
\pgfpathmoveto{\pgfqpoint{5.800000in}{136.259255in}}%
\pgfpathlineto{\pgfqpoint{5.800000in}{136.259255in}}%
\pgfpathlineto{\pgfqpoint{5.800000in}{136.362092in}}%
\pgfpathlineto{\pgfqpoint{5.800000in}{136.362092in}}%
\pgfpathclose%
\pgfusepath{fill}%
\end{pgfscope}%
\begin{pgfscope}%
\pgfpathrectangle{\pgfqpoint{5.800000in}{0.720000in}}{\pgfqpoint{1.400000in}{4.620000in}}%
\pgfusepath{clip}%
\pgfsetbuttcap%
\pgfsetmiterjoin%
\definecolor{currentfill}{rgb}{0.121569,0.466667,0.705882}%
\pgfsetfillcolor{currentfill}%
\pgfsetlinewidth{0.000000pt}%
\definecolor{currentstroke}{rgb}{0.000000,0.000000,0.000000}%
\pgfsetstrokecolor{currentstroke}%
\pgfsetstrokeopacity{0.000000}%
\pgfsetdash{}{0pt}%
\pgfpathmoveto{\pgfqpoint{5.800000in}{136.362092in}}%
\pgfpathlineto{\pgfqpoint{5.800000in}{136.362092in}}%
\pgfpathlineto{\pgfqpoint{5.800000in}{136.464929in}}%
\pgfpathlineto{\pgfqpoint{5.800000in}{136.464929in}}%
\pgfpathclose%
\pgfusepath{fill}%
\end{pgfscope}%
\begin{pgfscope}%
\pgfpathrectangle{\pgfqpoint{5.800000in}{0.720000in}}{\pgfqpoint{1.400000in}{4.620000in}}%
\pgfusepath{clip}%
\pgfsetbuttcap%
\pgfsetmiterjoin%
\definecolor{currentfill}{rgb}{0.121569,0.466667,0.705882}%
\pgfsetfillcolor{currentfill}%
\pgfsetlinewidth{0.000000pt}%
\definecolor{currentstroke}{rgb}{0.000000,0.000000,0.000000}%
\pgfsetstrokecolor{currentstroke}%
\pgfsetstrokeopacity{0.000000}%
\pgfsetdash{}{0pt}%
\pgfpathmoveto{\pgfqpoint{5.800000in}{136.464929in}}%
\pgfpathlineto{\pgfqpoint{5.800000in}{136.464929in}}%
\pgfpathlineto{\pgfqpoint{5.800000in}{136.567767in}}%
\pgfpathlineto{\pgfqpoint{5.800000in}{136.567767in}}%
\pgfpathclose%
\pgfusepath{fill}%
\end{pgfscope}%
\begin{pgfscope}%
\pgfpathrectangle{\pgfqpoint{5.800000in}{0.720000in}}{\pgfqpoint{1.400000in}{4.620000in}}%
\pgfusepath{clip}%
\pgfsetbuttcap%
\pgfsetmiterjoin%
\definecolor{currentfill}{rgb}{0.121569,0.466667,0.705882}%
\pgfsetfillcolor{currentfill}%
\pgfsetlinewidth{0.000000pt}%
\definecolor{currentstroke}{rgb}{0.000000,0.000000,0.000000}%
\pgfsetstrokecolor{currentstroke}%
\pgfsetstrokeopacity{0.000000}%
\pgfsetdash{}{0pt}%
\pgfpathmoveto{\pgfqpoint{5.800000in}{136.567767in}}%
\pgfpathlineto{\pgfqpoint{5.800000in}{136.567767in}}%
\pgfpathlineto{\pgfqpoint{5.800000in}{136.670604in}}%
\pgfpathlineto{\pgfqpoint{5.800000in}{136.670604in}}%
\pgfpathclose%
\pgfusepath{fill}%
\end{pgfscope}%
\begin{pgfscope}%
\pgfpathrectangle{\pgfqpoint{5.800000in}{0.720000in}}{\pgfqpoint{1.400000in}{4.620000in}}%
\pgfusepath{clip}%
\pgfsetbuttcap%
\pgfsetmiterjoin%
\definecolor{currentfill}{rgb}{0.121569,0.466667,0.705882}%
\pgfsetfillcolor{currentfill}%
\pgfsetlinewidth{0.000000pt}%
\definecolor{currentstroke}{rgb}{0.000000,0.000000,0.000000}%
\pgfsetstrokecolor{currentstroke}%
\pgfsetstrokeopacity{0.000000}%
\pgfsetdash{}{0pt}%
\pgfpathmoveto{\pgfqpoint{5.800000in}{136.670604in}}%
\pgfpathlineto{\pgfqpoint{5.800000in}{136.670604in}}%
\pgfpathlineto{\pgfqpoint{5.800000in}{136.773441in}}%
\pgfpathlineto{\pgfqpoint{5.800000in}{136.773441in}}%
\pgfpathclose%
\pgfusepath{fill}%
\end{pgfscope}%
\begin{pgfscope}%
\pgfpathrectangle{\pgfqpoint{5.800000in}{0.720000in}}{\pgfqpoint{1.400000in}{4.620000in}}%
\pgfusepath{clip}%
\pgfsetbuttcap%
\pgfsetmiterjoin%
\definecolor{currentfill}{rgb}{0.121569,0.466667,0.705882}%
\pgfsetfillcolor{currentfill}%
\pgfsetlinewidth{0.000000pt}%
\definecolor{currentstroke}{rgb}{0.000000,0.000000,0.000000}%
\pgfsetstrokecolor{currentstroke}%
\pgfsetstrokeopacity{0.000000}%
\pgfsetdash{}{0pt}%
\pgfpathmoveto{\pgfqpoint{5.800000in}{136.773441in}}%
\pgfpathlineto{\pgfqpoint{5.800000in}{136.773441in}}%
\pgfpathlineto{\pgfqpoint{5.800000in}{136.876278in}}%
\pgfpathlineto{\pgfqpoint{5.800000in}{136.876278in}}%
\pgfpathclose%
\pgfusepath{fill}%
\end{pgfscope}%
\begin{pgfscope}%
\pgfpathrectangle{\pgfqpoint{5.800000in}{0.720000in}}{\pgfqpoint{1.400000in}{4.620000in}}%
\pgfusepath{clip}%
\pgfsetbuttcap%
\pgfsetmiterjoin%
\definecolor{currentfill}{rgb}{0.121569,0.466667,0.705882}%
\pgfsetfillcolor{currentfill}%
\pgfsetlinewidth{0.000000pt}%
\definecolor{currentstroke}{rgb}{0.000000,0.000000,0.000000}%
\pgfsetstrokecolor{currentstroke}%
\pgfsetstrokeopacity{0.000000}%
\pgfsetdash{}{0pt}%
\pgfpathmoveto{\pgfqpoint{5.800000in}{136.876278in}}%
\pgfpathlineto{\pgfqpoint{5.800000in}{136.876278in}}%
\pgfpathlineto{\pgfqpoint{5.800000in}{136.979115in}}%
\pgfpathlineto{\pgfqpoint{5.800000in}{136.979115in}}%
\pgfpathclose%
\pgfusepath{fill}%
\end{pgfscope}%
\begin{pgfscope}%
\pgfpathrectangle{\pgfqpoint{5.800000in}{0.720000in}}{\pgfqpoint{1.400000in}{4.620000in}}%
\pgfusepath{clip}%
\pgfsetbuttcap%
\pgfsetmiterjoin%
\definecolor{currentfill}{rgb}{0.121569,0.466667,0.705882}%
\pgfsetfillcolor{currentfill}%
\pgfsetlinewidth{0.000000pt}%
\definecolor{currentstroke}{rgb}{0.000000,0.000000,0.000000}%
\pgfsetstrokecolor{currentstroke}%
\pgfsetstrokeopacity{0.000000}%
\pgfsetdash{}{0pt}%
\pgfpathmoveto{\pgfqpoint{5.800000in}{136.979115in}}%
\pgfpathlineto{\pgfqpoint{5.800000in}{136.979115in}}%
\pgfpathlineto{\pgfqpoint{5.800000in}{137.081952in}}%
\pgfpathlineto{\pgfqpoint{5.800000in}{137.081952in}}%
\pgfpathclose%
\pgfusepath{fill}%
\end{pgfscope}%
\begin{pgfscope}%
\pgfpathrectangle{\pgfqpoint{5.800000in}{0.720000in}}{\pgfqpoint{1.400000in}{4.620000in}}%
\pgfusepath{clip}%
\pgfsetbuttcap%
\pgfsetmiterjoin%
\definecolor{currentfill}{rgb}{0.121569,0.466667,0.705882}%
\pgfsetfillcolor{currentfill}%
\pgfsetlinewidth{0.000000pt}%
\definecolor{currentstroke}{rgb}{0.000000,0.000000,0.000000}%
\pgfsetstrokecolor{currentstroke}%
\pgfsetstrokeopacity{0.000000}%
\pgfsetdash{}{0pt}%
\pgfpathmoveto{\pgfqpoint{5.800000in}{137.081952in}}%
\pgfpathlineto{\pgfqpoint{5.800000in}{137.081952in}}%
\pgfpathlineto{\pgfqpoint{5.800000in}{137.184789in}}%
\pgfpathlineto{\pgfqpoint{5.800000in}{137.184789in}}%
\pgfpathclose%
\pgfusepath{fill}%
\end{pgfscope}%
\begin{pgfscope}%
\pgfpathrectangle{\pgfqpoint{5.800000in}{0.720000in}}{\pgfqpoint{1.400000in}{4.620000in}}%
\pgfusepath{clip}%
\pgfsetbuttcap%
\pgfsetmiterjoin%
\definecolor{currentfill}{rgb}{0.121569,0.466667,0.705882}%
\pgfsetfillcolor{currentfill}%
\pgfsetlinewidth{0.000000pt}%
\definecolor{currentstroke}{rgb}{0.000000,0.000000,0.000000}%
\pgfsetstrokecolor{currentstroke}%
\pgfsetstrokeopacity{0.000000}%
\pgfsetdash{}{0pt}%
\pgfpathmoveto{\pgfqpoint{5.800000in}{137.184789in}}%
\pgfpathlineto{\pgfqpoint{5.800000in}{137.184789in}}%
\pgfpathlineto{\pgfqpoint{5.800000in}{137.287626in}}%
\pgfpathlineto{\pgfqpoint{5.800000in}{137.287626in}}%
\pgfpathclose%
\pgfusepath{fill}%
\end{pgfscope}%
\begin{pgfscope}%
\pgfpathrectangle{\pgfqpoint{5.800000in}{0.720000in}}{\pgfqpoint{1.400000in}{4.620000in}}%
\pgfusepath{clip}%
\pgfsetbuttcap%
\pgfsetmiterjoin%
\definecolor{currentfill}{rgb}{0.121569,0.466667,0.705882}%
\pgfsetfillcolor{currentfill}%
\pgfsetlinewidth{0.000000pt}%
\definecolor{currentstroke}{rgb}{0.000000,0.000000,0.000000}%
\pgfsetstrokecolor{currentstroke}%
\pgfsetstrokeopacity{0.000000}%
\pgfsetdash{}{0pt}%
\pgfpathmoveto{\pgfqpoint{5.800000in}{137.287626in}}%
\pgfpathlineto{\pgfqpoint{5.800000in}{137.287626in}}%
\pgfpathlineto{\pgfqpoint{5.800000in}{137.390463in}}%
\pgfpathlineto{\pgfqpoint{5.800000in}{137.390463in}}%
\pgfpathclose%
\pgfusepath{fill}%
\end{pgfscope}%
\begin{pgfscope}%
\pgfpathrectangle{\pgfqpoint{5.800000in}{0.720000in}}{\pgfqpoint{1.400000in}{4.620000in}}%
\pgfusepath{clip}%
\pgfsetbuttcap%
\pgfsetmiterjoin%
\definecolor{currentfill}{rgb}{0.121569,0.466667,0.705882}%
\pgfsetfillcolor{currentfill}%
\pgfsetlinewidth{0.000000pt}%
\definecolor{currentstroke}{rgb}{0.000000,0.000000,0.000000}%
\pgfsetstrokecolor{currentstroke}%
\pgfsetstrokeopacity{0.000000}%
\pgfsetdash{}{0pt}%
\pgfpathmoveto{\pgfqpoint{5.800000in}{137.390463in}}%
\pgfpathlineto{\pgfqpoint{5.800000in}{137.390463in}}%
\pgfpathlineto{\pgfqpoint{5.800000in}{137.493300in}}%
\pgfpathlineto{\pgfqpoint{5.800000in}{137.493300in}}%
\pgfpathclose%
\pgfusepath{fill}%
\end{pgfscope}%
\begin{pgfscope}%
\pgfpathrectangle{\pgfqpoint{5.800000in}{0.720000in}}{\pgfqpoint{1.400000in}{4.620000in}}%
\pgfusepath{clip}%
\pgfsetbuttcap%
\pgfsetmiterjoin%
\definecolor{currentfill}{rgb}{0.121569,0.466667,0.705882}%
\pgfsetfillcolor{currentfill}%
\pgfsetlinewidth{0.000000pt}%
\definecolor{currentstroke}{rgb}{0.000000,0.000000,0.000000}%
\pgfsetstrokecolor{currentstroke}%
\pgfsetstrokeopacity{0.000000}%
\pgfsetdash{}{0pt}%
\pgfpathmoveto{\pgfqpoint{5.800000in}{137.493300in}}%
\pgfpathlineto{\pgfqpoint{5.800000in}{137.493300in}}%
\pgfpathlineto{\pgfqpoint{5.800000in}{137.596137in}}%
\pgfpathlineto{\pgfqpoint{5.800000in}{137.596137in}}%
\pgfpathclose%
\pgfusepath{fill}%
\end{pgfscope}%
\begin{pgfscope}%
\pgfpathrectangle{\pgfqpoint{5.800000in}{0.720000in}}{\pgfqpoint{1.400000in}{4.620000in}}%
\pgfusepath{clip}%
\pgfsetbuttcap%
\pgfsetmiterjoin%
\definecolor{currentfill}{rgb}{0.121569,0.466667,0.705882}%
\pgfsetfillcolor{currentfill}%
\pgfsetlinewidth{0.000000pt}%
\definecolor{currentstroke}{rgb}{0.000000,0.000000,0.000000}%
\pgfsetstrokecolor{currentstroke}%
\pgfsetstrokeopacity{0.000000}%
\pgfsetdash{}{0pt}%
\pgfpathmoveto{\pgfqpoint{5.800000in}{137.596137in}}%
\pgfpathlineto{\pgfqpoint{5.800000in}{137.596137in}}%
\pgfpathlineto{\pgfqpoint{5.800000in}{137.698974in}}%
\pgfpathlineto{\pgfqpoint{5.800000in}{137.698974in}}%
\pgfpathclose%
\pgfusepath{fill}%
\end{pgfscope}%
\begin{pgfscope}%
\pgfpathrectangle{\pgfqpoint{5.800000in}{0.720000in}}{\pgfqpoint{1.400000in}{4.620000in}}%
\pgfusepath{clip}%
\pgfsetbuttcap%
\pgfsetmiterjoin%
\definecolor{currentfill}{rgb}{0.121569,0.466667,0.705882}%
\pgfsetfillcolor{currentfill}%
\pgfsetlinewidth{0.000000pt}%
\definecolor{currentstroke}{rgb}{0.000000,0.000000,0.000000}%
\pgfsetstrokecolor{currentstroke}%
\pgfsetstrokeopacity{0.000000}%
\pgfsetdash{}{0pt}%
\pgfpathmoveto{\pgfqpoint{5.800000in}{137.698974in}}%
\pgfpathlineto{\pgfqpoint{5.800000in}{137.698974in}}%
\pgfpathlineto{\pgfqpoint{5.800000in}{137.801811in}}%
\pgfpathlineto{\pgfqpoint{5.800000in}{137.801811in}}%
\pgfpathclose%
\pgfusepath{fill}%
\end{pgfscope}%
\begin{pgfscope}%
\pgfpathrectangle{\pgfqpoint{5.800000in}{0.720000in}}{\pgfqpoint{1.400000in}{4.620000in}}%
\pgfusepath{clip}%
\pgfsetbuttcap%
\pgfsetmiterjoin%
\definecolor{currentfill}{rgb}{0.121569,0.466667,0.705882}%
\pgfsetfillcolor{currentfill}%
\pgfsetlinewidth{0.000000pt}%
\definecolor{currentstroke}{rgb}{0.000000,0.000000,0.000000}%
\pgfsetstrokecolor{currentstroke}%
\pgfsetstrokeopacity{0.000000}%
\pgfsetdash{}{0pt}%
\pgfpathmoveto{\pgfqpoint{5.800000in}{137.801811in}}%
\pgfpathlineto{\pgfqpoint{5.800000in}{137.801811in}}%
\pgfpathlineto{\pgfqpoint{5.800000in}{137.904648in}}%
\pgfpathlineto{\pgfqpoint{5.800000in}{137.904648in}}%
\pgfpathclose%
\pgfusepath{fill}%
\end{pgfscope}%
\begin{pgfscope}%
\pgfpathrectangle{\pgfqpoint{5.800000in}{0.720000in}}{\pgfqpoint{1.400000in}{4.620000in}}%
\pgfusepath{clip}%
\pgfsetbuttcap%
\pgfsetmiterjoin%
\definecolor{currentfill}{rgb}{0.121569,0.466667,0.705882}%
\pgfsetfillcolor{currentfill}%
\pgfsetlinewidth{0.000000pt}%
\definecolor{currentstroke}{rgb}{0.000000,0.000000,0.000000}%
\pgfsetstrokecolor{currentstroke}%
\pgfsetstrokeopacity{0.000000}%
\pgfsetdash{}{0pt}%
\pgfpathmoveto{\pgfqpoint{5.800000in}{137.904648in}}%
\pgfpathlineto{\pgfqpoint{5.800000in}{137.904648in}}%
\pgfpathlineto{\pgfqpoint{5.800000in}{138.007485in}}%
\pgfpathlineto{\pgfqpoint{5.800000in}{138.007485in}}%
\pgfpathclose%
\pgfusepath{fill}%
\end{pgfscope}%
\begin{pgfscope}%
\pgfpathrectangle{\pgfqpoint{5.800000in}{0.720000in}}{\pgfqpoint{1.400000in}{4.620000in}}%
\pgfusepath{clip}%
\pgfsetbuttcap%
\pgfsetmiterjoin%
\definecolor{currentfill}{rgb}{0.121569,0.466667,0.705882}%
\pgfsetfillcolor{currentfill}%
\pgfsetlinewidth{0.000000pt}%
\definecolor{currentstroke}{rgb}{0.000000,0.000000,0.000000}%
\pgfsetstrokecolor{currentstroke}%
\pgfsetstrokeopacity{0.000000}%
\pgfsetdash{}{0pt}%
\pgfpathmoveto{\pgfqpoint{5.800000in}{138.007485in}}%
\pgfpathlineto{\pgfqpoint{5.800000in}{138.007485in}}%
\pgfpathlineto{\pgfqpoint{5.800000in}{138.110323in}}%
\pgfpathlineto{\pgfqpoint{5.800000in}{138.110323in}}%
\pgfpathclose%
\pgfusepath{fill}%
\end{pgfscope}%
\begin{pgfscope}%
\pgfpathrectangle{\pgfqpoint{5.800000in}{0.720000in}}{\pgfqpoint{1.400000in}{4.620000in}}%
\pgfusepath{clip}%
\pgfsetbuttcap%
\pgfsetmiterjoin%
\definecolor{currentfill}{rgb}{0.121569,0.466667,0.705882}%
\pgfsetfillcolor{currentfill}%
\pgfsetlinewidth{0.000000pt}%
\definecolor{currentstroke}{rgb}{0.000000,0.000000,0.000000}%
\pgfsetstrokecolor{currentstroke}%
\pgfsetstrokeopacity{0.000000}%
\pgfsetdash{}{0pt}%
\pgfpathmoveto{\pgfqpoint{5.800000in}{138.110323in}}%
\pgfpathlineto{\pgfqpoint{5.800000in}{138.110323in}}%
\pgfpathlineto{\pgfqpoint{5.800000in}{138.213160in}}%
\pgfpathlineto{\pgfqpoint{5.800000in}{138.213160in}}%
\pgfpathclose%
\pgfusepath{fill}%
\end{pgfscope}%
\begin{pgfscope}%
\pgfpathrectangle{\pgfqpoint{5.800000in}{0.720000in}}{\pgfqpoint{1.400000in}{4.620000in}}%
\pgfusepath{clip}%
\pgfsetbuttcap%
\pgfsetmiterjoin%
\definecolor{currentfill}{rgb}{0.121569,0.466667,0.705882}%
\pgfsetfillcolor{currentfill}%
\pgfsetlinewidth{0.000000pt}%
\definecolor{currentstroke}{rgb}{0.000000,0.000000,0.000000}%
\pgfsetstrokecolor{currentstroke}%
\pgfsetstrokeopacity{0.000000}%
\pgfsetdash{}{0pt}%
\pgfpathmoveto{\pgfqpoint{5.800000in}{138.213160in}}%
\pgfpathlineto{\pgfqpoint{5.800000in}{138.213160in}}%
\pgfpathlineto{\pgfqpoint{5.800000in}{138.315997in}}%
\pgfpathlineto{\pgfqpoint{5.800000in}{138.315997in}}%
\pgfpathclose%
\pgfusepath{fill}%
\end{pgfscope}%
\begin{pgfscope}%
\pgfpathrectangle{\pgfqpoint{5.800000in}{0.720000in}}{\pgfqpoint{1.400000in}{4.620000in}}%
\pgfusepath{clip}%
\pgfsetbuttcap%
\pgfsetmiterjoin%
\definecolor{currentfill}{rgb}{0.121569,0.466667,0.705882}%
\pgfsetfillcolor{currentfill}%
\pgfsetlinewidth{0.000000pt}%
\definecolor{currentstroke}{rgb}{0.000000,0.000000,0.000000}%
\pgfsetstrokecolor{currentstroke}%
\pgfsetstrokeopacity{0.000000}%
\pgfsetdash{}{0pt}%
\pgfpathmoveto{\pgfqpoint{5.800000in}{138.315997in}}%
\pgfpathlineto{\pgfqpoint{5.800000in}{138.315997in}}%
\pgfpathlineto{\pgfqpoint{5.800000in}{138.418834in}}%
\pgfpathlineto{\pgfqpoint{5.800000in}{138.418834in}}%
\pgfpathclose%
\pgfusepath{fill}%
\end{pgfscope}%
\begin{pgfscope}%
\pgfpathrectangle{\pgfqpoint{5.800000in}{0.720000in}}{\pgfqpoint{1.400000in}{4.620000in}}%
\pgfusepath{clip}%
\pgfsetbuttcap%
\pgfsetmiterjoin%
\definecolor{currentfill}{rgb}{0.121569,0.466667,0.705882}%
\pgfsetfillcolor{currentfill}%
\pgfsetlinewidth{0.000000pt}%
\definecolor{currentstroke}{rgb}{0.000000,0.000000,0.000000}%
\pgfsetstrokecolor{currentstroke}%
\pgfsetstrokeopacity{0.000000}%
\pgfsetdash{}{0pt}%
\pgfpathmoveto{\pgfqpoint{5.800000in}{138.418834in}}%
\pgfpathlineto{\pgfqpoint{5.800000in}{138.418834in}}%
\pgfpathlineto{\pgfqpoint{5.800000in}{138.521671in}}%
\pgfpathlineto{\pgfqpoint{5.800000in}{138.521671in}}%
\pgfpathclose%
\pgfusepath{fill}%
\end{pgfscope}%
\begin{pgfscope}%
\pgfpathrectangle{\pgfqpoint{5.800000in}{0.720000in}}{\pgfqpoint{1.400000in}{4.620000in}}%
\pgfusepath{clip}%
\pgfsetbuttcap%
\pgfsetmiterjoin%
\definecolor{currentfill}{rgb}{0.121569,0.466667,0.705882}%
\pgfsetfillcolor{currentfill}%
\pgfsetlinewidth{0.000000pt}%
\definecolor{currentstroke}{rgb}{0.000000,0.000000,0.000000}%
\pgfsetstrokecolor{currentstroke}%
\pgfsetstrokeopacity{0.000000}%
\pgfsetdash{}{0pt}%
\pgfpathmoveto{\pgfqpoint{5.800000in}{138.521671in}}%
\pgfpathlineto{\pgfqpoint{5.800000in}{138.521671in}}%
\pgfpathlineto{\pgfqpoint{5.800000in}{138.624508in}}%
\pgfpathlineto{\pgfqpoint{5.800000in}{138.624508in}}%
\pgfpathclose%
\pgfusepath{fill}%
\end{pgfscope}%
\begin{pgfscope}%
\pgfpathrectangle{\pgfqpoint{5.800000in}{0.720000in}}{\pgfqpoint{1.400000in}{4.620000in}}%
\pgfusepath{clip}%
\pgfsetbuttcap%
\pgfsetmiterjoin%
\definecolor{currentfill}{rgb}{0.121569,0.466667,0.705882}%
\pgfsetfillcolor{currentfill}%
\pgfsetlinewidth{0.000000pt}%
\definecolor{currentstroke}{rgb}{0.000000,0.000000,0.000000}%
\pgfsetstrokecolor{currentstroke}%
\pgfsetstrokeopacity{0.000000}%
\pgfsetdash{}{0pt}%
\pgfpathmoveto{\pgfqpoint{5.800000in}{138.624508in}}%
\pgfpathlineto{\pgfqpoint{5.800000in}{138.624508in}}%
\pgfpathlineto{\pgfqpoint{5.800000in}{138.727345in}}%
\pgfpathlineto{\pgfqpoint{5.800000in}{138.727345in}}%
\pgfpathclose%
\pgfusepath{fill}%
\end{pgfscope}%
\begin{pgfscope}%
\pgfpathrectangle{\pgfqpoint{5.800000in}{0.720000in}}{\pgfqpoint{1.400000in}{4.620000in}}%
\pgfusepath{clip}%
\pgfsetbuttcap%
\pgfsetmiterjoin%
\definecolor{currentfill}{rgb}{0.121569,0.466667,0.705882}%
\pgfsetfillcolor{currentfill}%
\pgfsetlinewidth{0.000000pt}%
\definecolor{currentstroke}{rgb}{0.000000,0.000000,0.000000}%
\pgfsetstrokecolor{currentstroke}%
\pgfsetstrokeopacity{0.000000}%
\pgfsetdash{}{0pt}%
\pgfpathmoveto{\pgfqpoint{5.800000in}{138.727345in}}%
\pgfpathlineto{\pgfqpoint{5.800000in}{138.727345in}}%
\pgfpathlineto{\pgfqpoint{5.800000in}{138.830182in}}%
\pgfpathlineto{\pgfqpoint{5.800000in}{138.830182in}}%
\pgfpathclose%
\pgfusepath{fill}%
\end{pgfscope}%
\begin{pgfscope}%
\pgfpathrectangle{\pgfqpoint{5.800000in}{0.720000in}}{\pgfqpoint{1.400000in}{4.620000in}}%
\pgfusepath{clip}%
\pgfsetbuttcap%
\pgfsetmiterjoin%
\definecolor{currentfill}{rgb}{0.121569,0.466667,0.705882}%
\pgfsetfillcolor{currentfill}%
\pgfsetlinewidth{0.000000pt}%
\definecolor{currentstroke}{rgb}{0.000000,0.000000,0.000000}%
\pgfsetstrokecolor{currentstroke}%
\pgfsetstrokeopacity{0.000000}%
\pgfsetdash{}{0pt}%
\pgfpathmoveto{\pgfqpoint{5.800000in}{138.830182in}}%
\pgfpathlineto{\pgfqpoint{5.800000in}{138.830182in}}%
\pgfpathlineto{\pgfqpoint{5.800000in}{138.933019in}}%
\pgfpathlineto{\pgfqpoint{5.800000in}{138.933019in}}%
\pgfpathclose%
\pgfusepath{fill}%
\end{pgfscope}%
\begin{pgfscope}%
\pgfpathrectangle{\pgfqpoint{5.800000in}{0.720000in}}{\pgfqpoint{1.400000in}{4.620000in}}%
\pgfusepath{clip}%
\pgfsetbuttcap%
\pgfsetmiterjoin%
\definecolor{currentfill}{rgb}{0.121569,0.466667,0.705882}%
\pgfsetfillcolor{currentfill}%
\pgfsetlinewidth{0.000000pt}%
\definecolor{currentstroke}{rgb}{0.000000,0.000000,0.000000}%
\pgfsetstrokecolor{currentstroke}%
\pgfsetstrokeopacity{0.000000}%
\pgfsetdash{}{0pt}%
\pgfpathmoveto{\pgfqpoint{5.800000in}{138.933019in}}%
\pgfpathlineto{\pgfqpoint{5.800000in}{138.933019in}}%
\pgfpathlineto{\pgfqpoint{5.800000in}{139.035856in}}%
\pgfpathlineto{\pgfqpoint{5.800000in}{139.035856in}}%
\pgfpathclose%
\pgfusepath{fill}%
\end{pgfscope}%
\begin{pgfscope}%
\pgfpathrectangle{\pgfqpoint{5.800000in}{0.720000in}}{\pgfqpoint{1.400000in}{4.620000in}}%
\pgfusepath{clip}%
\pgfsetbuttcap%
\pgfsetmiterjoin%
\definecolor{currentfill}{rgb}{0.121569,0.466667,0.705882}%
\pgfsetfillcolor{currentfill}%
\pgfsetlinewidth{0.000000pt}%
\definecolor{currentstroke}{rgb}{0.000000,0.000000,0.000000}%
\pgfsetstrokecolor{currentstroke}%
\pgfsetstrokeopacity{0.000000}%
\pgfsetdash{}{0pt}%
\pgfpathmoveto{\pgfqpoint{5.800000in}{139.035856in}}%
\pgfpathlineto{\pgfqpoint{5.800000in}{139.035856in}}%
\pgfpathlineto{\pgfqpoint{5.800000in}{139.138693in}}%
\pgfpathlineto{\pgfqpoint{5.800000in}{139.138693in}}%
\pgfpathclose%
\pgfusepath{fill}%
\end{pgfscope}%
\begin{pgfscope}%
\pgfpathrectangle{\pgfqpoint{5.800000in}{0.720000in}}{\pgfqpoint{1.400000in}{4.620000in}}%
\pgfusepath{clip}%
\pgfsetbuttcap%
\pgfsetmiterjoin%
\definecolor{currentfill}{rgb}{0.121569,0.466667,0.705882}%
\pgfsetfillcolor{currentfill}%
\pgfsetlinewidth{0.000000pt}%
\definecolor{currentstroke}{rgb}{0.000000,0.000000,0.000000}%
\pgfsetstrokecolor{currentstroke}%
\pgfsetstrokeopacity{0.000000}%
\pgfsetdash{}{0pt}%
\pgfpathmoveto{\pgfqpoint{5.800000in}{139.138693in}}%
\pgfpathlineto{\pgfqpoint{5.800000in}{139.138693in}}%
\pgfpathlineto{\pgfqpoint{5.800000in}{139.241530in}}%
\pgfpathlineto{\pgfqpoint{5.800000in}{139.241530in}}%
\pgfpathclose%
\pgfusepath{fill}%
\end{pgfscope}%
\begin{pgfscope}%
\pgfpathrectangle{\pgfqpoint{5.800000in}{0.720000in}}{\pgfqpoint{1.400000in}{4.620000in}}%
\pgfusepath{clip}%
\pgfsetbuttcap%
\pgfsetmiterjoin%
\definecolor{currentfill}{rgb}{0.121569,0.466667,0.705882}%
\pgfsetfillcolor{currentfill}%
\pgfsetlinewidth{0.000000pt}%
\definecolor{currentstroke}{rgb}{0.000000,0.000000,0.000000}%
\pgfsetstrokecolor{currentstroke}%
\pgfsetstrokeopacity{0.000000}%
\pgfsetdash{}{0pt}%
\pgfpathmoveto{\pgfqpoint{5.800000in}{139.241530in}}%
\pgfpathlineto{\pgfqpoint{5.800000in}{139.241530in}}%
\pgfpathlineto{\pgfqpoint{5.800000in}{139.344367in}}%
\pgfpathlineto{\pgfqpoint{5.800000in}{139.344367in}}%
\pgfpathclose%
\pgfusepath{fill}%
\end{pgfscope}%
\begin{pgfscope}%
\pgfpathrectangle{\pgfqpoint{5.800000in}{0.720000in}}{\pgfqpoint{1.400000in}{4.620000in}}%
\pgfusepath{clip}%
\pgfsetbuttcap%
\pgfsetmiterjoin%
\definecolor{currentfill}{rgb}{0.121569,0.466667,0.705882}%
\pgfsetfillcolor{currentfill}%
\pgfsetlinewidth{0.000000pt}%
\definecolor{currentstroke}{rgb}{0.000000,0.000000,0.000000}%
\pgfsetstrokecolor{currentstroke}%
\pgfsetstrokeopacity{0.000000}%
\pgfsetdash{}{0pt}%
\pgfpathmoveto{\pgfqpoint{5.800000in}{139.344367in}}%
\pgfpathlineto{\pgfqpoint{5.800000in}{139.344367in}}%
\pgfpathlineto{\pgfqpoint{5.800000in}{139.447204in}}%
\pgfpathlineto{\pgfqpoint{5.800000in}{139.447204in}}%
\pgfpathclose%
\pgfusepath{fill}%
\end{pgfscope}%
\begin{pgfscope}%
\pgfpathrectangle{\pgfqpoint{5.800000in}{0.720000in}}{\pgfqpoint{1.400000in}{4.620000in}}%
\pgfusepath{clip}%
\pgfsetbuttcap%
\pgfsetmiterjoin%
\definecolor{currentfill}{rgb}{0.121569,0.466667,0.705882}%
\pgfsetfillcolor{currentfill}%
\pgfsetlinewidth{0.000000pt}%
\definecolor{currentstroke}{rgb}{0.000000,0.000000,0.000000}%
\pgfsetstrokecolor{currentstroke}%
\pgfsetstrokeopacity{0.000000}%
\pgfsetdash{}{0pt}%
\pgfpathmoveto{\pgfqpoint{5.800000in}{139.447204in}}%
\pgfpathlineto{\pgfqpoint{5.800000in}{139.447204in}}%
\pgfpathlineto{\pgfqpoint{5.800000in}{139.550041in}}%
\pgfpathlineto{\pgfqpoint{5.800000in}{139.550041in}}%
\pgfpathclose%
\pgfusepath{fill}%
\end{pgfscope}%
\begin{pgfscope}%
\pgfpathrectangle{\pgfqpoint{5.800000in}{0.720000in}}{\pgfqpoint{1.400000in}{4.620000in}}%
\pgfusepath{clip}%
\pgfsetbuttcap%
\pgfsetmiterjoin%
\definecolor{currentfill}{rgb}{0.121569,0.466667,0.705882}%
\pgfsetfillcolor{currentfill}%
\pgfsetlinewidth{0.000000pt}%
\definecolor{currentstroke}{rgb}{0.000000,0.000000,0.000000}%
\pgfsetstrokecolor{currentstroke}%
\pgfsetstrokeopacity{0.000000}%
\pgfsetdash{}{0pt}%
\pgfpathmoveto{\pgfqpoint{5.800000in}{139.550041in}}%
\pgfpathlineto{\pgfqpoint{5.800000in}{139.550041in}}%
\pgfpathlineto{\pgfqpoint{5.800000in}{139.652879in}}%
\pgfpathlineto{\pgfqpoint{5.800000in}{139.652879in}}%
\pgfpathclose%
\pgfusepath{fill}%
\end{pgfscope}%
\begin{pgfscope}%
\pgfpathrectangle{\pgfqpoint{5.800000in}{0.720000in}}{\pgfqpoint{1.400000in}{4.620000in}}%
\pgfusepath{clip}%
\pgfsetbuttcap%
\pgfsetmiterjoin%
\definecolor{currentfill}{rgb}{0.121569,0.466667,0.705882}%
\pgfsetfillcolor{currentfill}%
\pgfsetlinewidth{0.000000pt}%
\definecolor{currentstroke}{rgb}{0.000000,0.000000,0.000000}%
\pgfsetstrokecolor{currentstroke}%
\pgfsetstrokeopacity{0.000000}%
\pgfsetdash{}{0pt}%
\pgfpathmoveto{\pgfqpoint{5.800000in}{139.652879in}}%
\pgfpathlineto{\pgfqpoint{5.800000in}{139.652879in}}%
\pgfpathlineto{\pgfqpoint{5.800000in}{139.755716in}}%
\pgfpathlineto{\pgfqpoint{5.800000in}{139.755716in}}%
\pgfpathclose%
\pgfusepath{fill}%
\end{pgfscope}%
\begin{pgfscope}%
\pgfpathrectangle{\pgfqpoint{5.800000in}{0.720000in}}{\pgfqpoint{1.400000in}{4.620000in}}%
\pgfusepath{clip}%
\pgfsetbuttcap%
\pgfsetmiterjoin%
\definecolor{currentfill}{rgb}{0.121569,0.466667,0.705882}%
\pgfsetfillcolor{currentfill}%
\pgfsetlinewidth{0.000000pt}%
\definecolor{currentstroke}{rgb}{0.000000,0.000000,0.000000}%
\pgfsetstrokecolor{currentstroke}%
\pgfsetstrokeopacity{0.000000}%
\pgfsetdash{}{0pt}%
\pgfpathmoveto{\pgfqpoint{5.800000in}{139.755716in}}%
\pgfpathlineto{\pgfqpoint{5.800000in}{139.755716in}}%
\pgfpathlineto{\pgfqpoint{5.800000in}{139.858553in}}%
\pgfpathlineto{\pgfqpoint{5.800000in}{139.858553in}}%
\pgfpathclose%
\pgfusepath{fill}%
\end{pgfscope}%
\begin{pgfscope}%
\pgfpathrectangle{\pgfqpoint{5.800000in}{0.720000in}}{\pgfqpoint{1.400000in}{4.620000in}}%
\pgfusepath{clip}%
\pgfsetbuttcap%
\pgfsetmiterjoin%
\definecolor{currentfill}{rgb}{0.121569,0.466667,0.705882}%
\pgfsetfillcolor{currentfill}%
\pgfsetlinewidth{0.000000pt}%
\definecolor{currentstroke}{rgb}{0.000000,0.000000,0.000000}%
\pgfsetstrokecolor{currentstroke}%
\pgfsetstrokeopacity{0.000000}%
\pgfsetdash{}{0pt}%
\pgfpathmoveto{\pgfqpoint{5.800000in}{139.858553in}}%
\pgfpathlineto{\pgfqpoint{5.800000in}{139.858553in}}%
\pgfpathlineto{\pgfqpoint{5.800000in}{139.961390in}}%
\pgfpathlineto{\pgfqpoint{5.800000in}{139.961390in}}%
\pgfpathclose%
\pgfusepath{fill}%
\end{pgfscope}%
\begin{pgfscope}%
\pgfpathrectangle{\pgfqpoint{5.800000in}{0.720000in}}{\pgfqpoint{1.400000in}{4.620000in}}%
\pgfusepath{clip}%
\pgfsetbuttcap%
\pgfsetmiterjoin%
\definecolor{currentfill}{rgb}{0.121569,0.466667,0.705882}%
\pgfsetfillcolor{currentfill}%
\pgfsetlinewidth{0.000000pt}%
\definecolor{currentstroke}{rgb}{0.000000,0.000000,0.000000}%
\pgfsetstrokecolor{currentstroke}%
\pgfsetstrokeopacity{0.000000}%
\pgfsetdash{}{0pt}%
\pgfpathmoveto{\pgfqpoint{5.800000in}{139.961390in}}%
\pgfpathlineto{\pgfqpoint{5.800000in}{139.961390in}}%
\pgfpathlineto{\pgfqpoint{5.800000in}{140.064227in}}%
\pgfpathlineto{\pgfqpoint{5.800000in}{140.064227in}}%
\pgfpathclose%
\pgfusepath{fill}%
\end{pgfscope}%
\begin{pgfscope}%
\pgfpathrectangle{\pgfqpoint{5.800000in}{0.720000in}}{\pgfqpoint{1.400000in}{4.620000in}}%
\pgfusepath{clip}%
\pgfsetbuttcap%
\pgfsetmiterjoin%
\definecolor{currentfill}{rgb}{0.121569,0.466667,0.705882}%
\pgfsetfillcolor{currentfill}%
\pgfsetlinewidth{0.000000pt}%
\definecolor{currentstroke}{rgb}{0.000000,0.000000,0.000000}%
\pgfsetstrokecolor{currentstroke}%
\pgfsetstrokeopacity{0.000000}%
\pgfsetdash{}{0pt}%
\pgfpathmoveto{\pgfqpoint{5.800000in}{140.064227in}}%
\pgfpathlineto{\pgfqpoint{5.800000in}{140.064227in}}%
\pgfpathlineto{\pgfqpoint{5.800000in}{140.167064in}}%
\pgfpathlineto{\pgfqpoint{5.800000in}{140.167064in}}%
\pgfpathclose%
\pgfusepath{fill}%
\end{pgfscope}%
\begin{pgfscope}%
\pgfpathrectangle{\pgfqpoint{5.800000in}{0.720000in}}{\pgfqpoint{1.400000in}{4.620000in}}%
\pgfusepath{clip}%
\pgfsetbuttcap%
\pgfsetmiterjoin%
\definecolor{currentfill}{rgb}{0.121569,0.466667,0.705882}%
\pgfsetfillcolor{currentfill}%
\pgfsetlinewidth{0.000000pt}%
\definecolor{currentstroke}{rgb}{0.000000,0.000000,0.000000}%
\pgfsetstrokecolor{currentstroke}%
\pgfsetstrokeopacity{0.000000}%
\pgfsetdash{}{0pt}%
\pgfpathmoveto{\pgfqpoint{5.800000in}{140.167064in}}%
\pgfpathlineto{\pgfqpoint{5.800000in}{140.167064in}}%
\pgfpathlineto{\pgfqpoint{5.800000in}{140.269901in}}%
\pgfpathlineto{\pgfqpoint{5.800000in}{140.269901in}}%
\pgfpathclose%
\pgfusepath{fill}%
\end{pgfscope}%
\begin{pgfscope}%
\pgfpathrectangle{\pgfqpoint{5.800000in}{0.720000in}}{\pgfqpoint{1.400000in}{4.620000in}}%
\pgfusepath{clip}%
\pgfsetbuttcap%
\pgfsetmiterjoin%
\definecolor{currentfill}{rgb}{0.121569,0.466667,0.705882}%
\pgfsetfillcolor{currentfill}%
\pgfsetlinewidth{0.000000pt}%
\definecolor{currentstroke}{rgb}{0.000000,0.000000,0.000000}%
\pgfsetstrokecolor{currentstroke}%
\pgfsetstrokeopacity{0.000000}%
\pgfsetdash{}{0pt}%
\pgfpathmoveto{\pgfqpoint{5.800000in}{140.269901in}}%
\pgfpathlineto{\pgfqpoint{5.800000in}{140.269901in}}%
\pgfpathlineto{\pgfqpoint{5.800000in}{140.372738in}}%
\pgfpathlineto{\pgfqpoint{5.800000in}{140.372738in}}%
\pgfpathclose%
\pgfusepath{fill}%
\end{pgfscope}%
\begin{pgfscope}%
\pgfpathrectangle{\pgfqpoint{5.800000in}{0.720000in}}{\pgfqpoint{1.400000in}{4.620000in}}%
\pgfusepath{clip}%
\pgfsetbuttcap%
\pgfsetmiterjoin%
\definecolor{currentfill}{rgb}{0.121569,0.466667,0.705882}%
\pgfsetfillcolor{currentfill}%
\pgfsetlinewidth{0.000000pt}%
\definecolor{currentstroke}{rgb}{0.000000,0.000000,0.000000}%
\pgfsetstrokecolor{currentstroke}%
\pgfsetstrokeopacity{0.000000}%
\pgfsetdash{}{0pt}%
\pgfpathmoveto{\pgfqpoint{5.800000in}{140.372738in}}%
\pgfpathlineto{\pgfqpoint{5.800000in}{140.372738in}}%
\pgfpathlineto{\pgfqpoint{5.800000in}{140.475575in}}%
\pgfpathlineto{\pgfqpoint{5.800000in}{140.475575in}}%
\pgfpathclose%
\pgfusepath{fill}%
\end{pgfscope}%
\begin{pgfscope}%
\pgfpathrectangle{\pgfqpoint{5.800000in}{0.720000in}}{\pgfqpoint{1.400000in}{4.620000in}}%
\pgfusepath{clip}%
\pgfsetbuttcap%
\pgfsetmiterjoin%
\definecolor{currentfill}{rgb}{0.121569,0.466667,0.705882}%
\pgfsetfillcolor{currentfill}%
\pgfsetlinewidth{0.000000pt}%
\definecolor{currentstroke}{rgb}{0.000000,0.000000,0.000000}%
\pgfsetstrokecolor{currentstroke}%
\pgfsetstrokeopacity{0.000000}%
\pgfsetdash{}{0pt}%
\pgfpathmoveto{\pgfqpoint{5.800000in}{140.475575in}}%
\pgfpathlineto{\pgfqpoint{5.800000in}{140.475575in}}%
\pgfpathlineto{\pgfqpoint{5.800000in}{140.578412in}}%
\pgfpathlineto{\pgfqpoint{5.800000in}{140.578412in}}%
\pgfpathclose%
\pgfusepath{fill}%
\end{pgfscope}%
\begin{pgfscope}%
\pgfpathrectangle{\pgfqpoint{5.800000in}{0.720000in}}{\pgfqpoint{1.400000in}{4.620000in}}%
\pgfusepath{clip}%
\pgfsetbuttcap%
\pgfsetmiterjoin%
\definecolor{currentfill}{rgb}{0.121569,0.466667,0.705882}%
\pgfsetfillcolor{currentfill}%
\pgfsetlinewidth{0.000000pt}%
\definecolor{currentstroke}{rgb}{0.000000,0.000000,0.000000}%
\pgfsetstrokecolor{currentstroke}%
\pgfsetstrokeopacity{0.000000}%
\pgfsetdash{}{0pt}%
\pgfpathmoveto{\pgfqpoint{5.800000in}{140.578412in}}%
\pgfpathlineto{\pgfqpoint{5.800000in}{140.578412in}}%
\pgfpathlineto{\pgfqpoint{5.800000in}{140.681249in}}%
\pgfpathlineto{\pgfqpoint{5.800000in}{140.681249in}}%
\pgfpathclose%
\pgfusepath{fill}%
\end{pgfscope}%
\begin{pgfscope}%
\pgfpathrectangle{\pgfqpoint{5.800000in}{0.720000in}}{\pgfqpoint{1.400000in}{4.620000in}}%
\pgfusepath{clip}%
\pgfsetbuttcap%
\pgfsetmiterjoin%
\definecolor{currentfill}{rgb}{0.121569,0.466667,0.705882}%
\pgfsetfillcolor{currentfill}%
\pgfsetlinewidth{0.000000pt}%
\definecolor{currentstroke}{rgb}{0.000000,0.000000,0.000000}%
\pgfsetstrokecolor{currentstroke}%
\pgfsetstrokeopacity{0.000000}%
\pgfsetdash{}{0pt}%
\pgfpathmoveto{\pgfqpoint{5.800000in}{140.681249in}}%
\pgfpathlineto{\pgfqpoint{5.800000in}{140.681249in}}%
\pgfpathlineto{\pgfqpoint{5.800000in}{140.784086in}}%
\pgfpathlineto{\pgfqpoint{5.800000in}{140.784086in}}%
\pgfpathclose%
\pgfusepath{fill}%
\end{pgfscope}%
\begin{pgfscope}%
\pgfpathrectangle{\pgfqpoint{5.800000in}{0.720000in}}{\pgfqpoint{1.400000in}{4.620000in}}%
\pgfusepath{clip}%
\pgfsetbuttcap%
\pgfsetmiterjoin%
\definecolor{currentfill}{rgb}{0.121569,0.466667,0.705882}%
\pgfsetfillcolor{currentfill}%
\pgfsetlinewidth{0.000000pt}%
\definecolor{currentstroke}{rgb}{0.000000,0.000000,0.000000}%
\pgfsetstrokecolor{currentstroke}%
\pgfsetstrokeopacity{0.000000}%
\pgfsetdash{}{0pt}%
\pgfpathmoveto{\pgfqpoint{5.800000in}{140.784086in}}%
\pgfpathlineto{\pgfqpoint{5.800000in}{140.784086in}}%
\pgfpathlineto{\pgfqpoint{5.800000in}{140.886923in}}%
\pgfpathlineto{\pgfqpoint{5.800000in}{140.886923in}}%
\pgfpathclose%
\pgfusepath{fill}%
\end{pgfscope}%
\begin{pgfscope}%
\pgfpathrectangle{\pgfqpoint{5.800000in}{0.720000in}}{\pgfqpoint{1.400000in}{4.620000in}}%
\pgfusepath{clip}%
\pgfsetbuttcap%
\pgfsetmiterjoin%
\definecolor{currentfill}{rgb}{0.121569,0.466667,0.705882}%
\pgfsetfillcolor{currentfill}%
\pgfsetlinewidth{0.000000pt}%
\definecolor{currentstroke}{rgb}{0.000000,0.000000,0.000000}%
\pgfsetstrokecolor{currentstroke}%
\pgfsetstrokeopacity{0.000000}%
\pgfsetdash{}{0pt}%
\pgfpathmoveto{\pgfqpoint{5.800000in}{140.886923in}}%
\pgfpathlineto{\pgfqpoint{5.800000in}{140.886923in}}%
\pgfpathlineto{\pgfqpoint{5.800000in}{140.989760in}}%
\pgfpathlineto{\pgfqpoint{5.800000in}{140.989760in}}%
\pgfpathclose%
\pgfusepath{fill}%
\end{pgfscope}%
\begin{pgfscope}%
\pgfpathrectangle{\pgfqpoint{5.800000in}{0.720000in}}{\pgfqpoint{1.400000in}{4.620000in}}%
\pgfusepath{clip}%
\pgfsetbuttcap%
\pgfsetmiterjoin%
\definecolor{currentfill}{rgb}{0.121569,0.466667,0.705882}%
\pgfsetfillcolor{currentfill}%
\pgfsetlinewidth{0.000000pt}%
\definecolor{currentstroke}{rgb}{0.000000,0.000000,0.000000}%
\pgfsetstrokecolor{currentstroke}%
\pgfsetstrokeopacity{0.000000}%
\pgfsetdash{}{0pt}%
\pgfpathmoveto{\pgfqpoint{5.800000in}{140.989760in}}%
\pgfpathlineto{\pgfqpoint{5.800000in}{140.989760in}}%
\pgfpathlineto{\pgfqpoint{5.800000in}{141.092598in}}%
\pgfpathlineto{\pgfqpoint{5.800000in}{141.092598in}}%
\pgfpathclose%
\pgfusepath{fill}%
\end{pgfscope}%
\begin{pgfscope}%
\pgfpathrectangle{\pgfqpoint{5.800000in}{0.720000in}}{\pgfqpoint{1.400000in}{4.620000in}}%
\pgfusepath{clip}%
\pgfsetbuttcap%
\pgfsetmiterjoin%
\definecolor{currentfill}{rgb}{0.121569,0.466667,0.705882}%
\pgfsetfillcolor{currentfill}%
\pgfsetlinewidth{0.000000pt}%
\definecolor{currentstroke}{rgb}{0.000000,0.000000,0.000000}%
\pgfsetstrokecolor{currentstroke}%
\pgfsetstrokeopacity{0.000000}%
\pgfsetdash{}{0pt}%
\pgfpathmoveto{\pgfqpoint{5.800000in}{141.092598in}}%
\pgfpathlineto{\pgfqpoint{5.800000in}{141.092598in}}%
\pgfpathlineto{\pgfqpoint{5.800000in}{141.195435in}}%
\pgfpathlineto{\pgfqpoint{5.800000in}{141.195435in}}%
\pgfpathclose%
\pgfusepath{fill}%
\end{pgfscope}%
\begin{pgfscope}%
\pgfpathrectangle{\pgfqpoint{5.800000in}{0.720000in}}{\pgfqpoint{1.400000in}{4.620000in}}%
\pgfusepath{clip}%
\pgfsetbuttcap%
\pgfsetmiterjoin%
\definecolor{currentfill}{rgb}{0.121569,0.466667,0.705882}%
\pgfsetfillcolor{currentfill}%
\pgfsetlinewidth{0.000000pt}%
\definecolor{currentstroke}{rgb}{0.000000,0.000000,0.000000}%
\pgfsetstrokecolor{currentstroke}%
\pgfsetstrokeopacity{0.000000}%
\pgfsetdash{}{0pt}%
\pgfpathmoveto{\pgfqpoint{5.800000in}{141.195435in}}%
\pgfpathlineto{\pgfqpoint{5.800000in}{141.195435in}}%
\pgfpathlineto{\pgfqpoint{5.800000in}{141.298272in}}%
\pgfpathlineto{\pgfqpoint{5.800000in}{141.298272in}}%
\pgfpathclose%
\pgfusepath{fill}%
\end{pgfscope}%
\begin{pgfscope}%
\pgfpathrectangle{\pgfqpoint{5.800000in}{0.720000in}}{\pgfqpoint{1.400000in}{4.620000in}}%
\pgfusepath{clip}%
\pgfsetbuttcap%
\pgfsetmiterjoin%
\definecolor{currentfill}{rgb}{0.121569,0.466667,0.705882}%
\pgfsetfillcolor{currentfill}%
\pgfsetlinewidth{0.000000pt}%
\definecolor{currentstroke}{rgb}{0.000000,0.000000,0.000000}%
\pgfsetstrokecolor{currentstroke}%
\pgfsetstrokeopacity{0.000000}%
\pgfsetdash{}{0pt}%
\pgfpathmoveto{\pgfqpoint{5.800000in}{141.298272in}}%
\pgfpathlineto{\pgfqpoint{5.800000in}{141.298272in}}%
\pgfpathlineto{\pgfqpoint{5.800000in}{141.401109in}}%
\pgfpathlineto{\pgfqpoint{5.800000in}{141.401109in}}%
\pgfpathclose%
\pgfusepath{fill}%
\end{pgfscope}%
\begin{pgfscope}%
\pgfpathrectangle{\pgfqpoint{5.800000in}{0.720000in}}{\pgfqpoint{1.400000in}{4.620000in}}%
\pgfusepath{clip}%
\pgfsetbuttcap%
\pgfsetmiterjoin%
\definecolor{currentfill}{rgb}{0.121569,0.466667,0.705882}%
\pgfsetfillcolor{currentfill}%
\pgfsetlinewidth{0.000000pt}%
\definecolor{currentstroke}{rgb}{0.000000,0.000000,0.000000}%
\pgfsetstrokecolor{currentstroke}%
\pgfsetstrokeopacity{0.000000}%
\pgfsetdash{}{0pt}%
\pgfpathmoveto{\pgfqpoint{5.800000in}{141.401109in}}%
\pgfpathlineto{\pgfqpoint{5.800000in}{141.401109in}}%
\pgfpathlineto{\pgfqpoint{5.800000in}{141.503946in}}%
\pgfpathlineto{\pgfqpoint{5.800000in}{141.503946in}}%
\pgfpathclose%
\pgfusepath{fill}%
\end{pgfscope}%
\begin{pgfscope}%
\pgfpathrectangle{\pgfqpoint{5.800000in}{0.720000in}}{\pgfqpoint{1.400000in}{4.620000in}}%
\pgfusepath{clip}%
\pgfsetbuttcap%
\pgfsetmiterjoin%
\definecolor{currentfill}{rgb}{0.121569,0.466667,0.705882}%
\pgfsetfillcolor{currentfill}%
\pgfsetlinewidth{0.000000pt}%
\definecolor{currentstroke}{rgb}{0.000000,0.000000,0.000000}%
\pgfsetstrokecolor{currentstroke}%
\pgfsetstrokeopacity{0.000000}%
\pgfsetdash{}{0pt}%
\pgfpathmoveto{\pgfqpoint{5.800000in}{141.503946in}}%
\pgfpathlineto{\pgfqpoint{5.800000in}{141.503946in}}%
\pgfpathlineto{\pgfqpoint{5.800000in}{141.606783in}}%
\pgfpathlineto{\pgfqpoint{5.800000in}{141.606783in}}%
\pgfpathclose%
\pgfusepath{fill}%
\end{pgfscope}%
\begin{pgfscope}%
\pgfpathrectangle{\pgfqpoint{5.800000in}{0.720000in}}{\pgfqpoint{1.400000in}{4.620000in}}%
\pgfusepath{clip}%
\pgfsetbuttcap%
\pgfsetmiterjoin%
\definecolor{currentfill}{rgb}{0.121569,0.466667,0.705882}%
\pgfsetfillcolor{currentfill}%
\pgfsetlinewidth{0.000000pt}%
\definecolor{currentstroke}{rgb}{0.000000,0.000000,0.000000}%
\pgfsetstrokecolor{currentstroke}%
\pgfsetstrokeopacity{0.000000}%
\pgfsetdash{}{0pt}%
\pgfpathmoveto{\pgfqpoint{5.800000in}{141.606783in}}%
\pgfpathlineto{\pgfqpoint{5.800000in}{141.606783in}}%
\pgfpathlineto{\pgfqpoint{5.800000in}{141.709620in}}%
\pgfpathlineto{\pgfqpoint{5.800000in}{141.709620in}}%
\pgfpathclose%
\pgfusepath{fill}%
\end{pgfscope}%
\begin{pgfscope}%
\pgfpathrectangle{\pgfqpoint{5.800000in}{0.720000in}}{\pgfqpoint{1.400000in}{4.620000in}}%
\pgfusepath{clip}%
\pgfsetbuttcap%
\pgfsetmiterjoin%
\definecolor{currentfill}{rgb}{0.121569,0.466667,0.705882}%
\pgfsetfillcolor{currentfill}%
\pgfsetlinewidth{0.000000pt}%
\definecolor{currentstroke}{rgb}{0.000000,0.000000,0.000000}%
\pgfsetstrokecolor{currentstroke}%
\pgfsetstrokeopacity{0.000000}%
\pgfsetdash{}{0pt}%
\pgfpathmoveto{\pgfqpoint{5.800000in}{141.709620in}}%
\pgfpathlineto{\pgfqpoint{5.800000in}{141.709620in}}%
\pgfpathlineto{\pgfqpoint{5.800000in}{141.812457in}}%
\pgfpathlineto{\pgfqpoint{5.800000in}{141.812457in}}%
\pgfpathclose%
\pgfusepath{fill}%
\end{pgfscope}%
\begin{pgfscope}%
\pgfpathrectangle{\pgfqpoint{5.800000in}{0.720000in}}{\pgfqpoint{1.400000in}{4.620000in}}%
\pgfusepath{clip}%
\pgfsetbuttcap%
\pgfsetmiterjoin%
\definecolor{currentfill}{rgb}{0.121569,0.466667,0.705882}%
\pgfsetfillcolor{currentfill}%
\pgfsetlinewidth{0.000000pt}%
\definecolor{currentstroke}{rgb}{0.000000,0.000000,0.000000}%
\pgfsetstrokecolor{currentstroke}%
\pgfsetstrokeopacity{0.000000}%
\pgfsetdash{}{0pt}%
\pgfpathmoveto{\pgfqpoint{5.800000in}{141.812457in}}%
\pgfpathlineto{\pgfqpoint{5.800000in}{141.812457in}}%
\pgfpathlineto{\pgfqpoint{5.800000in}{141.915294in}}%
\pgfpathlineto{\pgfqpoint{5.800000in}{141.915294in}}%
\pgfpathclose%
\pgfusepath{fill}%
\end{pgfscope}%
\begin{pgfscope}%
\pgfpathrectangle{\pgfqpoint{5.800000in}{0.720000in}}{\pgfqpoint{1.400000in}{4.620000in}}%
\pgfusepath{clip}%
\pgfsetbuttcap%
\pgfsetmiterjoin%
\definecolor{currentfill}{rgb}{0.121569,0.466667,0.705882}%
\pgfsetfillcolor{currentfill}%
\pgfsetlinewidth{0.000000pt}%
\definecolor{currentstroke}{rgb}{0.000000,0.000000,0.000000}%
\pgfsetstrokecolor{currentstroke}%
\pgfsetstrokeopacity{0.000000}%
\pgfsetdash{}{0pt}%
\pgfpathmoveto{\pgfqpoint{5.800000in}{141.915294in}}%
\pgfpathlineto{\pgfqpoint{5.800000in}{141.915294in}}%
\pgfpathlineto{\pgfqpoint{5.800000in}{142.018131in}}%
\pgfpathlineto{\pgfqpoint{5.800000in}{142.018131in}}%
\pgfpathclose%
\pgfusepath{fill}%
\end{pgfscope}%
\begin{pgfscope}%
\pgfpathrectangle{\pgfqpoint{5.800000in}{0.720000in}}{\pgfqpoint{1.400000in}{4.620000in}}%
\pgfusepath{clip}%
\pgfsetbuttcap%
\pgfsetmiterjoin%
\definecolor{currentfill}{rgb}{0.121569,0.466667,0.705882}%
\pgfsetfillcolor{currentfill}%
\pgfsetlinewidth{0.000000pt}%
\definecolor{currentstroke}{rgb}{0.000000,0.000000,0.000000}%
\pgfsetstrokecolor{currentstroke}%
\pgfsetstrokeopacity{0.000000}%
\pgfsetdash{}{0pt}%
\pgfpathmoveto{\pgfqpoint{5.800000in}{142.018131in}}%
\pgfpathlineto{\pgfqpoint{5.800000in}{142.018131in}}%
\pgfpathlineto{\pgfqpoint{5.800000in}{142.120968in}}%
\pgfpathlineto{\pgfqpoint{5.800000in}{142.120968in}}%
\pgfpathclose%
\pgfusepath{fill}%
\end{pgfscope}%
\begin{pgfscope}%
\pgfpathrectangle{\pgfqpoint{5.800000in}{0.720000in}}{\pgfqpoint{1.400000in}{4.620000in}}%
\pgfusepath{clip}%
\pgfsetbuttcap%
\pgfsetmiterjoin%
\definecolor{currentfill}{rgb}{0.121569,0.466667,0.705882}%
\pgfsetfillcolor{currentfill}%
\pgfsetlinewidth{0.000000pt}%
\definecolor{currentstroke}{rgb}{0.000000,0.000000,0.000000}%
\pgfsetstrokecolor{currentstroke}%
\pgfsetstrokeopacity{0.000000}%
\pgfsetdash{}{0pt}%
\pgfpathmoveto{\pgfqpoint{5.800000in}{142.120968in}}%
\pgfpathlineto{\pgfqpoint{5.800000in}{142.120968in}}%
\pgfpathlineto{\pgfqpoint{5.800000in}{142.223805in}}%
\pgfpathlineto{\pgfqpoint{5.800000in}{142.223805in}}%
\pgfpathclose%
\pgfusepath{fill}%
\end{pgfscope}%
\begin{pgfscope}%
\pgfpathrectangle{\pgfqpoint{5.800000in}{0.720000in}}{\pgfqpoint{1.400000in}{4.620000in}}%
\pgfusepath{clip}%
\pgfsetbuttcap%
\pgfsetmiterjoin%
\definecolor{currentfill}{rgb}{0.121569,0.466667,0.705882}%
\pgfsetfillcolor{currentfill}%
\pgfsetlinewidth{0.000000pt}%
\definecolor{currentstroke}{rgb}{0.000000,0.000000,0.000000}%
\pgfsetstrokecolor{currentstroke}%
\pgfsetstrokeopacity{0.000000}%
\pgfsetdash{}{0pt}%
\pgfpathmoveto{\pgfqpoint{5.800000in}{142.223805in}}%
\pgfpathlineto{\pgfqpoint{5.800000in}{142.223805in}}%
\pgfpathlineto{\pgfqpoint{5.800000in}{142.326642in}}%
\pgfpathlineto{\pgfqpoint{5.800000in}{142.326642in}}%
\pgfpathclose%
\pgfusepath{fill}%
\end{pgfscope}%
\begin{pgfscope}%
\pgfpathrectangle{\pgfqpoint{5.800000in}{0.720000in}}{\pgfqpoint{1.400000in}{4.620000in}}%
\pgfusepath{clip}%
\pgfsetbuttcap%
\pgfsetmiterjoin%
\definecolor{currentfill}{rgb}{0.121569,0.466667,0.705882}%
\pgfsetfillcolor{currentfill}%
\pgfsetlinewidth{0.000000pt}%
\definecolor{currentstroke}{rgb}{0.000000,0.000000,0.000000}%
\pgfsetstrokecolor{currentstroke}%
\pgfsetstrokeopacity{0.000000}%
\pgfsetdash{}{0pt}%
\pgfpathmoveto{\pgfqpoint{5.800000in}{142.326642in}}%
\pgfpathlineto{\pgfqpoint{5.800000in}{142.326642in}}%
\pgfpathlineto{\pgfqpoint{5.800000in}{142.429479in}}%
\pgfpathlineto{\pgfqpoint{5.800000in}{142.429479in}}%
\pgfpathclose%
\pgfusepath{fill}%
\end{pgfscope}%
\begin{pgfscope}%
\pgfpathrectangle{\pgfqpoint{5.800000in}{0.720000in}}{\pgfqpoint{1.400000in}{4.620000in}}%
\pgfusepath{clip}%
\pgfsetbuttcap%
\pgfsetmiterjoin%
\definecolor{currentfill}{rgb}{0.121569,0.466667,0.705882}%
\pgfsetfillcolor{currentfill}%
\pgfsetlinewidth{0.000000pt}%
\definecolor{currentstroke}{rgb}{0.000000,0.000000,0.000000}%
\pgfsetstrokecolor{currentstroke}%
\pgfsetstrokeopacity{0.000000}%
\pgfsetdash{}{0pt}%
\pgfpathmoveto{\pgfqpoint{5.800000in}{142.429479in}}%
\pgfpathlineto{\pgfqpoint{5.800000in}{142.429479in}}%
\pgfpathlineto{\pgfqpoint{5.800000in}{142.532316in}}%
\pgfpathlineto{\pgfqpoint{5.800000in}{142.532316in}}%
\pgfpathclose%
\pgfusepath{fill}%
\end{pgfscope}%
\begin{pgfscope}%
\pgfpathrectangle{\pgfqpoint{5.800000in}{0.720000in}}{\pgfqpoint{1.400000in}{4.620000in}}%
\pgfusepath{clip}%
\pgfsetbuttcap%
\pgfsetmiterjoin%
\definecolor{currentfill}{rgb}{0.121569,0.466667,0.705882}%
\pgfsetfillcolor{currentfill}%
\pgfsetlinewidth{0.000000pt}%
\definecolor{currentstroke}{rgb}{0.000000,0.000000,0.000000}%
\pgfsetstrokecolor{currentstroke}%
\pgfsetstrokeopacity{0.000000}%
\pgfsetdash{}{0pt}%
\pgfpathmoveto{\pgfqpoint{5.800000in}{142.532316in}}%
\pgfpathlineto{\pgfqpoint{5.800000in}{142.532316in}}%
\pgfpathlineto{\pgfqpoint{5.800000in}{142.635154in}}%
\pgfpathlineto{\pgfqpoint{5.800000in}{142.635154in}}%
\pgfpathclose%
\pgfusepath{fill}%
\end{pgfscope}%
\begin{pgfscope}%
\pgfpathrectangle{\pgfqpoint{5.800000in}{0.720000in}}{\pgfqpoint{1.400000in}{4.620000in}}%
\pgfusepath{clip}%
\pgfsetbuttcap%
\pgfsetmiterjoin%
\definecolor{currentfill}{rgb}{0.121569,0.466667,0.705882}%
\pgfsetfillcolor{currentfill}%
\pgfsetlinewidth{0.000000pt}%
\definecolor{currentstroke}{rgb}{0.000000,0.000000,0.000000}%
\pgfsetstrokecolor{currentstroke}%
\pgfsetstrokeopacity{0.000000}%
\pgfsetdash{}{0pt}%
\pgfpathmoveto{\pgfqpoint{5.800000in}{142.635154in}}%
\pgfpathlineto{\pgfqpoint{5.800000in}{142.635154in}}%
\pgfpathlineto{\pgfqpoint{5.800000in}{142.737991in}}%
\pgfpathlineto{\pgfqpoint{5.800000in}{142.737991in}}%
\pgfpathclose%
\pgfusepath{fill}%
\end{pgfscope}%
\begin{pgfscope}%
\pgfpathrectangle{\pgfqpoint{5.800000in}{0.720000in}}{\pgfqpoint{1.400000in}{4.620000in}}%
\pgfusepath{clip}%
\pgfsetbuttcap%
\pgfsetmiterjoin%
\definecolor{currentfill}{rgb}{0.121569,0.466667,0.705882}%
\pgfsetfillcolor{currentfill}%
\pgfsetlinewidth{0.000000pt}%
\definecolor{currentstroke}{rgb}{0.000000,0.000000,0.000000}%
\pgfsetstrokecolor{currentstroke}%
\pgfsetstrokeopacity{0.000000}%
\pgfsetdash{}{0pt}%
\pgfpathmoveto{\pgfqpoint{5.800000in}{142.737991in}}%
\pgfpathlineto{\pgfqpoint{5.800000in}{142.737991in}}%
\pgfpathlineto{\pgfqpoint{5.800000in}{142.840828in}}%
\pgfpathlineto{\pgfqpoint{5.800000in}{142.840828in}}%
\pgfpathclose%
\pgfusepath{fill}%
\end{pgfscope}%
\begin{pgfscope}%
\pgfpathrectangle{\pgfqpoint{5.800000in}{0.720000in}}{\pgfqpoint{1.400000in}{4.620000in}}%
\pgfusepath{clip}%
\pgfsetbuttcap%
\pgfsetmiterjoin%
\definecolor{currentfill}{rgb}{0.121569,0.466667,0.705882}%
\pgfsetfillcolor{currentfill}%
\pgfsetlinewidth{0.000000pt}%
\definecolor{currentstroke}{rgb}{0.000000,0.000000,0.000000}%
\pgfsetstrokecolor{currentstroke}%
\pgfsetstrokeopacity{0.000000}%
\pgfsetdash{}{0pt}%
\pgfpathmoveto{\pgfqpoint{5.800000in}{142.840828in}}%
\pgfpathlineto{\pgfqpoint{5.800000in}{142.840828in}}%
\pgfpathlineto{\pgfqpoint{5.800000in}{142.943665in}}%
\pgfpathlineto{\pgfqpoint{5.800000in}{142.943665in}}%
\pgfpathclose%
\pgfusepath{fill}%
\end{pgfscope}%
\begin{pgfscope}%
\pgfpathrectangle{\pgfqpoint{5.800000in}{0.720000in}}{\pgfqpoint{1.400000in}{4.620000in}}%
\pgfusepath{clip}%
\pgfsetbuttcap%
\pgfsetmiterjoin%
\definecolor{currentfill}{rgb}{0.121569,0.466667,0.705882}%
\pgfsetfillcolor{currentfill}%
\pgfsetlinewidth{0.000000pt}%
\definecolor{currentstroke}{rgb}{0.000000,0.000000,0.000000}%
\pgfsetstrokecolor{currentstroke}%
\pgfsetstrokeopacity{0.000000}%
\pgfsetdash{}{0pt}%
\pgfpathmoveto{\pgfqpoint{5.800000in}{142.943665in}}%
\pgfpathlineto{\pgfqpoint{5.800000in}{142.943665in}}%
\pgfpathlineto{\pgfqpoint{5.800000in}{143.046502in}}%
\pgfpathlineto{\pgfqpoint{5.800000in}{143.046502in}}%
\pgfpathclose%
\pgfusepath{fill}%
\end{pgfscope}%
\begin{pgfscope}%
\pgfpathrectangle{\pgfqpoint{5.800000in}{0.720000in}}{\pgfqpoint{1.400000in}{4.620000in}}%
\pgfusepath{clip}%
\pgfsetbuttcap%
\pgfsetmiterjoin%
\definecolor{currentfill}{rgb}{0.121569,0.466667,0.705882}%
\pgfsetfillcolor{currentfill}%
\pgfsetlinewidth{0.000000pt}%
\definecolor{currentstroke}{rgb}{0.000000,0.000000,0.000000}%
\pgfsetstrokecolor{currentstroke}%
\pgfsetstrokeopacity{0.000000}%
\pgfsetdash{}{0pt}%
\pgfpathmoveto{\pgfqpoint{5.800000in}{143.046502in}}%
\pgfpathlineto{\pgfqpoint{5.800000in}{143.046502in}}%
\pgfpathlineto{\pgfqpoint{5.800000in}{143.149339in}}%
\pgfpathlineto{\pgfqpoint{5.800000in}{143.149339in}}%
\pgfpathclose%
\pgfusepath{fill}%
\end{pgfscope}%
\begin{pgfscope}%
\pgfpathrectangle{\pgfqpoint{5.800000in}{0.720000in}}{\pgfqpoint{1.400000in}{4.620000in}}%
\pgfusepath{clip}%
\pgfsetbuttcap%
\pgfsetmiterjoin%
\definecolor{currentfill}{rgb}{0.121569,0.466667,0.705882}%
\pgfsetfillcolor{currentfill}%
\pgfsetlinewidth{0.000000pt}%
\definecolor{currentstroke}{rgb}{0.000000,0.000000,0.000000}%
\pgfsetstrokecolor{currentstroke}%
\pgfsetstrokeopacity{0.000000}%
\pgfsetdash{}{0pt}%
\pgfpathmoveto{\pgfqpoint{5.800000in}{143.149339in}}%
\pgfpathlineto{\pgfqpoint{5.800000in}{143.149339in}}%
\pgfpathlineto{\pgfqpoint{5.800000in}{143.252176in}}%
\pgfpathlineto{\pgfqpoint{5.800000in}{143.252176in}}%
\pgfpathclose%
\pgfusepath{fill}%
\end{pgfscope}%
\begin{pgfscope}%
\pgfpathrectangle{\pgfqpoint{5.800000in}{0.720000in}}{\pgfqpoint{1.400000in}{4.620000in}}%
\pgfusepath{clip}%
\pgfsetbuttcap%
\pgfsetmiterjoin%
\definecolor{currentfill}{rgb}{0.121569,0.466667,0.705882}%
\pgfsetfillcolor{currentfill}%
\pgfsetlinewidth{0.000000pt}%
\definecolor{currentstroke}{rgb}{0.000000,0.000000,0.000000}%
\pgfsetstrokecolor{currentstroke}%
\pgfsetstrokeopacity{0.000000}%
\pgfsetdash{}{0pt}%
\pgfpathmoveto{\pgfqpoint{5.800000in}{143.252176in}}%
\pgfpathlineto{\pgfqpoint{5.800000in}{143.252176in}}%
\pgfpathlineto{\pgfqpoint{5.800000in}{143.355013in}}%
\pgfpathlineto{\pgfqpoint{5.800000in}{143.355013in}}%
\pgfpathclose%
\pgfusepath{fill}%
\end{pgfscope}%
\begin{pgfscope}%
\pgfpathrectangle{\pgfqpoint{5.800000in}{0.720000in}}{\pgfqpoint{1.400000in}{4.620000in}}%
\pgfusepath{clip}%
\pgfsetbuttcap%
\pgfsetmiterjoin%
\definecolor{currentfill}{rgb}{0.121569,0.466667,0.705882}%
\pgfsetfillcolor{currentfill}%
\pgfsetlinewidth{0.000000pt}%
\definecolor{currentstroke}{rgb}{0.000000,0.000000,0.000000}%
\pgfsetstrokecolor{currentstroke}%
\pgfsetstrokeopacity{0.000000}%
\pgfsetdash{}{0pt}%
\pgfpathmoveto{\pgfqpoint{5.800000in}{143.355013in}}%
\pgfpathlineto{\pgfqpoint{5.800000in}{143.355013in}}%
\pgfpathlineto{\pgfqpoint{5.800000in}{143.457850in}}%
\pgfpathlineto{\pgfqpoint{5.800000in}{143.457850in}}%
\pgfpathclose%
\pgfusepath{fill}%
\end{pgfscope}%
\begin{pgfscope}%
\pgfpathrectangle{\pgfqpoint{5.800000in}{0.720000in}}{\pgfqpoint{1.400000in}{4.620000in}}%
\pgfusepath{clip}%
\pgfsetbuttcap%
\pgfsetmiterjoin%
\definecolor{currentfill}{rgb}{0.121569,0.466667,0.705882}%
\pgfsetfillcolor{currentfill}%
\pgfsetlinewidth{0.000000pt}%
\definecolor{currentstroke}{rgb}{0.000000,0.000000,0.000000}%
\pgfsetstrokecolor{currentstroke}%
\pgfsetstrokeopacity{0.000000}%
\pgfsetdash{}{0pt}%
\pgfpathmoveto{\pgfqpoint{5.800000in}{143.457850in}}%
\pgfpathlineto{\pgfqpoint{5.800000in}{143.457850in}}%
\pgfpathlineto{\pgfqpoint{5.800000in}{143.560687in}}%
\pgfpathlineto{\pgfqpoint{5.800000in}{143.560687in}}%
\pgfpathclose%
\pgfusepath{fill}%
\end{pgfscope}%
\begin{pgfscope}%
\pgfpathrectangle{\pgfqpoint{5.800000in}{0.720000in}}{\pgfqpoint{1.400000in}{4.620000in}}%
\pgfusepath{clip}%
\pgfsetbuttcap%
\pgfsetmiterjoin%
\definecolor{currentfill}{rgb}{0.121569,0.466667,0.705882}%
\pgfsetfillcolor{currentfill}%
\pgfsetlinewidth{0.000000pt}%
\definecolor{currentstroke}{rgb}{0.000000,0.000000,0.000000}%
\pgfsetstrokecolor{currentstroke}%
\pgfsetstrokeopacity{0.000000}%
\pgfsetdash{}{0pt}%
\pgfpathmoveto{\pgfqpoint{5.800000in}{143.560687in}}%
\pgfpathlineto{\pgfqpoint{5.800000in}{143.560687in}}%
\pgfpathlineto{\pgfqpoint{5.800000in}{143.663524in}}%
\pgfpathlineto{\pgfqpoint{5.800000in}{143.663524in}}%
\pgfpathclose%
\pgfusepath{fill}%
\end{pgfscope}%
\begin{pgfscope}%
\pgfpathrectangle{\pgfqpoint{5.800000in}{0.720000in}}{\pgfqpoint{1.400000in}{4.620000in}}%
\pgfusepath{clip}%
\pgfsetbuttcap%
\pgfsetmiterjoin%
\definecolor{currentfill}{rgb}{0.121569,0.466667,0.705882}%
\pgfsetfillcolor{currentfill}%
\pgfsetlinewidth{0.000000pt}%
\definecolor{currentstroke}{rgb}{0.000000,0.000000,0.000000}%
\pgfsetstrokecolor{currentstroke}%
\pgfsetstrokeopacity{0.000000}%
\pgfsetdash{}{0pt}%
\pgfpathmoveto{\pgfqpoint{5.800000in}{143.663524in}}%
\pgfpathlineto{\pgfqpoint{5.800000in}{143.663524in}}%
\pgfpathlineto{\pgfqpoint{5.800000in}{143.766361in}}%
\pgfpathlineto{\pgfqpoint{5.800000in}{143.766361in}}%
\pgfpathclose%
\pgfusepath{fill}%
\end{pgfscope}%
\begin{pgfscope}%
\pgfpathrectangle{\pgfqpoint{5.800000in}{0.720000in}}{\pgfqpoint{1.400000in}{4.620000in}}%
\pgfusepath{clip}%
\pgfsetbuttcap%
\pgfsetmiterjoin%
\definecolor{currentfill}{rgb}{0.121569,0.466667,0.705882}%
\pgfsetfillcolor{currentfill}%
\pgfsetlinewidth{0.000000pt}%
\definecolor{currentstroke}{rgb}{0.000000,0.000000,0.000000}%
\pgfsetstrokecolor{currentstroke}%
\pgfsetstrokeopacity{0.000000}%
\pgfsetdash{}{0pt}%
\pgfpathmoveto{\pgfqpoint{5.800000in}{143.766361in}}%
\pgfpathlineto{\pgfqpoint{5.800000in}{143.766361in}}%
\pgfpathlineto{\pgfqpoint{5.800000in}{143.869198in}}%
\pgfpathlineto{\pgfqpoint{5.800000in}{143.869198in}}%
\pgfpathclose%
\pgfusepath{fill}%
\end{pgfscope}%
\begin{pgfscope}%
\pgfpathrectangle{\pgfqpoint{5.800000in}{0.720000in}}{\pgfqpoint{1.400000in}{4.620000in}}%
\pgfusepath{clip}%
\pgfsetbuttcap%
\pgfsetmiterjoin%
\definecolor{currentfill}{rgb}{0.121569,0.466667,0.705882}%
\pgfsetfillcolor{currentfill}%
\pgfsetlinewidth{0.000000pt}%
\definecolor{currentstroke}{rgb}{0.000000,0.000000,0.000000}%
\pgfsetstrokecolor{currentstroke}%
\pgfsetstrokeopacity{0.000000}%
\pgfsetdash{}{0pt}%
\pgfpathmoveto{\pgfqpoint{5.800000in}{143.869198in}}%
\pgfpathlineto{\pgfqpoint{5.800000in}{143.869198in}}%
\pgfpathlineto{\pgfqpoint{5.800000in}{143.972035in}}%
\pgfpathlineto{\pgfqpoint{5.800000in}{143.972035in}}%
\pgfpathclose%
\pgfusepath{fill}%
\end{pgfscope}%
\begin{pgfscope}%
\pgfpathrectangle{\pgfqpoint{5.800000in}{0.720000in}}{\pgfqpoint{1.400000in}{4.620000in}}%
\pgfusepath{clip}%
\pgfsetbuttcap%
\pgfsetmiterjoin%
\definecolor{currentfill}{rgb}{0.121569,0.466667,0.705882}%
\pgfsetfillcolor{currentfill}%
\pgfsetlinewidth{0.000000pt}%
\definecolor{currentstroke}{rgb}{0.000000,0.000000,0.000000}%
\pgfsetstrokecolor{currentstroke}%
\pgfsetstrokeopacity{0.000000}%
\pgfsetdash{}{0pt}%
\pgfpathmoveto{\pgfqpoint{5.800000in}{143.972035in}}%
\pgfpathlineto{\pgfqpoint{5.800000in}{143.972035in}}%
\pgfpathlineto{\pgfqpoint{5.800000in}{144.074872in}}%
\pgfpathlineto{\pgfqpoint{5.800000in}{144.074872in}}%
\pgfpathclose%
\pgfusepath{fill}%
\end{pgfscope}%
\begin{pgfscope}%
\pgfpathrectangle{\pgfqpoint{5.800000in}{0.720000in}}{\pgfqpoint{1.400000in}{4.620000in}}%
\pgfusepath{clip}%
\pgfsetbuttcap%
\pgfsetmiterjoin%
\definecolor{currentfill}{rgb}{0.121569,0.466667,0.705882}%
\pgfsetfillcolor{currentfill}%
\pgfsetlinewidth{0.000000pt}%
\definecolor{currentstroke}{rgb}{0.000000,0.000000,0.000000}%
\pgfsetstrokecolor{currentstroke}%
\pgfsetstrokeopacity{0.000000}%
\pgfsetdash{}{0pt}%
\pgfpathmoveto{\pgfqpoint{5.800000in}{144.074872in}}%
\pgfpathlineto{\pgfqpoint{5.800000in}{144.074872in}}%
\pgfpathlineto{\pgfqpoint{5.800000in}{144.177710in}}%
\pgfpathlineto{\pgfqpoint{5.800000in}{144.177710in}}%
\pgfpathclose%
\pgfusepath{fill}%
\end{pgfscope}%
\begin{pgfscope}%
\pgfpathrectangle{\pgfqpoint{5.800000in}{0.720000in}}{\pgfqpoint{1.400000in}{4.620000in}}%
\pgfusepath{clip}%
\pgfsetbuttcap%
\pgfsetmiterjoin%
\definecolor{currentfill}{rgb}{0.121569,0.466667,0.705882}%
\pgfsetfillcolor{currentfill}%
\pgfsetlinewidth{0.000000pt}%
\definecolor{currentstroke}{rgb}{0.000000,0.000000,0.000000}%
\pgfsetstrokecolor{currentstroke}%
\pgfsetstrokeopacity{0.000000}%
\pgfsetdash{}{0pt}%
\pgfpathmoveto{\pgfqpoint{5.800000in}{144.177710in}}%
\pgfpathlineto{\pgfqpoint{5.800000in}{144.177710in}}%
\pgfpathlineto{\pgfqpoint{5.800000in}{144.280547in}}%
\pgfpathlineto{\pgfqpoint{5.800000in}{144.280547in}}%
\pgfpathclose%
\pgfusepath{fill}%
\end{pgfscope}%
\begin{pgfscope}%
\pgfpathrectangle{\pgfqpoint{5.800000in}{0.720000in}}{\pgfqpoint{1.400000in}{4.620000in}}%
\pgfusepath{clip}%
\pgfsetbuttcap%
\pgfsetmiterjoin%
\definecolor{currentfill}{rgb}{0.121569,0.466667,0.705882}%
\pgfsetfillcolor{currentfill}%
\pgfsetlinewidth{0.000000pt}%
\definecolor{currentstroke}{rgb}{0.000000,0.000000,0.000000}%
\pgfsetstrokecolor{currentstroke}%
\pgfsetstrokeopacity{0.000000}%
\pgfsetdash{}{0pt}%
\pgfpathmoveto{\pgfqpoint{5.800000in}{144.280547in}}%
\pgfpathlineto{\pgfqpoint{5.800000in}{144.280547in}}%
\pgfpathlineto{\pgfqpoint{5.800000in}{144.383384in}}%
\pgfpathlineto{\pgfqpoint{5.800000in}{144.383384in}}%
\pgfpathclose%
\pgfusepath{fill}%
\end{pgfscope}%
\begin{pgfscope}%
\pgfpathrectangle{\pgfqpoint{5.800000in}{0.720000in}}{\pgfqpoint{1.400000in}{4.620000in}}%
\pgfusepath{clip}%
\pgfsetbuttcap%
\pgfsetmiterjoin%
\definecolor{currentfill}{rgb}{0.121569,0.466667,0.705882}%
\pgfsetfillcolor{currentfill}%
\pgfsetlinewidth{0.000000pt}%
\definecolor{currentstroke}{rgb}{0.000000,0.000000,0.000000}%
\pgfsetstrokecolor{currentstroke}%
\pgfsetstrokeopacity{0.000000}%
\pgfsetdash{}{0pt}%
\pgfpathmoveto{\pgfqpoint{5.800000in}{144.383384in}}%
\pgfpathlineto{\pgfqpoint{5.800000in}{144.383384in}}%
\pgfpathlineto{\pgfqpoint{5.800000in}{144.486221in}}%
\pgfpathlineto{\pgfqpoint{5.800000in}{144.486221in}}%
\pgfpathclose%
\pgfusepath{fill}%
\end{pgfscope}%
\begin{pgfscope}%
\pgfpathrectangle{\pgfqpoint{5.800000in}{0.720000in}}{\pgfqpoint{1.400000in}{4.620000in}}%
\pgfusepath{clip}%
\pgfsetbuttcap%
\pgfsetmiterjoin%
\definecolor{currentfill}{rgb}{0.121569,0.466667,0.705882}%
\pgfsetfillcolor{currentfill}%
\pgfsetlinewidth{0.000000pt}%
\definecolor{currentstroke}{rgb}{0.000000,0.000000,0.000000}%
\pgfsetstrokecolor{currentstroke}%
\pgfsetstrokeopacity{0.000000}%
\pgfsetdash{}{0pt}%
\pgfpathmoveto{\pgfqpoint{5.800000in}{144.486221in}}%
\pgfpathlineto{\pgfqpoint{5.800000in}{144.486221in}}%
\pgfpathlineto{\pgfqpoint{5.800000in}{144.589058in}}%
\pgfpathlineto{\pgfqpoint{5.800000in}{144.589058in}}%
\pgfpathclose%
\pgfusepath{fill}%
\end{pgfscope}%
\begin{pgfscope}%
\pgfpathrectangle{\pgfqpoint{5.800000in}{0.720000in}}{\pgfqpoint{1.400000in}{4.620000in}}%
\pgfusepath{clip}%
\pgfsetbuttcap%
\pgfsetmiterjoin%
\definecolor{currentfill}{rgb}{0.121569,0.466667,0.705882}%
\pgfsetfillcolor{currentfill}%
\pgfsetlinewidth{0.000000pt}%
\definecolor{currentstroke}{rgb}{0.000000,0.000000,0.000000}%
\pgfsetstrokecolor{currentstroke}%
\pgfsetstrokeopacity{0.000000}%
\pgfsetdash{}{0pt}%
\pgfpathmoveto{\pgfqpoint{5.800000in}{144.589058in}}%
\pgfpathlineto{\pgfqpoint{5.800000in}{144.589058in}}%
\pgfpathlineto{\pgfqpoint{5.800000in}{144.691895in}}%
\pgfpathlineto{\pgfqpoint{5.800000in}{144.691895in}}%
\pgfpathclose%
\pgfusepath{fill}%
\end{pgfscope}%
\begin{pgfscope}%
\pgfpathrectangle{\pgfqpoint{5.800000in}{0.720000in}}{\pgfqpoint{1.400000in}{4.620000in}}%
\pgfusepath{clip}%
\pgfsetbuttcap%
\pgfsetmiterjoin%
\definecolor{currentfill}{rgb}{0.121569,0.466667,0.705882}%
\pgfsetfillcolor{currentfill}%
\pgfsetlinewidth{0.000000pt}%
\definecolor{currentstroke}{rgb}{0.000000,0.000000,0.000000}%
\pgfsetstrokecolor{currentstroke}%
\pgfsetstrokeopacity{0.000000}%
\pgfsetdash{}{0pt}%
\pgfpathmoveto{\pgfqpoint{5.800000in}{144.691895in}}%
\pgfpathlineto{\pgfqpoint{5.800000in}{144.691895in}}%
\pgfpathlineto{\pgfqpoint{5.800000in}{144.794732in}}%
\pgfpathlineto{\pgfqpoint{5.800000in}{144.794732in}}%
\pgfpathclose%
\pgfusepath{fill}%
\end{pgfscope}%
\begin{pgfscope}%
\pgfpathrectangle{\pgfqpoint{5.800000in}{0.720000in}}{\pgfqpoint{1.400000in}{4.620000in}}%
\pgfusepath{clip}%
\pgfsetbuttcap%
\pgfsetmiterjoin%
\definecolor{currentfill}{rgb}{0.121569,0.466667,0.705882}%
\pgfsetfillcolor{currentfill}%
\pgfsetlinewidth{0.000000pt}%
\definecolor{currentstroke}{rgb}{0.000000,0.000000,0.000000}%
\pgfsetstrokecolor{currentstroke}%
\pgfsetstrokeopacity{0.000000}%
\pgfsetdash{}{0pt}%
\pgfpathmoveto{\pgfqpoint{5.800000in}{144.794732in}}%
\pgfpathlineto{\pgfqpoint{5.800000in}{144.794732in}}%
\pgfpathlineto{\pgfqpoint{5.800000in}{144.897569in}}%
\pgfpathlineto{\pgfqpoint{5.800000in}{144.897569in}}%
\pgfpathclose%
\pgfusepath{fill}%
\end{pgfscope}%
\begin{pgfscope}%
\pgfpathrectangle{\pgfqpoint{5.800000in}{0.720000in}}{\pgfqpoint{1.400000in}{4.620000in}}%
\pgfusepath{clip}%
\pgfsetbuttcap%
\pgfsetmiterjoin%
\definecolor{currentfill}{rgb}{0.121569,0.466667,0.705882}%
\pgfsetfillcolor{currentfill}%
\pgfsetlinewidth{0.000000pt}%
\definecolor{currentstroke}{rgb}{0.000000,0.000000,0.000000}%
\pgfsetstrokecolor{currentstroke}%
\pgfsetstrokeopacity{0.000000}%
\pgfsetdash{}{0pt}%
\pgfpathmoveto{\pgfqpoint{5.800000in}{144.897569in}}%
\pgfpathlineto{\pgfqpoint{5.800000in}{144.897569in}}%
\pgfpathlineto{\pgfqpoint{5.800000in}{145.000406in}}%
\pgfpathlineto{\pgfqpoint{5.800000in}{145.000406in}}%
\pgfpathclose%
\pgfusepath{fill}%
\end{pgfscope}%
\begin{pgfscope}%
\pgfpathrectangle{\pgfqpoint{5.800000in}{0.720000in}}{\pgfqpoint{1.400000in}{4.620000in}}%
\pgfusepath{clip}%
\pgfsetbuttcap%
\pgfsetmiterjoin%
\definecolor{currentfill}{rgb}{0.121569,0.466667,0.705882}%
\pgfsetfillcolor{currentfill}%
\pgfsetlinewidth{0.000000pt}%
\definecolor{currentstroke}{rgb}{0.000000,0.000000,0.000000}%
\pgfsetstrokecolor{currentstroke}%
\pgfsetstrokeopacity{0.000000}%
\pgfsetdash{}{0pt}%
\pgfpathmoveto{\pgfqpoint{5.800000in}{145.000406in}}%
\pgfpathlineto{\pgfqpoint{5.800000in}{145.000406in}}%
\pgfpathlineto{\pgfqpoint{5.800000in}{145.103243in}}%
\pgfpathlineto{\pgfqpoint{5.800000in}{145.103243in}}%
\pgfpathclose%
\pgfusepath{fill}%
\end{pgfscope}%
\begin{pgfscope}%
\pgfpathrectangle{\pgfqpoint{5.800000in}{0.720000in}}{\pgfqpoint{1.400000in}{4.620000in}}%
\pgfusepath{clip}%
\pgfsetbuttcap%
\pgfsetmiterjoin%
\definecolor{currentfill}{rgb}{0.121569,0.466667,0.705882}%
\pgfsetfillcolor{currentfill}%
\pgfsetlinewidth{0.000000pt}%
\definecolor{currentstroke}{rgb}{0.000000,0.000000,0.000000}%
\pgfsetstrokecolor{currentstroke}%
\pgfsetstrokeopacity{0.000000}%
\pgfsetdash{}{0pt}%
\pgfpathmoveto{\pgfqpoint{5.800000in}{145.103243in}}%
\pgfpathlineto{\pgfqpoint{5.800000in}{145.103243in}}%
\pgfpathlineto{\pgfqpoint{5.800000in}{145.206080in}}%
\pgfpathlineto{\pgfqpoint{5.800000in}{145.206080in}}%
\pgfpathclose%
\pgfusepath{fill}%
\end{pgfscope}%
\begin{pgfscope}%
\pgfpathrectangle{\pgfqpoint{5.800000in}{0.720000in}}{\pgfqpoint{1.400000in}{4.620000in}}%
\pgfusepath{clip}%
\pgfsetbuttcap%
\pgfsetmiterjoin%
\definecolor{currentfill}{rgb}{0.121569,0.466667,0.705882}%
\pgfsetfillcolor{currentfill}%
\pgfsetlinewidth{0.000000pt}%
\definecolor{currentstroke}{rgb}{0.000000,0.000000,0.000000}%
\pgfsetstrokecolor{currentstroke}%
\pgfsetstrokeopacity{0.000000}%
\pgfsetdash{}{0pt}%
\pgfpathmoveto{\pgfqpoint{5.800000in}{145.206080in}}%
\pgfpathlineto{\pgfqpoint{5.800000in}{145.206080in}}%
\pgfpathlineto{\pgfqpoint{5.800000in}{145.308917in}}%
\pgfpathlineto{\pgfqpoint{5.800000in}{145.308917in}}%
\pgfpathclose%
\pgfusepath{fill}%
\end{pgfscope}%
\begin{pgfscope}%
\pgfpathrectangle{\pgfqpoint{5.800000in}{0.720000in}}{\pgfqpoint{1.400000in}{4.620000in}}%
\pgfusepath{clip}%
\pgfsetbuttcap%
\pgfsetmiterjoin%
\definecolor{currentfill}{rgb}{0.121569,0.466667,0.705882}%
\pgfsetfillcolor{currentfill}%
\pgfsetlinewidth{0.000000pt}%
\definecolor{currentstroke}{rgb}{0.000000,0.000000,0.000000}%
\pgfsetstrokecolor{currentstroke}%
\pgfsetstrokeopacity{0.000000}%
\pgfsetdash{}{0pt}%
\pgfpathmoveto{\pgfqpoint{5.800000in}{145.308917in}}%
\pgfpathlineto{\pgfqpoint{5.800000in}{145.308917in}}%
\pgfpathlineto{\pgfqpoint{5.800000in}{145.411754in}}%
\pgfpathlineto{\pgfqpoint{5.800000in}{145.411754in}}%
\pgfpathclose%
\pgfusepath{fill}%
\end{pgfscope}%
\begin{pgfscope}%
\pgfpathrectangle{\pgfqpoint{5.800000in}{0.720000in}}{\pgfqpoint{1.400000in}{4.620000in}}%
\pgfusepath{clip}%
\pgfsetbuttcap%
\pgfsetmiterjoin%
\definecolor{currentfill}{rgb}{0.121569,0.466667,0.705882}%
\pgfsetfillcolor{currentfill}%
\pgfsetlinewidth{0.000000pt}%
\definecolor{currentstroke}{rgb}{0.000000,0.000000,0.000000}%
\pgfsetstrokecolor{currentstroke}%
\pgfsetstrokeopacity{0.000000}%
\pgfsetdash{}{0pt}%
\pgfpathmoveto{\pgfqpoint{5.800000in}{145.411754in}}%
\pgfpathlineto{\pgfqpoint{5.800000in}{145.411754in}}%
\pgfpathlineto{\pgfqpoint{5.800000in}{145.514591in}}%
\pgfpathlineto{\pgfqpoint{5.800000in}{145.514591in}}%
\pgfpathclose%
\pgfusepath{fill}%
\end{pgfscope}%
\begin{pgfscope}%
\pgfpathrectangle{\pgfqpoint{5.800000in}{0.720000in}}{\pgfqpoint{1.400000in}{4.620000in}}%
\pgfusepath{clip}%
\pgfsetbuttcap%
\pgfsetmiterjoin%
\definecolor{currentfill}{rgb}{0.121569,0.466667,0.705882}%
\pgfsetfillcolor{currentfill}%
\pgfsetlinewidth{0.000000pt}%
\definecolor{currentstroke}{rgb}{0.000000,0.000000,0.000000}%
\pgfsetstrokecolor{currentstroke}%
\pgfsetstrokeopacity{0.000000}%
\pgfsetdash{}{0pt}%
\pgfpathmoveto{\pgfqpoint{5.800000in}{145.514591in}}%
\pgfpathlineto{\pgfqpoint{5.800000in}{145.514591in}}%
\pgfpathlineto{\pgfqpoint{5.800000in}{145.617428in}}%
\pgfpathlineto{\pgfqpoint{5.800000in}{145.617428in}}%
\pgfpathclose%
\pgfusepath{fill}%
\end{pgfscope}%
\begin{pgfscope}%
\pgfpathrectangle{\pgfqpoint{5.800000in}{0.720000in}}{\pgfqpoint{1.400000in}{4.620000in}}%
\pgfusepath{clip}%
\pgfsetbuttcap%
\pgfsetmiterjoin%
\definecolor{currentfill}{rgb}{0.121569,0.466667,0.705882}%
\pgfsetfillcolor{currentfill}%
\pgfsetlinewidth{0.000000pt}%
\definecolor{currentstroke}{rgb}{0.000000,0.000000,0.000000}%
\pgfsetstrokecolor{currentstroke}%
\pgfsetstrokeopacity{0.000000}%
\pgfsetdash{}{0pt}%
\pgfpathmoveto{\pgfqpoint{5.800000in}{145.617428in}}%
\pgfpathlineto{\pgfqpoint{5.800000in}{145.617428in}}%
\pgfpathlineto{\pgfqpoint{5.800000in}{145.720266in}}%
\pgfpathlineto{\pgfqpoint{5.800000in}{145.720266in}}%
\pgfpathclose%
\pgfusepath{fill}%
\end{pgfscope}%
\begin{pgfscope}%
\pgfpathrectangle{\pgfqpoint{5.800000in}{0.720000in}}{\pgfqpoint{1.400000in}{4.620000in}}%
\pgfusepath{clip}%
\pgfsetbuttcap%
\pgfsetmiterjoin%
\definecolor{currentfill}{rgb}{0.121569,0.466667,0.705882}%
\pgfsetfillcolor{currentfill}%
\pgfsetlinewidth{0.000000pt}%
\definecolor{currentstroke}{rgb}{0.000000,0.000000,0.000000}%
\pgfsetstrokecolor{currentstroke}%
\pgfsetstrokeopacity{0.000000}%
\pgfsetdash{}{0pt}%
\pgfpathmoveto{\pgfqpoint{5.800000in}{145.720266in}}%
\pgfpathlineto{\pgfqpoint{5.800000in}{145.720266in}}%
\pgfpathlineto{\pgfqpoint{5.800000in}{145.823103in}}%
\pgfpathlineto{\pgfqpoint{5.800000in}{145.823103in}}%
\pgfpathclose%
\pgfusepath{fill}%
\end{pgfscope}%
\begin{pgfscope}%
\pgfpathrectangle{\pgfqpoint{5.800000in}{0.720000in}}{\pgfqpoint{1.400000in}{4.620000in}}%
\pgfusepath{clip}%
\pgfsetbuttcap%
\pgfsetmiterjoin%
\definecolor{currentfill}{rgb}{0.121569,0.466667,0.705882}%
\pgfsetfillcolor{currentfill}%
\pgfsetlinewidth{0.000000pt}%
\definecolor{currentstroke}{rgb}{0.000000,0.000000,0.000000}%
\pgfsetstrokecolor{currentstroke}%
\pgfsetstrokeopacity{0.000000}%
\pgfsetdash{}{0pt}%
\pgfpathmoveto{\pgfqpoint{5.800000in}{145.823103in}}%
\pgfpathlineto{\pgfqpoint{5.800000in}{145.823103in}}%
\pgfpathlineto{\pgfqpoint{5.800000in}{145.925940in}}%
\pgfpathlineto{\pgfqpoint{5.800000in}{145.925940in}}%
\pgfpathclose%
\pgfusepath{fill}%
\end{pgfscope}%
\begin{pgfscope}%
\pgfpathrectangle{\pgfqpoint{5.800000in}{0.720000in}}{\pgfqpoint{1.400000in}{4.620000in}}%
\pgfusepath{clip}%
\pgfsetbuttcap%
\pgfsetmiterjoin%
\definecolor{currentfill}{rgb}{0.121569,0.466667,0.705882}%
\pgfsetfillcolor{currentfill}%
\pgfsetlinewidth{0.000000pt}%
\definecolor{currentstroke}{rgb}{0.000000,0.000000,0.000000}%
\pgfsetstrokecolor{currentstroke}%
\pgfsetstrokeopacity{0.000000}%
\pgfsetdash{}{0pt}%
\pgfpathmoveto{\pgfqpoint{5.800000in}{145.925940in}}%
\pgfpathlineto{\pgfqpoint{5.800000in}{145.925940in}}%
\pgfpathlineto{\pgfqpoint{5.800000in}{146.028777in}}%
\pgfpathlineto{\pgfqpoint{5.800000in}{146.028777in}}%
\pgfpathclose%
\pgfusepath{fill}%
\end{pgfscope}%
\begin{pgfscope}%
\pgfpathrectangle{\pgfqpoint{5.800000in}{0.720000in}}{\pgfqpoint{1.400000in}{4.620000in}}%
\pgfusepath{clip}%
\pgfsetbuttcap%
\pgfsetmiterjoin%
\definecolor{currentfill}{rgb}{0.121569,0.466667,0.705882}%
\pgfsetfillcolor{currentfill}%
\pgfsetlinewidth{0.000000pt}%
\definecolor{currentstroke}{rgb}{0.000000,0.000000,0.000000}%
\pgfsetstrokecolor{currentstroke}%
\pgfsetstrokeopacity{0.000000}%
\pgfsetdash{}{0pt}%
\pgfpathmoveto{\pgfqpoint{5.800000in}{146.028777in}}%
\pgfpathlineto{\pgfqpoint{5.800000in}{146.028777in}}%
\pgfpathlineto{\pgfqpoint{5.800000in}{146.131614in}}%
\pgfpathlineto{\pgfqpoint{5.800000in}{146.131614in}}%
\pgfpathclose%
\pgfusepath{fill}%
\end{pgfscope}%
\begin{pgfscope}%
\pgfpathrectangle{\pgfqpoint{5.800000in}{0.720000in}}{\pgfqpoint{1.400000in}{4.620000in}}%
\pgfusepath{clip}%
\pgfsetbuttcap%
\pgfsetmiterjoin%
\definecolor{currentfill}{rgb}{0.121569,0.466667,0.705882}%
\pgfsetfillcolor{currentfill}%
\pgfsetlinewidth{0.000000pt}%
\definecolor{currentstroke}{rgb}{0.000000,0.000000,0.000000}%
\pgfsetstrokecolor{currentstroke}%
\pgfsetstrokeopacity{0.000000}%
\pgfsetdash{}{0pt}%
\pgfpathmoveto{\pgfqpoint{5.800000in}{146.131614in}}%
\pgfpathlineto{\pgfqpoint{5.800000in}{146.131614in}}%
\pgfpathlineto{\pgfqpoint{5.800000in}{146.234451in}}%
\pgfpathlineto{\pgfqpoint{5.800000in}{146.234451in}}%
\pgfpathclose%
\pgfusepath{fill}%
\end{pgfscope}%
\begin{pgfscope}%
\pgfpathrectangle{\pgfqpoint{5.800000in}{0.720000in}}{\pgfqpoint{1.400000in}{4.620000in}}%
\pgfusepath{clip}%
\pgfsetbuttcap%
\pgfsetmiterjoin%
\definecolor{currentfill}{rgb}{0.121569,0.466667,0.705882}%
\pgfsetfillcolor{currentfill}%
\pgfsetlinewidth{0.000000pt}%
\definecolor{currentstroke}{rgb}{0.000000,0.000000,0.000000}%
\pgfsetstrokecolor{currentstroke}%
\pgfsetstrokeopacity{0.000000}%
\pgfsetdash{}{0pt}%
\pgfpathmoveto{\pgfqpoint{5.800000in}{146.234451in}}%
\pgfpathlineto{\pgfqpoint{5.800000in}{146.234451in}}%
\pgfpathlineto{\pgfqpoint{5.800000in}{146.337288in}}%
\pgfpathlineto{\pgfqpoint{5.800000in}{146.337288in}}%
\pgfpathclose%
\pgfusepath{fill}%
\end{pgfscope}%
\begin{pgfscope}%
\pgfpathrectangle{\pgfqpoint{5.800000in}{0.720000in}}{\pgfqpoint{1.400000in}{4.620000in}}%
\pgfusepath{clip}%
\pgfsetbuttcap%
\pgfsetmiterjoin%
\definecolor{currentfill}{rgb}{0.121569,0.466667,0.705882}%
\pgfsetfillcolor{currentfill}%
\pgfsetlinewidth{0.000000pt}%
\definecolor{currentstroke}{rgb}{0.000000,0.000000,0.000000}%
\pgfsetstrokecolor{currentstroke}%
\pgfsetstrokeopacity{0.000000}%
\pgfsetdash{}{0pt}%
\pgfpathmoveto{\pgfqpoint{5.800000in}{146.337288in}}%
\pgfpathlineto{\pgfqpoint{5.800000in}{146.337288in}}%
\pgfpathlineto{\pgfqpoint{5.800000in}{146.440125in}}%
\pgfpathlineto{\pgfqpoint{5.800000in}{146.440125in}}%
\pgfpathclose%
\pgfusepath{fill}%
\end{pgfscope}%
\begin{pgfscope}%
\pgfpathrectangle{\pgfqpoint{5.800000in}{0.720000in}}{\pgfqpoint{1.400000in}{4.620000in}}%
\pgfusepath{clip}%
\pgfsetbuttcap%
\pgfsetmiterjoin%
\definecolor{currentfill}{rgb}{0.121569,0.466667,0.705882}%
\pgfsetfillcolor{currentfill}%
\pgfsetlinewidth{0.000000pt}%
\definecolor{currentstroke}{rgb}{0.000000,0.000000,0.000000}%
\pgfsetstrokecolor{currentstroke}%
\pgfsetstrokeopacity{0.000000}%
\pgfsetdash{}{0pt}%
\pgfpathmoveto{\pgfqpoint{5.800000in}{146.440125in}}%
\pgfpathlineto{\pgfqpoint{5.800000in}{146.440125in}}%
\pgfpathlineto{\pgfqpoint{5.800000in}{146.542962in}}%
\pgfpathlineto{\pgfqpoint{5.800000in}{146.542962in}}%
\pgfpathclose%
\pgfusepath{fill}%
\end{pgfscope}%
\begin{pgfscope}%
\pgfpathrectangle{\pgfqpoint{5.800000in}{0.720000in}}{\pgfqpoint{1.400000in}{4.620000in}}%
\pgfusepath{clip}%
\pgfsetbuttcap%
\pgfsetmiterjoin%
\definecolor{currentfill}{rgb}{0.121569,0.466667,0.705882}%
\pgfsetfillcolor{currentfill}%
\pgfsetlinewidth{0.000000pt}%
\definecolor{currentstroke}{rgb}{0.000000,0.000000,0.000000}%
\pgfsetstrokecolor{currentstroke}%
\pgfsetstrokeopacity{0.000000}%
\pgfsetdash{}{0pt}%
\pgfpathmoveto{\pgfqpoint{5.800000in}{146.542962in}}%
\pgfpathlineto{\pgfqpoint{5.800000in}{146.542962in}}%
\pgfpathlineto{\pgfqpoint{5.800000in}{146.645799in}}%
\pgfpathlineto{\pgfqpoint{5.800000in}{146.645799in}}%
\pgfpathclose%
\pgfusepath{fill}%
\end{pgfscope}%
\begin{pgfscope}%
\pgfpathrectangle{\pgfqpoint{5.800000in}{0.720000in}}{\pgfqpoint{1.400000in}{4.620000in}}%
\pgfusepath{clip}%
\pgfsetbuttcap%
\pgfsetmiterjoin%
\definecolor{currentfill}{rgb}{0.121569,0.466667,0.705882}%
\pgfsetfillcolor{currentfill}%
\pgfsetlinewidth{0.000000pt}%
\definecolor{currentstroke}{rgb}{0.000000,0.000000,0.000000}%
\pgfsetstrokecolor{currentstroke}%
\pgfsetstrokeopacity{0.000000}%
\pgfsetdash{}{0pt}%
\pgfpathmoveto{\pgfqpoint{5.800000in}{146.645799in}}%
\pgfpathlineto{\pgfqpoint{5.800000in}{146.645799in}}%
\pgfpathlineto{\pgfqpoint{5.800000in}{146.748636in}}%
\pgfpathlineto{\pgfqpoint{5.800000in}{146.748636in}}%
\pgfpathclose%
\pgfusepath{fill}%
\end{pgfscope}%
\begin{pgfscope}%
\pgfpathrectangle{\pgfqpoint{5.800000in}{0.720000in}}{\pgfqpoint{1.400000in}{4.620000in}}%
\pgfusepath{clip}%
\pgfsetbuttcap%
\pgfsetmiterjoin%
\definecolor{currentfill}{rgb}{0.121569,0.466667,0.705882}%
\pgfsetfillcolor{currentfill}%
\pgfsetlinewidth{0.000000pt}%
\definecolor{currentstroke}{rgb}{0.000000,0.000000,0.000000}%
\pgfsetstrokecolor{currentstroke}%
\pgfsetstrokeopacity{0.000000}%
\pgfsetdash{}{0pt}%
\pgfpathmoveto{\pgfqpoint{5.800000in}{146.748636in}}%
\pgfpathlineto{\pgfqpoint{5.800000in}{146.748636in}}%
\pgfpathlineto{\pgfqpoint{5.800000in}{146.851473in}}%
\pgfpathlineto{\pgfqpoint{5.800000in}{146.851473in}}%
\pgfpathclose%
\pgfusepath{fill}%
\end{pgfscope}%
\begin{pgfscope}%
\pgfpathrectangle{\pgfqpoint{5.800000in}{0.720000in}}{\pgfqpoint{1.400000in}{4.620000in}}%
\pgfusepath{clip}%
\pgfsetbuttcap%
\pgfsetmiterjoin%
\definecolor{currentfill}{rgb}{0.121569,0.466667,0.705882}%
\pgfsetfillcolor{currentfill}%
\pgfsetlinewidth{0.000000pt}%
\definecolor{currentstroke}{rgb}{0.000000,0.000000,0.000000}%
\pgfsetstrokecolor{currentstroke}%
\pgfsetstrokeopacity{0.000000}%
\pgfsetdash{}{0pt}%
\pgfpathmoveto{\pgfqpoint{5.800000in}{146.851473in}}%
\pgfpathlineto{\pgfqpoint{5.800000in}{146.851473in}}%
\pgfpathlineto{\pgfqpoint{5.800000in}{146.954310in}}%
\pgfpathlineto{\pgfqpoint{5.800000in}{146.954310in}}%
\pgfpathclose%
\pgfusepath{fill}%
\end{pgfscope}%
\begin{pgfscope}%
\pgfpathrectangle{\pgfqpoint{5.800000in}{0.720000in}}{\pgfqpoint{1.400000in}{4.620000in}}%
\pgfusepath{clip}%
\pgfsetbuttcap%
\pgfsetmiterjoin%
\definecolor{currentfill}{rgb}{0.121569,0.466667,0.705882}%
\pgfsetfillcolor{currentfill}%
\pgfsetlinewidth{0.000000pt}%
\definecolor{currentstroke}{rgb}{0.000000,0.000000,0.000000}%
\pgfsetstrokecolor{currentstroke}%
\pgfsetstrokeopacity{0.000000}%
\pgfsetdash{}{0pt}%
\pgfpathmoveto{\pgfqpoint{5.800000in}{146.954310in}}%
\pgfpathlineto{\pgfqpoint{5.800000in}{146.954310in}}%
\pgfpathlineto{\pgfqpoint{5.800000in}{147.057147in}}%
\pgfpathlineto{\pgfqpoint{5.800000in}{147.057147in}}%
\pgfpathclose%
\pgfusepath{fill}%
\end{pgfscope}%
\begin{pgfscope}%
\pgfpathrectangle{\pgfqpoint{5.800000in}{0.720000in}}{\pgfqpoint{1.400000in}{4.620000in}}%
\pgfusepath{clip}%
\pgfsetbuttcap%
\pgfsetmiterjoin%
\definecolor{currentfill}{rgb}{0.121569,0.466667,0.705882}%
\pgfsetfillcolor{currentfill}%
\pgfsetlinewidth{0.000000pt}%
\definecolor{currentstroke}{rgb}{0.000000,0.000000,0.000000}%
\pgfsetstrokecolor{currentstroke}%
\pgfsetstrokeopacity{0.000000}%
\pgfsetdash{}{0pt}%
\pgfpathmoveto{\pgfqpoint{5.800000in}{147.057147in}}%
\pgfpathlineto{\pgfqpoint{5.800000in}{147.057147in}}%
\pgfpathlineto{\pgfqpoint{5.800000in}{147.159985in}}%
\pgfpathlineto{\pgfqpoint{5.800000in}{147.159985in}}%
\pgfpathclose%
\pgfusepath{fill}%
\end{pgfscope}%
\begin{pgfscope}%
\pgfpathrectangle{\pgfqpoint{5.800000in}{0.720000in}}{\pgfqpoint{1.400000in}{4.620000in}}%
\pgfusepath{clip}%
\pgfsetbuttcap%
\pgfsetmiterjoin%
\definecolor{currentfill}{rgb}{0.121569,0.466667,0.705882}%
\pgfsetfillcolor{currentfill}%
\pgfsetlinewidth{0.000000pt}%
\definecolor{currentstroke}{rgb}{0.000000,0.000000,0.000000}%
\pgfsetstrokecolor{currentstroke}%
\pgfsetstrokeopacity{0.000000}%
\pgfsetdash{}{0pt}%
\pgfpathmoveto{\pgfqpoint{5.800000in}{147.159985in}}%
\pgfpathlineto{\pgfqpoint{5.800000in}{147.159985in}}%
\pgfpathlineto{\pgfqpoint{5.800000in}{147.262822in}}%
\pgfpathlineto{\pgfqpoint{5.800000in}{147.262822in}}%
\pgfpathclose%
\pgfusepath{fill}%
\end{pgfscope}%
\begin{pgfscope}%
\pgfpathrectangle{\pgfqpoint{5.800000in}{0.720000in}}{\pgfqpoint{1.400000in}{4.620000in}}%
\pgfusepath{clip}%
\pgfsetbuttcap%
\pgfsetmiterjoin%
\definecolor{currentfill}{rgb}{0.121569,0.466667,0.705882}%
\pgfsetfillcolor{currentfill}%
\pgfsetlinewidth{0.000000pt}%
\definecolor{currentstroke}{rgb}{0.000000,0.000000,0.000000}%
\pgfsetstrokecolor{currentstroke}%
\pgfsetstrokeopacity{0.000000}%
\pgfsetdash{}{0pt}%
\pgfpathmoveto{\pgfqpoint{5.800000in}{147.262822in}}%
\pgfpathlineto{\pgfqpoint{5.800000in}{147.262822in}}%
\pgfpathlineto{\pgfqpoint{5.800000in}{147.365659in}}%
\pgfpathlineto{\pgfqpoint{5.800000in}{147.365659in}}%
\pgfpathclose%
\pgfusepath{fill}%
\end{pgfscope}%
\begin{pgfscope}%
\pgfpathrectangle{\pgfqpoint{5.800000in}{0.720000in}}{\pgfqpoint{1.400000in}{4.620000in}}%
\pgfusepath{clip}%
\pgfsetbuttcap%
\pgfsetmiterjoin%
\definecolor{currentfill}{rgb}{0.121569,0.466667,0.705882}%
\pgfsetfillcolor{currentfill}%
\pgfsetlinewidth{0.000000pt}%
\definecolor{currentstroke}{rgb}{0.000000,0.000000,0.000000}%
\pgfsetstrokecolor{currentstroke}%
\pgfsetstrokeopacity{0.000000}%
\pgfsetdash{}{0pt}%
\pgfpathmoveto{\pgfqpoint{5.800000in}{147.365659in}}%
\pgfpathlineto{\pgfqpoint{5.800000in}{147.365659in}}%
\pgfpathlineto{\pgfqpoint{5.800000in}{147.468496in}}%
\pgfpathlineto{\pgfqpoint{5.800000in}{147.468496in}}%
\pgfpathclose%
\pgfusepath{fill}%
\end{pgfscope}%
\begin{pgfscope}%
\pgfpathrectangle{\pgfqpoint{5.800000in}{0.720000in}}{\pgfqpoint{1.400000in}{4.620000in}}%
\pgfusepath{clip}%
\pgfsetbuttcap%
\pgfsetmiterjoin%
\definecolor{currentfill}{rgb}{0.121569,0.466667,0.705882}%
\pgfsetfillcolor{currentfill}%
\pgfsetlinewidth{0.000000pt}%
\definecolor{currentstroke}{rgb}{0.000000,0.000000,0.000000}%
\pgfsetstrokecolor{currentstroke}%
\pgfsetstrokeopacity{0.000000}%
\pgfsetdash{}{0pt}%
\pgfpathmoveto{\pgfqpoint{5.800000in}{147.468496in}}%
\pgfpathlineto{\pgfqpoint{5.800000in}{147.468496in}}%
\pgfpathlineto{\pgfqpoint{5.800000in}{147.571333in}}%
\pgfpathlineto{\pgfqpoint{5.800000in}{147.571333in}}%
\pgfpathclose%
\pgfusepath{fill}%
\end{pgfscope}%
\begin{pgfscope}%
\pgfpathrectangle{\pgfqpoint{5.800000in}{0.720000in}}{\pgfqpoint{1.400000in}{4.620000in}}%
\pgfusepath{clip}%
\pgfsetbuttcap%
\pgfsetmiterjoin%
\definecolor{currentfill}{rgb}{0.121569,0.466667,0.705882}%
\pgfsetfillcolor{currentfill}%
\pgfsetlinewidth{0.000000pt}%
\definecolor{currentstroke}{rgb}{0.000000,0.000000,0.000000}%
\pgfsetstrokecolor{currentstroke}%
\pgfsetstrokeopacity{0.000000}%
\pgfsetdash{}{0pt}%
\pgfpathmoveto{\pgfqpoint{5.800000in}{147.571333in}}%
\pgfpathlineto{\pgfqpoint{5.800000in}{147.571333in}}%
\pgfpathlineto{\pgfqpoint{5.800000in}{147.674170in}}%
\pgfpathlineto{\pgfqpoint{5.800000in}{147.674170in}}%
\pgfpathclose%
\pgfusepath{fill}%
\end{pgfscope}%
\begin{pgfscope}%
\pgfpathrectangle{\pgfqpoint{5.800000in}{0.720000in}}{\pgfqpoint{1.400000in}{4.620000in}}%
\pgfusepath{clip}%
\pgfsetbuttcap%
\pgfsetmiterjoin%
\definecolor{currentfill}{rgb}{0.121569,0.466667,0.705882}%
\pgfsetfillcolor{currentfill}%
\pgfsetlinewidth{0.000000pt}%
\definecolor{currentstroke}{rgb}{0.000000,0.000000,0.000000}%
\pgfsetstrokecolor{currentstroke}%
\pgfsetstrokeopacity{0.000000}%
\pgfsetdash{}{0pt}%
\pgfpathmoveto{\pgfqpoint{5.800000in}{147.674170in}}%
\pgfpathlineto{\pgfqpoint{5.800000in}{147.674170in}}%
\pgfpathlineto{\pgfqpoint{5.800000in}{147.777007in}}%
\pgfpathlineto{\pgfqpoint{5.800000in}{147.777007in}}%
\pgfpathclose%
\pgfusepath{fill}%
\end{pgfscope}%
\begin{pgfscope}%
\pgfpathrectangle{\pgfqpoint{5.800000in}{0.720000in}}{\pgfqpoint{1.400000in}{4.620000in}}%
\pgfusepath{clip}%
\pgfsetbuttcap%
\pgfsetmiterjoin%
\definecolor{currentfill}{rgb}{0.121569,0.466667,0.705882}%
\pgfsetfillcolor{currentfill}%
\pgfsetlinewidth{0.000000pt}%
\definecolor{currentstroke}{rgb}{0.000000,0.000000,0.000000}%
\pgfsetstrokecolor{currentstroke}%
\pgfsetstrokeopacity{0.000000}%
\pgfsetdash{}{0pt}%
\pgfpathmoveto{\pgfqpoint{5.800000in}{147.777007in}}%
\pgfpathlineto{\pgfqpoint{5.800000in}{147.777007in}}%
\pgfpathlineto{\pgfqpoint{5.800000in}{147.879844in}}%
\pgfpathlineto{\pgfqpoint{5.800000in}{147.879844in}}%
\pgfpathclose%
\pgfusepath{fill}%
\end{pgfscope}%
\begin{pgfscope}%
\pgfpathrectangle{\pgfqpoint{5.800000in}{0.720000in}}{\pgfqpoint{1.400000in}{4.620000in}}%
\pgfusepath{clip}%
\pgfsetbuttcap%
\pgfsetmiterjoin%
\definecolor{currentfill}{rgb}{0.121569,0.466667,0.705882}%
\pgfsetfillcolor{currentfill}%
\pgfsetlinewidth{0.000000pt}%
\definecolor{currentstroke}{rgb}{0.000000,0.000000,0.000000}%
\pgfsetstrokecolor{currentstroke}%
\pgfsetstrokeopacity{0.000000}%
\pgfsetdash{}{0pt}%
\pgfpathmoveto{\pgfqpoint{5.800000in}{147.879844in}}%
\pgfpathlineto{\pgfqpoint{5.800000in}{147.879844in}}%
\pgfpathlineto{\pgfqpoint{5.800000in}{147.982681in}}%
\pgfpathlineto{\pgfqpoint{5.800000in}{147.982681in}}%
\pgfpathclose%
\pgfusepath{fill}%
\end{pgfscope}%
\begin{pgfscope}%
\pgfpathrectangle{\pgfqpoint{5.800000in}{0.720000in}}{\pgfqpoint{1.400000in}{4.620000in}}%
\pgfusepath{clip}%
\pgfsetbuttcap%
\pgfsetmiterjoin%
\definecolor{currentfill}{rgb}{0.121569,0.466667,0.705882}%
\pgfsetfillcolor{currentfill}%
\pgfsetlinewidth{0.000000pt}%
\definecolor{currentstroke}{rgb}{0.000000,0.000000,0.000000}%
\pgfsetstrokecolor{currentstroke}%
\pgfsetstrokeopacity{0.000000}%
\pgfsetdash{}{0pt}%
\pgfpathmoveto{\pgfqpoint{5.800000in}{147.982681in}}%
\pgfpathlineto{\pgfqpoint{5.800000in}{147.982681in}}%
\pgfpathlineto{\pgfqpoint{5.800000in}{148.085518in}}%
\pgfpathlineto{\pgfqpoint{5.800000in}{148.085518in}}%
\pgfpathclose%
\pgfusepath{fill}%
\end{pgfscope}%
\begin{pgfscope}%
\pgfpathrectangle{\pgfqpoint{5.800000in}{0.720000in}}{\pgfqpoint{1.400000in}{4.620000in}}%
\pgfusepath{clip}%
\pgfsetbuttcap%
\pgfsetmiterjoin%
\definecolor{currentfill}{rgb}{0.121569,0.466667,0.705882}%
\pgfsetfillcolor{currentfill}%
\pgfsetlinewidth{0.000000pt}%
\definecolor{currentstroke}{rgb}{0.000000,0.000000,0.000000}%
\pgfsetstrokecolor{currentstroke}%
\pgfsetstrokeopacity{0.000000}%
\pgfsetdash{}{0pt}%
\pgfpathmoveto{\pgfqpoint{5.800000in}{148.085518in}}%
\pgfpathlineto{\pgfqpoint{5.800000in}{148.085518in}}%
\pgfpathlineto{\pgfqpoint{5.800000in}{148.188355in}}%
\pgfpathlineto{\pgfqpoint{5.800000in}{148.188355in}}%
\pgfpathclose%
\pgfusepath{fill}%
\end{pgfscope}%
\begin{pgfscope}%
\pgfpathrectangle{\pgfqpoint{5.800000in}{0.720000in}}{\pgfqpoint{1.400000in}{4.620000in}}%
\pgfusepath{clip}%
\pgfsetbuttcap%
\pgfsetmiterjoin%
\definecolor{currentfill}{rgb}{0.121569,0.466667,0.705882}%
\pgfsetfillcolor{currentfill}%
\pgfsetlinewidth{0.000000pt}%
\definecolor{currentstroke}{rgb}{0.000000,0.000000,0.000000}%
\pgfsetstrokecolor{currentstroke}%
\pgfsetstrokeopacity{0.000000}%
\pgfsetdash{}{0pt}%
\pgfpathmoveto{\pgfqpoint{5.800000in}{148.188355in}}%
\pgfpathlineto{\pgfqpoint{5.800000in}{148.188355in}}%
\pgfpathlineto{\pgfqpoint{5.800000in}{148.291192in}}%
\pgfpathlineto{\pgfqpoint{5.800000in}{148.291192in}}%
\pgfpathclose%
\pgfusepath{fill}%
\end{pgfscope}%
\begin{pgfscope}%
\pgfpathrectangle{\pgfqpoint{5.800000in}{0.720000in}}{\pgfqpoint{1.400000in}{4.620000in}}%
\pgfusepath{clip}%
\pgfsetbuttcap%
\pgfsetmiterjoin%
\definecolor{currentfill}{rgb}{0.121569,0.466667,0.705882}%
\pgfsetfillcolor{currentfill}%
\pgfsetlinewidth{0.000000pt}%
\definecolor{currentstroke}{rgb}{0.000000,0.000000,0.000000}%
\pgfsetstrokecolor{currentstroke}%
\pgfsetstrokeopacity{0.000000}%
\pgfsetdash{}{0pt}%
\pgfpathmoveto{\pgfqpoint{5.800000in}{148.291192in}}%
\pgfpathlineto{\pgfqpoint{5.800000in}{148.291192in}}%
\pgfpathlineto{\pgfqpoint{5.800000in}{148.394029in}}%
\pgfpathlineto{\pgfqpoint{5.800000in}{148.394029in}}%
\pgfpathclose%
\pgfusepath{fill}%
\end{pgfscope}%
\begin{pgfscope}%
\pgfpathrectangle{\pgfqpoint{5.800000in}{0.720000in}}{\pgfqpoint{1.400000in}{4.620000in}}%
\pgfusepath{clip}%
\pgfsetbuttcap%
\pgfsetmiterjoin%
\definecolor{currentfill}{rgb}{0.121569,0.466667,0.705882}%
\pgfsetfillcolor{currentfill}%
\pgfsetlinewidth{0.000000pt}%
\definecolor{currentstroke}{rgb}{0.000000,0.000000,0.000000}%
\pgfsetstrokecolor{currentstroke}%
\pgfsetstrokeopacity{0.000000}%
\pgfsetdash{}{0pt}%
\pgfpathmoveto{\pgfqpoint{5.800000in}{148.394029in}}%
\pgfpathlineto{\pgfqpoint{5.800000in}{148.394029in}}%
\pgfpathlineto{\pgfqpoint{5.800000in}{148.496866in}}%
\pgfpathlineto{\pgfqpoint{5.800000in}{148.496866in}}%
\pgfpathclose%
\pgfusepath{fill}%
\end{pgfscope}%
\begin{pgfscope}%
\pgfpathrectangle{\pgfqpoint{5.800000in}{0.720000in}}{\pgfqpoint{1.400000in}{4.620000in}}%
\pgfusepath{clip}%
\pgfsetbuttcap%
\pgfsetmiterjoin%
\definecolor{currentfill}{rgb}{0.121569,0.466667,0.705882}%
\pgfsetfillcolor{currentfill}%
\pgfsetlinewidth{0.000000pt}%
\definecolor{currentstroke}{rgb}{0.000000,0.000000,0.000000}%
\pgfsetstrokecolor{currentstroke}%
\pgfsetstrokeopacity{0.000000}%
\pgfsetdash{}{0pt}%
\pgfpathmoveto{\pgfqpoint{5.800000in}{148.496866in}}%
\pgfpathlineto{\pgfqpoint{5.800000in}{148.496866in}}%
\pgfpathlineto{\pgfqpoint{5.800000in}{148.599703in}}%
\pgfpathlineto{\pgfqpoint{5.800000in}{148.599703in}}%
\pgfpathclose%
\pgfusepath{fill}%
\end{pgfscope}%
\begin{pgfscope}%
\pgfpathrectangle{\pgfqpoint{5.800000in}{0.720000in}}{\pgfqpoint{1.400000in}{4.620000in}}%
\pgfusepath{clip}%
\pgfsetbuttcap%
\pgfsetmiterjoin%
\definecolor{currentfill}{rgb}{0.121569,0.466667,0.705882}%
\pgfsetfillcolor{currentfill}%
\pgfsetlinewidth{0.000000pt}%
\definecolor{currentstroke}{rgb}{0.000000,0.000000,0.000000}%
\pgfsetstrokecolor{currentstroke}%
\pgfsetstrokeopacity{0.000000}%
\pgfsetdash{}{0pt}%
\pgfpathmoveto{\pgfqpoint{5.800000in}{148.599703in}}%
\pgfpathlineto{\pgfqpoint{5.800000in}{148.599703in}}%
\pgfpathlineto{\pgfqpoint{5.800000in}{148.702541in}}%
\pgfpathlineto{\pgfqpoint{5.800000in}{148.702541in}}%
\pgfpathclose%
\pgfusepath{fill}%
\end{pgfscope}%
\begin{pgfscope}%
\pgfpathrectangle{\pgfqpoint{5.800000in}{0.720000in}}{\pgfqpoint{1.400000in}{4.620000in}}%
\pgfusepath{clip}%
\pgfsetbuttcap%
\pgfsetmiterjoin%
\definecolor{currentfill}{rgb}{0.121569,0.466667,0.705882}%
\pgfsetfillcolor{currentfill}%
\pgfsetlinewidth{0.000000pt}%
\definecolor{currentstroke}{rgb}{0.000000,0.000000,0.000000}%
\pgfsetstrokecolor{currentstroke}%
\pgfsetstrokeopacity{0.000000}%
\pgfsetdash{}{0pt}%
\pgfpathmoveto{\pgfqpoint{5.800000in}{148.702541in}}%
\pgfpathlineto{\pgfqpoint{5.800000in}{148.702541in}}%
\pgfpathlineto{\pgfqpoint{5.800000in}{148.805378in}}%
\pgfpathlineto{\pgfqpoint{5.800000in}{148.805378in}}%
\pgfpathclose%
\pgfusepath{fill}%
\end{pgfscope}%
\begin{pgfscope}%
\pgfpathrectangle{\pgfqpoint{5.800000in}{0.720000in}}{\pgfqpoint{1.400000in}{4.620000in}}%
\pgfusepath{clip}%
\pgfsetbuttcap%
\pgfsetmiterjoin%
\definecolor{currentfill}{rgb}{0.121569,0.466667,0.705882}%
\pgfsetfillcolor{currentfill}%
\pgfsetlinewidth{0.000000pt}%
\definecolor{currentstroke}{rgb}{0.000000,0.000000,0.000000}%
\pgfsetstrokecolor{currentstroke}%
\pgfsetstrokeopacity{0.000000}%
\pgfsetdash{}{0pt}%
\pgfpathmoveto{\pgfqpoint{5.800000in}{148.805378in}}%
\pgfpathlineto{\pgfqpoint{5.800000in}{148.805378in}}%
\pgfpathlineto{\pgfqpoint{5.800000in}{148.908215in}}%
\pgfpathlineto{\pgfqpoint{5.800000in}{148.908215in}}%
\pgfpathclose%
\pgfusepath{fill}%
\end{pgfscope}%
\begin{pgfscope}%
\pgfpathrectangle{\pgfqpoint{5.800000in}{0.720000in}}{\pgfqpoint{1.400000in}{4.620000in}}%
\pgfusepath{clip}%
\pgfsetbuttcap%
\pgfsetmiterjoin%
\definecolor{currentfill}{rgb}{0.121569,0.466667,0.705882}%
\pgfsetfillcolor{currentfill}%
\pgfsetlinewidth{0.000000pt}%
\definecolor{currentstroke}{rgb}{0.000000,0.000000,0.000000}%
\pgfsetstrokecolor{currentstroke}%
\pgfsetstrokeopacity{0.000000}%
\pgfsetdash{}{0pt}%
\pgfpathmoveto{\pgfqpoint{5.800000in}{148.908215in}}%
\pgfpathlineto{\pgfqpoint{5.800000in}{148.908215in}}%
\pgfpathlineto{\pgfqpoint{5.800000in}{149.011052in}}%
\pgfpathlineto{\pgfqpoint{5.800000in}{149.011052in}}%
\pgfpathclose%
\pgfusepath{fill}%
\end{pgfscope}%
\begin{pgfscope}%
\pgfpathrectangle{\pgfqpoint{5.800000in}{0.720000in}}{\pgfqpoint{1.400000in}{4.620000in}}%
\pgfusepath{clip}%
\pgfsetbuttcap%
\pgfsetmiterjoin%
\definecolor{currentfill}{rgb}{0.121569,0.466667,0.705882}%
\pgfsetfillcolor{currentfill}%
\pgfsetlinewidth{0.000000pt}%
\definecolor{currentstroke}{rgb}{0.000000,0.000000,0.000000}%
\pgfsetstrokecolor{currentstroke}%
\pgfsetstrokeopacity{0.000000}%
\pgfsetdash{}{0pt}%
\pgfpathmoveto{\pgfqpoint{5.800000in}{149.011052in}}%
\pgfpathlineto{\pgfqpoint{5.800000in}{149.011052in}}%
\pgfpathlineto{\pgfqpoint{5.800000in}{149.113889in}}%
\pgfpathlineto{\pgfqpoint{5.800000in}{149.113889in}}%
\pgfpathclose%
\pgfusepath{fill}%
\end{pgfscope}%
\begin{pgfscope}%
\pgfpathrectangle{\pgfqpoint{5.800000in}{0.720000in}}{\pgfqpoint{1.400000in}{4.620000in}}%
\pgfusepath{clip}%
\pgfsetbuttcap%
\pgfsetmiterjoin%
\definecolor{currentfill}{rgb}{0.121569,0.466667,0.705882}%
\pgfsetfillcolor{currentfill}%
\pgfsetlinewidth{0.000000pt}%
\definecolor{currentstroke}{rgb}{0.000000,0.000000,0.000000}%
\pgfsetstrokecolor{currentstroke}%
\pgfsetstrokeopacity{0.000000}%
\pgfsetdash{}{0pt}%
\pgfpathmoveto{\pgfqpoint{5.800000in}{149.113889in}}%
\pgfpathlineto{\pgfqpoint{5.800000in}{149.113889in}}%
\pgfpathlineto{\pgfqpoint{5.800000in}{149.216726in}}%
\pgfpathlineto{\pgfqpoint{5.800000in}{149.216726in}}%
\pgfpathclose%
\pgfusepath{fill}%
\end{pgfscope}%
\begin{pgfscope}%
\pgfpathrectangle{\pgfqpoint{5.800000in}{0.720000in}}{\pgfqpoint{1.400000in}{4.620000in}}%
\pgfusepath{clip}%
\pgfsetbuttcap%
\pgfsetmiterjoin%
\definecolor{currentfill}{rgb}{0.121569,0.466667,0.705882}%
\pgfsetfillcolor{currentfill}%
\pgfsetlinewidth{0.000000pt}%
\definecolor{currentstroke}{rgb}{0.000000,0.000000,0.000000}%
\pgfsetstrokecolor{currentstroke}%
\pgfsetstrokeopacity{0.000000}%
\pgfsetdash{}{0pt}%
\pgfpathmoveto{\pgfqpoint{5.800000in}{149.216726in}}%
\pgfpathlineto{\pgfqpoint{5.800000in}{149.216726in}}%
\pgfpathlineto{\pgfqpoint{5.800000in}{149.319563in}}%
\pgfpathlineto{\pgfqpoint{5.800000in}{149.319563in}}%
\pgfpathclose%
\pgfusepath{fill}%
\end{pgfscope}%
\begin{pgfscope}%
\pgfpathrectangle{\pgfqpoint{5.800000in}{0.720000in}}{\pgfqpoint{1.400000in}{4.620000in}}%
\pgfusepath{clip}%
\pgfsetbuttcap%
\pgfsetmiterjoin%
\definecolor{currentfill}{rgb}{0.121569,0.466667,0.705882}%
\pgfsetfillcolor{currentfill}%
\pgfsetlinewidth{0.000000pt}%
\definecolor{currentstroke}{rgb}{0.000000,0.000000,0.000000}%
\pgfsetstrokecolor{currentstroke}%
\pgfsetstrokeopacity{0.000000}%
\pgfsetdash{}{0pt}%
\pgfpathmoveto{\pgfqpoint{5.800000in}{149.319563in}}%
\pgfpathlineto{\pgfqpoint{5.800000in}{149.319563in}}%
\pgfpathlineto{\pgfqpoint{5.800000in}{149.422400in}}%
\pgfpathlineto{\pgfqpoint{5.800000in}{149.422400in}}%
\pgfpathclose%
\pgfusepath{fill}%
\end{pgfscope}%
\begin{pgfscope}%
\pgfpathrectangle{\pgfqpoint{5.800000in}{0.720000in}}{\pgfqpoint{1.400000in}{4.620000in}}%
\pgfusepath{clip}%
\pgfsetbuttcap%
\pgfsetmiterjoin%
\definecolor{currentfill}{rgb}{0.121569,0.466667,0.705882}%
\pgfsetfillcolor{currentfill}%
\pgfsetlinewidth{0.000000pt}%
\definecolor{currentstroke}{rgb}{0.000000,0.000000,0.000000}%
\pgfsetstrokecolor{currentstroke}%
\pgfsetstrokeopacity{0.000000}%
\pgfsetdash{}{0pt}%
\pgfpathmoveto{\pgfqpoint{5.800000in}{149.422400in}}%
\pgfpathlineto{\pgfqpoint{5.800000in}{149.422400in}}%
\pgfpathlineto{\pgfqpoint{5.800000in}{149.525237in}}%
\pgfpathlineto{\pgfqpoint{5.800000in}{149.525237in}}%
\pgfpathclose%
\pgfusepath{fill}%
\end{pgfscope}%
\begin{pgfscope}%
\pgfpathrectangle{\pgfqpoint{5.800000in}{0.720000in}}{\pgfqpoint{1.400000in}{4.620000in}}%
\pgfusepath{clip}%
\pgfsetbuttcap%
\pgfsetmiterjoin%
\definecolor{currentfill}{rgb}{0.121569,0.466667,0.705882}%
\pgfsetfillcolor{currentfill}%
\pgfsetlinewidth{0.000000pt}%
\definecolor{currentstroke}{rgb}{0.000000,0.000000,0.000000}%
\pgfsetstrokecolor{currentstroke}%
\pgfsetstrokeopacity{0.000000}%
\pgfsetdash{}{0pt}%
\pgfpathmoveto{\pgfqpoint{5.800000in}{149.525237in}}%
\pgfpathlineto{\pgfqpoint{5.800000in}{149.525237in}}%
\pgfpathlineto{\pgfqpoint{5.800000in}{149.628074in}}%
\pgfpathlineto{\pgfqpoint{5.800000in}{149.628074in}}%
\pgfpathclose%
\pgfusepath{fill}%
\end{pgfscope}%
\begin{pgfscope}%
\pgfpathrectangle{\pgfqpoint{5.800000in}{0.720000in}}{\pgfqpoint{1.400000in}{4.620000in}}%
\pgfusepath{clip}%
\pgfsetbuttcap%
\pgfsetmiterjoin%
\definecolor{currentfill}{rgb}{0.121569,0.466667,0.705882}%
\pgfsetfillcolor{currentfill}%
\pgfsetlinewidth{0.000000pt}%
\definecolor{currentstroke}{rgb}{0.000000,0.000000,0.000000}%
\pgfsetstrokecolor{currentstroke}%
\pgfsetstrokeopacity{0.000000}%
\pgfsetdash{}{0pt}%
\pgfpathmoveto{\pgfqpoint{5.800000in}{149.628074in}}%
\pgfpathlineto{\pgfqpoint{5.800000in}{149.628074in}}%
\pgfpathlineto{\pgfqpoint{5.800000in}{149.730911in}}%
\pgfpathlineto{\pgfqpoint{5.800000in}{149.730911in}}%
\pgfpathclose%
\pgfusepath{fill}%
\end{pgfscope}%
\begin{pgfscope}%
\pgfpathrectangle{\pgfqpoint{5.800000in}{0.720000in}}{\pgfqpoint{1.400000in}{4.620000in}}%
\pgfusepath{clip}%
\pgfsetbuttcap%
\pgfsetmiterjoin%
\definecolor{currentfill}{rgb}{0.121569,0.466667,0.705882}%
\pgfsetfillcolor{currentfill}%
\pgfsetlinewidth{0.000000pt}%
\definecolor{currentstroke}{rgb}{0.000000,0.000000,0.000000}%
\pgfsetstrokecolor{currentstroke}%
\pgfsetstrokeopacity{0.000000}%
\pgfsetdash{}{0pt}%
\pgfpathmoveto{\pgfqpoint{5.800000in}{149.730911in}}%
\pgfpathlineto{\pgfqpoint{5.800000in}{149.730911in}}%
\pgfpathlineto{\pgfqpoint{5.800000in}{149.833748in}}%
\pgfpathlineto{\pgfqpoint{5.800000in}{149.833748in}}%
\pgfpathclose%
\pgfusepath{fill}%
\end{pgfscope}%
\begin{pgfscope}%
\pgfpathrectangle{\pgfqpoint{5.800000in}{0.720000in}}{\pgfqpoint{1.400000in}{4.620000in}}%
\pgfusepath{clip}%
\pgfsetbuttcap%
\pgfsetmiterjoin%
\definecolor{currentfill}{rgb}{0.121569,0.466667,0.705882}%
\pgfsetfillcolor{currentfill}%
\pgfsetlinewidth{0.000000pt}%
\definecolor{currentstroke}{rgb}{0.000000,0.000000,0.000000}%
\pgfsetstrokecolor{currentstroke}%
\pgfsetstrokeopacity{0.000000}%
\pgfsetdash{}{0pt}%
\pgfpathmoveto{\pgfqpoint{5.800000in}{149.833748in}}%
\pgfpathlineto{\pgfqpoint{5.800000in}{149.833748in}}%
\pgfpathlineto{\pgfqpoint{5.800000in}{149.936585in}}%
\pgfpathlineto{\pgfqpoint{5.800000in}{149.936585in}}%
\pgfpathclose%
\pgfusepath{fill}%
\end{pgfscope}%
\begin{pgfscope}%
\pgfpathrectangle{\pgfqpoint{5.800000in}{0.720000in}}{\pgfqpoint{1.400000in}{4.620000in}}%
\pgfusepath{clip}%
\pgfsetbuttcap%
\pgfsetmiterjoin%
\definecolor{currentfill}{rgb}{0.121569,0.466667,0.705882}%
\pgfsetfillcolor{currentfill}%
\pgfsetlinewidth{0.000000pt}%
\definecolor{currentstroke}{rgb}{0.000000,0.000000,0.000000}%
\pgfsetstrokecolor{currentstroke}%
\pgfsetstrokeopacity{0.000000}%
\pgfsetdash{}{0pt}%
\pgfpathmoveto{\pgfqpoint{5.800000in}{149.936585in}}%
\pgfpathlineto{\pgfqpoint{5.800000in}{149.936585in}}%
\pgfpathlineto{\pgfqpoint{5.800000in}{150.039422in}}%
\pgfpathlineto{\pgfqpoint{5.800000in}{150.039422in}}%
\pgfpathclose%
\pgfusepath{fill}%
\end{pgfscope}%
\begin{pgfscope}%
\pgfpathrectangle{\pgfqpoint{5.800000in}{0.720000in}}{\pgfqpoint{1.400000in}{4.620000in}}%
\pgfusepath{clip}%
\pgfsetbuttcap%
\pgfsetmiterjoin%
\definecolor{currentfill}{rgb}{0.121569,0.466667,0.705882}%
\pgfsetfillcolor{currentfill}%
\pgfsetlinewidth{0.000000pt}%
\definecolor{currentstroke}{rgb}{0.000000,0.000000,0.000000}%
\pgfsetstrokecolor{currentstroke}%
\pgfsetstrokeopacity{0.000000}%
\pgfsetdash{}{0pt}%
\pgfpathmoveto{\pgfqpoint{5.800000in}{150.039422in}}%
\pgfpathlineto{\pgfqpoint{5.800000in}{150.039422in}}%
\pgfpathlineto{\pgfqpoint{5.800000in}{150.142259in}}%
\pgfpathlineto{\pgfqpoint{5.800000in}{150.142259in}}%
\pgfpathclose%
\pgfusepath{fill}%
\end{pgfscope}%
\begin{pgfscope}%
\pgfpathrectangle{\pgfqpoint{5.800000in}{0.720000in}}{\pgfqpoint{1.400000in}{4.620000in}}%
\pgfusepath{clip}%
\pgfsetbuttcap%
\pgfsetmiterjoin%
\definecolor{currentfill}{rgb}{0.121569,0.466667,0.705882}%
\pgfsetfillcolor{currentfill}%
\pgfsetlinewidth{0.000000pt}%
\definecolor{currentstroke}{rgb}{0.000000,0.000000,0.000000}%
\pgfsetstrokecolor{currentstroke}%
\pgfsetstrokeopacity{0.000000}%
\pgfsetdash{}{0pt}%
\pgfpathmoveto{\pgfqpoint{5.800000in}{150.142259in}}%
\pgfpathlineto{\pgfqpoint{5.800000in}{150.142259in}}%
\pgfpathlineto{\pgfqpoint{5.800000in}{150.245097in}}%
\pgfpathlineto{\pgfqpoint{5.800000in}{150.245097in}}%
\pgfpathclose%
\pgfusepath{fill}%
\end{pgfscope}%
\begin{pgfscope}%
\pgfpathrectangle{\pgfqpoint{5.800000in}{0.720000in}}{\pgfqpoint{1.400000in}{4.620000in}}%
\pgfusepath{clip}%
\pgfsetbuttcap%
\pgfsetmiterjoin%
\definecolor{currentfill}{rgb}{0.121569,0.466667,0.705882}%
\pgfsetfillcolor{currentfill}%
\pgfsetlinewidth{0.000000pt}%
\definecolor{currentstroke}{rgb}{0.000000,0.000000,0.000000}%
\pgfsetstrokecolor{currentstroke}%
\pgfsetstrokeopacity{0.000000}%
\pgfsetdash{}{0pt}%
\pgfpathmoveto{\pgfqpoint{5.800000in}{150.245097in}}%
\pgfpathlineto{\pgfqpoint{5.800000in}{150.245097in}}%
\pgfpathlineto{\pgfqpoint{5.800000in}{150.347934in}}%
\pgfpathlineto{\pgfqpoint{5.800000in}{150.347934in}}%
\pgfpathclose%
\pgfusepath{fill}%
\end{pgfscope}%
\begin{pgfscope}%
\pgfpathrectangle{\pgfqpoint{5.800000in}{0.720000in}}{\pgfqpoint{1.400000in}{4.620000in}}%
\pgfusepath{clip}%
\pgfsetbuttcap%
\pgfsetmiterjoin%
\definecolor{currentfill}{rgb}{0.121569,0.466667,0.705882}%
\pgfsetfillcolor{currentfill}%
\pgfsetlinewidth{0.000000pt}%
\definecolor{currentstroke}{rgb}{0.000000,0.000000,0.000000}%
\pgfsetstrokecolor{currentstroke}%
\pgfsetstrokeopacity{0.000000}%
\pgfsetdash{}{0pt}%
\pgfpathmoveto{\pgfqpoint{5.800000in}{150.347934in}}%
\pgfpathlineto{\pgfqpoint{5.800000in}{150.347934in}}%
\pgfpathlineto{\pgfqpoint{5.800000in}{150.450771in}}%
\pgfpathlineto{\pgfqpoint{5.800000in}{150.450771in}}%
\pgfpathclose%
\pgfusepath{fill}%
\end{pgfscope}%
\begin{pgfscope}%
\pgfpathrectangle{\pgfqpoint{5.800000in}{0.720000in}}{\pgfqpoint{1.400000in}{4.620000in}}%
\pgfusepath{clip}%
\pgfsetbuttcap%
\pgfsetmiterjoin%
\definecolor{currentfill}{rgb}{0.121569,0.466667,0.705882}%
\pgfsetfillcolor{currentfill}%
\pgfsetlinewidth{0.000000pt}%
\definecolor{currentstroke}{rgb}{0.000000,0.000000,0.000000}%
\pgfsetstrokecolor{currentstroke}%
\pgfsetstrokeopacity{0.000000}%
\pgfsetdash{}{0pt}%
\pgfpathmoveto{\pgfqpoint{5.800000in}{150.450771in}}%
\pgfpathlineto{\pgfqpoint{5.800000in}{150.450771in}}%
\pgfpathlineto{\pgfqpoint{5.800000in}{150.553608in}}%
\pgfpathlineto{\pgfqpoint{5.800000in}{150.553608in}}%
\pgfpathclose%
\pgfusepath{fill}%
\end{pgfscope}%
\begin{pgfscope}%
\pgfpathrectangle{\pgfqpoint{5.800000in}{0.720000in}}{\pgfqpoint{1.400000in}{4.620000in}}%
\pgfusepath{clip}%
\pgfsetbuttcap%
\pgfsetmiterjoin%
\definecolor{currentfill}{rgb}{0.121569,0.466667,0.705882}%
\pgfsetfillcolor{currentfill}%
\pgfsetlinewidth{0.000000pt}%
\definecolor{currentstroke}{rgb}{0.000000,0.000000,0.000000}%
\pgfsetstrokecolor{currentstroke}%
\pgfsetstrokeopacity{0.000000}%
\pgfsetdash{}{0pt}%
\pgfpathmoveto{\pgfqpoint{5.800000in}{150.553608in}}%
\pgfpathlineto{\pgfqpoint{5.800000in}{150.553608in}}%
\pgfpathlineto{\pgfqpoint{5.800000in}{150.656445in}}%
\pgfpathlineto{\pgfqpoint{5.800000in}{150.656445in}}%
\pgfpathclose%
\pgfusepath{fill}%
\end{pgfscope}%
\begin{pgfscope}%
\pgfpathrectangle{\pgfqpoint{5.800000in}{0.720000in}}{\pgfqpoint{1.400000in}{4.620000in}}%
\pgfusepath{clip}%
\pgfsetbuttcap%
\pgfsetmiterjoin%
\definecolor{currentfill}{rgb}{0.121569,0.466667,0.705882}%
\pgfsetfillcolor{currentfill}%
\pgfsetlinewidth{0.000000pt}%
\definecolor{currentstroke}{rgb}{0.000000,0.000000,0.000000}%
\pgfsetstrokecolor{currentstroke}%
\pgfsetstrokeopacity{0.000000}%
\pgfsetdash{}{0pt}%
\pgfpathmoveto{\pgfqpoint{5.800000in}{150.656445in}}%
\pgfpathlineto{\pgfqpoint{5.800000in}{150.656445in}}%
\pgfpathlineto{\pgfqpoint{5.800000in}{150.759282in}}%
\pgfpathlineto{\pgfqpoint{5.800000in}{150.759282in}}%
\pgfpathclose%
\pgfusepath{fill}%
\end{pgfscope}%
\begin{pgfscope}%
\pgfpathrectangle{\pgfqpoint{5.800000in}{0.720000in}}{\pgfqpoint{1.400000in}{4.620000in}}%
\pgfusepath{clip}%
\pgfsetbuttcap%
\pgfsetmiterjoin%
\definecolor{currentfill}{rgb}{0.121569,0.466667,0.705882}%
\pgfsetfillcolor{currentfill}%
\pgfsetlinewidth{0.000000pt}%
\definecolor{currentstroke}{rgb}{0.000000,0.000000,0.000000}%
\pgfsetstrokecolor{currentstroke}%
\pgfsetstrokeopacity{0.000000}%
\pgfsetdash{}{0pt}%
\pgfpathmoveto{\pgfqpoint{5.800000in}{150.759282in}}%
\pgfpathlineto{\pgfqpoint{5.800000in}{150.759282in}}%
\pgfpathlineto{\pgfqpoint{5.800000in}{150.862119in}}%
\pgfpathlineto{\pgfqpoint{5.800000in}{150.862119in}}%
\pgfpathclose%
\pgfusepath{fill}%
\end{pgfscope}%
\begin{pgfscope}%
\pgfpathrectangle{\pgfqpoint{5.800000in}{0.720000in}}{\pgfqpoint{1.400000in}{4.620000in}}%
\pgfusepath{clip}%
\pgfsetbuttcap%
\pgfsetmiterjoin%
\definecolor{currentfill}{rgb}{0.121569,0.466667,0.705882}%
\pgfsetfillcolor{currentfill}%
\pgfsetlinewidth{0.000000pt}%
\definecolor{currentstroke}{rgb}{0.000000,0.000000,0.000000}%
\pgfsetstrokecolor{currentstroke}%
\pgfsetstrokeopacity{0.000000}%
\pgfsetdash{}{0pt}%
\pgfpathmoveto{\pgfqpoint{5.800000in}{150.862119in}}%
\pgfpathlineto{\pgfqpoint{5.800000in}{150.862119in}}%
\pgfpathlineto{\pgfqpoint{5.800000in}{150.964956in}}%
\pgfpathlineto{\pgfqpoint{5.800000in}{150.964956in}}%
\pgfpathclose%
\pgfusepath{fill}%
\end{pgfscope}%
\begin{pgfscope}%
\pgfpathrectangle{\pgfqpoint{5.800000in}{0.720000in}}{\pgfqpoint{1.400000in}{4.620000in}}%
\pgfusepath{clip}%
\pgfsetbuttcap%
\pgfsetmiterjoin%
\definecolor{currentfill}{rgb}{0.121569,0.466667,0.705882}%
\pgfsetfillcolor{currentfill}%
\pgfsetlinewidth{0.000000pt}%
\definecolor{currentstroke}{rgb}{0.000000,0.000000,0.000000}%
\pgfsetstrokecolor{currentstroke}%
\pgfsetstrokeopacity{0.000000}%
\pgfsetdash{}{0pt}%
\pgfpathmoveto{\pgfqpoint{5.800000in}{150.964956in}}%
\pgfpathlineto{\pgfqpoint{5.800000in}{150.964956in}}%
\pgfpathlineto{\pgfqpoint{5.800000in}{151.067793in}}%
\pgfpathlineto{\pgfqpoint{5.800000in}{151.067793in}}%
\pgfpathclose%
\pgfusepath{fill}%
\end{pgfscope}%
\begin{pgfscope}%
\pgfpathrectangle{\pgfqpoint{5.800000in}{0.720000in}}{\pgfqpoint{1.400000in}{4.620000in}}%
\pgfusepath{clip}%
\pgfsetbuttcap%
\pgfsetmiterjoin%
\definecolor{currentfill}{rgb}{0.121569,0.466667,0.705882}%
\pgfsetfillcolor{currentfill}%
\pgfsetlinewidth{0.000000pt}%
\definecolor{currentstroke}{rgb}{0.000000,0.000000,0.000000}%
\pgfsetstrokecolor{currentstroke}%
\pgfsetstrokeopacity{0.000000}%
\pgfsetdash{}{0pt}%
\pgfpathmoveto{\pgfqpoint{5.800000in}{151.067793in}}%
\pgfpathlineto{\pgfqpoint{5.800000in}{151.067793in}}%
\pgfpathlineto{\pgfqpoint{5.800000in}{151.170630in}}%
\pgfpathlineto{\pgfqpoint{5.800000in}{151.170630in}}%
\pgfpathclose%
\pgfusepath{fill}%
\end{pgfscope}%
\begin{pgfscope}%
\pgfpathrectangle{\pgfqpoint{5.800000in}{0.720000in}}{\pgfqpoint{1.400000in}{4.620000in}}%
\pgfusepath{clip}%
\pgfsetbuttcap%
\pgfsetmiterjoin%
\definecolor{currentfill}{rgb}{0.121569,0.466667,0.705882}%
\pgfsetfillcolor{currentfill}%
\pgfsetlinewidth{0.000000pt}%
\definecolor{currentstroke}{rgb}{0.000000,0.000000,0.000000}%
\pgfsetstrokecolor{currentstroke}%
\pgfsetstrokeopacity{0.000000}%
\pgfsetdash{}{0pt}%
\pgfpathmoveto{\pgfqpoint{5.800000in}{151.170630in}}%
\pgfpathlineto{\pgfqpoint{5.800000in}{151.170630in}}%
\pgfpathlineto{\pgfqpoint{5.800000in}{151.273467in}}%
\pgfpathlineto{\pgfqpoint{5.800000in}{151.273467in}}%
\pgfpathclose%
\pgfusepath{fill}%
\end{pgfscope}%
\begin{pgfscope}%
\pgfpathrectangle{\pgfqpoint{5.800000in}{0.720000in}}{\pgfqpoint{1.400000in}{4.620000in}}%
\pgfusepath{clip}%
\pgfsetbuttcap%
\pgfsetmiterjoin%
\definecolor{currentfill}{rgb}{0.121569,0.466667,0.705882}%
\pgfsetfillcolor{currentfill}%
\pgfsetlinewidth{0.000000pt}%
\definecolor{currentstroke}{rgb}{0.000000,0.000000,0.000000}%
\pgfsetstrokecolor{currentstroke}%
\pgfsetstrokeopacity{0.000000}%
\pgfsetdash{}{0pt}%
\pgfpathmoveto{\pgfqpoint{5.800000in}{151.273467in}}%
\pgfpathlineto{\pgfqpoint{5.800000in}{151.273467in}}%
\pgfpathlineto{\pgfqpoint{5.800000in}{151.376304in}}%
\pgfpathlineto{\pgfqpoint{5.800000in}{151.376304in}}%
\pgfpathclose%
\pgfusepath{fill}%
\end{pgfscope}%
\begin{pgfscope}%
\pgfpathrectangle{\pgfqpoint{5.800000in}{0.720000in}}{\pgfqpoint{1.400000in}{4.620000in}}%
\pgfusepath{clip}%
\pgfsetbuttcap%
\pgfsetmiterjoin%
\definecolor{currentfill}{rgb}{0.121569,0.466667,0.705882}%
\pgfsetfillcolor{currentfill}%
\pgfsetlinewidth{0.000000pt}%
\definecolor{currentstroke}{rgb}{0.000000,0.000000,0.000000}%
\pgfsetstrokecolor{currentstroke}%
\pgfsetstrokeopacity{0.000000}%
\pgfsetdash{}{0pt}%
\pgfpathmoveto{\pgfqpoint{5.800000in}{151.376304in}}%
\pgfpathlineto{\pgfqpoint{5.800000in}{151.376304in}}%
\pgfpathlineto{\pgfqpoint{5.800000in}{151.479141in}}%
\pgfpathlineto{\pgfqpoint{5.800000in}{151.479141in}}%
\pgfpathclose%
\pgfusepath{fill}%
\end{pgfscope}%
\begin{pgfscope}%
\pgfpathrectangle{\pgfqpoint{5.800000in}{0.720000in}}{\pgfqpoint{1.400000in}{4.620000in}}%
\pgfusepath{clip}%
\pgfsetbuttcap%
\pgfsetmiterjoin%
\definecolor{currentfill}{rgb}{0.121569,0.466667,0.705882}%
\pgfsetfillcolor{currentfill}%
\pgfsetlinewidth{0.000000pt}%
\definecolor{currentstroke}{rgb}{0.000000,0.000000,0.000000}%
\pgfsetstrokecolor{currentstroke}%
\pgfsetstrokeopacity{0.000000}%
\pgfsetdash{}{0pt}%
\pgfpathmoveto{\pgfqpoint{5.800000in}{151.479141in}}%
\pgfpathlineto{\pgfqpoint{5.800000in}{151.479141in}}%
\pgfpathlineto{\pgfqpoint{5.800000in}{151.581978in}}%
\pgfpathlineto{\pgfqpoint{5.800000in}{151.581978in}}%
\pgfpathclose%
\pgfusepath{fill}%
\end{pgfscope}%
\begin{pgfscope}%
\pgfpathrectangle{\pgfqpoint{5.800000in}{0.720000in}}{\pgfqpoint{1.400000in}{4.620000in}}%
\pgfusepath{clip}%
\pgfsetbuttcap%
\pgfsetmiterjoin%
\definecolor{currentfill}{rgb}{0.121569,0.466667,0.705882}%
\pgfsetfillcolor{currentfill}%
\pgfsetlinewidth{0.000000pt}%
\definecolor{currentstroke}{rgb}{0.000000,0.000000,0.000000}%
\pgfsetstrokecolor{currentstroke}%
\pgfsetstrokeopacity{0.000000}%
\pgfsetdash{}{0pt}%
\pgfpathmoveto{\pgfqpoint{5.800000in}{151.581978in}}%
\pgfpathlineto{\pgfqpoint{5.800000in}{151.581978in}}%
\pgfpathlineto{\pgfqpoint{5.800000in}{151.684815in}}%
\pgfpathlineto{\pgfqpoint{5.800000in}{151.684815in}}%
\pgfpathclose%
\pgfusepath{fill}%
\end{pgfscope}%
\begin{pgfscope}%
\pgfpathrectangle{\pgfqpoint{5.800000in}{0.720000in}}{\pgfqpoint{1.400000in}{4.620000in}}%
\pgfusepath{clip}%
\pgfsetbuttcap%
\pgfsetmiterjoin%
\definecolor{currentfill}{rgb}{0.121569,0.466667,0.705882}%
\pgfsetfillcolor{currentfill}%
\pgfsetlinewidth{0.000000pt}%
\definecolor{currentstroke}{rgb}{0.000000,0.000000,0.000000}%
\pgfsetstrokecolor{currentstroke}%
\pgfsetstrokeopacity{0.000000}%
\pgfsetdash{}{0pt}%
\pgfpathmoveto{\pgfqpoint{5.800000in}{151.684815in}}%
\pgfpathlineto{\pgfqpoint{5.800000in}{151.684815in}}%
\pgfpathlineto{\pgfqpoint{5.800000in}{151.787653in}}%
\pgfpathlineto{\pgfqpoint{5.800000in}{151.787653in}}%
\pgfpathclose%
\pgfusepath{fill}%
\end{pgfscope}%
\begin{pgfscope}%
\pgfpathrectangle{\pgfqpoint{5.800000in}{0.720000in}}{\pgfqpoint{1.400000in}{4.620000in}}%
\pgfusepath{clip}%
\pgfsetbuttcap%
\pgfsetmiterjoin%
\definecolor{currentfill}{rgb}{0.121569,0.466667,0.705882}%
\pgfsetfillcolor{currentfill}%
\pgfsetlinewidth{0.000000pt}%
\definecolor{currentstroke}{rgb}{0.000000,0.000000,0.000000}%
\pgfsetstrokecolor{currentstroke}%
\pgfsetstrokeopacity{0.000000}%
\pgfsetdash{}{0pt}%
\pgfpathmoveto{\pgfqpoint{5.800000in}{151.787653in}}%
\pgfpathlineto{\pgfqpoint{5.800000in}{151.787653in}}%
\pgfpathlineto{\pgfqpoint{5.800000in}{151.890490in}}%
\pgfpathlineto{\pgfqpoint{5.800000in}{151.890490in}}%
\pgfpathclose%
\pgfusepath{fill}%
\end{pgfscope}%
\begin{pgfscope}%
\pgfpathrectangle{\pgfqpoint{5.800000in}{0.720000in}}{\pgfqpoint{1.400000in}{4.620000in}}%
\pgfusepath{clip}%
\pgfsetbuttcap%
\pgfsetmiterjoin%
\definecolor{currentfill}{rgb}{0.121569,0.466667,0.705882}%
\pgfsetfillcolor{currentfill}%
\pgfsetlinewidth{0.000000pt}%
\definecolor{currentstroke}{rgb}{0.000000,0.000000,0.000000}%
\pgfsetstrokecolor{currentstroke}%
\pgfsetstrokeopacity{0.000000}%
\pgfsetdash{}{0pt}%
\pgfpathmoveto{\pgfqpoint{5.800000in}{151.890490in}}%
\pgfpathlineto{\pgfqpoint{5.800000in}{151.890490in}}%
\pgfpathlineto{\pgfqpoint{5.800000in}{151.993327in}}%
\pgfpathlineto{\pgfqpoint{5.800000in}{151.993327in}}%
\pgfpathclose%
\pgfusepath{fill}%
\end{pgfscope}%
\begin{pgfscope}%
\pgfpathrectangle{\pgfqpoint{5.800000in}{0.720000in}}{\pgfqpoint{1.400000in}{4.620000in}}%
\pgfusepath{clip}%
\pgfsetbuttcap%
\pgfsetmiterjoin%
\definecolor{currentfill}{rgb}{0.121569,0.466667,0.705882}%
\pgfsetfillcolor{currentfill}%
\pgfsetlinewidth{0.000000pt}%
\definecolor{currentstroke}{rgb}{0.000000,0.000000,0.000000}%
\pgfsetstrokecolor{currentstroke}%
\pgfsetstrokeopacity{0.000000}%
\pgfsetdash{}{0pt}%
\pgfpathmoveto{\pgfqpoint{5.800000in}{151.993327in}}%
\pgfpathlineto{\pgfqpoint{5.800000in}{151.993327in}}%
\pgfpathlineto{\pgfqpoint{5.800000in}{152.096164in}}%
\pgfpathlineto{\pgfqpoint{5.800000in}{152.096164in}}%
\pgfpathclose%
\pgfusepath{fill}%
\end{pgfscope}%
\begin{pgfscope}%
\pgfpathrectangle{\pgfqpoint{5.800000in}{0.720000in}}{\pgfqpoint{1.400000in}{4.620000in}}%
\pgfusepath{clip}%
\pgfsetbuttcap%
\pgfsetmiterjoin%
\definecolor{currentfill}{rgb}{0.121569,0.466667,0.705882}%
\pgfsetfillcolor{currentfill}%
\pgfsetlinewidth{0.000000pt}%
\definecolor{currentstroke}{rgb}{0.000000,0.000000,0.000000}%
\pgfsetstrokecolor{currentstroke}%
\pgfsetstrokeopacity{0.000000}%
\pgfsetdash{}{0pt}%
\pgfpathmoveto{\pgfqpoint{5.800000in}{152.096164in}}%
\pgfpathlineto{\pgfqpoint{5.800000in}{152.096164in}}%
\pgfpathlineto{\pgfqpoint{5.800000in}{152.199001in}}%
\pgfpathlineto{\pgfqpoint{5.800000in}{152.199001in}}%
\pgfpathclose%
\pgfusepath{fill}%
\end{pgfscope}%
\begin{pgfscope}%
\pgfpathrectangle{\pgfqpoint{5.800000in}{0.720000in}}{\pgfqpoint{1.400000in}{4.620000in}}%
\pgfusepath{clip}%
\pgfsetbuttcap%
\pgfsetmiterjoin%
\definecolor{currentfill}{rgb}{0.121569,0.466667,0.705882}%
\pgfsetfillcolor{currentfill}%
\pgfsetlinewidth{0.000000pt}%
\definecolor{currentstroke}{rgb}{0.000000,0.000000,0.000000}%
\pgfsetstrokecolor{currentstroke}%
\pgfsetstrokeopacity{0.000000}%
\pgfsetdash{}{0pt}%
\pgfpathmoveto{\pgfqpoint{5.800000in}{152.199001in}}%
\pgfpathlineto{\pgfqpoint{5.800000in}{152.199001in}}%
\pgfpathlineto{\pgfqpoint{5.800000in}{152.301838in}}%
\pgfpathlineto{\pgfqpoint{5.800000in}{152.301838in}}%
\pgfpathclose%
\pgfusepath{fill}%
\end{pgfscope}%
\begin{pgfscope}%
\pgfpathrectangle{\pgfqpoint{5.800000in}{0.720000in}}{\pgfqpoint{1.400000in}{4.620000in}}%
\pgfusepath{clip}%
\pgfsetbuttcap%
\pgfsetmiterjoin%
\definecolor{currentfill}{rgb}{0.121569,0.466667,0.705882}%
\pgfsetfillcolor{currentfill}%
\pgfsetlinewidth{0.000000pt}%
\definecolor{currentstroke}{rgb}{0.000000,0.000000,0.000000}%
\pgfsetstrokecolor{currentstroke}%
\pgfsetstrokeopacity{0.000000}%
\pgfsetdash{}{0pt}%
\pgfpathmoveto{\pgfqpoint{5.800000in}{152.301838in}}%
\pgfpathlineto{\pgfqpoint{5.800000in}{152.301838in}}%
\pgfpathlineto{\pgfqpoint{5.800000in}{152.404675in}}%
\pgfpathlineto{\pgfqpoint{5.800000in}{152.404675in}}%
\pgfpathclose%
\pgfusepath{fill}%
\end{pgfscope}%
\begin{pgfscope}%
\pgfpathrectangle{\pgfqpoint{5.800000in}{0.720000in}}{\pgfqpoint{1.400000in}{4.620000in}}%
\pgfusepath{clip}%
\pgfsetbuttcap%
\pgfsetmiterjoin%
\definecolor{currentfill}{rgb}{0.121569,0.466667,0.705882}%
\pgfsetfillcolor{currentfill}%
\pgfsetlinewidth{0.000000pt}%
\definecolor{currentstroke}{rgb}{0.000000,0.000000,0.000000}%
\pgfsetstrokecolor{currentstroke}%
\pgfsetstrokeopacity{0.000000}%
\pgfsetdash{}{0pt}%
\pgfpathmoveto{\pgfqpoint{5.800000in}{152.404675in}}%
\pgfpathlineto{\pgfqpoint{5.800000in}{152.404675in}}%
\pgfpathlineto{\pgfqpoint{5.800000in}{152.507512in}}%
\pgfpathlineto{\pgfqpoint{5.800000in}{152.507512in}}%
\pgfpathclose%
\pgfusepath{fill}%
\end{pgfscope}%
\begin{pgfscope}%
\pgfpathrectangle{\pgfqpoint{5.800000in}{0.720000in}}{\pgfqpoint{1.400000in}{4.620000in}}%
\pgfusepath{clip}%
\pgfsetbuttcap%
\pgfsetmiterjoin%
\definecolor{currentfill}{rgb}{0.121569,0.466667,0.705882}%
\pgfsetfillcolor{currentfill}%
\pgfsetlinewidth{0.000000pt}%
\definecolor{currentstroke}{rgb}{0.000000,0.000000,0.000000}%
\pgfsetstrokecolor{currentstroke}%
\pgfsetstrokeopacity{0.000000}%
\pgfsetdash{}{0pt}%
\pgfpathmoveto{\pgfqpoint{5.800000in}{152.507512in}}%
\pgfpathlineto{\pgfqpoint{5.800000in}{152.507512in}}%
\pgfpathlineto{\pgfqpoint{5.800000in}{152.610349in}}%
\pgfpathlineto{\pgfqpoint{5.800000in}{152.610349in}}%
\pgfpathclose%
\pgfusepath{fill}%
\end{pgfscope}%
\begin{pgfscope}%
\pgfpathrectangle{\pgfqpoint{5.800000in}{0.720000in}}{\pgfqpoint{1.400000in}{4.620000in}}%
\pgfusepath{clip}%
\pgfsetbuttcap%
\pgfsetmiterjoin%
\definecolor{currentfill}{rgb}{0.121569,0.466667,0.705882}%
\pgfsetfillcolor{currentfill}%
\pgfsetlinewidth{0.000000pt}%
\definecolor{currentstroke}{rgb}{0.000000,0.000000,0.000000}%
\pgfsetstrokecolor{currentstroke}%
\pgfsetstrokeopacity{0.000000}%
\pgfsetdash{}{0pt}%
\pgfpathmoveto{\pgfqpoint{5.800000in}{152.610349in}}%
\pgfpathlineto{\pgfqpoint{5.800000in}{152.610349in}}%
\pgfpathlineto{\pgfqpoint{5.800000in}{152.713186in}}%
\pgfpathlineto{\pgfqpoint{5.800000in}{152.713186in}}%
\pgfpathclose%
\pgfusepath{fill}%
\end{pgfscope}%
\begin{pgfscope}%
\pgfpathrectangle{\pgfqpoint{5.800000in}{0.720000in}}{\pgfqpoint{1.400000in}{4.620000in}}%
\pgfusepath{clip}%
\pgfsetbuttcap%
\pgfsetmiterjoin%
\definecolor{currentfill}{rgb}{0.121569,0.466667,0.705882}%
\pgfsetfillcolor{currentfill}%
\pgfsetlinewidth{0.000000pt}%
\definecolor{currentstroke}{rgb}{0.000000,0.000000,0.000000}%
\pgfsetstrokecolor{currentstroke}%
\pgfsetstrokeopacity{0.000000}%
\pgfsetdash{}{0pt}%
\pgfpathmoveto{\pgfqpoint{5.800000in}{152.713186in}}%
\pgfpathlineto{\pgfqpoint{5.800000in}{152.713186in}}%
\pgfpathlineto{\pgfqpoint{5.800000in}{152.816023in}}%
\pgfpathlineto{\pgfqpoint{5.800000in}{152.816023in}}%
\pgfpathclose%
\pgfusepath{fill}%
\end{pgfscope}%
\begin{pgfscope}%
\pgfpathrectangle{\pgfqpoint{5.800000in}{0.720000in}}{\pgfqpoint{1.400000in}{4.620000in}}%
\pgfusepath{clip}%
\pgfsetbuttcap%
\pgfsetmiterjoin%
\definecolor{currentfill}{rgb}{0.121569,0.466667,0.705882}%
\pgfsetfillcolor{currentfill}%
\pgfsetlinewidth{0.000000pt}%
\definecolor{currentstroke}{rgb}{0.000000,0.000000,0.000000}%
\pgfsetstrokecolor{currentstroke}%
\pgfsetstrokeopacity{0.000000}%
\pgfsetdash{}{0pt}%
\pgfpathmoveto{\pgfqpoint{5.800000in}{152.816023in}}%
\pgfpathlineto{\pgfqpoint{5.800000in}{152.816023in}}%
\pgfpathlineto{\pgfqpoint{5.800000in}{152.918860in}}%
\pgfpathlineto{\pgfqpoint{5.800000in}{152.918860in}}%
\pgfpathclose%
\pgfusepath{fill}%
\end{pgfscope}%
\begin{pgfscope}%
\pgfpathrectangle{\pgfqpoint{5.800000in}{0.720000in}}{\pgfqpoint{1.400000in}{4.620000in}}%
\pgfusepath{clip}%
\pgfsetbuttcap%
\pgfsetmiterjoin%
\definecolor{currentfill}{rgb}{0.121569,0.466667,0.705882}%
\pgfsetfillcolor{currentfill}%
\pgfsetlinewidth{0.000000pt}%
\definecolor{currentstroke}{rgb}{0.000000,0.000000,0.000000}%
\pgfsetstrokecolor{currentstroke}%
\pgfsetstrokeopacity{0.000000}%
\pgfsetdash{}{0pt}%
\pgfpathmoveto{\pgfqpoint{5.800000in}{152.918860in}}%
\pgfpathlineto{\pgfqpoint{5.800000in}{152.918860in}}%
\pgfpathlineto{\pgfqpoint{5.800000in}{153.021697in}}%
\pgfpathlineto{\pgfqpoint{5.800000in}{153.021697in}}%
\pgfpathclose%
\pgfusepath{fill}%
\end{pgfscope}%
\begin{pgfscope}%
\pgfpathrectangle{\pgfqpoint{5.800000in}{0.720000in}}{\pgfqpoint{1.400000in}{4.620000in}}%
\pgfusepath{clip}%
\pgfsetbuttcap%
\pgfsetmiterjoin%
\definecolor{currentfill}{rgb}{0.121569,0.466667,0.705882}%
\pgfsetfillcolor{currentfill}%
\pgfsetlinewidth{0.000000pt}%
\definecolor{currentstroke}{rgb}{0.000000,0.000000,0.000000}%
\pgfsetstrokecolor{currentstroke}%
\pgfsetstrokeopacity{0.000000}%
\pgfsetdash{}{0pt}%
\pgfpathmoveto{\pgfqpoint{5.800000in}{153.021697in}}%
\pgfpathlineto{\pgfqpoint{5.800000in}{153.021697in}}%
\pgfpathlineto{\pgfqpoint{5.800000in}{153.124534in}}%
\pgfpathlineto{\pgfqpoint{5.800000in}{153.124534in}}%
\pgfpathclose%
\pgfusepath{fill}%
\end{pgfscope}%
\begin{pgfscope}%
\pgfpathrectangle{\pgfqpoint{5.800000in}{0.720000in}}{\pgfqpoint{1.400000in}{4.620000in}}%
\pgfusepath{clip}%
\pgfsetbuttcap%
\pgfsetmiterjoin%
\definecolor{currentfill}{rgb}{0.121569,0.466667,0.705882}%
\pgfsetfillcolor{currentfill}%
\pgfsetlinewidth{0.000000pt}%
\definecolor{currentstroke}{rgb}{0.000000,0.000000,0.000000}%
\pgfsetstrokecolor{currentstroke}%
\pgfsetstrokeopacity{0.000000}%
\pgfsetdash{}{0pt}%
\pgfpathmoveto{\pgfqpoint{5.800000in}{153.124534in}}%
\pgfpathlineto{\pgfqpoint{5.800000in}{153.124534in}}%
\pgfpathlineto{\pgfqpoint{5.800000in}{153.227372in}}%
\pgfpathlineto{\pgfqpoint{5.800000in}{153.227372in}}%
\pgfpathclose%
\pgfusepath{fill}%
\end{pgfscope}%
\begin{pgfscope}%
\pgfpathrectangle{\pgfqpoint{5.800000in}{0.720000in}}{\pgfqpoint{1.400000in}{4.620000in}}%
\pgfusepath{clip}%
\pgfsetbuttcap%
\pgfsetmiterjoin%
\definecolor{currentfill}{rgb}{0.121569,0.466667,0.705882}%
\pgfsetfillcolor{currentfill}%
\pgfsetlinewidth{0.000000pt}%
\definecolor{currentstroke}{rgb}{0.000000,0.000000,0.000000}%
\pgfsetstrokecolor{currentstroke}%
\pgfsetstrokeopacity{0.000000}%
\pgfsetdash{}{0pt}%
\pgfpathmoveto{\pgfqpoint{5.800000in}{153.227372in}}%
\pgfpathlineto{\pgfqpoint{5.800000in}{153.227372in}}%
\pgfpathlineto{\pgfqpoint{5.800000in}{153.330209in}}%
\pgfpathlineto{\pgfqpoint{5.800000in}{153.330209in}}%
\pgfpathclose%
\pgfusepath{fill}%
\end{pgfscope}%
\begin{pgfscope}%
\pgfpathrectangle{\pgfqpoint{5.800000in}{0.720000in}}{\pgfqpoint{1.400000in}{4.620000in}}%
\pgfusepath{clip}%
\pgfsetbuttcap%
\pgfsetmiterjoin%
\definecolor{currentfill}{rgb}{0.121569,0.466667,0.705882}%
\pgfsetfillcolor{currentfill}%
\pgfsetlinewidth{0.000000pt}%
\definecolor{currentstroke}{rgb}{0.000000,0.000000,0.000000}%
\pgfsetstrokecolor{currentstroke}%
\pgfsetstrokeopacity{0.000000}%
\pgfsetdash{}{0pt}%
\pgfpathmoveto{\pgfqpoint{5.800000in}{153.330209in}}%
\pgfpathlineto{\pgfqpoint{5.800000in}{153.330209in}}%
\pgfpathlineto{\pgfqpoint{5.800000in}{153.433046in}}%
\pgfpathlineto{\pgfqpoint{5.800000in}{153.433046in}}%
\pgfpathclose%
\pgfusepath{fill}%
\end{pgfscope}%
\begin{pgfscope}%
\pgfpathrectangle{\pgfqpoint{5.800000in}{0.720000in}}{\pgfqpoint{1.400000in}{4.620000in}}%
\pgfusepath{clip}%
\pgfsetbuttcap%
\pgfsetmiterjoin%
\definecolor{currentfill}{rgb}{0.121569,0.466667,0.705882}%
\pgfsetfillcolor{currentfill}%
\pgfsetlinewidth{0.000000pt}%
\definecolor{currentstroke}{rgb}{0.000000,0.000000,0.000000}%
\pgfsetstrokecolor{currentstroke}%
\pgfsetstrokeopacity{0.000000}%
\pgfsetdash{}{0pt}%
\pgfpathmoveto{\pgfqpoint{5.800000in}{153.433046in}}%
\pgfpathlineto{\pgfqpoint{5.800000in}{153.433046in}}%
\pgfpathlineto{\pgfqpoint{5.800000in}{153.535883in}}%
\pgfpathlineto{\pgfqpoint{5.800000in}{153.535883in}}%
\pgfpathclose%
\pgfusepath{fill}%
\end{pgfscope}%
\begin{pgfscope}%
\pgfpathrectangle{\pgfqpoint{5.800000in}{0.720000in}}{\pgfqpoint{1.400000in}{4.620000in}}%
\pgfusepath{clip}%
\pgfsetbuttcap%
\pgfsetmiterjoin%
\definecolor{currentfill}{rgb}{0.121569,0.466667,0.705882}%
\pgfsetfillcolor{currentfill}%
\pgfsetlinewidth{0.000000pt}%
\definecolor{currentstroke}{rgb}{0.000000,0.000000,0.000000}%
\pgfsetstrokecolor{currentstroke}%
\pgfsetstrokeopacity{0.000000}%
\pgfsetdash{}{0pt}%
\pgfpathmoveto{\pgfqpoint{5.800000in}{153.535883in}}%
\pgfpathlineto{\pgfqpoint{5.800000in}{153.535883in}}%
\pgfpathlineto{\pgfqpoint{5.800000in}{153.638720in}}%
\pgfpathlineto{\pgfqpoint{5.800000in}{153.638720in}}%
\pgfpathclose%
\pgfusepath{fill}%
\end{pgfscope}%
\begin{pgfscope}%
\pgfpathrectangle{\pgfqpoint{5.800000in}{0.720000in}}{\pgfqpoint{1.400000in}{4.620000in}}%
\pgfusepath{clip}%
\pgfsetbuttcap%
\pgfsetmiterjoin%
\definecolor{currentfill}{rgb}{0.121569,0.466667,0.705882}%
\pgfsetfillcolor{currentfill}%
\pgfsetlinewidth{0.000000pt}%
\definecolor{currentstroke}{rgb}{0.000000,0.000000,0.000000}%
\pgfsetstrokecolor{currentstroke}%
\pgfsetstrokeopacity{0.000000}%
\pgfsetdash{}{0pt}%
\pgfpathmoveto{\pgfqpoint{5.800000in}{153.638720in}}%
\pgfpathlineto{\pgfqpoint{5.800000in}{153.638720in}}%
\pgfpathlineto{\pgfqpoint{5.800000in}{153.741557in}}%
\pgfpathlineto{\pgfqpoint{5.800000in}{153.741557in}}%
\pgfpathclose%
\pgfusepath{fill}%
\end{pgfscope}%
\begin{pgfscope}%
\pgfpathrectangle{\pgfqpoint{5.800000in}{0.720000in}}{\pgfqpoint{1.400000in}{4.620000in}}%
\pgfusepath{clip}%
\pgfsetbuttcap%
\pgfsetmiterjoin%
\definecolor{currentfill}{rgb}{0.121569,0.466667,0.705882}%
\pgfsetfillcolor{currentfill}%
\pgfsetlinewidth{0.000000pt}%
\definecolor{currentstroke}{rgb}{0.000000,0.000000,0.000000}%
\pgfsetstrokecolor{currentstroke}%
\pgfsetstrokeopacity{0.000000}%
\pgfsetdash{}{0pt}%
\pgfpathmoveto{\pgfqpoint{5.800000in}{153.741557in}}%
\pgfpathlineto{\pgfqpoint{5.800000in}{153.741557in}}%
\pgfpathlineto{\pgfqpoint{5.800000in}{153.844394in}}%
\pgfpathlineto{\pgfqpoint{5.800000in}{153.844394in}}%
\pgfpathclose%
\pgfusepath{fill}%
\end{pgfscope}%
\begin{pgfscope}%
\pgfpathrectangle{\pgfqpoint{5.800000in}{0.720000in}}{\pgfqpoint{1.400000in}{4.620000in}}%
\pgfusepath{clip}%
\pgfsetbuttcap%
\pgfsetmiterjoin%
\definecolor{currentfill}{rgb}{0.121569,0.466667,0.705882}%
\pgfsetfillcolor{currentfill}%
\pgfsetlinewidth{0.000000pt}%
\definecolor{currentstroke}{rgb}{0.000000,0.000000,0.000000}%
\pgfsetstrokecolor{currentstroke}%
\pgfsetstrokeopacity{0.000000}%
\pgfsetdash{}{0pt}%
\pgfpathmoveto{\pgfqpoint{5.800000in}{153.844394in}}%
\pgfpathlineto{\pgfqpoint{5.800000in}{153.844394in}}%
\pgfpathlineto{\pgfqpoint{5.800000in}{153.947231in}}%
\pgfpathlineto{\pgfqpoint{5.800000in}{153.947231in}}%
\pgfpathclose%
\pgfusepath{fill}%
\end{pgfscope}%
\begin{pgfscope}%
\pgfpathrectangle{\pgfqpoint{5.800000in}{0.720000in}}{\pgfqpoint{1.400000in}{4.620000in}}%
\pgfusepath{clip}%
\pgfsetbuttcap%
\pgfsetmiterjoin%
\definecolor{currentfill}{rgb}{0.121569,0.466667,0.705882}%
\pgfsetfillcolor{currentfill}%
\pgfsetlinewidth{0.000000pt}%
\definecolor{currentstroke}{rgb}{0.000000,0.000000,0.000000}%
\pgfsetstrokecolor{currentstroke}%
\pgfsetstrokeopacity{0.000000}%
\pgfsetdash{}{0pt}%
\pgfpathmoveto{\pgfqpoint{5.800000in}{153.947231in}}%
\pgfpathlineto{\pgfqpoint{5.800000in}{153.947231in}}%
\pgfpathlineto{\pgfqpoint{5.800000in}{154.050068in}}%
\pgfpathlineto{\pgfqpoint{5.800000in}{154.050068in}}%
\pgfpathclose%
\pgfusepath{fill}%
\end{pgfscope}%
\begin{pgfscope}%
\pgfpathrectangle{\pgfqpoint{5.800000in}{0.720000in}}{\pgfqpoint{1.400000in}{4.620000in}}%
\pgfusepath{clip}%
\pgfsetbuttcap%
\pgfsetmiterjoin%
\definecolor{currentfill}{rgb}{0.121569,0.466667,0.705882}%
\pgfsetfillcolor{currentfill}%
\pgfsetlinewidth{0.000000pt}%
\definecolor{currentstroke}{rgb}{0.000000,0.000000,0.000000}%
\pgfsetstrokecolor{currentstroke}%
\pgfsetstrokeopacity{0.000000}%
\pgfsetdash{}{0pt}%
\pgfpathmoveto{\pgfqpoint{5.800000in}{154.050068in}}%
\pgfpathlineto{\pgfqpoint{5.800000in}{154.050068in}}%
\pgfpathlineto{\pgfqpoint{5.800000in}{154.152905in}}%
\pgfpathlineto{\pgfqpoint{5.800000in}{154.152905in}}%
\pgfpathclose%
\pgfusepath{fill}%
\end{pgfscope}%
\begin{pgfscope}%
\pgfpathrectangle{\pgfqpoint{5.800000in}{0.720000in}}{\pgfqpoint{1.400000in}{4.620000in}}%
\pgfusepath{clip}%
\pgfsetbuttcap%
\pgfsetmiterjoin%
\definecolor{currentfill}{rgb}{0.121569,0.466667,0.705882}%
\pgfsetfillcolor{currentfill}%
\pgfsetlinewidth{0.000000pt}%
\definecolor{currentstroke}{rgb}{0.000000,0.000000,0.000000}%
\pgfsetstrokecolor{currentstroke}%
\pgfsetstrokeopacity{0.000000}%
\pgfsetdash{}{0pt}%
\pgfpathmoveto{\pgfqpoint{5.800000in}{154.152905in}}%
\pgfpathlineto{\pgfqpoint{5.800000in}{154.152905in}}%
\pgfpathlineto{\pgfqpoint{5.800000in}{154.255742in}}%
\pgfpathlineto{\pgfqpoint{5.800000in}{154.255742in}}%
\pgfpathclose%
\pgfusepath{fill}%
\end{pgfscope}%
\begin{pgfscope}%
\pgfpathrectangle{\pgfqpoint{5.800000in}{0.720000in}}{\pgfqpoint{1.400000in}{4.620000in}}%
\pgfusepath{clip}%
\pgfsetbuttcap%
\pgfsetmiterjoin%
\definecolor{currentfill}{rgb}{0.121569,0.466667,0.705882}%
\pgfsetfillcolor{currentfill}%
\pgfsetlinewidth{0.000000pt}%
\definecolor{currentstroke}{rgb}{0.000000,0.000000,0.000000}%
\pgfsetstrokecolor{currentstroke}%
\pgfsetstrokeopacity{0.000000}%
\pgfsetdash{}{0pt}%
\pgfpathmoveto{\pgfqpoint{5.800000in}{154.255742in}}%
\pgfpathlineto{\pgfqpoint{5.800000in}{154.255742in}}%
\pgfpathlineto{\pgfqpoint{5.800000in}{154.358579in}}%
\pgfpathlineto{\pgfqpoint{5.800000in}{154.358579in}}%
\pgfpathclose%
\pgfusepath{fill}%
\end{pgfscope}%
\begin{pgfscope}%
\pgfpathrectangle{\pgfqpoint{5.800000in}{0.720000in}}{\pgfqpoint{1.400000in}{4.620000in}}%
\pgfusepath{clip}%
\pgfsetbuttcap%
\pgfsetmiterjoin%
\definecolor{currentfill}{rgb}{0.121569,0.466667,0.705882}%
\pgfsetfillcolor{currentfill}%
\pgfsetlinewidth{0.000000pt}%
\definecolor{currentstroke}{rgb}{0.000000,0.000000,0.000000}%
\pgfsetstrokecolor{currentstroke}%
\pgfsetstrokeopacity{0.000000}%
\pgfsetdash{}{0pt}%
\pgfpathmoveto{\pgfqpoint{5.800000in}{154.358579in}}%
\pgfpathlineto{\pgfqpoint{5.800000in}{154.358579in}}%
\pgfpathlineto{\pgfqpoint{5.800000in}{154.461416in}}%
\pgfpathlineto{\pgfqpoint{5.800000in}{154.461416in}}%
\pgfpathclose%
\pgfusepath{fill}%
\end{pgfscope}%
\begin{pgfscope}%
\pgfpathrectangle{\pgfqpoint{5.800000in}{0.720000in}}{\pgfqpoint{1.400000in}{4.620000in}}%
\pgfusepath{clip}%
\pgfsetbuttcap%
\pgfsetmiterjoin%
\definecolor{currentfill}{rgb}{0.121569,0.466667,0.705882}%
\pgfsetfillcolor{currentfill}%
\pgfsetlinewidth{0.000000pt}%
\definecolor{currentstroke}{rgb}{0.000000,0.000000,0.000000}%
\pgfsetstrokecolor{currentstroke}%
\pgfsetstrokeopacity{0.000000}%
\pgfsetdash{}{0pt}%
\pgfpathmoveto{\pgfqpoint{5.800000in}{154.461416in}}%
\pgfpathlineto{\pgfqpoint{5.800000in}{154.461416in}}%
\pgfpathlineto{\pgfqpoint{5.800000in}{154.564253in}}%
\pgfpathlineto{\pgfqpoint{5.800000in}{154.564253in}}%
\pgfpathclose%
\pgfusepath{fill}%
\end{pgfscope}%
\begin{pgfscope}%
\pgfpathrectangle{\pgfqpoint{5.800000in}{0.720000in}}{\pgfqpoint{1.400000in}{4.620000in}}%
\pgfusepath{clip}%
\pgfsetbuttcap%
\pgfsetmiterjoin%
\definecolor{currentfill}{rgb}{0.121569,0.466667,0.705882}%
\pgfsetfillcolor{currentfill}%
\pgfsetlinewidth{0.000000pt}%
\definecolor{currentstroke}{rgb}{0.000000,0.000000,0.000000}%
\pgfsetstrokecolor{currentstroke}%
\pgfsetstrokeopacity{0.000000}%
\pgfsetdash{}{0pt}%
\pgfpathmoveto{\pgfqpoint{5.800000in}{154.564253in}}%
\pgfpathlineto{\pgfqpoint{5.800000in}{154.564253in}}%
\pgfpathlineto{\pgfqpoint{5.800000in}{154.667090in}}%
\pgfpathlineto{\pgfqpoint{5.800000in}{154.667090in}}%
\pgfpathclose%
\pgfusepath{fill}%
\end{pgfscope}%
\begin{pgfscope}%
\pgfpathrectangle{\pgfqpoint{5.800000in}{0.720000in}}{\pgfqpoint{1.400000in}{4.620000in}}%
\pgfusepath{clip}%
\pgfsetbuttcap%
\pgfsetmiterjoin%
\definecolor{currentfill}{rgb}{0.121569,0.466667,0.705882}%
\pgfsetfillcolor{currentfill}%
\pgfsetlinewidth{0.000000pt}%
\definecolor{currentstroke}{rgb}{0.000000,0.000000,0.000000}%
\pgfsetstrokecolor{currentstroke}%
\pgfsetstrokeopacity{0.000000}%
\pgfsetdash{}{0pt}%
\pgfpathmoveto{\pgfqpoint{5.800000in}{154.667090in}}%
\pgfpathlineto{\pgfqpoint{5.800000in}{154.667090in}}%
\pgfpathlineto{\pgfqpoint{5.800000in}{154.769928in}}%
\pgfpathlineto{\pgfqpoint{5.800000in}{154.769928in}}%
\pgfpathclose%
\pgfusepath{fill}%
\end{pgfscope}%
\begin{pgfscope}%
\pgfpathrectangle{\pgfqpoint{5.800000in}{0.720000in}}{\pgfqpoint{1.400000in}{4.620000in}}%
\pgfusepath{clip}%
\pgfsetbuttcap%
\pgfsetmiterjoin%
\definecolor{currentfill}{rgb}{0.121569,0.466667,0.705882}%
\pgfsetfillcolor{currentfill}%
\pgfsetlinewidth{0.000000pt}%
\definecolor{currentstroke}{rgb}{0.000000,0.000000,0.000000}%
\pgfsetstrokecolor{currentstroke}%
\pgfsetstrokeopacity{0.000000}%
\pgfsetdash{}{0pt}%
\pgfpathmoveto{\pgfqpoint{5.800000in}{154.769928in}}%
\pgfpathlineto{\pgfqpoint{5.800000in}{154.769928in}}%
\pgfpathlineto{\pgfqpoint{5.800000in}{154.872765in}}%
\pgfpathlineto{\pgfqpoint{5.800000in}{154.872765in}}%
\pgfpathclose%
\pgfusepath{fill}%
\end{pgfscope}%
\begin{pgfscope}%
\pgfpathrectangle{\pgfqpoint{5.800000in}{0.720000in}}{\pgfqpoint{1.400000in}{4.620000in}}%
\pgfusepath{clip}%
\pgfsetbuttcap%
\pgfsetmiterjoin%
\definecolor{currentfill}{rgb}{0.121569,0.466667,0.705882}%
\pgfsetfillcolor{currentfill}%
\pgfsetlinewidth{0.000000pt}%
\definecolor{currentstroke}{rgb}{0.000000,0.000000,0.000000}%
\pgfsetstrokecolor{currentstroke}%
\pgfsetstrokeopacity{0.000000}%
\pgfsetdash{}{0pt}%
\pgfpathmoveto{\pgfqpoint{5.800000in}{154.872765in}}%
\pgfpathlineto{\pgfqpoint{5.800000in}{154.872765in}}%
\pgfpathlineto{\pgfqpoint{5.800000in}{154.975602in}}%
\pgfpathlineto{\pgfqpoint{5.800000in}{154.975602in}}%
\pgfpathclose%
\pgfusepath{fill}%
\end{pgfscope}%
\begin{pgfscope}%
\pgfpathrectangle{\pgfqpoint{5.800000in}{0.720000in}}{\pgfqpoint{1.400000in}{4.620000in}}%
\pgfusepath{clip}%
\pgfsetbuttcap%
\pgfsetmiterjoin%
\definecolor{currentfill}{rgb}{0.121569,0.466667,0.705882}%
\pgfsetfillcolor{currentfill}%
\pgfsetlinewidth{0.000000pt}%
\definecolor{currentstroke}{rgb}{0.000000,0.000000,0.000000}%
\pgfsetstrokecolor{currentstroke}%
\pgfsetstrokeopacity{0.000000}%
\pgfsetdash{}{0pt}%
\pgfpathmoveto{\pgfqpoint{5.800000in}{154.975602in}}%
\pgfpathlineto{\pgfqpoint{5.800000in}{154.975602in}}%
\pgfpathlineto{\pgfqpoint{5.800000in}{155.078439in}}%
\pgfpathlineto{\pgfqpoint{5.800000in}{155.078439in}}%
\pgfpathclose%
\pgfusepath{fill}%
\end{pgfscope}%
\begin{pgfscope}%
\pgfpathrectangle{\pgfqpoint{5.800000in}{0.720000in}}{\pgfqpoint{1.400000in}{4.620000in}}%
\pgfusepath{clip}%
\pgfsetbuttcap%
\pgfsetmiterjoin%
\definecolor{currentfill}{rgb}{0.121569,0.466667,0.705882}%
\pgfsetfillcolor{currentfill}%
\pgfsetlinewidth{0.000000pt}%
\definecolor{currentstroke}{rgb}{0.000000,0.000000,0.000000}%
\pgfsetstrokecolor{currentstroke}%
\pgfsetstrokeopacity{0.000000}%
\pgfsetdash{}{0pt}%
\pgfpathmoveto{\pgfqpoint{5.800000in}{155.078439in}}%
\pgfpathlineto{\pgfqpoint{5.800000in}{155.078439in}}%
\pgfpathlineto{\pgfqpoint{5.800000in}{155.181276in}}%
\pgfpathlineto{\pgfqpoint{5.800000in}{155.181276in}}%
\pgfpathclose%
\pgfusepath{fill}%
\end{pgfscope}%
\begin{pgfscope}%
\pgfpathrectangle{\pgfqpoint{5.800000in}{0.720000in}}{\pgfqpoint{1.400000in}{4.620000in}}%
\pgfusepath{clip}%
\pgfsetbuttcap%
\pgfsetmiterjoin%
\definecolor{currentfill}{rgb}{0.121569,0.466667,0.705882}%
\pgfsetfillcolor{currentfill}%
\pgfsetlinewidth{0.000000pt}%
\definecolor{currentstroke}{rgb}{0.000000,0.000000,0.000000}%
\pgfsetstrokecolor{currentstroke}%
\pgfsetstrokeopacity{0.000000}%
\pgfsetdash{}{0pt}%
\pgfpathmoveto{\pgfqpoint{5.800000in}{155.181276in}}%
\pgfpathlineto{\pgfqpoint{5.800000in}{155.181276in}}%
\pgfpathlineto{\pgfqpoint{5.800000in}{155.284113in}}%
\pgfpathlineto{\pgfqpoint{5.800000in}{155.284113in}}%
\pgfpathclose%
\pgfusepath{fill}%
\end{pgfscope}%
\begin{pgfscope}%
\pgfpathrectangle{\pgfqpoint{5.800000in}{0.720000in}}{\pgfqpoint{1.400000in}{4.620000in}}%
\pgfusepath{clip}%
\pgfsetbuttcap%
\pgfsetmiterjoin%
\definecolor{currentfill}{rgb}{0.121569,0.466667,0.705882}%
\pgfsetfillcolor{currentfill}%
\pgfsetlinewidth{0.000000pt}%
\definecolor{currentstroke}{rgb}{0.000000,0.000000,0.000000}%
\pgfsetstrokecolor{currentstroke}%
\pgfsetstrokeopacity{0.000000}%
\pgfsetdash{}{0pt}%
\pgfpathmoveto{\pgfqpoint{5.800000in}{155.284113in}}%
\pgfpathlineto{\pgfqpoint{5.800000in}{155.284113in}}%
\pgfpathlineto{\pgfqpoint{5.800000in}{155.386950in}}%
\pgfpathlineto{\pgfqpoint{5.800000in}{155.386950in}}%
\pgfpathclose%
\pgfusepath{fill}%
\end{pgfscope}%
\begin{pgfscope}%
\pgfpathrectangle{\pgfqpoint{5.800000in}{0.720000in}}{\pgfqpoint{1.400000in}{4.620000in}}%
\pgfusepath{clip}%
\pgfsetbuttcap%
\pgfsetmiterjoin%
\definecolor{currentfill}{rgb}{0.121569,0.466667,0.705882}%
\pgfsetfillcolor{currentfill}%
\pgfsetlinewidth{0.000000pt}%
\definecolor{currentstroke}{rgb}{0.000000,0.000000,0.000000}%
\pgfsetstrokecolor{currentstroke}%
\pgfsetstrokeopacity{0.000000}%
\pgfsetdash{}{0pt}%
\pgfpathmoveto{\pgfqpoint{5.800000in}{155.386950in}}%
\pgfpathlineto{\pgfqpoint{5.800000in}{155.386950in}}%
\pgfpathlineto{\pgfqpoint{5.800000in}{155.489787in}}%
\pgfpathlineto{\pgfqpoint{5.800000in}{155.489787in}}%
\pgfpathclose%
\pgfusepath{fill}%
\end{pgfscope}%
\begin{pgfscope}%
\pgfpathrectangle{\pgfqpoint{5.800000in}{0.720000in}}{\pgfqpoint{1.400000in}{4.620000in}}%
\pgfusepath{clip}%
\pgfsetbuttcap%
\pgfsetmiterjoin%
\definecolor{currentfill}{rgb}{0.121569,0.466667,0.705882}%
\pgfsetfillcolor{currentfill}%
\pgfsetlinewidth{0.000000pt}%
\definecolor{currentstroke}{rgb}{0.000000,0.000000,0.000000}%
\pgfsetstrokecolor{currentstroke}%
\pgfsetstrokeopacity{0.000000}%
\pgfsetdash{}{0pt}%
\pgfpathmoveto{\pgfqpoint{5.800000in}{155.489787in}}%
\pgfpathlineto{\pgfqpoint{5.800000in}{155.489787in}}%
\pgfpathlineto{\pgfqpoint{5.800000in}{155.592624in}}%
\pgfpathlineto{\pgfqpoint{5.800000in}{155.592624in}}%
\pgfpathclose%
\pgfusepath{fill}%
\end{pgfscope}%
\begin{pgfscope}%
\pgfpathrectangle{\pgfqpoint{5.800000in}{0.720000in}}{\pgfqpoint{1.400000in}{4.620000in}}%
\pgfusepath{clip}%
\pgfsetbuttcap%
\pgfsetmiterjoin%
\definecolor{currentfill}{rgb}{0.121569,0.466667,0.705882}%
\pgfsetfillcolor{currentfill}%
\pgfsetlinewidth{0.000000pt}%
\definecolor{currentstroke}{rgb}{0.000000,0.000000,0.000000}%
\pgfsetstrokecolor{currentstroke}%
\pgfsetstrokeopacity{0.000000}%
\pgfsetdash{}{0pt}%
\pgfpathmoveto{\pgfqpoint{5.800000in}{155.592624in}}%
\pgfpathlineto{\pgfqpoint{5.800000in}{155.592624in}}%
\pgfpathlineto{\pgfqpoint{5.800000in}{155.695461in}}%
\pgfpathlineto{\pgfqpoint{5.800000in}{155.695461in}}%
\pgfpathclose%
\pgfusepath{fill}%
\end{pgfscope}%
\begin{pgfscope}%
\pgfpathrectangle{\pgfqpoint{5.800000in}{0.720000in}}{\pgfqpoint{1.400000in}{4.620000in}}%
\pgfusepath{clip}%
\pgfsetbuttcap%
\pgfsetmiterjoin%
\definecolor{currentfill}{rgb}{0.121569,0.466667,0.705882}%
\pgfsetfillcolor{currentfill}%
\pgfsetlinewidth{0.000000pt}%
\definecolor{currentstroke}{rgb}{0.000000,0.000000,0.000000}%
\pgfsetstrokecolor{currentstroke}%
\pgfsetstrokeopacity{0.000000}%
\pgfsetdash{}{0pt}%
\pgfpathmoveto{\pgfqpoint{5.800000in}{155.695461in}}%
\pgfpathlineto{\pgfqpoint{5.800000in}{155.695461in}}%
\pgfpathlineto{\pgfqpoint{5.800000in}{155.798298in}}%
\pgfpathlineto{\pgfqpoint{5.800000in}{155.798298in}}%
\pgfpathclose%
\pgfusepath{fill}%
\end{pgfscope}%
\begin{pgfscope}%
\pgfpathrectangle{\pgfqpoint{5.800000in}{0.720000in}}{\pgfqpoint{1.400000in}{4.620000in}}%
\pgfusepath{clip}%
\pgfsetbuttcap%
\pgfsetmiterjoin%
\definecolor{currentfill}{rgb}{0.121569,0.466667,0.705882}%
\pgfsetfillcolor{currentfill}%
\pgfsetlinewidth{0.000000pt}%
\definecolor{currentstroke}{rgb}{0.000000,0.000000,0.000000}%
\pgfsetstrokecolor{currentstroke}%
\pgfsetstrokeopacity{0.000000}%
\pgfsetdash{}{0pt}%
\pgfpathmoveto{\pgfqpoint{5.800000in}{155.798298in}}%
\pgfpathlineto{\pgfqpoint{5.800000in}{155.798298in}}%
\pgfpathlineto{\pgfqpoint{5.800000in}{155.901135in}}%
\pgfpathlineto{\pgfqpoint{5.800000in}{155.901135in}}%
\pgfpathclose%
\pgfusepath{fill}%
\end{pgfscope}%
\begin{pgfscope}%
\pgfpathrectangle{\pgfqpoint{5.800000in}{0.720000in}}{\pgfqpoint{1.400000in}{4.620000in}}%
\pgfusepath{clip}%
\pgfsetbuttcap%
\pgfsetmiterjoin%
\definecolor{currentfill}{rgb}{0.121569,0.466667,0.705882}%
\pgfsetfillcolor{currentfill}%
\pgfsetlinewidth{0.000000pt}%
\definecolor{currentstroke}{rgb}{0.000000,0.000000,0.000000}%
\pgfsetstrokecolor{currentstroke}%
\pgfsetstrokeopacity{0.000000}%
\pgfsetdash{}{0pt}%
\pgfpathmoveto{\pgfqpoint{5.800000in}{155.901135in}}%
\pgfpathlineto{\pgfqpoint{5.800000in}{155.901135in}}%
\pgfpathlineto{\pgfqpoint{5.800000in}{156.003972in}}%
\pgfpathlineto{\pgfqpoint{5.800000in}{156.003972in}}%
\pgfpathclose%
\pgfusepath{fill}%
\end{pgfscope}%
\begin{pgfscope}%
\pgfpathrectangle{\pgfqpoint{5.800000in}{0.720000in}}{\pgfqpoint{1.400000in}{4.620000in}}%
\pgfusepath{clip}%
\pgfsetbuttcap%
\pgfsetmiterjoin%
\definecolor{currentfill}{rgb}{0.121569,0.466667,0.705882}%
\pgfsetfillcolor{currentfill}%
\pgfsetlinewidth{0.000000pt}%
\definecolor{currentstroke}{rgb}{0.000000,0.000000,0.000000}%
\pgfsetstrokecolor{currentstroke}%
\pgfsetstrokeopacity{0.000000}%
\pgfsetdash{}{0pt}%
\pgfpathmoveto{\pgfqpoint{5.800000in}{156.003972in}}%
\pgfpathlineto{\pgfqpoint{5.800000in}{156.003972in}}%
\pgfpathlineto{\pgfqpoint{5.800000in}{156.106809in}}%
\pgfpathlineto{\pgfqpoint{5.800000in}{156.106809in}}%
\pgfpathclose%
\pgfusepath{fill}%
\end{pgfscope}%
\begin{pgfscope}%
\pgfpathrectangle{\pgfqpoint{5.800000in}{0.720000in}}{\pgfqpoint{1.400000in}{4.620000in}}%
\pgfusepath{clip}%
\pgfsetbuttcap%
\pgfsetmiterjoin%
\definecolor{currentfill}{rgb}{0.121569,0.466667,0.705882}%
\pgfsetfillcolor{currentfill}%
\pgfsetlinewidth{0.000000pt}%
\definecolor{currentstroke}{rgb}{0.000000,0.000000,0.000000}%
\pgfsetstrokecolor{currentstroke}%
\pgfsetstrokeopacity{0.000000}%
\pgfsetdash{}{0pt}%
\pgfpathmoveto{\pgfqpoint{5.800000in}{156.106809in}}%
\pgfpathlineto{\pgfqpoint{5.800000in}{156.106809in}}%
\pgfpathlineto{\pgfqpoint{5.800000in}{156.209646in}}%
\pgfpathlineto{\pgfqpoint{5.800000in}{156.209646in}}%
\pgfpathclose%
\pgfusepath{fill}%
\end{pgfscope}%
\begin{pgfscope}%
\pgfpathrectangle{\pgfqpoint{5.800000in}{0.720000in}}{\pgfqpoint{1.400000in}{4.620000in}}%
\pgfusepath{clip}%
\pgfsetbuttcap%
\pgfsetmiterjoin%
\definecolor{currentfill}{rgb}{0.121569,0.466667,0.705882}%
\pgfsetfillcolor{currentfill}%
\pgfsetlinewidth{0.000000pt}%
\definecolor{currentstroke}{rgb}{0.000000,0.000000,0.000000}%
\pgfsetstrokecolor{currentstroke}%
\pgfsetstrokeopacity{0.000000}%
\pgfsetdash{}{0pt}%
\pgfpathmoveto{\pgfqpoint{5.800000in}{156.209646in}}%
\pgfpathlineto{\pgfqpoint{5.800000in}{156.209646in}}%
\pgfpathlineto{\pgfqpoint{5.800000in}{156.312484in}}%
\pgfpathlineto{\pgfqpoint{5.800000in}{156.312484in}}%
\pgfpathclose%
\pgfusepath{fill}%
\end{pgfscope}%
\begin{pgfscope}%
\pgfpathrectangle{\pgfqpoint{5.800000in}{0.720000in}}{\pgfqpoint{1.400000in}{4.620000in}}%
\pgfusepath{clip}%
\pgfsetbuttcap%
\pgfsetmiterjoin%
\definecolor{currentfill}{rgb}{0.121569,0.466667,0.705882}%
\pgfsetfillcolor{currentfill}%
\pgfsetlinewidth{0.000000pt}%
\definecolor{currentstroke}{rgb}{0.000000,0.000000,0.000000}%
\pgfsetstrokecolor{currentstroke}%
\pgfsetstrokeopacity{0.000000}%
\pgfsetdash{}{0pt}%
\pgfpathmoveto{\pgfqpoint{5.800000in}{156.312484in}}%
\pgfpathlineto{\pgfqpoint{5.800000in}{156.312484in}}%
\pgfpathlineto{\pgfqpoint{5.800000in}{156.415321in}}%
\pgfpathlineto{\pgfqpoint{5.800000in}{156.415321in}}%
\pgfpathclose%
\pgfusepath{fill}%
\end{pgfscope}%
\begin{pgfscope}%
\pgfpathrectangle{\pgfqpoint{5.800000in}{0.720000in}}{\pgfqpoint{1.400000in}{4.620000in}}%
\pgfusepath{clip}%
\pgfsetbuttcap%
\pgfsetmiterjoin%
\definecolor{currentfill}{rgb}{0.121569,0.466667,0.705882}%
\pgfsetfillcolor{currentfill}%
\pgfsetlinewidth{0.000000pt}%
\definecolor{currentstroke}{rgb}{0.000000,0.000000,0.000000}%
\pgfsetstrokecolor{currentstroke}%
\pgfsetstrokeopacity{0.000000}%
\pgfsetdash{}{0pt}%
\pgfpathmoveto{\pgfqpoint{5.800000in}{156.415321in}}%
\pgfpathlineto{\pgfqpoint{5.800000in}{156.415321in}}%
\pgfpathlineto{\pgfqpoint{5.800000in}{156.518158in}}%
\pgfpathlineto{\pgfqpoint{5.800000in}{156.518158in}}%
\pgfpathclose%
\pgfusepath{fill}%
\end{pgfscope}%
\begin{pgfscope}%
\pgfpathrectangle{\pgfqpoint{5.800000in}{0.720000in}}{\pgfqpoint{1.400000in}{4.620000in}}%
\pgfusepath{clip}%
\pgfsetbuttcap%
\pgfsetmiterjoin%
\definecolor{currentfill}{rgb}{0.121569,0.466667,0.705882}%
\pgfsetfillcolor{currentfill}%
\pgfsetlinewidth{0.000000pt}%
\definecolor{currentstroke}{rgb}{0.000000,0.000000,0.000000}%
\pgfsetstrokecolor{currentstroke}%
\pgfsetstrokeopacity{0.000000}%
\pgfsetdash{}{0pt}%
\pgfpathmoveto{\pgfqpoint{5.800000in}{156.518158in}}%
\pgfpathlineto{\pgfqpoint{5.800000in}{156.518158in}}%
\pgfpathlineto{\pgfqpoint{5.800000in}{156.620995in}}%
\pgfpathlineto{\pgfqpoint{5.800000in}{156.620995in}}%
\pgfpathclose%
\pgfusepath{fill}%
\end{pgfscope}%
\begin{pgfscope}%
\pgfpathrectangle{\pgfqpoint{5.800000in}{0.720000in}}{\pgfqpoint{1.400000in}{4.620000in}}%
\pgfusepath{clip}%
\pgfsetbuttcap%
\pgfsetmiterjoin%
\definecolor{currentfill}{rgb}{0.121569,0.466667,0.705882}%
\pgfsetfillcolor{currentfill}%
\pgfsetlinewidth{0.000000pt}%
\definecolor{currentstroke}{rgb}{0.000000,0.000000,0.000000}%
\pgfsetstrokecolor{currentstroke}%
\pgfsetstrokeopacity{0.000000}%
\pgfsetdash{}{0pt}%
\pgfpathmoveto{\pgfqpoint{5.800000in}{156.620995in}}%
\pgfpathlineto{\pgfqpoint{5.800000in}{156.620995in}}%
\pgfpathlineto{\pgfqpoint{5.800000in}{156.723832in}}%
\pgfpathlineto{\pgfqpoint{5.800000in}{156.723832in}}%
\pgfpathclose%
\pgfusepath{fill}%
\end{pgfscope}%
\begin{pgfscope}%
\pgfpathrectangle{\pgfqpoint{5.800000in}{0.720000in}}{\pgfqpoint{1.400000in}{4.620000in}}%
\pgfusepath{clip}%
\pgfsetbuttcap%
\pgfsetmiterjoin%
\definecolor{currentfill}{rgb}{0.121569,0.466667,0.705882}%
\pgfsetfillcolor{currentfill}%
\pgfsetlinewidth{0.000000pt}%
\definecolor{currentstroke}{rgb}{0.000000,0.000000,0.000000}%
\pgfsetstrokecolor{currentstroke}%
\pgfsetstrokeopacity{0.000000}%
\pgfsetdash{}{0pt}%
\pgfpathmoveto{\pgfqpoint{5.800000in}{156.723832in}}%
\pgfpathlineto{\pgfqpoint{5.800000in}{156.723832in}}%
\pgfpathlineto{\pgfqpoint{5.800000in}{156.826669in}}%
\pgfpathlineto{\pgfqpoint{5.800000in}{156.826669in}}%
\pgfpathclose%
\pgfusepath{fill}%
\end{pgfscope}%
\begin{pgfscope}%
\pgfpathrectangle{\pgfqpoint{5.800000in}{0.720000in}}{\pgfqpoint{1.400000in}{4.620000in}}%
\pgfusepath{clip}%
\pgfsetbuttcap%
\pgfsetmiterjoin%
\definecolor{currentfill}{rgb}{0.121569,0.466667,0.705882}%
\pgfsetfillcolor{currentfill}%
\pgfsetlinewidth{0.000000pt}%
\definecolor{currentstroke}{rgb}{0.000000,0.000000,0.000000}%
\pgfsetstrokecolor{currentstroke}%
\pgfsetstrokeopacity{0.000000}%
\pgfsetdash{}{0pt}%
\pgfpathmoveto{\pgfqpoint{5.800000in}{156.826669in}}%
\pgfpathlineto{\pgfqpoint{5.800000in}{156.826669in}}%
\pgfpathlineto{\pgfqpoint{5.800000in}{156.929506in}}%
\pgfpathlineto{\pgfqpoint{5.800000in}{156.929506in}}%
\pgfpathclose%
\pgfusepath{fill}%
\end{pgfscope}%
\begin{pgfscope}%
\pgfpathrectangle{\pgfqpoint{5.800000in}{0.720000in}}{\pgfqpoint{1.400000in}{4.620000in}}%
\pgfusepath{clip}%
\pgfsetbuttcap%
\pgfsetmiterjoin%
\definecolor{currentfill}{rgb}{0.121569,0.466667,0.705882}%
\pgfsetfillcolor{currentfill}%
\pgfsetlinewidth{0.000000pt}%
\definecolor{currentstroke}{rgb}{0.000000,0.000000,0.000000}%
\pgfsetstrokecolor{currentstroke}%
\pgfsetstrokeopacity{0.000000}%
\pgfsetdash{}{0pt}%
\pgfpathmoveto{\pgfqpoint{5.800000in}{156.929506in}}%
\pgfpathlineto{\pgfqpoint{5.800000in}{156.929506in}}%
\pgfpathlineto{\pgfqpoint{5.800000in}{157.032343in}}%
\pgfpathlineto{\pgfqpoint{5.800000in}{157.032343in}}%
\pgfpathclose%
\pgfusepath{fill}%
\end{pgfscope}%
\begin{pgfscope}%
\pgfpathrectangle{\pgfqpoint{5.800000in}{0.720000in}}{\pgfqpoint{1.400000in}{4.620000in}}%
\pgfusepath{clip}%
\pgfsetbuttcap%
\pgfsetmiterjoin%
\definecolor{currentfill}{rgb}{0.121569,0.466667,0.705882}%
\pgfsetfillcolor{currentfill}%
\pgfsetlinewidth{0.000000pt}%
\definecolor{currentstroke}{rgb}{0.000000,0.000000,0.000000}%
\pgfsetstrokecolor{currentstroke}%
\pgfsetstrokeopacity{0.000000}%
\pgfsetdash{}{0pt}%
\pgfpathmoveto{\pgfqpoint{5.800000in}{157.032343in}}%
\pgfpathlineto{\pgfqpoint{5.800000in}{157.032343in}}%
\pgfpathlineto{\pgfqpoint{5.800000in}{157.135180in}}%
\pgfpathlineto{\pgfqpoint{5.800000in}{157.135180in}}%
\pgfpathclose%
\pgfusepath{fill}%
\end{pgfscope}%
\begin{pgfscope}%
\pgfpathrectangle{\pgfqpoint{5.800000in}{0.720000in}}{\pgfqpoint{1.400000in}{4.620000in}}%
\pgfusepath{clip}%
\pgfsetbuttcap%
\pgfsetmiterjoin%
\definecolor{currentfill}{rgb}{0.121569,0.466667,0.705882}%
\pgfsetfillcolor{currentfill}%
\pgfsetlinewidth{0.000000pt}%
\definecolor{currentstroke}{rgb}{0.000000,0.000000,0.000000}%
\pgfsetstrokecolor{currentstroke}%
\pgfsetstrokeopacity{0.000000}%
\pgfsetdash{}{0pt}%
\pgfpathmoveto{\pgfqpoint{5.800000in}{157.135180in}}%
\pgfpathlineto{\pgfqpoint{5.800000in}{157.135180in}}%
\pgfpathlineto{\pgfqpoint{5.800000in}{157.238017in}}%
\pgfpathlineto{\pgfqpoint{5.800000in}{157.238017in}}%
\pgfpathclose%
\pgfusepath{fill}%
\end{pgfscope}%
\begin{pgfscope}%
\pgfpathrectangle{\pgfqpoint{5.800000in}{0.720000in}}{\pgfqpoint{1.400000in}{4.620000in}}%
\pgfusepath{clip}%
\pgfsetbuttcap%
\pgfsetmiterjoin%
\definecolor{currentfill}{rgb}{0.121569,0.466667,0.705882}%
\pgfsetfillcolor{currentfill}%
\pgfsetlinewidth{0.000000pt}%
\definecolor{currentstroke}{rgb}{0.000000,0.000000,0.000000}%
\pgfsetstrokecolor{currentstroke}%
\pgfsetstrokeopacity{0.000000}%
\pgfsetdash{}{0pt}%
\pgfpathmoveto{\pgfqpoint{5.800000in}{157.238017in}}%
\pgfpathlineto{\pgfqpoint{5.800000in}{157.238017in}}%
\pgfpathlineto{\pgfqpoint{5.800000in}{157.340854in}}%
\pgfpathlineto{\pgfqpoint{5.800000in}{157.340854in}}%
\pgfpathclose%
\pgfusepath{fill}%
\end{pgfscope}%
\begin{pgfscope}%
\pgfpathrectangle{\pgfqpoint{5.800000in}{0.720000in}}{\pgfqpoint{1.400000in}{4.620000in}}%
\pgfusepath{clip}%
\pgfsetbuttcap%
\pgfsetmiterjoin%
\definecolor{currentfill}{rgb}{0.121569,0.466667,0.705882}%
\pgfsetfillcolor{currentfill}%
\pgfsetlinewidth{0.000000pt}%
\definecolor{currentstroke}{rgb}{0.000000,0.000000,0.000000}%
\pgfsetstrokecolor{currentstroke}%
\pgfsetstrokeopacity{0.000000}%
\pgfsetdash{}{0pt}%
\pgfpathmoveto{\pgfqpoint{5.800000in}{157.340854in}}%
\pgfpathlineto{\pgfqpoint{5.800000in}{157.340854in}}%
\pgfpathlineto{\pgfqpoint{5.800000in}{157.443691in}}%
\pgfpathlineto{\pgfqpoint{5.800000in}{157.443691in}}%
\pgfpathclose%
\pgfusepath{fill}%
\end{pgfscope}%
\begin{pgfscope}%
\pgfpathrectangle{\pgfqpoint{5.800000in}{0.720000in}}{\pgfqpoint{1.400000in}{4.620000in}}%
\pgfusepath{clip}%
\pgfsetbuttcap%
\pgfsetmiterjoin%
\definecolor{currentfill}{rgb}{0.121569,0.466667,0.705882}%
\pgfsetfillcolor{currentfill}%
\pgfsetlinewidth{0.000000pt}%
\definecolor{currentstroke}{rgb}{0.000000,0.000000,0.000000}%
\pgfsetstrokecolor{currentstroke}%
\pgfsetstrokeopacity{0.000000}%
\pgfsetdash{}{0pt}%
\pgfpathmoveto{\pgfqpoint{5.800000in}{157.443691in}}%
\pgfpathlineto{\pgfqpoint{5.800000in}{157.443691in}}%
\pgfpathlineto{\pgfqpoint{5.800000in}{157.546528in}}%
\pgfpathlineto{\pgfqpoint{5.800000in}{157.546528in}}%
\pgfpathclose%
\pgfusepath{fill}%
\end{pgfscope}%
\begin{pgfscope}%
\pgfpathrectangle{\pgfqpoint{5.800000in}{0.720000in}}{\pgfqpoint{1.400000in}{4.620000in}}%
\pgfusepath{clip}%
\pgfsetbuttcap%
\pgfsetmiterjoin%
\definecolor{currentfill}{rgb}{0.121569,0.466667,0.705882}%
\pgfsetfillcolor{currentfill}%
\pgfsetlinewidth{0.000000pt}%
\definecolor{currentstroke}{rgb}{0.000000,0.000000,0.000000}%
\pgfsetstrokecolor{currentstroke}%
\pgfsetstrokeopacity{0.000000}%
\pgfsetdash{}{0pt}%
\pgfpathmoveto{\pgfqpoint{5.800000in}{157.546528in}}%
\pgfpathlineto{\pgfqpoint{5.800000in}{157.546528in}}%
\pgfpathlineto{\pgfqpoint{5.800000in}{157.649365in}}%
\pgfpathlineto{\pgfqpoint{5.800000in}{157.649365in}}%
\pgfpathclose%
\pgfusepath{fill}%
\end{pgfscope}%
\begin{pgfscope}%
\pgfpathrectangle{\pgfqpoint{5.800000in}{0.720000in}}{\pgfqpoint{1.400000in}{4.620000in}}%
\pgfusepath{clip}%
\pgfsetbuttcap%
\pgfsetmiterjoin%
\definecolor{currentfill}{rgb}{0.121569,0.466667,0.705882}%
\pgfsetfillcolor{currentfill}%
\pgfsetlinewidth{0.000000pt}%
\definecolor{currentstroke}{rgb}{0.000000,0.000000,0.000000}%
\pgfsetstrokecolor{currentstroke}%
\pgfsetstrokeopacity{0.000000}%
\pgfsetdash{}{0pt}%
\pgfpathmoveto{\pgfqpoint{5.800000in}{157.649365in}}%
\pgfpathlineto{\pgfqpoint{5.800000in}{157.649365in}}%
\pgfpathlineto{\pgfqpoint{5.800000in}{157.752202in}}%
\pgfpathlineto{\pgfqpoint{5.800000in}{157.752202in}}%
\pgfpathclose%
\pgfusepath{fill}%
\end{pgfscope}%
\begin{pgfscope}%
\pgfpathrectangle{\pgfqpoint{5.800000in}{0.720000in}}{\pgfqpoint{1.400000in}{4.620000in}}%
\pgfusepath{clip}%
\pgfsetbuttcap%
\pgfsetmiterjoin%
\definecolor{currentfill}{rgb}{0.121569,0.466667,0.705882}%
\pgfsetfillcolor{currentfill}%
\pgfsetlinewidth{0.000000pt}%
\definecolor{currentstroke}{rgb}{0.000000,0.000000,0.000000}%
\pgfsetstrokecolor{currentstroke}%
\pgfsetstrokeopacity{0.000000}%
\pgfsetdash{}{0pt}%
\pgfpathmoveto{\pgfqpoint{5.800000in}{157.752202in}}%
\pgfpathlineto{\pgfqpoint{5.800000in}{157.752202in}}%
\pgfpathlineto{\pgfqpoint{5.800000in}{157.855040in}}%
\pgfpathlineto{\pgfqpoint{5.800000in}{157.855040in}}%
\pgfpathclose%
\pgfusepath{fill}%
\end{pgfscope}%
\begin{pgfscope}%
\pgfpathrectangle{\pgfqpoint{5.800000in}{0.720000in}}{\pgfqpoint{1.400000in}{4.620000in}}%
\pgfusepath{clip}%
\pgfsetbuttcap%
\pgfsetmiterjoin%
\definecolor{currentfill}{rgb}{0.121569,0.466667,0.705882}%
\pgfsetfillcolor{currentfill}%
\pgfsetlinewidth{0.000000pt}%
\definecolor{currentstroke}{rgb}{0.000000,0.000000,0.000000}%
\pgfsetstrokecolor{currentstroke}%
\pgfsetstrokeopacity{0.000000}%
\pgfsetdash{}{0pt}%
\pgfpathmoveto{\pgfqpoint{5.800000in}{157.855040in}}%
\pgfpathlineto{\pgfqpoint{5.800000in}{157.855040in}}%
\pgfpathlineto{\pgfqpoint{5.800000in}{157.957877in}}%
\pgfpathlineto{\pgfqpoint{5.800000in}{157.957877in}}%
\pgfpathclose%
\pgfusepath{fill}%
\end{pgfscope}%
\begin{pgfscope}%
\pgfpathrectangle{\pgfqpoint{5.800000in}{0.720000in}}{\pgfqpoint{1.400000in}{4.620000in}}%
\pgfusepath{clip}%
\pgfsetbuttcap%
\pgfsetmiterjoin%
\definecolor{currentfill}{rgb}{0.121569,0.466667,0.705882}%
\pgfsetfillcolor{currentfill}%
\pgfsetlinewidth{0.000000pt}%
\definecolor{currentstroke}{rgb}{0.000000,0.000000,0.000000}%
\pgfsetstrokecolor{currentstroke}%
\pgfsetstrokeopacity{0.000000}%
\pgfsetdash{}{0pt}%
\pgfpathmoveto{\pgfqpoint{5.800000in}{157.957877in}}%
\pgfpathlineto{\pgfqpoint{5.800000in}{157.957877in}}%
\pgfpathlineto{\pgfqpoint{5.800000in}{158.060714in}}%
\pgfpathlineto{\pgfqpoint{5.800000in}{158.060714in}}%
\pgfpathclose%
\pgfusepath{fill}%
\end{pgfscope}%
\begin{pgfscope}%
\pgfpathrectangle{\pgfqpoint{5.800000in}{0.720000in}}{\pgfqpoint{1.400000in}{4.620000in}}%
\pgfusepath{clip}%
\pgfsetbuttcap%
\pgfsetmiterjoin%
\definecolor{currentfill}{rgb}{0.121569,0.466667,0.705882}%
\pgfsetfillcolor{currentfill}%
\pgfsetlinewidth{0.000000pt}%
\definecolor{currentstroke}{rgb}{0.000000,0.000000,0.000000}%
\pgfsetstrokecolor{currentstroke}%
\pgfsetstrokeopacity{0.000000}%
\pgfsetdash{}{0pt}%
\pgfpathmoveto{\pgfqpoint{5.800000in}{158.060714in}}%
\pgfpathlineto{\pgfqpoint{5.800000in}{158.060714in}}%
\pgfpathlineto{\pgfqpoint{5.800000in}{158.163551in}}%
\pgfpathlineto{\pgfqpoint{5.800000in}{158.163551in}}%
\pgfpathclose%
\pgfusepath{fill}%
\end{pgfscope}%
\begin{pgfscope}%
\pgfpathrectangle{\pgfqpoint{5.800000in}{0.720000in}}{\pgfqpoint{1.400000in}{4.620000in}}%
\pgfusepath{clip}%
\pgfsetbuttcap%
\pgfsetmiterjoin%
\definecolor{currentfill}{rgb}{0.121569,0.466667,0.705882}%
\pgfsetfillcolor{currentfill}%
\pgfsetlinewidth{0.000000pt}%
\definecolor{currentstroke}{rgb}{0.000000,0.000000,0.000000}%
\pgfsetstrokecolor{currentstroke}%
\pgfsetstrokeopacity{0.000000}%
\pgfsetdash{}{0pt}%
\pgfpathmoveto{\pgfqpoint{5.800000in}{158.163551in}}%
\pgfpathlineto{\pgfqpoint{5.800000in}{158.163551in}}%
\pgfpathlineto{\pgfqpoint{5.800000in}{158.266388in}}%
\pgfpathlineto{\pgfqpoint{5.800000in}{158.266388in}}%
\pgfpathclose%
\pgfusepath{fill}%
\end{pgfscope}%
\begin{pgfscope}%
\pgfpathrectangle{\pgfqpoint{5.800000in}{0.720000in}}{\pgfqpoint{1.400000in}{4.620000in}}%
\pgfusepath{clip}%
\pgfsetbuttcap%
\pgfsetmiterjoin%
\definecolor{currentfill}{rgb}{0.121569,0.466667,0.705882}%
\pgfsetfillcolor{currentfill}%
\pgfsetlinewidth{0.000000pt}%
\definecolor{currentstroke}{rgb}{0.000000,0.000000,0.000000}%
\pgfsetstrokecolor{currentstroke}%
\pgfsetstrokeopacity{0.000000}%
\pgfsetdash{}{0pt}%
\pgfpathmoveto{\pgfqpoint{5.800000in}{158.266388in}}%
\pgfpathlineto{\pgfqpoint{5.800000in}{158.266388in}}%
\pgfpathlineto{\pgfqpoint{5.800000in}{158.369225in}}%
\pgfpathlineto{\pgfqpoint{5.800000in}{158.369225in}}%
\pgfpathclose%
\pgfusepath{fill}%
\end{pgfscope}%
\begin{pgfscope}%
\pgfpathrectangle{\pgfqpoint{5.800000in}{0.720000in}}{\pgfqpoint{1.400000in}{4.620000in}}%
\pgfusepath{clip}%
\pgfsetbuttcap%
\pgfsetmiterjoin%
\definecolor{currentfill}{rgb}{0.121569,0.466667,0.705882}%
\pgfsetfillcolor{currentfill}%
\pgfsetlinewidth{0.000000pt}%
\definecolor{currentstroke}{rgb}{0.000000,0.000000,0.000000}%
\pgfsetstrokecolor{currentstroke}%
\pgfsetstrokeopacity{0.000000}%
\pgfsetdash{}{0pt}%
\pgfpathmoveto{\pgfqpoint{5.800000in}{158.369225in}}%
\pgfpathlineto{\pgfqpoint{5.800000in}{158.369225in}}%
\pgfpathlineto{\pgfqpoint{5.800000in}{158.472062in}}%
\pgfpathlineto{\pgfqpoint{5.800000in}{158.472062in}}%
\pgfpathclose%
\pgfusepath{fill}%
\end{pgfscope}%
\begin{pgfscope}%
\pgfpathrectangle{\pgfqpoint{5.800000in}{0.720000in}}{\pgfqpoint{1.400000in}{4.620000in}}%
\pgfusepath{clip}%
\pgfsetbuttcap%
\pgfsetmiterjoin%
\definecolor{currentfill}{rgb}{0.121569,0.466667,0.705882}%
\pgfsetfillcolor{currentfill}%
\pgfsetlinewidth{0.000000pt}%
\definecolor{currentstroke}{rgb}{0.000000,0.000000,0.000000}%
\pgfsetstrokecolor{currentstroke}%
\pgfsetstrokeopacity{0.000000}%
\pgfsetdash{}{0pt}%
\pgfpathmoveto{\pgfqpoint{5.800000in}{158.472062in}}%
\pgfpathlineto{\pgfqpoint{5.800000in}{158.472062in}}%
\pgfpathlineto{\pgfqpoint{5.800000in}{158.574899in}}%
\pgfpathlineto{\pgfqpoint{5.800000in}{158.574899in}}%
\pgfpathclose%
\pgfusepath{fill}%
\end{pgfscope}%
\begin{pgfscope}%
\pgfpathrectangle{\pgfqpoint{5.800000in}{0.720000in}}{\pgfqpoint{1.400000in}{4.620000in}}%
\pgfusepath{clip}%
\pgfsetbuttcap%
\pgfsetmiterjoin%
\definecolor{currentfill}{rgb}{0.121569,0.466667,0.705882}%
\pgfsetfillcolor{currentfill}%
\pgfsetlinewidth{0.000000pt}%
\definecolor{currentstroke}{rgb}{0.000000,0.000000,0.000000}%
\pgfsetstrokecolor{currentstroke}%
\pgfsetstrokeopacity{0.000000}%
\pgfsetdash{}{0pt}%
\pgfpathmoveto{\pgfqpoint{5.800000in}{158.574899in}}%
\pgfpathlineto{\pgfqpoint{5.800000in}{158.574899in}}%
\pgfpathlineto{\pgfqpoint{5.800000in}{158.677736in}}%
\pgfpathlineto{\pgfqpoint{5.800000in}{158.677736in}}%
\pgfpathclose%
\pgfusepath{fill}%
\end{pgfscope}%
\begin{pgfscope}%
\pgfpathrectangle{\pgfqpoint{5.800000in}{0.720000in}}{\pgfqpoint{1.400000in}{4.620000in}}%
\pgfusepath{clip}%
\pgfsetbuttcap%
\pgfsetmiterjoin%
\definecolor{currentfill}{rgb}{0.121569,0.466667,0.705882}%
\pgfsetfillcolor{currentfill}%
\pgfsetlinewidth{0.000000pt}%
\definecolor{currentstroke}{rgb}{0.000000,0.000000,0.000000}%
\pgfsetstrokecolor{currentstroke}%
\pgfsetstrokeopacity{0.000000}%
\pgfsetdash{}{0pt}%
\pgfpathmoveto{\pgfqpoint{5.800000in}{158.677736in}}%
\pgfpathlineto{\pgfqpoint{5.800000in}{158.677736in}}%
\pgfpathlineto{\pgfqpoint{5.800000in}{158.780573in}}%
\pgfpathlineto{\pgfqpoint{5.800000in}{158.780573in}}%
\pgfpathclose%
\pgfusepath{fill}%
\end{pgfscope}%
\begin{pgfscope}%
\pgfpathrectangle{\pgfqpoint{5.800000in}{0.720000in}}{\pgfqpoint{1.400000in}{4.620000in}}%
\pgfusepath{clip}%
\pgfsetbuttcap%
\pgfsetmiterjoin%
\definecolor{currentfill}{rgb}{0.121569,0.466667,0.705882}%
\pgfsetfillcolor{currentfill}%
\pgfsetlinewidth{0.000000pt}%
\definecolor{currentstroke}{rgb}{0.000000,0.000000,0.000000}%
\pgfsetstrokecolor{currentstroke}%
\pgfsetstrokeopacity{0.000000}%
\pgfsetdash{}{0pt}%
\pgfpathmoveto{\pgfqpoint{5.800000in}{158.780573in}}%
\pgfpathlineto{\pgfqpoint{5.800000in}{158.780573in}}%
\pgfpathlineto{\pgfqpoint{5.800000in}{158.883410in}}%
\pgfpathlineto{\pgfqpoint{5.800000in}{158.883410in}}%
\pgfpathclose%
\pgfusepath{fill}%
\end{pgfscope}%
\begin{pgfscope}%
\pgfpathrectangle{\pgfqpoint{5.800000in}{0.720000in}}{\pgfqpoint{1.400000in}{4.620000in}}%
\pgfusepath{clip}%
\pgfsetbuttcap%
\pgfsetmiterjoin%
\definecolor{currentfill}{rgb}{0.121569,0.466667,0.705882}%
\pgfsetfillcolor{currentfill}%
\pgfsetlinewidth{0.000000pt}%
\definecolor{currentstroke}{rgb}{0.000000,0.000000,0.000000}%
\pgfsetstrokecolor{currentstroke}%
\pgfsetstrokeopacity{0.000000}%
\pgfsetdash{}{0pt}%
\pgfpathmoveto{\pgfqpoint{5.800000in}{158.883410in}}%
\pgfpathlineto{\pgfqpoint{5.800000in}{158.883410in}}%
\pgfpathlineto{\pgfqpoint{5.800000in}{158.986247in}}%
\pgfpathlineto{\pgfqpoint{5.800000in}{158.986247in}}%
\pgfpathclose%
\pgfusepath{fill}%
\end{pgfscope}%
\begin{pgfscope}%
\pgfpathrectangle{\pgfqpoint{5.800000in}{0.720000in}}{\pgfqpoint{1.400000in}{4.620000in}}%
\pgfusepath{clip}%
\pgfsetbuttcap%
\pgfsetmiterjoin%
\definecolor{currentfill}{rgb}{0.121569,0.466667,0.705882}%
\pgfsetfillcolor{currentfill}%
\pgfsetlinewidth{0.000000pt}%
\definecolor{currentstroke}{rgb}{0.000000,0.000000,0.000000}%
\pgfsetstrokecolor{currentstroke}%
\pgfsetstrokeopacity{0.000000}%
\pgfsetdash{}{0pt}%
\pgfpathmoveto{\pgfqpoint{5.800000in}{158.986247in}}%
\pgfpathlineto{\pgfqpoint{5.800000in}{158.986247in}}%
\pgfpathlineto{\pgfqpoint{5.800000in}{159.089084in}}%
\pgfpathlineto{\pgfqpoint{5.800000in}{159.089084in}}%
\pgfpathclose%
\pgfusepath{fill}%
\end{pgfscope}%
\begin{pgfscope}%
\pgfpathrectangle{\pgfqpoint{5.800000in}{0.720000in}}{\pgfqpoint{1.400000in}{4.620000in}}%
\pgfusepath{clip}%
\pgfsetbuttcap%
\pgfsetmiterjoin%
\definecolor{currentfill}{rgb}{0.121569,0.466667,0.705882}%
\pgfsetfillcolor{currentfill}%
\pgfsetlinewidth{0.000000pt}%
\definecolor{currentstroke}{rgb}{0.000000,0.000000,0.000000}%
\pgfsetstrokecolor{currentstroke}%
\pgfsetstrokeopacity{0.000000}%
\pgfsetdash{}{0pt}%
\pgfpathmoveto{\pgfqpoint{5.800000in}{159.089084in}}%
\pgfpathlineto{\pgfqpoint{5.800000in}{159.089084in}}%
\pgfpathlineto{\pgfqpoint{5.800000in}{159.191921in}}%
\pgfpathlineto{\pgfqpoint{5.800000in}{159.191921in}}%
\pgfpathclose%
\pgfusepath{fill}%
\end{pgfscope}%
\begin{pgfscope}%
\pgfpathrectangle{\pgfqpoint{5.800000in}{0.720000in}}{\pgfqpoint{1.400000in}{4.620000in}}%
\pgfusepath{clip}%
\pgfsetbuttcap%
\pgfsetmiterjoin%
\definecolor{currentfill}{rgb}{0.121569,0.466667,0.705882}%
\pgfsetfillcolor{currentfill}%
\pgfsetlinewidth{0.000000pt}%
\definecolor{currentstroke}{rgb}{0.000000,0.000000,0.000000}%
\pgfsetstrokecolor{currentstroke}%
\pgfsetstrokeopacity{0.000000}%
\pgfsetdash{}{0pt}%
\pgfpathmoveto{\pgfqpoint{5.800000in}{159.191921in}}%
\pgfpathlineto{\pgfqpoint{5.800000in}{159.191921in}}%
\pgfpathlineto{\pgfqpoint{5.800000in}{159.294759in}}%
\pgfpathlineto{\pgfqpoint{5.800000in}{159.294759in}}%
\pgfpathclose%
\pgfusepath{fill}%
\end{pgfscope}%
\begin{pgfscope}%
\pgfpathrectangle{\pgfqpoint{5.800000in}{0.720000in}}{\pgfqpoint{1.400000in}{4.620000in}}%
\pgfusepath{clip}%
\pgfsetbuttcap%
\pgfsetmiterjoin%
\definecolor{currentfill}{rgb}{0.121569,0.466667,0.705882}%
\pgfsetfillcolor{currentfill}%
\pgfsetlinewidth{0.000000pt}%
\definecolor{currentstroke}{rgb}{0.000000,0.000000,0.000000}%
\pgfsetstrokecolor{currentstroke}%
\pgfsetstrokeopacity{0.000000}%
\pgfsetdash{}{0pt}%
\pgfpathmoveto{\pgfqpoint{5.800000in}{159.294759in}}%
\pgfpathlineto{\pgfqpoint{5.800000in}{159.294759in}}%
\pgfpathlineto{\pgfqpoint{5.800000in}{159.397596in}}%
\pgfpathlineto{\pgfqpoint{5.800000in}{159.397596in}}%
\pgfpathclose%
\pgfusepath{fill}%
\end{pgfscope}%
\begin{pgfscope}%
\pgfpathrectangle{\pgfqpoint{5.800000in}{0.720000in}}{\pgfqpoint{1.400000in}{4.620000in}}%
\pgfusepath{clip}%
\pgfsetbuttcap%
\pgfsetmiterjoin%
\definecolor{currentfill}{rgb}{0.121569,0.466667,0.705882}%
\pgfsetfillcolor{currentfill}%
\pgfsetlinewidth{0.000000pt}%
\definecolor{currentstroke}{rgb}{0.000000,0.000000,0.000000}%
\pgfsetstrokecolor{currentstroke}%
\pgfsetstrokeopacity{0.000000}%
\pgfsetdash{}{0pt}%
\pgfpathmoveto{\pgfqpoint{5.800000in}{159.397596in}}%
\pgfpathlineto{\pgfqpoint{5.800000in}{159.397596in}}%
\pgfpathlineto{\pgfqpoint{5.800000in}{159.500433in}}%
\pgfpathlineto{\pgfqpoint{5.800000in}{159.500433in}}%
\pgfpathclose%
\pgfusepath{fill}%
\end{pgfscope}%
\begin{pgfscope}%
\pgfpathrectangle{\pgfqpoint{5.800000in}{0.720000in}}{\pgfqpoint{1.400000in}{4.620000in}}%
\pgfusepath{clip}%
\pgfsetbuttcap%
\pgfsetmiterjoin%
\definecolor{currentfill}{rgb}{0.121569,0.466667,0.705882}%
\pgfsetfillcolor{currentfill}%
\pgfsetlinewidth{0.000000pt}%
\definecolor{currentstroke}{rgb}{0.000000,0.000000,0.000000}%
\pgfsetstrokecolor{currentstroke}%
\pgfsetstrokeopacity{0.000000}%
\pgfsetdash{}{0pt}%
\pgfpathmoveto{\pgfqpoint{5.800000in}{159.500433in}}%
\pgfpathlineto{\pgfqpoint{5.800000in}{159.500433in}}%
\pgfpathlineto{\pgfqpoint{5.800000in}{159.603270in}}%
\pgfpathlineto{\pgfqpoint{5.800000in}{159.603270in}}%
\pgfpathclose%
\pgfusepath{fill}%
\end{pgfscope}%
\begin{pgfscope}%
\pgfpathrectangle{\pgfqpoint{5.800000in}{0.720000in}}{\pgfqpoint{1.400000in}{4.620000in}}%
\pgfusepath{clip}%
\pgfsetbuttcap%
\pgfsetmiterjoin%
\definecolor{currentfill}{rgb}{0.121569,0.466667,0.705882}%
\pgfsetfillcolor{currentfill}%
\pgfsetlinewidth{0.000000pt}%
\definecolor{currentstroke}{rgb}{0.000000,0.000000,0.000000}%
\pgfsetstrokecolor{currentstroke}%
\pgfsetstrokeopacity{0.000000}%
\pgfsetdash{}{0pt}%
\pgfpathmoveto{\pgfqpoint{5.800000in}{159.603270in}}%
\pgfpathlineto{\pgfqpoint{5.800000in}{159.603270in}}%
\pgfpathlineto{\pgfqpoint{5.800000in}{159.706107in}}%
\pgfpathlineto{\pgfqpoint{5.800000in}{159.706107in}}%
\pgfpathclose%
\pgfusepath{fill}%
\end{pgfscope}%
\begin{pgfscope}%
\pgfpathrectangle{\pgfqpoint{5.800000in}{0.720000in}}{\pgfqpoint{1.400000in}{4.620000in}}%
\pgfusepath{clip}%
\pgfsetbuttcap%
\pgfsetmiterjoin%
\definecolor{currentfill}{rgb}{0.121569,0.466667,0.705882}%
\pgfsetfillcolor{currentfill}%
\pgfsetlinewidth{0.000000pt}%
\definecolor{currentstroke}{rgb}{0.000000,0.000000,0.000000}%
\pgfsetstrokecolor{currentstroke}%
\pgfsetstrokeopacity{0.000000}%
\pgfsetdash{}{0pt}%
\pgfpathmoveto{\pgfqpoint{5.800000in}{159.706107in}}%
\pgfpathlineto{\pgfqpoint{5.800000in}{159.706107in}}%
\pgfpathlineto{\pgfqpoint{5.800000in}{159.808944in}}%
\pgfpathlineto{\pgfqpoint{5.800000in}{159.808944in}}%
\pgfpathclose%
\pgfusepath{fill}%
\end{pgfscope}%
\begin{pgfscope}%
\pgfpathrectangle{\pgfqpoint{5.800000in}{0.720000in}}{\pgfqpoint{1.400000in}{4.620000in}}%
\pgfusepath{clip}%
\pgfsetbuttcap%
\pgfsetmiterjoin%
\definecolor{currentfill}{rgb}{0.121569,0.466667,0.705882}%
\pgfsetfillcolor{currentfill}%
\pgfsetlinewidth{0.000000pt}%
\definecolor{currentstroke}{rgb}{0.000000,0.000000,0.000000}%
\pgfsetstrokecolor{currentstroke}%
\pgfsetstrokeopacity{0.000000}%
\pgfsetdash{}{0pt}%
\pgfpathmoveto{\pgfqpoint{5.800000in}{159.808944in}}%
\pgfpathlineto{\pgfqpoint{5.800000in}{159.808944in}}%
\pgfpathlineto{\pgfqpoint{5.800000in}{159.911781in}}%
\pgfpathlineto{\pgfqpoint{5.800000in}{159.911781in}}%
\pgfpathclose%
\pgfusepath{fill}%
\end{pgfscope}%
\begin{pgfscope}%
\pgfpathrectangle{\pgfqpoint{5.800000in}{0.720000in}}{\pgfqpoint{1.400000in}{4.620000in}}%
\pgfusepath{clip}%
\pgfsetbuttcap%
\pgfsetmiterjoin%
\definecolor{currentfill}{rgb}{0.121569,0.466667,0.705882}%
\pgfsetfillcolor{currentfill}%
\pgfsetlinewidth{0.000000pt}%
\definecolor{currentstroke}{rgb}{0.000000,0.000000,0.000000}%
\pgfsetstrokecolor{currentstroke}%
\pgfsetstrokeopacity{0.000000}%
\pgfsetdash{}{0pt}%
\pgfpathmoveto{\pgfqpoint{5.800000in}{159.911781in}}%
\pgfpathlineto{\pgfqpoint{5.800000in}{159.911781in}}%
\pgfpathlineto{\pgfqpoint{5.800000in}{160.014618in}}%
\pgfpathlineto{\pgfqpoint{5.800000in}{160.014618in}}%
\pgfpathclose%
\pgfusepath{fill}%
\end{pgfscope}%
\begin{pgfscope}%
\pgfpathrectangle{\pgfqpoint{5.800000in}{0.720000in}}{\pgfqpoint{1.400000in}{4.620000in}}%
\pgfusepath{clip}%
\pgfsetbuttcap%
\pgfsetmiterjoin%
\definecolor{currentfill}{rgb}{0.121569,0.466667,0.705882}%
\pgfsetfillcolor{currentfill}%
\pgfsetlinewidth{0.000000pt}%
\definecolor{currentstroke}{rgb}{0.000000,0.000000,0.000000}%
\pgfsetstrokecolor{currentstroke}%
\pgfsetstrokeopacity{0.000000}%
\pgfsetdash{}{0pt}%
\pgfpathmoveto{\pgfqpoint{5.800000in}{160.014618in}}%
\pgfpathlineto{\pgfqpoint{5.800000in}{160.014618in}}%
\pgfpathlineto{\pgfqpoint{5.800000in}{160.117455in}}%
\pgfpathlineto{\pgfqpoint{5.800000in}{160.117455in}}%
\pgfpathclose%
\pgfusepath{fill}%
\end{pgfscope}%
\begin{pgfscope}%
\pgfpathrectangle{\pgfqpoint{5.800000in}{0.720000in}}{\pgfqpoint{1.400000in}{4.620000in}}%
\pgfusepath{clip}%
\pgfsetbuttcap%
\pgfsetmiterjoin%
\definecolor{currentfill}{rgb}{0.121569,0.466667,0.705882}%
\pgfsetfillcolor{currentfill}%
\pgfsetlinewidth{0.000000pt}%
\definecolor{currentstroke}{rgb}{0.000000,0.000000,0.000000}%
\pgfsetstrokecolor{currentstroke}%
\pgfsetstrokeopacity{0.000000}%
\pgfsetdash{}{0pt}%
\pgfpathmoveto{\pgfqpoint{5.800000in}{160.117455in}}%
\pgfpathlineto{\pgfqpoint{5.800000in}{160.117455in}}%
\pgfpathlineto{\pgfqpoint{5.800000in}{160.220292in}}%
\pgfpathlineto{\pgfqpoint{5.800000in}{160.220292in}}%
\pgfpathclose%
\pgfusepath{fill}%
\end{pgfscope}%
\begin{pgfscope}%
\pgfpathrectangle{\pgfqpoint{5.800000in}{0.720000in}}{\pgfqpoint{1.400000in}{4.620000in}}%
\pgfusepath{clip}%
\pgfsetbuttcap%
\pgfsetmiterjoin%
\definecolor{currentfill}{rgb}{0.121569,0.466667,0.705882}%
\pgfsetfillcolor{currentfill}%
\pgfsetlinewidth{0.000000pt}%
\definecolor{currentstroke}{rgb}{0.000000,0.000000,0.000000}%
\pgfsetstrokecolor{currentstroke}%
\pgfsetstrokeopacity{0.000000}%
\pgfsetdash{}{0pt}%
\pgfpathmoveto{\pgfqpoint{5.800000in}{160.220292in}}%
\pgfpathlineto{\pgfqpoint{5.800000in}{160.220292in}}%
\pgfpathlineto{\pgfqpoint{5.800000in}{160.323129in}}%
\pgfpathlineto{\pgfqpoint{5.800000in}{160.323129in}}%
\pgfpathclose%
\pgfusepath{fill}%
\end{pgfscope}%
\begin{pgfscope}%
\pgfpathrectangle{\pgfqpoint{5.800000in}{0.720000in}}{\pgfqpoint{1.400000in}{4.620000in}}%
\pgfusepath{clip}%
\pgfsetbuttcap%
\pgfsetmiterjoin%
\definecolor{currentfill}{rgb}{0.121569,0.466667,0.705882}%
\pgfsetfillcolor{currentfill}%
\pgfsetlinewidth{0.000000pt}%
\definecolor{currentstroke}{rgb}{0.000000,0.000000,0.000000}%
\pgfsetstrokecolor{currentstroke}%
\pgfsetstrokeopacity{0.000000}%
\pgfsetdash{}{0pt}%
\pgfpathmoveto{\pgfqpoint{5.800000in}{160.323129in}}%
\pgfpathlineto{\pgfqpoint{5.800000in}{160.323129in}}%
\pgfpathlineto{\pgfqpoint{5.800000in}{160.425966in}}%
\pgfpathlineto{\pgfqpoint{5.800000in}{160.425966in}}%
\pgfpathclose%
\pgfusepath{fill}%
\end{pgfscope}%
\begin{pgfscope}%
\pgfpathrectangle{\pgfqpoint{5.800000in}{0.720000in}}{\pgfqpoint{1.400000in}{4.620000in}}%
\pgfusepath{clip}%
\pgfsetbuttcap%
\pgfsetmiterjoin%
\definecolor{currentfill}{rgb}{0.121569,0.466667,0.705882}%
\pgfsetfillcolor{currentfill}%
\pgfsetlinewidth{0.000000pt}%
\definecolor{currentstroke}{rgb}{0.000000,0.000000,0.000000}%
\pgfsetstrokecolor{currentstroke}%
\pgfsetstrokeopacity{0.000000}%
\pgfsetdash{}{0pt}%
\pgfpathmoveto{\pgfqpoint{5.800000in}{160.425966in}}%
\pgfpathlineto{\pgfqpoint{5.800000in}{160.425966in}}%
\pgfpathlineto{\pgfqpoint{5.800000in}{160.528803in}}%
\pgfpathlineto{\pgfqpoint{5.800000in}{160.528803in}}%
\pgfpathclose%
\pgfusepath{fill}%
\end{pgfscope}%
\begin{pgfscope}%
\pgfpathrectangle{\pgfqpoint{5.800000in}{0.720000in}}{\pgfqpoint{1.400000in}{4.620000in}}%
\pgfusepath{clip}%
\pgfsetbuttcap%
\pgfsetmiterjoin%
\definecolor{currentfill}{rgb}{0.121569,0.466667,0.705882}%
\pgfsetfillcolor{currentfill}%
\pgfsetlinewidth{0.000000pt}%
\definecolor{currentstroke}{rgb}{0.000000,0.000000,0.000000}%
\pgfsetstrokecolor{currentstroke}%
\pgfsetstrokeopacity{0.000000}%
\pgfsetdash{}{0pt}%
\pgfpathmoveto{\pgfqpoint{5.800000in}{160.528803in}}%
\pgfpathlineto{\pgfqpoint{5.800000in}{160.528803in}}%
\pgfpathlineto{\pgfqpoint{5.800000in}{160.631640in}}%
\pgfpathlineto{\pgfqpoint{5.800000in}{160.631640in}}%
\pgfpathclose%
\pgfusepath{fill}%
\end{pgfscope}%
\begin{pgfscope}%
\pgfpathrectangle{\pgfqpoint{5.800000in}{0.720000in}}{\pgfqpoint{1.400000in}{4.620000in}}%
\pgfusepath{clip}%
\pgfsetbuttcap%
\pgfsetmiterjoin%
\definecolor{currentfill}{rgb}{0.121569,0.466667,0.705882}%
\pgfsetfillcolor{currentfill}%
\pgfsetlinewidth{0.000000pt}%
\definecolor{currentstroke}{rgb}{0.000000,0.000000,0.000000}%
\pgfsetstrokecolor{currentstroke}%
\pgfsetstrokeopacity{0.000000}%
\pgfsetdash{}{0pt}%
\pgfpathmoveto{\pgfqpoint{5.800000in}{160.631640in}}%
\pgfpathlineto{\pgfqpoint{5.800000in}{160.631640in}}%
\pgfpathlineto{\pgfqpoint{5.800000in}{160.734477in}}%
\pgfpathlineto{\pgfqpoint{5.800000in}{160.734477in}}%
\pgfpathclose%
\pgfusepath{fill}%
\end{pgfscope}%
\begin{pgfscope}%
\pgfpathrectangle{\pgfqpoint{5.800000in}{0.720000in}}{\pgfqpoint{1.400000in}{4.620000in}}%
\pgfusepath{clip}%
\pgfsetbuttcap%
\pgfsetmiterjoin%
\definecolor{currentfill}{rgb}{0.121569,0.466667,0.705882}%
\pgfsetfillcolor{currentfill}%
\pgfsetlinewidth{0.000000pt}%
\definecolor{currentstroke}{rgb}{0.000000,0.000000,0.000000}%
\pgfsetstrokecolor{currentstroke}%
\pgfsetstrokeopacity{0.000000}%
\pgfsetdash{}{0pt}%
\pgfpathmoveto{\pgfqpoint{5.800000in}{160.734477in}}%
\pgfpathlineto{\pgfqpoint{5.800000in}{160.734477in}}%
\pgfpathlineto{\pgfqpoint{5.800000in}{160.837315in}}%
\pgfpathlineto{\pgfqpoint{5.800000in}{160.837315in}}%
\pgfpathclose%
\pgfusepath{fill}%
\end{pgfscope}%
\begin{pgfscope}%
\pgfpathrectangle{\pgfqpoint{5.800000in}{0.720000in}}{\pgfqpoint{1.400000in}{4.620000in}}%
\pgfusepath{clip}%
\pgfsetbuttcap%
\pgfsetmiterjoin%
\definecolor{currentfill}{rgb}{0.121569,0.466667,0.705882}%
\pgfsetfillcolor{currentfill}%
\pgfsetlinewidth{0.000000pt}%
\definecolor{currentstroke}{rgb}{0.000000,0.000000,0.000000}%
\pgfsetstrokecolor{currentstroke}%
\pgfsetstrokeopacity{0.000000}%
\pgfsetdash{}{0pt}%
\pgfpathmoveto{\pgfqpoint{5.800000in}{160.837315in}}%
\pgfpathlineto{\pgfqpoint{5.800000in}{160.837315in}}%
\pgfpathlineto{\pgfqpoint{5.800000in}{160.940152in}}%
\pgfpathlineto{\pgfqpoint{5.800000in}{160.940152in}}%
\pgfpathclose%
\pgfusepath{fill}%
\end{pgfscope}%
\begin{pgfscope}%
\pgfpathrectangle{\pgfqpoint{5.800000in}{0.720000in}}{\pgfqpoint{1.400000in}{4.620000in}}%
\pgfusepath{clip}%
\pgfsetbuttcap%
\pgfsetmiterjoin%
\definecolor{currentfill}{rgb}{0.121569,0.466667,0.705882}%
\pgfsetfillcolor{currentfill}%
\pgfsetlinewidth{0.000000pt}%
\definecolor{currentstroke}{rgb}{0.000000,0.000000,0.000000}%
\pgfsetstrokecolor{currentstroke}%
\pgfsetstrokeopacity{0.000000}%
\pgfsetdash{}{0pt}%
\pgfpathmoveto{\pgfqpoint{5.800000in}{160.940152in}}%
\pgfpathlineto{\pgfqpoint{5.800000in}{160.940152in}}%
\pgfpathlineto{\pgfqpoint{5.800000in}{161.042989in}}%
\pgfpathlineto{\pgfqpoint{5.800000in}{161.042989in}}%
\pgfpathclose%
\pgfusepath{fill}%
\end{pgfscope}%
\begin{pgfscope}%
\pgfpathrectangle{\pgfqpoint{5.800000in}{0.720000in}}{\pgfqpoint{1.400000in}{4.620000in}}%
\pgfusepath{clip}%
\pgfsetbuttcap%
\pgfsetmiterjoin%
\definecolor{currentfill}{rgb}{0.121569,0.466667,0.705882}%
\pgfsetfillcolor{currentfill}%
\pgfsetlinewidth{0.000000pt}%
\definecolor{currentstroke}{rgb}{0.000000,0.000000,0.000000}%
\pgfsetstrokecolor{currentstroke}%
\pgfsetstrokeopacity{0.000000}%
\pgfsetdash{}{0pt}%
\pgfpathmoveto{\pgfqpoint{5.800000in}{161.042989in}}%
\pgfpathlineto{\pgfqpoint{5.800000in}{161.042989in}}%
\pgfpathlineto{\pgfqpoint{5.800000in}{161.145826in}}%
\pgfpathlineto{\pgfqpoint{5.800000in}{161.145826in}}%
\pgfpathclose%
\pgfusepath{fill}%
\end{pgfscope}%
\begin{pgfscope}%
\pgfpathrectangle{\pgfqpoint{5.800000in}{0.720000in}}{\pgfqpoint{1.400000in}{4.620000in}}%
\pgfusepath{clip}%
\pgfsetbuttcap%
\pgfsetmiterjoin%
\definecolor{currentfill}{rgb}{0.121569,0.466667,0.705882}%
\pgfsetfillcolor{currentfill}%
\pgfsetlinewidth{0.000000pt}%
\definecolor{currentstroke}{rgb}{0.000000,0.000000,0.000000}%
\pgfsetstrokecolor{currentstroke}%
\pgfsetstrokeopacity{0.000000}%
\pgfsetdash{}{0pt}%
\pgfpathmoveto{\pgfqpoint{5.800000in}{161.145826in}}%
\pgfpathlineto{\pgfqpoint{5.800000in}{161.145826in}}%
\pgfpathlineto{\pgfqpoint{5.800000in}{161.248663in}}%
\pgfpathlineto{\pgfqpoint{5.800000in}{161.248663in}}%
\pgfpathclose%
\pgfusepath{fill}%
\end{pgfscope}%
\begin{pgfscope}%
\pgfpathrectangle{\pgfqpoint{5.800000in}{0.720000in}}{\pgfqpoint{1.400000in}{4.620000in}}%
\pgfusepath{clip}%
\pgfsetbuttcap%
\pgfsetmiterjoin%
\definecolor{currentfill}{rgb}{0.121569,0.466667,0.705882}%
\pgfsetfillcolor{currentfill}%
\pgfsetlinewidth{0.000000pt}%
\definecolor{currentstroke}{rgb}{0.000000,0.000000,0.000000}%
\pgfsetstrokecolor{currentstroke}%
\pgfsetstrokeopacity{0.000000}%
\pgfsetdash{}{0pt}%
\pgfpathmoveto{\pgfqpoint{5.800000in}{161.248663in}}%
\pgfpathlineto{\pgfqpoint{5.800000in}{161.248663in}}%
\pgfpathlineto{\pgfqpoint{5.800000in}{161.351500in}}%
\pgfpathlineto{\pgfqpoint{5.800000in}{161.351500in}}%
\pgfpathclose%
\pgfusepath{fill}%
\end{pgfscope}%
\begin{pgfscope}%
\pgfpathrectangle{\pgfqpoint{5.800000in}{0.720000in}}{\pgfqpoint{1.400000in}{4.620000in}}%
\pgfusepath{clip}%
\pgfsetbuttcap%
\pgfsetmiterjoin%
\definecolor{currentfill}{rgb}{0.121569,0.466667,0.705882}%
\pgfsetfillcolor{currentfill}%
\pgfsetlinewidth{0.000000pt}%
\definecolor{currentstroke}{rgb}{0.000000,0.000000,0.000000}%
\pgfsetstrokecolor{currentstroke}%
\pgfsetstrokeopacity{0.000000}%
\pgfsetdash{}{0pt}%
\pgfpathmoveto{\pgfqpoint{5.800000in}{161.351500in}}%
\pgfpathlineto{\pgfqpoint{5.800000in}{161.351500in}}%
\pgfpathlineto{\pgfqpoint{5.800000in}{161.454337in}}%
\pgfpathlineto{\pgfqpoint{5.800000in}{161.454337in}}%
\pgfpathclose%
\pgfusepath{fill}%
\end{pgfscope}%
\begin{pgfscope}%
\pgfpathrectangle{\pgfqpoint{5.800000in}{0.720000in}}{\pgfqpoint{1.400000in}{4.620000in}}%
\pgfusepath{clip}%
\pgfsetbuttcap%
\pgfsetmiterjoin%
\definecolor{currentfill}{rgb}{0.121569,0.466667,0.705882}%
\pgfsetfillcolor{currentfill}%
\pgfsetlinewidth{0.000000pt}%
\definecolor{currentstroke}{rgb}{0.000000,0.000000,0.000000}%
\pgfsetstrokecolor{currentstroke}%
\pgfsetstrokeopacity{0.000000}%
\pgfsetdash{}{0pt}%
\pgfpathmoveto{\pgfqpoint{5.800000in}{161.454337in}}%
\pgfpathlineto{\pgfqpoint{5.800000in}{161.454337in}}%
\pgfpathlineto{\pgfqpoint{5.800000in}{161.557174in}}%
\pgfpathlineto{\pgfqpoint{5.800000in}{161.557174in}}%
\pgfpathclose%
\pgfusepath{fill}%
\end{pgfscope}%
\begin{pgfscope}%
\pgfpathrectangle{\pgfqpoint{5.800000in}{0.720000in}}{\pgfqpoint{1.400000in}{4.620000in}}%
\pgfusepath{clip}%
\pgfsetbuttcap%
\pgfsetmiterjoin%
\definecolor{currentfill}{rgb}{0.121569,0.466667,0.705882}%
\pgfsetfillcolor{currentfill}%
\pgfsetlinewidth{0.000000pt}%
\definecolor{currentstroke}{rgb}{0.000000,0.000000,0.000000}%
\pgfsetstrokecolor{currentstroke}%
\pgfsetstrokeopacity{0.000000}%
\pgfsetdash{}{0pt}%
\pgfpathmoveto{\pgfqpoint{5.800000in}{161.557174in}}%
\pgfpathlineto{\pgfqpoint{5.800000in}{161.557174in}}%
\pgfpathlineto{\pgfqpoint{5.800000in}{161.660011in}}%
\pgfpathlineto{\pgfqpoint{5.800000in}{161.660011in}}%
\pgfpathclose%
\pgfusepath{fill}%
\end{pgfscope}%
\begin{pgfscope}%
\pgfpathrectangle{\pgfqpoint{5.800000in}{0.720000in}}{\pgfqpoint{1.400000in}{4.620000in}}%
\pgfusepath{clip}%
\pgfsetbuttcap%
\pgfsetmiterjoin%
\definecolor{currentfill}{rgb}{0.121569,0.466667,0.705882}%
\pgfsetfillcolor{currentfill}%
\pgfsetlinewidth{0.000000pt}%
\definecolor{currentstroke}{rgb}{0.000000,0.000000,0.000000}%
\pgfsetstrokecolor{currentstroke}%
\pgfsetstrokeopacity{0.000000}%
\pgfsetdash{}{0pt}%
\pgfpathmoveto{\pgfqpoint{5.800000in}{161.660011in}}%
\pgfpathlineto{\pgfqpoint{5.800000in}{161.660011in}}%
\pgfpathlineto{\pgfqpoint{5.800000in}{161.762848in}}%
\pgfpathlineto{\pgfqpoint{5.800000in}{161.762848in}}%
\pgfpathclose%
\pgfusepath{fill}%
\end{pgfscope}%
\begin{pgfscope}%
\pgfpathrectangle{\pgfqpoint{5.800000in}{0.720000in}}{\pgfqpoint{1.400000in}{4.620000in}}%
\pgfusepath{clip}%
\pgfsetbuttcap%
\pgfsetmiterjoin%
\definecolor{currentfill}{rgb}{0.121569,0.466667,0.705882}%
\pgfsetfillcolor{currentfill}%
\pgfsetlinewidth{0.000000pt}%
\definecolor{currentstroke}{rgb}{0.000000,0.000000,0.000000}%
\pgfsetstrokecolor{currentstroke}%
\pgfsetstrokeopacity{0.000000}%
\pgfsetdash{}{0pt}%
\pgfpathmoveto{\pgfqpoint{5.800000in}{161.762848in}}%
\pgfpathlineto{\pgfqpoint{5.800000in}{161.762848in}}%
\pgfpathlineto{\pgfqpoint{5.800000in}{161.865685in}}%
\pgfpathlineto{\pgfqpoint{5.800000in}{161.865685in}}%
\pgfpathclose%
\pgfusepath{fill}%
\end{pgfscope}%
\begin{pgfscope}%
\pgfpathrectangle{\pgfqpoint{5.800000in}{0.720000in}}{\pgfqpoint{1.400000in}{4.620000in}}%
\pgfusepath{clip}%
\pgfsetbuttcap%
\pgfsetmiterjoin%
\definecolor{currentfill}{rgb}{0.121569,0.466667,0.705882}%
\pgfsetfillcolor{currentfill}%
\pgfsetlinewidth{0.000000pt}%
\definecolor{currentstroke}{rgb}{0.000000,0.000000,0.000000}%
\pgfsetstrokecolor{currentstroke}%
\pgfsetstrokeopacity{0.000000}%
\pgfsetdash{}{0pt}%
\pgfpathmoveto{\pgfqpoint{5.800000in}{161.865685in}}%
\pgfpathlineto{\pgfqpoint{5.800000in}{161.865685in}}%
\pgfpathlineto{\pgfqpoint{5.800000in}{161.968522in}}%
\pgfpathlineto{\pgfqpoint{5.800000in}{161.968522in}}%
\pgfpathclose%
\pgfusepath{fill}%
\end{pgfscope}%
\begin{pgfscope}%
\pgfpathrectangle{\pgfqpoint{5.800000in}{0.720000in}}{\pgfqpoint{1.400000in}{4.620000in}}%
\pgfusepath{clip}%
\pgfsetbuttcap%
\pgfsetmiterjoin%
\definecolor{currentfill}{rgb}{0.121569,0.466667,0.705882}%
\pgfsetfillcolor{currentfill}%
\pgfsetlinewidth{0.000000pt}%
\definecolor{currentstroke}{rgb}{0.000000,0.000000,0.000000}%
\pgfsetstrokecolor{currentstroke}%
\pgfsetstrokeopacity{0.000000}%
\pgfsetdash{}{0pt}%
\pgfpathmoveto{\pgfqpoint{5.800000in}{161.968522in}}%
\pgfpathlineto{\pgfqpoint{5.800000in}{161.968522in}}%
\pgfpathlineto{\pgfqpoint{5.800000in}{162.071359in}}%
\pgfpathlineto{\pgfqpoint{5.800000in}{162.071359in}}%
\pgfpathclose%
\pgfusepath{fill}%
\end{pgfscope}%
\begin{pgfscope}%
\pgfpathrectangle{\pgfqpoint{5.800000in}{0.720000in}}{\pgfqpoint{1.400000in}{4.620000in}}%
\pgfusepath{clip}%
\pgfsetbuttcap%
\pgfsetmiterjoin%
\definecolor{currentfill}{rgb}{0.121569,0.466667,0.705882}%
\pgfsetfillcolor{currentfill}%
\pgfsetlinewidth{0.000000pt}%
\definecolor{currentstroke}{rgb}{0.000000,0.000000,0.000000}%
\pgfsetstrokecolor{currentstroke}%
\pgfsetstrokeopacity{0.000000}%
\pgfsetdash{}{0pt}%
\pgfpathmoveto{\pgfqpoint{5.800000in}{162.071359in}}%
\pgfpathlineto{\pgfqpoint{5.800000in}{162.071359in}}%
\pgfpathlineto{\pgfqpoint{5.800000in}{162.174196in}}%
\pgfpathlineto{\pgfqpoint{5.800000in}{162.174196in}}%
\pgfpathclose%
\pgfusepath{fill}%
\end{pgfscope}%
\begin{pgfscope}%
\pgfpathrectangle{\pgfqpoint{5.800000in}{0.720000in}}{\pgfqpoint{1.400000in}{4.620000in}}%
\pgfusepath{clip}%
\pgfsetbuttcap%
\pgfsetmiterjoin%
\definecolor{currentfill}{rgb}{0.121569,0.466667,0.705882}%
\pgfsetfillcolor{currentfill}%
\pgfsetlinewidth{0.000000pt}%
\definecolor{currentstroke}{rgb}{0.000000,0.000000,0.000000}%
\pgfsetstrokecolor{currentstroke}%
\pgfsetstrokeopacity{0.000000}%
\pgfsetdash{}{0pt}%
\pgfpathmoveto{\pgfqpoint{5.800000in}{162.174196in}}%
\pgfpathlineto{\pgfqpoint{5.800000in}{162.174196in}}%
\pgfpathlineto{\pgfqpoint{5.800000in}{162.277033in}}%
\pgfpathlineto{\pgfqpoint{5.800000in}{162.277033in}}%
\pgfpathclose%
\pgfusepath{fill}%
\end{pgfscope}%
\begin{pgfscope}%
\pgfpathrectangle{\pgfqpoint{5.800000in}{0.720000in}}{\pgfqpoint{1.400000in}{4.620000in}}%
\pgfusepath{clip}%
\pgfsetbuttcap%
\pgfsetmiterjoin%
\definecolor{currentfill}{rgb}{0.121569,0.466667,0.705882}%
\pgfsetfillcolor{currentfill}%
\pgfsetlinewidth{0.000000pt}%
\definecolor{currentstroke}{rgb}{0.000000,0.000000,0.000000}%
\pgfsetstrokecolor{currentstroke}%
\pgfsetstrokeopacity{0.000000}%
\pgfsetdash{}{0pt}%
\pgfpathmoveto{\pgfqpoint{5.800000in}{162.277033in}}%
\pgfpathlineto{\pgfqpoint{5.800000in}{162.277033in}}%
\pgfpathlineto{\pgfqpoint{5.800000in}{162.379871in}}%
\pgfpathlineto{\pgfqpoint{5.800000in}{162.379871in}}%
\pgfpathclose%
\pgfusepath{fill}%
\end{pgfscope}%
\begin{pgfscope}%
\pgfpathrectangle{\pgfqpoint{5.800000in}{0.720000in}}{\pgfqpoint{1.400000in}{4.620000in}}%
\pgfusepath{clip}%
\pgfsetbuttcap%
\pgfsetmiterjoin%
\definecolor{currentfill}{rgb}{0.121569,0.466667,0.705882}%
\pgfsetfillcolor{currentfill}%
\pgfsetlinewidth{0.000000pt}%
\definecolor{currentstroke}{rgb}{0.000000,0.000000,0.000000}%
\pgfsetstrokecolor{currentstroke}%
\pgfsetstrokeopacity{0.000000}%
\pgfsetdash{}{0pt}%
\pgfpathmoveto{\pgfqpoint{5.800000in}{162.379871in}}%
\pgfpathlineto{\pgfqpoint{5.800000in}{162.379871in}}%
\pgfpathlineto{\pgfqpoint{5.800000in}{162.482708in}}%
\pgfpathlineto{\pgfqpoint{5.800000in}{162.482708in}}%
\pgfpathclose%
\pgfusepath{fill}%
\end{pgfscope}%
\begin{pgfscope}%
\pgfpathrectangle{\pgfqpoint{5.800000in}{0.720000in}}{\pgfqpoint{1.400000in}{4.620000in}}%
\pgfusepath{clip}%
\pgfsetbuttcap%
\pgfsetmiterjoin%
\definecolor{currentfill}{rgb}{0.121569,0.466667,0.705882}%
\pgfsetfillcolor{currentfill}%
\pgfsetlinewidth{0.000000pt}%
\definecolor{currentstroke}{rgb}{0.000000,0.000000,0.000000}%
\pgfsetstrokecolor{currentstroke}%
\pgfsetstrokeopacity{0.000000}%
\pgfsetdash{}{0pt}%
\pgfpathmoveto{\pgfqpoint{5.800000in}{162.482708in}}%
\pgfpathlineto{\pgfqpoint{5.800000in}{162.482708in}}%
\pgfpathlineto{\pgfqpoint{5.800000in}{162.585545in}}%
\pgfpathlineto{\pgfqpoint{5.800000in}{162.585545in}}%
\pgfpathclose%
\pgfusepath{fill}%
\end{pgfscope}%
\begin{pgfscope}%
\pgfpathrectangle{\pgfqpoint{5.800000in}{0.720000in}}{\pgfqpoint{1.400000in}{4.620000in}}%
\pgfusepath{clip}%
\pgfsetbuttcap%
\pgfsetmiterjoin%
\definecolor{currentfill}{rgb}{0.121569,0.466667,0.705882}%
\pgfsetfillcolor{currentfill}%
\pgfsetlinewidth{0.000000pt}%
\definecolor{currentstroke}{rgb}{0.000000,0.000000,0.000000}%
\pgfsetstrokecolor{currentstroke}%
\pgfsetstrokeopacity{0.000000}%
\pgfsetdash{}{0pt}%
\pgfpathmoveto{\pgfqpoint{5.800000in}{162.585545in}}%
\pgfpathlineto{\pgfqpoint{5.800000in}{162.585545in}}%
\pgfpathlineto{\pgfqpoint{5.800000in}{162.688382in}}%
\pgfpathlineto{\pgfqpoint{5.800000in}{162.688382in}}%
\pgfpathclose%
\pgfusepath{fill}%
\end{pgfscope}%
\begin{pgfscope}%
\pgfpathrectangle{\pgfqpoint{5.800000in}{0.720000in}}{\pgfqpoint{1.400000in}{4.620000in}}%
\pgfusepath{clip}%
\pgfsetbuttcap%
\pgfsetmiterjoin%
\definecolor{currentfill}{rgb}{0.121569,0.466667,0.705882}%
\pgfsetfillcolor{currentfill}%
\pgfsetlinewidth{0.000000pt}%
\definecolor{currentstroke}{rgb}{0.000000,0.000000,0.000000}%
\pgfsetstrokecolor{currentstroke}%
\pgfsetstrokeopacity{0.000000}%
\pgfsetdash{}{0pt}%
\pgfpathmoveto{\pgfqpoint{5.800000in}{162.688382in}}%
\pgfpathlineto{\pgfqpoint{5.800000in}{162.688382in}}%
\pgfpathlineto{\pgfqpoint{5.800000in}{162.791219in}}%
\pgfpathlineto{\pgfqpoint{5.800000in}{162.791219in}}%
\pgfpathclose%
\pgfusepath{fill}%
\end{pgfscope}%
\begin{pgfscope}%
\pgfpathrectangle{\pgfqpoint{5.800000in}{0.720000in}}{\pgfqpoint{1.400000in}{4.620000in}}%
\pgfusepath{clip}%
\pgfsetbuttcap%
\pgfsetmiterjoin%
\definecolor{currentfill}{rgb}{0.121569,0.466667,0.705882}%
\pgfsetfillcolor{currentfill}%
\pgfsetlinewidth{0.000000pt}%
\definecolor{currentstroke}{rgb}{0.000000,0.000000,0.000000}%
\pgfsetstrokecolor{currentstroke}%
\pgfsetstrokeopacity{0.000000}%
\pgfsetdash{}{0pt}%
\pgfpathmoveto{\pgfqpoint{5.800000in}{162.791219in}}%
\pgfpathlineto{\pgfqpoint{5.800000in}{162.791219in}}%
\pgfpathlineto{\pgfqpoint{5.800000in}{162.894056in}}%
\pgfpathlineto{\pgfqpoint{5.800000in}{162.894056in}}%
\pgfpathclose%
\pgfusepath{fill}%
\end{pgfscope}%
\begin{pgfscope}%
\pgfpathrectangle{\pgfqpoint{5.800000in}{0.720000in}}{\pgfqpoint{1.400000in}{4.620000in}}%
\pgfusepath{clip}%
\pgfsetbuttcap%
\pgfsetmiterjoin%
\definecolor{currentfill}{rgb}{0.121569,0.466667,0.705882}%
\pgfsetfillcolor{currentfill}%
\pgfsetlinewidth{0.000000pt}%
\definecolor{currentstroke}{rgb}{0.000000,0.000000,0.000000}%
\pgfsetstrokecolor{currentstroke}%
\pgfsetstrokeopacity{0.000000}%
\pgfsetdash{}{0pt}%
\pgfpathmoveto{\pgfqpoint{5.800000in}{162.894056in}}%
\pgfpathlineto{\pgfqpoint{5.800000in}{162.894056in}}%
\pgfpathlineto{\pgfqpoint{5.800000in}{162.996893in}}%
\pgfpathlineto{\pgfqpoint{5.800000in}{162.996893in}}%
\pgfpathclose%
\pgfusepath{fill}%
\end{pgfscope}%
\begin{pgfscope}%
\pgfpathrectangle{\pgfqpoint{5.800000in}{0.720000in}}{\pgfqpoint{1.400000in}{4.620000in}}%
\pgfusepath{clip}%
\pgfsetbuttcap%
\pgfsetmiterjoin%
\definecolor{currentfill}{rgb}{0.121569,0.466667,0.705882}%
\pgfsetfillcolor{currentfill}%
\pgfsetlinewidth{0.000000pt}%
\definecolor{currentstroke}{rgb}{0.000000,0.000000,0.000000}%
\pgfsetstrokecolor{currentstroke}%
\pgfsetstrokeopacity{0.000000}%
\pgfsetdash{}{0pt}%
\pgfpathmoveto{\pgfqpoint{5.800000in}{162.996893in}}%
\pgfpathlineto{\pgfqpoint{5.800000in}{162.996893in}}%
\pgfpathlineto{\pgfqpoint{5.800000in}{163.099730in}}%
\pgfpathlineto{\pgfqpoint{5.800000in}{163.099730in}}%
\pgfpathclose%
\pgfusepath{fill}%
\end{pgfscope}%
\begin{pgfscope}%
\pgfpathrectangle{\pgfqpoint{5.800000in}{0.720000in}}{\pgfqpoint{1.400000in}{4.620000in}}%
\pgfusepath{clip}%
\pgfsetbuttcap%
\pgfsetmiterjoin%
\definecolor{currentfill}{rgb}{0.121569,0.466667,0.705882}%
\pgfsetfillcolor{currentfill}%
\pgfsetlinewidth{0.000000pt}%
\definecolor{currentstroke}{rgb}{0.000000,0.000000,0.000000}%
\pgfsetstrokecolor{currentstroke}%
\pgfsetstrokeopacity{0.000000}%
\pgfsetdash{}{0pt}%
\pgfpathmoveto{\pgfqpoint{5.800000in}{163.099730in}}%
\pgfpathlineto{\pgfqpoint{5.800000in}{163.099730in}}%
\pgfpathlineto{\pgfqpoint{5.800000in}{163.202567in}}%
\pgfpathlineto{\pgfqpoint{5.800000in}{163.202567in}}%
\pgfpathclose%
\pgfusepath{fill}%
\end{pgfscope}%
\begin{pgfscope}%
\pgfpathrectangle{\pgfqpoint{5.800000in}{0.720000in}}{\pgfqpoint{1.400000in}{4.620000in}}%
\pgfusepath{clip}%
\pgfsetbuttcap%
\pgfsetmiterjoin%
\definecolor{currentfill}{rgb}{0.121569,0.466667,0.705882}%
\pgfsetfillcolor{currentfill}%
\pgfsetlinewidth{0.000000pt}%
\definecolor{currentstroke}{rgb}{0.000000,0.000000,0.000000}%
\pgfsetstrokecolor{currentstroke}%
\pgfsetstrokeopacity{0.000000}%
\pgfsetdash{}{0pt}%
\pgfpathmoveto{\pgfqpoint{5.800000in}{163.202567in}}%
\pgfpathlineto{\pgfqpoint{5.800000in}{163.202567in}}%
\pgfpathlineto{\pgfqpoint{5.800000in}{163.305404in}}%
\pgfpathlineto{\pgfqpoint{5.800000in}{163.305404in}}%
\pgfpathclose%
\pgfusepath{fill}%
\end{pgfscope}%
\begin{pgfscope}%
\pgfpathrectangle{\pgfqpoint{5.800000in}{0.720000in}}{\pgfqpoint{1.400000in}{4.620000in}}%
\pgfusepath{clip}%
\pgfsetbuttcap%
\pgfsetmiterjoin%
\definecolor{currentfill}{rgb}{0.121569,0.466667,0.705882}%
\pgfsetfillcolor{currentfill}%
\pgfsetlinewidth{0.000000pt}%
\definecolor{currentstroke}{rgb}{0.000000,0.000000,0.000000}%
\pgfsetstrokecolor{currentstroke}%
\pgfsetstrokeopacity{0.000000}%
\pgfsetdash{}{0pt}%
\pgfpathmoveto{\pgfqpoint{5.800000in}{163.305404in}}%
\pgfpathlineto{\pgfqpoint{5.800000in}{163.305404in}}%
\pgfpathlineto{\pgfqpoint{5.800000in}{163.408241in}}%
\pgfpathlineto{\pgfqpoint{5.800000in}{163.408241in}}%
\pgfpathclose%
\pgfusepath{fill}%
\end{pgfscope}%
\begin{pgfscope}%
\pgfpathrectangle{\pgfqpoint{5.800000in}{0.720000in}}{\pgfqpoint{1.400000in}{4.620000in}}%
\pgfusepath{clip}%
\pgfsetbuttcap%
\pgfsetmiterjoin%
\definecolor{currentfill}{rgb}{0.121569,0.466667,0.705882}%
\pgfsetfillcolor{currentfill}%
\pgfsetlinewidth{0.000000pt}%
\definecolor{currentstroke}{rgb}{0.000000,0.000000,0.000000}%
\pgfsetstrokecolor{currentstroke}%
\pgfsetstrokeopacity{0.000000}%
\pgfsetdash{}{0pt}%
\pgfpathmoveto{\pgfqpoint{5.800000in}{163.408241in}}%
\pgfpathlineto{\pgfqpoint{5.800000in}{163.408241in}}%
\pgfpathlineto{\pgfqpoint{5.800000in}{163.511078in}}%
\pgfpathlineto{\pgfqpoint{5.800000in}{163.511078in}}%
\pgfpathclose%
\pgfusepath{fill}%
\end{pgfscope}%
\begin{pgfscope}%
\pgfpathrectangle{\pgfqpoint{5.800000in}{0.720000in}}{\pgfqpoint{1.400000in}{4.620000in}}%
\pgfusepath{clip}%
\pgfsetbuttcap%
\pgfsetmiterjoin%
\definecolor{currentfill}{rgb}{0.121569,0.466667,0.705882}%
\pgfsetfillcolor{currentfill}%
\pgfsetlinewidth{0.000000pt}%
\definecolor{currentstroke}{rgb}{0.000000,0.000000,0.000000}%
\pgfsetstrokecolor{currentstroke}%
\pgfsetstrokeopacity{0.000000}%
\pgfsetdash{}{0pt}%
\pgfpathmoveto{\pgfqpoint{5.800000in}{163.511078in}}%
\pgfpathlineto{\pgfqpoint{5.800000in}{163.511078in}}%
\pgfpathlineto{\pgfqpoint{5.800000in}{163.613915in}}%
\pgfpathlineto{\pgfqpoint{5.800000in}{163.613915in}}%
\pgfpathclose%
\pgfusepath{fill}%
\end{pgfscope}%
\begin{pgfscope}%
\pgfpathrectangle{\pgfqpoint{5.800000in}{0.720000in}}{\pgfqpoint{1.400000in}{4.620000in}}%
\pgfusepath{clip}%
\pgfsetbuttcap%
\pgfsetmiterjoin%
\definecolor{currentfill}{rgb}{0.121569,0.466667,0.705882}%
\pgfsetfillcolor{currentfill}%
\pgfsetlinewidth{0.000000pt}%
\definecolor{currentstroke}{rgb}{0.000000,0.000000,0.000000}%
\pgfsetstrokecolor{currentstroke}%
\pgfsetstrokeopacity{0.000000}%
\pgfsetdash{}{0pt}%
\pgfpathmoveto{\pgfqpoint{5.800000in}{163.613915in}}%
\pgfpathlineto{\pgfqpoint{5.800000in}{163.613915in}}%
\pgfpathlineto{\pgfqpoint{5.800000in}{163.716752in}}%
\pgfpathlineto{\pgfqpoint{5.800000in}{163.716752in}}%
\pgfpathclose%
\pgfusepath{fill}%
\end{pgfscope}%
\begin{pgfscope}%
\pgfpathrectangle{\pgfqpoint{5.800000in}{0.720000in}}{\pgfqpoint{1.400000in}{4.620000in}}%
\pgfusepath{clip}%
\pgfsetbuttcap%
\pgfsetmiterjoin%
\definecolor{currentfill}{rgb}{0.121569,0.466667,0.705882}%
\pgfsetfillcolor{currentfill}%
\pgfsetlinewidth{0.000000pt}%
\definecolor{currentstroke}{rgb}{0.000000,0.000000,0.000000}%
\pgfsetstrokecolor{currentstroke}%
\pgfsetstrokeopacity{0.000000}%
\pgfsetdash{}{0pt}%
\pgfpathmoveto{\pgfqpoint{5.800000in}{163.716752in}}%
\pgfpathlineto{\pgfqpoint{5.800000in}{163.716752in}}%
\pgfpathlineto{\pgfqpoint{5.800000in}{163.819589in}}%
\pgfpathlineto{\pgfqpoint{5.800000in}{163.819589in}}%
\pgfpathclose%
\pgfusepath{fill}%
\end{pgfscope}%
\begin{pgfscope}%
\pgfpathrectangle{\pgfqpoint{5.800000in}{0.720000in}}{\pgfqpoint{1.400000in}{4.620000in}}%
\pgfusepath{clip}%
\pgfsetbuttcap%
\pgfsetmiterjoin%
\definecolor{currentfill}{rgb}{0.121569,0.466667,0.705882}%
\pgfsetfillcolor{currentfill}%
\pgfsetlinewidth{0.000000pt}%
\definecolor{currentstroke}{rgb}{0.000000,0.000000,0.000000}%
\pgfsetstrokecolor{currentstroke}%
\pgfsetstrokeopacity{0.000000}%
\pgfsetdash{}{0pt}%
\pgfpathmoveto{\pgfqpoint{5.800000in}{163.819589in}}%
\pgfpathlineto{\pgfqpoint{5.800000in}{163.819589in}}%
\pgfpathlineto{\pgfqpoint{5.800000in}{163.922427in}}%
\pgfpathlineto{\pgfqpoint{5.800000in}{163.922427in}}%
\pgfpathclose%
\pgfusepath{fill}%
\end{pgfscope}%
\begin{pgfscope}%
\pgfpathrectangle{\pgfqpoint{5.800000in}{0.720000in}}{\pgfqpoint{1.400000in}{4.620000in}}%
\pgfusepath{clip}%
\pgfsetbuttcap%
\pgfsetmiterjoin%
\definecolor{currentfill}{rgb}{0.121569,0.466667,0.705882}%
\pgfsetfillcolor{currentfill}%
\pgfsetlinewidth{0.000000pt}%
\definecolor{currentstroke}{rgb}{0.000000,0.000000,0.000000}%
\pgfsetstrokecolor{currentstroke}%
\pgfsetstrokeopacity{0.000000}%
\pgfsetdash{}{0pt}%
\pgfpathmoveto{\pgfqpoint{5.800000in}{163.922427in}}%
\pgfpathlineto{\pgfqpoint{5.800000in}{163.922427in}}%
\pgfpathlineto{\pgfqpoint{5.800000in}{164.025264in}}%
\pgfpathlineto{\pgfqpoint{5.800000in}{164.025264in}}%
\pgfpathclose%
\pgfusepath{fill}%
\end{pgfscope}%
\begin{pgfscope}%
\pgfpathrectangle{\pgfqpoint{5.800000in}{0.720000in}}{\pgfqpoint{1.400000in}{4.620000in}}%
\pgfusepath{clip}%
\pgfsetbuttcap%
\pgfsetmiterjoin%
\definecolor{currentfill}{rgb}{0.121569,0.466667,0.705882}%
\pgfsetfillcolor{currentfill}%
\pgfsetlinewidth{0.000000pt}%
\definecolor{currentstroke}{rgb}{0.000000,0.000000,0.000000}%
\pgfsetstrokecolor{currentstroke}%
\pgfsetstrokeopacity{0.000000}%
\pgfsetdash{}{0pt}%
\pgfpathmoveto{\pgfqpoint{5.800000in}{164.025264in}}%
\pgfpathlineto{\pgfqpoint{5.800000in}{164.025264in}}%
\pgfpathlineto{\pgfqpoint{5.800000in}{164.128101in}}%
\pgfpathlineto{\pgfqpoint{5.800000in}{164.128101in}}%
\pgfpathclose%
\pgfusepath{fill}%
\end{pgfscope}%
\begin{pgfscope}%
\pgfpathrectangle{\pgfqpoint{5.800000in}{0.720000in}}{\pgfqpoint{1.400000in}{4.620000in}}%
\pgfusepath{clip}%
\pgfsetbuttcap%
\pgfsetmiterjoin%
\definecolor{currentfill}{rgb}{0.121569,0.466667,0.705882}%
\pgfsetfillcolor{currentfill}%
\pgfsetlinewidth{0.000000pt}%
\definecolor{currentstroke}{rgb}{0.000000,0.000000,0.000000}%
\pgfsetstrokecolor{currentstroke}%
\pgfsetstrokeopacity{0.000000}%
\pgfsetdash{}{0pt}%
\pgfpathmoveto{\pgfqpoint{5.800000in}{164.128101in}}%
\pgfpathlineto{\pgfqpoint{5.800000in}{164.128101in}}%
\pgfpathlineto{\pgfqpoint{5.800000in}{164.230938in}}%
\pgfpathlineto{\pgfqpoint{5.800000in}{164.230938in}}%
\pgfpathclose%
\pgfusepath{fill}%
\end{pgfscope}%
\begin{pgfscope}%
\pgfpathrectangle{\pgfqpoint{5.800000in}{0.720000in}}{\pgfqpoint{1.400000in}{4.620000in}}%
\pgfusepath{clip}%
\pgfsetbuttcap%
\pgfsetmiterjoin%
\definecolor{currentfill}{rgb}{0.121569,0.466667,0.705882}%
\pgfsetfillcolor{currentfill}%
\pgfsetlinewidth{0.000000pt}%
\definecolor{currentstroke}{rgb}{0.000000,0.000000,0.000000}%
\pgfsetstrokecolor{currentstroke}%
\pgfsetstrokeopacity{0.000000}%
\pgfsetdash{}{0pt}%
\pgfpathmoveto{\pgfqpoint{5.800000in}{164.230938in}}%
\pgfpathlineto{\pgfqpoint{5.800000in}{164.230938in}}%
\pgfpathlineto{\pgfqpoint{5.800000in}{164.333775in}}%
\pgfpathlineto{\pgfqpoint{5.800000in}{164.333775in}}%
\pgfpathclose%
\pgfusepath{fill}%
\end{pgfscope}%
\begin{pgfscope}%
\pgfpathrectangle{\pgfqpoint{5.800000in}{0.720000in}}{\pgfqpoint{1.400000in}{4.620000in}}%
\pgfusepath{clip}%
\pgfsetbuttcap%
\pgfsetmiterjoin%
\definecolor{currentfill}{rgb}{0.121569,0.466667,0.705882}%
\pgfsetfillcolor{currentfill}%
\pgfsetlinewidth{0.000000pt}%
\definecolor{currentstroke}{rgb}{0.000000,0.000000,0.000000}%
\pgfsetstrokecolor{currentstroke}%
\pgfsetstrokeopacity{0.000000}%
\pgfsetdash{}{0pt}%
\pgfpathmoveto{\pgfqpoint{5.800000in}{164.333775in}}%
\pgfpathlineto{\pgfqpoint{5.800000in}{164.333775in}}%
\pgfpathlineto{\pgfqpoint{5.800000in}{164.436612in}}%
\pgfpathlineto{\pgfqpoint{5.800000in}{164.436612in}}%
\pgfpathclose%
\pgfusepath{fill}%
\end{pgfscope}%
\begin{pgfscope}%
\pgfpathrectangle{\pgfqpoint{5.800000in}{0.720000in}}{\pgfqpoint{1.400000in}{4.620000in}}%
\pgfusepath{clip}%
\pgfsetbuttcap%
\pgfsetmiterjoin%
\definecolor{currentfill}{rgb}{0.121569,0.466667,0.705882}%
\pgfsetfillcolor{currentfill}%
\pgfsetlinewidth{0.000000pt}%
\definecolor{currentstroke}{rgb}{0.000000,0.000000,0.000000}%
\pgfsetstrokecolor{currentstroke}%
\pgfsetstrokeopacity{0.000000}%
\pgfsetdash{}{0pt}%
\pgfpathmoveto{\pgfqpoint{5.800000in}{164.436612in}}%
\pgfpathlineto{\pgfqpoint{5.800000in}{164.436612in}}%
\pgfpathlineto{\pgfqpoint{5.800000in}{164.539449in}}%
\pgfpathlineto{\pgfqpoint{5.800000in}{164.539449in}}%
\pgfpathclose%
\pgfusepath{fill}%
\end{pgfscope}%
\begin{pgfscope}%
\pgfpathrectangle{\pgfqpoint{5.800000in}{0.720000in}}{\pgfqpoint{1.400000in}{4.620000in}}%
\pgfusepath{clip}%
\pgfsetbuttcap%
\pgfsetmiterjoin%
\definecolor{currentfill}{rgb}{0.121569,0.466667,0.705882}%
\pgfsetfillcolor{currentfill}%
\pgfsetlinewidth{0.000000pt}%
\definecolor{currentstroke}{rgb}{0.000000,0.000000,0.000000}%
\pgfsetstrokecolor{currentstroke}%
\pgfsetstrokeopacity{0.000000}%
\pgfsetdash{}{0pt}%
\pgfpathmoveto{\pgfqpoint{5.800000in}{164.539449in}}%
\pgfpathlineto{\pgfqpoint{5.800000in}{164.539449in}}%
\pgfpathlineto{\pgfqpoint{5.800000in}{164.642286in}}%
\pgfpathlineto{\pgfqpoint{5.800000in}{164.642286in}}%
\pgfpathclose%
\pgfusepath{fill}%
\end{pgfscope}%
\begin{pgfscope}%
\pgfpathrectangle{\pgfqpoint{5.800000in}{0.720000in}}{\pgfqpoint{1.400000in}{4.620000in}}%
\pgfusepath{clip}%
\pgfsetbuttcap%
\pgfsetmiterjoin%
\definecolor{currentfill}{rgb}{0.121569,0.466667,0.705882}%
\pgfsetfillcolor{currentfill}%
\pgfsetlinewidth{0.000000pt}%
\definecolor{currentstroke}{rgb}{0.000000,0.000000,0.000000}%
\pgfsetstrokecolor{currentstroke}%
\pgfsetstrokeopacity{0.000000}%
\pgfsetdash{}{0pt}%
\pgfpathmoveto{\pgfqpoint{5.800000in}{164.642286in}}%
\pgfpathlineto{\pgfqpoint{5.800000in}{164.642286in}}%
\pgfpathlineto{\pgfqpoint{5.800000in}{164.745123in}}%
\pgfpathlineto{\pgfqpoint{5.800000in}{164.745123in}}%
\pgfpathclose%
\pgfusepath{fill}%
\end{pgfscope}%
\begin{pgfscope}%
\pgfpathrectangle{\pgfqpoint{5.800000in}{0.720000in}}{\pgfqpoint{1.400000in}{4.620000in}}%
\pgfusepath{clip}%
\pgfsetbuttcap%
\pgfsetmiterjoin%
\definecolor{currentfill}{rgb}{0.121569,0.466667,0.705882}%
\pgfsetfillcolor{currentfill}%
\pgfsetlinewidth{0.000000pt}%
\definecolor{currentstroke}{rgb}{0.000000,0.000000,0.000000}%
\pgfsetstrokecolor{currentstroke}%
\pgfsetstrokeopacity{0.000000}%
\pgfsetdash{}{0pt}%
\pgfpathmoveto{\pgfqpoint{5.800000in}{164.745123in}}%
\pgfpathlineto{\pgfqpoint{5.800000in}{164.745123in}}%
\pgfpathlineto{\pgfqpoint{5.800000in}{164.847960in}}%
\pgfpathlineto{\pgfqpoint{5.800000in}{164.847960in}}%
\pgfpathclose%
\pgfusepath{fill}%
\end{pgfscope}%
\begin{pgfscope}%
\pgfpathrectangle{\pgfqpoint{5.800000in}{0.720000in}}{\pgfqpoint{1.400000in}{4.620000in}}%
\pgfusepath{clip}%
\pgfsetbuttcap%
\pgfsetmiterjoin%
\definecolor{currentfill}{rgb}{0.121569,0.466667,0.705882}%
\pgfsetfillcolor{currentfill}%
\pgfsetlinewidth{0.000000pt}%
\definecolor{currentstroke}{rgb}{0.000000,0.000000,0.000000}%
\pgfsetstrokecolor{currentstroke}%
\pgfsetstrokeopacity{0.000000}%
\pgfsetdash{}{0pt}%
\pgfpathmoveto{\pgfqpoint{5.800000in}{164.847960in}}%
\pgfpathlineto{\pgfqpoint{5.800000in}{164.847960in}}%
\pgfpathlineto{\pgfqpoint{5.800000in}{164.950797in}}%
\pgfpathlineto{\pgfqpoint{5.800000in}{164.950797in}}%
\pgfpathclose%
\pgfusepath{fill}%
\end{pgfscope}%
\begin{pgfscope}%
\pgfpathrectangle{\pgfqpoint{5.800000in}{0.720000in}}{\pgfqpoint{1.400000in}{4.620000in}}%
\pgfusepath{clip}%
\pgfsetbuttcap%
\pgfsetmiterjoin%
\definecolor{currentfill}{rgb}{0.121569,0.466667,0.705882}%
\pgfsetfillcolor{currentfill}%
\pgfsetlinewidth{0.000000pt}%
\definecolor{currentstroke}{rgb}{0.000000,0.000000,0.000000}%
\pgfsetstrokecolor{currentstroke}%
\pgfsetstrokeopacity{0.000000}%
\pgfsetdash{}{0pt}%
\pgfpathmoveto{\pgfqpoint{5.800000in}{164.950797in}}%
\pgfpathlineto{\pgfqpoint{5.800000in}{164.950797in}}%
\pgfpathlineto{\pgfqpoint{5.800000in}{165.053634in}}%
\pgfpathlineto{\pgfqpoint{5.800000in}{165.053634in}}%
\pgfpathclose%
\pgfusepath{fill}%
\end{pgfscope}%
\begin{pgfscope}%
\pgfpathrectangle{\pgfqpoint{5.800000in}{0.720000in}}{\pgfqpoint{1.400000in}{4.620000in}}%
\pgfusepath{clip}%
\pgfsetbuttcap%
\pgfsetmiterjoin%
\definecolor{currentfill}{rgb}{0.121569,0.466667,0.705882}%
\pgfsetfillcolor{currentfill}%
\pgfsetlinewidth{0.000000pt}%
\definecolor{currentstroke}{rgb}{0.000000,0.000000,0.000000}%
\pgfsetstrokecolor{currentstroke}%
\pgfsetstrokeopacity{0.000000}%
\pgfsetdash{}{0pt}%
\pgfpathmoveto{\pgfqpoint{5.800000in}{165.053634in}}%
\pgfpathlineto{\pgfqpoint{5.800000in}{165.053634in}}%
\pgfpathlineto{\pgfqpoint{5.800000in}{165.156471in}}%
\pgfpathlineto{\pgfqpoint{5.800000in}{165.156471in}}%
\pgfpathclose%
\pgfusepath{fill}%
\end{pgfscope}%
\begin{pgfscope}%
\pgfpathrectangle{\pgfqpoint{5.800000in}{0.720000in}}{\pgfqpoint{1.400000in}{4.620000in}}%
\pgfusepath{clip}%
\pgfsetbuttcap%
\pgfsetmiterjoin%
\definecolor{currentfill}{rgb}{0.121569,0.466667,0.705882}%
\pgfsetfillcolor{currentfill}%
\pgfsetlinewidth{0.000000pt}%
\definecolor{currentstroke}{rgb}{0.000000,0.000000,0.000000}%
\pgfsetstrokecolor{currentstroke}%
\pgfsetstrokeopacity{0.000000}%
\pgfsetdash{}{0pt}%
\pgfpathmoveto{\pgfqpoint{5.800000in}{165.156471in}}%
\pgfpathlineto{\pgfqpoint{5.800000in}{165.156471in}}%
\pgfpathlineto{\pgfqpoint{5.800000in}{165.259308in}}%
\pgfpathlineto{\pgfqpoint{5.800000in}{165.259308in}}%
\pgfpathclose%
\pgfusepath{fill}%
\end{pgfscope}%
\begin{pgfscope}%
\pgfpathrectangle{\pgfqpoint{5.800000in}{0.720000in}}{\pgfqpoint{1.400000in}{4.620000in}}%
\pgfusepath{clip}%
\pgfsetbuttcap%
\pgfsetmiterjoin%
\definecolor{currentfill}{rgb}{0.121569,0.466667,0.705882}%
\pgfsetfillcolor{currentfill}%
\pgfsetlinewidth{0.000000pt}%
\definecolor{currentstroke}{rgb}{0.000000,0.000000,0.000000}%
\pgfsetstrokecolor{currentstroke}%
\pgfsetstrokeopacity{0.000000}%
\pgfsetdash{}{0pt}%
\pgfpathmoveto{\pgfqpoint{5.800000in}{165.259308in}}%
\pgfpathlineto{\pgfqpoint{5.800000in}{165.259308in}}%
\pgfpathlineto{\pgfqpoint{5.800000in}{165.362146in}}%
\pgfpathlineto{\pgfqpoint{5.800000in}{165.362146in}}%
\pgfpathclose%
\pgfusepath{fill}%
\end{pgfscope}%
\begin{pgfscope}%
\pgfpathrectangle{\pgfqpoint{5.800000in}{0.720000in}}{\pgfqpoint{1.400000in}{4.620000in}}%
\pgfusepath{clip}%
\pgfsetbuttcap%
\pgfsetmiterjoin%
\definecolor{currentfill}{rgb}{0.121569,0.466667,0.705882}%
\pgfsetfillcolor{currentfill}%
\pgfsetlinewidth{0.000000pt}%
\definecolor{currentstroke}{rgb}{0.000000,0.000000,0.000000}%
\pgfsetstrokecolor{currentstroke}%
\pgfsetstrokeopacity{0.000000}%
\pgfsetdash{}{0pt}%
\pgfpathmoveto{\pgfqpoint{5.800000in}{165.362146in}}%
\pgfpathlineto{\pgfqpoint{5.800000in}{165.362146in}}%
\pgfpathlineto{\pgfqpoint{5.800000in}{165.464983in}}%
\pgfpathlineto{\pgfqpoint{5.800000in}{165.464983in}}%
\pgfpathclose%
\pgfusepath{fill}%
\end{pgfscope}%
\begin{pgfscope}%
\pgfpathrectangle{\pgfqpoint{5.800000in}{0.720000in}}{\pgfqpoint{1.400000in}{4.620000in}}%
\pgfusepath{clip}%
\pgfsetbuttcap%
\pgfsetmiterjoin%
\definecolor{currentfill}{rgb}{0.121569,0.466667,0.705882}%
\pgfsetfillcolor{currentfill}%
\pgfsetlinewidth{0.000000pt}%
\definecolor{currentstroke}{rgb}{0.000000,0.000000,0.000000}%
\pgfsetstrokecolor{currentstroke}%
\pgfsetstrokeopacity{0.000000}%
\pgfsetdash{}{0pt}%
\pgfpathmoveto{\pgfqpoint{5.800000in}{165.464983in}}%
\pgfpathlineto{\pgfqpoint{5.800000in}{165.464983in}}%
\pgfpathlineto{\pgfqpoint{5.800000in}{165.567820in}}%
\pgfpathlineto{\pgfqpoint{5.800000in}{165.567820in}}%
\pgfpathclose%
\pgfusepath{fill}%
\end{pgfscope}%
\begin{pgfscope}%
\pgfpathrectangle{\pgfqpoint{5.800000in}{0.720000in}}{\pgfqpoint{1.400000in}{4.620000in}}%
\pgfusepath{clip}%
\pgfsetbuttcap%
\pgfsetmiterjoin%
\definecolor{currentfill}{rgb}{0.121569,0.466667,0.705882}%
\pgfsetfillcolor{currentfill}%
\pgfsetlinewidth{0.000000pt}%
\definecolor{currentstroke}{rgb}{0.000000,0.000000,0.000000}%
\pgfsetstrokecolor{currentstroke}%
\pgfsetstrokeopacity{0.000000}%
\pgfsetdash{}{0pt}%
\pgfpathmoveto{\pgfqpoint{5.800000in}{165.567820in}}%
\pgfpathlineto{\pgfqpoint{5.800000in}{165.567820in}}%
\pgfpathlineto{\pgfqpoint{5.800000in}{165.670657in}}%
\pgfpathlineto{\pgfqpoint{5.800000in}{165.670657in}}%
\pgfpathclose%
\pgfusepath{fill}%
\end{pgfscope}%
\begin{pgfscope}%
\pgfpathrectangle{\pgfqpoint{5.800000in}{0.720000in}}{\pgfqpoint{1.400000in}{4.620000in}}%
\pgfusepath{clip}%
\pgfsetbuttcap%
\pgfsetmiterjoin%
\definecolor{currentfill}{rgb}{0.121569,0.466667,0.705882}%
\pgfsetfillcolor{currentfill}%
\pgfsetlinewidth{0.000000pt}%
\definecolor{currentstroke}{rgb}{0.000000,0.000000,0.000000}%
\pgfsetstrokecolor{currentstroke}%
\pgfsetstrokeopacity{0.000000}%
\pgfsetdash{}{0pt}%
\pgfpathmoveto{\pgfqpoint{5.800000in}{165.670657in}}%
\pgfpathlineto{\pgfqpoint{5.800000in}{165.670657in}}%
\pgfpathlineto{\pgfqpoint{5.800000in}{165.773494in}}%
\pgfpathlineto{\pgfqpoint{5.800000in}{165.773494in}}%
\pgfpathclose%
\pgfusepath{fill}%
\end{pgfscope}%
\begin{pgfscope}%
\pgfpathrectangle{\pgfqpoint{5.800000in}{0.720000in}}{\pgfqpoint{1.400000in}{4.620000in}}%
\pgfusepath{clip}%
\pgfsetbuttcap%
\pgfsetmiterjoin%
\definecolor{currentfill}{rgb}{0.121569,0.466667,0.705882}%
\pgfsetfillcolor{currentfill}%
\pgfsetlinewidth{0.000000pt}%
\definecolor{currentstroke}{rgb}{0.000000,0.000000,0.000000}%
\pgfsetstrokecolor{currentstroke}%
\pgfsetstrokeopacity{0.000000}%
\pgfsetdash{}{0pt}%
\pgfpathmoveto{\pgfqpoint{5.800000in}{165.773494in}}%
\pgfpathlineto{\pgfqpoint{5.800000in}{165.773494in}}%
\pgfpathlineto{\pgfqpoint{5.800000in}{165.876331in}}%
\pgfpathlineto{\pgfqpoint{5.800000in}{165.876331in}}%
\pgfpathclose%
\pgfusepath{fill}%
\end{pgfscope}%
\begin{pgfscope}%
\pgfpathrectangle{\pgfqpoint{5.800000in}{0.720000in}}{\pgfqpoint{1.400000in}{4.620000in}}%
\pgfusepath{clip}%
\pgfsetbuttcap%
\pgfsetmiterjoin%
\definecolor{currentfill}{rgb}{0.121569,0.466667,0.705882}%
\pgfsetfillcolor{currentfill}%
\pgfsetlinewidth{0.000000pt}%
\definecolor{currentstroke}{rgb}{0.000000,0.000000,0.000000}%
\pgfsetstrokecolor{currentstroke}%
\pgfsetstrokeopacity{0.000000}%
\pgfsetdash{}{0pt}%
\pgfpathmoveto{\pgfqpoint{5.800000in}{165.876331in}}%
\pgfpathlineto{\pgfqpoint{5.800000in}{165.876331in}}%
\pgfpathlineto{\pgfqpoint{5.800000in}{165.979168in}}%
\pgfpathlineto{\pgfqpoint{5.800000in}{165.979168in}}%
\pgfpathclose%
\pgfusepath{fill}%
\end{pgfscope}%
\begin{pgfscope}%
\pgfpathrectangle{\pgfqpoint{5.800000in}{0.720000in}}{\pgfqpoint{1.400000in}{4.620000in}}%
\pgfusepath{clip}%
\pgfsetbuttcap%
\pgfsetmiterjoin%
\definecolor{currentfill}{rgb}{0.121569,0.466667,0.705882}%
\pgfsetfillcolor{currentfill}%
\pgfsetlinewidth{0.000000pt}%
\definecolor{currentstroke}{rgb}{0.000000,0.000000,0.000000}%
\pgfsetstrokecolor{currentstroke}%
\pgfsetstrokeopacity{0.000000}%
\pgfsetdash{}{0pt}%
\pgfpathmoveto{\pgfqpoint{5.800000in}{165.979168in}}%
\pgfpathlineto{\pgfqpoint{5.800000in}{165.979168in}}%
\pgfpathlineto{\pgfqpoint{5.800000in}{166.082005in}}%
\pgfpathlineto{\pgfqpoint{5.800000in}{166.082005in}}%
\pgfpathclose%
\pgfusepath{fill}%
\end{pgfscope}%
\begin{pgfscope}%
\pgfpathrectangle{\pgfqpoint{5.800000in}{0.720000in}}{\pgfqpoint{1.400000in}{4.620000in}}%
\pgfusepath{clip}%
\pgfsetbuttcap%
\pgfsetmiterjoin%
\definecolor{currentfill}{rgb}{0.121569,0.466667,0.705882}%
\pgfsetfillcolor{currentfill}%
\pgfsetlinewidth{0.000000pt}%
\definecolor{currentstroke}{rgb}{0.000000,0.000000,0.000000}%
\pgfsetstrokecolor{currentstroke}%
\pgfsetstrokeopacity{0.000000}%
\pgfsetdash{}{0pt}%
\pgfpathmoveto{\pgfqpoint{5.800000in}{166.082005in}}%
\pgfpathlineto{\pgfqpoint{5.800000in}{166.082005in}}%
\pgfpathlineto{\pgfqpoint{5.800000in}{166.184842in}}%
\pgfpathlineto{\pgfqpoint{5.800000in}{166.184842in}}%
\pgfpathclose%
\pgfusepath{fill}%
\end{pgfscope}%
\begin{pgfscope}%
\pgfpathrectangle{\pgfqpoint{5.800000in}{0.720000in}}{\pgfqpoint{1.400000in}{4.620000in}}%
\pgfusepath{clip}%
\pgfsetbuttcap%
\pgfsetmiterjoin%
\definecolor{currentfill}{rgb}{0.121569,0.466667,0.705882}%
\pgfsetfillcolor{currentfill}%
\pgfsetlinewidth{0.000000pt}%
\definecolor{currentstroke}{rgb}{0.000000,0.000000,0.000000}%
\pgfsetstrokecolor{currentstroke}%
\pgfsetstrokeopacity{0.000000}%
\pgfsetdash{}{0pt}%
\pgfpathmoveto{\pgfqpoint{5.800000in}{166.184842in}}%
\pgfpathlineto{\pgfqpoint{5.800000in}{166.184842in}}%
\pgfpathlineto{\pgfqpoint{5.800000in}{166.287679in}}%
\pgfpathlineto{\pgfqpoint{5.800000in}{166.287679in}}%
\pgfpathclose%
\pgfusepath{fill}%
\end{pgfscope}%
\begin{pgfscope}%
\pgfpathrectangle{\pgfqpoint{5.800000in}{0.720000in}}{\pgfqpoint{1.400000in}{4.620000in}}%
\pgfusepath{clip}%
\pgfsetbuttcap%
\pgfsetmiterjoin%
\definecolor{currentfill}{rgb}{0.121569,0.466667,0.705882}%
\pgfsetfillcolor{currentfill}%
\pgfsetlinewidth{0.000000pt}%
\definecolor{currentstroke}{rgb}{0.000000,0.000000,0.000000}%
\pgfsetstrokecolor{currentstroke}%
\pgfsetstrokeopacity{0.000000}%
\pgfsetdash{}{0pt}%
\pgfpathmoveto{\pgfqpoint{5.800000in}{166.287679in}}%
\pgfpathlineto{\pgfqpoint{5.800000in}{166.287679in}}%
\pgfpathlineto{\pgfqpoint{5.800000in}{166.390516in}}%
\pgfpathlineto{\pgfqpoint{5.800000in}{166.390516in}}%
\pgfpathclose%
\pgfusepath{fill}%
\end{pgfscope}%
\begin{pgfscope}%
\pgfpathrectangle{\pgfqpoint{5.800000in}{0.720000in}}{\pgfqpoint{1.400000in}{4.620000in}}%
\pgfusepath{clip}%
\pgfsetbuttcap%
\pgfsetmiterjoin%
\definecolor{currentfill}{rgb}{0.121569,0.466667,0.705882}%
\pgfsetfillcolor{currentfill}%
\pgfsetlinewidth{0.000000pt}%
\definecolor{currentstroke}{rgb}{0.000000,0.000000,0.000000}%
\pgfsetstrokecolor{currentstroke}%
\pgfsetstrokeopacity{0.000000}%
\pgfsetdash{}{0pt}%
\pgfpathmoveto{\pgfqpoint{5.800000in}{166.390516in}}%
\pgfpathlineto{\pgfqpoint{5.800000in}{166.390516in}}%
\pgfpathlineto{\pgfqpoint{5.800000in}{166.493353in}}%
\pgfpathlineto{\pgfqpoint{5.800000in}{166.493353in}}%
\pgfpathclose%
\pgfusepath{fill}%
\end{pgfscope}%
\begin{pgfscope}%
\pgfpathrectangle{\pgfqpoint{5.800000in}{0.720000in}}{\pgfqpoint{1.400000in}{4.620000in}}%
\pgfusepath{clip}%
\pgfsetbuttcap%
\pgfsetmiterjoin%
\definecolor{currentfill}{rgb}{0.121569,0.466667,0.705882}%
\pgfsetfillcolor{currentfill}%
\pgfsetlinewidth{0.000000pt}%
\definecolor{currentstroke}{rgb}{0.000000,0.000000,0.000000}%
\pgfsetstrokecolor{currentstroke}%
\pgfsetstrokeopacity{0.000000}%
\pgfsetdash{}{0pt}%
\pgfpathmoveto{\pgfqpoint{5.800000in}{166.493353in}}%
\pgfpathlineto{\pgfqpoint{5.800000in}{166.493353in}}%
\pgfpathlineto{\pgfqpoint{5.800000in}{166.596190in}}%
\pgfpathlineto{\pgfqpoint{5.800000in}{166.596190in}}%
\pgfpathclose%
\pgfusepath{fill}%
\end{pgfscope}%
\begin{pgfscope}%
\pgfpathrectangle{\pgfqpoint{5.800000in}{0.720000in}}{\pgfqpoint{1.400000in}{4.620000in}}%
\pgfusepath{clip}%
\pgfsetbuttcap%
\pgfsetmiterjoin%
\definecolor{currentfill}{rgb}{0.121569,0.466667,0.705882}%
\pgfsetfillcolor{currentfill}%
\pgfsetlinewidth{0.000000pt}%
\definecolor{currentstroke}{rgb}{0.000000,0.000000,0.000000}%
\pgfsetstrokecolor{currentstroke}%
\pgfsetstrokeopacity{0.000000}%
\pgfsetdash{}{0pt}%
\pgfpathmoveto{\pgfqpoint{5.800000in}{166.596190in}}%
\pgfpathlineto{\pgfqpoint{5.800000in}{166.596190in}}%
\pgfpathlineto{\pgfqpoint{5.800000in}{166.699027in}}%
\pgfpathlineto{\pgfqpoint{5.800000in}{166.699027in}}%
\pgfpathclose%
\pgfusepath{fill}%
\end{pgfscope}%
\begin{pgfscope}%
\pgfpathrectangle{\pgfqpoint{5.800000in}{0.720000in}}{\pgfqpoint{1.400000in}{4.620000in}}%
\pgfusepath{clip}%
\pgfsetbuttcap%
\pgfsetmiterjoin%
\definecolor{currentfill}{rgb}{0.121569,0.466667,0.705882}%
\pgfsetfillcolor{currentfill}%
\pgfsetlinewidth{0.000000pt}%
\definecolor{currentstroke}{rgb}{0.000000,0.000000,0.000000}%
\pgfsetstrokecolor{currentstroke}%
\pgfsetstrokeopacity{0.000000}%
\pgfsetdash{}{0pt}%
\pgfpathmoveto{\pgfqpoint{5.800000in}{166.699027in}}%
\pgfpathlineto{\pgfqpoint{5.800000in}{166.699027in}}%
\pgfpathlineto{\pgfqpoint{5.800000in}{166.801864in}}%
\pgfpathlineto{\pgfqpoint{5.800000in}{166.801864in}}%
\pgfpathclose%
\pgfusepath{fill}%
\end{pgfscope}%
\begin{pgfscope}%
\pgfpathrectangle{\pgfqpoint{5.800000in}{0.720000in}}{\pgfqpoint{1.400000in}{4.620000in}}%
\pgfusepath{clip}%
\pgfsetbuttcap%
\pgfsetmiterjoin%
\definecolor{currentfill}{rgb}{0.121569,0.466667,0.705882}%
\pgfsetfillcolor{currentfill}%
\pgfsetlinewidth{0.000000pt}%
\definecolor{currentstroke}{rgb}{0.000000,0.000000,0.000000}%
\pgfsetstrokecolor{currentstroke}%
\pgfsetstrokeopacity{0.000000}%
\pgfsetdash{}{0pt}%
\pgfpathmoveto{\pgfqpoint{5.800000in}{166.801864in}}%
\pgfpathlineto{\pgfqpoint{5.800000in}{166.801864in}}%
\pgfpathlineto{\pgfqpoint{5.800000in}{166.904702in}}%
\pgfpathlineto{\pgfqpoint{5.800000in}{166.904702in}}%
\pgfpathclose%
\pgfusepath{fill}%
\end{pgfscope}%
\begin{pgfscope}%
\pgfpathrectangle{\pgfqpoint{5.800000in}{0.720000in}}{\pgfqpoint{1.400000in}{4.620000in}}%
\pgfusepath{clip}%
\pgfsetbuttcap%
\pgfsetmiterjoin%
\definecolor{currentfill}{rgb}{0.121569,0.466667,0.705882}%
\pgfsetfillcolor{currentfill}%
\pgfsetlinewidth{0.000000pt}%
\definecolor{currentstroke}{rgb}{0.000000,0.000000,0.000000}%
\pgfsetstrokecolor{currentstroke}%
\pgfsetstrokeopacity{0.000000}%
\pgfsetdash{}{0pt}%
\pgfpathmoveto{\pgfqpoint{5.800000in}{166.904702in}}%
\pgfpathlineto{\pgfqpoint{5.800000in}{166.904702in}}%
\pgfpathlineto{\pgfqpoint{5.800000in}{167.007539in}}%
\pgfpathlineto{\pgfqpoint{5.800000in}{167.007539in}}%
\pgfpathclose%
\pgfusepath{fill}%
\end{pgfscope}%
\begin{pgfscope}%
\pgfpathrectangle{\pgfqpoint{5.800000in}{0.720000in}}{\pgfqpoint{1.400000in}{4.620000in}}%
\pgfusepath{clip}%
\pgfsetbuttcap%
\pgfsetmiterjoin%
\definecolor{currentfill}{rgb}{0.121569,0.466667,0.705882}%
\pgfsetfillcolor{currentfill}%
\pgfsetlinewidth{0.000000pt}%
\definecolor{currentstroke}{rgb}{0.000000,0.000000,0.000000}%
\pgfsetstrokecolor{currentstroke}%
\pgfsetstrokeopacity{0.000000}%
\pgfsetdash{}{0pt}%
\pgfpathmoveto{\pgfqpoint{5.800000in}{167.007539in}}%
\pgfpathlineto{\pgfqpoint{5.800000in}{167.007539in}}%
\pgfpathlineto{\pgfqpoint{5.800000in}{167.110376in}}%
\pgfpathlineto{\pgfqpoint{5.800000in}{167.110376in}}%
\pgfpathclose%
\pgfusepath{fill}%
\end{pgfscope}%
\begin{pgfscope}%
\pgfpathrectangle{\pgfqpoint{5.800000in}{0.720000in}}{\pgfqpoint{1.400000in}{4.620000in}}%
\pgfusepath{clip}%
\pgfsetbuttcap%
\pgfsetmiterjoin%
\definecolor{currentfill}{rgb}{0.121569,0.466667,0.705882}%
\pgfsetfillcolor{currentfill}%
\pgfsetlinewidth{0.000000pt}%
\definecolor{currentstroke}{rgb}{0.000000,0.000000,0.000000}%
\pgfsetstrokecolor{currentstroke}%
\pgfsetstrokeopacity{0.000000}%
\pgfsetdash{}{0pt}%
\pgfpathmoveto{\pgfqpoint{5.800000in}{167.110376in}}%
\pgfpathlineto{\pgfqpoint{5.800000in}{167.110376in}}%
\pgfpathlineto{\pgfqpoint{5.800000in}{167.213213in}}%
\pgfpathlineto{\pgfqpoint{5.800000in}{167.213213in}}%
\pgfpathclose%
\pgfusepath{fill}%
\end{pgfscope}%
\begin{pgfscope}%
\pgfpathrectangle{\pgfqpoint{5.800000in}{0.720000in}}{\pgfqpoint{1.400000in}{4.620000in}}%
\pgfusepath{clip}%
\pgfsetbuttcap%
\pgfsetmiterjoin%
\definecolor{currentfill}{rgb}{0.121569,0.466667,0.705882}%
\pgfsetfillcolor{currentfill}%
\pgfsetlinewidth{0.000000pt}%
\definecolor{currentstroke}{rgb}{0.000000,0.000000,0.000000}%
\pgfsetstrokecolor{currentstroke}%
\pgfsetstrokeopacity{0.000000}%
\pgfsetdash{}{0pt}%
\pgfpathmoveto{\pgfqpoint{5.800000in}{167.213213in}}%
\pgfpathlineto{\pgfqpoint{5.800000in}{167.213213in}}%
\pgfpathlineto{\pgfqpoint{5.800000in}{167.316050in}}%
\pgfpathlineto{\pgfqpoint{5.800000in}{167.316050in}}%
\pgfpathclose%
\pgfusepath{fill}%
\end{pgfscope}%
\begin{pgfscope}%
\pgfpathrectangle{\pgfqpoint{5.800000in}{0.720000in}}{\pgfqpoint{1.400000in}{4.620000in}}%
\pgfusepath{clip}%
\pgfsetbuttcap%
\pgfsetmiterjoin%
\definecolor{currentfill}{rgb}{0.121569,0.466667,0.705882}%
\pgfsetfillcolor{currentfill}%
\pgfsetlinewidth{0.000000pt}%
\definecolor{currentstroke}{rgb}{0.000000,0.000000,0.000000}%
\pgfsetstrokecolor{currentstroke}%
\pgfsetstrokeopacity{0.000000}%
\pgfsetdash{}{0pt}%
\pgfpathmoveto{\pgfqpoint{5.800000in}{167.316050in}}%
\pgfpathlineto{\pgfqpoint{5.800000in}{167.316050in}}%
\pgfpathlineto{\pgfqpoint{5.800000in}{167.418887in}}%
\pgfpathlineto{\pgfqpoint{5.800000in}{167.418887in}}%
\pgfpathclose%
\pgfusepath{fill}%
\end{pgfscope}%
\begin{pgfscope}%
\pgfpathrectangle{\pgfqpoint{5.800000in}{0.720000in}}{\pgfqpoint{1.400000in}{4.620000in}}%
\pgfusepath{clip}%
\pgfsetbuttcap%
\pgfsetmiterjoin%
\definecolor{currentfill}{rgb}{0.121569,0.466667,0.705882}%
\pgfsetfillcolor{currentfill}%
\pgfsetlinewidth{0.000000pt}%
\definecolor{currentstroke}{rgb}{0.000000,0.000000,0.000000}%
\pgfsetstrokecolor{currentstroke}%
\pgfsetstrokeopacity{0.000000}%
\pgfsetdash{}{0pt}%
\pgfpathmoveto{\pgfqpoint{5.800000in}{167.418887in}}%
\pgfpathlineto{\pgfqpoint{5.800000in}{167.418887in}}%
\pgfpathlineto{\pgfqpoint{5.800000in}{167.521724in}}%
\pgfpathlineto{\pgfqpoint{5.800000in}{167.521724in}}%
\pgfpathclose%
\pgfusepath{fill}%
\end{pgfscope}%
\begin{pgfscope}%
\pgfpathrectangle{\pgfqpoint{5.800000in}{0.720000in}}{\pgfqpoint{1.400000in}{4.620000in}}%
\pgfusepath{clip}%
\pgfsetbuttcap%
\pgfsetmiterjoin%
\definecolor{currentfill}{rgb}{0.121569,0.466667,0.705882}%
\pgfsetfillcolor{currentfill}%
\pgfsetlinewidth{0.000000pt}%
\definecolor{currentstroke}{rgb}{0.000000,0.000000,0.000000}%
\pgfsetstrokecolor{currentstroke}%
\pgfsetstrokeopacity{0.000000}%
\pgfsetdash{}{0pt}%
\pgfpathmoveto{\pgfqpoint{5.800000in}{167.521724in}}%
\pgfpathlineto{\pgfqpoint{5.800000in}{167.521724in}}%
\pgfpathlineto{\pgfqpoint{5.800000in}{167.624561in}}%
\pgfpathlineto{\pgfqpoint{5.800000in}{167.624561in}}%
\pgfpathclose%
\pgfusepath{fill}%
\end{pgfscope}%
\begin{pgfscope}%
\pgfpathrectangle{\pgfqpoint{5.800000in}{0.720000in}}{\pgfqpoint{1.400000in}{4.620000in}}%
\pgfusepath{clip}%
\pgfsetbuttcap%
\pgfsetmiterjoin%
\definecolor{currentfill}{rgb}{0.121569,0.466667,0.705882}%
\pgfsetfillcolor{currentfill}%
\pgfsetlinewidth{0.000000pt}%
\definecolor{currentstroke}{rgb}{0.000000,0.000000,0.000000}%
\pgfsetstrokecolor{currentstroke}%
\pgfsetstrokeopacity{0.000000}%
\pgfsetdash{}{0pt}%
\pgfpathmoveto{\pgfqpoint{5.800000in}{167.624561in}}%
\pgfpathlineto{\pgfqpoint{5.800000in}{167.624561in}}%
\pgfpathlineto{\pgfqpoint{5.800000in}{167.727398in}}%
\pgfpathlineto{\pgfqpoint{5.800000in}{167.727398in}}%
\pgfpathclose%
\pgfusepath{fill}%
\end{pgfscope}%
\begin{pgfscope}%
\pgfpathrectangle{\pgfqpoint{5.800000in}{0.720000in}}{\pgfqpoint{1.400000in}{4.620000in}}%
\pgfusepath{clip}%
\pgfsetbuttcap%
\pgfsetmiterjoin%
\definecolor{currentfill}{rgb}{0.121569,0.466667,0.705882}%
\pgfsetfillcolor{currentfill}%
\pgfsetlinewidth{0.000000pt}%
\definecolor{currentstroke}{rgb}{0.000000,0.000000,0.000000}%
\pgfsetstrokecolor{currentstroke}%
\pgfsetstrokeopacity{0.000000}%
\pgfsetdash{}{0pt}%
\pgfpathmoveto{\pgfqpoint{5.800000in}{167.727398in}}%
\pgfpathlineto{\pgfqpoint{5.800000in}{167.727398in}}%
\pgfpathlineto{\pgfqpoint{5.800000in}{167.830235in}}%
\pgfpathlineto{\pgfqpoint{5.800000in}{167.830235in}}%
\pgfpathclose%
\pgfusepath{fill}%
\end{pgfscope}%
\begin{pgfscope}%
\pgfpathrectangle{\pgfqpoint{5.800000in}{0.720000in}}{\pgfqpoint{1.400000in}{4.620000in}}%
\pgfusepath{clip}%
\pgfsetbuttcap%
\pgfsetmiterjoin%
\definecolor{currentfill}{rgb}{0.121569,0.466667,0.705882}%
\pgfsetfillcolor{currentfill}%
\pgfsetlinewidth{0.000000pt}%
\definecolor{currentstroke}{rgb}{0.000000,0.000000,0.000000}%
\pgfsetstrokecolor{currentstroke}%
\pgfsetstrokeopacity{0.000000}%
\pgfsetdash{}{0pt}%
\pgfpathmoveto{\pgfqpoint{5.800000in}{167.830235in}}%
\pgfpathlineto{\pgfqpoint{5.800000in}{167.830235in}}%
\pgfpathlineto{\pgfqpoint{5.800000in}{167.933072in}}%
\pgfpathlineto{\pgfqpoint{5.800000in}{167.933072in}}%
\pgfpathclose%
\pgfusepath{fill}%
\end{pgfscope}%
\begin{pgfscope}%
\pgfpathrectangle{\pgfqpoint{5.800000in}{0.720000in}}{\pgfqpoint{1.400000in}{4.620000in}}%
\pgfusepath{clip}%
\pgfsetbuttcap%
\pgfsetmiterjoin%
\definecolor{currentfill}{rgb}{0.121569,0.466667,0.705882}%
\pgfsetfillcolor{currentfill}%
\pgfsetlinewidth{0.000000pt}%
\definecolor{currentstroke}{rgb}{0.000000,0.000000,0.000000}%
\pgfsetstrokecolor{currentstroke}%
\pgfsetstrokeopacity{0.000000}%
\pgfsetdash{}{0pt}%
\pgfpathmoveto{\pgfqpoint{5.800000in}{167.933072in}}%
\pgfpathlineto{\pgfqpoint{5.800000in}{167.933072in}}%
\pgfpathlineto{\pgfqpoint{5.800000in}{168.035909in}}%
\pgfpathlineto{\pgfqpoint{5.800000in}{168.035909in}}%
\pgfpathclose%
\pgfusepath{fill}%
\end{pgfscope}%
\begin{pgfscope}%
\pgfpathrectangle{\pgfqpoint{5.800000in}{0.720000in}}{\pgfqpoint{1.400000in}{4.620000in}}%
\pgfusepath{clip}%
\pgfsetbuttcap%
\pgfsetmiterjoin%
\definecolor{currentfill}{rgb}{0.121569,0.466667,0.705882}%
\pgfsetfillcolor{currentfill}%
\pgfsetlinewidth{0.000000pt}%
\definecolor{currentstroke}{rgb}{0.000000,0.000000,0.000000}%
\pgfsetstrokecolor{currentstroke}%
\pgfsetstrokeopacity{0.000000}%
\pgfsetdash{}{0pt}%
\pgfpathmoveto{\pgfqpoint{5.800000in}{168.035909in}}%
\pgfpathlineto{\pgfqpoint{5.800000in}{168.035909in}}%
\pgfpathlineto{\pgfqpoint{5.800000in}{168.138746in}}%
\pgfpathlineto{\pgfqpoint{5.800000in}{168.138746in}}%
\pgfpathclose%
\pgfusepath{fill}%
\end{pgfscope}%
\begin{pgfscope}%
\pgfpathrectangle{\pgfqpoint{5.800000in}{0.720000in}}{\pgfqpoint{1.400000in}{4.620000in}}%
\pgfusepath{clip}%
\pgfsetbuttcap%
\pgfsetmiterjoin%
\definecolor{currentfill}{rgb}{0.121569,0.466667,0.705882}%
\pgfsetfillcolor{currentfill}%
\pgfsetlinewidth{0.000000pt}%
\definecolor{currentstroke}{rgb}{0.000000,0.000000,0.000000}%
\pgfsetstrokecolor{currentstroke}%
\pgfsetstrokeopacity{0.000000}%
\pgfsetdash{}{0pt}%
\pgfpathmoveto{\pgfqpoint{5.800000in}{168.138746in}}%
\pgfpathlineto{\pgfqpoint{5.800000in}{168.138746in}}%
\pgfpathlineto{\pgfqpoint{5.800000in}{168.241583in}}%
\pgfpathlineto{\pgfqpoint{5.800000in}{168.241583in}}%
\pgfpathclose%
\pgfusepath{fill}%
\end{pgfscope}%
\begin{pgfscope}%
\pgfpathrectangle{\pgfqpoint{5.800000in}{0.720000in}}{\pgfqpoint{1.400000in}{4.620000in}}%
\pgfusepath{clip}%
\pgfsetbuttcap%
\pgfsetmiterjoin%
\definecolor{currentfill}{rgb}{0.121569,0.466667,0.705882}%
\pgfsetfillcolor{currentfill}%
\pgfsetlinewidth{0.000000pt}%
\definecolor{currentstroke}{rgb}{0.000000,0.000000,0.000000}%
\pgfsetstrokecolor{currentstroke}%
\pgfsetstrokeopacity{0.000000}%
\pgfsetdash{}{0pt}%
\pgfpathmoveto{\pgfqpoint{5.800000in}{168.241583in}}%
\pgfpathlineto{\pgfqpoint{5.800000in}{168.241583in}}%
\pgfpathlineto{\pgfqpoint{5.800000in}{168.344420in}}%
\pgfpathlineto{\pgfqpoint{5.800000in}{168.344420in}}%
\pgfpathclose%
\pgfusepath{fill}%
\end{pgfscope}%
\begin{pgfscope}%
\pgfpathrectangle{\pgfqpoint{5.800000in}{0.720000in}}{\pgfqpoint{1.400000in}{4.620000in}}%
\pgfusepath{clip}%
\pgfsetbuttcap%
\pgfsetmiterjoin%
\definecolor{currentfill}{rgb}{0.121569,0.466667,0.705882}%
\pgfsetfillcolor{currentfill}%
\pgfsetlinewidth{0.000000pt}%
\definecolor{currentstroke}{rgb}{0.000000,0.000000,0.000000}%
\pgfsetstrokecolor{currentstroke}%
\pgfsetstrokeopacity{0.000000}%
\pgfsetdash{}{0pt}%
\pgfpathmoveto{\pgfqpoint{5.800000in}{168.344420in}}%
\pgfpathlineto{\pgfqpoint{5.800000in}{168.344420in}}%
\pgfpathlineto{\pgfqpoint{5.800000in}{168.447258in}}%
\pgfpathlineto{\pgfqpoint{5.800000in}{168.447258in}}%
\pgfpathclose%
\pgfusepath{fill}%
\end{pgfscope}%
\begin{pgfscope}%
\pgfpathrectangle{\pgfqpoint{5.800000in}{0.720000in}}{\pgfqpoint{1.400000in}{4.620000in}}%
\pgfusepath{clip}%
\pgfsetbuttcap%
\pgfsetmiterjoin%
\definecolor{currentfill}{rgb}{0.121569,0.466667,0.705882}%
\pgfsetfillcolor{currentfill}%
\pgfsetlinewidth{0.000000pt}%
\definecolor{currentstroke}{rgb}{0.000000,0.000000,0.000000}%
\pgfsetstrokecolor{currentstroke}%
\pgfsetstrokeopacity{0.000000}%
\pgfsetdash{}{0pt}%
\pgfpathmoveto{\pgfqpoint{5.800000in}{168.447258in}}%
\pgfpathlineto{\pgfqpoint{5.800000in}{168.447258in}}%
\pgfpathlineto{\pgfqpoint{5.800000in}{168.550095in}}%
\pgfpathlineto{\pgfqpoint{5.800000in}{168.550095in}}%
\pgfpathclose%
\pgfusepath{fill}%
\end{pgfscope}%
\begin{pgfscope}%
\pgfpathrectangle{\pgfqpoint{5.800000in}{0.720000in}}{\pgfqpoint{1.400000in}{4.620000in}}%
\pgfusepath{clip}%
\pgfsetbuttcap%
\pgfsetmiterjoin%
\definecolor{currentfill}{rgb}{0.121569,0.466667,0.705882}%
\pgfsetfillcolor{currentfill}%
\pgfsetlinewidth{0.000000pt}%
\definecolor{currentstroke}{rgb}{0.000000,0.000000,0.000000}%
\pgfsetstrokecolor{currentstroke}%
\pgfsetstrokeopacity{0.000000}%
\pgfsetdash{}{0pt}%
\pgfpathmoveto{\pgfqpoint{5.800000in}{168.550095in}}%
\pgfpathlineto{\pgfqpoint{5.800000in}{168.550095in}}%
\pgfpathlineto{\pgfqpoint{5.800000in}{168.652932in}}%
\pgfpathlineto{\pgfqpoint{5.800000in}{168.652932in}}%
\pgfpathclose%
\pgfusepath{fill}%
\end{pgfscope}%
\begin{pgfscope}%
\pgfpathrectangle{\pgfqpoint{5.800000in}{0.720000in}}{\pgfqpoint{1.400000in}{4.620000in}}%
\pgfusepath{clip}%
\pgfsetbuttcap%
\pgfsetmiterjoin%
\definecolor{currentfill}{rgb}{0.121569,0.466667,0.705882}%
\pgfsetfillcolor{currentfill}%
\pgfsetlinewidth{0.000000pt}%
\definecolor{currentstroke}{rgb}{0.000000,0.000000,0.000000}%
\pgfsetstrokecolor{currentstroke}%
\pgfsetstrokeopacity{0.000000}%
\pgfsetdash{}{0pt}%
\pgfpathmoveto{\pgfqpoint{5.800000in}{168.652932in}}%
\pgfpathlineto{\pgfqpoint{5.800000in}{168.652932in}}%
\pgfpathlineto{\pgfqpoint{5.800000in}{168.755769in}}%
\pgfpathlineto{\pgfqpoint{5.800000in}{168.755769in}}%
\pgfpathclose%
\pgfusepath{fill}%
\end{pgfscope}%
\begin{pgfscope}%
\pgfpathrectangle{\pgfqpoint{5.800000in}{0.720000in}}{\pgfqpoint{1.400000in}{4.620000in}}%
\pgfusepath{clip}%
\pgfsetbuttcap%
\pgfsetmiterjoin%
\definecolor{currentfill}{rgb}{0.121569,0.466667,0.705882}%
\pgfsetfillcolor{currentfill}%
\pgfsetlinewidth{0.000000pt}%
\definecolor{currentstroke}{rgb}{0.000000,0.000000,0.000000}%
\pgfsetstrokecolor{currentstroke}%
\pgfsetstrokeopacity{0.000000}%
\pgfsetdash{}{0pt}%
\pgfpathmoveto{\pgfqpoint{5.800000in}{168.755769in}}%
\pgfpathlineto{\pgfqpoint{5.800000in}{168.755769in}}%
\pgfpathlineto{\pgfqpoint{5.800000in}{168.858606in}}%
\pgfpathlineto{\pgfqpoint{5.800000in}{168.858606in}}%
\pgfpathclose%
\pgfusepath{fill}%
\end{pgfscope}%
\begin{pgfscope}%
\pgfpathrectangle{\pgfqpoint{5.800000in}{0.720000in}}{\pgfqpoint{1.400000in}{4.620000in}}%
\pgfusepath{clip}%
\pgfsetbuttcap%
\pgfsetmiterjoin%
\definecolor{currentfill}{rgb}{0.121569,0.466667,0.705882}%
\pgfsetfillcolor{currentfill}%
\pgfsetlinewidth{0.000000pt}%
\definecolor{currentstroke}{rgb}{0.000000,0.000000,0.000000}%
\pgfsetstrokecolor{currentstroke}%
\pgfsetstrokeopacity{0.000000}%
\pgfsetdash{}{0pt}%
\pgfpathmoveto{\pgfqpoint{5.800000in}{168.858606in}}%
\pgfpathlineto{\pgfqpoint{5.800000in}{168.858606in}}%
\pgfpathlineto{\pgfqpoint{5.800000in}{168.961443in}}%
\pgfpathlineto{\pgfqpoint{5.800000in}{168.961443in}}%
\pgfpathclose%
\pgfusepath{fill}%
\end{pgfscope}%
\begin{pgfscope}%
\pgfpathrectangle{\pgfqpoint{5.800000in}{0.720000in}}{\pgfqpoint{1.400000in}{4.620000in}}%
\pgfusepath{clip}%
\pgfsetbuttcap%
\pgfsetmiterjoin%
\definecolor{currentfill}{rgb}{0.121569,0.466667,0.705882}%
\pgfsetfillcolor{currentfill}%
\pgfsetlinewidth{0.000000pt}%
\definecolor{currentstroke}{rgb}{0.000000,0.000000,0.000000}%
\pgfsetstrokecolor{currentstroke}%
\pgfsetstrokeopacity{0.000000}%
\pgfsetdash{}{0pt}%
\pgfpathmoveto{\pgfqpoint{5.800000in}{168.961443in}}%
\pgfpathlineto{\pgfqpoint{5.800000in}{168.961443in}}%
\pgfpathlineto{\pgfqpoint{5.800000in}{169.064280in}}%
\pgfpathlineto{\pgfqpoint{5.800000in}{169.064280in}}%
\pgfpathclose%
\pgfusepath{fill}%
\end{pgfscope}%
\begin{pgfscope}%
\definecolor{textcolor}{rgb}{0.000000,0.000000,0.000000}%
\pgfsetstrokecolor{textcolor}%
\pgfsetfillcolor{textcolor}%
\pgftext[x=6.500000in,y=0.664444in,,top]{\color{textcolor}\sffamily\fontsize{20.000000}{24.000000}\selectfont \(\displaystyle \mathrm{arb.\ unit}\)}%
\end{pgfscope}%
\begin{pgfscope}%
\pgfsetrectcap%
\pgfsetmiterjoin%
\pgfsetlinewidth{0.803000pt}%
\definecolor{currentstroke}{rgb}{0.000000,0.000000,0.000000}%
\pgfsetstrokecolor{currentstroke}%
\pgfsetdash{}{0pt}%
\pgfpathmoveto{\pgfqpoint{5.800000in}{0.720000in}}%
\pgfpathlineto{\pgfqpoint{5.800000in}{5.340000in}}%
\pgfusepath{stroke}%
\end{pgfscope}%
\begin{pgfscope}%
\pgfsetrectcap%
\pgfsetmiterjoin%
\pgfsetlinewidth{0.803000pt}%
\definecolor{currentstroke}{rgb}{0.000000,0.000000,0.000000}%
\pgfsetstrokecolor{currentstroke}%
\pgfsetdash{}{0pt}%
\pgfpathmoveto{\pgfqpoint{7.200000in}{0.720000in}}%
\pgfpathlineto{\pgfqpoint{7.200000in}{5.340000in}}%
\pgfusepath{stroke}%
\end{pgfscope}%
\begin{pgfscope}%
\pgfsetrectcap%
\pgfsetmiterjoin%
\pgfsetlinewidth{0.803000pt}%
\definecolor{currentstroke}{rgb}{0.000000,0.000000,0.000000}%
\pgfsetstrokecolor{currentstroke}%
\pgfsetdash{}{0pt}%
\pgfpathmoveto{\pgfqpoint{5.800000in}{0.720000in}}%
\pgfpathlineto{\pgfqpoint{7.200000in}{0.720000in}}%
\pgfusepath{stroke}%
\end{pgfscope}%
\begin{pgfscope}%
\pgfsetrectcap%
\pgfsetmiterjoin%
\pgfsetlinewidth{0.803000pt}%
\definecolor{currentstroke}{rgb}{0.000000,0.000000,0.000000}%
\pgfsetstrokecolor{currentstroke}%
\pgfsetdash{}{0pt}%
\pgfpathmoveto{\pgfqpoint{5.800000in}{5.340000in}}%
\pgfpathlineto{\pgfqpoint{7.200000in}{5.340000in}}%
\pgfusepath{stroke}%
\end{pgfscope}%
\end{pgfpicture}%
\makeatother%
\endgroup%
}
    \caption{\label{fig:mcmc-npe} $D_\mathrm{w}$ histogram and its distributions conditioned \\ on $N_{\mathrm{PE}}$, errorbar explained in figure~\ref{fig:cnn-performance}.}
  \end{subfigure}
  \begin{subfigure}{.5\textwidth}
    \centering
    \resizebox{\textwidth}{!}{%% Creator: Matplotlib, PGF backend
%%
%% To include the figure in your LaTeX document, write
%%   \input{<filename>.pgf}
%%
%% Make sure the required packages are loaded in your preamble
%%   \usepackage{pgf}
%%
%% and, on pdftex
%%   \usepackage[utf8]{inputenc}\DeclareUnicodeCharacter{2212}{-}
%%
%% or, on luatex and xetex
%%   \usepackage{unicode-math}
%%
%% Figures using additional raster images can only be included by \input if
%% they are in the same directory as the main LaTeX file. For loading figures
%% from other directories you can use the `import` package
%%   \usepackage{import}
%%
%% and then include the figures with
%%   \import{<path to file>}{<filename>.pgf}
%%
%% Matplotlib used the following preamble
%%   \usepackage[detect-all,locale=DE]{siunitx}
%%
\begingroup%
\makeatletter%
\begin{pgfpicture}%
\pgfpathrectangle{\pgfpointorigin}{\pgfqpoint{8.000000in}{6.000000in}}%
\pgfusepath{use as bounding box, clip}%
\begin{pgfscope}%
\pgfsetbuttcap%
\pgfsetmiterjoin%
\definecolor{currentfill}{rgb}{1.000000,1.000000,1.000000}%
\pgfsetfillcolor{currentfill}%
\pgfsetlinewidth{0.000000pt}%
\definecolor{currentstroke}{rgb}{1.000000,1.000000,1.000000}%
\pgfsetstrokecolor{currentstroke}%
\pgfsetdash{}{0pt}%
\pgfpathmoveto{\pgfqpoint{0.000000in}{0.000000in}}%
\pgfpathlineto{\pgfqpoint{8.000000in}{0.000000in}}%
\pgfpathlineto{\pgfqpoint{8.000000in}{6.000000in}}%
\pgfpathlineto{\pgfqpoint{0.000000in}{6.000000in}}%
\pgfpathclose%
\pgfusepath{fill}%
\end{pgfscope}%
\begin{pgfscope}%
\pgfsetbuttcap%
\pgfsetmiterjoin%
\definecolor{currentfill}{rgb}{1.000000,1.000000,1.000000}%
\pgfsetfillcolor{currentfill}%
\pgfsetlinewidth{0.000000pt}%
\definecolor{currentstroke}{rgb}{0.000000,0.000000,0.000000}%
\pgfsetstrokecolor{currentstroke}%
\pgfsetstrokeopacity{0.000000}%
\pgfsetdash{}{0pt}%
\pgfpathmoveto{\pgfqpoint{1.000000in}{0.720000in}}%
\pgfpathlineto{\pgfqpoint{7.200000in}{0.720000in}}%
\pgfpathlineto{\pgfqpoint{7.200000in}{5.340000in}}%
\pgfpathlineto{\pgfqpoint{1.000000in}{5.340000in}}%
\pgfpathclose%
\pgfusepath{fill}%
\end{pgfscope}%
\begin{pgfscope}%
\pgfsetbuttcap%
\pgfsetroundjoin%
\definecolor{currentfill}{rgb}{0.000000,0.000000,0.000000}%
\pgfsetfillcolor{currentfill}%
\pgfsetlinewidth{0.803000pt}%
\definecolor{currentstroke}{rgb}{0.000000,0.000000,0.000000}%
\pgfsetstrokecolor{currentstroke}%
\pgfsetdash{}{0pt}%
\pgfsys@defobject{currentmarker}{\pgfqpoint{0.000000in}{-0.048611in}}{\pgfqpoint{0.000000in}{0.000000in}}{%
\pgfpathmoveto{\pgfqpoint{0.000000in}{0.000000in}}%
\pgfpathlineto{\pgfqpoint{0.000000in}{-0.048611in}}%
\pgfusepath{stroke,fill}%
}%
\begin{pgfscope}%
\pgfsys@transformshift{2.240000in}{0.720000in}%
\pgfsys@useobject{currentmarker}{}%
\end{pgfscope}%
\end{pgfscope}%
\begin{pgfscope}%
\definecolor{textcolor}{rgb}{0.000000,0.000000,0.000000}%
\pgfsetstrokecolor{textcolor}%
\pgfsetfillcolor{textcolor}%
\pgftext[x=2.240000in,y=0.622778in,,top]{\color{textcolor}\sffamily\fontsize{20.000000}{24.000000}\selectfont \(\displaystyle {200}\)}%
\end{pgfscope}%
\begin{pgfscope}%
\pgfsetbuttcap%
\pgfsetroundjoin%
\definecolor{currentfill}{rgb}{0.000000,0.000000,0.000000}%
\pgfsetfillcolor{currentfill}%
\pgfsetlinewidth{0.803000pt}%
\definecolor{currentstroke}{rgb}{0.000000,0.000000,0.000000}%
\pgfsetstrokecolor{currentstroke}%
\pgfsetdash{}{0pt}%
\pgfsys@defobject{currentmarker}{\pgfqpoint{0.000000in}{-0.048611in}}{\pgfqpoint{0.000000in}{0.000000in}}{%
\pgfpathmoveto{\pgfqpoint{0.000000in}{0.000000in}}%
\pgfpathlineto{\pgfqpoint{0.000000in}{-0.048611in}}%
\pgfusepath{stroke,fill}%
}%
\begin{pgfscope}%
\pgfsys@transformshift{3.790000in}{0.720000in}%
\pgfsys@useobject{currentmarker}{}%
\end{pgfscope}%
\end{pgfscope}%
\begin{pgfscope}%
\definecolor{textcolor}{rgb}{0.000000,0.000000,0.000000}%
\pgfsetstrokecolor{textcolor}%
\pgfsetfillcolor{textcolor}%
\pgftext[x=3.790000in,y=0.622778in,,top]{\color{textcolor}\sffamily\fontsize{20.000000}{24.000000}\selectfont \(\displaystyle {250}\)}%
\end{pgfscope}%
\begin{pgfscope}%
\pgfsetbuttcap%
\pgfsetroundjoin%
\definecolor{currentfill}{rgb}{0.000000,0.000000,0.000000}%
\pgfsetfillcolor{currentfill}%
\pgfsetlinewidth{0.803000pt}%
\definecolor{currentstroke}{rgb}{0.000000,0.000000,0.000000}%
\pgfsetstrokecolor{currentstroke}%
\pgfsetdash{}{0pt}%
\pgfsys@defobject{currentmarker}{\pgfqpoint{0.000000in}{-0.048611in}}{\pgfqpoint{0.000000in}{0.000000in}}{%
\pgfpathmoveto{\pgfqpoint{0.000000in}{0.000000in}}%
\pgfpathlineto{\pgfqpoint{0.000000in}{-0.048611in}}%
\pgfusepath{stroke,fill}%
}%
\begin{pgfscope}%
\pgfsys@transformshift{5.340000in}{0.720000in}%
\pgfsys@useobject{currentmarker}{}%
\end{pgfscope}%
\end{pgfscope}%
\begin{pgfscope}%
\definecolor{textcolor}{rgb}{0.000000,0.000000,0.000000}%
\pgfsetstrokecolor{textcolor}%
\pgfsetfillcolor{textcolor}%
\pgftext[x=5.340000in,y=0.622778in,,top]{\color{textcolor}\sffamily\fontsize{20.000000}{24.000000}\selectfont \(\displaystyle {300}\)}%
\end{pgfscope}%
\begin{pgfscope}%
\pgfsetbuttcap%
\pgfsetroundjoin%
\definecolor{currentfill}{rgb}{0.000000,0.000000,0.000000}%
\pgfsetfillcolor{currentfill}%
\pgfsetlinewidth{0.803000pt}%
\definecolor{currentstroke}{rgb}{0.000000,0.000000,0.000000}%
\pgfsetstrokecolor{currentstroke}%
\pgfsetdash{}{0pt}%
\pgfsys@defobject{currentmarker}{\pgfqpoint{0.000000in}{-0.048611in}}{\pgfqpoint{0.000000in}{0.000000in}}{%
\pgfpathmoveto{\pgfqpoint{0.000000in}{0.000000in}}%
\pgfpathlineto{\pgfqpoint{0.000000in}{-0.048611in}}%
\pgfusepath{stroke,fill}%
}%
\begin{pgfscope}%
\pgfsys@transformshift{6.890000in}{0.720000in}%
\pgfsys@useobject{currentmarker}{}%
\end{pgfscope}%
\end{pgfscope}%
\begin{pgfscope}%
\definecolor{textcolor}{rgb}{0.000000,0.000000,0.000000}%
\pgfsetstrokecolor{textcolor}%
\pgfsetfillcolor{textcolor}%
\pgftext[x=6.890000in,y=0.622778in,,top]{\color{textcolor}\sffamily\fontsize{20.000000}{24.000000}\selectfont \(\displaystyle {350}\)}%
\end{pgfscope}%
\begin{pgfscope}%
\definecolor{textcolor}{rgb}{0.000000,0.000000,0.000000}%
\pgfsetstrokecolor{textcolor}%
\pgfsetfillcolor{textcolor}%
\pgftext[x=4.100000in,y=0.311155in,,top]{\color{textcolor}\sffamily\fontsize{20.000000}{24.000000}\selectfont \(\displaystyle \mathrm{t}/\si{ns}\)}%
\end{pgfscope}%
\begin{pgfscope}%
\pgfsetbuttcap%
\pgfsetroundjoin%
\definecolor{currentfill}{rgb}{0.000000,0.000000,0.000000}%
\pgfsetfillcolor{currentfill}%
\pgfsetlinewidth{0.803000pt}%
\definecolor{currentstroke}{rgb}{0.000000,0.000000,0.000000}%
\pgfsetstrokecolor{currentstroke}%
\pgfsetdash{}{0pt}%
\pgfsys@defobject{currentmarker}{\pgfqpoint{-0.048611in}{0.000000in}}{\pgfqpoint{-0.000000in}{0.000000in}}{%
\pgfpathmoveto{\pgfqpoint{-0.000000in}{0.000000in}}%
\pgfpathlineto{\pgfqpoint{-0.048611in}{0.000000in}}%
\pgfusepath{stroke,fill}%
}%
\begin{pgfscope}%
\pgfsys@transformshift{1.000000in}{1.216628in}%
\pgfsys@useobject{currentmarker}{}%
\end{pgfscope}%
\end{pgfscope}%
\begin{pgfscope}%
\definecolor{textcolor}{rgb}{0.000000,0.000000,0.000000}%
\pgfsetstrokecolor{textcolor}%
\pgfsetfillcolor{textcolor}%
\pgftext[x=0.770670in, y=1.116609in, left, base]{\color{textcolor}\sffamily\fontsize{20.000000}{24.000000}\selectfont \(\displaystyle {0}\)}%
\end{pgfscope}%
\begin{pgfscope}%
\pgfsetbuttcap%
\pgfsetroundjoin%
\definecolor{currentfill}{rgb}{0.000000,0.000000,0.000000}%
\pgfsetfillcolor{currentfill}%
\pgfsetlinewidth{0.803000pt}%
\definecolor{currentstroke}{rgb}{0.000000,0.000000,0.000000}%
\pgfsetstrokecolor{currentstroke}%
\pgfsetdash{}{0pt}%
\pgfsys@defobject{currentmarker}{\pgfqpoint{-0.048611in}{0.000000in}}{\pgfqpoint{-0.000000in}{0.000000in}}{%
\pgfpathmoveto{\pgfqpoint{-0.000000in}{0.000000in}}%
\pgfpathlineto{\pgfqpoint{-0.048611in}{0.000000in}}%
\pgfusepath{stroke,fill}%
}%
\begin{pgfscope}%
\pgfsys@transformshift{1.000000in}{2.368924in}%
\pgfsys@useobject{currentmarker}{}%
\end{pgfscope}%
\end{pgfscope}%
\begin{pgfscope}%
\definecolor{textcolor}{rgb}{0.000000,0.000000,0.000000}%
\pgfsetstrokecolor{textcolor}%
\pgfsetfillcolor{textcolor}%
\pgftext[x=0.638563in, y=2.268904in, left, base]{\color{textcolor}\sffamily\fontsize{20.000000}{24.000000}\selectfont \(\displaystyle {10}\)}%
\end{pgfscope}%
\begin{pgfscope}%
\pgfsetbuttcap%
\pgfsetroundjoin%
\definecolor{currentfill}{rgb}{0.000000,0.000000,0.000000}%
\pgfsetfillcolor{currentfill}%
\pgfsetlinewidth{0.803000pt}%
\definecolor{currentstroke}{rgb}{0.000000,0.000000,0.000000}%
\pgfsetstrokecolor{currentstroke}%
\pgfsetdash{}{0pt}%
\pgfsys@defobject{currentmarker}{\pgfqpoint{-0.048611in}{0.000000in}}{\pgfqpoint{-0.000000in}{0.000000in}}{%
\pgfpathmoveto{\pgfqpoint{-0.000000in}{0.000000in}}%
\pgfpathlineto{\pgfqpoint{-0.048611in}{0.000000in}}%
\pgfusepath{stroke,fill}%
}%
\begin{pgfscope}%
\pgfsys@transformshift{1.000000in}{3.521219in}%
\pgfsys@useobject{currentmarker}{}%
\end{pgfscope}%
\end{pgfscope}%
\begin{pgfscope}%
\definecolor{textcolor}{rgb}{0.000000,0.000000,0.000000}%
\pgfsetstrokecolor{textcolor}%
\pgfsetfillcolor{textcolor}%
\pgftext[x=0.638563in, y=3.421200in, left, base]{\color{textcolor}\sffamily\fontsize{20.000000}{24.000000}\selectfont \(\displaystyle {20}\)}%
\end{pgfscope}%
\begin{pgfscope}%
\pgfsetbuttcap%
\pgfsetroundjoin%
\definecolor{currentfill}{rgb}{0.000000,0.000000,0.000000}%
\pgfsetfillcolor{currentfill}%
\pgfsetlinewidth{0.803000pt}%
\definecolor{currentstroke}{rgb}{0.000000,0.000000,0.000000}%
\pgfsetstrokecolor{currentstroke}%
\pgfsetdash{}{0pt}%
\pgfsys@defobject{currentmarker}{\pgfqpoint{-0.048611in}{0.000000in}}{\pgfqpoint{-0.000000in}{0.000000in}}{%
\pgfpathmoveto{\pgfqpoint{-0.000000in}{0.000000in}}%
\pgfpathlineto{\pgfqpoint{-0.048611in}{0.000000in}}%
\pgfusepath{stroke,fill}%
}%
\begin{pgfscope}%
\pgfsys@transformshift{1.000000in}{4.673515in}%
\pgfsys@useobject{currentmarker}{}%
\end{pgfscope}%
\end{pgfscope}%
\begin{pgfscope}%
\definecolor{textcolor}{rgb}{0.000000,0.000000,0.000000}%
\pgfsetstrokecolor{textcolor}%
\pgfsetfillcolor{textcolor}%
\pgftext[x=0.638563in, y=4.573495in, left, base]{\color{textcolor}\sffamily\fontsize{20.000000}{24.000000}\selectfont \(\displaystyle {30}\)}%
\end{pgfscope}%
\begin{pgfscope}%
\definecolor{textcolor}{rgb}{0.000000,0.000000,0.000000}%
\pgfsetstrokecolor{textcolor}%
\pgfsetfillcolor{textcolor}%
\pgftext[x=0.583007in,y=3.030000in,,bottom,rotate=90.000000]{\color{textcolor}\sffamily\fontsize{20.000000}{24.000000}\selectfont \(\displaystyle \mathrm{Voltage}/\si{mV}\)}%
\end{pgfscope}%
\begin{pgfscope}%
\pgfpathrectangle{\pgfqpoint{1.000000in}{0.720000in}}{\pgfqpoint{6.200000in}{4.620000in}}%
\pgfusepath{clip}%
\pgfsetbuttcap%
\pgfsetroundjoin%
\pgfsetlinewidth{2.007500pt}%
\definecolor{currentstroke}{rgb}{0.000000,0.500000,0.000000}%
\pgfsetstrokecolor{currentstroke}%
\pgfsetdash{}{0pt}%
\pgfpathmoveto{\pgfqpoint{0.990000in}{1.792776in}}%
\pgfpathlineto{\pgfqpoint{7.210000in}{1.792776in}}%
\pgfusepath{stroke}%
\end{pgfscope}%
\begin{pgfscope}%
\pgfpathrectangle{\pgfqpoint{1.000000in}{0.720000in}}{\pgfqpoint{6.200000in}{4.620000in}}%
\pgfusepath{clip}%
\pgfsetrectcap%
\pgfsetroundjoin%
\pgfsetlinewidth{2.007500pt}%
\definecolor{currentstroke}{rgb}{0.121569,0.466667,0.705882}%
\pgfsetstrokecolor{currentstroke}%
\pgfsetdash{}{0pt}%
\pgfpathmoveto{\pgfqpoint{0.990000in}{1.259166in}}%
\pgfpathlineto{\pgfqpoint{1.000000in}{1.324513in}}%
\pgfpathlineto{\pgfqpoint{1.031000in}{1.216365in}}%
\pgfpathlineto{\pgfqpoint{1.062000in}{1.062244in}}%
\pgfpathlineto{\pgfqpoint{1.093000in}{1.230703in}}%
\pgfpathlineto{\pgfqpoint{1.124000in}{1.041274in}}%
\pgfpathlineto{\pgfqpoint{1.155000in}{1.303116in}}%
\pgfpathlineto{\pgfqpoint{1.186000in}{1.244552in}}%
\pgfpathlineto{\pgfqpoint{1.217000in}{1.162208in}}%
\pgfpathlineto{\pgfqpoint{1.248000in}{1.179449in}}%
\pgfpathlineto{\pgfqpoint{1.279000in}{1.348020in}}%
\pgfpathlineto{\pgfqpoint{1.310000in}{1.185473in}}%
\pgfpathlineto{\pgfqpoint{1.341000in}{1.290854in}}%
\pgfpathlineto{\pgfqpoint{1.372000in}{1.270902in}}%
\pgfpathlineto{\pgfqpoint{1.403000in}{1.428783in}}%
\pgfpathlineto{\pgfqpoint{1.434000in}{1.348261in}}%
\pgfpathlineto{\pgfqpoint{1.465000in}{1.147818in}}%
\pgfpathlineto{\pgfqpoint{1.496000in}{1.253873in}}%
\pgfpathlineto{\pgfqpoint{1.527000in}{1.326430in}}%
\pgfpathlineto{\pgfqpoint{1.558000in}{1.186665in}}%
\pgfpathlineto{\pgfqpoint{1.589000in}{1.065293in}}%
\pgfpathlineto{\pgfqpoint{1.651000in}{1.114523in}}%
\pgfpathlineto{\pgfqpoint{1.682000in}{1.234776in}}%
\pgfpathlineto{\pgfqpoint{1.713000in}{1.304042in}}%
\pgfpathlineto{\pgfqpoint{1.744000in}{1.268083in}}%
\pgfpathlineto{\pgfqpoint{1.775000in}{1.089604in}}%
\pgfpathlineto{\pgfqpoint{1.806000in}{1.304014in}}%
\pgfpathlineto{\pgfqpoint{1.837000in}{1.375344in}}%
\pgfpathlineto{\pgfqpoint{1.868000in}{1.210887in}}%
\pgfpathlineto{\pgfqpoint{1.899000in}{1.294952in}}%
\pgfpathlineto{\pgfqpoint{1.930000in}{1.239778in}}%
\pgfpathlineto{\pgfqpoint{1.961000in}{1.241811in}}%
\pgfpathlineto{\pgfqpoint{1.992000in}{1.131081in}}%
\pgfpathlineto{\pgfqpoint{2.023000in}{1.090292in}}%
\pgfpathlineto{\pgfqpoint{2.054000in}{1.421251in}}%
\pgfpathlineto{\pgfqpoint{2.085000in}{1.099354in}}%
\pgfpathlineto{\pgfqpoint{2.116000in}{1.246851in}}%
\pgfpathlineto{\pgfqpoint{2.147000in}{1.408942in}}%
\pgfpathlineto{\pgfqpoint{2.178000in}{1.216381in}}%
\pgfpathlineto{\pgfqpoint{2.209000in}{1.192902in}}%
\pgfpathlineto{\pgfqpoint{2.240000in}{1.482274in}}%
\pgfpathlineto{\pgfqpoint{2.271000in}{1.190138in}}%
\pgfpathlineto{\pgfqpoint{2.302000in}{1.587539in}}%
\pgfpathlineto{\pgfqpoint{2.333000in}{1.815340in}}%
\pgfpathlineto{\pgfqpoint{2.364000in}{1.966713in}}%
\pgfpathlineto{\pgfqpoint{2.395000in}{2.070177in}}%
\pgfpathlineto{\pgfqpoint{2.426000in}{2.104790in}}%
\pgfpathlineto{\pgfqpoint{2.457000in}{2.209770in}}%
\pgfpathlineto{\pgfqpoint{2.488000in}{2.250044in}}%
\pgfpathlineto{\pgfqpoint{2.550000in}{2.048327in}}%
\pgfpathlineto{\pgfqpoint{2.581000in}{1.890512in}}%
\pgfpathlineto{\pgfqpoint{2.612000in}{1.726305in}}%
\pgfpathlineto{\pgfqpoint{2.643000in}{1.861973in}}%
\pgfpathlineto{\pgfqpoint{2.674000in}{1.494055in}}%
\pgfpathlineto{\pgfqpoint{2.705000in}{1.667500in}}%
\pgfpathlineto{\pgfqpoint{2.736000in}{1.997864in}}%
\pgfpathlineto{\pgfqpoint{2.767000in}{2.049254in}}%
\pgfpathlineto{\pgfqpoint{2.798000in}{2.570207in}}%
\pgfpathlineto{\pgfqpoint{2.829000in}{2.720641in}}%
\pgfpathlineto{\pgfqpoint{2.860000in}{2.741739in}}%
\pgfpathlineto{\pgfqpoint{2.891000in}{2.982150in}}%
\pgfpathlineto{\pgfqpoint{2.922000in}{2.479577in}}%
\pgfpathlineto{\pgfqpoint{2.953000in}{2.576149in}}%
\pgfpathlineto{\pgfqpoint{2.984000in}{2.577245in}}%
\pgfpathlineto{\pgfqpoint{3.015000in}{2.279583in}}%
\pgfpathlineto{\pgfqpoint{3.046000in}{2.748119in}}%
\pgfpathlineto{\pgfqpoint{3.077000in}{3.314676in}}%
\pgfpathlineto{\pgfqpoint{3.108000in}{3.267783in}}%
\pgfpathlineto{\pgfqpoint{3.139000in}{3.312337in}}%
\pgfpathlineto{\pgfqpoint{3.170000in}{3.340504in}}%
\pgfpathlineto{\pgfqpoint{3.201000in}{3.246880in}}%
\pgfpathlineto{\pgfqpoint{3.232000in}{3.434146in}}%
\pgfpathlineto{\pgfqpoint{3.263000in}{3.516186in}}%
\pgfpathlineto{\pgfqpoint{3.294000in}{3.304102in}}%
\pgfpathlineto{\pgfqpoint{3.325000in}{3.227273in}}%
\pgfpathlineto{\pgfqpoint{3.356000in}{3.146840in}}%
\pgfpathlineto{\pgfqpoint{3.387000in}{3.004278in}}%
\pgfpathlineto{\pgfqpoint{3.418000in}{2.782934in}}%
\pgfpathlineto{\pgfqpoint{3.449000in}{2.672557in}}%
\pgfpathlineto{\pgfqpoint{3.480000in}{2.307185in}}%
\pgfpathlineto{\pgfqpoint{3.511000in}{2.013032in}}%
\pgfpathlineto{\pgfqpoint{3.542000in}{1.973499in}}%
\pgfpathlineto{\pgfqpoint{3.573000in}{2.113799in}}%
\pgfpathlineto{\pgfqpoint{3.604000in}{1.871462in}}%
\pgfpathlineto{\pgfqpoint{3.635000in}{1.752360in}}%
\pgfpathlineto{\pgfqpoint{3.666000in}{1.492732in}}%
\pgfpathlineto{\pgfqpoint{3.697000in}{1.388334in}}%
\pgfpathlineto{\pgfqpoint{3.728000in}{1.423544in}}%
\pgfpathlineto{\pgfqpoint{3.759000in}{1.290665in}}%
\pgfpathlineto{\pgfqpoint{3.790000in}{1.584987in}}%
\pgfpathlineto{\pgfqpoint{3.821000in}{1.765486in}}%
\pgfpathlineto{\pgfqpoint{3.852000in}{2.006149in}}%
\pgfpathlineto{\pgfqpoint{3.883000in}{2.074590in}}%
\pgfpathlineto{\pgfqpoint{3.914000in}{1.996375in}}%
\pgfpathlineto{\pgfqpoint{3.945000in}{2.250922in}}%
\pgfpathlineto{\pgfqpoint{3.976000in}{2.156159in}}%
\pgfpathlineto{\pgfqpoint{4.007000in}{2.000141in}}%
\pgfpathlineto{\pgfqpoint{4.038000in}{1.967183in}}%
\pgfpathlineto{\pgfqpoint{4.069000in}{1.898653in}}%
\pgfpathlineto{\pgfqpoint{4.100000in}{1.772321in}}%
\pgfpathlineto{\pgfqpoint{4.131000in}{1.725336in}}%
\pgfpathlineto{\pgfqpoint{4.162000in}{1.452591in}}%
\pgfpathlineto{\pgfqpoint{4.193000in}{1.303933in}}%
\pgfpathlineto{\pgfqpoint{4.224000in}{1.500766in}}%
\pgfpathlineto{\pgfqpoint{4.255000in}{1.493246in}}%
\pgfpathlineto{\pgfqpoint{4.286000in}{1.454263in}}%
\pgfpathlineto{\pgfqpoint{4.317000in}{1.337234in}}%
\pgfpathlineto{\pgfqpoint{4.348000in}{1.330732in}}%
\pgfpathlineto{\pgfqpoint{4.379000in}{1.484354in}}%
\pgfpathlineto{\pgfqpoint{4.410000in}{1.176350in}}%
\pgfpathlineto{\pgfqpoint{4.441000in}{1.336339in}}%
\pgfpathlineto{\pgfqpoint{4.472000in}{1.220422in}}%
\pgfpathlineto{\pgfqpoint{4.503000in}{1.206405in}}%
\pgfpathlineto{\pgfqpoint{4.534000in}{1.294305in}}%
\pgfpathlineto{\pgfqpoint{4.565000in}{1.347051in}}%
\pgfpathlineto{\pgfqpoint{4.596000in}{1.174563in}}%
\pgfpathlineto{\pgfqpoint{4.627000in}{1.307704in}}%
\pgfpathlineto{\pgfqpoint{4.658000in}{1.352711in}}%
\pgfpathlineto{\pgfqpoint{4.689000in}{1.317780in}}%
\pgfpathlineto{\pgfqpoint{4.720000in}{1.195136in}}%
\pgfpathlineto{\pgfqpoint{4.751000in}{1.166124in}}%
\pgfpathlineto{\pgfqpoint{4.782000in}{0.959072in}}%
\pgfpathlineto{\pgfqpoint{4.813000in}{1.261171in}}%
\pgfpathlineto{\pgfqpoint{4.844000in}{1.296112in}}%
\pgfpathlineto{\pgfqpoint{4.875000in}{1.240588in}}%
\pgfpathlineto{\pgfqpoint{4.906000in}{1.067230in}}%
\pgfpathlineto{\pgfqpoint{4.937000in}{1.311625in}}%
\pgfpathlineto{\pgfqpoint{4.968000in}{1.036103in}}%
\pgfpathlineto{\pgfqpoint{4.999000in}{1.106575in}}%
\pgfpathlineto{\pgfqpoint{5.030000in}{1.217254in}}%
\pgfpathlineto{\pgfqpoint{5.061000in}{1.266262in}}%
\pgfpathlineto{\pgfqpoint{5.092000in}{1.143094in}}%
\pgfpathlineto{\pgfqpoint{5.123000in}{1.236623in}}%
\pgfpathlineto{\pgfqpoint{5.154000in}{1.215406in}}%
\pgfpathlineto{\pgfqpoint{5.185000in}{1.131646in}}%
\pgfpathlineto{\pgfqpoint{5.216000in}{1.241081in}}%
\pgfpathlineto{\pgfqpoint{5.247000in}{1.261583in}}%
\pgfpathlineto{\pgfqpoint{5.278000in}{1.225657in}}%
\pgfpathlineto{\pgfqpoint{5.309000in}{0.879922in}}%
\pgfpathlineto{\pgfqpoint{5.340000in}{1.327029in}}%
\pgfpathlineto{\pgfqpoint{5.371000in}{1.197321in}}%
\pgfpathlineto{\pgfqpoint{5.402000in}{1.359345in}}%
\pgfpathlineto{\pgfqpoint{5.433000in}{1.259374in}}%
\pgfpathlineto{\pgfqpoint{5.464000in}{1.300059in}}%
\pgfpathlineto{\pgfqpoint{5.495000in}{1.242462in}}%
\pgfpathlineto{\pgfqpoint{5.526000in}{1.277393in}}%
\pgfpathlineto{\pgfqpoint{5.557000in}{1.140668in}}%
\pgfpathlineto{\pgfqpoint{5.588000in}{1.191909in}}%
\pgfpathlineto{\pgfqpoint{5.619000in}{1.374748in}}%
\pgfpathlineto{\pgfqpoint{5.650000in}{1.253916in}}%
\pgfpathlineto{\pgfqpoint{5.681000in}{1.194244in}}%
\pgfpathlineto{\pgfqpoint{5.712000in}{1.091504in}}%
\pgfpathlineto{\pgfqpoint{5.743000in}{1.292385in}}%
\pgfpathlineto{\pgfqpoint{5.774000in}{1.263252in}}%
\pgfpathlineto{\pgfqpoint{5.805000in}{1.263979in}}%
\pgfpathlineto{\pgfqpoint{5.836000in}{1.126904in}}%
\pgfpathlineto{\pgfqpoint{5.867000in}{1.196833in}}%
\pgfpathlineto{\pgfqpoint{5.898000in}{1.346917in}}%
\pgfpathlineto{\pgfqpoint{5.929000in}{1.099555in}}%
\pgfpathlineto{\pgfqpoint{5.960000in}{1.470179in}}%
\pgfpathlineto{\pgfqpoint{5.991000in}{1.188071in}}%
\pgfpathlineto{\pgfqpoint{6.022000in}{1.232076in}}%
\pgfpathlineto{\pgfqpoint{6.053000in}{1.183611in}}%
\pgfpathlineto{\pgfqpoint{6.084000in}{1.189938in}}%
\pgfpathlineto{\pgfqpoint{6.115000in}{1.066875in}}%
\pgfpathlineto{\pgfqpoint{6.146000in}{1.194354in}}%
\pgfpathlineto{\pgfqpoint{6.177000in}{1.270541in}}%
\pgfpathlineto{\pgfqpoint{6.208000in}{1.326104in}}%
\pgfpathlineto{\pgfqpoint{6.239000in}{1.237888in}}%
\pgfpathlineto{\pgfqpoint{6.270000in}{1.144805in}}%
\pgfpathlineto{\pgfqpoint{6.301000in}{1.139158in}}%
\pgfpathlineto{\pgfqpoint{6.332000in}{1.256961in}}%
\pgfpathlineto{\pgfqpoint{6.363000in}{1.189772in}}%
\pgfpathlineto{\pgfqpoint{6.394000in}{1.335674in}}%
\pgfpathlineto{\pgfqpoint{6.425000in}{1.436643in}}%
\pgfpathlineto{\pgfqpoint{6.456000in}{1.208226in}}%
\pgfpathlineto{\pgfqpoint{6.487000in}{1.160169in}}%
\pgfpathlineto{\pgfqpoint{6.518000in}{1.355090in}}%
\pgfpathlineto{\pgfqpoint{6.549000in}{1.002682in}}%
\pgfpathlineto{\pgfqpoint{6.580000in}{1.268253in}}%
\pgfpathlineto{\pgfqpoint{6.611000in}{1.240896in}}%
\pgfpathlineto{\pgfqpoint{6.642000in}{1.220780in}}%
\pgfpathlineto{\pgfqpoint{6.673000in}{1.282894in}}%
\pgfpathlineto{\pgfqpoint{6.704000in}{1.218320in}}%
\pgfpathlineto{\pgfqpoint{6.735000in}{1.224396in}}%
\pgfpathlineto{\pgfqpoint{6.766000in}{1.158163in}}%
\pgfpathlineto{\pgfqpoint{6.797000in}{1.265718in}}%
\pgfpathlineto{\pgfqpoint{6.828000in}{1.226415in}}%
\pgfpathlineto{\pgfqpoint{6.859000in}{1.252724in}}%
\pgfpathlineto{\pgfqpoint{6.890000in}{1.089848in}}%
\pgfpathlineto{\pgfqpoint{6.921000in}{1.390178in}}%
\pgfpathlineto{\pgfqpoint{6.952000in}{1.384890in}}%
\pgfpathlineto{\pgfqpoint{6.983000in}{1.302144in}}%
\pgfpathlineto{\pgfqpoint{7.014000in}{1.320951in}}%
\pgfpathlineto{\pgfqpoint{7.045000in}{1.314784in}}%
\pgfpathlineto{\pgfqpoint{7.076000in}{1.006810in}}%
\pgfpathlineto{\pgfqpoint{7.107000in}{1.256666in}}%
\pgfpathlineto{\pgfqpoint{7.138000in}{1.235709in}}%
\pgfpathlineto{\pgfqpoint{7.169000in}{1.190949in}}%
\pgfpathlineto{\pgfqpoint{7.200000in}{1.253171in}}%
\pgfpathlineto{\pgfqpoint{7.210000in}{1.293453in}}%
\pgfpathlineto{\pgfqpoint{7.210000in}{1.293453in}}%
\pgfusepath{stroke}%
\end{pgfscope}%
\begin{pgfscope}%
\pgfsetrectcap%
\pgfsetmiterjoin%
\pgfsetlinewidth{0.803000pt}%
\definecolor{currentstroke}{rgb}{0.000000,0.000000,0.000000}%
\pgfsetstrokecolor{currentstroke}%
\pgfsetdash{}{0pt}%
\pgfpathmoveto{\pgfqpoint{1.000000in}{0.720000in}}%
\pgfpathlineto{\pgfqpoint{1.000000in}{5.340000in}}%
\pgfusepath{stroke}%
\end{pgfscope}%
\begin{pgfscope}%
\pgfsetrectcap%
\pgfsetmiterjoin%
\pgfsetlinewidth{0.803000pt}%
\definecolor{currentstroke}{rgb}{0.000000,0.000000,0.000000}%
\pgfsetstrokecolor{currentstroke}%
\pgfsetdash{}{0pt}%
\pgfpathmoveto{\pgfqpoint{7.200000in}{0.720000in}}%
\pgfpathlineto{\pgfqpoint{7.200000in}{5.340000in}}%
\pgfusepath{stroke}%
\end{pgfscope}%
\begin{pgfscope}%
\pgfsetrectcap%
\pgfsetmiterjoin%
\pgfsetlinewidth{0.803000pt}%
\definecolor{currentstroke}{rgb}{0.000000,0.000000,0.000000}%
\pgfsetstrokecolor{currentstroke}%
\pgfsetdash{}{0pt}%
\pgfpathmoveto{\pgfqpoint{1.000000in}{0.720000in}}%
\pgfpathlineto{\pgfqpoint{7.200000in}{0.720000in}}%
\pgfusepath{stroke}%
\end{pgfscope}%
\begin{pgfscope}%
\pgfsetrectcap%
\pgfsetmiterjoin%
\pgfsetlinewidth{0.803000pt}%
\definecolor{currentstroke}{rgb}{0.000000,0.000000,0.000000}%
\pgfsetstrokecolor{currentstroke}%
\pgfsetdash{}{0pt}%
\pgfpathmoveto{\pgfqpoint{1.000000in}{5.340000in}}%
\pgfpathlineto{\pgfqpoint{7.200000in}{5.340000in}}%
\pgfusepath{stroke}%
\end{pgfscope}%
\begin{pgfscope}%
\pgfsetbuttcap%
\pgfsetroundjoin%
\definecolor{currentfill}{rgb}{0.000000,0.000000,0.000000}%
\pgfsetfillcolor{currentfill}%
\pgfsetlinewidth{0.803000pt}%
\definecolor{currentstroke}{rgb}{0.000000,0.000000,0.000000}%
\pgfsetstrokecolor{currentstroke}%
\pgfsetdash{}{0pt}%
\pgfsys@defobject{currentmarker}{\pgfqpoint{0.000000in}{0.000000in}}{\pgfqpoint{0.048611in}{0.000000in}}{%
\pgfpathmoveto{\pgfqpoint{0.000000in}{0.000000in}}%
\pgfpathlineto{\pgfqpoint{0.048611in}{0.000000in}}%
\pgfusepath{stroke,fill}%
}%
\begin{pgfscope}%
\pgfsys@transformshift{7.200000in}{1.216628in}%
\pgfsys@useobject{currentmarker}{}%
\end{pgfscope}%
\end{pgfscope}%
\begin{pgfscope}%
\definecolor{textcolor}{rgb}{0.000000,0.000000,0.000000}%
\pgfsetstrokecolor{textcolor}%
\pgfsetfillcolor{textcolor}%
\pgftext[x=7.297222in, y=1.116609in, left, base]{\color{textcolor}\sffamily\fontsize{20.000000}{24.000000}\selectfont 0.0}%
\end{pgfscope}%
\begin{pgfscope}%
\pgfsetbuttcap%
\pgfsetroundjoin%
\definecolor{currentfill}{rgb}{0.000000,0.000000,0.000000}%
\pgfsetfillcolor{currentfill}%
\pgfsetlinewidth{0.803000pt}%
\definecolor{currentstroke}{rgb}{0.000000,0.000000,0.000000}%
\pgfsetstrokecolor{currentstroke}%
\pgfsetdash{}{0pt}%
\pgfsys@defobject{currentmarker}{\pgfqpoint{0.000000in}{0.000000in}}{\pgfqpoint{0.048611in}{0.000000in}}{%
\pgfpathmoveto{\pgfqpoint{0.000000in}{0.000000in}}%
\pgfpathlineto{\pgfqpoint{0.048611in}{0.000000in}}%
\pgfusepath{stroke,fill}%
}%
\begin{pgfscope}%
\pgfsys@transformshift{7.200000in}{1.844647in}%
\pgfsys@useobject{currentmarker}{}%
\end{pgfscope}%
\end{pgfscope}%
\begin{pgfscope}%
\definecolor{textcolor}{rgb}{0.000000,0.000000,0.000000}%
\pgfsetstrokecolor{textcolor}%
\pgfsetfillcolor{textcolor}%
\pgftext[x=7.297222in, y=1.744628in, left, base]{\color{textcolor}\sffamily\fontsize{20.000000}{24.000000}\selectfont 0.1}%
\end{pgfscope}%
\begin{pgfscope}%
\pgfsetbuttcap%
\pgfsetroundjoin%
\definecolor{currentfill}{rgb}{0.000000,0.000000,0.000000}%
\pgfsetfillcolor{currentfill}%
\pgfsetlinewidth{0.803000pt}%
\definecolor{currentstroke}{rgb}{0.000000,0.000000,0.000000}%
\pgfsetstrokecolor{currentstroke}%
\pgfsetdash{}{0pt}%
\pgfsys@defobject{currentmarker}{\pgfqpoint{0.000000in}{0.000000in}}{\pgfqpoint{0.048611in}{0.000000in}}{%
\pgfpathmoveto{\pgfqpoint{0.000000in}{0.000000in}}%
\pgfpathlineto{\pgfqpoint{0.048611in}{0.000000in}}%
\pgfusepath{stroke,fill}%
}%
\begin{pgfscope}%
\pgfsys@transformshift{7.200000in}{2.472666in}%
\pgfsys@useobject{currentmarker}{}%
\end{pgfscope}%
\end{pgfscope}%
\begin{pgfscope}%
\definecolor{textcolor}{rgb}{0.000000,0.000000,0.000000}%
\pgfsetstrokecolor{textcolor}%
\pgfsetfillcolor{textcolor}%
\pgftext[x=7.297222in, y=2.372646in, left, base]{\color{textcolor}\sffamily\fontsize{20.000000}{24.000000}\selectfont 0.2}%
\end{pgfscope}%
\begin{pgfscope}%
\pgfsetbuttcap%
\pgfsetroundjoin%
\definecolor{currentfill}{rgb}{0.000000,0.000000,0.000000}%
\pgfsetfillcolor{currentfill}%
\pgfsetlinewidth{0.803000pt}%
\definecolor{currentstroke}{rgb}{0.000000,0.000000,0.000000}%
\pgfsetstrokecolor{currentstroke}%
\pgfsetdash{}{0pt}%
\pgfsys@defobject{currentmarker}{\pgfqpoint{0.000000in}{0.000000in}}{\pgfqpoint{0.048611in}{0.000000in}}{%
\pgfpathmoveto{\pgfqpoint{0.000000in}{0.000000in}}%
\pgfpathlineto{\pgfqpoint{0.048611in}{0.000000in}}%
\pgfusepath{stroke,fill}%
}%
\begin{pgfscope}%
\pgfsys@transformshift{7.200000in}{3.100685in}%
\pgfsys@useobject{currentmarker}{}%
\end{pgfscope}%
\end{pgfscope}%
\begin{pgfscope}%
\definecolor{textcolor}{rgb}{0.000000,0.000000,0.000000}%
\pgfsetstrokecolor{textcolor}%
\pgfsetfillcolor{textcolor}%
\pgftext[x=7.297222in, y=3.000665in, left, base]{\color{textcolor}\sffamily\fontsize{20.000000}{24.000000}\selectfont 0.3}%
\end{pgfscope}%
\begin{pgfscope}%
\pgfsetbuttcap%
\pgfsetroundjoin%
\definecolor{currentfill}{rgb}{0.000000,0.000000,0.000000}%
\pgfsetfillcolor{currentfill}%
\pgfsetlinewidth{0.803000pt}%
\definecolor{currentstroke}{rgb}{0.000000,0.000000,0.000000}%
\pgfsetstrokecolor{currentstroke}%
\pgfsetdash{}{0pt}%
\pgfsys@defobject{currentmarker}{\pgfqpoint{0.000000in}{0.000000in}}{\pgfqpoint{0.048611in}{0.000000in}}{%
\pgfpathmoveto{\pgfqpoint{0.000000in}{0.000000in}}%
\pgfpathlineto{\pgfqpoint{0.048611in}{0.000000in}}%
\pgfusepath{stroke,fill}%
}%
\begin{pgfscope}%
\pgfsys@transformshift{7.200000in}{3.728703in}%
\pgfsys@useobject{currentmarker}{}%
\end{pgfscope}%
\end{pgfscope}%
\begin{pgfscope}%
\definecolor{textcolor}{rgb}{0.000000,0.000000,0.000000}%
\pgfsetstrokecolor{textcolor}%
\pgfsetfillcolor{textcolor}%
\pgftext[x=7.297222in, y=3.628684in, left, base]{\color{textcolor}\sffamily\fontsize{20.000000}{24.000000}\selectfont 0.4}%
\end{pgfscope}%
\begin{pgfscope}%
\pgfsetbuttcap%
\pgfsetroundjoin%
\definecolor{currentfill}{rgb}{0.000000,0.000000,0.000000}%
\pgfsetfillcolor{currentfill}%
\pgfsetlinewidth{0.803000pt}%
\definecolor{currentstroke}{rgb}{0.000000,0.000000,0.000000}%
\pgfsetstrokecolor{currentstroke}%
\pgfsetdash{}{0pt}%
\pgfsys@defobject{currentmarker}{\pgfqpoint{0.000000in}{0.000000in}}{\pgfqpoint{0.048611in}{0.000000in}}{%
\pgfpathmoveto{\pgfqpoint{0.000000in}{0.000000in}}%
\pgfpathlineto{\pgfqpoint{0.048611in}{0.000000in}}%
\pgfusepath{stroke,fill}%
}%
\begin{pgfscope}%
\pgfsys@transformshift{7.200000in}{4.356722in}%
\pgfsys@useobject{currentmarker}{}%
\end{pgfscope}%
\end{pgfscope}%
\begin{pgfscope}%
\definecolor{textcolor}{rgb}{0.000000,0.000000,0.000000}%
\pgfsetstrokecolor{textcolor}%
\pgfsetfillcolor{textcolor}%
\pgftext[x=7.297222in, y=4.256703in, left, base]{\color{textcolor}\sffamily\fontsize{20.000000}{24.000000}\selectfont 0.5}%
\end{pgfscope}%
\begin{pgfscope}%
\pgfsetbuttcap%
\pgfsetroundjoin%
\definecolor{currentfill}{rgb}{0.000000,0.000000,0.000000}%
\pgfsetfillcolor{currentfill}%
\pgfsetlinewidth{0.803000pt}%
\definecolor{currentstroke}{rgb}{0.000000,0.000000,0.000000}%
\pgfsetstrokecolor{currentstroke}%
\pgfsetdash{}{0pt}%
\pgfsys@defobject{currentmarker}{\pgfqpoint{0.000000in}{0.000000in}}{\pgfqpoint{0.048611in}{0.000000in}}{%
\pgfpathmoveto{\pgfqpoint{0.000000in}{0.000000in}}%
\pgfpathlineto{\pgfqpoint{0.048611in}{0.000000in}}%
\pgfusepath{stroke,fill}%
}%
\begin{pgfscope}%
\pgfsys@transformshift{7.200000in}{4.984741in}%
\pgfsys@useobject{currentmarker}{}%
\end{pgfscope}%
\end{pgfscope}%
\begin{pgfscope}%
\definecolor{textcolor}{rgb}{0.000000,0.000000,0.000000}%
\pgfsetstrokecolor{textcolor}%
\pgfsetfillcolor{textcolor}%
\pgftext[x=7.297222in, y=4.884722in, left, base]{\color{textcolor}\sffamily\fontsize{20.000000}{24.000000}\selectfont 0.6}%
\end{pgfscope}%
\begin{pgfscope}%
\definecolor{textcolor}{rgb}{0.000000,0.000000,0.000000}%
\pgfsetstrokecolor{textcolor}%
\pgfsetfillcolor{textcolor}%
\pgftext[x=7.698906in,y=3.030000in,,top,rotate=90.000000]{\color{textcolor}\sffamily\fontsize{20.000000}{24.000000}\selectfont \(\displaystyle \mathrm{Charge}\)}%
\end{pgfscope}%
\begin{pgfscope}%
\pgfpathrectangle{\pgfqpoint{1.000000in}{0.720000in}}{\pgfqpoint{6.200000in}{4.620000in}}%
\pgfusepath{clip}%
\pgfsetbuttcap%
\pgfsetroundjoin%
\pgfsetlinewidth{0.501875pt}%
\definecolor{currentstroke}{rgb}{1.000000,0.000000,0.000000}%
\pgfsetstrokecolor{currentstroke}%
\pgfsetdash{}{0pt}%
\pgfpathmoveto{\pgfqpoint{2.054000in}{1.216628in}}%
\pgfpathlineto{\pgfqpoint{2.054000in}{1.217097in}}%
\pgfusepath{stroke}%
\end{pgfscope}%
\begin{pgfscope}%
\pgfpathrectangle{\pgfqpoint{1.000000in}{0.720000in}}{\pgfqpoint{6.200000in}{4.620000in}}%
\pgfusepath{clip}%
\pgfsetbuttcap%
\pgfsetroundjoin%
\pgfsetlinewidth{0.501875pt}%
\definecolor{currentstroke}{rgb}{1.000000,0.000000,0.000000}%
\pgfsetstrokecolor{currentstroke}%
\pgfsetdash{}{0pt}%
\pgfpathmoveto{\pgfqpoint{2.069500in}{1.216628in}}%
\pgfpathlineto{\pgfqpoint{2.069500in}{1.217174in}}%
\pgfusepath{stroke}%
\end{pgfscope}%
\begin{pgfscope}%
\pgfpathrectangle{\pgfqpoint{1.000000in}{0.720000in}}{\pgfqpoint{6.200000in}{4.620000in}}%
\pgfusepath{clip}%
\pgfsetbuttcap%
\pgfsetroundjoin%
\pgfsetlinewidth{0.501875pt}%
\definecolor{currentstroke}{rgb}{1.000000,0.000000,0.000000}%
\pgfsetstrokecolor{currentstroke}%
\pgfsetdash{}{0pt}%
\pgfpathmoveto{\pgfqpoint{2.085000in}{1.216628in}}%
\pgfpathlineto{\pgfqpoint{2.085000in}{1.217055in}}%
\pgfusepath{stroke}%
\end{pgfscope}%
\begin{pgfscope}%
\pgfpathrectangle{\pgfqpoint{1.000000in}{0.720000in}}{\pgfqpoint{6.200000in}{4.620000in}}%
\pgfusepath{clip}%
\pgfsetbuttcap%
\pgfsetroundjoin%
\pgfsetlinewidth{0.501875pt}%
\definecolor{currentstroke}{rgb}{1.000000,0.000000,0.000000}%
\pgfsetstrokecolor{currentstroke}%
\pgfsetdash{}{0pt}%
\pgfpathmoveto{\pgfqpoint{2.100500in}{1.216628in}}%
\pgfpathlineto{\pgfqpoint{2.100500in}{1.216974in}}%
\pgfusepath{stroke}%
\end{pgfscope}%
\begin{pgfscope}%
\pgfpathrectangle{\pgfqpoint{1.000000in}{0.720000in}}{\pgfqpoint{6.200000in}{4.620000in}}%
\pgfusepath{clip}%
\pgfsetbuttcap%
\pgfsetroundjoin%
\pgfsetlinewidth{0.501875pt}%
\definecolor{currentstroke}{rgb}{1.000000,0.000000,0.000000}%
\pgfsetstrokecolor{currentstroke}%
\pgfsetdash{}{0pt}%
\pgfpathmoveto{\pgfqpoint{2.116000in}{1.216628in}}%
\pgfpathlineto{\pgfqpoint{2.116000in}{1.217072in}}%
\pgfusepath{stroke}%
\end{pgfscope}%
\begin{pgfscope}%
\pgfpathrectangle{\pgfqpoint{1.000000in}{0.720000in}}{\pgfqpoint{6.200000in}{4.620000in}}%
\pgfusepath{clip}%
\pgfsetbuttcap%
\pgfsetroundjoin%
\pgfsetlinewidth{0.501875pt}%
\definecolor{currentstroke}{rgb}{1.000000,0.000000,0.000000}%
\pgfsetstrokecolor{currentstroke}%
\pgfsetdash{}{0pt}%
\pgfpathmoveto{\pgfqpoint{2.131500in}{1.216628in}}%
\pgfpathlineto{\pgfqpoint{2.131500in}{1.216738in}}%
\pgfusepath{stroke}%
\end{pgfscope}%
\begin{pgfscope}%
\pgfpathrectangle{\pgfqpoint{1.000000in}{0.720000in}}{\pgfqpoint{6.200000in}{4.620000in}}%
\pgfusepath{clip}%
\pgfsetbuttcap%
\pgfsetroundjoin%
\pgfsetlinewidth{0.501875pt}%
\definecolor{currentstroke}{rgb}{1.000000,0.000000,0.000000}%
\pgfsetstrokecolor{currentstroke}%
\pgfsetdash{}{0pt}%
\pgfpathmoveto{\pgfqpoint{2.147000in}{1.216628in}}%
\pgfpathlineto{\pgfqpoint{2.147000in}{1.216650in}}%
\pgfusepath{stroke}%
\end{pgfscope}%
\begin{pgfscope}%
\pgfpathrectangle{\pgfqpoint{1.000000in}{0.720000in}}{\pgfqpoint{6.200000in}{4.620000in}}%
\pgfusepath{clip}%
\pgfsetbuttcap%
\pgfsetroundjoin%
\pgfsetlinewidth{0.501875pt}%
\definecolor{currentstroke}{rgb}{1.000000,0.000000,0.000000}%
\pgfsetstrokecolor{currentstroke}%
\pgfsetdash{}{0pt}%
\pgfpathmoveto{\pgfqpoint{2.162500in}{1.216628in}}%
\pgfpathlineto{\pgfqpoint{2.162500in}{1.216831in}}%
\pgfusepath{stroke}%
\end{pgfscope}%
\begin{pgfscope}%
\pgfpathrectangle{\pgfqpoint{1.000000in}{0.720000in}}{\pgfqpoint{6.200000in}{4.620000in}}%
\pgfusepath{clip}%
\pgfsetbuttcap%
\pgfsetroundjoin%
\pgfsetlinewidth{0.501875pt}%
\definecolor{currentstroke}{rgb}{1.000000,0.000000,0.000000}%
\pgfsetstrokecolor{currentstroke}%
\pgfsetdash{}{0pt}%
\pgfpathmoveto{\pgfqpoint{2.209000in}{1.216628in}}%
\pgfpathlineto{\pgfqpoint{2.209000in}{5.143649in}}%
\pgfusepath{stroke}%
\end{pgfscope}%
\begin{pgfscope}%
\pgfpathrectangle{\pgfqpoint{1.000000in}{0.720000in}}{\pgfqpoint{6.200000in}{4.620000in}}%
\pgfusepath{clip}%
\pgfsetbuttcap%
\pgfsetroundjoin%
\pgfsetlinewidth{0.501875pt}%
\definecolor{currentstroke}{rgb}{1.000000,0.000000,0.000000}%
\pgfsetstrokecolor{currentstroke}%
\pgfsetdash{}{0pt}%
\pgfpathmoveto{\pgfqpoint{2.240000in}{1.216628in}}%
\pgfpathlineto{\pgfqpoint{2.240000in}{1.216705in}}%
\pgfusepath{stroke}%
\end{pgfscope}%
\begin{pgfscope}%
\pgfpathrectangle{\pgfqpoint{1.000000in}{0.720000in}}{\pgfqpoint{6.200000in}{4.620000in}}%
\pgfusepath{clip}%
\pgfsetbuttcap%
\pgfsetroundjoin%
\pgfsetlinewidth{0.501875pt}%
\definecolor{currentstroke}{rgb}{1.000000,0.000000,0.000000}%
\pgfsetstrokecolor{currentstroke}%
\pgfsetdash{}{0pt}%
\pgfpathmoveto{\pgfqpoint{2.565500in}{1.216628in}}%
\pgfpathlineto{\pgfqpoint{2.565500in}{1.216788in}}%
\pgfusepath{stroke}%
\end{pgfscope}%
\begin{pgfscope}%
\pgfpathrectangle{\pgfqpoint{1.000000in}{0.720000in}}{\pgfqpoint{6.200000in}{4.620000in}}%
\pgfusepath{clip}%
\pgfsetbuttcap%
\pgfsetroundjoin%
\pgfsetlinewidth{0.501875pt}%
\definecolor{currentstroke}{rgb}{1.000000,0.000000,0.000000}%
\pgfsetstrokecolor{currentstroke}%
\pgfsetdash{}{0pt}%
\pgfpathmoveto{\pgfqpoint{2.581000in}{1.216628in}}%
\pgfpathlineto{\pgfqpoint{2.581000in}{1.216654in}}%
\pgfusepath{stroke}%
\end{pgfscope}%
\begin{pgfscope}%
\pgfpathrectangle{\pgfqpoint{1.000000in}{0.720000in}}{\pgfqpoint{6.200000in}{4.620000in}}%
\pgfusepath{clip}%
\pgfsetbuttcap%
\pgfsetroundjoin%
\pgfsetlinewidth{0.501875pt}%
\definecolor{currentstroke}{rgb}{1.000000,0.000000,0.000000}%
\pgfsetstrokecolor{currentstroke}%
\pgfsetdash{}{0pt}%
\pgfpathmoveto{\pgfqpoint{2.596500in}{1.216628in}}%
\pgfpathlineto{\pgfqpoint{2.596500in}{1.216633in}}%
\pgfusepath{stroke}%
\end{pgfscope}%
\begin{pgfscope}%
\pgfpathrectangle{\pgfqpoint{1.000000in}{0.720000in}}{\pgfqpoint{6.200000in}{4.620000in}}%
\pgfusepath{clip}%
\pgfsetbuttcap%
\pgfsetroundjoin%
\pgfsetlinewidth{0.501875pt}%
\definecolor{currentstroke}{rgb}{1.000000,0.000000,0.000000}%
\pgfsetstrokecolor{currentstroke}%
\pgfsetdash{}{0pt}%
\pgfpathmoveto{\pgfqpoint{2.612000in}{1.216628in}}%
\pgfpathlineto{\pgfqpoint{2.612000in}{3.752804in}}%
\pgfusepath{stroke}%
\end{pgfscope}%
\begin{pgfscope}%
\pgfpathrectangle{\pgfqpoint{1.000000in}{0.720000in}}{\pgfqpoint{6.200000in}{4.620000in}}%
\pgfusepath{clip}%
\pgfsetbuttcap%
\pgfsetroundjoin%
\pgfsetlinewidth{0.501875pt}%
\definecolor{currentstroke}{rgb}{1.000000,0.000000,0.000000}%
\pgfsetstrokecolor{currentstroke}%
\pgfsetdash{}{0pt}%
\pgfpathmoveto{\pgfqpoint{2.643000in}{1.216628in}}%
\pgfpathlineto{\pgfqpoint{2.643000in}{4.258724in}}%
\pgfusepath{stroke}%
\end{pgfscope}%
\begin{pgfscope}%
\pgfpathrectangle{\pgfqpoint{1.000000in}{0.720000in}}{\pgfqpoint{6.200000in}{4.620000in}}%
\pgfusepath{clip}%
\pgfsetbuttcap%
\pgfsetroundjoin%
\pgfsetlinewidth{0.501875pt}%
\definecolor{currentstroke}{rgb}{1.000000,0.000000,0.000000}%
\pgfsetstrokecolor{currentstroke}%
\pgfsetdash{}{0pt}%
\pgfpathmoveto{\pgfqpoint{2.674000in}{1.216628in}}%
\pgfpathlineto{\pgfqpoint{2.674000in}{1.216711in}}%
\pgfusepath{stroke}%
\end{pgfscope}%
\begin{pgfscope}%
\pgfpathrectangle{\pgfqpoint{1.000000in}{0.720000in}}{\pgfqpoint{6.200000in}{4.620000in}}%
\pgfusepath{clip}%
\pgfsetbuttcap%
\pgfsetroundjoin%
\pgfsetlinewidth{0.501875pt}%
\definecolor{currentstroke}{rgb}{1.000000,0.000000,0.000000}%
\pgfsetstrokecolor{currentstroke}%
\pgfsetdash{}{0pt}%
\pgfpathmoveto{\pgfqpoint{2.705000in}{1.216628in}}%
\pgfpathlineto{\pgfqpoint{2.705000in}{1.216702in}}%
\pgfusepath{stroke}%
\end{pgfscope}%
\begin{pgfscope}%
\pgfpathrectangle{\pgfqpoint{1.000000in}{0.720000in}}{\pgfqpoint{6.200000in}{4.620000in}}%
\pgfusepath{clip}%
\pgfsetbuttcap%
\pgfsetroundjoin%
\pgfsetlinewidth{0.501875pt}%
\definecolor{currentstroke}{rgb}{1.000000,0.000000,0.000000}%
\pgfsetstrokecolor{currentstroke}%
\pgfsetdash{}{0pt}%
\pgfpathmoveto{\pgfqpoint{2.720500in}{1.216628in}}%
\pgfpathlineto{\pgfqpoint{2.720500in}{1.216668in}}%
\pgfusepath{stroke}%
\end{pgfscope}%
\begin{pgfscope}%
\pgfpathrectangle{\pgfqpoint{1.000000in}{0.720000in}}{\pgfqpoint{6.200000in}{4.620000in}}%
\pgfusepath{clip}%
\pgfsetbuttcap%
\pgfsetroundjoin%
\pgfsetlinewidth{0.501875pt}%
\definecolor{currentstroke}{rgb}{1.000000,0.000000,0.000000}%
\pgfsetstrokecolor{currentstroke}%
\pgfsetdash{}{0pt}%
\pgfpathmoveto{\pgfqpoint{2.767000in}{1.216628in}}%
\pgfpathlineto{\pgfqpoint{2.767000in}{1.216668in}}%
\pgfusepath{stroke}%
\end{pgfscope}%
\begin{pgfscope}%
\pgfpathrectangle{\pgfqpoint{1.000000in}{0.720000in}}{\pgfqpoint{6.200000in}{4.620000in}}%
\pgfusepath{clip}%
\pgfsetbuttcap%
\pgfsetroundjoin%
\pgfsetlinewidth{0.501875pt}%
\definecolor{currentstroke}{rgb}{1.000000,0.000000,0.000000}%
\pgfsetstrokecolor{currentstroke}%
\pgfsetdash{}{0pt}%
\pgfpathmoveto{\pgfqpoint{2.782500in}{1.216628in}}%
\pgfpathlineto{\pgfqpoint{2.782500in}{1.216667in}}%
\pgfusepath{stroke}%
\end{pgfscope}%
\begin{pgfscope}%
\pgfpathrectangle{\pgfqpoint{1.000000in}{0.720000in}}{\pgfqpoint{6.200000in}{4.620000in}}%
\pgfusepath{clip}%
\pgfsetbuttcap%
\pgfsetroundjoin%
\pgfsetlinewidth{0.501875pt}%
\definecolor{currentstroke}{rgb}{1.000000,0.000000,0.000000}%
\pgfsetstrokecolor{currentstroke}%
\pgfsetdash{}{0pt}%
\pgfpathmoveto{\pgfqpoint{2.798000in}{1.216628in}}%
\pgfpathlineto{\pgfqpoint{2.798000in}{1.216671in}}%
\pgfusepath{stroke}%
\end{pgfscope}%
\begin{pgfscope}%
\pgfpathrectangle{\pgfqpoint{1.000000in}{0.720000in}}{\pgfqpoint{6.200000in}{4.620000in}}%
\pgfusepath{clip}%
\pgfsetbuttcap%
\pgfsetroundjoin%
\pgfsetlinewidth{0.501875pt}%
\definecolor{currentstroke}{rgb}{1.000000,0.000000,0.000000}%
\pgfsetstrokecolor{currentstroke}%
\pgfsetdash{}{0pt}%
\pgfpathmoveto{\pgfqpoint{2.844500in}{1.216628in}}%
\pgfpathlineto{\pgfqpoint{2.844500in}{1.216832in}}%
\pgfusepath{stroke}%
\end{pgfscope}%
\begin{pgfscope}%
\pgfpathrectangle{\pgfqpoint{1.000000in}{0.720000in}}{\pgfqpoint{6.200000in}{4.620000in}}%
\pgfusepath{clip}%
\pgfsetbuttcap%
\pgfsetroundjoin%
\pgfsetlinewidth{0.501875pt}%
\definecolor{currentstroke}{rgb}{1.000000,0.000000,0.000000}%
\pgfsetstrokecolor{currentstroke}%
\pgfsetdash{}{0pt}%
\pgfpathmoveto{\pgfqpoint{2.860000in}{1.216628in}}%
\pgfpathlineto{\pgfqpoint{2.860000in}{1.216657in}}%
\pgfusepath{stroke}%
\end{pgfscope}%
\begin{pgfscope}%
\pgfpathrectangle{\pgfqpoint{1.000000in}{0.720000in}}{\pgfqpoint{6.200000in}{4.620000in}}%
\pgfusepath{clip}%
\pgfsetbuttcap%
\pgfsetroundjoin%
\pgfsetlinewidth{0.501875pt}%
\definecolor{currentstroke}{rgb}{1.000000,0.000000,0.000000}%
\pgfsetstrokecolor{currentstroke}%
\pgfsetdash{}{0pt}%
\pgfpathmoveto{\pgfqpoint{2.906500in}{1.216628in}}%
\pgfpathlineto{\pgfqpoint{2.906500in}{4.773703in}}%
\pgfusepath{stroke}%
\end{pgfscope}%
\begin{pgfscope}%
\pgfpathrectangle{\pgfqpoint{1.000000in}{0.720000in}}{\pgfqpoint{6.200000in}{4.620000in}}%
\pgfusepath{clip}%
\pgfsetbuttcap%
\pgfsetroundjoin%
\pgfsetlinewidth{0.501875pt}%
\definecolor{currentstroke}{rgb}{1.000000,0.000000,0.000000}%
\pgfsetstrokecolor{currentstroke}%
\pgfsetdash{}{0pt}%
\pgfpathmoveto{\pgfqpoint{2.937500in}{1.216628in}}%
\pgfpathlineto{\pgfqpoint{2.937500in}{4.209337in}}%
\pgfusepath{stroke}%
\end{pgfscope}%
\begin{pgfscope}%
\pgfpathrectangle{\pgfqpoint{1.000000in}{0.720000in}}{\pgfqpoint{6.200000in}{4.620000in}}%
\pgfusepath{clip}%
\pgfsetbuttcap%
\pgfsetroundjoin%
\pgfsetlinewidth{0.501875pt}%
\definecolor{currentstroke}{rgb}{1.000000,0.000000,0.000000}%
\pgfsetstrokecolor{currentstroke}%
\pgfsetdash{}{0pt}%
\pgfpathmoveto{\pgfqpoint{2.984000in}{1.216628in}}%
\pgfpathlineto{\pgfqpoint{2.984000in}{1.216690in}}%
\pgfusepath{stroke}%
\end{pgfscope}%
\begin{pgfscope}%
\pgfpathrectangle{\pgfqpoint{1.000000in}{0.720000in}}{\pgfqpoint{6.200000in}{4.620000in}}%
\pgfusepath{clip}%
\pgfsetbuttcap%
\pgfsetroundjoin%
\pgfsetlinewidth{0.501875pt}%
\definecolor{currentstroke}{rgb}{1.000000,0.000000,0.000000}%
\pgfsetstrokecolor{currentstroke}%
\pgfsetdash{}{0pt}%
\pgfpathmoveto{\pgfqpoint{3.061500in}{1.216628in}}%
\pgfpathlineto{\pgfqpoint{3.061500in}{1.216689in}}%
\pgfusepath{stroke}%
\end{pgfscope}%
\begin{pgfscope}%
\pgfpathrectangle{\pgfqpoint{1.000000in}{0.720000in}}{\pgfqpoint{6.200000in}{4.620000in}}%
\pgfusepath{clip}%
\pgfsetbuttcap%
\pgfsetroundjoin%
\pgfsetlinewidth{0.501875pt}%
\definecolor{currentstroke}{rgb}{1.000000,0.000000,0.000000}%
\pgfsetstrokecolor{currentstroke}%
\pgfsetdash{}{0pt}%
\pgfpathmoveto{\pgfqpoint{3.077000in}{1.216628in}}%
\pgfpathlineto{\pgfqpoint{3.077000in}{1.216702in}}%
\pgfusepath{stroke}%
\end{pgfscope}%
\begin{pgfscope}%
\pgfpathrectangle{\pgfqpoint{1.000000in}{0.720000in}}{\pgfqpoint{6.200000in}{4.620000in}}%
\pgfusepath{clip}%
\pgfsetbuttcap%
\pgfsetroundjoin%
\pgfsetlinewidth{0.501875pt}%
\definecolor{currentstroke}{rgb}{1.000000,0.000000,0.000000}%
\pgfsetstrokecolor{currentstroke}%
\pgfsetdash{}{0pt}%
\pgfpathmoveto{\pgfqpoint{3.108000in}{1.216628in}}%
\pgfpathlineto{\pgfqpoint{3.108000in}{4.570799in}}%
\pgfusepath{stroke}%
\end{pgfscope}%
\begin{pgfscope}%
\pgfpathrectangle{\pgfqpoint{1.000000in}{0.720000in}}{\pgfqpoint{6.200000in}{4.620000in}}%
\pgfusepath{clip}%
\pgfsetbuttcap%
\pgfsetroundjoin%
\pgfsetlinewidth{0.501875pt}%
\definecolor{currentstroke}{rgb}{1.000000,0.000000,0.000000}%
\pgfsetstrokecolor{currentstroke}%
\pgfsetdash{}{0pt}%
\pgfpathmoveto{\pgfqpoint{3.201000in}{1.216628in}}%
\pgfpathlineto{\pgfqpoint{3.201000in}{1.216654in}}%
\pgfusepath{stroke}%
\end{pgfscope}%
\begin{pgfscope}%
\pgfpathrectangle{\pgfqpoint{1.000000in}{0.720000in}}{\pgfqpoint{6.200000in}{4.620000in}}%
\pgfusepath{clip}%
\pgfsetbuttcap%
\pgfsetroundjoin%
\pgfsetlinewidth{0.501875pt}%
\definecolor{currentstroke}{rgb}{1.000000,0.000000,0.000000}%
\pgfsetstrokecolor{currentstroke}%
\pgfsetdash{}{0pt}%
\pgfpathmoveto{\pgfqpoint{3.247500in}{1.216628in}}%
\pgfpathlineto{\pgfqpoint{3.247500in}{1.216665in}}%
\pgfusepath{stroke}%
\end{pgfscope}%
\begin{pgfscope}%
\pgfpathrectangle{\pgfqpoint{1.000000in}{0.720000in}}{\pgfqpoint{6.200000in}{4.620000in}}%
\pgfusepath{clip}%
\pgfsetbuttcap%
\pgfsetroundjoin%
\pgfsetlinewidth{0.501875pt}%
\definecolor{currentstroke}{rgb}{1.000000,0.000000,0.000000}%
\pgfsetstrokecolor{currentstroke}%
\pgfsetdash{}{0pt}%
\pgfpathmoveto{\pgfqpoint{3.666000in}{1.216628in}}%
\pgfpathlineto{\pgfqpoint{3.666000in}{1.216676in}}%
\pgfusepath{stroke}%
\end{pgfscope}%
\begin{pgfscope}%
\pgfpathrectangle{\pgfqpoint{1.000000in}{0.720000in}}{\pgfqpoint{6.200000in}{4.620000in}}%
\pgfusepath{clip}%
\pgfsetbuttcap%
\pgfsetroundjoin%
\pgfsetlinewidth{0.501875pt}%
\definecolor{currentstroke}{rgb}{1.000000,0.000000,0.000000}%
\pgfsetstrokecolor{currentstroke}%
\pgfsetdash{}{0pt}%
\pgfpathmoveto{\pgfqpoint{3.681500in}{1.216628in}}%
\pgfpathlineto{\pgfqpoint{3.681500in}{4.503043in}}%
\pgfusepath{stroke}%
\end{pgfscope}%
\begin{pgfscope}%
\pgfpathrectangle{\pgfqpoint{1.000000in}{0.720000in}}{\pgfqpoint{6.200000in}{4.620000in}}%
\pgfusepath{clip}%
\pgfsetbuttcap%
\pgfsetroundjoin%
\pgfsetlinewidth{0.501875pt}%
\definecolor{currentstroke}{rgb}{1.000000,0.000000,0.000000}%
\pgfsetstrokecolor{currentstroke}%
\pgfsetdash{}{0pt}%
\pgfpathmoveto{\pgfqpoint{3.712500in}{1.216628in}}%
\pgfpathlineto{\pgfqpoint{3.712500in}{1.216687in}}%
\pgfusepath{stroke}%
\end{pgfscope}%
\begin{pgfscope}%
\pgfpathrectangle{\pgfqpoint{1.000000in}{0.720000in}}{\pgfqpoint{6.200000in}{4.620000in}}%
\pgfusepath{clip}%
\pgfsetbuttcap%
\pgfsetroundjoin%
\pgfsetlinewidth{0.501875pt}%
\definecolor{currentstroke}{rgb}{1.000000,0.000000,0.000000}%
\pgfsetstrokecolor{currentstroke}%
\pgfsetdash{}{0pt}%
\pgfpathmoveto{\pgfqpoint{3.728000in}{1.216628in}}%
\pgfpathlineto{\pgfqpoint{3.728000in}{1.216636in}}%
\pgfusepath{stroke}%
\end{pgfscope}%
\begin{pgfscope}%
\pgfpathrectangle{\pgfqpoint{1.000000in}{0.720000in}}{\pgfqpoint{6.200000in}{4.620000in}}%
\pgfusepath{clip}%
\pgfsetbuttcap%
\pgfsetroundjoin%
\pgfsetlinewidth{0.501875pt}%
\definecolor{currentstroke}{rgb}{1.000000,0.000000,0.000000}%
\pgfsetstrokecolor{currentstroke}%
\pgfsetdash{}{0pt}%
\pgfpathmoveto{\pgfqpoint{3.743500in}{1.216628in}}%
\pgfpathlineto{\pgfqpoint{3.743500in}{1.216717in}}%
\pgfusepath{stroke}%
\end{pgfscope}%
\begin{pgfscope}%
\pgfpathrectangle{\pgfqpoint{1.000000in}{0.720000in}}{\pgfqpoint{6.200000in}{4.620000in}}%
\pgfusepath{clip}%
\pgfsetbuttcap%
\pgfsetroundjoin%
\pgfsetlinewidth{0.501875pt}%
\definecolor{currentstroke}{rgb}{1.000000,0.000000,0.000000}%
\pgfsetstrokecolor{currentstroke}%
\pgfsetdash{}{0pt}%
\pgfpathmoveto{\pgfqpoint{3.759000in}{1.216628in}}%
\pgfpathlineto{\pgfqpoint{3.759000in}{1.216687in}}%
\pgfusepath{stroke}%
\end{pgfscope}%
\begin{pgfscope}%
\pgfpathrectangle{\pgfqpoint{1.000000in}{0.720000in}}{\pgfqpoint{6.200000in}{4.620000in}}%
\pgfusepath{clip}%
\pgfsetbuttcap%
\pgfsetroundjoin%
\pgfsetlinewidth{0.501875pt}%
\definecolor{currentstroke}{rgb}{1.000000,0.000000,0.000000}%
\pgfsetstrokecolor{currentstroke}%
\pgfsetdash{}{0pt}%
\pgfpathmoveto{\pgfqpoint{3.790000in}{1.216628in}}%
\pgfpathlineto{\pgfqpoint{3.790000in}{1.216639in}}%
\pgfusepath{stroke}%
\end{pgfscope}%
\begin{pgfscope}%
\pgfsetrectcap%
\pgfsetmiterjoin%
\pgfsetlinewidth{0.803000pt}%
\definecolor{currentstroke}{rgb}{0.000000,0.000000,0.000000}%
\pgfsetstrokecolor{currentstroke}%
\pgfsetdash{}{0pt}%
\pgfpathmoveto{\pgfqpoint{1.000000in}{0.720000in}}%
\pgfpathlineto{\pgfqpoint{1.000000in}{5.340000in}}%
\pgfusepath{stroke}%
\end{pgfscope}%
\begin{pgfscope}%
\pgfsetrectcap%
\pgfsetmiterjoin%
\pgfsetlinewidth{0.803000pt}%
\definecolor{currentstroke}{rgb}{0.000000,0.000000,0.000000}%
\pgfsetstrokecolor{currentstroke}%
\pgfsetdash{}{0pt}%
\pgfpathmoveto{\pgfqpoint{7.200000in}{0.720000in}}%
\pgfpathlineto{\pgfqpoint{7.200000in}{5.340000in}}%
\pgfusepath{stroke}%
\end{pgfscope}%
\begin{pgfscope}%
\pgfsetrectcap%
\pgfsetmiterjoin%
\pgfsetlinewidth{0.803000pt}%
\definecolor{currentstroke}{rgb}{0.000000,0.000000,0.000000}%
\pgfsetstrokecolor{currentstroke}%
\pgfsetdash{}{0pt}%
\pgfpathmoveto{\pgfqpoint{1.000000in}{0.720000in}}%
\pgfpathlineto{\pgfqpoint{7.200000in}{0.720000in}}%
\pgfusepath{stroke}%
\end{pgfscope}%
\begin{pgfscope}%
\pgfsetrectcap%
\pgfsetmiterjoin%
\pgfsetlinewidth{0.803000pt}%
\definecolor{currentstroke}{rgb}{0.000000,0.000000,0.000000}%
\pgfsetstrokecolor{currentstroke}%
\pgfsetdash{}{0pt}%
\pgfpathmoveto{\pgfqpoint{1.000000in}{5.340000in}}%
\pgfpathlineto{\pgfqpoint{7.200000in}{5.340000in}}%
\pgfusepath{stroke}%
\end{pgfscope}%
\begin{pgfscope}%
\pgfsetbuttcap%
\pgfsetmiterjoin%
\definecolor{currentfill}{rgb}{1.000000,1.000000,1.000000}%
\pgfsetfillcolor{currentfill}%
\pgfsetfillopacity{0.800000}%
\pgfsetlinewidth{1.003750pt}%
\definecolor{currentstroke}{rgb}{0.800000,0.800000,0.800000}%
\pgfsetstrokecolor{currentstroke}%
\pgfsetstrokeopacity{0.800000}%
\pgfsetdash{}{0pt}%
\pgfpathmoveto{\pgfqpoint{4.976872in}{3.932908in}}%
\pgfpathlineto{\pgfqpoint{7.005556in}{3.932908in}}%
\pgfpathquadraticcurveto{\pgfqpoint{7.061111in}{3.932908in}}{\pgfqpoint{7.061111in}{3.988464in}}%
\pgfpathlineto{\pgfqpoint{7.061111in}{5.145556in}}%
\pgfpathquadraticcurveto{\pgfqpoint{7.061111in}{5.201111in}}{\pgfqpoint{7.005556in}{5.201111in}}%
\pgfpathlineto{\pgfqpoint{4.976872in}{5.201111in}}%
\pgfpathquadraticcurveto{\pgfqpoint{4.921317in}{5.201111in}}{\pgfqpoint{4.921317in}{5.145556in}}%
\pgfpathlineto{\pgfqpoint{4.921317in}{3.988464in}}%
\pgfpathquadraticcurveto{\pgfqpoint{4.921317in}{3.932908in}}{\pgfqpoint{4.976872in}{3.932908in}}%
\pgfpathclose%
\pgfusepath{stroke,fill}%
\end{pgfscope}%
\begin{pgfscope}%
\pgfsetrectcap%
\pgfsetroundjoin%
\pgfsetlinewidth{2.007500pt}%
\definecolor{currentstroke}{rgb}{0.121569,0.466667,0.705882}%
\pgfsetstrokecolor{currentstroke}%
\pgfsetdash{}{0pt}%
\pgfpathmoveto{\pgfqpoint{5.032428in}{4.987184in}}%
\pgfpathlineto{\pgfqpoint{5.587983in}{4.987184in}}%
\pgfusepath{stroke}%
\end{pgfscope}%
\begin{pgfscope}%
\definecolor{textcolor}{rgb}{0.000000,0.000000,0.000000}%
\pgfsetstrokecolor{textcolor}%
\pgfsetfillcolor{textcolor}%
\pgftext[x=5.810206in,y=4.889962in,left,base]{\color{textcolor}\sffamily\fontsize{20.000000}{24.000000}\selectfont Waveform}%
\end{pgfscope}%
\begin{pgfscope}%
\pgfsetbuttcap%
\pgfsetroundjoin%
\pgfsetlinewidth{2.007500pt}%
\definecolor{currentstroke}{rgb}{0.000000,0.500000,0.000000}%
\pgfsetstrokecolor{currentstroke}%
\pgfsetdash{}{0pt}%
\pgfpathmoveto{\pgfqpoint{5.032428in}{4.592227in}}%
\pgfpathlineto{\pgfqpoint{5.587983in}{4.592227in}}%
\pgfusepath{stroke}%
\end{pgfscope}%
\begin{pgfscope}%
\definecolor{textcolor}{rgb}{0.000000,0.000000,0.000000}%
\pgfsetstrokecolor{textcolor}%
\pgfsetfillcolor{textcolor}%
\pgftext[x=5.810206in,y=4.495005in,left,base]{\color{textcolor}\sffamily\fontsize{20.000000}{24.000000}\selectfont Threshold}%
\end{pgfscope}%
\begin{pgfscope}%
\pgfsetbuttcap%
\pgfsetroundjoin%
\pgfsetlinewidth{0.501875pt}%
\definecolor{currentstroke}{rgb}{1.000000,0.000000,0.000000}%
\pgfsetstrokecolor{currentstroke}%
\pgfsetdash{}{0pt}%
\pgfpathmoveto{\pgfqpoint{5.032428in}{4.197271in}}%
\pgfpathlineto{\pgfqpoint{5.587983in}{4.197271in}}%
\pgfusepath{stroke}%
\end{pgfscope}%
\begin{pgfscope}%
\definecolor{textcolor}{rgb}{0.000000,0.000000,0.000000}%
\pgfsetstrokecolor{textcolor}%
\pgfsetfillcolor{textcolor}%
\pgftext[x=5.810206in,y=4.100048in,left,base]{\color{textcolor}\sffamily\fontsize{20.000000}{24.000000}\selectfont Charge}%
\end{pgfscope}%
\end{pgfpicture}%
\makeatother%
\endgroup%
}
    \caption{\label{fig:mcmc}An example with \\ $\hat{t}_0 - t_0=\SI{2.67}{ns}$, $\mathrm{RSS}=\SI{8.83}{mV^2}$, $D_\mathrm{w}=\SI{0.62}{ns}$.}
  \end{subfigure}
  \caption{\label{fig:mcmc-performance}Demonstration of HMC with $\num[retain-unity-mantissa=false]{1e4}$ waveforms in~\subref{fig:mcmc-npe} and one waveform in~\subref{fig:mcmc} sampled from the same setup as figure~\ref{fig:method}.  Although using a more dedicated model, HMC performs worse than the direct charge fitting in figure~\ref{fig:dcf}. We suspect the Markov chain is not long enough.}
\end{figure}
Although we imposed a prior distribution in eq.~\eqref{eq:mixnormal} with $\E[q_j]=1$, The charges $\hat{q}_j$ in figure~\ref{fig:mcmc} are still less than 1.  The $D_\mathrm{w}$ marginal distribution in figure~\ref{fig:mcmc-npe} is less smooth than that of the direct charge fitting in figure~\ref{fig:fitting-npe}.  Similarly, RSS in figure~\ref{fig:mcmc} is slightly worse than that in figure~\ref{fig:fitting}.  We suspect the Markov chain has not finally converged due to the trans-dimensional property of eq.~\eqref{eq:mixnormal}.  Extending the chain is not a solution because MCMC is already much slower than direct fitting in section~\ref{sec:dcf}.  We need an algorithm that pertains to the model of eq.~\eqref{eq:mixnormal} but much faster than HMC.

\subsubsection{Fast stochastic matching pursuit}
\label{subsec:fsmp}
In reality, $w(t)$ is discretized as $\bm{w}$. If we rewrite the hierarchical model (eq.~\eqref{eq:mixnormal}) into a joint distribution, marginalizing out $\bm{q}'$ and $\bm{z}$ gives a flattened model,
\begin{equation}
  \label{eq:universe}
  \begin{aligned}
    p(\bm{w}, t_0, \mu) &= \sum_{\bm{z}} \int \mathrm{d}\bm{q}' p(\bm{w}, \bm{q}', \bm{z}, t_0, \mu) \\
    &= p(t_0, \mu) \sum_{\bm{z}} \left[\int \mathrm{d}\bm{q}' p(\bm{w}|\bm{q}') p(\bm{q}'|\bm{z}) \right] p(\bm{z}|t_0, \mu) \\
    &= p(t_0, \mu) \sum_{\bm{z}} p(\bm{w}|\bm{z}) p(\bm{z}|t_0, \mu) \\
    &= p(t_0, \mu) p(\bm{w}|t_0, \mu) \\
  \end{aligned}
\end{equation}
The integration over $\bm{q}'$ is the probability density of a multi-normal distribution $p(\bm{w}|\bm{z})$, with a fast algorithm to iteratively compute by Schniter~et al.~\cite{schniter_fast_2008}. The summation over $\bm{z}$, however, takes an exploding number of combinations.

Let's approximate the summation with a sample from $S = (\bm{s}_1, \bm{s}_2, \cdots, \bm{s}_M)$ by Metropolis-Hasings~\cite{metropolis_equation_1953, hastings_monte_1970, mackay_information_2003} from $p(\bm{z}) = C p(\bm{w} | \bm{z}) h( \bm{z})$. $C$ is independent of $\bm{z}$, and $h(\bm{z})$ is an educated guess for $p(\bm{z}|t_0, \mu)$ from a previous method like LucyDDM~(section~\ref{sec:lucyddm}). Then,
\begin{equation}
  \label{eq:mh}
  \begin{aligned}
    p(\bm{w}|\mu, t_0) &= \sum_{\bm{z}} p(\bm{w}|\bm{z}) p(\bm{z}|t_0, \mu) = \frac{1}{C}\sum_{\bm{z}} p(\bm{z}) \frac{p(\bm{z} | \mu, t_0)}{h(\bm{z})} \\
    &= \frac{1}{C} \E_{\bm{z}}\left[ \frac{p(\bm{z} | \mu, t_0)}{h(\bm{z})} \right] \approx \frac{1}{CM} \sum_{i=1}^M \frac{p(\bm{s}_i | \mu, t_0)}{h(\bm{s}_i)}. \\
  \end{aligned}
\end{equation}
Construct approximate MLEs for $t_0$, $\mu$ and $\bm{z}$, and the expectation estimation of $\hat{\bm{q}}$,
\begin{equation}
  \label{eq:fsmpcharge}
  \begin{aligned}
    \left(\hat{t}_0, \hat{\mu}\right) &= \arg\underset{t_0,\mu}{\max}~p(\bm{w}|\mu, t_0) = \arg\underset{t_0,\mu}{\max} \sum_{i=1}^M \frac{p(\bm{s}_i | \mu, t_0)}{h(\bm{s}_i)}\\
    \hat{\bm{z}} &= \arg \underset{\bm{s}_i \in S}{\max}~p(\bm{w}|\bm{s}_i) h(\bm{s}_i) \\
    \hat{\bm{q}}|{\hat{\bm{z}}} &= \E(\bm{q}'|\bm{w},\hat{\bm{z}})
  \end{aligned}
\end{equation}
RSS and $D_\mathrm{w}$ are calculated by eqs.~\eqref{eq:rss}, \eqref{eq:numerical}, \eqref{eq:gd-phi}.

We name the method \emph{fast stochastic matching pursuit}~(FSMP) after \emph{fast Bayesian matching pursuit}~(FBMP) by Schniter~et al.~\cite{schniter_fast_2008} and \emph{Bayesian stochastic matching pursuit} by Chen~et al.~\cite{chen_stochastic_2011}.  Here FSMP replaces the greedy search routine in FBMP with stochastic sampling.  With the help of Ekanadham~et al.'s function interpolation~\cite{ekanadham_recovery_2011}, FSMP straightforwardly extends $\bm{z}$ into an unbinned vector of PE locations $t_i$.  Geyer and Møller~\cite{geyer_simulation_1994} developed a similar sampler to handle trans-dimensionality in a Poisson point process.  $h(\bm{z})$ and the proposal distribution in Metropolis-Hastings steps could be tuned to improve sampling efficiency.  We shall leave the detailed study of the Markov chain convergence to our future publications.

\begin{figure}[h]
  \begin{subfigure}[b]{.45\textwidth}
    \centering
    \resizebox{1.05\textwidth}{!}{%% Creator: Matplotlib, PGF backend
%%
%% To include the figure in your LaTeX document, write
%%   \input{<filename>.pgf}
%%
%% Make sure the required packages are loaded in your preamble
%%   \usepackage{pgf}
%%
%% and, on pdftex
%%   \usepackage[utf8]{inputenc}\DeclareUnicodeCharacter{2212}{-}
%%
%% or, on luatex and xetex
%%   \usepackage{unicode-math}
%%
%% Figures using additional raster images can only be included by \input if
%% they are in the same directory as the main LaTeX file. For loading figures
%% from other directories you can use the `import` package
%%   \usepackage{import}
%%
%% and then include the figures with
%%   \import{<path to file>}{<filename>.pgf}
%%
%% Matplotlib used the following preamble
%%   \usepackage[detect-all,locale=DE]{siunitx}
%%
\begingroup%
\makeatletter%
\begin{pgfpicture}%
\pgfpathrectangle{\pgfpointorigin}{\pgfqpoint{8.000000in}{6.000000in}}%
\pgfusepath{use as bounding box, clip}%
\begin{pgfscope}%
\pgfsetbuttcap%
\pgfsetmiterjoin%
\definecolor{currentfill}{rgb}{1.000000,1.000000,1.000000}%
\pgfsetfillcolor{currentfill}%
\pgfsetlinewidth{0.000000pt}%
\definecolor{currentstroke}{rgb}{1.000000,1.000000,1.000000}%
\pgfsetstrokecolor{currentstroke}%
\pgfsetdash{}{0pt}%
\pgfpathmoveto{\pgfqpoint{0.000000in}{0.000000in}}%
\pgfpathlineto{\pgfqpoint{8.000000in}{0.000000in}}%
\pgfpathlineto{\pgfqpoint{8.000000in}{6.000000in}}%
\pgfpathlineto{\pgfqpoint{0.000000in}{6.000000in}}%
\pgfpathclose%
\pgfusepath{fill}%
\end{pgfscope}%
\begin{pgfscope}%
\pgfsetbuttcap%
\pgfsetmiterjoin%
\definecolor{currentfill}{rgb}{1.000000,1.000000,1.000000}%
\pgfsetfillcolor{currentfill}%
\pgfsetlinewidth{0.000000pt}%
\definecolor{currentstroke}{rgb}{0.000000,0.000000,0.000000}%
\pgfsetstrokecolor{currentstroke}%
\pgfsetstrokeopacity{0.000000}%
\pgfsetdash{}{0pt}%
\pgfpathmoveto{\pgfqpoint{1.000000in}{0.720000in}}%
\pgfpathlineto{\pgfqpoint{5.800000in}{0.720000in}}%
\pgfpathlineto{\pgfqpoint{5.800000in}{5.340000in}}%
\pgfpathlineto{\pgfqpoint{1.000000in}{5.340000in}}%
\pgfpathclose%
\pgfusepath{fill}%
\end{pgfscope}%
\begin{pgfscope}%
\pgfsetbuttcap%
\pgfsetroundjoin%
\definecolor{currentfill}{rgb}{0.000000,0.000000,0.000000}%
\pgfsetfillcolor{currentfill}%
\pgfsetlinewidth{0.803000pt}%
\definecolor{currentstroke}{rgb}{0.000000,0.000000,0.000000}%
\pgfsetstrokecolor{currentstroke}%
\pgfsetdash{}{0pt}%
\pgfsys@defobject{currentmarker}{\pgfqpoint{0.000000in}{-0.048611in}}{\pgfqpoint{0.000000in}{0.000000in}}{%
\pgfpathmoveto{\pgfqpoint{0.000000in}{0.000000in}}%
\pgfpathlineto{\pgfqpoint{0.000000in}{-0.048611in}}%
\pgfusepath{stroke,fill}%
}%
\begin{pgfscope}%
\pgfsys@transformshift{1.300000in}{0.720000in}%
\pgfsys@useobject{currentmarker}{}%
\end{pgfscope}%
\end{pgfscope}%
\begin{pgfscope}%
\definecolor{textcolor}{rgb}{0.000000,0.000000,0.000000}%
\pgfsetstrokecolor{textcolor}%
\pgfsetfillcolor{textcolor}%
\pgftext[x=1.300000in,y=0.622778in,,top]{\color{textcolor}\sffamily\fontsize{20.000000}{24.000000}\selectfont 1}%
\end{pgfscope}%
\begin{pgfscope}%
\pgfsetbuttcap%
\pgfsetroundjoin%
\definecolor{currentfill}{rgb}{0.000000,0.000000,0.000000}%
\pgfsetfillcolor{currentfill}%
\pgfsetlinewidth{0.803000pt}%
\definecolor{currentstroke}{rgb}{0.000000,0.000000,0.000000}%
\pgfsetstrokecolor{currentstroke}%
\pgfsetdash{}{0pt}%
\pgfsys@defobject{currentmarker}{\pgfqpoint{0.000000in}{-0.048611in}}{\pgfqpoint{0.000000in}{0.000000in}}{%
\pgfpathmoveto{\pgfqpoint{0.000000in}{0.000000in}}%
\pgfpathlineto{\pgfqpoint{0.000000in}{-0.048611in}}%
\pgfusepath{stroke,fill}%
}%
\begin{pgfscope}%
\pgfsys@transformshift{1.900000in}{0.720000in}%
\pgfsys@useobject{currentmarker}{}%
\end{pgfscope}%
\end{pgfscope}%
\begin{pgfscope}%
\definecolor{textcolor}{rgb}{0.000000,0.000000,0.000000}%
\pgfsetstrokecolor{textcolor}%
\pgfsetfillcolor{textcolor}%
\pgftext[x=1.900000in,y=0.622778in,,top]{\color{textcolor}\sffamily\fontsize{20.000000}{24.000000}\selectfont 3}%
\end{pgfscope}%
\begin{pgfscope}%
\pgfsetbuttcap%
\pgfsetroundjoin%
\definecolor{currentfill}{rgb}{0.000000,0.000000,0.000000}%
\pgfsetfillcolor{currentfill}%
\pgfsetlinewidth{0.803000pt}%
\definecolor{currentstroke}{rgb}{0.000000,0.000000,0.000000}%
\pgfsetstrokecolor{currentstroke}%
\pgfsetdash{}{0pt}%
\pgfsys@defobject{currentmarker}{\pgfqpoint{0.000000in}{-0.048611in}}{\pgfqpoint{0.000000in}{0.000000in}}{%
\pgfpathmoveto{\pgfqpoint{0.000000in}{0.000000in}}%
\pgfpathlineto{\pgfqpoint{0.000000in}{-0.048611in}}%
\pgfusepath{stroke,fill}%
}%
\begin{pgfscope}%
\pgfsys@transformshift{2.500000in}{0.720000in}%
\pgfsys@useobject{currentmarker}{}%
\end{pgfscope}%
\end{pgfscope}%
\begin{pgfscope}%
\definecolor{textcolor}{rgb}{0.000000,0.000000,0.000000}%
\pgfsetstrokecolor{textcolor}%
\pgfsetfillcolor{textcolor}%
\pgftext[x=2.500000in,y=0.622778in,,top]{\color{textcolor}\sffamily\fontsize{20.000000}{24.000000}\selectfont 5}%
\end{pgfscope}%
\begin{pgfscope}%
\pgfsetbuttcap%
\pgfsetroundjoin%
\definecolor{currentfill}{rgb}{0.000000,0.000000,0.000000}%
\pgfsetfillcolor{currentfill}%
\pgfsetlinewidth{0.803000pt}%
\definecolor{currentstroke}{rgb}{0.000000,0.000000,0.000000}%
\pgfsetstrokecolor{currentstroke}%
\pgfsetdash{}{0pt}%
\pgfsys@defobject{currentmarker}{\pgfqpoint{0.000000in}{-0.048611in}}{\pgfqpoint{0.000000in}{0.000000in}}{%
\pgfpathmoveto{\pgfqpoint{0.000000in}{0.000000in}}%
\pgfpathlineto{\pgfqpoint{0.000000in}{-0.048611in}}%
\pgfusepath{stroke,fill}%
}%
\begin{pgfscope}%
\pgfsys@transformshift{3.100000in}{0.720000in}%
\pgfsys@useobject{currentmarker}{}%
\end{pgfscope}%
\end{pgfscope}%
\begin{pgfscope}%
\definecolor{textcolor}{rgb}{0.000000,0.000000,0.000000}%
\pgfsetstrokecolor{textcolor}%
\pgfsetfillcolor{textcolor}%
\pgftext[x=3.100000in,y=0.622778in,,top]{\color{textcolor}\sffamily\fontsize{20.000000}{24.000000}\selectfont 7}%
\end{pgfscope}%
\begin{pgfscope}%
\pgfsetbuttcap%
\pgfsetroundjoin%
\definecolor{currentfill}{rgb}{0.000000,0.000000,0.000000}%
\pgfsetfillcolor{currentfill}%
\pgfsetlinewidth{0.803000pt}%
\definecolor{currentstroke}{rgb}{0.000000,0.000000,0.000000}%
\pgfsetstrokecolor{currentstroke}%
\pgfsetdash{}{0pt}%
\pgfsys@defobject{currentmarker}{\pgfqpoint{0.000000in}{-0.048611in}}{\pgfqpoint{0.000000in}{0.000000in}}{%
\pgfpathmoveto{\pgfqpoint{0.000000in}{0.000000in}}%
\pgfpathlineto{\pgfqpoint{0.000000in}{-0.048611in}}%
\pgfusepath{stroke,fill}%
}%
\begin{pgfscope}%
\pgfsys@transformshift{3.700000in}{0.720000in}%
\pgfsys@useobject{currentmarker}{}%
\end{pgfscope}%
\end{pgfscope}%
\begin{pgfscope}%
\definecolor{textcolor}{rgb}{0.000000,0.000000,0.000000}%
\pgfsetstrokecolor{textcolor}%
\pgfsetfillcolor{textcolor}%
\pgftext[x=3.700000in,y=0.622778in,,top]{\color{textcolor}\sffamily\fontsize{20.000000}{24.000000}\selectfont 9}%
\end{pgfscope}%
\begin{pgfscope}%
\pgfsetbuttcap%
\pgfsetroundjoin%
\definecolor{currentfill}{rgb}{0.000000,0.000000,0.000000}%
\pgfsetfillcolor{currentfill}%
\pgfsetlinewidth{0.803000pt}%
\definecolor{currentstroke}{rgb}{0.000000,0.000000,0.000000}%
\pgfsetstrokecolor{currentstroke}%
\pgfsetdash{}{0pt}%
\pgfsys@defobject{currentmarker}{\pgfqpoint{0.000000in}{-0.048611in}}{\pgfqpoint{0.000000in}{0.000000in}}{%
\pgfpathmoveto{\pgfqpoint{0.000000in}{0.000000in}}%
\pgfpathlineto{\pgfqpoint{0.000000in}{-0.048611in}}%
\pgfusepath{stroke,fill}%
}%
\begin{pgfscope}%
\pgfsys@transformshift{4.300000in}{0.720000in}%
\pgfsys@useobject{currentmarker}{}%
\end{pgfscope}%
\end{pgfscope}%
\begin{pgfscope}%
\definecolor{textcolor}{rgb}{0.000000,0.000000,0.000000}%
\pgfsetstrokecolor{textcolor}%
\pgfsetfillcolor{textcolor}%
\pgftext[x=4.300000in,y=0.622778in,,top]{\color{textcolor}\sffamily\fontsize{20.000000}{24.000000}\selectfont 11}%
\end{pgfscope}%
\begin{pgfscope}%
\pgfsetbuttcap%
\pgfsetroundjoin%
\definecolor{currentfill}{rgb}{0.000000,0.000000,0.000000}%
\pgfsetfillcolor{currentfill}%
\pgfsetlinewidth{0.803000pt}%
\definecolor{currentstroke}{rgb}{0.000000,0.000000,0.000000}%
\pgfsetstrokecolor{currentstroke}%
\pgfsetdash{}{0pt}%
\pgfsys@defobject{currentmarker}{\pgfqpoint{0.000000in}{-0.048611in}}{\pgfqpoint{0.000000in}{0.000000in}}{%
\pgfpathmoveto{\pgfqpoint{0.000000in}{0.000000in}}%
\pgfpathlineto{\pgfqpoint{0.000000in}{-0.048611in}}%
\pgfusepath{stroke,fill}%
}%
\begin{pgfscope}%
\pgfsys@transformshift{4.900000in}{0.720000in}%
\pgfsys@useobject{currentmarker}{}%
\end{pgfscope}%
\end{pgfscope}%
\begin{pgfscope}%
\definecolor{textcolor}{rgb}{0.000000,0.000000,0.000000}%
\pgfsetstrokecolor{textcolor}%
\pgfsetfillcolor{textcolor}%
\pgftext[x=4.900000in,y=0.622778in,,top]{\color{textcolor}\sffamily\fontsize{20.000000}{24.000000}\selectfont 13}%
\end{pgfscope}%
\begin{pgfscope}%
\pgfsetbuttcap%
\pgfsetroundjoin%
\definecolor{currentfill}{rgb}{0.000000,0.000000,0.000000}%
\pgfsetfillcolor{currentfill}%
\pgfsetlinewidth{0.803000pt}%
\definecolor{currentstroke}{rgb}{0.000000,0.000000,0.000000}%
\pgfsetstrokecolor{currentstroke}%
\pgfsetdash{}{0pt}%
\pgfsys@defobject{currentmarker}{\pgfqpoint{0.000000in}{-0.048611in}}{\pgfqpoint{0.000000in}{0.000000in}}{%
\pgfpathmoveto{\pgfqpoint{0.000000in}{0.000000in}}%
\pgfpathlineto{\pgfqpoint{0.000000in}{-0.048611in}}%
\pgfusepath{stroke,fill}%
}%
\begin{pgfscope}%
\pgfsys@transformshift{5.500000in}{0.720000in}%
\pgfsys@useobject{currentmarker}{}%
\end{pgfscope}%
\end{pgfscope}%
\begin{pgfscope}%
\definecolor{textcolor}{rgb}{0.000000,0.000000,0.000000}%
\pgfsetstrokecolor{textcolor}%
\pgfsetfillcolor{textcolor}%
\pgftext[x=5.500000in,y=0.622778in,,top]{\color{textcolor}\sffamily\fontsize{20.000000}{24.000000}\selectfont 15}%
\end{pgfscope}%
\begin{pgfscope}%
\definecolor{textcolor}{rgb}{0.000000,0.000000,0.000000}%
\pgfsetstrokecolor{textcolor}%
\pgfsetfillcolor{textcolor}%
\pgftext[x=3.400000in,y=0.311155in,,top]{\color{textcolor}\sffamily\fontsize{20.000000}{24.000000}\selectfont \(\displaystyle N_{\mathrm{PE}}\)}%
\end{pgfscope}%
\begin{pgfscope}%
\pgfsetbuttcap%
\pgfsetroundjoin%
\definecolor{currentfill}{rgb}{0.000000,0.000000,0.000000}%
\pgfsetfillcolor{currentfill}%
\pgfsetlinewidth{0.803000pt}%
\definecolor{currentstroke}{rgb}{0.000000,0.000000,0.000000}%
\pgfsetstrokecolor{currentstroke}%
\pgfsetdash{}{0pt}%
\pgfsys@defobject{currentmarker}{\pgfqpoint{-0.048611in}{0.000000in}}{\pgfqpoint{-0.000000in}{0.000000in}}{%
\pgfpathmoveto{\pgfqpoint{-0.000000in}{0.000000in}}%
\pgfpathlineto{\pgfqpoint{-0.048611in}{0.000000in}}%
\pgfusepath{stroke,fill}%
}%
\begin{pgfscope}%
\pgfsys@transformshift{1.000000in}{0.720000in}%
\pgfsys@useobject{currentmarker}{}%
\end{pgfscope}%
\end{pgfscope}%
\begin{pgfscope}%
\definecolor{textcolor}{rgb}{0.000000,0.000000,0.000000}%
\pgfsetstrokecolor{textcolor}%
\pgfsetfillcolor{textcolor}%
\pgftext[x=0.560215in, y=0.619981in, left, base]{\color{textcolor}\sffamily\fontsize{20.000000}{24.000000}\selectfont \(\displaystyle {0.0}\)}%
\end{pgfscope}%
\begin{pgfscope}%
\pgfsetbuttcap%
\pgfsetroundjoin%
\definecolor{currentfill}{rgb}{0.000000,0.000000,0.000000}%
\pgfsetfillcolor{currentfill}%
\pgfsetlinewidth{0.803000pt}%
\definecolor{currentstroke}{rgb}{0.000000,0.000000,0.000000}%
\pgfsetstrokecolor{currentstroke}%
\pgfsetdash{}{0pt}%
\pgfsys@defobject{currentmarker}{\pgfqpoint{-0.048611in}{0.000000in}}{\pgfqpoint{-0.000000in}{0.000000in}}{%
\pgfpathmoveto{\pgfqpoint{-0.000000in}{0.000000in}}%
\pgfpathlineto{\pgfqpoint{-0.048611in}{0.000000in}}%
\pgfusepath{stroke,fill}%
}%
\begin{pgfscope}%
\pgfsys@transformshift{1.000000in}{1.421250in}%
\pgfsys@useobject{currentmarker}{}%
\end{pgfscope}%
\end{pgfscope}%
\begin{pgfscope}%
\definecolor{textcolor}{rgb}{0.000000,0.000000,0.000000}%
\pgfsetstrokecolor{textcolor}%
\pgfsetfillcolor{textcolor}%
\pgftext[x=0.560215in, y=1.321230in, left, base]{\color{textcolor}\sffamily\fontsize{20.000000}{24.000000}\selectfont \(\displaystyle {0.2}\)}%
\end{pgfscope}%
\begin{pgfscope}%
\pgfsetbuttcap%
\pgfsetroundjoin%
\definecolor{currentfill}{rgb}{0.000000,0.000000,0.000000}%
\pgfsetfillcolor{currentfill}%
\pgfsetlinewidth{0.803000pt}%
\definecolor{currentstroke}{rgb}{0.000000,0.000000,0.000000}%
\pgfsetstrokecolor{currentstroke}%
\pgfsetdash{}{0pt}%
\pgfsys@defobject{currentmarker}{\pgfqpoint{-0.048611in}{0.000000in}}{\pgfqpoint{-0.000000in}{0.000000in}}{%
\pgfpathmoveto{\pgfqpoint{-0.000000in}{0.000000in}}%
\pgfpathlineto{\pgfqpoint{-0.048611in}{0.000000in}}%
\pgfusepath{stroke,fill}%
}%
\begin{pgfscope}%
\pgfsys@transformshift{1.000000in}{2.122499in}%
\pgfsys@useobject{currentmarker}{}%
\end{pgfscope}%
\end{pgfscope}%
\begin{pgfscope}%
\definecolor{textcolor}{rgb}{0.000000,0.000000,0.000000}%
\pgfsetstrokecolor{textcolor}%
\pgfsetfillcolor{textcolor}%
\pgftext[x=0.560215in, y=2.022480in, left, base]{\color{textcolor}\sffamily\fontsize{20.000000}{24.000000}\selectfont \(\displaystyle {0.4}\)}%
\end{pgfscope}%
\begin{pgfscope}%
\pgfsetbuttcap%
\pgfsetroundjoin%
\definecolor{currentfill}{rgb}{0.000000,0.000000,0.000000}%
\pgfsetfillcolor{currentfill}%
\pgfsetlinewidth{0.803000pt}%
\definecolor{currentstroke}{rgb}{0.000000,0.000000,0.000000}%
\pgfsetstrokecolor{currentstroke}%
\pgfsetdash{}{0pt}%
\pgfsys@defobject{currentmarker}{\pgfqpoint{-0.048611in}{0.000000in}}{\pgfqpoint{-0.000000in}{0.000000in}}{%
\pgfpathmoveto{\pgfqpoint{-0.000000in}{0.000000in}}%
\pgfpathlineto{\pgfqpoint{-0.048611in}{0.000000in}}%
\pgfusepath{stroke,fill}%
}%
\begin{pgfscope}%
\pgfsys@transformshift{1.000000in}{2.823749in}%
\pgfsys@useobject{currentmarker}{}%
\end{pgfscope}%
\end{pgfscope}%
\begin{pgfscope}%
\definecolor{textcolor}{rgb}{0.000000,0.000000,0.000000}%
\pgfsetstrokecolor{textcolor}%
\pgfsetfillcolor{textcolor}%
\pgftext[x=0.560215in, y=2.723730in, left, base]{\color{textcolor}\sffamily\fontsize{20.000000}{24.000000}\selectfont \(\displaystyle {0.6}\)}%
\end{pgfscope}%
\begin{pgfscope}%
\pgfsetbuttcap%
\pgfsetroundjoin%
\definecolor{currentfill}{rgb}{0.000000,0.000000,0.000000}%
\pgfsetfillcolor{currentfill}%
\pgfsetlinewidth{0.803000pt}%
\definecolor{currentstroke}{rgb}{0.000000,0.000000,0.000000}%
\pgfsetstrokecolor{currentstroke}%
\pgfsetdash{}{0pt}%
\pgfsys@defobject{currentmarker}{\pgfqpoint{-0.048611in}{0.000000in}}{\pgfqpoint{-0.000000in}{0.000000in}}{%
\pgfpathmoveto{\pgfqpoint{-0.000000in}{0.000000in}}%
\pgfpathlineto{\pgfqpoint{-0.048611in}{0.000000in}}%
\pgfusepath{stroke,fill}%
}%
\begin{pgfscope}%
\pgfsys@transformshift{1.000000in}{3.524999in}%
\pgfsys@useobject{currentmarker}{}%
\end{pgfscope}%
\end{pgfscope}%
\begin{pgfscope}%
\definecolor{textcolor}{rgb}{0.000000,0.000000,0.000000}%
\pgfsetstrokecolor{textcolor}%
\pgfsetfillcolor{textcolor}%
\pgftext[x=0.560215in, y=3.424979in, left, base]{\color{textcolor}\sffamily\fontsize{20.000000}{24.000000}\selectfont \(\displaystyle {0.8}\)}%
\end{pgfscope}%
\begin{pgfscope}%
\pgfsetbuttcap%
\pgfsetroundjoin%
\definecolor{currentfill}{rgb}{0.000000,0.000000,0.000000}%
\pgfsetfillcolor{currentfill}%
\pgfsetlinewidth{0.803000pt}%
\definecolor{currentstroke}{rgb}{0.000000,0.000000,0.000000}%
\pgfsetstrokecolor{currentstroke}%
\pgfsetdash{}{0pt}%
\pgfsys@defobject{currentmarker}{\pgfqpoint{-0.048611in}{0.000000in}}{\pgfqpoint{-0.000000in}{0.000000in}}{%
\pgfpathmoveto{\pgfqpoint{-0.000000in}{0.000000in}}%
\pgfpathlineto{\pgfqpoint{-0.048611in}{0.000000in}}%
\pgfusepath{stroke,fill}%
}%
\begin{pgfscope}%
\pgfsys@transformshift{1.000000in}{4.226248in}%
\pgfsys@useobject{currentmarker}{}%
\end{pgfscope}%
\end{pgfscope}%
\begin{pgfscope}%
\definecolor{textcolor}{rgb}{0.000000,0.000000,0.000000}%
\pgfsetstrokecolor{textcolor}%
\pgfsetfillcolor{textcolor}%
\pgftext[x=0.560215in, y=4.126229in, left, base]{\color{textcolor}\sffamily\fontsize{20.000000}{24.000000}\selectfont \(\displaystyle {1.0}\)}%
\end{pgfscope}%
\begin{pgfscope}%
\pgfsetbuttcap%
\pgfsetroundjoin%
\definecolor{currentfill}{rgb}{0.000000,0.000000,0.000000}%
\pgfsetfillcolor{currentfill}%
\pgfsetlinewidth{0.803000pt}%
\definecolor{currentstroke}{rgb}{0.000000,0.000000,0.000000}%
\pgfsetstrokecolor{currentstroke}%
\pgfsetdash{}{0pt}%
\pgfsys@defobject{currentmarker}{\pgfqpoint{-0.048611in}{0.000000in}}{\pgfqpoint{-0.000000in}{0.000000in}}{%
\pgfpathmoveto{\pgfqpoint{-0.000000in}{0.000000in}}%
\pgfpathlineto{\pgfqpoint{-0.048611in}{0.000000in}}%
\pgfusepath{stroke,fill}%
}%
\begin{pgfscope}%
\pgfsys@transformshift{1.000000in}{4.927498in}%
\pgfsys@useobject{currentmarker}{}%
\end{pgfscope}%
\end{pgfscope}%
\begin{pgfscope}%
\definecolor{textcolor}{rgb}{0.000000,0.000000,0.000000}%
\pgfsetstrokecolor{textcolor}%
\pgfsetfillcolor{textcolor}%
\pgftext[x=0.560215in, y=4.827479in, left, base]{\color{textcolor}\sffamily\fontsize{20.000000}{24.000000}\selectfont \(\displaystyle {1.2}\)}%
\end{pgfscope}%
\begin{pgfscope}%
\definecolor{textcolor}{rgb}{0.000000,0.000000,0.000000}%
\pgfsetstrokecolor{textcolor}%
\pgfsetfillcolor{textcolor}%
\pgftext[x=0.504660in,y=3.030000in,,bottom,rotate=90.000000]{\color{textcolor}\sffamily\fontsize{20.000000}{24.000000}\selectfont \(\displaystyle \mathrm{Wasserstein\ Distance}/\si{ns}\)}%
\end{pgfscope}%
\begin{pgfscope}%
\pgfpathrectangle{\pgfqpoint{1.000000in}{0.720000in}}{\pgfqpoint{4.800000in}{4.620000in}}%
\pgfusepath{clip}%
\pgfsetrectcap%
\pgfsetroundjoin%
\pgfsetlinewidth{1.003750pt}%
\definecolor{currentstroke}{rgb}{0.000000,0.000000,0.000000}%
\pgfsetstrokecolor{currentstroke}%
\pgfsetdash{}{0pt}%
\pgfpathmoveto{\pgfqpoint{1.300000in}{1.169035in}}%
\pgfpathlineto{\pgfqpoint{1.300000in}{0.720429in}}%
\pgfusepath{stroke}%
\end{pgfscope}%
\begin{pgfscope}%
\pgfpathrectangle{\pgfqpoint{1.000000in}{0.720000in}}{\pgfqpoint{4.800000in}{4.620000in}}%
\pgfusepath{clip}%
\pgfsetrectcap%
\pgfsetroundjoin%
\pgfsetlinewidth{1.003750pt}%
\definecolor{currentstroke}{rgb}{0.000000,0.000000,0.000000}%
\pgfsetstrokecolor{currentstroke}%
\pgfsetdash{}{0pt}%
\pgfpathmoveto{\pgfqpoint{1.300000in}{2.103806in}}%
\pgfpathlineto{\pgfqpoint{1.300000in}{3.407481in}}%
\pgfusepath{stroke}%
\end{pgfscope}%
\begin{pgfscope}%
\pgfpathrectangle{\pgfqpoint{1.000000in}{0.720000in}}{\pgfqpoint{4.800000in}{4.620000in}}%
\pgfusepath{clip}%
\pgfsetrectcap%
\pgfsetroundjoin%
\pgfsetlinewidth{1.003750pt}%
\definecolor{currentstroke}{rgb}{0.000000,0.000000,0.000000}%
\pgfsetstrokecolor{currentstroke}%
\pgfsetdash{}{0pt}%
\pgfpathmoveto{\pgfqpoint{1.262500in}{0.720429in}}%
\pgfpathlineto{\pgfqpoint{1.337500in}{0.720429in}}%
\pgfusepath{stroke}%
\end{pgfscope}%
\begin{pgfscope}%
\pgfpathrectangle{\pgfqpoint{1.000000in}{0.720000in}}{\pgfqpoint{4.800000in}{4.620000in}}%
\pgfusepath{clip}%
\pgfsetrectcap%
\pgfsetroundjoin%
\pgfsetlinewidth{1.003750pt}%
\definecolor{currentstroke}{rgb}{0.000000,0.000000,0.000000}%
\pgfsetstrokecolor{currentstroke}%
\pgfsetdash{}{0pt}%
\pgfpathmoveto{\pgfqpoint{1.262500in}{3.407481in}}%
\pgfpathlineto{\pgfqpoint{1.337500in}{3.407481in}}%
\pgfusepath{stroke}%
\end{pgfscope}%
\begin{pgfscope}%
\pgfpathrectangle{\pgfqpoint{1.000000in}{0.720000in}}{\pgfqpoint{4.800000in}{4.620000in}}%
\pgfusepath{clip}%
\pgfsetrectcap%
\pgfsetroundjoin%
\pgfsetlinewidth{1.003750pt}%
\definecolor{currentstroke}{rgb}{0.000000,0.000000,0.000000}%
\pgfsetstrokecolor{currentstroke}%
\pgfsetdash{}{0pt}%
\pgfpathmoveto{\pgfqpoint{1.600000in}{1.809859in}}%
\pgfpathlineto{\pgfqpoint{1.600000in}{0.808337in}}%
\pgfusepath{stroke}%
\end{pgfscope}%
\begin{pgfscope}%
\pgfpathrectangle{\pgfqpoint{1.000000in}{0.720000in}}{\pgfqpoint{4.800000in}{4.620000in}}%
\pgfusepath{clip}%
\pgfsetrectcap%
\pgfsetroundjoin%
\pgfsetlinewidth{1.003750pt}%
\definecolor{currentstroke}{rgb}{0.000000,0.000000,0.000000}%
\pgfsetstrokecolor{currentstroke}%
\pgfsetdash{}{0pt}%
\pgfpathmoveto{\pgfqpoint{1.600000in}{2.926791in}}%
\pgfpathlineto{\pgfqpoint{1.600000in}{4.561083in}}%
\pgfusepath{stroke}%
\end{pgfscope}%
\begin{pgfscope}%
\pgfpathrectangle{\pgfqpoint{1.000000in}{0.720000in}}{\pgfqpoint{4.800000in}{4.620000in}}%
\pgfusepath{clip}%
\pgfsetrectcap%
\pgfsetroundjoin%
\pgfsetlinewidth{1.003750pt}%
\definecolor{currentstroke}{rgb}{0.000000,0.000000,0.000000}%
\pgfsetstrokecolor{currentstroke}%
\pgfsetdash{}{0pt}%
\pgfpathmoveto{\pgfqpoint{1.562500in}{0.808337in}}%
\pgfpathlineto{\pgfqpoint{1.637500in}{0.808337in}}%
\pgfusepath{stroke}%
\end{pgfscope}%
\begin{pgfscope}%
\pgfpathrectangle{\pgfqpoint{1.000000in}{0.720000in}}{\pgfqpoint{4.800000in}{4.620000in}}%
\pgfusepath{clip}%
\pgfsetrectcap%
\pgfsetroundjoin%
\pgfsetlinewidth{1.003750pt}%
\definecolor{currentstroke}{rgb}{0.000000,0.000000,0.000000}%
\pgfsetstrokecolor{currentstroke}%
\pgfsetdash{}{0pt}%
\pgfpathmoveto{\pgfqpoint{1.562500in}{4.561083in}}%
\pgfpathlineto{\pgfqpoint{1.637500in}{4.561083in}}%
\pgfusepath{stroke}%
\end{pgfscope}%
\begin{pgfscope}%
\pgfpathrectangle{\pgfqpoint{1.000000in}{0.720000in}}{\pgfqpoint{4.800000in}{4.620000in}}%
\pgfusepath{clip}%
\pgfsetrectcap%
\pgfsetroundjoin%
\pgfsetlinewidth{1.003750pt}%
\definecolor{currentstroke}{rgb}{0.000000,0.000000,0.000000}%
\pgfsetstrokecolor{currentstroke}%
\pgfsetdash{}{0pt}%
\pgfpathmoveto{\pgfqpoint{1.900000in}{2.114223in}}%
\pgfpathlineto{\pgfqpoint{1.900000in}{1.084338in}}%
\pgfusepath{stroke}%
\end{pgfscope}%
\begin{pgfscope}%
\pgfpathrectangle{\pgfqpoint{1.000000in}{0.720000in}}{\pgfqpoint{4.800000in}{4.620000in}}%
\pgfusepath{clip}%
\pgfsetrectcap%
\pgfsetroundjoin%
\pgfsetlinewidth{1.003750pt}%
\definecolor{currentstroke}{rgb}{0.000000,0.000000,0.000000}%
\pgfsetstrokecolor{currentstroke}%
\pgfsetdash{}{0pt}%
\pgfpathmoveto{\pgfqpoint{1.900000in}{3.183098in}}%
\pgfpathlineto{\pgfqpoint{1.900000in}{4.785416in}}%
\pgfusepath{stroke}%
\end{pgfscope}%
\begin{pgfscope}%
\pgfpathrectangle{\pgfqpoint{1.000000in}{0.720000in}}{\pgfqpoint{4.800000in}{4.620000in}}%
\pgfusepath{clip}%
\pgfsetrectcap%
\pgfsetroundjoin%
\pgfsetlinewidth{1.003750pt}%
\definecolor{currentstroke}{rgb}{0.000000,0.000000,0.000000}%
\pgfsetstrokecolor{currentstroke}%
\pgfsetdash{}{0pt}%
\pgfpathmoveto{\pgfqpoint{1.862500in}{1.084338in}}%
\pgfpathlineto{\pgfqpoint{1.937500in}{1.084338in}}%
\pgfusepath{stroke}%
\end{pgfscope}%
\begin{pgfscope}%
\pgfpathrectangle{\pgfqpoint{1.000000in}{0.720000in}}{\pgfqpoint{4.800000in}{4.620000in}}%
\pgfusepath{clip}%
\pgfsetrectcap%
\pgfsetroundjoin%
\pgfsetlinewidth{1.003750pt}%
\definecolor{currentstroke}{rgb}{0.000000,0.000000,0.000000}%
\pgfsetstrokecolor{currentstroke}%
\pgfsetdash{}{0pt}%
\pgfpathmoveto{\pgfqpoint{1.862500in}{4.785416in}}%
\pgfpathlineto{\pgfqpoint{1.937500in}{4.785416in}}%
\pgfusepath{stroke}%
\end{pgfscope}%
\begin{pgfscope}%
\pgfpathrectangle{\pgfqpoint{1.000000in}{0.720000in}}{\pgfqpoint{4.800000in}{4.620000in}}%
\pgfusepath{clip}%
\pgfsetrectcap%
\pgfsetroundjoin%
\pgfsetlinewidth{1.003750pt}%
\definecolor{currentstroke}{rgb}{0.000000,0.000000,0.000000}%
\pgfsetstrokecolor{currentstroke}%
\pgfsetdash{}{0pt}%
\pgfpathmoveto{\pgfqpoint{2.200000in}{2.291483in}}%
\pgfpathlineto{\pgfqpoint{2.200000in}{0.988328in}}%
\pgfusepath{stroke}%
\end{pgfscope}%
\begin{pgfscope}%
\pgfpathrectangle{\pgfqpoint{1.000000in}{0.720000in}}{\pgfqpoint{4.800000in}{4.620000in}}%
\pgfusepath{clip}%
\pgfsetrectcap%
\pgfsetroundjoin%
\pgfsetlinewidth{1.003750pt}%
\definecolor{currentstroke}{rgb}{0.000000,0.000000,0.000000}%
\pgfsetstrokecolor{currentstroke}%
\pgfsetdash{}{0pt}%
\pgfpathmoveto{\pgfqpoint{2.200000in}{3.309486in}}%
\pgfpathlineto{\pgfqpoint{2.200000in}{4.801134in}}%
\pgfusepath{stroke}%
\end{pgfscope}%
\begin{pgfscope}%
\pgfpathrectangle{\pgfqpoint{1.000000in}{0.720000in}}{\pgfqpoint{4.800000in}{4.620000in}}%
\pgfusepath{clip}%
\pgfsetrectcap%
\pgfsetroundjoin%
\pgfsetlinewidth{1.003750pt}%
\definecolor{currentstroke}{rgb}{0.000000,0.000000,0.000000}%
\pgfsetstrokecolor{currentstroke}%
\pgfsetdash{}{0pt}%
\pgfpathmoveto{\pgfqpoint{2.162500in}{0.988328in}}%
\pgfpathlineto{\pgfqpoint{2.237500in}{0.988328in}}%
\pgfusepath{stroke}%
\end{pgfscope}%
\begin{pgfscope}%
\pgfpathrectangle{\pgfqpoint{1.000000in}{0.720000in}}{\pgfqpoint{4.800000in}{4.620000in}}%
\pgfusepath{clip}%
\pgfsetrectcap%
\pgfsetroundjoin%
\pgfsetlinewidth{1.003750pt}%
\definecolor{currentstroke}{rgb}{0.000000,0.000000,0.000000}%
\pgfsetstrokecolor{currentstroke}%
\pgfsetdash{}{0pt}%
\pgfpathmoveto{\pgfqpoint{2.162500in}{4.801134in}}%
\pgfpathlineto{\pgfqpoint{2.237500in}{4.801134in}}%
\pgfusepath{stroke}%
\end{pgfscope}%
\begin{pgfscope}%
\pgfpathrectangle{\pgfqpoint{1.000000in}{0.720000in}}{\pgfqpoint{4.800000in}{4.620000in}}%
\pgfusepath{clip}%
\pgfsetrectcap%
\pgfsetroundjoin%
\pgfsetlinewidth{1.003750pt}%
\definecolor{currentstroke}{rgb}{0.000000,0.000000,0.000000}%
\pgfsetstrokecolor{currentstroke}%
\pgfsetdash{}{0pt}%
\pgfpathmoveto{\pgfqpoint{2.500000in}{2.470818in}}%
\pgfpathlineto{\pgfqpoint{2.500000in}{1.200656in}}%
\pgfusepath{stroke}%
\end{pgfscope}%
\begin{pgfscope}%
\pgfpathrectangle{\pgfqpoint{1.000000in}{0.720000in}}{\pgfqpoint{4.800000in}{4.620000in}}%
\pgfusepath{clip}%
\pgfsetrectcap%
\pgfsetroundjoin%
\pgfsetlinewidth{1.003750pt}%
\definecolor{currentstroke}{rgb}{0.000000,0.000000,0.000000}%
\pgfsetstrokecolor{currentstroke}%
\pgfsetdash{}{0pt}%
\pgfpathmoveto{\pgfqpoint{2.500000in}{3.475065in}}%
\pgfpathlineto{\pgfqpoint{2.500000in}{4.969194in}}%
\pgfusepath{stroke}%
\end{pgfscope}%
\begin{pgfscope}%
\pgfpathrectangle{\pgfqpoint{1.000000in}{0.720000in}}{\pgfqpoint{4.800000in}{4.620000in}}%
\pgfusepath{clip}%
\pgfsetrectcap%
\pgfsetroundjoin%
\pgfsetlinewidth{1.003750pt}%
\definecolor{currentstroke}{rgb}{0.000000,0.000000,0.000000}%
\pgfsetstrokecolor{currentstroke}%
\pgfsetdash{}{0pt}%
\pgfpathmoveto{\pgfqpoint{2.462500in}{1.200656in}}%
\pgfpathlineto{\pgfqpoint{2.537500in}{1.200656in}}%
\pgfusepath{stroke}%
\end{pgfscope}%
\begin{pgfscope}%
\pgfpathrectangle{\pgfqpoint{1.000000in}{0.720000in}}{\pgfqpoint{4.800000in}{4.620000in}}%
\pgfusepath{clip}%
\pgfsetrectcap%
\pgfsetroundjoin%
\pgfsetlinewidth{1.003750pt}%
\definecolor{currentstroke}{rgb}{0.000000,0.000000,0.000000}%
\pgfsetstrokecolor{currentstroke}%
\pgfsetdash{}{0pt}%
\pgfpathmoveto{\pgfqpoint{2.462500in}{4.969194in}}%
\pgfpathlineto{\pgfqpoint{2.537500in}{4.969194in}}%
\pgfusepath{stroke}%
\end{pgfscope}%
\begin{pgfscope}%
\pgfpathrectangle{\pgfqpoint{1.000000in}{0.720000in}}{\pgfqpoint{4.800000in}{4.620000in}}%
\pgfusepath{clip}%
\pgfsetrectcap%
\pgfsetroundjoin%
\pgfsetlinewidth{1.003750pt}%
\definecolor{currentstroke}{rgb}{0.000000,0.000000,0.000000}%
\pgfsetstrokecolor{currentstroke}%
\pgfsetdash{}{0pt}%
\pgfpathmoveto{\pgfqpoint{2.800000in}{2.527181in}}%
\pgfpathlineto{\pgfqpoint{2.800000in}{1.411982in}}%
\pgfusepath{stroke}%
\end{pgfscope}%
\begin{pgfscope}%
\pgfpathrectangle{\pgfqpoint{1.000000in}{0.720000in}}{\pgfqpoint{4.800000in}{4.620000in}}%
\pgfusepath{clip}%
\pgfsetrectcap%
\pgfsetroundjoin%
\pgfsetlinewidth{1.003750pt}%
\definecolor{currentstroke}{rgb}{0.000000,0.000000,0.000000}%
\pgfsetstrokecolor{currentstroke}%
\pgfsetdash{}{0pt}%
\pgfpathmoveto{\pgfqpoint{2.800000in}{3.479811in}}%
\pgfpathlineto{\pgfqpoint{2.800000in}{4.902681in}}%
\pgfusepath{stroke}%
\end{pgfscope}%
\begin{pgfscope}%
\pgfpathrectangle{\pgfqpoint{1.000000in}{0.720000in}}{\pgfqpoint{4.800000in}{4.620000in}}%
\pgfusepath{clip}%
\pgfsetrectcap%
\pgfsetroundjoin%
\pgfsetlinewidth{1.003750pt}%
\definecolor{currentstroke}{rgb}{0.000000,0.000000,0.000000}%
\pgfsetstrokecolor{currentstroke}%
\pgfsetdash{}{0pt}%
\pgfpathmoveto{\pgfqpoint{2.762500in}{1.411982in}}%
\pgfpathlineto{\pgfqpoint{2.837500in}{1.411982in}}%
\pgfusepath{stroke}%
\end{pgfscope}%
\begin{pgfscope}%
\pgfpathrectangle{\pgfqpoint{1.000000in}{0.720000in}}{\pgfqpoint{4.800000in}{4.620000in}}%
\pgfusepath{clip}%
\pgfsetrectcap%
\pgfsetroundjoin%
\pgfsetlinewidth{1.003750pt}%
\definecolor{currentstroke}{rgb}{0.000000,0.000000,0.000000}%
\pgfsetstrokecolor{currentstroke}%
\pgfsetdash{}{0pt}%
\pgfpathmoveto{\pgfqpoint{2.762500in}{4.902681in}}%
\pgfpathlineto{\pgfqpoint{2.837500in}{4.902681in}}%
\pgfusepath{stroke}%
\end{pgfscope}%
\begin{pgfscope}%
\pgfpathrectangle{\pgfqpoint{1.000000in}{0.720000in}}{\pgfqpoint{4.800000in}{4.620000in}}%
\pgfusepath{clip}%
\pgfsetrectcap%
\pgfsetroundjoin%
\pgfsetlinewidth{1.003750pt}%
\definecolor{currentstroke}{rgb}{0.000000,0.000000,0.000000}%
\pgfsetstrokecolor{currentstroke}%
\pgfsetdash{}{0pt}%
\pgfpathmoveto{\pgfqpoint{3.100000in}{2.610220in}}%
\pgfpathlineto{\pgfqpoint{3.100000in}{1.568395in}}%
\pgfusepath{stroke}%
\end{pgfscope}%
\begin{pgfscope}%
\pgfpathrectangle{\pgfqpoint{1.000000in}{0.720000in}}{\pgfqpoint{4.800000in}{4.620000in}}%
\pgfusepath{clip}%
\pgfsetrectcap%
\pgfsetroundjoin%
\pgfsetlinewidth{1.003750pt}%
\definecolor{currentstroke}{rgb}{0.000000,0.000000,0.000000}%
\pgfsetstrokecolor{currentstroke}%
\pgfsetdash{}{0pt}%
\pgfpathmoveto{\pgfqpoint{3.100000in}{3.544204in}}%
\pgfpathlineto{\pgfqpoint{3.100000in}{4.937558in}}%
\pgfusepath{stroke}%
\end{pgfscope}%
\begin{pgfscope}%
\pgfpathrectangle{\pgfqpoint{1.000000in}{0.720000in}}{\pgfqpoint{4.800000in}{4.620000in}}%
\pgfusepath{clip}%
\pgfsetrectcap%
\pgfsetroundjoin%
\pgfsetlinewidth{1.003750pt}%
\definecolor{currentstroke}{rgb}{0.000000,0.000000,0.000000}%
\pgfsetstrokecolor{currentstroke}%
\pgfsetdash{}{0pt}%
\pgfpathmoveto{\pgfqpoint{3.062500in}{1.568395in}}%
\pgfpathlineto{\pgfqpoint{3.137500in}{1.568395in}}%
\pgfusepath{stroke}%
\end{pgfscope}%
\begin{pgfscope}%
\pgfpathrectangle{\pgfqpoint{1.000000in}{0.720000in}}{\pgfqpoint{4.800000in}{4.620000in}}%
\pgfusepath{clip}%
\pgfsetrectcap%
\pgfsetroundjoin%
\pgfsetlinewidth{1.003750pt}%
\definecolor{currentstroke}{rgb}{0.000000,0.000000,0.000000}%
\pgfsetstrokecolor{currentstroke}%
\pgfsetdash{}{0pt}%
\pgfpathmoveto{\pgfqpoint{3.062500in}{4.937558in}}%
\pgfpathlineto{\pgfqpoint{3.137500in}{4.937558in}}%
\pgfusepath{stroke}%
\end{pgfscope}%
\begin{pgfscope}%
\pgfpathrectangle{\pgfqpoint{1.000000in}{0.720000in}}{\pgfqpoint{4.800000in}{4.620000in}}%
\pgfusepath{clip}%
\pgfsetrectcap%
\pgfsetroundjoin%
\pgfsetlinewidth{1.003750pt}%
\definecolor{currentstroke}{rgb}{0.000000,0.000000,0.000000}%
\pgfsetstrokecolor{currentstroke}%
\pgfsetdash{}{0pt}%
\pgfpathmoveto{\pgfqpoint{3.400000in}{2.619069in}}%
\pgfpathlineto{\pgfqpoint{3.400000in}{1.679259in}}%
\pgfusepath{stroke}%
\end{pgfscope}%
\begin{pgfscope}%
\pgfpathrectangle{\pgfqpoint{1.000000in}{0.720000in}}{\pgfqpoint{4.800000in}{4.620000in}}%
\pgfusepath{clip}%
\pgfsetrectcap%
\pgfsetroundjoin%
\pgfsetlinewidth{1.003750pt}%
\definecolor{currentstroke}{rgb}{0.000000,0.000000,0.000000}%
\pgfsetstrokecolor{currentstroke}%
\pgfsetdash{}{0pt}%
\pgfpathmoveto{\pgfqpoint{3.400000in}{3.512155in}}%
\pgfpathlineto{\pgfqpoint{3.400000in}{4.638770in}}%
\pgfusepath{stroke}%
\end{pgfscope}%
\begin{pgfscope}%
\pgfpathrectangle{\pgfqpoint{1.000000in}{0.720000in}}{\pgfqpoint{4.800000in}{4.620000in}}%
\pgfusepath{clip}%
\pgfsetrectcap%
\pgfsetroundjoin%
\pgfsetlinewidth{1.003750pt}%
\definecolor{currentstroke}{rgb}{0.000000,0.000000,0.000000}%
\pgfsetstrokecolor{currentstroke}%
\pgfsetdash{}{0pt}%
\pgfpathmoveto{\pgfqpoint{3.362500in}{1.679259in}}%
\pgfpathlineto{\pgfqpoint{3.437500in}{1.679259in}}%
\pgfusepath{stroke}%
\end{pgfscope}%
\begin{pgfscope}%
\pgfpathrectangle{\pgfqpoint{1.000000in}{0.720000in}}{\pgfqpoint{4.800000in}{4.620000in}}%
\pgfusepath{clip}%
\pgfsetrectcap%
\pgfsetroundjoin%
\pgfsetlinewidth{1.003750pt}%
\definecolor{currentstroke}{rgb}{0.000000,0.000000,0.000000}%
\pgfsetstrokecolor{currentstroke}%
\pgfsetdash{}{0pt}%
\pgfpathmoveto{\pgfqpoint{3.362500in}{4.638770in}}%
\pgfpathlineto{\pgfqpoint{3.437500in}{4.638770in}}%
\pgfusepath{stroke}%
\end{pgfscope}%
\begin{pgfscope}%
\pgfpathrectangle{\pgfqpoint{1.000000in}{0.720000in}}{\pgfqpoint{4.800000in}{4.620000in}}%
\pgfusepath{clip}%
\pgfsetrectcap%
\pgfsetroundjoin%
\pgfsetlinewidth{1.003750pt}%
\definecolor{currentstroke}{rgb}{0.000000,0.000000,0.000000}%
\pgfsetstrokecolor{currentstroke}%
\pgfsetdash{}{0pt}%
\pgfpathmoveto{\pgfqpoint{3.700000in}{2.681560in}}%
\pgfpathlineto{\pgfqpoint{3.700000in}{1.781356in}}%
\pgfusepath{stroke}%
\end{pgfscope}%
\begin{pgfscope}%
\pgfpathrectangle{\pgfqpoint{1.000000in}{0.720000in}}{\pgfqpoint{4.800000in}{4.620000in}}%
\pgfusepath{clip}%
\pgfsetrectcap%
\pgfsetroundjoin%
\pgfsetlinewidth{1.003750pt}%
\definecolor{currentstroke}{rgb}{0.000000,0.000000,0.000000}%
\pgfsetstrokecolor{currentstroke}%
\pgfsetdash{}{0pt}%
\pgfpathmoveto{\pgfqpoint{3.700000in}{3.650320in}}%
\pgfpathlineto{\pgfqpoint{3.700000in}{4.795914in}}%
\pgfusepath{stroke}%
\end{pgfscope}%
\begin{pgfscope}%
\pgfpathrectangle{\pgfqpoint{1.000000in}{0.720000in}}{\pgfqpoint{4.800000in}{4.620000in}}%
\pgfusepath{clip}%
\pgfsetrectcap%
\pgfsetroundjoin%
\pgfsetlinewidth{1.003750pt}%
\definecolor{currentstroke}{rgb}{0.000000,0.000000,0.000000}%
\pgfsetstrokecolor{currentstroke}%
\pgfsetdash{}{0pt}%
\pgfpathmoveto{\pgfqpoint{3.662500in}{1.781356in}}%
\pgfpathlineto{\pgfqpoint{3.737500in}{1.781356in}}%
\pgfusepath{stroke}%
\end{pgfscope}%
\begin{pgfscope}%
\pgfpathrectangle{\pgfqpoint{1.000000in}{0.720000in}}{\pgfqpoint{4.800000in}{4.620000in}}%
\pgfusepath{clip}%
\pgfsetrectcap%
\pgfsetroundjoin%
\pgfsetlinewidth{1.003750pt}%
\definecolor{currentstroke}{rgb}{0.000000,0.000000,0.000000}%
\pgfsetstrokecolor{currentstroke}%
\pgfsetdash{}{0pt}%
\pgfpathmoveto{\pgfqpoint{3.662500in}{4.795914in}}%
\pgfpathlineto{\pgfqpoint{3.737500in}{4.795914in}}%
\pgfusepath{stroke}%
\end{pgfscope}%
\begin{pgfscope}%
\pgfpathrectangle{\pgfqpoint{1.000000in}{0.720000in}}{\pgfqpoint{4.800000in}{4.620000in}}%
\pgfusepath{clip}%
\pgfsetrectcap%
\pgfsetroundjoin%
\pgfsetlinewidth{1.003750pt}%
\definecolor{currentstroke}{rgb}{0.000000,0.000000,0.000000}%
\pgfsetstrokecolor{currentstroke}%
\pgfsetdash{}{0pt}%
\pgfpathmoveto{\pgfqpoint{4.000000in}{2.791569in}}%
\pgfpathlineto{\pgfqpoint{4.000000in}{2.034710in}}%
\pgfusepath{stroke}%
\end{pgfscope}%
\begin{pgfscope}%
\pgfpathrectangle{\pgfqpoint{1.000000in}{0.720000in}}{\pgfqpoint{4.800000in}{4.620000in}}%
\pgfusepath{clip}%
\pgfsetrectcap%
\pgfsetroundjoin%
\pgfsetlinewidth{1.003750pt}%
\definecolor{currentstroke}{rgb}{0.000000,0.000000,0.000000}%
\pgfsetstrokecolor{currentstroke}%
\pgfsetdash{}{0pt}%
\pgfpathmoveto{\pgfqpoint{4.000000in}{3.432691in}}%
\pgfpathlineto{\pgfqpoint{4.000000in}{4.144493in}}%
\pgfusepath{stroke}%
\end{pgfscope}%
\begin{pgfscope}%
\pgfpathrectangle{\pgfqpoint{1.000000in}{0.720000in}}{\pgfqpoint{4.800000in}{4.620000in}}%
\pgfusepath{clip}%
\pgfsetrectcap%
\pgfsetroundjoin%
\pgfsetlinewidth{1.003750pt}%
\definecolor{currentstroke}{rgb}{0.000000,0.000000,0.000000}%
\pgfsetstrokecolor{currentstroke}%
\pgfsetdash{}{0pt}%
\pgfpathmoveto{\pgfqpoint{3.962500in}{2.034710in}}%
\pgfpathlineto{\pgfqpoint{4.037500in}{2.034710in}}%
\pgfusepath{stroke}%
\end{pgfscope}%
\begin{pgfscope}%
\pgfpathrectangle{\pgfqpoint{1.000000in}{0.720000in}}{\pgfqpoint{4.800000in}{4.620000in}}%
\pgfusepath{clip}%
\pgfsetrectcap%
\pgfsetroundjoin%
\pgfsetlinewidth{1.003750pt}%
\definecolor{currentstroke}{rgb}{0.000000,0.000000,0.000000}%
\pgfsetstrokecolor{currentstroke}%
\pgfsetdash{}{0pt}%
\pgfpathmoveto{\pgfqpoint{3.962500in}{4.144493in}}%
\pgfpathlineto{\pgfqpoint{4.037500in}{4.144493in}}%
\pgfusepath{stroke}%
\end{pgfscope}%
\begin{pgfscope}%
\pgfpathrectangle{\pgfqpoint{1.000000in}{0.720000in}}{\pgfqpoint{4.800000in}{4.620000in}}%
\pgfusepath{clip}%
\pgfsetrectcap%
\pgfsetroundjoin%
\pgfsetlinewidth{1.003750pt}%
\definecolor{currentstroke}{rgb}{0.000000,0.000000,0.000000}%
\pgfsetstrokecolor{currentstroke}%
\pgfsetdash{}{0pt}%
\pgfpathmoveto{\pgfqpoint{4.300000in}{2.801023in}}%
\pgfpathlineto{\pgfqpoint{4.300000in}{2.001036in}}%
\pgfusepath{stroke}%
\end{pgfscope}%
\begin{pgfscope}%
\pgfpathrectangle{\pgfqpoint{1.000000in}{0.720000in}}{\pgfqpoint{4.800000in}{4.620000in}}%
\pgfusepath{clip}%
\pgfsetrectcap%
\pgfsetroundjoin%
\pgfsetlinewidth{1.003750pt}%
\definecolor{currentstroke}{rgb}{0.000000,0.000000,0.000000}%
\pgfsetstrokecolor{currentstroke}%
\pgfsetdash{}{0pt}%
\pgfpathmoveto{\pgfqpoint{4.300000in}{3.422301in}}%
\pgfpathlineto{\pgfqpoint{4.300000in}{3.990013in}}%
\pgfusepath{stroke}%
\end{pgfscope}%
\begin{pgfscope}%
\pgfpathrectangle{\pgfqpoint{1.000000in}{0.720000in}}{\pgfqpoint{4.800000in}{4.620000in}}%
\pgfusepath{clip}%
\pgfsetrectcap%
\pgfsetroundjoin%
\pgfsetlinewidth{1.003750pt}%
\definecolor{currentstroke}{rgb}{0.000000,0.000000,0.000000}%
\pgfsetstrokecolor{currentstroke}%
\pgfsetdash{}{0pt}%
\pgfpathmoveto{\pgfqpoint{4.262500in}{2.001036in}}%
\pgfpathlineto{\pgfqpoint{4.337500in}{2.001036in}}%
\pgfusepath{stroke}%
\end{pgfscope}%
\begin{pgfscope}%
\pgfpathrectangle{\pgfqpoint{1.000000in}{0.720000in}}{\pgfqpoint{4.800000in}{4.620000in}}%
\pgfusepath{clip}%
\pgfsetrectcap%
\pgfsetroundjoin%
\pgfsetlinewidth{1.003750pt}%
\definecolor{currentstroke}{rgb}{0.000000,0.000000,0.000000}%
\pgfsetstrokecolor{currentstroke}%
\pgfsetdash{}{0pt}%
\pgfpathmoveto{\pgfqpoint{4.262500in}{3.990013in}}%
\pgfpathlineto{\pgfqpoint{4.337500in}{3.990013in}}%
\pgfusepath{stroke}%
\end{pgfscope}%
\begin{pgfscope}%
\pgfpathrectangle{\pgfqpoint{1.000000in}{0.720000in}}{\pgfqpoint{4.800000in}{4.620000in}}%
\pgfusepath{clip}%
\pgfsetrectcap%
\pgfsetroundjoin%
\pgfsetlinewidth{1.003750pt}%
\definecolor{currentstroke}{rgb}{0.000000,0.000000,0.000000}%
\pgfsetstrokecolor{currentstroke}%
\pgfsetdash{}{0pt}%
\pgfpathmoveto{\pgfqpoint{4.600000in}{3.399568in}}%
\pgfpathlineto{\pgfqpoint{4.600000in}{2.890483in}}%
\pgfusepath{stroke}%
\end{pgfscope}%
\begin{pgfscope}%
\pgfpathrectangle{\pgfqpoint{1.000000in}{0.720000in}}{\pgfqpoint{4.800000in}{4.620000in}}%
\pgfusepath{clip}%
\pgfsetrectcap%
\pgfsetroundjoin%
\pgfsetlinewidth{1.003750pt}%
\definecolor{currentstroke}{rgb}{0.000000,0.000000,0.000000}%
\pgfsetstrokecolor{currentstroke}%
\pgfsetdash{}{0pt}%
\pgfpathmoveto{\pgfqpoint{4.600000in}{4.285052in}}%
\pgfpathlineto{\pgfqpoint{4.600000in}{5.120000in}}%
\pgfusepath{stroke}%
\end{pgfscope}%
\begin{pgfscope}%
\pgfpathrectangle{\pgfqpoint{1.000000in}{0.720000in}}{\pgfqpoint{4.800000in}{4.620000in}}%
\pgfusepath{clip}%
\pgfsetrectcap%
\pgfsetroundjoin%
\pgfsetlinewidth{1.003750pt}%
\definecolor{currentstroke}{rgb}{0.000000,0.000000,0.000000}%
\pgfsetstrokecolor{currentstroke}%
\pgfsetdash{}{0pt}%
\pgfpathmoveto{\pgfqpoint{4.562500in}{2.890483in}}%
\pgfpathlineto{\pgfqpoint{4.637500in}{2.890483in}}%
\pgfusepath{stroke}%
\end{pgfscope}%
\begin{pgfscope}%
\pgfpathrectangle{\pgfqpoint{1.000000in}{0.720000in}}{\pgfqpoint{4.800000in}{4.620000in}}%
\pgfusepath{clip}%
\pgfsetrectcap%
\pgfsetroundjoin%
\pgfsetlinewidth{1.003750pt}%
\definecolor{currentstroke}{rgb}{0.000000,0.000000,0.000000}%
\pgfsetstrokecolor{currentstroke}%
\pgfsetdash{}{0pt}%
\pgfpathmoveto{\pgfqpoint{4.562500in}{5.120000in}}%
\pgfpathlineto{\pgfqpoint{4.637500in}{5.120000in}}%
\pgfusepath{stroke}%
\end{pgfscope}%
\begin{pgfscope}%
\pgfpathrectangle{\pgfqpoint{1.000000in}{0.720000in}}{\pgfqpoint{4.800000in}{4.620000in}}%
\pgfusepath{clip}%
\pgfsetrectcap%
\pgfsetroundjoin%
\pgfsetlinewidth{1.003750pt}%
\definecolor{currentstroke}{rgb}{0.000000,0.000000,0.000000}%
\pgfsetstrokecolor{currentstroke}%
\pgfsetdash{}{0pt}%
\pgfpathmoveto{\pgfqpoint{4.900000in}{3.366848in}}%
\pgfpathlineto{\pgfqpoint{4.900000in}{3.366848in}}%
\pgfusepath{stroke}%
\end{pgfscope}%
\begin{pgfscope}%
\pgfpathrectangle{\pgfqpoint{1.000000in}{0.720000in}}{\pgfqpoint{4.800000in}{4.620000in}}%
\pgfusepath{clip}%
\pgfsetrectcap%
\pgfsetroundjoin%
\pgfsetlinewidth{1.003750pt}%
\definecolor{currentstroke}{rgb}{0.000000,0.000000,0.000000}%
\pgfsetstrokecolor{currentstroke}%
\pgfsetdash{}{0pt}%
\pgfpathmoveto{\pgfqpoint{4.900000in}{3.366848in}}%
\pgfpathlineto{\pgfqpoint{4.900000in}{3.366848in}}%
\pgfusepath{stroke}%
\end{pgfscope}%
\begin{pgfscope}%
\pgfpathrectangle{\pgfqpoint{1.000000in}{0.720000in}}{\pgfqpoint{4.800000in}{4.620000in}}%
\pgfusepath{clip}%
\pgfsetrectcap%
\pgfsetroundjoin%
\pgfsetlinewidth{1.003750pt}%
\definecolor{currentstroke}{rgb}{0.000000,0.000000,0.000000}%
\pgfsetstrokecolor{currentstroke}%
\pgfsetdash{}{0pt}%
\pgfpathmoveto{\pgfqpoint{4.862500in}{3.366848in}}%
\pgfpathlineto{\pgfqpoint{4.937500in}{3.366848in}}%
\pgfusepath{stroke}%
\end{pgfscope}%
\begin{pgfscope}%
\pgfpathrectangle{\pgfqpoint{1.000000in}{0.720000in}}{\pgfqpoint{4.800000in}{4.620000in}}%
\pgfusepath{clip}%
\pgfsetrectcap%
\pgfsetroundjoin%
\pgfsetlinewidth{1.003750pt}%
\definecolor{currentstroke}{rgb}{0.000000,0.000000,0.000000}%
\pgfsetstrokecolor{currentstroke}%
\pgfsetdash{}{0pt}%
\pgfpathmoveto{\pgfqpoint{4.862500in}{3.366848in}}%
\pgfpathlineto{\pgfqpoint{4.937500in}{3.366848in}}%
\pgfusepath{stroke}%
\end{pgfscope}%
\begin{pgfscope}%
\pgfpathrectangle{\pgfqpoint{1.000000in}{0.720000in}}{\pgfqpoint{4.800000in}{4.620000in}}%
\pgfusepath{clip}%
\pgfsetrectcap%
\pgfsetroundjoin%
\pgfsetlinewidth{1.003750pt}%
\definecolor{currentstroke}{rgb}{0.000000,0.000000,0.000000}%
\pgfsetstrokecolor{currentstroke}%
\pgfsetdash{}{0pt}%
\pgfpathmoveto{\pgfqpoint{5.200000in}{2.395066in}}%
\pgfpathlineto{\pgfqpoint{5.200000in}{2.395066in}}%
\pgfusepath{stroke}%
\end{pgfscope}%
\begin{pgfscope}%
\pgfpathrectangle{\pgfqpoint{1.000000in}{0.720000in}}{\pgfqpoint{4.800000in}{4.620000in}}%
\pgfusepath{clip}%
\pgfsetrectcap%
\pgfsetroundjoin%
\pgfsetlinewidth{1.003750pt}%
\definecolor{currentstroke}{rgb}{0.000000,0.000000,0.000000}%
\pgfsetstrokecolor{currentstroke}%
\pgfsetdash{}{0pt}%
\pgfpathmoveto{\pgfqpoint{5.200000in}{2.395066in}}%
\pgfpathlineto{\pgfqpoint{5.200000in}{2.395066in}}%
\pgfusepath{stroke}%
\end{pgfscope}%
\begin{pgfscope}%
\pgfpathrectangle{\pgfqpoint{1.000000in}{0.720000in}}{\pgfqpoint{4.800000in}{4.620000in}}%
\pgfusepath{clip}%
\pgfsetrectcap%
\pgfsetroundjoin%
\pgfsetlinewidth{1.003750pt}%
\definecolor{currentstroke}{rgb}{0.000000,0.000000,0.000000}%
\pgfsetstrokecolor{currentstroke}%
\pgfsetdash{}{0pt}%
\pgfpathmoveto{\pgfqpoint{5.162500in}{2.395066in}}%
\pgfpathlineto{\pgfqpoint{5.237500in}{2.395066in}}%
\pgfusepath{stroke}%
\end{pgfscope}%
\begin{pgfscope}%
\pgfpathrectangle{\pgfqpoint{1.000000in}{0.720000in}}{\pgfqpoint{4.800000in}{4.620000in}}%
\pgfusepath{clip}%
\pgfsetrectcap%
\pgfsetroundjoin%
\pgfsetlinewidth{1.003750pt}%
\definecolor{currentstroke}{rgb}{0.000000,0.000000,0.000000}%
\pgfsetstrokecolor{currentstroke}%
\pgfsetdash{}{0pt}%
\pgfpathmoveto{\pgfqpoint{5.162500in}{2.395066in}}%
\pgfpathlineto{\pgfqpoint{5.237500in}{2.395066in}}%
\pgfusepath{stroke}%
\end{pgfscope}%
\begin{pgfscope}%
\pgfpathrectangle{\pgfqpoint{1.000000in}{0.720000in}}{\pgfqpoint{4.800000in}{4.620000in}}%
\pgfusepath{clip}%
\pgfsetrectcap%
\pgfsetroundjoin%
\pgfsetlinewidth{1.003750pt}%
\definecolor{currentstroke}{rgb}{0.000000,0.000000,0.000000}%
\pgfsetstrokecolor{currentstroke}%
\pgfsetdash{}{0pt}%
\pgfpathmoveto{\pgfqpoint{5.500000in}{3.176830in}}%
\pgfpathlineto{\pgfqpoint{5.500000in}{3.176830in}}%
\pgfusepath{stroke}%
\end{pgfscope}%
\begin{pgfscope}%
\pgfpathrectangle{\pgfqpoint{1.000000in}{0.720000in}}{\pgfqpoint{4.800000in}{4.620000in}}%
\pgfusepath{clip}%
\pgfsetrectcap%
\pgfsetroundjoin%
\pgfsetlinewidth{1.003750pt}%
\definecolor{currentstroke}{rgb}{0.000000,0.000000,0.000000}%
\pgfsetstrokecolor{currentstroke}%
\pgfsetdash{}{0pt}%
\pgfpathmoveto{\pgfqpoint{5.500000in}{3.176830in}}%
\pgfpathlineto{\pgfqpoint{5.500000in}{3.176830in}}%
\pgfusepath{stroke}%
\end{pgfscope}%
\begin{pgfscope}%
\pgfpathrectangle{\pgfqpoint{1.000000in}{0.720000in}}{\pgfqpoint{4.800000in}{4.620000in}}%
\pgfusepath{clip}%
\pgfsetrectcap%
\pgfsetroundjoin%
\pgfsetlinewidth{1.003750pt}%
\definecolor{currentstroke}{rgb}{0.000000,0.000000,0.000000}%
\pgfsetstrokecolor{currentstroke}%
\pgfsetdash{}{0pt}%
\pgfpathmoveto{\pgfqpoint{5.462500in}{3.176830in}}%
\pgfpathlineto{\pgfqpoint{5.537500in}{3.176830in}}%
\pgfusepath{stroke}%
\end{pgfscope}%
\begin{pgfscope}%
\pgfpathrectangle{\pgfqpoint{1.000000in}{0.720000in}}{\pgfqpoint{4.800000in}{4.620000in}}%
\pgfusepath{clip}%
\pgfsetrectcap%
\pgfsetroundjoin%
\pgfsetlinewidth{1.003750pt}%
\definecolor{currentstroke}{rgb}{0.000000,0.000000,0.000000}%
\pgfsetstrokecolor{currentstroke}%
\pgfsetdash{}{0pt}%
\pgfpathmoveto{\pgfqpoint{5.462500in}{3.176830in}}%
\pgfpathlineto{\pgfqpoint{5.537500in}{3.176830in}}%
\pgfusepath{stroke}%
\end{pgfscope}%
\begin{pgfscope}%
\pgfpathrectangle{\pgfqpoint{1.000000in}{0.720000in}}{\pgfqpoint{4.800000in}{4.620000in}}%
\pgfusepath{clip}%
\pgfsetrectcap%
\pgfsetroundjoin%
\pgfsetlinewidth{2.007500pt}%
\definecolor{currentstroke}{rgb}{0.121569,0.466667,0.705882}%
\pgfsetstrokecolor{currentstroke}%
\pgfsetdash{}{0pt}%
\pgfpathmoveto{\pgfqpoint{1.300000in}{1.647459in}}%
\pgfpathlineto{\pgfqpoint{1.600000in}{2.291645in}}%
\pgfpathlineto{\pgfqpoint{1.900000in}{2.593085in}}%
\pgfpathlineto{\pgfqpoint{2.200000in}{2.736961in}}%
\pgfpathlineto{\pgfqpoint{2.500000in}{2.926272in}}%
\pgfpathlineto{\pgfqpoint{2.800000in}{2.987590in}}%
\pgfpathlineto{\pgfqpoint{3.100000in}{3.011750in}}%
\pgfpathlineto{\pgfqpoint{3.400000in}{3.045335in}}%
\pgfpathlineto{\pgfqpoint{3.700000in}{3.066931in}}%
\pgfpathlineto{\pgfqpoint{4.000000in}{3.074232in}}%
\pgfpathlineto{\pgfqpoint{4.300000in}{3.037917in}}%
\pgfpathlineto{\pgfqpoint{4.600000in}{3.536151in}}%
\pgfpathlineto{\pgfqpoint{4.900000in}{3.366848in}}%
\pgfpathlineto{\pgfqpoint{5.200000in}{2.395066in}}%
\pgfpathlineto{\pgfqpoint{5.500000in}{3.176830in}}%
\pgfusepath{stroke}%
\end{pgfscope}%
\begin{pgfscope}%
\pgfpathrectangle{\pgfqpoint{1.000000in}{0.720000in}}{\pgfqpoint{4.800000in}{4.620000in}}%
\pgfusepath{clip}%
\pgfsetbuttcap%
\pgfsetmiterjoin%
\definecolor{currentfill}{rgb}{0.121569,0.466667,0.705882}%
\pgfsetfillcolor{currentfill}%
\pgfsetlinewidth{1.003750pt}%
\definecolor{currentstroke}{rgb}{0.000000,0.000000,0.000000}%
\pgfsetstrokecolor{currentstroke}%
\pgfsetdash{}{0pt}%
\pgfpathmoveto{\pgfqpoint{1.225000in}{1.169035in}}%
\pgfpathlineto{\pgfqpoint{1.375000in}{1.169035in}}%
\pgfpathlineto{\pgfqpoint{1.375000in}{2.103806in}}%
\pgfpathlineto{\pgfqpoint{1.225000in}{2.103806in}}%
\pgfpathlineto{\pgfqpoint{1.225000in}{1.169035in}}%
\pgfpathclose%
\pgfusepath{stroke,fill}%
\end{pgfscope}%
\begin{pgfscope}%
\pgfpathrectangle{\pgfqpoint{1.000000in}{0.720000in}}{\pgfqpoint{4.800000in}{4.620000in}}%
\pgfusepath{clip}%
\pgfsetbuttcap%
\pgfsetmiterjoin%
\definecolor{currentfill}{rgb}{0.121569,0.466667,0.705882}%
\pgfsetfillcolor{currentfill}%
\pgfsetlinewidth{1.003750pt}%
\definecolor{currentstroke}{rgb}{0.000000,0.000000,0.000000}%
\pgfsetstrokecolor{currentstroke}%
\pgfsetdash{}{0pt}%
\pgfpathmoveto{\pgfqpoint{1.525000in}{1.809859in}}%
\pgfpathlineto{\pgfqpoint{1.675000in}{1.809859in}}%
\pgfpathlineto{\pgfqpoint{1.675000in}{2.926791in}}%
\pgfpathlineto{\pgfqpoint{1.525000in}{2.926791in}}%
\pgfpathlineto{\pgfqpoint{1.525000in}{1.809859in}}%
\pgfpathclose%
\pgfusepath{stroke,fill}%
\end{pgfscope}%
\begin{pgfscope}%
\pgfpathrectangle{\pgfqpoint{1.000000in}{0.720000in}}{\pgfqpoint{4.800000in}{4.620000in}}%
\pgfusepath{clip}%
\pgfsetbuttcap%
\pgfsetmiterjoin%
\definecolor{currentfill}{rgb}{0.121569,0.466667,0.705882}%
\pgfsetfillcolor{currentfill}%
\pgfsetlinewidth{1.003750pt}%
\definecolor{currentstroke}{rgb}{0.000000,0.000000,0.000000}%
\pgfsetstrokecolor{currentstroke}%
\pgfsetdash{}{0pt}%
\pgfpathmoveto{\pgfqpoint{1.825000in}{2.114223in}}%
\pgfpathlineto{\pgfqpoint{1.975000in}{2.114223in}}%
\pgfpathlineto{\pgfqpoint{1.975000in}{3.183098in}}%
\pgfpathlineto{\pgfqpoint{1.825000in}{3.183098in}}%
\pgfpathlineto{\pgfqpoint{1.825000in}{2.114223in}}%
\pgfpathclose%
\pgfusepath{stroke,fill}%
\end{pgfscope}%
\begin{pgfscope}%
\pgfpathrectangle{\pgfqpoint{1.000000in}{0.720000in}}{\pgfqpoint{4.800000in}{4.620000in}}%
\pgfusepath{clip}%
\pgfsetbuttcap%
\pgfsetmiterjoin%
\definecolor{currentfill}{rgb}{0.121569,0.466667,0.705882}%
\pgfsetfillcolor{currentfill}%
\pgfsetlinewidth{1.003750pt}%
\definecolor{currentstroke}{rgb}{0.000000,0.000000,0.000000}%
\pgfsetstrokecolor{currentstroke}%
\pgfsetdash{}{0pt}%
\pgfpathmoveto{\pgfqpoint{2.125000in}{2.291483in}}%
\pgfpathlineto{\pgfqpoint{2.275000in}{2.291483in}}%
\pgfpathlineto{\pgfqpoint{2.275000in}{3.309486in}}%
\pgfpathlineto{\pgfqpoint{2.125000in}{3.309486in}}%
\pgfpathlineto{\pgfqpoint{2.125000in}{2.291483in}}%
\pgfpathclose%
\pgfusepath{stroke,fill}%
\end{pgfscope}%
\begin{pgfscope}%
\pgfpathrectangle{\pgfqpoint{1.000000in}{0.720000in}}{\pgfqpoint{4.800000in}{4.620000in}}%
\pgfusepath{clip}%
\pgfsetbuttcap%
\pgfsetmiterjoin%
\definecolor{currentfill}{rgb}{0.121569,0.466667,0.705882}%
\pgfsetfillcolor{currentfill}%
\pgfsetlinewidth{1.003750pt}%
\definecolor{currentstroke}{rgb}{0.000000,0.000000,0.000000}%
\pgfsetstrokecolor{currentstroke}%
\pgfsetdash{}{0pt}%
\pgfpathmoveto{\pgfqpoint{2.425000in}{2.470818in}}%
\pgfpathlineto{\pgfqpoint{2.575000in}{2.470818in}}%
\pgfpathlineto{\pgfqpoint{2.575000in}{3.475065in}}%
\pgfpathlineto{\pgfqpoint{2.425000in}{3.475065in}}%
\pgfpathlineto{\pgfqpoint{2.425000in}{2.470818in}}%
\pgfpathclose%
\pgfusepath{stroke,fill}%
\end{pgfscope}%
\begin{pgfscope}%
\pgfpathrectangle{\pgfqpoint{1.000000in}{0.720000in}}{\pgfqpoint{4.800000in}{4.620000in}}%
\pgfusepath{clip}%
\pgfsetbuttcap%
\pgfsetmiterjoin%
\definecolor{currentfill}{rgb}{0.121569,0.466667,0.705882}%
\pgfsetfillcolor{currentfill}%
\pgfsetlinewidth{1.003750pt}%
\definecolor{currentstroke}{rgb}{0.000000,0.000000,0.000000}%
\pgfsetstrokecolor{currentstroke}%
\pgfsetdash{}{0pt}%
\pgfpathmoveto{\pgfqpoint{2.725000in}{2.527181in}}%
\pgfpathlineto{\pgfqpoint{2.875000in}{2.527181in}}%
\pgfpathlineto{\pgfqpoint{2.875000in}{3.479811in}}%
\pgfpathlineto{\pgfqpoint{2.725000in}{3.479811in}}%
\pgfpathlineto{\pgfqpoint{2.725000in}{2.527181in}}%
\pgfpathclose%
\pgfusepath{stroke,fill}%
\end{pgfscope}%
\begin{pgfscope}%
\pgfpathrectangle{\pgfqpoint{1.000000in}{0.720000in}}{\pgfqpoint{4.800000in}{4.620000in}}%
\pgfusepath{clip}%
\pgfsetbuttcap%
\pgfsetmiterjoin%
\definecolor{currentfill}{rgb}{0.121569,0.466667,0.705882}%
\pgfsetfillcolor{currentfill}%
\pgfsetlinewidth{1.003750pt}%
\definecolor{currentstroke}{rgb}{0.000000,0.000000,0.000000}%
\pgfsetstrokecolor{currentstroke}%
\pgfsetdash{}{0pt}%
\pgfpathmoveto{\pgfqpoint{3.025000in}{2.610220in}}%
\pgfpathlineto{\pgfqpoint{3.175000in}{2.610220in}}%
\pgfpathlineto{\pgfqpoint{3.175000in}{3.544204in}}%
\pgfpathlineto{\pgfqpoint{3.025000in}{3.544204in}}%
\pgfpathlineto{\pgfqpoint{3.025000in}{2.610220in}}%
\pgfpathclose%
\pgfusepath{stroke,fill}%
\end{pgfscope}%
\begin{pgfscope}%
\pgfpathrectangle{\pgfqpoint{1.000000in}{0.720000in}}{\pgfqpoint{4.800000in}{4.620000in}}%
\pgfusepath{clip}%
\pgfsetbuttcap%
\pgfsetmiterjoin%
\definecolor{currentfill}{rgb}{0.121569,0.466667,0.705882}%
\pgfsetfillcolor{currentfill}%
\pgfsetlinewidth{1.003750pt}%
\definecolor{currentstroke}{rgb}{0.000000,0.000000,0.000000}%
\pgfsetstrokecolor{currentstroke}%
\pgfsetdash{}{0pt}%
\pgfpathmoveto{\pgfqpoint{3.325000in}{2.619069in}}%
\pgfpathlineto{\pgfqpoint{3.475000in}{2.619069in}}%
\pgfpathlineto{\pgfqpoint{3.475000in}{3.512155in}}%
\pgfpathlineto{\pgfqpoint{3.325000in}{3.512155in}}%
\pgfpathlineto{\pgfqpoint{3.325000in}{2.619069in}}%
\pgfpathclose%
\pgfusepath{stroke,fill}%
\end{pgfscope}%
\begin{pgfscope}%
\pgfpathrectangle{\pgfqpoint{1.000000in}{0.720000in}}{\pgfqpoint{4.800000in}{4.620000in}}%
\pgfusepath{clip}%
\pgfsetbuttcap%
\pgfsetmiterjoin%
\definecolor{currentfill}{rgb}{0.121569,0.466667,0.705882}%
\pgfsetfillcolor{currentfill}%
\pgfsetlinewidth{1.003750pt}%
\definecolor{currentstroke}{rgb}{0.000000,0.000000,0.000000}%
\pgfsetstrokecolor{currentstroke}%
\pgfsetdash{}{0pt}%
\pgfpathmoveto{\pgfqpoint{3.625000in}{2.681560in}}%
\pgfpathlineto{\pgfqpoint{3.775000in}{2.681560in}}%
\pgfpathlineto{\pgfqpoint{3.775000in}{3.650320in}}%
\pgfpathlineto{\pgfqpoint{3.625000in}{3.650320in}}%
\pgfpathlineto{\pgfqpoint{3.625000in}{2.681560in}}%
\pgfpathclose%
\pgfusepath{stroke,fill}%
\end{pgfscope}%
\begin{pgfscope}%
\pgfpathrectangle{\pgfqpoint{1.000000in}{0.720000in}}{\pgfqpoint{4.800000in}{4.620000in}}%
\pgfusepath{clip}%
\pgfsetbuttcap%
\pgfsetmiterjoin%
\definecolor{currentfill}{rgb}{0.121569,0.466667,0.705882}%
\pgfsetfillcolor{currentfill}%
\pgfsetlinewidth{1.003750pt}%
\definecolor{currentstroke}{rgb}{0.000000,0.000000,0.000000}%
\pgfsetstrokecolor{currentstroke}%
\pgfsetdash{}{0pt}%
\pgfpathmoveto{\pgfqpoint{3.925000in}{2.791569in}}%
\pgfpathlineto{\pgfqpoint{4.075000in}{2.791569in}}%
\pgfpathlineto{\pgfqpoint{4.075000in}{3.432691in}}%
\pgfpathlineto{\pgfqpoint{3.925000in}{3.432691in}}%
\pgfpathlineto{\pgfqpoint{3.925000in}{2.791569in}}%
\pgfpathclose%
\pgfusepath{stroke,fill}%
\end{pgfscope}%
\begin{pgfscope}%
\pgfpathrectangle{\pgfqpoint{1.000000in}{0.720000in}}{\pgfqpoint{4.800000in}{4.620000in}}%
\pgfusepath{clip}%
\pgfsetbuttcap%
\pgfsetmiterjoin%
\definecolor{currentfill}{rgb}{0.121569,0.466667,0.705882}%
\pgfsetfillcolor{currentfill}%
\pgfsetlinewidth{1.003750pt}%
\definecolor{currentstroke}{rgb}{0.000000,0.000000,0.000000}%
\pgfsetstrokecolor{currentstroke}%
\pgfsetdash{}{0pt}%
\pgfpathmoveto{\pgfqpoint{4.225000in}{2.801023in}}%
\pgfpathlineto{\pgfqpoint{4.375000in}{2.801023in}}%
\pgfpathlineto{\pgfqpoint{4.375000in}{3.422301in}}%
\pgfpathlineto{\pgfqpoint{4.225000in}{3.422301in}}%
\pgfpathlineto{\pgfqpoint{4.225000in}{2.801023in}}%
\pgfpathclose%
\pgfusepath{stroke,fill}%
\end{pgfscope}%
\begin{pgfscope}%
\pgfpathrectangle{\pgfqpoint{1.000000in}{0.720000in}}{\pgfqpoint{4.800000in}{4.620000in}}%
\pgfusepath{clip}%
\pgfsetbuttcap%
\pgfsetmiterjoin%
\definecolor{currentfill}{rgb}{0.121569,0.466667,0.705882}%
\pgfsetfillcolor{currentfill}%
\pgfsetlinewidth{1.003750pt}%
\definecolor{currentstroke}{rgb}{0.000000,0.000000,0.000000}%
\pgfsetstrokecolor{currentstroke}%
\pgfsetdash{}{0pt}%
\pgfpathmoveto{\pgfqpoint{4.525000in}{3.399568in}}%
\pgfpathlineto{\pgfqpoint{4.675000in}{3.399568in}}%
\pgfpathlineto{\pgfqpoint{4.675000in}{4.285052in}}%
\pgfpathlineto{\pgfqpoint{4.525000in}{4.285052in}}%
\pgfpathlineto{\pgfqpoint{4.525000in}{3.399568in}}%
\pgfpathclose%
\pgfusepath{stroke,fill}%
\end{pgfscope}%
\begin{pgfscope}%
\pgfpathrectangle{\pgfqpoint{1.000000in}{0.720000in}}{\pgfqpoint{4.800000in}{4.620000in}}%
\pgfusepath{clip}%
\pgfsetbuttcap%
\pgfsetmiterjoin%
\definecolor{currentfill}{rgb}{0.121569,0.466667,0.705882}%
\pgfsetfillcolor{currentfill}%
\pgfsetlinewidth{1.003750pt}%
\definecolor{currentstroke}{rgb}{0.000000,0.000000,0.000000}%
\pgfsetstrokecolor{currentstroke}%
\pgfsetdash{}{0pt}%
\pgfpathmoveto{\pgfqpoint{4.825000in}{3.366848in}}%
\pgfpathlineto{\pgfqpoint{4.975000in}{3.366848in}}%
\pgfpathlineto{\pgfqpoint{4.975000in}{3.366848in}}%
\pgfpathlineto{\pgfqpoint{4.825000in}{3.366848in}}%
\pgfpathlineto{\pgfqpoint{4.825000in}{3.366848in}}%
\pgfpathclose%
\pgfusepath{stroke,fill}%
\end{pgfscope}%
\begin{pgfscope}%
\pgfpathrectangle{\pgfqpoint{1.000000in}{0.720000in}}{\pgfqpoint{4.800000in}{4.620000in}}%
\pgfusepath{clip}%
\pgfsetbuttcap%
\pgfsetmiterjoin%
\definecolor{currentfill}{rgb}{0.121569,0.466667,0.705882}%
\pgfsetfillcolor{currentfill}%
\pgfsetlinewidth{1.003750pt}%
\definecolor{currentstroke}{rgb}{0.000000,0.000000,0.000000}%
\pgfsetstrokecolor{currentstroke}%
\pgfsetdash{}{0pt}%
\pgfpathmoveto{\pgfqpoint{5.125000in}{2.395066in}}%
\pgfpathlineto{\pgfqpoint{5.275000in}{2.395066in}}%
\pgfpathlineto{\pgfqpoint{5.275000in}{2.395066in}}%
\pgfpathlineto{\pgfqpoint{5.125000in}{2.395066in}}%
\pgfpathlineto{\pgfqpoint{5.125000in}{2.395066in}}%
\pgfpathclose%
\pgfusepath{stroke,fill}%
\end{pgfscope}%
\begin{pgfscope}%
\pgfpathrectangle{\pgfqpoint{1.000000in}{0.720000in}}{\pgfqpoint{4.800000in}{4.620000in}}%
\pgfusepath{clip}%
\pgfsetbuttcap%
\pgfsetmiterjoin%
\definecolor{currentfill}{rgb}{0.121569,0.466667,0.705882}%
\pgfsetfillcolor{currentfill}%
\pgfsetlinewidth{1.003750pt}%
\definecolor{currentstroke}{rgb}{0.000000,0.000000,0.000000}%
\pgfsetstrokecolor{currentstroke}%
\pgfsetdash{}{0pt}%
\pgfpathmoveto{\pgfqpoint{5.425000in}{3.176830in}}%
\pgfpathlineto{\pgfqpoint{5.575000in}{3.176830in}}%
\pgfpathlineto{\pgfqpoint{5.575000in}{3.176830in}}%
\pgfpathlineto{\pgfqpoint{5.425000in}{3.176830in}}%
\pgfpathlineto{\pgfqpoint{5.425000in}{3.176830in}}%
\pgfpathclose%
\pgfusepath{stroke,fill}%
\end{pgfscope}%
\begin{pgfscope}%
\pgfpathrectangle{\pgfqpoint{1.000000in}{0.720000in}}{\pgfqpoint{4.800000in}{4.620000in}}%
\pgfusepath{clip}%
\pgfsetrectcap%
\pgfsetroundjoin%
\pgfsetlinewidth{1.003750pt}%
\definecolor{currentstroke}{rgb}{1.000000,0.498039,0.054902}%
\pgfsetstrokecolor{currentstroke}%
\pgfsetdash{}{0pt}%
\pgfpathmoveto{\pgfqpoint{1.225000in}{1.647459in}}%
\pgfpathlineto{\pgfqpoint{1.375000in}{1.647459in}}%
\pgfusepath{stroke}%
\end{pgfscope}%
\begin{pgfscope}%
\pgfpathrectangle{\pgfqpoint{1.000000in}{0.720000in}}{\pgfqpoint{4.800000in}{4.620000in}}%
\pgfusepath{clip}%
\pgfsetrectcap%
\pgfsetroundjoin%
\pgfsetlinewidth{1.003750pt}%
\definecolor{currentstroke}{rgb}{1.000000,0.498039,0.054902}%
\pgfsetstrokecolor{currentstroke}%
\pgfsetdash{}{0pt}%
\pgfpathmoveto{\pgfqpoint{1.525000in}{2.291645in}}%
\pgfpathlineto{\pgfqpoint{1.675000in}{2.291645in}}%
\pgfusepath{stroke}%
\end{pgfscope}%
\begin{pgfscope}%
\pgfpathrectangle{\pgfqpoint{1.000000in}{0.720000in}}{\pgfqpoint{4.800000in}{4.620000in}}%
\pgfusepath{clip}%
\pgfsetrectcap%
\pgfsetroundjoin%
\pgfsetlinewidth{1.003750pt}%
\definecolor{currentstroke}{rgb}{1.000000,0.498039,0.054902}%
\pgfsetstrokecolor{currentstroke}%
\pgfsetdash{}{0pt}%
\pgfpathmoveto{\pgfqpoint{1.825000in}{2.593085in}}%
\pgfpathlineto{\pgfqpoint{1.975000in}{2.593085in}}%
\pgfusepath{stroke}%
\end{pgfscope}%
\begin{pgfscope}%
\pgfpathrectangle{\pgfqpoint{1.000000in}{0.720000in}}{\pgfqpoint{4.800000in}{4.620000in}}%
\pgfusepath{clip}%
\pgfsetrectcap%
\pgfsetroundjoin%
\pgfsetlinewidth{1.003750pt}%
\definecolor{currentstroke}{rgb}{1.000000,0.498039,0.054902}%
\pgfsetstrokecolor{currentstroke}%
\pgfsetdash{}{0pt}%
\pgfpathmoveto{\pgfqpoint{2.125000in}{2.736961in}}%
\pgfpathlineto{\pgfqpoint{2.275000in}{2.736961in}}%
\pgfusepath{stroke}%
\end{pgfscope}%
\begin{pgfscope}%
\pgfpathrectangle{\pgfqpoint{1.000000in}{0.720000in}}{\pgfqpoint{4.800000in}{4.620000in}}%
\pgfusepath{clip}%
\pgfsetrectcap%
\pgfsetroundjoin%
\pgfsetlinewidth{1.003750pt}%
\definecolor{currentstroke}{rgb}{1.000000,0.498039,0.054902}%
\pgfsetstrokecolor{currentstroke}%
\pgfsetdash{}{0pt}%
\pgfpathmoveto{\pgfqpoint{2.425000in}{2.926272in}}%
\pgfpathlineto{\pgfqpoint{2.575000in}{2.926272in}}%
\pgfusepath{stroke}%
\end{pgfscope}%
\begin{pgfscope}%
\pgfpathrectangle{\pgfqpoint{1.000000in}{0.720000in}}{\pgfqpoint{4.800000in}{4.620000in}}%
\pgfusepath{clip}%
\pgfsetrectcap%
\pgfsetroundjoin%
\pgfsetlinewidth{1.003750pt}%
\definecolor{currentstroke}{rgb}{1.000000,0.498039,0.054902}%
\pgfsetstrokecolor{currentstroke}%
\pgfsetdash{}{0pt}%
\pgfpathmoveto{\pgfqpoint{2.725000in}{2.987590in}}%
\pgfpathlineto{\pgfqpoint{2.875000in}{2.987590in}}%
\pgfusepath{stroke}%
\end{pgfscope}%
\begin{pgfscope}%
\pgfpathrectangle{\pgfqpoint{1.000000in}{0.720000in}}{\pgfqpoint{4.800000in}{4.620000in}}%
\pgfusepath{clip}%
\pgfsetrectcap%
\pgfsetroundjoin%
\pgfsetlinewidth{1.003750pt}%
\definecolor{currentstroke}{rgb}{1.000000,0.498039,0.054902}%
\pgfsetstrokecolor{currentstroke}%
\pgfsetdash{}{0pt}%
\pgfpathmoveto{\pgfqpoint{3.025000in}{3.011750in}}%
\pgfpathlineto{\pgfqpoint{3.175000in}{3.011750in}}%
\pgfusepath{stroke}%
\end{pgfscope}%
\begin{pgfscope}%
\pgfpathrectangle{\pgfqpoint{1.000000in}{0.720000in}}{\pgfqpoint{4.800000in}{4.620000in}}%
\pgfusepath{clip}%
\pgfsetrectcap%
\pgfsetroundjoin%
\pgfsetlinewidth{1.003750pt}%
\definecolor{currentstroke}{rgb}{1.000000,0.498039,0.054902}%
\pgfsetstrokecolor{currentstroke}%
\pgfsetdash{}{0pt}%
\pgfpathmoveto{\pgfqpoint{3.325000in}{3.045335in}}%
\pgfpathlineto{\pgfqpoint{3.475000in}{3.045335in}}%
\pgfusepath{stroke}%
\end{pgfscope}%
\begin{pgfscope}%
\pgfpathrectangle{\pgfqpoint{1.000000in}{0.720000in}}{\pgfqpoint{4.800000in}{4.620000in}}%
\pgfusepath{clip}%
\pgfsetrectcap%
\pgfsetroundjoin%
\pgfsetlinewidth{1.003750pt}%
\definecolor{currentstroke}{rgb}{1.000000,0.498039,0.054902}%
\pgfsetstrokecolor{currentstroke}%
\pgfsetdash{}{0pt}%
\pgfpathmoveto{\pgfqpoint{3.625000in}{3.066931in}}%
\pgfpathlineto{\pgfqpoint{3.775000in}{3.066931in}}%
\pgfusepath{stroke}%
\end{pgfscope}%
\begin{pgfscope}%
\pgfpathrectangle{\pgfqpoint{1.000000in}{0.720000in}}{\pgfqpoint{4.800000in}{4.620000in}}%
\pgfusepath{clip}%
\pgfsetrectcap%
\pgfsetroundjoin%
\pgfsetlinewidth{1.003750pt}%
\definecolor{currentstroke}{rgb}{1.000000,0.498039,0.054902}%
\pgfsetstrokecolor{currentstroke}%
\pgfsetdash{}{0pt}%
\pgfpathmoveto{\pgfqpoint{3.925000in}{3.074232in}}%
\pgfpathlineto{\pgfqpoint{4.075000in}{3.074232in}}%
\pgfusepath{stroke}%
\end{pgfscope}%
\begin{pgfscope}%
\pgfpathrectangle{\pgfqpoint{1.000000in}{0.720000in}}{\pgfqpoint{4.800000in}{4.620000in}}%
\pgfusepath{clip}%
\pgfsetrectcap%
\pgfsetroundjoin%
\pgfsetlinewidth{1.003750pt}%
\definecolor{currentstroke}{rgb}{1.000000,0.498039,0.054902}%
\pgfsetstrokecolor{currentstroke}%
\pgfsetdash{}{0pt}%
\pgfpathmoveto{\pgfqpoint{4.225000in}{3.037917in}}%
\pgfpathlineto{\pgfqpoint{4.375000in}{3.037917in}}%
\pgfusepath{stroke}%
\end{pgfscope}%
\begin{pgfscope}%
\pgfpathrectangle{\pgfqpoint{1.000000in}{0.720000in}}{\pgfqpoint{4.800000in}{4.620000in}}%
\pgfusepath{clip}%
\pgfsetrectcap%
\pgfsetroundjoin%
\pgfsetlinewidth{1.003750pt}%
\definecolor{currentstroke}{rgb}{1.000000,0.498039,0.054902}%
\pgfsetstrokecolor{currentstroke}%
\pgfsetdash{}{0pt}%
\pgfpathmoveto{\pgfqpoint{4.525000in}{3.536151in}}%
\pgfpathlineto{\pgfqpoint{4.675000in}{3.536151in}}%
\pgfusepath{stroke}%
\end{pgfscope}%
\begin{pgfscope}%
\pgfpathrectangle{\pgfqpoint{1.000000in}{0.720000in}}{\pgfqpoint{4.800000in}{4.620000in}}%
\pgfusepath{clip}%
\pgfsetrectcap%
\pgfsetroundjoin%
\pgfsetlinewidth{1.003750pt}%
\definecolor{currentstroke}{rgb}{1.000000,0.498039,0.054902}%
\pgfsetstrokecolor{currentstroke}%
\pgfsetdash{}{0pt}%
\pgfpathmoveto{\pgfqpoint{4.825000in}{3.366848in}}%
\pgfpathlineto{\pgfqpoint{4.975000in}{3.366848in}}%
\pgfusepath{stroke}%
\end{pgfscope}%
\begin{pgfscope}%
\pgfpathrectangle{\pgfqpoint{1.000000in}{0.720000in}}{\pgfqpoint{4.800000in}{4.620000in}}%
\pgfusepath{clip}%
\pgfsetrectcap%
\pgfsetroundjoin%
\pgfsetlinewidth{1.003750pt}%
\definecolor{currentstroke}{rgb}{1.000000,0.498039,0.054902}%
\pgfsetstrokecolor{currentstroke}%
\pgfsetdash{}{0pt}%
\pgfpathmoveto{\pgfqpoint{5.125000in}{2.395066in}}%
\pgfpathlineto{\pgfqpoint{5.275000in}{2.395066in}}%
\pgfusepath{stroke}%
\end{pgfscope}%
\begin{pgfscope}%
\pgfpathrectangle{\pgfqpoint{1.000000in}{0.720000in}}{\pgfqpoint{4.800000in}{4.620000in}}%
\pgfusepath{clip}%
\pgfsetrectcap%
\pgfsetroundjoin%
\pgfsetlinewidth{1.003750pt}%
\definecolor{currentstroke}{rgb}{1.000000,0.498039,0.054902}%
\pgfsetstrokecolor{currentstroke}%
\pgfsetdash{}{0pt}%
\pgfpathmoveto{\pgfqpoint{5.425000in}{3.176830in}}%
\pgfpathlineto{\pgfqpoint{5.575000in}{3.176830in}}%
\pgfusepath{stroke}%
\end{pgfscope}%
\begin{pgfscope}%
\pgfsetrectcap%
\pgfsetmiterjoin%
\pgfsetlinewidth{0.803000pt}%
\definecolor{currentstroke}{rgb}{0.000000,0.000000,0.000000}%
\pgfsetstrokecolor{currentstroke}%
\pgfsetdash{}{0pt}%
\pgfpathmoveto{\pgfqpoint{1.000000in}{0.720000in}}%
\pgfpathlineto{\pgfqpoint{1.000000in}{5.340000in}}%
\pgfusepath{stroke}%
\end{pgfscope}%
\begin{pgfscope}%
\pgfsetrectcap%
\pgfsetmiterjoin%
\pgfsetlinewidth{0.803000pt}%
\definecolor{currentstroke}{rgb}{0.000000,0.000000,0.000000}%
\pgfsetstrokecolor{currentstroke}%
\pgfsetdash{}{0pt}%
\pgfpathmoveto{\pgfqpoint{5.800000in}{0.720000in}}%
\pgfpathlineto{\pgfqpoint{5.800000in}{5.340000in}}%
\pgfusepath{stroke}%
\end{pgfscope}%
\begin{pgfscope}%
\pgfsetrectcap%
\pgfsetmiterjoin%
\pgfsetlinewidth{0.803000pt}%
\definecolor{currentstroke}{rgb}{0.000000,0.000000,0.000000}%
\pgfsetstrokecolor{currentstroke}%
\pgfsetdash{}{0pt}%
\pgfpathmoveto{\pgfqpoint{1.000000in}{0.720000in}}%
\pgfpathlineto{\pgfqpoint{5.800000in}{0.720000in}}%
\pgfusepath{stroke}%
\end{pgfscope}%
\begin{pgfscope}%
\pgfsetrectcap%
\pgfsetmiterjoin%
\pgfsetlinewidth{0.803000pt}%
\definecolor{currentstroke}{rgb}{0.000000,0.000000,0.000000}%
\pgfsetstrokecolor{currentstroke}%
\pgfsetdash{}{0pt}%
\pgfpathmoveto{\pgfqpoint{1.000000in}{5.340000in}}%
\pgfpathlineto{\pgfqpoint{5.800000in}{5.340000in}}%
\pgfusepath{stroke}%
\end{pgfscope}%
\begin{pgfscope}%
\pgfsetbuttcap%
\pgfsetmiterjoin%
\definecolor{currentfill}{rgb}{1.000000,1.000000,1.000000}%
\pgfsetfillcolor{currentfill}%
\pgfsetfillopacity{0.800000}%
\pgfsetlinewidth{1.003750pt}%
\definecolor{currentstroke}{rgb}{0.800000,0.800000,0.800000}%
\pgfsetstrokecolor{currentstroke}%
\pgfsetstrokeopacity{0.800000}%
\pgfsetdash{}{0pt}%
\pgfpathmoveto{\pgfqpoint{4.330170in}{0.858889in}}%
\pgfpathlineto{\pgfqpoint{5.605556in}{0.858889in}}%
\pgfpathquadraticcurveto{\pgfqpoint{5.661111in}{0.858889in}}{\pgfqpoint{5.661111in}{0.914444in}}%
\pgfpathlineto{\pgfqpoint{5.661111in}{1.281623in}}%
\pgfpathquadraticcurveto{\pgfqpoint{5.661111in}{1.337179in}}{\pgfqpoint{5.605556in}{1.337179in}}%
\pgfpathlineto{\pgfqpoint{4.330170in}{1.337179in}}%
\pgfpathquadraticcurveto{\pgfqpoint{4.274615in}{1.337179in}}{\pgfqpoint{4.274615in}{1.281623in}}%
\pgfpathlineto{\pgfqpoint{4.274615in}{0.914444in}}%
\pgfpathquadraticcurveto{\pgfqpoint{4.274615in}{0.858889in}}{\pgfqpoint{4.330170in}{0.858889in}}%
\pgfpathclose%
\pgfusepath{stroke,fill}%
\end{pgfscope}%
\begin{pgfscope}%
\pgfsetrectcap%
\pgfsetroundjoin%
\pgfsetlinewidth{2.007500pt}%
\definecolor{currentstroke}{rgb}{0.121569,0.466667,0.705882}%
\pgfsetstrokecolor{currentstroke}%
\pgfsetdash{}{0pt}%
\pgfpathmoveto{\pgfqpoint{4.385726in}{1.123251in}}%
\pgfpathlineto{\pgfqpoint{4.941281in}{1.123251in}}%
\pgfusepath{stroke}%
\end{pgfscope}%
\begin{pgfscope}%
\definecolor{textcolor}{rgb}{0.000000,0.000000,0.000000}%
\pgfsetstrokecolor{textcolor}%
\pgfsetfillcolor{textcolor}%
\pgftext[x=5.163504in,y=1.026029in,left,base]{\color{textcolor}\sffamily\fontsize{20.000000}{24.000000}\selectfont \(\displaystyle D_w\)}%
\end{pgfscope}%
\begin{pgfscope}%
\pgfsetbuttcap%
\pgfsetmiterjoin%
\definecolor{currentfill}{rgb}{1.000000,1.000000,1.000000}%
\pgfsetfillcolor{currentfill}%
\pgfsetlinewidth{0.000000pt}%
\definecolor{currentstroke}{rgb}{0.000000,0.000000,0.000000}%
\pgfsetstrokecolor{currentstroke}%
\pgfsetstrokeopacity{0.000000}%
\pgfsetdash{}{0pt}%
\pgfpathmoveto{\pgfqpoint{5.800000in}{0.720000in}}%
\pgfpathlineto{\pgfqpoint{7.200000in}{0.720000in}}%
\pgfpathlineto{\pgfqpoint{7.200000in}{5.340000in}}%
\pgfpathlineto{\pgfqpoint{5.800000in}{5.340000in}}%
\pgfpathclose%
\pgfusepath{fill}%
\end{pgfscope}%
\begin{pgfscope}%
\pgfpathrectangle{\pgfqpoint{5.800000in}{0.720000in}}{\pgfqpoint{1.400000in}{4.620000in}}%
\pgfusepath{clip}%
\pgfsetbuttcap%
\pgfsetmiterjoin%
\definecolor{currentfill}{rgb}{0.121569,0.466667,0.705882}%
\pgfsetfillcolor{currentfill}%
\pgfsetlinewidth{0.000000pt}%
\definecolor{currentstroke}{rgb}{0.000000,0.000000,0.000000}%
\pgfsetstrokecolor{currentstroke}%
\pgfsetstrokeopacity{0.000000}%
\pgfsetdash{}{0pt}%
\pgfpathmoveto{\pgfqpoint{5.800000in}{0.720000in}}%
\pgfpathlineto{\pgfqpoint{5.905820in}{0.720000in}}%
\pgfpathlineto{\pgfqpoint{5.905820in}{0.837852in}}%
\pgfpathlineto{\pgfqpoint{5.800000in}{0.837852in}}%
\pgfpathclose%
\pgfusepath{fill}%
\end{pgfscope}%
\begin{pgfscope}%
\pgfpathrectangle{\pgfqpoint{5.800000in}{0.720000in}}{\pgfqpoint{1.400000in}{4.620000in}}%
\pgfusepath{clip}%
\pgfsetbuttcap%
\pgfsetmiterjoin%
\definecolor{currentfill}{rgb}{0.121569,0.466667,0.705882}%
\pgfsetfillcolor{currentfill}%
\pgfsetlinewidth{0.000000pt}%
\definecolor{currentstroke}{rgb}{0.000000,0.000000,0.000000}%
\pgfsetstrokecolor{currentstroke}%
\pgfsetstrokeopacity{0.000000}%
\pgfsetdash{}{0pt}%
\pgfpathmoveto{\pgfqpoint{5.800000in}{0.837852in}}%
\pgfpathlineto{\pgfqpoint{5.916618in}{0.837852in}}%
\pgfpathlineto{\pgfqpoint{5.916618in}{0.955704in}}%
\pgfpathlineto{\pgfqpoint{5.800000in}{0.955704in}}%
\pgfpathclose%
\pgfusepath{fill}%
\end{pgfscope}%
\begin{pgfscope}%
\pgfpathrectangle{\pgfqpoint{5.800000in}{0.720000in}}{\pgfqpoint{1.400000in}{4.620000in}}%
\pgfusepath{clip}%
\pgfsetbuttcap%
\pgfsetmiterjoin%
\definecolor{currentfill}{rgb}{0.121569,0.466667,0.705882}%
\pgfsetfillcolor{currentfill}%
\pgfsetlinewidth{0.000000pt}%
\definecolor{currentstroke}{rgb}{0.000000,0.000000,0.000000}%
\pgfsetstrokecolor{currentstroke}%
\pgfsetstrokeopacity{0.000000}%
\pgfsetdash{}{0pt}%
\pgfpathmoveto{\pgfqpoint{5.800000in}{0.955704in}}%
\pgfpathlineto{\pgfqpoint{5.923097in}{0.955704in}}%
\pgfpathlineto{\pgfqpoint{5.923097in}{1.073556in}}%
\pgfpathlineto{\pgfqpoint{5.800000in}{1.073556in}}%
\pgfpathclose%
\pgfusepath{fill}%
\end{pgfscope}%
\begin{pgfscope}%
\pgfpathrectangle{\pgfqpoint{5.800000in}{0.720000in}}{\pgfqpoint{1.400000in}{4.620000in}}%
\pgfusepath{clip}%
\pgfsetbuttcap%
\pgfsetmiterjoin%
\definecolor{currentfill}{rgb}{0.121569,0.466667,0.705882}%
\pgfsetfillcolor{currentfill}%
\pgfsetlinewidth{0.000000pt}%
\definecolor{currentstroke}{rgb}{0.000000,0.000000,0.000000}%
\pgfsetstrokecolor{currentstroke}%
\pgfsetstrokeopacity{0.000000}%
\pgfsetdash{}{0pt}%
\pgfpathmoveto{\pgfqpoint{5.800000in}{1.073556in}}%
\pgfpathlineto{\pgfqpoint{5.953331in}{1.073556in}}%
\pgfpathlineto{\pgfqpoint{5.953331in}{1.191407in}}%
\pgfpathlineto{\pgfqpoint{5.800000in}{1.191407in}}%
\pgfpathclose%
\pgfusepath{fill}%
\end{pgfscope}%
\begin{pgfscope}%
\pgfpathrectangle{\pgfqpoint{5.800000in}{0.720000in}}{\pgfqpoint{1.400000in}{4.620000in}}%
\pgfusepath{clip}%
\pgfsetbuttcap%
\pgfsetmiterjoin%
\definecolor{currentfill}{rgb}{0.121569,0.466667,0.705882}%
\pgfsetfillcolor{currentfill}%
\pgfsetlinewidth{0.000000pt}%
\definecolor{currentstroke}{rgb}{0.000000,0.000000,0.000000}%
\pgfsetstrokecolor{currentstroke}%
\pgfsetstrokeopacity{0.000000}%
\pgfsetdash{}{0pt}%
\pgfpathmoveto{\pgfqpoint{5.800000in}{1.191407in}}%
\pgfpathlineto{\pgfqpoint{6.031077in}{1.191407in}}%
\pgfpathlineto{\pgfqpoint{6.031077in}{1.309259in}}%
\pgfpathlineto{\pgfqpoint{5.800000in}{1.309259in}}%
\pgfpathclose%
\pgfusepath{fill}%
\end{pgfscope}%
\begin{pgfscope}%
\pgfpathrectangle{\pgfqpoint{5.800000in}{0.720000in}}{\pgfqpoint{1.400000in}{4.620000in}}%
\pgfusepath{clip}%
\pgfsetbuttcap%
\pgfsetmiterjoin%
\definecolor{currentfill}{rgb}{0.121569,0.466667,0.705882}%
\pgfsetfillcolor{currentfill}%
\pgfsetlinewidth{0.000000pt}%
\definecolor{currentstroke}{rgb}{0.000000,0.000000,0.000000}%
\pgfsetstrokecolor{currentstroke}%
\pgfsetstrokeopacity{0.000000}%
\pgfsetdash{}{0pt}%
\pgfpathmoveto{\pgfqpoint{5.800000in}{1.309259in}}%
\pgfpathlineto{\pgfqpoint{6.100184in}{1.309259in}}%
\pgfpathlineto{\pgfqpoint{6.100184in}{1.427111in}}%
\pgfpathlineto{\pgfqpoint{5.800000in}{1.427111in}}%
\pgfpathclose%
\pgfusepath{fill}%
\end{pgfscope}%
\begin{pgfscope}%
\pgfpathrectangle{\pgfqpoint{5.800000in}{0.720000in}}{\pgfqpoint{1.400000in}{4.620000in}}%
\pgfusepath{clip}%
\pgfsetbuttcap%
\pgfsetmiterjoin%
\definecolor{currentfill}{rgb}{0.121569,0.466667,0.705882}%
\pgfsetfillcolor{currentfill}%
\pgfsetlinewidth{0.000000pt}%
\definecolor{currentstroke}{rgb}{0.000000,0.000000,0.000000}%
\pgfsetstrokecolor{currentstroke}%
\pgfsetstrokeopacity{0.000000}%
\pgfsetdash{}{0pt}%
\pgfpathmoveto{\pgfqpoint{5.800000in}{1.427111in}}%
\pgfpathlineto{\pgfqpoint{6.210323in}{1.427111in}}%
\pgfpathlineto{\pgfqpoint{6.210323in}{1.544963in}}%
\pgfpathlineto{\pgfqpoint{5.800000in}{1.544963in}}%
\pgfpathclose%
\pgfusepath{fill}%
\end{pgfscope}%
\begin{pgfscope}%
\pgfpathrectangle{\pgfqpoint{5.800000in}{0.720000in}}{\pgfqpoint{1.400000in}{4.620000in}}%
\pgfusepath{clip}%
\pgfsetbuttcap%
\pgfsetmiterjoin%
\definecolor{currentfill}{rgb}{0.121569,0.466667,0.705882}%
\pgfsetfillcolor{currentfill}%
\pgfsetlinewidth{0.000000pt}%
\definecolor{currentstroke}{rgb}{0.000000,0.000000,0.000000}%
\pgfsetstrokecolor{currentstroke}%
\pgfsetstrokeopacity{0.000000}%
\pgfsetdash{}{0pt}%
\pgfpathmoveto{\pgfqpoint{5.800000in}{1.544963in}}%
\pgfpathlineto{\pgfqpoint{6.236238in}{1.544963in}}%
\pgfpathlineto{\pgfqpoint{6.236238in}{1.662815in}}%
\pgfpathlineto{\pgfqpoint{5.800000in}{1.662815in}}%
\pgfpathclose%
\pgfusepath{fill}%
\end{pgfscope}%
\begin{pgfscope}%
\pgfpathrectangle{\pgfqpoint{5.800000in}{0.720000in}}{\pgfqpoint{1.400000in}{4.620000in}}%
\pgfusepath{clip}%
\pgfsetbuttcap%
\pgfsetmiterjoin%
\definecolor{currentfill}{rgb}{0.121569,0.466667,0.705882}%
\pgfsetfillcolor{currentfill}%
\pgfsetlinewidth{0.000000pt}%
\definecolor{currentstroke}{rgb}{0.000000,0.000000,0.000000}%
\pgfsetstrokecolor{currentstroke}%
\pgfsetstrokeopacity{0.000000}%
\pgfsetdash{}{0pt}%
\pgfpathmoveto{\pgfqpoint{5.800000in}{1.662815in}}%
\pgfpathlineto{\pgfqpoint{6.428442in}{1.662815in}}%
\pgfpathlineto{\pgfqpoint{6.428442in}{1.780667in}}%
\pgfpathlineto{\pgfqpoint{5.800000in}{1.780667in}}%
\pgfpathclose%
\pgfusepath{fill}%
\end{pgfscope}%
\begin{pgfscope}%
\pgfpathrectangle{\pgfqpoint{5.800000in}{0.720000in}}{\pgfqpoint{1.400000in}{4.620000in}}%
\pgfusepath{clip}%
\pgfsetbuttcap%
\pgfsetmiterjoin%
\definecolor{currentfill}{rgb}{0.121569,0.466667,0.705882}%
\pgfsetfillcolor{currentfill}%
\pgfsetlinewidth{0.000000pt}%
\definecolor{currentstroke}{rgb}{0.000000,0.000000,0.000000}%
\pgfsetstrokecolor{currentstroke}%
\pgfsetstrokeopacity{0.000000}%
\pgfsetdash{}{0pt}%
\pgfpathmoveto{\pgfqpoint{5.800000in}{1.780667in}}%
\pgfpathlineto{\pgfqpoint{6.536422in}{1.780667in}}%
\pgfpathlineto{\pgfqpoint{6.536422in}{1.898519in}}%
\pgfpathlineto{\pgfqpoint{5.800000in}{1.898519in}}%
\pgfpathclose%
\pgfusepath{fill}%
\end{pgfscope}%
\begin{pgfscope}%
\pgfpathrectangle{\pgfqpoint{5.800000in}{0.720000in}}{\pgfqpoint{1.400000in}{4.620000in}}%
\pgfusepath{clip}%
\pgfsetbuttcap%
\pgfsetmiterjoin%
\definecolor{currentfill}{rgb}{0.121569,0.466667,0.705882}%
\pgfsetfillcolor{currentfill}%
\pgfsetlinewidth{0.000000pt}%
\definecolor{currentstroke}{rgb}{0.000000,0.000000,0.000000}%
\pgfsetstrokecolor{currentstroke}%
\pgfsetstrokeopacity{0.000000}%
\pgfsetdash{}{0pt}%
\pgfpathmoveto{\pgfqpoint{5.800000in}{1.898519in}}%
\pgfpathlineto{\pgfqpoint{6.635763in}{1.898519in}}%
\pgfpathlineto{\pgfqpoint{6.635763in}{2.016371in}}%
\pgfpathlineto{\pgfqpoint{5.800000in}{2.016371in}}%
\pgfpathclose%
\pgfusepath{fill}%
\end{pgfscope}%
\begin{pgfscope}%
\pgfpathrectangle{\pgfqpoint{5.800000in}{0.720000in}}{\pgfqpoint{1.400000in}{4.620000in}}%
\pgfusepath{clip}%
\pgfsetbuttcap%
\pgfsetmiterjoin%
\definecolor{currentfill}{rgb}{0.121569,0.466667,0.705882}%
\pgfsetfillcolor{currentfill}%
\pgfsetlinewidth{0.000000pt}%
\definecolor{currentstroke}{rgb}{0.000000,0.000000,0.000000}%
\pgfsetstrokecolor{currentstroke}%
\pgfsetstrokeopacity{0.000000}%
\pgfsetdash{}{0pt}%
\pgfpathmoveto{\pgfqpoint{5.800000in}{2.016371in}}%
\pgfpathlineto{\pgfqpoint{6.817169in}{2.016371in}}%
\pgfpathlineto{\pgfqpoint{6.817169in}{2.134222in}}%
\pgfpathlineto{\pgfqpoint{5.800000in}{2.134222in}}%
\pgfpathclose%
\pgfusepath{fill}%
\end{pgfscope}%
\begin{pgfscope}%
\pgfpathrectangle{\pgfqpoint{5.800000in}{0.720000in}}{\pgfqpoint{1.400000in}{4.620000in}}%
\pgfusepath{clip}%
\pgfsetbuttcap%
\pgfsetmiterjoin%
\definecolor{currentfill}{rgb}{0.121569,0.466667,0.705882}%
\pgfsetfillcolor{currentfill}%
\pgfsetlinewidth{0.000000pt}%
\definecolor{currentstroke}{rgb}{0.000000,0.000000,0.000000}%
\pgfsetstrokecolor{currentstroke}%
\pgfsetstrokeopacity{0.000000}%
\pgfsetdash{}{0pt}%
\pgfpathmoveto{\pgfqpoint{5.800000in}{2.134222in}}%
\pgfpathlineto{\pgfqpoint{6.838765in}{2.134222in}}%
\pgfpathlineto{\pgfqpoint{6.838765in}{2.252074in}}%
\pgfpathlineto{\pgfqpoint{5.800000in}{2.252074in}}%
\pgfpathclose%
\pgfusepath{fill}%
\end{pgfscope}%
\begin{pgfscope}%
\pgfpathrectangle{\pgfqpoint{5.800000in}{0.720000in}}{\pgfqpoint{1.400000in}{4.620000in}}%
\pgfusepath{clip}%
\pgfsetbuttcap%
\pgfsetmiterjoin%
\definecolor{currentfill}{rgb}{0.121569,0.466667,0.705882}%
\pgfsetfillcolor{currentfill}%
\pgfsetlinewidth{0.000000pt}%
\definecolor{currentstroke}{rgb}{0.000000,0.000000,0.000000}%
\pgfsetstrokecolor{currentstroke}%
\pgfsetstrokeopacity{0.000000}%
\pgfsetdash{}{0pt}%
\pgfpathmoveto{\pgfqpoint{5.800000in}{2.252074in}}%
\pgfpathlineto{\pgfqpoint{7.026649in}{2.252074in}}%
\pgfpathlineto{\pgfqpoint{7.026649in}{2.369926in}}%
\pgfpathlineto{\pgfqpoint{5.800000in}{2.369926in}}%
\pgfpathclose%
\pgfusepath{fill}%
\end{pgfscope}%
\begin{pgfscope}%
\pgfpathrectangle{\pgfqpoint{5.800000in}{0.720000in}}{\pgfqpoint{1.400000in}{4.620000in}}%
\pgfusepath{clip}%
\pgfsetbuttcap%
\pgfsetmiterjoin%
\definecolor{currentfill}{rgb}{0.121569,0.466667,0.705882}%
\pgfsetfillcolor{currentfill}%
\pgfsetlinewidth{0.000000pt}%
\definecolor{currentstroke}{rgb}{0.000000,0.000000,0.000000}%
\pgfsetstrokecolor{currentstroke}%
\pgfsetstrokeopacity{0.000000}%
\pgfsetdash{}{0pt}%
\pgfpathmoveto{\pgfqpoint{5.800000in}{2.369926in}}%
\pgfpathlineto{\pgfqpoint{7.030969in}{2.369926in}}%
\pgfpathlineto{\pgfqpoint{7.030969in}{2.487778in}}%
\pgfpathlineto{\pgfqpoint{5.800000in}{2.487778in}}%
\pgfpathclose%
\pgfusepath{fill}%
\end{pgfscope}%
\begin{pgfscope}%
\pgfpathrectangle{\pgfqpoint{5.800000in}{0.720000in}}{\pgfqpoint{1.400000in}{4.620000in}}%
\pgfusepath{clip}%
\pgfsetbuttcap%
\pgfsetmiterjoin%
\definecolor{currentfill}{rgb}{0.121569,0.466667,0.705882}%
\pgfsetfillcolor{currentfill}%
\pgfsetlinewidth{0.000000pt}%
\definecolor{currentstroke}{rgb}{0.000000,0.000000,0.000000}%
\pgfsetstrokecolor{currentstroke}%
\pgfsetstrokeopacity{0.000000}%
\pgfsetdash{}{0pt}%
\pgfpathmoveto{\pgfqpoint{5.800000in}{2.487778in}}%
\pgfpathlineto{\pgfqpoint{7.033128in}{2.487778in}}%
\pgfpathlineto{\pgfqpoint{7.033128in}{2.605630in}}%
\pgfpathlineto{\pgfqpoint{5.800000in}{2.605630in}}%
\pgfpathclose%
\pgfusepath{fill}%
\end{pgfscope}%
\begin{pgfscope}%
\pgfpathrectangle{\pgfqpoint{5.800000in}{0.720000in}}{\pgfqpoint{1.400000in}{4.620000in}}%
\pgfusepath{clip}%
\pgfsetbuttcap%
\pgfsetmiterjoin%
\definecolor{currentfill}{rgb}{0.121569,0.466667,0.705882}%
\pgfsetfillcolor{currentfill}%
\pgfsetlinewidth{0.000000pt}%
\definecolor{currentstroke}{rgb}{0.000000,0.000000,0.000000}%
\pgfsetstrokecolor{currentstroke}%
\pgfsetstrokeopacity{0.000000}%
\pgfsetdash{}{0pt}%
\pgfpathmoveto{\pgfqpoint{5.800000in}{2.605630in}}%
\pgfpathlineto{\pgfqpoint{7.069841in}{2.605630in}}%
\pgfpathlineto{\pgfqpoint{7.069841in}{2.723482in}}%
\pgfpathlineto{\pgfqpoint{5.800000in}{2.723482in}}%
\pgfpathclose%
\pgfusepath{fill}%
\end{pgfscope}%
\begin{pgfscope}%
\pgfpathrectangle{\pgfqpoint{5.800000in}{0.720000in}}{\pgfqpoint{1.400000in}{4.620000in}}%
\pgfusepath{clip}%
\pgfsetbuttcap%
\pgfsetmiterjoin%
\definecolor{currentfill}{rgb}{0.121569,0.466667,0.705882}%
\pgfsetfillcolor{currentfill}%
\pgfsetlinewidth{0.000000pt}%
\definecolor{currentstroke}{rgb}{0.000000,0.000000,0.000000}%
\pgfsetstrokecolor{currentstroke}%
\pgfsetstrokeopacity{0.000000}%
\pgfsetdash{}{0pt}%
\pgfpathmoveto{\pgfqpoint{5.800000in}{2.723482in}}%
\pgfpathlineto{\pgfqpoint{7.022330in}{2.723482in}}%
\pgfpathlineto{\pgfqpoint{7.022330in}{2.841334in}}%
\pgfpathlineto{\pgfqpoint{5.800000in}{2.841334in}}%
\pgfpathclose%
\pgfusepath{fill}%
\end{pgfscope}%
\begin{pgfscope}%
\pgfpathrectangle{\pgfqpoint{5.800000in}{0.720000in}}{\pgfqpoint{1.400000in}{4.620000in}}%
\pgfusepath{clip}%
\pgfsetbuttcap%
\pgfsetmiterjoin%
\definecolor{currentfill}{rgb}{0.121569,0.466667,0.705882}%
\pgfsetfillcolor{currentfill}%
\pgfsetlinewidth{0.000000pt}%
\definecolor{currentstroke}{rgb}{0.000000,0.000000,0.000000}%
\pgfsetstrokecolor{currentstroke}%
\pgfsetstrokeopacity{0.000000}%
\pgfsetdash{}{0pt}%
\pgfpathmoveto{\pgfqpoint{5.800000in}{2.841334in}}%
\pgfpathlineto{\pgfqpoint{6.955383in}{2.841334in}}%
\pgfpathlineto{\pgfqpoint{6.955383in}{2.959186in}}%
\pgfpathlineto{\pgfqpoint{5.800000in}{2.959186in}}%
\pgfpathclose%
\pgfusepath{fill}%
\end{pgfscope}%
\begin{pgfscope}%
\pgfpathrectangle{\pgfqpoint{5.800000in}{0.720000in}}{\pgfqpoint{1.400000in}{4.620000in}}%
\pgfusepath{clip}%
\pgfsetbuttcap%
\pgfsetmiterjoin%
\definecolor{currentfill}{rgb}{0.121569,0.466667,0.705882}%
\pgfsetfillcolor{currentfill}%
\pgfsetlinewidth{0.000000pt}%
\definecolor{currentstroke}{rgb}{0.000000,0.000000,0.000000}%
\pgfsetstrokecolor{currentstroke}%
\pgfsetstrokeopacity{0.000000}%
\pgfsetdash{}{0pt}%
\pgfpathmoveto{\pgfqpoint{5.800000in}{2.959186in}}%
\pgfpathlineto{\pgfqpoint{6.894914in}{2.959186in}}%
\pgfpathlineto{\pgfqpoint{6.894914in}{3.077037in}}%
\pgfpathlineto{\pgfqpoint{5.800000in}{3.077037in}}%
\pgfpathclose%
\pgfusepath{fill}%
\end{pgfscope}%
\begin{pgfscope}%
\pgfpathrectangle{\pgfqpoint{5.800000in}{0.720000in}}{\pgfqpoint{1.400000in}{4.620000in}}%
\pgfusepath{clip}%
\pgfsetbuttcap%
\pgfsetmiterjoin%
\definecolor{currentfill}{rgb}{0.121569,0.466667,0.705882}%
\pgfsetfillcolor{currentfill}%
\pgfsetlinewidth{0.000000pt}%
\definecolor{currentstroke}{rgb}{0.000000,0.000000,0.000000}%
\pgfsetstrokecolor{currentstroke}%
\pgfsetstrokeopacity{0.000000}%
\pgfsetdash{}{0pt}%
\pgfpathmoveto{\pgfqpoint{5.800000in}{3.077037in}}%
\pgfpathlineto{\pgfqpoint{6.685434in}{3.077037in}}%
\pgfpathlineto{\pgfqpoint{6.685434in}{3.194889in}}%
\pgfpathlineto{\pgfqpoint{5.800000in}{3.194889in}}%
\pgfpathclose%
\pgfusepath{fill}%
\end{pgfscope}%
\begin{pgfscope}%
\pgfpathrectangle{\pgfqpoint{5.800000in}{0.720000in}}{\pgfqpoint{1.400000in}{4.620000in}}%
\pgfusepath{clip}%
\pgfsetbuttcap%
\pgfsetmiterjoin%
\definecolor{currentfill}{rgb}{0.121569,0.466667,0.705882}%
\pgfsetfillcolor{currentfill}%
\pgfsetlinewidth{0.000000pt}%
\definecolor{currentstroke}{rgb}{0.000000,0.000000,0.000000}%
\pgfsetstrokecolor{currentstroke}%
\pgfsetstrokeopacity{0.000000}%
\pgfsetdash{}{0pt}%
\pgfpathmoveto{\pgfqpoint{5.800000in}{3.194889in}}%
\pgfpathlineto{\pgfqpoint{6.717827in}{3.194889in}}%
\pgfpathlineto{\pgfqpoint{6.717827in}{3.312741in}}%
\pgfpathlineto{\pgfqpoint{5.800000in}{3.312741in}}%
\pgfpathclose%
\pgfusepath{fill}%
\end{pgfscope}%
\begin{pgfscope}%
\pgfpathrectangle{\pgfqpoint{5.800000in}{0.720000in}}{\pgfqpoint{1.400000in}{4.620000in}}%
\pgfusepath{clip}%
\pgfsetbuttcap%
\pgfsetmiterjoin%
\definecolor{currentfill}{rgb}{0.121569,0.466667,0.705882}%
\pgfsetfillcolor{currentfill}%
\pgfsetlinewidth{0.000000pt}%
\definecolor{currentstroke}{rgb}{0.000000,0.000000,0.000000}%
\pgfsetstrokecolor{currentstroke}%
\pgfsetstrokeopacity{0.000000}%
\pgfsetdash{}{0pt}%
\pgfpathmoveto{\pgfqpoint{5.800000in}{3.312741in}}%
\pgfpathlineto{\pgfqpoint{6.542900in}{3.312741in}}%
\pgfpathlineto{\pgfqpoint{6.542900in}{3.430593in}}%
\pgfpathlineto{\pgfqpoint{5.800000in}{3.430593in}}%
\pgfpathclose%
\pgfusepath{fill}%
\end{pgfscope}%
\begin{pgfscope}%
\pgfpathrectangle{\pgfqpoint{5.800000in}{0.720000in}}{\pgfqpoint{1.400000in}{4.620000in}}%
\pgfusepath{clip}%
\pgfsetbuttcap%
\pgfsetmiterjoin%
\definecolor{currentfill}{rgb}{0.121569,0.466667,0.705882}%
\pgfsetfillcolor{currentfill}%
\pgfsetlinewidth{0.000000pt}%
\definecolor{currentstroke}{rgb}{0.000000,0.000000,0.000000}%
\pgfsetstrokecolor{currentstroke}%
\pgfsetstrokeopacity{0.000000}%
\pgfsetdash{}{0pt}%
\pgfpathmoveto{\pgfqpoint{5.800000in}{3.430593in}}%
\pgfpathlineto{\pgfqpoint{6.443559in}{3.430593in}}%
\pgfpathlineto{\pgfqpoint{6.443559in}{3.548445in}}%
\pgfpathlineto{\pgfqpoint{5.800000in}{3.548445in}}%
\pgfpathclose%
\pgfusepath{fill}%
\end{pgfscope}%
\begin{pgfscope}%
\pgfpathrectangle{\pgfqpoint{5.800000in}{0.720000in}}{\pgfqpoint{1.400000in}{4.620000in}}%
\pgfusepath{clip}%
\pgfsetbuttcap%
\pgfsetmiterjoin%
\definecolor{currentfill}{rgb}{0.121569,0.466667,0.705882}%
\pgfsetfillcolor{currentfill}%
\pgfsetlinewidth{0.000000pt}%
\definecolor{currentstroke}{rgb}{0.000000,0.000000,0.000000}%
\pgfsetstrokecolor{currentstroke}%
\pgfsetstrokeopacity{0.000000}%
\pgfsetdash{}{0pt}%
\pgfpathmoveto{\pgfqpoint{5.800000in}{3.548445in}}%
\pgfpathlineto{\pgfqpoint{6.378771in}{3.548445in}}%
\pgfpathlineto{\pgfqpoint{6.378771in}{3.666297in}}%
\pgfpathlineto{\pgfqpoint{5.800000in}{3.666297in}}%
\pgfpathclose%
\pgfusepath{fill}%
\end{pgfscope}%
\begin{pgfscope}%
\pgfpathrectangle{\pgfqpoint{5.800000in}{0.720000in}}{\pgfqpoint{1.400000in}{4.620000in}}%
\pgfusepath{clip}%
\pgfsetbuttcap%
\pgfsetmiterjoin%
\definecolor{currentfill}{rgb}{0.121569,0.466667,0.705882}%
\pgfsetfillcolor{currentfill}%
\pgfsetlinewidth{0.000000pt}%
\definecolor{currentstroke}{rgb}{0.000000,0.000000,0.000000}%
\pgfsetstrokecolor{currentstroke}%
\pgfsetstrokeopacity{0.000000}%
\pgfsetdash{}{0pt}%
\pgfpathmoveto{\pgfqpoint{5.800000in}{3.666297in}}%
\pgfpathlineto{\pgfqpoint{6.316143in}{3.666297in}}%
\pgfpathlineto{\pgfqpoint{6.316143in}{3.784149in}}%
\pgfpathlineto{\pgfqpoint{5.800000in}{3.784149in}}%
\pgfpathclose%
\pgfusepath{fill}%
\end{pgfscope}%
\begin{pgfscope}%
\pgfpathrectangle{\pgfqpoint{5.800000in}{0.720000in}}{\pgfqpoint{1.400000in}{4.620000in}}%
\pgfusepath{clip}%
\pgfsetbuttcap%
\pgfsetmiterjoin%
\definecolor{currentfill}{rgb}{0.121569,0.466667,0.705882}%
\pgfsetfillcolor{currentfill}%
\pgfsetlinewidth{0.000000pt}%
\definecolor{currentstroke}{rgb}{0.000000,0.000000,0.000000}%
\pgfsetstrokecolor{currentstroke}%
\pgfsetstrokeopacity{0.000000}%
\pgfsetdash{}{0pt}%
\pgfpathmoveto{\pgfqpoint{5.800000in}{3.784149in}}%
\pgfpathlineto{\pgfqpoint{6.216802in}{3.784149in}}%
\pgfpathlineto{\pgfqpoint{6.216802in}{3.902000in}}%
\pgfpathlineto{\pgfqpoint{5.800000in}{3.902000in}}%
\pgfpathclose%
\pgfusepath{fill}%
\end{pgfscope}%
\begin{pgfscope}%
\pgfpathrectangle{\pgfqpoint{5.800000in}{0.720000in}}{\pgfqpoint{1.400000in}{4.620000in}}%
\pgfusepath{clip}%
\pgfsetbuttcap%
\pgfsetmiterjoin%
\definecolor{currentfill}{rgb}{0.121569,0.466667,0.705882}%
\pgfsetfillcolor{currentfill}%
\pgfsetlinewidth{0.000000pt}%
\definecolor{currentstroke}{rgb}{0.000000,0.000000,0.000000}%
\pgfsetstrokecolor{currentstroke}%
\pgfsetstrokeopacity{0.000000}%
\pgfsetdash{}{0pt}%
\pgfpathmoveto{\pgfqpoint{5.800000in}{3.902000in}}%
\pgfpathlineto{\pgfqpoint{6.136897in}{3.902000in}}%
\pgfpathlineto{\pgfqpoint{6.136897in}{4.019852in}}%
\pgfpathlineto{\pgfqpoint{5.800000in}{4.019852in}}%
\pgfpathclose%
\pgfusepath{fill}%
\end{pgfscope}%
\begin{pgfscope}%
\pgfpathrectangle{\pgfqpoint{5.800000in}{0.720000in}}{\pgfqpoint{1.400000in}{4.620000in}}%
\pgfusepath{clip}%
\pgfsetbuttcap%
\pgfsetmiterjoin%
\definecolor{currentfill}{rgb}{0.121569,0.466667,0.705882}%
\pgfsetfillcolor{currentfill}%
\pgfsetlinewidth{0.000000pt}%
\definecolor{currentstroke}{rgb}{0.000000,0.000000,0.000000}%
\pgfsetstrokecolor{currentstroke}%
\pgfsetstrokeopacity{0.000000}%
\pgfsetdash{}{0pt}%
\pgfpathmoveto{\pgfqpoint{5.800000in}{4.019852in}}%
\pgfpathlineto{\pgfqpoint{6.065630in}{4.019852in}}%
\pgfpathlineto{\pgfqpoint{6.065630in}{4.137704in}}%
\pgfpathlineto{\pgfqpoint{5.800000in}{4.137704in}}%
\pgfpathclose%
\pgfusepath{fill}%
\end{pgfscope}%
\begin{pgfscope}%
\pgfpathrectangle{\pgfqpoint{5.800000in}{0.720000in}}{\pgfqpoint{1.400000in}{4.620000in}}%
\pgfusepath{clip}%
\pgfsetbuttcap%
\pgfsetmiterjoin%
\definecolor{currentfill}{rgb}{0.121569,0.466667,0.705882}%
\pgfsetfillcolor{currentfill}%
\pgfsetlinewidth{0.000000pt}%
\definecolor{currentstroke}{rgb}{0.000000,0.000000,0.000000}%
\pgfsetstrokecolor{currentstroke}%
\pgfsetstrokeopacity{0.000000}%
\pgfsetdash{}{0pt}%
\pgfpathmoveto{\pgfqpoint{5.800000in}{4.137704in}}%
\pgfpathlineto{\pgfqpoint{6.022438in}{4.137704in}}%
\pgfpathlineto{\pgfqpoint{6.022438in}{4.255556in}}%
\pgfpathlineto{\pgfqpoint{5.800000in}{4.255556in}}%
\pgfpathclose%
\pgfusepath{fill}%
\end{pgfscope}%
\begin{pgfscope}%
\pgfpathrectangle{\pgfqpoint{5.800000in}{0.720000in}}{\pgfqpoint{1.400000in}{4.620000in}}%
\pgfusepath{clip}%
\pgfsetbuttcap%
\pgfsetmiterjoin%
\definecolor{currentfill}{rgb}{0.121569,0.466667,0.705882}%
\pgfsetfillcolor{currentfill}%
\pgfsetlinewidth{0.000000pt}%
\definecolor{currentstroke}{rgb}{0.000000,0.000000,0.000000}%
\pgfsetstrokecolor{currentstroke}%
\pgfsetstrokeopacity{0.000000}%
\pgfsetdash{}{0pt}%
\pgfpathmoveto{\pgfqpoint{5.800000in}{4.255556in}}%
\pgfpathlineto{\pgfqpoint{6.024598in}{4.255556in}}%
\pgfpathlineto{\pgfqpoint{6.024598in}{4.373408in}}%
\pgfpathlineto{\pgfqpoint{5.800000in}{4.373408in}}%
\pgfpathclose%
\pgfusepath{fill}%
\end{pgfscope}%
\begin{pgfscope}%
\pgfpathrectangle{\pgfqpoint{5.800000in}{0.720000in}}{\pgfqpoint{1.400000in}{4.620000in}}%
\pgfusepath{clip}%
\pgfsetbuttcap%
\pgfsetmiterjoin%
\definecolor{currentfill}{rgb}{0.121569,0.466667,0.705882}%
\pgfsetfillcolor{currentfill}%
\pgfsetlinewidth{0.000000pt}%
\definecolor{currentstroke}{rgb}{0.000000,0.000000,0.000000}%
\pgfsetstrokecolor{currentstroke}%
\pgfsetstrokeopacity{0.000000}%
\pgfsetdash{}{0pt}%
\pgfpathmoveto{\pgfqpoint{5.800000in}{4.373408in}}%
\pgfpathlineto{\pgfqpoint{5.977087in}{4.373408in}}%
\pgfpathlineto{\pgfqpoint{5.977087in}{4.491260in}}%
\pgfpathlineto{\pgfqpoint{5.800000in}{4.491260in}}%
\pgfpathclose%
\pgfusepath{fill}%
\end{pgfscope}%
\begin{pgfscope}%
\pgfpathrectangle{\pgfqpoint{5.800000in}{0.720000in}}{\pgfqpoint{1.400000in}{4.620000in}}%
\pgfusepath{clip}%
\pgfsetbuttcap%
\pgfsetmiterjoin%
\definecolor{currentfill}{rgb}{0.121569,0.466667,0.705882}%
\pgfsetfillcolor{currentfill}%
\pgfsetlinewidth{0.000000pt}%
\definecolor{currentstroke}{rgb}{0.000000,0.000000,0.000000}%
\pgfsetstrokecolor{currentstroke}%
\pgfsetstrokeopacity{0.000000}%
\pgfsetdash{}{0pt}%
\pgfpathmoveto{\pgfqpoint{5.800000in}{4.491260in}}%
\pgfpathlineto{\pgfqpoint{5.916618in}{4.491260in}}%
\pgfpathlineto{\pgfqpoint{5.916618in}{4.609112in}}%
\pgfpathlineto{\pgfqpoint{5.800000in}{4.609112in}}%
\pgfpathclose%
\pgfusepath{fill}%
\end{pgfscope}%
\begin{pgfscope}%
\pgfpathrectangle{\pgfqpoint{5.800000in}{0.720000in}}{\pgfqpoint{1.400000in}{4.620000in}}%
\pgfusepath{clip}%
\pgfsetbuttcap%
\pgfsetmiterjoin%
\definecolor{currentfill}{rgb}{0.121569,0.466667,0.705882}%
\pgfsetfillcolor{currentfill}%
\pgfsetlinewidth{0.000000pt}%
\definecolor{currentstroke}{rgb}{0.000000,0.000000,0.000000}%
\pgfsetstrokecolor{currentstroke}%
\pgfsetstrokeopacity{0.000000}%
\pgfsetdash{}{0pt}%
\pgfpathmoveto{\pgfqpoint{5.800000in}{4.609112in}}%
\pgfpathlineto{\pgfqpoint{5.931735in}{4.609112in}}%
\pgfpathlineto{\pgfqpoint{5.931735in}{4.726964in}}%
\pgfpathlineto{\pgfqpoint{5.800000in}{4.726964in}}%
\pgfpathclose%
\pgfusepath{fill}%
\end{pgfscope}%
\begin{pgfscope}%
\pgfpathrectangle{\pgfqpoint{5.800000in}{0.720000in}}{\pgfqpoint{1.400000in}{4.620000in}}%
\pgfusepath{clip}%
\pgfsetbuttcap%
\pgfsetmiterjoin%
\definecolor{currentfill}{rgb}{0.121569,0.466667,0.705882}%
\pgfsetfillcolor{currentfill}%
\pgfsetlinewidth{0.000000pt}%
\definecolor{currentstroke}{rgb}{0.000000,0.000000,0.000000}%
\pgfsetstrokecolor{currentstroke}%
\pgfsetstrokeopacity{0.000000}%
\pgfsetdash{}{0pt}%
\pgfpathmoveto{\pgfqpoint{5.800000in}{4.726964in}}%
\pgfpathlineto{\pgfqpoint{5.882065in}{4.726964in}}%
\pgfpathlineto{\pgfqpoint{5.882065in}{4.844815in}}%
\pgfpathlineto{\pgfqpoint{5.800000in}{4.844815in}}%
\pgfpathclose%
\pgfusepath{fill}%
\end{pgfscope}%
\begin{pgfscope}%
\pgfpathrectangle{\pgfqpoint{5.800000in}{0.720000in}}{\pgfqpoint{1.400000in}{4.620000in}}%
\pgfusepath{clip}%
\pgfsetbuttcap%
\pgfsetmiterjoin%
\definecolor{currentfill}{rgb}{0.121569,0.466667,0.705882}%
\pgfsetfillcolor{currentfill}%
\pgfsetlinewidth{0.000000pt}%
\definecolor{currentstroke}{rgb}{0.000000,0.000000,0.000000}%
\pgfsetstrokecolor{currentstroke}%
\pgfsetstrokeopacity{0.000000}%
\pgfsetdash{}{0pt}%
\pgfpathmoveto{\pgfqpoint{5.800000in}{4.844815in}}%
\pgfpathlineto{\pgfqpoint{5.877745in}{4.844815in}}%
\pgfpathlineto{\pgfqpoint{5.877745in}{4.962667in}}%
\pgfpathlineto{\pgfqpoint{5.800000in}{4.962667in}}%
\pgfpathclose%
\pgfusepath{fill}%
\end{pgfscope}%
\begin{pgfscope}%
\pgfpathrectangle{\pgfqpoint{5.800000in}{0.720000in}}{\pgfqpoint{1.400000in}{4.620000in}}%
\pgfusepath{clip}%
\pgfsetbuttcap%
\pgfsetmiterjoin%
\definecolor{currentfill}{rgb}{0.121569,0.466667,0.705882}%
\pgfsetfillcolor{currentfill}%
\pgfsetlinewidth{0.000000pt}%
\definecolor{currentstroke}{rgb}{0.000000,0.000000,0.000000}%
\pgfsetstrokecolor{currentstroke}%
\pgfsetstrokeopacity{0.000000}%
\pgfsetdash{}{0pt}%
\pgfpathmoveto{\pgfqpoint{5.800000in}{4.962667in}}%
\pgfpathlineto{\pgfqpoint{5.869107in}{4.962667in}}%
\pgfpathlineto{\pgfqpoint{5.869107in}{5.080519in}}%
\pgfpathlineto{\pgfqpoint{5.800000in}{5.080519in}}%
\pgfpathclose%
\pgfusepath{fill}%
\end{pgfscope}%
\begin{pgfscope}%
\pgfpathrectangle{\pgfqpoint{5.800000in}{0.720000in}}{\pgfqpoint{1.400000in}{4.620000in}}%
\pgfusepath{clip}%
\pgfsetbuttcap%
\pgfsetmiterjoin%
\definecolor{currentfill}{rgb}{0.121569,0.466667,0.705882}%
\pgfsetfillcolor{currentfill}%
\pgfsetlinewidth{0.000000pt}%
\definecolor{currentstroke}{rgb}{0.000000,0.000000,0.000000}%
\pgfsetstrokecolor{currentstroke}%
\pgfsetstrokeopacity{0.000000}%
\pgfsetdash{}{0pt}%
\pgfpathmoveto{\pgfqpoint{5.800000in}{5.080519in}}%
\pgfpathlineto{\pgfqpoint{5.871267in}{5.080519in}}%
\pgfpathlineto{\pgfqpoint{5.871267in}{5.198371in}}%
\pgfpathlineto{\pgfqpoint{5.800000in}{5.198371in}}%
\pgfpathclose%
\pgfusepath{fill}%
\end{pgfscope}%
\begin{pgfscope}%
\pgfpathrectangle{\pgfqpoint{5.800000in}{0.720000in}}{\pgfqpoint{1.400000in}{4.620000in}}%
\pgfusepath{clip}%
\pgfsetbuttcap%
\pgfsetmiterjoin%
\definecolor{currentfill}{rgb}{0.121569,0.466667,0.705882}%
\pgfsetfillcolor{currentfill}%
\pgfsetlinewidth{0.000000pt}%
\definecolor{currentstroke}{rgb}{0.000000,0.000000,0.000000}%
\pgfsetstrokecolor{currentstroke}%
\pgfsetstrokeopacity{0.000000}%
\pgfsetdash{}{0pt}%
\pgfpathmoveto{\pgfqpoint{5.800000in}{5.198371in}}%
\pgfpathlineto{\pgfqpoint{5.845351in}{5.198371in}}%
\pgfpathlineto{\pgfqpoint{5.845351in}{5.316223in}}%
\pgfpathlineto{\pgfqpoint{5.800000in}{5.316223in}}%
\pgfpathclose%
\pgfusepath{fill}%
\end{pgfscope}%
\begin{pgfscope}%
\pgfpathrectangle{\pgfqpoint{5.800000in}{0.720000in}}{\pgfqpoint{1.400000in}{4.620000in}}%
\pgfusepath{clip}%
\pgfsetbuttcap%
\pgfsetmiterjoin%
\definecolor{currentfill}{rgb}{0.121569,0.466667,0.705882}%
\pgfsetfillcolor{currentfill}%
\pgfsetlinewidth{0.000000pt}%
\definecolor{currentstroke}{rgb}{0.000000,0.000000,0.000000}%
\pgfsetstrokecolor{currentstroke}%
\pgfsetstrokeopacity{0.000000}%
\pgfsetdash{}{0pt}%
\pgfpathmoveto{\pgfqpoint{5.800000in}{5.316223in}}%
\pgfpathlineto{\pgfqpoint{5.834554in}{5.316223in}}%
\pgfpathlineto{\pgfqpoint{5.834554in}{5.434075in}}%
\pgfpathlineto{\pgfqpoint{5.800000in}{5.434075in}}%
\pgfpathclose%
\pgfusepath{fill}%
\end{pgfscope}%
\begin{pgfscope}%
\pgfpathrectangle{\pgfqpoint{5.800000in}{0.720000in}}{\pgfqpoint{1.400000in}{4.620000in}}%
\pgfusepath{clip}%
\pgfsetbuttcap%
\pgfsetmiterjoin%
\definecolor{currentfill}{rgb}{0.121569,0.466667,0.705882}%
\pgfsetfillcolor{currentfill}%
\pgfsetlinewidth{0.000000pt}%
\definecolor{currentstroke}{rgb}{0.000000,0.000000,0.000000}%
\pgfsetstrokecolor{currentstroke}%
\pgfsetstrokeopacity{0.000000}%
\pgfsetdash{}{0pt}%
\pgfpathmoveto{\pgfqpoint{5.800000in}{5.434075in}}%
\pgfpathlineto{\pgfqpoint{5.834554in}{5.434075in}}%
\pgfpathlineto{\pgfqpoint{5.834554in}{5.551927in}}%
\pgfpathlineto{\pgfqpoint{5.800000in}{5.551927in}}%
\pgfpathclose%
\pgfusepath{fill}%
\end{pgfscope}%
\begin{pgfscope}%
\pgfpathrectangle{\pgfqpoint{5.800000in}{0.720000in}}{\pgfqpoint{1.400000in}{4.620000in}}%
\pgfusepath{clip}%
\pgfsetbuttcap%
\pgfsetmiterjoin%
\definecolor{currentfill}{rgb}{0.121569,0.466667,0.705882}%
\pgfsetfillcolor{currentfill}%
\pgfsetlinewidth{0.000000pt}%
\definecolor{currentstroke}{rgb}{0.000000,0.000000,0.000000}%
\pgfsetstrokecolor{currentstroke}%
\pgfsetstrokeopacity{0.000000}%
\pgfsetdash{}{0pt}%
\pgfpathmoveto{\pgfqpoint{5.800000in}{5.551927in}}%
\pgfpathlineto{\pgfqpoint{5.832394in}{5.551927in}}%
\pgfpathlineto{\pgfqpoint{5.832394in}{5.669779in}}%
\pgfpathlineto{\pgfqpoint{5.800000in}{5.669779in}}%
\pgfpathclose%
\pgfusepath{fill}%
\end{pgfscope}%
\begin{pgfscope}%
\pgfpathrectangle{\pgfqpoint{5.800000in}{0.720000in}}{\pgfqpoint{1.400000in}{4.620000in}}%
\pgfusepath{clip}%
\pgfsetbuttcap%
\pgfsetmiterjoin%
\definecolor{currentfill}{rgb}{0.121569,0.466667,0.705882}%
\pgfsetfillcolor{currentfill}%
\pgfsetlinewidth{0.000000pt}%
\definecolor{currentstroke}{rgb}{0.000000,0.000000,0.000000}%
\pgfsetstrokecolor{currentstroke}%
\pgfsetstrokeopacity{0.000000}%
\pgfsetdash{}{0pt}%
\pgfpathmoveto{\pgfqpoint{5.800000in}{5.669779in}}%
\pgfpathlineto{\pgfqpoint{5.817277in}{5.669779in}}%
\pgfpathlineto{\pgfqpoint{5.817277in}{5.787630in}}%
\pgfpathlineto{\pgfqpoint{5.800000in}{5.787630in}}%
\pgfpathclose%
\pgfusepath{fill}%
\end{pgfscope}%
\begin{pgfscope}%
\pgfpathrectangle{\pgfqpoint{5.800000in}{0.720000in}}{\pgfqpoint{1.400000in}{4.620000in}}%
\pgfusepath{clip}%
\pgfsetbuttcap%
\pgfsetmiterjoin%
\definecolor{currentfill}{rgb}{0.121569,0.466667,0.705882}%
\pgfsetfillcolor{currentfill}%
\pgfsetlinewidth{0.000000pt}%
\definecolor{currentstroke}{rgb}{0.000000,0.000000,0.000000}%
\pgfsetstrokecolor{currentstroke}%
\pgfsetstrokeopacity{0.000000}%
\pgfsetdash{}{0pt}%
\pgfpathmoveto{\pgfqpoint{5.800000in}{5.787630in}}%
\pgfpathlineto{\pgfqpoint{5.834554in}{5.787630in}}%
\pgfpathlineto{\pgfqpoint{5.834554in}{5.905482in}}%
\pgfpathlineto{\pgfqpoint{5.800000in}{5.905482in}}%
\pgfpathclose%
\pgfusepath{fill}%
\end{pgfscope}%
\begin{pgfscope}%
\pgfpathrectangle{\pgfqpoint{5.800000in}{0.720000in}}{\pgfqpoint{1.400000in}{4.620000in}}%
\pgfusepath{clip}%
\pgfsetbuttcap%
\pgfsetmiterjoin%
\definecolor{currentfill}{rgb}{0.121569,0.466667,0.705882}%
\pgfsetfillcolor{currentfill}%
\pgfsetlinewidth{0.000000pt}%
\definecolor{currentstroke}{rgb}{0.000000,0.000000,0.000000}%
\pgfsetstrokecolor{currentstroke}%
\pgfsetstrokeopacity{0.000000}%
\pgfsetdash{}{0pt}%
\pgfpathmoveto{\pgfqpoint{5.800000in}{5.905482in}}%
\pgfpathlineto{\pgfqpoint{5.815117in}{5.905482in}}%
\pgfpathlineto{\pgfqpoint{5.815117in}{6.023334in}}%
\pgfpathlineto{\pgfqpoint{5.800000in}{6.023334in}}%
\pgfpathclose%
\pgfusepath{fill}%
\end{pgfscope}%
\begin{pgfscope}%
\pgfpathrectangle{\pgfqpoint{5.800000in}{0.720000in}}{\pgfqpoint{1.400000in}{4.620000in}}%
\pgfusepath{clip}%
\pgfsetbuttcap%
\pgfsetmiterjoin%
\definecolor{currentfill}{rgb}{0.121569,0.466667,0.705882}%
\pgfsetfillcolor{currentfill}%
\pgfsetlinewidth{0.000000pt}%
\definecolor{currentstroke}{rgb}{0.000000,0.000000,0.000000}%
\pgfsetstrokecolor{currentstroke}%
\pgfsetstrokeopacity{0.000000}%
\pgfsetdash{}{0pt}%
\pgfpathmoveto{\pgfqpoint{5.800000in}{6.023334in}}%
\pgfpathlineto{\pgfqpoint{5.806479in}{6.023334in}}%
\pgfpathlineto{\pgfqpoint{5.806479in}{6.141186in}}%
\pgfpathlineto{\pgfqpoint{5.800000in}{6.141186in}}%
\pgfpathclose%
\pgfusepath{fill}%
\end{pgfscope}%
\begin{pgfscope}%
\pgfpathrectangle{\pgfqpoint{5.800000in}{0.720000in}}{\pgfqpoint{1.400000in}{4.620000in}}%
\pgfusepath{clip}%
\pgfsetbuttcap%
\pgfsetmiterjoin%
\definecolor{currentfill}{rgb}{0.121569,0.466667,0.705882}%
\pgfsetfillcolor{currentfill}%
\pgfsetlinewidth{0.000000pt}%
\definecolor{currentstroke}{rgb}{0.000000,0.000000,0.000000}%
\pgfsetstrokecolor{currentstroke}%
\pgfsetstrokeopacity{0.000000}%
\pgfsetdash{}{0pt}%
\pgfpathmoveto{\pgfqpoint{5.800000in}{6.141186in}}%
\pgfpathlineto{\pgfqpoint{5.832394in}{6.141186in}}%
\pgfpathlineto{\pgfqpoint{5.832394in}{6.259038in}}%
\pgfpathlineto{\pgfqpoint{5.800000in}{6.259038in}}%
\pgfpathclose%
\pgfusepath{fill}%
\end{pgfscope}%
\begin{pgfscope}%
\pgfpathrectangle{\pgfqpoint{5.800000in}{0.720000in}}{\pgfqpoint{1.400000in}{4.620000in}}%
\pgfusepath{clip}%
\pgfsetbuttcap%
\pgfsetmiterjoin%
\definecolor{currentfill}{rgb}{0.121569,0.466667,0.705882}%
\pgfsetfillcolor{currentfill}%
\pgfsetlinewidth{0.000000pt}%
\definecolor{currentstroke}{rgb}{0.000000,0.000000,0.000000}%
\pgfsetstrokecolor{currentstroke}%
\pgfsetstrokeopacity{0.000000}%
\pgfsetdash{}{0pt}%
\pgfpathmoveto{\pgfqpoint{5.800000in}{6.259038in}}%
\pgfpathlineto{\pgfqpoint{5.815117in}{6.259038in}}%
\pgfpathlineto{\pgfqpoint{5.815117in}{6.376890in}}%
\pgfpathlineto{\pgfqpoint{5.800000in}{6.376890in}}%
\pgfpathclose%
\pgfusepath{fill}%
\end{pgfscope}%
\begin{pgfscope}%
\pgfpathrectangle{\pgfqpoint{5.800000in}{0.720000in}}{\pgfqpoint{1.400000in}{4.620000in}}%
\pgfusepath{clip}%
\pgfsetbuttcap%
\pgfsetmiterjoin%
\definecolor{currentfill}{rgb}{0.121569,0.466667,0.705882}%
\pgfsetfillcolor{currentfill}%
\pgfsetlinewidth{0.000000pt}%
\definecolor{currentstroke}{rgb}{0.000000,0.000000,0.000000}%
\pgfsetstrokecolor{currentstroke}%
\pgfsetstrokeopacity{0.000000}%
\pgfsetdash{}{0pt}%
\pgfpathmoveto{\pgfqpoint{5.800000in}{6.376890in}}%
\pgfpathlineto{\pgfqpoint{5.806479in}{6.376890in}}%
\pgfpathlineto{\pgfqpoint{5.806479in}{6.494742in}}%
\pgfpathlineto{\pgfqpoint{5.800000in}{6.494742in}}%
\pgfpathclose%
\pgfusepath{fill}%
\end{pgfscope}%
\begin{pgfscope}%
\pgfpathrectangle{\pgfqpoint{5.800000in}{0.720000in}}{\pgfqpoint{1.400000in}{4.620000in}}%
\pgfusepath{clip}%
\pgfsetbuttcap%
\pgfsetmiterjoin%
\definecolor{currentfill}{rgb}{0.121569,0.466667,0.705882}%
\pgfsetfillcolor{currentfill}%
\pgfsetlinewidth{0.000000pt}%
\definecolor{currentstroke}{rgb}{0.000000,0.000000,0.000000}%
\pgfsetstrokecolor{currentstroke}%
\pgfsetstrokeopacity{0.000000}%
\pgfsetdash{}{0pt}%
\pgfpathmoveto{\pgfqpoint{5.800000in}{6.494742in}}%
\pgfpathlineto{\pgfqpoint{5.812958in}{6.494742in}}%
\pgfpathlineto{\pgfqpoint{5.812958in}{6.612594in}}%
\pgfpathlineto{\pgfqpoint{5.800000in}{6.612594in}}%
\pgfpathclose%
\pgfusepath{fill}%
\end{pgfscope}%
\begin{pgfscope}%
\pgfpathrectangle{\pgfqpoint{5.800000in}{0.720000in}}{\pgfqpoint{1.400000in}{4.620000in}}%
\pgfusepath{clip}%
\pgfsetbuttcap%
\pgfsetmiterjoin%
\definecolor{currentfill}{rgb}{0.121569,0.466667,0.705882}%
\pgfsetfillcolor{currentfill}%
\pgfsetlinewidth{0.000000pt}%
\definecolor{currentstroke}{rgb}{0.000000,0.000000,0.000000}%
\pgfsetstrokecolor{currentstroke}%
\pgfsetstrokeopacity{0.000000}%
\pgfsetdash{}{0pt}%
\pgfpathmoveto{\pgfqpoint{5.800000in}{6.612594in}}%
\pgfpathlineto{\pgfqpoint{5.819436in}{6.612594in}}%
\pgfpathlineto{\pgfqpoint{5.819436in}{6.730445in}}%
\pgfpathlineto{\pgfqpoint{5.800000in}{6.730445in}}%
\pgfpathclose%
\pgfusepath{fill}%
\end{pgfscope}%
\begin{pgfscope}%
\pgfpathrectangle{\pgfqpoint{5.800000in}{0.720000in}}{\pgfqpoint{1.400000in}{4.620000in}}%
\pgfusepath{clip}%
\pgfsetbuttcap%
\pgfsetmiterjoin%
\definecolor{currentfill}{rgb}{0.121569,0.466667,0.705882}%
\pgfsetfillcolor{currentfill}%
\pgfsetlinewidth{0.000000pt}%
\definecolor{currentstroke}{rgb}{0.000000,0.000000,0.000000}%
\pgfsetstrokecolor{currentstroke}%
\pgfsetstrokeopacity{0.000000}%
\pgfsetdash{}{0pt}%
\pgfpathmoveto{\pgfqpoint{5.800000in}{6.730445in}}%
\pgfpathlineto{\pgfqpoint{5.808638in}{6.730445in}}%
\pgfpathlineto{\pgfqpoint{5.808638in}{6.848297in}}%
\pgfpathlineto{\pgfqpoint{5.800000in}{6.848297in}}%
\pgfpathclose%
\pgfusepath{fill}%
\end{pgfscope}%
\begin{pgfscope}%
\pgfpathrectangle{\pgfqpoint{5.800000in}{0.720000in}}{\pgfqpoint{1.400000in}{4.620000in}}%
\pgfusepath{clip}%
\pgfsetbuttcap%
\pgfsetmiterjoin%
\definecolor{currentfill}{rgb}{0.121569,0.466667,0.705882}%
\pgfsetfillcolor{currentfill}%
\pgfsetlinewidth{0.000000pt}%
\definecolor{currentstroke}{rgb}{0.000000,0.000000,0.000000}%
\pgfsetstrokecolor{currentstroke}%
\pgfsetstrokeopacity{0.000000}%
\pgfsetdash{}{0pt}%
\pgfpathmoveto{\pgfqpoint{5.800000in}{6.848297in}}%
\pgfpathlineto{\pgfqpoint{5.804319in}{6.848297in}}%
\pgfpathlineto{\pgfqpoint{5.804319in}{6.966149in}}%
\pgfpathlineto{\pgfqpoint{5.800000in}{6.966149in}}%
\pgfpathclose%
\pgfusepath{fill}%
\end{pgfscope}%
\begin{pgfscope}%
\pgfpathrectangle{\pgfqpoint{5.800000in}{0.720000in}}{\pgfqpoint{1.400000in}{4.620000in}}%
\pgfusepath{clip}%
\pgfsetbuttcap%
\pgfsetmiterjoin%
\definecolor{currentfill}{rgb}{0.121569,0.466667,0.705882}%
\pgfsetfillcolor{currentfill}%
\pgfsetlinewidth{0.000000pt}%
\definecolor{currentstroke}{rgb}{0.000000,0.000000,0.000000}%
\pgfsetstrokecolor{currentstroke}%
\pgfsetstrokeopacity{0.000000}%
\pgfsetdash{}{0pt}%
\pgfpathmoveto{\pgfqpoint{5.800000in}{6.966149in}}%
\pgfpathlineto{\pgfqpoint{5.808638in}{6.966149in}}%
\pgfpathlineto{\pgfqpoint{5.808638in}{7.084001in}}%
\pgfpathlineto{\pgfqpoint{5.800000in}{7.084001in}}%
\pgfpathclose%
\pgfusepath{fill}%
\end{pgfscope}%
\begin{pgfscope}%
\pgfpathrectangle{\pgfqpoint{5.800000in}{0.720000in}}{\pgfqpoint{1.400000in}{4.620000in}}%
\pgfusepath{clip}%
\pgfsetbuttcap%
\pgfsetmiterjoin%
\definecolor{currentfill}{rgb}{0.121569,0.466667,0.705882}%
\pgfsetfillcolor{currentfill}%
\pgfsetlinewidth{0.000000pt}%
\definecolor{currentstroke}{rgb}{0.000000,0.000000,0.000000}%
\pgfsetstrokecolor{currentstroke}%
\pgfsetstrokeopacity{0.000000}%
\pgfsetdash{}{0pt}%
\pgfpathmoveto{\pgfqpoint{5.800000in}{7.084001in}}%
\pgfpathlineto{\pgfqpoint{5.802160in}{7.084001in}}%
\pgfpathlineto{\pgfqpoint{5.802160in}{7.201853in}}%
\pgfpathlineto{\pgfqpoint{5.800000in}{7.201853in}}%
\pgfpathclose%
\pgfusepath{fill}%
\end{pgfscope}%
\begin{pgfscope}%
\pgfpathrectangle{\pgfqpoint{5.800000in}{0.720000in}}{\pgfqpoint{1.400000in}{4.620000in}}%
\pgfusepath{clip}%
\pgfsetbuttcap%
\pgfsetmiterjoin%
\definecolor{currentfill}{rgb}{0.121569,0.466667,0.705882}%
\pgfsetfillcolor{currentfill}%
\pgfsetlinewidth{0.000000pt}%
\definecolor{currentstroke}{rgb}{0.000000,0.000000,0.000000}%
\pgfsetstrokecolor{currentstroke}%
\pgfsetstrokeopacity{0.000000}%
\pgfsetdash{}{0pt}%
\pgfpathmoveto{\pgfqpoint{5.800000in}{7.201853in}}%
\pgfpathlineto{\pgfqpoint{5.808638in}{7.201853in}}%
\pgfpathlineto{\pgfqpoint{5.808638in}{7.319705in}}%
\pgfpathlineto{\pgfqpoint{5.800000in}{7.319705in}}%
\pgfpathclose%
\pgfusepath{fill}%
\end{pgfscope}%
\begin{pgfscope}%
\pgfpathrectangle{\pgfqpoint{5.800000in}{0.720000in}}{\pgfqpoint{1.400000in}{4.620000in}}%
\pgfusepath{clip}%
\pgfsetbuttcap%
\pgfsetmiterjoin%
\definecolor{currentfill}{rgb}{0.121569,0.466667,0.705882}%
\pgfsetfillcolor{currentfill}%
\pgfsetlinewidth{0.000000pt}%
\definecolor{currentstroke}{rgb}{0.000000,0.000000,0.000000}%
\pgfsetstrokecolor{currentstroke}%
\pgfsetstrokeopacity{0.000000}%
\pgfsetdash{}{0pt}%
\pgfpathmoveto{\pgfqpoint{5.800000in}{7.319705in}}%
\pgfpathlineto{\pgfqpoint{5.806479in}{7.319705in}}%
\pgfpathlineto{\pgfqpoint{5.806479in}{7.437557in}}%
\pgfpathlineto{\pgfqpoint{5.800000in}{7.437557in}}%
\pgfpathclose%
\pgfusepath{fill}%
\end{pgfscope}%
\begin{pgfscope}%
\pgfpathrectangle{\pgfqpoint{5.800000in}{0.720000in}}{\pgfqpoint{1.400000in}{4.620000in}}%
\pgfusepath{clip}%
\pgfsetbuttcap%
\pgfsetmiterjoin%
\definecolor{currentfill}{rgb}{0.121569,0.466667,0.705882}%
\pgfsetfillcolor{currentfill}%
\pgfsetlinewidth{0.000000pt}%
\definecolor{currentstroke}{rgb}{0.000000,0.000000,0.000000}%
\pgfsetstrokecolor{currentstroke}%
\pgfsetstrokeopacity{0.000000}%
\pgfsetdash{}{0pt}%
\pgfpathmoveto{\pgfqpoint{5.800000in}{7.437557in}}%
\pgfpathlineto{\pgfqpoint{5.802160in}{7.437557in}}%
\pgfpathlineto{\pgfqpoint{5.802160in}{7.555408in}}%
\pgfpathlineto{\pgfqpoint{5.800000in}{7.555408in}}%
\pgfpathclose%
\pgfusepath{fill}%
\end{pgfscope}%
\begin{pgfscope}%
\pgfpathrectangle{\pgfqpoint{5.800000in}{0.720000in}}{\pgfqpoint{1.400000in}{4.620000in}}%
\pgfusepath{clip}%
\pgfsetbuttcap%
\pgfsetmiterjoin%
\definecolor{currentfill}{rgb}{0.121569,0.466667,0.705882}%
\pgfsetfillcolor{currentfill}%
\pgfsetlinewidth{0.000000pt}%
\definecolor{currentstroke}{rgb}{0.000000,0.000000,0.000000}%
\pgfsetstrokecolor{currentstroke}%
\pgfsetstrokeopacity{0.000000}%
\pgfsetdash{}{0pt}%
\pgfpathmoveto{\pgfqpoint{5.800000in}{7.555408in}}%
\pgfpathlineto{\pgfqpoint{5.804319in}{7.555408in}}%
\pgfpathlineto{\pgfqpoint{5.804319in}{7.673260in}}%
\pgfpathlineto{\pgfqpoint{5.800000in}{7.673260in}}%
\pgfpathclose%
\pgfusepath{fill}%
\end{pgfscope}%
\begin{pgfscope}%
\pgfpathrectangle{\pgfqpoint{5.800000in}{0.720000in}}{\pgfqpoint{1.400000in}{4.620000in}}%
\pgfusepath{clip}%
\pgfsetbuttcap%
\pgfsetmiterjoin%
\definecolor{currentfill}{rgb}{0.121569,0.466667,0.705882}%
\pgfsetfillcolor{currentfill}%
\pgfsetlinewidth{0.000000pt}%
\definecolor{currentstroke}{rgb}{0.000000,0.000000,0.000000}%
\pgfsetstrokecolor{currentstroke}%
\pgfsetstrokeopacity{0.000000}%
\pgfsetdash{}{0pt}%
\pgfpathmoveto{\pgfqpoint{5.800000in}{7.673260in}}%
\pgfpathlineto{\pgfqpoint{5.802160in}{7.673260in}}%
\pgfpathlineto{\pgfqpoint{5.802160in}{7.791112in}}%
\pgfpathlineto{\pgfqpoint{5.800000in}{7.791112in}}%
\pgfpathclose%
\pgfusepath{fill}%
\end{pgfscope}%
\begin{pgfscope}%
\pgfpathrectangle{\pgfqpoint{5.800000in}{0.720000in}}{\pgfqpoint{1.400000in}{4.620000in}}%
\pgfusepath{clip}%
\pgfsetbuttcap%
\pgfsetmiterjoin%
\definecolor{currentfill}{rgb}{0.121569,0.466667,0.705882}%
\pgfsetfillcolor{currentfill}%
\pgfsetlinewidth{0.000000pt}%
\definecolor{currentstroke}{rgb}{0.000000,0.000000,0.000000}%
\pgfsetstrokecolor{currentstroke}%
\pgfsetstrokeopacity{0.000000}%
\pgfsetdash{}{0pt}%
\pgfpathmoveto{\pgfqpoint{5.800000in}{7.791112in}}%
\pgfpathlineto{\pgfqpoint{5.806479in}{7.791112in}}%
\pgfpathlineto{\pgfqpoint{5.806479in}{7.908964in}}%
\pgfpathlineto{\pgfqpoint{5.800000in}{7.908964in}}%
\pgfpathclose%
\pgfusepath{fill}%
\end{pgfscope}%
\begin{pgfscope}%
\pgfpathrectangle{\pgfqpoint{5.800000in}{0.720000in}}{\pgfqpoint{1.400000in}{4.620000in}}%
\pgfusepath{clip}%
\pgfsetbuttcap%
\pgfsetmiterjoin%
\definecolor{currentfill}{rgb}{0.121569,0.466667,0.705882}%
\pgfsetfillcolor{currentfill}%
\pgfsetlinewidth{0.000000pt}%
\definecolor{currentstroke}{rgb}{0.000000,0.000000,0.000000}%
\pgfsetstrokecolor{currentstroke}%
\pgfsetstrokeopacity{0.000000}%
\pgfsetdash{}{0pt}%
\pgfpathmoveto{\pgfqpoint{5.800000in}{7.908964in}}%
\pgfpathlineto{\pgfqpoint{5.800000in}{7.908964in}}%
\pgfpathlineto{\pgfqpoint{5.800000in}{8.026816in}}%
\pgfpathlineto{\pgfqpoint{5.800000in}{8.026816in}}%
\pgfpathclose%
\pgfusepath{fill}%
\end{pgfscope}%
\begin{pgfscope}%
\pgfpathrectangle{\pgfqpoint{5.800000in}{0.720000in}}{\pgfqpoint{1.400000in}{4.620000in}}%
\pgfusepath{clip}%
\pgfsetbuttcap%
\pgfsetmiterjoin%
\definecolor{currentfill}{rgb}{0.121569,0.466667,0.705882}%
\pgfsetfillcolor{currentfill}%
\pgfsetlinewidth{0.000000pt}%
\definecolor{currentstroke}{rgb}{0.000000,0.000000,0.000000}%
\pgfsetstrokecolor{currentstroke}%
\pgfsetstrokeopacity{0.000000}%
\pgfsetdash{}{0pt}%
\pgfpathmoveto{\pgfqpoint{5.800000in}{8.026816in}}%
\pgfpathlineto{\pgfqpoint{5.800000in}{8.026816in}}%
\pgfpathlineto{\pgfqpoint{5.800000in}{8.144668in}}%
\pgfpathlineto{\pgfqpoint{5.800000in}{8.144668in}}%
\pgfpathclose%
\pgfusepath{fill}%
\end{pgfscope}%
\begin{pgfscope}%
\pgfpathrectangle{\pgfqpoint{5.800000in}{0.720000in}}{\pgfqpoint{1.400000in}{4.620000in}}%
\pgfusepath{clip}%
\pgfsetbuttcap%
\pgfsetmiterjoin%
\definecolor{currentfill}{rgb}{0.121569,0.466667,0.705882}%
\pgfsetfillcolor{currentfill}%
\pgfsetlinewidth{0.000000pt}%
\definecolor{currentstroke}{rgb}{0.000000,0.000000,0.000000}%
\pgfsetstrokecolor{currentstroke}%
\pgfsetstrokeopacity{0.000000}%
\pgfsetdash{}{0pt}%
\pgfpathmoveto{\pgfqpoint{5.800000in}{8.144668in}}%
\pgfpathlineto{\pgfqpoint{5.802160in}{8.144668in}}%
\pgfpathlineto{\pgfqpoint{5.802160in}{8.262520in}}%
\pgfpathlineto{\pgfqpoint{5.800000in}{8.262520in}}%
\pgfpathclose%
\pgfusepath{fill}%
\end{pgfscope}%
\begin{pgfscope}%
\pgfpathrectangle{\pgfqpoint{5.800000in}{0.720000in}}{\pgfqpoint{1.400000in}{4.620000in}}%
\pgfusepath{clip}%
\pgfsetbuttcap%
\pgfsetmiterjoin%
\definecolor{currentfill}{rgb}{0.121569,0.466667,0.705882}%
\pgfsetfillcolor{currentfill}%
\pgfsetlinewidth{0.000000pt}%
\definecolor{currentstroke}{rgb}{0.000000,0.000000,0.000000}%
\pgfsetstrokecolor{currentstroke}%
\pgfsetstrokeopacity{0.000000}%
\pgfsetdash{}{0pt}%
\pgfpathmoveto{\pgfqpoint{5.800000in}{8.262520in}}%
\pgfpathlineto{\pgfqpoint{5.802160in}{8.262520in}}%
\pgfpathlineto{\pgfqpoint{5.802160in}{8.380372in}}%
\pgfpathlineto{\pgfqpoint{5.800000in}{8.380372in}}%
\pgfpathclose%
\pgfusepath{fill}%
\end{pgfscope}%
\begin{pgfscope}%
\pgfpathrectangle{\pgfqpoint{5.800000in}{0.720000in}}{\pgfqpoint{1.400000in}{4.620000in}}%
\pgfusepath{clip}%
\pgfsetbuttcap%
\pgfsetmiterjoin%
\definecolor{currentfill}{rgb}{0.121569,0.466667,0.705882}%
\pgfsetfillcolor{currentfill}%
\pgfsetlinewidth{0.000000pt}%
\definecolor{currentstroke}{rgb}{0.000000,0.000000,0.000000}%
\pgfsetstrokecolor{currentstroke}%
\pgfsetstrokeopacity{0.000000}%
\pgfsetdash{}{0pt}%
\pgfpathmoveto{\pgfqpoint{5.800000in}{8.380372in}}%
\pgfpathlineto{\pgfqpoint{5.804319in}{8.380372in}}%
\pgfpathlineto{\pgfqpoint{5.804319in}{8.498223in}}%
\pgfpathlineto{\pgfqpoint{5.800000in}{8.498223in}}%
\pgfpathclose%
\pgfusepath{fill}%
\end{pgfscope}%
\begin{pgfscope}%
\pgfpathrectangle{\pgfqpoint{5.800000in}{0.720000in}}{\pgfqpoint{1.400000in}{4.620000in}}%
\pgfusepath{clip}%
\pgfsetbuttcap%
\pgfsetmiterjoin%
\definecolor{currentfill}{rgb}{0.121569,0.466667,0.705882}%
\pgfsetfillcolor{currentfill}%
\pgfsetlinewidth{0.000000pt}%
\definecolor{currentstroke}{rgb}{0.000000,0.000000,0.000000}%
\pgfsetstrokecolor{currentstroke}%
\pgfsetstrokeopacity{0.000000}%
\pgfsetdash{}{0pt}%
\pgfpathmoveto{\pgfqpoint{5.800000in}{8.498223in}}%
\pgfpathlineto{\pgfqpoint{5.808638in}{8.498223in}}%
\pgfpathlineto{\pgfqpoint{5.808638in}{8.616075in}}%
\pgfpathlineto{\pgfqpoint{5.800000in}{8.616075in}}%
\pgfpathclose%
\pgfusepath{fill}%
\end{pgfscope}%
\begin{pgfscope}%
\pgfpathrectangle{\pgfqpoint{5.800000in}{0.720000in}}{\pgfqpoint{1.400000in}{4.620000in}}%
\pgfusepath{clip}%
\pgfsetbuttcap%
\pgfsetmiterjoin%
\definecolor{currentfill}{rgb}{0.121569,0.466667,0.705882}%
\pgfsetfillcolor{currentfill}%
\pgfsetlinewidth{0.000000pt}%
\definecolor{currentstroke}{rgb}{0.000000,0.000000,0.000000}%
\pgfsetstrokecolor{currentstroke}%
\pgfsetstrokeopacity{0.000000}%
\pgfsetdash{}{0pt}%
\pgfpathmoveto{\pgfqpoint{5.800000in}{8.616075in}}%
\pgfpathlineto{\pgfqpoint{5.800000in}{8.616075in}}%
\pgfpathlineto{\pgfqpoint{5.800000in}{8.733927in}}%
\pgfpathlineto{\pgfqpoint{5.800000in}{8.733927in}}%
\pgfpathclose%
\pgfusepath{fill}%
\end{pgfscope}%
\begin{pgfscope}%
\pgfpathrectangle{\pgfqpoint{5.800000in}{0.720000in}}{\pgfqpoint{1.400000in}{4.620000in}}%
\pgfusepath{clip}%
\pgfsetbuttcap%
\pgfsetmiterjoin%
\definecolor{currentfill}{rgb}{0.121569,0.466667,0.705882}%
\pgfsetfillcolor{currentfill}%
\pgfsetlinewidth{0.000000pt}%
\definecolor{currentstroke}{rgb}{0.000000,0.000000,0.000000}%
\pgfsetstrokecolor{currentstroke}%
\pgfsetstrokeopacity{0.000000}%
\pgfsetdash{}{0pt}%
\pgfpathmoveto{\pgfqpoint{5.800000in}{8.733927in}}%
\pgfpathlineto{\pgfqpoint{5.804319in}{8.733927in}}%
\pgfpathlineto{\pgfqpoint{5.804319in}{8.851779in}}%
\pgfpathlineto{\pgfqpoint{5.800000in}{8.851779in}}%
\pgfpathclose%
\pgfusepath{fill}%
\end{pgfscope}%
\begin{pgfscope}%
\pgfpathrectangle{\pgfqpoint{5.800000in}{0.720000in}}{\pgfqpoint{1.400000in}{4.620000in}}%
\pgfusepath{clip}%
\pgfsetbuttcap%
\pgfsetmiterjoin%
\definecolor{currentfill}{rgb}{0.121569,0.466667,0.705882}%
\pgfsetfillcolor{currentfill}%
\pgfsetlinewidth{0.000000pt}%
\definecolor{currentstroke}{rgb}{0.000000,0.000000,0.000000}%
\pgfsetstrokecolor{currentstroke}%
\pgfsetstrokeopacity{0.000000}%
\pgfsetdash{}{0pt}%
\pgfpathmoveto{\pgfqpoint{5.800000in}{8.851779in}}%
\pgfpathlineto{\pgfqpoint{5.806479in}{8.851779in}}%
\pgfpathlineto{\pgfqpoint{5.806479in}{8.969631in}}%
\pgfpathlineto{\pgfqpoint{5.800000in}{8.969631in}}%
\pgfpathclose%
\pgfusepath{fill}%
\end{pgfscope}%
\begin{pgfscope}%
\pgfpathrectangle{\pgfqpoint{5.800000in}{0.720000in}}{\pgfqpoint{1.400000in}{4.620000in}}%
\pgfusepath{clip}%
\pgfsetbuttcap%
\pgfsetmiterjoin%
\definecolor{currentfill}{rgb}{0.121569,0.466667,0.705882}%
\pgfsetfillcolor{currentfill}%
\pgfsetlinewidth{0.000000pt}%
\definecolor{currentstroke}{rgb}{0.000000,0.000000,0.000000}%
\pgfsetstrokecolor{currentstroke}%
\pgfsetstrokeopacity{0.000000}%
\pgfsetdash{}{0pt}%
\pgfpathmoveto{\pgfqpoint{5.800000in}{8.969631in}}%
\pgfpathlineto{\pgfqpoint{5.802160in}{8.969631in}}%
\pgfpathlineto{\pgfqpoint{5.802160in}{9.087483in}}%
\pgfpathlineto{\pgfqpoint{5.800000in}{9.087483in}}%
\pgfpathclose%
\pgfusepath{fill}%
\end{pgfscope}%
\begin{pgfscope}%
\pgfpathrectangle{\pgfqpoint{5.800000in}{0.720000in}}{\pgfqpoint{1.400000in}{4.620000in}}%
\pgfusepath{clip}%
\pgfsetbuttcap%
\pgfsetmiterjoin%
\definecolor{currentfill}{rgb}{0.121569,0.466667,0.705882}%
\pgfsetfillcolor{currentfill}%
\pgfsetlinewidth{0.000000pt}%
\definecolor{currentstroke}{rgb}{0.000000,0.000000,0.000000}%
\pgfsetstrokecolor{currentstroke}%
\pgfsetstrokeopacity{0.000000}%
\pgfsetdash{}{0pt}%
\pgfpathmoveto{\pgfqpoint{5.800000in}{9.087483in}}%
\pgfpathlineto{\pgfqpoint{5.804319in}{9.087483in}}%
\pgfpathlineto{\pgfqpoint{5.804319in}{9.205335in}}%
\pgfpathlineto{\pgfqpoint{5.800000in}{9.205335in}}%
\pgfpathclose%
\pgfusepath{fill}%
\end{pgfscope}%
\begin{pgfscope}%
\pgfpathrectangle{\pgfqpoint{5.800000in}{0.720000in}}{\pgfqpoint{1.400000in}{4.620000in}}%
\pgfusepath{clip}%
\pgfsetbuttcap%
\pgfsetmiterjoin%
\definecolor{currentfill}{rgb}{0.121569,0.466667,0.705882}%
\pgfsetfillcolor{currentfill}%
\pgfsetlinewidth{0.000000pt}%
\definecolor{currentstroke}{rgb}{0.000000,0.000000,0.000000}%
\pgfsetstrokecolor{currentstroke}%
\pgfsetstrokeopacity{0.000000}%
\pgfsetdash{}{0pt}%
\pgfpathmoveto{\pgfqpoint{5.800000in}{9.205335in}}%
\pgfpathlineto{\pgfqpoint{5.802160in}{9.205335in}}%
\pgfpathlineto{\pgfqpoint{5.802160in}{9.323187in}}%
\pgfpathlineto{\pgfqpoint{5.800000in}{9.323187in}}%
\pgfpathclose%
\pgfusepath{fill}%
\end{pgfscope}%
\begin{pgfscope}%
\pgfpathrectangle{\pgfqpoint{5.800000in}{0.720000in}}{\pgfqpoint{1.400000in}{4.620000in}}%
\pgfusepath{clip}%
\pgfsetbuttcap%
\pgfsetmiterjoin%
\definecolor{currentfill}{rgb}{0.121569,0.466667,0.705882}%
\pgfsetfillcolor{currentfill}%
\pgfsetlinewidth{0.000000pt}%
\definecolor{currentstroke}{rgb}{0.000000,0.000000,0.000000}%
\pgfsetstrokecolor{currentstroke}%
\pgfsetstrokeopacity{0.000000}%
\pgfsetdash{}{0pt}%
\pgfpathmoveto{\pgfqpoint{5.800000in}{9.323187in}}%
\pgfpathlineto{\pgfqpoint{5.804319in}{9.323187in}}%
\pgfpathlineto{\pgfqpoint{5.804319in}{9.441038in}}%
\pgfpathlineto{\pgfqpoint{5.800000in}{9.441038in}}%
\pgfpathclose%
\pgfusepath{fill}%
\end{pgfscope}%
\begin{pgfscope}%
\pgfpathrectangle{\pgfqpoint{5.800000in}{0.720000in}}{\pgfqpoint{1.400000in}{4.620000in}}%
\pgfusepath{clip}%
\pgfsetbuttcap%
\pgfsetmiterjoin%
\definecolor{currentfill}{rgb}{0.121569,0.466667,0.705882}%
\pgfsetfillcolor{currentfill}%
\pgfsetlinewidth{0.000000pt}%
\definecolor{currentstroke}{rgb}{0.000000,0.000000,0.000000}%
\pgfsetstrokecolor{currentstroke}%
\pgfsetstrokeopacity{0.000000}%
\pgfsetdash{}{0pt}%
\pgfpathmoveto{\pgfqpoint{5.800000in}{9.441038in}}%
\pgfpathlineto{\pgfqpoint{5.800000in}{9.441038in}}%
\pgfpathlineto{\pgfqpoint{5.800000in}{9.558890in}}%
\pgfpathlineto{\pgfqpoint{5.800000in}{9.558890in}}%
\pgfpathclose%
\pgfusepath{fill}%
\end{pgfscope}%
\begin{pgfscope}%
\pgfpathrectangle{\pgfqpoint{5.800000in}{0.720000in}}{\pgfqpoint{1.400000in}{4.620000in}}%
\pgfusepath{clip}%
\pgfsetbuttcap%
\pgfsetmiterjoin%
\definecolor{currentfill}{rgb}{0.121569,0.466667,0.705882}%
\pgfsetfillcolor{currentfill}%
\pgfsetlinewidth{0.000000pt}%
\definecolor{currentstroke}{rgb}{0.000000,0.000000,0.000000}%
\pgfsetstrokecolor{currentstroke}%
\pgfsetstrokeopacity{0.000000}%
\pgfsetdash{}{0pt}%
\pgfpathmoveto{\pgfqpoint{5.800000in}{9.558890in}}%
\pgfpathlineto{\pgfqpoint{5.800000in}{9.558890in}}%
\pgfpathlineto{\pgfqpoint{5.800000in}{9.676742in}}%
\pgfpathlineto{\pgfqpoint{5.800000in}{9.676742in}}%
\pgfpathclose%
\pgfusepath{fill}%
\end{pgfscope}%
\begin{pgfscope}%
\pgfpathrectangle{\pgfqpoint{5.800000in}{0.720000in}}{\pgfqpoint{1.400000in}{4.620000in}}%
\pgfusepath{clip}%
\pgfsetbuttcap%
\pgfsetmiterjoin%
\definecolor{currentfill}{rgb}{0.121569,0.466667,0.705882}%
\pgfsetfillcolor{currentfill}%
\pgfsetlinewidth{0.000000pt}%
\definecolor{currentstroke}{rgb}{0.000000,0.000000,0.000000}%
\pgfsetstrokecolor{currentstroke}%
\pgfsetstrokeopacity{0.000000}%
\pgfsetdash{}{0pt}%
\pgfpathmoveto{\pgfqpoint{5.800000in}{9.676742in}}%
\pgfpathlineto{\pgfqpoint{5.802160in}{9.676742in}}%
\pgfpathlineto{\pgfqpoint{5.802160in}{9.794594in}}%
\pgfpathlineto{\pgfqpoint{5.800000in}{9.794594in}}%
\pgfpathclose%
\pgfusepath{fill}%
\end{pgfscope}%
\begin{pgfscope}%
\pgfpathrectangle{\pgfqpoint{5.800000in}{0.720000in}}{\pgfqpoint{1.400000in}{4.620000in}}%
\pgfusepath{clip}%
\pgfsetbuttcap%
\pgfsetmiterjoin%
\definecolor{currentfill}{rgb}{0.121569,0.466667,0.705882}%
\pgfsetfillcolor{currentfill}%
\pgfsetlinewidth{0.000000pt}%
\definecolor{currentstroke}{rgb}{0.000000,0.000000,0.000000}%
\pgfsetstrokecolor{currentstroke}%
\pgfsetstrokeopacity{0.000000}%
\pgfsetdash{}{0pt}%
\pgfpathmoveto{\pgfqpoint{5.800000in}{9.794594in}}%
\pgfpathlineto{\pgfqpoint{5.800000in}{9.794594in}}%
\pgfpathlineto{\pgfqpoint{5.800000in}{9.912446in}}%
\pgfpathlineto{\pgfqpoint{5.800000in}{9.912446in}}%
\pgfpathclose%
\pgfusepath{fill}%
\end{pgfscope}%
\begin{pgfscope}%
\pgfpathrectangle{\pgfqpoint{5.800000in}{0.720000in}}{\pgfqpoint{1.400000in}{4.620000in}}%
\pgfusepath{clip}%
\pgfsetbuttcap%
\pgfsetmiterjoin%
\definecolor{currentfill}{rgb}{0.121569,0.466667,0.705882}%
\pgfsetfillcolor{currentfill}%
\pgfsetlinewidth{0.000000pt}%
\definecolor{currentstroke}{rgb}{0.000000,0.000000,0.000000}%
\pgfsetstrokecolor{currentstroke}%
\pgfsetstrokeopacity{0.000000}%
\pgfsetdash{}{0pt}%
\pgfpathmoveto{\pgfqpoint{5.800000in}{9.912446in}}%
\pgfpathlineto{\pgfqpoint{5.800000in}{9.912446in}}%
\pgfpathlineto{\pgfqpoint{5.800000in}{10.030298in}}%
\pgfpathlineto{\pgfqpoint{5.800000in}{10.030298in}}%
\pgfpathclose%
\pgfusepath{fill}%
\end{pgfscope}%
\begin{pgfscope}%
\pgfpathrectangle{\pgfqpoint{5.800000in}{0.720000in}}{\pgfqpoint{1.400000in}{4.620000in}}%
\pgfusepath{clip}%
\pgfsetbuttcap%
\pgfsetmiterjoin%
\definecolor{currentfill}{rgb}{0.121569,0.466667,0.705882}%
\pgfsetfillcolor{currentfill}%
\pgfsetlinewidth{0.000000pt}%
\definecolor{currentstroke}{rgb}{0.000000,0.000000,0.000000}%
\pgfsetstrokecolor{currentstroke}%
\pgfsetstrokeopacity{0.000000}%
\pgfsetdash{}{0pt}%
\pgfpathmoveto{\pgfqpoint{5.800000in}{10.030298in}}%
\pgfpathlineto{\pgfqpoint{5.802160in}{10.030298in}}%
\pgfpathlineto{\pgfqpoint{5.802160in}{10.148150in}}%
\pgfpathlineto{\pgfqpoint{5.800000in}{10.148150in}}%
\pgfpathclose%
\pgfusepath{fill}%
\end{pgfscope}%
\begin{pgfscope}%
\pgfpathrectangle{\pgfqpoint{5.800000in}{0.720000in}}{\pgfqpoint{1.400000in}{4.620000in}}%
\pgfusepath{clip}%
\pgfsetbuttcap%
\pgfsetmiterjoin%
\definecolor{currentfill}{rgb}{0.121569,0.466667,0.705882}%
\pgfsetfillcolor{currentfill}%
\pgfsetlinewidth{0.000000pt}%
\definecolor{currentstroke}{rgb}{0.000000,0.000000,0.000000}%
\pgfsetstrokecolor{currentstroke}%
\pgfsetstrokeopacity{0.000000}%
\pgfsetdash{}{0pt}%
\pgfpathmoveto{\pgfqpoint{5.800000in}{10.148150in}}%
\pgfpathlineto{\pgfqpoint{5.802160in}{10.148150in}}%
\pgfpathlineto{\pgfqpoint{5.802160in}{10.266001in}}%
\pgfpathlineto{\pgfqpoint{5.800000in}{10.266001in}}%
\pgfpathclose%
\pgfusepath{fill}%
\end{pgfscope}%
\begin{pgfscope}%
\pgfpathrectangle{\pgfqpoint{5.800000in}{0.720000in}}{\pgfqpoint{1.400000in}{4.620000in}}%
\pgfusepath{clip}%
\pgfsetbuttcap%
\pgfsetmiterjoin%
\definecolor{currentfill}{rgb}{0.121569,0.466667,0.705882}%
\pgfsetfillcolor{currentfill}%
\pgfsetlinewidth{0.000000pt}%
\definecolor{currentstroke}{rgb}{0.000000,0.000000,0.000000}%
\pgfsetstrokecolor{currentstroke}%
\pgfsetstrokeopacity{0.000000}%
\pgfsetdash{}{0pt}%
\pgfpathmoveto{\pgfqpoint{5.800000in}{10.266001in}}%
\pgfpathlineto{\pgfqpoint{5.806479in}{10.266001in}}%
\pgfpathlineto{\pgfqpoint{5.806479in}{10.383853in}}%
\pgfpathlineto{\pgfqpoint{5.800000in}{10.383853in}}%
\pgfpathclose%
\pgfusepath{fill}%
\end{pgfscope}%
\begin{pgfscope}%
\pgfpathrectangle{\pgfqpoint{5.800000in}{0.720000in}}{\pgfqpoint{1.400000in}{4.620000in}}%
\pgfusepath{clip}%
\pgfsetbuttcap%
\pgfsetmiterjoin%
\definecolor{currentfill}{rgb}{0.121569,0.466667,0.705882}%
\pgfsetfillcolor{currentfill}%
\pgfsetlinewidth{0.000000pt}%
\definecolor{currentstroke}{rgb}{0.000000,0.000000,0.000000}%
\pgfsetstrokecolor{currentstroke}%
\pgfsetstrokeopacity{0.000000}%
\pgfsetdash{}{0pt}%
\pgfpathmoveto{\pgfqpoint{5.800000in}{10.383853in}}%
\pgfpathlineto{\pgfqpoint{5.802160in}{10.383853in}}%
\pgfpathlineto{\pgfqpoint{5.802160in}{10.501705in}}%
\pgfpathlineto{\pgfqpoint{5.800000in}{10.501705in}}%
\pgfpathclose%
\pgfusepath{fill}%
\end{pgfscope}%
\begin{pgfscope}%
\pgfpathrectangle{\pgfqpoint{5.800000in}{0.720000in}}{\pgfqpoint{1.400000in}{4.620000in}}%
\pgfusepath{clip}%
\pgfsetbuttcap%
\pgfsetmiterjoin%
\definecolor{currentfill}{rgb}{0.121569,0.466667,0.705882}%
\pgfsetfillcolor{currentfill}%
\pgfsetlinewidth{0.000000pt}%
\definecolor{currentstroke}{rgb}{0.000000,0.000000,0.000000}%
\pgfsetstrokecolor{currentstroke}%
\pgfsetstrokeopacity{0.000000}%
\pgfsetdash{}{0pt}%
\pgfpathmoveto{\pgfqpoint{5.800000in}{10.501705in}}%
\pgfpathlineto{\pgfqpoint{5.802160in}{10.501705in}}%
\pgfpathlineto{\pgfqpoint{5.802160in}{10.619557in}}%
\pgfpathlineto{\pgfqpoint{5.800000in}{10.619557in}}%
\pgfpathclose%
\pgfusepath{fill}%
\end{pgfscope}%
\begin{pgfscope}%
\pgfpathrectangle{\pgfqpoint{5.800000in}{0.720000in}}{\pgfqpoint{1.400000in}{4.620000in}}%
\pgfusepath{clip}%
\pgfsetbuttcap%
\pgfsetmiterjoin%
\definecolor{currentfill}{rgb}{0.121569,0.466667,0.705882}%
\pgfsetfillcolor{currentfill}%
\pgfsetlinewidth{0.000000pt}%
\definecolor{currentstroke}{rgb}{0.000000,0.000000,0.000000}%
\pgfsetstrokecolor{currentstroke}%
\pgfsetstrokeopacity{0.000000}%
\pgfsetdash{}{0pt}%
\pgfpathmoveto{\pgfqpoint{5.800000in}{10.619557in}}%
\pgfpathlineto{\pgfqpoint{5.800000in}{10.619557in}}%
\pgfpathlineto{\pgfqpoint{5.800000in}{10.737409in}}%
\pgfpathlineto{\pgfqpoint{5.800000in}{10.737409in}}%
\pgfpathclose%
\pgfusepath{fill}%
\end{pgfscope}%
\begin{pgfscope}%
\pgfpathrectangle{\pgfqpoint{5.800000in}{0.720000in}}{\pgfqpoint{1.400000in}{4.620000in}}%
\pgfusepath{clip}%
\pgfsetbuttcap%
\pgfsetmiterjoin%
\definecolor{currentfill}{rgb}{0.121569,0.466667,0.705882}%
\pgfsetfillcolor{currentfill}%
\pgfsetlinewidth{0.000000pt}%
\definecolor{currentstroke}{rgb}{0.000000,0.000000,0.000000}%
\pgfsetstrokecolor{currentstroke}%
\pgfsetstrokeopacity{0.000000}%
\pgfsetdash{}{0pt}%
\pgfpathmoveto{\pgfqpoint{5.800000in}{10.737409in}}%
\pgfpathlineto{\pgfqpoint{5.806479in}{10.737409in}}%
\pgfpathlineto{\pgfqpoint{5.806479in}{10.855261in}}%
\pgfpathlineto{\pgfqpoint{5.800000in}{10.855261in}}%
\pgfpathclose%
\pgfusepath{fill}%
\end{pgfscope}%
\begin{pgfscope}%
\pgfpathrectangle{\pgfqpoint{5.800000in}{0.720000in}}{\pgfqpoint{1.400000in}{4.620000in}}%
\pgfusepath{clip}%
\pgfsetbuttcap%
\pgfsetmiterjoin%
\definecolor{currentfill}{rgb}{0.121569,0.466667,0.705882}%
\pgfsetfillcolor{currentfill}%
\pgfsetlinewidth{0.000000pt}%
\definecolor{currentstroke}{rgb}{0.000000,0.000000,0.000000}%
\pgfsetstrokecolor{currentstroke}%
\pgfsetstrokeopacity{0.000000}%
\pgfsetdash{}{0pt}%
\pgfpathmoveto{\pgfqpoint{5.800000in}{10.855261in}}%
\pgfpathlineto{\pgfqpoint{5.804319in}{10.855261in}}%
\pgfpathlineto{\pgfqpoint{5.804319in}{10.973113in}}%
\pgfpathlineto{\pgfqpoint{5.800000in}{10.973113in}}%
\pgfpathclose%
\pgfusepath{fill}%
\end{pgfscope}%
\begin{pgfscope}%
\pgfpathrectangle{\pgfqpoint{5.800000in}{0.720000in}}{\pgfqpoint{1.400000in}{4.620000in}}%
\pgfusepath{clip}%
\pgfsetbuttcap%
\pgfsetmiterjoin%
\definecolor{currentfill}{rgb}{0.121569,0.466667,0.705882}%
\pgfsetfillcolor{currentfill}%
\pgfsetlinewidth{0.000000pt}%
\definecolor{currentstroke}{rgb}{0.000000,0.000000,0.000000}%
\pgfsetstrokecolor{currentstroke}%
\pgfsetstrokeopacity{0.000000}%
\pgfsetdash{}{0pt}%
\pgfpathmoveto{\pgfqpoint{5.800000in}{10.973113in}}%
\pgfpathlineto{\pgfqpoint{5.800000in}{10.973113in}}%
\pgfpathlineto{\pgfqpoint{5.800000in}{11.090965in}}%
\pgfpathlineto{\pgfqpoint{5.800000in}{11.090965in}}%
\pgfpathclose%
\pgfusepath{fill}%
\end{pgfscope}%
\begin{pgfscope}%
\pgfpathrectangle{\pgfqpoint{5.800000in}{0.720000in}}{\pgfqpoint{1.400000in}{4.620000in}}%
\pgfusepath{clip}%
\pgfsetbuttcap%
\pgfsetmiterjoin%
\definecolor{currentfill}{rgb}{0.121569,0.466667,0.705882}%
\pgfsetfillcolor{currentfill}%
\pgfsetlinewidth{0.000000pt}%
\definecolor{currentstroke}{rgb}{0.000000,0.000000,0.000000}%
\pgfsetstrokecolor{currentstroke}%
\pgfsetstrokeopacity{0.000000}%
\pgfsetdash{}{0pt}%
\pgfpathmoveto{\pgfqpoint{5.800000in}{11.090965in}}%
\pgfpathlineto{\pgfqpoint{5.804319in}{11.090965in}}%
\pgfpathlineto{\pgfqpoint{5.804319in}{11.208816in}}%
\pgfpathlineto{\pgfqpoint{5.800000in}{11.208816in}}%
\pgfpathclose%
\pgfusepath{fill}%
\end{pgfscope}%
\begin{pgfscope}%
\pgfpathrectangle{\pgfqpoint{5.800000in}{0.720000in}}{\pgfqpoint{1.400000in}{4.620000in}}%
\pgfusepath{clip}%
\pgfsetbuttcap%
\pgfsetmiterjoin%
\definecolor{currentfill}{rgb}{0.121569,0.466667,0.705882}%
\pgfsetfillcolor{currentfill}%
\pgfsetlinewidth{0.000000pt}%
\definecolor{currentstroke}{rgb}{0.000000,0.000000,0.000000}%
\pgfsetstrokecolor{currentstroke}%
\pgfsetstrokeopacity{0.000000}%
\pgfsetdash{}{0pt}%
\pgfpathmoveto{\pgfqpoint{5.800000in}{11.208816in}}%
\pgfpathlineto{\pgfqpoint{5.800000in}{11.208816in}}%
\pgfpathlineto{\pgfqpoint{5.800000in}{11.326668in}}%
\pgfpathlineto{\pgfqpoint{5.800000in}{11.326668in}}%
\pgfpathclose%
\pgfusepath{fill}%
\end{pgfscope}%
\begin{pgfscope}%
\pgfpathrectangle{\pgfqpoint{5.800000in}{0.720000in}}{\pgfqpoint{1.400000in}{4.620000in}}%
\pgfusepath{clip}%
\pgfsetbuttcap%
\pgfsetmiterjoin%
\definecolor{currentfill}{rgb}{0.121569,0.466667,0.705882}%
\pgfsetfillcolor{currentfill}%
\pgfsetlinewidth{0.000000pt}%
\definecolor{currentstroke}{rgb}{0.000000,0.000000,0.000000}%
\pgfsetstrokecolor{currentstroke}%
\pgfsetstrokeopacity{0.000000}%
\pgfsetdash{}{0pt}%
\pgfpathmoveto{\pgfqpoint{5.800000in}{11.326668in}}%
\pgfpathlineto{\pgfqpoint{5.800000in}{11.326668in}}%
\pgfpathlineto{\pgfqpoint{5.800000in}{11.444520in}}%
\pgfpathlineto{\pgfqpoint{5.800000in}{11.444520in}}%
\pgfpathclose%
\pgfusepath{fill}%
\end{pgfscope}%
\begin{pgfscope}%
\pgfpathrectangle{\pgfqpoint{5.800000in}{0.720000in}}{\pgfqpoint{1.400000in}{4.620000in}}%
\pgfusepath{clip}%
\pgfsetbuttcap%
\pgfsetmiterjoin%
\definecolor{currentfill}{rgb}{0.121569,0.466667,0.705882}%
\pgfsetfillcolor{currentfill}%
\pgfsetlinewidth{0.000000pt}%
\definecolor{currentstroke}{rgb}{0.000000,0.000000,0.000000}%
\pgfsetstrokecolor{currentstroke}%
\pgfsetstrokeopacity{0.000000}%
\pgfsetdash{}{0pt}%
\pgfpathmoveto{\pgfqpoint{5.800000in}{11.444520in}}%
\pgfpathlineto{\pgfqpoint{5.802160in}{11.444520in}}%
\pgfpathlineto{\pgfqpoint{5.802160in}{11.562372in}}%
\pgfpathlineto{\pgfqpoint{5.800000in}{11.562372in}}%
\pgfpathclose%
\pgfusepath{fill}%
\end{pgfscope}%
\begin{pgfscope}%
\pgfpathrectangle{\pgfqpoint{5.800000in}{0.720000in}}{\pgfqpoint{1.400000in}{4.620000in}}%
\pgfusepath{clip}%
\pgfsetbuttcap%
\pgfsetmiterjoin%
\definecolor{currentfill}{rgb}{0.121569,0.466667,0.705882}%
\pgfsetfillcolor{currentfill}%
\pgfsetlinewidth{0.000000pt}%
\definecolor{currentstroke}{rgb}{0.000000,0.000000,0.000000}%
\pgfsetstrokecolor{currentstroke}%
\pgfsetstrokeopacity{0.000000}%
\pgfsetdash{}{0pt}%
\pgfpathmoveto{\pgfqpoint{5.800000in}{11.562372in}}%
\pgfpathlineto{\pgfqpoint{5.802160in}{11.562372in}}%
\pgfpathlineto{\pgfqpoint{5.802160in}{11.680224in}}%
\pgfpathlineto{\pgfqpoint{5.800000in}{11.680224in}}%
\pgfpathclose%
\pgfusepath{fill}%
\end{pgfscope}%
\begin{pgfscope}%
\pgfpathrectangle{\pgfqpoint{5.800000in}{0.720000in}}{\pgfqpoint{1.400000in}{4.620000in}}%
\pgfusepath{clip}%
\pgfsetbuttcap%
\pgfsetmiterjoin%
\definecolor{currentfill}{rgb}{0.121569,0.466667,0.705882}%
\pgfsetfillcolor{currentfill}%
\pgfsetlinewidth{0.000000pt}%
\definecolor{currentstroke}{rgb}{0.000000,0.000000,0.000000}%
\pgfsetstrokecolor{currentstroke}%
\pgfsetstrokeopacity{0.000000}%
\pgfsetdash{}{0pt}%
\pgfpathmoveto{\pgfqpoint{5.800000in}{11.680224in}}%
\pgfpathlineto{\pgfqpoint{5.804319in}{11.680224in}}%
\pgfpathlineto{\pgfqpoint{5.804319in}{11.798076in}}%
\pgfpathlineto{\pgfqpoint{5.800000in}{11.798076in}}%
\pgfpathclose%
\pgfusepath{fill}%
\end{pgfscope}%
\begin{pgfscope}%
\pgfpathrectangle{\pgfqpoint{5.800000in}{0.720000in}}{\pgfqpoint{1.400000in}{4.620000in}}%
\pgfusepath{clip}%
\pgfsetbuttcap%
\pgfsetmiterjoin%
\definecolor{currentfill}{rgb}{0.121569,0.466667,0.705882}%
\pgfsetfillcolor{currentfill}%
\pgfsetlinewidth{0.000000pt}%
\definecolor{currentstroke}{rgb}{0.000000,0.000000,0.000000}%
\pgfsetstrokecolor{currentstroke}%
\pgfsetstrokeopacity{0.000000}%
\pgfsetdash{}{0pt}%
\pgfpathmoveto{\pgfqpoint{5.800000in}{11.798076in}}%
\pgfpathlineto{\pgfqpoint{5.800000in}{11.798076in}}%
\pgfpathlineto{\pgfqpoint{5.800000in}{11.915928in}}%
\pgfpathlineto{\pgfqpoint{5.800000in}{11.915928in}}%
\pgfpathclose%
\pgfusepath{fill}%
\end{pgfscope}%
\begin{pgfscope}%
\pgfpathrectangle{\pgfqpoint{5.800000in}{0.720000in}}{\pgfqpoint{1.400000in}{4.620000in}}%
\pgfusepath{clip}%
\pgfsetbuttcap%
\pgfsetmiterjoin%
\definecolor{currentfill}{rgb}{0.121569,0.466667,0.705882}%
\pgfsetfillcolor{currentfill}%
\pgfsetlinewidth{0.000000pt}%
\definecolor{currentstroke}{rgb}{0.000000,0.000000,0.000000}%
\pgfsetstrokecolor{currentstroke}%
\pgfsetstrokeopacity{0.000000}%
\pgfsetdash{}{0pt}%
\pgfpathmoveto{\pgfqpoint{5.800000in}{11.915928in}}%
\pgfpathlineto{\pgfqpoint{5.800000in}{11.915928in}}%
\pgfpathlineto{\pgfqpoint{5.800000in}{12.033780in}}%
\pgfpathlineto{\pgfqpoint{5.800000in}{12.033780in}}%
\pgfpathclose%
\pgfusepath{fill}%
\end{pgfscope}%
\begin{pgfscope}%
\pgfpathrectangle{\pgfqpoint{5.800000in}{0.720000in}}{\pgfqpoint{1.400000in}{4.620000in}}%
\pgfusepath{clip}%
\pgfsetbuttcap%
\pgfsetmiterjoin%
\definecolor{currentfill}{rgb}{0.121569,0.466667,0.705882}%
\pgfsetfillcolor{currentfill}%
\pgfsetlinewidth{0.000000pt}%
\definecolor{currentstroke}{rgb}{0.000000,0.000000,0.000000}%
\pgfsetstrokecolor{currentstroke}%
\pgfsetstrokeopacity{0.000000}%
\pgfsetdash{}{0pt}%
\pgfpathmoveto{\pgfqpoint{5.800000in}{12.033780in}}%
\pgfpathlineto{\pgfqpoint{5.800000in}{12.033780in}}%
\pgfpathlineto{\pgfqpoint{5.800000in}{12.151631in}}%
\pgfpathlineto{\pgfqpoint{5.800000in}{12.151631in}}%
\pgfpathclose%
\pgfusepath{fill}%
\end{pgfscope}%
\begin{pgfscope}%
\pgfpathrectangle{\pgfqpoint{5.800000in}{0.720000in}}{\pgfqpoint{1.400000in}{4.620000in}}%
\pgfusepath{clip}%
\pgfsetbuttcap%
\pgfsetmiterjoin%
\definecolor{currentfill}{rgb}{0.121569,0.466667,0.705882}%
\pgfsetfillcolor{currentfill}%
\pgfsetlinewidth{0.000000pt}%
\definecolor{currentstroke}{rgb}{0.000000,0.000000,0.000000}%
\pgfsetstrokecolor{currentstroke}%
\pgfsetstrokeopacity{0.000000}%
\pgfsetdash{}{0pt}%
\pgfpathmoveto{\pgfqpoint{5.800000in}{12.151631in}}%
\pgfpathlineto{\pgfqpoint{5.800000in}{12.151631in}}%
\pgfpathlineto{\pgfqpoint{5.800000in}{12.269483in}}%
\pgfpathlineto{\pgfqpoint{5.800000in}{12.269483in}}%
\pgfpathclose%
\pgfusepath{fill}%
\end{pgfscope}%
\begin{pgfscope}%
\pgfpathrectangle{\pgfqpoint{5.800000in}{0.720000in}}{\pgfqpoint{1.400000in}{4.620000in}}%
\pgfusepath{clip}%
\pgfsetbuttcap%
\pgfsetmiterjoin%
\definecolor{currentfill}{rgb}{0.121569,0.466667,0.705882}%
\pgfsetfillcolor{currentfill}%
\pgfsetlinewidth{0.000000pt}%
\definecolor{currentstroke}{rgb}{0.000000,0.000000,0.000000}%
\pgfsetstrokecolor{currentstroke}%
\pgfsetstrokeopacity{0.000000}%
\pgfsetdash{}{0pt}%
\pgfpathmoveto{\pgfqpoint{5.800000in}{12.269483in}}%
\pgfpathlineto{\pgfqpoint{5.802160in}{12.269483in}}%
\pgfpathlineto{\pgfqpoint{5.802160in}{12.387335in}}%
\pgfpathlineto{\pgfqpoint{5.800000in}{12.387335in}}%
\pgfpathclose%
\pgfusepath{fill}%
\end{pgfscope}%
\begin{pgfscope}%
\pgfpathrectangle{\pgfqpoint{5.800000in}{0.720000in}}{\pgfqpoint{1.400000in}{4.620000in}}%
\pgfusepath{clip}%
\pgfsetbuttcap%
\pgfsetmiterjoin%
\definecolor{currentfill}{rgb}{0.121569,0.466667,0.705882}%
\pgfsetfillcolor{currentfill}%
\pgfsetlinewidth{0.000000pt}%
\definecolor{currentstroke}{rgb}{0.000000,0.000000,0.000000}%
\pgfsetstrokecolor{currentstroke}%
\pgfsetstrokeopacity{0.000000}%
\pgfsetdash{}{0pt}%
\pgfpathmoveto{\pgfqpoint{5.800000in}{12.387335in}}%
\pgfpathlineto{\pgfqpoint{5.802160in}{12.387335in}}%
\pgfpathlineto{\pgfqpoint{5.802160in}{12.505187in}}%
\pgfpathlineto{\pgfqpoint{5.800000in}{12.505187in}}%
\pgfpathclose%
\pgfusepath{fill}%
\end{pgfscope}%
\begin{pgfscope}%
\pgfpathrectangle{\pgfqpoint{5.800000in}{0.720000in}}{\pgfqpoint{1.400000in}{4.620000in}}%
\pgfusepath{clip}%
\pgfsetbuttcap%
\pgfsetmiterjoin%
\definecolor{currentfill}{rgb}{0.121569,0.466667,0.705882}%
\pgfsetfillcolor{currentfill}%
\pgfsetlinewidth{0.000000pt}%
\definecolor{currentstroke}{rgb}{0.000000,0.000000,0.000000}%
\pgfsetstrokecolor{currentstroke}%
\pgfsetstrokeopacity{0.000000}%
\pgfsetdash{}{0pt}%
\pgfpathmoveto{\pgfqpoint{5.800000in}{12.505187in}}%
\pgfpathlineto{\pgfqpoint{5.800000in}{12.505187in}}%
\pgfpathlineto{\pgfqpoint{5.800000in}{12.623039in}}%
\pgfpathlineto{\pgfqpoint{5.800000in}{12.623039in}}%
\pgfpathclose%
\pgfusepath{fill}%
\end{pgfscope}%
\begin{pgfscope}%
\pgfpathrectangle{\pgfqpoint{5.800000in}{0.720000in}}{\pgfqpoint{1.400000in}{4.620000in}}%
\pgfusepath{clip}%
\pgfsetbuttcap%
\pgfsetmiterjoin%
\definecolor{currentfill}{rgb}{0.121569,0.466667,0.705882}%
\pgfsetfillcolor{currentfill}%
\pgfsetlinewidth{0.000000pt}%
\definecolor{currentstroke}{rgb}{0.000000,0.000000,0.000000}%
\pgfsetstrokecolor{currentstroke}%
\pgfsetstrokeopacity{0.000000}%
\pgfsetdash{}{0pt}%
\pgfpathmoveto{\pgfqpoint{5.800000in}{12.623039in}}%
\pgfpathlineto{\pgfqpoint{5.800000in}{12.623039in}}%
\pgfpathlineto{\pgfqpoint{5.800000in}{12.740891in}}%
\pgfpathlineto{\pgfqpoint{5.800000in}{12.740891in}}%
\pgfpathclose%
\pgfusepath{fill}%
\end{pgfscope}%
\begin{pgfscope}%
\pgfpathrectangle{\pgfqpoint{5.800000in}{0.720000in}}{\pgfqpoint{1.400000in}{4.620000in}}%
\pgfusepath{clip}%
\pgfsetbuttcap%
\pgfsetmiterjoin%
\definecolor{currentfill}{rgb}{0.121569,0.466667,0.705882}%
\pgfsetfillcolor{currentfill}%
\pgfsetlinewidth{0.000000pt}%
\definecolor{currentstroke}{rgb}{0.000000,0.000000,0.000000}%
\pgfsetstrokecolor{currentstroke}%
\pgfsetstrokeopacity{0.000000}%
\pgfsetdash{}{0pt}%
\pgfpathmoveto{\pgfqpoint{5.800000in}{12.740891in}}%
\pgfpathlineto{\pgfqpoint{5.802160in}{12.740891in}}%
\pgfpathlineto{\pgfqpoint{5.802160in}{12.858743in}}%
\pgfpathlineto{\pgfqpoint{5.800000in}{12.858743in}}%
\pgfpathclose%
\pgfusepath{fill}%
\end{pgfscope}%
\begin{pgfscope}%
\pgfpathrectangle{\pgfqpoint{5.800000in}{0.720000in}}{\pgfqpoint{1.400000in}{4.620000in}}%
\pgfusepath{clip}%
\pgfsetbuttcap%
\pgfsetmiterjoin%
\definecolor{currentfill}{rgb}{0.121569,0.466667,0.705882}%
\pgfsetfillcolor{currentfill}%
\pgfsetlinewidth{0.000000pt}%
\definecolor{currentstroke}{rgb}{0.000000,0.000000,0.000000}%
\pgfsetstrokecolor{currentstroke}%
\pgfsetstrokeopacity{0.000000}%
\pgfsetdash{}{0pt}%
\pgfpathmoveto{\pgfqpoint{5.800000in}{12.858743in}}%
\pgfpathlineto{\pgfqpoint{5.800000in}{12.858743in}}%
\pgfpathlineto{\pgfqpoint{5.800000in}{12.976595in}}%
\pgfpathlineto{\pgfqpoint{5.800000in}{12.976595in}}%
\pgfpathclose%
\pgfusepath{fill}%
\end{pgfscope}%
\begin{pgfscope}%
\pgfpathrectangle{\pgfqpoint{5.800000in}{0.720000in}}{\pgfqpoint{1.400000in}{4.620000in}}%
\pgfusepath{clip}%
\pgfsetbuttcap%
\pgfsetmiterjoin%
\definecolor{currentfill}{rgb}{0.121569,0.466667,0.705882}%
\pgfsetfillcolor{currentfill}%
\pgfsetlinewidth{0.000000pt}%
\definecolor{currentstroke}{rgb}{0.000000,0.000000,0.000000}%
\pgfsetstrokecolor{currentstroke}%
\pgfsetstrokeopacity{0.000000}%
\pgfsetdash{}{0pt}%
\pgfpathmoveto{\pgfqpoint{5.800000in}{12.976595in}}%
\pgfpathlineto{\pgfqpoint{5.800000in}{12.976595in}}%
\pgfpathlineto{\pgfqpoint{5.800000in}{13.094446in}}%
\pgfpathlineto{\pgfqpoint{5.800000in}{13.094446in}}%
\pgfpathclose%
\pgfusepath{fill}%
\end{pgfscope}%
\begin{pgfscope}%
\pgfpathrectangle{\pgfqpoint{5.800000in}{0.720000in}}{\pgfqpoint{1.400000in}{4.620000in}}%
\pgfusepath{clip}%
\pgfsetbuttcap%
\pgfsetmiterjoin%
\definecolor{currentfill}{rgb}{0.121569,0.466667,0.705882}%
\pgfsetfillcolor{currentfill}%
\pgfsetlinewidth{0.000000pt}%
\definecolor{currentstroke}{rgb}{0.000000,0.000000,0.000000}%
\pgfsetstrokecolor{currentstroke}%
\pgfsetstrokeopacity{0.000000}%
\pgfsetdash{}{0pt}%
\pgfpathmoveto{\pgfqpoint{5.800000in}{13.094446in}}%
\pgfpathlineto{\pgfqpoint{5.800000in}{13.094446in}}%
\pgfpathlineto{\pgfqpoint{5.800000in}{13.212298in}}%
\pgfpathlineto{\pgfqpoint{5.800000in}{13.212298in}}%
\pgfpathclose%
\pgfusepath{fill}%
\end{pgfscope}%
\begin{pgfscope}%
\pgfpathrectangle{\pgfqpoint{5.800000in}{0.720000in}}{\pgfqpoint{1.400000in}{4.620000in}}%
\pgfusepath{clip}%
\pgfsetbuttcap%
\pgfsetmiterjoin%
\definecolor{currentfill}{rgb}{0.121569,0.466667,0.705882}%
\pgfsetfillcolor{currentfill}%
\pgfsetlinewidth{0.000000pt}%
\definecolor{currentstroke}{rgb}{0.000000,0.000000,0.000000}%
\pgfsetstrokecolor{currentstroke}%
\pgfsetstrokeopacity{0.000000}%
\pgfsetdash{}{0pt}%
\pgfpathmoveto{\pgfqpoint{5.800000in}{13.212298in}}%
\pgfpathlineto{\pgfqpoint{5.802160in}{13.212298in}}%
\pgfpathlineto{\pgfqpoint{5.802160in}{13.330150in}}%
\pgfpathlineto{\pgfqpoint{5.800000in}{13.330150in}}%
\pgfpathclose%
\pgfusepath{fill}%
\end{pgfscope}%
\begin{pgfscope}%
\pgfpathrectangle{\pgfqpoint{5.800000in}{0.720000in}}{\pgfqpoint{1.400000in}{4.620000in}}%
\pgfusepath{clip}%
\pgfsetbuttcap%
\pgfsetmiterjoin%
\definecolor{currentfill}{rgb}{0.121569,0.466667,0.705882}%
\pgfsetfillcolor{currentfill}%
\pgfsetlinewidth{0.000000pt}%
\definecolor{currentstroke}{rgb}{0.000000,0.000000,0.000000}%
\pgfsetstrokecolor{currentstroke}%
\pgfsetstrokeopacity{0.000000}%
\pgfsetdash{}{0pt}%
\pgfpathmoveto{\pgfqpoint{5.800000in}{13.330150in}}%
\pgfpathlineto{\pgfqpoint{5.800000in}{13.330150in}}%
\pgfpathlineto{\pgfqpoint{5.800000in}{13.448002in}}%
\pgfpathlineto{\pgfqpoint{5.800000in}{13.448002in}}%
\pgfpathclose%
\pgfusepath{fill}%
\end{pgfscope}%
\begin{pgfscope}%
\pgfpathrectangle{\pgfqpoint{5.800000in}{0.720000in}}{\pgfqpoint{1.400000in}{4.620000in}}%
\pgfusepath{clip}%
\pgfsetbuttcap%
\pgfsetmiterjoin%
\definecolor{currentfill}{rgb}{0.121569,0.466667,0.705882}%
\pgfsetfillcolor{currentfill}%
\pgfsetlinewidth{0.000000pt}%
\definecolor{currentstroke}{rgb}{0.000000,0.000000,0.000000}%
\pgfsetstrokecolor{currentstroke}%
\pgfsetstrokeopacity{0.000000}%
\pgfsetdash{}{0pt}%
\pgfpathmoveto{\pgfqpoint{5.800000in}{13.448002in}}%
\pgfpathlineto{\pgfqpoint{5.800000in}{13.448002in}}%
\pgfpathlineto{\pgfqpoint{5.800000in}{13.565854in}}%
\pgfpathlineto{\pgfqpoint{5.800000in}{13.565854in}}%
\pgfpathclose%
\pgfusepath{fill}%
\end{pgfscope}%
\begin{pgfscope}%
\pgfpathrectangle{\pgfqpoint{5.800000in}{0.720000in}}{\pgfqpoint{1.400000in}{4.620000in}}%
\pgfusepath{clip}%
\pgfsetbuttcap%
\pgfsetmiterjoin%
\definecolor{currentfill}{rgb}{0.121569,0.466667,0.705882}%
\pgfsetfillcolor{currentfill}%
\pgfsetlinewidth{0.000000pt}%
\definecolor{currentstroke}{rgb}{0.000000,0.000000,0.000000}%
\pgfsetstrokecolor{currentstroke}%
\pgfsetstrokeopacity{0.000000}%
\pgfsetdash{}{0pt}%
\pgfpathmoveto{\pgfqpoint{5.800000in}{13.565854in}}%
\pgfpathlineto{\pgfqpoint{5.800000in}{13.565854in}}%
\pgfpathlineto{\pgfqpoint{5.800000in}{13.683706in}}%
\pgfpathlineto{\pgfqpoint{5.800000in}{13.683706in}}%
\pgfpathclose%
\pgfusepath{fill}%
\end{pgfscope}%
\begin{pgfscope}%
\pgfpathrectangle{\pgfqpoint{5.800000in}{0.720000in}}{\pgfqpoint{1.400000in}{4.620000in}}%
\pgfusepath{clip}%
\pgfsetbuttcap%
\pgfsetmiterjoin%
\definecolor{currentfill}{rgb}{0.121569,0.466667,0.705882}%
\pgfsetfillcolor{currentfill}%
\pgfsetlinewidth{0.000000pt}%
\definecolor{currentstroke}{rgb}{0.000000,0.000000,0.000000}%
\pgfsetstrokecolor{currentstroke}%
\pgfsetstrokeopacity{0.000000}%
\pgfsetdash{}{0pt}%
\pgfpathmoveto{\pgfqpoint{5.800000in}{13.683706in}}%
\pgfpathlineto{\pgfqpoint{5.802160in}{13.683706in}}%
\pgfpathlineto{\pgfqpoint{5.802160in}{13.801558in}}%
\pgfpathlineto{\pgfqpoint{5.800000in}{13.801558in}}%
\pgfpathclose%
\pgfusepath{fill}%
\end{pgfscope}%
\begin{pgfscope}%
\pgfpathrectangle{\pgfqpoint{5.800000in}{0.720000in}}{\pgfqpoint{1.400000in}{4.620000in}}%
\pgfusepath{clip}%
\pgfsetbuttcap%
\pgfsetmiterjoin%
\definecolor{currentfill}{rgb}{0.121569,0.466667,0.705882}%
\pgfsetfillcolor{currentfill}%
\pgfsetlinewidth{0.000000pt}%
\definecolor{currentstroke}{rgb}{0.000000,0.000000,0.000000}%
\pgfsetstrokecolor{currentstroke}%
\pgfsetstrokeopacity{0.000000}%
\pgfsetdash{}{0pt}%
\pgfpathmoveto{\pgfqpoint{5.800000in}{13.801558in}}%
\pgfpathlineto{\pgfqpoint{5.800000in}{13.801558in}}%
\pgfpathlineto{\pgfqpoint{5.800000in}{13.919409in}}%
\pgfpathlineto{\pgfqpoint{5.800000in}{13.919409in}}%
\pgfpathclose%
\pgfusepath{fill}%
\end{pgfscope}%
\begin{pgfscope}%
\pgfpathrectangle{\pgfqpoint{5.800000in}{0.720000in}}{\pgfqpoint{1.400000in}{4.620000in}}%
\pgfusepath{clip}%
\pgfsetbuttcap%
\pgfsetmiterjoin%
\definecolor{currentfill}{rgb}{0.121569,0.466667,0.705882}%
\pgfsetfillcolor{currentfill}%
\pgfsetlinewidth{0.000000pt}%
\definecolor{currentstroke}{rgb}{0.000000,0.000000,0.000000}%
\pgfsetstrokecolor{currentstroke}%
\pgfsetstrokeopacity{0.000000}%
\pgfsetdash{}{0pt}%
\pgfpathmoveto{\pgfqpoint{5.800000in}{13.919409in}}%
\pgfpathlineto{\pgfqpoint{5.800000in}{13.919409in}}%
\pgfpathlineto{\pgfqpoint{5.800000in}{14.037261in}}%
\pgfpathlineto{\pgfqpoint{5.800000in}{14.037261in}}%
\pgfpathclose%
\pgfusepath{fill}%
\end{pgfscope}%
\begin{pgfscope}%
\pgfpathrectangle{\pgfqpoint{5.800000in}{0.720000in}}{\pgfqpoint{1.400000in}{4.620000in}}%
\pgfusepath{clip}%
\pgfsetbuttcap%
\pgfsetmiterjoin%
\definecolor{currentfill}{rgb}{0.121569,0.466667,0.705882}%
\pgfsetfillcolor{currentfill}%
\pgfsetlinewidth{0.000000pt}%
\definecolor{currentstroke}{rgb}{0.000000,0.000000,0.000000}%
\pgfsetstrokecolor{currentstroke}%
\pgfsetstrokeopacity{0.000000}%
\pgfsetdash{}{0pt}%
\pgfpathmoveto{\pgfqpoint{5.800000in}{14.037261in}}%
\pgfpathlineto{\pgfqpoint{5.800000in}{14.037261in}}%
\pgfpathlineto{\pgfqpoint{5.800000in}{14.155113in}}%
\pgfpathlineto{\pgfqpoint{5.800000in}{14.155113in}}%
\pgfpathclose%
\pgfusepath{fill}%
\end{pgfscope}%
\begin{pgfscope}%
\pgfpathrectangle{\pgfqpoint{5.800000in}{0.720000in}}{\pgfqpoint{1.400000in}{4.620000in}}%
\pgfusepath{clip}%
\pgfsetbuttcap%
\pgfsetmiterjoin%
\definecolor{currentfill}{rgb}{0.121569,0.466667,0.705882}%
\pgfsetfillcolor{currentfill}%
\pgfsetlinewidth{0.000000pt}%
\definecolor{currentstroke}{rgb}{0.000000,0.000000,0.000000}%
\pgfsetstrokecolor{currentstroke}%
\pgfsetstrokeopacity{0.000000}%
\pgfsetdash{}{0pt}%
\pgfpathmoveto{\pgfqpoint{5.800000in}{14.155113in}}%
\pgfpathlineto{\pgfqpoint{5.800000in}{14.155113in}}%
\pgfpathlineto{\pgfqpoint{5.800000in}{14.272965in}}%
\pgfpathlineto{\pgfqpoint{5.800000in}{14.272965in}}%
\pgfpathclose%
\pgfusepath{fill}%
\end{pgfscope}%
\begin{pgfscope}%
\pgfpathrectangle{\pgfqpoint{5.800000in}{0.720000in}}{\pgfqpoint{1.400000in}{4.620000in}}%
\pgfusepath{clip}%
\pgfsetbuttcap%
\pgfsetmiterjoin%
\definecolor{currentfill}{rgb}{0.121569,0.466667,0.705882}%
\pgfsetfillcolor{currentfill}%
\pgfsetlinewidth{0.000000pt}%
\definecolor{currentstroke}{rgb}{0.000000,0.000000,0.000000}%
\pgfsetstrokecolor{currentstroke}%
\pgfsetstrokeopacity{0.000000}%
\pgfsetdash{}{0pt}%
\pgfpathmoveto{\pgfqpoint{5.800000in}{14.272965in}}%
\pgfpathlineto{\pgfqpoint{5.800000in}{14.272965in}}%
\pgfpathlineto{\pgfqpoint{5.800000in}{14.390817in}}%
\pgfpathlineto{\pgfqpoint{5.800000in}{14.390817in}}%
\pgfpathclose%
\pgfusepath{fill}%
\end{pgfscope}%
\begin{pgfscope}%
\pgfpathrectangle{\pgfqpoint{5.800000in}{0.720000in}}{\pgfqpoint{1.400000in}{4.620000in}}%
\pgfusepath{clip}%
\pgfsetbuttcap%
\pgfsetmiterjoin%
\definecolor{currentfill}{rgb}{0.121569,0.466667,0.705882}%
\pgfsetfillcolor{currentfill}%
\pgfsetlinewidth{0.000000pt}%
\definecolor{currentstroke}{rgb}{0.000000,0.000000,0.000000}%
\pgfsetstrokecolor{currentstroke}%
\pgfsetstrokeopacity{0.000000}%
\pgfsetdash{}{0pt}%
\pgfpathmoveto{\pgfqpoint{5.800000in}{14.390817in}}%
\pgfpathlineto{\pgfqpoint{5.800000in}{14.390817in}}%
\pgfpathlineto{\pgfqpoint{5.800000in}{14.508669in}}%
\pgfpathlineto{\pgfqpoint{5.800000in}{14.508669in}}%
\pgfpathclose%
\pgfusepath{fill}%
\end{pgfscope}%
\begin{pgfscope}%
\pgfpathrectangle{\pgfqpoint{5.800000in}{0.720000in}}{\pgfqpoint{1.400000in}{4.620000in}}%
\pgfusepath{clip}%
\pgfsetbuttcap%
\pgfsetmiterjoin%
\definecolor{currentfill}{rgb}{0.121569,0.466667,0.705882}%
\pgfsetfillcolor{currentfill}%
\pgfsetlinewidth{0.000000pt}%
\definecolor{currentstroke}{rgb}{0.000000,0.000000,0.000000}%
\pgfsetstrokecolor{currentstroke}%
\pgfsetstrokeopacity{0.000000}%
\pgfsetdash{}{0pt}%
\pgfpathmoveto{\pgfqpoint{5.800000in}{14.508669in}}%
\pgfpathlineto{\pgfqpoint{5.800000in}{14.508669in}}%
\pgfpathlineto{\pgfqpoint{5.800000in}{14.626521in}}%
\pgfpathlineto{\pgfqpoint{5.800000in}{14.626521in}}%
\pgfpathclose%
\pgfusepath{fill}%
\end{pgfscope}%
\begin{pgfscope}%
\pgfpathrectangle{\pgfqpoint{5.800000in}{0.720000in}}{\pgfqpoint{1.400000in}{4.620000in}}%
\pgfusepath{clip}%
\pgfsetbuttcap%
\pgfsetmiterjoin%
\definecolor{currentfill}{rgb}{0.121569,0.466667,0.705882}%
\pgfsetfillcolor{currentfill}%
\pgfsetlinewidth{0.000000pt}%
\definecolor{currentstroke}{rgb}{0.000000,0.000000,0.000000}%
\pgfsetstrokecolor{currentstroke}%
\pgfsetstrokeopacity{0.000000}%
\pgfsetdash{}{0pt}%
\pgfpathmoveto{\pgfqpoint{5.800000in}{14.626521in}}%
\pgfpathlineto{\pgfqpoint{5.800000in}{14.626521in}}%
\pgfpathlineto{\pgfqpoint{5.800000in}{14.744373in}}%
\pgfpathlineto{\pgfqpoint{5.800000in}{14.744373in}}%
\pgfpathclose%
\pgfusepath{fill}%
\end{pgfscope}%
\begin{pgfscope}%
\pgfpathrectangle{\pgfqpoint{5.800000in}{0.720000in}}{\pgfqpoint{1.400000in}{4.620000in}}%
\pgfusepath{clip}%
\pgfsetbuttcap%
\pgfsetmiterjoin%
\definecolor{currentfill}{rgb}{0.121569,0.466667,0.705882}%
\pgfsetfillcolor{currentfill}%
\pgfsetlinewidth{0.000000pt}%
\definecolor{currentstroke}{rgb}{0.000000,0.000000,0.000000}%
\pgfsetstrokecolor{currentstroke}%
\pgfsetstrokeopacity{0.000000}%
\pgfsetdash{}{0pt}%
\pgfpathmoveto{\pgfqpoint{5.800000in}{14.744373in}}%
\pgfpathlineto{\pgfqpoint{5.800000in}{14.744373in}}%
\pgfpathlineto{\pgfqpoint{5.800000in}{14.862224in}}%
\pgfpathlineto{\pgfqpoint{5.800000in}{14.862224in}}%
\pgfpathclose%
\pgfusepath{fill}%
\end{pgfscope}%
\begin{pgfscope}%
\pgfpathrectangle{\pgfqpoint{5.800000in}{0.720000in}}{\pgfqpoint{1.400000in}{4.620000in}}%
\pgfusepath{clip}%
\pgfsetbuttcap%
\pgfsetmiterjoin%
\definecolor{currentfill}{rgb}{0.121569,0.466667,0.705882}%
\pgfsetfillcolor{currentfill}%
\pgfsetlinewidth{0.000000pt}%
\definecolor{currentstroke}{rgb}{0.000000,0.000000,0.000000}%
\pgfsetstrokecolor{currentstroke}%
\pgfsetstrokeopacity{0.000000}%
\pgfsetdash{}{0pt}%
\pgfpathmoveto{\pgfqpoint{5.800000in}{14.862224in}}%
\pgfpathlineto{\pgfqpoint{5.800000in}{14.862224in}}%
\pgfpathlineto{\pgfqpoint{5.800000in}{14.980076in}}%
\pgfpathlineto{\pgfqpoint{5.800000in}{14.980076in}}%
\pgfpathclose%
\pgfusepath{fill}%
\end{pgfscope}%
\begin{pgfscope}%
\pgfpathrectangle{\pgfqpoint{5.800000in}{0.720000in}}{\pgfqpoint{1.400000in}{4.620000in}}%
\pgfusepath{clip}%
\pgfsetbuttcap%
\pgfsetmiterjoin%
\definecolor{currentfill}{rgb}{0.121569,0.466667,0.705882}%
\pgfsetfillcolor{currentfill}%
\pgfsetlinewidth{0.000000pt}%
\definecolor{currentstroke}{rgb}{0.000000,0.000000,0.000000}%
\pgfsetstrokecolor{currentstroke}%
\pgfsetstrokeopacity{0.000000}%
\pgfsetdash{}{0pt}%
\pgfpathmoveto{\pgfqpoint{5.800000in}{14.980076in}}%
\pgfpathlineto{\pgfqpoint{5.800000in}{14.980076in}}%
\pgfpathlineto{\pgfqpoint{5.800000in}{15.097928in}}%
\pgfpathlineto{\pgfqpoint{5.800000in}{15.097928in}}%
\pgfpathclose%
\pgfusepath{fill}%
\end{pgfscope}%
\begin{pgfscope}%
\pgfpathrectangle{\pgfqpoint{5.800000in}{0.720000in}}{\pgfqpoint{1.400000in}{4.620000in}}%
\pgfusepath{clip}%
\pgfsetbuttcap%
\pgfsetmiterjoin%
\definecolor{currentfill}{rgb}{0.121569,0.466667,0.705882}%
\pgfsetfillcolor{currentfill}%
\pgfsetlinewidth{0.000000pt}%
\definecolor{currentstroke}{rgb}{0.000000,0.000000,0.000000}%
\pgfsetstrokecolor{currentstroke}%
\pgfsetstrokeopacity{0.000000}%
\pgfsetdash{}{0pt}%
\pgfpathmoveto{\pgfqpoint{5.800000in}{15.097928in}}%
\pgfpathlineto{\pgfqpoint{5.800000in}{15.097928in}}%
\pgfpathlineto{\pgfqpoint{5.800000in}{15.215780in}}%
\pgfpathlineto{\pgfqpoint{5.800000in}{15.215780in}}%
\pgfpathclose%
\pgfusepath{fill}%
\end{pgfscope}%
\begin{pgfscope}%
\pgfpathrectangle{\pgfqpoint{5.800000in}{0.720000in}}{\pgfqpoint{1.400000in}{4.620000in}}%
\pgfusepath{clip}%
\pgfsetbuttcap%
\pgfsetmiterjoin%
\definecolor{currentfill}{rgb}{0.121569,0.466667,0.705882}%
\pgfsetfillcolor{currentfill}%
\pgfsetlinewidth{0.000000pt}%
\definecolor{currentstroke}{rgb}{0.000000,0.000000,0.000000}%
\pgfsetstrokecolor{currentstroke}%
\pgfsetstrokeopacity{0.000000}%
\pgfsetdash{}{0pt}%
\pgfpathmoveto{\pgfqpoint{5.800000in}{15.215780in}}%
\pgfpathlineto{\pgfqpoint{5.800000in}{15.215780in}}%
\pgfpathlineto{\pgfqpoint{5.800000in}{15.333632in}}%
\pgfpathlineto{\pgfqpoint{5.800000in}{15.333632in}}%
\pgfpathclose%
\pgfusepath{fill}%
\end{pgfscope}%
\begin{pgfscope}%
\pgfpathrectangle{\pgfqpoint{5.800000in}{0.720000in}}{\pgfqpoint{1.400000in}{4.620000in}}%
\pgfusepath{clip}%
\pgfsetbuttcap%
\pgfsetmiterjoin%
\definecolor{currentfill}{rgb}{0.121569,0.466667,0.705882}%
\pgfsetfillcolor{currentfill}%
\pgfsetlinewidth{0.000000pt}%
\definecolor{currentstroke}{rgb}{0.000000,0.000000,0.000000}%
\pgfsetstrokecolor{currentstroke}%
\pgfsetstrokeopacity{0.000000}%
\pgfsetdash{}{0pt}%
\pgfpathmoveto{\pgfqpoint{5.800000in}{15.333632in}}%
\pgfpathlineto{\pgfqpoint{5.802160in}{15.333632in}}%
\pgfpathlineto{\pgfqpoint{5.802160in}{15.451484in}}%
\pgfpathlineto{\pgfqpoint{5.800000in}{15.451484in}}%
\pgfpathclose%
\pgfusepath{fill}%
\end{pgfscope}%
\begin{pgfscope}%
\pgfpathrectangle{\pgfqpoint{5.800000in}{0.720000in}}{\pgfqpoint{1.400000in}{4.620000in}}%
\pgfusepath{clip}%
\pgfsetbuttcap%
\pgfsetmiterjoin%
\definecolor{currentfill}{rgb}{0.121569,0.466667,0.705882}%
\pgfsetfillcolor{currentfill}%
\pgfsetlinewidth{0.000000pt}%
\definecolor{currentstroke}{rgb}{0.000000,0.000000,0.000000}%
\pgfsetstrokecolor{currentstroke}%
\pgfsetstrokeopacity{0.000000}%
\pgfsetdash{}{0pt}%
\pgfpathmoveto{\pgfqpoint{5.800000in}{15.451484in}}%
\pgfpathlineto{\pgfqpoint{5.800000in}{15.451484in}}%
\pgfpathlineto{\pgfqpoint{5.800000in}{15.569336in}}%
\pgfpathlineto{\pgfqpoint{5.800000in}{15.569336in}}%
\pgfpathclose%
\pgfusepath{fill}%
\end{pgfscope}%
\begin{pgfscope}%
\pgfpathrectangle{\pgfqpoint{5.800000in}{0.720000in}}{\pgfqpoint{1.400000in}{4.620000in}}%
\pgfusepath{clip}%
\pgfsetbuttcap%
\pgfsetmiterjoin%
\definecolor{currentfill}{rgb}{0.121569,0.466667,0.705882}%
\pgfsetfillcolor{currentfill}%
\pgfsetlinewidth{0.000000pt}%
\definecolor{currentstroke}{rgb}{0.000000,0.000000,0.000000}%
\pgfsetstrokecolor{currentstroke}%
\pgfsetstrokeopacity{0.000000}%
\pgfsetdash{}{0pt}%
\pgfpathmoveto{\pgfqpoint{5.800000in}{15.569336in}}%
\pgfpathlineto{\pgfqpoint{5.800000in}{15.569336in}}%
\pgfpathlineto{\pgfqpoint{5.800000in}{15.687188in}}%
\pgfpathlineto{\pgfqpoint{5.800000in}{15.687188in}}%
\pgfpathclose%
\pgfusepath{fill}%
\end{pgfscope}%
\begin{pgfscope}%
\pgfpathrectangle{\pgfqpoint{5.800000in}{0.720000in}}{\pgfqpoint{1.400000in}{4.620000in}}%
\pgfusepath{clip}%
\pgfsetbuttcap%
\pgfsetmiterjoin%
\definecolor{currentfill}{rgb}{0.121569,0.466667,0.705882}%
\pgfsetfillcolor{currentfill}%
\pgfsetlinewidth{0.000000pt}%
\definecolor{currentstroke}{rgb}{0.000000,0.000000,0.000000}%
\pgfsetstrokecolor{currentstroke}%
\pgfsetstrokeopacity{0.000000}%
\pgfsetdash{}{0pt}%
\pgfpathmoveto{\pgfqpoint{5.800000in}{15.687188in}}%
\pgfpathlineto{\pgfqpoint{5.800000in}{15.687188in}}%
\pgfpathlineto{\pgfqpoint{5.800000in}{15.805039in}}%
\pgfpathlineto{\pgfqpoint{5.800000in}{15.805039in}}%
\pgfpathclose%
\pgfusepath{fill}%
\end{pgfscope}%
\begin{pgfscope}%
\pgfpathrectangle{\pgfqpoint{5.800000in}{0.720000in}}{\pgfqpoint{1.400000in}{4.620000in}}%
\pgfusepath{clip}%
\pgfsetbuttcap%
\pgfsetmiterjoin%
\definecolor{currentfill}{rgb}{0.121569,0.466667,0.705882}%
\pgfsetfillcolor{currentfill}%
\pgfsetlinewidth{0.000000pt}%
\definecolor{currentstroke}{rgb}{0.000000,0.000000,0.000000}%
\pgfsetstrokecolor{currentstroke}%
\pgfsetstrokeopacity{0.000000}%
\pgfsetdash{}{0pt}%
\pgfpathmoveto{\pgfqpoint{5.800000in}{15.805039in}}%
\pgfpathlineto{\pgfqpoint{5.800000in}{15.805039in}}%
\pgfpathlineto{\pgfqpoint{5.800000in}{15.922891in}}%
\pgfpathlineto{\pgfqpoint{5.800000in}{15.922891in}}%
\pgfpathclose%
\pgfusepath{fill}%
\end{pgfscope}%
\begin{pgfscope}%
\pgfpathrectangle{\pgfqpoint{5.800000in}{0.720000in}}{\pgfqpoint{1.400000in}{4.620000in}}%
\pgfusepath{clip}%
\pgfsetbuttcap%
\pgfsetmiterjoin%
\definecolor{currentfill}{rgb}{0.121569,0.466667,0.705882}%
\pgfsetfillcolor{currentfill}%
\pgfsetlinewidth{0.000000pt}%
\definecolor{currentstroke}{rgb}{0.000000,0.000000,0.000000}%
\pgfsetstrokecolor{currentstroke}%
\pgfsetstrokeopacity{0.000000}%
\pgfsetdash{}{0pt}%
\pgfpathmoveto{\pgfqpoint{5.800000in}{15.922891in}}%
\pgfpathlineto{\pgfqpoint{5.800000in}{15.922891in}}%
\pgfpathlineto{\pgfqpoint{5.800000in}{16.040743in}}%
\pgfpathlineto{\pgfqpoint{5.800000in}{16.040743in}}%
\pgfpathclose%
\pgfusepath{fill}%
\end{pgfscope}%
\begin{pgfscope}%
\pgfpathrectangle{\pgfqpoint{5.800000in}{0.720000in}}{\pgfqpoint{1.400000in}{4.620000in}}%
\pgfusepath{clip}%
\pgfsetbuttcap%
\pgfsetmiterjoin%
\definecolor{currentfill}{rgb}{0.121569,0.466667,0.705882}%
\pgfsetfillcolor{currentfill}%
\pgfsetlinewidth{0.000000pt}%
\definecolor{currentstroke}{rgb}{0.000000,0.000000,0.000000}%
\pgfsetstrokecolor{currentstroke}%
\pgfsetstrokeopacity{0.000000}%
\pgfsetdash{}{0pt}%
\pgfpathmoveto{\pgfqpoint{5.800000in}{16.040743in}}%
\pgfpathlineto{\pgfqpoint{5.802160in}{16.040743in}}%
\pgfpathlineto{\pgfqpoint{5.802160in}{16.158595in}}%
\pgfpathlineto{\pgfqpoint{5.800000in}{16.158595in}}%
\pgfpathclose%
\pgfusepath{fill}%
\end{pgfscope}%
\begin{pgfscope}%
\pgfpathrectangle{\pgfqpoint{5.800000in}{0.720000in}}{\pgfqpoint{1.400000in}{4.620000in}}%
\pgfusepath{clip}%
\pgfsetbuttcap%
\pgfsetmiterjoin%
\definecolor{currentfill}{rgb}{0.121569,0.466667,0.705882}%
\pgfsetfillcolor{currentfill}%
\pgfsetlinewidth{0.000000pt}%
\definecolor{currentstroke}{rgb}{0.000000,0.000000,0.000000}%
\pgfsetstrokecolor{currentstroke}%
\pgfsetstrokeopacity{0.000000}%
\pgfsetdash{}{0pt}%
\pgfpathmoveto{\pgfqpoint{5.800000in}{16.158595in}}%
\pgfpathlineto{\pgfqpoint{5.802160in}{16.158595in}}%
\pgfpathlineto{\pgfqpoint{5.802160in}{16.276447in}}%
\pgfpathlineto{\pgfqpoint{5.800000in}{16.276447in}}%
\pgfpathclose%
\pgfusepath{fill}%
\end{pgfscope}%
\begin{pgfscope}%
\pgfpathrectangle{\pgfqpoint{5.800000in}{0.720000in}}{\pgfqpoint{1.400000in}{4.620000in}}%
\pgfusepath{clip}%
\pgfsetbuttcap%
\pgfsetmiterjoin%
\definecolor{currentfill}{rgb}{0.121569,0.466667,0.705882}%
\pgfsetfillcolor{currentfill}%
\pgfsetlinewidth{0.000000pt}%
\definecolor{currentstroke}{rgb}{0.000000,0.000000,0.000000}%
\pgfsetstrokecolor{currentstroke}%
\pgfsetstrokeopacity{0.000000}%
\pgfsetdash{}{0pt}%
\pgfpathmoveto{\pgfqpoint{5.800000in}{16.276447in}}%
\pgfpathlineto{\pgfqpoint{5.802160in}{16.276447in}}%
\pgfpathlineto{\pgfqpoint{5.802160in}{16.394299in}}%
\pgfpathlineto{\pgfqpoint{5.800000in}{16.394299in}}%
\pgfpathclose%
\pgfusepath{fill}%
\end{pgfscope}%
\begin{pgfscope}%
\pgfpathrectangle{\pgfqpoint{5.800000in}{0.720000in}}{\pgfqpoint{1.400000in}{4.620000in}}%
\pgfusepath{clip}%
\pgfsetbuttcap%
\pgfsetmiterjoin%
\definecolor{currentfill}{rgb}{0.121569,0.466667,0.705882}%
\pgfsetfillcolor{currentfill}%
\pgfsetlinewidth{0.000000pt}%
\definecolor{currentstroke}{rgb}{0.000000,0.000000,0.000000}%
\pgfsetstrokecolor{currentstroke}%
\pgfsetstrokeopacity{0.000000}%
\pgfsetdash{}{0pt}%
\pgfpathmoveto{\pgfqpoint{5.800000in}{16.394299in}}%
\pgfpathlineto{\pgfqpoint{5.800000in}{16.394299in}}%
\pgfpathlineto{\pgfqpoint{5.800000in}{16.512151in}}%
\pgfpathlineto{\pgfqpoint{5.800000in}{16.512151in}}%
\pgfpathclose%
\pgfusepath{fill}%
\end{pgfscope}%
\begin{pgfscope}%
\pgfpathrectangle{\pgfqpoint{5.800000in}{0.720000in}}{\pgfqpoint{1.400000in}{4.620000in}}%
\pgfusepath{clip}%
\pgfsetbuttcap%
\pgfsetmiterjoin%
\definecolor{currentfill}{rgb}{0.121569,0.466667,0.705882}%
\pgfsetfillcolor{currentfill}%
\pgfsetlinewidth{0.000000pt}%
\definecolor{currentstroke}{rgb}{0.000000,0.000000,0.000000}%
\pgfsetstrokecolor{currentstroke}%
\pgfsetstrokeopacity{0.000000}%
\pgfsetdash{}{0pt}%
\pgfpathmoveto{\pgfqpoint{5.800000in}{16.512151in}}%
\pgfpathlineto{\pgfqpoint{5.800000in}{16.512151in}}%
\pgfpathlineto{\pgfqpoint{5.800000in}{16.630002in}}%
\pgfpathlineto{\pgfqpoint{5.800000in}{16.630002in}}%
\pgfpathclose%
\pgfusepath{fill}%
\end{pgfscope}%
\begin{pgfscope}%
\pgfpathrectangle{\pgfqpoint{5.800000in}{0.720000in}}{\pgfqpoint{1.400000in}{4.620000in}}%
\pgfusepath{clip}%
\pgfsetbuttcap%
\pgfsetmiterjoin%
\definecolor{currentfill}{rgb}{0.121569,0.466667,0.705882}%
\pgfsetfillcolor{currentfill}%
\pgfsetlinewidth{0.000000pt}%
\definecolor{currentstroke}{rgb}{0.000000,0.000000,0.000000}%
\pgfsetstrokecolor{currentstroke}%
\pgfsetstrokeopacity{0.000000}%
\pgfsetdash{}{0pt}%
\pgfpathmoveto{\pgfqpoint{5.800000in}{16.630002in}}%
\pgfpathlineto{\pgfqpoint{5.800000in}{16.630002in}}%
\pgfpathlineto{\pgfqpoint{5.800000in}{16.747854in}}%
\pgfpathlineto{\pgfqpoint{5.800000in}{16.747854in}}%
\pgfpathclose%
\pgfusepath{fill}%
\end{pgfscope}%
\begin{pgfscope}%
\pgfpathrectangle{\pgfqpoint{5.800000in}{0.720000in}}{\pgfqpoint{1.400000in}{4.620000in}}%
\pgfusepath{clip}%
\pgfsetbuttcap%
\pgfsetmiterjoin%
\definecolor{currentfill}{rgb}{0.121569,0.466667,0.705882}%
\pgfsetfillcolor{currentfill}%
\pgfsetlinewidth{0.000000pt}%
\definecolor{currentstroke}{rgb}{0.000000,0.000000,0.000000}%
\pgfsetstrokecolor{currentstroke}%
\pgfsetstrokeopacity{0.000000}%
\pgfsetdash{}{0pt}%
\pgfpathmoveto{\pgfqpoint{5.800000in}{16.747854in}}%
\pgfpathlineto{\pgfqpoint{5.800000in}{16.747854in}}%
\pgfpathlineto{\pgfqpoint{5.800000in}{16.865706in}}%
\pgfpathlineto{\pgfqpoint{5.800000in}{16.865706in}}%
\pgfpathclose%
\pgfusepath{fill}%
\end{pgfscope}%
\begin{pgfscope}%
\pgfpathrectangle{\pgfqpoint{5.800000in}{0.720000in}}{\pgfqpoint{1.400000in}{4.620000in}}%
\pgfusepath{clip}%
\pgfsetbuttcap%
\pgfsetmiterjoin%
\definecolor{currentfill}{rgb}{0.121569,0.466667,0.705882}%
\pgfsetfillcolor{currentfill}%
\pgfsetlinewidth{0.000000pt}%
\definecolor{currentstroke}{rgb}{0.000000,0.000000,0.000000}%
\pgfsetstrokecolor{currentstroke}%
\pgfsetstrokeopacity{0.000000}%
\pgfsetdash{}{0pt}%
\pgfpathmoveto{\pgfqpoint{5.800000in}{16.865706in}}%
\pgfpathlineto{\pgfqpoint{5.800000in}{16.865706in}}%
\pgfpathlineto{\pgfqpoint{5.800000in}{16.983558in}}%
\pgfpathlineto{\pgfqpoint{5.800000in}{16.983558in}}%
\pgfpathclose%
\pgfusepath{fill}%
\end{pgfscope}%
\begin{pgfscope}%
\pgfpathrectangle{\pgfqpoint{5.800000in}{0.720000in}}{\pgfqpoint{1.400000in}{4.620000in}}%
\pgfusepath{clip}%
\pgfsetbuttcap%
\pgfsetmiterjoin%
\definecolor{currentfill}{rgb}{0.121569,0.466667,0.705882}%
\pgfsetfillcolor{currentfill}%
\pgfsetlinewidth{0.000000pt}%
\definecolor{currentstroke}{rgb}{0.000000,0.000000,0.000000}%
\pgfsetstrokecolor{currentstroke}%
\pgfsetstrokeopacity{0.000000}%
\pgfsetdash{}{0pt}%
\pgfpathmoveto{\pgfqpoint{5.800000in}{16.983558in}}%
\pgfpathlineto{\pgfqpoint{5.802160in}{16.983558in}}%
\pgfpathlineto{\pgfqpoint{5.802160in}{17.101410in}}%
\pgfpathlineto{\pgfqpoint{5.800000in}{17.101410in}}%
\pgfpathclose%
\pgfusepath{fill}%
\end{pgfscope}%
\begin{pgfscope}%
\pgfpathrectangle{\pgfqpoint{5.800000in}{0.720000in}}{\pgfqpoint{1.400000in}{4.620000in}}%
\pgfusepath{clip}%
\pgfsetbuttcap%
\pgfsetmiterjoin%
\definecolor{currentfill}{rgb}{0.121569,0.466667,0.705882}%
\pgfsetfillcolor{currentfill}%
\pgfsetlinewidth{0.000000pt}%
\definecolor{currentstroke}{rgb}{0.000000,0.000000,0.000000}%
\pgfsetstrokecolor{currentstroke}%
\pgfsetstrokeopacity{0.000000}%
\pgfsetdash{}{0pt}%
\pgfpathmoveto{\pgfqpoint{5.800000in}{17.101410in}}%
\pgfpathlineto{\pgfqpoint{5.800000in}{17.101410in}}%
\pgfpathlineto{\pgfqpoint{5.800000in}{17.219262in}}%
\pgfpathlineto{\pgfqpoint{5.800000in}{17.219262in}}%
\pgfpathclose%
\pgfusepath{fill}%
\end{pgfscope}%
\begin{pgfscope}%
\pgfpathrectangle{\pgfqpoint{5.800000in}{0.720000in}}{\pgfqpoint{1.400000in}{4.620000in}}%
\pgfusepath{clip}%
\pgfsetbuttcap%
\pgfsetmiterjoin%
\definecolor{currentfill}{rgb}{0.121569,0.466667,0.705882}%
\pgfsetfillcolor{currentfill}%
\pgfsetlinewidth{0.000000pt}%
\definecolor{currentstroke}{rgb}{0.000000,0.000000,0.000000}%
\pgfsetstrokecolor{currentstroke}%
\pgfsetstrokeopacity{0.000000}%
\pgfsetdash{}{0pt}%
\pgfpathmoveto{\pgfqpoint{5.800000in}{17.219262in}}%
\pgfpathlineto{\pgfqpoint{5.800000in}{17.219262in}}%
\pgfpathlineto{\pgfqpoint{5.800000in}{17.337114in}}%
\pgfpathlineto{\pgfqpoint{5.800000in}{17.337114in}}%
\pgfpathclose%
\pgfusepath{fill}%
\end{pgfscope}%
\begin{pgfscope}%
\pgfpathrectangle{\pgfqpoint{5.800000in}{0.720000in}}{\pgfqpoint{1.400000in}{4.620000in}}%
\pgfusepath{clip}%
\pgfsetbuttcap%
\pgfsetmiterjoin%
\definecolor{currentfill}{rgb}{0.121569,0.466667,0.705882}%
\pgfsetfillcolor{currentfill}%
\pgfsetlinewidth{0.000000pt}%
\definecolor{currentstroke}{rgb}{0.000000,0.000000,0.000000}%
\pgfsetstrokecolor{currentstroke}%
\pgfsetstrokeopacity{0.000000}%
\pgfsetdash{}{0pt}%
\pgfpathmoveto{\pgfqpoint{5.800000in}{17.337114in}}%
\pgfpathlineto{\pgfqpoint{5.800000in}{17.337114in}}%
\pgfpathlineto{\pgfqpoint{5.800000in}{17.454966in}}%
\pgfpathlineto{\pgfqpoint{5.800000in}{17.454966in}}%
\pgfpathclose%
\pgfusepath{fill}%
\end{pgfscope}%
\begin{pgfscope}%
\pgfpathrectangle{\pgfqpoint{5.800000in}{0.720000in}}{\pgfqpoint{1.400000in}{4.620000in}}%
\pgfusepath{clip}%
\pgfsetbuttcap%
\pgfsetmiterjoin%
\definecolor{currentfill}{rgb}{0.121569,0.466667,0.705882}%
\pgfsetfillcolor{currentfill}%
\pgfsetlinewidth{0.000000pt}%
\definecolor{currentstroke}{rgb}{0.000000,0.000000,0.000000}%
\pgfsetstrokecolor{currentstroke}%
\pgfsetstrokeopacity{0.000000}%
\pgfsetdash{}{0pt}%
\pgfpathmoveto{\pgfqpoint{5.800000in}{17.454966in}}%
\pgfpathlineto{\pgfqpoint{5.800000in}{17.454966in}}%
\pgfpathlineto{\pgfqpoint{5.800000in}{17.572817in}}%
\pgfpathlineto{\pgfqpoint{5.800000in}{17.572817in}}%
\pgfpathclose%
\pgfusepath{fill}%
\end{pgfscope}%
\begin{pgfscope}%
\pgfpathrectangle{\pgfqpoint{5.800000in}{0.720000in}}{\pgfqpoint{1.400000in}{4.620000in}}%
\pgfusepath{clip}%
\pgfsetbuttcap%
\pgfsetmiterjoin%
\definecolor{currentfill}{rgb}{0.121569,0.466667,0.705882}%
\pgfsetfillcolor{currentfill}%
\pgfsetlinewidth{0.000000pt}%
\definecolor{currentstroke}{rgb}{0.000000,0.000000,0.000000}%
\pgfsetstrokecolor{currentstroke}%
\pgfsetstrokeopacity{0.000000}%
\pgfsetdash{}{0pt}%
\pgfpathmoveto{\pgfqpoint{5.800000in}{17.572817in}}%
\pgfpathlineto{\pgfqpoint{5.804319in}{17.572817in}}%
\pgfpathlineto{\pgfqpoint{5.804319in}{17.690669in}}%
\pgfpathlineto{\pgfqpoint{5.800000in}{17.690669in}}%
\pgfpathclose%
\pgfusepath{fill}%
\end{pgfscope}%
\begin{pgfscope}%
\pgfpathrectangle{\pgfqpoint{5.800000in}{0.720000in}}{\pgfqpoint{1.400000in}{4.620000in}}%
\pgfusepath{clip}%
\pgfsetbuttcap%
\pgfsetmiterjoin%
\definecolor{currentfill}{rgb}{0.121569,0.466667,0.705882}%
\pgfsetfillcolor{currentfill}%
\pgfsetlinewidth{0.000000pt}%
\definecolor{currentstroke}{rgb}{0.000000,0.000000,0.000000}%
\pgfsetstrokecolor{currentstroke}%
\pgfsetstrokeopacity{0.000000}%
\pgfsetdash{}{0pt}%
\pgfpathmoveto{\pgfqpoint{5.800000in}{17.690669in}}%
\pgfpathlineto{\pgfqpoint{5.800000in}{17.690669in}}%
\pgfpathlineto{\pgfqpoint{5.800000in}{17.808521in}}%
\pgfpathlineto{\pgfqpoint{5.800000in}{17.808521in}}%
\pgfpathclose%
\pgfusepath{fill}%
\end{pgfscope}%
\begin{pgfscope}%
\pgfpathrectangle{\pgfqpoint{5.800000in}{0.720000in}}{\pgfqpoint{1.400000in}{4.620000in}}%
\pgfusepath{clip}%
\pgfsetbuttcap%
\pgfsetmiterjoin%
\definecolor{currentfill}{rgb}{0.121569,0.466667,0.705882}%
\pgfsetfillcolor{currentfill}%
\pgfsetlinewidth{0.000000pt}%
\definecolor{currentstroke}{rgb}{0.000000,0.000000,0.000000}%
\pgfsetstrokecolor{currentstroke}%
\pgfsetstrokeopacity{0.000000}%
\pgfsetdash{}{0pt}%
\pgfpathmoveto{\pgfqpoint{5.800000in}{17.808521in}}%
\pgfpathlineto{\pgfqpoint{5.800000in}{17.808521in}}%
\pgfpathlineto{\pgfqpoint{5.800000in}{17.926373in}}%
\pgfpathlineto{\pgfqpoint{5.800000in}{17.926373in}}%
\pgfpathclose%
\pgfusepath{fill}%
\end{pgfscope}%
\begin{pgfscope}%
\pgfpathrectangle{\pgfqpoint{5.800000in}{0.720000in}}{\pgfqpoint{1.400000in}{4.620000in}}%
\pgfusepath{clip}%
\pgfsetbuttcap%
\pgfsetmiterjoin%
\definecolor{currentfill}{rgb}{0.121569,0.466667,0.705882}%
\pgfsetfillcolor{currentfill}%
\pgfsetlinewidth{0.000000pt}%
\definecolor{currentstroke}{rgb}{0.000000,0.000000,0.000000}%
\pgfsetstrokecolor{currentstroke}%
\pgfsetstrokeopacity{0.000000}%
\pgfsetdash{}{0pt}%
\pgfpathmoveto{\pgfqpoint{5.800000in}{17.926373in}}%
\pgfpathlineto{\pgfqpoint{5.800000in}{17.926373in}}%
\pgfpathlineto{\pgfqpoint{5.800000in}{18.044225in}}%
\pgfpathlineto{\pgfqpoint{5.800000in}{18.044225in}}%
\pgfpathclose%
\pgfusepath{fill}%
\end{pgfscope}%
\begin{pgfscope}%
\pgfpathrectangle{\pgfqpoint{5.800000in}{0.720000in}}{\pgfqpoint{1.400000in}{4.620000in}}%
\pgfusepath{clip}%
\pgfsetbuttcap%
\pgfsetmiterjoin%
\definecolor{currentfill}{rgb}{0.121569,0.466667,0.705882}%
\pgfsetfillcolor{currentfill}%
\pgfsetlinewidth{0.000000pt}%
\definecolor{currentstroke}{rgb}{0.000000,0.000000,0.000000}%
\pgfsetstrokecolor{currentstroke}%
\pgfsetstrokeopacity{0.000000}%
\pgfsetdash{}{0pt}%
\pgfpathmoveto{\pgfqpoint{5.800000in}{18.044225in}}%
\pgfpathlineto{\pgfqpoint{5.800000in}{18.044225in}}%
\pgfpathlineto{\pgfqpoint{5.800000in}{18.162077in}}%
\pgfpathlineto{\pgfqpoint{5.800000in}{18.162077in}}%
\pgfpathclose%
\pgfusepath{fill}%
\end{pgfscope}%
\begin{pgfscope}%
\pgfpathrectangle{\pgfqpoint{5.800000in}{0.720000in}}{\pgfqpoint{1.400000in}{4.620000in}}%
\pgfusepath{clip}%
\pgfsetbuttcap%
\pgfsetmiterjoin%
\definecolor{currentfill}{rgb}{0.121569,0.466667,0.705882}%
\pgfsetfillcolor{currentfill}%
\pgfsetlinewidth{0.000000pt}%
\definecolor{currentstroke}{rgb}{0.000000,0.000000,0.000000}%
\pgfsetstrokecolor{currentstroke}%
\pgfsetstrokeopacity{0.000000}%
\pgfsetdash{}{0pt}%
\pgfpathmoveto{\pgfqpoint{5.800000in}{18.162077in}}%
\pgfpathlineto{\pgfqpoint{5.800000in}{18.162077in}}%
\pgfpathlineto{\pgfqpoint{5.800000in}{18.279929in}}%
\pgfpathlineto{\pgfqpoint{5.800000in}{18.279929in}}%
\pgfpathclose%
\pgfusepath{fill}%
\end{pgfscope}%
\begin{pgfscope}%
\pgfpathrectangle{\pgfqpoint{5.800000in}{0.720000in}}{\pgfqpoint{1.400000in}{4.620000in}}%
\pgfusepath{clip}%
\pgfsetbuttcap%
\pgfsetmiterjoin%
\definecolor{currentfill}{rgb}{0.121569,0.466667,0.705882}%
\pgfsetfillcolor{currentfill}%
\pgfsetlinewidth{0.000000pt}%
\definecolor{currentstroke}{rgb}{0.000000,0.000000,0.000000}%
\pgfsetstrokecolor{currentstroke}%
\pgfsetstrokeopacity{0.000000}%
\pgfsetdash{}{0pt}%
\pgfpathmoveto{\pgfqpoint{5.800000in}{18.279929in}}%
\pgfpathlineto{\pgfqpoint{5.800000in}{18.279929in}}%
\pgfpathlineto{\pgfqpoint{5.800000in}{18.397781in}}%
\pgfpathlineto{\pgfqpoint{5.800000in}{18.397781in}}%
\pgfpathclose%
\pgfusepath{fill}%
\end{pgfscope}%
\begin{pgfscope}%
\pgfpathrectangle{\pgfqpoint{5.800000in}{0.720000in}}{\pgfqpoint{1.400000in}{4.620000in}}%
\pgfusepath{clip}%
\pgfsetbuttcap%
\pgfsetmiterjoin%
\definecolor{currentfill}{rgb}{0.121569,0.466667,0.705882}%
\pgfsetfillcolor{currentfill}%
\pgfsetlinewidth{0.000000pt}%
\definecolor{currentstroke}{rgb}{0.000000,0.000000,0.000000}%
\pgfsetstrokecolor{currentstroke}%
\pgfsetstrokeopacity{0.000000}%
\pgfsetdash{}{0pt}%
\pgfpathmoveto{\pgfqpoint{5.800000in}{18.397781in}}%
\pgfpathlineto{\pgfqpoint{5.800000in}{18.397781in}}%
\pgfpathlineto{\pgfqpoint{5.800000in}{18.515632in}}%
\pgfpathlineto{\pgfqpoint{5.800000in}{18.515632in}}%
\pgfpathclose%
\pgfusepath{fill}%
\end{pgfscope}%
\begin{pgfscope}%
\pgfpathrectangle{\pgfqpoint{5.800000in}{0.720000in}}{\pgfqpoint{1.400000in}{4.620000in}}%
\pgfusepath{clip}%
\pgfsetbuttcap%
\pgfsetmiterjoin%
\definecolor{currentfill}{rgb}{0.121569,0.466667,0.705882}%
\pgfsetfillcolor{currentfill}%
\pgfsetlinewidth{0.000000pt}%
\definecolor{currentstroke}{rgb}{0.000000,0.000000,0.000000}%
\pgfsetstrokecolor{currentstroke}%
\pgfsetstrokeopacity{0.000000}%
\pgfsetdash{}{0pt}%
\pgfpathmoveto{\pgfqpoint{5.800000in}{18.515632in}}%
\pgfpathlineto{\pgfqpoint{5.800000in}{18.515632in}}%
\pgfpathlineto{\pgfqpoint{5.800000in}{18.633484in}}%
\pgfpathlineto{\pgfqpoint{5.800000in}{18.633484in}}%
\pgfpathclose%
\pgfusepath{fill}%
\end{pgfscope}%
\begin{pgfscope}%
\pgfpathrectangle{\pgfqpoint{5.800000in}{0.720000in}}{\pgfqpoint{1.400000in}{4.620000in}}%
\pgfusepath{clip}%
\pgfsetbuttcap%
\pgfsetmiterjoin%
\definecolor{currentfill}{rgb}{0.121569,0.466667,0.705882}%
\pgfsetfillcolor{currentfill}%
\pgfsetlinewidth{0.000000pt}%
\definecolor{currentstroke}{rgb}{0.000000,0.000000,0.000000}%
\pgfsetstrokecolor{currentstroke}%
\pgfsetstrokeopacity{0.000000}%
\pgfsetdash{}{0pt}%
\pgfpathmoveto{\pgfqpoint{5.800000in}{18.633484in}}%
\pgfpathlineto{\pgfqpoint{5.800000in}{18.633484in}}%
\pgfpathlineto{\pgfqpoint{5.800000in}{18.751336in}}%
\pgfpathlineto{\pgfqpoint{5.800000in}{18.751336in}}%
\pgfpathclose%
\pgfusepath{fill}%
\end{pgfscope}%
\begin{pgfscope}%
\pgfpathrectangle{\pgfqpoint{5.800000in}{0.720000in}}{\pgfqpoint{1.400000in}{4.620000in}}%
\pgfusepath{clip}%
\pgfsetbuttcap%
\pgfsetmiterjoin%
\definecolor{currentfill}{rgb}{0.121569,0.466667,0.705882}%
\pgfsetfillcolor{currentfill}%
\pgfsetlinewidth{0.000000pt}%
\definecolor{currentstroke}{rgb}{0.000000,0.000000,0.000000}%
\pgfsetstrokecolor{currentstroke}%
\pgfsetstrokeopacity{0.000000}%
\pgfsetdash{}{0pt}%
\pgfpathmoveto{\pgfqpoint{5.800000in}{18.751336in}}%
\pgfpathlineto{\pgfqpoint{5.800000in}{18.751336in}}%
\pgfpathlineto{\pgfqpoint{5.800000in}{18.869188in}}%
\pgfpathlineto{\pgfqpoint{5.800000in}{18.869188in}}%
\pgfpathclose%
\pgfusepath{fill}%
\end{pgfscope}%
\begin{pgfscope}%
\pgfpathrectangle{\pgfqpoint{5.800000in}{0.720000in}}{\pgfqpoint{1.400000in}{4.620000in}}%
\pgfusepath{clip}%
\pgfsetbuttcap%
\pgfsetmiterjoin%
\definecolor{currentfill}{rgb}{0.121569,0.466667,0.705882}%
\pgfsetfillcolor{currentfill}%
\pgfsetlinewidth{0.000000pt}%
\definecolor{currentstroke}{rgb}{0.000000,0.000000,0.000000}%
\pgfsetstrokecolor{currentstroke}%
\pgfsetstrokeopacity{0.000000}%
\pgfsetdash{}{0pt}%
\pgfpathmoveto{\pgfqpoint{5.800000in}{18.869188in}}%
\pgfpathlineto{\pgfqpoint{5.800000in}{18.869188in}}%
\pgfpathlineto{\pgfqpoint{5.800000in}{18.987040in}}%
\pgfpathlineto{\pgfqpoint{5.800000in}{18.987040in}}%
\pgfpathclose%
\pgfusepath{fill}%
\end{pgfscope}%
\begin{pgfscope}%
\pgfpathrectangle{\pgfqpoint{5.800000in}{0.720000in}}{\pgfqpoint{1.400000in}{4.620000in}}%
\pgfusepath{clip}%
\pgfsetbuttcap%
\pgfsetmiterjoin%
\definecolor{currentfill}{rgb}{0.121569,0.466667,0.705882}%
\pgfsetfillcolor{currentfill}%
\pgfsetlinewidth{0.000000pt}%
\definecolor{currentstroke}{rgb}{0.000000,0.000000,0.000000}%
\pgfsetstrokecolor{currentstroke}%
\pgfsetstrokeopacity{0.000000}%
\pgfsetdash{}{0pt}%
\pgfpathmoveto{\pgfqpoint{5.800000in}{18.987040in}}%
\pgfpathlineto{\pgfqpoint{5.800000in}{18.987040in}}%
\pgfpathlineto{\pgfqpoint{5.800000in}{19.104892in}}%
\pgfpathlineto{\pgfqpoint{5.800000in}{19.104892in}}%
\pgfpathclose%
\pgfusepath{fill}%
\end{pgfscope}%
\begin{pgfscope}%
\pgfpathrectangle{\pgfqpoint{5.800000in}{0.720000in}}{\pgfqpoint{1.400000in}{4.620000in}}%
\pgfusepath{clip}%
\pgfsetbuttcap%
\pgfsetmiterjoin%
\definecolor{currentfill}{rgb}{0.121569,0.466667,0.705882}%
\pgfsetfillcolor{currentfill}%
\pgfsetlinewidth{0.000000pt}%
\definecolor{currentstroke}{rgb}{0.000000,0.000000,0.000000}%
\pgfsetstrokecolor{currentstroke}%
\pgfsetstrokeopacity{0.000000}%
\pgfsetdash{}{0pt}%
\pgfpathmoveto{\pgfqpoint{5.800000in}{19.104892in}}%
\pgfpathlineto{\pgfqpoint{5.800000in}{19.104892in}}%
\pgfpathlineto{\pgfqpoint{5.800000in}{19.222744in}}%
\pgfpathlineto{\pgfqpoint{5.800000in}{19.222744in}}%
\pgfpathclose%
\pgfusepath{fill}%
\end{pgfscope}%
\begin{pgfscope}%
\pgfpathrectangle{\pgfqpoint{5.800000in}{0.720000in}}{\pgfqpoint{1.400000in}{4.620000in}}%
\pgfusepath{clip}%
\pgfsetbuttcap%
\pgfsetmiterjoin%
\definecolor{currentfill}{rgb}{0.121569,0.466667,0.705882}%
\pgfsetfillcolor{currentfill}%
\pgfsetlinewidth{0.000000pt}%
\definecolor{currentstroke}{rgb}{0.000000,0.000000,0.000000}%
\pgfsetstrokecolor{currentstroke}%
\pgfsetstrokeopacity{0.000000}%
\pgfsetdash{}{0pt}%
\pgfpathmoveto{\pgfqpoint{5.800000in}{19.222744in}}%
\pgfpathlineto{\pgfqpoint{5.800000in}{19.222744in}}%
\pgfpathlineto{\pgfqpoint{5.800000in}{19.340596in}}%
\pgfpathlineto{\pgfqpoint{5.800000in}{19.340596in}}%
\pgfpathclose%
\pgfusepath{fill}%
\end{pgfscope}%
\begin{pgfscope}%
\pgfpathrectangle{\pgfqpoint{5.800000in}{0.720000in}}{\pgfqpoint{1.400000in}{4.620000in}}%
\pgfusepath{clip}%
\pgfsetbuttcap%
\pgfsetmiterjoin%
\definecolor{currentfill}{rgb}{0.121569,0.466667,0.705882}%
\pgfsetfillcolor{currentfill}%
\pgfsetlinewidth{0.000000pt}%
\definecolor{currentstroke}{rgb}{0.000000,0.000000,0.000000}%
\pgfsetstrokecolor{currentstroke}%
\pgfsetstrokeopacity{0.000000}%
\pgfsetdash{}{0pt}%
\pgfpathmoveto{\pgfqpoint{5.800000in}{19.340596in}}%
\pgfpathlineto{\pgfqpoint{5.800000in}{19.340596in}}%
\pgfpathlineto{\pgfqpoint{5.800000in}{19.458447in}}%
\pgfpathlineto{\pgfqpoint{5.800000in}{19.458447in}}%
\pgfpathclose%
\pgfusepath{fill}%
\end{pgfscope}%
\begin{pgfscope}%
\pgfpathrectangle{\pgfqpoint{5.800000in}{0.720000in}}{\pgfqpoint{1.400000in}{4.620000in}}%
\pgfusepath{clip}%
\pgfsetbuttcap%
\pgfsetmiterjoin%
\definecolor{currentfill}{rgb}{0.121569,0.466667,0.705882}%
\pgfsetfillcolor{currentfill}%
\pgfsetlinewidth{0.000000pt}%
\definecolor{currentstroke}{rgb}{0.000000,0.000000,0.000000}%
\pgfsetstrokecolor{currentstroke}%
\pgfsetstrokeopacity{0.000000}%
\pgfsetdash{}{0pt}%
\pgfpathmoveto{\pgfqpoint{5.800000in}{19.458447in}}%
\pgfpathlineto{\pgfqpoint{5.802160in}{19.458447in}}%
\pgfpathlineto{\pgfqpoint{5.802160in}{19.576299in}}%
\pgfpathlineto{\pgfqpoint{5.800000in}{19.576299in}}%
\pgfpathclose%
\pgfusepath{fill}%
\end{pgfscope}%
\begin{pgfscope}%
\pgfpathrectangle{\pgfqpoint{5.800000in}{0.720000in}}{\pgfqpoint{1.400000in}{4.620000in}}%
\pgfusepath{clip}%
\pgfsetbuttcap%
\pgfsetmiterjoin%
\definecolor{currentfill}{rgb}{0.121569,0.466667,0.705882}%
\pgfsetfillcolor{currentfill}%
\pgfsetlinewidth{0.000000pt}%
\definecolor{currentstroke}{rgb}{0.000000,0.000000,0.000000}%
\pgfsetstrokecolor{currentstroke}%
\pgfsetstrokeopacity{0.000000}%
\pgfsetdash{}{0pt}%
\pgfpathmoveto{\pgfqpoint{5.800000in}{19.576299in}}%
\pgfpathlineto{\pgfqpoint{5.802160in}{19.576299in}}%
\pgfpathlineto{\pgfqpoint{5.802160in}{19.694151in}}%
\pgfpathlineto{\pgfqpoint{5.800000in}{19.694151in}}%
\pgfpathclose%
\pgfusepath{fill}%
\end{pgfscope}%
\begin{pgfscope}%
\pgfpathrectangle{\pgfqpoint{5.800000in}{0.720000in}}{\pgfqpoint{1.400000in}{4.620000in}}%
\pgfusepath{clip}%
\pgfsetbuttcap%
\pgfsetmiterjoin%
\definecolor{currentfill}{rgb}{0.121569,0.466667,0.705882}%
\pgfsetfillcolor{currentfill}%
\pgfsetlinewidth{0.000000pt}%
\definecolor{currentstroke}{rgb}{0.000000,0.000000,0.000000}%
\pgfsetstrokecolor{currentstroke}%
\pgfsetstrokeopacity{0.000000}%
\pgfsetdash{}{0pt}%
\pgfpathmoveto{\pgfqpoint{5.800000in}{19.694151in}}%
\pgfpathlineto{\pgfqpoint{5.800000in}{19.694151in}}%
\pgfpathlineto{\pgfqpoint{5.800000in}{19.812003in}}%
\pgfpathlineto{\pgfqpoint{5.800000in}{19.812003in}}%
\pgfpathclose%
\pgfusepath{fill}%
\end{pgfscope}%
\begin{pgfscope}%
\pgfpathrectangle{\pgfqpoint{5.800000in}{0.720000in}}{\pgfqpoint{1.400000in}{4.620000in}}%
\pgfusepath{clip}%
\pgfsetbuttcap%
\pgfsetmiterjoin%
\definecolor{currentfill}{rgb}{0.121569,0.466667,0.705882}%
\pgfsetfillcolor{currentfill}%
\pgfsetlinewidth{0.000000pt}%
\definecolor{currentstroke}{rgb}{0.000000,0.000000,0.000000}%
\pgfsetstrokecolor{currentstroke}%
\pgfsetstrokeopacity{0.000000}%
\pgfsetdash{}{0pt}%
\pgfpathmoveto{\pgfqpoint{5.800000in}{19.812003in}}%
\pgfpathlineto{\pgfqpoint{5.800000in}{19.812003in}}%
\pgfpathlineto{\pgfqpoint{5.800000in}{19.929855in}}%
\pgfpathlineto{\pgfqpoint{5.800000in}{19.929855in}}%
\pgfpathclose%
\pgfusepath{fill}%
\end{pgfscope}%
\begin{pgfscope}%
\pgfpathrectangle{\pgfqpoint{5.800000in}{0.720000in}}{\pgfqpoint{1.400000in}{4.620000in}}%
\pgfusepath{clip}%
\pgfsetbuttcap%
\pgfsetmiterjoin%
\definecolor{currentfill}{rgb}{0.121569,0.466667,0.705882}%
\pgfsetfillcolor{currentfill}%
\pgfsetlinewidth{0.000000pt}%
\definecolor{currentstroke}{rgb}{0.000000,0.000000,0.000000}%
\pgfsetstrokecolor{currentstroke}%
\pgfsetstrokeopacity{0.000000}%
\pgfsetdash{}{0pt}%
\pgfpathmoveto{\pgfqpoint{5.800000in}{19.929855in}}%
\pgfpathlineto{\pgfqpoint{5.800000in}{19.929855in}}%
\pgfpathlineto{\pgfqpoint{5.800000in}{20.047707in}}%
\pgfpathlineto{\pgfqpoint{5.800000in}{20.047707in}}%
\pgfpathclose%
\pgfusepath{fill}%
\end{pgfscope}%
\begin{pgfscope}%
\pgfpathrectangle{\pgfqpoint{5.800000in}{0.720000in}}{\pgfqpoint{1.400000in}{4.620000in}}%
\pgfusepath{clip}%
\pgfsetbuttcap%
\pgfsetmiterjoin%
\definecolor{currentfill}{rgb}{0.121569,0.466667,0.705882}%
\pgfsetfillcolor{currentfill}%
\pgfsetlinewidth{0.000000pt}%
\definecolor{currentstroke}{rgb}{0.000000,0.000000,0.000000}%
\pgfsetstrokecolor{currentstroke}%
\pgfsetstrokeopacity{0.000000}%
\pgfsetdash{}{0pt}%
\pgfpathmoveto{\pgfqpoint{5.800000in}{20.047707in}}%
\pgfpathlineto{\pgfqpoint{5.800000in}{20.047707in}}%
\pgfpathlineto{\pgfqpoint{5.800000in}{20.165559in}}%
\pgfpathlineto{\pgfqpoint{5.800000in}{20.165559in}}%
\pgfpathclose%
\pgfusepath{fill}%
\end{pgfscope}%
\begin{pgfscope}%
\pgfpathrectangle{\pgfqpoint{5.800000in}{0.720000in}}{\pgfqpoint{1.400000in}{4.620000in}}%
\pgfusepath{clip}%
\pgfsetbuttcap%
\pgfsetmiterjoin%
\definecolor{currentfill}{rgb}{0.121569,0.466667,0.705882}%
\pgfsetfillcolor{currentfill}%
\pgfsetlinewidth{0.000000pt}%
\definecolor{currentstroke}{rgb}{0.000000,0.000000,0.000000}%
\pgfsetstrokecolor{currentstroke}%
\pgfsetstrokeopacity{0.000000}%
\pgfsetdash{}{0pt}%
\pgfpathmoveto{\pgfqpoint{5.800000in}{20.165559in}}%
\pgfpathlineto{\pgfqpoint{5.800000in}{20.165559in}}%
\pgfpathlineto{\pgfqpoint{5.800000in}{20.283410in}}%
\pgfpathlineto{\pgfqpoint{5.800000in}{20.283410in}}%
\pgfpathclose%
\pgfusepath{fill}%
\end{pgfscope}%
\begin{pgfscope}%
\pgfpathrectangle{\pgfqpoint{5.800000in}{0.720000in}}{\pgfqpoint{1.400000in}{4.620000in}}%
\pgfusepath{clip}%
\pgfsetbuttcap%
\pgfsetmiterjoin%
\definecolor{currentfill}{rgb}{0.121569,0.466667,0.705882}%
\pgfsetfillcolor{currentfill}%
\pgfsetlinewidth{0.000000pt}%
\definecolor{currentstroke}{rgb}{0.000000,0.000000,0.000000}%
\pgfsetstrokecolor{currentstroke}%
\pgfsetstrokeopacity{0.000000}%
\pgfsetdash{}{0pt}%
\pgfpathmoveto{\pgfqpoint{5.800000in}{20.283410in}}%
\pgfpathlineto{\pgfqpoint{5.800000in}{20.283410in}}%
\pgfpathlineto{\pgfqpoint{5.800000in}{20.401262in}}%
\pgfpathlineto{\pgfqpoint{5.800000in}{20.401262in}}%
\pgfpathclose%
\pgfusepath{fill}%
\end{pgfscope}%
\begin{pgfscope}%
\pgfpathrectangle{\pgfqpoint{5.800000in}{0.720000in}}{\pgfqpoint{1.400000in}{4.620000in}}%
\pgfusepath{clip}%
\pgfsetbuttcap%
\pgfsetmiterjoin%
\definecolor{currentfill}{rgb}{0.121569,0.466667,0.705882}%
\pgfsetfillcolor{currentfill}%
\pgfsetlinewidth{0.000000pt}%
\definecolor{currentstroke}{rgb}{0.000000,0.000000,0.000000}%
\pgfsetstrokecolor{currentstroke}%
\pgfsetstrokeopacity{0.000000}%
\pgfsetdash{}{0pt}%
\pgfpathmoveto{\pgfqpoint{5.800000in}{20.401262in}}%
\pgfpathlineto{\pgfqpoint{5.802160in}{20.401262in}}%
\pgfpathlineto{\pgfqpoint{5.802160in}{20.519114in}}%
\pgfpathlineto{\pgfqpoint{5.800000in}{20.519114in}}%
\pgfpathclose%
\pgfusepath{fill}%
\end{pgfscope}%
\begin{pgfscope}%
\pgfpathrectangle{\pgfqpoint{5.800000in}{0.720000in}}{\pgfqpoint{1.400000in}{4.620000in}}%
\pgfusepath{clip}%
\pgfsetbuttcap%
\pgfsetmiterjoin%
\definecolor{currentfill}{rgb}{0.121569,0.466667,0.705882}%
\pgfsetfillcolor{currentfill}%
\pgfsetlinewidth{0.000000pt}%
\definecolor{currentstroke}{rgb}{0.000000,0.000000,0.000000}%
\pgfsetstrokecolor{currentstroke}%
\pgfsetstrokeopacity{0.000000}%
\pgfsetdash{}{0pt}%
\pgfpathmoveto{\pgfqpoint{5.800000in}{20.519114in}}%
\pgfpathlineto{\pgfqpoint{5.800000in}{20.519114in}}%
\pgfpathlineto{\pgfqpoint{5.800000in}{20.636966in}}%
\pgfpathlineto{\pgfqpoint{5.800000in}{20.636966in}}%
\pgfpathclose%
\pgfusepath{fill}%
\end{pgfscope}%
\begin{pgfscope}%
\pgfpathrectangle{\pgfqpoint{5.800000in}{0.720000in}}{\pgfqpoint{1.400000in}{4.620000in}}%
\pgfusepath{clip}%
\pgfsetbuttcap%
\pgfsetmiterjoin%
\definecolor{currentfill}{rgb}{0.121569,0.466667,0.705882}%
\pgfsetfillcolor{currentfill}%
\pgfsetlinewidth{0.000000pt}%
\definecolor{currentstroke}{rgb}{0.000000,0.000000,0.000000}%
\pgfsetstrokecolor{currentstroke}%
\pgfsetstrokeopacity{0.000000}%
\pgfsetdash{}{0pt}%
\pgfpathmoveto{\pgfqpoint{5.800000in}{20.636966in}}%
\pgfpathlineto{\pgfqpoint{5.800000in}{20.636966in}}%
\pgfpathlineto{\pgfqpoint{5.800000in}{20.754818in}}%
\pgfpathlineto{\pgfqpoint{5.800000in}{20.754818in}}%
\pgfpathclose%
\pgfusepath{fill}%
\end{pgfscope}%
\begin{pgfscope}%
\pgfpathrectangle{\pgfqpoint{5.800000in}{0.720000in}}{\pgfqpoint{1.400000in}{4.620000in}}%
\pgfusepath{clip}%
\pgfsetbuttcap%
\pgfsetmiterjoin%
\definecolor{currentfill}{rgb}{0.121569,0.466667,0.705882}%
\pgfsetfillcolor{currentfill}%
\pgfsetlinewidth{0.000000pt}%
\definecolor{currentstroke}{rgb}{0.000000,0.000000,0.000000}%
\pgfsetstrokecolor{currentstroke}%
\pgfsetstrokeopacity{0.000000}%
\pgfsetdash{}{0pt}%
\pgfpathmoveto{\pgfqpoint{5.800000in}{20.754818in}}%
\pgfpathlineto{\pgfqpoint{5.800000in}{20.754818in}}%
\pgfpathlineto{\pgfqpoint{5.800000in}{20.872670in}}%
\pgfpathlineto{\pgfqpoint{5.800000in}{20.872670in}}%
\pgfpathclose%
\pgfusepath{fill}%
\end{pgfscope}%
\begin{pgfscope}%
\pgfpathrectangle{\pgfqpoint{5.800000in}{0.720000in}}{\pgfqpoint{1.400000in}{4.620000in}}%
\pgfusepath{clip}%
\pgfsetbuttcap%
\pgfsetmiterjoin%
\definecolor{currentfill}{rgb}{0.121569,0.466667,0.705882}%
\pgfsetfillcolor{currentfill}%
\pgfsetlinewidth{0.000000pt}%
\definecolor{currentstroke}{rgb}{0.000000,0.000000,0.000000}%
\pgfsetstrokecolor{currentstroke}%
\pgfsetstrokeopacity{0.000000}%
\pgfsetdash{}{0pt}%
\pgfpathmoveto{\pgfqpoint{5.800000in}{20.872670in}}%
\pgfpathlineto{\pgfqpoint{5.802160in}{20.872670in}}%
\pgfpathlineto{\pgfqpoint{5.802160in}{20.990522in}}%
\pgfpathlineto{\pgfqpoint{5.800000in}{20.990522in}}%
\pgfpathclose%
\pgfusepath{fill}%
\end{pgfscope}%
\begin{pgfscope}%
\pgfpathrectangle{\pgfqpoint{5.800000in}{0.720000in}}{\pgfqpoint{1.400000in}{4.620000in}}%
\pgfusepath{clip}%
\pgfsetbuttcap%
\pgfsetmiterjoin%
\definecolor{currentfill}{rgb}{0.121569,0.466667,0.705882}%
\pgfsetfillcolor{currentfill}%
\pgfsetlinewidth{0.000000pt}%
\definecolor{currentstroke}{rgb}{0.000000,0.000000,0.000000}%
\pgfsetstrokecolor{currentstroke}%
\pgfsetstrokeopacity{0.000000}%
\pgfsetdash{}{0pt}%
\pgfpathmoveto{\pgfqpoint{5.800000in}{20.990522in}}%
\pgfpathlineto{\pgfqpoint{5.802160in}{20.990522in}}%
\pgfpathlineto{\pgfqpoint{5.802160in}{21.108374in}}%
\pgfpathlineto{\pgfqpoint{5.800000in}{21.108374in}}%
\pgfpathclose%
\pgfusepath{fill}%
\end{pgfscope}%
\begin{pgfscope}%
\pgfpathrectangle{\pgfqpoint{5.800000in}{0.720000in}}{\pgfqpoint{1.400000in}{4.620000in}}%
\pgfusepath{clip}%
\pgfsetbuttcap%
\pgfsetmiterjoin%
\definecolor{currentfill}{rgb}{0.121569,0.466667,0.705882}%
\pgfsetfillcolor{currentfill}%
\pgfsetlinewidth{0.000000pt}%
\definecolor{currentstroke}{rgb}{0.000000,0.000000,0.000000}%
\pgfsetstrokecolor{currentstroke}%
\pgfsetstrokeopacity{0.000000}%
\pgfsetdash{}{0pt}%
\pgfpathmoveto{\pgfqpoint{5.800000in}{21.108374in}}%
\pgfpathlineto{\pgfqpoint{5.800000in}{21.108374in}}%
\pgfpathlineto{\pgfqpoint{5.800000in}{21.226225in}}%
\pgfpathlineto{\pgfqpoint{5.800000in}{21.226225in}}%
\pgfpathclose%
\pgfusepath{fill}%
\end{pgfscope}%
\begin{pgfscope}%
\pgfpathrectangle{\pgfqpoint{5.800000in}{0.720000in}}{\pgfqpoint{1.400000in}{4.620000in}}%
\pgfusepath{clip}%
\pgfsetbuttcap%
\pgfsetmiterjoin%
\definecolor{currentfill}{rgb}{0.121569,0.466667,0.705882}%
\pgfsetfillcolor{currentfill}%
\pgfsetlinewidth{0.000000pt}%
\definecolor{currentstroke}{rgb}{0.000000,0.000000,0.000000}%
\pgfsetstrokecolor{currentstroke}%
\pgfsetstrokeopacity{0.000000}%
\pgfsetdash{}{0pt}%
\pgfpathmoveto{\pgfqpoint{5.800000in}{21.226225in}}%
\pgfpathlineto{\pgfqpoint{5.800000in}{21.226225in}}%
\pgfpathlineto{\pgfqpoint{5.800000in}{21.344077in}}%
\pgfpathlineto{\pgfqpoint{5.800000in}{21.344077in}}%
\pgfpathclose%
\pgfusepath{fill}%
\end{pgfscope}%
\begin{pgfscope}%
\pgfpathrectangle{\pgfqpoint{5.800000in}{0.720000in}}{\pgfqpoint{1.400000in}{4.620000in}}%
\pgfusepath{clip}%
\pgfsetbuttcap%
\pgfsetmiterjoin%
\definecolor{currentfill}{rgb}{0.121569,0.466667,0.705882}%
\pgfsetfillcolor{currentfill}%
\pgfsetlinewidth{0.000000pt}%
\definecolor{currentstroke}{rgb}{0.000000,0.000000,0.000000}%
\pgfsetstrokecolor{currentstroke}%
\pgfsetstrokeopacity{0.000000}%
\pgfsetdash{}{0pt}%
\pgfpathmoveto{\pgfqpoint{5.800000in}{21.344077in}}%
\pgfpathlineto{\pgfqpoint{5.802160in}{21.344077in}}%
\pgfpathlineto{\pgfqpoint{5.802160in}{21.461929in}}%
\pgfpathlineto{\pgfqpoint{5.800000in}{21.461929in}}%
\pgfpathclose%
\pgfusepath{fill}%
\end{pgfscope}%
\begin{pgfscope}%
\pgfpathrectangle{\pgfqpoint{5.800000in}{0.720000in}}{\pgfqpoint{1.400000in}{4.620000in}}%
\pgfusepath{clip}%
\pgfsetbuttcap%
\pgfsetmiterjoin%
\definecolor{currentfill}{rgb}{0.121569,0.466667,0.705882}%
\pgfsetfillcolor{currentfill}%
\pgfsetlinewidth{0.000000pt}%
\definecolor{currentstroke}{rgb}{0.000000,0.000000,0.000000}%
\pgfsetstrokecolor{currentstroke}%
\pgfsetstrokeopacity{0.000000}%
\pgfsetdash{}{0pt}%
\pgfpathmoveto{\pgfqpoint{5.800000in}{21.461929in}}%
\pgfpathlineto{\pgfqpoint{5.800000in}{21.461929in}}%
\pgfpathlineto{\pgfqpoint{5.800000in}{21.579781in}}%
\pgfpathlineto{\pgfqpoint{5.800000in}{21.579781in}}%
\pgfpathclose%
\pgfusepath{fill}%
\end{pgfscope}%
\begin{pgfscope}%
\pgfpathrectangle{\pgfqpoint{5.800000in}{0.720000in}}{\pgfqpoint{1.400000in}{4.620000in}}%
\pgfusepath{clip}%
\pgfsetbuttcap%
\pgfsetmiterjoin%
\definecolor{currentfill}{rgb}{0.121569,0.466667,0.705882}%
\pgfsetfillcolor{currentfill}%
\pgfsetlinewidth{0.000000pt}%
\definecolor{currentstroke}{rgb}{0.000000,0.000000,0.000000}%
\pgfsetstrokecolor{currentstroke}%
\pgfsetstrokeopacity{0.000000}%
\pgfsetdash{}{0pt}%
\pgfpathmoveto{\pgfqpoint{5.800000in}{21.579781in}}%
\pgfpathlineto{\pgfqpoint{5.800000in}{21.579781in}}%
\pgfpathlineto{\pgfqpoint{5.800000in}{21.697633in}}%
\pgfpathlineto{\pgfqpoint{5.800000in}{21.697633in}}%
\pgfpathclose%
\pgfusepath{fill}%
\end{pgfscope}%
\begin{pgfscope}%
\pgfpathrectangle{\pgfqpoint{5.800000in}{0.720000in}}{\pgfqpoint{1.400000in}{4.620000in}}%
\pgfusepath{clip}%
\pgfsetbuttcap%
\pgfsetmiterjoin%
\definecolor{currentfill}{rgb}{0.121569,0.466667,0.705882}%
\pgfsetfillcolor{currentfill}%
\pgfsetlinewidth{0.000000pt}%
\definecolor{currentstroke}{rgb}{0.000000,0.000000,0.000000}%
\pgfsetstrokecolor{currentstroke}%
\pgfsetstrokeopacity{0.000000}%
\pgfsetdash{}{0pt}%
\pgfpathmoveto{\pgfqpoint{5.800000in}{21.697633in}}%
\pgfpathlineto{\pgfqpoint{5.802160in}{21.697633in}}%
\pgfpathlineto{\pgfqpoint{5.802160in}{21.815485in}}%
\pgfpathlineto{\pgfqpoint{5.800000in}{21.815485in}}%
\pgfpathclose%
\pgfusepath{fill}%
\end{pgfscope}%
\begin{pgfscope}%
\pgfpathrectangle{\pgfqpoint{5.800000in}{0.720000in}}{\pgfqpoint{1.400000in}{4.620000in}}%
\pgfusepath{clip}%
\pgfsetbuttcap%
\pgfsetmiterjoin%
\definecolor{currentfill}{rgb}{0.121569,0.466667,0.705882}%
\pgfsetfillcolor{currentfill}%
\pgfsetlinewidth{0.000000pt}%
\definecolor{currentstroke}{rgb}{0.000000,0.000000,0.000000}%
\pgfsetstrokecolor{currentstroke}%
\pgfsetstrokeopacity{0.000000}%
\pgfsetdash{}{0pt}%
\pgfpathmoveto{\pgfqpoint{5.800000in}{21.815485in}}%
\pgfpathlineto{\pgfqpoint{5.800000in}{21.815485in}}%
\pgfpathlineto{\pgfqpoint{5.800000in}{21.933337in}}%
\pgfpathlineto{\pgfqpoint{5.800000in}{21.933337in}}%
\pgfpathclose%
\pgfusepath{fill}%
\end{pgfscope}%
\begin{pgfscope}%
\pgfpathrectangle{\pgfqpoint{5.800000in}{0.720000in}}{\pgfqpoint{1.400000in}{4.620000in}}%
\pgfusepath{clip}%
\pgfsetbuttcap%
\pgfsetmiterjoin%
\definecolor{currentfill}{rgb}{0.121569,0.466667,0.705882}%
\pgfsetfillcolor{currentfill}%
\pgfsetlinewidth{0.000000pt}%
\definecolor{currentstroke}{rgb}{0.000000,0.000000,0.000000}%
\pgfsetstrokecolor{currentstroke}%
\pgfsetstrokeopacity{0.000000}%
\pgfsetdash{}{0pt}%
\pgfpathmoveto{\pgfqpoint{5.800000in}{21.933337in}}%
\pgfpathlineto{\pgfqpoint{5.800000in}{21.933337in}}%
\pgfpathlineto{\pgfqpoint{5.800000in}{22.051189in}}%
\pgfpathlineto{\pgfqpoint{5.800000in}{22.051189in}}%
\pgfpathclose%
\pgfusepath{fill}%
\end{pgfscope}%
\begin{pgfscope}%
\pgfpathrectangle{\pgfqpoint{5.800000in}{0.720000in}}{\pgfqpoint{1.400000in}{4.620000in}}%
\pgfusepath{clip}%
\pgfsetbuttcap%
\pgfsetmiterjoin%
\definecolor{currentfill}{rgb}{0.121569,0.466667,0.705882}%
\pgfsetfillcolor{currentfill}%
\pgfsetlinewidth{0.000000pt}%
\definecolor{currentstroke}{rgb}{0.000000,0.000000,0.000000}%
\pgfsetstrokecolor{currentstroke}%
\pgfsetstrokeopacity{0.000000}%
\pgfsetdash{}{0pt}%
\pgfpathmoveto{\pgfqpoint{5.800000in}{22.051189in}}%
\pgfpathlineto{\pgfqpoint{5.802160in}{22.051189in}}%
\pgfpathlineto{\pgfqpoint{5.802160in}{22.169040in}}%
\pgfpathlineto{\pgfqpoint{5.800000in}{22.169040in}}%
\pgfpathclose%
\pgfusepath{fill}%
\end{pgfscope}%
\begin{pgfscope}%
\pgfpathrectangle{\pgfqpoint{5.800000in}{0.720000in}}{\pgfqpoint{1.400000in}{4.620000in}}%
\pgfusepath{clip}%
\pgfsetbuttcap%
\pgfsetmiterjoin%
\definecolor{currentfill}{rgb}{0.121569,0.466667,0.705882}%
\pgfsetfillcolor{currentfill}%
\pgfsetlinewidth{0.000000pt}%
\definecolor{currentstroke}{rgb}{0.000000,0.000000,0.000000}%
\pgfsetstrokecolor{currentstroke}%
\pgfsetstrokeopacity{0.000000}%
\pgfsetdash{}{0pt}%
\pgfpathmoveto{\pgfqpoint{5.800000in}{22.169040in}}%
\pgfpathlineto{\pgfqpoint{5.800000in}{22.169040in}}%
\pgfpathlineto{\pgfqpoint{5.800000in}{22.286892in}}%
\pgfpathlineto{\pgfqpoint{5.800000in}{22.286892in}}%
\pgfpathclose%
\pgfusepath{fill}%
\end{pgfscope}%
\begin{pgfscope}%
\pgfpathrectangle{\pgfqpoint{5.800000in}{0.720000in}}{\pgfqpoint{1.400000in}{4.620000in}}%
\pgfusepath{clip}%
\pgfsetbuttcap%
\pgfsetmiterjoin%
\definecolor{currentfill}{rgb}{0.121569,0.466667,0.705882}%
\pgfsetfillcolor{currentfill}%
\pgfsetlinewidth{0.000000pt}%
\definecolor{currentstroke}{rgb}{0.000000,0.000000,0.000000}%
\pgfsetstrokecolor{currentstroke}%
\pgfsetstrokeopacity{0.000000}%
\pgfsetdash{}{0pt}%
\pgfpathmoveto{\pgfqpoint{5.800000in}{22.286892in}}%
\pgfpathlineto{\pgfqpoint{5.800000in}{22.286892in}}%
\pgfpathlineto{\pgfqpoint{5.800000in}{22.404744in}}%
\pgfpathlineto{\pgfqpoint{5.800000in}{22.404744in}}%
\pgfpathclose%
\pgfusepath{fill}%
\end{pgfscope}%
\begin{pgfscope}%
\pgfpathrectangle{\pgfqpoint{5.800000in}{0.720000in}}{\pgfqpoint{1.400000in}{4.620000in}}%
\pgfusepath{clip}%
\pgfsetbuttcap%
\pgfsetmiterjoin%
\definecolor{currentfill}{rgb}{0.121569,0.466667,0.705882}%
\pgfsetfillcolor{currentfill}%
\pgfsetlinewidth{0.000000pt}%
\definecolor{currentstroke}{rgb}{0.000000,0.000000,0.000000}%
\pgfsetstrokecolor{currentstroke}%
\pgfsetstrokeopacity{0.000000}%
\pgfsetdash{}{0pt}%
\pgfpathmoveto{\pgfqpoint{5.800000in}{22.404744in}}%
\pgfpathlineto{\pgfqpoint{5.802160in}{22.404744in}}%
\pgfpathlineto{\pgfqpoint{5.802160in}{22.522596in}}%
\pgfpathlineto{\pgfqpoint{5.800000in}{22.522596in}}%
\pgfpathclose%
\pgfusepath{fill}%
\end{pgfscope}%
\begin{pgfscope}%
\pgfpathrectangle{\pgfqpoint{5.800000in}{0.720000in}}{\pgfqpoint{1.400000in}{4.620000in}}%
\pgfusepath{clip}%
\pgfsetbuttcap%
\pgfsetmiterjoin%
\definecolor{currentfill}{rgb}{0.121569,0.466667,0.705882}%
\pgfsetfillcolor{currentfill}%
\pgfsetlinewidth{0.000000pt}%
\definecolor{currentstroke}{rgb}{0.000000,0.000000,0.000000}%
\pgfsetstrokecolor{currentstroke}%
\pgfsetstrokeopacity{0.000000}%
\pgfsetdash{}{0pt}%
\pgfpathmoveto{\pgfqpoint{5.800000in}{22.522596in}}%
\pgfpathlineto{\pgfqpoint{5.802160in}{22.522596in}}%
\pgfpathlineto{\pgfqpoint{5.802160in}{22.640448in}}%
\pgfpathlineto{\pgfqpoint{5.800000in}{22.640448in}}%
\pgfpathclose%
\pgfusepath{fill}%
\end{pgfscope}%
\begin{pgfscope}%
\pgfpathrectangle{\pgfqpoint{5.800000in}{0.720000in}}{\pgfqpoint{1.400000in}{4.620000in}}%
\pgfusepath{clip}%
\pgfsetbuttcap%
\pgfsetmiterjoin%
\definecolor{currentfill}{rgb}{0.121569,0.466667,0.705882}%
\pgfsetfillcolor{currentfill}%
\pgfsetlinewidth{0.000000pt}%
\definecolor{currentstroke}{rgb}{0.000000,0.000000,0.000000}%
\pgfsetstrokecolor{currentstroke}%
\pgfsetstrokeopacity{0.000000}%
\pgfsetdash{}{0pt}%
\pgfpathmoveto{\pgfqpoint{5.800000in}{22.640448in}}%
\pgfpathlineto{\pgfqpoint{5.800000in}{22.640448in}}%
\pgfpathlineto{\pgfqpoint{5.800000in}{22.758300in}}%
\pgfpathlineto{\pgfqpoint{5.800000in}{22.758300in}}%
\pgfpathclose%
\pgfusepath{fill}%
\end{pgfscope}%
\begin{pgfscope}%
\pgfpathrectangle{\pgfqpoint{5.800000in}{0.720000in}}{\pgfqpoint{1.400000in}{4.620000in}}%
\pgfusepath{clip}%
\pgfsetbuttcap%
\pgfsetmiterjoin%
\definecolor{currentfill}{rgb}{0.121569,0.466667,0.705882}%
\pgfsetfillcolor{currentfill}%
\pgfsetlinewidth{0.000000pt}%
\definecolor{currentstroke}{rgb}{0.000000,0.000000,0.000000}%
\pgfsetstrokecolor{currentstroke}%
\pgfsetstrokeopacity{0.000000}%
\pgfsetdash{}{0pt}%
\pgfpathmoveto{\pgfqpoint{5.800000in}{22.758300in}}%
\pgfpathlineto{\pgfqpoint{5.800000in}{22.758300in}}%
\pgfpathlineto{\pgfqpoint{5.800000in}{22.876152in}}%
\pgfpathlineto{\pgfqpoint{5.800000in}{22.876152in}}%
\pgfpathclose%
\pgfusepath{fill}%
\end{pgfscope}%
\begin{pgfscope}%
\pgfpathrectangle{\pgfqpoint{5.800000in}{0.720000in}}{\pgfqpoint{1.400000in}{4.620000in}}%
\pgfusepath{clip}%
\pgfsetbuttcap%
\pgfsetmiterjoin%
\definecolor{currentfill}{rgb}{0.121569,0.466667,0.705882}%
\pgfsetfillcolor{currentfill}%
\pgfsetlinewidth{0.000000pt}%
\definecolor{currentstroke}{rgb}{0.000000,0.000000,0.000000}%
\pgfsetstrokecolor{currentstroke}%
\pgfsetstrokeopacity{0.000000}%
\pgfsetdash{}{0pt}%
\pgfpathmoveto{\pgfqpoint{5.800000in}{22.876152in}}%
\pgfpathlineto{\pgfqpoint{5.800000in}{22.876152in}}%
\pgfpathlineto{\pgfqpoint{5.800000in}{22.994003in}}%
\pgfpathlineto{\pgfqpoint{5.800000in}{22.994003in}}%
\pgfpathclose%
\pgfusepath{fill}%
\end{pgfscope}%
\begin{pgfscope}%
\pgfpathrectangle{\pgfqpoint{5.800000in}{0.720000in}}{\pgfqpoint{1.400000in}{4.620000in}}%
\pgfusepath{clip}%
\pgfsetbuttcap%
\pgfsetmiterjoin%
\definecolor{currentfill}{rgb}{0.121569,0.466667,0.705882}%
\pgfsetfillcolor{currentfill}%
\pgfsetlinewidth{0.000000pt}%
\definecolor{currentstroke}{rgb}{0.000000,0.000000,0.000000}%
\pgfsetstrokecolor{currentstroke}%
\pgfsetstrokeopacity{0.000000}%
\pgfsetdash{}{0pt}%
\pgfpathmoveto{\pgfqpoint{5.800000in}{22.994003in}}%
\pgfpathlineto{\pgfqpoint{5.800000in}{22.994003in}}%
\pgfpathlineto{\pgfqpoint{5.800000in}{23.111855in}}%
\pgfpathlineto{\pgfqpoint{5.800000in}{23.111855in}}%
\pgfpathclose%
\pgfusepath{fill}%
\end{pgfscope}%
\begin{pgfscope}%
\pgfpathrectangle{\pgfqpoint{5.800000in}{0.720000in}}{\pgfqpoint{1.400000in}{4.620000in}}%
\pgfusepath{clip}%
\pgfsetbuttcap%
\pgfsetmiterjoin%
\definecolor{currentfill}{rgb}{0.121569,0.466667,0.705882}%
\pgfsetfillcolor{currentfill}%
\pgfsetlinewidth{0.000000pt}%
\definecolor{currentstroke}{rgb}{0.000000,0.000000,0.000000}%
\pgfsetstrokecolor{currentstroke}%
\pgfsetstrokeopacity{0.000000}%
\pgfsetdash{}{0pt}%
\pgfpathmoveto{\pgfqpoint{5.800000in}{23.111855in}}%
\pgfpathlineto{\pgfqpoint{5.800000in}{23.111855in}}%
\pgfpathlineto{\pgfqpoint{5.800000in}{23.229707in}}%
\pgfpathlineto{\pgfqpoint{5.800000in}{23.229707in}}%
\pgfpathclose%
\pgfusepath{fill}%
\end{pgfscope}%
\begin{pgfscope}%
\pgfpathrectangle{\pgfqpoint{5.800000in}{0.720000in}}{\pgfqpoint{1.400000in}{4.620000in}}%
\pgfusepath{clip}%
\pgfsetbuttcap%
\pgfsetmiterjoin%
\definecolor{currentfill}{rgb}{0.121569,0.466667,0.705882}%
\pgfsetfillcolor{currentfill}%
\pgfsetlinewidth{0.000000pt}%
\definecolor{currentstroke}{rgb}{0.000000,0.000000,0.000000}%
\pgfsetstrokecolor{currentstroke}%
\pgfsetstrokeopacity{0.000000}%
\pgfsetdash{}{0pt}%
\pgfpathmoveto{\pgfqpoint{5.800000in}{23.229707in}}%
\pgfpathlineto{\pgfqpoint{5.802160in}{23.229707in}}%
\pgfpathlineto{\pgfqpoint{5.802160in}{23.347559in}}%
\pgfpathlineto{\pgfqpoint{5.800000in}{23.347559in}}%
\pgfpathclose%
\pgfusepath{fill}%
\end{pgfscope}%
\begin{pgfscope}%
\pgfpathrectangle{\pgfqpoint{5.800000in}{0.720000in}}{\pgfqpoint{1.400000in}{4.620000in}}%
\pgfusepath{clip}%
\pgfsetbuttcap%
\pgfsetmiterjoin%
\definecolor{currentfill}{rgb}{0.121569,0.466667,0.705882}%
\pgfsetfillcolor{currentfill}%
\pgfsetlinewidth{0.000000pt}%
\definecolor{currentstroke}{rgb}{0.000000,0.000000,0.000000}%
\pgfsetstrokecolor{currentstroke}%
\pgfsetstrokeopacity{0.000000}%
\pgfsetdash{}{0pt}%
\pgfpathmoveto{\pgfqpoint{5.800000in}{23.347559in}}%
\pgfpathlineto{\pgfqpoint{5.800000in}{23.347559in}}%
\pgfpathlineto{\pgfqpoint{5.800000in}{23.465411in}}%
\pgfpathlineto{\pgfqpoint{5.800000in}{23.465411in}}%
\pgfpathclose%
\pgfusepath{fill}%
\end{pgfscope}%
\begin{pgfscope}%
\pgfpathrectangle{\pgfqpoint{5.800000in}{0.720000in}}{\pgfqpoint{1.400000in}{4.620000in}}%
\pgfusepath{clip}%
\pgfsetbuttcap%
\pgfsetmiterjoin%
\definecolor{currentfill}{rgb}{0.121569,0.466667,0.705882}%
\pgfsetfillcolor{currentfill}%
\pgfsetlinewidth{0.000000pt}%
\definecolor{currentstroke}{rgb}{0.000000,0.000000,0.000000}%
\pgfsetstrokecolor{currentstroke}%
\pgfsetstrokeopacity{0.000000}%
\pgfsetdash{}{0pt}%
\pgfpathmoveto{\pgfqpoint{5.800000in}{23.465411in}}%
\pgfpathlineto{\pgfqpoint{5.800000in}{23.465411in}}%
\pgfpathlineto{\pgfqpoint{5.800000in}{23.583263in}}%
\pgfpathlineto{\pgfqpoint{5.800000in}{23.583263in}}%
\pgfpathclose%
\pgfusepath{fill}%
\end{pgfscope}%
\begin{pgfscope}%
\pgfpathrectangle{\pgfqpoint{5.800000in}{0.720000in}}{\pgfqpoint{1.400000in}{4.620000in}}%
\pgfusepath{clip}%
\pgfsetbuttcap%
\pgfsetmiterjoin%
\definecolor{currentfill}{rgb}{0.121569,0.466667,0.705882}%
\pgfsetfillcolor{currentfill}%
\pgfsetlinewidth{0.000000pt}%
\definecolor{currentstroke}{rgb}{0.000000,0.000000,0.000000}%
\pgfsetstrokecolor{currentstroke}%
\pgfsetstrokeopacity{0.000000}%
\pgfsetdash{}{0pt}%
\pgfpathmoveto{\pgfqpoint{5.800000in}{23.583263in}}%
\pgfpathlineto{\pgfqpoint{5.802160in}{23.583263in}}%
\pgfpathlineto{\pgfqpoint{5.802160in}{23.701115in}}%
\pgfpathlineto{\pgfqpoint{5.800000in}{23.701115in}}%
\pgfpathclose%
\pgfusepath{fill}%
\end{pgfscope}%
\begin{pgfscope}%
\pgfpathrectangle{\pgfqpoint{5.800000in}{0.720000in}}{\pgfqpoint{1.400000in}{4.620000in}}%
\pgfusepath{clip}%
\pgfsetbuttcap%
\pgfsetmiterjoin%
\definecolor{currentfill}{rgb}{0.121569,0.466667,0.705882}%
\pgfsetfillcolor{currentfill}%
\pgfsetlinewidth{0.000000pt}%
\definecolor{currentstroke}{rgb}{0.000000,0.000000,0.000000}%
\pgfsetstrokecolor{currentstroke}%
\pgfsetstrokeopacity{0.000000}%
\pgfsetdash{}{0pt}%
\pgfpathmoveto{\pgfqpoint{5.800000in}{23.701115in}}%
\pgfpathlineto{\pgfqpoint{5.800000in}{23.701115in}}%
\pgfpathlineto{\pgfqpoint{5.800000in}{23.818967in}}%
\pgfpathlineto{\pgfqpoint{5.800000in}{23.818967in}}%
\pgfpathclose%
\pgfusepath{fill}%
\end{pgfscope}%
\begin{pgfscope}%
\pgfpathrectangle{\pgfqpoint{5.800000in}{0.720000in}}{\pgfqpoint{1.400000in}{4.620000in}}%
\pgfusepath{clip}%
\pgfsetbuttcap%
\pgfsetmiterjoin%
\definecolor{currentfill}{rgb}{0.121569,0.466667,0.705882}%
\pgfsetfillcolor{currentfill}%
\pgfsetlinewidth{0.000000pt}%
\definecolor{currentstroke}{rgb}{0.000000,0.000000,0.000000}%
\pgfsetstrokecolor{currentstroke}%
\pgfsetstrokeopacity{0.000000}%
\pgfsetdash{}{0pt}%
\pgfpathmoveto{\pgfqpoint{5.800000in}{23.818967in}}%
\pgfpathlineto{\pgfqpoint{5.802160in}{23.818967in}}%
\pgfpathlineto{\pgfqpoint{5.802160in}{23.936818in}}%
\pgfpathlineto{\pgfqpoint{5.800000in}{23.936818in}}%
\pgfpathclose%
\pgfusepath{fill}%
\end{pgfscope}%
\begin{pgfscope}%
\pgfpathrectangle{\pgfqpoint{5.800000in}{0.720000in}}{\pgfqpoint{1.400000in}{4.620000in}}%
\pgfusepath{clip}%
\pgfsetbuttcap%
\pgfsetmiterjoin%
\definecolor{currentfill}{rgb}{0.121569,0.466667,0.705882}%
\pgfsetfillcolor{currentfill}%
\pgfsetlinewidth{0.000000pt}%
\definecolor{currentstroke}{rgb}{0.000000,0.000000,0.000000}%
\pgfsetstrokecolor{currentstroke}%
\pgfsetstrokeopacity{0.000000}%
\pgfsetdash{}{0pt}%
\pgfpathmoveto{\pgfqpoint{5.800000in}{23.936818in}}%
\pgfpathlineto{\pgfqpoint{5.800000in}{23.936818in}}%
\pgfpathlineto{\pgfqpoint{5.800000in}{24.054670in}}%
\pgfpathlineto{\pgfqpoint{5.800000in}{24.054670in}}%
\pgfpathclose%
\pgfusepath{fill}%
\end{pgfscope}%
\begin{pgfscope}%
\pgfpathrectangle{\pgfqpoint{5.800000in}{0.720000in}}{\pgfqpoint{1.400000in}{4.620000in}}%
\pgfusepath{clip}%
\pgfsetbuttcap%
\pgfsetmiterjoin%
\definecolor{currentfill}{rgb}{0.121569,0.466667,0.705882}%
\pgfsetfillcolor{currentfill}%
\pgfsetlinewidth{0.000000pt}%
\definecolor{currentstroke}{rgb}{0.000000,0.000000,0.000000}%
\pgfsetstrokecolor{currentstroke}%
\pgfsetstrokeopacity{0.000000}%
\pgfsetdash{}{0pt}%
\pgfpathmoveto{\pgfqpoint{5.800000in}{24.054670in}}%
\pgfpathlineto{\pgfqpoint{5.800000in}{24.054670in}}%
\pgfpathlineto{\pgfqpoint{5.800000in}{24.172522in}}%
\pgfpathlineto{\pgfqpoint{5.800000in}{24.172522in}}%
\pgfpathclose%
\pgfusepath{fill}%
\end{pgfscope}%
\begin{pgfscope}%
\pgfpathrectangle{\pgfqpoint{5.800000in}{0.720000in}}{\pgfqpoint{1.400000in}{4.620000in}}%
\pgfusepath{clip}%
\pgfsetbuttcap%
\pgfsetmiterjoin%
\definecolor{currentfill}{rgb}{0.121569,0.466667,0.705882}%
\pgfsetfillcolor{currentfill}%
\pgfsetlinewidth{0.000000pt}%
\definecolor{currentstroke}{rgb}{0.000000,0.000000,0.000000}%
\pgfsetstrokecolor{currentstroke}%
\pgfsetstrokeopacity{0.000000}%
\pgfsetdash{}{0pt}%
\pgfpathmoveto{\pgfqpoint{5.800000in}{24.172522in}}%
\pgfpathlineto{\pgfqpoint{5.800000in}{24.172522in}}%
\pgfpathlineto{\pgfqpoint{5.800000in}{24.290374in}}%
\pgfpathlineto{\pgfqpoint{5.800000in}{24.290374in}}%
\pgfpathclose%
\pgfusepath{fill}%
\end{pgfscope}%
\begin{pgfscope}%
\pgfpathrectangle{\pgfqpoint{5.800000in}{0.720000in}}{\pgfqpoint{1.400000in}{4.620000in}}%
\pgfusepath{clip}%
\pgfsetbuttcap%
\pgfsetmiterjoin%
\definecolor{currentfill}{rgb}{0.121569,0.466667,0.705882}%
\pgfsetfillcolor{currentfill}%
\pgfsetlinewidth{0.000000pt}%
\definecolor{currentstroke}{rgb}{0.000000,0.000000,0.000000}%
\pgfsetstrokecolor{currentstroke}%
\pgfsetstrokeopacity{0.000000}%
\pgfsetdash{}{0pt}%
\pgfpathmoveto{\pgfqpoint{5.800000in}{24.290374in}}%
\pgfpathlineto{\pgfqpoint{5.800000in}{24.290374in}}%
\pgfpathlineto{\pgfqpoint{5.800000in}{24.408226in}}%
\pgfpathlineto{\pgfqpoint{5.800000in}{24.408226in}}%
\pgfpathclose%
\pgfusepath{fill}%
\end{pgfscope}%
\begin{pgfscope}%
\pgfpathrectangle{\pgfqpoint{5.800000in}{0.720000in}}{\pgfqpoint{1.400000in}{4.620000in}}%
\pgfusepath{clip}%
\pgfsetbuttcap%
\pgfsetmiterjoin%
\definecolor{currentfill}{rgb}{0.121569,0.466667,0.705882}%
\pgfsetfillcolor{currentfill}%
\pgfsetlinewidth{0.000000pt}%
\definecolor{currentstroke}{rgb}{0.000000,0.000000,0.000000}%
\pgfsetstrokecolor{currentstroke}%
\pgfsetstrokeopacity{0.000000}%
\pgfsetdash{}{0pt}%
\pgfpathmoveto{\pgfqpoint{5.800000in}{24.408226in}}%
\pgfpathlineto{\pgfqpoint{5.800000in}{24.408226in}}%
\pgfpathlineto{\pgfqpoint{5.800000in}{24.526078in}}%
\pgfpathlineto{\pgfqpoint{5.800000in}{24.526078in}}%
\pgfpathclose%
\pgfusepath{fill}%
\end{pgfscope}%
\begin{pgfscope}%
\pgfpathrectangle{\pgfqpoint{5.800000in}{0.720000in}}{\pgfqpoint{1.400000in}{4.620000in}}%
\pgfusepath{clip}%
\pgfsetbuttcap%
\pgfsetmiterjoin%
\definecolor{currentfill}{rgb}{0.121569,0.466667,0.705882}%
\pgfsetfillcolor{currentfill}%
\pgfsetlinewidth{0.000000pt}%
\definecolor{currentstroke}{rgb}{0.000000,0.000000,0.000000}%
\pgfsetstrokecolor{currentstroke}%
\pgfsetstrokeopacity{0.000000}%
\pgfsetdash{}{0pt}%
\pgfpathmoveto{\pgfqpoint{5.800000in}{24.526078in}}%
\pgfpathlineto{\pgfqpoint{5.800000in}{24.526078in}}%
\pgfpathlineto{\pgfqpoint{5.800000in}{24.643930in}}%
\pgfpathlineto{\pgfqpoint{5.800000in}{24.643930in}}%
\pgfpathclose%
\pgfusepath{fill}%
\end{pgfscope}%
\begin{pgfscope}%
\pgfpathrectangle{\pgfqpoint{5.800000in}{0.720000in}}{\pgfqpoint{1.400000in}{4.620000in}}%
\pgfusepath{clip}%
\pgfsetbuttcap%
\pgfsetmiterjoin%
\definecolor{currentfill}{rgb}{0.121569,0.466667,0.705882}%
\pgfsetfillcolor{currentfill}%
\pgfsetlinewidth{0.000000pt}%
\definecolor{currentstroke}{rgb}{0.000000,0.000000,0.000000}%
\pgfsetstrokecolor{currentstroke}%
\pgfsetstrokeopacity{0.000000}%
\pgfsetdash{}{0pt}%
\pgfpathmoveto{\pgfqpoint{5.800000in}{24.643930in}}%
\pgfpathlineto{\pgfqpoint{5.800000in}{24.643930in}}%
\pgfpathlineto{\pgfqpoint{5.800000in}{24.761782in}}%
\pgfpathlineto{\pgfqpoint{5.800000in}{24.761782in}}%
\pgfpathclose%
\pgfusepath{fill}%
\end{pgfscope}%
\begin{pgfscope}%
\pgfpathrectangle{\pgfqpoint{5.800000in}{0.720000in}}{\pgfqpoint{1.400000in}{4.620000in}}%
\pgfusepath{clip}%
\pgfsetbuttcap%
\pgfsetmiterjoin%
\definecolor{currentfill}{rgb}{0.121569,0.466667,0.705882}%
\pgfsetfillcolor{currentfill}%
\pgfsetlinewidth{0.000000pt}%
\definecolor{currentstroke}{rgb}{0.000000,0.000000,0.000000}%
\pgfsetstrokecolor{currentstroke}%
\pgfsetstrokeopacity{0.000000}%
\pgfsetdash{}{0pt}%
\pgfpathmoveto{\pgfqpoint{5.800000in}{24.761782in}}%
\pgfpathlineto{\pgfqpoint{5.800000in}{24.761782in}}%
\pgfpathlineto{\pgfqpoint{5.800000in}{24.879633in}}%
\pgfpathlineto{\pgfqpoint{5.800000in}{24.879633in}}%
\pgfpathclose%
\pgfusepath{fill}%
\end{pgfscope}%
\begin{pgfscope}%
\pgfpathrectangle{\pgfqpoint{5.800000in}{0.720000in}}{\pgfqpoint{1.400000in}{4.620000in}}%
\pgfusepath{clip}%
\pgfsetbuttcap%
\pgfsetmiterjoin%
\definecolor{currentfill}{rgb}{0.121569,0.466667,0.705882}%
\pgfsetfillcolor{currentfill}%
\pgfsetlinewidth{0.000000pt}%
\definecolor{currentstroke}{rgb}{0.000000,0.000000,0.000000}%
\pgfsetstrokecolor{currentstroke}%
\pgfsetstrokeopacity{0.000000}%
\pgfsetdash{}{0pt}%
\pgfpathmoveto{\pgfqpoint{5.800000in}{24.879633in}}%
\pgfpathlineto{\pgfqpoint{5.800000in}{24.879633in}}%
\pgfpathlineto{\pgfqpoint{5.800000in}{24.997485in}}%
\pgfpathlineto{\pgfqpoint{5.800000in}{24.997485in}}%
\pgfpathclose%
\pgfusepath{fill}%
\end{pgfscope}%
\begin{pgfscope}%
\pgfpathrectangle{\pgfqpoint{5.800000in}{0.720000in}}{\pgfqpoint{1.400000in}{4.620000in}}%
\pgfusepath{clip}%
\pgfsetbuttcap%
\pgfsetmiterjoin%
\definecolor{currentfill}{rgb}{0.121569,0.466667,0.705882}%
\pgfsetfillcolor{currentfill}%
\pgfsetlinewidth{0.000000pt}%
\definecolor{currentstroke}{rgb}{0.000000,0.000000,0.000000}%
\pgfsetstrokecolor{currentstroke}%
\pgfsetstrokeopacity{0.000000}%
\pgfsetdash{}{0pt}%
\pgfpathmoveto{\pgfqpoint{5.800000in}{24.997485in}}%
\pgfpathlineto{\pgfqpoint{5.800000in}{24.997485in}}%
\pgfpathlineto{\pgfqpoint{5.800000in}{25.115337in}}%
\pgfpathlineto{\pgfqpoint{5.800000in}{25.115337in}}%
\pgfpathclose%
\pgfusepath{fill}%
\end{pgfscope}%
\begin{pgfscope}%
\pgfpathrectangle{\pgfqpoint{5.800000in}{0.720000in}}{\pgfqpoint{1.400000in}{4.620000in}}%
\pgfusepath{clip}%
\pgfsetbuttcap%
\pgfsetmiterjoin%
\definecolor{currentfill}{rgb}{0.121569,0.466667,0.705882}%
\pgfsetfillcolor{currentfill}%
\pgfsetlinewidth{0.000000pt}%
\definecolor{currentstroke}{rgb}{0.000000,0.000000,0.000000}%
\pgfsetstrokecolor{currentstroke}%
\pgfsetstrokeopacity{0.000000}%
\pgfsetdash{}{0pt}%
\pgfpathmoveto{\pgfqpoint{5.800000in}{25.115337in}}%
\pgfpathlineto{\pgfqpoint{5.802160in}{25.115337in}}%
\pgfpathlineto{\pgfqpoint{5.802160in}{25.233189in}}%
\pgfpathlineto{\pgfqpoint{5.800000in}{25.233189in}}%
\pgfpathclose%
\pgfusepath{fill}%
\end{pgfscope}%
\begin{pgfscope}%
\pgfpathrectangle{\pgfqpoint{5.800000in}{0.720000in}}{\pgfqpoint{1.400000in}{4.620000in}}%
\pgfusepath{clip}%
\pgfsetbuttcap%
\pgfsetmiterjoin%
\definecolor{currentfill}{rgb}{0.121569,0.466667,0.705882}%
\pgfsetfillcolor{currentfill}%
\pgfsetlinewidth{0.000000pt}%
\definecolor{currentstroke}{rgb}{0.000000,0.000000,0.000000}%
\pgfsetstrokecolor{currentstroke}%
\pgfsetstrokeopacity{0.000000}%
\pgfsetdash{}{0pt}%
\pgfpathmoveto{\pgfqpoint{5.800000in}{25.233189in}}%
\pgfpathlineto{\pgfqpoint{5.800000in}{25.233189in}}%
\pgfpathlineto{\pgfqpoint{5.800000in}{25.351041in}}%
\pgfpathlineto{\pgfqpoint{5.800000in}{25.351041in}}%
\pgfpathclose%
\pgfusepath{fill}%
\end{pgfscope}%
\begin{pgfscope}%
\pgfpathrectangle{\pgfqpoint{5.800000in}{0.720000in}}{\pgfqpoint{1.400000in}{4.620000in}}%
\pgfusepath{clip}%
\pgfsetbuttcap%
\pgfsetmiterjoin%
\definecolor{currentfill}{rgb}{0.121569,0.466667,0.705882}%
\pgfsetfillcolor{currentfill}%
\pgfsetlinewidth{0.000000pt}%
\definecolor{currentstroke}{rgb}{0.000000,0.000000,0.000000}%
\pgfsetstrokecolor{currentstroke}%
\pgfsetstrokeopacity{0.000000}%
\pgfsetdash{}{0pt}%
\pgfpathmoveto{\pgfqpoint{5.800000in}{25.351041in}}%
\pgfpathlineto{\pgfqpoint{5.800000in}{25.351041in}}%
\pgfpathlineto{\pgfqpoint{5.800000in}{25.468893in}}%
\pgfpathlineto{\pgfqpoint{5.800000in}{25.468893in}}%
\pgfpathclose%
\pgfusepath{fill}%
\end{pgfscope}%
\begin{pgfscope}%
\pgfpathrectangle{\pgfqpoint{5.800000in}{0.720000in}}{\pgfqpoint{1.400000in}{4.620000in}}%
\pgfusepath{clip}%
\pgfsetbuttcap%
\pgfsetmiterjoin%
\definecolor{currentfill}{rgb}{0.121569,0.466667,0.705882}%
\pgfsetfillcolor{currentfill}%
\pgfsetlinewidth{0.000000pt}%
\definecolor{currentstroke}{rgb}{0.000000,0.000000,0.000000}%
\pgfsetstrokecolor{currentstroke}%
\pgfsetstrokeopacity{0.000000}%
\pgfsetdash{}{0pt}%
\pgfpathmoveto{\pgfqpoint{5.800000in}{25.468893in}}%
\pgfpathlineto{\pgfqpoint{5.800000in}{25.468893in}}%
\pgfpathlineto{\pgfqpoint{5.800000in}{25.586745in}}%
\pgfpathlineto{\pgfqpoint{5.800000in}{25.586745in}}%
\pgfpathclose%
\pgfusepath{fill}%
\end{pgfscope}%
\begin{pgfscope}%
\pgfpathrectangle{\pgfqpoint{5.800000in}{0.720000in}}{\pgfqpoint{1.400000in}{4.620000in}}%
\pgfusepath{clip}%
\pgfsetbuttcap%
\pgfsetmiterjoin%
\definecolor{currentfill}{rgb}{0.121569,0.466667,0.705882}%
\pgfsetfillcolor{currentfill}%
\pgfsetlinewidth{0.000000pt}%
\definecolor{currentstroke}{rgb}{0.000000,0.000000,0.000000}%
\pgfsetstrokecolor{currentstroke}%
\pgfsetstrokeopacity{0.000000}%
\pgfsetdash{}{0pt}%
\pgfpathmoveto{\pgfqpoint{5.800000in}{25.586745in}}%
\pgfpathlineto{\pgfqpoint{5.800000in}{25.586745in}}%
\pgfpathlineto{\pgfqpoint{5.800000in}{25.704597in}}%
\pgfpathlineto{\pgfqpoint{5.800000in}{25.704597in}}%
\pgfpathclose%
\pgfusepath{fill}%
\end{pgfscope}%
\begin{pgfscope}%
\pgfpathrectangle{\pgfqpoint{5.800000in}{0.720000in}}{\pgfqpoint{1.400000in}{4.620000in}}%
\pgfusepath{clip}%
\pgfsetbuttcap%
\pgfsetmiterjoin%
\definecolor{currentfill}{rgb}{0.121569,0.466667,0.705882}%
\pgfsetfillcolor{currentfill}%
\pgfsetlinewidth{0.000000pt}%
\definecolor{currentstroke}{rgb}{0.000000,0.000000,0.000000}%
\pgfsetstrokecolor{currentstroke}%
\pgfsetstrokeopacity{0.000000}%
\pgfsetdash{}{0pt}%
\pgfpathmoveto{\pgfqpoint{5.800000in}{25.704597in}}%
\pgfpathlineto{\pgfqpoint{5.800000in}{25.704597in}}%
\pgfpathlineto{\pgfqpoint{5.800000in}{25.822448in}}%
\pgfpathlineto{\pgfqpoint{5.800000in}{25.822448in}}%
\pgfpathclose%
\pgfusepath{fill}%
\end{pgfscope}%
\begin{pgfscope}%
\pgfpathrectangle{\pgfqpoint{5.800000in}{0.720000in}}{\pgfqpoint{1.400000in}{4.620000in}}%
\pgfusepath{clip}%
\pgfsetbuttcap%
\pgfsetmiterjoin%
\definecolor{currentfill}{rgb}{0.121569,0.466667,0.705882}%
\pgfsetfillcolor{currentfill}%
\pgfsetlinewidth{0.000000pt}%
\definecolor{currentstroke}{rgb}{0.000000,0.000000,0.000000}%
\pgfsetstrokecolor{currentstroke}%
\pgfsetstrokeopacity{0.000000}%
\pgfsetdash{}{0pt}%
\pgfpathmoveto{\pgfqpoint{5.800000in}{25.822448in}}%
\pgfpathlineto{\pgfqpoint{5.800000in}{25.822448in}}%
\pgfpathlineto{\pgfqpoint{5.800000in}{25.940300in}}%
\pgfpathlineto{\pgfqpoint{5.800000in}{25.940300in}}%
\pgfpathclose%
\pgfusepath{fill}%
\end{pgfscope}%
\begin{pgfscope}%
\pgfpathrectangle{\pgfqpoint{5.800000in}{0.720000in}}{\pgfqpoint{1.400000in}{4.620000in}}%
\pgfusepath{clip}%
\pgfsetbuttcap%
\pgfsetmiterjoin%
\definecolor{currentfill}{rgb}{0.121569,0.466667,0.705882}%
\pgfsetfillcolor{currentfill}%
\pgfsetlinewidth{0.000000pt}%
\definecolor{currentstroke}{rgb}{0.000000,0.000000,0.000000}%
\pgfsetstrokecolor{currentstroke}%
\pgfsetstrokeopacity{0.000000}%
\pgfsetdash{}{0pt}%
\pgfpathmoveto{\pgfqpoint{5.800000in}{25.940300in}}%
\pgfpathlineto{\pgfqpoint{5.802160in}{25.940300in}}%
\pgfpathlineto{\pgfqpoint{5.802160in}{26.058152in}}%
\pgfpathlineto{\pgfqpoint{5.800000in}{26.058152in}}%
\pgfpathclose%
\pgfusepath{fill}%
\end{pgfscope}%
\begin{pgfscope}%
\pgfpathrectangle{\pgfqpoint{5.800000in}{0.720000in}}{\pgfqpoint{1.400000in}{4.620000in}}%
\pgfusepath{clip}%
\pgfsetbuttcap%
\pgfsetmiterjoin%
\definecolor{currentfill}{rgb}{0.121569,0.466667,0.705882}%
\pgfsetfillcolor{currentfill}%
\pgfsetlinewidth{0.000000pt}%
\definecolor{currentstroke}{rgb}{0.000000,0.000000,0.000000}%
\pgfsetstrokecolor{currentstroke}%
\pgfsetstrokeopacity{0.000000}%
\pgfsetdash{}{0pt}%
\pgfpathmoveto{\pgfqpoint{5.800000in}{26.058152in}}%
\pgfpathlineto{\pgfqpoint{5.800000in}{26.058152in}}%
\pgfpathlineto{\pgfqpoint{5.800000in}{26.176004in}}%
\pgfpathlineto{\pgfqpoint{5.800000in}{26.176004in}}%
\pgfpathclose%
\pgfusepath{fill}%
\end{pgfscope}%
\begin{pgfscope}%
\pgfpathrectangle{\pgfqpoint{5.800000in}{0.720000in}}{\pgfqpoint{1.400000in}{4.620000in}}%
\pgfusepath{clip}%
\pgfsetbuttcap%
\pgfsetmiterjoin%
\definecolor{currentfill}{rgb}{0.121569,0.466667,0.705882}%
\pgfsetfillcolor{currentfill}%
\pgfsetlinewidth{0.000000pt}%
\definecolor{currentstroke}{rgb}{0.000000,0.000000,0.000000}%
\pgfsetstrokecolor{currentstroke}%
\pgfsetstrokeopacity{0.000000}%
\pgfsetdash{}{0pt}%
\pgfpathmoveto{\pgfqpoint{5.800000in}{26.176004in}}%
\pgfpathlineto{\pgfqpoint{5.800000in}{26.176004in}}%
\pgfpathlineto{\pgfqpoint{5.800000in}{26.293856in}}%
\pgfpathlineto{\pgfqpoint{5.800000in}{26.293856in}}%
\pgfpathclose%
\pgfusepath{fill}%
\end{pgfscope}%
\begin{pgfscope}%
\pgfpathrectangle{\pgfqpoint{5.800000in}{0.720000in}}{\pgfqpoint{1.400000in}{4.620000in}}%
\pgfusepath{clip}%
\pgfsetbuttcap%
\pgfsetmiterjoin%
\definecolor{currentfill}{rgb}{0.121569,0.466667,0.705882}%
\pgfsetfillcolor{currentfill}%
\pgfsetlinewidth{0.000000pt}%
\definecolor{currentstroke}{rgb}{0.000000,0.000000,0.000000}%
\pgfsetstrokecolor{currentstroke}%
\pgfsetstrokeopacity{0.000000}%
\pgfsetdash{}{0pt}%
\pgfpathmoveto{\pgfqpoint{5.800000in}{26.293856in}}%
\pgfpathlineto{\pgfqpoint{5.800000in}{26.293856in}}%
\pgfpathlineto{\pgfqpoint{5.800000in}{26.411708in}}%
\pgfpathlineto{\pgfqpoint{5.800000in}{26.411708in}}%
\pgfpathclose%
\pgfusepath{fill}%
\end{pgfscope}%
\begin{pgfscope}%
\pgfpathrectangle{\pgfqpoint{5.800000in}{0.720000in}}{\pgfqpoint{1.400000in}{4.620000in}}%
\pgfusepath{clip}%
\pgfsetbuttcap%
\pgfsetmiterjoin%
\definecolor{currentfill}{rgb}{0.121569,0.466667,0.705882}%
\pgfsetfillcolor{currentfill}%
\pgfsetlinewidth{0.000000pt}%
\definecolor{currentstroke}{rgb}{0.000000,0.000000,0.000000}%
\pgfsetstrokecolor{currentstroke}%
\pgfsetstrokeopacity{0.000000}%
\pgfsetdash{}{0pt}%
\pgfpathmoveto{\pgfqpoint{5.800000in}{26.411708in}}%
\pgfpathlineto{\pgfqpoint{5.800000in}{26.411708in}}%
\pgfpathlineto{\pgfqpoint{5.800000in}{26.529560in}}%
\pgfpathlineto{\pgfqpoint{5.800000in}{26.529560in}}%
\pgfpathclose%
\pgfusepath{fill}%
\end{pgfscope}%
\begin{pgfscope}%
\pgfpathrectangle{\pgfqpoint{5.800000in}{0.720000in}}{\pgfqpoint{1.400000in}{4.620000in}}%
\pgfusepath{clip}%
\pgfsetbuttcap%
\pgfsetmiterjoin%
\definecolor{currentfill}{rgb}{0.121569,0.466667,0.705882}%
\pgfsetfillcolor{currentfill}%
\pgfsetlinewidth{0.000000pt}%
\definecolor{currentstroke}{rgb}{0.000000,0.000000,0.000000}%
\pgfsetstrokecolor{currentstroke}%
\pgfsetstrokeopacity{0.000000}%
\pgfsetdash{}{0pt}%
\pgfpathmoveto{\pgfqpoint{5.800000in}{26.529560in}}%
\pgfpathlineto{\pgfqpoint{5.800000in}{26.529560in}}%
\pgfpathlineto{\pgfqpoint{5.800000in}{26.647411in}}%
\pgfpathlineto{\pgfqpoint{5.800000in}{26.647411in}}%
\pgfpathclose%
\pgfusepath{fill}%
\end{pgfscope}%
\begin{pgfscope}%
\pgfpathrectangle{\pgfqpoint{5.800000in}{0.720000in}}{\pgfqpoint{1.400000in}{4.620000in}}%
\pgfusepath{clip}%
\pgfsetbuttcap%
\pgfsetmiterjoin%
\definecolor{currentfill}{rgb}{0.121569,0.466667,0.705882}%
\pgfsetfillcolor{currentfill}%
\pgfsetlinewidth{0.000000pt}%
\definecolor{currentstroke}{rgb}{0.000000,0.000000,0.000000}%
\pgfsetstrokecolor{currentstroke}%
\pgfsetstrokeopacity{0.000000}%
\pgfsetdash{}{0pt}%
\pgfpathmoveto{\pgfqpoint{5.800000in}{26.647411in}}%
\pgfpathlineto{\pgfqpoint{5.800000in}{26.647411in}}%
\pgfpathlineto{\pgfqpoint{5.800000in}{26.765263in}}%
\pgfpathlineto{\pgfqpoint{5.800000in}{26.765263in}}%
\pgfpathclose%
\pgfusepath{fill}%
\end{pgfscope}%
\begin{pgfscope}%
\pgfpathrectangle{\pgfqpoint{5.800000in}{0.720000in}}{\pgfqpoint{1.400000in}{4.620000in}}%
\pgfusepath{clip}%
\pgfsetbuttcap%
\pgfsetmiterjoin%
\definecolor{currentfill}{rgb}{0.121569,0.466667,0.705882}%
\pgfsetfillcolor{currentfill}%
\pgfsetlinewidth{0.000000pt}%
\definecolor{currentstroke}{rgb}{0.000000,0.000000,0.000000}%
\pgfsetstrokecolor{currentstroke}%
\pgfsetstrokeopacity{0.000000}%
\pgfsetdash{}{0pt}%
\pgfpathmoveto{\pgfqpoint{5.800000in}{26.765263in}}%
\pgfpathlineto{\pgfqpoint{5.800000in}{26.765263in}}%
\pgfpathlineto{\pgfqpoint{5.800000in}{26.883115in}}%
\pgfpathlineto{\pgfqpoint{5.800000in}{26.883115in}}%
\pgfpathclose%
\pgfusepath{fill}%
\end{pgfscope}%
\begin{pgfscope}%
\pgfpathrectangle{\pgfqpoint{5.800000in}{0.720000in}}{\pgfqpoint{1.400000in}{4.620000in}}%
\pgfusepath{clip}%
\pgfsetbuttcap%
\pgfsetmiterjoin%
\definecolor{currentfill}{rgb}{0.121569,0.466667,0.705882}%
\pgfsetfillcolor{currentfill}%
\pgfsetlinewidth{0.000000pt}%
\definecolor{currentstroke}{rgb}{0.000000,0.000000,0.000000}%
\pgfsetstrokecolor{currentstroke}%
\pgfsetstrokeopacity{0.000000}%
\pgfsetdash{}{0pt}%
\pgfpathmoveto{\pgfqpoint{5.800000in}{26.883115in}}%
\pgfpathlineto{\pgfqpoint{5.800000in}{26.883115in}}%
\pgfpathlineto{\pgfqpoint{5.800000in}{27.000967in}}%
\pgfpathlineto{\pgfqpoint{5.800000in}{27.000967in}}%
\pgfpathclose%
\pgfusepath{fill}%
\end{pgfscope}%
\begin{pgfscope}%
\pgfpathrectangle{\pgfqpoint{5.800000in}{0.720000in}}{\pgfqpoint{1.400000in}{4.620000in}}%
\pgfusepath{clip}%
\pgfsetbuttcap%
\pgfsetmiterjoin%
\definecolor{currentfill}{rgb}{0.121569,0.466667,0.705882}%
\pgfsetfillcolor{currentfill}%
\pgfsetlinewidth{0.000000pt}%
\definecolor{currentstroke}{rgb}{0.000000,0.000000,0.000000}%
\pgfsetstrokecolor{currentstroke}%
\pgfsetstrokeopacity{0.000000}%
\pgfsetdash{}{0pt}%
\pgfpathmoveto{\pgfqpoint{5.800000in}{27.000967in}}%
\pgfpathlineto{\pgfqpoint{5.800000in}{27.000967in}}%
\pgfpathlineto{\pgfqpoint{5.800000in}{27.118819in}}%
\pgfpathlineto{\pgfqpoint{5.800000in}{27.118819in}}%
\pgfpathclose%
\pgfusepath{fill}%
\end{pgfscope}%
\begin{pgfscope}%
\pgfpathrectangle{\pgfqpoint{5.800000in}{0.720000in}}{\pgfqpoint{1.400000in}{4.620000in}}%
\pgfusepath{clip}%
\pgfsetbuttcap%
\pgfsetmiterjoin%
\definecolor{currentfill}{rgb}{0.121569,0.466667,0.705882}%
\pgfsetfillcolor{currentfill}%
\pgfsetlinewidth{0.000000pt}%
\definecolor{currentstroke}{rgb}{0.000000,0.000000,0.000000}%
\pgfsetstrokecolor{currentstroke}%
\pgfsetstrokeopacity{0.000000}%
\pgfsetdash{}{0pt}%
\pgfpathmoveto{\pgfqpoint{5.800000in}{27.118819in}}%
\pgfpathlineto{\pgfqpoint{5.800000in}{27.118819in}}%
\pgfpathlineto{\pgfqpoint{5.800000in}{27.236671in}}%
\pgfpathlineto{\pgfqpoint{5.800000in}{27.236671in}}%
\pgfpathclose%
\pgfusepath{fill}%
\end{pgfscope}%
\begin{pgfscope}%
\pgfpathrectangle{\pgfqpoint{5.800000in}{0.720000in}}{\pgfqpoint{1.400000in}{4.620000in}}%
\pgfusepath{clip}%
\pgfsetbuttcap%
\pgfsetmiterjoin%
\definecolor{currentfill}{rgb}{0.121569,0.466667,0.705882}%
\pgfsetfillcolor{currentfill}%
\pgfsetlinewidth{0.000000pt}%
\definecolor{currentstroke}{rgb}{0.000000,0.000000,0.000000}%
\pgfsetstrokecolor{currentstroke}%
\pgfsetstrokeopacity{0.000000}%
\pgfsetdash{}{0pt}%
\pgfpathmoveto{\pgfqpoint{5.800000in}{27.236671in}}%
\pgfpathlineto{\pgfqpoint{5.800000in}{27.236671in}}%
\pgfpathlineto{\pgfqpoint{5.800000in}{27.354523in}}%
\pgfpathlineto{\pgfqpoint{5.800000in}{27.354523in}}%
\pgfpathclose%
\pgfusepath{fill}%
\end{pgfscope}%
\begin{pgfscope}%
\pgfpathrectangle{\pgfqpoint{5.800000in}{0.720000in}}{\pgfqpoint{1.400000in}{4.620000in}}%
\pgfusepath{clip}%
\pgfsetbuttcap%
\pgfsetmiterjoin%
\definecolor{currentfill}{rgb}{0.121569,0.466667,0.705882}%
\pgfsetfillcolor{currentfill}%
\pgfsetlinewidth{0.000000pt}%
\definecolor{currentstroke}{rgb}{0.000000,0.000000,0.000000}%
\pgfsetstrokecolor{currentstroke}%
\pgfsetstrokeopacity{0.000000}%
\pgfsetdash{}{0pt}%
\pgfpathmoveto{\pgfqpoint{5.800000in}{27.354523in}}%
\pgfpathlineto{\pgfqpoint{5.800000in}{27.354523in}}%
\pgfpathlineto{\pgfqpoint{5.800000in}{27.472375in}}%
\pgfpathlineto{\pgfqpoint{5.800000in}{27.472375in}}%
\pgfpathclose%
\pgfusepath{fill}%
\end{pgfscope}%
\begin{pgfscope}%
\pgfpathrectangle{\pgfqpoint{5.800000in}{0.720000in}}{\pgfqpoint{1.400000in}{4.620000in}}%
\pgfusepath{clip}%
\pgfsetbuttcap%
\pgfsetmiterjoin%
\definecolor{currentfill}{rgb}{0.121569,0.466667,0.705882}%
\pgfsetfillcolor{currentfill}%
\pgfsetlinewidth{0.000000pt}%
\definecolor{currentstroke}{rgb}{0.000000,0.000000,0.000000}%
\pgfsetstrokecolor{currentstroke}%
\pgfsetstrokeopacity{0.000000}%
\pgfsetdash{}{0pt}%
\pgfpathmoveto{\pgfqpoint{5.800000in}{27.472375in}}%
\pgfpathlineto{\pgfqpoint{5.800000in}{27.472375in}}%
\pgfpathlineto{\pgfqpoint{5.800000in}{27.590226in}}%
\pgfpathlineto{\pgfqpoint{5.800000in}{27.590226in}}%
\pgfpathclose%
\pgfusepath{fill}%
\end{pgfscope}%
\begin{pgfscope}%
\pgfpathrectangle{\pgfqpoint{5.800000in}{0.720000in}}{\pgfqpoint{1.400000in}{4.620000in}}%
\pgfusepath{clip}%
\pgfsetbuttcap%
\pgfsetmiterjoin%
\definecolor{currentfill}{rgb}{0.121569,0.466667,0.705882}%
\pgfsetfillcolor{currentfill}%
\pgfsetlinewidth{0.000000pt}%
\definecolor{currentstroke}{rgb}{0.000000,0.000000,0.000000}%
\pgfsetstrokecolor{currentstroke}%
\pgfsetstrokeopacity{0.000000}%
\pgfsetdash{}{0pt}%
\pgfpathmoveto{\pgfqpoint{5.800000in}{27.590226in}}%
\pgfpathlineto{\pgfqpoint{5.800000in}{27.590226in}}%
\pgfpathlineto{\pgfqpoint{5.800000in}{27.708078in}}%
\pgfpathlineto{\pgfqpoint{5.800000in}{27.708078in}}%
\pgfpathclose%
\pgfusepath{fill}%
\end{pgfscope}%
\begin{pgfscope}%
\pgfpathrectangle{\pgfqpoint{5.800000in}{0.720000in}}{\pgfqpoint{1.400000in}{4.620000in}}%
\pgfusepath{clip}%
\pgfsetbuttcap%
\pgfsetmiterjoin%
\definecolor{currentfill}{rgb}{0.121569,0.466667,0.705882}%
\pgfsetfillcolor{currentfill}%
\pgfsetlinewidth{0.000000pt}%
\definecolor{currentstroke}{rgb}{0.000000,0.000000,0.000000}%
\pgfsetstrokecolor{currentstroke}%
\pgfsetstrokeopacity{0.000000}%
\pgfsetdash{}{0pt}%
\pgfpathmoveto{\pgfqpoint{5.800000in}{27.708078in}}%
\pgfpathlineto{\pgfqpoint{5.800000in}{27.708078in}}%
\pgfpathlineto{\pgfqpoint{5.800000in}{27.825930in}}%
\pgfpathlineto{\pgfqpoint{5.800000in}{27.825930in}}%
\pgfpathclose%
\pgfusepath{fill}%
\end{pgfscope}%
\begin{pgfscope}%
\pgfpathrectangle{\pgfqpoint{5.800000in}{0.720000in}}{\pgfqpoint{1.400000in}{4.620000in}}%
\pgfusepath{clip}%
\pgfsetbuttcap%
\pgfsetmiterjoin%
\definecolor{currentfill}{rgb}{0.121569,0.466667,0.705882}%
\pgfsetfillcolor{currentfill}%
\pgfsetlinewidth{0.000000pt}%
\definecolor{currentstroke}{rgb}{0.000000,0.000000,0.000000}%
\pgfsetstrokecolor{currentstroke}%
\pgfsetstrokeopacity{0.000000}%
\pgfsetdash{}{0pt}%
\pgfpathmoveto{\pgfqpoint{5.800000in}{27.825930in}}%
\pgfpathlineto{\pgfqpoint{5.800000in}{27.825930in}}%
\pgfpathlineto{\pgfqpoint{5.800000in}{27.943782in}}%
\pgfpathlineto{\pgfqpoint{5.800000in}{27.943782in}}%
\pgfpathclose%
\pgfusepath{fill}%
\end{pgfscope}%
\begin{pgfscope}%
\pgfpathrectangle{\pgfqpoint{5.800000in}{0.720000in}}{\pgfqpoint{1.400000in}{4.620000in}}%
\pgfusepath{clip}%
\pgfsetbuttcap%
\pgfsetmiterjoin%
\definecolor{currentfill}{rgb}{0.121569,0.466667,0.705882}%
\pgfsetfillcolor{currentfill}%
\pgfsetlinewidth{0.000000pt}%
\definecolor{currentstroke}{rgb}{0.000000,0.000000,0.000000}%
\pgfsetstrokecolor{currentstroke}%
\pgfsetstrokeopacity{0.000000}%
\pgfsetdash{}{0pt}%
\pgfpathmoveto{\pgfqpoint{5.800000in}{27.943782in}}%
\pgfpathlineto{\pgfqpoint{5.800000in}{27.943782in}}%
\pgfpathlineto{\pgfqpoint{5.800000in}{28.061634in}}%
\pgfpathlineto{\pgfqpoint{5.800000in}{28.061634in}}%
\pgfpathclose%
\pgfusepath{fill}%
\end{pgfscope}%
\begin{pgfscope}%
\pgfpathrectangle{\pgfqpoint{5.800000in}{0.720000in}}{\pgfqpoint{1.400000in}{4.620000in}}%
\pgfusepath{clip}%
\pgfsetbuttcap%
\pgfsetmiterjoin%
\definecolor{currentfill}{rgb}{0.121569,0.466667,0.705882}%
\pgfsetfillcolor{currentfill}%
\pgfsetlinewidth{0.000000pt}%
\definecolor{currentstroke}{rgb}{0.000000,0.000000,0.000000}%
\pgfsetstrokecolor{currentstroke}%
\pgfsetstrokeopacity{0.000000}%
\pgfsetdash{}{0pt}%
\pgfpathmoveto{\pgfqpoint{5.800000in}{28.061634in}}%
\pgfpathlineto{\pgfqpoint{5.800000in}{28.061634in}}%
\pgfpathlineto{\pgfqpoint{5.800000in}{28.179486in}}%
\pgfpathlineto{\pgfqpoint{5.800000in}{28.179486in}}%
\pgfpathclose%
\pgfusepath{fill}%
\end{pgfscope}%
\begin{pgfscope}%
\pgfpathrectangle{\pgfqpoint{5.800000in}{0.720000in}}{\pgfqpoint{1.400000in}{4.620000in}}%
\pgfusepath{clip}%
\pgfsetbuttcap%
\pgfsetmiterjoin%
\definecolor{currentfill}{rgb}{0.121569,0.466667,0.705882}%
\pgfsetfillcolor{currentfill}%
\pgfsetlinewidth{0.000000pt}%
\definecolor{currentstroke}{rgb}{0.000000,0.000000,0.000000}%
\pgfsetstrokecolor{currentstroke}%
\pgfsetstrokeopacity{0.000000}%
\pgfsetdash{}{0pt}%
\pgfpathmoveto{\pgfqpoint{5.800000in}{28.179486in}}%
\pgfpathlineto{\pgfqpoint{5.800000in}{28.179486in}}%
\pgfpathlineto{\pgfqpoint{5.800000in}{28.297338in}}%
\pgfpathlineto{\pgfqpoint{5.800000in}{28.297338in}}%
\pgfpathclose%
\pgfusepath{fill}%
\end{pgfscope}%
\begin{pgfscope}%
\pgfpathrectangle{\pgfqpoint{5.800000in}{0.720000in}}{\pgfqpoint{1.400000in}{4.620000in}}%
\pgfusepath{clip}%
\pgfsetbuttcap%
\pgfsetmiterjoin%
\definecolor{currentfill}{rgb}{0.121569,0.466667,0.705882}%
\pgfsetfillcolor{currentfill}%
\pgfsetlinewidth{0.000000pt}%
\definecolor{currentstroke}{rgb}{0.000000,0.000000,0.000000}%
\pgfsetstrokecolor{currentstroke}%
\pgfsetstrokeopacity{0.000000}%
\pgfsetdash{}{0pt}%
\pgfpathmoveto{\pgfqpoint{5.800000in}{28.297338in}}%
\pgfpathlineto{\pgfqpoint{5.800000in}{28.297338in}}%
\pgfpathlineto{\pgfqpoint{5.800000in}{28.415190in}}%
\pgfpathlineto{\pgfqpoint{5.800000in}{28.415190in}}%
\pgfpathclose%
\pgfusepath{fill}%
\end{pgfscope}%
\begin{pgfscope}%
\pgfpathrectangle{\pgfqpoint{5.800000in}{0.720000in}}{\pgfqpoint{1.400000in}{4.620000in}}%
\pgfusepath{clip}%
\pgfsetbuttcap%
\pgfsetmiterjoin%
\definecolor{currentfill}{rgb}{0.121569,0.466667,0.705882}%
\pgfsetfillcolor{currentfill}%
\pgfsetlinewidth{0.000000pt}%
\definecolor{currentstroke}{rgb}{0.000000,0.000000,0.000000}%
\pgfsetstrokecolor{currentstroke}%
\pgfsetstrokeopacity{0.000000}%
\pgfsetdash{}{0pt}%
\pgfpathmoveto{\pgfqpoint{5.800000in}{28.415190in}}%
\pgfpathlineto{\pgfqpoint{5.800000in}{28.415190in}}%
\pgfpathlineto{\pgfqpoint{5.800000in}{28.533041in}}%
\pgfpathlineto{\pgfqpoint{5.800000in}{28.533041in}}%
\pgfpathclose%
\pgfusepath{fill}%
\end{pgfscope}%
\begin{pgfscope}%
\pgfpathrectangle{\pgfqpoint{5.800000in}{0.720000in}}{\pgfqpoint{1.400000in}{4.620000in}}%
\pgfusepath{clip}%
\pgfsetbuttcap%
\pgfsetmiterjoin%
\definecolor{currentfill}{rgb}{0.121569,0.466667,0.705882}%
\pgfsetfillcolor{currentfill}%
\pgfsetlinewidth{0.000000pt}%
\definecolor{currentstroke}{rgb}{0.000000,0.000000,0.000000}%
\pgfsetstrokecolor{currentstroke}%
\pgfsetstrokeopacity{0.000000}%
\pgfsetdash{}{0pt}%
\pgfpathmoveto{\pgfqpoint{5.800000in}{28.533041in}}%
\pgfpathlineto{\pgfqpoint{5.800000in}{28.533041in}}%
\pgfpathlineto{\pgfqpoint{5.800000in}{28.650893in}}%
\pgfpathlineto{\pgfqpoint{5.800000in}{28.650893in}}%
\pgfpathclose%
\pgfusepath{fill}%
\end{pgfscope}%
\begin{pgfscope}%
\pgfpathrectangle{\pgfqpoint{5.800000in}{0.720000in}}{\pgfqpoint{1.400000in}{4.620000in}}%
\pgfusepath{clip}%
\pgfsetbuttcap%
\pgfsetmiterjoin%
\definecolor{currentfill}{rgb}{0.121569,0.466667,0.705882}%
\pgfsetfillcolor{currentfill}%
\pgfsetlinewidth{0.000000pt}%
\definecolor{currentstroke}{rgb}{0.000000,0.000000,0.000000}%
\pgfsetstrokecolor{currentstroke}%
\pgfsetstrokeopacity{0.000000}%
\pgfsetdash{}{0pt}%
\pgfpathmoveto{\pgfqpoint{5.800000in}{28.650893in}}%
\pgfpathlineto{\pgfqpoint{5.800000in}{28.650893in}}%
\pgfpathlineto{\pgfqpoint{5.800000in}{28.768745in}}%
\pgfpathlineto{\pgfqpoint{5.800000in}{28.768745in}}%
\pgfpathclose%
\pgfusepath{fill}%
\end{pgfscope}%
\begin{pgfscope}%
\pgfpathrectangle{\pgfqpoint{5.800000in}{0.720000in}}{\pgfqpoint{1.400000in}{4.620000in}}%
\pgfusepath{clip}%
\pgfsetbuttcap%
\pgfsetmiterjoin%
\definecolor{currentfill}{rgb}{0.121569,0.466667,0.705882}%
\pgfsetfillcolor{currentfill}%
\pgfsetlinewidth{0.000000pt}%
\definecolor{currentstroke}{rgb}{0.000000,0.000000,0.000000}%
\pgfsetstrokecolor{currentstroke}%
\pgfsetstrokeopacity{0.000000}%
\pgfsetdash{}{0pt}%
\pgfpathmoveto{\pgfqpoint{5.800000in}{28.768745in}}%
\pgfpathlineto{\pgfqpoint{5.800000in}{28.768745in}}%
\pgfpathlineto{\pgfqpoint{5.800000in}{28.886597in}}%
\pgfpathlineto{\pgfqpoint{5.800000in}{28.886597in}}%
\pgfpathclose%
\pgfusepath{fill}%
\end{pgfscope}%
\begin{pgfscope}%
\pgfpathrectangle{\pgfqpoint{5.800000in}{0.720000in}}{\pgfqpoint{1.400000in}{4.620000in}}%
\pgfusepath{clip}%
\pgfsetbuttcap%
\pgfsetmiterjoin%
\definecolor{currentfill}{rgb}{0.121569,0.466667,0.705882}%
\pgfsetfillcolor{currentfill}%
\pgfsetlinewidth{0.000000pt}%
\definecolor{currentstroke}{rgb}{0.000000,0.000000,0.000000}%
\pgfsetstrokecolor{currentstroke}%
\pgfsetstrokeopacity{0.000000}%
\pgfsetdash{}{0pt}%
\pgfpathmoveto{\pgfqpoint{5.800000in}{28.886597in}}%
\pgfpathlineto{\pgfqpoint{5.800000in}{28.886597in}}%
\pgfpathlineto{\pgfqpoint{5.800000in}{29.004449in}}%
\pgfpathlineto{\pgfqpoint{5.800000in}{29.004449in}}%
\pgfpathclose%
\pgfusepath{fill}%
\end{pgfscope}%
\begin{pgfscope}%
\pgfpathrectangle{\pgfqpoint{5.800000in}{0.720000in}}{\pgfqpoint{1.400000in}{4.620000in}}%
\pgfusepath{clip}%
\pgfsetbuttcap%
\pgfsetmiterjoin%
\definecolor{currentfill}{rgb}{0.121569,0.466667,0.705882}%
\pgfsetfillcolor{currentfill}%
\pgfsetlinewidth{0.000000pt}%
\definecolor{currentstroke}{rgb}{0.000000,0.000000,0.000000}%
\pgfsetstrokecolor{currentstroke}%
\pgfsetstrokeopacity{0.000000}%
\pgfsetdash{}{0pt}%
\pgfpathmoveto{\pgfqpoint{5.800000in}{29.004449in}}%
\pgfpathlineto{\pgfqpoint{5.800000in}{29.004449in}}%
\pgfpathlineto{\pgfqpoint{5.800000in}{29.122301in}}%
\pgfpathlineto{\pgfqpoint{5.800000in}{29.122301in}}%
\pgfpathclose%
\pgfusepath{fill}%
\end{pgfscope}%
\begin{pgfscope}%
\pgfpathrectangle{\pgfqpoint{5.800000in}{0.720000in}}{\pgfqpoint{1.400000in}{4.620000in}}%
\pgfusepath{clip}%
\pgfsetbuttcap%
\pgfsetmiterjoin%
\definecolor{currentfill}{rgb}{0.121569,0.466667,0.705882}%
\pgfsetfillcolor{currentfill}%
\pgfsetlinewidth{0.000000pt}%
\definecolor{currentstroke}{rgb}{0.000000,0.000000,0.000000}%
\pgfsetstrokecolor{currentstroke}%
\pgfsetstrokeopacity{0.000000}%
\pgfsetdash{}{0pt}%
\pgfpathmoveto{\pgfqpoint{5.800000in}{29.122301in}}%
\pgfpathlineto{\pgfqpoint{5.800000in}{29.122301in}}%
\pgfpathlineto{\pgfqpoint{5.800000in}{29.240153in}}%
\pgfpathlineto{\pgfqpoint{5.800000in}{29.240153in}}%
\pgfpathclose%
\pgfusepath{fill}%
\end{pgfscope}%
\begin{pgfscope}%
\pgfpathrectangle{\pgfqpoint{5.800000in}{0.720000in}}{\pgfqpoint{1.400000in}{4.620000in}}%
\pgfusepath{clip}%
\pgfsetbuttcap%
\pgfsetmiterjoin%
\definecolor{currentfill}{rgb}{0.121569,0.466667,0.705882}%
\pgfsetfillcolor{currentfill}%
\pgfsetlinewidth{0.000000pt}%
\definecolor{currentstroke}{rgb}{0.000000,0.000000,0.000000}%
\pgfsetstrokecolor{currentstroke}%
\pgfsetstrokeopacity{0.000000}%
\pgfsetdash{}{0pt}%
\pgfpathmoveto{\pgfqpoint{5.800000in}{29.240153in}}%
\pgfpathlineto{\pgfqpoint{5.800000in}{29.240153in}}%
\pgfpathlineto{\pgfqpoint{5.800000in}{29.358004in}}%
\pgfpathlineto{\pgfqpoint{5.800000in}{29.358004in}}%
\pgfpathclose%
\pgfusepath{fill}%
\end{pgfscope}%
\begin{pgfscope}%
\pgfpathrectangle{\pgfqpoint{5.800000in}{0.720000in}}{\pgfqpoint{1.400000in}{4.620000in}}%
\pgfusepath{clip}%
\pgfsetbuttcap%
\pgfsetmiterjoin%
\definecolor{currentfill}{rgb}{0.121569,0.466667,0.705882}%
\pgfsetfillcolor{currentfill}%
\pgfsetlinewidth{0.000000pt}%
\definecolor{currentstroke}{rgb}{0.000000,0.000000,0.000000}%
\pgfsetstrokecolor{currentstroke}%
\pgfsetstrokeopacity{0.000000}%
\pgfsetdash{}{0pt}%
\pgfpathmoveto{\pgfqpoint{5.800000in}{29.358004in}}%
\pgfpathlineto{\pgfqpoint{5.800000in}{29.358004in}}%
\pgfpathlineto{\pgfqpoint{5.800000in}{29.475856in}}%
\pgfpathlineto{\pgfqpoint{5.800000in}{29.475856in}}%
\pgfpathclose%
\pgfusepath{fill}%
\end{pgfscope}%
\begin{pgfscope}%
\pgfpathrectangle{\pgfqpoint{5.800000in}{0.720000in}}{\pgfqpoint{1.400000in}{4.620000in}}%
\pgfusepath{clip}%
\pgfsetbuttcap%
\pgfsetmiterjoin%
\definecolor{currentfill}{rgb}{0.121569,0.466667,0.705882}%
\pgfsetfillcolor{currentfill}%
\pgfsetlinewidth{0.000000pt}%
\definecolor{currentstroke}{rgb}{0.000000,0.000000,0.000000}%
\pgfsetstrokecolor{currentstroke}%
\pgfsetstrokeopacity{0.000000}%
\pgfsetdash{}{0pt}%
\pgfpathmoveto{\pgfqpoint{5.800000in}{29.475856in}}%
\pgfpathlineto{\pgfqpoint{5.800000in}{29.475856in}}%
\pgfpathlineto{\pgfqpoint{5.800000in}{29.593708in}}%
\pgfpathlineto{\pgfqpoint{5.800000in}{29.593708in}}%
\pgfpathclose%
\pgfusepath{fill}%
\end{pgfscope}%
\begin{pgfscope}%
\pgfpathrectangle{\pgfqpoint{5.800000in}{0.720000in}}{\pgfqpoint{1.400000in}{4.620000in}}%
\pgfusepath{clip}%
\pgfsetbuttcap%
\pgfsetmiterjoin%
\definecolor{currentfill}{rgb}{0.121569,0.466667,0.705882}%
\pgfsetfillcolor{currentfill}%
\pgfsetlinewidth{0.000000pt}%
\definecolor{currentstroke}{rgb}{0.000000,0.000000,0.000000}%
\pgfsetstrokecolor{currentstroke}%
\pgfsetstrokeopacity{0.000000}%
\pgfsetdash{}{0pt}%
\pgfpathmoveto{\pgfqpoint{5.800000in}{29.593708in}}%
\pgfpathlineto{\pgfqpoint{5.800000in}{29.593708in}}%
\pgfpathlineto{\pgfqpoint{5.800000in}{29.711560in}}%
\pgfpathlineto{\pgfqpoint{5.800000in}{29.711560in}}%
\pgfpathclose%
\pgfusepath{fill}%
\end{pgfscope}%
\begin{pgfscope}%
\pgfpathrectangle{\pgfqpoint{5.800000in}{0.720000in}}{\pgfqpoint{1.400000in}{4.620000in}}%
\pgfusepath{clip}%
\pgfsetbuttcap%
\pgfsetmiterjoin%
\definecolor{currentfill}{rgb}{0.121569,0.466667,0.705882}%
\pgfsetfillcolor{currentfill}%
\pgfsetlinewidth{0.000000pt}%
\definecolor{currentstroke}{rgb}{0.000000,0.000000,0.000000}%
\pgfsetstrokecolor{currentstroke}%
\pgfsetstrokeopacity{0.000000}%
\pgfsetdash{}{0pt}%
\pgfpathmoveto{\pgfqpoint{5.800000in}{29.711560in}}%
\pgfpathlineto{\pgfqpoint{5.800000in}{29.711560in}}%
\pgfpathlineto{\pgfqpoint{5.800000in}{29.829412in}}%
\pgfpathlineto{\pgfqpoint{5.800000in}{29.829412in}}%
\pgfpathclose%
\pgfusepath{fill}%
\end{pgfscope}%
\begin{pgfscope}%
\pgfpathrectangle{\pgfqpoint{5.800000in}{0.720000in}}{\pgfqpoint{1.400000in}{4.620000in}}%
\pgfusepath{clip}%
\pgfsetbuttcap%
\pgfsetmiterjoin%
\definecolor{currentfill}{rgb}{0.121569,0.466667,0.705882}%
\pgfsetfillcolor{currentfill}%
\pgfsetlinewidth{0.000000pt}%
\definecolor{currentstroke}{rgb}{0.000000,0.000000,0.000000}%
\pgfsetstrokecolor{currentstroke}%
\pgfsetstrokeopacity{0.000000}%
\pgfsetdash{}{0pt}%
\pgfpathmoveto{\pgfqpoint{5.800000in}{29.829412in}}%
\pgfpathlineto{\pgfqpoint{5.802160in}{29.829412in}}%
\pgfpathlineto{\pgfqpoint{5.802160in}{29.947264in}}%
\pgfpathlineto{\pgfqpoint{5.800000in}{29.947264in}}%
\pgfpathclose%
\pgfusepath{fill}%
\end{pgfscope}%
\begin{pgfscope}%
\pgfpathrectangle{\pgfqpoint{5.800000in}{0.720000in}}{\pgfqpoint{1.400000in}{4.620000in}}%
\pgfusepath{clip}%
\pgfsetbuttcap%
\pgfsetmiterjoin%
\definecolor{currentfill}{rgb}{0.121569,0.466667,0.705882}%
\pgfsetfillcolor{currentfill}%
\pgfsetlinewidth{0.000000pt}%
\definecolor{currentstroke}{rgb}{0.000000,0.000000,0.000000}%
\pgfsetstrokecolor{currentstroke}%
\pgfsetstrokeopacity{0.000000}%
\pgfsetdash{}{0pt}%
\pgfpathmoveto{\pgfqpoint{5.800000in}{29.947264in}}%
\pgfpathlineto{\pgfqpoint{5.800000in}{29.947264in}}%
\pgfpathlineto{\pgfqpoint{5.800000in}{30.065116in}}%
\pgfpathlineto{\pgfqpoint{5.800000in}{30.065116in}}%
\pgfpathclose%
\pgfusepath{fill}%
\end{pgfscope}%
\begin{pgfscope}%
\pgfpathrectangle{\pgfqpoint{5.800000in}{0.720000in}}{\pgfqpoint{1.400000in}{4.620000in}}%
\pgfusepath{clip}%
\pgfsetbuttcap%
\pgfsetmiterjoin%
\definecolor{currentfill}{rgb}{0.121569,0.466667,0.705882}%
\pgfsetfillcolor{currentfill}%
\pgfsetlinewidth{0.000000pt}%
\definecolor{currentstroke}{rgb}{0.000000,0.000000,0.000000}%
\pgfsetstrokecolor{currentstroke}%
\pgfsetstrokeopacity{0.000000}%
\pgfsetdash{}{0pt}%
\pgfpathmoveto{\pgfqpoint{5.800000in}{30.065116in}}%
\pgfpathlineto{\pgfqpoint{5.800000in}{30.065116in}}%
\pgfpathlineto{\pgfqpoint{5.800000in}{30.182968in}}%
\pgfpathlineto{\pgfqpoint{5.800000in}{30.182968in}}%
\pgfpathclose%
\pgfusepath{fill}%
\end{pgfscope}%
\begin{pgfscope}%
\pgfpathrectangle{\pgfqpoint{5.800000in}{0.720000in}}{\pgfqpoint{1.400000in}{4.620000in}}%
\pgfusepath{clip}%
\pgfsetbuttcap%
\pgfsetmiterjoin%
\definecolor{currentfill}{rgb}{0.121569,0.466667,0.705882}%
\pgfsetfillcolor{currentfill}%
\pgfsetlinewidth{0.000000pt}%
\definecolor{currentstroke}{rgb}{0.000000,0.000000,0.000000}%
\pgfsetstrokecolor{currentstroke}%
\pgfsetstrokeopacity{0.000000}%
\pgfsetdash{}{0pt}%
\pgfpathmoveto{\pgfqpoint{5.800000in}{30.182968in}}%
\pgfpathlineto{\pgfqpoint{5.800000in}{30.182968in}}%
\pgfpathlineto{\pgfqpoint{5.800000in}{30.300819in}}%
\pgfpathlineto{\pgfqpoint{5.800000in}{30.300819in}}%
\pgfpathclose%
\pgfusepath{fill}%
\end{pgfscope}%
\begin{pgfscope}%
\pgfpathrectangle{\pgfqpoint{5.800000in}{0.720000in}}{\pgfqpoint{1.400000in}{4.620000in}}%
\pgfusepath{clip}%
\pgfsetbuttcap%
\pgfsetmiterjoin%
\definecolor{currentfill}{rgb}{0.121569,0.466667,0.705882}%
\pgfsetfillcolor{currentfill}%
\pgfsetlinewidth{0.000000pt}%
\definecolor{currentstroke}{rgb}{0.000000,0.000000,0.000000}%
\pgfsetstrokecolor{currentstroke}%
\pgfsetstrokeopacity{0.000000}%
\pgfsetdash{}{0pt}%
\pgfpathmoveto{\pgfqpoint{5.800000in}{30.300819in}}%
\pgfpathlineto{\pgfqpoint{5.800000in}{30.300819in}}%
\pgfpathlineto{\pgfqpoint{5.800000in}{30.418671in}}%
\pgfpathlineto{\pgfqpoint{5.800000in}{30.418671in}}%
\pgfpathclose%
\pgfusepath{fill}%
\end{pgfscope}%
\begin{pgfscope}%
\pgfpathrectangle{\pgfqpoint{5.800000in}{0.720000in}}{\pgfqpoint{1.400000in}{4.620000in}}%
\pgfusepath{clip}%
\pgfsetbuttcap%
\pgfsetmiterjoin%
\definecolor{currentfill}{rgb}{0.121569,0.466667,0.705882}%
\pgfsetfillcolor{currentfill}%
\pgfsetlinewidth{0.000000pt}%
\definecolor{currentstroke}{rgb}{0.000000,0.000000,0.000000}%
\pgfsetstrokecolor{currentstroke}%
\pgfsetstrokeopacity{0.000000}%
\pgfsetdash{}{0pt}%
\pgfpathmoveto{\pgfqpoint{5.800000in}{30.418671in}}%
\pgfpathlineto{\pgfqpoint{5.800000in}{30.418671in}}%
\pgfpathlineto{\pgfqpoint{5.800000in}{30.536523in}}%
\pgfpathlineto{\pgfqpoint{5.800000in}{30.536523in}}%
\pgfpathclose%
\pgfusepath{fill}%
\end{pgfscope}%
\begin{pgfscope}%
\pgfpathrectangle{\pgfqpoint{5.800000in}{0.720000in}}{\pgfqpoint{1.400000in}{4.620000in}}%
\pgfusepath{clip}%
\pgfsetbuttcap%
\pgfsetmiterjoin%
\definecolor{currentfill}{rgb}{0.121569,0.466667,0.705882}%
\pgfsetfillcolor{currentfill}%
\pgfsetlinewidth{0.000000pt}%
\definecolor{currentstroke}{rgb}{0.000000,0.000000,0.000000}%
\pgfsetstrokecolor{currentstroke}%
\pgfsetstrokeopacity{0.000000}%
\pgfsetdash{}{0pt}%
\pgfpathmoveto{\pgfqpoint{5.800000in}{30.536523in}}%
\pgfpathlineto{\pgfqpoint{5.800000in}{30.536523in}}%
\pgfpathlineto{\pgfqpoint{5.800000in}{30.654375in}}%
\pgfpathlineto{\pgfqpoint{5.800000in}{30.654375in}}%
\pgfpathclose%
\pgfusepath{fill}%
\end{pgfscope}%
\begin{pgfscope}%
\pgfpathrectangle{\pgfqpoint{5.800000in}{0.720000in}}{\pgfqpoint{1.400000in}{4.620000in}}%
\pgfusepath{clip}%
\pgfsetbuttcap%
\pgfsetmiterjoin%
\definecolor{currentfill}{rgb}{0.121569,0.466667,0.705882}%
\pgfsetfillcolor{currentfill}%
\pgfsetlinewidth{0.000000pt}%
\definecolor{currentstroke}{rgb}{0.000000,0.000000,0.000000}%
\pgfsetstrokecolor{currentstroke}%
\pgfsetstrokeopacity{0.000000}%
\pgfsetdash{}{0pt}%
\pgfpathmoveto{\pgfqpoint{5.800000in}{30.654375in}}%
\pgfpathlineto{\pgfqpoint{5.800000in}{30.654375in}}%
\pgfpathlineto{\pgfqpoint{5.800000in}{30.772227in}}%
\pgfpathlineto{\pgfqpoint{5.800000in}{30.772227in}}%
\pgfpathclose%
\pgfusepath{fill}%
\end{pgfscope}%
\begin{pgfscope}%
\pgfpathrectangle{\pgfqpoint{5.800000in}{0.720000in}}{\pgfqpoint{1.400000in}{4.620000in}}%
\pgfusepath{clip}%
\pgfsetbuttcap%
\pgfsetmiterjoin%
\definecolor{currentfill}{rgb}{0.121569,0.466667,0.705882}%
\pgfsetfillcolor{currentfill}%
\pgfsetlinewidth{0.000000pt}%
\definecolor{currentstroke}{rgb}{0.000000,0.000000,0.000000}%
\pgfsetstrokecolor{currentstroke}%
\pgfsetstrokeopacity{0.000000}%
\pgfsetdash{}{0pt}%
\pgfpathmoveto{\pgfqpoint{5.800000in}{30.772227in}}%
\pgfpathlineto{\pgfqpoint{5.800000in}{30.772227in}}%
\pgfpathlineto{\pgfqpoint{5.800000in}{30.890079in}}%
\pgfpathlineto{\pgfqpoint{5.800000in}{30.890079in}}%
\pgfpathclose%
\pgfusepath{fill}%
\end{pgfscope}%
\begin{pgfscope}%
\pgfpathrectangle{\pgfqpoint{5.800000in}{0.720000in}}{\pgfqpoint{1.400000in}{4.620000in}}%
\pgfusepath{clip}%
\pgfsetbuttcap%
\pgfsetmiterjoin%
\definecolor{currentfill}{rgb}{0.121569,0.466667,0.705882}%
\pgfsetfillcolor{currentfill}%
\pgfsetlinewidth{0.000000pt}%
\definecolor{currentstroke}{rgb}{0.000000,0.000000,0.000000}%
\pgfsetstrokecolor{currentstroke}%
\pgfsetstrokeopacity{0.000000}%
\pgfsetdash{}{0pt}%
\pgfpathmoveto{\pgfqpoint{5.800000in}{30.890079in}}%
\pgfpathlineto{\pgfqpoint{5.800000in}{30.890079in}}%
\pgfpathlineto{\pgfqpoint{5.800000in}{31.007931in}}%
\pgfpathlineto{\pgfqpoint{5.800000in}{31.007931in}}%
\pgfpathclose%
\pgfusepath{fill}%
\end{pgfscope}%
\begin{pgfscope}%
\pgfpathrectangle{\pgfqpoint{5.800000in}{0.720000in}}{\pgfqpoint{1.400000in}{4.620000in}}%
\pgfusepath{clip}%
\pgfsetbuttcap%
\pgfsetmiterjoin%
\definecolor{currentfill}{rgb}{0.121569,0.466667,0.705882}%
\pgfsetfillcolor{currentfill}%
\pgfsetlinewidth{0.000000pt}%
\definecolor{currentstroke}{rgb}{0.000000,0.000000,0.000000}%
\pgfsetstrokecolor{currentstroke}%
\pgfsetstrokeopacity{0.000000}%
\pgfsetdash{}{0pt}%
\pgfpathmoveto{\pgfqpoint{5.800000in}{31.007931in}}%
\pgfpathlineto{\pgfqpoint{5.800000in}{31.007931in}}%
\pgfpathlineto{\pgfqpoint{5.800000in}{31.125783in}}%
\pgfpathlineto{\pgfqpoint{5.800000in}{31.125783in}}%
\pgfpathclose%
\pgfusepath{fill}%
\end{pgfscope}%
\begin{pgfscope}%
\pgfpathrectangle{\pgfqpoint{5.800000in}{0.720000in}}{\pgfqpoint{1.400000in}{4.620000in}}%
\pgfusepath{clip}%
\pgfsetbuttcap%
\pgfsetmiterjoin%
\definecolor{currentfill}{rgb}{0.121569,0.466667,0.705882}%
\pgfsetfillcolor{currentfill}%
\pgfsetlinewidth{0.000000pt}%
\definecolor{currentstroke}{rgb}{0.000000,0.000000,0.000000}%
\pgfsetstrokecolor{currentstroke}%
\pgfsetstrokeopacity{0.000000}%
\pgfsetdash{}{0pt}%
\pgfpathmoveto{\pgfqpoint{5.800000in}{31.125783in}}%
\pgfpathlineto{\pgfqpoint{5.800000in}{31.125783in}}%
\pgfpathlineto{\pgfqpoint{5.800000in}{31.243634in}}%
\pgfpathlineto{\pgfqpoint{5.800000in}{31.243634in}}%
\pgfpathclose%
\pgfusepath{fill}%
\end{pgfscope}%
\begin{pgfscope}%
\pgfpathrectangle{\pgfqpoint{5.800000in}{0.720000in}}{\pgfqpoint{1.400000in}{4.620000in}}%
\pgfusepath{clip}%
\pgfsetbuttcap%
\pgfsetmiterjoin%
\definecolor{currentfill}{rgb}{0.121569,0.466667,0.705882}%
\pgfsetfillcolor{currentfill}%
\pgfsetlinewidth{0.000000pt}%
\definecolor{currentstroke}{rgb}{0.000000,0.000000,0.000000}%
\pgfsetstrokecolor{currentstroke}%
\pgfsetstrokeopacity{0.000000}%
\pgfsetdash{}{0pt}%
\pgfpathmoveto{\pgfqpoint{5.800000in}{31.243634in}}%
\pgfpathlineto{\pgfqpoint{5.800000in}{31.243634in}}%
\pgfpathlineto{\pgfqpoint{5.800000in}{31.361486in}}%
\pgfpathlineto{\pgfqpoint{5.800000in}{31.361486in}}%
\pgfpathclose%
\pgfusepath{fill}%
\end{pgfscope}%
\begin{pgfscope}%
\pgfpathrectangle{\pgfqpoint{5.800000in}{0.720000in}}{\pgfqpoint{1.400000in}{4.620000in}}%
\pgfusepath{clip}%
\pgfsetbuttcap%
\pgfsetmiterjoin%
\definecolor{currentfill}{rgb}{0.121569,0.466667,0.705882}%
\pgfsetfillcolor{currentfill}%
\pgfsetlinewidth{0.000000pt}%
\definecolor{currentstroke}{rgb}{0.000000,0.000000,0.000000}%
\pgfsetstrokecolor{currentstroke}%
\pgfsetstrokeopacity{0.000000}%
\pgfsetdash{}{0pt}%
\pgfpathmoveto{\pgfqpoint{5.800000in}{31.361486in}}%
\pgfpathlineto{\pgfqpoint{5.800000in}{31.361486in}}%
\pgfpathlineto{\pgfqpoint{5.800000in}{31.479338in}}%
\pgfpathlineto{\pgfqpoint{5.800000in}{31.479338in}}%
\pgfpathclose%
\pgfusepath{fill}%
\end{pgfscope}%
\begin{pgfscope}%
\pgfpathrectangle{\pgfqpoint{5.800000in}{0.720000in}}{\pgfqpoint{1.400000in}{4.620000in}}%
\pgfusepath{clip}%
\pgfsetbuttcap%
\pgfsetmiterjoin%
\definecolor{currentfill}{rgb}{0.121569,0.466667,0.705882}%
\pgfsetfillcolor{currentfill}%
\pgfsetlinewidth{0.000000pt}%
\definecolor{currentstroke}{rgb}{0.000000,0.000000,0.000000}%
\pgfsetstrokecolor{currentstroke}%
\pgfsetstrokeopacity{0.000000}%
\pgfsetdash{}{0pt}%
\pgfpathmoveto{\pgfqpoint{5.800000in}{31.479338in}}%
\pgfpathlineto{\pgfqpoint{5.800000in}{31.479338in}}%
\pgfpathlineto{\pgfqpoint{5.800000in}{31.597190in}}%
\pgfpathlineto{\pgfqpoint{5.800000in}{31.597190in}}%
\pgfpathclose%
\pgfusepath{fill}%
\end{pgfscope}%
\begin{pgfscope}%
\pgfpathrectangle{\pgfqpoint{5.800000in}{0.720000in}}{\pgfqpoint{1.400000in}{4.620000in}}%
\pgfusepath{clip}%
\pgfsetbuttcap%
\pgfsetmiterjoin%
\definecolor{currentfill}{rgb}{0.121569,0.466667,0.705882}%
\pgfsetfillcolor{currentfill}%
\pgfsetlinewidth{0.000000pt}%
\definecolor{currentstroke}{rgb}{0.000000,0.000000,0.000000}%
\pgfsetstrokecolor{currentstroke}%
\pgfsetstrokeopacity{0.000000}%
\pgfsetdash{}{0pt}%
\pgfpathmoveto{\pgfqpoint{5.800000in}{31.597190in}}%
\pgfpathlineto{\pgfqpoint{5.800000in}{31.597190in}}%
\pgfpathlineto{\pgfqpoint{5.800000in}{31.715042in}}%
\pgfpathlineto{\pgfqpoint{5.800000in}{31.715042in}}%
\pgfpathclose%
\pgfusepath{fill}%
\end{pgfscope}%
\begin{pgfscope}%
\pgfpathrectangle{\pgfqpoint{5.800000in}{0.720000in}}{\pgfqpoint{1.400000in}{4.620000in}}%
\pgfusepath{clip}%
\pgfsetbuttcap%
\pgfsetmiterjoin%
\definecolor{currentfill}{rgb}{0.121569,0.466667,0.705882}%
\pgfsetfillcolor{currentfill}%
\pgfsetlinewidth{0.000000pt}%
\definecolor{currentstroke}{rgb}{0.000000,0.000000,0.000000}%
\pgfsetstrokecolor{currentstroke}%
\pgfsetstrokeopacity{0.000000}%
\pgfsetdash{}{0pt}%
\pgfpathmoveto{\pgfqpoint{5.800000in}{31.715042in}}%
\pgfpathlineto{\pgfqpoint{5.800000in}{31.715042in}}%
\pgfpathlineto{\pgfqpoint{5.800000in}{31.832894in}}%
\pgfpathlineto{\pgfqpoint{5.800000in}{31.832894in}}%
\pgfpathclose%
\pgfusepath{fill}%
\end{pgfscope}%
\begin{pgfscope}%
\pgfpathrectangle{\pgfqpoint{5.800000in}{0.720000in}}{\pgfqpoint{1.400000in}{4.620000in}}%
\pgfusepath{clip}%
\pgfsetbuttcap%
\pgfsetmiterjoin%
\definecolor{currentfill}{rgb}{0.121569,0.466667,0.705882}%
\pgfsetfillcolor{currentfill}%
\pgfsetlinewidth{0.000000pt}%
\definecolor{currentstroke}{rgb}{0.000000,0.000000,0.000000}%
\pgfsetstrokecolor{currentstroke}%
\pgfsetstrokeopacity{0.000000}%
\pgfsetdash{}{0pt}%
\pgfpathmoveto{\pgfqpoint{5.800000in}{31.832894in}}%
\pgfpathlineto{\pgfqpoint{5.800000in}{31.832894in}}%
\pgfpathlineto{\pgfqpoint{5.800000in}{31.950746in}}%
\pgfpathlineto{\pgfqpoint{5.800000in}{31.950746in}}%
\pgfpathclose%
\pgfusepath{fill}%
\end{pgfscope}%
\begin{pgfscope}%
\pgfpathrectangle{\pgfqpoint{5.800000in}{0.720000in}}{\pgfqpoint{1.400000in}{4.620000in}}%
\pgfusepath{clip}%
\pgfsetbuttcap%
\pgfsetmiterjoin%
\definecolor{currentfill}{rgb}{0.121569,0.466667,0.705882}%
\pgfsetfillcolor{currentfill}%
\pgfsetlinewidth{0.000000pt}%
\definecolor{currentstroke}{rgb}{0.000000,0.000000,0.000000}%
\pgfsetstrokecolor{currentstroke}%
\pgfsetstrokeopacity{0.000000}%
\pgfsetdash{}{0pt}%
\pgfpathmoveto{\pgfqpoint{5.800000in}{31.950746in}}%
\pgfpathlineto{\pgfqpoint{5.802160in}{31.950746in}}%
\pgfpathlineto{\pgfqpoint{5.802160in}{32.068598in}}%
\pgfpathlineto{\pgfqpoint{5.800000in}{32.068598in}}%
\pgfpathclose%
\pgfusepath{fill}%
\end{pgfscope}%
\begin{pgfscope}%
\pgfpathrectangle{\pgfqpoint{5.800000in}{0.720000in}}{\pgfqpoint{1.400000in}{4.620000in}}%
\pgfusepath{clip}%
\pgfsetbuttcap%
\pgfsetmiterjoin%
\definecolor{currentfill}{rgb}{0.121569,0.466667,0.705882}%
\pgfsetfillcolor{currentfill}%
\pgfsetlinewidth{0.000000pt}%
\definecolor{currentstroke}{rgb}{0.000000,0.000000,0.000000}%
\pgfsetstrokecolor{currentstroke}%
\pgfsetstrokeopacity{0.000000}%
\pgfsetdash{}{0pt}%
\pgfpathmoveto{\pgfqpoint{5.800000in}{32.068598in}}%
\pgfpathlineto{\pgfqpoint{5.800000in}{32.068598in}}%
\pgfpathlineto{\pgfqpoint{5.800000in}{32.186449in}}%
\pgfpathlineto{\pgfqpoint{5.800000in}{32.186449in}}%
\pgfpathclose%
\pgfusepath{fill}%
\end{pgfscope}%
\begin{pgfscope}%
\pgfpathrectangle{\pgfqpoint{5.800000in}{0.720000in}}{\pgfqpoint{1.400000in}{4.620000in}}%
\pgfusepath{clip}%
\pgfsetbuttcap%
\pgfsetmiterjoin%
\definecolor{currentfill}{rgb}{0.121569,0.466667,0.705882}%
\pgfsetfillcolor{currentfill}%
\pgfsetlinewidth{0.000000pt}%
\definecolor{currentstroke}{rgb}{0.000000,0.000000,0.000000}%
\pgfsetstrokecolor{currentstroke}%
\pgfsetstrokeopacity{0.000000}%
\pgfsetdash{}{0pt}%
\pgfpathmoveto{\pgfqpoint{5.800000in}{32.186449in}}%
\pgfpathlineto{\pgfqpoint{5.800000in}{32.186449in}}%
\pgfpathlineto{\pgfqpoint{5.800000in}{32.304301in}}%
\pgfpathlineto{\pgfqpoint{5.800000in}{32.304301in}}%
\pgfpathclose%
\pgfusepath{fill}%
\end{pgfscope}%
\begin{pgfscope}%
\pgfpathrectangle{\pgfqpoint{5.800000in}{0.720000in}}{\pgfqpoint{1.400000in}{4.620000in}}%
\pgfusepath{clip}%
\pgfsetbuttcap%
\pgfsetmiterjoin%
\definecolor{currentfill}{rgb}{0.121569,0.466667,0.705882}%
\pgfsetfillcolor{currentfill}%
\pgfsetlinewidth{0.000000pt}%
\definecolor{currentstroke}{rgb}{0.000000,0.000000,0.000000}%
\pgfsetstrokecolor{currentstroke}%
\pgfsetstrokeopacity{0.000000}%
\pgfsetdash{}{0pt}%
\pgfpathmoveto{\pgfqpoint{5.800000in}{32.304301in}}%
\pgfpathlineto{\pgfqpoint{5.800000in}{32.304301in}}%
\pgfpathlineto{\pgfqpoint{5.800000in}{32.422153in}}%
\pgfpathlineto{\pgfqpoint{5.800000in}{32.422153in}}%
\pgfpathclose%
\pgfusepath{fill}%
\end{pgfscope}%
\begin{pgfscope}%
\pgfpathrectangle{\pgfqpoint{5.800000in}{0.720000in}}{\pgfqpoint{1.400000in}{4.620000in}}%
\pgfusepath{clip}%
\pgfsetbuttcap%
\pgfsetmiterjoin%
\definecolor{currentfill}{rgb}{0.121569,0.466667,0.705882}%
\pgfsetfillcolor{currentfill}%
\pgfsetlinewidth{0.000000pt}%
\definecolor{currentstroke}{rgb}{0.000000,0.000000,0.000000}%
\pgfsetstrokecolor{currentstroke}%
\pgfsetstrokeopacity{0.000000}%
\pgfsetdash{}{0pt}%
\pgfpathmoveto{\pgfqpoint{5.800000in}{32.422153in}}%
\pgfpathlineto{\pgfqpoint{5.800000in}{32.422153in}}%
\pgfpathlineto{\pgfqpoint{5.800000in}{32.540005in}}%
\pgfpathlineto{\pgfqpoint{5.800000in}{32.540005in}}%
\pgfpathclose%
\pgfusepath{fill}%
\end{pgfscope}%
\begin{pgfscope}%
\pgfpathrectangle{\pgfqpoint{5.800000in}{0.720000in}}{\pgfqpoint{1.400000in}{4.620000in}}%
\pgfusepath{clip}%
\pgfsetbuttcap%
\pgfsetmiterjoin%
\definecolor{currentfill}{rgb}{0.121569,0.466667,0.705882}%
\pgfsetfillcolor{currentfill}%
\pgfsetlinewidth{0.000000pt}%
\definecolor{currentstroke}{rgb}{0.000000,0.000000,0.000000}%
\pgfsetstrokecolor{currentstroke}%
\pgfsetstrokeopacity{0.000000}%
\pgfsetdash{}{0pt}%
\pgfpathmoveto{\pgfqpoint{5.800000in}{32.540005in}}%
\pgfpathlineto{\pgfqpoint{5.800000in}{32.540005in}}%
\pgfpathlineto{\pgfqpoint{5.800000in}{32.657857in}}%
\pgfpathlineto{\pgfqpoint{5.800000in}{32.657857in}}%
\pgfpathclose%
\pgfusepath{fill}%
\end{pgfscope}%
\begin{pgfscope}%
\pgfpathrectangle{\pgfqpoint{5.800000in}{0.720000in}}{\pgfqpoint{1.400000in}{4.620000in}}%
\pgfusepath{clip}%
\pgfsetbuttcap%
\pgfsetmiterjoin%
\definecolor{currentfill}{rgb}{0.121569,0.466667,0.705882}%
\pgfsetfillcolor{currentfill}%
\pgfsetlinewidth{0.000000pt}%
\definecolor{currentstroke}{rgb}{0.000000,0.000000,0.000000}%
\pgfsetstrokecolor{currentstroke}%
\pgfsetstrokeopacity{0.000000}%
\pgfsetdash{}{0pt}%
\pgfpathmoveto{\pgfqpoint{5.800000in}{32.657857in}}%
\pgfpathlineto{\pgfqpoint{5.800000in}{32.657857in}}%
\pgfpathlineto{\pgfqpoint{5.800000in}{32.775709in}}%
\pgfpathlineto{\pgfqpoint{5.800000in}{32.775709in}}%
\pgfpathclose%
\pgfusepath{fill}%
\end{pgfscope}%
\begin{pgfscope}%
\pgfpathrectangle{\pgfqpoint{5.800000in}{0.720000in}}{\pgfqpoint{1.400000in}{4.620000in}}%
\pgfusepath{clip}%
\pgfsetbuttcap%
\pgfsetmiterjoin%
\definecolor{currentfill}{rgb}{0.121569,0.466667,0.705882}%
\pgfsetfillcolor{currentfill}%
\pgfsetlinewidth{0.000000pt}%
\definecolor{currentstroke}{rgb}{0.000000,0.000000,0.000000}%
\pgfsetstrokecolor{currentstroke}%
\pgfsetstrokeopacity{0.000000}%
\pgfsetdash{}{0pt}%
\pgfpathmoveto{\pgfqpoint{5.800000in}{32.775709in}}%
\pgfpathlineto{\pgfqpoint{5.800000in}{32.775709in}}%
\pgfpathlineto{\pgfqpoint{5.800000in}{32.893561in}}%
\pgfpathlineto{\pgfqpoint{5.800000in}{32.893561in}}%
\pgfpathclose%
\pgfusepath{fill}%
\end{pgfscope}%
\begin{pgfscope}%
\pgfpathrectangle{\pgfqpoint{5.800000in}{0.720000in}}{\pgfqpoint{1.400000in}{4.620000in}}%
\pgfusepath{clip}%
\pgfsetbuttcap%
\pgfsetmiterjoin%
\definecolor{currentfill}{rgb}{0.121569,0.466667,0.705882}%
\pgfsetfillcolor{currentfill}%
\pgfsetlinewidth{0.000000pt}%
\definecolor{currentstroke}{rgb}{0.000000,0.000000,0.000000}%
\pgfsetstrokecolor{currentstroke}%
\pgfsetstrokeopacity{0.000000}%
\pgfsetdash{}{0pt}%
\pgfpathmoveto{\pgfqpoint{5.800000in}{32.893561in}}%
\pgfpathlineto{\pgfqpoint{5.800000in}{32.893561in}}%
\pgfpathlineto{\pgfqpoint{5.800000in}{33.011412in}}%
\pgfpathlineto{\pgfqpoint{5.800000in}{33.011412in}}%
\pgfpathclose%
\pgfusepath{fill}%
\end{pgfscope}%
\begin{pgfscope}%
\pgfpathrectangle{\pgfqpoint{5.800000in}{0.720000in}}{\pgfqpoint{1.400000in}{4.620000in}}%
\pgfusepath{clip}%
\pgfsetbuttcap%
\pgfsetmiterjoin%
\definecolor{currentfill}{rgb}{0.121569,0.466667,0.705882}%
\pgfsetfillcolor{currentfill}%
\pgfsetlinewidth{0.000000pt}%
\definecolor{currentstroke}{rgb}{0.000000,0.000000,0.000000}%
\pgfsetstrokecolor{currentstroke}%
\pgfsetstrokeopacity{0.000000}%
\pgfsetdash{}{0pt}%
\pgfpathmoveto{\pgfqpoint{5.800000in}{33.011412in}}%
\pgfpathlineto{\pgfqpoint{5.800000in}{33.011412in}}%
\pgfpathlineto{\pgfqpoint{5.800000in}{33.129264in}}%
\pgfpathlineto{\pgfqpoint{5.800000in}{33.129264in}}%
\pgfpathclose%
\pgfusepath{fill}%
\end{pgfscope}%
\begin{pgfscope}%
\pgfpathrectangle{\pgfqpoint{5.800000in}{0.720000in}}{\pgfqpoint{1.400000in}{4.620000in}}%
\pgfusepath{clip}%
\pgfsetbuttcap%
\pgfsetmiterjoin%
\definecolor{currentfill}{rgb}{0.121569,0.466667,0.705882}%
\pgfsetfillcolor{currentfill}%
\pgfsetlinewidth{0.000000pt}%
\definecolor{currentstroke}{rgb}{0.000000,0.000000,0.000000}%
\pgfsetstrokecolor{currentstroke}%
\pgfsetstrokeopacity{0.000000}%
\pgfsetdash{}{0pt}%
\pgfpathmoveto{\pgfqpoint{5.800000in}{33.129264in}}%
\pgfpathlineto{\pgfqpoint{5.800000in}{33.129264in}}%
\pgfpathlineto{\pgfqpoint{5.800000in}{33.247116in}}%
\pgfpathlineto{\pgfqpoint{5.800000in}{33.247116in}}%
\pgfpathclose%
\pgfusepath{fill}%
\end{pgfscope}%
\begin{pgfscope}%
\pgfpathrectangle{\pgfqpoint{5.800000in}{0.720000in}}{\pgfqpoint{1.400000in}{4.620000in}}%
\pgfusepath{clip}%
\pgfsetbuttcap%
\pgfsetmiterjoin%
\definecolor{currentfill}{rgb}{0.121569,0.466667,0.705882}%
\pgfsetfillcolor{currentfill}%
\pgfsetlinewidth{0.000000pt}%
\definecolor{currentstroke}{rgb}{0.000000,0.000000,0.000000}%
\pgfsetstrokecolor{currentstroke}%
\pgfsetstrokeopacity{0.000000}%
\pgfsetdash{}{0pt}%
\pgfpathmoveto{\pgfqpoint{5.800000in}{33.247116in}}%
\pgfpathlineto{\pgfqpoint{5.800000in}{33.247116in}}%
\pgfpathlineto{\pgfqpoint{5.800000in}{33.364968in}}%
\pgfpathlineto{\pgfqpoint{5.800000in}{33.364968in}}%
\pgfpathclose%
\pgfusepath{fill}%
\end{pgfscope}%
\begin{pgfscope}%
\pgfpathrectangle{\pgfqpoint{5.800000in}{0.720000in}}{\pgfqpoint{1.400000in}{4.620000in}}%
\pgfusepath{clip}%
\pgfsetbuttcap%
\pgfsetmiterjoin%
\definecolor{currentfill}{rgb}{0.121569,0.466667,0.705882}%
\pgfsetfillcolor{currentfill}%
\pgfsetlinewidth{0.000000pt}%
\definecolor{currentstroke}{rgb}{0.000000,0.000000,0.000000}%
\pgfsetstrokecolor{currentstroke}%
\pgfsetstrokeopacity{0.000000}%
\pgfsetdash{}{0pt}%
\pgfpathmoveto{\pgfqpoint{5.800000in}{33.364968in}}%
\pgfpathlineto{\pgfqpoint{5.800000in}{33.364968in}}%
\pgfpathlineto{\pgfqpoint{5.800000in}{33.482820in}}%
\pgfpathlineto{\pgfqpoint{5.800000in}{33.482820in}}%
\pgfpathclose%
\pgfusepath{fill}%
\end{pgfscope}%
\begin{pgfscope}%
\pgfpathrectangle{\pgfqpoint{5.800000in}{0.720000in}}{\pgfqpoint{1.400000in}{4.620000in}}%
\pgfusepath{clip}%
\pgfsetbuttcap%
\pgfsetmiterjoin%
\definecolor{currentfill}{rgb}{0.121569,0.466667,0.705882}%
\pgfsetfillcolor{currentfill}%
\pgfsetlinewidth{0.000000pt}%
\definecolor{currentstroke}{rgb}{0.000000,0.000000,0.000000}%
\pgfsetstrokecolor{currentstroke}%
\pgfsetstrokeopacity{0.000000}%
\pgfsetdash{}{0pt}%
\pgfpathmoveto{\pgfqpoint{5.800000in}{33.482820in}}%
\pgfpathlineto{\pgfqpoint{5.800000in}{33.482820in}}%
\pgfpathlineto{\pgfqpoint{5.800000in}{33.600672in}}%
\pgfpathlineto{\pgfqpoint{5.800000in}{33.600672in}}%
\pgfpathclose%
\pgfusepath{fill}%
\end{pgfscope}%
\begin{pgfscope}%
\pgfpathrectangle{\pgfqpoint{5.800000in}{0.720000in}}{\pgfqpoint{1.400000in}{4.620000in}}%
\pgfusepath{clip}%
\pgfsetbuttcap%
\pgfsetmiterjoin%
\definecolor{currentfill}{rgb}{0.121569,0.466667,0.705882}%
\pgfsetfillcolor{currentfill}%
\pgfsetlinewidth{0.000000pt}%
\definecolor{currentstroke}{rgb}{0.000000,0.000000,0.000000}%
\pgfsetstrokecolor{currentstroke}%
\pgfsetstrokeopacity{0.000000}%
\pgfsetdash{}{0pt}%
\pgfpathmoveto{\pgfqpoint{5.800000in}{33.600672in}}%
\pgfpathlineto{\pgfqpoint{5.800000in}{33.600672in}}%
\pgfpathlineto{\pgfqpoint{5.800000in}{33.718524in}}%
\pgfpathlineto{\pgfqpoint{5.800000in}{33.718524in}}%
\pgfpathclose%
\pgfusepath{fill}%
\end{pgfscope}%
\begin{pgfscope}%
\pgfpathrectangle{\pgfqpoint{5.800000in}{0.720000in}}{\pgfqpoint{1.400000in}{4.620000in}}%
\pgfusepath{clip}%
\pgfsetbuttcap%
\pgfsetmiterjoin%
\definecolor{currentfill}{rgb}{0.121569,0.466667,0.705882}%
\pgfsetfillcolor{currentfill}%
\pgfsetlinewidth{0.000000pt}%
\definecolor{currentstroke}{rgb}{0.000000,0.000000,0.000000}%
\pgfsetstrokecolor{currentstroke}%
\pgfsetstrokeopacity{0.000000}%
\pgfsetdash{}{0pt}%
\pgfpathmoveto{\pgfqpoint{5.800000in}{33.718524in}}%
\pgfpathlineto{\pgfqpoint{5.800000in}{33.718524in}}%
\pgfpathlineto{\pgfqpoint{5.800000in}{33.836376in}}%
\pgfpathlineto{\pgfqpoint{5.800000in}{33.836376in}}%
\pgfpathclose%
\pgfusepath{fill}%
\end{pgfscope}%
\begin{pgfscope}%
\pgfpathrectangle{\pgfqpoint{5.800000in}{0.720000in}}{\pgfqpoint{1.400000in}{4.620000in}}%
\pgfusepath{clip}%
\pgfsetbuttcap%
\pgfsetmiterjoin%
\definecolor{currentfill}{rgb}{0.121569,0.466667,0.705882}%
\pgfsetfillcolor{currentfill}%
\pgfsetlinewidth{0.000000pt}%
\definecolor{currentstroke}{rgb}{0.000000,0.000000,0.000000}%
\pgfsetstrokecolor{currentstroke}%
\pgfsetstrokeopacity{0.000000}%
\pgfsetdash{}{0pt}%
\pgfpathmoveto{\pgfqpoint{5.800000in}{33.836376in}}%
\pgfpathlineto{\pgfqpoint{5.800000in}{33.836376in}}%
\pgfpathlineto{\pgfqpoint{5.800000in}{33.954227in}}%
\pgfpathlineto{\pgfqpoint{5.800000in}{33.954227in}}%
\pgfpathclose%
\pgfusepath{fill}%
\end{pgfscope}%
\begin{pgfscope}%
\pgfpathrectangle{\pgfqpoint{5.800000in}{0.720000in}}{\pgfqpoint{1.400000in}{4.620000in}}%
\pgfusepath{clip}%
\pgfsetbuttcap%
\pgfsetmiterjoin%
\definecolor{currentfill}{rgb}{0.121569,0.466667,0.705882}%
\pgfsetfillcolor{currentfill}%
\pgfsetlinewidth{0.000000pt}%
\definecolor{currentstroke}{rgb}{0.000000,0.000000,0.000000}%
\pgfsetstrokecolor{currentstroke}%
\pgfsetstrokeopacity{0.000000}%
\pgfsetdash{}{0pt}%
\pgfpathmoveto{\pgfqpoint{5.800000in}{33.954227in}}%
\pgfpathlineto{\pgfqpoint{5.800000in}{33.954227in}}%
\pgfpathlineto{\pgfqpoint{5.800000in}{34.072079in}}%
\pgfpathlineto{\pgfqpoint{5.800000in}{34.072079in}}%
\pgfpathclose%
\pgfusepath{fill}%
\end{pgfscope}%
\begin{pgfscope}%
\pgfpathrectangle{\pgfqpoint{5.800000in}{0.720000in}}{\pgfqpoint{1.400000in}{4.620000in}}%
\pgfusepath{clip}%
\pgfsetbuttcap%
\pgfsetmiterjoin%
\definecolor{currentfill}{rgb}{0.121569,0.466667,0.705882}%
\pgfsetfillcolor{currentfill}%
\pgfsetlinewidth{0.000000pt}%
\definecolor{currentstroke}{rgb}{0.000000,0.000000,0.000000}%
\pgfsetstrokecolor{currentstroke}%
\pgfsetstrokeopacity{0.000000}%
\pgfsetdash{}{0pt}%
\pgfpathmoveto{\pgfqpoint{5.800000in}{34.072079in}}%
\pgfpathlineto{\pgfqpoint{5.800000in}{34.072079in}}%
\pgfpathlineto{\pgfqpoint{5.800000in}{34.189931in}}%
\pgfpathlineto{\pgfqpoint{5.800000in}{34.189931in}}%
\pgfpathclose%
\pgfusepath{fill}%
\end{pgfscope}%
\begin{pgfscope}%
\pgfpathrectangle{\pgfqpoint{5.800000in}{0.720000in}}{\pgfqpoint{1.400000in}{4.620000in}}%
\pgfusepath{clip}%
\pgfsetbuttcap%
\pgfsetmiterjoin%
\definecolor{currentfill}{rgb}{0.121569,0.466667,0.705882}%
\pgfsetfillcolor{currentfill}%
\pgfsetlinewidth{0.000000pt}%
\definecolor{currentstroke}{rgb}{0.000000,0.000000,0.000000}%
\pgfsetstrokecolor{currentstroke}%
\pgfsetstrokeopacity{0.000000}%
\pgfsetdash{}{0pt}%
\pgfpathmoveto{\pgfqpoint{5.800000in}{34.189931in}}%
\pgfpathlineto{\pgfqpoint{5.800000in}{34.189931in}}%
\pgfpathlineto{\pgfqpoint{5.800000in}{34.307783in}}%
\pgfpathlineto{\pgfqpoint{5.800000in}{34.307783in}}%
\pgfpathclose%
\pgfusepath{fill}%
\end{pgfscope}%
\begin{pgfscope}%
\pgfpathrectangle{\pgfqpoint{5.800000in}{0.720000in}}{\pgfqpoint{1.400000in}{4.620000in}}%
\pgfusepath{clip}%
\pgfsetbuttcap%
\pgfsetmiterjoin%
\definecolor{currentfill}{rgb}{0.121569,0.466667,0.705882}%
\pgfsetfillcolor{currentfill}%
\pgfsetlinewidth{0.000000pt}%
\definecolor{currentstroke}{rgb}{0.000000,0.000000,0.000000}%
\pgfsetstrokecolor{currentstroke}%
\pgfsetstrokeopacity{0.000000}%
\pgfsetdash{}{0pt}%
\pgfpathmoveto{\pgfqpoint{5.800000in}{34.307783in}}%
\pgfpathlineto{\pgfqpoint{5.800000in}{34.307783in}}%
\pgfpathlineto{\pgfqpoint{5.800000in}{34.425635in}}%
\pgfpathlineto{\pgfqpoint{5.800000in}{34.425635in}}%
\pgfpathclose%
\pgfusepath{fill}%
\end{pgfscope}%
\begin{pgfscope}%
\pgfpathrectangle{\pgfqpoint{5.800000in}{0.720000in}}{\pgfqpoint{1.400000in}{4.620000in}}%
\pgfusepath{clip}%
\pgfsetbuttcap%
\pgfsetmiterjoin%
\definecolor{currentfill}{rgb}{0.121569,0.466667,0.705882}%
\pgfsetfillcolor{currentfill}%
\pgfsetlinewidth{0.000000pt}%
\definecolor{currentstroke}{rgb}{0.000000,0.000000,0.000000}%
\pgfsetstrokecolor{currentstroke}%
\pgfsetstrokeopacity{0.000000}%
\pgfsetdash{}{0pt}%
\pgfpathmoveto{\pgfqpoint{5.800000in}{34.425635in}}%
\pgfpathlineto{\pgfqpoint{5.800000in}{34.425635in}}%
\pgfpathlineto{\pgfqpoint{5.800000in}{34.543487in}}%
\pgfpathlineto{\pgfqpoint{5.800000in}{34.543487in}}%
\pgfpathclose%
\pgfusepath{fill}%
\end{pgfscope}%
\begin{pgfscope}%
\pgfpathrectangle{\pgfqpoint{5.800000in}{0.720000in}}{\pgfqpoint{1.400000in}{4.620000in}}%
\pgfusepath{clip}%
\pgfsetbuttcap%
\pgfsetmiterjoin%
\definecolor{currentfill}{rgb}{0.121569,0.466667,0.705882}%
\pgfsetfillcolor{currentfill}%
\pgfsetlinewidth{0.000000pt}%
\definecolor{currentstroke}{rgb}{0.000000,0.000000,0.000000}%
\pgfsetstrokecolor{currentstroke}%
\pgfsetstrokeopacity{0.000000}%
\pgfsetdash{}{0pt}%
\pgfpathmoveto{\pgfqpoint{5.800000in}{34.543487in}}%
\pgfpathlineto{\pgfqpoint{5.800000in}{34.543487in}}%
\pgfpathlineto{\pgfqpoint{5.800000in}{34.661339in}}%
\pgfpathlineto{\pgfqpoint{5.800000in}{34.661339in}}%
\pgfpathclose%
\pgfusepath{fill}%
\end{pgfscope}%
\begin{pgfscope}%
\pgfpathrectangle{\pgfqpoint{5.800000in}{0.720000in}}{\pgfqpoint{1.400000in}{4.620000in}}%
\pgfusepath{clip}%
\pgfsetbuttcap%
\pgfsetmiterjoin%
\definecolor{currentfill}{rgb}{0.121569,0.466667,0.705882}%
\pgfsetfillcolor{currentfill}%
\pgfsetlinewidth{0.000000pt}%
\definecolor{currentstroke}{rgb}{0.000000,0.000000,0.000000}%
\pgfsetstrokecolor{currentstroke}%
\pgfsetstrokeopacity{0.000000}%
\pgfsetdash{}{0pt}%
\pgfpathmoveto{\pgfqpoint{5.800000in}{34.661339in}}%
\pgfpathlineto{\pgfqpoint{5.800000in}{34.661339in}}%
\pgfpathlineto{\pgfqpoint{5.800000in}{34.779191in}}%
\pgfpathlineto{\pgfqpoint{5.800000in}{34.779191in}}%
\pgfpathclose%
\pgfusepath{fill}%
\end{pgfscope}%
\begin{pgfscope}%
\pgfpathrectangle{\pgfqpoint{5.800000in}{0.720000in}}{\pgfqpoint{1.400000in}{4.620000in}}%
\pgfusepath{clip}%
\pgfsetbuttcap%
\pgfsetmiterjoin%
\definecolor{currentfill}{rgb}{0.121569,0.466667,0.705882}%
\pgfsetfillcolor{currentfill}%
\pgfsetlinewidth{0.000000pt}%
\definecolor{currentstroke}{rgb}{0.000000,0.000000,0.000000}%
\pgfsetstrokecolor{currentstroke}%
\pgfsetstrokeopacity{0.000000}%
\pgfsetdash{}{0pt}%
\pgfpathmoveto{\pgfqpoint{5.800000in}{34.779191in}}%
\pgfpathlineto{\pgfqpoint{5.800000in}{34.779191in}}%
\pgfpathlineto{\pgfqpoint{5.800000in}{34.897042in}}%
\pgfpathlineto{\pgfqpoint{5.800000in}{34.897042in}}%
\pgfpathclose%
\pgfusepath{fill}%
\end{pgfscope}%
\begin{pgfscope}%
\pgfpathrectangle{\pgfqpoint{5.800000in}{0.720000in}}{\pgfqpoint{1.400000in}{4.620000in}}%
\pgfusepath{clip}%
\pgfsetbuttcap%
\pgfsetmiterjoin%
\definecolor{currentfill}{rgb}{0.121569,0.466667,0.705882}%
\pgfsetfillcolor{currentfill}%
\pgfsetlinewidth{0.000000pt}%
\definecolor{currentstroke}{rgb}{0.000000,0.000000,0.000000}%
\pgfsetstrokecolor{currentstroke}%
\pgfsetstrokeopacity{0.000000}%
\pgfsetdash{}{0pt}%
\pgfpathmoveto{\pgfqpoint{5.800000in}{34.897042in}}%
\pgfpathlineto{\pgfqpoint{5.800000in}{34.897042in}}%
\pgfpathlineto{\pgfqpoint{5.800000in}{35.014894in}}%
\pgfpathlineto{\pgfqpoint{5.800000in}{35.014894in}}%
\pgfpathclose%
\pgfusepath{fill}%
\end{pgfscope}%
\begin{pgfscope}%
\pgfpathrectangle{\pgfqpoint{5.800000in}{0.720000in}}{\pgfqpoint{1.400000in}{4.620000in}}%
\pgfusepath{clip}%
\pgfsetbuttcap%
\pgfsetmiterjoin%
\definecolor{currentfill}{rgb}{0.121569,0.466667,0.705882}%
\pgfsetfillcolor{currentfill}%
\pgfsetlinewidth{0.000000pt}%
\definecolor{currentstroke}{rgb}{0.000000,0.000000,0.000000}%
\pgfsetstrokecolor{currentstroke}%
\pgfsetstrokeopacity{0.000000}%
\pgfsetdash{}{0pt}%
\pgfpathmoveto{\pgfqpoint{5.800000in}{35.014894in}}%
\pgfpathlineto{\pgfqpoint{5.800000in}{35.014894in}}%
\pgfpathlineto{\pgfqpoint{5.800000in}{35.132746in}}%
\pgfpathlineto{\pgfqpoint{5.800000in}{35.132746in}}%
\pgfpathclose%
\pgfusepath{fill}%
\end{pgfscope}%
\begin{pgfscope}%
\pgfpathrectangle{\pgfqpoint{5.800000in}{0.720000in}}{\pgfqpoint{1.400000in}{4.620000in}}%
\pgfusepath{clip}%
\pgfsetbuttcap%
\pgfsetmiterjoin%
\definecolor{currentfill}{rgb}{0.121569,0.466667,0.705882}%
\pgfsetfillcolor{currentfill}%
\pgfsetlinewidth{0.000000pt}%
\definecolor{currentstroke}{rgb}{0.000000,0.000000,0.000000}%
\pgfsetstrokecolor{currentstroke}%
\pgfsetstrokeopacity{0.000000}%
\pgfsetdash{}{0pt}%
\pgfpathmoveto{\pgfqpoint{5.800000in}{35.132746in}}%
\pgfpathlineto{\pgfqpoint{5.802160in}{35.132746in}}%
\pgfpathlineto{\pgfqpoint{5.802160in}{35.250598in}}%
\pgfpathlineto{\pgfqpoint{5.800000in}{35.250598in}}%
\pgfpathclose%
\pgfusepath{fill}%
\end{pgfscope}%
\begin{pgfscope}%
\pgfpathrectangle{\pgfqpoint{5.800000in}{0.720000in}}{\pgfqpoint{1.400000in}{4.620000in}}%
\pgfusepath{clip}%
\pgfsetbuttcap%
\pgfsetmiterjoin%
\definecolor{currentfill}{rgb}{0.121569,0.466667,0.705882}%
\pgfsetfillcolor{currentfill}%
\pgfsetlinewidth{0.000000pt}%
\definecolor{currentstroke}{rgb}{0.000000,0.000000,0.000000}%
\pgfsetstrokecolor{currentstroke}%
\pgfsetstrokeopacity{0.000000}%
\pgfsetdash{}{0pt}%
\pgfpathmoveto{\pgfqpoint{5.800000in}{35.250598in}}%
\pgfpathlineto{\pgfqpoint{5.800000in}{35.250598in}}%
\pgfpathlineto{\pgfqpoint{5.800000in}{35.368450in}}%
\pgfpathlineto{\pgfqpoint{5.800000in}{35.368450in}}%
\pgfpathclose%
\pgfusepath{fill}%
\end{pgfscope}%
\begin{pgfscope}%
\pgfpathrectangle{\pgfqpoint{5.800000in}{0.720000in}}{\pgfqpoint{1.400000in}{4.620000in}}%
\pgfusepath{clip}%
\pgfsetbuttcap%
\pgfsetmiterjoin%
\definecolor{currentfill}{rgb}{0.121569,0.466667,0.705882}%
\pgfsetfillcolor{currentfill}%
\pgfsetlinewidth{0.000000pt}%
\definecolor{currentstroke}{rgb}{0.000000,0.000000,0.000000}%
\pgfsetstrokecolor{currentstroke}%
\pgfsetstrokeopacity{0.000000}%
\pgfsetdash{}{0pt}%
\pgfpathmoveto{\pgfqpoint{5.800000in}{35.368450in}}%
\pgfpathlineto{\pgfqpoint{5.800000in}{35.368450in}}%
\pgfpathlineto{\pgfqpoint{5.800000in}{35.486302in}}%
\pgfpathlineto{\pgfqpoint{5.800000in}{35.486302in}}%
\pgfpathclose%
\pgfusepath{fill}%
\end{pgfscope}%
\begin{pgfscope}%
\pgfpathrectangle{\pgfqpoint{5.800000in}{0.720000in}}{\pgfqpoint{1.400000in}{4.620000in}}%
\pgfusepath{clip}%
\pgfsetbuttcap%
\pgfsetmiterjoin%
\definecolor{currentfill}{rgb}{0.121569,0.466667,0.705882}%
\pgfsetfillcolor{currentfill}%
\pgfsetlinewidth{0.000000pt}%
\definecolor{currentstroke}{rgb}{0.000000,0.000000,0.000000}%
\pgfsetstrokecolor{currentstroke}%
\pgfsetstrokeopacity{0.000000}%
\pgfsetdash{}{0pt}%
\pgfpathmoveto{\pgfqpoint{5.800000in}{35.486302in}}%
\pgfpathlineto{\pgfqpoint{5.800000in}{35.486302in}}%
\pgfpathlineto{\pgfqpoint{5.800000in}{35.604154in}}%
\pgfpathlineto{\pgfqpoint{5.800000in}{35.604154in}}%
\pgfpathclose%
\pgfusepath{fill}%
\end{pgfscope}%
\begin{pgfscope}%
\pgfpathrectangle{\pgfqpoint{5.800000in}{0.720000in}}{\pgfqpoint{1.400000in}{4.620000in}}%
\pgfusepath{clip}%
\pgfsetbuttcap%
\pgfsetmiterjoin%
\definecolor{currentfill}{rgb}{0.121569,0.466667,0.705882}%
\pgfsetfillcolor{currentfill}%
\pgfsetlinewidth{0.000000pt}%
\definecolor{currentstroke}{rgb}{0.000000,0.000000,0.000000}%
\pgfsetstrokecolor{currentstroke}%
\pgfsetstrokeopacity{0.000000}%
\pgfsetdash{}{0pt}%
\pgfpathmoveto{\pgfqpoint{5.800000in}{35.604154in}}%
\pgfpathlineto{\pgfqpoint{5.800000in}{35.604154in}}%
\pgfpathlineto{\pgfqpoint{5.800000in}{35.722005in}}%
\pgfpathlineto{\pgfqpoint{5.800000in}{35.722005in}}%
\pgfpathclose%
\pgfusepath{fill}%
\end{pgfscope}%
\begin{pgfscope}%
\pgfpathrectangle{\pgfqpoint{5.800000in}{0.720000in}}{\pgfqpoint{1.400000in}{4.620000in}}%
\pgfusepath{clip}%
\pgfsetbuttcap%
\pgfsetmiterjoin%
\definecolor{currentfill}{rgb}{0.121569,0.466667,0.705882}%
\pgfsetfillcolor{currentfill}%
\pgfsetlinewidth{0.000000pt}%
\definecolor{currentstroke}{rgb}{0.000000,0.000000,0.000000}%
\pgfsetstrokecolor{currentstroke}%
\pgfsetstrokeopacity{0.000000}%
\pgfsetdash{}{0pt}%
\pgfpathmoveto{\pgfqpoint{5.800000in}{35.722005in}}%
\pgfpathlineto{\pgfqpoint{5.800000in}{35.722005in}}%
\pgfpathlineto{\pgfqpoint{5.800000in}{35.839857in}}%
\pgfpathlineto{\pgfqpoint{5.800000in}{35.839857in}}%
\pgfpathclose%
\pgfusepath{fill}%
\end{pgfscope}%
\begin{pgfscope}%
\pgfpathrectangle{\pgfqpoint{5.800000in}{0.720000in}}{\pgfqpoint{1.400000in}{4.620000in}}%
\pgfusepath{clip}%
\pgfsetbuttcap%
\pgfsetmiterjoin%
\definecolor{currentfill}{rgb}{0.121569,0.466667,0.705882}%
\pgfsetfillcolor{currentfill}%
\pgfsetlinewidth{0.000000pt}%
\definecolor{currentstroke}{rgb}{0.000000,0.000000,0.000000}%
\pgfsetstrokecolor{currentstroke}%
\pgfsetstrokeopacity{0.000000}%
\pgfsetdash{}{0pt}%
\pgfpathmoveto{\pgfqpoint{5.800000in}{35.839857in}}%
\pgfpathlineto{\pgfqpoint{5.800000in}{35.839857in}}%
\pgfpathlineto{\pgfqpoint{5.800000in}{35.957709in}}%
\pgfpathlineto{\pgfqpoint{5.800000in}{35.957709in}}%
\pgfpathclose%
\pgfusepath{fill}%
\end{pgfscope}%
\begin{pgfscope}%
\pgfpathrectangle{\pgfqpoint{5.800000in}{0.720000in}}{\pgfqpoint{1.400000in}{4.620000in}}%
\pgfusepath{clip}%
\pgfsetbuttcap%
\pgfsetmiterjoin%
\definecolor{currentfill}{rgb}{0.121569,0.466667,0.705882}%
\pgfsetfillcolor{currentfill}%
\pgfsetlinewidth{0.000000pt}%
\definecolor{currentstroke}{rgb}{0.000000,0.000000,0.000000}%
\pgfsetstrokecolor{currentstroke}%
\pgfsetstrokeopacity{0.000000}%
\pgfsetdash{}{0pt}%
\pgfpathmoveto{\pgfqpoint{5.800000in}{35.957709in}}%
\pgfpathlineto{\pgfqpoint{5.800000in}{35.957709in}}%
\pgfpathlineto{\pgfqpoint{5.800000in}{36.075561in}}%
\pgfpathlineto{\pgfqpoint{5.800000in}{36.075561in}}%
\pgfpathclose%
\pgfusepath{fill}%
\end{pgfscope}%
\begin{pgfscope}%
\pgfpathrectangle{\pgfqpoint{5.800000in}{0.720000in}}{\pgfqpoint{1.400000in}{4.620000in}}%
\pgfusepath{clip}%
\pgfsetbuttcap%
\pgfsetmiterjoin%
\definecolor{currentfill}{rgb}{0.121569,0.466667,0.705882}%
\pgfsetfillcolor{currentfill}%
\pgfsetlinewidth{0.000000pt}%
\definecolor{currentstroke}{rgb}{0.000000,0.000000,0.000000}%
\pgfsetstrokecolor{currentstroke}%
\pgfsetstrokeopacity{0.000000}%
\pgfsetdash{}{0pt}%
\pgfpathmoveto{\pgfqpoint{5.800000in}{36.075561in}}%
\pgfpathlineto{\pgfqpoint{5.800000in}{36.075561in}}%
\pgfpathlineto{\pgfqpoint{5.800000in}{36.193413in}}%
\pgfpathlineto{\pgfqpoint{5.800000in}{36.193413in}}%
\pgfpathclose%
\pgfusepath{fill}%
\end{pgfscope}%
\begin{pgfscope}%
\pgfpathrectangle{\pgfqpoint{5.800000in}{0.720000in}}{\pgfqpoint{1.400000in}{4.620000in}}%
\pgfusepath{clip}%
\pgfsetbuttcap%
\pgfsetmiterjoin%
\definecolor{currentfill}{rgb}{0.121569,0.466667,0.705882}%
\pgfsetfillcolor{currentfill}%
\pgfsetlinewidth{0.000000pt}%
\definecolor{currentstroke}{rgb}{0.000000,0.000000,0.000000}%
\pgfsetstrokecolor{currentstroke}%
\pgfsetstrokeopacity{0.000000}%
\pgfsetdash{}{0pt}%
\pgfpathmoveto{\pgfqpoint{5.800000in}{36.193413in}}%
\pgfpathlineto{\pgfqpoint{5.800000in}{36.193413in}}%
\pgfpathlineto{\pgfqpoint{5.800000in}{36.311265in}}%
\pgfpathlineto{\pgfqpoint{5.800000in}{36.311265in}}%
\pgfpathclose%
\pgfusepath{fill}%
\end{pgfscope}%
\begin{pgfscope}%
\pgfpathrectangle{\pgfqpoint{5.800000in}{0.720000in}}{\pgfqpoint{1.400000in}{4.620000in}}%
\pgfusepath{clip}%
\pgfsetbuttcap%
\pgfsetmiterjoin%
\definecolor{currentfill}{rgb}{0.121569,0.466667,0.705882}%
\pgfsetfillcolor{currentfill}%
\pgfsetlinewidth{0.000000pt}%
\definecolor{currentstroke}{rgb}{0.000000,0.000000,0.000000}%
\pgfsetstrokecolor{currentstroke}%
\pgfsetstrokeopacity{0.000000}%
\pgfsetdash{}{0pt}%
\pgfpathmoveto{\pgfqpoint{5.800000in}{36.311265in}}%
\pgfpathlineto{\pgfqpoint{5.800000in}{36.311265in}}%
\pgfpathlineto{\pgfqpoint{5.800000in}{36.429117in}}%
\pgfpathlineto{\pgfqpoint{5.800000in}{36.429117in}}%
\pgfpathclose%
\pgfusepath{fill}%
\end{pgfscope}%
\begin{pgfscope}%
\pgfpathrectangle{\pgfqpoint{5.800000in}{0.720000in}}{\pgfqpoint{1.400000in}{4.620000in}}%
\pgfusepath{clip}%
\pgfsetbuttcap%
\pgfsetmiterjoin%
\definecolor{currentfill}{rgb}{0.121569,0.466667,0.705882}%
\pgfsetfillcolor{currentfill}%
\pgfsetlinewidth{0.000000pt}%
\definecolor{currentstroke}{rgb}{0.000000,0.000000,0.000000}%
\pgfsetstrokecolor{currentstroke}%
\pgfsetstrokeopacity{0.000000}%
\pgfsetdash{}{0pt}%
\pgfpathmoveto{\pgfqpoint{5.800000in}{36.429117in}}%
\pgfpathlineto{\pgfqpoint{5.800000in}{36.429117in}}%
\pgfpathlineto{\pgfqpoint{5.800000in}{36.546969in}}%
\pgfpathlineto{\pgfqpoint{5.800000in}{36.546969in}}%
\pgfpathclose%
\pgfusepath{fill}%
\end{pgfscope}%
\begin{pgfscope}%
\pgfpathrectangle{\pgfqpoint{5.800000in}{0.720000in}}{\pgfqpoint{1.400000in}{4.620000in}}%
\pgfusepath{clip}%
\pgfsetbuttcap%
\pgfsetmiterjoin%
\definecolor{currentfill}{rgb}{0.121569,0.466667,0.705882}%
\pgfsetfillcolor{currentfill}%
\pgfsetlinewidth{0.000000pt}%
\definecolor{currentstroke}{rgb}{0.000000,0.000000,0.000000}%
\pgfsetstrokecolor{currentstroke}%
\pgfsetstrokeopacity{0.000000}%
\pgfsetdash{}{0pt}%
\pgfpathmoveto{\pgfqpoint{5.800000in}{36.546969in}}%
\pgfpathlineto{\pgfqpoint{5.800000in}{36.546969in}}%
\pgfpathlineto{\pgfqpoint{5.800000in}{36.664820in}}%
\pgfpathlineto{\pgfqpoint{5.800000in}{36.664820in}}%
\pgfpathclose%
\pgfusepath{fill}%
\end{pgfscope}%
\begin{pgfscope}%
\pgfpathrectangle{\pgfqpoint{5.800000in}{0.720000in}}{\pgfqpoint{1.400000in}{4.620000in}}%
\pgfusepath{clip}%
\pgfsetbuttcap%
\pgfsetmiterjoin%
\definecolor{currentfill}{rgb}{0.121569,0.466667,0.705882}%
\pgfsetfillcolor{currentfill}%
\pgfsetlinewidth{0.000000pt}%
\definecolor{currentstroke}{rgb}{0.000000,0.000000,0.000000}%
\pgfsetstrokecolor{currentstroke}%
\pgfsetstrokeopacity{0.000000}%
\pgfsetdash{}{0pt}%
\pgfpathmoveto{\pgfqpoint{5.800000in}{36.664820in}}%
\pgfpathlineto{\pgfqpoint{5.800000in}{36.664820in}}%
\pgfpathlineto{\pgfqpoint{5.800000in}{36.782672in}}%
\pgfpathlineto{\pgfqpoint{5.800000in}{36.782672in}}%
\pgfpathclose%
\pgfusepath{fill}%
\end{pgfscope}%
\begin{pgfscope}%
\pgfpathrectangle{\pgfqpoint{5.800000in}{0.720000in}}{\pgfqpoint{1.400000in}{4.620000in}}%
\pgfusepath{clip}%
\pgfsetbuttcap%
\pgfsetmiterjoin%
\definecolor{currentfill}{rgb}{0.121569,0.466667,0.705882}%
\pgfsetfillcolor{currentfill}%
\pgfsetlinewidth{0.000000pt}%
\definecolor{currentstroke}{rgb}{0.000000,0.000000,0.000000}%
\pgfsetstrokecolor{currentstroke}%
\pgfsetstrokeopacity{0.000000}%
\pgfsetdash{}{0pt}%
\pgfpathmoveto{\pgfqpoint{5.800000in}{36.782672in}}%
\pgfpathlineto{\pgfqpoint{5.800000in}{36.782672in}}%
\pgfpathlineto{\pgfqpoint{5.800000in}{36.900524in}}%
\pgfpathlineto{\pgfqpoint{5.800000in}{36.900524in}}%
\pgfpathclose%
\pgfusepath{fill}%
\end{pgfscope}%
\begin{pgfscope}%
\pgfpathrectangle{\pgfqpoint{5.800000in}{0.720000in}}{\pgfqpoint{1.400000in}{4.620000in}}%
\pgfusepath{clip}%
\pgfsetbuttcap%
\pgfsetmiterjoin%
\definecolor{currentfill}{rgb}{0.121569,0.466667,0.705882}%
\pgfsetfillcolor{currentfill}%
\pgfsetlinewidth{0.000000pt}%
\definecolor{currentstroke}{rgb}{0.000000,0.000000,0.000000}%
\pgfsetstrokecolor{currentstroke}%
\pgfsetstrokeopacity{0.000000}%
\pgfsetdash{}{0pt}%
\pgfpathmoveto{\pgfqpoint{5.800000in}{36.900524in}}%
\pgfpathlineto{\pgfqpoint{5.800000in}{36.900524in}}%
\pgfpathlineto{\pgfqpoint{5.800000in}{37.018376in}}%
\pgfpathlineto{\pgfqpoint{5.800000in}{37.018376in}}%
\pgfpathclose%
\pgfusepath{fill}%
\end{pgfscope}%
\begin{pgfscope}%
\pgfpathrectangle{\pgfqpoint{5.800000in}{0.720000in}}{\pgfqpoint{1.400000in}{4.620000in}}%
\pgfusepath{clip}%
\pgfsetbuttcap%
\pgfsetmiterjoin%
\definecolor{currentfill}{rgb}{0.121569,0.466667,0.705882}%
\pgfsetfillcolor{currentfill}%
\pgfsetlinewidth{0.000000pt}%
\definecolor{currentstroke}{rgb}{0.000000,0.000000,0.000000}%
\pgfsetstrokecolor{currentstroke}%
\pgfsetstrokeopacity{0.000000}%
\pgfsetdash{}{0pt}%
\pgfpathmoveto{\pgfqpoint{5.800000in}{37.018376in}}%
\pgfpathlineto{\pgfqpoint{5.800000in}{37.018376in}}%
\pgfpathlineto{\pgfqpoint{5.800000in}{37.136228in}}%
\pgfpathlineto{\pgfqpoint{5.800000in}{37.136228in}}%
\pgfpathclose%
\pgfusepath{fill}%
\end{pgfscope}%
\begin{pgfscope}%
\pgfpathrectangle{\pgfqpoint{5.800000in}{0.720000in}}{\pgfqpoint{1.400000in}{4.620000in}}%
\pgfusepath{clip}%
\pgfsetbuttcap%
\pgfsetmiterjoin%
\definecolor{currentfill}{rgb}{0.121569,0.466667,0.705882}%
\pgfsetfillcolor{currentfill}%
\pgfsetlinewidth{0.000000pt}%
\definecolor{currentstroke}{rgb}{0.000000,0.000000,0.000000}%
\pgfsetstrokecolor{currentstroke}%
\pgfsetstrokeopacity{0.000000}%
\pgfsetdash{}{0pt}%
\pgfpathmoveto{\pgfqpoint{5.800000in}{37.136228in}}%
\pgfpathlineto{\pgfqpoint{5.800000in}{37.136228in}}%
\pgfpathlineto{\pgfqpoint{5.800000in}{37.254080in}}%
\pgfpathlineto{\pgfqpoint{5.800000in}{37.254080in}}%
\pgfpathclose%
\pgfusepath{fill}%
\end{pgfscope}%
\begin{pgfscope}%
\pgfpathrectangle{\pgfqpoint{5.800000in}{0.720000in}}{\pgfqpoint{1.400000in}{4.620000in}}%
\pgfusepath{clip}%
\pgfsetbuttcap%
\pgfsetmiterjoin%
\definecolor{currentfill}{rgb}{0.121569,0.466667,0.705882}%
\pgfsetfillcolor{currentfill}%
\pgfsetlinewidth{0.000000pt}%
\definecolor{currentstroke}{rgb}{0.000000,0.000000,0.000000}%
\pgfsetstrokecolor{currentstroke}%
\pgfsetstrokeopacity{0.000000}%
\pgfsetdash{}{0pt}%
\pgfpathmoveto{\pgfqpoint{5.800000in}{37.254080in}}%
\pgfpathlineto{\pgfqpoint{5.800000in}{37.254080in}}%
\pgfpathlineto{\pgfqpoint{5.800000in}{37.371932in}}%
\pgfpathlineto{\pgfqpoint{5.800000in}{37.371932in}}%
\pgfpathclose%
\pgfusepath{fill}%
\end{pgfscope}%
\begin{pgfscope}%
\pgfpathrectangle{\pgfqpoint{5.800000in}{0.720000in}}{\pgfqpoint{1.400000in}{4.620000in}}%
\pgfusepath{clip}%
\pgfsetbuttcap%
\pgfsetmiterjoin%
\definecolor{currentfill}{rgb}{0.121569,0.466667,0.705882}%
\pgfsetfillcolor{currentfill}%
\pgfsetlinewidth{0.000000pt}%
\definecolor{currentstroke}{rgb}{0.000000,0.000000,0.000000}%
\pgfsetstrokecolor{currentstroke}%
\pgfsetstrokeopacity{0.000000}%
\pgfsetdash{}{0pt}%
\pgfpathmoveto{\pgfqpoint{5.800000in}{37.371932in}}%
\pgfpathlineto{\pgfqpoint{5.800000in}{37.371932in}}%
\pgfpathlineto{\pgfqpoint{5.800000in}{37.489784in}}%
\pgfpathlineto{\pgfqpoint{5.800000in}{37.489784in}}%
\pgfpathclose%
\pgfusepath{fill}%
\end{pgfscope}%
\begin{pgfscope}%
\pgfpathrectangle{\pgfqpoint{5.800000in}{0.720000in}}{\pgfqpoint{1.400000in}{4.620000in}}%
\pgfusepath{clip}%
\pgfsetbuttcap%
\pgfsetmiterjoin%
\definecolor{currentfill}{rgb}{0.121569,0.466667,0.705882}%
\pgfsetfillcolor{currentfill}%
\pgfsetlinewidth{0.000000pt}%
\definecolor{currentstroke}{rgb}{0.000000,0.000000,0.000000}%
\pgfsetstrokecolor{currentstroke}%
\pgfsetstrokeopacity{0.000000}%
\pgfsetdash{}{0pt}%
\pgfpathmoveto{\pgfqpoint{5.800000in}{37.489784in}}%
\pgfpathlineto{\pgfqpoint{5.800000in}{37.489784in}}%
\pgfpathlineto{\pgfqpoint{5.800000in}{37.607635in}}%
\pgfpathlineto{\pgfqpoint{5.800000in}{37.607635in}}%
\pgfpathclose%
\pgfusepath{fill}%
\end{pgfscope}%
\begin{pgfscope}%
\pgfpathrectangle{\pgfqpoint{5.800000in}{0.720000in}}{\pgfqpoint{1.400000in}{4.620000in}}%
\pgfusepath{clip}%
\pgfsetbuttcap%
\pgfsetmiterjoin%
\definecolor{currentfill}{rgb}{0.121569,0.466667,0.705882}%
\pgfsetfillcolor{currentfill}%
\pgfsetlinewidth{0.000000pt}%
\definecolor{currentstroke}{rgb}{0.000000,0.000000,0.000000}%
\pgfsetstrokecolor{currentstroke}%
\pgfsetstrokeopacity{0.000000}%
\pgfsetdash{}{0pt}%
\pgfpathmoveto{\pgfqpoint{5.800000in}{37.607635in}}%
\pgfpathlineto{\pgfqpoint{5.800000in}{37.607635in}}%
\pgfpathlineto{\pgfqpoint{5.800000in}{37.725487in}}%
\pgfpathlineto{\pgfqpoint{5.800000in}{37.725487in}}%
\pgfpathclose%
\pgfusepath{fill}%
\end{pgfscope}%
\begin{pgfscope}%
\pgfpathrectangle{\pgfqpoint{5.800000in}{0.720000in}}{\pgfqpoint{1.400000in}{4.620000in}}%
\pgfusepath{clip}%
\pgfsetbuttcap%
\pgfsetmiterjoin%
\definecolor{currentfill}{rgb}{0.121569,0.466667,0.705882}%
\pgfsetfillcolor{currentfill}%
\pgfsetlinewidth{0.000000pt}%
\definecolor{currentstroke}{rgb}{0.000000,0.000000,0.000000}%
\pgfsetstrokecolor{currentstroke}%
\pgfsetstrokeopacity{0.000000}%
\pgfsetdash{}{0pt}%
\pgfpathmoveto{\pgfqpoint{5.800000in}{37.725487in}}%
\pgfpathlineto{\pgfqpoint{5.800000in}{37.725487in}}%
\pgfpathlineto{\pgfqpoint{5.800000in}{37.843339in}}%
\pgfpathlineto{\pgfqpoint{5.800000in}{37.843339in}}%
\pgfpathclose%
\pgfusepath{fill}%
\end{pgfscope}%
\begin{pgfscope}%
\pgfpathrectangle{\pgfqpoint{5.800000in}{0.720000in}}{\pgfqpoint{1.400000in}{4.620000in}}%
\pgfusepath{clip}%
\pgfsetbuttcap%
\pgfsetmiterjoin%
\definecolor{currentfill}{rgb}{0.121569,0.466667,0.705882}%
\pgfsetfillcolor{currentfill}%
\pgfsetlinewidth{0.000000pt}%
\definecolor{currentstroke}{rgb}{0.000000,0.000000,0.000000}%
\pgfsetstrokecolor{currentstroke}%
\pgfsetstrokeopacity{0.000000}%
\pgfsetdash{}{0pt}%
\pgfpathmoveto{\pgfqpoint{5.800000in}{37.843339in}}%
\pgfpathlineto{\pgfqpoint{5.800000in}{37.843339in}}%
\pgfpathlineto{\pgfqpoint{5.800000in}{37.961191in}}%
\pgfpathlineto{\pgfqpoint{5.800000in}{37.961191in}}%
\pgfpathclose%
\pgfusepath{fill}%
\end{pgfscope}%
\begin{pgfscope}%
\pgfpathrectangle{\pgfqpoint{5.800000in}{0.720000in}}{\pgfqpoint{1.400000in}{4.620000in}}%
\pgfusepath{clip}%
\pgfsetbuttcap%
\pgfsetmiterjoin%
\definecolor{currentfill}{rgb}{0.121569,0.466667,0.705882}%
\pgfsetfillcolor{currentfill}%
\pgfsetlinewidth{0.000000pt}%
\definecolor{currentstroke}{rgb}{0.000000,0.000000,0.000000}%
\pgfsetstrokecolor{currentstroke}%
\pgfsetstrokeopacity{0.000000}%
\pgfsetdash{}{0pt}%
\pgfpathmoveto{\pgfqpoint{5.800000in}{37.961191in}}%
\pgfpathlineto{\pgfqpoint{5.800000in}{37.961191in}}%
\pgfpathlineto{\pgfqpoint{5.800000in}{38.079043in}}%
\pgfpathlineto{\pgfqpoint{5.800000in}{38.079043in}}%
\pgfpathclose%
\pgfusepath{fill}%
\end{pgfscope}%
\begin{pgfscope}%
\pgfpathrectangle{\pgfqpoint{5.800000in}{0.720000in}}{\pgfqpoint{1.400000in}{4.620000in}}%
\pgfusepath{clip}%
\pgfsetbuttcap%
\pgfsetmiterjoin%
\definecolor{currentfill}{rgb}{0.121569,0.466667,0.705882}%
\pgfsetfillcolor{currentfill}%
\pgfsetlinewidth{0.000000pt}%
\definecolor{currentstroke}{rgb}{0.000000,0.000000,0.000000}%
\pgfsetstrokecolor{currentstroke}%
\pgfsetstrokeopacity{0.000000}%
\pgfsetdash{}{0pt}%
\pgfpathmoveto{\pgfqpoint{5.800000in}{38.079043in}}%
\pgfpathlineto{\pgfqpoint{5.800000in}{38.079043in}}%
\pgfpathlineto{\pgfqpoint{5.800000in}{38.196895in}}%
\pgfpathlineto{\pgfqpoint{5.800000in}{38.196895in}}%
\pgfpathclose%
\pgfusepath{fill}%
\end{pgfscope}%
\begin{pgfscope}%
\pgfpathrectangle{\pgfqpoint{5.800000in}{0.720000in}}{\pgfqpoint{1.400000in}{4.620000in}}%
\pgfusepath{clip}%
\pgfsetbuttcap%
\pgfsetmiterjoin%
\definecolor{currentfill}{rgb}{0.121569,0.466667,0.705882}%
\pgfsetfillcolor{currentfill}%
\pgfsetlinewidth{0.000000pt}%
\definecolor{currentstroke}{rgb}{0.000000,0.000000,0.000000}%
\pgfsetstrokecolor{currentstroke}%
\pgfsetstrokeopacity{0.000000}%
\pgfsetdash{}{0pt}%
\pgfpathmoveto{\pgfqpoint{5.800000in}{38.196895in}}%
\pgfpathlineto{\pgfqpoint{5.800000in}{38.196895in}}%
\pgfpathlineto{\pgfqpoint{5.800000in}{38.314747in}}%
\pgfpathlineto{\pgfqpoint{5.800000in}{38.314747in}}%
\pgfpathclose%
\pgfusepath{fill}%
\end{pgfscope}%
\begin{pgfscope}%
\pgfpathrectangle{\pgfqpoint{5.800000in}{0.720000in}}{\pgfqpoint{1.400000in}{4.620000in}}%
\pgfusepath{clip}%
\pgfsetbuttcap%
\pgfsetmiterjoin%
\definecolor{currentfill}{rgb}{0.121569,0.466667,0.705882}%
\pgfsetfillcolor{currentfill}%
\pgfsetlinewidth{0.000000pt}%
\definecolor{currentstroke}{rgb}{0.000000,0.000000,0.000000}%
\pgfsetstrokecolor{currentstroke}%
\pgfsetstrokeopacity{0.000000}%
\pgfsetdash{}{0pt}%
\pgfpathmoveto{\pgfqpoint{5.800000in}{38.314747in}}%
\pgfpathlineto{\pgfqpoint{5.800000in}{38.314747in}}%
\pgfpathlineto{\pgfqpoint{5.800000in}{38.432598in}}%
\pgfpathlineto{\pgfqpoint{5.800000in}{38.432598in}}%
\pgfpathclose%
\pgfusepath{fill}%
\end{pgfscope}%
\begin{pgfscope}%
\pgfpathrectangle{\pgfqpoint{5.800000in}{0.720000in}}{\pgfqpoint{1.400000in}{4.620000in}}%
\pgfusepath{clip}%
\pgfsetbuttcap%
\pgfsetmiterjoin%
\definecolor{currentfill}{rgb}{0.121569,0.466667,0.705882}%
\pgfsetfillcolor{currentfill}%
\pgfsetlinewidth{0.000000pt}%
\definecolor{currentstroke}{rgb}{0.000000,0.000000,0.000000}%
\pgfsetstrokecolor{currentstroke}%
\pgfsetstrokeopacity{0.000000}%
\pgfsetdash{}{0pt}%
\pgfpathmoveto{\pgfqpoint{5.800000in}{38.432598in}}%
\pgfpathlineto{\pgfqpoint{5.800000in}{38.432598in}}%
\pgfpathlineto{\pgfqpoint{5.800000in}{38.550450in}}%
\pgfpathlineto{\pgfqpoint{5.800000in}{38.550450in}}%
\pgfpathclose%
\pgfusepath{fill}%
\end{pgfscope}%
\begin{pgfscope}%
\pgfpathrectangle{\pgfqpoint{5.800000in}{0.720000in}}{\pgfqpoint{1.400000in}{4.620000in}}%
\pgfusepath{clip}%
\pgfsetbuttcap%
\pgfsetmiterjoin%
\definecolor{currentfill}{rgb}{0.121569,0.466667,0.705882}%
\pgfsetfillcolor{currentfill}%
\pgfsetlinewidth{0.000000pt}%
\definecolor{currentstroke}{rgb}{0.000000,0.000000,0.000000}%
\pgfsetstrokecolor{currentstroke}%
\pgfsetstrokeopacity{0.000000}%
\pgfsetdash{}{0pt}%
\pgfpathmoveto{\pgfqpoint{5.800000in}{38.550450in}}%
\pgfpathlineto{\pgfqpoint{5.800000in}{38.550450in}}%
\pgfpathlineto{\pgfqpoint{5.800000in}{38.668302in}}%
\pgfpathlineto{\pgfqpoint{5.800000in}{38.668302in}}%
\pgfpathclose%
\pgfusepath{fill}%
\end{pgfscope}%
\begin{pgfscope}%
\pgfpathrectangle{\pgfqpoint{5.800000in}{0.720000in}}{\pgfqpoint{1.400000in}{4.620000in}}%
\pgfusepath{clip}%
\pgfsetbuttcap%
\pgfsetmiterjoin%
\definecolor{currentfill}{rgb}{0.121569,0.466667,0.705882}%
\pgfsetfillcolor{currentfill}%
\pgfsetlinewidth{0.000000pt}%
\definecolor{currentstroke}{rgb}{0.000000,0.000000,0.000000}%
\pgfsetstrokecolor{currentstroke}%
\pgfsetstrokeopacity{0.000000}%
\pgfsetdash{}{0pt}%
\pgfpathmoveto{\pgfqpoint{5.800000in}{38.668302in}}%
\pgfpathlineto{\pgfqpoint{5.800000in}{38.668302in}}%
\pgfpathlineto{\pgfqpoint{5.800000in}{38.786154in}}%
\pgfpathlineto{\pgfqpoint{5.800000in}{38.786154in}}%
\pgfpathclose%
\pgfusepath{fill}%
\end{pgfscope}%
\begin{pgfscope}%
\pgfpathrectangle{\pgfqpoint{5.800000in}{0.720000in}}{\pgfqpoint{1.400000in}{4.620000in}}%
\pgfusepath{clip}%
\pgfsetbuttcap%
\pgfsetmiterjoin%
\definecolor{currentfill}{rgb}{0.121569,0.466667,0.705882}%
\pgfsetfillcolor{currentfill}%
\pgfsetlinewidth{0.000000pt}%
\definecolor{currentstroke}{rgb}{0.000000,0.000000,0.000000}%
\pgfsetstrokecolor{currentstroke}%
\pgfsetstrokeopacity{0.000000}%
\pgfsetdash{}{0pt}%
\pgfpathmoveto{\pgfqpoint{5.800000in}{38.786154in}}%
\pgfpathlineto{\pgfqpoint{5.800000in}{38.786154in}}%
\pgfpathlineto{\pgfqpoint{5.800000in}{38.904006in}}%
\pgfpathlineto{\pgfqpoint{5.800000in}{38.904006in}}%
\pgfpathclose%
\pgfusepath{fill}%
\end{pgfscope}%
\begin{pgfscope}%
\pgfpathrectangle{\pgfqpoint{5.800000in}{0.720000in}}{\pgfqpoint{1.400000in}{4.620000in}}%
\pgfusepath{clip}%
\pgfsetbuttcap%
\pgfsetmiterjoin%
\definecolor{currentfill}{rgb}{0.121569,0.466667,0.705882}%
\pgfsetfillcolor{currentfill}%
\pgfsetlinewidth{0.000000pt}%
\definecolor{currentstroke}{rgb}{0.000000,0.000000,0.000000}%
\pgfsetstrokecolor{currentstroke}%
\pgfsetstrokeopacity{0.000000}%
\pgfsetdash{}{0pt}%
\pgfpathmoveto{\pgfqpoint{5.800000in}{38.904006in}}%
\pgfpathlineto{\pgfqpoint{5.800000in}{38.904006in}}%
\pgfpathlineto{\pgfqpoint{5.800000in}{39.021858in}}%
\pgfpathlineto{\pgfqpoint{5.800000in}{39.021858in}}%
\pgfpathclose%
\pgfusepath{fill}%
\end{pgfscope}%
\begin{pgfscope}%
\pgfpathrectangle{\pgfqpoint{5.800000in}{0.720000in}}{\pgfqpoint{1.400000in}{4.620000in}}%
\pgfusepath{clip}%
\pgfsetbuttcap%
\pgfsetmiterjoin%
\definecolor{currentfill}{rgb}{0.121569,0.466667,0.705882}%
\pgfsetfillcolor{currentfill}%
\pgfsetlinewidth{0.000000pt}%
\definecolor{currentstroke}{rgb}{0.000000,0.000000,0.000000}%
\pgfsetstrokecolor{currentstroke}%
\pgfsetstrokeopacity{0.000000}%
\pgfsetdash{}{0pt}%
\pgfpathmoveto{\pgfqpoint{5.800000in}{39.021858in}}%
\pgfpathlineto{\pgfqpoint{5.800000in}{39.021858in}}%
\pgfpathlineto{\pgfqpoint{5.800000in}{39.139710in}}%
\pgfpathlineto{\pgfqpoint{5.800000in}{39.139710in}}%
\pgfpathclose%
\pgfusepath{fill}%
\end{pgfscope}%
\begin{pgfscope}%
\pgfpathrectangle{\pgfqpoint{5.800000in}{0.720000in}}{\pgfqpoint{1.400000in}{4.620000in}}%
\pgfusepath{clip}%
\pgfsetbuttcap%
\pgfsetmiterjoin%
\definecolor{currentfill}{rgb}{0.121569,0.466667,0.705882}%
\pgfsetfillcolor{currentfill}%
\pgfsetlinewidth{0.000000pt}%
\definecolor{currentstroke}{rgb}{0.000000,0.000000,0.000000}%
\pgfsetstrokecolor{currentstroke}%
\pgfsetstrokeopacity{0.000000}%
\pgfsetdash{}{0pt}%
\pgfpathmoveto{\pgfqpoint{5.800000in}{39.139710in}}%
\pgfpathlineto{\pgfqpoint{5.800000in}{39.139710in}}%
\pgfpathlineto{\pgfqpoint{5.800000in}{39.257562in}}%
\pgfpathlineto{\pgfqpoint{5.800000in}{39.257562in}}%
\pgfpathclose%
\pgfusepath{fill}%
\end{pgfscope}%
\begin{pgfscope}%
\pgfpathrectangle{\pgfqpoint{5.800000in}{0.720000in}}{\pgfqpoint{1.400000in}{4.620000in}}%
\pgfusepath{clip}%
\pgfsetbuttcap%
\pgfsetmiterjoin%
\definecolor{currentfill}{rgb}{0.121569,0.466667,0.705882}%
\pgfsetfillcolor{currentfill}%
\pgfsetlinewidth{0.000000pt}%
\definecolor{currentstroke}{rgb}{0.000000,0.000000,0.000000}%
\pgfsetstrokecolor{currentstroke}%
\pgfsetstrokeopacity{0.000000}%
\pgfsetdash{}{0pt}%
\pgfpathmoveto{\pgfqpoint{5.800000in}{39.257562in}}%
\pgfpathlineto{\pgfqpoint{5.800000in}{39.257562in}}%
\pgfpathlineto{\pgfqpoint{5.800000in}{39.375413in}}%
\pgfpathlineto{\pgfqpoint{5.800000in}{39.375413in}}%
\pgfpathclose%
\pgfusepath{fill}%
\end{pgfscope}%
\begin{pgfscope}%
\pgfpathrectangle{\pgfqpoint{5.800000in}{0.720000in}}{\pgfqpoint{1.400000in}{4.620000in}}%
\pgfusepath{clip}%
\pgfsetbuttcap%
\pgfsetmiterjoin%
\definecolor{currentfill}{rgb}{0.121569,0.466667,0.705882}%
\pgfsetfillcolor{currentfill}%
\pgfsetlinewidth{0.000000pt}%
\definecolor{currentstroke}{rgb}{0.000000,0.000000,0.000000}%
\pgfsetstrokecolor{currentstroke}%
\pgfsetstrokeopacity{0.000000}%
\pgfsetdash{}{0pt}%
\pgfpathmoveto{\pgfqpoint{5.800000in}{39.375413in}}%
\pgfpathlineto{\pgfqpoint{5.800000in}{39.375413in}}%
\pgfpathlineto{\pgfqpoint{5.800000in}{39.493265in}}%
\pgfpathlineto{\pgfqpoint{5.800000in}{39.493265in}}%
\pgfpathclose%
\pgfusepath{fill}%
\end{pgfscope}%
\begin{pgfscope}%
\pgfpathrectangle{\pgfqpoint{5.800000in}{0.720000in}}{\pgfqpoint{1.400000in}{4.620000in}}%
\pgfusepath{clip}%
\pgfsetbuttcap%
\pgfsetmiterjoin%
\definecolor{currentfill}{rgb}{0.121569,0.466667,0.705882}%
\pgfsetfillcolor{currentfill}%
\pgfsetlinewidth{0.000000pt}%
\definecolor{currentstroke}{rgb}{0.000000,0.000000,0.000000}%
\pgfsetstrokecolor{currentstroke}%
\pgfsetstrokeopacity{0.000000}%
\pgfsetdash{}{0pt}%
\pgfpathmoveto{\pgfqpoint{5.800000in}{39.493265in}}%
\pgfpathlineto{\pgfqpoint{5.800000in}{39.493265in}}%
\pgfpathlineto{\pgfqpoint{5.800000in}{39.611117in}}%
\pgfpathlineto{\pgfqpoint{5.800000in}{39.611117in}}%
\pgfpathclose%
\pgfusepath{fill}%
\end{pgfscope}%
\begin{pgfscope}%
\pgfpathrectangle{\pgfqpoint{5.800000in}{0.720000in}}{\pgfqpoint{1.400000in}{4.620000in}}%
\pgfusepath{clip}%
\pgfsetbuttcap%
\pgfsetmiterjoin%
\definecolor{currentfill}{rgb}{0.121569,0.466667,0.705882}%
\pgfsetfillcolor{currentfill}%
\pgfsetlinewidth{0.000000pt}%
\definecolor{currentstroke}{rgb}{0.000000,0.000000,0.000000}%
\pgfsetstrokecolor{currentstroke}%
\pgfsetstrokeopacity{0.000000}%
\pgfsetdash{}{0pt}%
\pgfpathmoveto{\pgfqpoint{5.800000in}{39.611117in}}%
\pgfpathlineto{\pgfqpoint{5.800000in}{39.611117in}}%
\pgfpathlineto{\pgfqpoint{5.800000in}{39.728969in}}%
\pgfpathlineto{\pgfqpoint{5.800000in}{39.728969in}}%
\pgfpathclose%
\pgfusepath{fill}%
\end{pgfscope}%
\begin{pgfscope}%
\pgfpathrectangle{\pgfqpoint{5.800000in}{0.720000in}}{\pgfqpoint{1.400000in}{4.620000in}}%
\pgfusepath{clip}%
\pgfsetbuttcap%
\pgfsetmiterjoin%
\definecolor{currentfill}{rgb}{0.121569,0.466667,0.705882}%
\pgfsetfillcolor{currentfill}%
\pgfsetlinewidth{0.000000pt}%
\definecolor{currentstroke}{rgb}{0.000000,0.000000,0.000000}%
\pgfsetstrokecolor{currentstroke}%
\pgfsetstrokeopacity{0.000000}%
\pgfsetdash{}{0pt}%
\pgfpathmoveto{\pgfqpoint{5.800000in}{39.728969in}}%
\pgfpathlineto{\pgfqpoint{5.800000in}{39.728969in}}%
\pgfpathlineto{\pgfqpoint{5.800000in}{39.846821in}}%
\pgfpathlineto{\pgfqpoint{5.800000in}{39.846821in}}%
\pgfpathclose%
\pgfusepath{fill}%
\end{pgfscope}%
\begin{pgfscope}%
\pgfpathrectangle{\pgfqpoint{5.800000in}{0.720000in}}{\pgfqpoint{1.400000in}{4.620000in}}%
\pgfusepath{clip}%
\pgfsetbuttcap%
\pgfsetmiterjoin%
\definecolor{currentfill}{rgb}{0.121569,0.466667,0.705882}%
\pgfsetfillcolor{currentfill}%
\pgfsetlinewidth{0.000000pt}%
\definecolor{currentstroke}{rgb}{0.000000,0.000000,0.000000}%
\pgfsetstrokecolor{currentstroke}%
\pgfsetstrokeopacity{0.000000}%
\pgfsetdash{}{0pt}%
\pgfpathmoveto{\pgfqpoint{5.800000in}{39.846821in}}%
\pgfpathlineto{\pgfqpoint{5.800000in}{39.846821in}}%
\pgfpathlineto{\pgfqpoint{5.800000in}{39.964673in}}%
\pgfpathlineto{\pgfqpoint{5.800000in}{39.964673in}}%
\pgfpathclose%
\pgfusepath{fill}%
\end{pgfscope}%
\begin{pgfscope}%
\pgfpathrectangle{\pgfqpoint{5.800000in}{0.720000in}}{\pgfqpoint{1.400000in}{4.620000in}}%
\pgfusepath{clip}%
\pgfsetbuttcap%
\pgfsetmiterjoin%
\definecolor{currentfill}{rgb}{0.121569,0.466667,0.705882}%
\pgfsetfillcolor{currentfill}%
\pgfsetlinewidth{0.000000pt}%
\definecolor{currentstroke}{rgb}{0.000000,0.000000,0.000000}%
\pgfsetstrokecolor{currentstroke}%
\pgfsetstrokeopacity{0.000000}%
\pgfsetdash{}{0pt}%
\pgfpathmoveto{\pgfqpoint{5.800000in}{39.964673in}}%
\pgfpathlineto{\pgfqpoint{5.800000in}{39.964673in}}%
\pgfpathlineto{\pgfqpoint{5.800000in}{40.082525in}}%
\pgfpathlineto{\pgfqpoint{5.800000in}{40.082525in}}%
\pgfpathclose%
\pgfusepath{fill}%
\end{pgfscope}%
\begin{pgfscope}%
\pgfpathrectangle{\pgfqpoint{5.800000in}{0.720000in}}{\pgfqpoint{1.400000in}{4.620000in}}%
\pgfusepath{clip}%
\pgfsetbuttcap%
\pgfsetmiterjoin%
\definecolor{currentfill}{rgb}{0.121569,0.466667,0.705882}%
\pgfsetfillcolor{currentfill}%
\pgfsetlinewidth{0.000000pt}%
\definecolor{currentstroke}{rgb}{0.000000,0.000000,0.000000}%
\pgfsetstrokecolor{currentstroke}%
\pgfsetstrokeopacity{0.000000}%
\pgfsetdash{}{0pt}%
\pgfpathmoveto{\pgfqpoint{5.800000in}{40.082525in}}%
\pgfpathlineto{\pgfqpoint{5.800000in}{40.082525in}}%
\pgfpathlineto{\pgfqpoint{5.800000in}{40.200377in}}%
\pgfpathlineto{\pgfqpoint{5.800000in}{40.200377in}}%
\pgfpathclose%
\pgfusepath{fill}%
\end{pgfscope}%
\begin{pgfscope}%
\pgfpathrectangle{\pgfqpoint{5.800000in}{0.720000in}}{\pgfqpoint{1.400000in}{4.620000in}}%
\pgfusepath{clip}%
\pgfsetbuttcap%
\pgfsetmiterjoin%
\definecolor{currentfill}{rgb}{0.121569,0.466667,0.705882}%
\pgfsetfillcolor{currentfill}%
\pgfsetlinewidth{0.000000pt}%
\definecolor{currentstroke}{rgb}{0.000000,0.000000,0.000000}%
\pgfsetstrokecolor{currentstroke}%
\pgfsetstrokeopacity{0.000000}%
\pgfsetdash{}{0pt}%
\pgfpathmoveto{\pgfqpoint{5.800000in}{40.200377in}}%
\pgfpathlineto{\pgfqpoint{5.800000in}{40.200377in}}%
\pgfpathlineto{\pgfqpoint{5.800000in}{40.318228in}}%
\pgfpathlineto{\pgfqpoint{5.800000in}{40.318228in}}%
\pgfpathclose%
\pgfusepath{fill}%
\end{pgfscope}%
\begin{pgfscope}%
\pgfpathrectangle{\pgfqpoint{5.800000in}{0.720000in}}{\pgfqpoint{1.400000in}{4.620000in}}%
\pgfusepath{clip}%
\pgfsetbuttcap%
\pgfsetmiterjoin%
\definecolor{currentfill}{rgb}{0.121569,0.466667,0.705882}%
\pgfsetfillcolor{currentfill}%
\pgfsetlinewidth{0.000000pt}%
\definecolor{currentstroke}{rgb}{0.000000,0.000000,0.000000}%
\pgfsetstrokecolor{currentstroke}%
\pgfsetstrokeopacity{0.000000}%
\pgfsetdash{}{0pt}%
\pgfpathmoveto{\pgfqpoint{5.800000in}{40.318228in}}%
\pgfpathlineto{\pgfqpoint{5.800000in}{40.318228in}}%
\pgfpathlineto{\pgfqpoint{5.800000in}{40.436080in}}%
\pgfpathlineto{\pgfqpoint{5.800000in}{40.436080in}}%
\pgfpathclose%
\pgfusepath{fill}%
\end{pgfscope}%
\begin{pgfscope}%
\pgfpathrectangle{\pgfqpoint{5.800000in}{0.720000in}}{\pgfqpoint{1.400000in}{4.620000in}}%
\pgfusepath{clip}%
\pgfsetbuttcap%
\pgfsetmiterjoin%
\definecolor{currentfill}{rgb}{0.121569,0.466667,0.705882}%
\pgfsetfillcolor{currentfill}%
\pgfsetlinewidth{0.000000pt}%
\definecolor{currentstroke}{rgb}{0.000000,0.000000,0.000000}%
\pgfsetstrokecolor{currentstroke}%
\pgfsetstrokeopacity{0.000000}%
\pgfsetdash{}{0pt}%
\pgfpathmoveto{\pgfqpoint{5.800000in}{40.436080in}}%
\pgfpathlineto{\pgfqpoint{5.800000in}{40.436080in}}%
\pgfpathlineto{\pgfqpoint{5.800000in}{40.553932in}}%
\pgfpathlineto{\pgfqpoint{5.800000in}{40.553932in}}%
\pgfpathclose%
\pgfusepath{fill}%
\end{pgfscope}%
\begin{pgfscope}%
\pgfpathrectangle{\pgfqpoint{5.800000in}{0.720000in}}{\pgfqpoint{1.400000in}{4.620000in}}%
\pgfusepath{clip}%
\pgfsetbuttcap%
\pgfsetmiterjoin%
\definecolor{currentfill}{rgb}{0.121569,0.466667,0.705882}%
\pgfsetfillcolor{currentfill}%
\pgfsetlinewidth{0.000000pt}%
\definecolor{currentstroke}{rgb}{0.000000,0.000000,0.000000}%
\pgfsetstrokecolor{currentstroke}%
\pgfsetstrokeopacity{0.000000}%
\pgfsetdash{}{0pt}%
\pgfpathmoveto{\pgfqpoint{5.800000in}{40.553932in}}%
\pgfpathlineto{\pgfqpoint{5.800000in}{40.553932in}}%
\pgfpathlineto{\pgfqpoint{5.800000in}{40.671784in}}%
\pgfpathlineto{\pgfqpoint{5.800000in}{40.671784in}}%
\pgfpathclose%
\pgfusepath{fill}%
\end{pgfscope}%
\begin{pgfscope}%
\pgfpathrectangle{\pgfqpoint{5.800000in}{0.720000in}}{\pgfqpoint{1.400000in}{4.620000in}}%
\pgfusepath{clip}%
\pgfsetbuttcap%
\pgfsetmiterjoin%
\definecolor{currentfill}{rgb}{0.121569,0.466667,0.705882}%
\pgfsetfillcolor{currentfill}%
\pgfsetlinewidth{0.000000pt}%
\definecolor{currentstroke}{rgb}{0.000000,0.000000,0.000000}%
\pgfsetstrokecolor{currentstroke}%
\pgfsetstrokeopacity{0.000000}%
\pgfsetdash{}{0pt}%
\pgfpathmoveto{\pgfqpoint{5.800000in}{40.671784in}}%
\pgfpathlineto{\pgfqpoint{5.800000in}{40.671784in}}%
\pgfpathlineto{\pgfqpoint{5.800000in}{40.789636in}}%
\pgfpathlineto{\pgfqpoint{5.800000in}{40.789636in}}%
\pgfpathclose%
\pgfusepath{fill}%
\end{pgfscope}%
\begin{pgfscope}%
\pgfpathrectangle{\pgfqpoint{5.800000in}{0.720000in}}{\pgfqpoint{1.400000in}{4.620000in}}%
\pgfusepath{clip}%
\pgfsetbuttcap%
\pgfsetmiterjoin%
\definecolor{currentfill}{rgb}{0.121569,0.466667,0.705882}%
\pgfsetfillcolor{currentfill}%
\pgfsetlinewidth{0.000000pt}%
\definecolor{currentstroke}{rgb}{0.000000,0.000000,0.000000}%
\pgfsetstrokecolor{currentstroke}%
\pgfsetstrokeopacity{0.000000}%
\pgfsetdash{}{0pt}%
\pgfpathmoveto{\pgfqpoint{5.800000in}{40.789636in}}%
\pgfpathlineto{\pgfqpoint{5.800000in}{40.789636in}}%
\pgfpathlineto{\pgfqpoint{5.800000in}{40.907488in}}%
\pgfpathlineto{\pgfqpoint{5.800000in}{40.907488in}}%
\pgfpathclose%
\pgfusepath{fill}%
\end{pgfscope}%
\begin{pgfscope}%
\pgfpathrectangle{\pgfqpoint{5.800000in}{0.720000in}}{\pgfqpoint{1.400000in}{4.620000in}}%
\pgfusepath{clip}%
\pgfsetbuttcap%
\pgfsetmiterjoin%
\definecolor{currentfill}{rgb}{0.121569,0.466667,0.705882}%
\pgfsetfillcolor{currentfill}%
\pgfsetlinewidth{0.000000pt}%
\definecolor{currentstroke}{rgb}{0.000000,0.000000,0.000000}%
\pgfsetstrokecolor{currentstroke}%
\pgfsetstrokeopacity{0.000000}%
\pgfsetdash{}{0pt}%
\pgfpathmoveto{\pgfqpoint{5.800000in}{40.907488in}}%
\pgfpathlineto{\pgfqpoint{5.800000in}{40.907488in}}%
\pgfpathlineto{\pgfqpoint{5.800000in}{41.025340in}}%
\pgfpathlineto{\pgfqpoint{5.800000in}{41.025340in}}%
\pgfpathclose%
\pgfusepath{fill}%
\end{pgfscope}%
\begin{pgfscope}%
\pgfpathrectangle{\pgfqpoint{5.800000in}{0.720000in}}{\pgfqpoint{1.400000in}{4.620000in}}%
\pgfusepath{clip}%
\pgfsetbuttcap%
\pgfsetmiterjoin%
\definecolor{currentfill}{rgb}{0.121569,0.466667,0.705882}%
\pgfsetfillcolor{currentfill}%
\pgfsetlinewidth{0.000000pt}%
\definecolor{currentstroke}{rgb}{0.000000,0.000000,0.000000}%
\pgfsetstrokecolor{currentstroke}%
\pgfsetstrokeopacity{0.000000}%
\pgfsetdash{}{0pt}%
\pgfpathmoveto{\pgfqpoint{5.800000in}{41.025340in}}%
\pgfpathlineto{\pgfqpoint{5.800000in}{41.025340in}}%
\pgfpathlineto{\pgfqpoint{5.800000in}{41.143192in}}%
\pgfpathlineto{\pgfqpoint{5.800000in}{41.143192in}}%
\pgfpathclose%
\pgfusepath{fill}%
\end{pgfscope}%
\begin{pgfscope}%
\pgfpathrectangle{\pgfqpoint{5.800000in}{0.720000in}}{\pgfqpoint{1.400000in}{4.620000in}}%
\pgfusepath{clip}%
\pgfsetbuttcap%
\pgfsetmiterjoin%
\definecolor{currentfill}{rgb}{0.121569,0.466667,0.705882}%
\pgfsetfillcolor{currentfill}%
\pgfsetlinewidth{0.000000pt}%
\definecolor{currentstroke}{rgb}{0.000000,0.000000,0.000000}%
\pgfsetstrokecolor{currentstroke}%
\pgfsetstrokeopacity{0.000000}%
\pgfsetdash{}{0pt}%
\pgfpathmoveto{\pgfqpoint{5.800000in}{41.143192in}}%
\pgfpathlineto{\pgfqpoint{5.800000in}{41.143192in}}%
\pgfpathlineto{\pgfqpoint{5.800000in}{41.261043in}}%
\pgfpathlineto{\pgfqpoint{5.800000in}{41.261043in}}%
\pgfpathclose%
\pgfusepath{fill}%
\end{pgfscope}%
\begin{pgfscope}%
\pgfpathrectangle{\pgfqpoint{5.800000in}{0.720000in}}{\pgfqpoint{1.400000in}{4.620000in}}%
\pgfusepath{clip}%
\pgfsetbuttcap%
\pgfsetmiterjoin%
\definecolor{currentfill}{rgb}{0.121569,0.466667,0.705882}%
\pgfsetfillcolor{currentfill}%
\pgfsetlinewidth{0.000000pt}%
\definecolor{currentstroke}{rgb}{0.000000,0.000000,0.000000}%
\pgfsetstrokecolor{currentstroke}%
\pgfsetstrokeopacity{0.000000}%
\pgfsetdash{}{0pt}%
\pgfpathmoveto{\pgfqpoint{5.800000in}{41.261043in}}%
\pgfpathlineto{\pgfqpoint{5.800000in}{41.261043in}}%
\pgfpathlineto{\pgfqpoint{5.800000in}{41.378895in}}%
\pgfpathlineto{\pgfqpoint{5.800000in}{41.378895in}}%
\pgfpathclose%
\pgfusepath{fill}%
\end{pgfscope}%
\begin{pgfscope}%
\pgfpathrectangle{\pgfqpoint{5.800000in}{0.720000in}}{\pgfqpoint{1.400000in}{4.620000in}}%
\pgfusepath{clip}%
\pgfsetbuttcap%
\pgfsetmiterjoin%
\definecolor{currentfill}{rgb}{0.121569,0.466667,0.705882}%
\pgfsetfillcolor{currentfill}%
\pgfsetlinewidth{0.000000pt}%
\definecolor{currentstroke}{rgb}{0.000000,0.000000,0.000000}%
\pgfsetstrokecolor{currentstroke}%
\pgfsetstrokeopacity{0.000000}%
\pgfsetdash{}{0pt}%
\pgfpathmoveto{\pgfqpoint{5.800000in}{41.378895in}}%
\pgfpathlineto{\pgfqpoint{5.800000in}{41.378895in}}%
\pgfpathlineto{\pgfqpoint{5.800000in}{41.496747in}}%
\pgfpathlineto{\pgfqpoint{5.800000in}{41.496747in}}%
\pgfpathclose%
\pgfusepath{fill}%
\end{pgfscope}%
\begin{pgfscope}%
\pgfpathrectangle{\pgfqpoint{5.800000in}{0.720000in}}{\pgfqpoint{1.400000in}{4.620000in}}%
\pgfusepath{clip}%
\pgfsetbuttcap%
\pgfsetmiterjoin%
\definecolor{currentfill}{rgb}{0.121569,0.466667,0.705882}%
\pgfsetfillcolor{currentfill}%
\pgfsetlinewidth{0.000000pt}%
\definecolor{currentstroke}{rgb}{0.000000,0.000000,0.000000}%
\pgfsetstrokecolor{currentstroke}%
\pgfsetstrokeopacity{0.000000}%
\pgfsetdash{}{0pt}%
\pgfpathmoveto{\pgfqpoint{5.800000in}{41.496747in}}%
\pgfpathlineto{\pgfqpoint{5.800000in}{41.496747in}}%
\pgfpathlineto{\pgfqpoint{5.800000in}{41.614599in}}%
\pgfpathlineto{\pgfqpoint{5.800000in}{41.614599in}}%
\pgfpathclose%
\pgfusepath{fill}%
\end{pgfscope}%
\begin{pgfscope}%
\pgfpathrectangle{\pgfqpoint{5.800000in}{0.720000in}}{\pgfqpoint{1.400000in}{4.620000in}}%
\pgfusepath{clip}%
\pgfsetbuttcap%
\pgfsetmiterjoin%
\definecolor{currentfill}{rgb}{0.121569,0.466667,0.705882}%
\pgfsetfillcolor{currentfill}%
\pgfsetlinewidth{0.000000pt}%
\definecolor{currentstroke}{rgb}{0.000000,0.000000,0.000000}%
\pgfsetstrokecolor{currentstroke}%
\pgfsetstrokeopacity{0.000000}%
\pgfsetdash{}{0pt}%
\pgfpathmoveto{\pgfqpoint{5.800000in}{41.614599in}}%
\pgfpathlineto{\pgfqpoint{5.800000in}{41.614599in}}%
\pgfpathlineto{\pgfqpoint{5.800000in}{41.732451in}}%
\pgfpathlineto{\pgfqpoint{5.800000in}{41.732451in}}%
\pgfpathclose%
\pgfusepath{fill}%
\end{pgfscope}%
\begin{pgfscope}%
\pgfpathrectangle{\pgfqpoint{5.800000in}{0.720000in}}{\pgfqpoint{1.400000in}{4.620000in}}%
\pgfusepath{clip}%
\pgfsetbuttcap%
\pgfsetmiterjoin%
\definecolor{currentfill}{rgb}{0.121569,0.466667,0.705882}%
\pgfsetfillcolor{currentfill}%
\pgfsetlinewidth{0.000000pt}%
\definecolor{currentstroke}{rgb}{0.000000,0.000000,0.000000}%
\pgfsetstrokecolor{currentstroke}%
\pgfsetstrokeopacity{0.000000}%
\pgfsetdash{}{0pt}%
\pgfpathmoveto{\pgfqpoint{5.800000in}{41.732451in}}%
\pgfpathlineto{\pgfqpoint{5.802160in}{41.732451in}}%
\pgfpathlineto{\pgfqpoint{5.802160in}{41.850303in}}%
\pgfpathlineto{\pgfqpoint{5.800000in}{41.850303in}}%
\pgfpathclose%
\pgfusepath{fill}%
\end{pgfscope}%
\begin{pgfscope}%
\pgfpathrectangle{\pgfqpoint{5.800000in}{0.720000in}}{\pgfqpoint{1.400000in}{4.620000in}}%
\pgfusepath{clip}%
\pgfsetbuttcap%
\pgfsetmiterjoin%
\definecolor{currentfill}{rgb}{0.121569,0.466667,0.705882}%
\pgfsetfillcolor{currentfill}%
\pgfsetlinewidth{0.000000pt}%
\definecolor{currentstroke}{rgb}{0.000000,0.000000,0.000000}%
\pgfsetstrokecolor{currentstroke}%
\pgfsetstrokeopacity{0.000000}%
\pgfsetdash{}{0pt}%
\pgfpathmoveto{\pgfqpoint{5.800000in}{41.850303in}}%
\pgfpathlineto{\pgfqpoint{5.800000in}{41.850303in}}%
\pgfpathlineto{\pgfqpoint{5.800000in}{41.968155in}}%
\pgfpathlineto{\pgfqpoint{5.800000in}{41.968155in}}%
\pgfpathclose%
\pgfusepath{fill}%
\end{pgfscope}%
\begin{pgfscope}%
\pgfpathrectangle{\pgfqpoint{5.800000in}{0.720000in}}{\pgfqpoint{1.400000in}{4.620000in}}%
\pgfusepath{clip}%
\pgfsetbuttcap%
\pgfsetmiterjoin%
\definecolor{currentfill}{rgb}{0.121569,0.466667,0.705882}%
\pgfsetfillcolor{currentfill}%
\pgfsetlinewidth{0.000000pt}%
\definecolor{currentstroke}{rgb}{0.000000,0.000000,0.000000}%
\pgfsetstrokecolor{currentstroke}%
\pgfsetstrokeopacity{0.000000}%
\pgfsetdash{}{0pt}%
\pgfpathmoveto{\pgfqpoint{5.800000in}{41.968155in}}%
\pgfpathlineto{\pgfqpoint{5.800000in}{41.968155in}}%
\pgfpathlineto{\pgfqpoint{5.800000in}{42.086006in}}%
\pgfpathlineto{\pgfqpoint{5.800000in}{42.086006in}}%
\pgfpathclose%
\pgfusepath{fill}%
\end{pgfscope}%
\begin{pgfscope}%
\pgfpathrectangle{\pgfqpoint{5.800000in}{0.720000in}}{\pgfqpoint{1.400000in}{4.620000in}}%
\pgfusepath{clip}%
\pgfsetbuttcap%
\pgfsetmiterjoin%
\definecolor{currentfill}{rgb}{0.121569,0.466667,0.705882}%
\pgfsetfillcolor{currentfill}%
\pgfsetlinewidth{0.000000pt}%
\definecolor{currentstroke}{rgb}{0.000000,0.000000,0.000000}%
\pgfsetstrokecolor{currentstroke}%
\pgfsetstrokeopacity{0.000000}%
\pgfsetdash{}{0pt}%
\pgfpathmoveto{\pgfqpoint{5.800000in}{42.086006in}}%
\pgfpathlineto{\pgfqpoint{5.800000in}{42.086006in}}%
\pgfpathlineto{\pgfqpoint{5.800000in}{42.203858in}}%
\pgfpathlineto{\pgfqpoint{5.800000in}{42.203858in}}%
\pgfpathclose%
\pgfusepath{fill}%
\end{pgfscope}%
\begin{pgfscope}%
\pgfpathrectangle{\pgfqpoint{5.800000in}{0.720000in}}{\pgfqpoint{1.400000in}{4.620000in}}%
\pgfusepath{clip}%
\pgfsetbuttcap%
\pgfsetmiterjoin%
\definecolor{currentfill}{rgb}{0.121569,0.466667,0.705882}%
\pgfsetfillcolor{currentfill}%
\pgfsetlinewidth{0.000000pt}%
\definecolor{currentstroke}{rgb}{0.000000,0.000000,0.000000}%
\pgfsetstrokecolor{currentstroke}%
\pgfsetstrokeopacity{0.000000}%
\pgfsetdash{}{0pt}%
\pgfpathmoveto{\pgfqpoint{5.800000in}{42.203858in}}%
\pgfpathlineto{\pgfqpoint{5.800000in}{42.203858in}}%
\pgfpathlineto{\pgfqpoint{5.800000in}{42.321710in}}%
\pgfpathlineto{\pgfqpoint{5.800000in}{42.321710in}}%
\pgfpathclose%
\pgfusepath{fill}%
\end{pgfscope}%
\begin{pgfscope}%
\pgfpathrectangle{\pgfqpoint{5.800000in}{0.720000in}}{\pgfqpoint{1.400000in}{4.620000in}}%
\pgfusepath{clip}%
\pgfsetbuttcap%
\pgfsetmiterjoin%
\definecolor{currentfill}{rgb}{0.121569,0.466667,0.705882}%
\pgfsetfillcolor{currentfill}%
\pgfsetlinewidth{0.000000pt}%
\definecolor{currentstroke}{rgb}{0.000000,0.000000,0.000000}%
\pgfsetstrokecolor{currentstroke}%
\pgfsetstrokeopacity{0.000000}%
\pgfsetdash{}{0pt}%
\pgfpathmoveto{\pgfqpoint{5.800000in}{42.321710in}}%
\pgfpathlineto{\pgfqpoint{5.800000in}{42.321710in}}%
\pgfpathlineto{\pgfqpoint{5.800000in}{42.439562in}}%
\pgfpathlineto{\pgfqpoint{5.800000in}{42.439562in}}%
\pgfpathclose%
\pgfusepath{fill}%
\end{pgfscope}%
\begin{pgfscope}%
\pgfpathrectangle{\pgfqpoint{5.800000in}{0.720000in}}{\pgfqpoint{1.400000in}{4.620000in}}%
\pgfusepath{clip}%
\pgfsetbuttcap%
\pgfsetmiterjoin%
\definecolor{currentfill}{rgb}{0.121569,0.466667,0.705882}%
\pgfsetfillcolor{currentfill}%
\pgfsetlinewidth{0.000000pt}%
\definecolor{currentstroke}{rgb}{0.000000,0.000000,0.000000}%
\pgfsetstrokecolor{currentstroke}%
\pgfsetstrokeopacity{0.000000}%
\pgfsetdash{}{0pt}%
\pgfpathmoveto{\pgfqpoint{5.800000in}{42.439562in}}%
\pgfpathlineto{\pgfqpoint{5.800000in}{42.439562in}}%
\pgfpathlineto{\pgfqpoint{5.800000in}{42.557414in}}%
\pgfpathlineto{\pgfqpoint{5.800000in}{42.557414in}}%
\pgfpathclose%
\pgfusepath{fill}%
\end{pgfscope}%
\begin{pgfscope}%
\pgfpathrectangle{\pgfqpoint{5.800000in}{0.720000in}}{\pgfqpoint{1.400000in}{4.620000in}}%
\pgfusepath{clip}%
\pgfsetbuttcap%
\pgfsetmiterjoin%
\definecolor{currentfill}{rgb}{0.121569,0.466667,0.705882}%
\pgfsetfillcolor{currentfill}%
\pgfsetlinewidth{0.000000pt}%
\definecolor{currentstroke}{rgb}{0.000000,0.000000,0.000000}%
\pgfsetstrokecolor{currentstroke}%
\pgfsetstrokeopacity{0.000000}%
\pgfsetdash{}{0pt}%
\pgfpathmoveto{\pgfqpoint{5.800000in}{42.557414in}}%
\pgfpathlineto{\pgfqpoint{5.800000in}{42.557414in}}%
\pgfpathlineto{\pgfqpoint{5.800000in}{42.675266in}}%
\pgfpathlineto{\pgfqpoint{5.800000in}{42.675266in}}%
\pgfpathclose%
\pgfusepath{fill}%
\end{pgfscope}%
\begin{pgfscope}%
\pgfpathrectangle{\pgfqpoint{5.800000in}{0.720000in}}{\pgfqpoint{1.400000in}{4.620000in}}%
\pgfusepath{clip}%
\pgfsetbuttcap%
\pgfsetmiterjoin%
\definecolor{currentfill}{rgb}{0.121569,0.466667,0.705882}%
\pgfsetfillcolor{currentfill}%
\pgfsetlinewidth{0.000000pt}%
\definecolor{currentstroke}{rgb}{0.000000,0.000000,0.000000}%
\pgfsetstrokecolor{currentstroke}%
\pgfsetstrokeopacity{0.000000}%
\pgfsetdash{}{0pt}%
\pgfpathmoveto{\pgfqpoint{5.800000in}{42.675266in}}%
\pgfpathlineto{\pgfqpoint{5.800000in}{42.675266in}}%
\pgfpathlineto{\pgfqpoint{5.800000in}{42.793118in}}%
\pgfpathlineto{\pgfqpoint{5.800000in}{42.793118in}}%
\pgfpathclose%
\pgfusepath{fill}%
\end{pgfscope}%
\begin{pgfscope}%
\pgfpathrectangle{\pgfqpoint{5.800000in}{0.720000in}}{\pgfqpoint{1.400000in}{4.620000in}}%
\pgfusepath{clip}%
\pgfsetbuttcap%
\pgfsetmiterjoin%
\definecolor{currentfill}{rgb}{0.121569,0.466667,0.705882}%
\pgfsetfillcolor{currentfill}%
\pgfsetlinewidth{0.000000pt}%
\definecolor{currentstroke}{rgb}{0.000000,0.000000,0.000000}%
\pgfsetstrokecolor{currentstroke}%
\pgfsetstrokeopacity{0.000000}%
\pgfsetdash{}{0pt}%
\pgfpathmoveto{\pgfqpoint{5.800000in}{42.793118in}}%
\pgfpathlineto{\pgfqpoint{5.800000in}{42.793118in}}%
\pgfpathlineto{\pgfqpoint{5.800000in}{42.910970in}}%
\pgfpathlineto{\pgfqpoint{5.800000in}{42.910970in}}%
\pgfpathclose%
\pgfusepath{fill}%
\end{pgfscope}%
\begin{pgfscope}%
\pgfpathrectangle{\pgfqpoint{5.800000in}{0.720000in}}{\pgfqpoint{1.400000in}{4.620000in}}%
\pgfusepath{clip}%
\pgfsetbuttcap%
\pgfsetmiterjoin%
\definecolor{currentfill}{rgb}{0.121569,0.466667,0.705882}%
\pgfsetfillcolor{currentfill}%
\pgfsetlinewidth{0.000000pt}%
\definecolor{currentstroke}{rgb}{0.000000,0.000000,0.000000}%
\pgfsetstrokecolor{currentstroke}%
\pgfsetstrokeopacity{0.000000}%
\pgfsetdash{}{0pt}%
\pgfpathmoveto{\pgfqpoint{5.800000in}{42.910970in}}%
\pgfpathlineto{\pgfqpoint{5.800000in}{42.910970in}}%
\pgfpathlineto{\pgfqpoint{5.800000in}{43.028821in}}%
\pgfpathlineto{\pgfqpoint{5.800000in}{43.028821in}}%
\pgfpathclose%
\pgfusepath{fill}%
\end{pgfscope}%
\begin{pgfscope}%
\pgfpathrectangle{\pgfqpoint{5.800000in}{0.720000in}}{\pgfqpoint{1.400000in}{4.620000in}}%
\pgfusepath{clip}%
\pgfsetbuttcap%
\pgfsetmiterjoin%
\definecolor{currentfill}{rgb}{0.121569,0.466667,0.705882}%
\pgfsetfillcolor{currentfill}%
\pgfsetlinewidth{0.000000pt}%
\definecolor{currentstroke}{rgb}{0.000000,0.000000,0.000000}%
\pgfsetstrokecolor{currentstroke}%
\pgfsetstrokeopacity{0.000000}%
\pgfsetdash{}{0pt}%
\pgfpathmoveto{\pgfqpoint{5.800000in}{43.028821in}}%
\pgfpathlineto{\pgfqpoint{5.800000in}{43.028821in}}%
\pgfpathlineto{\pgfqpoint{5.800000in}{43.146673in}}%
\pgfpathlineto{\pgfqpoint{5.800000in}{43.146673in}}%
\pgfpathclose%
\pgfusepath{fill}%
\end{pgfscope}%
\begin{pgfscope}%
\pgfpathrectangle{\pgfqpoint{5.800000in}{0.720000in}}{\pgfqpoint{1.400000in}{4.620000in}}%
\pgfusepath{clip}%
\pgfsetbuttcap%
\pgfsetmiterjoin%
\definecolor{currentfill}{rgb}{0.121569,0.466667,0.705882}%
\pgfsetfillcolor{currentfill}%
\pgfsetlinewidth{0.000000pt}%
\definecolor{currentstroke}{rgb}{0.000000,0.000000,0.000000}%
\pgfsetstrokecolor{currentstroke}%
\pgfsetstrokeopacity{0.000000}%
\pgfsetdash{}{0pt}%
\pgfpathmoveto{\pgfqpoint{5.800000in}{43.146673in}}%
\pgfpathlineto{\pgfqpoint{5.800000in}{43.146673in}}%
\pgfpathlineto{\pgfqpoint{5.800000in}{43.264525in}}%
\pgfpathlineto{\pgfqpoint{5.800000in}{43.264525in}}%
\pgfpathclose%
\pgfusepath{fill}%
\end{pgfscope}%
\begin{pgfscope}%
\pgfpathrectangle{\pgfqpoint{5.800000in}{0.720000in}}{\pgfqpoint{1.400000in}{4.620000in}}%
\pgfusepath{clip}%
\pgfsetbuttcap%
\pgfsetmiterjoin%
\definecolor{currentfill}{rgb}{0.121569,0.466667,0.705882}%
\pgfsetfillcolor{currentfill}%
\pgfsetlinewidth{0.000000pt}%
\definecolor{currentstroke}{rgb}{0.000000,0.000000,0.000000}%
\pgfsetstrokecolor{currentstroke}%
\pgfsetstrokeopacity{0.000000}%
\pgfsetdash{}{0pt}%
\pgfpathmoveto{\pgfqpoint{5.800000in}{43.264525in}}%
\pgfpathlineto{\pgfqpoint{5.800000in}{43.264525in}}%
\pgfpathlineto{\pgfqpoint{5.800000in}{43.382377in}}%
\pgfpathlineto{\pgfqpoint{5.800000in}{43.382377in}}%
\pgfpathclose%
\pgfusepath{fill}%
\end{pgfscope}%
\begin{pgfscope}%
\pgfpathrectangle{\pgfqpoint{5.800000in}{0.720000in}}{\pgfqpoint{1.400000in}{4.620000in}}%
\pgfusepath{clip}%
\pgfsetbuttcap%
\pgfsetmiterjoin%
\definecolor{currentfill}{rgb}{0.121569,0.466667,0.705882}%
\pgfsetfillcolor{currentfill}%
\pgfsetlinewidth{0.000000pt}%
\definecolor{currentstroke}{rgb}{0.000000,0.000000,0.000000}%
\pgfsetstrokecolor{currentstroke}%
\pgfsetstrokeopacity{0.000000}%
\pgfsetdash{}{0pt}%
\pgfpathmoveto{\pgfqpoint{5.800000in}{43.382377in}}%
\pgfpathlineto{\pgfqpoint{5.800000in}{43.382377in}}%
\pgfpathlineto{\pgfqpoint{5.800000in}{43.500229in}}%
\pgfpathlineto{\pgfqpoint{5.800000in}{43.500229in}}%
\pgfpathclose%
\pgfusepath{fill}%
\end{pgfscope}%
\begin{pgfscope}%
\pgfpathrectangle{\pgfqpoint{5.800000in}{0.720000in}}{\pgfqpoint{1.400000in}{4.620000in}}%
\pgfusepath{clip}%
\pgfsetbuttcap%
\pgfsetmiterjoin%
\definecolor{currentfill}{rgb}{0.121569,0.466667,0.705882}%
\pgfsetfillcolor{currentfill}%
\pgfsetlinewidth{0.000000pt}%
\definecolor{currentstroke}{rgb}{0.000000,0.000000,0.000000}%
\pgfsetstrokecolor{currentstroke}%
\pgfsetstrokeopacity{0.000000}%
\pgfsetdash{}{0pt}%
\pgfpathmoveto{\pgfqpoint{5.800000in}{43.500229in}}%
\pgfpathlineto{\pgfqpoint{5.800000in}{43.500229in}}%
\pgfpathlineto{\pgfqpoint{5.800000in}{43.618081in}}%
\pgfpathlineto{\pgfqpoint{5.800000in}{43.618081in}}%
\pgfpathclose%
\pgfusepath{fill}%
\end{pgfscope}%
\begin{pgfscope}%
\pgfpathrectangle{\pgfqpoint{5.800000in}{0.720000in}}{\pgfqpoint{1.400000in}{4.620000in}}%
\pgfusepath{clip}%
\pgfsetbuttcap%
\pgfsetmiterjoin%
\definecolor{currentfill}{rgb}{0.121569,0.466667,0.705882}%
\pgfsetfillcolor{currentfill}%
\pgfsetlinewidth{0.000000pt}%
\definecolor{currentstroke}{rgb}{0.000000,0.000000,0.000000}%
\pgfsetstrokecolor{currentstroke}%
\pgfsetstrokeopacity{0.000000}%
\pgfsetdash{}{0pt}%
\pgfpathmoveto{\pgfqpoint{5.800000in}{43.618081in}}%
\pgfpathlineto{\pgfqpoint{5.800000in}{43.618081in}}%
\pgfpathlineto{\pgfqpoint{5.800000in}{43.735933in}}%
\pgfpathlineto{\pgfqpoint{5.800000in}{43.735933in}}%
\pgfpathclose%
\pgfusepath{fill}%
\end{pgfscope}%
\begin{pgfscope}%
\definecolor{textcolor}{rgb}{0.000000,0.000000,0.000000}%
\pgfsetstrokecolor{textcolor}%
\pgfsetfillcolor{textcolor}%
\pgftext[x=6.500000in,y=0.664444in,,top]{\color{textcolor}\sffamily\fontsize{20.000000}{24.000000}\selectfont \(\displaystyle \mathrm{arb.\ unit}\)}%
\end{pgfscope}%
\begin{pgfscope}%
\pgfsetrectcap%
\pgfsetmiterjoin%
\pgfsetlinewidth{0.803000pt}%
\definecolor{currentstroke}{rgb}{0.000000,0.000000,0.000000}%
\pgfsetstrokecolor{currentstroke}%
\pgfsetdash{}{0pt}%
\pgfpathmoveto{\pgfqpoint{5.800000in}{0.720000in}}%
\pgfpathlineto{\pgfqpoint{5.800000in}{5.340000in}}%
\pgfusepath{stroke}%
\end{pgfscope}%
\begin{pgfscope}%
\pgfsetrectcap%
\pgfsetmiterjoin%
\pgfsetlinewidth{0.803000pt}%
\definecolor{currentstroke}{rgb}{0.000000,0.000000,0.000000}%
\pgfsetstrokecolor{currentstroke}%
\pgfsetdash{}{0pt}%
\pgfpathmoveto{\pgfqpoint{7.200000in}{0.720000in}}%
\pgfpathlineto{\pgfqpoint{7.200000in}{5.340000in}}%
\pgfusepath{stroke}%
\end{pgfscope}%
\begin{pgfscope}%
\pgfsetrectcap%
\pgfsetmiterjoin%
\pgfsetlinewidth{0.803000pt}%
\definecolor{currentstroke}{rgb}{0.000000,0.000000,0.000000}%
\pgfsetstrokecolor{currentstroke}%
\pgfsetdash{}{0pt}%
\pgfpathmoveto{\pgfqpoint{5.800000in}{0.720000in}}%
\pgfpathlineto{\pgfqpoint{7.200000in}{0.720000in}}%
\pgfusepath{stroke}%
\end{pgfscope}%
\begin{pgfscope}%
\pgfsetrectcap%
\pgfsetmiterjoin%
\pgfsetlinewidth{0.803000pt}%
\definecolor{currentstroke}{rgb}{0.000000,0.000000,0.000000}%
\pgfsetstrokecolor{currentstroke}%
\pgfsetdash{}{0pt}%
\pgfpathmoveto{\pgfqpoint{5.800000in}{5.340000in}}%
\pgfpathlineto{\pgfqpoint{7.200000in}{5.340000in}}%
\pgfusepath{stroke}%
\end{pgfscope}%
\end{pgfpicture}%
\makeatother%
\endgroup%
}
    \caption{\label{fig:fsmp-npe} $D_\mathrm{w}$ histogram and distributions conditioned on $N_{\mathrm{PE}}$, errorbar explained in figure~\ref{fig:cnn-performance}.}
  \end{subfigure}
  \hspace{0.5em}
  \begin{subfigure}[b]{.55\textwidth}
    \centering
    \resizebox{\textwidth}{!}{%% Creator: Matplotlib, PGF backend
%%
%% To include the figure in your LaTeX document, write
%%   \input{<filename>.pgf}
%%
%% Make sure the required packages are loaded in your preamble
%%   \usepackage{pgf}
%%
%% and, on pdftex
%%   \usepackage[utf8]{inputenc}\DeclareUnicodeCharacter{2212}{-}
%%
%% or, on luatex and xetex
%%   \usepackage{unicode-math}
%%
%% Figures using additional raster images can only be included by \input if
%% they are in the same directory as the main LaTeX file. For loading figures
%% from other directories you can use the `import` package
%%   \usepackage{import}
%%
%% and then include the figures with
%%   \import{<path to file>}{<filename>.pgf}
%%
%% Matplotlib used the following preamble
%%   \usepackage[detect-all,locale=DE]{siunitx}
%%
\begingroup%
\makeatletter%
\begin{pgfpicture}%
\pgfpathrectangle{\pgfpointorigin}{\pgfqpoint{10.000000in}{10.000000in}}%
\pgfusepath{use as bounding box, clip}%
\begin{pgfscope}%
\pgfsetbuttcap%
\pgfsetmiterjoin%
\definecolor{currentfill}{rgb}{1.000000,1.000000,1.000000}%
\pgfsetfillcolor{currentfill}%
\pgfsetlinewidth{0.000000pt}%
\definecolor{currentstroke}{rgb}{1.000000,1.000000,1.000000}%
\pgfsetstrokecolor{currentstroke}%
\pgfsetdash{}{0pt}%
\pgfpathmoveto{\pgfqpoint{0.000000in}{0.000000in}}%
\pgfpathlineto{\pgfqpoint{10.000000in}{0.000000in}}%
\pgfpathlineto{\pgfqpoint{10.000000in}{10.000000in}}%
\pgfpathlineto{\pgfqpoint{0.000000in}{10.000000in}}%
\pgfpathclose%
\pgfusepath{fill}%
\end{pgfscope}%
\begin{pgfscope}%
\pgfsetbuttcap%
\pgfsetmiterjoin%
\definecolor{currentfill}{rgb}{1.000000,1.000000,1.000000}%
\pgfsetfillcolor{currentfill}%
\pgfsetlinewidth{0.000000pt}%
\definecolor{currentstroke}{rgb}{0.000000,0.000000,0.000000}%
\pgfsetstrokecolor{currentstroke}%
\pgfsetstrokeopacity{0.000000}%
\pgfsetdash{}{0pt}%
\pgfpathmoveto{\pgfqpoint{1.000000in}{2.000000in}}%
\pgfpathlineto{\pgfqpoint{9.500000in}{2.000000in}}%
\pgfpathlineto{\pgfqpoint{9.500000in}{5.000000in}}%
\pgfpathlineto{\pgfqpoint{1.000000in}{5.000000in}}%
\pgfpathclose%
\pgfusepath{fill}%
\end{pgfscope}%
\begin{pgfscope}%
\pgfpathrectangle{\pgfqpoint{1.000000in}{2.000000in}}{\pgfqpoint{8.500000in}{3.000000in}}%
\pgfusepath{clip}%
\pgfsetrectcap%
\pgfsetroundjoin%
\pgfsetlinewidth{0.803000pt}%
\definecolor{currentstroke}{rgb}{0.690196,0.690196,0.690196}%
\pgfsetstrokecolor{currentstroke}%
\pgfsetdash{}{0pt}%
\pgfpathmoveto{\pgfqpoint{2.725210in}{2.000000in}}%
\pgfpathlineto{\pgfqpoint{2.725210in}{5.000000in}}%
\pgfusepath{stroke}%
\end{pgfscope}%
\begin{pgfscope}%
\pgfsetbuttcap%
\pgfsetroundjoin%
\definecolor{currentfill}{rgb}{0.000000,0.000000,0.000000}%
\pgfsetfillcolor{currentfill}%
\pgfsetlinewidth{0.803000pt}%
\definecolor{currentstroke}{rgb}{0.000000,0.000000,0.000000}%
\pgfsetstrokecolor{currentstroke}%
\pgfsetdash{}{0pt}%
\pgfsys@defobject{currentmarker}{\pgfqpoint{0.000000in}{-0.048611in}}{\pgfqpoint{0.000000in}{0.000000in}}{%
\pgfpathmoveto{\pgfqpoint{0.000000in}{0.000000in}}%
\pgfpathlineto{\pgfqpoint{0.000000in}{-0.048611in}}%
\pgfusepath{stroke,fill}%
}%
\begin{pgfscope}%
\pgfsys@transformshift{2.725210in}{2.000000in}%
\pgfsys@useobject{currentmarker}{}%
\end{pgfscope}%
\end{pgfscope}%
\begin{pgfscope}%
\pgfpathrectangle{\pgfqpoint{1.000000in}{2.000000in}}{\pgfqpoint{8.500000in}{3.000000in}}%
\pgfusepath{clip}%
\pgfsetrectcap%
\pgfsetroundjoin%
\pgfsetlinewidth{0.803000pt}%
\definecolor{currentstroke}{rgb}{0.690196,0.690196,0.690196}%
\pgfsetstrokecolor{currentstroke}%
\pgfsetdash{}{0pt}%
\pgfpathmoveto{\pgfqpoint{4.506292in}{2.000000in}}%
\pgfpathlineto{\pgfqpoint{4.506292in}{5.000000in}}%
\pgfusepath{stroke}%
\end{pgfscope}%
\begin{pgfscope}%
\pgfsetbuttcap%
\pgfsetroundjoin%
\definecolor{currentfill}{rgb}{0.000000,0.000000,0.000000}%
\pgfsetfillcolor{currentfill}%
\pgfsetlinewidth{0.803000pt}%
\definecolor{currentstroke}{rgb}{0.000000,0.000000,0.000000}%
\pgfsetstrokecolor{currentstroke}%
\pgfsetdash{}{0pt}%
\pgfsys@defobject{currentmarker}{\pgfqpoint{0.000000in}{-0.048611in}}{\pgfqpoint{0.000000in}{0.000000in}}{%
\pgfpathmoveto{\pgfqpoint{0.000000in}{0.000000in}}%
\pgfpathlineto{\pgfqpoint{0.000000in}{-0.048611in}}%
\pgfusepath{stroke,fill}%
}%
\begin{pgfscope}%
\pgfsys@transformshift{4.506292in}{2.000000in}%
\pgfsys@useobject{currentmarker}{}%
\end{pgfscope}%
\end{pgfscope}%
\begin{pgfscope}%
\pgfpathrectangle{\pgfqpoint{1.000000in}{2.000000in}}{\pgfqpoint{8.500000in}{3.000000in}}%
\pgfusepath{clip}%
\pgfsetrectcap%
\pgfsetroundjoin%
\pgfsetlinewidth{0.803000pt}%
\definecolor{currentstroke}{rgb}{0.690196,0.690196,0.690196}%
\pgfsetstrokecolor{currentstroke}%
\pgfsetdash{}{0pt}%
\pgfpathmoveto{\pgfqpoint{6.287373in}{2.000000in}}%
\pgfpathlineto{\pgfqpoint{6.287373in}{5.000000in}}%
\pgfusepath{stroke}%
\end{pgfscope}%
\begin{pgfscope}%
\pgfsetbuttcap%
\pgfsetroundjoin%
\definecolor{currentfill}{rgb}{0.000000,0.000000,0.000000}%
\pgfsetfillcolor{currentfill}%
\pgfsetlinewidth{0.803000pt}%
\definecolor{currentstroke}{rgb}{0.000000,0.000000,0.000000}%
\pgfsetstrokecolor{currentstroke}%
\pgfsetdash{}{0pt}%
\pgfsys@defobject{currentmarker}{\pgfqpoint{0.000000in}{-0.048611in}}{\pgfqpoint{0.000000in}{0.000000in}}{%
\pgfpathmoveto{\pgfqpoint{0.000000in}{0.000000in}}%
\pgfpathlineto{\pgfqpoint{0.000000in}{-0.048611in}}%
\pgfusepath{stroke,fill}%
}%
\begin{pgfscope}%
\pgfsys@transformshift{6.287373in}{2.000000in}%
\pgfsys@useobject{currentmarker}{}%
\end{pgfscope}%
\end{pgfscope}%
\begin{pgfscope}%
\pgfpathrectangle{\pgfqpoint{1.000000in}{2.000000in}}{\pgfqpoint{8.500000in}{3.000000in}}%
\pgfusepath{clip}%
\pgfsetrectcap%
\pgfsetroundjoin%
\pgfsetlinewidth{0.803000pt}%
\definecolor{currentstroke}{rgb}{0.690196,0.690196,0.690196}%
\pgfsetstrokecolor{currentstroke}%
\pgfsetdash{}{0pt}%
\pgfpathmoveto{\pgfqpoint{8.068455in}{2.000000in}}%
\pgfpathlineto{\pgfqpoint{8.068455in}{5.000000in}}%
\pgfusepath{stroke}%
\end{pgfscope}%
\begin{pgfscope}%
\pgfsetbuttcap%
\pgfsetroundjoin%
\definecolor{currentfill}{rgb}{0.000000,0.000000,0.000000}%
\pgfsetfillcolor{currentfill}%
\pgfsetlinewidth{0.803000pt}%
\definecolor{currentstroke}{rgb}{0.000000,0.000000,0.000000}%
\pgfsetstrokecolor{currentstroke}%
\pgfsetdash{}{0pt}%
\pgfsys@defobject{currentmarker}{\pgfqpoint{0.000000in}{-0.048611in}}{\pgfqpoint{0.000000in}{0.000000in}}{%
\pgfpathmoveto{\pgfqpoint{0.000000in}{0.000000in}}%
\pgfpathlineto{\pgfqpoint{0.000000in}{-0.048611in}}%
\pgfusepath{stroke,fill}%
}%
\begin{pgfscope}%
\pgfsys@transformshift{8.068455in}{2.000000in}%
\pgfsys@useobject{currentmarker}{}%
\end{pgfscope}%
\end{pgfscope}%
\begin{pgfscope}%
\pgfpathrectangle{\pgfqpoint{1.000000in}{2.000000in}}{\pgfqpoint{8.500000in}{3.000000in}}%
\pgfusepath{clip}%
\pgfsetrectcap%
\pgfsetroundjoin%
\pgfsetlinewidth{0.803000pt}%
\definecolor{currentstroke}{rgb}{0.690196,0.690196,0.690196}%
\pgfsetstrokecolor{currentstroke}%
\pgfsetdash{}{0pt}%
\pgfpathmoveto{\pgfqpoint{1.000000in}{2.545455in}}%
\pgfpathlineto{\pgfqpoint{9.500000in}{2.545455in}}%
\pgfusepath{stroke}%
\end{pgfscope}%
\begin{pgfscope}%
\pgfsetbuttcap%
\pgfsetroundjoin%
\definecolor{currentfill}{rgb}{0.000000,0.000000,0.000000}%
\pgfsetfillcolor{currentfill}%
\pgfsetlinewidth{0.803000pt}%
\definecolor{currentstroke}{rgb}{0.000000,0.000000,0.000000}%
\pgfsetstrokecolor{currentstroke}%
\pgfsetdash{}{0pt}%
\pgfsys@defobject{currentmarker}{\pgfqpoint{-0.048611in}{0.000000in}}{\pgfqpoint{-0.000000in}{0.000000in}}{%
\pgfpathmoveto{\pgfqpoint{-0.000000in}{0.000000in}}%
\pgfpathlineto{\pgfqpoint{-0.048611in}{0.000000in}}%
\pgfusepath{stroke,fill}%
}%
\begin{pgfscope}%
\pgfsys@transformshift{1.000000in}{2.545455in}%
\pgfsys@useobject{currentmarker}{}%
\end{pgfscope}%
\end{pgfscope}%
\begin{pgfscope}%
\definecolor{textcolor}{rgb}{0.000000,0.000000,0.000000}%
\pgfsetstrokecolor{textcolor}%
\pgfsetfillcolor{textcolor}%
\pgftext[x=0.770670in, y=2.445435in, left, base]{\color{textcolor}\sffamily\fontsize{20.000000}{24.000000}\selectfont \(\displaystyle {0}\)}%
\end{pgfscope}%
\begin{pgfscope}%
\pgfpathrectangle{\pgfqpoint{1.000000in}{2.000000in}}{\pgfqpoint{8.500000in}{3.000000in}}%
\pgfusepath{clip}%
\pgfsetrectcap%
\pgfsetroundjoin%
\pgfsetlinewidth{0.803000pt}%
\definecolor{currentstroke}{rgb}{0.690196,0.690196,0.690196}%
\pgfsetstrokecolor{currentstroke}%
\pgfsetdash{}{0pt}%
\pgfpathmoveto{\pgfqpoint{1.000000in}{3.227273in}}%
\pgfpathlineto{\pgfqpoint{9.500000in}{3.227273in}}%
\pgfusepath{stroke}%
\end{pgfscope}%
\begin{pgfscope}%
\pgfsetbuttcap%
\pgfsetroundjoin%
\definecolor{currentfill}{rgb}{0.000000,0.000000,0.000000}%
\pgfsetfillcolor{currentfill}%
\pgfsetlinewidth{0.803000pt}%
\definecolor{currentstroke}{rgb}{0.000000,0.000000,0.000000}%
\pgfsetstrokecolor{currentstroke}%
\pgfsetdash{}{0pt}%
\pgfsys@defobject{currentmarker}{\pgfqpoint{-0.048611in}{0.000000in}}{\pgfqpoint{-0.000000in}{0.000000in}}{%
\pgfpathmoveto{\pgfqpoint{-0.000000in}{0.000000in}}%
\pgfpathlineto{\pgfqpoint{-0.048611in}{0.000000in}}%
\pgfusepath{stroke,fill}%
}%
\begin{pgfscope}%
\pgfsys@transformshift{1.000000in}{3.227273in}%
\pgfsys@useobject{currentmarker}{}%
\end{pgfscope}%
\end{pgfscope}%
\begin{pgfscope}%
\definecolor{textcolor}{rgb}{0.000000,0.000000,0.000000}%
\pgfsetstrokecolor{textcolor}%
\pgfsetfillcolor{textcolor}%
\pgftext[x=0.638563in, y=3.127253in, left, base]{\color{textcolor}\sffamily\fontsize{20.000000}{24.000000}\selectfont \(\displaystyle {10}\)}%
\end{pgfscope}%
\begin{pgfscope}%
\pgfpathrectangle{\pgfqpoint{1.000000in}{2.000000in}}{\pgfqpoint{8.500000in}{3.000000in}}%
\pgfusepath{clip}%
\pgfsetrectcap%
\pgfsetroundjoin%
\pgfsetlinewidth{0.803000pt}%
\definecolor{currentstroke}{rgb}{0.690196,0.690196,0.690196}%
\pgfsetstrokecolor{currentstroke}%
\pgfsetdash{}{0pt}%
\pgfpathmoveto{\pgfqpoint{1.000000in}{3.909091in}}%
\pgfpathlineto{\pgfqpoint{9.500000in}{3.909091in}}%
\pgfusepath{stroke}%
\end{pgfscope}%
\begin{pgfscope}%
\pgfsetbuttcap%
\pgfsetroundjoin%
\definecolor{currentfill}{rgb}{0.000000,0.000000,0.000000}%
\pgfsetfillcolor{currentfill}%
\pgfsetlinewidth{0.803000pt}%
\definecolor{currentstroke}{rgb}{0.000000,0.000000,0.000000}%
\pgfsetstrokecolor{currentstroke}%
\pgfsetdash{}{0pt}%
\pgfsys@defobject{currentmarker}{\pgfqpoint{-0.048611in}{0.000000in}}{\pgfqpoint{-0.000000in}{0.000000in}}{%
\pgfpathmoveto{\pgfqpoint{-0.000000in}{0.000000in}}%
\pgfpathlineto{\pgfqpoint{-0.048611in}{0.000000in}}%
\pgfusepath{stroke,fill}%
}%
\begin{pgfscope}%
\pgfsys@transformshift{1.000000in}{3.909091in}%
\pgfsys@useobject{currentmarker}{}%
\end{pgfscope}%
\end{pgfscope}%
\begin{pgfscope}%
\definecolor{textcolor}{rgb}{0.000000,0.000000,0.000000}%
\pgfsetstrokecolor{textcolor}%
\pgfsetfillcolor{textcolor}%
\pgftext[x=0.638563in, y=3.809072in, left, base]{\color{textcolor}\sffamily\fontsize{20.000000}{24.000000}\selectfont \(\displaystyle {20}\)}%
\end{pgfscope}%
\begin{pgfscope}%
\pgfpathrectangle{\pgfqpoint{1.000000in}{2.000000in}}{\pgfqpoint{8.500000in}{3.000000in}}%
\pgfusepath{clip}%
\pgfsetrectcap%
\pgfsetroundjoin%
\pgfsetlinewidth{0.803000pt}%
\definecolor{currentstroke}{rgb}{0.690196,0.690196,0.690196}%
\pgfsetstrokecolor{currentstroke}%
\pgfsetdash{}{0pt}%
\pgfpathmoveto{\pgfqpoint{1.000000in}{4.590909in}}%
\pgfpathlineto{\pgfqpoint{9.500000in}{4.590909in}}%
\pgfusepath{stroke}%
\end{pgfscope}%
\begin{pgfscope}%
\pgfsetbuttcap%
\pgfsetroundjoin%
\definecolor{currentfill}{rgb}{0.000000,0.000000,0.000000}%
\pgfsetfillcolor{currentfill}%
\pgfsetlinewidth{0.803000pt}%
\definecolor{currentstroke}{rgb}{0.000000,0.000000,0.000000}%
\pgfsetstrokecolor{currentstroke}%
\pgfsetdash{}{0pt}%
\pgfsys@defobject{currentmarker}{\pgfqpoint{-0.048611in}{0.000000in}}{\pgfqpoint{-0.000000in}{0.000000in}}{%
\pgfpathmoveto{\pgfqpoint{-0.000000in}{0.000000in}}%
\pgfpathlineto{\pgfqpoint{-0.048611in}{0.000000in}}%
\pgfusepath{stroke,fill}%
}%
\begin{pgfscope}%
\pgfsys@transformshift{1.000000in}{4.590909in}%
\pgfsys@useobject{currentmarker}{}%
\end{pgfscope}%
\end{pgfscope}%
\begin{pgfscope}%
\definecolor{textcolor}{rgb}{0.000000,0.000000,0.000000}%
\pgfsetstrokecolor{textcolor}%
\pgfsetfillcolor{textcolor}%
\pgftext[x=0.638563in, y=4.490890in, left, base]{\color{textcolor}\sffamily\fontsize{20.000000}{24.000000}\selectfont \(\displaystyle {30}\)}%
\end{pgfscope}%
\begin{pgfscope}%
\definecolor{textcolor}{rgb}{0.000000,0.000000,0.000000}%
\pgfsetstrokecolor{textcolor}%
\pgfsetfillcolor{textcolor}%
\pgftext[x=0.583007in,y=3.500000in,,bottom,rotate=90.000000]{\color{textcolor}\sffamily\fontsize{20.000000}{24.000000}\selectfont \(\displaystyle \mathrm{Voltage}/\si{mV}\)}%
\end{pgfscope}%
\begin{pgfscope}%
\pgfpathrectangle{\pgfqpoint{1.000000in}{2.000000in}}{\pgfqpoint{8.500000in}{3.000000in}}%
\pgfusepath{clip}%
\pgfsetbuttcap%
\pgfsetroundjoin%
\pgfsetlinewidth{2.007500pt}%
\definecolor{currentstroke}{rgb}{0.000000,0.750000,0.750000}%
\pgfsetstrokecolor{currentstroke}%
\pgfsetdash{}{0pt}%
\pgfpathmoveto{\pgfqpoint{0.990000in}{2.886364in}}%
\pgfpathlineto{\pgfqpoint{9.510000in}{2.886364in}}%
\pgfusepath{stroke}%
\end{pgfscope}%
\begin{pgfscope}%
\pgfpathrectangle{\pgfqpoint{1.000000in}{2.000000in}}{\pgfqpoint{8.500000in}{3.000000in}}%
\pgfusepath{clip}%
\pgfsetrectcap%
\pgfsetroundjoin%
\pgfsetlinewidth{2.007500pt}%
\definecolor{currentstroke}{rgb}{0.000000,0.000000,1.000000}%
\pgfsetstrokecolor{currentstroke}%
\pgfsetdash{}{0pt}%
\pgfpathmoveto{\pgfqpoint{0.990000in}{2.487563in}}%
\pgfpathlineto{\pgfqpoint{1.015372in}{2.681818in}}%
\pgfpathlineto{\pgfqpoint{1.050994in}{2.477273in}}%
\pgfpathlineto{\pgfqpoint{1.122237in}{2.613636in}}%
\pgfpathlineto{\pgfqpoint{1.157859in}{2.477273in}}%
\pgfpathlineto{\pgfqpoint{1.193480in}{2.545455in}}%
\pgfpathlineto{\pgfqpoint{1.264724in}{2.545455in}}%
\pgfpathlineto{\pgfqpoint{1.300345in}{2.613636in}}%
\pgfpathlineto{\pgfqpoint{1.371588in}{2.613636in}}%
\pgfpathlineto{\pgfqpoint{1.478453in}{2.409091in}}%
\pgfpathlineto{\pgfqpoint{1.514075in}{2.613636in}}%
\pgfpathlineto{\pgfqpoint{1.549697in}{2.613636in}}%
\pgfpathlineto{\pgfqpoint{1.585318in}{2.545455in}}%
\pgfpathlineto{\pgfqpoint{1.656562in}{2.545455in}}%
\pgfpathlineto{\pgfqpoint{1.692183in}{2.477273in}}%
\pgfpathlineto{\pgfqpoint{1.727805in}{2.613636in}}%
\pgfpathlineto{\pgfqpoint{1.799048in}{2.613636in}}%
\pgfpathlineto{\pgfqpoint{1.834670in}{2.545455in}}%
\pgfpathlineto{\pgfqpoint{1.905913in}{2.545455in}}%
\pgfpathlineto{\pgfqpoint{1.941535in}{2.613636in}}%
\pgfpathlineto{\pgfqpoint{1.977156in}{2.477273in}}%
\pgfpathlineto{\pgfqpoint{2.048399in}{2.477273in}}%
\pgfpathlineto{\pgfqpoint{2.084021in}{2.613636in}}%
\pgfpathlineto{\pgfqpoint{2.119643in}{2.477273in}}%
\pgfpathlineto{\pgfqpoint{2.155264in}{2.477273in}}%
\pgfpathlineto{\pgfqpoint{2.190886in}{2.613636in}}%
\pgfpathlineto{\pgfqpoint{2.226508in}{2.545455in}}%
\pgfpathlineto{\pgfqpoint{2.333372in}{2.545455in}}%
\pgfpathlineto{\pgfqpoint{2.368994in}{2.477273in}}%
\pgfpathlineto{\pgfqpoint{2.404616in}{2.613636in}}%
\pgfpathlineto{\pgfqpoint{2.440237in}{2.477273in}}%
\pgfpathlineto{\pgfqpoint{2.475859in}{2.477273in}}%
\pgfpathlineto{\pgfqpoint{2.511481in}{2.545455in}}%
\pgfpathlineto{\pgfqpoint{2.582724in}{2.545455in}}%
\pgfpathlineto{\pgfqpoint{2.618345in}{2.477273in}}%
\pgfpathlineto{\pgfqpoint{2.653967in}{2.613636in}}%
\pgfpathlineto{\pgfqpoint{2.689589in}{2.545455in}}%
\pgfpathlineto{\pgfqpoint{2.725210in}{2.613636in}}%
\pgfpathlineto{\pgfqpoint{2.760832in}{2.545455in}}%
\pgfpathlineto{\pgfqpoint{2.796454in}{2.613636in}}%
\pgfpathlineto{\pgfqpoint{2.867697in}{2.613636in}}%
\pgfpathlineto{\pgfqpoint{2.903318in}{2.886364in}}%
\pgfpathlineto{\pgfqpoint{2.938940in}{3.227273in}}%
\pgfpathlineto{\pgfqpoint{2.974562in}{3.295455in}}%
\pgfpathlineto{\pgfqpoint{3.010183in}{3.636364in}}%
\pgfpathlineto{\pgfqpoint{3.045805in}{3.636364in}}%
\pgfpathlineto{\pgfqpoint{3.081427in}{3.568182in}}%
\pgfpathlineto{\pgfqpoint{3.117048in}{3.568182in}}%
\pgfpathlineto{\pgfqpoint{3.152670in}{3.500000in}}%
\pgfpathlineto{\pgfqpoint{3.188292in}{3.840909in}}%
\pgfpathlineto{\pgfqpoint{3.223913in}{3.840909in}}%
\pgfpathlineto{\pgfqpoint{3.259535in}{4.318182in}}%
\pgfpathlineto{\pgfqpoint{3.295156in}{4.386364in}}%
\pgfpathlineto{\pgfqpoint{3.330778in}{4.386364in}}%
\pgfpathlineto{\pgfqpoint{3.366400in}{4.454545in}}%
\pgfpathlineto{\pgfqpoint{3.402021in}{4.250000in}}%
\pgfpathlineto{\pgfqpoint{3.437643in}{3.977273in}}%
\pgfpathlineto{\pgfqpoint{3.473265in}{3.909091in}}%
\pgfpathlineto{\pgfqpoint{3.508886in}{3.772727in}}%
\pgfpathlineto{\pgfqpoint{3.544508in}{4.045455in}}%
\pgfpathlineto{\pgfqpoint{3.580129in}{4.250000in}}%
\pgfpathlineto{\pgfqpoint{3.615751in}{4.113636in}}%
\pgfpathlineto{\pgfqpoint{3.651373in}{4.181818in}}%
\pgfpathlineto{\pgfqpoint{3.722616in}{4.590909in}}%
\pgfpathlineto{\pgfqpoint{3.758238in}{4.659091in}}%
\pgfpathlineto{\pgfqpoint{3.793859in}{4.659091in}}%
\pgfpathlineto{\pgfqpoint{3.829481in}{4.454545in}}%
\pgfpathlineto{\pgfqpoint{3.865102in}{4.386364in}}%
\pgfpathlineto{\pgfqpoint{3.900724in}{4.250000in}}%
\pgfpathlineto{\pgfqpoint{3.936346in}{3.977273in}}%
\pgfpathlineto{\pgfqpoint{3.971967in}{3.840909in}}%
\pgfpathlineto{\pgfqpoint{4.007589in}{3.772727in}}%
\pgfpathlineto{\pgfqpoint{4.043211in}{3.568182in}}%
\pgfpathlineto{\pgfqpoint{4.078832in}{3.295455in}}%
\pgfpathlineto{\pgfqpoint{4.150075in}{3.022727in}}%
\pgfpathlineto{\pgfqpoint{4.185697in}{2.954545in}}%
\pgfpathlineto{\pgfqpoint{4.256940in}{2.954545in}}%
\pgfpathlineto{\pgfqpoint{4.292562in}{3.159091in}}%
\pgfpathlineto{\pgfqpoint{4.328184in}{3.500000in}}%
\pgfpathlineto{\pgfqpoint{4.363805in}{3.772727in}}%
\pgfpathlineto{\pgfqpoint{4.399427in}{3.772727in}}%
\pgfpathlineto{\pgfqpoint{4.435048in}{4.045455in}}%
\pgfpathlineto{\pgfqpoint{4.470670in}{3.909091in}}%
\pgfpathlineto{\pgfqpoint{4.506292in}{3.704545in}}%
\pgfpathlineto{\pgfqpoint{4.541913in}{3.568182in}}%
\pgfpathlineto{\pgfqpoint{4.577535in}{3.500000in}}%
\pgfpathlineto{\pgfqpoint{4.648778in}{3.227273in}}%
\pgfpathlineto{\pgfqpoint{4.684400in}{3.227273in}}%
\pgfpathlineto{\pgfqpoint{4.720021in}{2.954545in}}%
\pgfpathlineto{\pgfqpoint{4.755643in}{2.954545in}}%
\pgfpathlineto{\pgfqpoint{4.826886in}{2.818182in}}%
\pgfpathlineto{\pgfqpoint{4.862508in}{2.818182in}}%
\pgfpathlineto{\pgfqpoint{4.898130in}{2.750000in}}%
\pgfpathlineto{\pgfqpoint{4.969373in}{2.750000in}}%
\pgfpathlineto{\pgfqpoint{5.004995in}{2.477273in}}%
\pgfpathlineto{\pgfqpoint{5.040616in}{2.613636in}}%
\pgfpathlineto{\pgfqpoint{5.076238in}{2.613636in}}%
\pgfpathlineto{\pgfqpoint{5.111859in}{2.545455in}}%
\pgfpathlineto{\pgfqpoint{5.147481in}{2.613636in}}%
\pgfpathlineto{\pgfqpoint{5.183103in}{2.613636in}}%
\pgfpathlineto{\pgfqpoint{5.218724in}{2.545455in}}%
\pgfpathlineto{\pgfqpoint{5.254346in}{2.613636in}}%
\pgfpathlineto{\pgfqpoint{5.289968in}{2.545455in}}%
\pgfpathlineto{\pgfqpoint{5.325589in}{2.681818in}}%
\pgfpathlineto{\pgfqpoint{5.361211in}{2.613636in}}%
\pgfpathlineto{\pgfqpoint{5.396832in}{2.681818in}}%
\pgfpathlineto{\pgfqpoint{5.432454in}{2.545455in}}%
\pgfpathlineto{\pgfqpoint{5.468076in}{2.613636in}}%
\pgfpathlineto{\pgfqpoint{5.503697in}{2.477273in}}%
\pgfpathlineto{\pgfqpoint{5.539319in}{2.613636in}}%
\pgfpathlineto{\pgfqpoint{5.574941in}{2.477273in}}%
\pgfpathlineto{\pgfqpoint{5.610562in}{2.477273in}}%
\pgfpathlineto{\pgfqpoint{5.646184in}{2.409091in}}%
\pgfpathlineto{\pgfqpoint{5.681805in}{2.681818in}}%
\pgfpathlineto{\pgfqpoint{5.717427in}{2.613636in}}%
\pgfpathlineto{\pgfqpoint{5.753049in}{2.477273in}}%
\pgfpathlineto{\pgfqpoint{5.788670in}{2.545455in}}%
\pgfpathlineto{\pgfqpoint{5.824292in}{2.681818in}}%
\pgfpathlineto{\pgfqpoint{5.895535in}{2.409091in}}%
\pgfpathlineto{\pgfqpoint{5.931157in}{2.545455in}}%
\pgfpathlineto{\pgfqpoint{5.966778in}{2.613636in}}%
\pgfpathlineto{\pgfqpoint{6.002400in}{2.477273in}}%
\pgfpathlineto{\pgfqpoint{6.038022in}{2.613636in}}%
\pgfpathlineto{\pgfqpoint{6.073643in}{2.409091in}}%
\pgfpathlineto{\pgfqpoint{6.109265in}{2.545455in}}%
\pgfpathlineto{\pgfqpoint{6.144887in}{2.545455in}}%
\pgfpathlineto{\pgfqpoint{6.180508in}{2.613636in}}%
\pgfpathlineto{\pgfqpoint{6.216130in}{2.477273in}}%
\pgfpathlineto{\pgfqpoint{6.251751in}{2.545455in}}%
\pgfpathlineto{\pgfqpoint{6.287373in}{2.545455in}}%
\pgfpathlineto{\pgfqpoint{6.322995in}{2.613636in}}%
\pgfpathlineto{\pgfqpoint{6.358616in}{2.545455in}}%
\pgfpathlineto{\pgfqpoint{6.394238in}{2.545455in}}%
\pgfpathlineto{\pgfqpoint{6.429860in}{2.477273in}}%
\pgfpathlineto{\pgfqpoint{6.465481in}{2.545455in}}%
\pgfpathlineto{\pgfqpoint{6.501103in}{2.409091in}}%
\pgfpathlineto{\pgfqpoint{6.536725in}{2.545455in}}%
\pgfpathlineto{\pgfqpoint{6.572346in}{2.545455in}}%
\pgfpathlineto{\pgfqpoint{6.607968in}{2.613636in}}%
\pgfpathlineto{\pgfqpoint{6.643589in}{2.409091in}}%
\pgfpathlineto{\pgfqpoint{6.679211in}{2.681818in}}%
\pgfpathlineto{\pgfqpoint{6.714833in}{2.477273in}}%
\pgfpathlineto{\pgfqpoint{6.750454in}{2.545455in}}%
\pgfpathlineto{\pgfqpoint{6.786076in}{2.681818in}}%
\pgfpathlineto{\pgfqpoint{6.821698in}{2.545455in}}%
\pgfpathlineto{\pgfqpoint{6.857319in}{2.613636in}}%
\pgfpathlineto{\pgfqpoint{6.892941in}{2.613636in}}%
\pgfpathlineto{\pgfqpoint{6.928562in}{2.477273in}}%
\pgfpathlineto{\pgfqpoint{6.964184in}{2.545455in}}%
\pgfpathlineto{\pgfqpoint{6.999806in}{2.545455in}}%
\pgfpathlineto{\pgfqpoint{7.035427in}{2.613636in}}%
\pgfpathlineto{\pgfqpoint{7.106671in}{2.477273in}}%
\pgfpathlineto{\pgfqpoint{7.142292in}{2.613636in}}%
\pgfpathlineto{\pgfqpoint{7.177914in}{2.613636in}}%
\pgfpathlineto{\pgfqpoint{7.213535in}{2.681818in}}%
\pgfpathlineto{\pgfqpoint{7.249157in}{2.477273in}}%
\pgfpathlineto{\pgfqpoint{7.284779in}{2.545455in}}%
\pgfpathlineto{\pgfqpoint{7.320400in}{2.409091in}}%
\pgfpathlineto{\pgfqpoint{7.356022in}{2.613636in}}%
\pgfpathlineto{\pgfqpoint{7.391644in}{2.613636in}}%
\pgfpathlineto{\pgfqpoint{7.462887in}{2.477273in}}%
\pgfpathlineto{\pgfqpoint{7.498508in}{2.477273in}}%
\pgfpathlineto{\pgfqpoint{7.534130in}{2.613636in}}%
\pgfpathlineto{\pgfqpoint{7.569752in}{2.477273in}}%
\pgfpathlineto{\pgfqpoint{7.605373in}{2.545455in}}%
\pgfpathlineto{\pgfqpoint{7.640995in}{2.477273in}}%
\pgfpathlineto{\pgfqpoint{7.712238in}{2.613636in}}%
\pgfpathlineto{\pgfqpoint{7.747860in}{2.545455in}}%
\pgfpathlineto{\pgfqpoint{7.783481in}{2.409091in}}%
\pgfpathlineto{\pgfqpoint{7.854725in}{2.545455in}}%
\pgfpathlineto{\pgfqpoint{7.890346in}{2.477273in}}%
\pgfpathlineto{\pgfqpoint{7.925968in}{2.613636in}}%
\pgfpathlineto{\pgfqpoint{7.961590in}{2.545455in}}%
\pgfpathlineto{\pgfqpoint{7.997211in}{2.545455in}}%
\pgfpathlineto{\pgfqpoint{8.068455in}{2.409091in}}%
\pgfpathlineto{\pgfqpoint{8.104076in}{2.545455in}}%
\pgfpathlineto{\pgfqpoint{8.175319in}{2.545455in}}%
\pgfpathlineto{\pgfqpoint{8.210941in}{2.613636in}}%
\pgfpathlineto{\pgfqpoint{8.246563in}{2.613636in}}%
\pgfpathlineto{\pgfqpoint{8.282184in}{2.477273in}}%
\pgfpathlineto{\pgfqpoint{8.317806in}{2.409091in}}%
\pgfpathlineto{\pgfqpoint{8.353428in}{2.477273in}}%
\pgfpathlineto{\pgfqpoint{8.389049in}{2.613636in}}%
\pgfpathlineto{\pgfqpoint{8.424671in}{2.545455in}}%
\pgfpathlineto{\pgfqpoint{8.460292in}{2.613636in}}%
\pgfpathlineto{\pgfqpoint{8.495914in}{2.477273in}}%
\pgfpathlineto{\pgfqpoint{8.531536in}{2.545455in}}%
\pgfpathlineto{\pgfqpoint{8.567157in}{2.477273in}}%
\pgfpathlineto{\pgfqpoint{8.638401in}{2.613636in}}%
\pgfpathlineto{\pgfqpoint{8.674022in}{2.545455in}}%
\pgfpathlineto{\pgfqpoint{8.709644in}{2.545455in}}%
\pgfpathlineto{\pgfqpoint{8.745265in}{2.409091in}}%
\pgfpathlineto{\pgfqpoint{8.780887in}{2.545455in}}%
\pgfpathlineto{\pgfqpoint{8.816509in}{2.477273in}}%
\pgfpathlineto{\pgfqpoint{8.887752in}{2.477273in}}%
\pgfpathlineto{\pgfqpoint{8.923374in}{2.545455in}}%
\pgfpathlineto{\pgfqpoint{9.137103in}{2.545455in}}%
\pgfpathlineto{\pgfqpoint{9.172725in}{2.613636in}}%
\pgfpathlineto{\pgfqpoint{9.208347in}{2.613636in}}%
\pgfpathlineto{\pgfqpoint{9.243968in}{2.477273in}}%
\pgfpathlineto{\pgfqpoint{9.279590in}{2.477273in}}%
\pgfpathlineto{\pgfqpoint{9.315211in}{2.613636in}}%
\pgfpathlineto{\pgfqpoint{9.350833in}{2.477273in}}%
\pgfpathlineto{\pgfqpoint{9.386455in}{2.613636in}}%
\pgfpathlineto{\pgfqpoint{9.422076in}{2.477273in}}%
\pgfpathlineto{\pgfqpoint{9.457698in}{2.545455in}}%
\pgfpathlineto{\pgfqpoint{9.493320in}{2.477273in}}%
\pgfpathlineto{\pgfqpoint{9.510000in}{2.541127in}}%
\pgfpathlineto{\pgfqpoint{9.510000in}{2.541127in}}%
\pgfusepath{stroke}%
\end{pgfscope}%
\begin{pgfscope}%
\pgfpathrectangle{\pgfqpoint{1.000000in}{2.000000in}}{\pgfqpoint{8.500000in}{3.000000in}}%
\pgfusepath{clip}%
\pgfsetrectcap%
\pgfsetroundjoin%
\pgfsetlinewidth{2.007500pt}%
\definecolor{currentstroke}{rgb}{0.000000,0.000000,0.000000}%
\pgfsetstrokecolor{currentstroke}%
\pgfsetdash{}{0pt}%
\pgfpathmoveto{\pgfqpoint{0.990000in}{2.545455in}}%
\pgfpathlineto{\pgfqpoint{2.796454in}{2.545455in}}%
\pgfpathlineto{\pgfqpoint{2.832075in}{2.548459in}}%
\pgfpathlineto{\pgfqpoint{2.867697in}{2.611155in}}%
\pgfpathlineto{\pgfqpoint{2.903318in}{2.812862in}}%
\pgfpathlineto{\pgfqpoint{2.938940in}{3.103160in}}%
\pgfpathlineto{\pgfqpoint{2.974562in}{3.375760in}}%
\pgfpathlineto{\pgfqpoint{3.010183in}{3.564298in}}%
\pgfpathlineto{\pgfqpoint{3.045805in}{3.654000in}}%
\pgfpathlineto{\pgfqpoint{3.081427in}{3.659941in}}%
\pgfpathlineto{\pgfqpoint{3.117048in}{3.608361in}}%
\pgfpathlineto{\pgfqpoint{3.152670in}{3.574626in}}%
\pgfpathlineto{\pgfqpoint{3.188292in}{3.691743in}}%
\pgfpathlineto{\pgfqpoint{3.259535in}{4.194172in}}%
\pgfpathlineto{\pgfqpoint{3.295156in}{4.360256in}}%
\pgfpathlineto{\pgfqpoint{3.330778in}{4.411138in}}%
\pgfpathlineto{\pgfqpoint{3.366400in}{4.361519in}}%
\pgfpathlineto{\pgfqpoint{3.402021in}{4.241584in}}%
\pgfpathlineto{\pgfqpoint{3.473265in}{3.920983in}}%
\pgfpathlineto{\pgfqpoint{3.508886in}{3.866682in}}%
\pgfpathlineto{\pgfqpoint{3.544508in}{3.960020in}}%
\pgfpathlineto{\pgfqpoint{3.580129in}{4.112593in}}%
\pgfpathlineto{\pgfqpoint{3.615751in}{4.229592in}}%
\pgfpathlineto{\pgfqpoint{3.651373in}{4.306836in}}%
\pgfpathlineto{\pgfqpoint{3.686994in}{4.413237in}}%
\pgfpathlineto{\pgfqpoint{3.722616in}{4.537410in}}%
\pgfpathlineto{\pgfqpoint{3.758238in}{4.617582in}}%
\pgfpathlineto{\pgfqpoint{3.793859in}{4.616812in}}%
\pgfpathlineto{\pgfqpoint{3.829481in}{4.534912in}}%
\pgfpathlineto{\pgfqpoint{3.865102in}{4.392063in}}%
\pgfpathlineto{\pgfqpoint{3.900724in}{4.213268in}}%
\pgfpathlineto{\pgfqpoint{3.971967in}{3.828464in}}%
\pgfpathlineto{\pgfqpoint{4.007589in}{3.648045in}}%
\pgfpathlineto{\pgfqpoint{4.043211in}{3.484288in}}%
\pgfpathlineto{\pgfqpoint{4.078832in}{3.339381in}}%
\pgfpathlineto{\pgfqpoint{4.114454in}{3.213469in}}%
\pgfpathlineto{\pgfqpoint{4.150075in}{3.105503in}}%
\pgfpathlineto{\pgfqpoint{4.185697in}{3.013855in}}%
\pgfpathlineto{\pgfqpoint{4.221319in}{2.951060in}}%
\pgfpathlineto{\pgfqpoint{4.256940in}{3.006035in}}%
\pgfpathlineto{\pgfqpoint{4.292562in}{3.215910in}}%
\pgfpathlineto{\pgfqpoint{4.328184in}{3.485379in}}%
\pgfpathlineto{\pgfqpoint{4.363805in}{3.709387in}}%
\pgfpathlineto{\pgfqpoint{4.399427in}{3.838631in}}%
\pgfpathlineto{\pgfqpoint{4.435048in}{3.871472in}}%
\pgfpathlineto{\pgfqpoint{4.470670in}{3.829070in}}%
\pgfpathlineto{\pgfqpoint{4.506292in}{3.737629in}}%
\pgfpathlineto{\pgfqpoint{4.541913in}{3.619935in}}%
\pgfpathlineto{\pgfqpoint{4.613157in}{3.367273in}}%
\pgfpathlineto{\pgfqpoint{4.648778in}{3.249891in}}%
\pgfpathlineto{\pgfqpoint{4.684400in}{3.143957in}}%
\pgfpathlineto{\pgfqpoint{4.720021in}{3.050665in}}%
\pgfpathlineto{\pgfqpoint{4.755643in}{2.969916in}}%
\pgfpathlineto{\pgfqpoint{4.791265in}{2.900892in}}%
\pgfpathlineto{\pgfqpoint{4.826886in}{2.842424in}}%
\pgfpathlineto{\pgfqpoint{4.862508in}{2.793222in}}%
\pgfpathlineto{\pgfqpoint{4.898130in}{2.752016in}}%
\pgfpathlineto{\pgfqpoint{4.933751in}{2.717621in}}%
\pgfpathlineto{\pgfqpoint{4.969373in}{2.688978in}}%
\pgfpathlineto{\pgfqpoint{5.004995in}{2.665159in}}%
\pgfpathlineto{\pgfqpoint{5.040616in}{2.645369in}}%
\pgfpathlineto{\pgfqpoint{5.076238in}{2.628931in}}%
\pgfpathlineto{\pgfqpoint{5.111859in}{2.615275in}}%
\pgfpathlineto{\pgfqpoint{5.147481in}{2.603927in}}%
\pgfpathlineto{\pgfqpoint{5.183103in}{2.594489in}}%
\pgfpathlineto{\pgfqpoint{5.218724in}{2.586633in}}%
\pgfpathlineto{\pgfqpoint{5.289968in}{2.574627in}}%
\pgfpathlineto{\pgfqpoint{5.361211in}{2.566250in}}%
\pgfpathlineto{\pgfqpoint{5.432454in}{2.560374in}}%
\pgfpathlineto{\pgfqpoint{5.539319in}{2.554629in}}%
\pgfpathlineto{\pgfqpoint{5.646184in}{2.551177in}}%
\pgfpathlineto{\pgfqpoint{5.824292in}{2.548140in}}%
\pgfpathlineto{\pgfqpoint{6.073643in}{2.546442in}}%
\pgfpathlineto{\pgfqpoint{6.607968in}{2.545596in}}%
\pgfpathlineto{\pgfqpoint{9.350833in}{2.545455in}}%
\pgfpathlineto{\pgfqpoint{9.510000in}{2.545455in}}%
\pgfpathlineto{\pgfqpoint{9.510000in}{2.545455in}}%
\pgfusepath{stroke}%
\end{pgfscope}%
\begin{pgfscope}%
\pgfpathrectangle{\pgfqpoint{1.000000in}{2.000000in}}{\pgfqpoint{8.500000in}{3.000000in}}%
\pgfusepath{clip}%
\pgfsetrectcap%
\pgfsetroundjoin%
\pgfsetlinewidth{2.007500pt}%
\definecolor{currentstroke}{rgb}{0.000000,0.500000,0.000000}%
\pgfsetstrokecolor{currentstroke}%
\pgfsetdash{}{0pt}%
\pgfpathmoveto{\pgfqpoint{0.990000in}{2.545455in}}%
\pgfpathlineto{\pgfqpoint{2.796454in}{2.545639in}}%
\pgfpathlineto{\pgfqpoint{2.832075in}{2.568002in}}%
\pgfpathlineto{\pgfqpoint{2.867697in}{2.699187in}}%
\pgfpathlineto{\pgfqpoint{2.903318in}{2.948221in}}%
\pgfpathlineto{\pgfqpoint{2.938940in}{3.222309in}}%
\pgfpathlineto{\pgfqpoint{2.974562in}{3.437700in}}%
\pgfpathlineto{\pgfqpoint{3.010183in}{3.561427in}}%
\pgfpathlineto{\pgfqpoint{3.045805in}{3.598312in}}%
\pgfpathlineto{\pgfqpoint{3.081427in}{3.569645in}}%
\pgfpathlineto{\pgfqpoint{3.117048in}{3.516276in}}%
\pgfpathlineto{\pgfqpoint{3.152670in}{3.539686in}}%
\pgfpathlineto{\pgfqpoint{3.188292in}{3.716332in}}%
\pgfpathlineto{\pgfqpoint{3.223913in}{3.988611in}}%
\pgfpathlineto{\pgfqpoint{3.259535in}{4.237345in}}%
\pgfpathlineto{\pgfqpoint{3.295156in}{4.386169in}}%
\pgfpathlineto{\pgfqpoint{3.330778in}{4.418868in}}%
\pgfpathlineto{\pgfqpoint{3.366400in}{4.354992in}}%
\pgfpathlineto{\pgfqpoint{3.402021in}{4.225747in}}%
\pgfpathlineto{\pgfqpoint{3.437643in}{4.060769in}}%
\pgfpathlineto{\pgfqpoint{3.473265in}{3.907510in}}%
\pgfpathlineto{\pgfqpoint{3.508886in}{3.879072in}}%
\pgfpathlineto{\pgfqpoint{3.544508in}{3.994841in}}%
\pgfpathlineto{\pgfqpoint{3.580129in}{4.154806in}}%
\pgfpathlineto{\pgfqpoint{3.615751in}{4.266354in}}%
\pgfpathlineto{\pgfqpoint{3.651373in}{4.311086in}}%
\pgfpathlineto{\pgfqpoint{3.686994in}{4.367700in}}%
\pgfpathlineto{\pgfqpoint{3.722616in}{4.466203in}}%
\pgfpathlineto{\pgfqpoint{3.758238in}{4.551698in}}%
\pgfpathlineto{\pgfqpoint{3.793859in}{4.572733in}}%
\pgfpathlineto{\pgfqpoint{3.829481in}{4.515418in}}%
\pgfpathlineto{\pgfqpoint{3.865102in}{4.392856in}}%
\pgfpathlineto{\pgfqpoint{3.900724in}{4.228016in}}%
\pgfpathlineto{\pgfqpoint{4.007589in}{3.675278in}}%
\pgfpathlineto{\pgfqpoint{4.043211in}{3.510332in}}%
\pgfpathlineto{\pgfqpoint{4.078832in}{3.363226in}}%
\pgfpathlineto{\pgfqpoint{4.114454in}{3.234673in}}%
\pgfpathlineto{\pgfqpoint{4.150075in}{3.123979in}}%
\pgfpathlineto{\pgfqpoint{4.185697in}{3.029897in}}%
\pgfpathlineto{\pgfqpoint{4.221319in}{2.975759in}}%
\pgfpathlineto{\pgfqpoint{4.256940in}{3.058679in}}%
\pgfpathlineto{\pgfqpoint{4.292562in}{3.287173in}}%
\pgfpathlineto{\pgfqpoint{4.328184in}{3.553572in}}%
\pgfpathlineto{\pgfqpoint{4.363805in}{3.760729in}}%
\pgfpathlineto{\pgfqpoint{4.399427in}{3.869711in}}%
\pgfpathlineto{\pgfqpoint{4.435048in}{3.884941in}}%
\pgfpathlineto{\pgfqpoint{4.470670in}{3.829658in}}%
\pgfpathlineto{\pgfqpoint{4.506292in}{3.729962in}}%
\pgfpathlineto{\pgfqpoint{4.541913in}{3.607691in}}%
\pgfpathlineto{\pgfqpoint{4.613157in}{3.352762in}}%
\pgfpathlineto{\pgfqpoint{4.648778in}{3.236085in}}%
\pgfpathlineto{\pgfqpoint{4.684400in}{3.131380in}}%
\pgfpathlineto{\pgfqpoint{4.720021in}{3.039533in}}%
\pgfpathlineto{\pgfqpoint{4.755643in}{2.960258in}}%
\pgfpathlineto{\pgfqpoint{4.791265in}{2.892633in}}%
\pgfpathlineto{\pgfqpoint{4.826886in}{2.835433in}}%
\pgfpathlineto{\pgfqpoint{4.862508in}{2.787350in}}%
\pgfpathlineto{\pgfqpoint{4.898130in}{2.747110in}}%
\pgfpathlineto{\pgfqpoint{4.933751in}{2.713539in}}%
\pgfpathlineto{\pgfqpoint{4.969373in}{2.685591in}}%
\pgfpathlineto{\pgfqpoint{5.004995in}{2.662355in}}%
\pgfpathlineto{\pgfqpoint{5.040616in}{2.643049in}}%
\pgfpathlineto{\pgfqpoint{5.076238in}{2.627013in}}%
\pgfpathlineto{\pgfqpoint{5.111859in}{2.613691in}}%
\pgfpathlineto{\pgfqpoint{5.147481in}{2.602617in}}%
\pgfpathlineto{\pgfqpoint{5.183103in}{2.593406in}}%
\pgfpathlineto{\pgfqpoint{5.218724in}{2.585737in}}%
\pgfpathlineto{\pgfqpoint{5.289968in}{2.574012in}}%
\pgfpathlineto{\pgfqpoint{5.361211in}{2.565827in}}%
\pgfpathlineto{\pgfqpoint{5.432454in}{2.560080in}}%
\pgfpathlineto{\pgfqpoint{5.539319in}{2.554459in}}%
\pgfpathlineto{\pgfqpoint{5.681805in}{2.550274in}}%
\pgfpathlineto{\pgfqpoint{5.859914in}{2.547737in}}%
\pgfpathlineto{\pgfqpoint{6.144887in}{2.546195in}}%
\pgfpathlineto{\pgfqpoint{6.821698in}{2.545523in}}%
\pgfpathlineto{\pgfqpoint{9.510000in}{2.545455in}}%
\pgfpathlineto{\pgfqpoint{9.510000in}{2.545455in}}%
\pgfusepath{stroke}%
\end{pgfscope}%
\begin{pgfscope}%
\pgfsetrectcap%
\pgfsetmiterjoin%
\pgfsetlinewidth{0.803000pt}%
\definecolor{currentstroke}{rgb}{0.000000,0.000000,0.000000}%
\pgfsetstrokecolor{currentstroke}%
\pgfsetdash{}{0pt}%
\pgfpathmoveto{\pgfqpoint{1.000000in}{2.000000in}}%
\pgfpathlineto{\pgfqpoint{1.000000in}{5.000000in}}%
\pgfusepath{stroke}%
\end{pgfscope}%
\begin{pgfscope}%
\pgfsetrectcap%
\pgfsetmiterjoin%
\pgfsetlinewidth{0.803000pt}%
\definecolor{currentstroke}{rgb}{0.000000,0.000000,0.000000}%
\pgfsetstrokecolor{currentstroke}%
\pgfsetdash{}{0pt}%
\pgfpathmoveto{\pgfqpoint{9.500000in}{2.000000in}}%
\pgfpathlineto{\pgfqpoint{9.500000in}{5.000000in}}%
\pgfusepath{stroke}%
\end{pgfscope}%
\begin{pgfscope}%
\pgfsetrectcap%
\pgfsetmiterjoin%
\pgfsetlinewidth{0.803000pt}%
\definecolor{currentstroke}{rgb}{0.000000,0.000000,0.000000}%
\pgfsetstrokecolor{currentstroke}%
\pgfsetdash{}{0pt}%
\pgfpathmoveto{\pgfqpoint{1.000000in}{2.000000in}}%
\pgfpathlineto{\pgfqpoint{9.500000in}{2.000000in}}%
\pgfusepath{stroke}%
\end{pgfscope}%
\begin{pgfscope}%
\pgfsetrectcap%
\pgfsetmiterjoin%
\pgfsetlinewidth{0.803000pt}%
\definecolor{currentstroke}{rgb}{0.000000,0.000000,0.000000}%
\pgfsetstrokecolor{currentstroke}%
\pgfsetdash{}{0pt}%
\pgfpathmoveto{\pgfqpoint{1.000000in}{5.000000in}}%
\pgfpathlineto{\pgfqpoint{9.500000in}{5.000000in}}%
\pgfusepath{stroke}%
\end{pgfscope}%
\begin{pgfscope}%
\pgfsetbuttcap%
\pgfsetmiterjoin%
\definecolor{currentfill}{rgb}{1.000000,1.000000,1.000000}%
\pgfsetfillcolor{currentfill}%
\pgfsetfillopacity{0.800000}%
\pgfsetlinewidth{1.003750pt}%
\definecolor{currentstroke}{rgb}{0.800000,0.800000,0.800000}%
\pgfsetstrokecolor{currentstroke}%
\pgfsetstrokeopacity{0.800000}%
\pgfsetdash{}{0pt}%
\pgfpathmoveto{\pgfqpoint{7.151502in}{3.197952in}}%
\pgfpathlineto{\pgfqpoint{9.305556in}{3.197952in}}%
\pgfpathquadraticcurveto{\pgfqpoint{9.361111in}{3.197952in}}{\pgfqpoint{9.361111in}{3.253507in}}%
\pgfpathlineto{\pgfqpoint{9.361111in}{4.805556in}}%
\pgfpathquadraticcurveto{\pgfqpoint{9.361111in}{4.861111in}}{\pgfqpoint{9.305556in}{4.861111in}}%
\pgfpathlineto{\pgfqpoint{7.151502in}{4.861111in}}%
\pgfpathquadraticcurveto{\pgfqpoint{7.095946in}{4.861111in}}{\pgfqpoint{7.095946in}{4.805556in}}%
\pgfpathlineto{\pgfqpoint{7.095946in}{3.253507in}}%
\pgfpathquadraticcurveto{\pgfqpoint{7.095946in}{3.197952in}}{\pgfqpoint{7.151502in}{3.197952in}}%
\pgfpathclose%
\pgfusepath{stroke,fill}%
\end{pgfscope}%
\begin{pgfscope}%
\pgfsetrectcap%
\pgfsetroundjoin%
\pgfsetlinewidth{2.007500pt}%
\definecolor{currentstroke}{rgb}{0.000000,0.000000,1.000000}%
\pgfsetstrokecolor{currentstroke}%
\pgfsetdash{}{0pt}%
\pgfpathmoveto{\pgfqpoint{7.207057in}{4.647184in}}%
\pgfpathlineto{\pgfqpoint{7.762613in}{4.647184in}}%
\pgfusepath{stroke}%
\end{pgfscope}%
\begin{pgfscope}%
\definecolor{textcolor}{rgb}{0.000000,0.000000,0.000000}%
\pgfsetstrokecolor{textcolor}%
\pgfsetfillcolor{textcolor}%
\pgftext[x=7.984835in,y=4.549962in,left,base]{\color{textcolor}\sffamily\fontsize{20.000000}{24.000000}\selectfont origin wave}%
\end{pgfscope}%
\begin{pgfscope}%
\pgfsetrectcap%
\pgfsetroundjoin%
\pgfsetlinewidth{2.007500pt}%
\definecolor{currentstroke}{rgb}{0.000000,0.000000,0.000000}%
\pgfsetstrokecolor{currentstroke}%
\pgfsetdash{}{0pt}%
\pgfpathmoveto{\pgfqpoint{7.207057in}{4.252227in}}%
\pgfpathlineto{\pgfqpoint{7.762613in}{4.252227in}}%
\pgfusepath{stroke}%
\end{pgfscope}%
\begin{pgfscope}%
\definecolor{textcolor}{rgb}{0.000000,0.000000,0.000000}%
\pgfsetstrokecolor{textcolor}%
\pgfsetfillcolor{textcolor}%
\pgftext[x=7.984835in,y=4.155005in,left,base]{\color{textcolor}\sffamily\fontsize{20.000000}{24.000000}\selectfont truth wave}%
\end{pgfscope}%
\begin{pgfscope}%
\pgfsetrectcap%
\pgfsetroundjoin%
\pgfsetlinewidth{2.007500pt}%
\definecolor{currentstroke}{rgb}{0.000000,0.500000,0.000000}%
\pgfsetstrokecolor{currentstroke}%
\pgfsetdash{}{0pt}%
\pgfpathmoveto{\pgfqpoint{7.207057in}{3.857271in}}%
\pgfpathlineto{\pgfqpoint{7.762613in}{3.857271in}}%
\pgfusepath{stroke}%
\end{pgfscope}%
\begin{pgfscope}%
\definecolor{textcolor}{rgb}{0.000000,0.000000,0.000000}%
\pgfsetstrokecolor{textcolor}%
\pgfsetfillcolor{textcolor}%
\pgftext[x=7.984835in,y=3.760048in,left,base]{\color{textcolor}\sffamily\fontsize{20.000000}{24.000000}\selectfont recon wave}%
\end{pgfscope}%
\begin{pgfscope}%
\pgfsetbuttcap%
\pgfsetroundjoin%
\pgfsetlinewidth{2.007500pt}%
\definecolor{currentstroke}{rgb}{0.000000,0.750000,0.750000}%
\pgfsetstrokecolor{currentstroke}%
\pgfsetdash{}{0pt}%
\pgfpathmoveto{\pgfqpoint{7.207057in}{3.462314in}}%
\pgfpathlineto{\pgfqpoint{7.762613in}{3.462314in}}%
\pgfusepath{stroke}%
\end{pgfscope}%
\begin{pgfscope}%
\definecolor{textcolor}{rgb}{0.000000,0.000000,0.000000}%
\pgfsetstrokecolor{textcolor}%
\pgfsetfillcolor{textcolor}%
\pgftext[x=7.984835in,y=3.365092in,left,base]{\color{textcolor}\sffamily\fontsize{20.000000}{24.000000}\selectfont threshold}%
\end{pgfscope}%
\begin{pgfscope}%
\pgfsetbuttcap%
\pgfsetmiterjoin%
\definecolor{currentfill}{rgb}{1.000000,1.000000,1.000000}%
\pgfsetfillcolor{currentfill}%
\pgfsetlinewidth{0.000000pt}%
\definecolor{currentstroke}{rgb}{0.000000,0.000000,0.000000}%
\pgfsetstrokecolor{currentstroke}%
\pgfsetstrokeopacity{0.000000}%
\pgfsetdash{}{0pt}%
\pgfpathmoveto{\pgfqpoint{1.000000in}{5.000000in}}%
\pgfpathlineto{\pgfqpoint{9.500000in}{5.000000in}}%
\pgfpathlineto{\pgfqpoint{9.500000in}{7.000000in}}%
\pgfpathlineto{\pgfqpoint{1.000000in}{7.000000in}}%
\pgfpathclose%
\pgfusepath{fill}%
\end{pgfscope}%
\begin{pgfscope}%
\pgfpathrectangle{\pgfqpoint{1.000000in}{5.000000in}}{\pgfqpoint{8.500000in}{2.000000in}}%
\pgfusepath{clip}%
\pgfsetrectcap%
\pgfsetroundjoin%
\pgfsetlinewidth{0.803000pt}%
\definecolor{currentstroke}{rgb}{0.690196,0.690196,0.690196}%
\pgfsetstrokecolor{currentstroke}%
\pgfsetdash{}{0pt}%
\pgfpathmoveto{\pgfqpoint{2.725210in}{5.000000in}}%
\pgfpathlineto{\pgfqpoint{2.725210in}{7.000000in}}%
\pgfusepath{stroke}%
\end{pgfscope}%
\begin{pgfscope}%
\pgfsetbuttcap%
\pgfsetroundjoin%
\definecolor{currentfill}{rgb}{0.000000,0.000000,0.000000}%
\pgfsetfillcolor{currentfill}%
\pgfsetlinewidth{0.803000pt}%
\definecolor{currentstroke}{rgb}{0.000000,0.000000,0.000000}%
\pgfsetstrokecolor{currentstroke}%
\pgfsetdash{}{0pt}%
\pgfsys@defobject{currentmarker}{\pgfqpoint{0.000000in}{-0.048611in}}{\pgfqpoint{0.000000in}{0.000000in}}{%
\pgfpathmoveto{\pgfqpoint{0.000000in}{0.000000in}}%
\pgfpathlineto{\pgfqpoint{0.000000in}{-0.048611in}}%
\pgfusepath{stroke,fill}%
}%
\begin{pgfscope}%
\pgfsys@transformshift{2.725210in}{5.000000in}%
\pgfsys@useobject{currentmarker}{}%
\end{pgfscope}%
\end{pgfscope}%
\begin{pgfscope}%
\pgfpathrectangle{\pgfqpoint{1.000000in}{5.000000in}}{\pgfqpoint{8.500000in}{2.000000in}}%
\pgfusepath{clip}%
\pgfsetrectcap%
\pgfsetroundjoin%
\pgfsetlinewidth{0.803000pt}%
\definecolor{currentstroke}{rgb}{0.690196,0.690196,0.690196}%
\pgfsetstrokecolor{currentstroke}%
\pgfsetdash{}{0pt}%
\pgfpathmoveto{\pgfqpoint{4.506292in}{5.000000in}}%
\pgfpathlineto{\pgfqpoint{4.506292in}{7.000000in}}%
\pgfusepath{stroke}%
\end{pgfscope}%
\begin{pgfscope}%
\pgfsetbuttcap%
\pgfsetroundjoin%
\definecolor{currentfill}{rgb}{0.000000,0.000000,0.000000}%
\pgfsetfillcolor{currentfill}%
\pgfsetlinewidth{0.803000pt}%
\definecolor{currentstroke}{rgb}{0.000000,0.000000,0.000000}%
\pgfsetstrokecolor{currentstroke}%
\pgfsetdash{}{0pt}%
\pgfsys@defobject{currentmarker}{\pgfqpoint{0.000000in}{-0.048611in}}{\pgfqpoint{0.000000in}{0.000000in}}{%
\pgfpathmoveto{\pgfqpoint{0.000000in}{0.000000in}}%
\pgfpathlineto{\pgfqpoint{0.000000in}{-0.048611in}}%
\pgfusepath{stroke,fill}%
}%
\begin{pgfscope}%
\pgfsys@transformshift{4.506292in}{5.000000in}%
\pgfsys@useobject{currentmarker}{}%
\end{pgfscope}%
\end{pgfscope}%
\begin{pgfscope}%
\pgfpathrectangle{\pgfqpoint{1.000000in}{5.000000in}}{\pgfqpoint{8.500000in}{2.000000in}}%
\pgfusepath{clip}%
\pgfsetrectcap%
\pgfsetroundjoin%
\pgfsetlinewidth{0.803000pt}%
\definecolor{currentstroke}{rgb}{0.690196,0.690196,0.690196}%
\pgfsetstrokecolor{currentstroke}%
\pgfsetdash{}{0pt}%
\pgfpathmoveto{\pgfqpoint{6.287373in}{5.000000in}}%
\pgfpathlineto{\pgfqpoint{6.287373in}{7.000000in}}%
\pgfusepath{stroke}%
\end{pgfscope}%
\begin{pgfscope}%
\pgfsetbuttcap%
\pgfsetroundjoin%
\definecolor{currentfill}{rgb}{0.000000,0.000000,0.000000}%
\pgfsetfillcolor{currentfill}%
\pgfsetlinewidth{0.803000pt}%
\definecolor{currentstroke}{rgb}{0.000000,0.000000,0.000000}%
\pgfsetstrokecolor{currentstroke}%
\pgfsetdash{}{0pt}%
\pgfsys@defobject{currentmarker}{\pgfqpoint{0.000000in}{-0.048611in}}{\pgfqpoint{0.000000in}{0.000000in}}{%
\pgfpathmoveto{\pgfqpoint{0.000000in}{0.000000in}}%
\pgfpathlineto{\pgfqpoint{0.000000in}{-0.048611in}}%
\pgfusepath{stroke,fill}%
}%
\begin{pgfscope}%
\pgfsys@transformshift{6.287373in}{5.000000in}%
\pgfsys@useobject{currentmarker}{}%
\end{pgfscope}%
\end{pgfscope}%
\begin{pgfscope}%
\pgfpathrectangle{\pgfqpoint{1.000000in}{5.000000in}}{\pgfqpoint{8.500000in}{2.000000in}}%
\pgfusepath{clip}%
\pgfsetrectcap%
\pgfsetroundjoin%
\pgfsetlinewidth{0.803000pt}%
\definecolor{currentstroke}{rgb}{0.690196,0.690196,0.690196}%
\pgfsetstrokecolor{currentstroke}%
\pgfsetdash{}{0pt}%
\pgfpathmoveto{\pgfqpoint{8.068455in}{5.000000in}}%
\pgfpathlineto{\pgfqpoint{8.068455in}{7.000000in}}%
\pgfusepath{stroke}%
\end{pgfscope}%
\begin{pgfscope}%
\pgfsetbuttcap%
\pgfsetroundjoin%
\definecolor{currentfill}{rgb}{0.000000,0.000000,0.000000}%
\pgfsetfillcolor{currentfill}%
\pgfsetlinewidth{0.803000pt}%
\definecolor{currentstroke}{rgb}{0.000000,0.000000,0.000000}%
\pgfsetstrokecolor{currentstroke}%
\pgfsetdash{}{0pt}%
\pgfsys@defobject{currentmarker}{\pgfqpoint{0.000000in}{-0.048611in}}{\pgfqpoint{0.000000in}{0.000000in}}{%
\pgfpathmoveto{\pgfqpoint{0.000000in}{0.000000in}}%
\pgfpathlineto{\pgfqpoint{0.000000in}{-0.048611in}}%
\pgfusepath{stroke,fill}%
}%
\begin{pgfscope}%
\pgfsys@transformshift{8.068455in}{5.000000in}%
\pgfsys@useobject{currentmarker}{}%
\end{pgfscope}%
\end{pgfscope}%
\begin{pgfscope}%
\pgfpathrectangle{\pgfqpoint{1.000000in}{5.000000in}}{\pgfqpoint{8.500000in}{2.000000in}}%
\pgfusepath{clip}%
\pgfsetrectcap%
\pgfsetroundjoin%
\pgfsetlinewidth{0.803000pt}%
\definecolor{currentstroke}{rgb}{0.690196,0.690196,0.690196}%
\pgfsetstrokecolor{currentstroke}%
\pgfsetdash{}{0pt}%
\pgfpathmoveto{\pgfqpoint{1.000000in}{5.000000in}}%
\pgfpathlineto{\pgfqpoint{9.500000in}{5.000000in}}%
\pgfusepath{stroke}%
\end{pgfscope}%
\begin{pgfscope}%
\pgfsetbuttcap%
\pgfsetroundjoin%
\definecolor{currentfill}{rgb}{0.000000,0.000000,0.000000}%
\pgfsetfillcolor{currentfill}%
\pgfsetlinewidth{0.803000pt}%
\definecolor{currentstroke}{rgb}{0.000000,0.000000,0.000000}%
\pgfsetstrokecolor{currentstroke}%
\pgfsetdash{}{0pt}%
\pgfsys@defobject{currentmarker}{\pgfqpoint{-0.048611in}{0.000000in}}{\pgfqpoint{-0.000000in}{0.000000in}}{%
\pgfpathmoveto{\pgfqpoint{-0.000000in}{0.000000in}}%
\pgfpathlineto{\pgfqpoint{-0.048611in}{0.000000in}}%
\pgfusepath{stroke,fill}%
}%
\begin{pgfscope}%
\pgfsys@transformshift{1.000000in}{5.000000in}%
\pgfsys@useobject{currentmarker}{}%
\end{pgfscope}%
\end{pgfscope}%
\begin{pgfscope}%
\definecolor{textcolor}{rgb}{0.000000,0.000000,0.000000}%
\pgfsetstrokecolor{textcolor}%
\pgfsetfillcolor{textcolor}%
\pgftext[x=0.560215in, y=4.899981in, left, base]{\color{textcolor}\sffamily\fontsize{20.000000}{24.000000}\selectfont \(\displaystyle {0.0}\)}%
\end{pgfscope}%
\begin{pgfscope}%
\pgfpathrectangle{\pgfqpoint{1.000000in}{5.000000in}}{\pgfqpoint{8.500000in}{2.000000in}}%
\pgfusepath{clip}%
\pgfsetrectcap%
\pgfsetroundjoin%
\pgfsetlinewidth{0.803000pt}%
\definecolor{currentstroke}{rgb}{0.690196,0.690196,0.690196}%
\pgfsetstrokecolor{currentstroke}%
\pgfsetdash{}{0pt}%
\pgfpathmoveto{\pgfqpoint{1.000000in}{5.245724in}}%
\pgfpathlineto{\pgfqpoint{9.500000in}{5.245724in}}%
\pgfusepath{stroke}%
\end{pgfscope}%
\begin{pgfscope}%
\pgfsetbuttcap%
\pgfsetroundjoin%
\definecolor{currentfill}{rgb}{0.000000,0.000000,0.000000}%
\pgfsetfillcolor{currentfill}%
\pgfsetlinewidth{0.803000pt}%
\definecolor{currentstroke}{rgb}{0.000000,0.000000,0.000000}%
\pgfsetstrokecolor{currentstroke}%
\pgfsetdash{}{0pt}%
\pgfsys@defobject{currentmarker}{\pgfqpoint{-0.048611in}{0.000000in}}{\pgfqpoint{-0.000000in}{0.000000in}}{%
\pgfpathmoveto{\pgfqpoint{-0.000000in}{0.000000in}}%
\pgfpathlineto{\pgfqpoint{-0.048611in}{0.000000in}}%
\pgfusepath{stroke,fill}%
}%
\begin{pgfscope}%
\pgfsys@transformshift{1.000000in}{5.245724in}%
\pgfsys@useobject{currentmarker}{}%
\end{pgfscope}%
\end{pgfscope}%
\begin{pgfscope}%
\definecolor{textcolor}{rgb}{0.000000,0.000000,0.000000}%
\pgfsetstrokecolor{textcolor}%
\pgfsetfillcolor{textcolor}%
\pgftext[x=0.560215in, y=5.145705in, left, base]{\color{textcolor}\sffamily\fontsize{20.000000}{24.000000}\selectfont \(\displaystyle {0.2}\)}%
\end{pgfscope}%
\begin{pgfscope}%
\pgfpathrectangle{\pgfqpoint{1.000000in}{5.000000in}}{\pgfqpoint{8.500000in}{2.000000in}}%
\pgfusepath{clip}%
\pgfsetrectcap%
\pgfsetroundjoin%
\pgfsetlinewidth{0.803000pt}%
\definecolor{currentstroke}{rgb}{0.690196,0.690196,0.690196}%
\pgfsetstrokecolor{currentstroke}%
\pgfsetdash{}{0pt}%
\pgfpathmoveto{\pgfqpoint{1.000000in}{5.491449in}}%
\pgfpathlineto{\pgfqpoint{9.500000in}{5.491449in}}%
\pgfusepath{stroke}%
\end{pgfscope}%
\begin{pgfscope}%
\pgfsetbuttcap%
\pgfsetroundjoin%
\definecolor{currentfill}{rgb}{0.000000,0.000000,0.000000}%
\pgfsetfillcolor{currentfill}%
\pgfsetlinewidth{0.803000pt}%
\definecolor{currentstroke}{rgb}{0.000000,0.000000,0.000000}%
\pgfsetstrokecolor{currentstroke}%
\pgfsetdash{}{0pt}%
\pgfsys@defobject{currentmarker}{\pgfqpoint{-0.048611in}{0.000000in}}{\pgfqpoint{-0.000000in}{0.000000in}}{%
\pgfpathmoveto{\pgfqpoint{-0.000000in}{0.000000in}}%
\pgfpathlineto{\pgfqpoint{-0.048611in}{0.000000in}}%
\pgfusepath{stroke,fill}%
}%
\begin{pgfscope}%
\pgfsys@transformshift{1.000000in}{5.491449in}%
\pgfsys@useobject{currentmarker}{}%
\end{pgfscope}%
\end{pgfscope}%
\begin{pgfscope}%
\definecolor{textcolor}{rgb}{0.000000,0.000000,0.000000}%
\pgfsetstrokecolor{textcolor}%
\pgfsetfillcolor{textcolor}%
\pgftext[x=0.560215in, y=5.391430in, left, base]{\color{textcolor}\sffamily\fontsize{20.000000}{24.000000}\selectfont \(\displaystyle {0.4}\)}%
\end{pgfscope}%
\begin{pgfscope}%
\pgfpathrectangle{\pgfqpoint{1.000000in}{5.000000in}}{\pgfqpoint{8.500000in}{2.000000in}}%
\pgfusepath{clip}%
\pgfsetrectcap%
\pgfsetroundjoin%
\pgfsetlinewidth{0.803000pt}%
\definecolor{currentstroke}{rgb}{0.690196,0.690196,0.690196}%
\pgfsetstrokecolor{currentstroke}%
\pgfsetdash{}{0pt}%
\pgfpathmoveto{\pgfqpoint{1.000000in}{5.737173in}}%
\pgfpathlineto{\pgfqpoint{9.500000in}{5.737173in}}%
\pgfusepath{stroke}%
\end{pgfscope}%
\begin{pgfscope}%
\pgfsetbuttcap%
\pgfsetroundjoin%
\definecolor{currentfill}{rgb}{0.000000,0.000000,0.000000}%
\pgfsetfillcolor{currentfill}%
\pgfsetlinewidth{0.803000pt}%
\definecolor{currentstroke}{rgb}{0.000000,0.000000,0.000000}%
\pgfsetstrokecolor{currentstroke}%
\pgfsetdash{}{0pt}%
\pgfsys@defobject{currentmarker}{\pgfqpoint{-0.048611in}{0.000000in}}{\pgfqpoint{-0.000000in}{0.000000in}}{%
\pgfpathmoveto{\pgfqpoint{-0.000000in}{0.000000in}}%
\pgfpathlineto{\pgfqpoint{-0.048611in}{0.000000in}}%
\pgfusepath{stroke,fill}%
}%
\begin{pgfscope}%
\pgfsys@transformshift{1.000000in}{5.737173in}%
\pgfsys@useobject{currentmarker}{}%
\end{pgfscope}%
\end{pgfscope}%
\begin{pgfscope}%
\definecolor{textcolor}{rgb}{0.000000,0.000000,0.000000}%
\pgfsetstrokecolor{textcolor}%
\pgfsetfillcolor{textcolor}%
\pgftext[x=0.560215in, y=5.637154in, left, base]{\color{textcolor}\sffamily\fontsize{20.000000}{24.000000}\selectfont \(\displaystyle {0.6}\)}%
\end{pgfscope}%
\begin{pgfscope}%
\pgfpathrectangle{\pgfqpoint{1.000000in}{5.000000in}}{\pgfqpoint{8.500000in}{2.000000in}}%
\pgfusepath{clip}%
\pgfsetrectcap%
\pgfsetroundjoin%
\pgfsetlinewidth{0.803000pt}%
\definecolor{currentstroke}{rgb}{0.690196,0.690196,0.690196}%
\pgfsetstrokecolor{currentstroke}%
\pgfsetdash{}{0pt}%
\pgfpathmoveto{\pgfqpoint{1.000000in}{5.982898in}}%
\pgfpathlineto{\pgfqpoint{9.500000in}{5.982898in}}%
\pgfusepath{stroke}%
\end{pgfscope}%
\begin{pgfscope}%
\pgfsetbuttcap%
\pgfsetroundjoin%
\definecolor{currentfill}{rgb}{0.000000,0.000000,0.000000}%
\pgfsetfillcolor{currentfill}%
\pgfsetlinewidth{0.803000pt}%
\definecolor{currentstroke}{rgb}{0.000000,0.000000,0.000000}%
\pgfsetstrokecolor{currentstroke}%
\pgfsetdash{}{0pt}%
\pgfsys@defobject{currentmarker}{\pgfqpoint{-0.048611in}{0.000000in}}{\pgfqpoint{-0.000000in}{0.000000in}}{%
\pgfpathmoveto{\pgfqpoint{-0.000000in}{0.000000in}}%
\pgfpathlineto{\pgfqpoint{-0.048611in}{0.000000in}}%
\pgfusepath{stroke,fill}%
}%
\begin{pgfscope}%
\pgfsys@transformshift{1.000000in}{5.982898in}%
\pgfsys@useobject{currentmarker}{}%
\end{pgfscope}%
\end{pgfscope}%
\begin{pgfscope}%
\definecolor{textcolor}{rgb}{0.000000,0.000000,0.000000}%
\pgfsetstrokecolor{textcolor}%
\pgfsetfillcolor{textcolor}%
\pgftext[x=0.560215in, y=5.882879in, left, base]{\color{textcolor}\sffamily\fontsize{20.000000}{24.000000}\selectfont \(\displaystyle {0.8}\)}%
\end{pgfscope}%
\begin{pgfscope}%
\pgfpathrectangle{\pgfqpoint{1.000000in}{5.000000in}}{\pgfqpoint{8.500000in}{2.000000in}}%
\pgfusepath{clip}%
\pgfsetrectcap%
\pgfsetroundjoin%
\pgfsetlinewidth{0.803000pt}%
\definecolor{currentstroke}{rgb}{0.690196,0.690196,0.690196}%
\pgfsetstrokecolor{currentstroke}%
\pgfsetdash{}{0pt}%
\pgfpathmoveto{\pgfqpoint{1.000000in}{6.228622in}}%
\pgfpathlineto{\pgfqpoint{9.500000in}{6.228622in}}%
\pgfusepath{stroke}%
\end{pgfscope}%
\begin{pgfscope}%
\pgfsetbuttcap%
\pgfsetroundjoin%
\definecolor{currentfill}{rgb}{0.000000,0.000000,0.000000}%
\pgfsetfillcolor{currentfill}%
\pgfsetlinewidth{0.803000pt}%
\definecolor{currentstroke}{rgb}{0.000000,0.000000,0.000000}%
\pgfsetstrokecolor{currentstroke}%
\pgfsetdash{}{0pt}%
\pgfsys@defobject{currentmarker}{\pgfqpoint{-0.048611in}{0.000000in}}{\pgfqpoint{-0.000000in}{0.000000in}}{%
\pgfpathmoveto{\pgfqpoint{-0.000000in}{0.000000in}}%
\pgfpathlineto{\pgfqpoint{-0.048611in}{0.000000in}}%
\pgfusepath{stroke,fill}%
}%
\begin{pgfscope}%
\pgfsys@transformshift{1.000000in}{6.228622in}%
\pgfsys@useobject{currentmarker}{}%
\end{pgfscope}%
\end{pgfscope}%
\begin{pgfscope}%
\definecolor{textcolor}{rgb}{0.000000,0.000000,0.000000}%
\pgfsetstrokecolor{textcolor}%
\pgfsetfillcolor{textcolor}%
\pgftext[x=0.560215in, y=6.128603in, left, base]{\color{textcolor}\sffamily\fontsize{20.000000}{24.000000}\selectfont \(\displaystyle {1.0}\)}%
\end{pgfscope}%
\begin{pgfscope}%
\pgfpathrectangle{\pgfqpoint{1.000000in}{5.000000in}}{\pgfqpoint{8.500000in}{2.000000in}}%
\pgfusepath{clip}%
\pgfsetrectcap%
\pgfsetroundjoin%
\pgfsetlinewidth{0.803000pt}%
\definecolor{currentstroke}{rgb}{0.690196,0.690196,0.690196}%
\pgfsetstrokecolor{currentstroke}%
\pgfsetdash{}{0pt}%
\pgfpathmoveto{\pgfqpoint{1.000000in}{6.474347in}}%
\pgfpathlineto{\pgfqpoint{9.500000in}{6.474347in}}%
\pgfusepath{stroke}%
\end{pgfscope}%
\begin{pgfscope}%
\pgfsetbuttcap%
\pgfsetroundjoin%
\definecolor{currentfill}{rgb}{0.000000,0.000000,0.000000}%
\pgfsetfillcolor{currentfill}%
\pgfsetlinewidth{0.803000pt}%
\definecolor{currentstroke}{rgb}{0.000000,0.000000,0.000000}%
\pgfsetstrokecolor{currentstroke}%
\pgfsetdash{}{0pt}%
\pgfsys@defobject{currentmarker}{\pgfqpoint{-0.048611in}{0.000000in}}{\pgfqpoint{-0.000000in}{0.000000in}}{%
\pgfpathmoveto{\pgfqpoint{-0.000000in}{0.000000in}}%
\pgfpathlineto{\pgfqpoint{-0.048611in}{0.000000in}}%
\pgfusepath{stroke,fill}%
}%
\begin{pgfscope}%
\pgfsys@transformshift{1.000000in}{6.474347in}%
\pgfsys@useobject{currentmarker}{}%
\end{pgfscope}%
\end{pgfscope}%
\begin{pgfscope}%
\definecolor{textcolor}{rgb}{0.000000,0.000000,0.000000}%
\pgfsetstrokecolor{textcolor}%
\pgfsetfillcolor{textcolor}%
\pgftext[x=0.560215in, y=6.374327in, left, base]{\color{textcolor}\sffamily\fontsize{20.000000}{24.000000}\selectfont \(\displaystyle {1.2}\)}%
\end{pgfscope}%
\begin{pgfscope}%
\pgfpathrectangle{\pgfqpoint{1.000000in}{5.000000in}}{\pgfqpoint{8.500000in}{2.000000in}}%
\pgfusepath{clip}%
\pgfsetrectcap%
\pgfsetroundjoin%
\pgfsetlinewidth{0.803000pt}%
\definecolor{currentstroke}{rgb}{0.690196,0.690196,0.690196}%
\pgfsetstrokecolor{currentstroke}%
\pgfsetdash{}{0pt}%
\pgfpathmoveto{\pgfqpoint{1.000000in}{6.720071in}}%
\pgfpathlineto{\pgfqpoint{9.500000in}{6.720071in}}%
\pgfusepath{stroke}%
\end{pgfscope}%
\begin{pgfscope}%
\pgfsetbuttcap%
\pgfsetroundjoin%
\definecolor{currentfill}{rgb}{0.000000,0.000000,0.000000}%
\pgfsetfillcolor{currentfill}%
\pgfsetlinewidth{0.803000pt}%
\definecolor{currentstroke}{rgb}{0.000000,0.000000,0.000000}%
\pgfsetstrokecolor{currentstroke}%
\pgfsetdash{}{0pt}%
\pgfsys@defobject{currentmarker}{\pgfqpoint{-0.048611in}{0.000000in}}{\pgfqpoint{-0.000000in}{0.000000in}}{%
\pgfpathmoveto{\pgfqpoint{-0.000000in}{0.000000in}}%
\pgfpathlineto{\pgfqpoint{-0.048611in}{0.000000in}}%
\pgfusepath{stroke,fill}%
}%
\begin{pgfscope}%
\pgfsys@transformshift{1.000000in}{6.720071in}%
\pgfsys@useobject{currentmarker}{}%
\end{pgfscope}%
\end{pgfscope}%
\begin{pgfscope}%
\definecolor{textcolor}{rgb}{0.000000,0.000000,0.000000}%
\pgfsetstrokecolor{textcolor}%
\pgfsetfillcolor{textcolor}%
\pgftext[x=0.560215in, y=6.620052in, left, base]{\color{textcolor}\sffamily\fontsize{20.000000}{24.000000}\selectfont \(\displaystyle {1.4}\)}%
\end{pgfscope}%
\begin{pgfscope}%
\definecolor{textcolor}{rgb}{0.000000,0.000000,0.000000}%
\pgfsetstrokecolor{textcolor}%
\pgfsetfillcolor{textcolor}%
\pgftext[x=0.504660in,y=6.000000in,,bottom,rotate=90.000000]{\color{textcolor}\sffamily\fontsize{20.000000}{24.000000}\selectfont \(\displaystyle \mathrm{Charge}\)}%
\end{pgfscope}%
\begin{pgfscope}%
\pgfpathrectangle{\pgfqpoint{1.000000in}{5.000000in}}{\pgfqpoint{8.500000in}{2.000000in}}%
\pgfusepath{clip}%
\pgfsetbuttcap%
\pgfsetroundjoin%
\pgfsetlinewidth{2.007500pt}%
\definecolor{currentstroke}{rgb}{0.000000,0.000000,0.000000}%
\pgfsetstrokecolor{currentstroke}%
\pgfsetdash{}{0pt}%
\pgfpathmoveto{\pgfqpoint{2.781081in}{5.000000in}}%
\pgfpathlineto{\pgfqpoint{2.781081in}{6.433962in}}%
\pgfusepath{stroke}%
\end{pgfscope}%
\begin{pgfscope}%
\pgfpathrectangle{\pgfqpoint{1.000000in}{5.000000in}}{\pgfqpoint{8.500000in}{2.000000in}}%
\pgfusepath{clip}%
\pgfsetbuttcap%
\pgfsetroundjoin%
\pgfsetlinewidth{2.007500pt}%
\definecolor{currentstroke}{rgb}{0.000000,0.000000,0.000000}%
\pgfsetstrokecolor{currentstroke}%
\pgfsetdash{}{0pt}%
\pgfpathmoveto{\pgfqpoint{3.072831in}{5.000000in}}%
\pgfpathlineto{\pgfqpoint{3.072831in}{6.818182in}}%
\pgfusepath{stroke}%
\end{pgfscope}%
\begin{pgfscope}%
\pgfpathrectangle{\pgfqpoint{1.000000in}{5.000000in}}{\pgfqpoint{8.500000in}{2.000000in}}%
\pgfusepath{clip}%
\pgfsetbuttcap%
\pgfsetroundjoin%
\pgfsetlinewidth{2.007500pt}%
\definecolor{currentstroke}{rgb}{0.000000,0.000000,0.000000}%
\pgfsetstrokecolor{currentstroke}%
\pgfsetdash{}{0pt}%
\pgfpathmoveto{\pgfqpoint{3.407148in}{5.000000in}}%
\pgfpathlineto{\pgfqpoint{3.407148in}{6.480365in}}%
\pgfusepath{stroke}%
\end{pgfscope}%
\begin{pgfscope}%
\pgfpathrectangle{\pgfqpoint{1.000000in}{5.000000in}}{\pgfqpoint{8.500000in}{2.000000in}}%
\pgfusepath{clip}%
\pgfsetbuttcap%
\pgfsetroundjoin%
\pgfsetlinewidth{2.007500pt}%
\definecolor{currentstroke}{rgb}{0.000000,0.000000,0.000000}%
\pgfsetstrokecolor{currentstroke}%
\pgfsetdash{}{0pt}%
\pgfpathmoveto{\pgfqpoint{3.567899in}{5.000000in}}%
\pgfpathlineto{\pgfqpoint{3.567899in}{6.158068in}}%
\pgfusepath{stroke}%
\end{pgfscope}%
\begin{pgfscope}%
\pgfpathrectangle{\pgfqpoint{1.000000in}{5.000000in}}{\pgfqpoint{8.500000in}{2.000000in}}%
\pgfusepath{clip}%
\pgfsetbuttcap%
\pgfsetroundjoin%
\pgfsetlinewidth{2.007500pt}%
\definecolor{currentstroke}{rgb}{0.000000,0.000000,0.000000}%
\pgfsetstrokecolor{currentstroke}%
\pgfsetdash{}{0pt}%
\pgfpathmoveto{\pgfqpoint{4.156756in}{5.000000in}}%
\pgfpathlineto{\pgfqpoint{4.156756in}{6.529536in}}%
\pgfusepath{stroke}%
\end{pgfscope}%
\begin{pgfscope}%
\pgfsetrectcap%
\pgfsetmiterjoin%
\pgfsetlinewidth{0.803000pt}%
\definecolor{currentstroke}{rgb}{0.000000,0.000000,0.000000}%
\pgfsetstrokecolor{currentstroke}%
\pgfsetdash{}{0pt}%
\pgfpathmoveto{\pgfqpoint{1.000000in}{5.000000in}}%
\pgfpathlineto{\pgfqpoint{1.000000in}{7.000000in}}%
\pgfusepath{stroke}%
\end{pgfscope}%
\begin{pgfscope}%
\pgfsetrectcap%
\pgfsetmiterjoin%
\pgfsetlinewidth{0.803000pt}%
\definecolor{currentstroke}{rgb}{0.000000,0.000000,0.000000}%
\pgfsetstrokecolor{currentstroke}%
\pgfsetdash{}{0pt}%
\pgfpathmoveto{\pgfqpoint{9.500000in}{5.000000in}}%
\pgfpathlineto{\pgfqpoint{9.500000in}{7.000000in}}%
\pgfusepath{stroke}%
\end{pgfscope}%
\begin{pgfscope}%
\pgfsetrectcap%
\pgfsetmiterjoin%
\pgfsetlinewidth{0.803000pt}%
\definecolor{currentstroke}{rgb}{0.000000,0.000000,0.000000}%
\pgfsetstrokecolor{currentstroke}%
\pgfsetdash{}{0pt}%
\pgfpathmoveto{\pgfqpoint{1.000000in}{5.000000in}}%
\pgfpathlineto{\pgfqpoint{9.500000in}{5.000000in}}%
\pgfusepath{stroke}%
\end{pgfscope}%
\begin{pgfscope}%
\pgfsetrectcap%
\pgfsetmiterjoin%
\pgfsetlinewidth{0.803000pt}%
\definecolor{currentstroke}{rgb}{0.000000,0.000000,0.000000}%
\pgfsetstrokecolor{currentstroke}%
\pgfsetdash{}{0pt}%
\pgfpathmoveto{\pgfqpoint{1.000000in}{7.000000in}}%
\pgfpathlineto{\pgfqpoint{9.500000in}{7.000000in}}%
\pgfusepath{stroke}%
\end{pgfscope}%
\begin{pgfscope}%
\pgfsetbuttcap%
\pgfsetmiterjoin%
\definecolor{currentfill}{rgb}{1.000000,1.000000,1.000000}%
\pgfsetfillcolor{currentfill}%
\pgfsetfillopacity{0.800000}%
\pgfsetlinewidth{1.003750pt}%
\definecolor{currentstroke}{rgb}{0.800000,0.800000,0.800000}%
\pgfsetstrokecolor{currentstroke}%
\pgfsetstrokeopacity{0.800000}%
\pgfsetdash{}{0pt}%
\pgfpathmoveto{\pgfqpoint{6.976817in}{6.382821in}}%
\pgfpathlineto{\pgfqpoint{9.305556in}{6.382821in}}%
\pgfpathquadraticcurveto{\pgfqpoint{9.361111in}{6.382821in}}{\pgfqpoint{9.361111in}{6.438377in}}%
\pgfpathlineto{\pgfqpoint{9.361111in}{6.805556in}}%
\pgfpathquadraticcurveto{\pgfqpoint{9.361111in}{6.861111in}}{\pgfqpoint{9.305556in}{6.861111in}}%
\pgfpathlineto{\pgfqpoint{6.976817in}{6.861111in}}%
\pgfpathquadraticcurveto{\pgfqpoint{6.921262in}{6.861111in}}{\pgfqpoint{6.921262in}{6.805556in}}%
\pgfpathlineto{\pgfqpoint{6.921262in}{6.438377in}}%
\pgfpathquadraticcurveto{\pgfqpoint{6.921262in}{6.382821in}}{\pgfqpoint{6.976817in}{6.382821in}}%
\pgfpathclose%
\pgfusepath{stroke,fill}%
\end{pgfscope}%
\begin{pgfscope}%
\pgfsetbuttcap%
\pgfsetroundjoin%
\pgfsetlinewidth{2.007500pt}%
\definecolor{currentstroke}{rgb}{0.000000,0.000000,0.000000}%
\pgfsetstrokecolor{currentstroke}%
\pgfsetdash{}{0pt}%
\pgfpathmoveto{\pgfqpoint{7.032373in}{6.647184in}}%
\pgfpathlineto{\pgfqpoint{7.587928in}{6.647184in}}%
\pgfusepath{stroke}%
\end{pgfscope}%
\begin{pgfscope}%
\definecolor{textcolor}{rgb}{0.000000,0.000000,0.000000}%
\pgfsetstrokecolor{textcolor}%
\pgfsetfillcolor{textcolor}%
\pgftext[x=7.810150in,y=6.549962in,left,base]{\color{textcolor}\sffamily\fontsize{20.000000}{24.000000}\selectfont truth Charge}%
\end{pgfscope}%
\begin{pgfscope}%
\pgfsetbuttcap%
\pgfsetmiterjoin%
\definecolor{currentfill}{rgb}{1.000000,1.000000,1.000000}%
\pgfsetfillcolor{currentfill}%
\pgfsetlinewidth{0.000000pt}%
\definecolor{currentstroke}{rgb}{0.000000,0.000000,0.000000}%
\pgfsetstrokecolor{currentstroke}%
\pgfsetstrokeopacity{0.000000}%
\pgfsetdash{}{0pt}%
\pgfpathmoveto{\pgfqpoint{1.000000in}{7.000000in}}%
\pgfpathlineto{\pgfqpoint{9.500000in}{7.000000in}}%
\pgfpathlineto{\pgfqpoint{9.500000in}{9.000000in}}%
\pgfpathlineto{\pgfqpoint{1.000000in}{9.000000in}}%
\pgfpathclose%
\pgfusepath{fill}%
\end{pgfscope}%
\begin{pgfscope}%
\pgfpathrectangle{\pgfqpoint{1.000000in}{7.000000in}}{\pgfqpoint{8.500000in}{2.000000in}}%
\pgfusepath{clip}%
\pgfsetrectcap%
\pgfsetroundjoin%
\pgfsetlinewidth{0.803000pt}%
\definecolor{currentstroke}{rgb}{0.690196,0.690196,0.690196}%
\pgfsetstrokecolor{currentstroke}%
\pgfsetdash{}{0pt}%
\pgfpathmoveto{\pgfqpoint{2.725210in}{7.000000in}}%
\pgfpathlineto{\pgfqpoint{2.725210in}{9.000000in}}%
\pgfusepath{stroke}%
\end{pgfscope}%
\begin{pgfscope}%
\pgfsetbuttcap%
\pgfsetroundjoin%
\definecolor{currentfill}{rgb}{0.000000,0.000000,0.000000}%
\pgfsetfillcolor{currentfill}%
\pgfsetlinewidth{0.803000pt}%
\definecolor{currentstroke}{rgb}{0.000000,0.000000,0.000000}%
\pgfsetstrokecolor{currentstroke}%
\pgfsetdash{}{0pt}%
\pgfsys@defobject{currentmarker}{\pgfqpoint{0.000000in}{-0.048611in}}{\pgfqpoint{0.000000in}{0.000000in}}{%
\pgfpathmoveto{\pgfqpoint{0.000000in}{0.000000in}}%
\pgfpathlineto{\pgfqpoint{0.000000in}{-0.048611in}}%
\pgfusepath{stroke,fill}%
}%
\begin{pgfscope}%
\pgfsys@transformshift{2.725210in}{7.000000in}%
\pgfsys@useobject{currentmarker}{}%
\end{pgfscope}%
\end{pgfscope}%
\begin{pgfscope}%
\pgfpathrectangle{\pgfqpoint{1.000000in}{7.000000in}}{\pgfqpoint{8.500000in}{2.000000in}}%
\pgfusepath{clip}%
\pgfsetrectcap%
\pgfsetroundjoin%
\pgfsetlinewidth{0.803000pt}%
\definecolor{currentstroke}{rgb}{0.690196,0.690196,0.690196}%
\pgfsetstrokecolor{currentstroke}%
\pgfsetdash{}{0pt}%
\pgfpathmoveto{\pgfqpoint{4.506292in}{7.000000in}}%
\pgfpathlineto{\pgfqpoint{4.506292in}{9.000000in}}%
\pgfusepath{stroke}%
\end{pgfscope}%
\begin{pgfscope}%
\pgfsetbuttcap%
\pgfsetroundjoin%
\definecolor{currentfill}{rgb}{0.000000,0.000000,0.000000}%
\pgfsetfillcolor{currentfill}%
\pgfsetlinewidth{0.803000pt}%
\definecolor{currentstroke}{rgb}{0.000000,0.000000,0.000000}%
\pgfsetstrokecolor{currentstroke}%
\pgfsetdash{}{0pt}%
\pgfsys@defobject{currentmarker}{\pgfqpoint{0.000000in}{-0.048611in}}{\pgfqpoint{0.000000in}{0.000000in}}{%
\pgfpathmoveto{\pgfqpoint{0.000000in}{0.000000in}}%
\pgfpathlineto{\pgfqpoint{0.000000in}{-0.048611in}}%
\pgfusepath{stroke,fill}%
}%
\begin{pgfscope}%
\pgfsys@transformshift{4.506292in}{7.000000in}%
\pgfsys@useobject{currentmarker}{}%
\end{pgfscope}%
\end{pgfscope}%
\begin{pgfscope}%
\pgfpathrectangle{\pgfqpoint{1.000000in}{7.000000in}}{\pgfqpoint{8.500000in}{2.000000in}}%
\pgfusepath{clip}%
\pgfsetrectcap%
\pgfsetroundjoin%
\pgfsetlinewidth{0.803000pt}%
\definecolor{currentstroke}{rgb}{0.690196,0.690196,0.690196}%
\pgfsetstrokecolor{currentstroke}%
\pgfsetdash{}{0pt}%
\pgfpathmoveto{\pgfqpoint{6.287373in}{7.000000in}}%
\pgfpathlineto{\pgfqpoint{6.287373in}{9.000000in}}%
\pgfusepath{stroke}%
\end{pgfscope}%
\begin{pgfscope}%
\pgfsetbuttcap%
\pgfsetroundjoin%
\definecolor{currentfill}{rgb}{0.000000,0.000000,0.000000}%
\pgfsetfillcolor{currentfill}%
\pgfsetlinewidth{0.803000pt}%
\definecolor{currentstroke}{rgb}{0.000000,0.000000,0.000000}%
\pgfsetstrokecolor{currentstroke}%
\pgfsetdash{}{0pt}%
\pgfsys@defobject{currentmarker}{\pgfqpoint{0.000000in}{-0.048611in}}{\pgfqpoint{0.000000in}{0.000000in}}{%
\pgfpathmoveto{\pgfqpoint{0.000000in}{0.000000in}}%
\pgfpathlineto{\pgfqpoint{0.000000in}{-0.048611in}}%
\pgfusepath{stroke,fill}%
}%
\begin{pgfscope}%
\pgfsys@transformshift{6.287373in}{7.000000in}%
\pgfsys@useobject{currentmarker}{}%
\end{pgfscope}%
\end{pgfscope}%
\begin{pgfscope}%
\pgfpathrectangle{\pgfqpoint{1.000000in}{7.000000in}}{\pgfqpoint{8.500000in}{2.000000in}}%
\pgfusepath{clip}%
\pgfsetrectcap%
\pgfsetroundjoin%
\pgfsetlinewidth{0.803000pt}%
\definecolor{currentstroke}{rgb}{0.690196,0.690196,0.690196}%
\pgfsetstrokecolor{currentstroke}%
\pgfsetdash{}{0pt}%
\pgfpathmoveto{\pgfqpoint{8.068455in}{7.000000in}}%
\pgfpathlineto{\pgfqpoint{8.068455in}{9.000000in}}%
\pgfusepath{stroke}%
\end{pgfscope}%
\begin{pgfscope}%
\pgfsetbuttcap%
\pgfsetroundjoin%
\definecolor{currentfill}{rgb}{0.000000,0.000000,0.000000}%
\pgfsetfillcolor{currentfill}%
\pgfsetlinewidth{0.803000pt}%
\definecolor{currentstroke}{rgb}{0.000000,0.000000,0.000000}%
\pgfsetstrokecolor{currentstroke}%
\pgfsetdash{}{0pt}%
\pgfsys@defobject{currentmarker}{\pgfqpoint{0.000000in}{-0.048611in}}{\pgfqpoint{0.000000in}{0.000000in}}{%
\pgfpathmoveto{\pgfqpoint{0.000000in}{0.000000in}}%
\pgfpathlineto{\pgfqpoint{0.000000in}{-0.048611in}}%
\pgfusepath{stroke,fill}%
}%
\begin{pgfscope}%
\pgfsys@transformshift{8.068455in}{7.000000in}%
\pgfsys@useobject{currentmarker}{}%
\end{pgfscope}%
\end{pgfscope}%
\begin{pgfscope}%
\pgfpathrectangle{\pgfqpoint{1.000000in}{7.000000in}}{\pgfqpoint{8.500000in}{2.000000in}}%
\pgfusepath{clip}%
\pgfsetrectcap%
\pgfsetroundjoin%
\pgfsetlinewidth{0.803000pt}%
\definecolor{currentstroke}{rgb}{0.690196,0.690196,0.690196}%
\pgfsetstrokecolor{currentstroke}%
\pgfsetdash{}{0pt}%
\pgfpathmoveto{\pgfqpoint{1.000000in}{7.000000in}}%
\pgfpathlineto{\pgfqpoint{9.500000in}{7.000000in}}%
\pgfusepath{stroke}%
\end{pgfscope}%
\begin{pgfscope}%
\pgfsetbuttcap%
\pgfsetroundjoin%
\definecolor{currentfill}{rgb}{0.000000,0.000000,0.000000}%
\pgfsetfillcolor{currentfill}%
\pgfsetlinewidth{0.803000pt}%
\definecolor{currentstroke}{rgb}{0.000000,0.000000,0.000000}%
\pgfsetstrokecolor{currentstroke}%
\pgfsetdash{}{0pt}%
\pgfsys@defobject{currentmarker}{\pgfqpoint{-0.048611in}{0.000000in}}{\pgfqpoint{-0.000000in}{0.000000in}}{%
\pgfpathmoveto{\pgfqpoint{-0.000000in}{0.000000in}}%
\pgfpathlineto{\pgfqpoint{-0.048611in}{0.000000in}}%
\pgfusepath{stroke,fill}%
}%
\begin{pgfscope}%
\pgfsys@transformshift{1.000000in}{7.000000in}%
\pgfsys@useobject{currentmarker}{}%
\end{pgfscope}%
\end{pgfscope}%
\begin{pgfscope}%
\definecolor{textcolor}{rgb}{0.000000,0.000000,0.000000}%
\pgfsetstrokecolor{textcolor}%
\pgfsetfillcolor{textcolor}%
\pgftext[x=0.560215in, y=6.899981in, left, base]{\color{textcolor}\sffamily\fontsize{20.000000}{24.000000}\selectfont \(\displaystyle {0.0}\)}%
\end{pgfscope}%
\begin{pgfscope}%
\pgfpathrectangle{\pgfqpoint{1.000000in}{7.000000in}}{\pgfqpoint{8.500000in}{2.000000in}}%
\pgfusepath{clip}%
\pgfsetrectcap%
\pgfsetroundjoin%
\pgfsetlinewidth{0.803000pt}%
\definecolor{currentstroke}{rgb}{0.690196,0.690196,0.690196}%
\pgfsetstrokecolor{currentstroke}%
\pgfsetdash{}{0pt}%
\pgfpathmoveto{\pgfqpoint{1.000000in}{7.245724in}}%
\pgfpathlineto{\pgfqpoint{9.500000in}{7.245724in}}%
\pgfusepath{stroke}%
\end{pgfscope}%
\begin{pgfscope}%
\pgfsetbuttcap%
\pgfsetroundjoin%
\definecolor{currentfill}{rgb}{0.000000,0.000000,0.000000}%
\pgfsetfillcolor{currentfill}%
\pgfsetlinewidth{0.803000pt}%
\definecolor{currentstroke}{rgb}{0.000000,0.000000,0.000000}%
\pgfsetstrokecolor{currentstroke}%
\pgfsetdash{}{0pt}%
\pgfsys@defobject{currentmarker}{\pgfqpoint{-0.048611in}{0.000000in}}{\pgfqpoint{-0.000000in}{0.000000in}}{%
\pgfpathmoveto{\pgfqpoint{-0.000000in}{0.000000in}}%
\pgfpathlineto{\pgfqpoint{-0.048611in}{0.000000in}}%
\pgfusepath{stroke,fill}%
}%
\begin{pgfscope}%
\pgfsys@transformshift{1.000000in}{7.245724in}%
\pgfsys@useobject{currentmarker}{}%
\end{pgfscope}%
\end{pgfscope}%
\begin{pgfscope}%
\definecolor{textcolor}{rgb}{0.000000,0.000000,0.000000}%
\pgfsetstrokecolor{textcolor}%
\pgfsetfillcolor{textcolor}%
\pgftext[x=0.560215in, y=7.145705in, left, base]{\color{textcolor}\sffamily\fontsize{20.000000}{24.000000}\selectfont \(\displaystyle {0.2}\)}%
\end{pgfscope}%
\begin{pgfscope}%
\pgfpathrectangle{\pgfqpoint{1.000000in}{7.000000in}}{\pgfqpoint{8.500000in}{2.000000in}}%
\pgfusepath{clip}%
\pgfsetrectcap%
\pgfsetroundjoin%
\pgfsetlinewidth{0.803000pt}%
\definecolor{currentstroke}{rgb}{0.690196,0.690196,0.690196}%
\pgfsetstrokecolor{currentstroke}%
\pgfsetdash{}{0pt}%
\pgfpathmoveto{\pgfqpoint{1.000000in}{7.491449in}}%
\pgfpathlineto{\pgfqpoint{9.500000in}{7.491449in}}%
\pgfusepath{stroke}%
\end{pgfscope}%
\begin{pgfscope}%
\pgfsetbuttcap%
\pgfsetroundjoin%
\definecolor{currentfill}{rgb}{0.000000,0.000000,0.000000}%
\pgfsetfillcolor{currentfill}%
\pgfsetlinewidth{0.803000pt}%
\definecolor{currentstroke}{rgb}{0.000000,0.000000,0.000000}%
\pgfsetstrokecolor{currentstroke}%
\pgfsetdash{}{0pt}%
\pgfsys@defobject{currentmarker}{\pgfqpoint{-0.048611in}{0.000000in}}{\pgfqpoint{-0.000000in}{0.000000in}}{%
\pgfpathmoveto{\pgfqpoint{-0.000000in}{0.000000in}}%
\pgfpathlineto{\pgfqpoint{-0.048611in}{0.000000in}}%
\pgfusepath{stroke,fill}%
}%
\begin{pgfscope}%
\pgfsys@transformshift{1.000000in}{7.491449in}%
\pgfsys@useobject{currentmarker}{}%
\end{pgfscope}%
\end{pgfscope}%
\begin{pgfscope}%
\definecolor{textcolor}{rgb}{0.000000,0.000000,0.000000}%
\pgfsetstrokecolor{textcolor}%
\pgfsetfillcolor{textcolor}%
\pgftext[x=0.560215in, y=7.391430in, left, base]{\color{textcolor}\sffamily\fontsize{20.000000}{24.000000}\selectfont \(\displaystyle {0.4}\)}%
\end{pgfscope}%
\begin{pgfscope}%
\pgfpathrectangle{\pgfqpoint{1.000000in}{7.000000in}}{\pgfqpoint{8.500000in}{2.000000in}}%
\pgfusepath{clip}%
\pgfsetrectcap%
\pgfsetroundjoin%
\pgfsetlinewidth{0.803000pt}%
\definecolor{currentstroke}{rgb}{0.690196,0.690196,0.690196}%
\pgfsetstrokecolor{currentstroke}%
\pgfsetdash{}{0pt}%
\pgfpathmoveto{\pgfqpoint{1.000000in}{7.737173in}}%
\pgfpathlineto{\pgfqpoint{9.500000in}{7.737173in}}%
\pgfusepath{stroke}%
\end{pgfscope}%
\begin{pgfscope}%
\pgfsetbuttcap%
\pgfsetroundjoin%
\definecolor{currentfill}{rgb}{0.000000,0.000000,0.000000}%
\pgfsetfillcolor{currentfill}%
\pgfsetlinewidth{0.803000pt}%
\definecolor{currentstroke}{rgb}{0.000000,0.000000,0.000000}%
\pgfsetstrokecolor{currentstroke}%
\pgfsetdash{}{0pt}%
\pgfsys@defobject{currentmarker}{\pgfqpoint{-0.048611in}{0.000000in}}{\pgfqpoint{-0.000000in}{0.000000in}}{%
\pgfpathmoveto{\pgfqpoint{-0.000000in}{0.000000in}}%
\pgfpathlineto{\pgfqpoint{-0.048611in}{0.000000in}}%
\pgfusepath{stroke,fill}%
}%
\begin{pgfscope}%
\pgfsys@transformshift{1.000000in}{7.737173in}%
\pgfsys@useobject{currentmarker}{}%
\end{pgfscope}%
\end{pgfscope}%
\begin{pgfscope}%
\definecolor{textcolor}{rgb}{0.000000,0.000000,0.000000}%
\pgfsetstrokecolor{textcolor}%
\pgfsetfillcolor{textcolor}%
\pgftext[x=0.560215in, y=7.637154in, left, base]{\color{textcolor}\sffamily\fontsize{20.000000}{24.000000}\selectfont \(\displaystyle {0.6}\)}%
\end{pgfscope}%
\begin{pgfscope}%
\pgfpathrectangle{\pgfqpoint{1.000000in}{7.000000in}}{\pgfqpoint{8.500000in}{2.000000in}}%
\pgfusepath{clip}%
\pgfsetrectcap%
\pgfsetroundjoin%
\pgfsetlinewidth{0.803000pt}%
\definecolor{currentstroke}{rgb}{0.690196,0.690196,0.690196}%
\pgfsetstrokecolor{currentstroke}%
\pgfsetdash{}{0pt}%
\pgfpathmoveto{\pgfqpoint{1.000000in}{7.982898in}}%
\pgfpathlineto{\pgfqpoint{9.500000in}{7.982898in}}%
\pgfusepath{stroke}%
\end{pgfscope}%
\begin{pgfscope}%
\pgfsetbuttcap%
\pgfsetroundjoin%
\definecolor{currentfill}{rgb}{0.000000,0.000000,0.000000}%
\pgfsetfillcolor{currentfill}%
\pgfsetlinewidth{0.803000pt}%
\definecolor{currentstroke}{rgb}{0.000000,0.000000,0.000000}%
\pgfsetstrokecolor{currentstroke}%
\pgfsetdash{}{0pt}%
\pgfsys@defobject{currentmarker}{\pgfqpoint{-0.048611in}{0.000000in}}{\pgfqpoint{-0.000000in}{0.000000in}}{%
\pgfpathmoveto{\pgfqpoint{-0.000000in}{0.000000in}}%
\pgfpathlineto{\pgfqpoint{-0.048611in}{0.000000in}}%
\pgfusepath{stroke,fill}%
}%
\begin{pgfscope}%
\pgfsys@transformshift{1.000000in}{7.982898in}%
\pgfsys@useobject{currentmarker}{}%
\end{pgfscope}%
\end{pgfscope}%
\begin{pgfscope}%
\definecolor{textcolor}{rgb}{0.000000,0.000000,0.000000}%
\pgfsetstrokecolor{textcolor}%
\pgfsetfillcolor{textcolor}%
\pgftext[x=0.560215in, y=7.882879in, left, base]{\color{textcolor}\sffamily\fontsize{20.000000}{24.000000}\selectfont \(\displaystyle {0.8}\)}%
\end{pgfscope}%
\begin{pgfscope}%
\pgfpathrectangle{\pgfqpoint{1.000000in}{7.000000in}}{\pgfqpoint{8.500000in}{2.000000in}}%
\pgfusepath{clip}%
\pgfsetrectcap%
\pgfsetroundjoin%
\pgfsetlinewidth{0.803000pt}%
\definecolor{currentstroke}{rgb}{0.690196,0.690196,0.690196}%
\pgfsetstrokecolor{currentstroke}%
\pgfsetdash{}{0pt}%
\pgfpathmoveto{\pgfqpoint{1.000000in}{8.228622in}}%
\pgfpathlineto{\pgfqpoint{9.500000in}{8.228622in}}%
\pgfusepath{stroke}%
\end{pgfscope}%
\begin{pgfscope}%
\pgfsetbuttcap%
\pgfsetroundjoin%
\definecolor{currentfill}{rgb}{0.000000,0.000000,0.000000}%
\pgfsetfillcolor{currentfill}%
\pgfsetlinewidth{0.803000pt}%
\definecolor{currentstroke}{rgb}{0.000000,0.000000,0.000000}%
\pgfsetstrokecolor{currentstroke}%
\pgfsetdash{}{0pt}%
\pgfsys@defobject{currentmarker}{\pgfqpoint{-0.048611in}{0.000000in}}{\pgfqpoint{-0.000000in}{0.000000in}}{%
\pgfpathmoveto{\pgfqpoint{-0.000000in}{0.000000in}}%
\pgfpathlineto{\pgfqpoint{-0.048611in}{0.000000in}}%
\pgfusepath{stroke,fill}%
}%
\begin{pgfscope}%
\pgfsys@transformshift{1.000000in}{8.228622in}%
\pgfsys@useobject{currentmarker}{}%
\end{pgfscope}%
\end{pgfscope}%
\begin{pgfscope}%
\definecolor{textcolor}{rgb}{0.000000,0.000000,0.000000}%
\pgfsetstrokecolor{textcolor}%
\pgfsetfillcolor{textcolor}%
\pgftext[x=0.560215in, y=8.128603in, left, base]{\color{textcolor}\sffamily\fontsize{20.000000}{24.000000}\selectfont \(\displaystyle {1.0}\)}%
\end{pgfscope}%
\begin{pgfscope}%
\pgfpathrectangle{\pgfqpoint{1.000000in}{7.000000in}}{\pgfqpoint{8.500000in}{2.000000in}}%
\pgfusepath{clip}%
\pgfsetrectcap%
\pgfsetroundjoin%
\pgfsetlinewidth{0.803000pt}%
\definecolor{currentstroke}{rgb}{0.690196,0.690196,0.690196}%
\pgfsetstrokecolor{currentstroke}%
\pgfsetdash{}{0pt}%
\pgfpathmoveto{\pgfqpoint{1.000000in}{8.474347in}}%
\pgfpathlineto{\pgfqpoint{9.500000in}{8.474347in}}%
\pgfusepath{stroke}%
\end{pgfscope}%
\begin{pgfscope}%
\pgfsetbuttcap%
\pgfsetroundjoin%
\definecolor{currentfill}{rgb}{0.000000,0.000000,0.000000}%
\pgfsetfillcolor{currentfill}%
\pgfsetlinewidth{0.803000pt}%
\definecolor{currentstroke}{rgb}{0.000000,0.000000,0.000000}%
\pgfsetstrokecolor{currentstroke}%
\pgfsetdash{}{0pt}%
\pgfsys@defobject{currentmarker}{\pgfqpoint{-0.048611in}{0.000000in}}{\pgfqpoint{-0.000000in}{0.000000in}}{%
\pgfpathmoveto{\pgfqpoint{-0.000000in}{0.000000in}}%
\pgfpathlineto{\pgfqpoint{-0.048611in}{0.000000in}}%
\pgfusepath{stroke,fill}%
}%
\begin{pgfscope}%
\pgfsys@transformshift{1.000000in}{8.474347in}%
\pgfsys@useobject{currentmarker}{}%
\end{pgfscope}%
\end{pgfscope}%
\begin{pgfscope}%
\definecolor{textcolor}{rgb}{0.000000,0.000000,0.000000}%
\pgfsetstrokecolor{textcolor}%
\pgfsetfillcolor{textcolor}%
\pgftext[x=0.560215in, y=8.374327in, left, base]{\color{textcolor}\sffamily\fontsize{20.000000}{24.000000}\selectfont \(\displaystyle {1.2}\)}%
\end{pgfscope}%
\begin{pgfscope}%
\pgfpathrectangle{\pgfqpoint{1.000000in}{7.000000in}}{\pgfqpoint{8.500000in}{2.000000in}}%
\pgfusepath{clip}%
\pgfsetrectcap%
\pgfsetroundjoin%
\pgfsetlinewidth{0.803000pt}%
\definecolor{currentstroke}{rgb}{0.690196,0.690196,0.690196}%
\pgfsetstrokecolor{currentstroke}%
\pgfsetdash{}{0pt}%
\pgfpathmoveto{\pgfqpoint{1.000000in}{8.720071in}}%
\pgfpathlineto{\pgfqpoint{9.500000in}{8.720071in}}%
\pgfusepath{stroke}%
\end{pgfscope}%
\begin{pgfscope}%
\pgfsetbuttcap%
\pgfsetroundjoin%
\definecolor{currentfill}{rgb}{0.000000,0.000000,0.000000}%
\pgfsetfillcolor{currentfill}%
\pgfsetlinewidth{0.803000pt}%
\definecolor{currentstroke}{rgb}{0.000000,0.000000,0.000000}%
\pgfsetstrokecolor{currentstroke}%
\pgfsetdash{}{0pt}%
\pgfsys@defobject{currentmarker}{\pgfqpoint{-0.048611in}{0.000000in}}{\pgfqpoint{-0.000000in}{0.000000in}}{%
\pgfpathmoveto{\pgfqpoint{-0.000000in}{0.000000in}}%
\pgfpathlineto{\pgfqpoint{-0.048611in}{0.000000in}}%
\pgfusepath{stroke,fill}%
}%
\begin{pgfscope}%
\pgfsys@transformshift{1.000000in}{8.720071in}%
\pgfsys@useobject{currentmarker}{}%
\end{pgfscope}%
\end{pgfscope}%
\begin{pgfscope}%
\definecolor{textcolor}{rgb}{0.000000,0.000000,0.000000}%
\pgfsetstrokecolor{textcolor}%
\pgfsetfillcolor{textcolor}%
\pgftext[x=0.560215in, y=8.620052in, left, base]{\color{textcolor}\sffamily\fontsize{20.000000}{24.000000}\selectfont \(\displaystyle {1.4}\)}%
\end{pgfscope}%
\begin{pgfscope}%
\definecolor{textcolor}{rgb}{0.000000,0.000000,0.000000}%
\pgfsetstrokecolor{textcolor}%
\pgfsetfillcolor{textcolor}%
\pgftext[x=0.504660in,y=8.000000in,,bottom,rotate=90.000000]{\color{textcolor}\sffamily\fontsize{20.000000}{24.000000}\selectfont \(\displaystyle \mathrm{Charge}\)}%
\end{pgfscope}%
\begin{pgfscope}%
\pgfpathrectangle{\pgfqpoint{1.000000in}{7.000000in}}{\pgfqpoint{8.500000in}{2.000000in}}%
\pgfusepath{clip}%
\pgfsetbuttcap%
\pgfsetroundjoin%
\pgfsetlinewidth{2.007500pt}%
\definecolor{currentstroke}{rgb}{0.000000,0.500000,0.000000}%
\pgfsetstrokecolor{currentstroke}%
\pgfsetdash{}{0pt}%
\pgfpathmoveto{\pgfqpoint{2.760832in}{7.000000in}}%
\pgfpathlineto{\pgfqpoint{2.760832in}{8.347208in}}%
\pgfusepath{stroke}%
\end{pgfscope}%
\begin{pgfscope}%
\pgfpathrectangle{\pgfqpoint{1.000000in}{7.000000in}}{\pgfqpoint{8.500000in}{2.000000in}}%
\pgfusepath{clip}%
\pgfsetbuttcap%
\pgfsetroundjoin%
\pgfsetlinewidth{2.007500pt}%
\definecolor{currentstroke}{rgb}{0.000000,0.500000,0.000000}%
\pgfsetstrokecolor{currentstroke}%
\pgfsetdash{}{0pt}%
\pgfpathmoveto{\pgfqpoint{3.045805in}{7.000000in}}%
\pgfpathlineto{\pgfqpoint{3.045805in}{8.054271in}}%
\pgfusepath{stroke}%
\end{pgfscope}%
\begin{pgfscope}%
\pgfpathrectangle{\pgfqpoint{1.000000in}{7.000000in}}{\pgfqpoint{8.500000in}{2.000000in}}%
\pgfusepath{clip}%
\pgfsetbuttcap%
\pgfsetroundjoin%
\pgfsetlinewidth{2.007500pt}%
\definecolor{currentstroke}{rgb}{0.000000,0.500000,0.000000}%
\pgfsetstrokecolor{currentstroke}%
\pgfsetdash{}{0pt}%
\pgfpathmoveto{\pgfqpoint{3.081427in}{7.000000in}}%
\pgfpathlineto{\pgfqpoint{3.081427in}{7.857574in}}%
\pgfusepath{stroke}%
\end{pgfscope}%
\begin{pgfscope}%
\pgfpathrectangle{\pgfqpoint{1.000000in}{7.000000in}}{\pgfqpoint{8.500000in}{2.000000in}}%
\pgfusepath{clip}%
\pgfsetbuttcap%
\pgfsetroundjoin%
\pgfsetlinewidth{2.007500pt}%
\definecolor{currentstroke}{rgb}{0.000000,0.500000,0.000000}%
\pgfsetstrokecolor{currentstroke}%
\pgfsetdash{}{0pt}%
\pgfpathmoveto{\pgfqpoint{3.402021in}{7.000000in}}%
\pgfpathlineto{\pgfqpoint{3.402021in}{8.520901in}}%
\pgfusepath{stroke}%
\end{pgfscope}%
\begin{pgfscope}%
\pgfpathrectangle{\pgfqpoint{1.000000in}{7.000000in}}{\pgfqpoint{8.500000in}{2.000000in}}%
\pgfusepath{clip}%
\pgfsetbuttcap%
\pgfsetroundjoin%
\pgfsetlinewidth{2.007500pt}%
\definecolor{currentstroke}{rgb}{0.000000,0.500000,0.000000}%
\pgfsetstrokecolor{currentstroke}%
\pgfsetdash{}{0pt}%
\pgfpathmoveto{\pgfqpoint{3.580129in}{7.000000in}}%
\pgfpathlineto{\pgfqpoint{3.580129in}{8.156271in}}%
\pgfusepath{stroke}%
\end{pgfscope}%
\begin{pgfscope}%
\pgfpathrectangle{\pgfqpoint{1.000000in}{7.000000in}}{\pgfqpoint{8.500000in}{2.000000in}}%
\pgfusepath{clip}%
\pgfsetbuttcap%
\pgfsetroundjoin%
\pgfsetlinewidth{2.007500pt}%
\definecolor{currentstroke}{rgb}{0.000000,0.500000,0.000000}%
\pgfsetstrokecolor{currentstroke}%
\pgfsetdash{}{0pt}%
\pgfpathmoveto{\pgfqpoint{4.150075in}{7.000000in}}%
\pgfpathlineto{\pgfqpoint{4.150075in}{8.539188in}}%
\pgfusepath{stroke}%
\end{pgfscope}%
\begin{pgfscope}%
\pgfsetrectcap%
\pgfsetmiterjoin%
\pgfsetlinewidth{0.803000pt}%
\definecolor{currentstroke}{rgb}{0.000000,0.000000,0.000000}%
\pgfsetstrokecolor{currentstroke}%
\pgfsetdash{}{0pt}%
\pgfpathmoveto{\pgfqpoint{1.000000in}{7.000000in}}%
\pgfpathlineto{\pgfqpoint{1.000000in}{9.000000in}}%
\pgfusepath{stroke}%
\end{pgfscope}%
\begin{pgfscope}%
\pgfsetrectcap%
\pgfsetmiterjoin%
\pgfsetlinewidth{0.803000pt}%
\definecolor{currentstroke}{rgb}{0.000000,0.000000,0.000000}%
\pgfsetstrokecolor{currentstroke}%
\pgfsetdash{}{0pt}%
\pgfpathmoveto{\pgfqpoint{9.500000in}{7.000000in}}%
\pgfpathlineto{\pgfqpoint{9.500000in}{9.000000in}}%
\pgfusepath{stroke}%
\end{pgfscope}%
\begin{pgfscope}%
\pgfsetrectcap%
\pgfsetmiterjoin%
\pgfsetlinewidth{0.803000pt}%
\definecolor{currentstroke}{rgb}{0.000000,0.000000,0.000000}%
\pgfsetstrokecolor{currentstroke}%
\pgfsetdash{}{0pt}%
\pgfpathmoveto{\pgfqpoint{1.000000in}{7.000000in}}%
\pgfpathlineto{\pgfqpoint{9.500000in}{7.000000in}}%
\pgfusepath{stroke}%
\end{pgfscope}%
\begin{pgfscope}%
\pgfsetrectcap%
\pgfsetmiterjoin%
\pgfsetlinewidth{0.803000pt}%
\definecolor{currentstroke}{rgb}{0.000000,0.000000,0.000000}%
\pgfsetstrokecolor{currentstroke}%
\pgfsetdash{}{0pt}%
\pgfpathmoveto{\pgfqpoint{1.000000in}{9.000000in}}%
\pgfpathlineto{\pgfqpoint{9.500000in}{9.000000in}}%
\pgfusepath{stroke}%
\end{pgfscope}%
\begin{pgfscope}%
\pgfsetbuttcap%
\pgfsetmiterjoin%
\definecolor{currentfill}{rgb}{1.000000,1.000000,1.000000}%
\pgfsetfillcolor{currentfill}%
\pgfsetfillopacity{0.800000}%
\pgfsetlinewidth{1.003750pt}%
\definecolor{currentstroke}{rgb}{0.800000,0.800000,0.800000}%
\pgfsetstrokecolor{currentstroke}%
\pgfsetstrokeopacity{0.800000}%
\pgfsetdash{}{0pt}%
\pgfpathmoveto{\pgfqpoint{6.935606in}{8.382821in}}%
\pgfpathlineto{\pgfqpoint{9.305556in}{8.382821in}}%
\pgfpathquadraticcurveto{\pgfqpoint{9.361111in}{8.382821in}}{\pgfqpoint{9.361111in}{8.438377in}}%
\pgfpathlineto{\pgfqpoint{9.361111in}{8.805556in}}%
\pgfpathquadraticcurveto{\pgfqpoint{9.361111in}{8.861111in}}{\pgfqpoint{9.305556in}{8.861111in}}%
\pgfpathlineto{\pgfqpoint{6.935606in}{8.861111in}}%
\pgfpathquadraticcurveto{\pgfqpoint{6.880050in}{8.861111in}}{\pgfqpoint{6.880050in}{8.805556in}}%
\pgfpathlineto{\pgfqpoint{6.880050in}{8.438377in}}%
\pgfpathquadraticcurveto{\pgfqpoint{6.880050in}{8.382821in}}{\pgfqpoint{6.935606in}{8.382821in}}%
\pgfpathclose%
\pgfusepath{stroke,fill}%
\end{pgfscope}%
\begin{pgfscope}%
\pgfsetbuttcap%
\pgfsetroundjoin%
\pgfsetlinewidth{2.007500pt}%
\definecolor{currentstroke}{rgb}{0.000000,0.500000,0.000000}%
\pgfsetstrokecolor{currentstroke}%
\pgfsetdash{}{0pt}%
\pgfpathmoveto{\pgfqpoint{6.991161in}{8.647184in}}%
\pgfpathlineto{\pgfqpoint{7.546717in}{8.647184in}}%
\pgfusepath{stroke}%
\end{pgfscope}%
\begin{pgfscope}%
\definecolor{textcolor}{rgb}{0.000000,0.000000,0.000000}%
\pgfsetstrokecolor{textcolor}%
\pgfsetfillcolor{textcolor}%
\pgftext[x=7.768939in,y=8.549962in,left,base]{\color{textcolor}\sffamily\fontsize{20.000000}{24.000000}\selectfont recon Charge}%
\end{pgfscope}%
\begin{pgfscope}%
\pgfsetbuttcap%
\pgfsetmiterjoin%
\definecolor{currentfill}{rgb}{1.000000,1.000000,1.000000}%
\pgfsetfillcolor{currentfill}%
\pgfsetlinewidth{0.000000pt}%
\definecolor{currentstroke}{rgb}{0.000000,0.000000,0.000000}%
\pgfsetstrokecolor{currentstroke}%
\pgfsetstrokeopacity{0.000000}%
\pgfsetdash{}{0pt}%
\pgfpathmoveto{\pgfqpoint{1.000000in}{1.000000in}}%
\pgfpathlineto{\pgfqpoint{9.500000in}{1.000000in}}%
\pgfpathlineto{\pgfqpoint{9.500000in}{2.000000in}}%
\pgfpathlineto{\pgfqpoint{1.000000in}{2.000000in}}%
\pgfpathclose%
\pgfusepath{fill}%
\end{pgfscope}%
\begin{pgfscope}%
\pgfpathrectangle{\pgfqpoint{1.000000in}{1.000000in}}{\pgfqpoint{8.500000in}{1.000000in}}%
\pgfusepath{clip}%
\pgfsetbuttcap%
\pgfsetroundjoin%
\definecolor{currentfill}{rgb}{0.000000,0.000000,0.000000}%
\pgfsetfillcolor{currentfill}%
\pgfsetlinewidth{1.003750pt}%
\definecolor{currentstroke}{rgb}{0.000000,0.000000,0.000000}%
\pgfsetstrokecolor{currentstroke}%
\pgfsetdash{}{0pt}%
\pgfsys@defobject{currentmarker}{\pgfqpoint{-0.013889in}{-0.013889in}}{\pgfqpoint{0.013889in}{0.013889in}}{%
\pgfpathmoveto{\pgfqpoint{0.000000in}{-0.013889in}}%
\pgfpathcurveto{\pgfqpoint{0.003683in}{-0.013889in}}{\pgfqpoint{0.007216in}{-0.012425in}}{\pgfqpoint{0.009821in}{-0.009821in}}%
\pgfpathcurveto{\pgfqpoint{0.012425in}{-0.007216in}}{\pgfqpoint{0.013889in}{-0.003683in}}{\pgfqpoint{0.013889in}{0.000000in}}%
\pgfpathcurveto{\pgfqpoint{0.013889in}{0.003683in}}{\pgfqpoint{0.012425in}{0.007216in}}{\pgfqpoint{0.009821in}{0.009821in}}%
\pgfpathcurveto{\pgfqpoint{0.007216in}{0.012425in}}{\pgfqpoint{0.003683in}{0.013889in}}{\pgfqpoint{0.000000in}{0.013889in}}%
\pgfpathcurveto{\pgfqpoint{-0.003683in}{0.013889in}}{\pgfqpoint{-0.007216in}{0.012425in}}{\pgfqpoint{-0.009821in}{0.009821in}}%
\pgfpathcurveto{\pgfqpoint{-0.012425in}{0.007216in}}{\pgfqpoint{-0.013889in}{0.003683in}}{\pgfqpoint{-0.013889in}{0.000000in}}%
\pgfpathcurveto{\pgfqpoint{-0.013889in}{-0.003683in}}{\pgfqpoint{-0.012425in}{-0.007216in}}{\pgfqpoint{-0.009821in}{-0.009821in}}%
\pgfpathcurveto{\pgfqpoint{-0.007216in}{-0.012425in}}{\pgfqpoint{-0.003683in}{-0.013889in}}{\pgfqpoint{0.000000in}{-0.013889in}}%
\pgfpathclose%
\pgfusepath{stroke,fill}%
}%
\begin{pgfscope}%
\pgfsys@transformshift{-15.085604in}{1.300000in}%
\pgfsys@useobject{currentmarker}{}%
\end{pgfscope}%
\begin{pgfscope}%
\pgfsys@transformshift{-15.049982in}{1.600000in}%
\pgfsys@useobject{currentmarker}{}%
\end{pgfscope}%
\begin{pgfscope}%
\pgfsys@transformshift{-15.014360in}{1.600000in}%
\pgfsys@useobject{currentmarker}{}%
\end{pgfscope}%
\begin{pgfscope}%
\pgfsys@transformshift{-14.978739in}{1.400000in}%
\pgfsys@useobject{currentmarker}{}%
\end{pgfscope}%
\begin{pgfscope}%
\pgfsys@transformshift{-14.943117in}{1.600000in}%
\pgfsys@useobject{currentmarker}{}%
\end{pgfscope}%
\begin{pgfscope}%
\pgfsys@transformshift{-14.907495in}{1.400000in}%
\pgfsys@useobject{currentmarker}{}%
\end{pgfscope}%
\begin{pgfscope}%
\pgfsys@transformshift{-14.871874in}{1.500000in}%
\pgfsys@useobject{currentmarker}{}%
\end{pgfscope}%
\begin{pgfscope}%
\pgfsys@transformshift{-14.836252in}{1.500000in}%
\pgfsys@useobject{currentmarker}{}%
\end{pgfscope}%
\begin{pgfscope}%
\pgfsys@transformshift{-14.800631in}{1.400000in}%
\pgfsys@useobject{currentmarker}{}%
\end{pgfscope}%
\begin{pgfscope}%
\pgfsys@transformshift{-14.765009in}{1.400000in}%
\pgfsys@useobject{currentmarker}{}%
\end{pgfscope}%
\begin{pgfscope}%
\pgfsys@transformshift{-14.729387in}{1.500000in}%
\pgfsys@useobject{currentmarker}{}%
\end{pgfscope}%
\begin{pgfscope}%
\pgfsys@transformshift{-14.693766in}{1.600000in}%
\pgfsys@useobject{currentmarker}{}%
\end{pgfscope}%
\begin{pgfscope}%
\pgfsys@transformshift{-14.658144in}{1.600000in}%
\pgfsys@useobject{currentmarker}{}%
\end{pgfscope}%
\begin{pgfscope}%
\pgfsys@transformshift{-14.622522in}{1.500000in}%
\pgfsys@useobject{currentmarker}{}%
\end{pgfscope}%
\begin{pgfscope}%
\pgfsys@transformshift{-14.586901in}{1.700000in}%
\pgfsys@useobject{currentmarker}{}%
\end{pgfscope}%
\begin{pgfscope}%
\pgfsys@transformshift{-14.551279in}{1.500000in}%
\pgfsys@useobject{currentmarker}{}%
\end{pgfscope}%
\begin{pgfscope}%
\pgfsys@transformshift{-14.515658in}{1.600000in}%
\pgfsys@useobject{currentmarker}{}%
\end{pgfscope}%
\begin{pgfscope}%
\pgfsys@transformshift{-14.480036in}{1.300000in}%
\pgfsys@useobject{currentmarker}{}%
\end{pgfscope}%
\begin{pgfscope}%
\pgfsys@transformshift{-14.444414in}{1.500000in}%
\pgfsys@useobject{currentmarker}{}%
\end{pgfscope}%
\begin{pgfscope}%
\pgfsys@transformshift{-14.408793in}{1.400000in}%
\pgfsys@useobject{currentmarker}{}%
\end{pgfscope}%
\begin{pgfscope}%
\pgfsys@transformshift{-14.373171in}{1.300000in}%
\pgfsys@useobject{currentmarker}{}%
\end{pgfscope}%
\begin{pgfscope}%
\pgfsys@transformshift{-14.337549in}{1.500000in}%
\pgfsys@useobject{currentmarker}{}%
\end{pgfscope}%
\begin{pgfscope}%
\pgfsys@transformshift{-14.301928in}{1.500000in}%
\pgfsys@useobject{currentmarker}{}%
\end{pgfscope}%
\begin{pgfscope}%
\pgfsys@transformshift{-14.266306in}{1.600000in}%
\pgfsys@useobject{currentmarker}{}%
\end{pgfscope}%
\begin{pgfscope}%
\pgfsys@transformshift{-14.230685in}{1.500000in}%
\pgfsys@useobject{currentmarker}{}%
\end{pgfscope}%
\begin{pgfscope}%
\pgfsys@transformshift{-14.195063in}{1.500000in}%
\pgfsys@useobject{currentmarker}{}%
\end{pgfscope}%
\begin{pgfscope}%
\pgfsys@transformshift{-14.159441in}{1.600000in}%
\pgfsys@useobject{currentmarker}{}%
\end{pgfscope}%
\begin{pgfscope}%
\pgfsys@transformshift{-14.123820in}{1.300000in}%
\pgfsys@useobject{currentmarker}{}%
\end{pgfscope}%
\begin{pgfscope}%
\pgfsys@transformshift{-14.088198in}{1.500000in}%
\pgfsys@useobject{currentmarker}{}%
\end{pgfscope}%
\begin{pgfscope}%
\pgfsys@transformshift{-14.052576in}{1.400000in}%
\pgfsys@useobject{currentmarker}{}%
\end{pgfscope}%
\begin{pgfscope}%
\pgfsys@transformshift{-14.016955in}{1.500000in}%
\pgfsys@useobject{currentmarker}{}%
\end{pgfscope}%
\begin{pgfscope}%
\pgfsys@transformshift{-13.981333in}{1.400000in}%
\pgfsys@useobject{currentmarker}{}%
\end{pgfscope}%
\begin{pgfscope}%
\pgfsys@transformshift{-13.945712in}{1.500000in}%
\pgfsys@useobject{currentmarker}{}%
\end{pgfscope}%
\begin{pgfscope}%
\pgfsys@transformshift{-13.910090in}{1.400000in}%
\pgfsys@useobject{currentmarker}{}%
\end{pgfscope}%
\begin{pgfscope}%
\pgfsys@transformshift{-13.874468in}{1.600000in}%
\pgfsys@useobject{currentmarker}{}%
\end{pgfscope}%
\begin{pgfscope}%
\pgfsys@transformshift{-13.838847in}{1.600000in}%
\pgfsys@useobject{currentmarker}{}%
\end{pgfscope}%
\begin{pgfscope}%
\pgfsys@transformshift{-13.803225in}{1.400000in}%
\pgfsys@useobject{currentmarker}{}%
\end{pgfscope}%
\begin{pgfscope}%
\pgfsys@transformshift{-13.767603in}{1.600000in}%
\pgfsys@useobject{currentmarker}{}%
\end{pgfscope}%
\begin{pgfscope}%
\pgfsys@transformshift{-13.731982in}{1.500000in}%
\pgfsys@useobject{currentmarker}{}%
\end{pgfscope}%
\begin{pgfscope}%
\pgfsys@transformshift{-13.696360in}{1.500000in}%
\pgfsys@useobject{currentmarker}{}%
\end{pgfscope}%
\begin{pgfscope}%
\pgfsys@transformshift{-13.660738in}{1.400000in}%
\pgfsys@useobject{currentmarker}{}%
\end{pgfscope}%
\begin{pgfscope}%
\pgfsys@transformshift{-13.625117in}{1.500000in}%
\pgfsys@useobject{currentmarker}{}%
\end{pgfscope}%
\begin{pgfscope}%
\pgfsys@transformshift{-13.589495in}{1.300000in}%
\pgfsys@useobject{currentmarker}{}%
\end{pgfscope}%
\begin{pgfscope}%
\pgfsys@transformshift{-13.553874in}{1.400000in}%
\pgfsys@useobject{currentmarker}{}%
\end{pgfscope}%
\begin{pgfscope}%
\pgfsys@transformshift{-13.518252in}{1.500000in}%
\pgfsys@useobject{currentmarker}{}%
\end{pgfscope}%
\begin{pgfscope}%
\pgfsys@transformshift{-13.482630in}{1.500000in}%
\pgfsys@useobject{currentmarker}{}%
\end{pgfscope}%
\begin{pgfscope}%
\pgfsys@transformshift{-13.447009in}{1.400000in}%
\pgfsys@useobject{currentmarker}{}%
\end{pgfscope}%
\begin{pgfscope}%
\pgfsys@transformshift{-13.411387in}{1.400000in}%
\pgfsys@useobject{currentmarker}{}%
\end{pgfscope}%
\begin{pgfscope}%
\pgfsys@transformshift{-13.375765in}{1.400000in}%
\pgfsys@useobject{currentmarker}{}%
\end{pgfscope}%
\begin{pgfscope}%
\pgfsys@transformshift{-13.340144in}{1.500000in}%
\pgfsys@useobject{currentmarker}{}%
\end{pgfscope}%
\begin{pgfscope}%
\pgfsys@transformshift{-13.304522in}{1.600000in}%
\pgfsys@useobject{currentmarker}{}%
\end{pgfscope}%
\begin{pgfscope}%
\pgfsys@transformshift{-13.268901in}{1.500000in}%
\pgfsys@useobject{currentmarker}{}%
\end{pgfscope}%
\begin{pgfscope}%
\pgfsys@transformshift{-13.233279in}{1.400000in}%
\pgfsys@useobject{currentmarker}{}%
\end{pgfscope}%
\begin{pgfscope}%
\pgfsys@transformshift{-13.197657in}{1.600000in}%
\pgfsys@useobject{currentmarker}{}%
\end{pgfscope}%
\begin{pgfscope}%
\pgfsys@transformshift{-13.162036in}{1.400000in}%
\pgfsys@useobject{currentmarker}{}%
\end{pgfscope}%
\begin{pgfscope}%
\pgfsys@transformshift{-13.126414in}{1.400000in}%
\pgfsys@useobject{currentmarker}{}%
\end{pgfscope}%
\begin{pgfscope}%
\pgfsys@transformshift{-13.090792in}{1.500000in}%
\pgfsys@useobject{currentmarker}{}%
\end{pgfscope}%
\begin{pgfscope}%
\pgfsys@transformshift{-13.055171in}{1.700000in}%
\pgfsys@useobject{currentmarker}{}%
\end{pgfscope}%
\begin{pgfscope}%
\pgfsys@transformshift{-13.019549in}{1.500000in}%
\pgfsys@useobject{currentmarker}{}%
\end{pgfscope}%
\begin{pgfscope}%
\pgfsys@transformshift{-12.983928in}{1.400000in}%
\pgfsys@useobject{currentmarker}{}%
\end{pgfscope}%
\begin{pgfscope}%
\pgfsys@transformshift{-12.948306in}{1.500000in}%
\pgfsys@useobject{currentmarker}{}%
\end{pgfscope}%
\begin{pgfscope}%
\pgfsys@transformshift{-12.912684in}{1.600000in}%
\pgfsys@useobject{currentmarker}{}%
\end{pgfscope}%
\begin{pgfscope}%
\pgfsys@transformshift{-12.877063in}{1.500000in}%
\pgfsys@useobject{currentmarker}{}%
\end{pgfscope}%
\begin{pgfscope}%
\pgfsys@transformshift{-12.841441in}{1.600000in}%
\pgfsys@useobject{currentmarker}{}%
\end{pgfscope}%
\begin{pgfscope}%
\pgfsys@transformshift{-12.805819in}{1.400000in}%
\pgfsys@useobject{currentmarker}{}%
\end{pgfscope}%
\begin{pgfscope}%
\pgfsys@transformshift{-12.770198in}{1.500000in}%
\pgfsys@useobject{currentmarker}{}%
\end{pgfscope}%
\begin{pgfscope}%
\pgfsys@transformshift{-12.734576in}{1.300000in}%
\pgfsys@useobject{currentmarker}{}%
\end{pgfscope}%
\begin{pgfscope}%
\pgfsys@transformshift{-12.698955in}{1.600000in}%
\pgfsys@useobject{currentmarker}{}%
\end{pgfscope}%
\begin{pgfscope}%
\pgfsys@transformshift{-12.663333in}{1.500000in}%
\pgfsys@useobject{currentmarker}{}%
\end{pgfscope}%
\begin{pgfscope}%
\pgfsys@transformshift{-12.627711in}{1.500000in}%
\pgfsys@useobject{currentmarker}{}%
\end{pgfscope}%
\begin{pgfscope}%
\pgfsys@transformshift{-12.592090in}{1.700000in}%
\pgfsys@useobject{currentmarker}{}%
\end{pgfscope}%
\begin{pgfscope}%
\pgfsys@transformshift{-12.556468in}{1.400000in}%
\pgfsys@useobject{currentmarker}{}%
\end{pgfscope}%
\begin{pgfscope}%
\pgfsys@transformshift{-12.520846in}{1.500000in}%
\pgfsys@useobject{currentmarker}{}%
\end{pgfscope}%
\begin{pgfscope}%
\pgfsys@transformshift{-12.485225in}{1.500000in}%
\pgfsys@useobject{currentmarker}{}%
\end{pgfscope}%
\begin{pgfscope}%
\pgfsys@transformshift{-12.449603in}{1.500000in}%
\pgfsys@useobject{currentmarker}{}%
\end{pgfscope}%
\begin{pgfscope}%
\pgfsys@transformshift{-12.413982in}{1.400000in}%
\pgfsys@useobject{currentmarker}{}%
\end{pgfscope}%
\begin{pgfscope}%
\pgfsys@transformshift{-12.378360in}{1.400000in}%
\pgfsys@useobject{currentmarker}{}%
\end{pgfscope}%
\begin{pgfscope}%
\pgfsys@transformshift{-12.342738in}{1.500000in}%
\pgfsys@useobject{currentmarker}{}%
\end{pgfscope}%
\begin{pgfscope}%
\pgfsys@transformshift{-12.307117in}{1.300000in}%
\pgfsys@useobject{currentmarker}{}%
\end{pgfscope}%
\begin{pgfscope}%
\pgfsys@transformshift{-12.271495in}{1.400000in}%
\pgfsys@useobject{currentmarker}{}%
\end{pgfscope}%
\begin{pgfscope}%
\pgfsys@transformshift{-12.235873in}{1.400000in}%
\pgfsys@useobject{currentmarker}{}%
\end{pgfscope}%
\begin{pgfscope}%
\pgfsys@transformshift{-12.200252in}{1.600000in}%
\pgfsys@useobject{currentmarker}{}%
\end{pgfscope}%
\begin{pgfscope}%
\pgfsys@transformshift{-12.164630in}{1.400000in}%
\pgfsys@useobject{currentmarker}{}%
\end{pgfscope}%
\begin{pgfscope}%
\pgfsys@transformshift{-12.129008in}{1.500000in}%
\pgfsys@useobject{currentmarker}{}%
\end{pgfscope}%
\begin{pgfscope}%
\pgfsys@transformshift{-12.093387in}{1.600000in}%
\pgfsys@useobject{currentmarker}{}%
\end{pgfscope}%
\begin{pgfscope}%
\pgfsys@transformshift{-12.057765in}{1.500000in}%
\pgfsys@useobject{currentmarker}{}%
\end{pgfscope}%
\begin{pgfscope}%
\pgfsys@transformshift{-12.022144in}{1.600000in}%
\pgfsys@useobject{currentmarker}{}%
\end{pgfscope}%
\begin{pgfscope}%
\pgfsys@transformshift{-11.986522in}{1.700000in}%
\pgfsys@useobject{currentmarker}{}%
\end{pgfscope}%
\begin{pgfscope}%
\pgfsys@transformshift{-11.950900in}{1.500000in}%
\pgfsys@useobject{currentmarker}{}%
\end{pgfscope}%
\begin{pgfscope}%
\pgfsys@transformshift{-11.915279in}{1.600000in}%
\pgfsys@useobject{currentmarker}{}%
\end{pgfscope}%
\begin{pgfscope}%
\pgfsys@transformshift{-11.879657in}{1.500000in}%
\pgfsys@useobject{currentmarker}{}%
\end{pgfscope}%
\begin{pgfscope}%
\pgfsys@transformshift{-11.844035in}{1.300000in}%
\pgfsys@useobject{currentmarker}{}%
\end{pgfscope}%
\begin{pgfscope}%
\pgfsys@transformshift{-11.808414in}{1.500000in}%
\pgfsys@useobject{currentmarker}{}%
\end{pgfscope}%
\begin{pgfscope}%
\pgfsys@transformshift{-11.772792in}{1.500000in}%
\pgfsys@useobject{currentmarker}{}%
\end{pgfscope}%
\begin{pgfscope}%
\pgfsys@transformshift{-11.737171in}{1.500000in}%
\pgfsys@useobject{currentmarker}{}%
\end{pgfscope}%
\begin{pgfscope}%
\pgfsys@transformshift{-11.701549in}{1.600000in}%
\pgfsys@useobject{currentmarker}{}%
\end{pgfscope}%
\begin{pgfscope}%
\pgfsys@transformshift{-11.665927in}{1.700000in}%
\pgfsys@useobject{currentmarker}{}%
\end{pgfscope}%
\begin{pgfscope}%
\pgfsys@transformshift{-11.630306in}{1.600000in}%
\pgfsys@useobject{currentmarker}{}%
\end{pgfscope}%
\begin{pgfscope}%
\pgfsys@transformshift{-11.594684in}{1.500000in}%
\pgfsys@useobject{currentmarker}{}%
\end{pgfscope}%
\begin{pgfscope}%
\pgfsys@transformshift{-11.559062in}{1.600000in}%
\pgfsys@useobject{currentmarker}{}%
\end{pgfscope}%
\begin{pgfscope}%
\pgfsys@transformshift{-11.523441in}{1.500000in}%
\pgfsys@useobject{currentmarker}{}%
\end{pgfscope}%
\begin{pgfscope}%
\pgfsys@transformshift{-11.487819in}{1.500000in}%
\pgfsys@useobject{currentmarker}{}%
\end{pgfscope}%
\begin{pgfscope}%
\pgfsys@transformshift{-11.452198in}{1.600000in}%
\pgfsys@useobject{currentmarker}{}%
\end{pgfscope}%
\begin{pgfscope}%
\pgfsys@transformshift{-11.416576in}{1.600000in}%
\pgfsys@useobject{currentmarker}{}%
\end{pgfscope}%
\begin{pgfscope}%
\pgfsys@transformshift{-11.380954in}{1.400000in}%
\pgfsys@useobject{currentmarker}{}%
\end{pgfscope}%
\begin{pgfscope}%
\pgfsys@transformshift{-11.345333in}{1.500000in}%
\pgfsys@useobject{currentmarker}{}%
\end{pgfscope}%
\begin{pgfscope}%
\pgfsys@transformshift{-11.309711in}{1.600000in}%
\pgfsys@useobject{currentmarker}{}%
\end{pgfscope}%
\begin{pgfscope}%
\pgfsys@transformshift{-11.274089in}{1.600000in}%
\pgfsys@useobject{currentmarker}{}%
\end{pgfscope}%
\begin{pgfscope}%
\pgfsys@transformshift{-11.238468in}{1.400000in}%
\pgfsys@useobject{currentmarker}{}%
\end{pgfscope}%
\begin{pgfscope}%
\pgfsys@transformshift{-11.202846in}{1.600000in}%
\pgfsys@useobject{currentmarker}{}%
\end{pgfscope}%
\begin{pgfscope}%
\pgfsys@transformshift{-11.167225in}{1.500000in}%
\pgfsys@useobject{currentmarker}{}%
\end{pgfscope}%
\begin{pgfscope}%
\pgfsys@transformshift{-11.131603in}{1.600000in}%
\pgfsys@useobject{currentmarker}{}%
\end{pgfscope}%
\begin{pgfscope}%
\pgfsys@transformshift{-11.095981in}{1.500000in}%
\pgfsys@useobject{currentmarker}{}%
\end{pgfscope}%
\begin{pgfscope}%
\pgfsys@transformshift{-11.060360in}{1.600000in}%
\pgfsys@useobject{currentmarker}{}%
\end{pgfscope}%
\begin{pgfscope}%
\pgfsys@transformshift{-11.024738in}{1.700000in}%
\pgfsys@useobject{currentmarker}{}%
\end{pgfscope}%
\begin{pgfscope}%
\pgfsys@transformshift{-10.989116in}{1.400000in}%
\pgfsys@useobject{currentmarker}{}%
\end{pgfscope}%
\begin{pgfscope}%
\pgfsys@transformshift{-10.953495in}{1.500000in}%
\pgfsys@useobject{currentmarker}{}%
\end{pgfscope}%
\begin{pgfscope}%
\pgfsys@transformshift{-10.917873in}{1.500000in}%
\pgfsys@useobject{currentmarker}{}%
\end{pgfscope}%
\begin{pgfscope}%
\pgfsys@transformshift{-10.882252in}{1.400000in}%
\pgfsys@useobject{currentmarker}{}%
\end{pgfscope}%
\begin{pgfscope}%
\pgfsys@transformshift{-10.846630in}{1.500000in}%
\pgfsys@useobject{currentmarker}{}%
\end{pgfscope}%
\begin{pgfscope}%
\pgfsys@transformshift{-10.811008in}{1.400000in}%
\pgfsys@useobject{currentmarker}{}%
\end{pgfscope}%
\begin{pgfscope}%
\pgfsys@transformshift{-10.775387in}{1.600000in}%
\pgfsys@useobject{currentmarker}{}%
\end{pgfscope}%
\begin{pgfscope}%
\pgfsys@transformshift{-10.739765in}{1.500000in}%
\pgfsys@useobject{currentmarker}{}%
\end{pgfscope}%
\begin{pgfscope}%
\pgfsys@transformshift{-10.704143in}{1.700000in}%
\pgfsys@useobject{currentmarker}{}%
\end{pgfscope}%
\begin{pgfscope}%
\pgfsys@transformshift{-10.668522in}{1.500000in}%
\pgfsys@useobject{currentmarker}{}%
\end{pgfscope}%
\begin{pgfscope}%
\pgfsys@transformshift{-10.632900in}{1.500000in}%
\pgfsys@useobject{currentmarker}{}%
\end{pgfscope}%
\begin{pgfscope}%
\pgfsys@transformshift{-10.597278in}{1.200000in}%
\pgfsys@useobject{currentmarker}{}%
\end{pgfscope}%
\begin{pgfscope}%
\pgfsys@transformshift{-10.561657in}{1.500000in}%
\pgfsys@useobject{currentmarker}{}%
\end{pgfscope}%
\begin{pgfscope}%
\pgfsys@transformshift{-10.526035in}{1.500000in}%
\pgfsys@useobject{currentmarker}{}%
\end{pgfscope}%
\begin{pgfscope}%
\pgfsys@transformshift{-10.490414in}{1.400000in}%
\pgfsys@useobject{currentmarker}{}%
\end{pgfscope}%
\begin{pgfscope}%
\pgfsys@transformshift{-10.454792in}{1.500000in}%
\pgfsys@useobject{currentmarker}{}%
\end{pgfscope}%
\begin{pgfscope}%
\pgfsys@transformshift{-10.419170in}{1.500000in}%
\pgfsys@useobject{currentmarker}{}%
\end{pgfscope}%
\begin{pgfscope}%
\pgfsys@transformshift{-10.383549in}{1.600000in}%
\pgfsys@useobject{currentmarker}{}%
\end{pgfscope}%
\begin{pgfscope}%
\pgfsys@transformshift{-10.347927in}{1.300000in}%
\pgfsys@useobject{currentmarker}{}%
\end{pgfscope}%
\begin{pgfscope}%
\pgfsys@transformshift{-10.312305in}{1.500000in}%
\pgfsys@useobject{currentmarker}{}%
\end{pgfscope}%
\begin{pgfscope}%
\pgfsys@transformshift{-10.276684in}{1.600000in}%
\pgfsys@useobject{currentmarker}{}%
\end{pgfscope}%
\begin{pgfscope}%
\pgfsys@transformshift{-10.241062in}{1.400000in}%
\pgfsys@useobject{currentmarker}{}%
\end{pgfscope}%
\begin{pgfscope}%
\pgfsys@transformshift{-10.205441in}{1.600000in}%
\pgfsys@useobject{currentmarker}{}%
\end{pgfscope}%
\begin{pgfscope}%
\pgfsys@transformshift{-10.169819in}{1.500000in}%
\pgfsys@useobject{currentmarker}{}%
\end{pgfscope}%
\begin{pgfscope}%
\pgfsys@transformshift{-10.134197in}{1.600000in}%
\pgfsys@useobject{currentmarker}{}%
\end{pgfscope}%
\begin{pgfscope}%
\pgfsys@transformshift{-10.098576in}{1.500000in}%
\pgfsys@useobject{currentmarker}{}%
\end{pgfscope}%
\begin{pgfscope}%
\pgfsys@transformshift{-10.062954in}{1.500000in}%
\pgfsys@useobject{currentmarker}{}%
\end{pgfscope}%
\begin{pgfscope}%
\pgfsys@transformshift{-10.027332in}{1.500000in}%
\pgfsys@useobject{currentmarker}{}%
\end{pgfscope}%
\begin{pgfscope}%
\pgfsys@transformshift{-9.991711in}{1.600000in}%
\pgfsys@useobject{currentmarker}{}%
\end{pgfscope}%
\begin{pgfscope}%
\pgfsys@transformshift{-9.956089in}{1.500000in}%
\pgfsys@useobject{currentmarker}{}%
\end{pgfscope}%
\begin{pgfscope}%
\pgfsys@transformshift{-9.920468in}{1.600000in}%
\pgfsys@useobject{currentmarker}{}%
\end{pgfscope}%
\begin{pgfscope}%
\pgfsys@transformshift{-9.884846in}{1.500000in}%
\pgfsys@useobject{currentmarker}{}%
\end{pgfscope}%
\begin{pgfscope}%
\pgfsys@transformshift{-9.849224in}{1.600000in}%
\pgfsys@useobject{currentmarker}{}%
\end{pgfscope}%
\begin{pgfscope}%
\pgfsys@transformshift{-9.813603in}{1.600000in}%
\pgfsys@useobject{currentmarker}{}%
\end{pgfscope}%
\begin{pgfscope}%
\pgfsys@transformshift{-9.777981in}{1.500000in}%
\pgfsys@useobject{currentmarker}{}%
\end{pgfscope}%
\begin{pgfscope}%
\pgfsys@transformshift{-9.742359in}{1.500000in}%
\pgfsys@useobject{currentmarker}{}%
\end{pgfscope}%
\begin{pgfscope}%
\pgfsys@transformshift{-9.706738in}{1.500000in}%
\pgfsys@useobject{currentmarker}{}%
\end{pgfscope}%
\begin{pgfscope}%
\pgfsys@transformshift{-9.671116in}{1.400000in}%
\pgfsys@useobject{currentmarker}{}%
\end{pgfscope}%
\begin{pgfscope}%
\pgfsys@transformshift{-9.635495in}{1.600000in}%
\pgfsys@useobject{currentmarker}{}%
\end{pgfscope}%
\begin{pgfscope}%
\pgfsys@transformshift{-9.599873in}{1.500000in}%
\pgfsys@useobject{currentmarker}{}%
\end{pgfscope}%
\begin{pgfscope}%
\pgfsys@transformshift{-9.564251in}{1.600000in}%
\pgfsys@useobject{currentmarker}{}%
\end{pgfscope}%
\begin{pgfscope}%
\pgfsys@transformshift{-9.528630in}{1.600000in}%
\pgfsys@useobject{currentmarker}{}%
\end{pgfscope}%
\begin{pgfscope}%
\pgfsys@transformshift{-9.493008in}{1.600000in}%
\pgfsys@useobject{currentmarker}{}%
\end{pgfscope}%
\begin{pgfscope}%
\pgfsys@transformshift{-9.457386in}{1.700000in}%
\pgfsys@useobject{currentmarker}{}%
\end{pgfscope}%
\begin{pgfscope}%
\pgfsys@transformshift{-9.421765in}{1.500000in}%
\pgfsys@useobject{currentmarker}{}%
\end{pgfscope}%
\begin{pgfscope}%
\pgfsys@transformshift{-9.386143in}{1.400000in}%
\pgfsys@useobject{currentmarker}{}%
\end{pgfscope}%
\begin{pgfscope}%
\pgfsys@transformshift{-9.350522in}{1.600000in}%
\pgfsys@useobject{currentmarker}{}%
\end{pgfscope}%
\begin{pgfscope}%
\pgfsys@transformshift{-9.314900in}{1.500000in}%
\pgfsys@useobject{currentmarker}{}%
\end{pgfscope}%
\begin{pgfscope}%
\pgfsys@transformshift{-9.279278in}{1.400000in}%
\pgfsys@useobject{currentmarker}{}%
\end{pgfscope}%
\begin{pgfscope}%
\pgfsys@transformshift{-9.243657in}{1.600000in}%
\pgfsys@useobject{currentmarker}{}%
\end{pgfscope}%
\begin{pgfscope}%
\pgfsys@transformshift{-9.208035in}{1.600000in}%
\pgfsys@useobject{currentmarker}{}%
\end{pgfscope}%
\begin{pgfscope}%
\pgfsys@transformshift{-9.172413in}{1.500000in}%
\pgfsys@useobject{currentmarker}{}%
\end{pgfscope}%
\begin{pgfscope}%
\pgfsys@transformshift{-9.136792in}{1.500000in}%
\pgfsys@useobject{currentmarker}{}%
\end{pgfscope}%
\begin{pgfscope}%
\pgfsys@transformshift{-9.101170in}{1.200000in}%
\pgfsys@useobject{currentmarker}{}%
\end{pgfscope}%
\begin{pgfscope}%
\pgfsys@transformshift{-9.065548in}{1.500000in}%
\pgfsys@useobject{currentmarker}{}%
\end{pgfscope}%
\begin{pgfscope}%
\pgfsys@transformshift{-9.029927in}{1.500000in}%
\pgfsys@useobject{currentmarker}{}%
\end{pgfscope}%
\begin{pgfscope}%
\pgfsys@transformshift{-8.994305in}{1.700000in}%
\pgfsys@useobject{currentmarker}{}%
\end{pgfscope}%
\begin{pgfscope}%
\pgfsys@transformshift{-8.958684in}{1.600000in}%
\pgfsys@useobject{currentmarker}{}%
\end{pgfscope}%
\begin{pgfscope}%
\pgfsys@transformshift{-8.923062in}{1.500000in}%
\pgfsys@useobject{currentmarker}{}%
\end{pgfscope}%
\begin{pgfscope}%
\pgfsys@transformshift{-8.887440in}{1.400000in}%
\pgfsys@useobject{currentmarker}{}%
\end{pgfscope}%
\begin{pgfscope}%
\pgfsys@transformshift{-8.851819in}{1.400000in}%
\pgfsys@useobject{currentmarker}{}%
\end{pgfscope}%
\begin{pgfscope}%
\pgfsys@transformshift{-8.816197in}{1.600000in}%
\pgfsys@useobject{currentmarker}{}%
\end{pgfscope}%
\begin{pgfscope}%
\pgfsys@transformshift{-8.780575in}{1.600000in}%
\pgfsys@useobject{currentmarker}{}%
\end{pgfscope}%
\begin{pgfscope}%
\pgfsys@transformshift{-8.744954in}{1.400000in}%
\pgfsys@useobject{currentmarker}{}%
\end{pgfscope}%
\begin{pgfscope}%
\pgfsys@transformshift{-8.709332in}{1.500000in}%
\pgfsys@useobject{currentmarker}{}%
\end{pgfscope}%
\begin{pgfscope}%
\pgfsys@transformshift{-8.673711in}{1.500000in}%
\pgfsys@useobject{currentmarker}{}%
\end{pgfscope}%
\begin{pgfscope}%
\pgfsys@transformshift{-8.638089in}{1.500000in}%
\pgfsys@useobject{currentmarker}{}%
\end{pgfscope}%
\begin{pgfscope}%
\pgfsys@transformshift{-8.602467in}{1.300000in}%
\pgfsys@useobject{currentmarker}{}%
\end{pgfscope}%
\begin{pgfscope}%
\pgfsys@transformshift{-8.566846in}{1.600000in}%
\pgfsys@useobject{currentmarker}{}%
\end{pgfscope}%
\begin{pgfscope}%
\pgfsys@transformshift{-8.531224in}{1.700000in}%
\pgfsys@useobject{currentmarker}{}%
\end{pgfscope}%
\begin{pgfscope}%
\pgfsys@transformshift{-8.495602in}{1.500000in}%
\pgfsys@useobject{currentmarker}{}%
\end{pgfscope}%
\begin{pgfscope}%
\pgfsys@transformshift{-8.459981in}{1.600000in}%
\pgfsys@useobject{currentmarker}{}%
\end{pgfscope}%
\begin{pgfscope}%
\pgfsys@transformshift{-8.424359in}{1.600000in}%
\pgfsys@useobject{currentmarker}{}%
\end{pgfscope}%
\begin{pgfscope}%
\pgfsys@transformshift{-8.388738in}{1.700000in}%
\pgfsys@useobject{currentmarker}{}%
\end{pgfscope}%
\begin{pgfscope}%
\pgfsys@transformshift{-8.353116in}{1.700000in}%
\pgfsys@useobject{currentmarker}{}%
\end{pgfscope}%
\begin{pgfscope}%
\pgfsys@transformshift{-8.317494in}{1.500000in}%
\pgfsys@useobject{currentmarker}{}%
\end{pgfscope}%
\begin{pgfscope}%
\pgfsys@transformshift{-8.281873in}{1.700000in}%
\pgfsys@useobject{currentmarker}{}%
\end{pgfscope}%
\begin{pgfscope}%
\pgfsys@transformshift{-8.246251in}{1.600000in}%
\pgfsys@useobject{currentmarker}{}%
\end{pgfscope}%
\begin{pgfscope}%
\pgfsys@transformshift{-8.210629in}{1.500000in}%
\pgfsys@useobject{currentmarker}{}%
\end{pgfscope}%
\begin{pgfscope}%
\pgfsys@transformshift{-8.175008in}{1.500000in}%
\pgfsys@useobject{currentmarker}{}%
\end{pgfscope}%
\begin{pgfscope}%
\pgfsys@transformshift{-8.139386in}{1.500000in}%
\pgfsys@useobject{currentmarker}{}%
\end{pgfscope}%
\begin{pgfscope}%
\pgfsys@transformshift{-8.103765in}{1.300000in}%
\pgfsys@useobject{currentmarker}{}%
\end{pgfscope}%
\begin{pgfscope}%
\pgfsys@transformshift{-8.068143in}{1.400000in}%
\pgfsys@useobject{currentmarker}{}%
\end{pgfscope}%
\begin{pgfscope}%
\pgfsys@transformshift{-8.032521in}{1.500000in}%
\pgfsys@useobject{currentmarker}{}%
\end{pgfscope}%
\begin{pgfscope}%
\pgfsys@transformshift{-7.996900in}{1.500000in}%
\pgfsys@useobject{currentmarker}{}%
\end{pgfscope}%
\begin{pgfscope}%
\pgfsys@transformshift{-7.961278in}{1.500000in}%
\pgfsys@useobject{currentmarker}{}%
\end{pgfscope}%
\begin{pgfscope}%
\pgfsys@transformshift{-7.925656in}{1.500000in}%
\pgfsys@useobject{currentmarker}{}%
\end{pgfscope}%
\begin{pgfscope}%
\pgfsys@transformshift{-7.890035in}{1.400000in}%
\pgfsys@useobject{currentmarker}{}%
\end{pgfscope}%
\begin{pgfscope}%
\pgfsys@transformshift{-7.854413in}{1.600000in}%
\pgfsys@useobject{currentmarker}{}%
\end{pgfscope}%
\begin{pgfscope}%
\pgfsys@transformshift{-7.818792in}{1.400000in}%
\pgfsys@useobject{currentmarker}{}%
\end{pgfscope}%
\begin{pgfscope}%
\pgfsys@transformshift{-7.783170in}{1.500000in}%
\pgfsys@useobject{currentmarker}{}%
\end{pgfscope}%
\begin{pgfscope}%
\pgfsys@transformshift{-7.747548in}{1.600000in}%
\pgfsys@useobject{currentmarker}{}%
\end{pgfscope}%
\begin{pgfscope}%
\pgfsys@transformshift{-7.711927in}{1.500000in}%
\pgfsys@useobject{currentmarker}{}%
\end{pgfscope}%
\begin{pgfscope}%
\pgfsys@transformshift{-7.676305in}{1.400000in}%
\pgfsys@useobject{currentmarker}{}%
\end{pgfscope}%
\begin{pgfscope}%
\pgfsys@transformshift{-7.640683in}{1.500000in}%
\pgfsys@useobject{currentmarker}{}%
\end{pgfscope}%
\begin{pgfscope}%
\pgfsys@transformshift{-7.605062in}{1.600000in}%
\pgfsys@useobject{currentmarker}{}%
\end{pgfscope}%
\begin{pgfscope}%
\pgfsys@transformshift{-7.569440in}{1.600000in}%
\pgfsys@useobject{currentmarker}{}%
\end{pgfscope}%
\begin{pgfscope}%
\pgfsys@transformshift{-7.533818in}{1.500000in}%
\pgfsys@useobject{currentmarker}{}%
\end{pgfscope}%
\begin{pgfscope}%
\pgfsys@transformshift{-7.498197in}{1.400000in}%
\pgfsys@useobject{currentmarker}{}%
\end{pgfscope}%
\begin{pgfscope}%
\pgfsys@transformshift{-7.462575in}{1.500000in}%
\pgfsys@useobject{currentmarker}{}%
\end{pgfscope}%
\begin{pgfscope}%
\pgfsys@transformshift{-7.426954in}{1.600000in}%
\pgfsys@useobject{currentmarker}{}%
\end{pgfscope}%
\begin{pgfscope}%
\pgfsys@transformshift{-7.391332in}{1.600000in}%
\pgfsys@useobject{currentmarker}{}%
\end{pgfscope}%
\begin{pgfscope}%
\pgfsys@transformshift{-7.355710in}{1.600000in}%
\pgfsys@useobject{currentmarker}{}%
\end{pgfscope}%
\begin{pgfscope}%
\pgfsys@transformshift{-7.320089in}{1.500000in}%
\pgfsys@useobject{currentmarker}{}%
\end{pgfscope}%
\begin{pgfscope}%
\pgfsys@transformshift{-7.284467in}{1.400000in}%
\pgfsys@useobject{currentmarker}{}%
\end{pgfscope}%
\begin{pgfscope}%
\pgfsys@transformshift{-7.248845in}{1.600000in}%
\pgfsys@useobject{currentmarker}{}%
\end{pgfscope}%
\begin{pgfscope}%
\pgfsys@transformshift{-7.213224in}{1.400000in}%
\pgfsys@useobject{currentmarker}{}%
\end{pgfscope}%
\begin{pgfscope}%
\pgfsys@transformshift{-7.177602in}{1.500000in}%
\pgfsys@useobject{currentmarker}{}%
\end{pgfscope}%
\begin{pgfscope}%
\pgfsys@transformshift{-7.141981in}{1.500000in}%
\pgfsys@useobject{currentmarker}{}%
\end{pgfscope}%
\begin{pgfscope}%
\pgfsys@transformshift{-7.106359in}{1.400000in}%
\pgfsys@useobject{currentmarker}{}%
\end{pgfscope}%
\begin{pgfscope}%
\pgfsys@transformshift{-7.070737in}{1.700000in}%
\pgfsys@useobject{currentmarker}{}%
\end{pgfscope}%
\begin{pgfscope}%
\pgfsys@transformshift{-7.035116in}{1.600000in}%
\pgfsys@useobject{currentmarker}{}%
\end{pgfscope}%
\begin{pgfscope}%
\pgfsys@transformshift{-6.999494in}{1.400000in}%
\pgfsys@useobject{currentmarker}{}%
\end{pgfscope}%
\begin{pgfscope}%
\pgfsys@transformshift{-6.963872in}{1.500000in}%
\pgfsys@useobject{currentmarker}{}%
\end{pgfscope}%
\begin{pgfscope}%
\pgfsys@transformshift{-6.928251in}{1.500000in}%
\pgfsys@useobject{currentmarker}{}%
\end{pgfscope}%
\begin{pgfscope}%
\pgfsys@transformshift{-6.892629in}{1.500000in}%
\pgfsys@useobject{currentmarker}{}%
\end{pgfscope}%
\begin{pgfscope}%
\pgfsys@transformshift{-6.857008in}{1.500000in}%
\pgfsys@useobject{currentmarker}{}%
\end{pgfscope}%
\begin{pgfscope}%
\pgfsys@transformshift{-6.821386in}{1.500000in}%
\pgfsys@useobject{currentmarker}{}%
\end{pgfscope}%
\begin{pgfscope}%
\pgfsys@transformshift{-6.785764in}{1.500000in}%
\pgfsys@useobject{currentmarker}{}%
\end{pgfscope}%
\begin{pgfscope}%
\pgfsys@transformshift{-6.750143in}{1.400000in}%
\pgfsys@useobject{currentmarker}{}%
\end{pgfscope}%
\begin{pgfscope}%
\pgfsys@transformshift{-6.714521in}{1.700000in}%
\pgfsys@useobject{currentmarker}{}%
\end{pgfscope}%
\begin{pgfscope}%
\pgfsys@transformshift{-6.678899in}{1.800000in}%
\pgfsys@useobject{currentmarker}{}%
\end{pgfscope}%
\begin{pgfscope}%
\pgfsys@transformshift{-6.643278in}{1.500000in}%
\pgfsys@useobject{currentmarker}{}%
\end{pgfscope}%
\begin{pgfscope}%
\pgfsys@transformshift{-6.607656in}{1.500000in}%
\pgfsys@useobject{currentmarker}{}%
\end{pgfscope}%
\begin{pgfscope}%
\pgfsys@transformshift{-6.572035in}{1.500000in}%
\pgfsys@useobject{currentmarker}{}%
\end{pgfscope}%
\begin{pgfscope}%
\pgfsys@transformshift{-6.536413in}{1.400000in}%
\pgfsys@useobject{currentmarker}{}%
\end{pgfscope}%
\begin{pgfscope}%
\pgfsys@transformshift{-6.500791in}{1.700000in}%
\pgfsys@useobject{currentmarker}{}%
\end{pgfscope}%
\begin{pgfscope}%
\pgfsys@transformshift{-6.465170in}{1.400000in}%
\pgfsys@useobject{currentmarker}{}%
\end{pgfscope}%
\begin{pgfscope}%
\pgfsys@transformshift{-6.429548in}{1.400000in}%
\pgfsys@useobject{currentmarker}{}%
\end{pgfscope}%
\begin{pgfscope}%
\pgfsys@transformshift{-6.393926in}{1.300000in}%
\pgfsys@useobject{currentmarker}{}%
\end{pgfscope}%
\begin{pgfscope}%
\pgfsys@transformshift{-6.358305in}{1.600000in}%
\pgfsys@useobject{currentmarker}{}%
\end{pgfscope}%
\begin{pgfscope}%
\pgfsys@transformshift{-6.322683in}{1.500000in}%
\pgfsys@useobject{currentmarker}{}%
\end{pgfscope}%
\begin{pgfscope}%
\pgfsys@transformshift{-6.287062in}{1.400000in}%
\pgfsys@useobject{currentmarker}{}%
\end{pgfscope}%
\begin{pgfscope}%
\pgfsys@transformshift{-6.251440in}{1.600000in}%
\pgfsys@useobject{currentmarker}{}%
\end{pgfscope}%
\begin{pgfscope}%
\pgfsys@transformshift{-6.215818in}{1.600000in}%
\pgfsys@useobject{currentmarker}{}%
\end{pgfscope}%
\begin{pgfscope}%
\pgfsys@transformshift{-6.180197in}{1.400000in}%
\pgfsys@useobject{currentmarker}{}%
\end{pgfscope}%
\begin{pgfscope}%
\pgfsys@transformshift{-6.144575in}{1.400000in}%
\pgfsys@useobject{currentmarker}{}%
\end{pgfscope}%
\begin{pgfscope}%
\pgfsys@transformshift{-6.108953in}{1.500000in}%
\pgfsys@useobject{currentmarker}{}%
\end{pgfscope}%
\begin{pgfscope}%
\pgfsys@transformshift{-6.073332in}{1.500000in}%
\pgfsys@useobject{currentmarker}{}%
\end{pgfscope}%
\begin{pgfscope}%
\pgfsys@transformshift{-6.037710in}{1.400000in}%
\pgfsys@useobject{currentmarker}{}%
\end{pgfscope}%
\begin{pgfscope}%
\pgfsys@transformshift{-6.002088in}{1.400000in}%
\pgfsys@useobject{currentmarker}{}%
\end{pgfscope}%
\begin{pgfscope}%
\pgfsys@transformshift{-5.966467in}{1.600000in}%
\pgfsys@useobject{currentmarker}{}%
\end{pgfscope}%
\begin{pgfscope}%
\pgfsys@transformshift{-5.930845in}{1.500000in}%
\pgfsys@useobject{currentmarker}{}%
\end{pgfscope}%
\begin{pgfscope}%
\pgfsys@transformshift{-5.895224in}{1.500000in}%
\pgfsys@useobject{currentmarker}{}%
\end{pgfscope}%
\begin{pgfscope}%
\pgfsys@transformshift{-5.859602in}{1.500000in}%
\pgfsys@useobject{currentmarker}{}%
\end{pgfscope}%
\begin{pgfscope}%
\pgfsys@transformshift{-5.823980in}{1.500000in}%
\pgfsys@useobject{currentmarker}{}%
\end{pgfscope}%
\begin{pgfscope}%
\pgfsys@transformshift{-5.788359in}{1.400000in}%
\pgfsys@useobject{currentmarker}{}%
\end{pgfscope}%
\begin{pgfscope}%
\pgfsys@transformshift{-5.752737in}{1.400000in}%
\pgfsys@useobject{currentmarker}{}%
\end{pgfscope}%
\begin{pgfscope}%
\pgfsys@transformshift{-5.717115in}{1.800000in}%
\pgfsys@useobject{currentmarker}{}%
\end{pgfscope}%
\begin{pgfscope}%
\pgfsys@transformshift{-5.681494in}{1.500000in}%
\pgfsys@useobject{currentmarker}{}%
\end{pgfscope}%
\begin{pgfscope}%
\pgfsys@transformshift{-5.645872in}{1.600000in}%
\pgfsys@useobject{currentmarker}{}%
\end{pgfscope}%
\begin{pgfscope}%
\pgfsys@transformshift{-5.610251in}{1.500000in}%
\pgfsys@useobject{currentmarker}{}%
\end{pgfscope}%
\begin{pgfscope}%
\pgfsys@transformshift{-5.574629in}{1.400000in}%
\pgfsys@useobject{currentmarker}{}%
\end{pgfscope}%
\begin{pgfscope}%
\pgfsys@transformshift{-5.539007in}{1.500000in}%
\pgfsys@useobject{currentmarker}{}%
\end{pgfscope}%
\begin{pgfscope}%
\pgfsys@transformshift{-5.503386in}{1.600000in}%
\pgfsys@useobject{currentmarker}{}%
\end{pgfscope}%
\begin{pgfscope}%
\pgfsys@transformshift{-5.467764in}{1.500000in}%
\pgfsys@useobject{currentmarker}{}%
\end{pgfscope}%
\begin{pgfscope}%
\pgfsys@transformshift{-5.432142in}{1.400000in}%
\pgfsys@useobject{currentmarker}{}%
\end{pgfscope}%
\begin{pgfscope}%
\pgfsys@transformshift{-5.396521in}{1.500000in}%
\pgfsys@useobject{currentmarker}{}%
\end{pgfscope}%
\begin{pgfscope}%
\pgfsys@transformshift{-5.360899in}{1.600000in}%
\pgfsys@useobject{currentmarker}{}%
\end{pgfscope}%
\begin{pgfscope}%
\pgfsys@transformshift{-5.325278in}{1.500000in}%
\pgfsys@useobject{currentmarker}{}%
\end{pgfscope}%
\begin{pgfscope}%
\pgfsys@transformshift{-5.289656in}{1.400000in}%
\pgfsys@useobject{currentmarker}{}%
\end{pgfscope}%
\begin{pgfscope}%
\pgfsys@transformshift{-5.254034in}{1.500000in}%
\pgfsys@useobject{currentmarker}{}%
\end{pgfscope}%
\begin{pgfscope}%
\pgfsys@transformshift{-5.218413in}{1.400000in}%
\pgfsys@useobject{currentmarker}{}%
\end{pgfscope}%
\begin{pgfscope}%
\pgfsys@transformshift{-5.182791in}{1.400000in}%
\pgfsys@useobject{currentmarker}{}%
\end{pgfscope}%
\begin{pgfscope}%
\pgfsys@transformshift{-5.147169in}{1.500000in}%
\pgfsys@useobject{currentmarker}{}%
\end{pgfscope}%
\begin{pgfscope}%
\pgfsys@transformshift{-5.111548in}{1.500000in}%
\pgfsys@useobject{currentmarker}{}%
\end{pgfscope}%
\begin{pgfscope}%
\pgfsys@transformshift{-5.075926in}{1.400000in}%
\pgfsys@useobject{currentmarker}{}%
\end{pgfscope}%
\begin{pgfscope}%
\pgfsys@transformshift{-5.040305in}{1.500000in}%
\pgfsys@useobject{currentmarker}{}%
\end{pgfscope}%
\begin{pgfscope}%
\pgfsys@transformshift{-5.004683in}{1.400000in}%
\pgfsys@useobject{currentmarker}{}%
\end{pgfscope}%
\begin{pgfscope}%
\pgfsys@transformshift{-4.969061in}{1.600000in}%
\pgfsys@useobject{currentmarker}{}%
\end{pgfscope}%
\begin{pgfscope}%
\pgfsys@transformshift{-4.933440in}{1.500000in}%
\pgfsys@useobject{currentmarker}{}%
\end{pgfscope}%
\begin{pgfscope}%
\pgfsys@transformshift{-4.897818in}{1.600000in}%
\pgfsys@useobject{currentmarker}{}%
\end{pgfscope}%
\begin{pgfscope}%
\pgfsys@transformshift{-4.862196in}{1.600000in}%
\pgfsys@useobject{currentmarker}{}%
\end{pgfscope}%
\begin{pgfscope}%
\pgfsys@transformshift{-4.826575in}{1.600000in}%
\pgfsys@useobject{currentmarker}{}%
\end{pgfscope}%
\begin{pgfscope}%
\pgfsys@transformshift{-4.790953in}{1.400000in}%
\pgfsys@useobject{currentmarker}{}%
\end{pgfscope}%
\begin{pgfscope}%
\pgfsys@transformshift{-4.755332in}{1.500000in}%
\pgfsys@useobject{currentmarker}{}%
\end{pgfscope}%
\begin{pgfscope}%
\pgfsys@transformshift{-4.719710in}{1.500000in}%
\pgfsys@useobject{currentmarker}{}%
\end{pgfscope}%
\begin{pgfscope}%
\pgfsys@transformshift{-4.684088in}{1.600000in}%
\pgfsys@useobject{currentmarker}{}%
\end{pgfscope}%
\begin{pgfscope}%
\pgfsys@transformshift{-4.648467in}{1.500000in}%
\pgfsys@useobject{currentmarker}{}%
\end{pgfscope}%
\begin{pgfscope}%
\pgfsys@transformshift{-4.612845in}{1.600000in}%
\pgfsys@useobject{currentmarker}{}%
\end{pgfscope}%
\begin{pgfscope}%
\pgfsys@transformshift{-4.577223in}{1.400000in}%
\pgfsys@useobject{currentmarker}{}%
\end{pgfscope}%
\begin{pgfscope}%
\pgfsys@transformshift{-4.541602in}{1.500000in}%
\pgfsys@useobject{currentmarker}{}%
\end{pgfscope}%
\begin{pgfscope}%
\pgfsys@transformshift{-4.505980in}{1.300000in}%
\pgfsys@useobject{currentmarker}{}%
\end{pgfscope}%
\begin{pgfscope}%
\pgfsys@transformshift{-4.470358in}{1.400000in}%
\pgfsys@useobject{currentmarker}{}%
\end{pgfscope}%
\begin{pgfscope}%
\pgfsys@transformshift{-4.434737in}{1.600000in}%
\pgfsys@useobject{currentmarker}{}%
\end{pgfscope}%
\begin{pgfscope}%
\pgfsys@transformshift{-4.399115in}{1.300000in}%
\pgfsys@useobject{currentmarker}{}%
\end{pgfscope}%
\begin{pgfscope}%
\pgfsys@transformshift{-4.363494in}{1.500000in}%
\pgfsys@useobject{currentmarker}{}%
\end{pgfscope}%
\begin{pgfscope}%
\pgfsys@transformshift{-4.327872in}{1.500000in}%
\pgfsys@useobject{currentmarker}{}%
\end{pgfscope}%
\begin{pgfscope}%
\pgfsys@transformshift{-4.292250in}{1.500000in}%
\pgfsys@useobject{currentmarker}{}%
\end{pgfscope}%
\begin{pgfscope}%
\pgfsys@transformshift{-4.256629in}{1.500000in}%
\pgfsys@useobject{currentmarker}{}%
\end{pgfscope}%
\begin{pgfscope}%
\pgfsys@transformshift{-4.221007in}{1.400000in}%
\pgfsys@useobject{currentmarker}{}%
\end{pgfscope}%
\begin{pgfscope}%
\pgfsys@transformshift{-4.185385in}{1.500000in}%
\pgfsys@useobject{currentmarker}{}%
\end{pgfscope}%
\begin{pgfscope}%
\pgfsys@transformshift{-4.149764in}{1.600000in}%
\pgfsys@useobject{currentmarker}{}%
\end{pgfscope}%
\begin{pgfscope}%
\pgfsys@transformshift{-4.114142in}{1.500000in}%
\pgfsys@useobject{currentmarker}{}%
\end{pgfscope}%
\begin{pgfscope}%
\pgfsys@transformshift{-4.078521in}{1.600000in}%
\pgfsys@useobject{currentmarker}{}%
\end{pgfscope}%
\begin{pgfscope}%
\pgfsys@transformshift{-4.042899in}{1.500000in}%
\pgfsys@useobject{currentmarker}{}%
\end{pgfscope}%
\begin{pgfscope}%
\pgfsys@transformshift{-4.007277in}{1.300000in}%
\pgfsys@useobject{currentmarker}{}%
\end{pgfscope}%
\begin{pgfscope}%
\pgfsys@transformshift{-3.971656in}{1.600000in}%
\pgfsys@useobject{currentmarker}{}%
\end{pgfscope}%
\begin{pgfscope}%
\pgfsys@transformshift{-3.936034in}{1.400000in}%
\pgfsys@useobject{currentmarker}{}%
\end{pgfscope}%
\begin{pgfscope}%
\pgfsys@transformshift{-3.900412in}{1.700000in}%
\pgfsys@useobject{currentmarker}{}%
\end{pgfscope}%
\begin{pgfscope}%
\pgfsys@transformshift{-3.864791in}{1.500000in}%
\pgfsys@useobject{currentmarker}{}%
\end{pgfscope}%
\begin{pgfscope}%
\pgfsys@transformshift{-3.829169in}{1.500000in}%
\pgfsys@useobject{currentmarker}{}%
\end{pgfscope}%
\begin{pgfscope}%
\pgfsys@transformshift{-3.793548in}{1.400000in}%
\pgfsys@useobject{currentmarker}{}%
\end{pgfscope}%
\begin{pgfscope}%
\pgfsys@transformshift{-3.757926in}{1.700000in}%
\pgfsys@useobject{currentmarker}{}%
\end{pgfscope}%
\begin{pgfscope}%
\pgfsys@transformshift{-3.722304in}{1.600000in}%
\pgfsys@useobject{currentmarker}{}%
\end{pgfscope}%
\begin{pgfscope}%
\pgfsys@transformshift{-3.686683in}{1.500000in}%
\pgfsys@useobject{currentmarker}{}%
\end{pgfscope}%
\begin{pgfscope}%
\pgfsys@transformshift{-3.651061in}{1.500000in}%
\pgfsys@useobject{currentmarker}{}%
\end{pgfscope}%
\begin{pgfscope}%
\pgfsys@transformshift{-3.615439in}{1.500000in}%
\pgfsys@useobject{currentmarker}{}%
\end{pgfscope}%
\begin{pgfscope}%
\pgfsys@transformshift{-3.579818in}{1.600000in}%
\pgfsys@useobject{currentmarker}{}%
\end{pgfscope}%
\begin{pgfscope}%
\pgfsys@transformshift{-3.544196in}{1.700000in}%
\pgfsys@useobject{currentmarker}{}%
\end{pgfscope}%
\begin{pgfscope}%
\pgfsys@transformshift{-3.508575in}{1.700000in}%
\pgfsys@useobject{currentmarker}{}%
\end{pgfscope}%
\begin{pgfscope}%
\pgfsys@transformshift{-3.472953in}{1.400000in}%
\pgfsys@useobject{currentmarker}{}%
\end{pgfscope}%
\begin{pgfscope}%
\pgfsys@transformshift{-3.437331in}{1.400000in}%
\pgfsys@useobject{currentmarker}{}%
\end{pgfscope}%
\begin{pgfscope}%
\pgfsys@transformshift{-3.401710in}{1.400000in}%
\pgfsys@useobject{currentmarker}{}%
\end{pgfscope}%
\begin{pgfscope}%
\pgfsys@transformshift{-3.366088in}{1.500000in}%
\pgfsys@useobject{currentmarker}{}%
\end{pgfscope}%
\begin{pgfscope}%
\pgfsys@transformshift{-3.330466in}{1.400000in}%
\pgfsys@useobject{currentmarker}{}%
\end{pgfscope}%
\begin{pgfscope}%
\pgfsys@transformshift{-3.294845in}{1.500000in}%
\pgfsys@useobject{currentmarker}{}%
\end{pgfscope}%
\begin{pgfscope}%
\pgfsys@transformshift{-3.259223in}{1.400000in}%
\pgfsys@useobject{currentmarker}{}%
\end{pgfscope}%
\begin{pgfscope}%
\pgfsys@transformshift{-3.223602in}{1.400000in}%
\pgfsys@useobject{currentmarker}{}%
\end{pgfscope}%
\begin{pgfscope}%
\pgfsys@transformshift{-3.187980in}{1.500000in}%
\pgfsys@useobject{currentmarker}{}%
\end{pgfscope}%
\begin{pgfscope}%
\pgfsys@transformshift{-3.152358in}{1.400000in}%
\pgfsys@useobject{currentmarker}{}%
\end{pgfscope}%
\begin{pgfscope}%
\pgfsys@transformshift{-3.116737in}{1.400000in}%
\pgfsys@useobject{currentmarker}{}%
\end{pgfscope}%
\begin{pgfscope}%
\pgfsys@transformshift{-3.081115in}{1.400000in}%
\pgfsys@useobject{currentmarker}{}%
\end{pgfscope}%
\begin{pgfscope}%
\pgfsys@transformshift{-3.045493in}{1.500000in}%
\pgfsys@useobject{currentmarker}{}%
\end{pgfscope}%
\begin{pgfscope}%
\pgfsys@transformshift{-3.009872in}{1.500000in}%
\pgfsys@useobject{currentmarker}{}%
\end{pgfscope}%
\begin{pgfscope}%
\pgfsys@transformshift{-2.974250in}{1.300000in}%
\pgfsys@useobject{currentmarker}{}%
\end{pgfscope}%
\begin{pgfscope}%
\pgfsys@transformshift{-2.938628in}{1.500000in}%
\pgfsys@useobject{currentmarker}{}%
\end{pgfscope}%
\begin{pgfscope}%
\pgfsys@transformshift{-2.903007in}{1.500000in}%
\pgfsys@useobject{currentmarker}{}%
\end{pgfscope}%
\begin{pgfscope}%
\pgfsys@transformshift{-2.867385in}{1.600000in}%
\pgfsys@useobject{currentmarker}{}%
\end{pgfscope}%
\begin{pgfscope}%
\pgfsys@transformshift{-2.831764in}{1.500000in}%
\pgfsys@useobject{currentmarker}{}%
\end{pgfscope}%
\begin{pgfscope}%
\pgfsys@transformshift{-2.796142in}{1.500000in}%
\pgfsys@useobject{currentmarker}{}%
\end{pgfscope}%
\begin{pgfscope}%
\pgfsys@transformshift{-2.760520in}{1.700000in}%
\pgfsys@useobject{currentmarker}{}%
\end{pgfscope}%
\begin{pgfscope}%
\pgfsys@transformshift{-2.724899in}{1.400000in}%
\pgfsys@useobject{currentmarker}{}%
\end{pgfscope}%
\begin{pgfscope}%
\pgfsys@transformshift{-2.689277in}{1.500000in}%
\pgfsys@useobject{currentmarker}{}%
\end{pgfscope}%
\begin{pgfscope}%
\pgfsys@transformshift{-2.653655in}{1.500000in}%
\pgfsys@useobject{currentmarker}{}%
\end{pgfscope}%
\begin{pgfscope}%
\pgfsys@transformshift{-2.618034in}{1.400000in}%
\pgfsys@useobject{currentmarker}{}%
\end{pgfscope}%
\begin{pgfscope}%
\pgfsys@transformshift{-2.582412in}{1.500000in}%
\pgfsys@useobject{currentmarker}{}%
\end{pgfscope}%
\begin{pgfscope}%
\pgfsys@transformshift{-2.546791in}{1.700000in}%
\pgfsys@useobject{currentmarker}{}%
\end{pgfscope}%
\begin{pgfscope}%
\pgfsys@transformshift{-2.511169in}{1.600000in}%
\pgfsys@useobject{currentmarker}{}%
\end{pgfscope}%
\begin{pgfscope}%
\pgfsys@transformshift{-2.475547in}{1.300000in}%
\pgfsys@useobject{currentmarker}{}%
\end{pgfscope}%
\begin{pgfscope}%
\pgfsys@transformshift{-2.439926in}{1.400000in}%
\pgfsys@useobject{currentmarker}{}%
\end{pgfscope}%
\begin{pgfscope}%
\pgfsys@transformshift{-2.404304in}{1.500000in}%
\pgfsys@useobject{currentmarker}{}%
\end{pgfscope}%
\begin{pgfscope}%
\pgfsys@transformshift{-2.368682in}{1.700000in}%
\pgfsys@useobject{currentmarker}{}%
\end{pgfscope}%
\begin{pgfscope}%
\pgfsys@transformshift{-2.333061in}{1.500000in}%
\pgfsys@useobject{currentmarker}{}%
\end{pgfscope}%
\begin{pgfscope}%
\pgfsys@transformshift{-2.297439in}{1.400000in}%
\pgfsys@useobject{currentmarker}{}%
\end{pgfscope}%
\begin{pgfscope}%
\pgfsys@transformshift{-2.261818in}{1.600000in}%
\pgfsys@useobject{currentmarker}{}%
\end{pgfscope}%
\begin{pgfscope}%
\pgfsys@transformshift{-2.226196in}{1.500000in}%
\pgfsys@useobject{currentmarker}{}%
\end{pgfscope}%
\begin{pgfscope}%
\pgfsys@transformshift{-2.190574in}{1.500000in}%
\pgfsys@useobject{currentmarker}{}%
\end{pgfscope}%
\begin{pgfscope}%
\pgfsys@transformshift{-2.154953in}{1.600000in}%
\pgfsys@useobject{currentmarker}{}%
\end{pgfscope}%
\begin{pgfscope}%
\pgfsys@transformshift{-2.119331in}{1.600000in}%
\pgfsys@useobject{currentmarker}{}%
\end{pgfscope}%
\begin{pgfscope}%
\pgfsys@transformshift{-2.083709in}{1.400000in}%
\pgfsys@useobject{currentmarker}{}%
\end{pgfscope}%
\begin{pgfscope}%
\pgfsys@transformshift{-2.048088in}{1.500000in}%
\pgfsys@useobject{currentmarker}{}%
\end{pgfscope}%
\begin{pgfscope}%
\pgfsys@transformshift{-2.012466in}{1.500000in}%
\pgfsys@useobject{currentmarker}{}%
\end{pgfscope}%
\begin{pgfscope}%
\pgfsys@transformshift{-1.976845in}{1.500000in}%
\pgfsys@useobject{currentmarker}{}%
\end{pgfscope}%
\begin{pgfscope}%
\pgfsys@transformshift{-1.941223in}{1.500000in}%
\pgfsys@useobject{currentmarker}{}%
\end{pgfscope}%
\begin{pgfscope}%
\pgfsys@transformshift{-1.905601in}{1.700000in}%
\pgfsys@useobject{currentmarker}{}%
\end{pgfscope}%
\begin{pgfscope}%
\pgfsys@transformshift{-1.869980in}{1.500000in}%
\pgfsys@useobject{currentmarker}{}%
\end{pgfscope}%
\begin{pgfscope}%
\pgfsys@transformshift{-1.834358in}{1.400000in}%
\pgfsys@useobject{currentmarker}{}%
\end{pgfscope}%
\begin{pgfscope}%
\pgfsys@transformshift{-1.798736in}{1.400000in}%
\pgfsys@useobject{currentmarker}{}%
\end{pgfscope}%
\begin{pgfscope}%
\pgfsys@transformshift{-1.763115in}{1.500000in}%
\pgfsys@useobject{currentmarker}{}%
\end{pgfscope}%
\begin{pgfscope}%
\pgfsys@transformshift{-1.727493in}{1.400000in}%
\pgfsys@useobject{currentmarker}{}%
\end{pgfscope}%
\begin{pgfscope}%
\pgfsys@transformshift{-1.691872in}{1.700000in}%
\pgfsys@useobject{currentmarker}{}%
\end{pgfscope}%
\begin{pgfscope}%
\pgfsys@transformshift{-1.656250in}{1.300000in}%
\pgfsys@useobject{currentmarker}{}%
\end{pgfscope}%
\begin{pgfscope}%
\pgfsys@transformshift{-1.620628in}{1.600000in}%
\pgfsys@useobject{currentmarker}{}%
\end{pgfscope}%
\begin{pgfscope}%
\pgfsys@transformshift{-1.585007in}{1.500000in}%
\pgfsys@useobject{currentmarker}{}%
\end{pgfscope}%
\begin{pgfscope}%
\pgfsys@transformshift{-1.549385in}{1.500000in}%
\pgfsys@useobject{currentmarker}{}%
\end{pgfscope}%
\begin{pgfscope}%
\pgfsys@transformshift{-1.513763in}{1.700000in}%
\pgfsys@useobject{currentmarker}{}%
\end{pgfscope}%
\begin{pgfscope}%
\pgfsys@transformshift{-1.478142in}{1.500000in}%
\pgfsys@useobject{currentmarker}{}%
\end{pgfscope}%
\begin{pgfscope}%
\pgfsys@transformshift{-1.442520in}{1.500000in}%
\pgfsys@useobject{currentmarker}{}%
\end{pgfscope}%
\begin{pgfscope}%
\pgfsys@transformshift{-1.406898in}{1.500000in}%
\pgfsys@useobject{currentmarker}{}%
\end{pgfscope}%
\begin{pgfscope}%
\pgfsys@transformshift{-1.371277in}{1.500000in}%
\pgfsys@useobject{currentmarker}{}%
\end{pgfscope}%
\begin{pgfscope}%
\pgfsys@transformshift{-1.335655in}{1.600000in}%
\pgfsys@useobject{currentmarker}{}%
\end{pgfscope}%
\begin{pgfscope}%
\pgfsys@transformshift{-1.300034in}{1.500000in}%
\pgfsys@useobject{currentmarker}{}%
\end{pgfscope}%
\begin{pgfscope}%
\pgfsys@transformshift{-1.264412in}{1.500000in}%
\pgfsys@useobject{currentmarker}{}%
\end{pgfscope}%
\begin{pgfscope}%
\pgfsys@transformshift{-1.228790in}{1.400000in}%
\pgfsys@useobject{currentmarker}{}%
\end{pgfscope}%
\begin{pgfscope}%
\pgfsys@transformshift{-1.193169in}{1.500000in}%
\pgfsys@useobject{currentmarker}{}%
\end{pgfscope}%
\begin{pgfscope}%
\pgfsys@transformshift{-1.157547in}{1.700000in}%
\pgfsys@useobject{currentmarker}{}%
\end{pgfscope}%
\begin{pgfscope}%
\pgfsys@transformshift{-1.121925in}{1.400000in}%
\pgfsys@useobject{currentmarker}{}%
\end{pgfscope}%
\begin{pgfscope}%
\pgfsys@transformshift{-1.086304in}{1.400000in}%
\pgfsys@useobject{currentmarker}{}%
\end{pgfscope}%
\begin{pgfscope}%
\pgfsys@transformshift{-1.050682in}{1.500000in}%
\pgfsys@useobject{currentmarker}{}%
\end{pgfscope}%
\begin{pgfscope}%
\pgfsys@transformshift{-1.015061in}{1.500000in}%
\pgfsys@useobject{currentmarker}{}%
\end{pgfscope}%
\begin{pgfscope}%
\pgfsys@transformshift{-0.979439in}{1.600000in}%
\pgfsys@useobject{currentmarker}{}%
\end{pgfscope}%
\begin{pgfscope}%
\pgfsys@transformshift{-0.943817in}{1.500000in}%
\pgfsys@useobject{currentmarker}{}%
\end{pgfscope}%
\begin{pgfscope}%
\pgfsys@transformshift{-0.908196in}{1.600000in}%
\pgfsys@useobject{currentmarker}{}%
\end{pgfscope}%
\begin{pgfscope}%
\pgfsys@transformshift{-0.872574in}{1.400000in}%
\pgfsys@useobject{currentmarker}{}%
\end{pgfscope}%
\begin{pgfscope}%
\pgfsys@transformshift{-0.836952in}{1.500000in}%
\pgfsys@useobject{currentmarker}{}%
\end{pgfscope}%
\begin{pgfscope}%
\pgfsys@transformshift{-0.801331in}{1.500000in}%
\pgfsys@useobject{currentmarker}{}%
\end{pgfscope}%
\begin{pgfscope}%
\pgfsys@transformshift{-0.765709in}{1.500000in}%
\pgfsys@useobject{currentmarker}{}%
\end{pgfscope}%
\begin{pgfscope}%
\pgfsys@transformshift{-0.730088in}{1.500000in}%
\pgfsys@useobject{currentmarker}{}%
\end{pgfscope}%
\begin{pgfscope}%
\pgfsys@transformshift{-0.694466in}{1.500000in}%
\pgfsys@useobject{currentmarker}{}%
\end{pgfscope}%
\begin{pgfscope}%
\pgfsys@transformshift{-0.658844in}{1.500000in}%
\pgfsys@useobject{currentmarker}{}%
\end{pgfscope}%
\begin{pgfscope}%
\pgfsys@transformshift{-0.623223in}{1.500000in}%
\pgfsys@useobject{currentmarker}{}%
\end{pgfscope}%
\begin{pgfscope}%
\pgfsys@transformshift{-0.587601in}{1.500000in}%
\pgfsys@useobject{currentmarker}{}%
\end{pgfscope}%
\begin{pgfscope}%
\pgfsys@transformshift{-0.551979in}{1.400000in}%
\pgfsys@useobject{currentmarker}{}%
\end{pgfscope}%
\begin{pgfscope}%
\pgfsys@transformshift{-0.516358in}{1.600000in}%
\pgfsys@useobject{currentmarker}{}%
\end{pgfscope}%
\begin{pgfscope}%
\pgfsys@transformshift{-0.480736in}{1.500000in}%
\pgfsys@useobject{currentmarker}{}%
\end{pgfscope}%
\begin{pgfscope}%
\pgfsys@transformshift{-0.445115in}{1.500000in}%
\pgfsys@useobject{currentmarker}{}%
\end{pgfscope}%
\begin{pgfscope}%
\pgfsys@transformshift{-0.409493in}{1.400000in}%
\pgfsys@useobject{currentmarker}{}%
\end{pgfscope}%
\begin{pgfscope}%
\pgfsys@transformshift{-0.373871in}{1.700000in}%
\pgfsys@useobject{currentmarker}{}%
\end{pgfscope}%
\begin{pgfscope}%
\pgfsys@transformshift{-0.338250in}{1.600000in}%
\pgfsys@useobject{currentmarker}{}%
\end{pgfscope}%
\begin{pgfscope}%
\pgfsys@transformshift{-0.302628in}{1.500000in}%
\pgfsys@useobject{currentmarker}{}%
\end{pgfscope}%
\begin{pgfscope}%
\pgfsys@transformshift{-0.267006in}{1.400000in}%
\pgfsys@useobject{currentmarker}{}%
\end{pgfscope}%
\begin{pgfscope}%
\pgfsys@transformshift{-0.231385in}{1.600000in}%
\pgfsys@useobject{currentmarker}{}%
\end{pgfscope}%
\begin{pgfscope}%
\pgfsys@transformshift{-0.195763in}{1.400000in}%
\pgfsys@useobject{currentmarker}{}%
\end{pgfscope}%
\begin{pgfscope}%
\pgfsys@transformshift{-0.160142in}{1.400000in}%
\pgfsys@useobject{currentmarker}{}%
\end{pgfscope}%
\begin{pgfscope}%
\pgfsys@transformshift{-0.124520in}{1.500000in}%
\pgfsys@useobject{currentmarker}{}%
\end{pgfscope}%
\begin{pgfscope}%
\pgfsys@transformshift{-0.088898in}{1.500000in}%
\pgfsys@useobject{currentmarker}{}%
\end{pgfscope}%
\begin{pgfscope}%
\pgfsys@transformshift{-0.053277in}{1.600000in}%
\pgfsys@useobject{currentmarker}{}%
\end{pgfscope}%
\begin{pgfscope}%
\pgfsys@transformshift{-0.017655in}{1.400000in}%
\pgfsys@useobject{currentmarker}{}%
\end{pgfscope}%
\begin{pgfscope}%
\pgfsys@transformshift{0.017967in}{1.600000in}%
\pgfsys@useobject{currentmarker}{}%
\end{pgfscope}%
\begin{pgfscope}%
\pgfsys@transformshift{0.053588in}{1.500000in}%
\pgfsys@useobject{currentmarker}{}%
\end{pgfscope}%
\begin{pgfscope}%
\pgfsys@transformshift{0.089210in}{1.500000in}%
\pgfsys@useobject{currentmarker}{}%
\end{pgfscope}%
\begin{pgfscope}%
\pgfsys@transformshift{0.124832in}{1.500000in}%
\pgfsys@useobject{currentmarker}{}%
\end{pgfscope}%
\begin{pgfscope}%
\pgfsys@transformshift{0.160453in}{1.500000in}%
\pgfsys@useobject{currentmarker}{}%
\end{pgfscope}%
\begin{pgfscope}%
\pgfsys@transformshift{0.196075in}{1.600000in}%
\pgfsys@useobject{currentmarker}{}%
\end{pgfscope}%
\begin{pgfscope}%
\pgfsys@transformshift{0.231696in}{1.200000in}%
\pgfsys@useobject{currentmarker}{}%
\end{pgfscope}%
\begin{pgfscope}%
\pgfsys@transformshift{0.267318in}{1.500000in}%
\pgfsys@useobject{currentmarker}{}%
\end{pgfscope}%
\begin{pgfscope}%
\pgfsys@transformshift{0.302940in}{1.600000in}%
\pgfsys@useobject{currentmarker}{}%
\end{pgfscope}%
\begin{pgfscope}%
\pgfsys@transformshift{0.338561in}{1.500000in}%
\pgfsys@useobject{currentmarker}{}%
\end{pgfscope}%
\begin{pgfscope}%
\pgfsys@transformshift{0.374183in}{1.500000in}%
\pgfsys@useobject{currentmarker}{}%
\end{pgfscope}%
\begin{pgfscope}%
\pgfsys@transformshift{0.409805in}{1.500000in}%
\pgfsys@useobject{currentmarker}{}%
\end{pgfscope}%
\begin{pgfscope}%
\pgfsys@transformshift{0.445426in}{1.600000in}%
\pgfsys@useobject{currentmarker}{}%
\end{pgfscope}%
\begin{pgfscope}%
\pgfsys@transformshift{0.481048in}{1.400000in}%
\pgfsys@useobject{currentmarker}{}%
\end{pgfscope}%
\begin{pgfscope}%
\pgfsys@transformshift{0.516669in}{1.600000in}%
\pgfsys@useobject{currentmarker}{}%
\end{pgfscope}%
\begin{pgfscope}%
\pgfsys@transformshift{0.552291in}{1.600000in}%
\pgfsys@useobject{currentmarker}{}%
\end{pgfscope}%
\begin{pgfscope}%
\pgfsys@transformshift{0.587913in}{1.300000in}%
\pgfsys@useobject{currentmarker}{}%
\end{pgfscope}%
\begin{pgfscope}%
\pgfsys@transformshift{0.623534in}{1.400000in}%
\pgfsys@useobject{currentmarker}{}%
\end{pgfscope}%
\begin{pgfscope}%
\pgfsys@transformshift{0.659156in}{1.500000in}%
\pgfsys@useobject{currentmarker}{}%
\end{pgfscope}%
\begin{pgfscope}%
\pgfsys@transformshift{0.694778in}{1.700000in}%
\pgfsys@useobject{currentmarker}{}%
\end{pgfscope}%
\begin{pgfscope}%
\pgfsys@transformshift{0.730399in}{1.600000in}%
\pgfsys@useobject{currentmarker}{}%
\end{pgfscope}%
\begin{pgfscope}%
\pgfsys@transformshift{0.766021in}{1.400000in}%
\pgfsys@useobject{currentmarker}{}%
\end{pgfscope}%
\begin{pgfscope}%
\pgfsys@transformshift{0.801642in}{1.600000in}%
\pgfsys@useobject{currentmarker}{}%
\end{pgfscope}%
\begin{pgfscope}%
\pgfsys@transformshift{0.837264in}{1.500000in}%
\pgfsys@useobject{currentmarker}{}%
\end{pgfscope}%
\begin{pgfscope}%
\pgfsys@transformshift{0.872886in}{1.400000in}%
\pgfsys@useobject{currentmarker}{}%
\end{pgfscope}%
\begin{pgfscope}%
\pgfsys@transformshift{0.908507in}{1.400000in}%
\pgfsys@useobject{currentmarker}{}%
\end{pgfscope}%
\begin{pgfscope}%
\pgfsys@transformshift{0.944129in}{1.500000in}%
\pgfsys@useobject{currentmarker}{}%
\end{pgfscope}%
\begin{pgfscope}%
\pgfsys@transformshift{0.979751in}{1.700000in}%
\pgfsys@useobject{currentmarker}{}%
\end{pgfscope}%
\begin{pgfscope}%
\pgfsys@transformshift{1.015372in}{1.300000in}%
\pgfsys@useobject{currentmarker}{}%
\end{pgfscope}%
\begin{pgfscope}%
\pgfsys@transformshift{1.050994in}{1.600000in}%
\pgfsys@useobject{currentmarker}{}%
\end{pgfscope}%
\begin{pgfscope}%
\pgfsys@transformshift{1.086615in}{1.500000in}%
\pgfsys@useobject{currentmarker}{}%
\end{pgfscope}%
\begin{pgfscope}%
\pgfsys@transformshift{1.122237in}{1.400000in}%
\pgfsys@useobject{currentmarker}{}%
\end{pgfscope}%
\begin{pgfscope}%
\pgfsys@transformshift{1.157859in}{1.600000in}%
\pgfsys@useobject{currentmarker}{}%
\end{pgfscope}%
\begin{pgfscope}%
\pgfsys@transformshift{1.193480in}{1.500000in}%
\pgfsys@useobject{currentmarker}{}%
\end{pgfscope}%
\begin{pgfscope}%
\pgfsys@transformshift{1.229102in}{1.500000in}%
\pgfsys@useobject{currentmarker}{}%
\end{pgfscope}%
\begin{pgfscope}%
\pgfsys@transformshift{1.264724in}{1.500000in}%
\pgfsys@useobject{currentmarker}{}%
\end{pgfscope}%
\begin{pgfscope}%
\pgfsys@transformshift{1.300345in}{1.400000in}%
\pgfsys@useobject{currentmarker}{}%
\end{pgfscope}%
\begin{pgfscope}%
\pgfsys@transformshift{1.335967in}{1.400000in}%
\pgfsys@useobject{currentmarker}{}%
\end{pgfscope}%
\begin{pgfscope}%
\pgfsys@transformshift{1.371588in}{1.400000in}%
\pgfsys@useobject{currentmarker}{}%
\end{pgfscope}%
\begin{pgfscope}%
\pgfsys@transformshift{1.407210in}{1.500000in}%
\pgfsys@useobject{currentmarker}{}%
\end{pgfscope}%
\begin{pgfscope}%
\pgfsys@transformshift{1.442832in}{1.600000in}%
\pgfsys@useobject{currentmarker}{}%
\end{pgfscope}%
\begin{pgfscope}%
\pgfsys@transformshift{1.478453in}{1.700000in}%
\pgfsys@useobject{currentmarker}{}%
\end{pgfscope}%
\begin{pgfscope}%
\pgfsys@transformshift{1.514075in}{1.400000in}%
\pgfsys@useobject{currentmarker}{}%
\end{pgfscope}%
\begin{pgfscope}%
\pgfsys@transformshift{1.549697in}{1.400000in}%
\pgfsys@useobject{currentmarker}{}%
\end{pgfscope}%
\begin{pgfscope}%
\pgfsys@transformshift{1.585318in}{1.500000in}%
\pgfsys@useobject{currentmarker}{}%
\end{pgfscope}%
\begin{pgfscope}%
\pgfsys@transformshift{1.620940in}{1.500000in}%
\pgfsys@useobject{currentmarker}{}%
\end{pgfscope}%
\begin{pgfscope}%
\pgfsys@transformshift{1.656562in}{1.500000in}%
\pgfsys@useobject{currentmarker}{}%
\end{pgfscope}%
\begin{pgfscope}%
\pgfsys@transformshift{1.692183in}{1.600000in}%
\pgfsys@useobject{currentmarker}{}%
\end{pgfscope}%
\begin{pgfscope}%
\pgfsys@transformshift{1.727805in}{1.400000in}%
\pgfsys@useobject{currentmarker}{}%
\end{pgfscope}%
\begin{pgfscope}%
\pgfsys@transformshift{1.763426in}{1.400000in}%
\pgfsys@useobject{currentmarker}{}%
\end{pgfscope}%
\begin{pgfscope}%
\pgfsys@transformshift{1.799048in}{1.400000in}%
\pgfsys@useobject{currentmarker}{}%
\end{pgfscope}%
\begin{pgfscope}%
\pgfsys@transformshift{1.834670in}{1.500000in}%
\pgfsys@useobject{currentmarker}{}%
\end{pgfscope}%
\begin{pgfscope}%
\pgfsys@transformshift{1.870291in}{1.500000in}%
\pgfsys@useobject{currentmarker}{}%
\end{pgfscope}%
\begin{pgfscope}%
\pgfsys@transformshift{1.905913in}{1.500000in}%
\pgfsys@useobject{currentmarker}{}%
\end{pgfscope}%
\begin{pgfscope}%
\pgfsys@transformshift{1.941535in}{1.400000in}%
\pgfsys@useobject{currentmarker}{}%
\end{pgfscope}%
\begin{pgfscope}%
\pgfsys@transformshift{1.977156in}{1.600000in}%
\pgfsys@useobject{currentmarker}{}%
\end{pgfscope}%
\begin{pgfscope}%
\pgfsys@transformshift{2.012778in}{1.600000in}%
\pgfsys@useobject{currentmarker}{}%
\end{pgfscope}%
\begin{pgfscope}%
\pgfsys@transformshift{2.048399in}{1.600000in}%
\pgfsys@useobject{currentmarker}{}%
\end{pgfscope}%
\begin{pgfscope}%
\pgfsys@transformshift{2.084021in}{1.400000in}%
\pgfsys@useobject{currentmarker}{}%
\end{pgfscope}%
\begin{pgfscope}%
\pgfsys@transformshift{2.119643in}{1.600000in}%
\pgfsys@useobject{currentmarker}{}%
\end{pgfscope}%
\begin{pgfscope}%
\pgfsys@transformshift{2.155264in}{1.600000in}%
\pgfsys@useobject{currentmarker}{}%
\end{pgfscope}%
\begin{pgfscope}%
\pgfsys@transformshift{2.190886in}{1.400000in}%
\pgfsys@useobject{currentmarker}{}%
\end{pgfscope}%
\begin{pgfscope}%
\pgfsys@transformshift{2.226508in}{1.500000in}%
\pgfsys@useobject{currentmarker}{}%
\end{pgfscope}%
\begin{pgfscope}%
\pgfsys@transformshift{2.262129in}{1.500000in}%
\pgfsys@useobject{currentmarker}{}%
\end{pgfscope}%
\begin{pgfscope}%
\pgfsys@transformshift{2.297751in}{1.500000in}%
\pgfsys@useobject{currentmarker}{}%
\end{pgfscope}%
\begin{pgfscope}%
\pgfsys@transformshift{2.333372in}{1.500000in}%
\pgfsys@useobject{currentmarker}{}%
\end{pgfscope}%
\begin{pgfscope}%
\pgfsys@transformshift{2.368994in}{1.600000in}%
\pgfsys@useobject{currentmarker}{}%
\end{pgfscope}%
\begin{pgfscope}%
\pgfsys@transformshift{2.404616in}{1.400000in}%
\pgfsys@useobject{currentmarker}{}%
\end{pgfscope}%
\begin{pgfscope}%
\pgfsys@transformshift{2.440237in}{1.600000in}%
\pgfsys@useobject{currentmarker}{}%
\end{pgfscope}%
\begin{pgfscope}%
\pgfsys@transformshift{2.475859in}{1.600000in}%
\pgfsys@useobject{currentmarker}{}%
\end{pgfscope}%
\begin{pgfscope}%
\pgfsys@transformshift{2.511481in}{1.500000in}%
\pgfsys@useobject{currentmarker}{}%
\end{pgfscope}%
\begin{pgfscope}%
\pgfsys@transformshift{2.547102in}{1.500000in}%
\pgfsys@useobject{currentmarker}{}%
\end{pgfscope}%
\begin{pgfscope}%
\pgfsys@transformshift{2.582724in}{1.500000in}%
\pgfsys@useobject{currentmarker}{}%
\end{pgfscope}%
\begin{pgfscope}%
\pgfsys@transformshift{2.618345in}{1.600000in}%
\pgfsys@useobject{currentmarker}{}%
\end{pgfscope}%
\begin{pgfscope}%
\pgfsys@transformshift{2.653967in}{1.400000in}%
\pgfsys@useobject{currentmarker}{}%
\end{pgfscope}%
\begin{pgfscope}%
\pgfsys@transformshift{2.689589in}{1.500000in}%
\pgfsys@useobject{currentmarker}{}%
\end{pgfscope}%
\begin{pgfscope}%
\pgfsys@transformshift{2.725210in}{1.400000in}%
\pgfsys@useobject{currentmarker}{}%
\end{pgfscope}%
\begin{pgfscope}%
\pgfsys@transformshift{2.760832in}{1.500000in}%
\pgfsys@useobject{currentmarker}{}%
\end{pgfscope}%
\begin{pgfscope}%
\pgfsys@transformshift{2.796454in}{1.400271in}%
\pgfsys@useobject{currentmarker}{}%
\end{pgfscope}%
\begin{pgfscope}%
\pgfsys@transformshift{2.832075in}{1.433070in}%
\pgfsys@useobject{currentmarker}{}%
\end{pgfscope}%
\begin{pgfscope}%
\pgfsys@transformshift{2.867697in}{1.625474in}%
\pgfsys@useobject{currentmarker}{}%
\end{pgfscope}%
\begin{pgfscope}%
\pgfsys@transformshift{2.903318in}{1.590724in}%
\pgfsys@useobject{currentmarker}{}%
\end{pgfscope}%
\begin{pgfscope}%
\pgfsys@transformshift{2.938940in}{1.492720in}%
\pgfsys@useobject{currentmarker}{}%
\end{pgfscope}%
\begin{pgfscope}%
\pgfsys@transformshift{2.974562in}{1.708626in}%
\pgfsys@useobject{currentmarker}{}%
\end{pgfscope}%
\begin{pgfscope}%
\pgfsys@transformshift{3.010183in}{1.390093in}%
\pgfsys@useobject{currentmarker}{}%
\end{pgfscope}%
\begin{pgfscope}%
\pgfsys@transformshift{3.045805in}{1.444190in}%
\pgfsys@useobject{currentmarker}{}%
\end{pgfscope}%
\begin{pgfscope}%
\pgfsys@transformshift{3.081427in}{1.502147in}%
\pgfsys@useobject{currentmarker}{}%
\end{pgfscope}%
\begin{pgfscope}%
\pgfsys@transformshift{3.117048in}{1.423872in}%
\pgfsys@useobject{currentmarker}{}%
\end{pgfscope}%
\begin{pgfscope}%
\pgfsys@transformshift{3.152670in}{1.558206in}%
\pgfsys@useobject{currentmarker}{}%
\end{pgfscope}%
\begin{pgfscope}%
\pgfsys@transformshift{3.188292in}{1.317287in}%
\pgfsys@useobject{currentmarker}{}%
\end{pgfscope}%
\begin{pgfscope}%
\pgfsys@transformshift{3.223913in}{1.716629in}%
\pgfsys@useobject{currentmarker}{}%
\end{pgfscope}%
\begin{pgfscope}%
\pgfsys@transformshift{3.259535in}{1.381439in}%
\pgfsys@useobject{currentmarker}{}%
\end{pgfscope}%
\begin{pgfscope}%
\pgfsys@transformshift{3.295156in}{1.499715in}%
\pgfsys@useobject{currentmarker}{}%
\end{pgfscope}%
\begin{pgfscope}%
\pgfsys@transformshift{3.330778in}{1.547673in}%
\pgfsys@useobject{currentmarker}{}%
\end{pgfscope}%
\begin{pgfscope}%
\pgfsys@transformshift{3.366400in}{1.353988in}%
\pgfsys@useobject{currentmarker}{}%
\end{pgfscope}%
\begin{pgfscope}%
\pgfsys@transformshift{3.402021in}{1.464429in}%
\pgfsys@useobject{currentmarker}{}%
\end{pgfscope}%
\begin{pgfscope}%
\pgfsys@transformshift{3.437643in}{1.622461in}%
\pgfsys@useobject{currentmarker}{}%
\end{pgfscope}%
\begin{pgfscope}%
\pgfsys@transformshift{3.473265in}{1.497682in}%
\pgfsys@useobject{currentmarker}{}%
\end{pgfscope}%
\begin{pgfscope}%
\pgfsys@transformshift{3.508886in}{1.655973in}%
\pgfsys@useobject{currentmarker}{}%
\end{pgfscope}%
\begin{pgfscope}%
\pgfsys@transformshift{3.544508in}{1.425767in}%
\pgfsys@useobject{currentmarker}{}%
\end{pgfscope}%
\begin{pgfscope}%
\pgfsys@transformshift{3.580129in}{1.360382in}%
\pgfsys@useobject{currentmarker}{}%
\end{pgfscope}%
\begin{pgfscope}%
\pgfsys@transformshift{3.615751in}{1.723986in}%
\pgfsys@useobject{currentmarker}{}%
\end{pgfscope}%
\begin{pgfscope}%
\pgfsys@transformshift{3.651373in}{1.689592in}%
\pgfsys@useobject{currentmarker}{}%
\end{pgfscope}%
\begin{pgfscope}%
\pgfsys@transformshift{3.686994in}{1.472627in}%
\pgfsys@useobject{currentmarker}{}%
\end{pgfscope}%
\begin{pgfscope}%
\pgfsys@transformshift{3.722616in}{1.317097in}%
\pgfsys@useobject{currentmarker}{}%
\end{pgfscope}%
\begin{pgfscope}%
\pgfsys@transformshift{3.758238in}{1.342491in}%
\pgfsys@useobject{currentmarker}{}%
\end{pgfscope}%
\begin{pgfscope}%
\pgfsys@transformshift{3.793859in}{1.373342in}%
\pgfsys@useobject{currentmarker}{}%
\end{pgfscope}%
\begin{pgfscope}%
\pgfsys@transformshift{3.829481in}{1.589280in}%
\pgfsys@useobject{currentmarker}{}%
\end{pgfscope}%
\begin{pgfscope}%
\pgfsys@transformshift{3.865102in}{1.509521in}%
\pgfsys@useobject{currentmarker}{}%
\end{pgfscope}%
\begin{pgfscope}%
\pgfsys@transformshift{3.900724in}{1.467756in}%
\pgfsys@useobject{currentmarker}{}%
\end{pgfscope}%
\begin{pgfscope}%
\pgfsys@transformshift{3.936346in}{1.596481in}%
\pgfsys@useobject{currentmarker}{}%
\end{pgfscope}%
\begin{pgfscope}%
\pgfsys@transformshift{3.971967in}{1.520711in}%
\pgfsys@useobject{currentmarker}{}%
\end{pgfscope}%
\begin{pgfscope}%
\pgfsys@transformshift{4.007589in}{1.357074in}%
\pgfsys@useobject{currentmarker}{}%
\end{pgfscope}%
\begin{pgfscope}%
\pgfsys@transformshift{4.043211in}{1.415154in}%
\pgfsys@useobject{currentmarker}{}%
\end{pgfscope}%
\begin{pgfscope}%
\pgfsys@transformshift{4.078832in}{1.599399in}%
\pgfsys@useobject{currentmarker}{}%
\end{pgfscope}%
\begin{pgfscope}%
\pgfsys@transformshift{4.114454in}{1.610854in}%
\pgfsys@useobject{currentmarker}{}%
\end{pgfscope}%
\begin{pgfscope}%
\pgfsys@transformshift{4.150075in}{1.648503in}%
\pgfsys@useobject{currentmarker}{}%
\end{pgfscope}%
\begin{pgfscope}%
\pgfsys@transformshift{4.185697in}{1.610516in}%
\pgfsys@useobject{currentmarker}{}%
\end{pgfscope}%
\begin{pgfscope}%
\pgfsys@transformshift{4.221319in}{1.531113in}%
\pgfsys@useobject{currentmarker}{}%
\end{pgfscope}%
\begin{pgfscope}%
\pgfsys@transformshift{4.256940in}{1.652729in}%
\pgfsys@useobject{currentmarker}{}%
\end{pgfscope}%
\begin{pgfscope}%
\pgfsys@transformshift{4.292562in}{1.687853in}%
\pgfsys@useobject{currentmarker}{}%
\end{pgfscope}%
\begin{pgfscope}%
\pgfsys@transformshift{4.328184in}{1.578572in}%
\pgfsys@useobject{currentmarker}{}%
\end{pgfscope}%
\begin{pgfscope}%
\pgfsys@transformshift{4.363805in}{1.482403in}%
\pgfsys@useobject{currentmarker}{}%
\end{pgfscope}%
\begin{pgfscope}%
\pgfsys@transformshift{4.399427in}{1.642243in}%
\pgfsys@useobject{currentmarker}{}%
\end{pgfscope}%
\begin{pgfscope}%
\pgfsys@transformshift{4.435048in}{1.264579in}%
\pgfsys@useobject{currentmarker}{}%
\end{pgfscope}%
\begin{pgfscope}%
\pgfsys@transformshift{4.470670in}{1.383498in}%
\pgfsys@useobject{currentmarker}{}%
\end{pgfscope}%
\begin{pgfscope}%
\pgfsys@transformshift{4.506292in}{1.537277in}%
\pgfsys@useobject{currentmarker}{}%
\end{pgfscope}%
\begin{pgfscope}%
\pgfsys@transformshift{4.541913in}{1.557947in}%
\pgfsys@useobject{currentmarker}{}%
\end{pgfscope}%
\begin{pgfscope}%
\pgfsys@transformshift{4.577535in}{1.468579in}%
\pgfsys@useobject{currentmarker}{}%
\end{pgfscope}%
\begin{pgfscope}%
\pgfsys@transformshift{4.613157in}{1.484050in}%
\pgfsys@useobject{currentmarker}{}%
\end{pgfscope}%
\begin{pgfscope}%
\pgfsys@transformshift{4.648778in}{1.512925in}%
\pgfsys@useobject{currentmarker}{}%
\end{pgfscope}%
\begin{pgfscope}%
\pgfsys@transformshift{4.684400in}{1.359358in}%
\pgfsys@useobject{currentmarker}{}%
\end{pgfscope}%
\begin{pgfscope}%
\pgfsys@transformshift{4.720021in}{1.624648in}%
\pgfsys@useobject{currentmarker}{}%
\end{pgfscope}%
\begin{pgfscope}%
\pgfsys@transformshift{4.755643in}{1.508379in}%
\pgfsys@useobject{currentmarker}{}%
\end{pgfscope}%
\begin{pgfscope}%
\pgfsys@transformshift{4.791265in}{1.509195in}%
\pgfsys@useobject{currentmarker}{}%
\end{pgfscope}%
\begin{pgfscope}%
\pgfsys@transformshift{4.826886in}{1.525302in}%
\pgfsys@useobject{currentmarker}{}%
\end{pgfscope}%
\begin{pgfscope}%
\pgfsys@transformshift{4.862508in}{1.454780in}%
\pgfsys@useobject{currentmarker}{}%
\end{pgfscope}%
\begin{pgfscope}%
\pgfsys@transformshift{4.898130in}{1.495762in}%
\pgfsys@useobject{currentmarker}{}%
\end{pgfscope}%
\begin{pgfscope}%
\pgfsys@transformshift{4.933751in}{1.446525in}%
\pgfsys@useobject{currentmarker}{}%
\end{pgfscope}%
\begin{pgfscope}%
\pgfsys@transformshift{4.969373in}{1.405534in}%
\pgfsys@useobject{currentmarker}{}%
\end{pgfscope}%
\begin{pgfscope}%
\pgfsys@transformshift{5.004995in}{1.771454in}%
\pgfsys@useobject{currentmarker}{}%
\end{pgfscope}%
\begin{pgfscope}%
\pgfsys@transformshift{5.040616in}{1.543139in}%
\pgfsys@useobject{currentmarker}{}%
\end{pgfscope}%
\begin{pgfscope}%
\pgfsys@transformshift{5.076238in}{1.519620in}%
\pgfsys@useobject{currentmarker}{}%
\end{pgfscope}%
\begin{pgfscope}%
\pgfsys@transformshift{5.111859in}{1.600080in}%
\pgfsys@useobject{currentmarker}{}%
\end{pgfscope}%
\begin{pgfscope}%
\pgfsys@transformshift{5.147481in}{1.483839in}%
\pgfsys@useobject{currentmarker}{}%
\end{pgfscope}%
\begin{pgfscope}%
\pgfsys@transformshift{5.183103in}{1.470329in}%
\pgfsys@useobject{currentmarker}{}%
\end{pgfscope}%
\begin{pgfscope}%
\pgfsys@transformshift{5.218724in}{1.559082in}%
\pgfsys@useobject{currentmarker}{}%
\end{pgfscope}%
\begin{pgfscope}%
\pgfsys@transformshift{5.254346in}{1.449707in}%
\pgfsys@useobject{currentmarker}{}%
\end{pgfscope}%
\begin{pgfscope}%
\pgfsys@transformshift{5.289968in}{1.541884in}%
\pgfsys@useobject{currentmarker}{}%
\end{pgfscope}%
\begin{pgfscope}%
\pgfsys@transformshift{5.325589in}{1.335348in}%
\pgfsys@useobject{currentmarker}{}%
\end{pgfscope}%
\begin{pgfscope}%
\pgfsys@transformshift{5.361211in}{1.429879in}%
\pgfsys@useobject{currentmarker}{}%
\end{pgfscope}%
\begin{pgfscope}%
\pgfsys@transformshift{5.396832in}{1.325297in}%
\pgfsys@useobject{currentmarker}{}%
\end{pgfscope}%
\begin{pgfscope}%
\pgfsys@transformshift{5.432454in}{1.521451in}%
\pgfsys@useobject{currentmarker}{}%
\end{pgfscope}%
\begin{pgfscope}%
\pgfsys@transformshift{5.468076in}{1.418219in}%
\pgfsys@useobject{currentmarker}{}%
\end{pgfscope}%
\begin{pgfscope}%
\pgfsys@transformshift{5.503697in}{1.615499in}%
\pgfsys@useobject{currentmarker}{}%
\end{pgfscope}%
\begin{pgfscope}%
\pgfsys@transformshift{5.539319in}{1.413206in}%
\pgfsys@useobject{currentmarker}{}%
\end{pgfscope}%
\begin{pgfscope}%
\pgfsys@transformshift{5.574941in}{1.611270in}%
\pgfsys@useobject{currentmarker}{}%
\end{pgfscope}%
\begin{pgfscope}%
\pgfsys@transformshift{5.610562in}{1.609632in}%
\pgfsys@useobject{currentmarker}{}%
\end{pgfscope}%
\begin{pgfscope}%
\pgfsys@transformshift{5.646184in}{1.708245in}%
\pgfsys@useobject{currentmarker}{}%
\end{pgfscope}%
\begin{pgfscope}%
\pgfsys@transformshift{5.681805in}{1.307069in}%
\pgfsys@useobject{currentmarker}{}%
\end{pgfscope}%
\begin{pgfscope}%
\pgfsys@transformshift{5.717427in}{1.406070in}%
\pgfsys@useobject{currentmarker}{}%
\end{pgfscope}%
\begin{pgfscope}%
\pgfsys@transformshift{5.753049in}{1.605219in}%
\pgfsys@useobject{currentmarker}{}%
\end{pgfscope}%
\begin{pgfscope}%
\pgfsys@transformshift{5.788670in}{1.504494in}%
\pgfsys@useobject{currentmarker}{}%
\end{pgfscope}%
\begin{pgfscope}%
\pgfsys@transformshift{5.824292in}{1.303876in}%
\pgfsys@useobject{currentmarker}{}%
\end{pgfscope}%
\begin{pgfscope}%
\pgfsys@transformshift{5.859914in}{1.503347in}%
\pgfsys@useobject{currentmarker}{}%
\end{pgfscope}%
\begin{pgfscope}%
\pgfsys@transformshift{5.895535in}{1.702895in}%
\pgfsys@useobject{currentmarker}{}%
\end{pgfscope}%
\begin{pgfscope}%
\pgfsys@transformshift{5.931157in}{1.502507in}%
\pgfsys@useobject{currentmarker}{}%
\end{pgfscope}%
\begin{pgfscope}%
\pgfsys@transformshift{5.966778in}{1.402174in}%
\pgfsys@useobject{currentmarker}{}%
\end{pgfscope}%
\begin{pgfscope}%
\pgfsys@transformshift{6.002400in}{1.601887in}%
\pgfsys@useobject{currentmarker}{}%
\end{pgfscope}%
\begin{pgfscope}%
\pgfsys@transformshift{6.038022in}{1.401641in}%
\pgfsys@useobject{currentmarker}{}%
\end{pgfscope}%
\begin{pgfscope}%
\pgfsys@transformshift{6.073643in}{1.701429in}%
\pgfsys@useobject{currentmarker}{}%
\end{pgfscope}%
\begin{pgfscope}%
\pgfsys@transformshift{6.109265in}{1.501245in}%
\pgfsys@useobject{currentmarker}{}%
\end{pgfscope}%
\begin{pgfscope}%
\pgfsys@transformshift{6.144887in}{1.501087in}%
\pgfsys@useobject{currentmarker}{}%
\end{pgfscope}%
\begin{pgfscope}%
\pgfsys@transformshift{6.180508in}{1.400950in}%
\pgfsys@useobject{currentmarker}{}%
\end{pgfscope}%
\begin{pgfscope}%
\pgfsys@transformshift{6.216130in}{1.600831in}%
\pgfsys@useobject{currentmarker}{}%
\end{pgfscope}%
\begin{pgfscope}%
\pgfsys@transformshift{6.251751in}{1.500728in}%
\pgfsys@useobject{currentmarker}{}%
\end{pgfscope}%
\begin{pgfscope}%
\pgfsys@transformshift{6.287373in}{1.500638in}%
\pgfsys@useobject{currentmarker}{}%
\end{pgfscope}%
\begin{pgfscope}%
\pgfsys@transformshift{6.322995in}{1.400560in}%
\pgfsys@useobject{currentmarker}{}%
\end{pgfscope}%
\begin{pgfscope}%
\pgfsys@transformshift{6.358616in}{1.500492in}%
\pgfsys@useobject{currentmarker}{}%
\end{pgfscope}%
\begin{pgfscope}%
\pgfsys@transformshift{6.394238in}{1.500433in}%
\pgfsys@useobject{currentmarker}{}%
\end{pgfscope}%
\begin{pgfscope}%
\pgfsys@transformshift{6.429860in}{1.600381in}%
\pgfsys@useobject{currentmarker}{}%
\end{pgfscope}%
\begin{pgfscope}%
\pgfsys@transformshift{6.465481in}{1.500336in}%
\pgfsys@useobject{currentmarker}{}%
\end{pgfscope}%
\begin{pgfscope}%
\pgfsys@transformshift{6.501103in}{1.700297in}%
\pgfsys@useobject{currentmarker}{}%
\end{pgfscope}%
\begin{pgfscope}%
\pgfsys@transformshift{6.536725in}{1.500262in}%
\pgfsys@useobject{currentmarker}{}%
\end{pgfscope}%
\begin{pgfscope}%
\pgfsys@transformshift{6.572346in}{1.500232in}%
\pgfsys@useobject{currentmarker}{}%
\end{pgfscope}%
\begin{pgfscope}%
\pgfsys@transformshift{6.607968in}{1.400205in}%
\pgfsys@useobject{currentmarker}{}%
\end{pgfscope}%
\begin{pgfscope}%
\pgfsys@transformshift{6.643589in}{1.700182in}%
\pgfsys@useobject{currentmarker}{}%
\end{pgfscope}%
\begin{pgfscope}%
\pgfsys@transformshift{6.679211in}{1.300161in}%
\pgfsys@useobject{currentmarker}{}%
\end{pgfscope}%
\begin{pgfscope}%
\pgfsys@transformshift{6.714833in}{1.600143in}%
\pgfsys@useobject{currentmarker}{}%
\end{pgfscope}%
\begin{pgfscope}%
\pgfsys@transformshift{6.750454in}{1.500127in}%
\pgfsys@useobject{currentmarker}{}%
\end{pgfscope}%
\begin{pgfscope}%
\pgfsys@transformshift{6.786076in}{1.300113in}%
\pgfsys@useobject{currentmarker}{}%
\end{pgfscope}%
\begin{pgfscope}%
\pgfsys@transformshift{6.821698in}{1.500101in}%
\pgfsys@useobject{currentmarker}{}%
\end{pgfscope}%
\begin{pgfscope}%
\pgfsys@transformshift{6.857319in}{1.400090in}%
\pgfsys@useobject{currentmarker}{}%
\end{pgfscope}%
\begin{pgfscope}%
\pgfsys@transformshift{6.892941in}{1.400080in}%
\pgfsys@useobject{currentmarker}{}%
\end{pgfscope}%
\begin{pgfscope}%
\pgfsys@transformshift{6.928562in}{1.600071in}%
\pgfsys@useobject{currentmarker}{}%
\end{pgfscope}%
\begin{pgfscope}%
\pgfsys@transformshift{6.964184in}{1.500064in}%
\pgfsys@useobject{currentmarker}{}%
\end{pgfscope}%
\begin{pgfscope}%
\pgfsys@transformshift{6.999806in}{1.500057in}%
\pgfsys@useobject{currentmarker}{}%
\end{pgfscope}%
\begin{pgfscope}%
\pgfsys@transformshift{7.035427in}{1.400051in}%
\pgfsys@useobject{currentmarker}{}%
\end{pgfscope}%
\begin{pgfscope}%
\pgfsys@transformshift{7.071049in}{1.500046in}%
\pgfsys@useobject{currentmarker}{}%
\end{pgfscope}%
\begin{pgfscope}%
\pgfsys@transformshift{7.106671in}{1.600041in}%
\pgfsys@useobject{currentmarker}{}%
\end{pgfscope}%
\begin{pgfscope}%
\pgfsys@transformshift{7.142292in}{1.400037in}%
\pgfsys@useobject{currentmarker}{}%
\end{pgfscope}%
\begin{pgfscope}%
\pgfsys@transformshift{7.177914in}{1.400033in}%
\pgfsys@useobject{currentmarker}{}%
\end{pgfscope}%
\begin{pgfscope}%
\pgfsys@transformshift{7.213535in}{1.300030in}%
\pgfsys@useobject{currentmarker}{}%
\end{pgfscope}%
\begin{pgfscope}%
\pgfsys@transformshift{7.249157in}{1.600027in}%
\pgfsys@useobject{currentmarker}{}%
\end{pgfscope}%
\begin{pgfscope}%
\pgfsys@transformshift{7.284779in}{1.500024in}%
\pgfsys@useobject{currentmarker}{}%
\end{pgfscope}%
\begin{pgfscope}%
\pgfsys@transformshift{7.320400in}{1.700022in}%
\pgfsys@useobject{currentmarker}{}%
\end{pgfscope}%
\begin{pgfscope}%
\pgfsys@transformshift{7.356022in}{1.400019in}%
\pgfsys@useobject{currentmarker}{}%
\end{pgfscope}%
\begin{pgfscope}%
\pgfsys@transformshift{7.391644in}{1.400017in}%
\pgfsys@useobject{currentmarker}{}%
\end{pgfscope}%
\begin{pgfscope}%
\pgfsys@transformshift{7.427265in}{1.500016in}%
\pgfsys@useobject{currentmarker}{}%
\end{pgfscope}%
\begin{pgfscope}%
\pgfsys@transformshift{7.462887in}{1.600014in}%
\pgfsys@useobject{currentmarker}{}%
\end{pgfscope}%
\begin{pgfscope}%
\pgfsys@transformshift{7.498508in}{1.600013in}%
\pgfsys@useobject{currentmarker}{}%
\end{pgfscope}%
\begin{pgfscope}%
\pgfsys@transformshift{7.534130in}{1.400012in}%
\pgfsys@useobject{currentmarker}{}%
\end{pgfscope}%
\begin{pgfscope}%
\pgfsys@transformshift{7.569752in}{1.600011in}%
\pgfsys@useobject{currentmarker}{}%
\end{pgfscope}%
\begin{pgfscope}%
\pgfsys@transformshift{7.605373in}{1.500010in}%
\pgfsys@useobject{currentmarker}{}%
\end{pgfscope}%
\begin{pgfscope}%
\pgfsys@transformshift{7.640995in}{1.600009in}%
\pgfsys@useobject{currentmarker}{}%
\end{pgfscope}%
\begin{pgfscope}%
\pgfsys@transformshift{7.676617in}{1.500008in}%
\pgfsys@useobject{currentmarker}{}%
\end{pgfscope}%
\begin{pgfscope}%
\pgfsys@transformshift{7.712238in}{1.400007in}%
\pgfsys@useobject{currentmarker}{}%
\end{pgfscope}%
\begin{pgfscope}%
\pgfsys@transformshift{7.747860in}{1.500006in}%
\pgfsys@useobject{currentmarker}{}%
\end{pgfscope}%
\begin{pgfscope}%
\pgfsys@transformshift{7.783481in}{1.700006in}%
\pgfsys@useobject{currentmarker}{}%
\end{pgfscope}%
\begin{pgfscope}%
\pgfsys@transformshift{7.819103in}{1.600005in}%
\pgfsys@useobject{currentmarker}{}%
\end{pgfscope}%
\begin{pgfscope}%
\pgfsys@transformshift{7.854725in}{1.500005in}%
\pgfsys@useobject{currentmarker}{}%
\end{pgfscope}%
\begin{pgfscope}%
\pgfsys@transformshift{7.890346in}{1.600004in}%
\pgfsys@useobject{currentmarker}{}%
\end{pgfscope}%
\begin{pgfscope}%
\pgfsys@transformshift{7.925968in}{1.400004in}%
\pgfsys@useobject{currentmarker}{}%
\end{pgfscope}%
\begin{pgfscope}%
\pgfsys@transformshift{7.961590in}{1.500004in}%
\pgfsys@useobject{currentmarker}{}%
\end{pgfscope}%
\begin{pgfscope}%
\pgfsys@transformshift{7.997211in}{1.500003in}%
\pgfsys@useobject{currentmarker}{}%
\end{pgfscope}%
\begin{pgfscope}%
\pgfsys@transformshift{8.032833in}{1.600003in}%
\pgfsys@useobject{currentmarker}{}%
\end{pgfscope}%
\begin{pgfscope}%
\pgfsys@transformshift{8.068455in}{1.700003in}%
\pgfsys@useobject{currentmarker}{}%
\end{pgfscope}%
\begin{pgfscope}%
\pgfsys@transformshift{8.104076in}{1.500002in}%
\pgfsys@useobject{currentmarker}{}%
\end{pgfscope}%
\begin{pgfscope}%
\pgfsys@transformshift{8.139698in}{1.500002in}%
\pgfsys@useobject{currentmarker}{}%
\end{pgfscope}%
\begin{pgfscope}%
\pgfsys@transformshift{8.175319in}{1.500002in}%
\pgfsys@useobject{currentmarker}{}%
\end{pgfscope}%
\begin{pgfscope}%
\pgfsys@transformshift{8.210941in}{1.400002in}%
\pgfsys@useobject{currentmarker}{}%
\end{pgfscope}%
\begin{pgfscope}%
\pgfsys@transformshift{8.246563in}{1.400002in}%
\pgfsys@useobject{currentmarker}{}%
\end{pgfscope}%
\begin{pgfscope}%
\pgfsys@transformshift{8.282184in}{1.600002in}%
\pgfsys@useobject{currentmarker}{}%
\end{pgfscope}%
\begin{pgfscope}%
\pgfsys@transformshift{8.317806in}{1.700001in}%
\pgfsys@useobject{currentmarker}{}%
\end{pgfscope}%
\begin{pgfscope}%
\pgfsys@transformshift{8.353428in}{1.600001in}%
\pgfsys@useobject{currentmarker}{}%
\end{pgfscope}%
\begin{pgfscope}%
\pgfsys@transformshift{8.389049in}{1.400001in}%
\pgfsys@useobject{currentmarker}{}%
\end{pgfscope}%
\begin{pgfscope}%
\pgfsys@transformshift{8.424671in}{1.500001in}%
\pgfsys@useobject{currentmarker}{}%
\end{pgfscope}%
\begin{pgfscope}%
\pgfsys@transformshift{8.460292in}{1.400001in}%
\pgfsys@useobject{currentmarker}{}%
\end{pgfscope}%
\begin{pgfscope}%
\pgfsys@transformshift{8.495914in}{1.600001in}%
\pgfsys@useobject{currentmarker}{}%
\end{pgfscope}%
\begin{pgfscope}%
\pgfsys@transformshift{8.531536in}{1.500001in}%
\pgfsys@useobject{currentmarker}{}%
\end{pgfscope}%
\begin{pgfscope}%
\pgfsys@transformshift{8.567157in}{1.600001in}%
\pgfsys@useobject{currentmarker}{}%
\end{pgfscope}%
\begin{pgfscope}%
\pgfsys@transformshift{8.602779in}{1.500001in}%
\pgfsys@useobject{currentmarker}{}%
\end{pgfscope}%
\begin{pgfscope}%
\pgfsys@transformshift{8.638401in}{1.400001in}%
\pgfsys@useobject{currentmarker}{}%
\end{pgfscope}%
\begin{pgfscope}%
\pgfsys@transformshift{8.674022in}{1.500001in}%
\pgfsys@useobject{currentmarker}{}%
\end{pgfscope}%
\begin{pgfscope}%
\pgfsys@transformshift{8.709644in}{1.500001in}%
\pgfsys@useobject{currentmarker}{}%
\end{pgfscope}%
\begin{pgfscope}%
\pgfsys@transformshift{8.745265in}{1.700001in}%
\pgfsys@useobject{currentmarker}{}%
\end{pgfscope}%
\begin{pgfscope}%
\pgfsys@transformshift{8.780887in}{1.500000in}%
\pgfsys@useobject{currentmarker}{}%
\end{pgfscope}%
\begin{pgfscope}%
\pgfsys@transformshift{8.816509in}{1.600000in}%
\pgfsys@useobject{currentmarker}{}%
\end{pgfscope}%
\begin{pgfscope}%
\pgfsys@transformshift{8.852130in}{1.600000in}%
\pgfsys@useobject{currentmarker}{}%
\end{pgfscope}%
\begin{pgfscope}%
\pgfsys@transformshift{8.887752in}{1.600000in}%
\pgfsys@useobject{currentmarker}{}%
\end{pgfscope}%
\begin{pgfscope}%
\pgfsys@transformshift{8.923374in}{1.500000in}%
\pgfsys@useobject{currentmarker}{}%
\end{pgfscope}%
\begin{pgfscope}%
\pgfsys@transformshift{8.958995in}{1.500000in}%
\pgfsys@useobject{currentmarker}{}%
\end{pgfscope}%
\begin{pgfscope}%
\pgfsys@transformshift{8.994617in}{1.500000in}%
\pgfsys@useobject{currentmarker}{}%
\end{pgfscope}%
\begin{pgfscope}%
\pgfsys@transformshift{9.030238in}{1.500000in}%
\pgfsys@useobject{currentmarker}{}%
\end{pgfscope}%
\begin{pgfscope}%
\pgfsys@transformshift{9.065860in}{1.500000in}%
\pgfsys@useobject{currentmarker}{}%
\end{pgfscope}%
\begin{pgfscope}%
\pgfsys@transformshift{9.101482in}{1.500000in}%
\pgfsys@useobject{currentmarker}{}%
\end{pgfscope}%
\begin{pgfscope}%
\pgfsys@transformshift{9.137103in}{1.500000in}%
\pgfsys@useobject{currentmarker}{}%
\end{pgfscope}%
\begin{pgfscope}%
\pgfsys@transformshift{9.172725in}{1.400000in}%
\pgfsys@useobject{currentmarker}{}%
\end{pgfscope}%
\begin{pgfscope}%
\pgfsys@transformshift{9.208347in}{1.400000in}%
\pgfsys@useobject{currentmarker}{}%
\end{pgfscope}%
\begin{pgfscope}%
\pgfsys@transformshift{9.243968in}{1.600000in}%
\pgfsys@useobject{currentmarker}{}%
\end{pgfscope}%
\begin{pgfscope}%
\pgfsys@transformshift{9.279590in}{1.600000in}%
\pgfsys@useobject{currentmarker}{}%
\end{pgfscope}%
\begin{pgfscope}%
\pgfsys@transformshift{9.315211in}{1.400000in}%
\pgfsys@useobject{currentmarker}{}%
\end{pgfscope}%
\begin{pgfscope}%
\pgfsys@transformshift{9.350833in}{1.600000in}%
\pgfsys@useobject{currentmarker}{}%
\end{pgfscope}%
\begin{pgfscope}%
\pgfsys@transformshift{9.386455in}{1.400000in}%
\pgfsys@useobject{currentmarker}{}%
\end{pgfscope}%
\begin{pgfscope}%
\pgfsys@transformshift{9.422076in}{1.600000in}%
\pgfsys@useobject{currentmarker}{}%
\end{pgfscope}%
\begin{pgfscope}%
\pgfsys@transformshift{9.457698in}{1.500000in}%
\pgfsys@useobject{currentmarker}{}%
\end{pgfscope}%
\begin{pgfscope}%
\pgfsys@transformshift{9.493320in}{1.600000in}%
\pgfsys@useobject{currentmarker}{}%
\end{pgfscope}%
\begin{pgfscope}%
\pgfsys@transformshift{9.528941in}{1.400000in}%
\pgfsys@useobject{currentmarker}{}%
\end{pgfscope}%
\begin{pgfscope}%
\pgfsys@transformshift{9.564563in}{1.400000in}%
\pgfsys@useobject{currentmarker}{}%
\end{pgfscope}%
\begin{pgfscope}%
\pgfsys@transformshift{9.600185in}{1.300000in}%
\pgfsys@useobject{currentmarker}{}%
\end{pgfscope}%
\begin{pgfscope}%
\pgfsys@transformshift{9.635806in}{1.300000in}%
\pgfsys@useobject{currentmarker}{}%
\end{pgfscope}%
\begin{pgfscope}%
\pgfsys@transformshift{9.671428in}{1.600000in}%
\pgfsys@useobject{currentmarker}{}%
\end{pgfscope}%
\begin{pgfscope}%
\pgfsys@transformshift{9.707049in}{1.400000in}%
\pgfsys@useobject{currentmarker}{}%
\end{pgfscope}%
\begin{pgfscope}%
\pgfsys@transformshift{9.742671in}{1.600000in}%
\pgfsys@useobject{currentmarker}{}%
\end{pgfscope}%
\begin{pgfscope}%
\pgfsys@transformshift{9.778293in}{1.300000in}%
\pgfsys@useobject{currentmarker}{}%
\end{pgfscope}%
\begin{pgfscope}%
\pgfsys@transformshift{9.813914in}{1.600000in}%
\pgfsys@useobject{currentmarker}{}%
\end{pgfscope}%
\begin{pgfscope}%
\pgfsys@transformshift{9.849536in}{1.300000in}%
\pgfsys@useobject{currentmarker}{}%
\end{pgfscope}%
\begin{pgfscope}%
\pgfsys@transformshift{9.885158in}{1.600000in}%
\pgfsys@useobject{currentmarker}{}%
\end{pgfscope}%
\begin{pgfscope}%
\pgfsys@transformshift{9.920779in}{1.700000in}%
\pgfsys@useobject{currentmarker}{}%
\end{pgfscope}%
\begin{pgfscope}%
\pgfsys@transformshift{9.956401in}{1.600000in}%
\pgfsys@useobject{currentmarker}{}%
\end{pgfscope}%
\begin{pgfscope}%
\pgfsys@transformshift{9.992022in}{1.500000in}%
\pgfsys@useobject{currentmarker}{}%
\end{pgfscope}%
\begin{pgfscope}%
\pgfsys@transformshift{10.027644in}{1.600000in}%
\pgfsys@useobject{currentmarker}{}%
\end{pgfscope}%
\begin{pgfscope}%
\pgfsys@transformshift{10.063266in}{1.400000in}%
\pgfsys@useobject{currentmarker}{}%
\end{pgfscope}%
\begin{pgfscope}%
\pgfsys@transformshift{10.098887in}{1.800000in}%
\pgfsys@useobject{currentmarker}{}%
\end{pgfscope}%
\begin{pgfscope}%
\pgfsys@transformshift{10.134509in}{1.500000in}%
\pgfsys@useobject{currentmarker}{}%
\end{pgfscope}%
\begin{pgfscope}%
\pgfsys@transformshift{10.170131in}{1.500000in}%
\pgfsys@useobject{currentmarker}{}%
\end{pgfscope}%
\begin{pgfscope}%
\pgfsys@transformshift{10.205752in}{1.500000in}%
\pgfsys@useobject{currentmarker}{}%
\end{pgfscope}%
\begin{pgfscope}%
\pgfsys@transformshift{10.241374in}{1.700000in}%
\pgfsys@useobject{currentmarker}{}%
\end{pgfscope}%
\begin{pgfscope}%
\pgfsys@transformshift{10.276995in}{1.500000in}%
\pgfsys@useobject{currentmarker}{}%
\end{pgfscope}%
\begin{pgfscope}%
\pgfsys@transformshift{10.312617in}{1.400000in}%
\pgfsys@useobject{currentmarker}{}%
\end{pgfscope}%
\begin{pgfscope}%
\pgfsys@transformshift{10.348239in}{1.400000in}%
\pgfsys@useobject{currentmarker}{}%
\end{pgfscope}%
\begin{pgfscope}%
\pgfsys@transformshift{10.383860in}{1.500000in}%
\pgfsys@useobject{currentmarker}{}%
\end{pgfscope}%
\begin{pgfscope}%
\pgfsys@transformshift{10.419482in}{1.600000in}%
\pgfsys@useobject{currentmarker}{}%
\end{pgfscope}%
\begin{pgfscope}%
\pgfsys@transformshift{10.455104in}{1.400000in}%
\pgfsys@useobject{currentmarker}{}%
\end{pgfscope}%
\begin{pgfscope}%
\pgfsys@transformshift{10.490725in}{1.700000in}%
\pgfsys@useobject{currentmarker}{}%
\end{pgfscope}%
\begin{pgfscope}%
\pgfsys@transformshift{10.526347in}{1.600000in}%
\pgfsys@useobject{currentmarker}{}%
\end{pgfscope}%
\begin{pgfscope}%
\pgfsys@transformshift{10.561968in}{1.600000in}%
\pgfsys@useobject{currentmarker}{}%
\end{pgfscope}%
\begin{pgfscope}%
\pgfsys@transformshift{10.597590in}{1.600000in}%
\pgfsys@useobject{currentmarker}{}%
\end{pgfscope}%
\begin{pgfscope}%
\pgfsys@transformshift{10.633212in}{1.300000in}%
\pgfsys@useobject{currentmarker}{}%
\end{pgfscope}%
\begin{pgfscope}%
\pgfsys@transformshift{10.668833in}{1.600000in}%
\pgfsys@useobject{currentmarker}{}%
\end{pgfscope}%
\begin{pgfscope}%
\pgfsys@transformshift{10.704455in}{1.500000in}%
\pgfsys@useobject{currentmarker}{}%
\end{pgfscope}%
\begin{pgfscope}%
\pgfsys@transformshift{10.740077in}{1.500000in}%
\pgfsys@useobject{currentmarker}{}%
\end{pgfscope}%
\begin{pgfscope}%
\pgfsys@transformshift{10.775698in}{1.500000in}%
\pgfsys@useobject{currentmarker}{}%
\end{pgfscope}%
\begin{pgfscope}%
\pgfsys@transformshift{10.811320in}{1.300000in}%
\pgfsys@useobject{currentmarker}{}%
\end{pgfscope}%
\begin{pgfscope}%
\pgfsys@transformshift{10.846941in}{1.500000in}%
\pgfsys@useobject{currentmarker}{}%
\end{pgfscope}%
\begin{pgfscope}%
\pgfsys@transformshift{10.882563in}{1.700000in}%
\pgfsys@useobject{currentmarker}{}%
\end{pgfscope}%
\begin{pgfscope}%
\pgfsys@transformshift{10.918185in}{1.500000in}%
\pgfsys@useobject{currentmarker}{}%
\end{pgfscope}%
\begin{pgfscope}%
\pgfsys@transformshift{10.953806in}{1.400000in}%
\pgfsys@useobject{currentmarker}{}%
\end{pgfscope}%
\begin{pgfscope}%
\pgfsys@transformshift{10.989428in}{1.400000in}%
\pgfsys@useobject{currentmarker}{}%
\end{pgfscope}%
\begin{pgfscope}%
\pgfsys@transformshift{11.025050in}{1.500000in}%
\pgfsys@useobject{currentmarker}{}%
\end{pgfscope}%
\begin{pgfscope}%
\pgfsys@transformshift{11.060671in}{1.500000in}%
\pgfsys@useobject{currentmarker}{}%
\end{pgfscope}%
\begin{pgfscope}%
\pgfsys@transformshift{11.096293in}{1.600000in}%
\pgfsys@useobject{currentmarker}{}%
\end{pgfscope}%
\begin{pgfscope}%
\pgfsys@transformshift{11.131915in}{1.400000in}%
\pgfsys@useobject{currentmarker}{}%
\end{pgfscope}%
\begin{pgfscope}%
\pgfsys@transformshift{11.167536in}{1.400000in}%
\pgfsys@useobject{currentmarker}{}%
\end{pgfscope}%
\begin{pgfscope}%
\pgfsys@transformshift{11.203158in}{1.600000in}%
\pgfsys@useobject{currentmarker}{}%
\end{pgfscope}%
\begin{pgfscope}%
\pgfsys@transformshift{11.238779in}{1.700000in}%
\pgfsys@useobject{currentmarker}{}%
\end{pgfscope}%
\begin{pgfscope}%
\pgfsys@transformshift{11.274401in}{1.500000in}%
\pgfsys@useobject{currentmarker}{}%
\end{pgfscope}%
\begin{pgfscope}%
\pgfsys@transformshift{11.310023in}{1.600000in}%
\pgfsys@useobject{currentmarker}{}%
\end{pgfscope}%
\begin{pgfscope}%
\pgfsys@transformshift{11.345644in}{1.400000in}%
\pgfsys@useobject{currentmarker}{}%
\end{pgfscope}%
\begin{pgfscope}%
\pgfsys@transformshift{11.381266in}{1.500000in}%
\pgfsys@useobject{currentmarker}{}%
\end{pgfscope}%
\begin{pgfscope}%
\pgfsys@transformshift{11.416888in}{1.600000in}%
\pgfsys@useobject{currentmarker}{}%
\end{pgfscope}%
\begin{pgfscope}%
\pgfsys@transformshift{11.452509in}{1.500000in}%
\pgfsys@useobject{currentmarker}{}%
\end{pgfscope}%
\begin{pgfscope}%
\pgfsys@transformshift{11.488131in}{1.300000in}%
\pgfsys@useobject{currentmarker}{}%
\end{pgfscope}%
\begin{pgfscope}%
\pgfsys@transformshift{11.523752in}{1.600000in}%
\pgfsys@useobject{currentmarker}{}%
\end{pgfscope}%
\begin{pgfscope}%
\pgfsys@transformshift{11.559374in}{1.500000in}%
\pgfsys@useobject{currentmarker}{}%
\end{pgfscope}%
\begin{pgfscope}%
\pgfsys@transformshift{11.594996in}{1.500000in}%
\pgfsys@useobject{currentmarker}{}%
\end{pgfscope}%
\begin{pgfscope}%
\pgfsys@transformshift{11.630617in}{1.500000in}%
\pgfsys@useobject{currentmarker}{}%
\end{pgfscope}%
\begin{pgfscope}%
\pgfsys@transformshift{11.666239in}{1.400000in}%
\pgfsys@useobject{currentmarker}{}%
\end{pgfscope}%
\begin{pgfscope}%
\pgfsys@transformshift{11.701861in}{1.500000in}%
\pgfsys@useobject{currentmarker}{}%
\end{pgfscope}%
\begin{pgfscope}%
\pgfsys@transformshift{11.737482in}{1.400000in}%
\pgfsys@useobject{currentmarker}{}%
\end{pgfscope}%
\begin{pgfscope}%
\pgfsys@transformshift{11.773104in}{1.500000in}%
\pgfsys@useobject{currentmarker}{}%
\end{pgfscope}%
\begin{pgfscope}%
\pgfsys@transformshift{11.808725in}{1.400000in}%
\pgfsys@useobject{currentmarker}{}%
\end{pgfscope}%
\begin{pgfscope}%
\pgfsys@transformshift{11.844347in}{1.600000in}%
\pgfsys@useobject{currentmarker}{}%
\end{pgfscope}%
\begin{pgfscope}%
\pgfsys@transformshift{11.879969in}{1.500000in}%
\pgfsys@useobject{currentmarker}{}%
\end{pgfscope}%
\begin{pgfscope}%
\pgfsys@transformshift{11.915590in}{1.400000in}%
\pgfsys@useobject{currentmarker}{}%
\end{pgfscope}%
\begin{pgfscope}%
\pgfsys@transformshift{11.951212in}{1.600000in}%
\pgfsys@useobject{currentmarker}{}%
\end{pgfscope}%
\begin{pgfscope}%
\pgfsys@transformshift{11.986834in}{1.600000in}%
\pgfsys@useobject{currentmarker}{}%
\end{pgfscope}%
\begin{pgfscope}%
\pgfsys@transformshift{12.022455in}{1.200000in}%
\pgfsys@useobject{currentmarker}{}%
\end{pgfscope}%
\begin{pgfscope}%
\pgfsys@transformshift{12.058077in}{1.500000in}%
\pgfsys@useobject{currentmarker}{}%
\end{pgfscope}%
\begin{pgfscope}%
\pgfsys@transformshift{12.093698in}{1.600000in}%
\pgfsys@useobject{currentmarker}{}%
\end{pgfscope}%
\begin{pgfscope}%
\pgfsys@transformshift{12.129320in}{1.400000in}%
\pgfsys@useobject{currentmarker}{}%
\end{pgfscope}%
\begin{pgfscope}%
\pgfsys@transformshift{12.164942in}{1.400000in}%
\pgfsys@useobject{currentmarker}{}%
\end{pgfscope}%
\begin{pgfscope}%
\pgfsys@transformshift{12.200563in}{1.500000in}%
\pgfsys@useobject{currentmarker}{}%
\end{pgfscope}%
\begin{pgfscope}%
\pgfsys@transformshift{12.236185in}{1.700000in}%
\pgfsys@useobject{currentmarker}{}%
\end{pgfscope}%
\begin{pgfscope}%
\pgfsys@transformshift{12.271807in}{1.300000in}%
\pgfsys@useobject{currentmarker}{}%
\end{pgfscope}%
\begin{pgfscope}%
\pgfsys@transformshift{12.307428in}{1.500000in}%
\pgfsys@useobject{currentmarker}{}%
\end{pgfscope}%
\begin{pgfscope}%
\pgfsys@transformshift{12.343050in}{1.500000in}%
\pgfsys@useobject{currentmarker}{}%
\end{pgfscope}%
\begin{pgfscope}%
\pgfsys@transformshift{12.378671in}{1.500000in}%
\pgfsys@useobject{currentmarker}{}%
\end{pgfscope}%
\begin{pgfscope}%
\pgfsys@transformshift{12.414293in}{1.400000in}%
\pgfsys@useobject{currentmarker}{}%
\end{pgfscope}%
\begin{pgfscope}%
\pgfsys@transformshift{12.449915in}{1.400000in}%
\pgfsys@useobject{currentmarker}{}%
\end{pgfscope}%
\begin{pgfscope}%
\pgfsys@transformshift{12.485536in}{1.600000in}%
\pgfsys@useobject{currentmarker}{}%
\end{pgfscope}%
\begin{pgfscope}%
\pgfsys@transformshift{12.521158in}{1.500000in}%
\pgfsys@useobject{currentmarker}{}%
\end{pgfscope}%
\begin{pgfscope}%
\pgfsys@transformshift{12.556780in}{1.500000in}%
\pgfsys@useobject{currentmarker}{}%
\end{pgfscope}%
\begin{pgfscope}%
\pgfsys@transformshift{12.592401in}{1.600000in}%
\pgfsys@useobject{currentmarker}{}%
\end{pgfscope}%
\begin{pgfscope}%
\pgfsys@transformshift{12.628023in}{1.400000in}%
\pgfsys@useobject{currentmarker}{}%
\end{pgfscope}%
\begin{pgfscope}%
\pgfsys@transformshift{12.663645in}{1.600000in}%
\pgfsys@useobject{currentmarker}{}%
\end{pgfscope}%
\begin{pgfscope}%
\pgfsys@transformshift{12.699266in}{1.400000in}%
\pgfsys@useobject{currentmarker}{}%
\end{pgfscope}%
\begin{pgfscope}%
\pgfsys@transformshift{12.734888in}{1.400000in}%
\pgfsys@useobject{currentmarker}{}%
\end{pgfscope}%
\begin{pgfscope}%
\pgfsys@transformshift{12.770509in}{1.500000in}%
\pgfsys@useobject{currentmarker}{}%
\end{pgfscope}%
\begin{pgfscope}%
\pgfsys@transformshift{12.806131in}{1.500000in}%
\pgfsys@useobject{currentmarker}{}%
\end{pgfscope}%
\begin{pgfscope}%
\pgfsys@transformshift{12.841753in}{1.400000in}%
\pgfsys@useobject{currentmarker}{}%
\end{pgfscope}%
\begin{pgfscope}%
\pgfsys@transformshift{12.877374in}{1.400000in}%
\pgfsys@useobject{currentmarker}{}%
\end{pgfscope}%
\begin{pgfscope}%
\pgfsys@transformshift{12.912996in}{1.500000in}%
\pgfsys@useobject{currentmarker}{}%
\end{pgfscope}%
\begin{pgfscope}%
\pgfsys@transformshift{12.948618in}{1.300000in}%
\pgfsys@useobject{currentmarker}{}%
\end{pgfscope}%
\begin{pgfscope}%
\pgfsys@transformshift{12.984239in}{1.600000in}%
\pgfsys@useobject{currentmarker}{}%
\end{pgfscope}%
\begin{pgfscope}%
\pgfsys@transformshift{13.019861in}{1.400000in}%
\pgfsys@useobject{currentmarker}{}%
\end{pgfscope}%
\begin{pgfscope}%
\pgfsys@transformshift{13.055482in}{1.800000in}%
\pgfsys@useobject{currentmarker}{}%
\end{pgfscope}%
\begin{pgfscope}%
\pgfsys@transformshift{13.091104in}{1.400000in}%
\pgfsys@useobject{currentmarker}{}%
\end{pgfscope}%
\begin{pgfscope}%
\pgfsys@transformshift{13.126726in}{1.800000in}%
\pgfsys@useobject{currentmarker}{}%
\end{pgfscope}%
\begin{pgfscope}%
\pgfsys@transformshift{13.162347in}{1.700000in}%
\pgfsys@useobject{currentmarker}{}%
\end{pgfscope}%
\begin{pgfscope}%
\pgfsys@transformshift{13.197969in}{1.500000in}%
\pgfsys@useobject{currentmarker}{}%
\end{pgfscope}%
\begin{pgfscope}%
\pgfsys@transformshift{13.233591in}{1.400000in}%
\pgfsys@useobject{currentmarker}{}%
\end{pgfscope}%
\begin{pgfscope}%
\pgfsys@transformshift{13.269212in}{1.500000in}%
\pgfsys@useobject{currentmarker}{}%
\end{pgfscope}%
\begin{pgfscope}%
\pgfsys@transformshift{13.304834in}{1.600000in}%
\pgfsys@useobject{currentmarker}{}%
\end{pgfscope}%
\begin{pgfscope}%
\pgfsys@transformshift{13.340455in}{1.700000in}%
\pgfsys@useobject{currentmarker}{}%
\end{pgfscope}%
\begin{pgfscope}%
\pgfsys@transformshift{13.376077in}{1.400000in}%
\pgfsys@useobject{currentmarker}{}%
\end{pgfscope}%
\begin{pgfscope}%
\pgfsys@transformshift{13.411699in}{1.400000in}%
\pgfsys@useobject{currentmarker}{}%
\end{pgfscope}%
\begin{pgfscope}%
\pgfsys@transformshift{13.447320in}{1.500000in}%
\pgfsys@useobject{currentmarker}{}%
\end{pgfscope}%
\begin{pgfscope}%
\pgfsys@transformshift{13.482942in}{1.400000in}%
\pgfsys@useobject{currentmarker}{}%
\end{pgfscope}%
\begin{pgfscope}%
\pgfsys@transformshift{13.518564in}{1.400000in}%
\pgfsys@useobject{currentmarker}{}%
\end{pgfscope}%
\begin{pgfscope}%
\pgfsys@transformshift{13.554185in}{1.400000in}%
\pgfsys@useobject{currentmarker}{}%
\end{pgfscope}%
\begin{pgfscope}%
\pgfsys@transformshift{13.589807in}{1.500000in}%
\pgfsys@useobject{currentmarker}{}%
\end{pgfscope}%
\begin{pgfscope}%
\pgfsys@transformshift{13.625428in}{1.600000in}%
\pgfsys@useobject{currentmarker}{}%
\end{pgfscope}%
\begin{pgfscope}%
\pgfsys@transformshift{13.661050in}{1.300000in}%
\pgfsys@useobject{currentmarker}{}%
\end{pgfscope}%
\begin{pgfscope}%
\pgfsys@transformshift{13.696672in}{1.500000in}%
\pgfsys@useobject{currentmarker}{}%
\end{pgfscope}%
\begin{pgfscope}%
\pgfsys@transformshift{13.732293in}{1.600000in}%
\pgfsys@useobject{currentmarker}{}%
\end{pgfscope}%
\begin{pgfscope}%
\pgfsys@transformshift{13.767915in}{1.300000in}%
\pgfsys@useobject{currentmarker}{}%
\end{pgfscope}%
\begin{pgfscope}%
\pgfsys@transformshift{13.803537in}{1.500000in}%
\pgfsys@useobject{currentmarker}{}%
\end{pgfscope}%
\begin{pgfscope}%
\pgfsys@transformshift{13.839158in}{1.500000in}%
\pgfsys@useobject{currentmarker}{}%
\end{pgfscope}%
\begin{pgfscope}%
\pgfsys@transformshift{13.874780in}{1.500000in}%
\pgfsys@useobject{currentmarker}{}%
\end{pgfscope}%
\begin{pgfscope}%
\pgfsys@transformshift{13.910401in}{1.400000in}%
\pgfsys@useobject{currentmarker}{}%
\end{pgfscope}%
\begin{pgfscope}%
\pgfsys@transformshift{13.946023in}{1.400000in}%
\pgfsys@useobject{currentmarker}{}%
\end{pgfscope}%
\begin{pgfscope}%
\pgfsys@transformshift{13.981645in}{1.500000in}%
\pgfsys@useobject{currentmarker}{}%
\end{pgfscope}%
\begin{pgfscope}%
\pgfsys@transformshift{14.017266in}{1.600000in}%
\pgfsys@useobject{currentmarker}{}%
\end{pgfscope}%
\begin{pgfscope}%
\pgfsys@transformshift{14.052888in}{1.500000in}%
\pgfsys@useobject{currentmarker}{}%
\end{pgfscope}%
\begin{pgfscope}%
\pgfsys@transformshift{14.088510in}{1.700000in}%
\pgfsys@useobject{currentmarker}{}%
\end{pgfscope}%
\begin{pgfscope}%
\pgfsys@transformshift{14.124131in}{1.500000in}%
\pgfsys@useobject{currentmarker}{}%
\end{pgfscope}%
\begin{pgfscope}%
\pgfsys@transformshift{14.159753in}{1.500000in}%
\pgfsys@useobject{currentmarker}{}%
\end{pgfscope}%
\begin{pgfscope}%
\pgfsys@transformshift{14.195375in}{1.400000in}%
\pgfsys@useobject{currentmarker}{}%
\end{pgfscope}%
\begin{pgfscope}%
\pgfsys@transformshift{14.230996in}{1.500000in}%
\pgfsys@useobject{currentmarker}{}%
\end{pgfscope}%
\begin{pgfscope}%
\pgfsys@transformshift{14.266618in}{1.400000in}%
\pgfsys@useobject{currentmarker}{}%
\end{pgfscope}%
\begin{pgfscope}%
\pgfsys@transformshift{14.302239in}{1.500000in}%
\pgfsys@useobject{currentmarker}{}%
\end{pgfscope}%
\begin{pgfscope}%
\pgfsys@transformshift{14.337861in}{1.600000in}%
\pgfsys@useobject{currentmarker}{}%
\end{pgfscope}%
\begin{pgfscope}%
\pgfsys@transformshift{14.373483in}{1.400000in}%
\pgfsys@useobject{currentmarker}{}%
\end{pgfscope}%
\begin{pgfscope}%
\pgfsys@transformshift{14.409104in}{1.600000in}%
\pgfsys@useobject{currentmarker}{}%
\end{pgfscope}%
\begin{pgfscope}%
\pgfsys@transformshift{14.444726in}{1.500000in}%
\pgfsys@useobject{currentmarker}{}%
\end{pgfscope}%
\begin{pgfscope}%
\pgfsys@transformshift{14.480348in}{1.400000in}%
\pgfsys@useobject{currentmarker}{}%
\end{pgfscope}%
\begin{pgfscope}%
\pgfsys@transformshift{14.515969in}{1.600000in}%
\pgfsys@useobject{currentmarker}{}%
\end{pgfscope}%
\begin{pgfscope}%
\pgfsys@transformshift{14.551591in}{1.600000in}%
\pgfsys@useobject{currentmarker}{}%
\end{pgfscope}%
\begin{pgfscope}%
\pgfsys@transformshift{14.587212in}{1.400000in}%
\pgfsys@useobject{currentmarker}{}%
\end{pgfscope}%
\begin{pgfscope}%
\pgfsys@transformshift{14.622834in}{1.400000in}%
\pgfsys@useobject{currentmarker}{}%
\end{pgfscope}%
\begin{pgfscope}%
\pgfsys@transformshift{14.658456in}{1.300000in}%
\pgfsys@useobject{currentmarker}{}%
\end{pgfscope}%
\begin{pgfscope}%
\pgfsys@transformshift{14.694077in}{1.600000in}%
\pgfsys@useobject{currentmarker}{}%
\end{pgfscope}%
\begin{pgfscope}%
\pgfsys@transformshift{14.729699in}{1.600000in}%
\pgfsys@useobject{currentmarker}{}%
\end{pgfscope}%
\begin{pgfscope}%
\pgfsys@transformshift{14.765321in}{1.600000in}%
\pgfsys@useobject{currentmarker}{}%
\end{pgfscope}%
\begin{pgfscope}%
\pgfsys@transformshift{14.800942in}{1.500000in}%
\pgfsys@useobject{currentmarker}{}%
\end{pgfscope}%
\begin{pgfscope}%
\pgfsys@transformshift{14.836564in}{1.400000in}%
\pgfsys@useobject{currentmarker}{}%
\end{pgfscope}%
\begin{pgfscope}%
\pgfsys@transformshift{14.872185in}{1.500000in}%
\pgfsys@useobject{currentmarker}{}%
\end{pgfscope}%
\begin{pgfscope}%
\pgfsys@transformshift{14.907807in}{1.600000in}%
\pgfsys@useobject{currentmarker}{}%
\end{pgfscope}%
\begin{pgfscope}%
\pgfsys@transformshift{14.943429in}{1.600000in}%
\pgfsys@useobject{currentmarker}{}%
\end{pgfscope}%
\begin{pgfscope}%
\pgfsys@transformshift{14.979050in}{1.500000in}%
\pgfsys@useobject{currentmarker}{}%
\end{pgfscope}%
\begin{pgfscope}%
\pgfsys@transformshift{15.014672in}{1.600000in}%
\pgfsys@useobject{currentmarker}{}%
\end{pgfscope}%
\begin{pgfscope}%
\pgfsys@transformshift{15.050294in}{1.600000in}%
\pgfsys@useobject{currentmarker}{}%
\end{pgfscope}%
\begin{pgfscope}%
\pgfsys@transformshift{15.085915in}{1.200000in}%
\pgfsys@useobject{currentmarker}{}%
\end{pgfscope}%
\begin{pgfscope}%
\pgfsys@transformshift{15.121537in}{1.500000in}%
\pgfsys@useobject{currentmarker}{}%
\end{pgfscope}%
\begin{pgfscope}%
\pgfsys@transformshift{15.157158in}{1.400000in}%
\pgfsys@useobject{currentmarker}{}%
\end{pgfscope}%
\begin{pgfscope}%
\pgfsys@transformshift{15.192780in}{1.500000in}%
\pgfsys@useobject{currentmarker}{}%
\end{pgfscope}%
\begin{pgfscope}%
\pgfsys@transformshift{15.228402in}{1.500000in}%
\pgfsys@useobject{currentmarker}{}%
\end{pgfscope}%
\begin{pgfscope}%
\pgfsys@transformshift{15.264023in}{1.600000in}%
\pgfsys@useobject{currentmarker}{}%
\end{pgfscope}%
\begin{pgfscope}%
\pgfsys@transformshift{15.299645in}{1.600000in}%
\pgfsys@useobject{currentmarker}{}%
\end{pgfscope}%
\begin{pgfscope}%
\pgfsys@transformshift{15.335267in}{1.500000in}%
\pgfsys@useobject{currentmarker}{}%
\end{pgfscope}%
\begin{pgfscope}%
\pgfsys@transformshift{15.370888in}{1.500000in}%
\pgfsys@useobject{currentmarker}{}%
\end{pgfscope}%
\begin{pgfscope}%
\pgfsys@transformshift{15.406510in}{1.500000in}%
\pgfsys@useobject{currentmarker}{}%
\end{pgfscope}%
\begin{pgfscope}%
\pgfsys@transformshift{15.442131in}{1.600000in}%
\pgfsys@useobject{currentmarker}{}%
\end{pgfscope}%
\begin{pgfscope}%
\pgfsys@transformshift{15.477753in}{1.600000in}%
\pgfsys@useobject{currentmarker}{}%
\end{pgfscope}%
\begin{pgfscope}%
\pgfsys@transformshift{15.513375in}{1.500000in}%
\pgfsys@useobject{currentmarker}{}%
\end{pgfscope}%
\begin{pgfscope}%
\pgfsys@transformshift{15.548996in}{1.500000in}%
\pgfsys@useobject{currentmarker}{}%
\end{pgfscope}%
\begin{pgfscope}%
\pgfsys@transformshift{15.584618in}{1.500000in}%
\pgfsys@useobject{currentmarker}{}%
\end{pgfscope}%
\begin{pgfscope}%
\pgfsys@transformshift{15.620240in}{1.500000in}%
\pgfsys@useobject{currentmarker}{}%
\end{pgfscope}%
\begin{pgfscope}%
\pgfsys@transformshift{15.655861in}{1.400000in}%
\pgfsys@useobject{currentmarker}{}%
\end{pgfscope}%
\begin{pgfscope}%
\pgfsys@transformshift{15.691483in}{1.500000in}%
\pgfsys@useobject{currentmarker}{}%
\end{pgfscope}%
\begin{pgfscope}%
\pgfsys@transformshift{15.727105in}{1.500000in}%
\pgfsys@useobject{currentmarker}{}%
\end{pgfscope}%
\begin{pgfscope}%
\pgfsys@transformshift{15.762726in}{1.500000in}%
\pgfsys@useobject{currentmarker}{}%
\end{pgfscope}%
\begin{pgfscope}%
\pgfsys@transformshift{15.798348in}{1.700000in}%
\pgfsys@useobject{currentmarker}{}%
\end{pgfscope}%
\begin{pgfscope}%
\pgfsys@transformshift{15.833969in}{1.500000in}%
\pgfsys@useobject{currentmarker}{}%
\end{pgfscope}%
\begin{pgfscope}%
\pgfsys@transformshift{15.869591in}{1.600000in}%
\pgfsys@useobject{currentmarker}{}%
\end{pgfscope}%
\begin{pgfscope}%
\pgfsys@transformshift{15.905213in}{1.400000in}%
\pgfsys@useobject{currentmarker}{}%
\end{pgfscope}%
\begin{pgfscope}%
\pgfsys@transformshift{15.940834in}{1.400000in}%
\pgfsys@useobject{currentmarker}{}%
\end{pgfscope}%
\begin{pgfscope}%
\pgfsys@transformshift{15.976456in}{1.200000in}%
\pgfsys@useobject{currentmarker}{}%
\end{pgfscope}%
\begin{pgfscope}%
\pgfsys@transformshift{16.012078in}{1.500000in}%
\pgfsys@useobject{currentmarker}{}%
\end{pgfscope}%
\begin{pgfscope}%
\pgfsys@transformshift{16.047699in}{1.400000in}%
\pgfsys@useobject{currentmarker}{}%
\end{pgfscope}%
\begin{pgfscope}%
\pgfsys@transformshift{16.083321in}{1.400000in}%
\pgfsys@useobject{currentmarker}{}%
\end{pgfscope}%
\begin{pgfscope}%
\pgfsys@transformshift{16.118942in}{1.400000in}%
\pgfsys@useobject{currentmarker}{}%
\end{pgfscope}%
\begin{pgfscope}%
\pgfsys@transformshift{16.154564in}{1.600000in}%
\pgfsys@useobject{currentmarker}{}%
\end{pgfscope}%
\begin{pgfscope}%
\pgfsys@transformshift{16.190186in}{1.500000in}%
\pgfsys@useobject{currentmarker}{}%
\end{pgfscope}%
\begin{pgfscope}%
\pgfsys@transformshift{16.225807in}{1.600000in}%
\pgfsys@useobject{currentmarker}{}%
\end{pgfscope}%
\begin{pgfscope}%
\pgfsys@transformshift{16.261429in}{1.300000in}%
\pgfsys@useobject{currentmarker}{}%
\end{pgfscope}%
\begin{pgfscope}%
\pgfsys@transformshift{16.297051in}{1.700000in}%
\pgfsys@useobject{currentmarker}{}%
\end{pgfscope}%
\begin{pgfscope}%
\pgfsys@transformshift{16.332672in}{1.300000in}%
\pgfsys@useobject{currentmarker}{}%
\end{pgfscope}%
\begin{pgfscope}%
\pgfsys@transformshift{16.368294in}{1.400000in}%
\pgfsys@useobject{currentmarker}{}%
\end{pgfscope}%
\begin{pgfscope}%
\pgfsys@transformshift{16.403915in}{1.500000in}%
\pgfsys@useobject{currentmarker}{}%
\end{pgfscope}%
\begin{pgfscope}%
\pgfsys@transformshift{16.439537in}{1.500000in}%
\pgfsys@useobject{currentmarker}{}%
\end{pgfscope}%
\begin{pgfscope}%
\pgfsys@transformshift{16.475159in}{1.500000in}%
\pgfsys@useobject{currentmarker}{}%
\end{pgfscope}%
\begin{pgfscope}%
\pgfsys@transformshift{16.510780in}{1.400000in}%
\pgfsys@useobject{currentmarker}{}%
\end{pgfscope}%
\begin{pgfscope}%
\pgfsys@transformshift{16.546402in}{1.500000in}%
\pgfsys@useobject{currentmarker}{}%
\end{pgfscope}%
\begin{pgfscope}%
\pgfsys@transformshift{16.582024in}{1.500000in}%
\pgfsys@useobject{currentmarker}{}%
\end{pgfscope}%
\begin{pgfscope}%
\pgfsys@transformshift{16.617645in}{1.500000in}%
\pgfsys@useobject{currentmarker}{}%
\end{pgfscope}%
\begin{pgfscope}%
\pgfsys@transformshift{16.653267in}{1.600000in}%
\pgfsys@useobject{currentmarker}{}%
\end{pgfscope}%
\begin{pgfscope}%
\pgfsys@transformshift{16.688888in}{1.400000in}%
\pgfsys@useobject{currentmarker}{}%
\end{pgfscope}%
\begin{pgfscope}%
\pgfsys@transformshift{16.724510in}{1.300000in}%
\pgfsys@useobject{currentmarker}{}%
\end{pgfscope}%
\begin{pgfscope}%
\pgfsys@transformshift{16.760132in}{1.600000in}%
\pgfsys@useobject{currentmarker}{}%
\end{pgfscope}%
\begin{pgfscope}%
\pgfsys@transformshift{16.795753in}{1.500000in}%
\pgfsys@useobject{currentmarker}{}%
\end{pgfscope}%
\begin{pgfscope}%
\pgfsys@transformshift{16.831375in}{1.600000in}%
\pgfsys@useobject{currentmarker}{}%
\end{pgfscope}%
\begin{pgfscope}%
\pgfsys@transformshift{16.866997in}{1.700000in}%
\pgfsys@useobject{currentmarker}{}%
\end{pgfscope}%
\begin{pgfscope}%
\pgfsys@transformshift{16.902618in}{1.400000in}%
\pgfsys@useobject{currentmarker}{}%
\end{pgfscope}%
\begin{pgfscope}%
\pgfsys@transformshift{16.938240in}{1.400000in}%
\pgfsys@useobject{currentmarker}{}%
\end{pgfscope}%
\begin{pgfscope}%
\pgfsys@transformshift{16.973861in}{1.500000in}%
\pgfsys@useobject{currentmarker}{}%
\end{pgfscope}%
\begin{pgfscope}%
\pgfsys@transformshift{17.009483in}{1.600000in}%
\pgfsys@useobject{currentmarker}{}%
\end{pgfscope}%
\begin{pgfscope}%
\pgfsys@transformshift{17.045105in}{1.600000in}%
\pgfsys@useobject{currentmarker}{}%
\end{pgfscope}%
\begin{pgfscope}%
\pgfsys@transformshift{17.080726in}{1.500000in}%
\pgfsys@useobject{currentmarker}{}%
\end{pgfscope}%
\begin{pgfscope}%
\pgfsys@transformshift{17.116348in}{1.400000in}%
\pgfsys@useobject{currentmarker}{}%
\end{pgfscope}%
\begin{pgfscope}%
\pgfsys@transformshift{17.151970in}{1.500000in}%
\pgfsys@useobject{currentmarker}{}%
\end{pgfscope}%
\begin{pgfscope}%
\pgfsys@transformshift{17.187591in}{1.700000in}%
\pgfsys@useobject{currentmarker}{}%
\end{pgfscope}%
\begin{pgfscope}%
\pgfsys@transformshift{17.223213in}{1.600000in}%
\pgfsys@useobject{currentmarker}{}%
\end{pgfscope}%
\begin{pgfscope}%
\pgfsys@transformshift{17.258835in}{1.400000in}%
\pgfsys@useobject{currentmarker}{}%
\end{pgfscope}%
\begin{pgfscope}%
\pgfsys@transformshift{17.294456in}{1.400000in}%
\pgfsys@useobject{currentmarker}{}%
\end{pgfscope}%
\begin{pgfscope}%
\pgfsys@transformshift{17.330078in}{1.700000in}%
\pgfsys@useobject{currentmarker}{}%
\end{pgfscope}%
\begin{pgfscope}%
\pgfsys@transformshift{17.365699in}{1.600000in}%
\pgfsys@useobject{currentmarker}{}%
\end{pgfscope}%
\begin{pgfscope}%
\pgfsys@transformshift{17.401321in}{1.600000in}%
\pgfsys@useobject{currentmarker}{}%
\end{pgfscope}%
\begin{pgfscope}%
\pgfsys@transformshift{17.436943in}{1.500000in}%
\pgfsys@useobject{currentmarker}{}%
\end{pgfscope}%
\begin{pgfscope}%
\pgfsys@transformshift{17.472564in}{1.500000in}%
\pgfsys@useobject{currentmarker}{}%
\end{pgfscope}%
\begin{pgfscope}%
\pgfsys@transformshift{17.508186in}{1.400000in}%
\pgfsys@useobject{currentmarker}{}%
\end{pgfscope}%
\begin{pgfscope}%
\pgfsys@transformshift{17.543808in}{1.400000in}%
\pgfsys@useobject{currentmarker}{}%
\end{pgfscope}%
\begin{pgfscope}%
\pgfsys@transformshift{17.579429in}{1.500000in}%
\pgfsys@useobject{currentmarker}{}%
\end{pgfscope}%
\begin{pgfscope}%
\pgfsys@transformshift{17.615051in}{1.500000in}%
\pgfsys@useobject{currentmarker}{}%
\end{pgfscope}%
\begin{pgfscope}%
\pgfsys@transformshift{17.650672in}{1.400000in}%
\pgfsys@useobject{currentmarker}{}%
\end{pgfscope}%
\begin{pgfscope}%
\pgfsys@transformshift{17.686294in}{1.500000in}%
\pgfsys@useobject{currentmarker}{}%
\end{pgfscope}%
\begin{pgfscope}%
\pgfsys@transformshift{17.721916in}{1.500000in}%
\pgfsys@useobject{currentmarker}{}%
\end{pgfscope}%
\begin{pgfscope}%
\pgfsys@transformshift{17.757537in}{1.700000in}%
\pgfsys@useobject{currentmarker}{}%
\end{pgfscope}%
\begin{pgfscope}%
\pgfsys@transformshift{17.793159in}{1.400000in}%
\pgfsys@useobject{currentmarker}{}%
\end{pgfscope}%
\begin{pgfscope}%
\pgfsys@transformshift{17.828781in}{1.600000in}%
\pgfsys@useobject{currentmarker}{}%
\end{pgfscope}%
\begin{pgfscope}%
\pgfsys@transformshift{17.864402in}{1.600000in}%
\pgfsys@useobject{currentmarker}{}%
\end{pgfscope}%
\begin{pgfscope}%
\pgfsys@transformshift{17.900024in}{1.700000in}%
\pgfsys@useobject{currentmarker}{}%
\end{pgfscope}%
\begin{pgfscope}%
\pgfsys@transformshift{17.935645in}{1.500000in}%
\pgfsys@useobject{currentmarker}{}%
\end{pgfscope}%
\begin{pgfscope}%
\pgfsys@transformshift{17.971267in}{1.300000in}%
\pgfsys@useobject{currentmarker}{}%
\end{pgfscope}%
\begin{pgfscope}%
\pgfsys@transformshift{18.006889in}{1.600000in}%
\pgfsys@useobject{currentmarker}{}%
\end{pgfscope}%
\begin{pgfscope}%
\pgfsys@transformshift{18.042510in}{1.400000in}%
\pgfsys@useobject{currentmarker}{}%
\end{pgfscope}%
\begin{pgfscope}%
\pgfsys@transformshift{18.078132in}{1.600000in}%
\pgfsys@useobject{currentmarker}{}%
\end{pgfscope}%
\begin{pgfscope}%
\pgfsys@transformshift{18.113754in}{1.400000in}%
\pgfsys@useobject{currentmarker}{}%
\end{pgfscope}%
\begin{pgfscope}%
\pgfsys@transformshift{18.149375in}{1.600000in}%
\pgfsys@useobject{currentmarker}{}%
\end{pgfscope}%
\begin{pgfscope}%
\pgfsys@transformshift{18.184997in}{1.500000in}%
\pgfsys@useobject{currentmarker}{}%
\end{pgfscope}%
\begin{pgfscope}%
\pgfsys@transformshift{18.220618in}{1.600000in}%
\pgfsys@useobject{currentmarker}{}%
\end{pgfscope}%
\begin{pgfscope}%
\pgfsys@transformshift{18.256240in}{1.400000in}%
\pgfsys@useobject{currentmarker}{}%
\end{pgfscope}%
\begin{pgfscope}%
\pgfsys@transformshift{18.291862in}{1.500000in}%
\pgfsys@useobject{currentmarker}{}%
\end{pgfscope}%
\begin{pgfscope}%
\pgfsys@transformshift{18.327483in}{1.400000in}%
\pgfsys@useobject{currentmarker}{}%
\end{pgfscope}%
\begin{pgfscope}%
\pgfsys@transformshift{18.363105in}{1.600000in}%
\pgfsys@useobject{currentmarker}{}%
\end{pgfscope}%
\begin{pgfscope}%
\pgfsys@transformshift{18.398727in}{1.400000in}%
\pgfsys@useobject{currentmarker}{}%
\end{pgfscope}%
\begin{pgfscope}%
\pgfsys@transformshift{18.434348in}{1.500000in}%
\pgfsys@useobject{currentmarker}{}%
\end{pgfscope}%
\begin{pgfscope}%
\pgfsys@transformshift{18.469970in}{1.500000in}%
\pgfsys@useobject{currentmarker}{}%
\end{pgfscope}%
\begin{pgfscope}%
\pgfsys@transformshift{18.505591in}{1.600000in}%
\pgfsys@useobject{currentmarker}{}%
\end{pgfscope}%
\begin{pgfscope}%
\pgfsys@transformshift{18.541213in}{1.600000in}%
\pgfsys@useobject{currentmarker}{}%
\end{pgfscope}%
\begin{pgfscope}%
\pgfsys@transformshift{18.576835in}{1.600000in}%
\pgfsys@useobject{currentmarker}{}%
\end{pgfscope}%
\begin{pgfscope}%
\pgfsys@transformshift{18.612456in}{1.500000in}%
\pgfsys@useobject{currentmarker}{}%
\end{pgfscope}%
\begin{pgfscope}%
\pgfsys@transformshift{18.648078in}{1.600000in}%
\pgfsys@useobject{currentmarker}{}%
\end{pgfscope}%
\begin{pgfscope}%
\pgfsys@transformshift{18.683700in}{1.600000in}%
\pgfsys@useobject{currentmarker}{}%
\end{pgfscope}%
\begin{pgfscope}%
\pgfsys@transformshift{18.719321in}{1.500000in}%
\pgfsys@useobject{currentmarker}{}%
\end{pgfscope}%
\begin{pgfscope}%
\pgfsys@transformshift{18.754943in}{1.600000in}%
\pgfsys@useobject{currentmarker}{}%
\end{pgfscope}%
\begin{pgfscope}%
\pgfsys@transformshift{18.790565in}{1.700000in}%
\pgfsys@useobject{currentmarker}{}%
\end{pgfscope}%
\begin{pgfscope}%
\pgfsys@transformshift{18.826186in}{1.700000in}%
\pgfsys@useobject{currentmarker}{}%
\end{pgfscope}%
\begin{pgfscope}%
\pgfsys@transformshift{18.861808in}{1.400000in}%
\pgfsys@useobject{currentmarker}{}%
\end{pgfscope}%
\begin{pgfscope}%
\pgfsys@transformshift{18.897429in}{1.400000in}%
\pgfsys@useobject{currentmarker}{}%
\end{pgfscope}%
\begin{pgfscope}%
\pgfsys@transformshift{18.933051in}{1.600000in}%
\pgfsys@useobject{currentmarker}{}%
\end{pgfscope}%
\begin{pgfscope}%
\pgfsys@transformshift{18.968673in}{1.400000in}%
\pgfsys@useobject{currentmarker}{}%
\end{pgfscope}%
\begin{pgfscope}%
\pgfsys@transformshift{19.004294in}{1.300000in}%
\pgfsys@useobject{currentmarker}{}%
\end{pgfscope}%
\begin{pgfscope}%
\pgfsys@transformshift{19.039916in}{1.500000in}%
\pgfsys@useobject{currentmarker}{}%
\end{pgfscope}%
\begin{pgfscope}%
\pgfsys@transformshift{19.075538in}{1.500000in}%
\pgfsys@useobject{currentmarker}{}%
\end{pgfscope}%
\begin{pgfscope}%
\pgfsys@transformshift{19.111159in}{1.600000in}%
\pgfsys@useobject{currentmarker}{}%
\end{pgfscope}%
\begin{pgfscope}%
\pgfsys@transformshift{19.146781in}{1.400000in}%
\pgfsys@useobject{currentmarker}{}%
\end{pgfscope}%
\begin{pgfscope}%
\pgfsys@transformshift{19.182402in}{1.600000in}%
\pgfsys@useobject{currentmarker}{}%
\end{pgfscope}%
\begin{pgfscope}%
\pgfsys@transformshift{19.218024in}{1.500000in}%
\pgfsys@useobject{currentmarker}{}%
\end{pgfscope}%
\begin{pgfscope}%
\pgfsys@transformshift{19.253646in}{1.700000in}%
\pgfsys@useobject{currentmarker}{}%
\end{pgfscope}%
\begin{pgfscope}%
\pgfsys@transformshift{19.289267in}{1.600000in}%
\pgfsys@useobject{currentmarker}{}%
\end{pgfscope}%
\begin{pgfscope}%
\pgfsys@transformshift{19.324889in}{1.700000in}%
\pgfsys@useobject{currentmarker}{}%
\end{pgfscope}%
\begin{pgfscope}%
\pgfsys@transformshift{19.360511in}{1.600000in}%
\pgfsys@useobject{currentmarker}{}%
\end{pgfscope}%
\begin{pgfscope}%
\pgfsys@transformshift{19.396132in}{1.400000in}%
\pgfsys@useobject{currentmarker}{}%
\end{pgfscope}%
\begin{pgfscope}%
\pgfsys@transformshift{19.431754in}{1.800000in}%
\pgfsys@useobject{currentmarker}{}%
\end{pgfscope}%
\begin{pgfscope}%
\pgfsys@transformshift{19.467375in}{1.500000in}%
\pgfsys@useobject{currentmarker}{}%
\end{pgfscope}%
\begin{pgfscope}%
\pgfsys@transformshift{19.502997in}{1.500000in}%
\pgfsys@useobject{currentmarker}{}%
\end{pgfscope}%
\begin{pgfscope}%
\pgfsys@transformshift{19.538619in}{1.400000in}%
\pgfsys@useobject{currentmarker}{}%
\end{pgfscope}%
\begin{pgfscope}%
\pgfsys@transformshift{19.574240in}{1.500000in}%
\pgfsys@useobject{currentmarker}{}%
\end{pgfscope}%
\begin{pgfscope}%
\pgfsys@transformshift{19.609862in}{1.700000in}%
\pgfsys@useobject{currentmarker}{}%
\end{pgfscope}%
\begin{pgfscope}%
\pgfsys@transformshift{19.645484in}{1.400000in}%
\pgfsys@useobject{currentmarker}{}%
\end{pgfscope}%
\begin{pgfscope}%
\pgfsys@transformshift{19.681105in}{1.500000in}%
\pgfsys@useobject{currentmarker}{}%
\end{pgfscope}%
\begin{pgfscope}%
\pgfsys@transformshift{19.716727in}{1.500000in}%
\pgfsys@useobject{currentmarker}{}%
\end{pgfscope}%
\begin{pgfscope}%
\pgfsys@transformshift{19.752348in}{1.500000in}%
\pgfsys@useobject{currentmarker}{}%
\end{pgfscope}%
\begin{pgfscope}%
\pgfsys@transformshift{19.787970in}{1.400000in}%
\pgfsys@useobject{currentmarker}{}%
\end{pgfscope}%
\begin{pgfscope}%
\pgfsys@transformshift{19.823592in}{1.400000in}%
\pgfsys@useobject{currentmarker}{}%
\end{pgfscope}%
\begin{pgfscope}%
\pgfsys@transformshift{19.859213in}{1.300000in}%
\pgfsys@useobject{currentmarker}{}%
\end{pgfscope}%
\begin{pgfscope}%
\pgfsys@transformshift{19.894835in}{1.600000in}%
\pgfsys@useobject{currentmarker}{}%
\end{pgfscope}%
\begin{pgfscope}%
\pgfsys@transformshift{19.930457in}{1.500000in}%
\pgfsys@useobject{currentmarker}{}%
\end{pgfscope}%
\begin{pgfscope}%
\pgfsys@transformshift{19.966078in}{1.600000in}%
\pgfsys@useobject{currentmarker}{}%
\end{pgfscope}%
\begin{pgfscope}%
\pgfsys@transformshift{20.001700in}{1.500000in}%
\pgfsys@useobject{currentmarker}{}%
\end{pgfscope}%
\begin{pgfscope}%
\pgfsys@transformshift{20.037321in}{1.500000in}%
\pgfsys@useobject{currentmarker}{}%
\end{pgfscope}%
\begin{pgfscope}%
\pgfsys@transformshift{20.072943in}{1.300000in}%
\pgfsys@useobject{currentmarker}{}%
\end{pgfscope}%
\begin{pgfscope}%
\pgfsys@transformshift{20.108565in}{1.300000in}%
\pgfsys@useobject{currentmarker}{}%
\end{pgfscope}%
\begin{pgfscope}%
\pgfsys@transformshift{20.144186in}{1.400000in}%
\pgfsys@useobject{currentmarker}{}%
\end{pgfscope}%
\begin{pgfscope}%
\pgfsys@transformshift{20.179808in}{1.600000in}%
\pgfsys@useobject{currentmarker}{}%
\end{pgfscope}%
\begin{pgfscope}%
\pgfsys@transformshift{20.215430in}{1.500000in}%
\pgfsys@useobject{currentmarker}{}%
\end{pgfscope}%
\begin{pgfscope}%
\pgfsys@transformshift{20.251051in}{1.700000in}%
\pgfsys@useobject{currentmarker}{}%
\end{pgfscope}%
\begin{pgfscope}%
\pgfsys@transformshift{20.286673in}{1.400000in}%
\pgfsys@useobject{currentmarker}{}%
\end{pgfscope}%
\begin{pgfscope}%
\pgfsys@transformshift{20.322295in}{1.500000in}%
\pgfsys@useobject{currentmarker}{}%
\end{pgfscope}%
\begin{pgfscope}%
\pgfsys@transformshift{20.357916in}{1.500000in}%
\pgfsys@useobject{currentmarker}{}%
\end{pgfscope}%
\begin{pgfscope}%
\pgfsys@transformshift{20.393538in}{1.500000in}%
\pgfsys@useobject{currentmarker}{}%
\end{pgfscope}%
\begin{pgfscope}%
\pgfsys@transformshift{20.429159in}{1.600000in}%
\pgfsys@useobject{currentmarker}{}%
\end{pgfscope}%
\begin{pgfscope}%
\pgfsys@transformshift{20.464781in}{1.200000in}%
\pgfsys@useobject{currentmarker}{}%
\end{pgfscope}%
\begin{pgfscope}%
\pgfsys@transformshift{20.500403in}{1.700000in}%
\pgfsys@useobject{currentmarker}{}%
\end{pgfscope}%
\begin{pgfscope}%
\pgfsys@transformshift{20.536024in}{1.600000in}%
\pgfsys@useobject{currentmarker}{}%
\end{pgfscope}%
\begin{pgfscope}%
\pgfsys@transformshift{20.571646in}{1.600000in}%
\pgfsys@useobject{currentmarker}{}%
\end{pgfscope}%
\begin{pgfscope}%
\pgfsys@transformshift{20.607268in}{1.600000in}%
\pgfsys@useobject{currentmarker}{}%
\end{pgfscope}%
\begin{pgfscope}%
\pgfsys@transformshift{20.642889in}{1.600000in}%
\pgfsys@useobject{currentmarker}{}%
\end{pgfscope}%
\begin{pgfscope}%
\pgfsys@transformshift{20.678511in}{1.500000in}%
\pgfsys@useobject{currentmarker}{}%
\end{pgfscope}%
\begin{pgfscope}%
\pgfsys@transformshift{20.714132in}{1.500000in}%
\pgfsys@useobject{currentmarker}{}%
\end{pgfscope}%
\begin{pgfscope}%
\pgfsys@transformshift{20.749754in}{1.400000in}%
\pgfsys@useobject{currentmarker}{}%
\end{pgfscope}%
\begin{pgfscope}%
\pgfsys@transformshift{20.785376in}{1.500000in}%
\pgfsys@useobject{currentmarker}{}%
\end{pgfscope}%
\begin{pgfscope}%
\pgfsys@transformshift{20.820997in}{1.500000in}%
\pgfsys@useobject{currentmarker}{}%
\end{pgfscope}%
\begin{pgfscope}%
\pgfsys@transformshift{20.856619in}{1.700000in}%
\pgfsys@useobject{currentmarker}{}%
\end{pgfscope}%
\begin{pgfscope}%
\pgfsys@transformshift{20.892241in}{1.500000in}%
\pgfsys@useobject{currentmarker}{}%
\end{pgfscope}%
\begin{pgfscope}%
\pgfsys@transformshift{20.927862in}{1.500000in}%
\pgfsys@useobject{currentmarker}{}%
\end{pgfscope}%
\begin{pgfscope}%
\pgfsys@transformshift{20.963484in}{1.700000in}%
\pgfsys@useobject{currentmarker}{}%
\end{pgfscope}%
\begin{pgfscope}%
\pgfsys@transformshift{20.999105in}{1.500000in}%
\pgfsys@useobject{currentmarker}{}%
\end{pgfscope}%
\begin{pgfscope}%
\pgfsys@transformshift{21.034727in}{1.500000in}%
\pgfsys@useobject{currentmarker}{}%
\end{pgfscope}%
\begin{pgfscope}%
\pgfsys@transformshift{21.070349in}{1.400000in}%
\pgfsys@useobject{currentmarker}{}%
\end{pgfscope}%
\begin{pgfscope}%
\pgfsys@transformshift{21.105970in}{1.500000in}%
\pgfsys@useobject{currentmarker}{}%
\end{pgfscope}%
\begin{pgfscope}%
\pgfsys@transformshift{21.141592in}{1.600000in}%
\pgfsys@useobject{currentmarker}{}%
\end{pgfscope}%
\begin{pgfscope}%
\pgfsys@transformshift{21.177214in}{1.400000in}%
\pgfsys@useobject{currentmarker}{}%
\end{pgfscope}%
\begin{pgfscope}%
\pgfsys@transformshift{21.212835in}{1.400000in}%
\pgfsys@useobject{currentmarker}{}%
\end{pgfscope}%
\begin{pgfscope}%
\pgfsys@transformshift{21.248457in}{1.500000in}%
\pgfsys@useobject{currentmarker}{}%
\end{pgfscope}%
\begin{pgfscope}%
\pgfsys@transformshift{21.284078in}{1.600000in}%
\pgfsys@useobject{currentmarker}{}%
\end{pgfscope}%
\begin{pgfscope}%
\pgfsys@transformshift{21.319700in}{1.500000in}%
\pgfsys@useobject{currentmarker}{}%
\end{pgfscope}%
\begin{pgfscope}%
\pgfsys@transformshift{21.355322in}{1.400000in}%
\pgfsys@useobject{currentmarker}{}%
\end{pgfscope}%
\begin{pgfscope}%
\pgfsys@transformshift{21.390943in}{1.400000in}%
\pgfsys@useobject{currentmarker}{}%
\end{pgfscope}%
\begin{pgfscope}%
\pgfsys@transformshift{21.426565in}{1.600000in}%
\pgfsys@useobject{currentmarker}{}%
\end{pgfscope}%
\begin{pgfscope}%
\pgfsys@transformshift{21.462187in}{1.700000in}%
\pgfsys@useobject{currentmarker}{}%
\end{pgfscope}%
\begin{pgfscope}%
\pgfsys@transformshift{21.497808in}{1.600000in}%
\pgfsys@useobject{currentmarker}{}%
\end{pgfscope}%
\begin{pgfscope}%
\pgfsys@transformshift{21.533430in}{1.500000in}%
\pgfsys@useobject{currentmarker}{}%
\end{pgfscope}%
\end{pgfscope}%
\begin{pgfscope}%
\pgfpathrectangle{\pgfqpoint{1.000000in}{1.000000in}}{\pgfqpoint{8.500000in}{1.000000in}}%
\pgfusepath{clip}%
\pgfsetrectcap%
\pgfsetroundjoin%
\pgfsetlinewidth{0.803000pt}%
\definecolor{currentstroke}{rgb}{0.690196,0.690196,0.690196}%
\pgfsetstrokecolor{currentstroke}%
\pgfsetdash{}{0pt}%
\pgfpathmoveto{\pgfqpoint{2.725210in}{1.000000in}}%
\pgfpathlineto{\pgfqpoint{2.725210in}{2.000000in}}%
\pgfusepath{stroke}%
\end{pgfscope}%
\begin{pgfscope}%
\pgfsetbuttcap%
\pgfsetroundjoin%
\definecolor{currentfill}{rgb}{0.000000,0.000000,0.000000}%
\pgfsetfillcolor{currentfill}%
\pgfsetlinewidth{0.803000pt}%
\definecolor{currentstroke}{rgb}{0.000000,0.000000,0.000000}%
\pgfsetstrokecolor{currentstroke}%
\pgfsetdash{}{0pt}%
\pgfsys@defobject{currentmarker}{\pgfqpoint{0.000000in}{-0.048611in}}{\pgfqpoint{0.000000in}{0.000000in}}{%
\pgfpathmoveto{\pgfqpoint{0.000000in}{0.000000in}}%
\pgfpathlineto{\pgfqpoint{0.000000in}{-0.048611in}}%
\pgfusepath{stroke,fill}%
}%
\begin{pgfscope}%
\pgfsys@transformshift{2.725210in}{1.000000in}%
\pgfsys@useobject{currentmarker}{}%
\end{pgfscope}%
\end{pgfscope}%
\begin{pgfscope}%
\definecolor{textcolor}{rgb}{0.000000,0.000000,0.000000}%
\pgfsetstrokecolor{textcolor}%
\pgfsetfillcolor{textcolor}%
\pgftext[x=2.725210in,y=0.902778in,,top]{\color{textcolor}\sffamily\fontsize{20.000000}{24.000000}\selectfont \(\displaystyle {500}\)}%
\end{pgfscope}%
\begin{pgfscope}%
\pgfpathrectangle{\pgfqpoint{1.000000in}{1.000000in}}{\pgfqpoint{8.500000in}{1.000000in}}%
\pgfusepath{clip}%
\pgfsetrectcap%
\pgfsetroundjoin%
\pgfsetlinewidth{0.803000pt}%
\definecolor{currentstroke}{rgb}{0.690196,0.690196,0.690196}%
\pgfsetstrokecolor{currentstroke}%
\pgfsetdash{}{0pt}%
\pgfpathmoveto{\pgfqpoint{4.506292in}{1.000000in}}%
\pgfpathlineto{\pgfqpoint{4.506292in}{2.000000in}}%
\pgfusepath{stroke}%
\end{pgfscope}%
\begin{pgfscope}%
\pgfsetbuttcap%
\pgfsetroundjoin%
\definecolor{currentfill}{rgb}{0.000000,0.000000,0.000000}%
\pgfsetfillcolor{currentfill}%
\pgfsetlinewidth{0.803000pt}%
\definecolor{currentstroke}{rgb}{0.000000,0.000000,0.000000}%
\pgfsetstrokecolor{currentstroke}%
\pgfsetdash{}{0pt}%
\pgfsys@defobject{currentmarker}{\pgfqpoint{0.000000in}{-0.048611in}}{\pgfqpoint{0.000000in}{0.000000in}}{%
\pgfpathmoveto{\pgfqpoint{0.000000in}{0.000000in}}%
\pgfpathlineto{\pgfqpoint{0.000000in}{-0.048611in}}%
\pgfusepath{stroke,fill}%
}%
\begin{pgfscope}%
\pgfsys@transformshift{4.506292in}{1.000000in}%
\pgfsys@useobject{currentmarker}{}%
\end{pgfscope}%
\end{pgfscope}%
\begin{pgfscope}%
\definecolor{textcolor}{rgb}{0.000000,0.000000,0.000000}%
\pgfsetstrokecolor{textcolor}%
\pgfsetfillcolor{textcolor}%
\pgftext[x=4.506292in,y=0.902778in,,top]{\color{textcolor}\sffamily\fontsize{20.000000}{24.000000}\selectfont \(\displaystyle {550}\)}%
\end{pgfscope}%
\begin{pgfscope}%
\pgfpathrectangle{\pgfqpoint{1.000000in}{1.000000in}}{\pgfqpoint{8.500000in}{1.000000in}}%
\pgfusepath{clip}%
\pgfsetrectcap%
\pgfsetroundjoin%
\pgfsetlinewidth{0.803000pt}%
\definecolor{currentstroke}{rgb}{0.690196,0.690196,0.690196}%
\pgfsetstrokecolor{currentstroke}%
\pgfsetdash{}{0pt}%
\pgfpathmoveto{\pgfqpoint{6.287373in}{1.000000in}}%
\pgfpathlineto{\pgfqpoint{6.287373in}{2.000000in}}%
\pgfusepath{stroke}%
\end{pgfscope}%
\begin{pgfscope}%
\pgfsetbuttcap%
\pgfsetroundjoin%
\definecolor{currentfill}{rgb}{0.000000,0.000000,0.000000}%
\pgfsetfillcolor{currentfill}%
\pgfsetlinewidth{0.803000pt}%
\definecolor{currentstroke}{rgb}{0.000000,0.000000,0.000000}%
\pgfsetstrokecolor{currentstroke}%
\pgfsetdash{}{0pt}%
\pgfsys@defobject{currentmarker}{\pgfqpoint{0.000000in}{-0.048611in}}{\pgfqpoint{0.000000in}{0.000000in}}{%
\pgfpathmoveto{\pgfqpoint{0.000000in}{0.000000in}}%
\pgfpathlineto{\pgfqpoint{0.000000in}{-0.048611in}}%
\pgfusepath{stroke,fill}%
}%
\begin{pgfscope}%
\pgfsys@transformshift{6.287373in}{1.000000in}%
\pgfsys@useobject{currentmarker}{}%
\end{pgfscope}%
\end{pgfscope}%
\begin{pgfscope}%
\definecolor{textcolor}{rgb}{0.000000,0.000000,0.000000}%
\pgfsetstrokecolor{textcolor}%
\pgfsetfillcolor{textcolor}%
\pgftext[x=6.287373in,y=0.902778in,,top]{\color{textcolor}\sffamily\fontsize{20.000000}{24.000000}\selectfont \(\displaystyle {600}\)}%
\end{pgfscope}%
\begin{pgfscope}%
\pgfpathrectangle{\pgfqpoint{1.000000in}{1.000000in}}{\pgfqpoint{8.500000in}{1.000000in}}%
\pgfusepath{clip}%
\pgfsetrectcap%
\pgfsetroundjoin%
\pgfsetlinewidth{0.803000pt}%
\definecolor{currentstroke}{rgb}{0.690196,0.690196,0.690196}%
\pgfsetstrokecolor{currentstroke}%
\pgfsetdash{}{0pt}%
\pgfpathmoveto{\pgfqpoint{8.068455in}{1.000000in}}%
\pgfpathlineto{\pgfqpoint{8.068455in}{2.000000in}}%
\pgfusepath{stroke}%
\end{pgfscope}%
\begin{pgfscope}%
\pgfsetbuttcap%
\pgfsetroundjoin%
\definecolor{currentfill}{rgb}{0.000000,0.000000,0.000000}%
\pgfsetfillcolor{currentfill}%
\pgfsetlinewidth{0.803000pt}%
\definecolor{currentstroke}{rgb}{0.000000,0.000000,0.000000}%
\pgfsetstrokecolor{currentstroke}%
\pgfsetdash{}{0pt}%
\pgfsys@defobject{currentmarker}{\pgfqpoint{0.000000in}{-0.048611in}}{\pgfqpoint{0.000000in}{0.000000in}}{%
\pgfpathmoveto{\pgfqpoint{0.000000in}{0.000000in}}%
\pgfpathlineto{\pgfqpoint{0.000000in}{-0.048611in}}%
\pgfusepath{stroke,fill}%
}%
\begin{pgfscope}%
\pgfsys@transformshift{8.068455in}{1.000000in}%
\pgfsys@useobject{currentmarker}{}%
\end{pgfscope}%
\end{pgfscope}%
\begin{pgfscope}%
\definecolor{textcolor}{rgb}{0.000000,0.000000,0.000000}%
\pgfsetstrokecolor{textcolor}%
\pgfsetfillcolor{textcolor}%
\pgftext[x=8.068455in,y=0.902778in,,top]{\color{textcolor}\sffamily\fontsize{20.000000}{24.000000}\selectfont \(\displaystyle {650}\)}%
\end{pgfscope}%
\begin{pgfscope}%
\definecolor{textcolor}{rgb}{0.000000,0.000000,0.000000}%
\pgfsetstrokecolor{textcolor}%
\pgfsetfillcolor{textcolor}%
\pgftext[x=5.250000in,y=0.591155in,,top]{\color{textcolor}\sffamily\fontsize{20.000000}{24.000000}\selectfont \(\displaystyle \mathrm{t}/\si{ns}\)}%
\end{pgfscope}%
\begin{pgfscope}%
\pgfpathrectangle{\pgfqpoint{1.000000in}{1.000000in}}{\pgfqpoint{8.500000in}{1.000000in}}%
\pgfusepath{clip}%
\pgfsetrectcap%
\pgfsetroundjoin%
\pgfsetlinewidth{0.803000pt}%
\definecolor{currentstroke}{rgb}{0.690196,0.690196,0.690196}%
\pgfsetstrokecolor{currentstroke}%
\pgfsetdash{}{0pt}%
\pgfpathmoveto{\pgfqpoint{1.000000in}{1.000000in}}%
\pgfpathlineto{\pgfqpoint{9.500000in}{1.000000in}}%
\pgfusepath{stroke}%
\end{pgfscope}%
\begin{pgfscope}%
\pgfsetbuttcap%
\pgfsetroundjoin%
\definecolor{currentfill}{rgb}{0.000000,0.000000,0.000000}%
\pgfsetfillcolor{currentfill}%
\pgfsetlinewidth{0.803000pt}%
\definecolor{currentstroke}{rgb}{0.000000,0.000000,0.000000}%
\pgfsetstrokecolor{currentstroke}%
\pgfsetdash{}{0pt}%
\pgfsys@defobject{currentmarker}{\pgfqpoint{-0.048611in}{0.000000in}}{\pgfqpoint{-0.000000in}{0.000000in}}{%
\pgfpathmoveto{\pgfqpoint{-0.000000in}{0.000000in}}%
\pgfpathlineto{\pgfqpoint{-0.048611in}{0.000000in}}%
\pgfusepath{stroke,fill}%
}%
\begin{pgfscope}%
\pgfsys@transformshift{1.000000in}{1.000000in}%
\pgfsys@useobject{currentmarker}{}%
\end{pgfscope}%
\end{pgfscope}%
\begin{pgfscope}%
\definecolor{textcolor}{rgb}{0.000000,0.000000,0.000000}%
\pgfsetstrokecolor{textcolor}%
\pgfsetfillcolor{textcolor}%
\pgftext[x=0.546626in, y=0.899981in, left, base]{\color{textcolor}\sffamily\fontsize{20.000000}{24.000000}\selectfont \(\displaystyle {-5}\)}%
\end{pgfscope}%
\begin{pgfscope}%
\pgfpathrectangle{\pgfqpoint{1.000000in}{1.000000in}}{\pgfqpoint{8.500000in}{1.000000in}}%
\pgfusepath{clip}%
\pgfsetrectcap%
\pgfsetroundjoin%
\pgfsetlinewidth{0.803000pt}%
\definecolor{currentstroke}{rgb}{0.690196,0.690196,0.690196}%
\pgfsetstrokecolor{currentstroke}%
\pgfsetdash{}{0pt}%
\pgfpathmoveto{\pgfqpoint{1.000000in}{1.500000in}}%
\pgfpathlineto{\pgfqpoint{9.500000in}{1.500000in}}%
\pgfusepath{stroke}%
\end{pgfscope}%
\begin{pgfscope}%
\pgfsetbuttcap%
\pgfsetroundjoin%
\definecolor{currentfill}{rgb}{0.000000,0.000000,0.000000}%
\pgfsetfillcolor{currentfill}%
\pgfsetlinewidth{0.803000pt}%
\definecolor{currentstroke}{rgb}{0.000000,0.000000,0.000000}%
\pgfsetstrokecolor{currentstroke}%
\pgfsetdash{}{0pt}%
\pgfsys@defobject{currentmarker}{\pgfqpoint{-0.048611in}{0.000000in}}{\pgfqpoint{-0.000000in}{0.000000in}}{%
\pgfpathmoveto{\pgfqpoint{-0.000000in}{0.000000in}}%
\pgfpathlineto{\pgfqpoint{-0.048611in}{0.000000in}}%
\pgfusepath{stroke,fill}%
}%
\begin{pgfscope}%
\pgfsys@transformshift{1.000000in}{1.500000in}%
\pgfsys@useobject{currentmarker}{}%
\end{pgfscope}%
\end{pgfscope}%
\begin{pgfscope}%
\definecolor{textcolor}{rgb}{0.000000,0.000000,0.000000}%
\pgfsetstrokecolor{textcolor}%
\pgfsetfillcolor{textcolor}%
\pgftext[x=0.770670in, y=1.399981in, left, base]{\color{textcolor}\sffamily\fontsize{20.000000}{24.000000}\selectfont \(\displaystyle {0}\)}%
\end{pgfscope}%
\begin{pgfscope}%
\pgfpathrectangle{\pgfqpoint{1.000000in}{1.000000in}}{\pgfqpoint{8.500000in}{1.000000in}}%
\pgfusepath{clip}%
\pgfsetrectcap%
\pgfsetroundjoin%
\pgfsetlinewidth{0.803000pt}%
\definecolor{currentstroke}{rgb}{0.690196,0.690196,0.690196}%
\pgfsetstrokecolor{currentstroke}%
\pgfsetdash{}{0pt}%
\pgfpathmoveto{\pgfqpoint{1.000000in}{2.000000in}}%
\pgfpathlineto{\pgfqpoint{9.500000in}{2.000000in}}%
\pgfusepath{stroke}%
\end{pgfscope}%
\begin{pgfscope}%
\pgfsetbuttcap%
\pgfsetroundjoin%
\definecolor{currentfill}{rgb}{0.000000,0.000000,0.000000}%
\pgfsetfillcolor{currentfill}%
\pgfsetlinewidth{0.803000pt}%
\definecolor{currentstroke}{rgb}{0.000000,0.000000,0.000000}%
\pgfsetstrokecolor{currentstroke}%
\pgfsetdash{}{0pt}%
\pgfsys@defobject{currentmarker}{\pgfqpoint{-0.048611in}{0.000000in}}{\pgfqpoint{-0.000000in}{0.000000in}}{%
\pgfpathmoveto{\pgfqpoint{-0.000000in}{0.000000in}}%
\pgfpathlineto{\pgfqpoint{-0.048611in}{0.000000in}}%
\pgfusepath{stroke,fill}%
}%
\begin{pgfscope}%
\pgfsys@transformshift{1.000000in}{2.000000in}%
\pgfsys@useobject{currentmarker}{}%
\end{pgfscope}%
\end{pgfscope}%
\begin{pgfscope}%
\definecolor{textcolor}{rgb}{0.000000,0.000000,0.000000}%
\pgfsetstrokecolor{textcolor}%
\pgfsetfillcolor{textcolor}%
\pgftext[x=0.770670in, y=1.899981in, left, base]{\color{textcolor}\sffamily\fontsize{20.000000}{24.000000}\selectfont \(\displaystyle {5}\)}%
\end{pgfscope}%
\begin{pgfscope}%
\definecolor{textcolor}{rgb}{0.000000,0.000000,0.000000}%
\pgfsetstrokecolor{textcolor}%
\pgfsetfillcolor{textcolor}%
\pgftext[x=0.491071in,y=1.500000in,,bottom,rotate=90.000000]{\color{textcolor}\sffamily\fontsize{20.000000}{24.000000}\selectfont \(\displaystyle \mathrm{Voltage}/\si{mV}\)}%
\end{pgfscope}%
\begin{pgfscope}%
\pgfsetrectcap%
\pgfsetmiterjoin%
\pgfsetlinewidth{0.803000pt}%
\definecolor{currentstroke}{rgb}{0.000000,0.000000,0.000000}%
\pgfsetstrokecolor{currentstroke}%
\pgfsetdash{}{0pt}%
\pgfpathmoveto{\pgfqpoint{1.000000in}{1.000000in}}%
\pgfpathlineto{\pgfqpoint{1.000000in}{2.000000in}}%
\pgfusepath{stroke}%
\end{pgfscope}%
\begin{pgfscope}%
\pgfsetrectcap%
\pgfsetmiterjoin%
\pgfsetlinewidth{0.803000pt}%
\definecolor{currentstroke}{rgb}{0.000000,0.000000,0.000000}%
\pgfsetstrokecolor{currentstroke}%
\pgfsetdash{}{0pt}%
\pgfpathmoveto{\pgfqpoint{9.500000in}{1.000000in}}%
\pgfpathlineto{\pgfqpoint{9.500000in}{2.000000in}}%
\pgfusepath{stroke}%
\end{pgfscope}%
\begin{pgfscope}%
\pgfsetrectcap%
\pgfsetmiterjoin%
\pgfsetlinewidth{0.803000pt}%
\definecolor{currentstroke}{rgb}{0.000000,0.000000,0.000000}%
\pgfsetstrokecolor{currentstroke}%
\pgfsetdash{}{0pt}%
\pgfpathmoveto{\pgfqpoint{1.000000in}{1.000000in}}%
\pgfpathlineto{\pgfqpoint{9.500000in}{1.000000in}}%
\pgfusepath{stroke}%
\end{pgfscope}%
\begin{pgfscope}%
\pgfsetrectcap%
\pgfsetmiterjoin%
\pgfsetlinewidth{0.803000pt}%
\definecolor{currentstroke}{rgb}{0.000000,0.000000,0.000000}%
\pgfsetstrokecolor{currentstroke}%
\pgfsetdash{}{0pt}%
\pgfpathmoveto{\pgfqpoint{1.000000in}{2.000000in}}%
\pgfpathlineto{\pgfqpoint{9.500000in}{2.000000in}}%
\pgfusepath{stroke}%
\end{pgfscope}%
\begin{pgfscope}%
\pgfsetbuttcap%
\pgfsetmiterjoin%
\definecolor{currentfill}{rgb}{1.000000,1.000000,1.000000}%
\pgfsetfillcolor{currentfill}%
\pgfsetfillopacity{0.800000}%
\pgfsetlinewidth{1.003750pt}%
\definecolor{currentstroke}{rgb}{0.800000,0.800000,0.800000}%
\pgfsetstrokecolor{currentstroke}%
\pgfsetstrokeopacity{0.800000}%
\pgfsetdash{}{0pt}%
\pgfpathmoveto{\pgfqpoint{6.921366in}{1.382821in}}%
\pgfpathlineto{\pgfqpoint{9.305556in}{1.382821in}}%
\pgfpathquadraticcurveto{\pgfqpoint{9.361111in}{1.382821in}}{\pgfqpoint{9.361111in}{1.438377in}}%
\pgfpathlineto{\pgfqpoint{9.361111in}{1.805556in}}%
\pgfpathquadraticcurveto{\pgfqpoint{9.361111in}{1.861111in}}{\pgfqpoint{9.305556in}{1.861111in}}%
\pgfpathlineto{\pgfqpoint{6.921366in}{1.861111in}}%
\pgfpathquadraticcurveto{\pgfqpoint{6.865810in}{1.861111in}}{\pgfqpoint{6.865810in}{1.805556in}}%
\pgfpathlineto{\pgfqpoint{6.865810in}{1.438377in}}%
\pgfpathquadraticcurveto{\pgfqpoint{6.865810in}{1.382821in}}{\pgfqpoint{6.921366in}{1.382821in}}%
\pgfpathclose%
\pgfusepath{stroke,fill}%
\end{pgfscope}%
\begin{pgfscope}%
\pgfsetbuttcap%
\pgfsetroundjoin%
\definecolor{currentfill}{rgb}{0.000000,0.000000,0.000000}%
\pgfsetfillcolor{currentfill}%
\pgfsetlinewidth{1.003750pt}%
\definecolor{currentstroke}{rgb}{0.000000,0.000000,0.000000}%
\pgfsetstrokecolor{currentstroke}%
\pgfsetdash{}{0pt}%
\pgfsys@defobject{currentmarker}{\pgfqpoint{-0.013889in}{-0.013889in}}{\pgfqpoint{0.013889in}{0.013889in}}{%
\pgfpathmoveto{\pgfqpoint{0.000000in}{-0.013889in}}%
\pgfpathcurveto{\pgfqpoint{0.003683in}{-0.013889in}}{\pgfqpoint{0.007216in}{-0.012425in}}{\pgfqpoint{0.009821in}{-0.009821in}}%
\pgfpathcurveto{\pgfqpoint{0.012425in}{-0.007216in}}{\pgfqpoint{0.013889in}{-0.003683in}}{\pgfqpoint{0.013889in}{0.000000in}}%
\pgfpathcurveto{\pgfqpoint{0.013889in}{0.003683in}}{\pgfqpoint{0.012425in}{0.007216in}}{\pgfqpoint{0.009821in}{0.009821in}}%
\pgfpathcurveto{\pgfqpoint{0.007216in}{0.012425in}}{\pgfqpoint{0.003683in}{0.013889in}}{\pgfqpoint{0.000000in}{0.013889in}}%
\pgfpathcurveto{\pgfqpoint{-0.003683in}{0.013889in}}{\pgfqpoint{-0.007216in}{0.012425in}}{\pgfqpoint{-0.009821in}{0.009821in}}%
\pgfpathcurveto{\pgfqpoint{-0.012425in}{0.007216in}}{\pgfqpoint{-0.013889in}{0.003683in}}{\pgfqpoint{-0.013889in}{0.000000in}}%
\pgfpathcurveto{\pgfqpoint{-0.013889in}{-0.003683in}}{\pgfqpoint{-0.012425in}{-0.007216in}}{\pgfqpoint{-0.009821in}{-0.009821in}}%
\pgfpathcurveto{\pgfqpoint{-0.007216in}{-0.012425in}}{\pgfqpoint{-0.003683in}{-0.013889in}}{\pgfqpoint{0.000000in}{-0.013889in}}%
\pgfpathclose%
\pgfusepath{stroke,fill}%
}%
\begin{pgfscope}%
\pgfsys@transformshift{7.254699in}{1.622878in}%
\pgfsys@useobject{currentmarker}{}%
\end{pgfscope}%
\end{pgfscope}%
\begin{pgfscope}%
\definecolor{textcolor}{rgb}{0.000000,0.000000,0.000000}%
\pgfsetstrokecolor{textcolor}%
\pgfsetfillcolor{textcolor}%
\pgftext[x=7.754699in,y=1.549962in,left,base]{\color{textcolor}\sffamily\fontsize{20.000000}{24.000000}\selectfont residual wave}%
\end{pgfscope}%
\end{pgfpicture}%
\makeatother%
\endgroup%
}
    \caption{\label{fig:fsmp} An example giving \\ $\hat{t}_0 - t_0=\SI{1.93}{ns}$, $\mathrm{RSS}=\SI{32.5}{mV^2}$, $D_\mathrm{w}=\SI{0.71}{ns}$.}
  \end{subfigure}
  \caption{\label{fig:fsmp-performance}Demonstration of FSMP with $\num[retain-unity-mantissa=false]{1e4}$ waveforms in~\subref{fig:fsmp-npe} and one waveform in~\subref{fig:fsmp} sampled from the same setup as figure~\ref{fig:method}.  FSMP reconstructs the waveform and charges flawlessly.}
\end{figure}
In terms of $D_\mathrm{w}$, figure~\ref{fig:fsmp-npe} shows that FSMP is on par with CNN in figure~\ref{fig:cnn-npe}.  Figure~\ref{fig:fsmp} is a perfect reconstruction example where the true and reconstructed charges and waveforms overlap.  A bias of $\hat{t}_0 - t_0=\SI{-3.97}{ns}$ aligns with $\hat{t}_\mathrm{ALL}$ in eq.~\eqref{eq:2}, which will be covered in section~\ref{subsec:timeresolution}.  The superior performance of FSMP attributes to sparsity and positiveness of $q'_i$, correct modeling of $V_\mathrm{PE}$, $q'$ distribution and white noise.

Estimators for $t_0$ and $\mu$ in eq.~\eqref{eq:fsmpcharge} is an elegant interface to event reconstruction, eliminating the need of $\hat{t}_\mathrm{KL}$ and $\hat{\mu}_\mathrm{KL}$ in section~\ref{sec:pseudo}.

\section{Summary and Discussion} % (fold)
\label{sec:discussion}

\subsection{Program Efficiency}

The efficiency of 5 performant methods on our simulation dataset show in Table~\ref{fig:efficiency}. The CNN on GPU is faster than LucyDDM and FBMP, and much faster than gradient descent.  Number of iteration for LucyDDM and gradient descent and $t'_i$ sampling frequency of FBMP and MCMC can be decreased to trade for speed.  The time consumed by FBMP contains the initialization time of LucyDDM pre-conditioner.

\begin{table}[H]
    \centering
    \caption{\label{fig:efficiency} Reconstruction Efficiency.}
    \begin{tabular}{cc}
        \hline
        & Performance/$10^{5}$Waveform \\
        \hline
        CNN & 7.5s (GPU\tablefootnote{one graphics card of NVIDIA\textsuperscript{\textregistered} Tesla\textsuperscript{\textregistered} K80.}) \\
        FBMP & 432.7s (CPU\tablefootnote{100 CPU cores of AMD EYPC\texttrademark\ 7702}) \\
        Gradient Descent & 1072.5s (CPU) \\
        LucyDDM & 362.7s (CPU) \\
        MCMC & 15539.1s (CPU) \\
        \hline
    \end{tabular}
\end{table}

\subsection{Posterior Charge Distribution}

\begin{figure}[H]
    \centering
    \resizebox{0.6\textwidth}{!}{%% Creator: Matplotlib, PGF backend
%%
%% To include the figure in your LaTeX document, write
%%   \input{<filename>.pgf}
%%
%% Make sure the required packages are loaded in your preamble
%%   \usepackage{pgf}
%%
%% and, on pdftex
%%   \usepackage[utf8]{inputenc}\DeclareUnicodeCharacter{2212}{-}
%%
%% or, on luatex and xetex
%%   \usepackage{unicode-math}
%%
%% Figures using additional raster images can only be included by \input if
%% they are in the same directory as the main LaTeX file. For loading figures
%% from other directories you can use the `import` package
%%   \usepackage{import}
%%
%% and then include the figures with
%%   \import{<path to file>}{<filename>.pgf}
%%
%% Matplotlib used the following preamble
%%   \usepackage[detect-all,locale=DE]{siunitx}
%%
\begingroup%
\makeatletter%
\begin{pgfpicture}%
\pgfpathrectangle{\pgfpointorigin}{\pgfqpoint{8.000000in}{6.000000in}}%
\pgfusepath{use as bounding box, clip}%
\begin{pgfscope}%
\pgfsetbuttcap%
\pgfsetmiterjoin%
\definecolor{currentfill}{rgb}{1.000000,1.000000,1.000000}%
\pgfsetfillcolor{currentfill}%
\pgfsetlinewidth{0.000000pt}%
\definecolor{currentstroke}{rgb}{1.000000,1.000000,1.000000}%
\pgfsetstrokecolor{currentstroke}%
\pgfsetdash{}{0pt}%
\pgfpathmoveto{\pgfqpoint{0.000000in}{0.000000in}}%
\pgfpathlineto{\pgfqpoint{8.000000in}{0.000000in}}%
\pgfpathlineto{\pgfqpoint{8.000000in}{6.000000in}}%
\pgfpathlineto{\pgfqpoint{0.000000in}{6.000000in}}%
\pgfpathclose%
\pgfusepath{fill}%
\end{pgfscope}%
\begin{pgfscope}%
\pgfsetbuttcap%
\pgfsetmiterjoin%
\definecolor{currentfill}{rgb}{1.000000,1.000000,1.000000}%
\pgfsetfillcolor{currentfill}%
\pgfsetlinewidth{0.000000pt}%
\definecolor{currentstroke}{rgb}{0.000000,0.000000,0.000000}%
\pgfsetstrokecolor{currentstroke}%
\pgfsetstrokeopacity{0.000000}%
\pgfsetdash{}{0pt}%
\pgfpathmoveto{\pgfqpoint{1.000000in}{0.660000in}}%
\pgfpathlineto{\pgfqpoint{7.200000in}{0.660000in}}%
\pgfpathlineto{\pgfqpoint{7.200000in}{5.280000in}}%
\pgfpathlineto{\pgfqpoint{1.000000in}{5.280000in}}%
\pgfpathclose%
\pgfusepath{fill}%
\end{pgfscope}%
\begin{pgfscope}%
\pgfsetbuttcap%
\pgfsetroundjoin%
\definecolor{currentfill}{rgb}{0.000000,0.000000,0.000000}%
\pgfsetfillcolor{currentfill}%
\pgfsetlinewidth{0.803000pt}%
\definecolor{currentstroke}{rgb}{0.000000,0.000000,0.000000}%
\pgfsetstrokecolor{currentstroke}%
\pgfsetdash{}{0pt}%
\pgfsys@defobject{currentmarker}{\pgfqpoint{0.000000in}{-0.048611in}}{\pgfqpoint{0.000000in}{0.000000in}}{%
\pgfpathmoveto{\pgfqpoint{0.000000in}{0.000000in}}%
\pgfpathlineto{\pgfqpoint{0.000000in}{-0.048611in}}%
\pgfusepath{stroke,fill}%
}%
\begin{pgfscope}%
\pgfsys@transformshift{1.000000in}{0.660000in}%
\pgfsys@useobject{currentmarker}{}%
\end{pgfscope}%
\end{pgfscope}%
\begin{pgfscope}%
\definecolor{textcolor}{rgb}{0.000000,0.000000,0.000000}%
\pgfsetstrokecolor{textcolor}%
\pgfsetfillcolor{textcolor}%
\pgftext[x=1.000000in,y=0.562778in,,top]{\color{textcolor}\sffamily\fontsize{20.000000}{24.000000}\selectfont \(\displaystyle {0}\)}%
\end{pgfscope}%
\begin{pgfscope}%
\pgfsetbuttcap%
\pgfsetroundjoin%
\definecolor{currentfill}{rgb}{0.000000,0.000000,0.000000}%
\pgfsetfillcolor{currentfill}%
\pgfsetlinewidth{0.803000pt}%
\definecolor{currentstroke}{rgb}{0.000000,0.000000,0.000000}%
\pgfsetstrokecolor{currentstroke}%
\pgfsetdash{}{0pt}%
\pgfsys@defobject{currentmarker}{\pgfqpoint{0.000000in}{-0.048611in}}{\pgfqpoint{0.000000in}{0.000000in}}{%
\pgfpathmoveto{\pgfqpoint{0.000000in}{0.000000in}}%
\pgfpathlineto{\pgfqpoint{0.000000in}{-0.048611in}}%
\pgfusepath{stroke,fill}%
}%
\begin{pgfscope}%
\pgfsys@transformshift{1.885714in}{0.660000in}%
\pgfsys@useobject{currentmarker}{}%
\end{pgfscope}%
\end{pgfscope}%
\begin{pgfscope}%
\definecolor{textcolor}{rgb}{0.000000,0.000000,0.000000}%
\pgfsetstrokecolor{textcolor}%
\pgfsetfillcolor{textcolor}%
\pgftext[x=1.885714in,y=0.562778in,,top]{\color{textcolor}\sffamily\fontsize{20.000000}{24.000000}\selectfont \(\displaystyle {50}\)}%
\end{pgfscope}%
\begin{pgfscope}%
\pgfsetbuttcap%
\pgfsetroundjoin%
\definecolor{currentfill}{rgb}{0.000000,0.000000,0.000000}%
\pgfsetfillcolor{currentfill}%
\pgfsetlinewidth{0.803000pt}%
\definecolor{currentstroke}{rgb}{0.000000,0.000000,0.000000}%
\pgfsetstrokecolor{currentstroke}%
\pgfsetdash{}{0pt}%
\pgfsys@defobject{currentmarker}{\pgfqpoint{0.000000in}{-0.048611in}}{\pgfqpoint{0.000000in}{0.000000in}}{%
\pgfpathmoveto{\pgfqpoint{0.000000in}{0.000000in}}%
\pgfpathlineto{\pgfqpoint{0.000000in}{-0.048611in}}%
\pgfusepath{stroke,fill}%
}%
\begin{pgfscope}%
\pgfsys@transformshift{2.771429in}{0.660000in}%
\pgfsys@useobject{currentmarker}{}%
\end{pgfscope}%
\end{pgfscope}%
\begin{pgfscope}%
\definecolor{textcolor}{rgb}{0.000000,0.000000,0.000000}%
\pgfsetstrokecolor{textcolor}%
\pgfsetfillcolor{textcolor}%
\pgftext[x=2.771429in,y=0.562778in,,top]{\color{textcolor}\sffamily\fontsize{20.000000}{24.000000}\selectfont \(\displaystyle {100}\)}%
\end{pgfscope}%
\begin{pgfscope}%
\pgfsetbuttcap%
\pgfsetroundjoin%
\definecolor{currentfill}{rgb}{0.000000,0.000000,0.000000}%
\pgfsetfillcolor{currentfill}%
\pgfsetlinewidth{0.803000pt}%
\definecolor{currentstroke}{rgb}{0.000000,0.000000,0.000000}%
\pgfsetstrokecolor{currentstroke}%
\pgfsetdash{}{0pt}%
\pgfsys@defobject{currentmarker}{\pgfqpoint{0.000000in}{-0.048611in}}{\pgfqpoint{0.000000in}{0.000000in}}{%
\pgfpathmoveto{\pgfqpoint{0.000000in}{0.000000in}}%
\pgfpathlineto{\pgfqpoint{0.000000in}{-0.048611in}}%
\pgfusepath{stroke,fill}%
}%
\begin{pgfscope}%
\pgfsys@transformshift{3.657143in}{0.660000in}%
\pgfsys@useobject{currentmarker}{}%
\end{pgfscope}%
\end{pgfscope}%
\begin{pgfscope}%
\definecolor{textcolor}{rgb}{0.000000,0.000000,0.000000}%
\pgfsetstrokecolor{textcolor}%
\pgfsetfillcolor{textcolor}%
\pgftext[x=3.657143in,y=0.562778in,,top]{\color{textcolor}\sffamily\fontsize{20.000000}{24.000000}\selectfont \(\displaystyle {150}\)}%
\end{pgfscope}%
\begin{pgfscope}%
\pgfsetbuttcap%
\pgfsetroundjoin%
\definecolor{currentfill}{rgb}{0.000000,0.000000,0.000000}%
\pgfsetfillcolor{currentfill}%
\pgfsetlinewidth{0.803000pt}%
\definecolor{currentstroke}{rgb}{0.000000,0.000000,0.000000}%
\pgfsetstrokecolor{currentstroke}%
\pgfsetdash{}{0pt}%
\pgfsys@defobject{currentmarker}{\pgfqpoint{0.000000in}{-0.048611in}}{\pgfqpoint{0.000000in}{0.000000in}}{%
\pgfpathmoveto{\pgfqpoint{0.000000in}{0.000000in}}%
\pgfpathlineto{\pgfqpoint{0.000000in}{-0.048611in}}%
\pgfusepath{stroke,fill}%
}%
\begin{pgfscope}%
\pgfsys@transformshift{4.542857in}{0.660000in}%
\pgfsys@useobject{currentmarker}{}%
\end{pgfscope}%
\end{pgfscope}%
\begin{pgfscope}%
\definecolor{textcolor}{rgb}{0.000000,0.000000,0.000000}%
\pgfsetstrokecolor{textcolor}%
\pgfsetfillcolor{textcolor}%
\pgftext[x=4.542857in,y=0.562778in,,top]{\color{textcolor}\sffamily\fontsize{20.000000}{24.000000}\selectfont \(\displaystyle {200}\)}%
\end{pgfscope}%
\begin{pgfscope}%
\pgfsetbuttcap%
\pgfsetroundjoin%
\definecolor{currentfill}{rgb}{0.000000,0.000000,0.000000}%
\pgfsetfillcolor{currentfill}%
\pgfsetlinewidth{0.803000pt}%
\definecolor{currentstroke}{rgb}{0.000000,0.000000,0.000000}%
\pgfsetstrokecolor{currentstroke}%
\pgfsetdash{}{0pt}%
\pgfsys@defobject{currentmarker}{\pgfqpoint{0.000000in}{-0.048611in}}{\pgfqpoint{0.000000in}{0.000000in}}{%
\pgfpathmoveto{\pgfqpoint{0.000000in}{0.000000in}}%
\pgfpathlineto{\pgfqpoint{0.000000in}{-0.048611in}}%
\pgfusepath{stroke,fill}%
}%
\begin{pgfscope}%
\pgfsys@transformshift{5.428571in}{0.660000in}%
\pgfsys@useobject{currentmarker}{}%
\end{pgfscope}%
\end{pgfscope}%
\begin{pgfscope}%
\definecolor{textcolor}{rgb}{0.000000,0.000000,0.000000}%
\pgfsetstrokecolor{textcolor}%
\pgfsetfillcolor{textcolor}%
\pgftext[x=5.428571in,y=0.562778in,,top]{\color{textcolor}\sffamily\fontsize{20.000000}{24.000000}\selectfont \(\displaystyle {250}\)}%
\end{pgfscope}%
\begin{pgfscope}%
\pgfsetbuttcap%
\pgfsetroundjoin%
\definecolor{currentfill}{rgb}{0.000000,0.000000,0.000000}%
\pgfsetfillcolor{currentfill}%
\pgfsetlinewidth{0.803000pt}%
\definecolor{currentstroke}{rgb}{0.000000,0.000000,0.000000}%
\pgfsetstrokecolor{currentstroke}%
\pgfsetdash{}{0pt}%
\pgfsys@defobject{currentmarker}{\pgfqpoint{0.000000in}{-0.048611in}}{\pgfqpoint{0.000000in}{0.000000in}}{%
\pgfpathmoveto{\pgfqpoint{0.000000in}{0.000000in}}%
\pgfpathlineto{\pgfqpoint{0.000000in}{-0.048611in}}%
\pgfusepath{stroke,fill}%
}%
\begin{pgfscope}%
\pgfsys@transformshift{6.314286in}{0.660000in}%
\pgfsys@useobject{currentmarker}{}%
\end{pgfscope}%
\end{pgfscope}%
\begin{pgfscope}%
\definecolor{textcolor}{rgb}{0.000000,0.000000,0.000000}%
\pgfsetstrokecolor{textcolor}%
\pgfsetfillcolor{textcolor}%
\pgftext[x=6.314286in,y=0.562778in,,top]{\color{textcolor}\sffamily\fontsize{20.000000}{24.000000}\selectfont \(\displaystyle {300}\)}%
\end{pgfscope}%
\begin{pgfscope}%
\pgfsetbuttcap%
\pgfsetroundjoin%
\definecolor{currentfill}{rgb}{0.000000,0.000000,0.000000}%
\pgfsetfillcolor{currentfill}%
\pgfsetlinewidth{0.803000pt}%
\definecolor{currentstroke}{rgb}{0.000000,0.000000,0.000000}%
\pgfsetstrokecolor{currentstroke}%
\pgfsetdash{}{0pt}%
\pgfsys@defobject{currentmarker}{\pgfqpoint{0.000000in}{-0.048611in}}{\pgfqpoint{0.000000in}{0.000000in}}{%
\pgfpathmoveto{\pgfqpoint{0.000000in}{0.000000in}}%
\pgfpathlineto{\pgfqpoint{0.000000in}{-0.048611in}}%
\pgfusepath{stroke,fill}%
}%
\begin{pgfscope}%
\pgfsys@transformshift{7.200000in}{0.660000in}%
\pgfsys@useobject{currentmarker}{}%
\end{pgfscope}%
\end{pgfscope}%
\begin{pgfscope}%
\definecolor{textcolor}{rgb}{0.000000,0.000000,0.000000}%
\pgfsetstrokecolor{textcolor}%
\pgfsetfillcolor{textcolor}%
\pgftext[x=7.200000in,y=0.562778in,,top]{\color{textcolor}\sffamily\fontsize{20.000000}{24.000000}\selectfont \(\displaystyle {350}\)}%
\end{pgfscope}%
\begin{pgfscope}%
\definecolor{textcolor}{rgb}{0.000000,0.000000,0.000000}%
\pgfsetstrokecolor{textcolor}%
\pgfsetfillcolor{textcolor}%
\pgftext[x=4.100000in,y=0.251155in,,top]{\color{textcolor}\sffamily\fontsize{20.000000}{24.000000}\selectfont \(\displaystyle \mathrm{Charge}/\si{mV\cdot ns}\)}%
\end{pgfscope}%
\begin{pgfscope}%
\pgfsetbuttcap%
\pgfsetroundjoin%
\definecolor{currentfill}{rgb}{0.000000,0.000000,0.000000}%
\pgfsetfillcolor{currentfill}%
\pgfsetlinewidth{0.803000pt}%
\definecolor{currentstroke}{rgb}{0.000000,0.000000,0.000000}%
\pgfsetstrokecolor{currentstroke}%
\pgfsetdash{}{0pt}%
\pgfsys@defobject{currentmarker}{\pgfqpoint{-0.048611in}{0.000000in}}{\pgfqpoint{-0.000000in}{0.000000in}}{%
\pgfpathmoveto{\pgfqpoint{-0.000000in}{0.000000in}}%
\pgfpathlineto{\pgfqpoint{-0.048611in}{0.000000in}}%
\pgfusepath{stroke,fill}%
}%
\begin{pgfscope}%
\pgfsys@transformshift{1.000000in}{0.660000in}%
\pgfsys@useobject{currentmarker}{}%
\end{pgfscope}%
\end{pgfscope}%
\begin{pgfscope}%
\definecolor{textcolor}{rgb}{0.000000,0.000000,0.000000}%
\pgfsetstrokecolor{textcolor}%
\pgfsetfillcolor{textcolor}%
\pgftext[x=0.296001in, y=0.559981in, left, base]{\color{textcolor}\sffamily\fontsize{20.000000}{24.000000}\selectfont \(\displaystyle {0.000}\)}%
\end{pgfscope}%
\begin{pgfscope}%
\pgfsetbuttcap%
\pgfsetroundjoin%
\definecolor{currentfill}{rgb}{0.000000,0.000000,0.000000}%
\pgfsetfillcolor{currentfill}%
\pgfsetlinewidth{0.803000pt}%
\definecolor{currentstroke}{rgb}{0.000000,0.000000,0.000000}%
\pgfsetstrokecolor{currentstroke}%
\pgfsetdash{}{0pt}%
\pgfsys@defobject{currentmarker}{\pgfqpoint{-0.048611in}{0.000000in}}{\pgfqpoint{-0.000000in}{0.000000in}}{%
\pgfpathmoveto{\pgfqpoint{-0.000000in}{0.000000in}}%
\pgfpathlineto{\pgfqpoint{-0.048611in}{0.000000in}}%
\pgfusepath{stroke,fill}%
}%
\begin{pgfscope}%
\pgfsys@transformshift{1.000000in}{1.370769in}%
\pgfsys@useobject{currentmarker}{}%
\end{pgfscope}%
\end{pgfscope}%
\begin{pgfscope}%
\definecolor{textcolor}{rgb}{0.000000,0.000000,0.000000}%
\pgfsetstrokecolor{textcolor}%
\pgfsetfillcolor{textcolor}%
\pgftext[x=0.296001in, y=1.270750in, left, base]{\color{textcolor}\sffamily\fontsize{20.000000}{24.000000}\selectfont \(\displaystyle {0.002}\)}%
\end{pgfscope}%
\begin{pgfscope}%
\pgfsetbuttcap%
\pgfsetroundjoin%
\definecolor{currentfill}{rgb}{0.000000,0.000000,0.000000}%
\pgfsetfillcolor{currentfill}%
\pgfsetlinewidth{0.803000pt}%
\definecolor{currentstroke}{rgb}{0.000000,0.000000,0.000000}%
\pgfsetstrokecolor{currentstroke}%
\pgfsetdash{}{0pt}%
\pgfsys@defobject{currentmarker}{\pgfqpoint{-0.048611in}{0.000000in}}{\pgfqpoint{-0.000000in}{0.000000in}}{%
\pgfpathmoveto{\pgfqpoint{-0.000000in}{0.000000in}}%
\pgfpathlineto{\pgfqpoint{-0.048611in}{0.000000in}}%
\pgfusepath{stroke,fill}%
}%
\begin{pgfscope}%
\pgfsys@transformshift{1.000000in}{2.081538in}%
\pgfsys@useobject{currentmarker}{}%
\end{pgfscope}%
\end{pgfscope}%
\begin{pgfscope}%
\definecolor{textcolor}{rgb}{0.000000,0.000000,0.000000}%
\pgfsetstrokecolor{textcolor}%
\pgfsetfillcolor{textcolor}%
\pgftext[x=0.296001in, y=1.981519in, left, base]{\color{textcolor}\sffamily\fontsize{20.000000}{24.000000}\selectfont \(\displaystyle {0.004}\)}%
\end{pgfscope}%
\begin{pgfscope}%
\pgfsetbuttcap%
\pgfsetroundjoin%
\definecolor{currentfill}{rgb}{0.000000,0.000000,0.000000}%
\pgfsetfillcolor{currentfill}%
\pgfsetlinewidth{0.803000pt}%
\definecolor{currentstroke}{rgb}{0.000000,0.000000,0.000000}%
\pgfsetstrokecolor{currentstroke}%
\pgfsetdash{}{0pt}%
\pgfsys@defobject{currentmarker}{\pgfqpoint{-0.048611in}{0.000000in}}{\pgfqpoint{-0.000000in}{0.000000in}}{%
\pgfpathmoveto{\pgfqpoint{-0.000000in}{0.000000in}}%
\pgfpathlineto{\pgfqpoint{-0.048611in}{0.000000in}}%
\pgfusepath{stroke,fill}%
}%
\begin{pgfscope}%
\pgfsys@transformshift{1.000000in}{2.792308in}%
\pgfsys@useobject{currentmarker}{}%
\end{pgfscope}%
\end{pgfscope}%
\begin{pgfscope}%
\definecolor{textcolor}{rgb}{0.000000,0.000000,0.000000}%
\pgfsetstrokecolor{textcolor}%
\pgfsetfillcolor{textcolor}%
\pgftext[x=0.296001in, y=2.692288in, left, base]{\color{textcolor}\sffamily\fontsize{20.000000}{24.000000}\selectfont \(\displaystyle {0.006}\)}%
\end{pgfscope}%
\begin{pgfscope}%
\pgfsetbuttcap%
\pgfsetroundjoin%
\definecolor{currentfill}{rgb}{0.000000,0.000000,0.000000}%
\pgfsetfillcolor{currentfill}%
\pgfsetlinewidth{0.803000pt}%
\definecolor{currentstroke}{rgb}{0.000000,0.000000,0.000000}%
\pgfsetstrokecolor{currentstroke}%
\pgfsetdash{}{0pt}%
\pgfsys@defobject{currentmarker}{\pgfqpoint{-0.048611in}{0.000000in}}{\pgfqpoint{-0.000000in}{0.000000in}}{%
\pgfpathmoveto{\pgfqpoint{-0.000000in}{0.000000in}}%
\pgfpathlineto{\pgfqpoint{-0.048611in}{0.000000in}}%
\pgfusepath{stroke,fill}%
}%
\begin{pgfscope}%
\pgfsys@transformshift{1.000000in}{3.503077in}%
\pgfsys@useobject{currentmarker}{}%
\end{pgfscope}%
\end{pgfscope}%
\begin{pgfscope}%
\definecolor{textcolor}{rgb}{0.000000,0.000000,0.000000}%
\pgfsetstrokecolor{textcolor}%
\pgfsetfillcolor{textcolor}%
\pgftext[x=0.296001in, y=3.403058in, left, base]{\color{textcolor}\sffamily\fontsize{20.000000}{24.000000}\selectfont \(\displaystyle {0.008}\)}%
\end{pgfscope}%
\begin{pgfscope}%
\pgfsetbuttcap%
\pgfsetroundjoin%
\definecolor{currentfill}{rgb}{0.000000,0.000000,0.000000}%
\pgfsetfillcolor{currentfill}%
\pgfsetlinewidth{0.803000pt}%
\definecolor{currentstroke}{rgb}{0.000000,0.000000,0.000000}%
\pgfsetstrokecolor{currentstroke}%
\pgfsetdash{}{0pt}%
\pgfsys@defobject{currentmarker}{\pgfqpoint{-0.048611in}{0.000000in}}{\pgfqpoint{-0.000000in}{0.000000in}}{%
\pgfpathmoveto{\pgfqpoint{-0.000000in}{0.000000in}}%
\pgfpathlineto{\pgfqpoint{-0.048611in}{0.000000in}}%
\pgfusepath{stroke,fill}%
}%
\begin{pgfscope}%
\pgfsys@transformshift{1.000000in}{4.213846in}%
\pgfsys@useobject{currentmarker}{}%
\end{pgfscope}%
\end{pgfscope}%
\begin{pgfscope}%
\definecolor{textcolor}{rgb}{0.000000,0.000000,0.000000}%
\pgfsetstrokecolor{textcolor}%
\pgfsetfillcolor{textcolor}%
\pgftext[x=0.296001in, y=4.113827in, left, base]{\color{textcolor}\sffamily\fontsize{20.000000}{24.000000}\selectfont \(\displaystyle {0.010}\)}%
\end{pgfscope}%
\begin{pgfscope}%
\pgfsetbuttcap%
\pgfsetroundjoin%
\definecolor{currentfill}{rgb}{0.000000,0.000000,0.000000}%
\pgfsetfillcolor{currentfill}%
\pgfsetlinewidth{0.803000pt}%
\definecolor{currentstroke}{rgb}{0.000000,0.000000,0.000000}%
\pgfsetstrokecolor{currentstroke}%
\pgfsetdash{}{0pt}%
\pgfsys@defobject{currentmarker}{\pgfqpoint{-0.048611in}{0.000000in}}{\pgfqpoint{-0.000000in}{0.000000in}}{%
\pgfpathmoveto{\pgfqpoint{-0.000000in}{0.000000in}}%
\pgfpathlineto{\pgfqpoint{-0.048611in}{0.000000in}}%
\pgfusepath{stroke,fill}%
}%
\begin{pgfscope}%
\pgfsys@transformshift{1.000000in}{4.924615in}%
\pgfsys@useobject{currentmarker}{}%
\end{pgfscope}%
\end{pgfscope}%
\begin{pgfscope}%
\definecolor{textcolor}{rgb}{0.000000,0.000000,0.000000}%
\pgfsetstrokecolor{textcolor}%
\pgfsetfillcolor{textcolor}%
\pgftext[x=0.296001in, y=4.824596in, left, base]{\color{textcolor}\sffamily\fontsize{20.000000}{24.000000}\selectfont \(\displaystyle {0.012}\)}%
\end{pgfscope}%
\begin{pgfscope}%
\definecolor{textcolor}{rgb}{0.000000,0.000000,0.000000}%
\pgfsetstrokecolor{textcolor}%
\pgfsetfillcolor{textcolor}%
\pgftext[x=0.240445in,y=2.970000in,,bottom,rotate=90.000000]{\color{textcolor}\sffamily\fontsize{20.000000}{24.000000}\selectfont \(\displaystyle \mathrm{Normalized\ Count}\)}%
\end{pgfscope}%
\begin{pgfscope}%
\pgfpathrectangle{\pgfqpoint{1.000000in}{0.660000in}}{\pgfqpoint{6.200000in}{4.620000in}}%
\pgfusepath{clip}%
\pgfsetbuttcap%
\pgfsetmiterjoin%
\pgfsetlinewidth{1.003750pt}%
\definecolor{currentstroke}{rgb}{0.000000,0.500000,0.000000}%
\pgfsetstrokecolor{currentstroke}%
\pgfsetstrokeopacity{0.500000}%
\pgfsetdash{}{0pt}%
\pgfpathmoveto{\pgfqpoint{1.001417in}{0.660000in}}%
\pgfpathlineto{\pgfqpoint{1.178560in}{0.660000in}}%
\pgfpathlineto{\pgfqpoint{1.178560in}{0.661190in}}%
\pgfpathlineto{\pgfqpoint{1.267131in}{0.661190in}}%
\pgfpathlineto{\pgfqpoint{1.267131in}{0.665951in}}%
\pgfpathlineto{\pgfqpoint{1.355703in}{0.665951in}}%
\pgfpathlineto{\pgfqpoint{1.355703in}{0.673092in}}%
\pgfpathlineto{\pgfqpoint{1.444274in}{0.673092in}}%
\pgfpathlineto{\pgfqpoint{1.444274in}{0.700465in}}%
\pgfpathlineto{\pgfqpoint{1.532846in}{0.700465in}}%
\pgfpathlineto{\pgfqpoint{1.532846in}{1.284829in}}%
\pgfpathlineto{\pgfqpoint{1.621417in}{1.284829in}}%
\pgfpathlineto{\pgfqpoint{1.621417in}{3.623472in}}%
\pgfpathlineto{\pgfqpoint{1.709989in}{3.623472in}}%
\pgfpathlineto{\pgfqpoint{1.709989in}{4.551789in}}%
\pgfpathlineto{\pgfqpoint{1.798560in}{4.551789in}}%
\pgfpathlineto{\pgfqpoint{1.798560in}{4.324471in}}%
\pgfpathlineto{\pgfqpoint{1.887131in}{4.324471in}}%
\pgfpathlineto{\pgfqpoint{1.887131in}{4.287576in}}%
\pgfpathlineto{\pgfqpoint{1.975703in}{4.287576in}}%
\pgfpathlineto{\pgfqpoint{1.975703in}{3.915059in}}%
\pgfpathlineto{\pgfqpoint{2.064274in}{3.915059in}}%
\pgfpathlineto{\pgfqpoint{2.064274in}{3.794854in}}%
\pgfpathlineto{\pgfqpoint{2.152846in}{3.794854in}}%
\pgfpathlineto{\pgfqpoint{2.152846in}{3.772241in}}%
\pgfpathlineto{\pgfqpoint{2.241417in}{3.772241in}}%
\pgfpathlineto{\pgfqpoint{2.241417in}{3.583007in}}%
\pgfpathlineto{\pgfqpoint{2.329989in}{3.583007in}}%
\pgfpathlineto{\pgfqpoint{2.329989in}{3.411626in}}%
\pgfpathlineto{\pgfqpoint{2.418560in}{3.411626in}}%
\pgfpathlineto{\pgfqpoint{2.418560in}{3.296181in}}%
\pgfpathlineto{\pgfqpoint{2.507131in}{3.296181in}}%
\pgfpathlineto{\pgfqpoint{2.507131in}{3.195019in}}%
\pgfpathlineto{\pgfqpoint{2.595703in}{3.195019in}}%
\pgfpathlineto{\pgfqpoint{2.595703in}{3.106948in}}%
\pgfpathlineto{\pgfqpoint{2.684274in}{3.106948in}}%
\pgfpathlineto{\pgfqpoint{2.684274in}{2.922474in}}%
\pgfpathlineto{\pgfqpoint{2.772846in}{2.922474in}}%
\pgfpathlineto{\pgfqpoint{2.772846in}{2.804650in}}%
\pgfpathlineto{\pgfqpoint{2.861417in}{2.804650in}}%
\pgfpathlineto{\pgfqpoint{2.861417in}{2.657071in}}%
\pgfpathlineto{\pgfqpoint{2.949989in}{2.657071in}}%
\pgfpathlineto{\pgfqpoint{2.949989in}{2.592803in}}%
\pgfpathlineto{\pgfqpoint{3.038560in}{2.592803in}}%
\pgfpathlineto{\pgfqpoint{3.038560in}{2.451175in}}%
\pgfpathlineto{\pgfqpoint{3.127131in}{2.451175in}}%
\pgfpathlineto{\pgfqpoint{3.127131in}{2.408330in}}%
\pgfpathlineto{\pgfqpoint{3.215703in}{2.408330in}}%
\pgfpathlineto{\pgfqpoint{3.215703in}{2.258371in}}%
\pgfpathlineto{\pgfqpoint{3.304274in}{2.258371in}}%
\pgfpathlineto{\pgfqpoint{3.304274in}{2.181011in}}%
\pgfpathlineto{\pgfqpoint{3.392846in}{2.181011in}}%
\pgfpathlineto{\pgfqpoint{3.392846in}{2.067947in}}%
\pgfpathlineto{\pgfqpoint{3.481417in}{2.067947in}}%
\pgfpathlineto{\pgfqpoint{3.481417in}{1.997728in}}%
\pgfpathlineto{\pgfqpoint{3.569989in}{1.997728in}}%
\pgfpathlineto{\pgfqpoint{3.569989in}{1.862051in}}%
\pgfpathlineto{\pgfqpoint{3.658560in}{1.862051in}}%
\pgfpathlineto{\pgfqpoint{3.658560in}{1.853720in}}%
\pgfpathlineto{\pgfqpoint{3.747131in}{1.853720in}}%
\pgfpathlineto{\pgfqpoint{3.747131in}{1.768029in}}%
\pgfpathlineto{\pgfqpoint{3.835703in}{1.768029in}}%
\pgfpathlineto{\pgfqpoint{3.835703in}{1.652585in}}%
\pgfpathlineto{\pgfqpoint{3.924274in}{1.652585in}}%
\pgfpathlineto{\pgfqpoint{3.924274in}{1.610930in}}%
\pgfpathlineto{\pgfqpoint{4.012846in}{1.610930in}}%
\pgfpathlineto{\pgfqpoint{4.012846in}{1.513337in}}%
\pgfpathlineto{\pgfqpoint{4.101417in}{1.513337in}}%
\pgfpathlineto{\pgfqpoint{4.101417in}{1.421696in}}%
\pgfpathlineto{\pgfqpoint{4.189989in}{1.421696in}}%
\pgfpathlineto{\pgfqpoint{4.189989in}{1.445499in}}%
\pgfpathlineto{\pgfqpoint{4.278560in}{1.445499in}}%
\pgfpathlineto{\pgfqpoint{4.278560in}{1.311012in}}%
\pgfpathlineto{\pgfqpoint{4.367131in}{1.311012in}}%
\pgfpathlineto{\pgfqpoint{4.367131in}{1.293160in}}%
\pgfpathlineto{\pgfqpoint{4.455703in}{1.293160in}}%
\pgfpathlineto{\pgfqpoint{4.455703in}{1.227701in}}%
\pgfpathlineto{\pgfqpoint{4.544274in}{1.227701in}}%
\pgfpathlineto{\pgfqpoint{4.544274in}{1.175335in}}%
\pgfpathlineto{\pgfqpoint{4.632846in}{1.175335in}}%
\pgfpathlineto{\pgfqpoint{4.632846in}{1.096785in}}%
\pgfpathlineto{\pgfqpoint{4.721417in}{1.096785in}}%
\pgfpathlineto{\pgfqpoint{4.721417in}{1.095595in}}%
\pgfpathlineto{\pgfqpoint{4.809989in}{1.095595in}}%
\pgfpathlineto{\pgfqpoint{4.809989in}{1.044418in}}%
\pgfpathlineto{\pgfqpoint{4.898560in}{1.044418in}}%
\pgfpathlineto{\pgfqpoint{4.898560in}{1.030137in}}%
\pgfpathlineto{\pgfqpoint{4.987131in}{1.030137in}}%
\pgfpathlineto{\pgfqpoint{4.987131in}{0.958728in}}%
\pgfpathlineto{\pgfqpoint{5.075703in}{0.958728in}}%
\pgfpathlineto{\pgfqpoint{5.075703in}{0.945636in}}%
\pgfpathlineto{\pgfqpoint{5.164274in}{0.945636in}}%
\pgfpathlineto{\pgfqpoint{5.164274in}{0.923023in}}%
\pgfpathlineto{\pgfqpoint{5.252846in}{0.923023in}}%
\pgfpathlineto{\pgfqpoint{5.252846in}{0.893269in}}%
\pgfpathlineto{\pgfqpoint{5.341417in}{0.893269in}}%
\pgfpathlineto{\pgfqpoint{5.341417in}{0.883748in}}%
\pgfpathlineto{\pgfqpoint{5.429989in}{0.883748in}}%
\pgfpathlineto{\pgfqpoint{5.429989in}{0.858755in}}%
\pgfpathlineto{\pgfqpoint{5.518560in}{0.858755in}}%
\pgfpathlineto{\pgfqpoint{5.518560in}{0.823050in}}%
\pgfpathlineto{\pgfqpoint{5.607131in}{0.823050in}}%
\pgfpathlineto{\pgfqpoint{5.607131in}{0.839713in}}%
\pgfpathlineto{\pgfqpoint{5.695703in}{0.839713in}}%
\pgfpathlineto{\pgfqpoint{5.695703in}{0.824241in}}%
\pgfpathlineto{\pgfqpoint{5.784274in}{0.824241in}}%
\pgfpathlineto{\pgfqpoint{5.784274in}{0.804008in}}%
\pgfpathlineto{\pgfqpoint{5.872846in}{0.804008in}}%
\pgfpathlineto{\pgfqpoint{5.872846in}{0.788536in}}%
\pgfpathlineto{\pgfqpoint{5.961417in}{0.788536in}}%
\pgfpathlineto{\pgfqpoint{5.961417in}{0.781395in}}%
\pgfpathlineto{\pgfqpoint{6.049989in}{0.781395in}}%
\pgfpathlineto{\pgfqpoint{6.049989in}{0.783776in}}%
\pgfpathlineto{\pgfqpoint{6.138560in}{0.783776in}}%
\pgfpathlineto{\pgfqpoint{6.138560in}{0.781395in}}%
\pgfpathlineto{\pgfqpoint{6.227131in}{0.781395in}}%
\pgfpathlineto{\pgfqpoint{6.227131in}{0.761163in}}%
\pgfpathlineto{\pgfqpoint{6.315703in}{0.761163in}}%
\pgfpathlineto{\pgfqpoint{6.315703in}{0.771874in}}%
\pgfpathlineto{\pgfqpoint{6.404274in}{0.771874in}}%
\pgfpathlineto{\pgfqpoint{6.404274in}{0.756402in}}%
\pgfpathlineto{\pgfqpoint{6.492846in}{0.756402in}}%
\pgfpathlineto{\pgfqpoint{6.492846in}{0.754022in}}%
\pgfpathlineto{\pgfqpoint{6.581417in}{0.754022in}}%
\pgfpathlineto{\pgfqpoint{6.581417in}{0.726648in}}%
\pgfpathlineto{\pgfqpoint{6.669989in}{0.726648in}}%
\pgfpathlineto{\pgfqpoint{6.669989in}{0.736170in}}%
\pgfpathlineto{\pgfqpoint{6.758560in}{0.736170in}}%
\pgfpathlineto{\pgfqpoint{6.758560in}{0.714747in}}%
\pgfpathlineto{\pgfqpoint{6.847131in}{0.714747in}}%
\pgfpathlineto{\pgfqpoint{6.847131in}{0.718317in}}%
\pgfpathlineto{\pgfqpoint{7.024274in}{0.718317in}}%
\pgfpathlineto{\pgfqpoint{7.024274in}{0.714747in}}%
\pgfpathlineto{\pgfqpoint{7.112846in}{0.714747in}}%
\pgfpathlineto{\pgfqpoint{7.112846in}{0.721888in}}%
\pgfpathlineto{\pgfqpoint{7.201417in}{0.721888in}}%
\pgfpathlineto{\pgfqpoint{7.201417in}{0.712367in}}%
\pgfpathlineto{\pgfqpoint{7.210000in}{0.712367in}}%
\pgfpathlineto{\pgfqpoint{7.210000in}{0.712367in}}%
\pgfusepath{stroke}%
\end{pgfscope}%
\begin{pgfscope}%
\pgfpathrectangle{\pgfqpoint{1.000000in}{0.660000in}}{\pgfqpoint{6.200000in}{4.620000in}}%
\pgfusepath{clip}%
\pgfsetbuttcap%
\pgfsetmiterjoin%
\pgfsetlinewidth{1.003750pt}%
\definecolor{currentstroke}{rgb}{0.750000,0.000000,0.750000}%
\pgfsetstrokecolor{currentstroke}%
\pgfsetstrokeopacity{0.500000}%
\pgfsetdash{}{0pt}%
\pgfpathmoveto{\pgfqpoint{1.001417in}{0.660000in}}%
\pgfpathlineto{\pgfqpoint{1.444274in}{0.660000in}}%
\pgfpathlineto{\pgfqpoint{1.444274in}{0.662304in}}%
\pgfpathlineto{\pgfqpoint{1.532846in}{0.662304in}}%
\pgfpathlineto{\pgfqpoint{1.532846in}{1.366230in}}%
\pgfpathlineto{\pgfqpoint{1.621417in}{1.366230in}}%
\pgfpathlineto{\pgfqpoint{1.621417in}{3.676167in}}%
\pgfpathlineto{\pgfqpoint{1.709989in}{3.676167in}}%
\pgfpathlineto{\pgfqpoint{1.709989in}{4.078245in}}%
\pgfpathlineto{\pgfqpoint{1.798560in}{4.078245in}}%
\pgfpathlineto{\pgfqpoint{1.798560in}{4.109352in}}%
\pgfpathlineto{\pgfqpoint{1.887131in}{4.109352in}}%
\pgfpathlineto{\pgfqpoint{1.887131in}{4.090918in}}%
\pgfpathlineto{\pgfqpoint{1.975703in}{4.090918in}}%
\pgfpathlineto{\pgfqpoint{1.975703in}{3.949212in}}%
\pgfpathlineto{\pgfqpoint{2.064274in}{3.949212in}}%
\pgfpathlineto{\pgfqpoint{2.064274in}{3.995295in}}%
\pgfpathlineto{\pgfqpoint{2.152846in}{3.995295in}}%
\pgfpathlineto{\pgfqpoint{2.152846in}{3.851284in}}%
\pgfpathlineto{\pgfqpoint{2.241417in}{3.851284in}}%
\pgfpathlineto{\pgfqpoint{2.241417in}{3.782159in}}%
\pgfpathlineto{\pgfqpoint{2.329989in}{3.782159in}}%
\pgfpathlineto{\pgfqpoint{2.329989in}{3.595520in}}%
\pgfpathlineto{\pgfqpoint{2.418560in}{3.595520in}}%
\pgfpathlineto{\pgfqpoint{2.418560in}{3.645060in}}%
\pgfpathlineto{\pgfqpoint{2.507131in}{3.645060in}}%
\pgfpathlineto{\pgfqpoint{2.507131in}{3.344365in}}%
\pgfpathlineto{\pgfqpoint{2.595703in}{3.344365in}}%
\pgfpathlineto{\pgfqpoint{2.595703in}{3.211875in}}%
\pgfpathlineto{\pgfqpoint{2.684274in}{3.211875in}}%
\pgfpathlineto{\pgfqpoint{2.684274in}{3.184225in}}%
\pgfpathlineto{\pgfqpoint{2.772846in}{3.184225in}}%
\pgfpathlineto{\pgfqpoint{2.772846in}{2.912332in}}%
\pgfpathlineto{\pgfqpoint{2.861417in}{2.912332in}}%
\pgfpathlineto{\pgfqpoint{2.861417in}{2.782146in}}%
\pgfpathlineto{\pgfqpoint{2.949989in}{2.782146in}}%
\pgfpathlineto{\pgfqpoint{2.949989in}{2.738367in}}%
\pgfpathlineto{\pgfqpoint{3.038560in}{2.738367in}}%
\pgfpathlineto{\pgfqpoint{3.038560in}{2.486060in}}%
\pgfpathlineto{\pgfqpoint{3.127131in}{2.486060in}}%
\pgfpathlineto{\pgfqpoint{3.127131in}{2.448041in}}%
\pgfpathlineto{\pgfqpoint{3.215703in}{2.448041in}}%
\pgfpathlineto{\pgfqpoint{3.215703in}{2.261402in}}%
\pgfpathlineto{\pgfqpoint{3.304274in}{2.261402in}}%
\pgfpathlineto{\pgfqpoint{3.304274in}{2.173844in}}%
\pgfpathlineto{\pgfqpoint{3.392846in}{2.173844in}}%
\pgfpathlineto{\pgfqpoint{3.392846in}{2.127760in}}%
\pgfpathlineto{\pgfqpoint{3.481417in}{2.127760in}}%
\pgfpathlineto{\pgfqpoint{3.481417in}{2.044810in}}%
\pgfpathlineto{\pgfqpoint{3.569989in}{2.044810in}}%
\pgfpathlineto{\pgfqpoint{3.569989in}{1.889278in}}%
\pgfpathlineto{\pgfqpoint{3.658560in}{1.889278in}}%
\pgfpathlineto{\pgfqpoint{3.658560in}{1.844347in}}%
\pgfpathlineto{\pgfqpoint{3.747131in}{1.844347in}}%
\pgfpathlineto{\pgfqpoint{3.747131in}{1.812088in}}%
\pgfpathlineto{\pgfqpoint{3.835703in}{1.812088in}}%
\pgfpathlineto{\pgfqpoint{3.835703in}{1.700336in}}%
\pgfpathlineto{\pgfqpoint{3.924274in}{1.700336in}}%
\pgfpathlineto{\pgfqpoint{3.924274in}{1.555172in}}%
\pgfpathlineto{\pgfqpoint{4.012846in}{1.555172in}}%
\pgfpathlineto{\pgfqpoint{4.012846in}{1.497568in}}%
\pgfpathlineto{\pgfqpoint{4.101417in}{1.497568in}}%
\pgfpathlineto{\pgfqpoint{4.101417in}{1.459549in}}%
\pgfpathlineto{\pgfqpoint{4.189989in}{1.459549in}}%
\pgfpathlineto{\pgfqpoint{4.189989in}{1.328211in}}%
\pgfpathlineto{\pgfqpoint{4.278560in}{1.328211in}}%
\pgfpathlineto{\pgfqpoint{4.278560in}{1.279823in}}%
\pgfpathlineto{\pgfqpoint{4.367131in}{1.279823in}}%
\pgfpathlineto{\pgfqpoint{4.367131in}{1.196873in}}%
\pgfpathlineto{\pgfqpoint{4.455703in}{1.196873in}}%
\pgfpathlineto{\pgfqpoint{4.455703in}{1.161158in}}%
\pgfpathlineto{\pgfqpoint{4.544274in}{1.161158in}}%
\pgfpathlineto{\pgfqpoint{4.544274in}{1.155398in}}%
\pgfpathlineto{\pgfqpoint{4.632846in}{1.155398in}}%
\pgfpathlineto{\pgfqpoint{4.632846in}{1.083968in}}%
\pgfpathlineto{\pgfqpoint{4.721417in}{1.083968in}}%
\pgfpathlineto{\pgfqpoint{4.721417in}{0.999866in}}%
\pgfpathlineto{\pgfqpoint{4.809989in}{0.999866in}}%
\pgfpathlineto{\pgfqpoint{4.809989in}{1.007931in}}%
\pgfpathlineto{\pgfqpoint{4.898560in}{1.007931in}}%
\pgfpathlineto{\pgfqpoint{4.898560in}{0.961847in}}%
\pgfpathlineto{\pgfqpoint{4.987131in}{0.961847in}}%
\pgfpathlineto{\pgfqpoint{4.987131in}{0.945718in}}%
\pgfpathlineto{\pgfqpoint{5.075703in}{0.945718in}}%
\pgfpathlineto{\pgfqpoint{5.075703in}{0.899634in}}%
\pgfpathlineto{\pgfqpoint{5.164274in}{0.899634in}}%
\pgfpathlineto{\pgfqpoint{5.164274in}{0.891570in}}%
\pgfpathlineto{\pgfqpoint{5.252846in}{0.891570in}}%
\pgfpathlineto{\pgfqpoint{5.252846in}{0.854703in}}%
\pgfpathlineto{\pgfqpoint{5.341417in}{0.854703in}}%
\pgfpathlineto{\pgfqpoint{5.341417in}{0.837422in}}%
\pgfpathlineto{\pgfqpoint{5.429989in}{0.837422in}}%
\pgfpathlineto{\pgfqpoint{5.429989in}{0.801707in}}%
\pgfpathlineto{\pgfqpoint{5.518560in}{0.801707in}}%
\pgfpathlineto{\pgfqpoint{5.518560in}{0.798251in}}%
\pgfpathlineto{\pgfqpoint{5.607131in}{0.798251in}}%
\pgfpathlineto{\pgfqpoint{5.607131in}{0.814380in}}%
\pgfpathlineto{\pgfqpoint{5.695703in}{0.814380in}}%
\pgfpathlineto{\pgfqpoint{5.695703in}{0.782121in}}%
\pgfpathlineto{\pgfqpoint{5.784274in}{0.782121in}}%
\pgfpathlineto{\pgfqpoint{5.784274in}{0.764840in}}%
\pgfpathlineto{\pgfqpoint{5.872846in}{0.764840in}}%
\pgfpathlineto{\pgfqpoint{5.872846in}{0.746407in}}%
\pgfpathlineto{\pgfqpoint{5.961417in}{0.746407in}}%
\pgfpathlineto{\pgfqpoint{5.961417in}{0.771753in}}%
\pgfpathlineto{\pgfqpoint{6.049989in}{0.771753in}}%
\pgfpathlineto{\pgfqpoint{6.049989in}{0.775209in}}%
\pgfpathlineto{\pgfqpoint{6.138560in}{0.775209in}}%
\pgfpathlineto{\pgfqpoint{6.138560in}{0.738342in}}%
\pgfpathlineto{\pgfqpoint{6.227131in}{0.738342in}}%
\pgfpathlineto{\pgfqpoint{6.227131in}{0.723365in}}%
\pgfpathlineto{\pgfqpoint{6.315703in}{0.723365in}}%
\pgfpathlineto{\pgfqpoint{6.315703in}{0.733734in}}%
\pgfpathlineto{\pgfqpoint{6.492846in}{0.733734in}}%
\pgfpathlineto{\pgfqpoint{6.492846in}{0.723365in}}%
\pgfpathlineto{\pgfqpoint{6.581417in}{0.723365in}}%
\pgfpathlineto{\pgfqpoint{6.581417in}{0.724517in}}%
\pgfpathlineto{\pgfqpoint{6.669989in}{0.724517in}}%
\pgfpathlineto{\pgfqpoint{6.669989in}{0.716452in}}%
\pgfpathlineto{\pgfqpoint{6.758560in}{0.716452in}}%
\pgfpathlineto{\pgfqpoint{6.758560in}{0.725669in}}%
\pgfpathlineto{\pgfqpoint{6.847131in}{0.725669in}}%
\pgfpathlineto{\pgfqpoint{6.847131in}{0.703779in}}%
\pgfpathlineto{\pgfqpoint{6.935703in}{0.703779in}}%
\pgfpathlineto{\pgfqpoint{6.935703in}{0.704931in}}%
\pgfpathlineto{\pgfqpoint{7.024274in}{0.704931in}}%
\pgfpathlineto{\pgfqpoint{7.024274in}{0.699171in}}%
\pgfpathlineto{\pgfqpoint{7.112846in}{0.699171in}}%
\pgfpathlineto{\pgfqpoint{7.112846in}{0.692258in}}%
\pgfpathlineto{\pgfqpoint{7.201417in}{0.692258in}}%
\pgfpathlineto{\pgfqpoint{7.201417in}{0.681890in}}%
\pgfpathlineto{\pgfqpoint{7.210000in}{0.681890in}}%
\pgfpathlineto{\pgfqpoint{7.210000in}{0.681890in}}%
\pgfusepath{stroke}%
\end{pgfscope}%
\begin{pgfscope}%
\pgfpathrectangle{\pgfqpoint{1.000000in}{0.660000in}}{\pgfqpoint{6.200000in}{4.620000in}}%
\pgfusepath{clip}%
\pgfsetbuttcap%
\pgfsetmiterjoin%
\pgfsetlinewidth{1.003750pt}%
\definecolor{currentstroke}{rgb}{0.750000,0.750000,0.000000}%
\pgfsetstrokecolor{currentstroke}%
\pgfsetstrokeopacity{0.500000}%
\pgfsetdash{}{0pt}%
\pgfpathmoveto{\pgfqpoint{1.001417in}{0.660000in}}%
\pgfpathlineto{\pgfqpoint{1.089989in}{0.660000in}}%
\pgfpathlineto{\pgfqpoint{1.089989in}{2.070893in}}%
\pgfpathlineto{\pgfqpoint{1.178560in}{2.070893in}}%
\pgfpathlineto{\pgfqpoint{1.178560in}{3.424067in}}%
\pgfpathlineto{\pgfqpoint{1.267131in}{3.424067in}}%
\pgfpathlineto{\pgfqpoint{1.267131in}{2.630119in}}%
\pgfpathlineto{\pgfqpoint{1.355703in}{2.630119in}}%
\pgfpathlineto{\pgfqpoint{1.355703in}{2.476204in}}%
\pgfpathlineto{\pgfqpoint{1.444274in}{2.476204in}}%
\pgfpathlineto{\pgfqpoint{1.444274in}{2.291505in}}%
\pgfpathlineto{\pgfqpoint{1.532846in}{2.291505in}}%
\pgfpathlineto{\pgfqpoint{1.532846in}{2.201721in}}%
\pgfpathlineto{\pgfqpoint{1.621417in}{2.201721in}}%
\pgfpathlineto{\pgfqpoint{1.621417in}{2.129894in}}%
\pgfpathlineto{\pgfqpoint{1.709989in}{2.129894in}}%
\pgfpathlineto{\pgfqpoint{1.709989in}{2.019587in}}%
\pgfpathlineto{\pgfqpoint{1.798560in}{2.019587in}}%
\pgfpathlineto{\pgfqpoint{1.798560in}{1.909281in}}%
\pgfpathlineto{\pgfqpoint{1.887131in}{1.909281in}}%
\pgfpathlineto{\pgfqpoint{1.887131in}{1.878498in}}%
\pgfpathlineto{\pgfqpoint{1.975703in}{1.878498in}}%
\pgfpathlineto{\pgfqpoint{1.975703in}{1.929803in}}%
\pgfpathlineto{\pgfqpoint{2.064274in}{1.929803in}}%
\pgfpathlineto{\pgfqpoint{2.064274in}{1.816932in}}%
\pgfpathlineto{\pgfqpoint{2.152846in}{1.816932in}}%
\pgfpathlineto{\pgfqpoint{2.152846in}{1.848998in}}%
\pgfpathlineto{\pgfqpoint{2.241417in}{1.848998in}}%
\pgfpathlineto{\pgfqpoint{2.241417in}{1.855411in}}%
\pgfpathlineto{\pgfqpoint{2.329989in}{1.855411in}}%
\pgfpathlineto{\pgfqpoint{2.329989in}{1.907999in}}%
\pgfpathlineto{\pgfqpoint{2.418560in}{1.907999in}}%
\pgfpathlineto{\pgfqpoint{2.418560in}{1.902868in}}%
\pgfpathlineto{\pgfqpoint{2.507131in}{1.902868in}}%
\pgfpathlineto{\pgfqpoint{2.507131in}{1.895172in}}%
\pgfpathlineto{\pgfqpoint{2.595703in}{1.895172in}}%
\pgfpathlineto{\pgfqpoint{2.595703in}{2.070893in}}%
\pgfpathlineto{\pgfqpoint{2.684274in}{2.070893in}}%
\pgfpathlineto{\pgfqpoint{2.684274in}{2.132459in}}%
\pgfpathlineto{\pgfqpoint{2.772846in}{2.132459in}}%
\pgfpathlineto{\pgfqpoint{2.772846in}{2.024718in}}%
\pgfpathlineto{\pgfqpoint{2.861417in}{2.024718in}}%
\pgfpathlineto{\pgfqpoint{2.861417in}{2.263287in}}%
\pgfpathlineto{\pgfqpoint{2.949989in}{2.263287in}}%
\pgfpathlineto{\pgfqpoint{2.949989in}{2.322288in}}%
\pgfpathlineto{\pgfqpoint{3.038560in}{2.322288in}}%
\pgfpathlineto{\pgfqpoint{3.038560in}{2.396681in}}%
\pgfpathlineto{\pgfqpoint{3.127131in}{2.396681in}}%
\pgfpathlineto{\pgfqpoint{3.127131in}{2.541618in}}%
\pgfpathlineto{\pgfqpoint{3.215703in}{2.541618in}}%
\pgfpathlineto{\pgfqpoint{3.215703in}{2.577531in}}%
\pgfpathlineto{\pgfqpoint{3.304274in}{2.577531in}}%
\pgfpathlineto{\pgfqpoint{3.304274in}{2.631402in}}%
\pgfpathlineto{\pgfqpoint{3.392846in}{2.631402in}}%
\pgfpathlineto{\pgfqpoint{3.392846in}{2.748121in}}%
\pgfpathlineto{\pgfqpoint{3.481417in}{2.748121in}}%
\pgfpathlineto{\pgfqpoint{3.481417in}{2.787883in}}%
\pgfpathlineto{\pgfqpoint{3.569989in}{2.787883in}}%
\pgfpathlineto{\pgfqpoint{3.569989in}{2.926407in}}%
\pgfpathlineto{\pgfqpoint{3.658560in}{2.926407in}}%
\pgfpathlineto{\pgfqpoint{3.658560in}{2.986690in}}%
\pgfpathlineto{\pgfqpoint{3.747131in}{2.986690in}}%
\pgfpathlineto{\pgfqpoint{3.747131in}{2.931537in}}%
\pgfpathlineto{\pgfqpoint{3.835703in}{2.931537in}}%
\pgfpathlineto{\pgfqpoint{3.835703in}{2.836623in}}%
\pgfpathlineto{\pgfqpoint{3.924274in}{2.836623in}}%
\pgfpathlineto{\pgfqpoint{3.924274in}{2.698099in}}%
\pgfpathlineto{\pgfqpoint{4.012846in}{2.698099in}}%
\pgfpathlineto{\pgfqpoint{4.012846in}{2.568553in}}%
\pgfpathlineto{\pgfqpoint{4.101417in}{2.568553in}}%
\pgfpathlineto{\pgfqpoint{4.101417in}{2.356919in}}%
\pgfpathlineto{\pgfqpoint{4.189989in}{2.356919in}}%
\pgfpathlineto{\pgfqpoint{4.189989in}{2.338962in}}%
\pgfpathlineto{\pgfqpoint{4.278560in}{2.338962in}}%
\pgfpathlineto{\pgfqpoint{4.278560in}{2.133741in}}%
\pgfpathlineto{\pgfqpoint{4.367131in}{2.133741in}}%
\pgfpathlineto{\pgfqpoint{4.367131in}{1.995217in}}%
\pgfpathlineto{\pgfqpoint{4.455703in}{1.995217in}}%
\pgfpathlineto{\pgfqpoint{4.455703in}{1.811801in}}%
\pgfpathlineto{\pgfqpoint{4.544274in}{1.811801in}}%
\pgfpathlineto{\pgfqpoint{4.544274in}{1.641212in}}%
\pgfpathlineto{\pgfqpoint{4.632846in}{1.641212in}}%
\pgfpathlineto{\pgfqpoint{4.632846in}{1.452665in}}%
\pgfpathlineto{\pgfqpoint{4.721417in}{1.452665in}}%
\pgfpathlineto{\pgfqpoint{4.721417in}{1.439839in}}%
\pgfpathlineto{\pgfqpoint{4.809989in}{1.439839in}}%
\pgfpathlineto{\pgfqpoint{4.809989in}{1.283358in}}%
\pgfpathlineto{\pgfqpoint{4.898560in}{1.283358in}}%
\pgfpathlineto{\pgfqpoint{4.898560in}{1.165356in}}%
\pgfpathlineto{\pgfqpoint{4.987131in}{1.165356in}}%
\pgfpathlineto{\pgfqpoint{4.987131in}{1.071724in}}%
\pgfpathlineto{\pgfqpoint{5.075703in}{1.071724in}}%
\pgfpathlineto{\pgfqpoint{5.075703in}{1.003745in}}%
\pgfpathlineto{\pgfqpoint{5.164274in}{1.003745in}}%
\pgfpathlineto{\pgfqpoint{5.164274in}{0.983223in}}%
\pgfpathlineto{\pgfqpoint{5.252846in}{0.983223in}}%
\pgfpathlineto{\pgfqpoint{5.252846in}{0.907548in}}%
\pgfpathlineto{\pgfqpoint{5.341417in}{0.907548in}}%
\pgfpathlineto{\pgfqpoint{5.341417in}{0.863938in}}%
\pgfpathlineto{\pgfqpoint{5.429989in}{0.863938in}}%
\pgfpathlineto{\pgfqpoint{5.429989in}{0.844699in}}%
\pgfpathlineto{\pgfqpoint{5.518560in}{0.844699in}}%
\pgfpathlineto{\pgfqpoint{5.518560in}{0.795959in}}%
\pgfpathlineto{\pgfqpoint{5.607131in}{0.795959in}}%
\pgfpathlineto{\pgfqpoint{5.607131in}{0.788263in}}%
\pgfpathlineto{\pgfqpoint{5.695703in}{0.788263in}}%
\pgfpathlineto{\pgfqpoint{5.695703in}{0.769024in}}%
\pgfpathlineto{\pgfqpoint{5.784274in}{0.769024in}}%
\pgfpathlineto{\pgfqpoint{5.784274in}{0.736958in}}%
\pgfpathlineto{\pgfqpoint{5.872846in}{0.736958in}}%
\pgfpathlineto{\pgfqpoint{5.872846in}{0.735675in}}%
\pgfpathlineto{\pgfqpoint{5.961417in}{0.735675in}}%
\pgfpathlineto{\pgfqpoint{5.961417in}{0.725414in}}%
\pgfpathlineto{\pgfqpoint{6.049989in}{0.725414in}}%
\pgfpathlineto{\pgfqpoint{6.049989in}{0.722849in}}%
\pgfpathlineto{\pgfqpoint{6.138560in}{0.722849in}}%
\pgfpathlineto{\pgfqpoint{6.138560in}{0.720284in}}%
\pgfpathlineto{\pgfqpoint{6.227131in}{0.720284in}}%
\pgfpathlineto{\pgfqpoint{6.227131in}{0.715153in}}%
\pgfpathlineto{\pgfqpoint{6.315703in}{0.715153in}}%
\pgfpathlineto{\pgfqpoint{6.315703in}{0.710023in}}%
\pgfpathlineto{\pgfqpoint{6.404274in}{0.710023in}}%
\pgfpathlineto{\pgfqpoint{6.404274in}{0.701044in}}%
\pgfpathlineto{\pgfqpoint{6.492846in}{0.701044in}}%
\pgfpathlineto{\pgfqpoint{6.492846in}{0.697196in}}%
\pgfpathlineto{\pgfqpoint{6.581417in}{0.697196in}}%
\pgfpathlineto{\pgfqpoint{6.581417in}{0.710023in}}%
\pgfpathlineto{\pgfqpoint{6.669989in}{0.710023in}}%
\pgfpathlineto{\pgfqpoint{6.669989in}{0.686935in}}%
\pgfpathlineto{\pgfqpoint{6.758560in}{0.686935in}}%
\pgfpathlineto{\pgfqpoint{6.758560in}{0.683087in}}%
\pgfpathlineto{\pgfqpoint{6.935703in}{0.683087in}}%
\pgfpathlineto{\pgfqpoint{6.935703in}{0.676674in}}%
\pgfpathlineto{\pgfqpoint{7.024274in}{0.676674in}}%
\pgfpathlineto{\pgfqpoint{7.024274in}{0.680522in}}%
\pgfpathlineto{\pgfqpoint{7.112846in}{0.680522in}}%
\pgfpathlineto{\pgfqpoint{7.112846in}{0.672826in}}%
\pgfpathlineto{\pgfqpoint{7.201417in}{0.672826in}}%
\pgfpathlineto{\pgfqpoint{7.201417in}{0.665131in}}%
\pgfpathlineto{\pgfqpoint{7.210000in}{0.665131in}}%
\pgfpathlineto{\pgfqpoint{7.210000in}{0.665131in}}%
\pgfusepath{stroke}%
\end{pgfscope}%
\begin{pgfscope}%
\pgfpathrectangle{\pgfqpoint{1.000000in}{0.660000in}}{\pgfqpoint{6.200000in}{4.620000in}}%
\pgfusepath{clip}%
\pgfsetbuttcap%
\pgfsetmiterjoin%
\pgfsetlinewidth{1.003750pt}%
\definecolor{currentstroke}{rgb}{0.000000,0.000000,1.000000}%
\pgfsetstrokecolor{currentstroke}%
\pgfsetstrokeopacity{0.500000}%
\pgfsetdash{}{0pt}%
\pgfpathmoveto{\pgfqpoint{1.001417in}{0.660000in}}%
\pgfpathlineto{\pgfqpoint{1.267131in}{0.660000in}}%
\pgfpathlineto{\pgfqpoint{1.267131in}{0.663533in}}%
\pgfpathlineto{\pgfqpoint{1.444274in}{0.663533in}}%
\pgfpathlineto{\pgfqpoint{1.444274in}{0.670600in}}%
\pgfpathlineto{\pgfqpoint{1.621417in}{0.670600in}}%
\pgfpathlineto{\pgfqpoint{1.621417in}{0.688265in}}%
\pgfpathlineto{\pgfqpoint{1.709989in}{0.688265in}}%
\pgfpathlineto{\pgfqpoint{1.709989in}{0.707698in}}%
\pgfpathlineto{\pgfqpoint{1.798560in}{0.707698in}}%
\pgfpathlineto{\pgfqpoint{1.798560in}{0.744796in}}%
\pgfpathlineto{\pgfqpoint{1.887131in}{0.744796in}}%
\pgfpathlineto{\pgfqpoint{1.887131in}{0.764229in}}%
\pgfpathlineto{\pgfqpoint{1.975703in}{0.764229in}}%
\pgfpathlineto{\pgfqpoint{1.975703in}{0.815460in}}%
\pgfpathlineto{\pgfqpoint{2.064274in}{0.815460in}}%
\pgfpathlineto{\pgfqpoint{2.064274in}{0.905556in}}%
\pgfpathlineto{\pgfqpoint{2.152846in}{0.905556in}}%
\pgfpathlineto{\pgfqpoint{2.152846in}{0.910856in}}%
\pgfpathlineto{\pgfqpoint{2.241417in}{0.910856in}}%
\pgfpathlineto{\pgfqpoint{2.241417in}{0.983286in}}%
\pgfpathlineto{\pgfqpoint{2.329989in}{0.983286in}}%
\pgfpathlineto{\pgfqpoint{2.329989in}{1.133446in}}%
\pgfpathlineto{\pgfqpoint{2.418560in}{1.133446in}}%
\pgfpathlineto{\pgfqpoint{2.418560in}{1.198810in}}%
\pgfpathlineto{\pgfqpoint{2.507131in}{1.198810in}}%
\pgfpathlineto{\pgfqpoint{2.507131in}{1.386068in}}%
\pgfpathlineto{\pgfqpoint{2.595703in}{1.386068in}}%
\pgfpathlineto{\pgfqpoint{2.595703in}{1.492063in}}%
\pgfpathlineto{\pgfqpoint{2.684274in}{1.492063in}}%
\pgfpathlineto{\pgfqpoint{2.684274in}{1.608658in}}%
\pgfpathlineto{\pgfqpoint{2.772846in}{1.608658in}}%
\pgfpathlineto{\pgfqpoint{2.772846in}{1.852447in}}%
\pgfpathlineto{\pgfqpoint{2.861417in}{1.852447in}}%
\pgfpathlineto{\pgfqpoint{2.861417in}{2.034406in}}%
\pgfpathlineto{\pgfqpoint{2.949989in}{2.034406in}}%
\pgfpathlineto{\pgfqpoint{2.949989in}{2.304694in}}%
\pgfpathlineto{\pgfqpoint{3.038560in}{2.304694in}}%
\pgfpathlineto{\pgfqpoint{3.038560in}{2.513151in}}%
\pgfpathlineto{\pgfqpoint{3.127131in}{2.513151in}}%
\pgfpathlineto{\pgfqpoint{3.127131in}{2.718076in}}%
\pgfpathlineto{\pgfqpoint{3.215703in}{2.718076in}}%
\pgfpathlineto{\pgfqpoint{3.215703in}{2.926533in}}%
\pgfpathlineto{\pgfqpoint{3.304274in}{2.926533in}}%
\pgfpathlineto{\pgfqpoint{3.304274in}{3.142057in}}%
\pgfpathlineto{\pgfqpoint{3.392846in}{3.142057in}}%
\pgfpathlineto{\pgfqpoint{3.392846in}{3.387613in}}%
\pgfpathlineto{\pgfqpoint{3.481417in}{3.387613in}}%
\pgfpathlineto{\pgfqpoint{3.481417in}{3.760363in}}%
\pgfpathlineto{\pgfqpoint{3.569989in}{3.760363in}}%
\pgfpathlineto{\pgfqpoint{3.569989in}{3.952921in}}%
\pgfpathlineto{\pgfqpoint{3.658560in}{3.952921in}}%
\pgfpathlineto{\pgfqpoint{3.658560in}{4.189644in}}%
\pgfpathlineto{\pgfqpoint{3.747131in}{4.189644in}}%
\pgfpathlineto{\pgfqpoint{3.747131in}{4.313305in}}%
\pgfpathlineto{\pgfqpoint{3.835703in}{4.313305in}}%
\pgfpathlineto{\pgfqpoint{3.835703in}{4.419300in}}%
\pgfpathlineto{\pgfqpoint{3.924274in}{4.419300in}}%
\pgfpathlineto{\pgfqpoint{3.924274in}{4.447566in}}%
\pgfpathlineto{\pgfqpoint{4.012846in}{4.447566in}}%
\pgfpathlineto{\pgfqpoint{4.012846in}{4.166678in}}%
\pgfpathlineto{\pgfqpoint{4.101417in}{4.166678in}}%
\pgfpathlineto{\pgfqpoint{4.101417in}{4.097781in}}%
\pgfpathlineto{\pgfqpoint{4.189989in}{4.097781in}}%
\pgfpathlineto{\pgfqpoint{4.189989in}{3.792161in}}%
\pgfpathlineto{\pgfqpoint{4.278560in}{3.792161in}}%
\pgfpathlineto{\pgfqpoint{4.278560in}{3.456510in}}%
\pgfpathlineto{\pgfqpoint{4.367131in}{3.456510in}}%
\pgfpathlineto{\pgfqpoint{4.367131in}{3.251585in}}%
\pgfpathlineto{\pgfqpoint{4.455703in}{3.251585in}}%
\pgfpathlineto{\pgfqpoint{4.455703in}{2.894734in}}%
\pgfpathlineto{\pgfqpoint{4.544274in}{2.894734in}}%
\pgfpathlineto{\pgfqpoint{4.544274in}{2.606781in}}%
\pgfpathlineto{\pgfqpoint{4.632846in}{2.606781in}}%
\pgfpathlineto{\pgfqpoint{4.632846in}{2.359458in}}%
\pgfpathlineto{\pgfqpoint{4.721417in}{2.359458in}}%
\pgfpathlineto{\pgfqpoint{4.721417in}{2.110369in}}%
\pgfpathlineto{\pgfqpoint{4.809989in}{2.110369in}}%
\pgfpathlineto{\pgfqpoint{4.809989in}{1.891312in}}%
\pgfpathlineto{\pgfqpoint{4.898560in}{1.891312in}}%
\pgfpathlineto{\pgfqpoint{4.898560in}{1.771184in}}%
\pgfpathlineto{\pgfqpoint{4.987131in}{1.771184in}}%
\pgfpathlineto{\pgfqpoint{4.987131in}{1.548594in}}%
\pgfpathlineto{\pgfqpoint{5.075703in}{1.548594in}}%
\pgfpathlineto{\pgfqpoint{5.075703in}{1.416100in}}%
\pgfpathlineto{\pgfqpoint{5.164274in}{1.416100in}}%
\pgfpathlineto{\pgfqpoint{5.164274in}{1.250041in}}%
\pgfpathlineto{\pgfqpoint{5.252846in}{1.250041in}}%
\pgfpathlineto{\pgfqpoint{5.252846in}{1.147578in}}%
\pgfpathlineto{\pgfqpoint{5.341417in}{1.147578in}}%
\pgfpathlineto{\pgfqpoint{5.341417in}{1.034517in}}%
\pgfpathlineto{\pgfqpoint{5.429989in}{1.034517in}}%
\pgfpathlineto{\pgfqpoint{5.429989in}{0.953254in}}%
\pgfpathlineto{\pgfqpoint{5.518560in}{0.953254in}}%
\pgfpathlineto{\pgfqpoint{5.518560in}{0.910856in}}%
\pgfpathlineto{\pgfqpoint{5.607131in}{0.910856in}}%
\pgfpathlineto{\pgfqpoint{5.607131in}{0.868457in}}%
\pgfpathlineto{\pgfqpoint{5.695703in}{0.868457in}}%
\pgfpathlineto{\pgfqpoint{5.695703in}{0.833126in}}%
\pgfpathlineto{\pgfqpoint{5.784274in}{0.833126in}}%
\pgfpathlineto{\pgfqpoint{5.784274in}{0.826059in}}%
\pgfpathlineto{\pgfqpoint{5.872846in}{0.826059in}}%
\pgfpathlineto{\pgfqpoint{5.872846in}{0.774828in}}%
\pgfpathlineto{\pgfqpoint{5.961417in}{0.774828in}}%
\pgfpathlineto{\pgfqpoint{5.961417in}{0.730664in}}%
\pgfpathlineto{\pgfqpoint{6.049989in}{0.730664in}}%
\pgfpathlineto{\pgfqpoint{6.049989in}{0.679432in}}%
\pgfpathlineto{\pgfqpoint{6.227131in}{0.679432in}}%
\pgfpathlineto{\pgfqpoint{6.227131in}{0.667066in}}%
\pgfpathlineto{\pgfqpoint{6.404274in}{0.667066in}}%
\pgfpathlineto{\pgfqpoint{6.404274in}{0.661767in}}%
\pgfpathlineto{\pgfqpoint{6.492846in}{0.661767in}}%
\pgfpathlineto{\pgfqpoint{6.492846in}{0.660000in}}%
\pgfpathlineto{\pgfqpoint{7.210000in}{0.660000in}}%
\pgfpathlineto{\pgfqpoint{7.210000in}{0.660000in}}%
\pgfusepath{stroke}%
\end{pgfscope}%
\begin{pgfscope}%
\pgfpathrectangle{\pgfqpoint{1.000000in}{0.660000in}}{\pgfqpoint{6.200000in}{4.620000in}}%
\pgfusepath{clip}%
\pgfsetbuttcap%
\pgfsetmiterjoin%
\pgfsetlinewidth{1.003750pt}%
\definecolor{currentstroke}{rgb}{1.000000,0.000000,0.000000}%
\pgfsetstrokecolor{currentstroke}%
\pgfsetstrokeopacity{0.500000}%
\pgfsetdash{}{0pt}%
\pgfpathmoveto{\pgfqpoint{1.001417in}{0.660000in}}%
\pgfpathlineto{\pgfqpoint{1.001417in}{5.290000in}}%
\pgfpathmoveto{\pgfqpoint{1.089989in}{5.290000in}}%
\pgfpathlineto{\pgfqpoint{1.089989in}{1.130194in}}%
\pgfpathlineto{\pgfqpoint{1.178560in}{1.130194in}}%
\pgfpathlineto{\pgfqpoint{1.178560in}{1.028171in}}%
\pgfpathlineto{\pgfqpoint{1.267131in}{1.028171in}}%
\pgfpathlineto{\pgfqpoint{1.267131in}{0.962077in}}%
\pgfpathlineto{\pgfqpoint{1.355703in}{0.962077in}}%
\pgfpathlineto{\pgfqpoint{1.355703in}{0.931027in}}%
\pgfpathlineto{\pgfqpoint{1.444274in}{0.931027in}}%
\pgfpathlineto{\pgfqpoint{1.444274in}{0.939899in}}%
\pgfpathlineto{\pgfqpoint{1.532846in}{0.939899in}}%
\pgfpathlineto{\pgfqpoint{1.532846in}{0.931027in}}%
\pgfpathlineto{\pgfqpoint{1.621417in}{0.931027in}}%
\pgfpathlineto{\pgfqpoint{1.621417in}{0.977603in}}%
\pgfpathlineto{\pgfqpoint{1.709989in}{0.977603in}}%
\pgfpathlineto{\pgfqpoint{1.709989in}{1.016638in}}%
\pgfpathlineto{\pgfqpoint{1.798560in}{1.016638in}}%
\pgfpathlineto{\pgfqpoint{1.798560in}{1.140396in}}%
\pgfpathlineto{\pgfqpoint{1.887131in}{1.140396in}}%
\pgfpathlineto{\pgfqpoint{1.887131in}{1.275244in}}%
\pgfpathlineto{\pgfqpoint{1.975703in}{1.275244in}}%
\pgfpathlineto{\pgfqpoint{1.975703in}{1.473968in}}%
\pgfpathlineto{\pgfqpoint{2.064274in}{1.473968in}}%
\pgfpathlineto{\pgfqpoint{2.064274in}{1.639423in}}%
\pgfpathlineto{\pgfqpoint{2.152846in}{1.639423in}}%
\pgfpathlineto{\pgfqpoint{2.152846in}{1.812419in}}%
\pgfpathlineto{\pgfqpoint{2.241417in}{1.812419in}}%
\pgfpathlineto{\pgfqpoint{2.241417in}{1.986302in}}%
\pgfpathlineto{\pgfqpoint{2.329989in}{1.986302in}}%
\pgfpathlineto{\pgfqpoint{2.329989in}{2.185469in}}%
\pgfpathlineto{\pgfqpoint{2.418560in}{2.185469in}}%
\pgfpathlineto{\pgfqpoint{2.418560in}{2.321204in}}%
\pgfpathlineto{\pgfqpoint{2.507131in}{2.321204in}}%
\pgfpathlineto{\pgfqpoint{2.507131in}{2.410364in}}%
\pgfpathlineto{\pgfqpoint{2.595703in}{2.410364in}}%
\pgfpathlineto{\pgfqpoint{2.595703in}{2.448068in}}%
\pgfpathlineto{\pgfqpoint{2.684274in}{2.448068in}}%
\pgfpathlineto{\pgfqpoint{2.684274in}{2.462706in}}%
\pgfpathlineto{\pgfqpoint{2.772846in}{2.462706in}}%
\pgfpathlineto{\pgfqpoint{2.772846in}{2.441858in}}%
\pgfpathlineto{\pgfqpoint{2.861417in}{2.441858in}}%
\pgfpathlineto{\pgfqpoint{2.861417in}{2.415243in}}%
\pgfpathlineto{\pgfqpoint{2.949989in}{2.415243in}}%
\pgfpathlineto{\pgfqpoint{2.949989in}{2.270193in}}%
\pgfpathlineto{\pgfqpoint{3.038560in}{2.270193in}}%
\pgfpathlineto{\pgfqpoint{3.038560in}{2.150870in}}%
\pgfpathlineto{\pgfqpoint{3.127131in}{2.150870in}}%
\pgfpathlineto{\pgfqpoint{3.127131in}{1.969002in}}%
\pgfpathlineto{\pgfqpoint{3.215703in}{1.969002in}}%
\pgfpathlineto{\pgfqpoint{3.215703in}{1.804878in}}%
\pgfpathlineto{\pgfqpoint{3.304274in}{1.804878in}}%
\pgfpathlineto{\pgfqpoint{3.304274in}{1.608372in}}%
\pgfpathlineto{\pgfqpoint{3.392846in}{1.608372in}}%
\pgfpathlineto{\pgfqpoint{3.392846in}{1.454007in}}%
\pgfpathlineto{\pgfqpoint{3.481417in}{1.454007in}}%
\pgfpathlineto{\pgfqpoint{3.481417in}{1.308069in}}%
\pgfpathlineto{\pgfqpoint{3.569989in}{1.308069in}}%
\pgfpathlineto{\pgfqpoint{3.569989in}{1.217579in}}%
\pgfpathlineto{\pgfqpoint{3.658560in}{1.217579in}}%
\pgfpathlineto{\pgfqpoint{3.658560in}{1.078295in}}%
\pgfpathlineto{\pgfqpoint{3.747131in}{1.078295in}}%
\pgfpathlineto{\pgfqpoint{3.747131in}{0.981151in}}%
\pgfpathlineto{\pgfqpoint{3.835703in}{0.981151in}}%
\pgfpathlineto{\pgfqpoint{3.835703in}{0.903081in}}%
\pgfpathlineto{\pgfqpoint{3.924274in}{0.903081in}}%
\pgfpathlineto{\pgfqpoint{3.924274in}{0.862716in}}%
\pgfpathlineto{\pgfqpoint{4.012846in}{0.862716in}}%
\pgfpathlineto{\pgfqpoint{4.012846in}{0.802389in}}%
\pgfpathlineto{\pgfqpoint{4.101417in}{0.802389in}}%
\pgfpathlineto{\pgfqpoint{4.101417in}{0.785089in}}%
\pgfpathlineto{\pgfqpoint{4.189989in}{0.785089in}}%
\pgfpathlineto{\pgfqpoint{4.189989in}{0.748716in}}%
\pgfpathlineto{\pgfqpoint{4.278560in}{0.748716in}}%
\pgfpathlineto{\pgfqpoint{4.278560in}{0.729642in}}%
\pgfpathlineto{\pgfqpoint{4.367131in}{0.729642in}}%
\pgfpathlineto{\pgfqpoint{4.367131in}{0.705245in}}%
\pgfpathlineto{\pgfqpoint{4.455703in}{0.705245in}}%
\pgfpathlineto{\pgfqpoint{4.455703in}{0.698591in}}%
\pgfpathlineto{\pgfqpoint{4.544274in}{0.698591in}}%
\pgfpathlineto{\pgfqpoint{4.544274in}{0.690163in}}%
\pgfpathlineto{\pgfqpoint{4.632846in}{0.690163in}}%
\pgfpathlineto{\pgfqpoint{4.632846in}{0.679961in}}%
\pgfpathlineto{\pgfqpoint{4.721417in}{0.679961in}}%
\pgfpathlineto{\pgfqpoint{4.721417in}{0.672864in}}%
\pgfpathlineto{\pgfqpoint{4.809989in}{0.672864in}}%
\pgfpathlineto{\pgfqpoint{4.809989in}{0.670646in}}%
\pgfpathlineto{\pgfqpoint{4.898560in}{0.670646in}}%
\pgfpathlineto{\pgfqpoint{4.898560in}{0.668872in}}%
\pgfpathlineto{\pgfqpoint{4.987131in}{0.668872in}}%
\pgfpathlineto{\pgfqpoint{4.987131in}{0.667541in}}%
\pgfpathlineto{\pgfqpoint{5.075703in}{0.667541in}}%
\pgfpathlineto{\pgfqpoint{5.075703in}{0.664436in}}%
\pgfpathlineto{\pgfqpoint{5.341417in}{0.663992in}}%
\pgfpathlineto{\pgfqpoint{5.341417in}{0.661774in}}%
\pgfpathlineto{\pgfqpoint{5.607131in}{0.660887in}}%
\pgfpathlineto{\pgfqpoint{5.607131in}{0.660444in}}%
\pgfpathlineto{\pgfqpoint{5.695703in}{0.660444in}}%
\pgfpathlineto{\pgfqpoint{5.695703in}{0.662218in}}%
\pgfpathlineto{\pgfqpoint{5.784274in}{0.662218in}}%
\pgfpathlineto{\pgfqpoint{5.784274in}{0.660444in}}%
\pgfpathlineto{\pgfqpoint{7.210000in}{0.660000in}}%
\pgfpathlineto{\pgfqpoint{7.210000in}{0.660000in}}%
\pgfusepath{stroke}%
\end{pgfscope}%
\begin{pgfscope}%
\pgfpathrectangle{\pgfqpoint{1.000000in}{0.660000in}}{\pgfqpoint{6.200000in}{4.620000in}}%
\pgfusepath{clip}%
\pgfsetrectcap%
\pgfsetroundjoin%
\pgfsetlinewidth{2.007500pt}%
\definecolor{currentstroke}{rgb}{0.000000,0.000000,0.000000}%
\pgfsetstrokecolor{currentstroke}%
\pgfsetdash{}{0pt}%
\pgfpathmoveto{\pgfqpoint{1.000000in}{0.661189in}}%
\pgfpathlineto{\pgfqpoint{1.209029in}{0.663705in}}%
\pgfpathlineto{\pgfqpoint{1.347200in}{0.667487in}}%
\pgfpathlineto{\pgfqpoint{1.453486in}{0.672533in}}%
\pgfpathlineto{\pgfqpoint{1.542057in}{0.678926in}}%
\pgfpathlineto{\pgfqpoint{1.618229in}{0.686643in}}%
\pgfpathlineto{\pgfqpoint{1.685543in}{0.695700in}}%
\pgfpathlineto{\pgfqpoint{1.745771in}{0.706030in}}%
\pgfpathlineto{\pgfqpoint{1.800686in}{0.717669in}}%
\pgfpathlineto{\pgfqpoint{1.852057in}{0.730822in}}%
\pgfpathlineto{\pgfqpoint{1.899886in}{0.745347in}}%
\pgfpathlineto{\pgfqpoint{1.945943in}{0.761703in}}%
\pgfpathlineto{\pgfqpoint{1.988457in}{0.779123in}}%
\pgfpathlineto{\pgfqpoint{2.029200in}{0.798143in}}%
\pgfpathlineto{\pgfqpoint{2.069943in}{0.819672in}}%
\pgfpathlineto{\pgfqpoint{2.108914in}{0.842830in}}%
\pgfpathlineto{\pgfqpoint{2.146114in}{0.867476in}}%
\pgfpathlineto{\pgfqpoint{2.183314in}{0.894797in}}%
\pgfpathlineto{\pgfqpoint{2.220514in}{0.924983in}}%
\pgfpathlineto{\pgfqpoint{2.255943in}{0.956572in}}%
\pgfpathlineto{\pgfqpoint{2.291371in}{0.991099in}}%
\pgfpathlineto{\pgfqpoint{2.326800in}{1.028721in}}%
\pgfpathlineto{\pgfqpoint{2.362229in}{1.069594in}}%
\pgfpathlineto{\pgfqpoint{2.397657in}{1.113861in}}%
\pgfpathlineto{\pgfqpoint{2.433086in}{1.161657in}}%
\pgfpathlineto{\pgfqpoint{2.468514in}{1.213101in}}%
\pgfpathlineto{\pgfqpoint{2.505714in}{1.271159in}}%
\pgfpathlineto{\pgfqpoint{2.542914in}{1.333452in}}%
\pgfpathlineto{\pgfqpoint{2.580114in}{1.400051in}}%
\pgfpathlineto{\pgfqpoint{2.619086in}{1.474486in}}%
\pgfpathlineto{\pgfqpoint{2.658057in}{1.553700in}}%
\pgfpathlineto{\pgfqpoint{2.698800in}{1.641583in}}%
\pgfpathlineto{\pgfqpoint{2.741314in}{1.738705in}}%
\pgfpathlineto{\pgfqpoint{2.785600in}{1.845559in}}%
\pgfpathlineto{\pgfqpoint{2.831657in}{1.962521in}}%
\pgfpathlineto{\pgfqpoint{2.881257in}{2.094623in}}%
\pgfpathlineto{\pgfqpoint{2.936171in}{2.247461in}}%
\pgfpathlineto{\pgfqpoint{2.998171in}{2.426878in}}%
\pgfpathlineto{\pgfqpoint{3.076114in}{2.659634in}}%
\pgfpathlineto{\pgfqpoint{3.265657in}{3.228729in}}%
\pgfpathlineto{\pgfqpoint{3.322343in}{3.390299in}}%
\pgfpathlineto{\pgfqpoint{3.370171in}{3.520229in}}%
\pgfpathlineto{\pgfqpoint{3.412686in}{3.629524in}}%
\pgfpathlineto{\pgfqpoint{3.451657in}{3.723672in}}%
\pgfpathlineto{\pgfqpoint{3.487086in}{3.803586in}}%
\pgfpathlineto{\pgfqpoint{3.520743in}{3.873984in}}%
\pgfpathlineto{\pgfqpoint{3.550857in}{3.932042in}}%
\pgfpathlineto{\pgfqpoint{3.579200in}{3.982157in}}%
\pgfpathlineto{\pgfqpoint{3.605771in}{4.024944in}}%
\pgfpathlineto{\pgfqpoint{3.632343in}{4.063491in}}%
\pgfpathlineto{\pgfqpoint{3.657143in}{4.095506in}}%
\pgfpathlineto{\pgfqpoint{3.680171in}{4.121705in}}%
\pgfpathlineto{\pgfqpoint{3.701429in}{4.142799in}}%
\pgfpathlineto{\pgfqpoint{3.722686in}{4.160869in}}%
\pgfpathlineto{\pgfqpoint{3.742171in}{4.174736in}}%
\pgfpathlineto{\pgfqpoint{3.761657in}{4.185990in}}%
\pgfpathlineto{\pgfqpoint{3.781143in}{4.194606in}}%
\pgfpathlineto{\pgfqpoint{3.798857in}{4.200133in}}%
\pgfpathlineto{\pgfqpoint{3.816571in}{4.203453in}}%
\pgfpathlineto{\pgfqpoint{3.834286in}{4.204561in}}%
\pgfpathlineto{\pgfqpoint{3.852000in}{4.203453in}}%
\pgfpathlineto{\pgfqpoint{3.869714in}{4.200133in}}%
\pgfpathlineto{\pgfqpoint{3.887429in}{4.194606in}}%
\pgfpathlineto{\pgfqpoint{3.905143in}{4.186882in}}%
\pgfpathlineto{\pgfqpoint{3.924629in}{4.175867in}}%
\pgfpathlineto{\pgfqpoint{3.944114in}{4.162237in}}%
\pgfpathlineto{\pgfqpoint{3.965371in}{4.144421in}}%
\pgfpathlineto{\pgfqpoint{3.986629in}{4.123577in}}%
\pgfpathlineto{\pgfqpoint{4.009657in}{4.097643in}}%
\pgfpathlineto{\pgfqpoint{4.032686in}{4.068302in}}%
\pgfpathlineto{\pgfqpoint{4.057486in}{4.032997in}}%
\pgfpathlineto{\pgfqpoint{4.084057in}{3.991045in}}%
\pgfpathlineto{\pgfqpoint{4.112400in}{3.941781in}}%
\pgfpathlineto{\pgfqpoint{4.140743in}{3.888076in}}%
\pgfpathlineto{\pgfqpoint{4.172629in}{3.822653in}}%
\pgfpathlineto{\pgfqpoint{4.206286in}{3.748241in}}%
\pgfpathlineto{\pgfqpoint{4.241714in}{3.664472in}}%
\pgfpathlineto{\pgfqpoint{4.280686in}{3.566543in}}%
\pgfpathlineto{\pgfqpoint{4.323200in}{3.453693in}}%
\pgfpathlineto{\pgfqpoint{4.372800in}{3.315450in}}%
\pgfpathlineto{\pgfqpoint{4.431257in}{3.145593in}}%
\pgfpathlineto{\pgfqpoint{4.512743in}{2.901199in}}%
\pgfpathlineto{\pgfqpoint{4.673943in}{2.416462in}}%
\pgfpathlineto{\pgfqpoint{4.739486in}{2.227389in}}%
\pgfpathlineto{\pgfqpoint{4.796171in}{2.070594in}}%
\pgfpathlineto{\pgfqpoint{4.847543in}{1.935023in}}%
\pgfpathlineto{\pgfqpoint{4.895371in}{1.815070in}}%
\pgfpathlineto{\pgfqpoint{4.939657in}{1.709815in}}%
\pgfpathlineto{\pgfqpoint{4.982171in}{1.614292in}}%
\pgfpathlineto{\pgfqpoint{5.022914in}{1.527978in}}%
\pgfpathlineto{\pgfqpoint{5.063657in}{1.446865in}}%
\pgfpathlineto{\pgfqpoint{5.102629in}{1.374169in}}%
\pgfpathlineto{\pgfqpoint{5.141600in}{1.306231in}}%
\pgfpathlineto{\pgfqpoint{5.178800in}{1.245764in}}%
\pgfpathlineto{\pgfqpoint{5.216000in}{1.189493in}}%
\pgfpathlineto{\pgfqpoint{5.253200in}{1.137310in}}%
\pgfpathlineto{\pgfqpoint{5.290400in}{1.089086in}}%
\pgfpathlineto{\pgfqpoint{5.325829in}{1.046703in}}%
\pgfpathlineto{\pgfqpoint{5.361257in}{1.007636in}}%
\pgfpathlineto{\pgfqpoint{5.396686in}{0.971736in}}%
\pgfpathlineto{\pgfqpoint{5.432114in}{0.938845in}}%
\pgfpathlineto{\pgfqpoint{5.469314in}{0.907372in}}%
\pgfpathlineto{\pgfqpoint{5.506514in}{0.878846in}}%
\pgfpathlineto{\pgfqpoint{5.543714in}{0.853077in}}%
\pgfpathlineto{\pgfqpoint{5.582686in}{0.828830in}}%
\pgfpathlineto{\pgfqpoint{5.621657in}{0.807181in}}%
\pgfpathlineto{\pgfqpoint{5.662400in}{0.787099in}}%
\pgfpathlineto{\pgfqpoint{5.704914in}{0.768676in}}%
\pgfpathlineto{\pgfqpoint{5.749200in}{0.751966in}}%
\pgfpathlineto{\pgfqpoint{5.795257in}{0.736987in}}%
\pgfpathlineto{\pgfqpoint{5.844857in}{0.723273in}}%
\pgfpathlineto{\pgfqpoint{5.896229in}{0.711373in}}%
\pgfpathlineto{\pgfqpoint{5.952914in}{0.700574in}}%
\pgfpathlineto{\pgfqpoint{6.014914in}{0.691114in}}%
\pgfpathlineto{\pgfqpoint{6.082229in}{0.683122in}}%
\pgfpathlineto{\pgfqpoint{6.158400in}{0.676346in}}%
\pgfpathlineto{\pgfqpoint{6.246971in}{0.670763in}}%
\pgfpathlineto{\pgfqpoint{6.353257in}{0.666386in}}%
\pgfpathlineto{\pgfqpoint{6.487886in}{0.663192in}}%
\pgfpathlineto{\pgfqpoint{6.675657in}{0.661142in}}%
\pgfpathlineto{\pgfqpoint{6.999829in}{0.660164in}}%
\pgfpathlineto{\pgfqpoint{7.201771in}{0.660044in}}%
\pgfpathlineto{\pgfqpoint{7.201771in}{0.660044in}}%
\pgfusepath{stroke}%
\end{pgfscope}%
\begin{pgfscope}%
\pgfsetrectcap%
\pgfsetmiterjoin%
\pgfsetlinewidth{0.803000pt}%
\definecolor{currentstroke}{rgb}{0.000000,0.000000,0.000000}%
\pgfsetstrokecolor{currentstroke}%
\pgfsetdash{}{0pt}%
\pgfpathmoveto{\pgfqpoint{1.000000in}{0.660000in}}%
\pgfpathlineto{\pgfqpoint{1.000000in}{5.280000in}}%
\pgfusepath{stroke}%
\end{pgfscope}%
\begin{pgfscope}%
\pgfsetrectcap%
\pgfsetmiterjoin%
\pgfsetlinewidth{0.803000pt}%
\definecolor{currentstroke}{rgb}{0.000000,0.000000,0.000000}%
\pgfsetstrokecolor{currentstroke}%
\pgfsetdash{}{0pt}%
\pgfpathmoveto{\pgfqpoint{7.200000in}{0.660000in}}%
\pgfpathlineto{\pgfqpoint{7.200000in}{5.280000in}}%
\pgfusepath{stroke}%
\end{pgfscope}%
\begin{pgfscope}%
\pgfsetrectcap%
\pgfsetmiterjoin%
\pgfsetlinewidth{0.803000pt}%
\definecolor{currentstroke}{rgb}{0.000000,0.000000,0.000000}%
\pgfsetstrokecolor{currentstroke}%
\pgfsetdash{}{0pt}%
\pgfpathmoveto{\pgfqpoint{1.000000in}{0.660000in}}%
\pgfpathlineto{\pgfqpoint{7.200000in}{0.660000in}}%
\pgfusepath{stroke}%
\end{pgfscope}%
\begin{pgfscope}%
\pgfsetrectcap%
\pgfsetmiterjoin%
\pgfsetlinewidth{0.803000pt}%
\definecolor{currentstroke}{rgb}{0.000000,0.000000,0.000000}%
\pgfsetstrokecolor{currentstroke}%
\pgfsetdash{}{0pt}%
\pgfpathmoveto{\pgfqpoint{1.000000in}{5.280000in}}%
\pgfpathlineto{\pgfqpoint{7.200000in}{5.280000in}}%
\pgfusepath{stroke}%
\end{pgfscope}%
\begin{pgfscope}%
\pgfsetbuttcap%
\pgfsetmiterjoin%
\definecolor{currentfill}{rgb}{1.000000,1.000000,1.000000}%
\pgfsetfillcolor{currentfill}%
\pgfsetfillopacity{0.800000}%
\pgfsetlinewidth{1.003750pt}%
\definecolor{currentstroke}{rgb}{0.800000,0.800000,0.800000}%
\pgfsetstrokecolor{currentstroke}%
\pgfsetstrokeopacity{0.800000}%
\pgfsetdash{}{0pt}%
\pgfpathmoveto{\pgfqpoint{4.735997in}{2.688039in}}%
\pgfpathlineto{\pgfqpoint{7.005556in}{2.688039in}}%
\pgfpathquadraticcurveto{\pgfqpoint{7.061111in}{2.688039in}}{\pgfqpoint{7.061111in}{2.743594in}}%
\pgfpathlineto{\pgfqpoint{7.061111in}{5.085556in}}%
\pgfpathquadraticcurveto{\pgfqpoint{7.061111in}{5.141111in}}{\pgfqpoint{7.005556in}{5.141111in}}%
\pgfpathlineto{\pgfqpoint{4.735997in}{5.141111in}}%
\pgfpathquadraticcurveto{\pgfqpoint{4.680441in}{5.141111in}}{\pgfqpoint{4.680441in}{5.085556in}}%
\pgfpathlineto{\pgfqpoint{4.680441in}{2.743594in}}%
\pgfpathquadraticcurveto{\pgfqpoint{4.680441in}{2.688039in}}{\pgfqpoint{4.735997in}{2.688039in}}%
\pgfpathclose%
\pgfusepath{stroke,fill}%
\end{pgfscope}%
\begin{pgfscope}%
\pgfsetrectcap%
\pgfsetroundjoin%
\pgfsetlinewidth{2.007500pt}%
\definecolor{currentstroke}{rgb}{0.000000,0.000000,0.000000}%
\pgfsetstrokecolor{currentstroke}%
\pgfsetdash{}{0pt}%
\pgfpathmoveto{\pgfqpoint{4.791552in}{4.927184in}}%
\pgfpathlineto{\pgfqpoint{5.347108in}{4.927184in}}%
\pgfusepath{stroke}%
\end{pgfscope}%
\begin{pgfscope}%
\definecolor{textcolor}{rgb}{0.000000,0.000000,0.000000}%
\pgfsetstrokecolor{textcolor}%
\pgfsetfillcolor{textcolor}%
\pgftext[x=5.569330in,y=4.829962in,left,base]{\color{textcolor}\sffamily\fontsize{20.000000}{24.000000}\selectfont \(\displaystyle \mathrm{ChargePDF}\)}%
\end{pgfscope}%
\begin{pgfscope}%
\pgfsetbuttcap%
\pgfsetmiterjoin%
\pgfsetlinewidth{1.003750pt}%
\definecolor{currentstroke}{rgb}{0.000000,0.500000,0.000000}%
\pgfsetstrokecolor{currentstroke}%
\pgfsetstrokeopacity{0.500000}%
\pgfsetdash{}{0pt}%
\pgfpathmoveto{\pgfqpoint{4.791552in}{4.435005in}}%
\pgfpathlineto{\pgfqpoint{5.347108in}{4.435005in}}%
\pgfpathlineto{\pgfqpoint{5.347108in}{4.629449in}}%
\pgfpathlineto{\pgfqpoint{4.791552in}{4.629449in}}%
\pgfpathclose%
\pgfusepath{stroke}%
\end{pgfscope}%
\begin{pgfscope}%
\definecolor{textcolor}{rgb}{0.000000,0.000000,0.000000}%
\pgfsetstrokecolor{textcolor}%
\pgfsetfillcolor{textcolor}%
\pgftext[x=5.569330in,y=4.435005in,left,base]{\color{textcolor}\sffamily\fontsize{20.000000}{24.000000}\selectfont \(\displaystyle \mathrm{LucyDDM}\)}%
\end{pgfscope}%
\begin{pgfscope}%
\pgfsetbuttcap%
\pgfsetmiterjoin%
\pgfsetlinewidth{1.003750pt}%
\definecolor{currentstroke}{rgb}{0.750000,0.000000,0.750000}%
\pgfsetstrokecolor{currentstroke}%
\pgfsetstrokeopacity{0.500000}%
\pgfsetdash{}{0pt}%
\pgfpathmoveto{\pgfqpoint{4.791552in}{4.040048in}}%
\pgfpathlineto{\pgfqpoint{5.347108in}{4.040048in}}%
\pgfpathlineto{\pgfqpoint{5.347108in}{4.234493in}}%
\pgfpathlineto{\pgfqpoint{4.791552in}{4.234493in}}%
\pgfpathclose%
\pgfusepath{stroke}%
\end{pgfscope}%
\begin{pgfscope}%
\definecolor{textcolor}{rgb}{0.000000,0.000000,0.000000}%
\pgfsetstrokecolor{textcolor}%
\pgfsetfillcolor{textcolor}%
\pgftext[x=5.569330in,y=4.040048in,left,base]{\color{textcolor}\sffamily\fontsize{20.000000}{24.000000}\selectfont \(\displaystyle \mathrm{Fitting}\)}%
\end{pgfscope}%
\begin{pgfscope}%
\pgfsetbuttcap%
\pgfsetmiterjoin%
\pgfsetlinewidth{1.003750pt}%
\definecolor{currentstroke}{rgb}{0.750000,0.750000,0.000000}%
\pgfsetstrokecolor{currentstroke}%
\pgfsetstrokeopacity{0.500000}%
\pgfsetdash{}{0pt}%
\pgfpathmoveto{\pgfqpoint{4.791552in}{3.645092in}}%
\pgfpathlineto{\pgfqpoint{5.347108in}{3.645092in}}%
\pgfpathlineto{\pgfqpoint{5.347108in}{3.839536in}}%
\pgfpathlineto{\pgfqpoint{4.791552in}{3.839536in}}%
\pgfpathclose%
\pgfusepath{stroke}%
\end{pgfscope}%
\begin{pgfscope}%
\definecolor{textcolor}{rgb}{0.000000,0.000000,0.000000}%
\pgfsetstrokecolor{textcolor}%
\pgfsetfillcolor{textcolor}%
\pgftext[x=5.569330in,y=3.645092in,left,base]{\color{textcolor}\sffamily\fontsize{20.000000}{24.000000}\selectfont \(\displaystyle \mathrm{CNN}\)}%
\end{pgfscope}%
\begin{pgfscope}%
\pgfsetbuttcap%
\pgfsetmiterjoin%
\pgfsetlinewidth{1.003750pt}%
\definecolor{currentstroke}{rgb}{0.000000,0.000000,1.000000}%
\pgfsetstrokecolor{currentstroke}%
\pgfsetstrokeopacity{0.500000}%
\pgfsetdash{}{0pt}%
\pgfpathmoveto{\pgfqpoint{4.791552in}{3.250135in}}%
\pgfpathlineto{\pgfqpoint{5.347108in}{3.250135in}}%
\pgfpathlineto{\pgfqpoint{5.347108in}{3.444580in}}%
\pgfpathlineto{\pgfqpoint{4.791552in}{3.444580in}}%
\pgfpathclose%
\pgfusepath{stroke}%
\end{pgfscope}%
\begin{pgfscope}%
\definecolor{textcolor}{rgb}{0.000000,0.000000,0.000000}%
\pgfsetstrokecolor{textcolor}%
\pgfsetfillcolor{textcolor}%
\pgftext[x=5.569330in,y=3.250135in,left,base]{\color{textcolor}\sffamily\fontsize{20.000000}{24.000000}\selectfont \(\displaystyle \mathrm{FBMP}\)}%
\end{pgfscope}%
\begin{pgfscope}%
\pgfsetbuttcap%
\pgfsetmiterjoin%
\pgfsetlinewidth{1.003750pt}%
\definecolor{currentstroke}{rgb}{1.000000,0.000000,0.000000}%
\pgfsetstrokecolor{currentstroke}%
\pgfsetstrokeopacity{0.500000}%
\pgfsetdash{}{0pt}%
\pgfpathmoveto{\pgfqpoint{4.791552in}{2.855179in}}%
\pgfpathlineto{\pgfqpoint{5.347108in}{2.855179in}}%
\pgfpathlineto{\pgfqpoint{5.347108in}{3.049623in}}%
\pgfpathlineto{\pgfqpoint{4.791552in}{3.049623in}}%
\pgfpathclose%
\pgfusepath{stroke}%
\end{pgfscope}%
\begin{pgfscope}%
\definecolor{textcolor}{rgb}{0.000000,0.000000,0.000000}%
\pgfsetstrokecolor{textcolor}%
\pgfsetfillcolor{textcolor}%
\pgftext[x=5.569330in,y=2.855179in,left,base]{\color{textcolor}\sffamily\fontsize{20.000000}{24.000000}\selectfont \(\displaystyle \mathrm{MCMC}\)}%
\end{pgfscope}%
\end{pgfpicture}%
\makeatother%
\endgroup%
}
    \caption{\label{fig:recchargehist}  $\hat{q}$ distribution, $\mu=5, \tau=\SI{20}{ns}, \sigma=\SI{5}{ns}$}
\end{figure}

The charge of reconstruction, $\hat{q}$, result distribution when $\mu=5, \tau=20\mathrm{ns}, \sigma=10\mathrm{ns}$ shows in figure (see figure~\ref{fig:recchargehist}). 

The steep edge near 0 in these distributions results from the cut of the original $\hat{q}$, which is intended to remove fragment values of $\hat{q}$. But as we can see in figure~\ref{fig:recchargehist}, the charge distribution of LucyDDM, Fitting, CNN, and MCMC are severely distorted, while the charge distribution of FBMP partly retains the shape of the original (or prior) distribution of charge and only some skewing persists. 

During optimization of Wasserstein distance between $\hat{q}(t)$ and $q_{tru}(t)$ (e.g. CNN) or RSS between the origin wave $v_{w}(t)$ and reconstructed wave $v_{r}(t)$ (e.g. Fitting and LucyDDM), the total number of hittime which is the degree of freedom (DOF) of parameters is a constant, and the analysis process is not treated as a sparse regression problem. So a lot of parameters in $q$ turn out to be fragment values, which distorts the $\hat{q}$ distribution. So the reduction of DOF before estimation of parameters in FBMP gives relatively good results in sparsity, though it is still noticeable that the skewness of charge distribution of FBMP persists. 

% Bayesian

As the origin waveform $v_{w}(t)$ is contaminated by Gaussian noise, the optimization a of single loss might be disturbed by the degeneracy of reconstruction results, which means for different $\hat{q}(t)$, the W-dist or RSS can result in the identity or very close. So it is adequate if we can obtain several samples from parameter space of $q$, and even superior if we can give the posterior probability of these samples. Additionally, the expansibility of the Bayesian method will allow us to retain sufficient information in subsequent steps such as event reconstruction. 

% Time bin

But as we see, the reconstruction results of hittime, $\hat{t}$, are discrete values (time bin edges in a DAQ window). To obtain continuous values of $\hat{t}$ may result in better analysis results. The results in the study show that when using refined time bins, the time resolution, the sparsity of reconstructed PEs, and all figures of merits (Wasserstein distance and RSS) are better, however, a lost efficiency. 

% Pedestal & Hardware

The evaluation of the pedestal of waveforms is not included in this work, which is a potential future work to do. Additionally, widely equipped PMTs bring the storage pressure of readout data. This problem can be solved by a voltage data pre-process hardware where waveform analysis is implemented, which is equivalent, or even more effective than data compression. 

\subsection{Timing Resolution}

% Likelihood

Formula \eqref{eq:pseudo} is the likelihood to estimate $t_{0}$ in formula \eqref{eq:time-pro}. But $q_{i}$ is the charge in each time bin $t_{i}$ rather the number of PE. 

The timing resolution $\delta$ using waveform analysis results (charge $\hat{q}$) of these methods are computed and collected. The $t_{0}$ reconstruction process is also MLE, which introduced by the formula \eqref{eq:pseudo}. Additionally, MCMC can provide an estimation of $t_{0}$ directly, which is also collected, and FBMP provides dozens of samples of $q$, which allows us to derive MLE estimation of $t_{0}$. The comparison between timing resolution using the first PE, all PE, and the 5 methods is shown in figure~\ref{fig:deltamethods}. 

\begin{figure}[H]
    \centering
    \resizebox{\textwidth}{!}{%% Creator: Matplotlib, PGF backend
%%
%% To include the figure in your LaTeX document, write
%%   \input{<filename>.pgf}
%%
%% Make sure the required packages are loaded in your preamble
%%   \usepackage{pgf}
%%
%% and, on pdftex
%%   \usepackage[utf8]{inputenc}\DeclareUnicodeCharacter{2212}{-}
%%
%% or, on luatex and xetex
%%   \usepackage{unicode-math}
%%
%% Figures using additional raster images can only be included by \input if
%% they are in the same directory as the main LaTeX file. For loading figures
%% from other directories you can use the `import` package
%%   \usepackage{import}
%%
%% and then include the figures with
%%   \import{<path to file>}{<filename>.pgf}
%%
%% Matplotlib used the following preamble
%%   \usepackage[detect-all,locale=DE]{siunitx}
%%
\begingroup%
\makeatletter%
\begin{pgfpicture}%
\pgfpathrectangle{\pgfpointorigin}{\pgfqpoint{10.000000in}{3.000000in}}%
\pgfusepath{use as bounding box, clip}%
\begin{pgfscope}%
\pgfsetbuttcap%
\pgfsetmiterjoin%
\definecolor{currentfill}{rgb}{1.000000,1.000000,1.000000}%
\pgfsetfillcolor{currentfill}%
\pgfsetlinewidth{0.000000pt}%
\definecolor{currentstroke}{rgb}{1.000000,1.000000,1.000000}%
\pgfsetstrokecolor{currentstroke}%
\pgfsetdash{}{0pt}%
\pgfpathmoveto{\pgfqpoint{0.000000in}{0.000000in}}%
\pgfpathlineto{\pgfqpoint{10.000000in}{0.000000in}}%
\pgfpathlineto{\pgfqpoint{10.000000in}{3.000000in}}%
\pgfpathlineto{\pgfqpoint{0.000000in}{3.000000in}}%
\pgfpathclose%
\pgfusepath{fill}%
\end{pgfscope}%
\begin{pgfscope}%
\pgfsetbuttcap%
\pgfsetmiterjoin%
\definecolor{currentfill}{rgb}{1.000000,1.000000,1.000000}%
\pgfsetfillcolor{currentfill}%
\pgfsetlinewidth{0.000000pt}%
\definecolor{currentstroke}{rgb}{0.000000,0.000000,0.000000}%
\pgfsetstrokecolor{currentstroke}%
\pgfsetstrokeopacity{0.000000}%
\pgfsetdash{}{0pt}%
\pgfpathmoveto{\pgfqpoint{1.000000in}{0.450000in}}%
\pgfpathlineto{\pgfqpoint{4.043478in}{0.450000in}}%
\pgfpathlineto{\pgfqpoint{4.043478in}{2.760000in}}%
\pgfpathlineto{\pgfqpoint{1.000000in}{2.760000in}}%
\pgfpathclose%
\pgfusepath{fill}%
\end{pgfscope}%
\begin{pgfscope}%
\pgfpathrectangle{\pgfqpoint{1.000000in}{0.450000in}}{\pgfqpoint{3.043478in}{2.310000in}}%
\pgfusepath{clip}%
\pgfsetrectcap%
\pgfsetroundjoin%
\pgfsetlinewidth{0.803000pt}%
\definecolor{currentstroke}{rgb}{0.690196,0.690196,0.690196}%
\pgfsetstrokecolor{currentstroke}%
\pgfsetdash{}{0pt}%
\pgfpathmoveto{\pgfqpoint{1.101077in}{0.450000in}}%
\pgfpathlineto{\pgfqpoint{1.101077in}{2.760000in}}%
\pgfusepath{stroke}%
\end{pgfscope}%
\begin{pgfscope}%
\pgfsetbuttcap%
\pgfsetroundjoin%
\definecolor{currentfill}{rgb}{0.000000,0.000000,0.000000}%
\pgfsetfillcolor{currentfill}%
\pgfsetlinewidth{0.803000pt}%
\definecolor{currentstroke}{rgb}{0.000000,0.000000,0.000000}%
\pgfsetstrokecolor{currentstroke}%
\pgfsetdash{}{0pt}%
\pgfsys@defobject{currentmarker}{\pgfqpoint{0.000000in}{-0.048611in}}{\pgfqpoint{0.000000in}{0.000000in}}{%
\pgfpathmoveto{\pgfqpoint{0.000000in}{0.000000in}}%
\pgfpathlineto{\pgfqpoint{0.000000in}{-0.048611in}}%
\pgfusepath{stroke,fill}%
}%
\begin{pgfscope}%
\pgfsys@transformshift{1.101077in}{0.450000in}%
\pgfsys@useobject{currentmarker}{}%
\end{pgfscope}%
\end{pgfscope}%
\begin{pgfscope}%
\definecolor{textcolor}{rgb}{0.000000,0.000000,0.000000}%
\pgfsetstrokecolor{textcolor}%
\pgfsetfillcolor{textcolor}%
\pgftext[x=1.101077in,y=0.352778in,,top]{\color{textcolor}\sffamily\fontsize{12.000000}{14.400000}\selectfont \(\displaystyle {0}\)}%
\end{pgfscope}%
\begin{pgfscope}%
\pgfpathrectangle{\pgfqpoint{1.000000in}{0.450000in}}{\pgfqpoint{3.043478in}{2.310000in}}%
\pgfusepath{clip}%
\pgfsetrectcap%
\pgfsetroundjoin%
\pgfsetlinewidth{0.803000pt}%
\definecolor{currentstroke}{rgb}{0.690196,0.690196,0.690196}%
\pgfsetstrokecolor{currentstroke}%
\pgfsetdash{}{0pt}%
\pgfpathmoveto{\pgfqpoint{2.032659in}{0.450000in}}%
\pgfpathlineto{\pgfqpoint{2.032659in}{2.760000in}}%
\pgfusepath{stroke}%
\end{pgfscope}%
\begin{pgfscope}%
\pgfsetbuttcap%
\pgfsetroundjoin%
\definecolor{currentfill}{rgb}{0.000000,0.000000,0.000000}%
\pgfsetfillcolor{currentfill}%
\pgfsetlinewidth{0.803000pt}%
\definecolor{currentstroke}{rgb}{0.000000,0.000000,0.000000}%
\pgfsetstrokecolor{currentstroke}%
\pgfsetdash{}{0pt}%
\pgfsys@defobject{currentmarker}{\pgfqpoint{0.000000in}{-0.048611in}}{\pgfqpoint{0.000000in}{0.000000in}}{%
\pgfpathmoveto{\pgfqpoint{0.000000in}{0.000000in}}%
\pgfpathlineto{\pgfqpoint{0.000000in}{-0.048611in}}%
\pgfusepath{stroke,fill}%
}%
\begin{pgfscope}%
\pgfsys@transformshift{2.032659in}{0.450000in}%
\pgfsys@useobject{currentmarker}{}%
\end{pgfscope}%
\end{pgfscope}%
\begin{pgfscope}%
\definecolor{textcolor}{rgb}{0.000000,0.000000,0.000000}%
\pgfsetstrokecolor{textcolor}%
\pgfsetfillcolor{textcolor}%
\pgftext[x=2.032659in,y=0.352778in,,top]{\color{textcolor}\sffamily\fontsize{12.000000}{14.400000}\selectfont \(\displaystyle {10}\)}%
\end{pgfscope}%
\begin{pgfscope}%
\pgfpathrectangle{\pgfqpoint{1.000000in}{0.450000in}}{\pgfqpoint{3.043478in}{2.310000in}}%
\pgfusepath{clip}%
\pgfsetrectcap%
\pgfsetroundjoin%
\pgfsetlinewidth{0.803000pt}%
\definecolor{currentstroke}{rgb}{0.690196,0.690196,0.690196}%
\pgfsetstrokecolor{currentstroke}%
\pgfsetdash{}{0pt}%
\pgfpathmoveto{\pgfqpoint{2.964241in}{0.450000in}}%
\pgfpathlineto{\pgfqpoint{2.964241in}{2.760000in}}%
\pgfusepath{stroke}%
\end{pgfscope}%
\begin{pgfscope}%
\pgfsetbuttcap%
\pgfsetroundjoin%
\definecolor{currentfill}{rgb}{0.000000,0.000000,0.000000}%
\pgfsetfillcolor{currentfill}%
\pgfsetlinewidth{0.803000pt}%
\definecolor{currentstroke}{rgb}{0.000000,0.000000,0.000000}%
\pgfsetstrokecolor{currentstroke}%
\pgfsetdash{}{0pt}%
\pgfsys@defobject{currentmarker}{\pgfqpoint{0.000000in}{-0.048611in}}{\pgfqpoint{0.000000in}{0.000000in}}{%
\pgfpathmoveto{\pgfqpoint{0.000000in}{0.000000in}}%
\pgfpathlineto{\pgfqpoint{0.000000in}{-0.048611in}}%
\pgfusepath{stroke,fill}%
}%
\begin{pgfscope}%
\pgfsys@transformshift{2.964241in}{0.450000in}%
\pgfsys@useobject{currentmarker}{}%
\end{pgfscope}%
\end{pgfscope}%
\begin{pgfscope}%
\definecolor{textcolor}{rgb}{0.000000,0.000000,0.000000}%
\pgfsetstrokecolor{textcolor}%
\pgfsetfillcolor{textcolor}%
\pgftext[x=2.964241in,y=0.352778in,,top]{\color{textcolor}\sffamily\fontsize{12.000000}{14.400000}\selectfont \(\displaystyle {20}\)}%
\end{pgfscope}%
\begin{pgfscope}%
\pgfpathrectangle{\pgfqpoint{1.000000in}{0.450000in}}{\pgfqpoint{3.043478in}{2.310000in}}%
\pgfusepath{clip}%
\pgfsetrectcap%
\pgfsetroundjoin%
\pgfsetlinewidth{0.803000pt}%
\definecolor{currentstroke}{rgb}{0.690196,0.690196,0.690196}%
\pgfsetstrokecolor{currentstroke}%
\pgfsetdash{}{0pt}%
\pgfpathmoveto{\pgfqpoint{3.895823in}{0.450000in}}%
\pgfpathlineto{\pgfqpoint{3.895823in}{2.760000in}}%
\pgfusepath{stroke}%
\end{pgfscope}%
\begin{pgfscope}%
\pgfsetbuttcap%
\pgfsetroundjoin%
\definecolor{currentfill}{rgb}{0.000000,0.000000,0.000000}%
\pgfsetfillcolor{currentfill}%
\pgfsetlinewidth{0.803000pt}%
\definecolor{currentstroke}{rgb}{0.000000,0.000000,0.000000}%
\pgfsetstrokecolor{currentstroke}%
\pgfsetdash{}{0pt}%
\pgfsys@defobject{currentmarker}{\pgfqpoint{0.000000in}{-0.048611in}}{\pgfqpoint{0.000000in}{0.000000in}}{%
\pgfpathmoveto{\pgfqpoint{0.000000in}{0.000000in}}%
\pgfpathlineto{\pgfqpoint{0.000000in}{-0.048611in}}%
\pgfusepath{stroke,fill}%
}%
\begin{pgfscope}%
\pgfsys@transformshift{3.895823in}{0.450000in}%
\pgfsys@useobject{currentmarker}{}%
\end{pgfscope}%
\end{pgfscope}%
\begin{pgfscope}%
\definecolor{textcolor}{rgb}{0.000000,0.000000,0.000000}%
\pgfsetstrokecolor{textcolor}%
\pgfsetfillcolor{textcolor}%
\pgftext[x=3.895823in,y=0.352778in,,top]{\color{textcolor}\sffamily\fontsize{12.000000}{14.400000}\selectfont \(\displaystyle {30}\)}%
\end{pgfscope}%
\begin{pgfscope}%
\definecolor{textcolor}{rgb}{0.000000,0.000000,0.000000}%
\pgfsetstrokecolor{textcolor}%
\pgfsetfillcolor{textcolor}%
\pgftext[x=2.521739in,y=0.149075in,,top]{\color{textcolor}\sffamily\fontsize{12.000000}{14.400000}\selectfont \(\displaystyle \mu\)}%
\end{pgfscope}%
\begin{pgfscope}%
\pgfpathrectangle{\pgfqpoint{1.000000in}{0.450000in}}{\pgfqpoint{3.043478in}{2.310000in}}%
\pgfusepath{clip}%
\pgfsetrectcap%
\pgfsetroundjoin%
\pgfsetlinewidth{0.803000pt}%
\definecolor{currentstroke}{rgb}{0.690196,0.690196,0.690196}%
\pgfsetstrokecolor{currentstroke}%
\pgfsetdash{}{0pt}%
\pgfpathmoveto{\pgfqpoint{1.000000in}{0.824872in}}%
\pgfpathlineto{\pgfqpoint{4.043478in}{0.824872in}}%
\pgfusepath{stroke}%
\end{pgfscope}%
\begin{pgfscope}%
\pgfsetbuttcap%
\pgfsetroundjoin%
\definecolor{currentfill}{rgb}{0.000000,0.000000,0.000000}%
\pgfsetfillcolor{currentfill}%
\pgfsetlinewidth{0.803000pt}%
\definecolor{currentstroke}{rgb}{0.000000,0.000000,0.000000}%
\pgfsetstrokecolor{currentstroke}%
\pgfsetdash{}{0pt}%
\pgfsys@defobject{currentmarker}{\pgfqpoint{-0.048611in}{0.000000in}}{\pgfqpoint{-0.000000in}{0.000000in}}{%
\pgfpathmoveto{\pgfqpoint{-0.000000in}{0.000000in}}%
\pgfpathlineto{\pgfqpoint{-0.048611in}{0.000000in}}%
\pgfusepath{stroke,fill}%
}%
\begin{pgfscope}%
\pgfsys@transformshift{1.000000in}{0.824872in}%
\pgfsys@useobject{currentmarker}{}%
\end{pgfscope}%
\end{pgfscope}%
\begin{pgfscope}%
\definecolor{textcolor}{rgb}{0.000000,0.000000,0.000000}%
\pgfsetstrokecolor{textcolor}%
\pgfsetfillcolor{textcolor}%
\pgftext[x=0.612657in, y=0.767001in, left, base]{\color{textcolor}\sffamily\fontsize{12.000000}{14.400000}\selectfont \(\displaystyle {1.00}\)}%
\end{pgfscope}%
\begin{pgfscope}%
\pgfpathrectangle{\pgfqpoint{1.000000in}{0.450000in}}{\pgfqpoint{3.043478in}{2.310000in}}%
\pgfusepath{clip}%
\pgfsetrectcap%
\pgfsetroundjoin%
\pgfsetlinewidth{0.803000pt}%
\definecolor{currentstroke}{rgb}{0.690196,0.690196,0.690196}%
\pgfsetstrokecolor{currentstroke}%
\pgfsetdash{}{0pt}%
\pgfpathmoveto{\pgfqpoint{1.000000in}{1.446904in}}%
\pgfpathlineto{\pgfqpoint{4.043478in}{1.446904in}}%
\pgfusepath{stroke}%
\end{pgfscope}%
\begin{pgfscope}%
\pgfsetbuttcap%
\pgfsetroundjoin%
\definecolor{currentfill}{rgb}{0.000000,0.000000,0.000000}%
\pgfsetfillcolor{currentfill}%
\pgfsetlinewidth{0.803000pt}%
\definecolor{currentstroke}{rgb}{0.000000,0.000000,0.000000}%
\pgfsetstrokecolor{currentstroke}%
\pgfsetdash{}{0pt}%
\pgfsys@defobject{currentmarker}{\pgfqpoint{-0.048611in}{0.000000in}}{\pgfqpoint{-0.000000in}{0.000000in}}{%
\pgfpathmoveto{\pgfqpoint{-0.000000in}{0.000000in}}%
\pgfpathlineto{\pgfqpoint{-0.048611in}{0.000000in}}%
\pgfusepath{stroke,fill}%
}%
\begin{pgfscope}%
\pgfsys@transformshift{1.000000in}{1.446904in}%
\pgfsys@useobject{currentmarker}{}%
\end{pgfscope}%
\end{pgfscope}%
\begin{pgfscope}%
\definecolor{textcolor}{rgb}{0.000000,0.000000,0.000000}%
\pgfsetstrokecolor{textcolor}%
\pgfsetfillcolor{textcolor}%
\pgftext[x=0.612657in, y=1.389034in, left, base]{\color{textcolor}\sffamily\fontsize{12.000000}{14.400000}\selectfont \(\displaystyle {1.02}\)}%
\end{pgfscope}%
\begin{pgfscope}%
\pgfpathrectangle{\pgfqpoint{1.000000in}{0.450000in}}{\pgfqpoint{3.043478in}{2.310000in}}%
\pgfusepath{clip}%
\pgfsetrectcap%
\pgfsetroundjoin%
\pgfsetlinewidth{0.803000pt}%
\definecolor{currentstroke}{rgb}{0.690196,0.690196,0.690196}%
\pgfsetstrokecolor{currentstroke}%
\pgfsetdash{}{0pt}%
\pgfpathmoveto{\pgfqpoint{1.000000in}{2.068936in}}%
\pgfpathlineto{\pgfqpoint{4.043478in}{2.068936in}}%
\pgfusepath{stroke}%
\end{pgfscope}%
\begin{pgfscope}%
\pgfsetbuttcap%
\pgfsetroundjoin%
\definecolor{currentfill}{rgb}{0.000000,0.000000,0.000000}%
\pgfsetfillcolor{currentfill}%
\pgfsetlinewidth{0.803000pt}%
\definecolor{currentstroke}{rgb}{0.000000,0.000000,0.000000}%
\pgfsetstrokecolor{currentstroke}%
\pgfsetdash{}{0pt}%
\pgfsys@defobject{currentmarker}{\pgfqpoint{-0.048611in}{0.000000in}}{\pgfqpoint{-0.000000in}{0.000000in}}{%
\pgfpathmoveto{\pgfqpoint{-0.000000in}{0.000000in}}%
\pgfpathlineto{\pgfqpoint{-0.048611in}{0.000000in}}%
\pgfusepath{stroke,fill}%
}%
\begin{pgfscope}%
\pgfsys@transformshift{1.000000in}{2.068936in}%
\pgfsys@useobject{currentmarker}{}%
\end{pgfscope}%
\end{pgfscope}%
\begin{pgfscope}%
\definecolor{textcolor}{rgb}{0.000000,0.000000,0.000000}%
\pgfsetstrokecolor{textcolor}%
\pgfsetfillcolor{textcolor}%
\pgftext[x=0.612657in, y=2.011066in, left, base]{\color{textcolor}\sffamily\fontsize{12.000000}{14.400000}\selectfont \(\displaystyle {1.04}\)}%
\end{pgfscope}%
\begin{pgfscope}%
\pgfpathrectangle{\pgfqpoint{1.000000in}{0.450000in}}{\pgfqpoint{3.043478in}{2.310000in}}%
\pgfusepath{clip}%
\pgfsetrectcap%
\pgfsetroundjoin%
\pgfsetlinewidth{0.803000pt}%
\definecolor{currentstroke}{rgb}{0.690196,0.690196,0.690196}%
\pgfsetstrokecolor{currentstroke}%
\pgfsetdash{}{0pt}%
\pgfpathmoveto{\pgfqpoint{1.000000in}{2.690968in}}%
\pgfpathlineto{\pgfqpoint{4.043478in}{2.690968in}}%
\pgfusepath{stroke}%
\end{pgfscope}%
\begin{pgfscope}%
\pgfsetbuttcap%
\pgfsetroundjoin%
\definecolor{currentfill}{rgb}{0.000000,0.000000,0.000000}%
\pgfsetfillcolor{currentfill}%
\pgfsetlinewidth{0.803000pt}%
\definecolor{currentstroke}{rgb}{0.000000,0.000000,0.000000}%
\pgfsetstrokecolor{currentstroke}%
\pgfsetdash{}{0pt}%
\pgfsys@defobject{currentmarker}{\pgfqpoint{-0.048611in}{0.000000in}}{\pgfqpoint{-0.000000in}{0.000000in}}{%
\pgfpathmoveto{\pgfqpoint{-0.000000in}{0.000000in}}%
\pgfpathlineto{\pgfqpoint{-0.048611in}{0.000000in}}%
\pgfusepath{stroke,fill}%
}%
\begin{pgfscope}%
\pgfsys@transformshift{1.000000in}{2.690968in}%
\pgfsys@useobject{currentmarker}{}%
\end{pgfscope}%
\end{pgfscope}%
\begin{pgfscope}%
\definecolor{textcolor}{rgb}{0.000000,0.000000,0.000000}%
\pgfsetstrokecolor{textcolor}%
\pgfsetfillcolor{textcolor}%
\pgftext[x=0.612657in, y=2.633098in, left, base]{\color{textcolor}\sffamily\fontsize{12.000000}{14.400000}\selectfont \(\displaystyle {1.06}\)}%
\end{pgfscope}%
\begin{pgfscope}%
\definecolor{textcolor}{rgb}{0.000000,0.000000,0.000000}%
\pgfsetstrokecolor{textcolor}%
\pgfsetfillcolor{textcolor}%
\pgftext[x=0.557102in,y=1.605000in,,bottom,rotate=90.000000]{\color{textcolor}\sffamily\fontsize{12.000000}{14.400000}\selectfont \(\displaystyle \mathrm{ratio}\)}%
\end{pgfscope}%
\begin{pgfscope}%
\pgfpathrectangle{\pgfqpoint{1.000000in}{0.450000in}}{\pgfqpoint{3.043478in}{2.310000in}}%
\pgfusepath{clip}%
\pgfsetbuttcap%
\pgfsetroundjoin%
\pgfsetlinewidth{2.007500pt}%
\definecolor{currentstroke}{rgb}{0.750000,0.750000,0.000000}%
\pgfsetstrokecolor{currentstroke}%
\pgfsetdash{}{0pt}%
\pgfpathmoveto{\pgfqpoint{1.147656in}{0.693841in}}%
\pgfpathlineto{\pgfqpoint{1.147656in}{1.320395in}}%
\pgfusepath{stroke}%
\end{pgfscope}%
\begin{pgfscope}%
\pgfpathrectangle{\pgfqpoint{1.000000in}{0.450000in}}{\pgfqpoint{3.043478in}{2.310000in}}%
\pgfusepath{clip}%
\pgfsetbuttcap%
\pgfsetroundjoin%
\pgfsetlinewidth{2.007500pt}%
\definecolor{currentstroke}{rgb}{0.750000,0.750000,0.000000}%
\pgfsetstrokecolor{currentstroke}%
\pgfsetdash{}{0pt}%
\pgfpathmoveto{\pgfqpoint{1.194235in}{0.850362in}}%
\pgfpathlineto{\pgfqpoint{1.194235in}{1.479924in}}%
\pgfusepath{stroke}%
\end{pgfscope}%
\begin{pgfscope}%
\pgfpathrectangle{\pgfqpoint{1.000000in}{0.450000in}}{\pgfqpoint{3.043478in}{2.310000in}}%
\pgfusepath{clip}%
\pgfsetbuttcap%
\pgfsetroundjoin%
\pgfsetlinewidth{2.007500pt}%
\definecolor{currentstroke}{rgb}{0.750000,0.750000,0.000000}%
\pgfsetstrokecolor{currentstroke}%
\pgfsetdash{}{0pt}%
\pgfpathmoveto{\pgfqpoint{1.240814in}{1.033684in}}%
\pgfpathlineto{\pgfqpoint{1.240814in}{1.666762in}}%
\pgfusepath{stroke}%
\end{pgfscope}%
\begin{pgfscope}%
\pgfpathrectangle{\pgfqpoint{1.000000in}{0.450000in}}{\pgfqpoint{3.043478in}{2.310000in}}%
\pgfusepath{clip}%
\pgfsetbuttcap%
\pgfsetroundjoin%
\pgfsetlinewidth{2.007500pt}%
\definecolor{currentstroke}{rgb}{0.750000,0.750000,0.000000}%
\pgfsetstrokecolor{currentstroke}%
\pgfsetdash{}{0pt}%
\pgfpathmoveto{\pgfqpoint{1.287393in}{1.025735in}}%
\pgfpathlineto{\pgfqpoint{1.287393in}{1.658364in}}%
\pgfusepath{stroke}%
\end{pgfscope}%
\begin{pgfscope}%
\pgfpathrectangle{\pgfqpoint{1.000000in}{0.450000in}}{\pgfqpoint{3.043478in}{2.310000in}}%
\pgfusepath{clip}%
\pgfsetbuttcap%
\pgfsetroundjoin%
\pgfsetlinewidth{2.007500pt}%
\definecolor{currentstroke}{rgb}{0.750000,0.750000,0.000000}%
\pgfsetstrokecolor{currentstroke}%
\pgfsetdash{}{0pt}%
\pgfpathmoveto{\pgfqpoint{1.333972in}{1.148432in}}%
\pgfpathlineto{\pgfqpoint{1.333972in}{1.783412in}}%
\pgfusepath{stroke}%
\end{pgfscope}%
\begin{pgfscope}%
\pgfpathrectangle{\pgfqpoint{1.000000in}{0.450000in}}{\pgfqpoint{3.043478in}{2.310000in}}%
\pgfusepath{clip}%
\pgfsetbuttcap%
\pgfsetroundjoin%
\pgfsetlinewidth{2.007500pt}%
\definecolor{currentstroke}{rgb}{0.750000,0.750000,0.000000}%
\pgfsetstrokecolor{currentstroke}%
\pgfsetdash{}{0pt}%
\pgfpathmoveto{\pgfqpoint{1.380551in}{1.320075in}}%
\pgfpathlineto{\pgfqpoint{1.380551in}{1.958523in}}%
\pgfusepath{stroke}%
\end{pgfscope}%
\begin{pgfscope}%
\pgfpathrectangle{\pgfqpoint{1.000000in}{0.450000in}}{\pgfqpoint{3.043478in}{2.310000in}}%
\pgfusepath{clip}%
\pgfsetbuttcap%
\pgfsetroundjoin%
\pgfsetlinewidth{2.007500pt}%
\definecolor{currentstroke}{rgb}{0.750000,0.750000,0.000000}%
\pgfsetstrokecolor{currentstroke}%
\pgfsetdash{}{0pt}%
\pgfpathmoveto{\pgfqpoint{1.427130in}{1.481159in}}%
\pgfpathlineto{\pgfqpoint{1.427130in}{2.122797in}}%
\pgfusepath{stroke}%
\end{pgfscope}%
\begin{pgfscope}%
\pgfpathrectangle{\pgfqpoint{1.000000in}{0.450000in}}{\pgfqpoint{3.043478in}{2.310000in}}%
\pgfusepath{clip}%
\pgfsetbuttcap%
\pgfsetroundjoin%
\pgfsetlinewidth{2.007500pt}%
\definecolor{currentstroke}{rgb}{0.750000,0.750000,0.000000}%
\pgfsetstrokecolor{currentstroke}%
\pgfsetdash{}{0pt}%
\pgfpathmoveto{\pgfqpoint{1.473709in}{1.630673in}}%
\pgfpathlineto{\pgfqpoint{1.473709in}{2.275300in}}%
\pgfusepath{stroke}%
\end{pgfscope}%
\begin{pgfscope}%
\pgfpathrectangle{\pgfqpoint{1.000000in}{0.450000in}}{\pgfqpoint{3.043478in}{2.310000in}}%
\pgfusepath{clip}%
\pgfsetbuttcap%
\pgfsetroundjoin%
\pgfsetlinewidth{2.007500pt}%
\definecolor{currentstroke}{rgb}{0.750000,0.750000,0.000000}%
\pgfsetstrokecolor{currentstroke}%
\pgfsetdash{}{0pt}%
\pgfpathmoveto{\pgfqpoint{1.660026in}{1.767931in}}%
\pgfpathlineto{\pgfqpoint{1.660026in}{2.415363in}}%
\pgfusepath{stroke}%
\end{pgfscope}%
\begin{pgfscope}%
\pgfpathrectangle{\pgfqpoint{1.000000in}{0.450000in}}{\pgfqpoint{3.043478in}{2.310000in}}%
\pgfusepath{clip}%
\pgfsetbuttcap%
\pgfsetroundjoin%
\pgfsetlinewidth{2.007500pt}%
\definecolor{currentstroke}{rgb}{0.750000,0.750000,0.000000}%
\pgfsetstrokecolor{currentstroke}%
\pgfsetdash{}{0pt}%
\pgfpathmoveto{\pgfqpoint{1.846342in}{2.002854in}}%
\pgfpathlineto{\pgfqpoint{1.846342in}{2.655000in}}%
\pgfusepath{stroke}%
\end{pgfscope}%
\begin{pgfscope}%
\pgfpathrectangle{\pgfqpoint{1.000000in}{0.450000in}}{\pgfqpoint{3.043478in}{2.310000in}}%
\pgfusepath{clip}%
\pgfsetbuttcap%
\pgfsetroundjoin%
\pgfsetlinewidth{2.007500pt}%
\definecolor{currentstroke}{rgb}{0.750000,0.750000,0.000000}%
\pgfsetstrokecolor{currentstroke}%
\pgfsetdash{}{0pt}%
\pgfpathmoveto{\pgfqpoint{2.032659in}{1.823906in}}%
\pgfpathlineto{\pgfqpoint{2.032659in}{2.472404in}}%
\pgfusepath{stroke}%
\end{pgfscope}%
\begin{pgfscope}%
\pgfpathrectangle{\pgfqpoint{1.000000in}{0.450000in}}{\pgfqpoint{3.043478in}{2.310000in}}%
\pgfusepath{clip}%
\pgfsetbuttcap%
\pgfsetroundjoin%
\pgfsetlinewidth{2.007500pt}%
\definecolor{currentstroke}{rgb}{0.750000,0.750000,0.000000}%
\pgfsetstrokecolor{currentstroke}%
\pgfsetdash{}{0pt}%
\pgfpathmoveto{\pgfqpoint{2.498450in}{1.669844in}}%
\pgfpathlineto{\pgfqpoint{2.498450in}{2.315229in}}%
\pgfusepath{stroke}%
\end{pgfscope}%
\begin{pgfscope}%
\pgfpathrectangle{\pgfqpoint{1.000000in}{0.450000in}}{\pgfqpoint{3.043478in}{2.310000in}}%
\pgfusepath{clip}%
\pgfsetbuttcap%
\pgfsetroundjoin%
\pgfsetlinewidth{2.007500pt}%
\definecolor{currentstroke}{rgb}{0.750000,0.750000,0.000000}%
\pgfsetstrokecolor{currentstroke}%
\pgfsetdash{}{0pt}%
\pgfpathmoveto{\pgfqpoint{2.964241in}{1.796310in}}%
\pgfpathlineto{\pgfqpoint{2.964241in}{2.444250in}}%
\pgfusepath{stroke}%
\end{pgfscope}%
\begin{pgfscope}%
\pgfpathrectangle{\pgfqpoint{1.000000in}{0.450000in}}{\pgfqpoint{3.043478in}{2.310000in}}%
\pgfusepath{clip}%
\pgfsetbuttcap%
\pgfsetroundjoin%
\pgfsetlinewidth{2.007500pt}%
\definecolor{currentstroke}{rgb}{0.750000,0.750000,0.000000}%
\pgfsetstrokecolor{currentstroke}%
\pgfsetdash{}{0pt}%
\pgfpathmoveto{\pgfqpoint{3.430032in}{1.580103in}}%
\pgfpathlineto{\pgfqpoint{3.430032in}{2.223675in}}%
\pgfusepath{stroke}%
\end{pgfscope}%
\begin{pgfscope}%
\pgfpathrectangle{\pgfqpoint{1.000000in}{0.450000in}}{\pgfqpoint{3.043478in}{2.310000in}}%
\pgfusepath{clip}%
\pgfsetbuttcap%
\pgfsetroundjoin%
\pgfsetlinewidth{2.007500pt}%
\definecolor{currentstroke}{rgb}{0.750000,0.750000,0.000000}%
\pgfsetstrokecolor{currentstroke}%
\pgfsetdash{}{0pt}%
\pgfpathmoveto{\pgfqpoint{3.895823in}{1.702214in}}%
\pgfpathlineto{\pgfqpoint{3.895823in}{2.348253in}}%
\pgfusepath{stroke}%
\end{pgfscope}%
\begin{pgfscope}%
\pgfpathrectangle{\pgfqpoint{1.000000in}{0.450000in}}{\pgfqpoint{3.043478in}{2.310000in}}%
\pgfusepath{clip}%
\pgfsetbuttcap%
\pgfsetroundjoin%
\pgfsetlinewidth{2.007500pt}%
\definecolor{currentstroke}{rgb}{0.121569,0.466667,0.705882}%
\pgfsetstrokecolor{currentstroke}%
\pgfsetdash{}{0pt}%
\pgfpathmoveto{\pgfqpoint{1.156972in}{0.642922in}}%
\pgfpathlineto{\pgfqpoint{1.156972in}{1.268446in}}%
\pgfusepath{stroke}%
\end{pgfscope}%
\begin{pgfscope}%
\pgfpathrectangle{\pgfqpoint{1.000000in}{0.450000in}}{\pgfqpoint{3.043478in}{2.310000in}}%
\pgfusepath{clip}%
\pgfsetbuttcap%
\pgfsetroundjoin%
\pgfsetlinewidth{2.007500pt}%
\definecolor{currentstroke}{rgb}{0.121569,0.466667,0.705882}%
\pgfsetstrokecolor{currentstroke}%
\pgfsetdash{}{0pt}%
\pgfpathmoveto{\pgfqpoint{1.203551in}{0.744310in}}%
\pgfpathlineto{\pgfqpoint{1.203551in}{1.371727in}}%
\pgfusepath{stroke}%
\end{pgfscope}%
\begin{pgfscope}%
\pgfpathrectangle{\pgfqpoint{1.000000in}{0.450000in}}{\pgfqpoint{3.043478in}{2.310000in}}%
\pgfusepath{clip}%
\pgfsetbuttcap%
\pgfsetroundjoin%
\pgfsetlinewidth{2.007500pt}%
\definecolor{currentstroke}{rgb}{0.121569,0.466667,0.705882}%
\pgfsetstrokecolor{currentstroke}%
\pgfsetdash{}{0pt}%
\pgfpathmoveto{\pgfqpoint{1.250130in}{0.739184in}}%
\pgfpathlineto{\pgfqpoint{1.250130in}{1.366306in}}%
\pgfusepath{stroke}%
\end{pgfscope}%
\begin{pgfscope}%
\pgfpathrectangle{\pgfqpoint{1.000000in}{0.450000in}}{\pgfqpoint{3.043478in}{2.310000in}}%
\pgfusepath{clip}%
\pgfsetbuttcap%
\pgfsetroundjoin%
\pgfsetlinewidth{2.007500pt}%
\definecolor{currentstroke}{rgb}{0.121569,0.466667,0.705882}%
\pgfsetstrokecolor{currentstroke}%
\pgfsetdash{}{0pt}%
\pgfpathmoveto{\pgfqpoint{1.296709in}{0.817144in}}%
\pgfpathlineto{\pgfqpoint{1.296709in}{1.445557in}}%
\pgfusepath{stroke}%
\end{pgfscope}%
\begin{pgfscope}%
\pgfpathrectangle{\pgfqpoint{1.000000in}{0.450000in}}{\pgfqpoint{3.043478in}{2.310000in}}%
\pgfusepath{clip}%
\pgfsetbuttcap%
\pgfsetroundjoin%
\pgfsetlinewidth{2.007500pt}%
\definecolor{currentstroke}{rgb}{0.121569,0.466667,0.705882}%
\pgfsetstrokecolor{currentstroke}%
\pgfsetdash{}{0pt}%
\pgfpathmoveto{\pgfqpoint{1.343288in}{0.842480in}}%
\pgfpathlineto{\pgfqpoint{1.343288in}{1.471278in}}%
\pgfusepath{stroke}%
\end{pgfscope}%
\begin{pgfscope}%
\pgfpathrectangle{\pgfqpoint{1.000000in}{0.450000in}}{\pgfqpoint{3.043478in}{2.310000in}}%
\pgfusepath{clip}%
\pgfsetbuttcap%
\pgfsetroundjoin%
\pgfsetlinewidth{2.007500pt}%
\definecolor{currentstroke}{rgb}{0.121569,0.466667,0.705882}%
\pgfsetstrokecolor{currentstroke}%
\pgfsetdash{}{0pt}%
\pgfpathmoveto{\pgfqpoint{1.389867in}{0.978455in}}%
\pgfpathlineto{\pgfqpoint{1.389867in}{1.610000in}}%
\pgfusepath{stroke}%
\end{pgfscope}%
\begin{pgfscope}%
\pgfpathrectangle{\pgfqpoint{1.000000in}{0.450000in}}{\pgfqpoint{3.043478in}{2.310000in}}%
\pgfusepath{clip}%
\pgfsetbuttcap%
\pgfsetroundjoin%
\pgfsetlinewidth{2.007500pt}%
\definecolor{currentstroke}{rgb}{0.121569,0.466667,0.705882}%
\pgfsetstrokecolor{currentstroke}%
\pgfsetdash{}{0pt}%
\pgfpathmoveto{\pgfqpoint{1.436446in}{1.095647in}}%
\pgfpathlineto{\pgfqpoint{1.436446in}{1.729497in}}%
\pgfusepath{stroke}%
\end{pgfscope}%
\begin{pgfscope}%
\pgfpathrectangle{\pgfqpoint{1.000000in}{0.450000in}}{\pgfqpoint{3.043478in}{2.310000in}}%
\pgfusepath{clip}%
\pgfsetbuttcap%
\pgfsetroundjoin%
\pgfsetlinewidth{2.007500pt}%
\definecolor{currentstroke}{rgb}{0.121569,0.466667,0.705882}%
\pgfsetstrokecolor{currentstroke}%
\pgfsetdash{}{0pt}%
\pgfpathmoveto{\pgfqpoint{1.483025in}{1.172220in}}%
\pgfpathlineto{\pgfqpoint{1.483025in}{1.807584in}}%
\pgfusepath{stroke}%
\end{pgfscope}%
\begin{pgfscope}%
\pgfpathrectangle{\pgfqpoint{1.000000in}{0.450000in}}{\pgfqpoint{3.043478in}{2.310000in}}%
\pgfusepath{clip}%
\pgfsetbuttcap%
\pgfsetroundjoin%
\pgfsetlinewidth{2.007500pt}%
\definecolor{currentstroke}{rgb}{0.121569,0.466667,0.705882}%
\pgfsetstrokecolor{currentstroke}%
\pgfsetdash{}{0pt}%
\pgfpathmoveto{\pgfqpoint{1.669342in}{1.276490in}}%
\pgfpathlineto{\pgfqpoint{1.669342in}{1.913993in}}%
\pgfusepath{stroke}%
\end{pgfscope}%
\begin{pgfscope}%
\pgfpathrectangle{\pgfqpoint{1.000000in}{0.450000in}}{\pgfqpoint{3.043478in}{2.310000in}}%
\pgfusepath{clip}%
\pgfsetbuttcap%
\pgfsetroundjoin%
\pgfsetlinewidth{2.007500pt}%
\definecolor{currentstroke}{rgb}{0.121569,0.466667,0.705882}%
\pgfsetstrokecolor{currentstroke}%
\pgfsetdash{}{0pt}%
\pgfpathmoveto{\pgfqpoint{1.855658in}{1.518219in}}%
\pgfpathlineto{\pgfqpoint{1.855658in}{2.160573in}}%
\pgfusepath{stroke}%
\end{pgfscope}%
\begin{pgfscope}%
\pgfpathrectangle{\pgfqpoint{1.000000in}{0.450000in}}{\pgfqpoint{3.043478in}{2.310000in}}%
\pgfusepath{clip}%
\pgfsetbuttcap%
\pgfsetroundjoin%
\pgfsetlinewidth{2.007500pt}%
\definecolor{currentstroke}{rgb}{0.121569,0.466667,0.705882}%
\pgfsetstrokecolor{currentstroke}%
\pgfsetdash{}{0pt}%
\pgfpathmoveto{\pgfqpoint{2.041974in}{1.351429in}}%
\pgfpathlineto{\pgfqpoint{2.041974in}{1.990382in}}%
\pgfusepath{stroke}%
\end{pgfscope}%
\begin{pgfscope}%
\pgfpathrectangle{\pgfqpoint{1.000000in}{0.450000in}}{\pgfqpoint{3.043478in}{2.310000in}}%
\pgfusepath{clip}%
\pgfsetbuttcap%
\pgfsetroundjoin%
\pgfsetlinewidth{2.007500pt}%
\definecolor{currentstroke}{rgb}{0.121569,0.466667,0.705882}%
\pgfsetstrokecolor{currentstroke}%
\pgfsetdash{}{0pt}%
\pgfpathmoveto{\pgfqpoint{2.507765in}{1.328052in}}%
\pgfpathlineto{\pgfqpoint{2.507765in}{1.966532in}}%
\pgfusepath{stroke}%
\end{pgfscope}%
\begin{pgfscope}%
\pgfpathrectangle{\pgfqpoint{1.000000in}{0.450000in}}{\pgfqpoint{3.043478in}{2.310000in}}%
\pgfusepath{clip}%
\pgfsetbuttcap%
\pgfsetroundjoin%
\pgfsetlinewidth{2.007500pt}%
\definecolor{currentstroke}{rgb}{0.121569,0.466667,0.705882}%
\pgfsetstrokecolor{currentstroke}%
\pgfsetdash{}{0pt}%
\pgfpathmoveto{\pgfqpoint{2.973556in}{1.566520in}}%
\pgfpathlineto{\pgfqpoint{2.973556in}{2.209818in}}%
\pgfusepath{stroke}%
\end{pgfscope}%
\begin{pgfscope}%
\pgfpathrectangle{\pgfqpoint{1.000000in}{0.450000in}}{\pgfqpoint{3.043478in}{2.310000in}}%
\pgfusepath{clip}%
\pgfsetbuttcap%
\pgfsetroundjoin%
\pgfsetlinewidth{2.007500pt}%
\definecolor{currentstroke}{rgb}{0.121569,0.466667,0.705882}%
\pgfsetstrokecolor{currentstroke}%
\pgfsetdash{}{0pt}%
\pgfpathmoveto{\pgfqpoint{3.439347in}{1.353699in}}%
\pgfpathlineto{\pgfqpoint{3.439347in}{1.992698in}}%
\pgfusepath{stroke}%
\end{pgfscope}%
\begin{pgfscope}%
\pgfpathrectangle{\pgfqpoint{1.000000in}{0.450000in}}{\pgfqpoint{3.043478in}{2.310000in}}%
\pgfusepath{clip}%
\pgfsetbuttcap%
\pgfsetroundjoin%
\pgfsetlinewidth{2.007500pt}%
\definecolor{currentstroke}{rgb}{0.121569,0.466667,0.705882}%
\pgfsetstrokecolor{currentstroke}%
\pgfsetdash{}{0pt}%
\pgfpathmoveto{\pgfqpoint{3.905138in}{1.565966in}}%
\pgfpathlineto{\pgfqpoint{3.905138in}{2.209253in}}%
\pgfusepath{stroke}%
\end{pgfscope}%
\begin{pgfscope}%
\pgfpathrectangle{\pgfqpoint{1.000000in}{0.450000in}}{\pgfqpoint{3.043478in}{2.310000in}}%
\pgfusepath{clip}%
\pgfsetbuttcap%
\pgfsetroundjoin%
\pgfsetlinewidth{2.007500pt}%
\definecolor{currentstroke}{rgb}{0.000000,0.750000,0.750000}%
\pgfsetstrokecolor{currentstroke}%
\pgfsetdash{}{0pt}%
\pgfpathmoveto{\pgfqpoint{1.152314in}{0.607780in}}%
\pgfpathlineto{\pgfqpoint{1.152314in}{1.232593in}}%
\pgfusepath{stroke}%
\end{pgfscope}%
\begin{pgfscope}%
\pgfpathrectangle{\pgfqpoint{1.000000in}{0.450000in}}{\pgfqpoint{3.043478in}{2.310000in}}%
\pgfusepath{clip}%
\pgfsetbuttcap%
\pgfsetroundjoin%
\pgfsetlinewidth{2.007500pt}%
\definecolor{currentstroke}{rgb}{0.000000,0.750000,0.750000}%
\pgfsetstrokecolor{currentstroke}%
\pgfsetdash{}{0pt}%
\pgfpathmoveto{\pgfqpoint{1.198893in}{0.697379in}}%
\pgfpathlineto{\pgfqpoint{1.198893in}{1.323846in}}%
\pgfusepath{stroke}%
\end{pgfscope}%
\begin{pgfscope}%
\pgfpathrectangle{\pgfqpoint{1.000000in}{0.450000in}}{\pgfqpoint{3.043478in}{2.310000in}}%
\pgfusepath{clip}%
\pgfsetbuttcap%
\pgfsetroundjoin%
\pgfsetlinewidth{2.007500pt}%
\definecolor{currentstroke}{rgb}{0.000000,0.750000,0.750000}%
\pgfsetstrokecolor{currentstroke}%
\pgfsetdash{}{0pt}%
\pgfpathmoveto{\pgfqpoint{1.245472in}{0.760656in}}%
\pgfpathlineto{\pgfqpoint{1.245472in}{1.388212in}}%
\pgfusepath{stroke}%
\end{pgfscope}%
\begin{pgfscope}%
\pgfpathrectangle{\pgfqpoint{1.000000in}{0.450000in}}{\pgfqpoint{3.043478in}{2.310000in}}%
\pgfusepath{clip}%
\pgfsetbuttcap%
\pgfsetroundjoin%
\pgfsetlinewidth{2.007500pt}%
\definecolor{currentstroke}{rgb}{0.000000,0.750000,0.750000}%
\pgfsetstrokecolor{currentstroke}%
\pgfsetdash{}{0pt}%
\pgfpathmoveto{\pgfqpoint{1.292051in}{0.788101in}}%
\pgfpathlineto{\pgfqpoint{1.292051in}{1.415927in}}%
\pgfusepath{stroke}%
\end{pgfscope}%
\begin{pgfscope}%
\pgfpathrectangle{\pgfqpoint{1.000000in}{0.450000in}}{\pgfqpoint{3.043478in}{2.310000in}}%
\pgfusepath{clip}%
\pgfsetbuttcap%
\pgfsetroundjoin%
\pgfsetlinewidth{2.007500pt}%
\definecolor{currentstroke}{rgb}{0.000000,0.750000,0.750000}%
\pgfsetstrokecolor{currentstroke}%
\pgfsetdash{}{0pt}%
\pgfpathmoveto{\pgfqpoint{1.338630in}{0.856394in}}%
\pgfpathlineto{\pgfqpoint{1.338630in}{1.485473in}}%
\pgfusepath{stroke}%
\end{pgfscope}%
\begin{pgfscope}%
\pgfpathrectangle{\pgfqpoint{1.000000in}{0.450000in}}{\pgfqpoint{3.043478in}{2.310000in}}%
\pgfusepath{clip}%
\pgfsetbuttcap%
\pgfsetroundjoin%
\pgfsetlinewidth{2.007500pt}%
\definecolor{currentstroke}{rgb}{0.000000,0.750000,0.750000}%
\pgfsetstrokecolor{currentstroke}%
\pgfsetdash{}{0pt}%
\pgfpathmoveto{\pgfqpoint{1.385209in}{0.981833in}}%
\pgfpathlineto{\pgfqpoint{1.385209in}{1.613447in}}%
\pgfusepath{stroke}%
\end{pgfscope}%
\begin{pgfscope}%
\pgfpathrectangle{\pgfqpoint{1.000000in}{0.450000in}}{\pgfqpoint{3.043478in}{2.310000in}}%
\pgfusepath{clip}%
\pgfsetbuttcap%
\pgfsetroundjoin%
\pgfsetlinewidth{2.007500pt}%
\definecolor{currentstroke}{rgb}{0.000000,0.750000,0.750000}%
\pgfsetstrokecolor{currentstroke}%
\pgfsetdash{}{0pt}%
\pgfpathmoveto{\pgfqpoint{1.431788in}{1.093680in}}%
\pgfpathlineto{\pgfqpoint{1.431788in}{1.727490in}}%
\pgfusepath{stroke}%
\end{pgfscope}%
\begin{pgfscope}%
\pgfpathrectangle{\pgfqpoint{1.000000in}{0.450000in}}{\pgfqpoint{3.043478in}{2.310000in}}%
\pgfusepath{clip}%
\pgfsetbuttcap%
\pgfsetroundjoin%
\pgfsetlinewidth{2.007500pt}%
\definecolor{currentstroke}{rgb}{0.000000,0.750000,0.750000}%
\pgfsetstrokecolor{currentstroke}%
\pgfsetdash{}{0pt}%
\pgfpathmoveto{\pgfqpoint{1.478367in}{1.181349in}}%
\pgfpathlineto{\pgfqpoint{1.478367in}{1.816898in}}%
\pgfusepath{stroke}%
\end{pgfscope}%
\begin{pgfscope}%
\pgfpathrectangle{\pgfqpoint{1.000000in}{0.450000in}}{\pgfqpoint{3.043478in}{2.310000in}}%
\pgfusepath{clip}%
\pgfsetbuttcap%
\pgfsetroundjoin%
\pgfsetlinewidth{2.007500pt}%
\definecolor{currentstroke}{rgb}{0.000000,0.750000,0.750000}%
\pgfsetstrokecolor{currentstroke}%
\pgfsetdash{}{0pt}%
\pgfpathmoveto{\pgfqpoint{1.664684in}{1.244878in}}%
\pgfpathlineto{\pgfqpoint{1.664684in}{1.881742in}}%
\pgfusepath{stroke}%
\end{pgfscope}%
\begin{pgfscope}%
\pgfpathrectangle{\pgfqpoint{1.000000in}{0.450000in}}{\pgfqpoint{3.043478in}{2.310000in}}%
\pgfusepath{clip}%
\pgfsetbuttcap%
\pgfsetroundjoin%
\pgfsetlinewidth{2.007500pt}%
\definecolor{currentstroke}{rgb}{0.000000,0.750000,0.750000}%
\pgfsetstrokecolor{currentstroke}%
\pgfsetdash{}{0pt}%
\pgfpathmoveto{\pgfqpoint{1.851000in}{1.542983in}}%
\pgfpathlineto{\pgfqpoint{1.851000in}{2.185838in}}%
\pgfusepath{stroke}%
\end{pgfscope}%
\begin{pgfscope}%
\pgfpathrectangle{\pgfqpoint{1.000000in}{0.450000in}}{\pgfqpoint{3.043478in}{2.310000in}}%
\pgfusepath{clip}%
\pgfsetbuttcap%
\pgfsetroundjoin%
\pgfsetlinewidth{2.007500pt}%
\definecolor{currentstroke}{rgb}{0.000000,0.750000,0.750000}%
\pgfsetstrokecolor{currentstroke}%
\pgfsetdash{}{0pt}%
\pgfpathmoveto{\pgfqpoint{2.037317in}{1.344262in}}%
\pgfpathlineto{\pgfqpoint{2.037317in}{1.983069in}}%
\pgfusepath{stroke}%
\end{pgfscope}%
\begin{pgfscope}%
\pgfpathrectangle{\pgfqpoint{1.000000in}{0.450000in}}{\pgfqpoint{3.043478in}{2.310000in}}%
\pgfusepath{clip}%
\pgfsetbuttcap%
\pgfsetroundjoin%
\pgfsetlinewidth{2.007500pt}%
\definecolor{currentstroke}{rgb}{0.000000,0.750000,0.750000}%
\pgfsetstrokecolor{currentstroke}%
\pgfsetdash{}{0pt}%
\pgfpathmoveto{\pgfqpoint{2.503107in}{1.332515in}}%
\pgfpathlineto{\pgfqpoint{2.503107in}{1.971086in}}%
\pgfusepath{stroke}%
\end{pgfscope}%
\begin{pgfscope}%
\pgfpathrectangle{\pgfqpoint{1.000000in}{0.450000in}}{\pgfqpoint{3.043478in}{2.310000in}}%
\pgfusepath{clip}%
\pgfsetbuttcap%
\pgfsetroundjoin%
\pgfsetlinewidth{2.007500pt}%
\definecolor{currentstroke}{rgb}{0.000000,0.750000,0.750000}%
\pgfsetstrokecolor{currentstroke}%
\pgfsetdash{}{0pt}%
\pgfpathmoveto{\pgfqpoint{2.968898in}{1.570067in}}%
\pgfpathlineto{\pgfqpoint{2.968898in}{2.213436in}}%
\pgfusepath{stroke}%
\end{pgfscope}%
\begin{pgfscope}%
\pgfpathrectangle{\pgfqpoint{1.000000in}{0.450000in}}{\pgfqpoint{3.043478in}{2.310000in}}%
\pgfusepath{clip}%
\pgfsetbuttcap%
\pgfsetroundjoin%
\pgfsetlinewidth{2.007500pt}%
\definecolor{currentstroke}{rgb}{0.000000,0.750000,0.750000}%
\pgfsetstrokecolor{currentstroke}%
\pgfsetdash{}{0pt}%
\pgfpathmoveto{\pgfqpoint{3.434689in}{1.360514in}}%
\pgfpathlineto{\pgfqpoint{3.434689in}{1.999650in}}%
\pgfusepath{stroke}%
\end{pgfscope}%
\begin{pgfscope}%
\pgfpathrectangle{\pgfqpoint{1.000000in}{0.450000in}}{\pgfqpoint{3.043478in}{2.310000in}}%
\pgfusepath{clip}%
\pgfsetbuttcap%
\pgfsetroundjoin%
\pgfsetlinewidth{2.007500pt}%
\definecolor{currentstroke}{rgb}{0.000000,0.750000,0.750000}%
\pgfsetstrokecolor{currentstroke}%
\pgfsetdash{}{0pt}%
\pgfpathmoveto{\pgfqpoint{3.900480in}{1.554530in}}%
\pgfpathlineto{\pgfqpoint{3.900480in}{2.197585in}}%
\pgfusepath{stroke}%
\end{pgfscope}%
\begin{pgfscope}%
\pgfpathrectangle{\pgfqpoint{1.000000in}{0.450000in}}{\pgfqpoint{3.043478in}{2.310000in}}%
\pgfusepath{clip}%
\pgfsetbuttcap%
\pgfsetroundjoin%
\pgfsetlinewidth{2.007500pt}%
\definecolor{currentstroke}{rgb}{1.000000,0.000000,0.000000}%
\pgfsetstrokecolor{currentstroke}%
\pgfsetdash{}{0pt}%
\pgfpathmoveto{\pgfqpoint{1.138340in}{0.573078in}}%
\pgfpathlineto{\pgfqpoint{1.138340in}{1.197189in}}%
\pgfusepath{stroke}%
\end{pgfscope}%
\begin{pgfscope}%
\pgfpathrectangle{\pgfqpoint{1.000000in}{0.450000in}}{\pgfqpoint{3.043478in}{2.310000in}}%
\pgfusepath{clip}%
\pgfsetbuttcap%
\pgfsetroundjoin%
\pgfsetlinewidth{2.007500pt}%
\definecolor{currentstroke}{rgb}{1.000000,0.000000,0.000000}%
\pgfsetstrokecolor{currentstroke}%
\pgfsetdash{}{0pt}%
\pgfpathmoveto{\pgfqpoint{1.184919in}{0.555000in}}%
\pgfpathlineto{\pgfqpoint{1.184919in}{1.178588in}}%
\pgfusepath{stroke}%
\end{pgfscope}%
\begin{pgfscope}%
\pgfpathrectangle{\pgfqpoint{1.000000in}{0.450000in}}{\pgfqpoint{3.043478in}{2.310000in}}%
\pgfusepath{clip}%
\pgfsetbuttcap%
\pgfsetroundjoin%
\pgfsetlinewidth{2.007500pt}%
\definecolor{currentstroke}{rgb}{1.000000,0.000000,0.000000}%
\pgfsetstrokecolor{currentstroke}%
\pgfsetdash{}{0pt}%
\pgfpathmoveto{\pgfqpoint{1.231498in}{0.620177in}}%
\pgfpathlineto{\pgfqpoint{1.231498in}{1.244893in}}%
\pgfusepath{stroke}%
\end{pgfscope}%
\begin{pgfscope}%
\pgfpathrectangle{\pgfqpoint{1.000000in}{0.450000in}}{\pgfqpoint{3.043478in}{2.310000in}}%
\pgfusepath{clip}%
\pgfsetbuttcap%
\pgfsetroundjoin%
\pgfsetlinewidth{2.007500pt}%
\definecolor{currentstroke}{rgb}{1.000000,0.000000,0.000000}%
\pgfsetstrokecolor{currentstroke}%
\pgfsetdash{}{0pt}%
\pgfpathmoveto{\pgfqpoint{1.278077in}{0.652130in}}%
\pgfpathlineto{\pgfqpoint{1.278077in}{1.277208in}}%
\pgfusepath{stroke}%
\end{pgfscope}%
\begin{pgfscope}%
\pgfpathrectangle{\pgfqpoint{1.000000in}{0.450000in}}{\pgfqpoint{3.043478in}{2.310000in}}%
\pgfusepath{clip}%
\pgfsetbuttcap%
\pgfsetroundjoin%
\pgfsetlinewidth{2.007500pt}%
\definecolor{currentstroke}{rgb}{1.000000,0.000000,0.000000}%
\pgfsetstrokecolor{currentstroke}%
\pgfsetdash{}{0pt}%
\pgfpathmoveto{\pgfqpoint{1.324656in}{0.723213in}}%
\pgfpathlineto{\pgfqpoint{1.324656in}{1.349601in}}%
\pgfusepath{stroke}%
\end{pgfscope}%
\begin{pgfscope}%
\pgfpathrectangle{\pgfqpoint{1.000000in}{0.450000in}}{\pgfqpoint{3.043478in}{2.310000in}}%
\pgfusepath{clip}%
\pgfsetbuttcap%
\pgfsetroundjoin%
\pgfsetlinewidth{2.007500pt}%
\definecolor{currentstroke}{rgb}{1.000000,0.000000,0.000000}%
\pgfsetstrokecolor{currentstroke}%
\pgfsetdash{}{0pt}%
\pgfpathmoveto{\pgfqpoint{1.371235in}{0.749878in}}%
\pgfpathlineto{\pgfqpoint{1.371235in}{1.376805in}}%
\pgfusepath{stroke}%
\end{pgfscope}%
\begin{pgfscope}%
\pgfpathrectangle{\pgfqpoint{1.000000in}{0.450000in}}{\pgfqpoint{3.043478in}{2.310000in}}%
\pgfusepath{clip}%
\pgfsetbuttcap%
\pgfsetroundjoin%
\pgfsetlinewidth{2.007500pt}%
\definecolor{currentstroke}{rgb}{1.000000,0.000000,0.000000}%
\pgfsetstrokecolor{currentstroke}%
\pgfsetdash{}{0pt}%
\pgfpathmoveto{\pgfqpoint{1.417815in}{0.913135in}}%
\pgfpathlineto{\pgfqpoint{1.417815in}{1.543296in}}%
\pgfusepath{stroke}%
\end{pgfscope}%
\begin{pgfscope}%
\pgfpathrectangle{\pgfqpoint{1.000000in}{0.450000in}}{\pgfqpoint{3.043478in}{2.310000in}}%
\pgfusepath{clip}%
\pgfsetbuttcap%
\pgfsetroundjoin%
\pgfsetlinewidth{2.007500pt}%
\definecolor{currentstroke}{rgb}{1.000000,0.000000,0.000000}%
\pgfsetstrokecolor{currentstroke}%
\pgfsetdash{}{0pt}%
\pgfpathmoveto{\pgfqpoint{1.464394in}{0.908400in}}%
\pgfpathlineto{\pgfqpoint{1.464394in}{1.538434in}}%
\pgfusepath{stroke}%
\end{pgfscope}%
\begin{pgfscope}%
\pgfpathrectangle{\pgfqpoint{1.000000in}{0.450000in}}{\pgfqpoint{3.043478in}{2.310000in}}%
\pgfusepath{clip}%
\pgfsetbuttcap%
\pgfsetroundjoin%
\pgfsetlinewidth{2.007500pt}%
\definecolor{currentstroke}{rgb}{1.000000,0.000000,0.000000}%
\pgfsetstrokecolor{currentstroke}%
\pgfsetdash{}{0pt}%
\pgfpathmoveto{\pgfqpoint{1.650710in}{1.017777in}}%
\pgfpathlineto{\pgfqpoint{1.650710in}{1.650053in}}%
\pgfusepath{stroke}%
\end{pgfscope}%
\begin{pgfscope}%
\pgfpathrectangle{\pgfqpoint{1.000000in}{0.450000in}}{\pgfqpoint{3.043478in}{2.310000in}}%
\pgfusepath{clip}%
\pgfsetbuttcap%
\pgfsetroundjoin%
\pgfsetlinewidth{2.007500pt}%
\definecolor{currentstroke}{rgb}{1.000000,0.000000,0.000000}%
\pgfsetstrokecolor{currentstroke}%
\pgfsetdash{}{0pt}%
\pgfpathmoveto{\pgfqpoint{1.837026in}{1.341930in}}%
\pgfpathlineto{\pgfqpoint{1.837026in}{1.980724in}}%
\pgfusepath{stroke}%
\end{pgfscope}%
\begin{pgfscope}%
\pgfpathrectangle{\pgfqpoint{1.000000in}{0.450000in}}{\pgfqpoint{3.043478in}{2.310000in}}%
\pgfusepath{clip}%
\pgfsetbuttcap%
\pgfsetroundjoin%
\pgfsetlinewidth{2.007500pt}%
\definecolor{currentstroke}{rgb}{1.000000,0.000000,0.000000}%
\pgfsetstrokecolor{currentstroke}%
\pgfsetdash{}{0pt}%
\pgfpathmoveto{\pgfqpoint{2.023343in}{1.304098in}}%
\pgfpathlineto{\pgfqpoint{2.023343in}{1.942094in}}%
\pgfusepath{stroke}%
\end{pgfscope}%
\begin{pgfscope}%
\pgfpathrectangle{\pgfqpoint{1.000000in}{0.450000in}}{\pgfqpoint{3.043478in}{2.310000in}}%
\pgfusepath{clip}%
\pgfsetbuttcap%
\pgfsetroundjoin%
\pgfsetlinewidth{2.007500pt}%
\definecolor{currentstroke}{rgb}{1.000000,0.000000,0.000000}%
\pgfsetstrokecolor{currentstroke}%
\pgfsetdash{}{0pt}%
\pgfpathmoveto{\pgfqpoint{2.489134in}{1.297245in}}%
\pgfpathlineto{\pgfqpoint{2.489134in}{1.935103in}}%
\pgfusepath{stroke}%
\end{pgfscope}%
\begin{pgfscope}%
\pgfpathrectangle{\pgfqpoint{1.000000in}{0.450000in}}{\pgfqpoint{3.043478in}{2.310000in}}%
\pgfusepath{clip}%
\pgfsetbuttcap%
\pgfsetroundjoin%
\pgfsetlinewidth{2.007500pt}%
\definecolor{currentstroke}{rgb}{1.000000,0.000000,0.000000}%
\pgfsetstrokecolor{currentstroke}%
\pgfsetdash{}{0pt}%
\pgfpathmoveto{\pgfqpoint{2.954925in}{1.591485in}}%
\pgfpathlineto{\pgfqpoint{2.954925in}{2.235288in}}%
\pgfusepath{stroke}%
\end{pgfscope}%
\begin{pgfscope}%
\pgfpathrectangle{\pgfqpoint{1.000000in}{0.450000in}}{\pgfqpoint{3.043478in}{2.310000in}}%
\pgfusepath{clip}%
\pgfsetbuttcap%
\pgfsetroundjoin%
\pgfsetlinewidth{2.007500pt}%
\definecolor{currentstroke}{rgb}{1.000000,0.000000,0.000000}%
\pgfsetstrokecolor{currentstroke}%
\pgfsetdash{}{0pt}%
\pgfpathmoveto{\pgfqpoint{3.420716in}{1.425436in}}%
\pgfpathlineto{\pgfqpoint{3.420716in}{2.065884in}}%
\pgfusepath{stroke}%
\end{pgfscope}%
\begin{pgfscope}%
\pgfpathrectangle{\pgfqpoint{1.000000in}{0.450000in}}{\pgfqpoint{3.043478in}{2.310000in}}%
\pgfusepath{clip}%
\pgfsetbuttcap%
\pgfsetroundjoin%
\pgfsetlinewidth{2.007500pt}%
\definecolor{currentstroke}{rgb}{1.000000,0.000000,0.000000}%
\pgfsetstrokecolor{currentstroke}%
\pgfsetdash{}{0pt}%
\pgfpathmoveto{\pgfqpoint{3.886507in}{1.645884in}}%
\pgfpathlineto{\pgfqpoint{3.886507in}{2.290785in}}%
\pgfusepath{stroke}%
\end{pgfscope}%
\begin{pgfscope}%
\pgfpathrectangle{\pgfqpoint{1.000000in}{0.450000in}}{\pgfqpoint{3.043478in}{2.310000in}}%
\pgfusepath{clip}%
\pgfsetrectcap%
\pgfsetroundjoin%
\pgfsetlinewidth{2.007500pt}%
\definecolor{currentstroke}{rgb}{0.750000,0.750000,0.000000}%
\pgfsetstrokecolor{currentstroke}%
\pgfsetdash{}{0pt}%
\pgfpathmoveto{\pgfqpoint{1.147656in}{1.007118in}}%
\pgfpathlineto{\pgfqpoint{1.194235in}{1.165143in}}%
\pgfpathlineto{\pgfqpoint{1.240814in}{1.350223in}}%
\pgfpathlineto{\pgfqpoint{1.287393in}{1.342050in}}%
\pgfpathlineto{\pgfqpoint{1.333972in}{1.465922in}}%
\pgfpathlineto{\pgfqpoint{1.380551in}{1.639299in}}%
\pgfpathlineto{\pgfqpoint{1.427130in}{1.801978in}}%
\pgfpathlineto{\pgfqpoint{1.473709in}{1.952987in}}%
\pgfpathlineto{\pgfqpoint{1.660026in}{2.091647in}}%
\pgfpathlineto{\pgfqpoint{1.846342in}{2.328927in}}%
\pgfpathlineto{\pgfqpoint{2.032659in}{2.148155in}}%
\pgfpathlineto{\pgfqpoint{2.498450in}{1.992536in}}%
\pgfpathlineto{\pgfqpoint{2.964241in}{2.120280in}}%
\pgfpathlineto{\pgfqpoint{3.430032in}{1.901889in}}%
\pgfpathlineto{\pgfqpoint{3.895823in}{2.025233in}}%
\pgfusepath{stroke}%
\end{pgfscope}%
\begin{pgfscope}%
\pgfpathrectangle{\pgfqpoint{1.000000in}{0.450000in}}{\pgfqpoint{3.043478in}{2.310000in}}%
\pgfusepath{clip}%
\pgfsetbuttcap%
\pgfsetmiterjoin%
\definecolor{currentfill}{rgb}{0.750000,0.750000,0.000000}%
\pgfsetfillcolor{currentfill}%
\pgfsetlinewidth{1.003750pt}%
\definecolor{currentstroke}{rgb}{0.750000,0.750000,0.000000}%
\pgfsetstrokecolor{currentstroke}%
\pgfsetdash{}{0pt}%
\pgfsys@defobject{currentmarker}{\pgfqpoint{-0.026418in}{-0.022473in}}{\pgfqpoint{0.026418in}{0.027778in}}{%
\pgfpathmoveto{\pgfqpoint{0.000000in}{0.027778in}}%
\pgfpathlineto{\pgfqpoint{-0.026418in}{0.008584in}}%
\pgfpathlineto{\pgfqpoint{-0.016327in}{-0.022473in}}%
\pgfpathlineto{\pgfqpoint{0.016327in}{-0.022473in}}%
\pgfpathlineto{\pgfqpoint{0.026418in}{0.008584in}}%
\pgfpathclose%
\pgfusepath{stroke,fill}%
}%
\begin{pgfscope}%
\pgfsys@transformshift{1.147656in}{1.007118in}%
\pgfsys@useobject{currentmarker}{}%
\end{pgfscope}%
\begin{pgfscope}%
\pgfsys@transformshift{1.194235in}{1.165143in}%
\pgfsys@useobject{currentmarker}{}%
\end{pgfscope}%
\begin{pgfscope}%
\pgfsys@transformshift{1.240814in}{1.350223in}%
\pgfsys@useobject{currentmarker}{}%
\end{pgfscope}%
\begin{pgfscope}%
\pgfsys@transformshift{1.287393in}{1.342050in}%
\pgfsys@useobject{currentmarker}{}%
\end{pgfscope}%
\begin{pgfscope}%
\pgfsys@transformshift{1.333972in}{1.465922in}%
\pgfsys@useobject{currentmarker}{}%
\end{pgfscope}%
\begin{pgfscope}%
\pgfsys@transformshift{1.380551in}{1.639299in}%
\pgfsys@useobject{currentmarker}{}%
\end{pgfscope}%
\begin{pgfscope}%
\pgfsys@transformshift{1.427130in}{1.801978in}%
\pgfsys@useobject{currentmarker}{}%
\end{pgfscope}%
\begin{pgfscope}%
\pgfsys@transformshift{1.473709in}{1.952987in}%
\pgfsys@useobject{currentmarker}{}%
\end{pgfscope}%
\begin{pgfscope}%
\pgfsys@transformshift{1.660026in}{2.091647in}%
\pgfsys@useobject{currentmarker}{}%
\end{pgfscope}%
\begin{pgfscope}%
\pgfsys@transformshift{1.846342in}{2.328927in}%
\pgfsys@useobject{currentmarker}{}%
\end{pgfscope}%
\begin{pgfscope}%
\pgfsys@transformshift{2.032659in}{2.148155in}%
\pgfsys@useobject{currentmarker}{}%
\end{pgfscope}%
\begin{pgfscope}%
\pgfsys@transformshift{2.498450in}{1.992536in}%
\pgfsys@useobject{currentmarker}{}%
\end{pgfscope}%
\begin{pgfscope}%
\pgfsys@transformshift{2.964241in}{2.120280in}%
\pgfsys@useobject{currentmarker}{}%
\end{pgfscope}%
\begin{pgfscope}%
\pgfsys@transformshift{3.430032in}{1.901889in}%
\pgfsys@useobject{currentmarker}{}%
\end{pgfscope}%
\begin{pgfscope}%
\pgfsys@transformshift{3.895823in}{2.025233in}%
\pgfsys@useobject{currentmarker}{}%
\end{pgfscope}%
\end{pgfscope}%
\begin{pgfscope}%
\pgfpathrectangle{\pgfqpoint{1.000000in}{0.450000in}}{\pgfqpoint{3.043478in}{2.310000in}}%
\pgfusepath{clip}%
\pgfsetrectcap%
\pgfsetroundjoin%
\pgfsetlinewidth{2.007500pt}%
\definecolor{currentstroke}{rgb}{0.121569,0.466667,0.705882}%
\pgfsetstrokecolor{currentstroke}%
\pgfsetdash{}{0pt}%
\pgfpathmoveto{\pgfqpoint{1.156972in}{0.955684in}}%
\pgfpathlineto{\pgfqpoint{1.203551in}{1.058018in}}%
\pgfpathlineto{\pgfqpoint{1.250130in}{1.052745in}}%
\pgfpathlineto{\pgfqpoint{1.296709in}{1.131350in}}%
\pgfpathlineto{\pgfqpoint{1.343288in}{1.156879in}}%
\pgfpathlineto{\pgfqpoint{1.389867in}{1.294227in}}%
\pgfpathlineto{\pgfqpoint{1.436446in}{1.412572in}}%
\pgfpathlineto{\pgfqpoint{1.483025in}{1.489902in}}%
\pgfpathlineto{\pgfqpoint{1.669342in}{1.595242in}}%
\pgfpathlineto{\pgfqpoint{1.855658in}{1.839396in}}%
\pgfpathlineto{\pgfqpoint{2.041974in}{1.670905in}}%
\pgfpathlineto{\pgfqpoint{2.507765in}{1.647292in}}%
\pgfpathlineto{\pgfqpoint{2.973556in}{1.888169in}}%
\pgfpathlineto{\pgfqpoint{3.439347in}{1.673199in}}%
\pgfpathlineto{\pgfqpoint{3.905138in}{1.887610in}}%
\pgfusepath{stroke}%
\end{pgfscope}%
\begin{pgfscope}%
\pgfpathrectangle{\pgfqpoint{1.000000in}{0.450000in}}{\pgfqpoint{3.043478in}{2.310000in}}%
\pgfusepath{clip}%
\pgfsetbuttcap%
\pgfsetroundjoin%
\definecolor{currentfill}{rgb}{0.121569,0.466667,0.705882}%
\pgfsetfillcolor{currentfill}%
\pgfsetlinewidth{1.003750pt}%
\definecolor{currentstroke}{rgb}{0.121569,0.466667,0.705882}%
\pgfsetstrokecolor{currentstroke}%
\pgfsetdash{}{0pt}%
\pgfsys@defobject{currentmarker}{\pgfqpoint{-0.027778in}{-0.027778in}}{\pgfqpoint{0.027778in}{0.027778in}}{%
\pgfpathmoveto{\pgfqpoint{-0.027778in}{-0.027778in}}%
\pgfpathlineto{\pgfqpoint{0.027778in}{0.027778in}}%
\pgfpathmoveto{\pgfqpoint{-0.027778in}{0.027778in}}%
\pgfpathlineto{\pgfqpoint{0.027778in}{-0.027778in}}%
\pgfusepath{stroke,fill}%
}%
\begin{pgfscope}%
\pgfsys@transformshift{1.156972in}{0.955684in}%
\pgfsys@useobject{currentmarker}{}%
\end{pgfscope}%
\begin{pgfscope}%
\pgfsys@transformshift{1.203551in}{1.058018in}%
\pgfsys@useobject{currentmarker}{}%
\end{pgfscope}%
\begin{pgfscope}%
\pgfsys@transformshift{1.250130in}{1.052745in}%
\pgfsys@useobject{currentmarker}{}%
\end{pgfscope}%
\begin{pgfscope}%
\pgfsys@transformshift{1.296709in}{1.131350in}%
\pgfsys@useobject{currentmarker}{}%
\end{pgfscope}%
\begin{pgfscope}%
\pgfsys@transformshift{1.343288in}{1.156879in}%
\pgfsys@useobject{currentmarker}{}%
\end{pgfscope}%
\begin{pgfscope}%
\pgfsys@transformshift{1.389867in}{1.294227in}%
\pgfsys@useobject{currentmarker}{}%
\end{pgfscope}%
\begin{pgfscope}%
\pgfsys@transformshift{1.436446in}{1.412572in}%
\pgfsys@useobject{currentmarker}{}%
\end{pgfscope}%
\begin{pgfscope}%
\pgfsys@transformshift{1.483025in}{1.489902in}%
\pgfsys@useobject{currentmarker}{}%
\end{pgfscope}%
\begin{pgfscope}%
\pgfsys@transformshift{1.669342in}{1.595242in}%
\pgfsys@useobject{currentmarker}{}%
\end{pgfscope}%
\begin{pgfscope}%
\pgfsys@transformshift{1.855658in}{1.839396in}%
\pgfsys@useobject{currentmarker}{}%
\end{pgfscope}%
\begin{pgfscope}%
\pgfsys@transformshift{2.041974in}{1.670905in}%
\pgfsys@useobject{currentmarker}{}%
\end{pgfscope}%
\begin{pgfscope}%
\pgfsys@transformshift{2.507765in}{1.647292in}%
\pgfsys@useobject{currentmarker}{}%
\end{pgfscope}%
\begin{pgfscope}%
\pgfsys@transformshift{2.973556in}{1.888169in}%
\pgfsys@useobject{currentmarker}{}%
\end{pgfscope}%
\begin{pgfscope}%
\pgfsys@transformshift{3.439347in}{1.673199in}%
\pgfsys@useobject{currentmarker}{}%
\end{pgfscope}%
\begin{pgfscope}%
\pgfsys@transformshift{3.905138in}{1.887610in}%
\pgfsys@useobject{currentmarker}{}%
\end{pgfscope}%
\end{pgfscope}%
\begin{pgfscope}%
\pgfpathrectangle{\pgfqpoint{1.000000in}{0.450000in}}{\pgfqpoint{3.043478in}{2.310000in}}%
\pgfusepath{clip}%
\pgfsetrectcap%
\pgfsetroundjoin%
\pgfsetlinewidth{2.007500pt}%
\definecolor{currentstroke}{rgb}{0.000000,0.750000,0.750000}%
\pgfsetstrokecolor{currentstroke}%
\pgfsetdash{}{0pt}%
\pgfpathmoveto{\pgfqpoint{1.152314in}{0.920186in}}%
\pgfpathlineto{\pgfqpoint{1.198893in}{1.010612in}}%
\pgfpathlineto{\pgfqpoint{1.245472in}{1.074434in}}%
\pgfpathlineto{\pgfqpoint{1.292051in}{1.102014in}}%
\pgfpathlineto{\pgfqpoint{1.338630in}{1.170934in}}%
\pgfpathlineto{\pgfqpoint{1.385209in}{1.297640in}}%
\pgfpathlineto{\pgfqpoint{1.431788in}{1.410585in}}%
\pgfpathlineto{\pgfqpoint{1.478367in}{1.499124in}}%
\pgfpathlineto{\pgfqpoint{1.664684in}{1.563310in}}%
\pgfpathlineto{\pgfqpoint{1.851000in}{1.864411in}}%
\pgfpathlineto{\pgfqpoint{2.037317in}{1.663666in}}%
\pgfpathlineto{\pgfqpoint{2.503107in}{1.651800in}}%
\pgfpathlineto{\pgfqpoint{2.968898in}{1.891751in}}%
\pgfpathlineto{\pgfqpoint{3.434689in}{1.680082in}}%
\pgfpathlineto{\pgfqpoint{3.900480in}{1.876057in}}%
\pgfusepath{stroke}%
\end{pgfscope}%
\begin{pgfscope}%
\pgfpathrectangle{\pgfqpoint{1.000000in}{0.450000in}}{\pgfqpoint{3.043478in}{2.310000in}}%
\pgfusepath{clip}%
\pgfsetbuttcap%
\pgfsetbeveljoin%
\definecolor{currentfill}{rgb}{0.000000,0.750000,0.750000}%
\pgfsetfillcolor{currentfill}%
\pgfsetlinewidth{1.003750pt}%
\definecolor{currentstroke}{rgb}{0.000000,0.750000,0.750000}%
\pgfsetstrokecolor{currentstroke}%
\pgfsetdash{}{0pt}%
\pgfsys@defobject{currentmarker}{\pgfqpoint{-0.026418in}{-0.022473in}}{\pgfqpoint{0.026418in}{0.027778in}}{%
\pgfpathmoveto{\pgfqpoint{0.000000in}{0.027778in}}%
\pgfpathlineto{\pgfqpoint{-0.006236in}{0.008584in}}%
\pgfpathlineto{\pgfqpoint{-0.026418in}{0.008584in}}%
\pgfpathlineto{\pgfqpoint{-0.010091in}{-0.003279in}}%
\pgfpathlineto{\pgfqpoint{-0.016327in}{-0.022473in}}%
\pgfpathlineto{\pgfqpoint{-0.000000in}{-0.010610in}}%
\pgfpathlineto{\pgfqpoint{0.016327in}{-0.022473in}}%
\pgfpathlineto{\pgfqpoint{0.010091in}{-0.003279in}}%
\pgfpathlineto{\pgfqpoint{0.026418in}{0.008584in}}%
\pgfpathlineto{\pgfqpoint{0.006236in}{0.008584in}}%
\pgfpathclose%
\pgfusepath{stroke,fill}%
}%
\begin{pgfscope}%
\pgfsys@transformshift{1.152314in}{0.920186in}%
\pgfsys@useobject{currentmarker}{}%
\end{pgfscope}%
\begin{pgfscope}%
\pgfsys@transformshift{1.198893in}{1.010612in}%
\pgfsys@useobject{currentmarker}{}%
\end{pgfscope}%
\begin{pgfscope}%
\pgfsys@transformshift{1.245472in}{1.074434in}%
\pgfsys@useobject{currentmarker}{}%
\end{pgfscope}%
\begin{pgfscope}%
\pgfsys@transformshift{1.292051in}{1.102014in}%
\pgfsys@useobject{currentmarker}{}%
\end{pgfscope}%
\begin{pgfscope}%
\pgfsys@transformshift{1.338630in}{1.170934in}%
\pgfsys@useobject{currentmarker}{}%
\end{pgfscope}%
\begin{pgfscope}%
\pgfsys@transformshift{1.385209in}{1.297640in}%
\pgfsys@useobject{currentmarker}{}%
\end{pgfscope}%
\begin{pgfscope}%
\pgfsys@transformshift{1.431788in}{1.410585in}%
\pgfsys@useobject{currentmarker}{}%
\end{pgfscope}%
\begin{pgfscope}%
\pgfsys@transformshift{1.478367in}{1.499124in}%
\pgfsys@useobject{currentmarker}{}%
\end{pgfscope}%
\begin{pgfscope}%
\pgfsys@transformshift{1.664684in}{1.563310in}%
\pgfsys@useobject{currentmarker}{}%
\end{pgfscope}%
\begin{pgfscope}%
\pgfsys@transformshift{1.851000in}{1.864411in}%
\pgfsys@useobject{currentmarker}{}%
\end{pgfscope}%
\begin{pgfscope}%
\pgfsys@transformshift{2.037317in}{1.663666in}%
\pgfsys@useobject{currentmarker}{}%
\end{pgfscope}%
\begin{pgfscope}%
\pgfsys@transformshift{2.503107in}{1.651800in}%
\pgfsys@useobject{currentmarker}{}%
\end{pgfscope}%
\begin{pgfscope}%
\pgfsys@transformshift{2.968898in}{1.891751in}%
\pgfsys@useobject{currentmarker}{}%
\end{pgfscope}%
\begin{pgfscope}%
\pgfsys@transformshift{3.434689in}{1.680082in}%
\pgfsys@useobject{currentmarker}{}%
\end{pgfscope}%
\begin{pgfscope}%
\pgfsys@transformshift{3.900480in}{1.876057in}%
\pgfsys@useobject{currentmarker}{}%
\end{pgfscope}%
\end{pgfscope}%
\begin{pgfscope}%
\pgfpathrectangle{\pgfqpoint{1.000000in}{0.450000in}}{\pgfqpoint{3.043478in}{2.310000in}}%
\pgfusepath{clip}%
\pgfsetrectcap%
\pgfsetroundjoin%
\pgfsetlinewidth{2.007500pt}%
\definecolor{currentstroke}{rgb}{1.000000,0.000000,0.000000}%
\pgfsetstrokecolor{currentstroke}%
\pgfsetdash{}{0pt}%
\pgfpathmoveto{\pgfqpoint{1.138340in}{0.885134in}}%
\pgfpathlineto{\pgfqpoint{1.184919in}{0.866794in}}%
\pgfpathlineto{\pgfqpoint{1.231498in}{0.932535in}}%
\pgfpathlineto{\pgfqpoint{1.278077in}{0.964669in}}%
\pgfpathlineto{\pgfqpoint{1.324656in}{1.036407in}}%
\pgfpathlineto{\pgfqpoint{1.371235in}{1.063342in}}%
\pgfpathlineto{\pgfqpoint{1.417815in}{1.228215in}}%
\pgfpathlineto{\pgfqpoint{1.464394in}{1.223417in}}%
\pgfpathlineto{\pgfqpoint{1.650710in}{1.333915in}}%
\pgfpathlineto{\pgfqpoint{1.837026in}{1.661327in}}%
\pgfpathlineto{\pgfqpoint{2.023343in}{1.623096in}}%
\pgfpathlineto{\pgfqpoint{2.489134in}{1.616174in}}%
\pgfpathlineto{\pgfqpoint{2.954925in}{1.913387in}}%
\pgfpathlineto{\pgfqpoint{3.420716in}{1.745660in}}%
\pgfpathlineto{\pgfqpoint{3.886507in}{1.968334in}}%
\pgfusepath{stroke}%
\end{pgfscope}%
\begin{pgfscope}%
\pgfpathrectangle{\pgfqpoint{1.000000in}{0.450000in}}{\pgfqpoint{3.043478in}{2.310000in}}%
\pgfusepath{clip}%
\pgfsetbuttcap%
\pgfsetmiterjoin%
\definecolor{currentfill}{rgb}{1.000000,0.000000,0.000000}%
\pgfsetfillcolor{currentfill}%
\pgfsetlinewidth{1.003750pt}%
\definecolor{currentstroke}{rgb}{1.000000,0.000000,0.000000}%
\pgfsetstrokecolor{currentstroke}%
\pgfsetdash{}{0pt}%
\pgfsys@defobject{currentmarker}{\pgfqpoint{-0.027778in}{-0.027778in}}{\pgfqpoint{0.027778in}{0.027778in}}{%
\pgfpathmoveto{\pgfqpoint{-0.027778in}{-0.027778in}}%
\pgfpathlineto{\pgfqpoint{0.027778in}{-0.027778in}}%
\pgfpathlineto{\pgfqpoint{0.027778in}{0.027778in}}%
\pgfpathlineto{\pgfqpoint{-0.027778in}{0.027778in}}%
\pgfpathclose%
\pgfusepath{stroke,fill}%
}%
\begin{pgfscope}%
\pgfsys@transformshift{1.138340in}{0.885134in}%
\pgfsys@useobject{currentmarker}{}%
\end{pgfscope}%
\begin{pgfscope}%
\pgfsys@transformshift{1.184919in}{0.866794in}%
\pgfsys@useobject{currentmarker}{}%
\end{pgfscope}%
\begin{pgfscope}%
\pgfsys@transformshift{1.231498in}{0.932535in}%
\pgfsys@useobject{currentmarker}{}%
\end{pgfscope}%
\begin{pgfscope}%
\pgfsys@transformshift{1.278077in}{0.964669in}%
\pgfsys@useobject{currentmarker}{}%
\end{pgfscope}%
\begin{pgfscope}%
\pgfsys@transformshift{1.324656in}{1.036407in}%
\pgfsys@useobject{currentmarker}{}%
\end{pgfscope}%
\begin{pgfscope}%
\pgfsys@transformshift{1.371235in}{1.063342in}%
\pgfsys@useobject{currentmarker}{}%
\end{pgfscope}%
\begin{pgfscope}%
\pgfsys@transformshift{1.417815in}{1.228215in}%
\pgfsys@useobject{currentmarker}{}%
\end{pgfscope}%
\begin{pgfscope}%
\pgfsys@transformshift{1.464394in}{1.223417in}%
\pgfsys@useobject{currentmarker}{}%
\end{pgfscope}%
\begin{pgfscope}%
\pgfsys@transformshift{1.650710in}{1.333915in}%
\pgfsys@useobject{currentmarker}{}%
\end{pgfscope}%
\begin{pgfscope}%
\pgfsys@transformshift{1.837026in}{1.661327in}%
\pgfsys@useobject{currentmarker}{}%
\end{pgfscope}%
\begin{pgfscope}%
\pgfsys@transformshift{2.023343in}{1.623096in}%
\pgfsys@useobject{currentmarker}{}%
\end{pgfscope}%
\begin{pgfscope}%
\pgfsys@transformshift{2.489134in}{1.616174in}%
\pgfsys@useobject{currentmarker}{}%
\end{pgfscope}%
\begin{pgfscope}%
\pgfsys@transformshift{2.954925in}{1.913387in}%
\pgfsys@useobject{currentmarker}{}%
\end{pgfscope}%
\begin{pgfscope}%
\pgfsys@transformshift{3.420716in}{1.745660in}%
\pgfsys@useobject{currentmarker}{}%
\end{pgfscope}%
\begin{pgfscope}%
\pgfsys@transformshift{3.886507in}{1.968334in}%
\pgfsys@useobject{currentmarker}{}%
\end{pgfscope}%
\end{pgfscope}%
\begin{pgfscope}%
\pgfsetrectcap%
\pgfsetmiterjoin%
\pgfsetlinewidth{0.803000pt}%
\definecolor{currentstroke}{rgb}{0.000000,0.000000,0.000000}%
\pgfsetstrokecolor{currentstroke}%
\pgfsetdash{}{0pt}%
\pgfpathmoveto{\pgfqpoint{1.000000in}{0.450000in}}%
\pgfpathlineto{\pgfqpoint{1.000000in}{2.760000in}}%
\pgfusepath{stroke}%
\end{pgfscope}%
\begin{pgfscope}%
\pgfsetrectcap%
\pgfsetmiterjoin%
\pgfsetlinewidth{0.803000pt}%
\definecolor{currentstroke}{rgb}{0.000000,0.000000,0.000000}%
\pgfsetstrokecolor{currentstroke}%
\pgfsetdash{}{0pt}%
\pgfpathmoveto{\pgfqpoint{4.043478in}{0.450000in}}%
\pgfpathlineto{\pgfqpoint{4.043478in}{2.760000in}}%
\pgfusepath{stroke}%
\end{pgfscope}%
\begin{pgfscope}%
\pgfsetrectcap%
\pgfsetmiterjoin%
\pgfsetlinewidth{0.803000pt}%
\definecolor{currentstroke}{rgb}{0.000000,0.000000,0.000000}%
\pgfsetstrokecolor{currentstroke}%
\pgfsetdash{}{0pt}%
\pgfpathmoveto{\pgfqpoint{1.000000in}{0.450000in}}%
\pgfpathlineto{\pgfqpoint{4.043478in}{0.450000in}}%
\pgfusepath{stroke}%
\end{pgfscope}%
\begin{pgfscope}%
\pgfsetrectcap%
\pgfsetmiterjoin%
\pgfsetlinewidth{0.803000pt}%
\definecolor{currentstroke}{rgb}{0.000000,0.000000,0.000000}%
\pgfsetstrokecolor{currentstroke}%
\pgfsetdash{}{0pt}%
\pgfpathmoveto{\pgfqpoint{1.000000in}{2.760000in}}%
\pgfpathlineto{\pgfqpoint{4.043478in}{2.760000in}}%
\pgfusepath{stroke}%
\end{pgfscope}%
\begin{pgfscope}%
\definecolor{textcolor}{rgb}{0.000000,0.000000,0.000000}%
\pgfsetstrokecolor{textcolor}%
\pgfsetfillcolor{textcolor}%
\pgftext[x=2.521739in,y=2.843333in,,base]{\color{textcolor}\sffamily\fontsize{14.400000}{17.280000}\selectfont \(\displaystyle \tau_l=0.0\si{ns},\,\sigma_l=5.0\si{ns}\)}%
\end{pgfscope}%
\begin{pgfscope}%
\pgfsetbuttcap%
\pgfsetmiterjoin%
\definecolor{currentfill}{rgb}{1.000000,1.000000,1.000000}%
\pgfsetfillcolor{currentfill}%
\pgfsetlinewidth{0.000000pt}%
\definecolor{currentstroke}{rgb}{0.000000,0.000000,0.000000}%
\pgfsetstrokecolor{currentstroke}%
\pgfsetstrokeopacity{0.000000}%
\pgfsetdash{}{0pt}%
\pgfpathmoveto{\pgfqpoint{4.956522in}{0.450000in}}%
\pgfpathlineto{\pgfqpoint{8.000000in}{0.450000in}}%
\pgfpathlineto{\pgfqpoint{8.000000in}{2.760000in}}%
\pgfpathlineto{\pgfqpoint{4.956522in}{2.760000in}}%
\pgfpathclose%
\pgfusepath{fill}%
\end{pgfscope}%
\begin{pgfscope}%
\pgfpathrectangle{\pgfqpoint{4.956522in}{0.450000in}}{\pgfqpoint{3.043478in}{2.310000in}}%
\pgfusepath{clip}%
\pgfsetrectcap%
\pgfsetroundjoin%
\pgfsetlinewidth{0.803000pt}%
\definecolor{currentstroke}{rgb}{0.690196,0.690196,0.690196}%
\pgfsetstrokecolor{currentstroke}%
\pgfsetdash{}{0pt}%
\pgfpathmoveto{\pgfqpoint{5.057598in}{0.450000in}}%
\pgfpathlineto{\pgfqpoint{5.057598in}{2.760000in}}%
\pgfusepath{stroke}%
\end{pgfscope}%
\begin{pgfscope}%
\pgfsetbuttcap%
\pgfsetroundjoin%
\definecolor{currentfill}{rgb}{0.000000,0.000000,0.000000}%
\pgfsetfillcolor{currentfill}%
\pgfsetlinewidth{0.803000pt}%
\definecolor{currentstroke}{rgb}{0.000000,0.000000,0.000000}%
\pgfsetstrokecolor{currentstroke}%
\pgfsetdash{}{0pt}%
\pgfsys@defobject{currentmarker}{\pgfqpoint{0.000000in}{-0.048611in}}{\pgfqpoint{0.000000in}{0.000000in}}{%
\pgfpathmoveto{\pgfqpoint{0.000000in}{0.000000in}}%
\pgfpathlineto{\pgfqpoint{0.000000in}{-0.048611in}}%
\pgfusepath{stroke,fill}%
}%
\begin{pgfscope}%
\pgfsys@transformshift{5.057598in}{0.450000in}%
\pgfsys@useobject{currentmarker}{}%
\end{pgfscope}%
\end{pgfscope}%
\begin{pgfscope}%
\definecolor{textcolor}{rgb}{0.000000,0.000000,0.000000}%
\pgfsetstrokecolor{textcolor}%
\pgfsetfillcolor{textcolor}%
\pgftext[x=5.057598in,y=0.352778in,,top]{\color{textcolor}\sffamily\fontsize{12.000000}{14.400000}\selectfont \(\displaystyle {0}\)}%
\end{pgfscope}%
\begin{pgfscope}%
\pgfpathrectangle{\pgfqpoint{4.956522in}{0.450000in}}{\pgfqpoint{3.043478in}{2.310000in}}%
\pgfusepath{clip}%
\pgfsetrectcap%
\pgfsetroundjoin%
\pgfsetlinewidth{0.803000pt}%
\definecolor{currentstroke}{rgb}{0.690196,0.690196,0.690196}%
\pgfsetstrokecolor{currentstroke}%
\pgfsetdash{}{0pt}%
\pgfpathmoveto{\pgfqpoint{5.989180in}{0.450000in}}%
\pgfpathlineto{\pgfqpoint{5.989180in}{2.760000in}}%
\pgfusepath{stroke}%
\end{pgfscope}%
\begin{pgfscope}%
\pgfsetbuttcap%
\pgfsetroundjoin%
\definecolor{currentfill}{rgb}{0.000000,0.000000,0.000000}%
\pgfsetfillcolor{currentfill}%
\pgfsetlinewidth{0.803000pt}%
\definecolor{currentstroke}{rgb}{0.000000,0.000000,0.000000}%
\pgfsetstrokecolor{currentstroke}%
\pgfsetdash{}{0pt}%
\pgfsys@defobject{currentmarker}{\pgfqpoint{0.000000in}{-0.048611in}}{\pgfqpoint{0.000000in}{0.000000in}}{%
\pgfpathmoveto{\pgfqpoint{0.000000in}{0.000000in}}%
\pgfpathlineto{\pgfqpoint{0.000000in}{-0.048611in}}%
\pgfusepath{stroke,fill}%
}%
\begin{pgfscope}%
\pgfsys@transformshift{5.989180in}{0.450000in}%
\pgfsys@useobject{currentmarker}{}%
\end{pgfscope}%
\end{pgfscope}%
\begin{pgfscope}%
\definecolor{textcolor}{rgb}{0.000000,0.000000,0.000000}%
\pgfsetstrokecolor{textcolor}%
\pgfsetfillcolor{textcolor}%
\pgftext[x=5.989180in,y=0.352778in,,top]{\color{textcolor}\sffamily\fontsize{12.000000}{14.400000}\selectfont \(\displaystyle {10}\)}%
\end{pgfscope}%
\begin{pgfscope}%
\pgfpathrectangle{\pgfqpoint{4.956522in}{0.450000in}}{\pgfqpoint{3.043478in}{2.310000in}}%
\pgfusepath{clip}%
\pgfsetrectcap%
\pgfsetroundjoin%
\pgfsetlinewidth{0.803000pt}%
\definecolor{currentstroke}{rgb}{0.690196,0.690196,0.690196}%
\pgfsetstrokecolor{currentstroke}%
\pgfsetdash{}{0pt}%
\pgfpathmoveto{\pgfqpoint{6.920762in}{0.450000in}}%
\pgfpathlineto{\pgfqpoint{6.920762in}{2.760000in}}%
\pgfusepath{stroke}%
\end{pgfscope}%
\begin{pgfscope}%
\pgfsetbuttcap%
\pgfsetroundjoin%
\definecolor{currentfill}{rgb}{0.000000,0.000000,0.000000}%
\pgfsetfillcolor{currentfill}%
\pgfsetlinewidth{0.803000pt}%
\definecolor{currentstroke}{rgb}{0.000000,0.000000,0.000000}%
\pgfsetstrokecolor{currentstroke}%
\pgfsetdash{}{0pt}%
\pgfsys@defobject{currentmarker}{\pgfqpoint{0.000000in}{-0.048611in}}{\pgfqpoint{0.000000in}{0.000000in}}{%
\pgfpathmoveto{\pgfqpoint{0.000000in}{0.000000in}}%
\pgfpathlineto{\pgfqpoint{0.000000in}{-0.048611in}}%
\pgfusepath{stroke,fill}%
}%
\begin{pgfscope}%
\pgfsys@transformshift{6.920762in}{0.450000in}%
\pgfsys@useobject{currentmarker}{}%
\end{pgfscope}%
\end{pgfscope}%
\begin{pgfscope}%
\definecolor{textcolor}{rgb}{0.000000,0.000000,0.000000}%
\pgfsetstrokecolor{textcolor}%
\pgfsetfillcolor{textcolor}%
\pgftext[x=6.920762in,y=0.352778in,,top]{\color{textcolor}\sffamily\fontsize{12.000000}{14.400000}\selectfont \(\displaystyle {20}\)}%
\end{pgfscope}%
\begin{pgfscope}%
\pgfpathrectangle{\pgfqpoint{4.956522in}{0.450000in}}{\pgfqpoint{3.043478in}{2.310000in}}%
\pgfusepath{clip}%
\pgfsetrectcap%
\pgfsetroundjoin%
\pgfsetlinewidth{0.803000pt}%
\definecolor{currentstroke}{rgb}{0.690196,0.690196,0.690196}%
\pgfsetstrokecolor{currentstroke}%
\pgfsetdash{}{0pt}%
\pgfpathmoveto{\pgfqpoint{7.852344in}{0.450000in}}%
\pgfpathlineto{\pgfqpoint{7.852344in}{2.760000in}}%
\pgfusepath{stroke}%
\end{pgfscope}%
\begin{pgfscope}%
\pgfsetbuttcap%
\pgfsetroundjoin%
\definecolor{currentfill}{rgb}{0.000000,0.000000,0.000000}%
\pgfsetfillcolor{currentfill}%
\pgfsetlinewidth{0.803000pt}%
\definecolor{currentstroke}{rgb}{0.000000,0.000000,0.000000}%
\pgfsetstrokecolor{currentstroke}%
\pgfsetdash{}{0pt}%
\pgfsys@defobject{currentmarker}{\pgfqpoint{0.000000in}{-0.048611in}}{\pgfqpoint{0.000000in}{0.000000in}}{%
\pgfpathmoveto{\pgfqpoint{0.000000in}{0.000000in}}%
\pgfpathlineto{\pgfqpoint{0.000000in}{-0.048611in}}%
\pgfusepath{stroke,fill}%
}%
\begin{pgfscope}%
\pgfsys@transformshift{7.852344in}{0.450000in}%
\pgfsys@useobject{currentmarker}{}%
\end{pgfscope}%
\end{pgfscope}%
\begin{pgfscope}%
\definecolor{textcolor}{rgb}{0.000000,0.000000,0.000000}%
\pgfsetstrokecolor{textcolor}%
\pgfsetfillcolor{textcolor}%
\pgftext[x=7.852344in,y=0.352778in,,top]{\color{textcolor}\sffamily\fontsize{12.000000}{14.400000}\selectfont \(\displaystyle {30}\)}%
\end{pgfscope}%
\begin{pgfscope}%
\definecolor{textcolor}{rgb}{0.000000,0.000000,0.000000}%
\pgfsetstrokecolor{textcolor}%
\pgfsetfillcolor{textcolor}%
\pgftext[x=6.478261in,y=0.149075in,,top]{\color{textcolor}\sffamily\fontsize{12.000000}{14.400000}\selectfont \(\displaystyle \mu\)}%
\end{pgfscope}%
\begin{pgfscope}%
\pgfpathrectangle{\pgfqpoint{4.956522in}{0.450000in}}{\pgfqpoint{3.043478in}{2.310000in}}%
\pgfusepath{clip}%
\pgfsetrectcap%
\pgfsetroundjoin%
\pgfsetlinewidth{0.803000pt}%
\definecolor{currentstroke}{rgb}{0.690196,0.690196,0.690196}%
\pgfsetstrokecolor{currentstroke}%
\pgfsetdash{}{0pt}%
\pgfpathmoveto{\pgfqpoint{4.956522in}{0.872475in}}%
\pgfpathlineto{\pgfqpoint{8.000000in}{0.872475in}}%
\pgfusepath{stroke}%
\end{pgfscope}%
\begin{pgfscope}%
\pgfsetbuttcap%
\pgfsetroundjoin%
\definecolor{currentfill}{rgb}{0.000000,0.000000,0.000000}%
\pgfsetfillcolor{currentfill}%
\pgfsetlinewidth{0.803000pt}%
\definecolor{currentstroke}{rgb}{0.000000,0.000000,0.000000}%
\pgfsetstrokecolor{currentstroke}%
\pgfsetdash{}{0pt}%
\pgfsys@defobject{currentmarker}{\pgfqpoint{-0.048611in}{0.000000in}}{\pgfqpoint{-0.000000in}{0.000000in}}{%
\pgfpathmoveto{\pgfqpoint{-0.000000in}{0.000000in}}%
\pgfpathlineto{\pgfqpoint{-0.048611in}{0.000000in}}%
\pgfusepath{stroke,fill}%
}%
\begin{pgfscope}%
\pgfsys@transformshift{4.956522in}{0.872475in}%
\pgfsys@useobject{currentmarker}{}%
\end{pgfscope}%
\end{pgfscope}%
\begin{pgfscope}%
\definecolor{textcolor}{rgb}{0.000000,0.000000,0.000000}%
\pgfsetstrokecolor{textcolor}%
\pgfsetfillcolor{textcolor}%
\pgftext[x=4.569179in, y=0.814605in, left, base]{\color{textcolor}\sffamily\fontsize{12.000000}{14.400000}\selectfont \(\displaystyle {1.00}\)}%
\end{pgfscope}%
\begin{pgfscope}%
\pgfpathrectangle{\pgfqpoint{4.956522in}{0.450000in}}{\pgfqpoint{3.043478in}{2.310000in}}%
\pgfusepath{clip}%
\pgfsetrectcap%
\pgfsetroundjoin%
\pgfsetlinewidth{0.803000pt}%
\definecolor{currentstroke}{rgb}{0.690196,0.690196,0.690196}%
\pgfsetstrokecolor{currentstroke}%
\pgfsetdash{}{0pt}%
\pgfpathmoveto{\pgfqpoint{4.956522in}{1.525984in}}%
\pgfpathlineto{\pgfqpoint{8.000000in}{1.525984in}}%
\pgfusepath{stroke}%
\end{pgfscope}%
\begin{pgfscope}%
\pgfsetbuttcap%
\pgfsetroundjoin%
\definecolor{currentfill}{rgb}{0.000000,0.000000,0.000000}%
\pgfsetfillcolor{currentfill}%
\pgfsetlinewidth{0.803000pt}%
\definecolor{currentstroke}{rgb}{0.000000,0.000000,0.000000}%
\pgfsetstrokecolor{currentstroke}%
\pgfsetdash{}{0pt}%
\pgfsys@defobject{currentmarker}{\pgfqpoint{-0.048611in}{0.000000in}}{\pgfqpoint{-0.000000in}{0.000000in}}{%
\pgfpathmoveto{\pgfqpoint{-0.000000in}{0.000000in}}%
\pgfpathlineto{\pgfqpoint{-0.048611in}{0.000000in}}%
\pgfusepath{stroke,fill}%
}%
\begin{pgfscope}%
\pgfsys@transformshift{4.956522in}{1.525984in}%
\pgfsys@useobject{currentmarker}{}%
\end{pgfscope}%
\end{pgfscope}%
\begin{pgfscope}%
\definecolor{textcolor}{rgb}{0.000000,0.000000,0.000000}%
\pgfsetstrokecolor{textcolor}%
\pgfsetfillcolor{textcolor}%
\pgftext[x=4.569179in, y=1.468114in, left, base]{\color{textcolor}\sffamily\fontsize{12.000000}{14.400000}\selectfont \(\displaystyle {1.02}\)}%
\end{pgfscope}%
\begin{pgfscope}%
\pgfpathrectangle{\pgfqpoint{4.956522in}{0.450000in}}{\pgfqpoint{3.043478in}{2.310000in}}%
\pgfusepath{clip}%
\pgfsetrectcap%
\pgfsetroundjoin%
\pgfsetlinewidth{0.803000pt}%
\definecolor{currentstroke}{rgb}{0.690196,0.690196,0.690196}%
\pgfsetstrokecolor{currentstroke}%
\pgfsetdash{}{0pt}%
\pgfpathmoveto{\pgfqpoint{4.956522in}{2.179493in}}%
\pgfpathlineto{\pgfqpoint{8.000000in}{2.179493in}}%
\pgfusepath{stroke}%
\end{pgfscope}%
\begin{pgfscope}%
\pgfsetbuttcap%
\pgfsetroundjoin%
\definecolor{currentfill}{rgb}{0.000000,0.000000,0.000000}%
\pgfsetfillcolor{currentfill}%
\pgfsetlinewidth{0.803000pt}%
\definecolor{currentstroke}{rgb}{0.000000,0.000000,0.000000}%
\pgfsetstrokecolor{currentstroke}%
\pgfsetdash{}{0pt}%
\pgfsys@defobject{currentmarker}{\pgfqpoint{-0.048611in}{0.000000in}}{\pgfqpoint{-0.000000in}{0.000000in}}{%
\pgfpathmoveto{\pgfqpoint{-0.000000in}{0.000000in}}%
\pgfpathlineto{\pgfqpoint{-0.048611in}{0.000000in}}%
\pgfusepath{stroke,fill}%
}%
\begin{pgfscope}%
\pgfsys@transformshift{4.956522in}{2.179493in}%
\pgfsys@useobject{currentmarker}{}%
\end{pgfscope}%
\end{pgfscope}%
\begin{pgfscope}%
\definecolor{textcolor}{rgb}{0.000000,0.000000,0.000000}%
\pgfsetstrokecolor{textcolor}%
\pgfsetfillcolor{textcolor}%
\pgftext[x=4.569179in, y=2.121622in, left, base]{\color{textcolor}\sffamily\fontsize{12.000000}{14.400000}\selectfont \(\displaystyle {1.04}\)}%
\end{pgfscope}%
\begin{pgfscope}%
\definecolor{textcolor}{rgb}{0.000000,0.000000,0.000000}%
\pgfsetstrokecolor{textcolor}%
\pgfsetfillcolor{textcolor}%
\pgftext[x=4.513624in,y=1.605000in,,bottom,rotate=90.000000]{\color{textcolor}\sffamily\fontsize{12.000000}{14.400000}\selectfont \(\displaystyle \mathrm{ratio}\)}%
\end{pgfscope}%
\begin{pgfscope}%
\pgfpathrectangle{\pgfqpoint{4.956522in}{0.450000in}}{\pgfqpoint{3.043478in}{2.310000in}}%
\pgfusepath{clip}%
\pgfsetbuttcap%
\pgfsetroundjoin%
\pgfsetlinewidth{2.007500pt}%
\definecolor{currentstroke}{rgb}{0.750000,0.750000,0.000000}%
\pgfsetstrokecolor{currentstroke}%
\pgfsetdash{}{0pt}%
\pgfpathmoveto{\pgfqpoint{5.104177in}{0.598124in}}%
\pgfpathlineto{\pgfqpoint{5.104177in}{1.253586in}}%
\pgfusepath{stroke}%
\end{pgfscope}%
\begin{pgfscope}%
\pgfpathrectangle{\pgfqpoint{4.956522in}{0.450000in}}{\pgfqpoint{3.043478in}{2.310000in}}%
\pgfusepath{clip}%
\pgfsetbuttcap%
\pgfsetroundjoin%
\pgfsetlinewidth{2.007500pt}%
\definecolor{currentstroke}{rgb}{0.750000,0.750000,0.000000}%
\pgfsetstrokecolor{currentstroke}%
\pgfsetdash{}{0pt}%
\pgfpathmoveto{\pgfqpoint{5.150757in}{0.627222in}}%
\pgfpathlineto{\pgfqpoint{5.150757in}{1.282874in}}%
\pgfusepath{stroke}%
\end{pgfscope}%
\begin{pgfscope}%
\pgfpathrectangle{\pgfqpoint{4.956522in}{0.450000in}}{\pgfqpoint{3.043478in}{2.310000in}}%
\pgfusepath{clip}%
\pgfsetbuttcap%
\pgfsetroundjoin%
\pgfsetlinewidth{2.007500pt}%
\definecolor{currentstroke}{rgb}{0.750000,0.750000,0.000000}%
\pgfsetstrokecolor{currentstroke}%
\pgfsetdash{}{0pt}%
\pgfpathmoveto{\pgfqpoint{5.197336in}{0.616503in}}%
\pgfpathlineto{\pgfqpoint{5.197336in}{1.272104in}}%
\pgfusepath{stroke}%
\end{pgfscope}%
\begin{pgfscope}%
\pgfpathrectangle{\pgfqpoint{4.956522in}{0.450000in}}{\pgfqpoint{3.043478in}{2.310000in}}%
\pgfusepath{clip}%
\pgfsetbuttcap%
\pgfsetroundjoin%
\pgfsetlinewidth{2.007500pt}%
\definecolor{currentstroke}{rgb}{0.750000,0.750000,0.000000}%
\pgfsetstrokecolor{currentstroke}%
\pgfsetdash{}{0pt}%
\pgfpathmoveto{\pgfqpoint{5.243915in}{0.797570in}}%
\pgfpathlineto{\pgfqpoint{5.243915in}{1.456732in}}%
\pgfusepath{stroke}%
\end{pgfscope}%
\begin{pgfscope}%
\pgfpathrectangle{\pgfqpoint{4.956522in}{0.450000in}}{\pgfqpoint{3.043478in}{2.310000in}}%
\pgfusepath{clip}%
\pgfsetbuttcap%
\pgfsetroundjoin%
\pgfsetlinewidth{2.007500pt}%
\definecolor{currentstroke}{rgb}{0.750000,0.750000,0.000000}%
\pgfsetstrokecolor{currentstroke}%
\pgfsetdash{}{0pt}%
\pgfpathmoveto{\pgfqpoint{5.290494in}{0.710098in}}%
\pgfpathlineto{\pgfqpoint{5.290494in}{1.367858in}}%
\pgfusepath{stroke}%
\end{pgfscope}%
\begin{pgfscope}%
\pgfpathrectangle{\pgfqpoint{4.956522in}{0.450000in}}{\pgfqpoint{3.043478in}{2.310000in}}%
\pgfusepath{clip}%
\pgfsetbuttcap%
\pgfsetroundjoin%
\pgfsetlinewidth{2.007500pt}%
\definecolor{currentstroke}{rgb}{0.750000,0.750000,0.000000}%
\pgfsetstrokecolor{currentstroke}%
\pgfsetdash{}{0pt}%
\pgfpathmoveto{\pgfqpoint{5.337073in}{0.786869in}}%
\pgfpathlineto{\pgfqpoint{5.337073in}{1.445715in}}%
\pgfusepath{stroke}%
\end{pgfscope}%
\begin{pgfscope}%
\pgfpathrectangle{\pgfqpoint{4.956522in}{0.450000in}}{\pgfqpoint{3.043478in}{2.310000in}}%
\pgfusepath{clip}%
\pgfsetbuttcap%
\pgfsetroundjoin%
\pgfsetlinewidth{2.007500pt}%
\definecolor{currentstroke}{rgb}{0.750000,0.750000,0.000000}%
\pgfsetstrokecolor{currentstroke}%
\pgfsetdash{}{0pt}%
\pgfpathmoveto{\pgfqpoint{5.383652in}{1.238740in}}%
\pgfpathlineto{\pgfqpoint{5.383652in}{1.906958in}}%
\pgfusepath{stroke}%
\end{pgfscope}%
\begin{pgfscope}%
\pgfpathrectangle{\pgfqpoint{4.956522in}{0.450000in}}{\pgfqpoint{3.043478in}{2.310000in}}%
\pgfusepath{clip}%
\pgfsetbuttcap%
\pgfsetroundjoin%
\pgfsetlinewidth{2.007500pt}%
\definecolor{currentstroke}{rgb}{0.750000,0.750000,0.000000}%
\pgfsetstrokecolor{currentstroke}%
\pgfsetdash{}{0pt}%
\pgfpathmoveto{\pgfqpoint{5.430231in}{1.029904in}}%
\pgfpathlineto{\pgfqpoint{5.430231in}{1.693832in}}%
\pgfusepath{stroke}%
\end{pgfscope}%
\begin{pgfscope}%
\pgfpathrectangle{\pgfqpoint{4.956522in}{0.450000in}}{\pgfqpoint{3.043478in}{2.310000in}}%
\pgfusepath{clip}%
\pgfsetbuttcap%
\pgfsetroundjoin%
\pgfsetlinewidth{2.007500pt}%
\definecolor{currentstroke}{rgb}{0.750000,0.750000,0.000000}%
\pgfsetstrokecolor{currentstroke}%
\pgfsetdash{}{0pt}%
\pgfpathmoveto{\pgfqpoint{5.616548in}{1.632711in}}%
\pgfpathlineto{\pgfqpoint{5.616548in}{2.308486in}}%
\pgfusepath{stroke}%
\end{pgfscope}%
\begin{pgfscope}%
\pgfpathrectangle{\pgfqpoint{4.956522in}{0.450000in}}{\pgfqpoint{3.043478in}{2.310000in}}%
\pgfusepath{clip}%
\pgfsetbuttcap%
\pgfsetroundjoin%
\pgfsetlinewidth{2.007500pt}%
\definecolor{currentstroke}{rgb}{0.750000,0.750000,0.000000}%
\pgfsetstrokecolor{currentstroke}%
\pgfsetdash{}{0pt}%
\pgfpathmoveto{\pgfqpoint{5.802864in}{1.729969in}}%
\pgfpathlineto{\pgfqpoint{5.802864in}{2.407573in}}%
\pgfusepath{stroke}%
\end{pgfscope}%
\begin{pgfscope}%
\pgfpathrectangle{\pgfqpoint{4.956522in}{0.450000in}}{\pgfqpoint{3.043478in}{2.310000in}}%
\pgfusepath{clip}%
\pgfsetbuttcap%
\pgfsetroundjoin%
\pgfsetlinewidth{2.007500pt}%
\definecolor{currentstroke}{rgb}{0.750000,0.750000,0.000000}%
\pgfsetstrokecolor{currentstroke}%
\pgfsetdash{}{0pt}%
\pgfpathmoveto{\pgfqpoint{5.989180in}{1.972563in}}%
\pgfpathlineto{\pgfqpoint{5.989180in}{2.655000in}}%
\pgfusepath{stroke}%
\end{pgfscope}%
\begin{pgfscope}%
\pgfpathrectangle{\pgfqpoint{4.956522in}{0.450000in}}{\pgfqpoint{3.043478in}{2.310000in}}%
\pgfusepath{clip}%
\pgfsetbuttcap%
\pgfsetroundjoin%
\pgfsetlinewidth{2.007500pt}%
\definecolor{currentstroke}{rgb}{0.750000,0.750000,0.000000}%
\pgfsetstrokecolor{currentstroke}%
\pgfsetdash{}{0pt}%
\pgfpathmoveto{\pgfqpoint{6.454971in}{1.837748in}}%
\pgfpathlineto{\pgfqpoint{6.454971in}{2.517393in}}%
\pgfusepath{stroke}%
\end{pgfscope}%
\begin{pgfscope}%
\pgfpathrectangle{\pgfqpoint{4.956522in}{0.450000in}}{\pgfqpoint{3.043478in}{2.310000in}}%
\pgfusepath{clip}%
\pgfsetbuttcap%
\pgfsetroundjoin%
\pgfsetlinewidth{2.007500pt}%
\definecolor{currentstroke}{rgb}{0.750000,0.750000,0.000000}%
\pgfsetstrokecolor{currentstroke}%
\pgfsetdash{}{0pt}%
\pgfpathmoveto{\pgfqpoint{6.920762in}{1.714483in}}%
\pgfpathlineto{\pgfqpoint{6.920762in}{2.391603in}}%
\pgfusepath{stroke}%
\end{pgfscope}%
\begin{pgfscope}%
\pgfpathrectangle{\pgfqpoint{4.956522in}{0.450000in}}{\pgfqpoint{3.043478in}{2.310000in}}%
\pgfusepath{clip}%
\pgfsetbuttcap%
\pgfsetroundjoin%
\pgfsetlinewidth{2.007500pt}%
\definecolor{currentstroke}{rgb}{0.750000,0.750000,0.000000}%
\pgfsetstrokecolor{currentstroke}%
\pgfsetdash{}{0pt}%
\pgfpathmoveto{\pgfqpoint{7.386553in}{1.755786in}}%
\pgfpathlineto{\pgfqpoint{7.386553in}{2.433741in}}%
\pgfusepath{stroke}%
\end{pgfscope}%
\begin{pgfscope}%
\pgfpathrectangle{\pgfqpoint{4.956522in}{0.450000in}}{\pgfqpoint{3.043478in}{2.310000in}}%
\pgfusepath{clip}%
\pgfsetbuttcap%
\pgfsetroundjoin%
\pgfsetlinewidth{2.007500pt}%
\definecolor{currentstroke}{rgb}{0.750000,0.750000,0.000000}%
\pgfsetstrokecolor{currentstroke}%
\pgfsetdash{}{0pt}%
\pgfpathmoveto{\pgfqpoint{7.852344in}{1.618662in}}%
\pgfpathlineto{\pgfqpoint{7.852344in}{2.293846in}}%
\pgfusepath{stroke}%
\end{pgfscope}%
\begin{pgfscope}%
\pgfpathrectangle{\pgfqpoint{4.956522in}{0.450000in}}{\pgfqpoint{3.043478in}{2.310000in}}%
\pgfusepath{clip}%
\pgfsetbuttcap%
\pgfsetroundjoin%
\pgfsetlinewidth{2.007500pt}%
\definecolor{currentstroke}{rgb}{0.121569,0.466667,0.705882}%
\pgfsetstrokecolor{currentstroke}%
\pgfsetdash{}{0pt}%
\pgfpathmoveto{\pgfqpoint{5.113493in}{0.555000in}}%
\pgfpathlineto{\pgfqpoint{5.113493in}{1.209589in}}%
\pgfusepath{stroke}%
\end{pgfscope}%
\begin{pgfscope}%
\pgfpathrectangle{\pgfqpoint{4.956522in}{0.450000in}}{\pgfqpoint{3.043478in}{2.310000in}}%
\pgfusepath{clip}%
\pgfsetbuttcap%
\pgfsetroundjoin%
\pgfsetlinewidth{2.007500pt}%
\definecolor{currentstroke}{rgb}{0.121569,0.466667,0.705882}%
\pgfsetstrokecolor{currentstroke}%
\pgfsetdash{}{0pt}%
\pgfpathmoveto{\pgfqpoint{5.160072in}{0.568689in}}%
\pgfpathlineto{\pgfqpoint{5.160072in}{1.223158in}}%
\pgfusepath{stroke}%
\end{pgfscope}%
\begin{pgfscope}%
\pgfpathrectangle{\pgfqpoint{4.956522in}{0.450000in}}{\pgfqpoint{3.043478in}{2.310000in}}%
\pgfusepath{clip}%
\pgfsetbuttcap%
\pgfsetroundjoin%
\pgfsetlinewidth{2.007500pt}%
\definecolor{currentstroke}{rgb}{0.121569,0.466667,0.705882}%
\pgfsetstrokecolor{currentstroke}%
\pgfsetdash{}{0pt}%
\pgfpathmoveto{\pgfqpoint{5.206651in}{0.566892in}}%
\pgfpathlineto{\pgfqpoint{5.206651in}{1.221490in}}%
\pgfusepath{stroke}%
\end{pgfscope}%
\begin{pgfscope}%
\pgfpathrectangle{\pgfqpoint{4.956522in}{0.450000in}}{\pgfqpoint{3.043478in}{2.310000in}}%
\pgfusepath{clip}%
\pgfsetbuttcap%
\pgfsetroundjoin%
\pgfsetlinewidth{2.007500pt}%
\definecolor{currentstroke}{rgb}{0.121569,0.466667,0.705882}%
\pgfsetstrokecolor{currentstroke}%
\pgfsetdash{}{0pt}%
\pgfpathmoveto{\pgfqpoint{5.253231in}{0.561282in}}%
\pgfpathlineto{\pgfqpoint{5.253231in}{1.215667in}}%
\pgfusepath{stroke}%
\end{pgfscope}%
\begin{pgfscope}%
\pgfpathrectangle{\pgfqpoint{4.956522in}{0.450000in}}{\pgfqpoint{3.043478in}{2.310000in}}%
\pgfusepath{clip}%
\pgfsetbuttcap%
\pgfsetroundjoin%
\pgfsetlinewidth{2.007500pt}%
\definecolor{currentstroke}{rgb}{0.121569,0.466667,0.705882}%
\pgfsetstrokecolor{currentstroke}%
\pgfsetdash{}{0pt}%
\pgfpathmoveto{\pgfqpoint{5.299810in}{0.594506in}}%
\pgfpathlineto{\pgfqpoint{5.299810in}{1.249927in}}%
\pgfusepath{stroke}%
\end{pgfscope}%
\begin{pgfscope}%
\pgfpathrectangle{\pgfqpoint{4.956522in}{0.450000in}}{\pgfqpoint{3.043478in}{2.310000in}}%
\pgfusepath{clip}%
\pgfsetbuttcap%
\pgfsetroundjoin%
\pgfsetlinewidth{2.007500pt}%
\definecolor{currentstroke}{rgb}{0.121569,0.466667,0.705882}%
\pgfsetstrokecolor{currentstroke}%
\pgfsetdash{}{0pt}%
\pgfpathmoveto{\pgfqpoint{5.346389in}{0.613960in}}%
\pgfpathlineto{\pgfqpoint{5.346389in}{1.269311in}}%
\pgfusepath{stroke}%
\end{pgfscope}%
\begin{pgfscope}%
\pgfpathrectangle{\pgfqpoint{4.956522in}{0.450000in}}{\pgfqpoint{3.043478in}{2.310000in}}%
\pgfusepath{clip}%
\pgfsetbuttcap%
\pgfsetroundjoin%
\pgfsetlinewidth{2.007500pt}%
\definecolor{currentstroke}{rgb}{0.121569,0.466667,0.705882}%
\pgfsetstrokecolor{currentstroke}%
\pgfsetdash{}{0pt}%
\pgfpathmoveto{\pgfqpoint{5.392968in}{0.714079in}}%
\pgfpathlineto{\pgfqpoint{5.392968in}{1.371687in}}%
\pgfusepath{stroke}%
\end{pgfscope}%
\begin{pgfscope}%
\pgfpathrectangle{\pgfqpoint{4.956522in}{0.450000in}}{\pgfqpoint{3.043478in}{2.310000in}}%
\pgfusepath{clip}%
\pgfsetbuttcap%
\pgfsetroundjoin%
\pgfsetlinewidth{2.007500pt}%
\definecolor{currentstroke}{rgb}{0.121569,0.466667,0.705882}%
\pgfsetstrokecolor{currentstroke}%
\pgfsetdash{}{0pt}%
\pgfpathmoveto{\pgfqpoint{5.439547in}{0.747431in}}%
\pgfpathlineto{\pgfqpoint{5.439547in}{1.405646in}}%
\pgfusepath{stroke}%
\end{pgfscope}%
\begin{pgfscope}%
\pgfpathrectangle{\pgfqpoint{4.956522in}{0.450000in}}{\pgfqpoint{3.043478in}{2.310000in}}%
\pgfusepath{clip}%
\pgfsetbuttcap%
\pgfsetroundjoin%
\pgfsetlinewidth{2.007500pt}%
\definecolor{currentstroke}{rgb}{0.121569,0.466667,0.705882}%
\pgfsetstrokecolor{currentstroke}%
\pgfsetdash{}{0pt}%
\pgfpathmoveto{\pgfqpoint{5.625863in}{1.130011in}}%
\pgfpathlineto{\pgfqpoint{5.625863in}{1.795626in}}%
\pgfusepath{stroke}%
\end{pgfscope}%
\begin{pgfscope}%
\pgfpathrectangle{\pgfqpoint{4.956522in}{0.450000in}}{\pgfqpoint{3.043478in}{2.310000in}}%
\pgfusepath{clip}%
\pgfsetbuttcap%
\pgfsetroundjoin%
\pgfsetlinewidth{2.007500pt}%
\definecolor{currentstroke}{rgb}{0.121569,0.466667,0.705882}%
\pgfsetstrokecolor{currentstroke}%
\pgfsetdash{}{0pt}%
\pgfpathmoveto{\pgfqpoint{5.812180in}{1.246480in}}%
\pgfpathlineto{\pgfqpoint{5.812180in}{1.914315in}}%
\pgfusepath{stroke}%
\end{pgfscope}%
\begin{pgfscope}%
\pgfpathrectangle{\pgfqpoint{4.956522in}{0.450000in}}{\pgfqpoint{3.043478in}{2.310000in}}%
\pgfusepath{clip}%
\pgfsetbuttcap%
\pgfsetroundjoin%
\pgfsetlinewidth{2.007500pt}%
\definecolor{currentstroke}{rgb}{0.121569,0.466667,0.705882}%
\pgfsetstrokecolor{currentstroke}%
\pgfsetdash{}{0pt}%
\pgfpathmoveto{\pgfqpoint{5.998496in}{1.495018in}}%
\pgfpathlineto{\pgfqpoint{5.998496in}{2.167807in}}%
\pgfusepath{stroke}%
\end{pgfscope}%
\begin{pgfscope}%
\pgfpathrectangle{\pgfqpoint{4.956522in}{0.450000in}}{\pgfqpoint{3.043478in}{2.310000in}}%
\pgfusepath{clip}%
\pgfsetbuttcap%
\pgfsetroundjoin%
\pgfsetlinewidth{2.007500pt}%
\definecolor{currentstroke}{rgb}{0.121569,0.466667,0.705882}%
\pgfsetstrokecolor{currentstroke}%
\pgfsetdash{}{0pt}%
\pgfpathmoveto{\pgfqpoint{6.464287in}{1.470113in}}%
\pgfpathlineto{\pgfqpoint{6.464287in}{2.142330in}}%
\pgfusepath{stroke}%
\end{pgfscope}%
\begin{pgfscope}%
\pgfpathrectangle{\pgfqpoint{4.956522in}{0.450000in}}{\pgfqpoint{3.043478in}{2.310000in}}%
\pgfusepath{clip}%
\pgfsetbuttcap%
\pgfsetroundjoin%
\pgfsetlinewidth{2.007500pt}%
\definecolor{currentstroke}{rgb}{0.121569,0.466667,0.705882}%
\pgfsetstrokecolor{currentstroke}%
\pgfsetdash{}{0pt}%
\pgfpathmoveto{\pgfqpoint{6.930078in}{1.416756in}}%
\pgfpathlineto{\pgfqpoint{6.930078in}{2.087861in}}%
\pgfusepath{stroke}%
\end{pgfscope}%
\begin{pgfscope}%
\pgfpathrectangle{\pgfqpoint{4.956522in}{0.450000in}}{\pgfqpoint{3.043478in}{2.310000in}}%
\pgfusepath{clip}%
\pgfsetbuttcap%
\pgfsetroundjoin%
\pgfsetlinewidth{2.007500pt}%
\definecolor{currentstroke}{rgb}{0.121569,0.466667,0.705882}%
\pgfsetstrokecolor{currentstroke}%
\pgfsetdash{}{0pt}%
\pgfpathmoveto{\pgfqpoint{7.395869in}{1.438703in}}%
\pgfpathlineto{\pgfqpoint{7.395869in}{2.110252in}}%
\pgfusepath{stroke}%
\end{pgfscope}%
\begin{pgfscope}%
\pgfpathrectangle{\pgfqpoint{4.956522in}{0.450000in}}{\pgfqpoint{3.043478in}{2.310000in}}%
\pgfusepath{clip}%
\pgfsetbuttcap%
\pgfsetroundjoin%
\pgfsetlinewidth{2.007500pt}%
\definecolor{currentstroke}{rgb}{0.121569,0.466667,0.705882}%
\pgfsetstrokecolor{currentstroke}%
\pgfsetdash{}{0pt}%
\pgfpathmoveto{\pgfqpoint{7.861660in}{1.389134in}}%
\pgfpathlineto{\pgfqpoint{7.861660in}{2.059681in}}%
\pgfusepath{stroke}%
\end{pgfscope}%
\begin{pgfscope}%
\pgfpathrectangle{\pgfqpoint{4.956522in}{0.450000in}}{\pgfqpoint{3.043478in}{2.310000in}}%
\pgfusepath{clip}%
\pgfsetbuttcap%
\pgfsetroundjoin%
\pgfsetlinewidth{2.007500pt}%
\definecolor{currentstroke}{rgb}{0.000000,0.750000,0.750000}%
\pgfsetstrokecolor{currentstroke}%
\pgfsetdash{}{0pt}%
\pgfpathmoveto{\pgfqpoint{5.108835in}{0.576272in}}%
\pgfpathlineto{\pgfqpoint{5.108835in}{1.231292in}}%
\pgfusepath{stroke}%
\end{pgfscope}%
\begin{pgfscope}%
\pgfpathrectangle{\pgfqpoint{4.956522in}{0.450000in}}{\pgfqpoint{3.043478in}{2.310000in}}%
\pgfusepath{clip}%
\pgfsetbuttcap%
\pgfsetroundjoin%
\pgfsetlinewidth{2.007500pt}%
\definecolor{currentstroke}{rgb}{0.000000,0.750000,0.750000}%
\pgfsetstrokecolor{currentstroke}%
\pgfsetdash{}{0pt}%
\pgfpathmoveto{\pgfqpoint{5.155414in}{0.588721in}}%
\pgfpathlineto{\pgfqpoint{5.155414in}{1.243595in}}%
\pgfusepath{stroke}%
\end{pgfscope}%
\begin{pgfscope}%
\pgfpathrectangle{\pgfqpoint{4.956522in}{0.450000in}}{\pgfqpoint{3.043478in}{2.310000in}}%
\pgfusepath{clip}%
\pgfsetbuttcap%
\pgfsetroundjoin%
\pgfsetlinewidth{2.007500pt}%
\definecolor{currentstroke}{rgb}{0.000000,0.750000,0.750000}%
\pgfsetstrokecolor{currentstroke}%
\pgfsetdash{}{0pt}%
\pgfpathmoveto{\pgfqpoint{5.201994in}{0.560290in}}%
\pgfpathlineto{\pgfqpoint{5.201994in}{1.214754in}}%
\pgfusepath{stroke}%
\end{pgfscope}%
\begin{pgfscope}%
\pgfpathrectangle{\pgfqpoint{4.956522in}{0.450000in}}{\pgfqpoint{3.043478in}{2.310000in}}%
\pgfusepath{clip}%
\pgfsetbuttcap%
\pgfsetroundjoin%
\pgfsetlinewidth{2.007500pt}%
\definecolor{currentstroke}{rgb}{0.000000,0.750000,0.750000}%
\pgfsetstrokecolor{currentstroke}%
\pgfsetdash{}{0pt}%
\pgfpathmoveto{\pgfqpoint{5.248573in}{0.725442in}}%
\pgfpathlineto{\pgfqpoint{5.248573in}{1.383147in}}%
\pgfusepath{stroke}%
\end{pgfscope}%
\begin{pgfscope}%
\pgfpathrectangle{\pgfqpoint{4.956522in}{0.450000in}}{\pgfqpoint{3.043478in}{2.310000in}}%
\pgfusepath{clip}%
\pgfsetbuttcap%
\pgfsetroundjoin%
\pgfsetlinewidth{2.007500pt}%
\definecolor{currentstroke}{rgb}{0.000000,0.750000,0.750000}%
\pgfsetstrokecolor{currentstroke}%
\pgfsetdash{}{0pt}%
\pgfpathmoveto{\pgfqpoint{5.295152in}{0.611312in}}%
\pgfpathlineto{\pgfqpoint{5.295152in}{1.267073in}}%
\pgfusepath{stroke}%
\end{pgfscope}%
\begin{pgfscope}%
\pgfpathrectangle{\pgfqpoint{4.956522in}{0.450000in}}{\pgfqpoint{3.043478in}{2.310000in}}%
\pgfusepath{clip}%
\pgfsetbuttcap%
\pgfsetroundjoin%
\pgfsetlinewidth{2.007500pt}%
\definecolor{currentstroke}{rgb}{0.000000,0.750000,0.750000}%
\pgfsetstrokecolor{currentstroke}%
\pgfsetdash{}{0pt}%
\pgfpathmoveto{\pgfqpoint{5.341731in}{0.655318in}}%
\pgfpathlineto{\pgfqpoint{5.341731in}{1.311505in}}%
\pgfusepath{stroke}%
\end{pgfscope}%
\begin{pgfscope}%
\pgfpathrectangle{\pgfqpoint{4.956522in}{0.450000in}}{\pgfqpoint{3.043478in}{2.310000in}}%
\pgfusepath{clip}%
\pgfsetbuttcap%
\pgfsetroundjoin%
\pgfsetlinewidth{2.007500pt}%
\definecolor{currentstroke}{rgb}{0.000000,0.750000,0.750000}%
\pgfsetstrokecolor{currentstroke}%
\pgfsetdash{}{0pt}%
\pgfpathmoveto{\pgfqpoint{5.388310in}{1.056979in}}%
\pgfpathlineto{\pgfqpoint{5.388310in}{1.721521in}}%
\pgfusepath{stroke}%
\end{pgfscope}%
\begin{pgfscope}%
\pgfpathrectangle{\pgfqpoint{4.956522in}{0.450000in}}{\pgfqpoint{3.043478in}{2.310000in}}%
\pgfusepath{clip}%
\pgfsetbuttcap%
\pgfsetroundjoin%
\pgfsetlinewidth{2.007500pt}%
\definecolor{currentstroke}{rgb}{0.000000,0.750000,0.750000}%
\pgfsetstrokecolor{currentstroke}%
\pgfsetdash{}{0pt}%
\pgfpathmoveto{\pgfqpoint{5.434889in}{0.797372in}}%
\pgfpathlineto{\pgfqpoint{5.434889in}{1.456597in}}%
\pgfusepath{stroke}%
\end{pgfscope}%
\begin{pgfscope}%
\pgfpathrectangle{\pgfqpoint{4.956522in}{0.450000in}}{\pgfqpoint{3.043478in}{2.310000in}}%
\pgfusepath{clip}%
\pgfsetbuttcap%
\pgfsetroundjoin%
\pgfsetlinewidth{2.007500pt}%
\definecolor{currentstroke}{rgb}{0.000000,0.750000,0.750000}%
\pgfsetstrokecolor{currentstroke}%
\pgfsetdash{}{0pt}%
\pgfpathmoveto{\pgfqpoint{5.621205in}{1.249459in}}%
\pgfpathlineto{\pgfqpoint{5.621205in}{1.917489in}}%
\pgfusepath{stroke}%
\end{pgfscope}%
\begin{pgfscope}%
\pgfpathrectangle{\pgfqpoint{4.956522in}{0.450000in}}{\pgfqpoint{3.043478in}{2.310000in}}%
\pgfusepath{clip}%
\pgfsetbuttcap%
\pgfsetroundjoin%
\pgfsetlinewidth{2.007500pt}%
\definecolor{currentstroke}{rgb}{0.000000,0.750000,0.750000}%
\pgfsetstrokecolor{currentstroke}%
\pgfsetdash{}{0pt}%
\pgfpathmoveto{\pgfqpoint{5.807522in}{1.304178in}}%
\pgfpathlineto{\pgfqpoint{5.807522in}{1.973178in}}%
\pgfusepath{stroke}%
\end{pgfscope}%
\begin{pgfscope}%
\pgfpathrectangle{\pgfqpoint{4.956522in}{0.450000in}}{\pgfqpoint{3.043478in}{2.310000in}}%
\pgfusepath{clip}%
\pgfsetbuttcap%
\pgfsetroundjoin%
\pgfsetlinewidth{2.007500pt}%
\definecolor{currentstroke}{rgb}{0.000000,0.750000,0.750000}%
\pgfsetstrokecolor{currentstroke}%
\pgfsetdash{}{0pt}%
\pgfpathmoveto{\pgfqpoint{5.993838in}{1.486597in}}%
\pgfpathlineto{\pgfqpoint{5.993838in}{2.159215in}}%
\pgfusepath{stroke}%
\end{pgfscope}%
\begin{pgfscope}%
\pgfpathrectangle{\pgfqpoint{4.956522in}{0.450000in}}{\pgfqpoint{3.043478in}{2.310000in}}%
\pgfusepath{clip}%
\pgfsetbuttcap%
\pgfsetroundjoin%
\pgfsetlinewidth{2.007500pt}%
\definecolor{currentstroke}{rgb}{0.000000,0.750000,0.750000}%
\pgfsetstrokecolor{currentstroke}%
\pgfsetdash{}{0pt}%
\pgfpathmoveto{\pgfqpoint{6.459629in}{1.546761in}}%
\pgfpathlineto{\pgfqpoint{6.459629in}{2.220526in}}%
\pgfusepath{stroke}%
\end{pgfscope}%
\begin{pgfscope}%
\pgfpathrectangle{\pgfqpoint{4.956522in}{0.450000in}}{\pgfqpoint{3.043478in}{2.310000in}}%
\pgfusepath{clip}%
\pgfsetbuttcap%
\pgfsetroundjoin%
\pgfsetlinewidth{2.007500pt}%
\definecolor{currentstroke}{rgb}{0.000000,0.750000,0.750000}%
\pgfsetstrokecolor{currentstroke}%
\pgfsetdash{}{0pt}%
\pgfpathmoveto{\pgfqpoint{6.925420in}{1.416819in}}%
\pgfpathlineto{\pgfqpoint{6.925420in}{2.087926in}}%
\pgfusepath{stroke}%
\end{pgfscope}%
\begin{pgfscope}%
\pgfpathrectangle{\pgfqpoint{4.956522in}{0.450000in}}{\pgfqpoint{3.043478in}{2.310000in}}%
\pgfusepath{clip}%
\pgfsetbuttcap%
\pgfsetroundjoin%
\pgfsetlinewidth{2.007500pt}%
\definecolor{currentstroke}{rgb}{0.000000,0.750000,0.750000}%
\pgfsetstrokecolor{currentstroke}%
\pgfsetdash{}{0pt}%
\pgfpathmoveto{\pgfqpoint{7.391211in}{1.430757in}}%
\pgfpathlineto{\pgfqpoint{7.391211in}{2.102145in}}%
\pgfusepath{stroke}%
\end{pgfscope}%
\begin{pgfscope}%
\pgfpathrectangle{\pgfqpoint{4.956522in}{0.450000in}}{\pgfqpoint{3.043478in}{2.310000in}}%
\pgfusepath{clip}%
\pgfsetbuttcap%
\pgfsetroundjoin%
\pgfsetlinewidth{2.007500pt}%
\definecolor{currentstroke}{rgb}{0.000000,0.750000,0.750000}%
\pgfsetstrokecolor{currentstroke}%
\pgfsetdash{}{0pt}%
\pgfpathmoveto{\pgfqpoint{7.857002in}{1.388000in}}%
\pgfpathlineto{\pgfqpoint{7.857002in}{2.058524in}}%
\pgfusepath{stroke}%
\end{pgfscope}%
\begin{pgfscope}%
\pgfpathrectangle{\pgfqpoint{4.956522in}{0.450000in}}{\pgfqpoint{3.043478in}{2.310000in}}%
\pgfusepath{clip}%
\pgfsetbuttcap%
\pgfsetroundjoin%
\pgfsetlinewidth{2.007500pt}%
\definecolor{currentstroke}{rgb}{1.000000,0.000000,0.000000}%
\pgfsetstrokecolor{currentstroke}%
\pgfsetdash{}{0pt}%
\pgfpathmoveto{\pgfqpoint{5.094862in}{0.584263in}}%
\pgfpathlineto{\pgfqpoint{5.094862in}{1.239445in}}%
\pgfusepath{stroke}%
\end{pgfscope}%
\begin{pgfscope}%
\pgfpathrectangle{\pgfqpoint{4.956522in}{0.450000in}}{\pgfqpoint{3.043478in}{2.310000in}}%
\pgfusepath{clip}%
\pgfsetbuttcap%
\pgfsetroundjoin%
\pgfsetlinewidth{2.007500pt}%
\definecolor{currentstroke}{rgb}{1.000000,0.000000,0.000000}%
\pgfsetstrokecolor{currentstroke}%
\pgfsetdash{}{0pt}%
\pgfpathmoveto{\pgfqpoint{5.141441in}{0.584979in}}%
\pgfpathlineto{\pgfqpoint{5.141441in}{1.239777in}}%
\pgfusepath{stroke}%
\end{pgfscope}%
\begin{pgfscope}%
\pgfpathrectangle{\pgfqpoint{4.956522in}{0.450000in}}{\pgfqpoint{3.043478in}{2.310000in}}%
\pgfusepath{clip}%
\pgfsetbuttcap%
\pgfsetroundjoin%
\pgfsetlinewidth{2.007500pt}%
\definecolor{currentstroke}{rgb}{1.000000,0.000000,0.000000}%
\pgfsetstrokecolor{currentstroke}%
\pgfsetdash{}{0pt}%
\pgfpathmoveto{\pgfqpoint{5.188020in}{0.564393in}}%
\pgfpathlineto{\pgfqpoint{5.188020in}{1.218940in}}%
\pgfusepath{stroke}%
\end{pgfscope}%
\begin{pgfscope}%
\pgfpathrectangle{\pgfqpoint{4.956522in}{0.450000in}}{\pgfqpoint{3.043478in}{2.310000in}}%
\pgfusepath{clip}%
\pgfsetbuttcap%
\pgfsetroundjoin%
\pgfsetlinewidth{2.007500pt}%
\definecolor{currentstroke}{rgb}{1.000000,0.000000,0.000000}%
\pgfsetstrokecolor{currentstroke}%
\pgfsetdash{}{0pt}%
\pgfpathmoveto{\pgfqpoint{5.234599in}{0.725102in}}%
\pgfpathlineto{\pgfqpoint{5.234599in}{1.382800in}}%
\pgfusepath{stroke}%
\end{pgfscope}%
\begin{pgfscope}%
\pgfpathrectangle{\pgfqpoint{4.956522in}{0.450000in}}{\pgfqpoint{3.043478in}{2.310000in}}%
\pgfusepath{clip}%
\pgfsetbuttcap%
\pgfsetroundjoin%
\pgfsetlinewidth{2.007500pt}%
\definecolor{currentstroke}{rgb}{1.000000,0.000000,0.000000}%
\pgfsetstrokecolor{currentstroke}%
\pgfsetdash{}{0pt}%
\pgfpathmoveto{\pgfqpoint{5.281178in}{0.600459in}}%
\pgfpathlineto{\pgfqpoint{5.281178in}{1.256002in}}%
\pgfusepath{stroke}%
\end{pgfscope}%
\begin{pgfscope}%
\pgfpathrectangle{\pgfqpoint{4.956522in}{0.450000in}}{\pgfqpoint{3.043478in}{2.310000in}}%
\pgfusepath{clip}%
\pgfsetbuttcap%
\pgfsetroundjoin%
\pgfsetlinewidth{2.007500pt}%
\definecolor{currentstroke}{rgb}{1.000000,0.000000,0.000000}%
\pgfsetstrokecolor{currentstroke}%
\pgfsetdash{}{0pt}%
\pgfpathmoveto{\pgfqpoint{5.327757in}{0.645903in}}%
\pgfpathlineto{\pgfqpoint{5.327757in}{1.301899in}}%
\pgfusepath{stroke}%
\end{pgfscope}%
\begin{pgfscope}%
\pgfpathrectangle{\pgfqpoint{4.956522in}{0.450000in}}{\pgfqpoint{3.043478in}{2.310000in}}%
\pgfusepath{clip}%
\pgfsetbuttcap%
\pgfsetroundjoin%
\pgfsetlinewidth{2.007500pt}%
\definecolor{currentstroke}{rgb}{1.000000,0.000000,0.000000}%
\pgfsetstrokecolor{currentstroke}%
\pgfsetdash{}{0pt}%
\pgfpathmoveto{\pgfqpoint{5.374336in}{0.989746in}}%
\pgfpathlineto{\pgfqpoint{5.374336in}{1.652928in}}%
\pgfusepath{stroke}%
\end{pgfscope}%
\begin{pgfscope}%
\pgfpathrectangle{\pgfqpoint{4.956522in}{0.450000in}}{\pgfqpoint{3.043478in}{2.310000in}}%
\pgfusepath{clip}%
\pgfsetbuttcap%
\pgfsetroundjoin%
\pgfsetlinewidth{2.007500pt}%
\definecolor{currentstroke}{rgb}{1.000000,0.000000,0.000000}%
\pgfsetstrokecolor{currentstroke}%
\pgfsetdash{}{0pt}%
\pgfpathmoveto{\pgfqpoint{5.420915in}{0.730590in}}%
\pgfpathlineto{\pgfqpoint{5.420915in}{1.388465in}}%
\pgfusepath{stroke}%
\end{pgfscope}%
\begin{pgfscope}%
\pgfpathrectangle{\pgfqpoint{4.956522in}{0.450000in}}{\pgfqpoint{3.043478in}{2.310000in}}%
\pgfusepath{clip}%
\pgfsetbuttcap%
\pgfsetroundjoin%
\pgfsetlinewidth{2.007500pt}%
\definecolor{currentstroke}{rgb}{1.000000,0.000000,0.000000}%
\pgfsetstrokecolor{currentstroke}%
\pgfsetdash{}{0pt}%
\pgfpathmoveto{\pgfqpoint{5.607232in}{1.056505in}}%
\pgfpathlineto{\pgfqpoint{5.607232in}{1.720634in}}%
\pgfusepath{stroke}%
\end{pgfscope}%
\begin{pgfscope}%
\pgfpathrectangle{\pgfqpoint{4.956522in}{0.450000in}}{\pgfqpoint{3.043478in}{2.310000in}}%
\pgfusepath{clip}%
\pgfsetbuttcap%
\pgfsetroundjoin%
\pgfsetlinewidth{2.007500pt}%
\definecolor{currentstroke}{rgb}{1.000000,0.000000,0.000000}%
\pgfsetstrokecolor{currentstroke}%
\pgfsetdash{}{0pt}%
\pgfpathmoveto{\pgfqpoint{5.793548in}{1.233927in}}%
\pgfpathlineto{\pgfqpoint{5.793548in}{1.901508in}}%
\pgfusepath{stroke}%
\end{pgfscope}%
\begin{pgfscope}%
\pgfpathrectangle{\pgfqpoint{4.956522in}{0.450000in}}{\pgfqpoint{3.043478in}{2.310000in}}%
\pgfusepath{clip}%
\pgfsetbuttcap%
\pgfsetroundjoin%
\pgfsetlinewidth{2.007500pt}%
\definecolor{currentstroke}{rgb}{1.000000,0.000000,0.000000}%
\pgfsetstrokecolor{currentstroke}%
\pgfsetdash{}{0pt}%
\pgfpathmoveto{\pgfqpoint{5.979865in}{1.362338in}}%
\pgfpathlineto{\pgfqpoint{5.979865in}{2.032446in}}%
\pgfusepath{stroke}%
\end{pgfscope}%
\begin{pgfscope}%
\pgfpathrectangle{\pgfqpoint{4.956522in}{0.450000in}}{\pgfqpoint{3.043478in}{2.310000in}}%
\pgfusepath{clip}%
\pgfsetbuttcap%
\pgfsetroundjoin%
\pgfsetlinewidth{2.007500pt}%
\definecolor{currentstroke}{rgb}{1.000000,0.000000,0.000000}%
\pgfsetstrokecolor{currentstroke}%
\pgfsetdash{}{0pt}%
\pgfpathmoveto{\pgfqpoint{6.445656in}{1.768515in}}%
\pgfpathlineto{\pgfqpoint{6.445656in}{2.446761in}}%
\pgfusepath{stroke}%
\end{pgfscope}%
\begin{pgfscope}%
\pgfpathrectangle{\pgfqpoint{4.956522in}{0.450000in}}{\pgfqpoint{3.043478in}{2.310000in}}%
\pgfusepath{clip}%
\pgfsetbuttcap%
\pgfsetroundjoin%
\pgfsetlinewidth{2.007500pt}%
\definecolor{currentstroke}{rgb}{1.000000,0.000000,0.000000}%
\pgfsetstrokecolor{currentstroke}%
\pgfsetdash{}{0pt}%
\pgfpathmoveto{\pgfqpoint{6.911446in}{1.657235in}}%
\pgfpathlineto{\pgfqpoint{6.911446in}{2.333199in}}%
\pgfusepath{stroke}%
\end{pgfscope}%
\begin{pgfscope}%
\pgfpathrectangle{\pgfqpoint{4.956522in}{0.450000in}}{\pgfqpoint{3.043478in}{2.310000in}}%
\pgfusepath{clip}%
\pgfsetbuttcap%
\pgfsetroundjoin%
\pgfsetlinewidth{2.007500pt}%
\definecolor{currentstroke}{rgb}{1.000000,0.000000,0.000000}%
\pgfsetstrokecolor{currentstroke}%
\pgfsetdash{}{0pt}%
\pgfpathmoveto{\pgfqpoint{7.377237in}{1.701324in}}%
\pgfpathlineto{\pgfqpoint{7.377237in}{2.378179in}}%
\pgfusepath{stroke}%
\end{pgfscope}%
\begin{pgfscope}%
\pgfpathrectangle{\pgfqpoint{4.956522in}{0.450000in}}{\pgfqpoint{3.043478in}{2.310000in}}%
\pgfusepath{clip}%
\pgfsetbuttcap%
\pgfsetroundjoin%
\pgfsetlinewidth{2.007500pt}%
\definecolor{currentstroke}{rgb}{1.000000,0.000000,0.000000}%
\pgfsetstrokecolor{currentstroke}%
\pgfsetdash{}{0pt}%
\pgfpathmoveto{\pgfqpoint{7.843028in}{1.765598in}}%
\pgfpathlineto{\pgfqpoint{7.843028in}{2.443750in}}%
\pgfusepath{stroke}%
\end{pgfscope}%
\begin{pgfscope}%
\pgfpathrectangle{\pgfqpoint{4.956522in}{0.450000in}}{\pgfqpoint{3.043478in}{2.310000in}}%
\pgfusepath{clip}%
\pgfsetrectcap%
\pgfsetroundjoin%
\pgfsetlinewidth{2.007500pt}%
\definecolor{currentstroke}{rgb}{0.750000,0.750000,0.000000}%
\pgfsetstrokecolor{currentstroke}%
\pgfsetdash{}{0pt}%
\pgfpathmoveto{\pgfqpoint{5.104177in}{0.925855in}}%
\pgfpathlineto{\pgfqpoint{5.150757in}{0.955048in}}%
\pgfpathlineto{\pgfqpoint{5.197336in}{0.944303in}}%
\pgfpathlineto{\pgfqpoint{5.243915in}{1.127151in}}%
\pgfpathlineto{\pgfqpoint{5.290494in}{1.038978in}}%
\pgfpathlineto{\pgfqpoint{5.337073in}{1.116292in}}%
\pgfpathlineto{\pgfqpoint{5.383652in}{1.572849in}}%
\pgfpathlineto{\pgfqpoint{5.430231in}{1.361868in}}%
\pgfpathlineto{\pgfqpoint{5.616548in}{1.970598in}}%
\pgfpathlineto{\pgfqpoint{5.802864in}{2.068771in}}%
\pgfpathlineto{\pgfqpoint{5.989180in}{2.313781in}}%
\pgfpathlineto{\pgfqpoint{6.454971in}{2.177571in}}%
\pgfpathlineto{\pgfqpoint{6.920762in}{2.053043in}}%
\pgfpathlineto{\pgfqpoint{7.386553in}{2.094763in}}%
\pgfpathlineto{\pgfqpoint{7.852344in}{1.956254in}}%
\pgfusepath{stroke}%
\end{pgfscope}%
\begin{pgfscope}%
\pgfpathrectangle{\pgfqpoint{4.956522in}{0.450000in}}{\pgfqpoint{3.043478in}{2.310000in}}%
\pgfusepath{clip}%
\pgfsetbuttcap%
\pgfsetmiterjoin%
\definecolor{currentfill}{rgb}{0.750000,0.750000,0.000000}%
\pgfsetfillcolor{currentfill}%
\pgfsetlinewidth{1.003750pt}%
\definecolor{currentstroke}{rgb}{0.750000,0.750000,0.000000}%
\pgfsetstrokecolor{currentstroke}%
\pgfsetdash{}{0pt}%
\pgfsys@defobject{currentmarker}{\pgfqpoint{-0.026418in}{-0.022473in}}{\pgfqpoint{0.026418in}{0.027778in}}{%
\pgfpathmoveto{\pgfqpoint{0.000000in}{0.027778in}}%
\pgfpathlineto{\pgfqpoint{-0.026418in}{0.008584in}}%
\pgfpathlineto{\pgfqpoint{-0.016327in}{-0.022473in}}%
\pgfpathlineto{\pgfqpoint{0.016327in}{-0.022473in}}%
\pgfpathlineto{\pgfqpoint{0.026418in}{0.008584in}}%
\pgfpathclose%
\pgfusepath{stroke,fill}%
}%
\begin{pgfscope}%
\pgfsys@transformshift{5.104177in}{0.925855in}%
\pgfsys@useobject{currentmarker}{}%
\end{pgfscope}%
\begin{pgfscope}%
\pgfsys@transformshift{5.150757in}{0.955048in}%
\pgfsys@useobject{currentmarker}{}%
\end{pgfscope}%
\begin{pgfscope}%
\pgfsys@transformshift{5.197336in}{0.944303in}%
\pgfsys@useobject{currentmarker}{}%
\end{pgfscope}%
\begin{pgfscope}%
\pgfsys@transformshift{5.243915in}{1.127151in}%
\pgfsys@useobject{currentmarker}{}%
\end{pgfscope}%
\begin{pgfscope}%
\pgfsys@transformshift{5.290494in}{1.038978in}%
\pgfsys@useobject{currentmarker}{}%
\end{pgfscope}%
\begin{pgfscope}%
\pgfsys@transformshift{5.337073in}{1.116292in}%
\pgfsys@useobject{currentmarker}{}%
\end{pgfscope}%
\begin{pgfscope}%
\pgfsys@transformshift{5.383652in}{1.572849in}%
\pgfsys@useobject{currentmarker}{}%
\end{pgfscope}%
\begin{pgfscope}%
\pgfsys@transformshift{5.430231in}{1.361868in}%
\pgfsys@useobject{currentmarker}{}%
\end{pgfscope}%
\begin{pgfscope}%
\pgfsys@transformshift{5.616548in}{1.970598in}%
\pgfsys@useobject{currentmarker}{}%
\end{pgfscope}%
\begin{pgfscope}%
\pgfsys@transformshift{5.802864in}{2.068771in}%
\pgfsys@useobject{currentmarker}{}%
\end{pgfscope}%
\begin{pgfscope}%
\pgfsys@transformshift{5.989180in}{2.313781in}%
\pgfsys@useobject{currentmarker}{}%
\end{pgfscope}%
\begin{pgfscope}%
\pgfsys@transformshift{6.454971in}{2.177571in}%
\pgfsys@useobject{currentmarker}{}%
\end{pgfscope}%
\begin{pgfscope}%
\pgfsys@transformshift{6.920762in}{2.053043in}%
\pgfsys@useobject{currentmarker}{}%
\end{pgfscope}%
\begin{pgfscope}%
\pgfsys@transformshift{7.386553in}{2.094763in}%
\pgfsys@useobject{currentmarker}{}%
\end{pgfscope}%
\begin{pgfscope}%
\pgfsys@transformshift{7.852344in}{1.956254in}%
\pgfsys@useobject{currentmarker}{}%
\end{pgfscope}%
\end{pgfscope}%
\begin{pgfscope}%
\pgfpathrectangle{\pgfqpoint{4.956522in}{0.450000in}}{\pgfqpoint{3.043478in}{2.310000in}}%
\pgfusepath{clip}%
\pgfsetrectcap%
\pgfsetroundjoin%
\pgfsetlinewidth{2.007500pt}%
\definecolor{currentstroke}{rgb}{0.121569,0.466667,0.705882}%
\pgfsetstrokecolor{currentstroke}%
\pgfsetdash{}{0pt}%
\pgfpathmoveto{\pgfqpoint{5.113493in}{0.882295in}}%
\pgfpathlineto{\pgfqpoint{5.160072in}{0.895924in}}%
\pgfpathlineto{\pgfqpoint{5.206651in}{0.894191in}}%
\pgfpathlineto{\pgfqpoint{5.253231in}{0.888475in}}%
\pgfpathlineto{\pgfqpoint{5.299810in}{0.922216in}}%
\pgfpathlineto{\pgfqpoint{5.346389in}{0.941636in}}%
\pgfpathlineto{\pgfqpoint{5.392968in}{1.042883in}}%
\pgfpathlineto{\pgfqpoint{5.439547in}{1.076539in}}%
\pgfpathlineto{\pgfqpoint{5.625863in}{1.462819in}}%
\pgfpathlineto{\pgfqpoint{5.812180in}{1.580398in}}%
\pgfpathlineto{\pgfqpoint{5.998496in}{1.831412in}}%
\pgfpathlineto{\pgfqpoint{6.464287in}{1.806221in}}%
\pgfpathlineto{\pgfqpoint{6.930078in}{1.752308in}}%
\pgfpathlineto{\pgfqpoint{7.395869in}{1.774478in}}%
\pgfpathlineto{\pgfqpoint{7.861660in}{1.724407in}}%
\pgfusepath{stroke}%
\end{pgfscope}%
\begin{pgfscope}%
\pgfpathrectangle{\pgfqpoint{4.956522in}{0.450000in}}{\pgfqpoint{3.043478in}{2.310000in}}%
\pgfusepath{clip}%
\pgfsetbuttcap%
\pgfsetroundjoin%
\definecolor{currentfill}{rgb}{0.121569,0.466667,0.705882}%
\pgfsetfillcolor{currentfill}%
\pgfsetlinewidth{1.003750pt}%
\definecolor{currentstroke}{rgb}{0.121569,0.466667,0.705882}%
\pgfsetstrokecolor{currentstroke}%
\pgfsetdash{}{0pt}%
\pgfsys@defobject{currentmarker}{\pgfqpoint{-0.027778in}{-0.027778in}}{\pgfqpoint{0.027778in}{0.027778in}}{%
\pgfpathmoveto{\pgfqpoint{-0.027778in}{-0.027778in}}%
\pgfpathlineto{\pgfqpoint{0.027778in}{0.027778in}}%
\pgfpathmoveto{\pgfqpoint{-0.027778in}{0.027778in}}%
\pgfpathlineto{\pgfqpoint{0.027778in}{-0.027778in}}%
\pgfusepath{stroke,fill}%
}%
\begin{pgfscope}%
\pgfsys@transformshift{5.113493in}{0.882295in}%
\pgfsys@useobject{currentmarker}{}%
\end{pgfscope}%
\begin{pgfscope}%
\pgfsys@transformshift{5.160072in}{0.895924in}%
\pgfsys@useobject{currentmarker}{}%
\end{pgfscope}%
\begin{pgfscope}%
\pgfsys@transformshift{5.206651in}{0.894191in}%
\pgfsys@useobject{currentmarker}{}%
\end{pgfscope}%
\begin{pgfscope}%
\pgfsys@transformshift{5.253231in}{0.888475in}%
\pgfsys@useobject{currentmarker}{}%
\end{pgfscope}%
\begin{pgfscope}%
\pgfsys@transformshift{5.299810in}{0.922216in}%
\pgfsys@useobject{currentmarker}{}%
\end{pgfscope}%
\begin{pgfscope}%
\pgfsys@transformshift{5.346389in}{0.941636in}%
\pgfsys@useobject{currentmarker}{}%
\end{pgfscope}%
\begin{pgfscope}%
\pgfsys@transformshift{5.392968in}{1.042883in}%
\pgfsys@useobject{currentmarker}{}%
\end{pgfscope}%
\begin{pgfscope}%
\pgfsys@transformshift{5.439547in}{1.076539in}%
\pgfsys@useobject{currentmarker}{}%
\end{pgfscope}%
\begin{pgfscope}%
\pgfsys@transformshift{5.625863in}{1.462819in}%
\pgfsys@useobject{currentmarker}{}%
\end{pgfscope}%
\begin{pgfscope}%
\pgfsys@transformshift{5.812180in}{1.580398in}%
\pgfsys@useobject{currentmarker}{}%
\end{pgfscope}%
\begin{pgfscope}%
\pgfsys@transformshift{5.998496in}{1.831412in}%
\pgfsys@useobject{currentmarker}{}%
\end{pgfscope}%
\begin{pgfscope}%
\pgfsys@transformshift{6.464287in}{1.806221in}%
\pgfsys@useobject{currentmarker}{}%
\end{pgfscope}%
\begin{pgfscope}%
\pgfsys@transformshift{6.930078in}{1.752308in}%
\pgfsys@useobject{currentmarker}{}%
\end{pgfscope}%
\begin{pgfscope}%
\pgfsys@transformshift{7.395869in}{1.774478in}%
\pgfsys@useobject{currentmarker}{}%
\end{pgfscope}%
\begin{pgfscope}%
\pgfsys@transformshift{7.861660in}{1.724407in}%
\pgfsys@useobject{currentmarker}{}%
\end{pgfscope}%
\end{pgfscope}%
\begin{pgfscope}%
\pgfpathrectangle{\pgfqpoint{4.956522in}{0.450000in}}{\pgfqpoint{3.043478in}{2.310000in}}%
\pgfusepath{clip}%
\pgfsetrectcap%
\pgfsetroundjoin%
\pgfsetlinewidth{2.007500pt}%
\definecolor{currentstroke}{rgb}{0.000000,0.750000,0.750000}%
\pgfsetstrokecolor{currentstroke}%
\pgfsetdash{}{0pt}%
\pgfpathmoveto{\pgfqpoint{5.108835in}{0.903782in}}%
\pgfpathlineto{\pgfqpoint{5.155414in}{0.916158in}}%
\pgfpathlineto{\pgfqpoint{5.201994in}{0.887522in}}%
\pgfpathlineto{\pgfqpoint{5.248573in}{1.054295in}}%
\pgfpathlineto{\pgfqpoint{5.295152in}{0.939193in}}%
\pgfpathlineto{\pgfqpoint{5.341731in}{0.983411in}}%
\pgfpathlineto{\pgfqpoint{5.388310in}{1.389250in}}%
\pgfpathlineto{\pgfqpoint{5.434889in}{1.126984in}}%
\pgfpathlineto{\pgfqpoint{5.621205in}{1.583474in}}%
\pgfpathlineto{\pgfqpoint{5.807522in}{1.638678in}}%
\pgfpathlineto{\pgfqpoint{5.993838in}{1.822906in}}%
\pgfpathlineto{\pgfqpoint{6.459629in}{1.883643in}}%
\pgfpathlineto{\pgfqpoint{6.925420in}{1.752372in}}%
\pgfpathlineto{\pgfqpoint{7.391211in}{1.766451in}}%
\pgfpathlineto{\pgfqpoint{7.857002in}{1.723262in}}%
\pgfusepath{stroke}%
\end{pgfscope}%
\begin{pgfscope}%
\pgfpathrectangle{\pgfqpoint{4.956522in}{0.450000in}}{\pgfqpoint{3.043478in}{2.310000in}}%
\pgfusepath{clip}%
\pgfsetbuttcap%
\pgfsetbeveljoin%
\definecolor{currentfill}{rgb}{0.000000,0.750000,0.750000}%
\pgfsetfillcolor{currentfill}%
\pgfsetlinewidth{1.003750pt}%
\definecolor{currentstroke}{rgb}{0.000000,0.750000,0.750000}%
\pgfsetstrokecolor{currentstroke}%
\pgfsetdash{}{0pt}%
\pgfsys@defobject{currentmarker}{\pgfqpoint{-0.026418in}{-0.022473in}}{\pgfqpoint{0.026418in}{0.027778in}}{%
\pgfpathmoveto{\pgfqpoint{0.000000in}{0.027778in}}%
\pgfpathlineto{\pgfqpoint{-0.006236in}{0.008584in}}%
\pgfpathlineto{\pgfqpoint{-0.026418in}{0.008584in}}%
\pgfpathlineto{\pgfqpoint{-0.010091in}{-0.003279in}}%
\pgfpathlineto{\pgfqpoint{-0.016327in}{-0.022473in}}%
\pgfpathlineto{\pgfqpoint{-0.000000in}{-0.010610in}}%
\pgfpathlineto{\pgfqpoint{0.016327in}{-0.022473in}}%
\pgfpathlineto{\pgfqpoint{0.010091in}{-0.003279in}}%
\pgfpathlineto{\pgfqpoint{0.026418in}{0.008584in}}%
\pgfpathlineto{\pgfqpoint{0.006236in}{0.008584in}}%
\pgfpathclose%
\pgfusepath{stroke,fill}%
}%
\begin{pgfscope}%
\pgfsys@transformshift{5.108835in}{0.903782in}%
\pgfsys@useobject{currentmarker}{}%
\end{pgfscope}%
\begin{pgfscope}%
\pgfsys@transformshift{5.155414in}{0.916158in}%
\pgfsys@useobject{currentmarker}{}%
\end{pgfscope}%
\begin{pgfscope}%
\pgfsys@transformshift{5.201994in}{0.887522in}%
\pgfsys@useobject{currentmarker}{}%
\end{pgfscope}%
\begin{pgfscope}%
\pgfsys@transformshift{5.248573in}{1.054295in}%
\pgfsys@useobject{currentmarker}{}%
\end{pgfscope}%
\begin{pgfscope}%
\pgfsys@transformshift{5.295152in}{0.939193in}%
\pgfsys@useobject{currentmarker}{}%
\end{pgfscope}%
\begin{pgfscope}%
\pgfsys@transformshift{5.341731in}{0.983411in}%
\pgfsys@useobject{currentmarker}{}%
\end{pgfscope}%
\begin{pgfscope}%
\pgfsys@transformshift{5.388310in}{1.389250in}%
\pgfsys@useobject{currentmarker}{}%
\end{pgfscope}%
\begin{pgfscope}%
\pgfsys@transformshift{5.434889in}{1.126984in}%
\pgfsys@useobject{currentmarker}{}%
\end{pgfscope}%
\begin{pgfscope}%
\pgfsys@transformshift{5.621205in}{1.583474in}%
\pgfsys@useobject{currentmarker}{}%
\end{pgfscope}%
\begin{pgfscope}%
\pgfsys@transformshift{5.807522in}{1.638678in}%
\pgfsys@useobject{currentmarker}{}%
\end{pgfscope}%
\begin{pgfscope}%
\pgfsys@transformshift{5.993838in}{1.822906in}%
\pgfsys@useobject{currentmarker}{}%
\end{pgfscope}%
\begin{pgfscope}%
\pgfsys@transformshift{6.459629in}{1.883643in}%
\pgfsys@useobject{currentmarker}{}%
\end{pgfscope}%
\begin{pgfscope}%
\pgfsys@transformshift{6.925420in}{1.752372in}%
\pgfsys@useobject{currentmarker}{}%
\end{pgfscope}%
\begin{pgfscope}%
\pgfsys@transformshift{7.391211in}{1.766451in}%
\pgfsys@useobject{currentmarker}{}%
\end{pgfscope}%
\begin{pgfscope}%
\pgfsys@transformshift{7.857002in}{1.723262in}%
\pgfsys@useobject{currentmarker}{}%
\end{pgfscope}%
\end{pgfscope}%
\begin{pgfscope}%
\pgfpathrectangle{\pgfqpoint{4.956522in}{0.450000in}}{\pgfqpoint{3.043478in}{2.310000in}}%
\pgfusepath{clip}%
\pgfsetrectcap%
\pgfsetroundjoin%
\pgfsetlinewidth{2.007500pt}%
\definecolor{currentstroke}{rgb}{1.000000,0.000000,0.000000}%
\pgfsetstrokecolor{currentstroke}%
\pgfsetdash{}{0pt}%
\pgfpathmoveto{\pgfqpoint{5.094862in}{0.911854in}}%
\pgfpathlineto{\pgfqpoint{5.141441in}{0.912378in}}%
\pgfpathlineto{\pgfqpoint{5.188020in}{0.891666in}}%
\pgfpathlineto{\pgfqpoint{5.234599in}{1.053951in}}%
\pgfpathlineto{\pgfqpoint{5.281178in}{0.928230in}}%
\pgfpathlineto{\pgfqpoint{5.327757in}{0.973901in}}%
\pgfpathlineto{\pgfqpoint{5.374336in}{1.321337in}}%
\pgfpathlineto{\pgfqpoint{5.420915in}{1.059528in}}%
\pgfpathlineto{\pgfqpoint{5.607232in}{1.388569in}}%
\pgfpathlineto{\pgfqpoint{5.793548in}{1.567717in}}%
\pgfpathlineto{\pgfqpoint{5.979865in}{1.697392in}}%
\pgfpathlineto{\pgfqpoint{6.445656in}{2.107638in}}%
\pgfpathlineto{\pgfqpoint{6.911446in}{1.995217in}}%
\pgfpathlineto{\pgfqpoint{7.377237in}{2.039751in}}%
\pgfpathlineto{\pgfqpoint{7.843028in}{2.104674in}}%
\pgfusepath{stroke}%
\end{pgfscope}%
\begin{pgfscope}%
\pgfpathrectangle{\pgfqpoint{4.956522in}{0.450000in}}{\pgfqpoint{3.043478in}{2.310000in}}%
\pgfusepath{clip}%
\pgfsetbuttcap%
\pgfsetmiterjoin%
\definecolor{currentfill}{rgb}{1.000000,0.000000,0.000000}%
\pgfsetfillcolor{currentfill}%
\pgfsetlinewidth{1.003750pt}%
\definecolor{currentstroke}{rgb}{1.000000,0.000000,0.000000}%
\pgfsetstrokecolor{currentstroke}%
\pgfsetdash{}{0pt}%
\pgfsys@defobject{currentmarker}{\pgfqpoint{-0.027778in}{-0.027778in}}{\pgfqpoint{0.027778in}{0.027778in}}{%
\pgfpathmoveto{\pgfqpoint{-0.027778in}{-0.027778in}}%
\pgfpathlineto{\pgfqpoint{0.027778in}{-0.027778in}}%
\pgfpathlineto{\pgfqpoint{0.027778in}{0.027778in}}%
\pgfpathlineto{\pgfqpoint{-0.027778in}{0.027778in}}%
\pgfpathclose%
\pgfusepath{stroke,fill}%
}%
\begin{pgfscope}%
\pgfsys@transformshift{5.094862in}{0.911854in}%
\pgfsys@useobject{currentmarker}{}%
\end{pgfscope}%
\begin{pgfscope}%
\pgfsys@transformshift{5.141441in}{0.912378in}%
\pgfsys@useobject{currentmarker}{}%
\end{pgfscope}%
\begin{pgfscope}%
\pgfsys@transformshift{5.188020in}{0.891666in}%
\pgfsys@useobject{currentmarker}{}%
\end{pgfscope}%
\begin{pgfscope}%
\pgfsys@transformshift{5.234599in}{1.053951in}%
\pgfsys@useobject{currentmarker}{}%
\end{pgfscope}%
\begin{pgfscope}%
\pgfsys@transformshift{5.281178in}{0.928230in}%
\pgfsys@useobject{currentmarker}{}%
\end{pgfscope}%
\begin{pgfscope}%
\pgfsys@transformshift{5.327757in}{0.973901in}%
\pgfsys@useobject{currentmarker}{}%
\end{pgfscope}%
\begin{pgfscope}%
\pgfsys@transformshift{5.374336in}{1.321337in}%
\pgfsys@useobject{currentmarker}{}%
\end{pgfscope}%
\begin{pgfscope}%
\pgfsys@transformshift{5.420915in}{1.059528in}%
\pgfsys@useobject{currentmarker}{}%
\end{pgfscope}%
\begin{pgfscope}%
\pgfsys@transformshift{5.607232in}{1.388569in}%
\pgfsys@useobject{currentmarker}{}%
\end{pgfscope}%
\begin{pgfscope}%
\pgfsys@transformshift{5.793548in}{1.567717in}%
\pgfsys@useobject{currentmarker}{}%
\end{pgfscope}%
\begin{pgfscope}%
\pgfsys@transformshift{5.979865in}{1.697392in}%
\pgfsys@useobject{currentmarker}{}%
\end{pgfscope}%
\begin{pgfscope}%
\pgfsys@transformshift{6.445656in}{2.107638in}%
\pgfsys@useobject{currentmarker}{}%
\end{pgfscope}%
\begin{pgfscope}%
\pgfsys@transformshift{6.911446in}{1.995217in}%
\pgfsys@useobject{currentmarker}{}%
\end{pgfscope}%
\begin{pgfscope}%
\pgfsys@transformshift{7.377237in}{2.039751in}%
\pgfsys@useobject{currentmarker}{}%
\end{pgfscope}%
\begin{pgfscope}%
\pgfsys@transformshift{7.843028in}{2.104674in}%
\pgfsys@useobject{currentmarker}{}%
\end{pgfscope}%
\end{pgfscope}%
\begin{pgfscope}%
\pgfsetrectcap%
\pgfsetmiterjoin%
\pgfsetlinewidth{0.803000pt}%
\definecolor{currentstroke}{rgb}{0.000000,0.000000,0.000000}%
\pgfsetstrokecolor{currentstroke}%
\pgfsetdash{}{0pt}%
\pgfpathmoveto{\pgfqpoint{4.956522in}{0.450000in}}%
\pgfpathlineto{\pgfqpoint{4.956522in}{2.760000in}}%
\pgfusepath{stroke}%
\end{pgfscope}%
\begin{pgfscope}%
\pgfsetrectcap%
\pgfsetmiterjoin%
\pgfsetlinewidth{0.803000pt}%
\definecolor{currentstroke}{rgb}{0.000000,0.000000,0.000000}%
\pgfsetstrokecolor{currentstroke}%
\pgfsetdash{}{0pt}%
\pgfpathmoveto{\pgfqpoint{8.000000in}{0.450000in}}%
\pgfpathlineto{\pgfqpoint{8.000000in}{2.760000in}}%
\pgfusepath{stroke}%
\end{pgfscope}%
\begin{pgfscope}%
\pgfsetrectcap%
\pgfsetmiterjoin%
\pgfsetlinewidth{0.803000pt}%
\definecolor{currentstroke}{rgb}{0.000000,0.000000,0.000000}%
\pgfsetstrokecolor{currentstroke}%
\pgfsetdash{}{0pt}%
\pgfpathmoveto{\pgfqpoint{4.956522in}{0.450000in}}%
\pgfpathlineto{\pgfqpoint{8.000000in}{0.450000in}}%
\pgfusepath{stroke}%
\end{pgfscope}%
\begin{pgfscope}%
\pgfsetrectcap%
\pgfsetmiterjoin%
\pgfsetlinewidth{0.803000pt}%
\definecolor{currentstroke}{rgb}{0.000000,0.000000,0.000000}%
\pgfsetstrokecolor{currentstroke}%
\pgfsetdash{}{0pt}%
\pgfpathmoveto{\pgfqpoint{4.956522in}{2.760000in}}%
\pgfpathlineto{\pgfqpoint{8.000000in}{2.760000in}}%
\pgfusepath{stroke}%
\end{pgfscope}%
\begin{pgfscope}%
\definecolor{textcolor}{rgb}{0.000000,0.000000,0.000000}%
\pgfsetstrokecolor{textcolor}%
\pgfsetfillcolor{textcolor}%
\pgftext[x=6.478261in,y=2.843333in,,base]{\color{textcolor}\sffamily\fontsize{14.400000}{17.280000}\selectfont \(\displaystyle \tau_l=20.0\si{ns},\,\sigma_l=5.0\si{ns}\)}%
\end{pgfscope}%
\begin{pgfscope}%
\pgfsetbuttcap%
\pgfsetmiterjoin%
\definecolor{currentfill}{rgb}{1.000000,1.000000,1.000000}%
\pgfsetfillcolor{currentfill}%
\pgfsetfillopacity{0.800000}%
\pgfsetlinewidth{1.003750pt}%
\definecolor{currentstroke}{rgb}{0.800000,0.800000,0.800000}%
\pgfsetstrokecolor{currentstroke}%
\pgfsetstrokeopacity{0.800000}%
\pgfsetdash{}{0pt}%
\pgfpathmoveto{\pgfqpoint{8.116667in}{1.466038in}}%
\pgfpathlineto{\pgfqpoint{9.400089in}{1.466038in}}%
\pgfpathquadraticcurveto{\pgfqpoint{9.433422in}{1.466038in}}{\pgfqpoint{9.433422in}{1.499372in}}%
\pgfpathlineto{\pgfqpoint{9.433422in}{2.412333in}}%
\pgfpathquadraticcurveto{\pgfqpoint{9.433422in}{2.445667in}}{\pgfqpoint{9.400089in}{2.445667in}}%
\pgfpathlineto{\pgfqpoint{8.116667in}{2.445667in}}%
\pgfpathquadraticcurveto{\pgfqpoint{8.083333in}{2.445667in}}{\pgfqpoint{8.083333in}{2.412333in}}%
\pgfpathlineto{\pgfqpoint{8.083333in}{1.499372in}}%
\pgfpathquadraticcurveto{\pgfqpoint{8.083333in}{1.466038in}}{\pgfqpoint{8.116667in}{1.466038in}}%
\pgfpathclose%
\pgfusepath{stroke,fill}%
\end{pgfscope}%
\begin{pgfscope}%
\pgfsetbuttcap%
\pgfsetroundjoin%
\pgfsetlinewidth{2.007500pt}%
\definecolor{currentstroke}{rgb}{0.750000,0.750000,0.000000}%
\pgfsetstrokecolor{currentstroke}%
\pgfsetdash{}{0pt}%
\pgfpathmoveto{\pgfqpoint{8.316667in}{2.237333in}}%
\pgfpathlineto{\pgfqpoint{8.316667in}{2.404000in}}%
\pgfusepath{stroke}%
\end{pgfscope}%
\begin{pgfscope}%
\pgfsetrectcap%
\pgfsetroundjoin%
\pgfsetlinewidth{2.007500pt}%
\definecolor{currentstroke}{rgb}{0.750000,0.750000,0.000000}%
\pgfsetstrokecolor{currentstroke}%
\pgfsetdash{}{0pt}%
\pgfpathmoveto{\pgfqpoint{8.150000in}{2.320667in}}%
\pgfpathlineto{\pgfqpoint{8.483333in}{2.320667in}}%
\pgfusepath{stroke}%
\end{pgfscope}%
\begin{pgfscope}%
\pgfsetbuttcap%
\pgfsetmiterjoin%
\definecolor{currentfill}{rgb}{0.750000,0.750000,0.000000}%
\pgfsetfillcolor{currentfill}%
\pgfsetlinewidth{1.003750pt}%
\definecolor{currentstroke}{rgb}{0.750000,0.750000,0.000000}%
\pgfsetstrokecolor{currentstroke}%
\pgfsetdash{}{0pt}%
\pgfsys@defobject{currentmarker}{\pgfqpoint{-0.026418in}{-0.022473in}}{\pgfqpoint{0.026418in}{0.027778in}}{%
\pgfpathmoveto{\pgfqpoint{0.000000in}{0.027778in}}%
\pgfpathlineto{\pgfqpoint{-0.026418in}{0.008584in}}%
\pgfpathlineto{\pgfqpoint{-0.016327in}{-0.022473in}}%
\pgfpathlineto{\pgfqpoint{0.016327in}{-0.022473in}}%
\pgfpathlineto{\pgfqpoint{0.026418in}{0.008584in}}%
\pgfpathclose%
\pgfusepath{stroke,fill}%
}%
\begin{pgfscope}%
\pgfsys@transformshift{8.316667in}{2.320667in}%
\pgfsys@useobject{currentmarker}{}%
\end{pgfscope}%
\end{pgfscope}%
\begin{pgfscope}%
\definecolor{textcolor}{rgb}{0.000000,0.000000,0.000000}%
\pgfsetstrokecolor{textcolor}%
\pgfsetfillcolor{textcolor}%
\pgftext[x=8.616667in,y=2.262333in,left,base]{\color{textcolor}\sffamily\fontsize{12.000000}{14.400000}\selectfont \(\displaystyle \mathrm{LucyDDM}\)}%
\end{pgfscope}%
\begin{pgfscope}%
\pgfsetbuttcap%
\pgfsetroundjoin%
\pgfsetlinewidth{2.007500pt}%
\definecolor{currentstroke}{rgb}{0.121569,0.466667,0.705882}%
\pgfsetstrokecolor{currentstroke}%
\pgfsetdash{}{0pt}%
\pgfpathmoveto{\pgfqpoint{8.316667in}{2.004926in}}%
\pgfpathlineto{\pgfqpoint{8.316667in}{2.171593in}}%
\pgfusepath{stroke}%
\end{pgfscope}%
\begin{pgfscope}%
\pgfsetrectcap%
\pgfsetroundjoin%
\pgfsetlinewidth{2.007500pt}%
\definecolor{currentstroke}{rgb}{0.121569,0.466667,0.705882}%
\pgfsetstrokecolor{currentstroke}%
\pgfsetdash{}{0pt}%
\pgfpathmoveto{\pgfqpoint{8.150000in}{2.088260in}}%
\pgfpathlineto{\pgfqpoint{8.483333in}{2.088260in}}%
\pgfusepath{stroke}%
\end{pgfscope}%
\begin{pgfscope}%
\pgfsetbuttcap%
\pgfsetroundjoin%
\definecolor{currentfill}{rgb}{0.121569,0.466667,0.705882}%
\pgfsetfillcolor{currentfill}%
\pgfsetlinewidth{1.003750pt}%
\definecolor{currentstroke}{rgb}{0.121569,0.466667,0.705882}%
\pgfsetstrokecolor{currentstroke}%
\pgfsetdash{}{0pt}%
\pgfsys@defobject{currentmarker}{\pgfqpoint{-0.027778in}{-0.027778in}}{\pgfqpoint{0.027778in}{0.027778in}}{%
\pgfpathmoveto{\pgfqpoint{-0.027778in}{-0.027778in}}%
\pgfpathlineto{\pgfqpoint{0.027778in}{0.027778in}}%
\pgfpathmoveto{\pgfqpoint{-0.027778in}{0.027778in}}%
\pgfpathlineto{\pgfqpoint{0.027778in}{-0.027778in}}%
\pgfusepath{stroke,fill}%
}%
\begin{pgfscope}%
\pgfsys@transformshift{8.316667in}{2.088260in}%
\pgfsys@useobject{currentmarker}{}%
\end{pgfscope}%
\end{pgfscope}%
\begin{pgfscope}%
\definecolor{textcolor}{rgb}{0.000000,0.000000,0.000000}%
\pgfsetstrokecolor{textcolor}%
\pgfsetfillcolor{textcolor}%
\pgftext[x=8.616667in,y=2.029926in,left,base]{\color{textcolor}\sffamily\fontsize{12.000000}{14.400000}\selectfont \(\displaystyle \mathrm{CNN}\)}%
\end{pgfscope}%
\begin{pgfscope}%
\pgfsetbuttcap%
\pgfsetroundjoin%
\pgfsetlinewidth{2.007500pt}%
\definecolor{currentstroke}{rgb}{0.000000,0.750000,0.750000}%
\pgfsetstrokecolor{currentstroke}%
\pgfsetdash{}{0pt}%
\pgfpathmoveto{\pgfqpoint{8.316667in}{1.772519in}}%
\pgfpathlineto{\pgfqpoint{8.316667in}{1.939186in}}%
\pgfusepath{stroke}%
\end{pgfscope}%
\begin{pgfscope}%
\pgfsetrectcap%
\pgfsetroundjoin%
\pgfsetlinewidth{2.007500pt}%
\definecolor{currentstroke}{rgb}{0.000000,0.750000,0.750000}%
\pgfsetstrokecolor{currentstroke}%
\pgfsetdash{}{0pt}%
\pgfpathmoveto{\pgfqpoint{8.150000in}{1.855853in}}%
\pgfpathlineto{\pgfqpoint{8.483333in}{1.855853in}}%
\pgfusepath{stroke}%
\end{pgfscope}%
\begin{pgfscope}%
\pgfsetbuttcap%
\pgfsetbeveljoin%
\definecolor{currentfill}{rgb}{0.000000,0.750000,0.750000}%
\pgfsetfillcolor{currentfill}%
\pgfsetlinewidth{1.003750pt}%
\definecolor{currentstroke}{rgb}{0.000000,0.750000,0.750000}%
\pgfsetstrokecolor{currentstroke}%
\pgfsetdash{}{0pt}%
\pgfsys@defobject{currentmarker}{\pgfqpoint{-0.026418in}{-0.022473in}}{\pgfqpoint{0.026418in}{0.027778in}}{%
\pgfpathmoveto{\pgfqpoint{0.000000in}{0.027778in}}%
\pgfpathlineto{\pgfqpoint{-0.006236in}{0.008584in}}%
\pgfpathlineto{\pgfqpoint{-0.026418in}{0.008584in}}%
\pgfpathlineto{\pgfqpoint{-0.010091in}{-0.003279in}}%
\pgfpathlineto{\pgfqpoint{-0.016327in}{-0.022473in}}%
\pgfpathlineto{\pgfqpoint{-0.000000in}{-0.010610in}}%
\pgfpathlineto{\pgfqpoint{0.016327in}{-0.022473in}}%
\pgfpathlineto{\pgfqpoint{0.010091in}{-0.003279in}}%
\pgfpathlineto{\pgfqpoint{0.026418in}{0.008584in}}%
\pgfpathlineto{\pgfqpoint{0.006236in}{0.008584in}}%
\pgfpathclose%
\pgfusepath{stroke,fill}%
}%
\begin{pgfscope}%
\pgfsys@transformshift{8.316667in}{1.855853in}%
\pgfsys@useobject{currentmarker}{}%
\end{pgfscope}%
\end{pgfscope}%
\begin{pgfscope}%
\definecolor{textcolor}{rgb}{0.000000,0.000000,0.000000}%
\pgfsetstrokecolor{textcolor}%
\pgfsetfillcolor{textcolor}%
\pgftext[x=8.616667in,y=1.797519in,left,base]{\color{textcolor}\sffamily\fontsize{12.000000}{14.400000}\selectfont \(\displaystyle \mathrm{Fitting}\)}%
\end{pgfscope}%
\begin{pgfscope}%
\pgfsetbuttcap%
\pgfsetroundjoin%
\pgfsetlinewidth{2.007500pt}%
\definecolor{currentstroke}{rgb}{1.000000,0.000000,0.000000}%
\pgfsetstrokecolor{currentstroke}%
\pgfsetdash{}{0pt}%
\pgfpathmoveto{\pgfqpoint{8.316667in}{1.540112in}}%
\pgfpathlineto{\pgfqpoint{8.316667in}{1.706779in}}%
\pgfusepath{stroke}%
\end{pgfscope}%
\begin{pgfscope}%
\pgfsetrectcap%
\pgfsetroundjoin%
\pgfsetlinewidth{2.007500pt}%
\definecolor{currentstroke}{rgb}{1.000000,0.000000,0.000000}%
\pgfsetstrokecolor{currentstroke}%
\pgfsetdash{}{0pt}%
\pgfpathmoveto{\pgfqpoint{8.150000in}{1.623445in}}%
\pgfpathlineto{\pgfqpoint{8.483333in}{1.623445in}}%
\pgfusepath{stroke}%
\end{pgfscope}%
\begin{pgfscope}%
\pgfsetbuttcap%
\pgfsetmiterjoin%
\definecolor{currentfill}{rgb}{1.000000,0.000000,0.000000}%
\pgfsetfillcolor{currentfill}%
\pgfsetlinewidth{1.003750pt}%
\definecolor{currentstroke}{rgb}{1.000000,0.000000,0.000000}%
\pgfsetstrokecolor{currentstroke}%
\pgfsetdash{}{0pt}%
\pgfsys@defobject{currentmarker}{\pgfqpoint{-0.027778in}{-0.027778in}}{\pgfqpoint{0.027778in}{0.027778in}}{%
\pgfpathmoveto{\pgfqpoint{-0.027778in}{-0.027778in}}%
\pgfpathlineto{\pgfqpoint{0.027778in}{-0.027778in}}%
\pgfpathlineto{\pgfqpoint{0.027778in}{0.027778in}}%
\pgfpathlineto{\pgfqpoint{-0.027778in}{0.027778in}}%
\pgfpathclose%
\pgfusepath{stroke,fill}%
}%
\begin{pgfscope}%
\pgfsys@transformshift{8.316667in}{1.623445in}%
\pgfsys@useobject{currentmarker}{}%
\end{pgfscope}%
\end{pgfscope}%
\begin{pgfscope}%
\definecolor{textcolor}{rgb}{0.000000,0.000000,0.000000}%
\pgfsetstrokecolor{textcolor}%
\pgfsetfillcolor{textcolor}%
\pgftext[x=8.616667in,y=1.565112in,left,base]{\color{textcolor}\sffamily\fontsize{12.000000}{14.400000}\selectfont \(\displaystyle \mathrm{FBMP}\)}%
\end{pgfscope}%
\end{pgfpicture}%
\makeatother%
\endgroup%
}
    \caption{\label{fig:deltamethods} Timing resolution ratio $\delta_{\mathrm{all}}/\delta_{\mathrm{1st}}$ of methods}
\end{figure}

In the figure, $\mathrm{MCMCcha}$ corresponds to reconstruction with recorded $\hat{q}$ during sampling and $\mathrm{MCMCt0}$ corresponds to reconstruction with $t_{0}$, which is directly collected in the MCMC method. $\mathrm{FBMPcha}$ corresponds to reconstruction with maximum posterior probability model's $\hat{q}$ and $\mathrm{FBMPt0}$ corresponds to the estimation of $t_{0}$ with all samples provided by FBMP. 

All 5 methods (LucyDDM, Fitting, MCMC, CNN, FBMP) provide better timing resolution $\delta$ than only using the first PE during construction. 

\subsection{Charge Reconstruction}

% section Discussion (end)

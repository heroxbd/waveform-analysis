Wasserstein distance $D_\mathrm{w}$ is a metric between two distributions, either of which can be discrete or continuous. It can capture the difference between a waveform analysis result $\hat{\phi}$ and the sampled light curve $\tilde{\phi}(t)$ in eq.~\eqref{eq:lc-sample}.
\begin{equation}
  D_\mathrm{w}\left[\hat{\phi}_*, \tilde{\phi}_*\right] = \inf_{\gamma \in \Gamma} \left[\int \left\vert t_1 - t_2 \right\vert \gamma(t_1, t_2)\mathrm{d}t_1\mathrm{d}t_2\right],
\end{equation}
where $*$ denotes the normalized light curves and $\Gamma$ is the collection of joint distributions with marginals $\hat{\phi}_*(t)$ and $\tilde{\phi}_*(t)$,
\begin{equation*}
  \label{eq:joint}
  \Gamma = \left\{\gamma(t_1, t_2) ~\middle\vert~ \int\gamma(t_1,t_2)\mathrm{d}t_1 = \tilde{\phi}_*(t_2) , \int\gamma(t_1,t_2)\mathrm{d}t_2 = \hat{\phi}_*(t_1) \right\}.
\end{equation*}
It is also known as the \textit{earth mover's distance}~\cite{levina_earth_2001}, ecoding the minimum cost to transport mass from one distribution to the other, as shown in figure~\ref{fig:l2}.

Alternatively, we can calculate $D_\mathrm{w}$ from cumulative distribution functions (CDF). Let $\hat\Phi(t)$ and $\tilde\Phi(t)$ denote the CDF of $\hat{\phi}_*(t)$ and $\tilde{\phi}_*(t)$, respectively. Then $D_w$ is equivalent to the $\ell_1$-metric between the two CDFs,
\begin{equation}
    D_\mathrm{w}\left[\hat{\phi}_*, \tilde{\phi}_*\right] = \int\left|\hat{\Phi}(t) - \tilde{\Phi}(t)\right| \mathrm{d}t.
    \label{eq:numerical}
\end{equation}
\documentclass{beamer}

\usetheme{Madrid}
\usecolortheme{default}

\definecolor{THUpurple}{RGB}{102,8,116}

\usepackage{amsmath}
\usepackage{mathtools}
\usepackage{caption}
\usepackage{listings}
\usepackage{lmodern}
\usepackage{xcolor}
\lstset{language=Python,keywordstyle={\bfseries \color{blue}}}
\usepackage{pdfpages}
\usepackage{makecell}
\usepackage[EULERGREEK]{sansmath}
\usepackage{float}
\usepackage{hyperref}
\usepackage{tikz}
\usetikzlibrary{shapes,arrows,positioning}
\usepackage[subrefformat=parens]{subcaption}
\usepackage[none]{hyphenat}
\usepackage[binary-units=true,per-mode=symbol]{siunitx}
\usepackage{CJK}
\usepackage{textcomp}
\usepackage{adjustbox}
\usepackage{pgfplots}
\usepackage{bm}
\usepackage{tablefootnote}
\DeclareMathOperator{\erf}{erf}
\usefonttheme[onlymath]{serif}

\newcommand{\dd}{\mathrm{d}}
\newcommand{\mev}{\mathrm{MeV}}
\newcommand{\gev}{\mathrm{GeV}}

\setbeamercolor{structure}{fg=THUpurple}
\setbeamersize{text margin left=10mm,text margin right=10mm}
% \setlength{\belowcaptionskip}{-2mm}
\title[Waveform Analysis]{Accurate and robust PMT waveform analysis with fast Bayesian matching pursuit}
\date[JUNO]{August 6, 2021}

\AtBeginSection[]
{
    \begin{frame}[noframenumbering]
        \frametitle{Outline}
        \thispagestyle{empty}
        \tableofcontents[currentsection]
    \end{frame}
}

\begin{document}
\captionsetup[figure]{labelfont={bf},name={Fig}}
\setbeamertemplate{frametitle}
{\begin{beamercolorbox}[wd=\paperwidth]{frametitle}
    \strut\hspace{0.5em}\insertframetitle\strut
    \hfill
    \raisebox{-2mm}{\includegraphics[width=1cm]{img/180px-Junologo.jpg}}
\end{beamercolorbox}
}
\tikzstyle{every picture}+=[remember picture]

\begin{CJK*}{UTF8}{gbsn}
\author[Dacheng Xu]{Dacheng~Xu~(徐大成) \and Erjin~Bao~(宝尔金) \and Yiyang~Wu~(武益阳) \and Benda~Xu~(续本达) \and Yu~Xu~(徐宇) \and Geliang~Zhang~(张戈亮) et.al \\ [4mm] \includegraphics[height=2cm]{img/Tsinghua_University_Logo.png}}

\frame{\titlepage}

\begin{frame}[noframenumbering]
\frametitle{Outline}
\thispagestyle{empty}
\tableofcontents
\end{frame}

\section{Motivation}

\begin{frame}
\frametitle{Motivation}
\begin{columns}
\column{0.425\textwidth}
\begin{figure}
    \centering
    \includegraphics[width=1.0\linewidth]{img/10-Figure7-1.png}
    \caption{An Event in JUNO Detector\cite{zhu_method_2019}}
\end{figure}
\column{0.575\textwidth}
\begin{figure}
    \centering
    \resizebox{\textwidth}{!}{%% Creator: Matplotlib, PGF backend
%%
%% To include the figure in your LaTeX document, write
%%   \input{<filename>.pgf}
%%
%% Make sure the required packages are loaded in your preamble
%%   \usepackage{pgf}
%%
%% and, on pdftex
%%   \usepackage[utf8]{inputenc}\DeclareUnicodeCharacter{2212}{-}
%%
%% or, on luatex and xetex
%%   \usepackage{unicode-math}
%%
%% Figures using additional raster images can only be included by \input if
%% they are in the same directory as the main LaTeX file. For loading figures
%% from other directories you can use the `import` package
%%   \usepackage{import}
%%
%% and then include the figures with
%%   \import{<path to file>}{<filename>.pgf}
%%
%% Matplotlib used the following preamble
%%   \usepackage[detect-all,locale=DE]{siunitx}
%%
\begingroup%
\makeatletter%
\begin{pgfpicture}%
\pgfpathrectangle{\pgfpointorigin}{\pgfqpoint{8.000000in}{6.000000in}}%
\pgfusepath{use as bounding box, clip}%
\begin{pgfscope}%
\pgfsetbuttcap%
\pgfsetmiterjoin%
\definecolor{currentfill}{rgb}{1.000000,1.000000,1.000000}%
\pgfsetfillcolor{currentfill}%
\pgfsetlinewidth{0.000000pt}%
\definecolor{currentstroke}{rgb}{1.000000,1.000000,1.000000}%
\pgfsetstrokecolor{currentstroke}%
\pgfsetdash{}{0pt}%
\pgfpathmoveto{\pgfqpoint{0.000000in}{0.000000in}}%
\pgfpathlineto{\pgfqpoint{8.000000in}{0.000000in}}%
\pgfpathlineto{\pgfqpoint{8.000000in}{6.000000in}}%
\pgfpathlineto{\pgfqpoint{0.000000in}{6.000000in}}%
\pgfpathclose%
\pgfusepath{fill}%
\end{pgfscope}%
\begin{pgfscope}%
\pgfsetbuttcap%
\pgfsetmiterjoin%
\definecolor{currentfill}{rgb}{1.000000,1.000000,1.000000}%
\pgfsetfillcolor{currentfill}%
\pgfsetlinewidth{0.000000pt}%
\definecolor{currentstroke}{rgb}{0.000000,0.000000,0.000000}%
\pgfsetstrokecolor{currentstroke}%
\pgfsetstrokeopacity{0.000000}%
\pgfsetdash{}{0pt}%
\pgfpathmoveto{\pgfqpoint{1.200000in}{0.900000in}}%
\pgfpathlineto{\pgfqpoint{6.800000in}{0.900000in}}%
\pgfpathlineto{\pgfqpoint{6.800000in}{5.700000in}}%
\pgfpathlineto{\pgfqpoint{1.200000in}{5.700000in}}%
\pgfpathclose%
\pgfusepath{fill}%
\end{pgfscope}%
\begin{pgfscope}%
\pgfsetbuttcap%
\pgfsetroundjoin%
\definecolor{currentfill}{rgb}{0.000000,0.000000,0.000000}%
\pgfsetfillcolor{currentfill}%
\pgfsetlinewidth{0.803000pt}%
\definecolor{currentstroke}{rgb}{0.000000,0.000000,0.000000}%
\pgfsetstrokecolor{currentstroke}%
\pgfsetdash{}{0pt}%
\pgfsys@defobject{currentmarker}{\pgfqpoint{0.000000in}{-0.048611in}}{\pgfqpoint{0.000000in}{0.000000in}}{%
\pgfpathmoveto{\pgfqpoint{0.000000in}{0.000000in}}%
\pgfpathlineto{\pgfqpoint{0.000000in}{-0.048611in}}%
\pgfusepath{stroke,fill}%
}%
\begin{pgfscope}%
\pgfsys@transformshift{1.200000in}{0.900000in}%
\pgfsys@useobject{currentmarker}{}%
\end{pgfscope}%
\end{pgfscope}%
\begin{pgfscope}%
\definecolor{textcolor}{rgb}{0.000000,0.000000,0.000000}%
\pgfsetstrokecolor{textcolor}%
\pgfsetfillcolor{textcolor}%
\pgftext[x=1.200000in,y=0.802778in,,top]{\color{textcolor}\sffamily\fontsize{20.000000}{24.000000}\selectfont \(\displaystyle {0}\)}%
\end{pgfscope}%
\begin{pgfscope}%
\pgfsetbuttcap%
\pgfsetroundjoin%
\definecolor{currentfill}{rgb}{0.000000,0.000000,0.000000}%
\pgfsetfillcolor{currentfill}%
\pgfsetlinewidth{0.803000pt}%
\definecolor{currentstroke}{rgb}{0.000000,0.000000,0.000000}%
\pgfsetstrokecolor{currentstroke}%
\pgfsetdash{}{0pt}%
\pgfsys@defobject{currentmarker}{\pgfqpoint{0.000000in}{-0.048611in}}{\pgfqpoint{0.000000in}{0.000000in}}{%
\pgfpathmoveto{\pgfqpoint{0.000000in}{0.000000in}}%
\pgfpathlineto{\pgfqpoint{0.000000in}{-0.048611in}}%
\pgfusepath{stroke,fill}%
}%
\begin{pgfscope}%
\pgfsys@transformshift{2.288435in}{0.900000in}%
\pgfsys@useobject{currentmarker}{}%
\end{pgfscope}%
\end{pgfscope}%
\begin{pgfscope}%
\definecolor{textcolor}{rgb}{0.000000,0.000000,0.000000}%
\pgfsetstrokecolor{textcolor}%
\pgfsetfillcolor{textcolor}%
\pgftext[x=2.288435in,y=0.802778in,,top]{\color{textcolor}\sffamily\fontsize{20.000000}{24.000000}\selectfont \(\displaystyle {200}\)}%
\end{pgfscope}%
\begin{pgfscope}%
\pgfsetbuttcap%
\pgfsetroundjoin%
\definecolor{currentfill}{rgb}{0.000000,0.000000,0.000000}%
\pgfsetfillcolor{currentfill}%
\pgfsetlinewidth{0.803000pt}%
\definecolor{currentstroke}{rgb}{0.000000,0.000000,0.000000}%
\pgfsetstrokecolor{currentstroke}%
\pgfsetdash{}{0pt}%
\pgfsys@defobject{currentmarker}{\pgfqpoint{0.000000in}{-0.048611in}}{\pgfqpoint{0.000000in}{0.000000in}}{%
\pgfpathmoveto{\pgfqpoint{0.000000in}{0.000000in}}%
\pgfpathlineto{\pgfqpoint{0.000000in}{-0.048611in}}%
\pgfusepath{stroke,fill}%
}%
\begin{pgfscope}%
\pgfsys@transformshift{3.376871in}{0.900000in}%
\pgfsys@useobject{currentmarker}{}%
\end{pgfscope}%
\end{pgfscope}%
\begin{pgfscope}%
\definecolor{textcolor}{rgb}{0.000000,0.000000,0.000000}%
\pgfsetstrokecolor{textcolor}%
\pgfsetfillcolor{textcolor}%
\pgftext[x=3.376871in,y=0.802778in,,top]{\color{textcolor}\sffamily\fontsize{20.000000}{24.000000}\selectfont \(\displaystyle {400}\)}%
\end{pgfscope}%
\begin{pgfscope}%
\pgfsetbuttcap%
\pgfsetroundjoin%
\definecolor{currentfill}{rgb}{0.000000,0.000000,0.000000}%
\pgfsetfillcolor{currentfill}%
\pgfsetlinewidth{0.803000pt}%
\definecolor{currentstroke}{rgb}{0.000000,0.000000,0.000000}%
\pgfsetstrokecolor{currentstroke}%
\pgfsetdash{}{0pt}%
\pgfsys@defobject{currentmarker}{\pgfqpoint{0.000000in}{-0.048611in}}{\pgfqpoint{0.000000in}{0.000000in}}{%
\pgfpathmoveto{\pgfqpoint{0.000000in}{0.000000in}}%
\pgfpathlineto{\pgfqpoint{0.000000in}{-0.048611in}}%
\pgfusepath{stroke,fill}%
}%
\begin{pgfscope}%
\pgfsys@transformshift{4.465306in}{0.900000in}%
\pgfsys@useobject{currentmarker}{}%
\end{pgfscope}%
\end{pgfscope}%
\begin{pgfscope}%
\definecolor{textcolor}{rgb}{0.000000,0.000000,0.000000}%
\pgfsetstrokecolor{textcolor}%
\pgfsetfillcolor{textcolor}%
\pgftext[x=4.465306in,y=0.802778in,,top]{\color{textcolor}\sffamily\fontsize{20.000000}{24.000000}\selectfont \(\displaystyle {600}\)}%
\end{pgfscope}%
\begin{pgfscope}%
\pgfsetbuttcap%
\pgfsetroundjoin%
\definecolor{currentfill}{rgb}{0.000000,0.000000,0.000000}%
\pgfsetfillcolor{currentfill}%
\pgfsetlinewidth{0.803000pt}%
\definecolor{currentstroke}{rgb}{0.000000,0.000000,0.000000}%
\pgfsetstrokecolor{currentstroke}%
\pgfsetdash{}{0pt}%
\pgfsys@defobject{currentmarker}{\pgfqpoint{0.000000in}{-0.048611in}}{\pgfqpoint{0.000000in}{0.000000in}}{%
\pgfpathmoveto{\pgfqpoint{0.000000in}{0.000000in}}%
\pgfpathlineto{\pgfqpoint{0.000000in}{-0.048611in}}%
\pgfusepath{stroke,fill}%
}%
\begin{pgfscope}%
\pgfsys@transformshift{5.553741in}{0.900000in}%
\pgfsys@useobject{currentmarker}{}%
\end{pgfscope}%
\end{pgfscope}%
\begin{pgfscope}%
\definecolor{textcolor}{rgb}{0.000000,0.000000,0.000000}%
\pgfsetstrokecolor{textcolor}%
\pgfsetfillcolor{textcolor}%
\pgftext[x=5.553741in,y=0.802778in,,top]{\color{textcolor}\sffamily\fontsize{20.000000}{24.000000}\selectfont \(\displaystyle {800}\)}%
\end{pgfscope}%
\begin{pgfscope}%
\pgfsetbuttcap%
\pgfsetroundjoin%
\definecolor{currentfill}{rgb}{0.000000,0.000000,0.000000}%
\pgfsetfillcolor{currentfill}%
\pgfsetlinewidth{0.803000pt}%
\definecolor{currentstroke}{rgb}{0.000000,0.000000,0.000000}%
\pgfsetstrokecolor{currentstroke}%
\pgfsetdash{}{0pt}%
\pgfsys@defobject{currentmarker}{\pgfqpoint{0.000000in}{-0.048611in}}{\pgfqpoint{0.000000in}{0.000000in}}{%
\pgfpathmoveto{\pgfqpoint{0.000000in}{0.000000in}}%
\pgfpathlineto{\pgfqpoint{0.000000in}{-0.048611in}}%
\pgfusepath{stroke,fill}%
}%
\begin{pgfscope}%
\pgfsys@transformshift{6.642177in}{0.900000in}%
\pgfsys@useobject{currentmarker}{}%
\end{pgfscope}%
\end{pgfscope}%
\begin{pgfscope}%
\definecolor{textcolor}{rgb}{0.000000,0.000000,0.000000}%
\pgfsetstrokecolor{textcolor}%
\pgfsetfillcolor{textcolor}%
\pgftext[x=6.642177in,y=0.802778in,,top]{\color{textcolor}\sffamily\fontsize{20.000000}{24.000000}\selectfont \(\displaystyle {1000}\)}%
\end{pgfscope}%
\begin{pgfscope}%
\definecolor{textcolor}{rgb}{0.000000,0.000000,0.000000}%
\pgfsetstrokecolor{textcolor}%
\pgfsetfillcolor{textcolor}%
\pgftext[x=4.000000in,y=0.491155in,,top]{\color{textcolor}\sffamily\fontsize{20.000000}{24.000000}\selectfont \(\displaystyle \mathrm{t}/\si{ns}\)}%
\end{pgfscope}%
\begin{pgfscope}%
\pgfsetbuttcap%
\pgfsetroundjoin%
\definecolor{currentfill}{rgb}{0.000000,0.000000,0.000000}%
\pgfsetfillcolor{currentfill}%
\pgfsetlinewidth{0.803000pt}%
\definecolor{currentstroke}{rgb}{0.000000,0.000000,0.000000}%
\pgfsetstrokecolor{currentstroke}%
\pgfsetdash{}{0pt}%
\pgfsys@defobject{currentmarker}{\pgfqpoint{-0.048611in}{0.000000in}}{\pgfqpoint{-0.000000in}{0.000000in}}{%
\pgfpathmoveto{\pgfqpoint{-0.000000in}{0.000000in}}%
\pgfpathlineto{\pgfqpoint{-0.048611in}{0.000000in}}%
\pgfusepath{stroke,fill}%
}%
\begin{pgfscope}%
\pgfsys@transformshift{1.200000in}{0.900000in}%
\pgfsys@useobject{currentmarker}{}%
\end{pgfscope}%
\end{pgfscope}%
\begin{pgfscope}%
\definecolor{textcolor}{rgb}{0.000000,0.000000,0.000000}%
\pgfsetstrokecolor{textcolor}%
\pgfsetfillcolor{textcolor}%
\pgftext[x=0.970670in, y=0.799981in, left, base]{\color{textcolor}\sffamily\fontsize{20.000000}{24.000000}\selectfont \(\displaystyle {0}\)}%
\end{pgfscope}%
\begin{pgfscope}%
\pgfsetbuttcap%
\pgfsetroundjoin%
\definecolor{currentfill}{rgb}{0.000000,0.000000,0.000000}%
\pgfsetfillcolor{currentfill}%
\pgfsetlinewidth{0.803000pt}%
\definecolor{currentstroke}{rgb}{0.000000,0.000000,0.000000}%
\pgfsetstrokecolor{currentstroke}%
\pgfsetdash{}{0pt}%
\pgfsys@defobject{currentmarker}{\pgfqpoint{-0.048611in}{0.000000in}}{\pgfqpoint{-0.000000in}{0.000000in}}{%
\pgfpathmoveto{\pgfqpoint{-0.000000in}{0.000000in}}%
\pgfpathlineto{\pgfqpoint{-0.048611in}{0.000000in}}%
\pgfusepath{stroke,fill}%
}%
\begin{pgfscope}%
\pgfsys@transformshift{1.200000in}{1.684122in}%
\pgfsys@useobject{currentmarker}{}%
\end{pgfscope}%
\end{pgfscope}%
\begin{pgfscope}%
\definecolor{textcolor}{rgb}{0.000000,0.000000,0.000000}%
\pgfsetstrokecolor{textcolor}%
\pgfsetfillcolor{textcolor}%
\pgftext[x=0.970670in, y=1.584102in, left, base]{\color{textcolor}\sffamily\fontsize{20.000000}{24.000000}\selectfont \(\displaystyle {5}\)}%
\end{pgfscope}%
\begin{pgfscope}%
\pgfsetbuttcap%
\pgfsetroundjoin%
\definecolor{currentfill}{rgb}{0.000000,0.000000,0.000000}%
\pgfsetfillcolor{currentfill}%
\pgfsetlinewidth{0.803000pt}%
\definecolor{currentstroke}{rgb}{0.000000,0.000000,0.000000}%
\pgfsetstrokecolor{currentstroke}%
\pgfsetdash{}{0pt}%
\pgfsys@defobject{currentmarker}{\pgfqpoint{-0.048611in}{0.000000in}}{\pgfqpoint{-0.000000in}{0.000000in}}{%
\pgfpathmoveto{\pgfqpoint{-0.000000in}{0.000000in}}%
\pgfpathlineto{\pgfqpoint{-0.048611in}{0.000000in}}%
\pgfusepath{stroke,fill}%
}%
\begin{pgfscope}%
\pgfsys@transformshift{1.200000in}{2.468243in}%
\pgfsys@useobject{currentmarker}{}%
\end{pgfscope}%
\end{pgfscope}%
\begin{pgfscope}%
\definecolor{textcolor}{rgb}{0.000000,0.000000,0.000000}%
\pgfsetstrokecolor{textcolor}%
\pgfsetfillcolor{textcolor}%
\pgftext[x=0.838563in, y=2.368224in, left, base]{\color{textcolor}\sffamily\fontsize{20.000000}{24.000000}\selectfont \(\displaystyle {10}\)}%
\end{pgfscope}%
\begin{pgfscope}%
\pgfsetbuttcap%
\pgfsetroundjoin%
\definecolor{currentfill}{rgb}{0.000000,0.000000,0.000000}%
\pgfsetfillcolor{currentfill}%
\pgfsetlinewidth{0.803000pt}%
\definecolor{currentstroke}{rgb}{0.000000,0.000000,0.000000}%
\pgfsetstrokecolor{currentstroke}%
\pgfsetdash{}{0pt}%
\pgfsys@defobject{currentmarker}{\pgfqpoint{-0.048611in}{0.000000in}}{\pgfqpoint{-0.000000in}{0.000000in}}{%
\pgfpathmoveto{\pgfqpoint{-0.000000in}{0.000000in}}%
\pgfpathlineto{\pgfqpoint{-0.048611in}{0.000000in}}%
\pgfusepath{stroke,fill}%
}%
\begin{pgfscope}%
\pgfsys@transformshift{1.200000in}{3.252365in}%
\pgfsys@useobject{currentmarker}{}%
\end{pgfscope}%
\end{pgfscope}%
\begin{pgfscope}%
\definecolor{textcolor}{rgb}{0.000000,0.000000,0.000000}%
\pgfsetstrokecolor{textcolor}%
\pgfsetfillcolor{textcolor}%
\pgftext[x=0.838563in, y=3.152345in, left, base]{\color{textcolor}\sffamily\fontsize{20.000000}{24.000000}\selectfont \(\displaystyle {15}\)}%
\end{pgfscope}%
\begin{pgfscope}%
\pgfsetbuttcap%
\pgfsetroundjoin%
\definecolor{currentfill}{rgb}{0.000000,0.000000,0.000000}%
\pgfsetfillcolor{currentfill}%
\pgfsetlinewidth{0.803000pt}%
\definecolor{currentstroke}{rgb}{0.000000,0.000000,0.000000}%
\pgfsetstrokecolor{currentstroke}%
\pgfsetdash{}{0pt}%
\pgfsys@defobject{currentmarker}{\pgfqpoint{-0.048611in}{0.000000in}}{\pgfqpoint{-0.000000in}{0.000000in}}{%
\pgfpathmoveto{\pgfqpoint{-0.000000in}{0.000000in}}%
\pgfpathlineto{\pgfqpoint{-0.048611in}{0.000000in}}%
\pgfusepath{stroke,fill}%
}%
\begin{pgfscope}%
\pgfsys@transformshift{1.200000in}{4.036486in}%
\pgfsys@useobject{currentmarker}{}%
\end{pgfscope}%
\end{pgfscope}%
\begin{pgfscope}%
\definecolor{textcolor}{rgb}{0.000000,0.000000,0.000000}%
\pgfsetstrokecolor{textcolor}%
\pgfsetfillcolor{textcolor}%
\pgftext[x=0.838563in, y=3.936467in, left, base]{\color{textcolor}\sffamily\fontsize{20.000000}{24.000000}\selectfont \(\displaystyle {20}\)}%
\end{pgfscope}%
\begin{pgfscope}%
\pgfsetbuttcap%
\pgfsetroundjoin%
\definecolor{currentfill}{rgb}{0.000000,0.000000,0.000000}%
\pgfsetfillcolor{currentfill}%
\pgfsetlinewidth{0.803000pt}%
\definecolor{currentstroke}{rgb}{0.000000,0.000000,0.000000}%
\pgfsetstrokecolor{currentstroke}%
\pgfsetdash{}{0pt}%
\pgfsys@defobject{currentmarker}{\pgfqpoint{-0.048611in}{0.000000in}}{\pgfqpoint{-0.000000in}{0.000000in}}{%
\pgfpathmoveto{\pgfqpoint{-0.000000in}{0.000000in}}%
\pgfpathlineto{\pgfqpoint{-0.048611in}{0.000000in}}%
\pgfusepath{stroke,fill}%
}%
\begin{pgfscope}%
\pgfsys@transformshift{1.200000in}{4.820608in}%
\pgfsys@useobject{currentmarker}{}%
\end{pgfscope}%
\end{pgfscope}%
\begin{pgfscope}%
\definecolor{textcolor}{rgb}{0.000000,0.000000,0.000000}%
\pgfsetstrokecolor{textcolor}%
\pgfsetfillcolor{textcolor}%
\pgftext[x=0.838563in, y=4.720588in, left, base]{\color{textcolor}\sffamily\fontsize{20.000000}{24.000000}\selectfont \(\displaystyle {25}\)}%
\end{pgfscope}%
\begin{pgfscope}%
\pgfsetbuttcap%
\pgfsetroundjoin%
\definecolor{currentfill}{rgb}{0.000000,0.000000,0.000000}%
\pgfsetfillcolor{currentfill}%
\pgfsetlinewidth{0.803000pt}%
\definecolor{currentstroke}{rgb}{0.000000,0.000000,0.000000}%
\pgfsetstrokecolor{currentstroke}%
\pgfsetdash{}{0pt}%
\pgfsys@defobject{currentmarker}{\pgfqpoint{-0.048611in}{0.000000in}}{\pgfqpoint{-0.000000in}{0.000000in}}{%
\pgfpathmoveto{\pgfqpoint{-0.000000in}{0.000000in}}%
\pgfpathlineto{\pgfqpoint{-0.048611in}{0.000000in}}%
\pgfusepath{stroke,fill}%
}%
\begin{pgfscope}%
\pgfsys@transformshift{1.200000in}{5.604729in}%
\pgfsys@useobject{currentmarker}{}%
\end{pgfscope}%
\end{pgfscope}%
\begin{pgfscope}%
\definecolor{textcolor}{rgb}{0.000000,0.000000,0.000000}%
\pgfsetstrokecolor{textcolor}%
\pgfsetfillcolor{textcolor}%
\pgftext[x=0.838563in, y=5.504710in, left, base]{\color{textcolor}\sffamily\fontsize{20.000000}{24.000000}\selectfont \(\displaystyle {30}\)}%
\end{pgfscope}%
\begin{pgfscope}%
\definecolor{textcolor}{rgb}{0.000000,0.000000,0.000000}%
\pgfsetstrokecolor{textcolor}%
\pgfsetfillcolor{textcolor}%
\pgftext[x=0.783007in,y=3.300000in,,bottom,rotate=90.000000]{\color{textcolor}\sffamily\fontsize{20.000000}{24.000000}\selectfont \(\displaystyle \mathrm{Voltage}/\si{mV}\)}%
\end{pgfscope}%
\begin{pgfscope}%
\pgfpathrectangle{\pgfqpoint{1.200000in}{0.900000in}}{\pgfqpoint{5.600000in}{4.800000in}}%
\pgfusepath{clip}%
\pgfsetrectcap%
\pgfsetroundjoin%
\pgfsetlinewidth{2.007500pt}%
\definecolor{currentstroke}{rgb}{0.121569,0.466667,0.705882}%
\pgfsetstrokecolor{currentstroke}%
\pgfsetdash{}{0pt}%
\pgfpathmoveto{\pgfqpoint{1.200000in}{4.663783in}}%
\pgfpathlineto{\pgfqpoint{1.205442in}{4.820608in}}%
\pgfpathlineto{\pgfqpoint{1.210884in}{4.820608in}}%
\pgfpathlineto{\pgfqpoint{1.216327in}{4.663783in}}%
\pgfpathlineto{\pgfqpoint{1.221769in}{4.820608in}}%
\pgfpathlineto{\pgfqpoint{1.227211in}{5.134256in}}%
\pgfpathlineto{\pgfqpoint{1.232653in}{4.820608in}}%
\pgfpathlineto{\pgfqpoint{1.238095in}{4.977432in}}%
\pgfpathlineto{\pgfqpoint{1.243537in}{4.820608in}}%
\pgfpathlineto{\pgfqpoint{1.248980in}{4.977432in}}%
\pgfpathlineto{\pgfqpoint{1.254422in}{4.820608in}}%
\pgfpathlineto{\pgfqpoint{1.259864in}{4.977432in}}%
\pgfpathlineto{\pgfqpoint{1.270748in}{4.977432in}}%
\pgfpathlineto{\pgfqpoint{1.276190in}{5.134256in}}%
\pgfpathlineto{\pgfqpoint{1.287075in}{4.506959in}}%
\pgfpathlineto{\pgfqpoint{1.292517in}{5.134256in}}%
\pgfpathlineto{\pgfqpoint{1.297959in}{4.820608in}}%
\pgfpathlineto{\pgfqpoint{1.303401in}{4.820608in}}%
\pgfpathlineto{\pgfqpoint{1.308844in}{4.663783in}}%
\pgfpathlineto{\pgfqpoint{1.314286in}{4.977432in}}%
\pgfpathlineto{\pgfqpoint{1.325170in}{4.977432in}}%
\pgfpathlineto{\pgfqpoint{1.336054in}{4.663783in}}%
\pgfpathlineto{\pgfqpoint{1.341497in}{4.820608in}}%
\pgfpathlineto{\pgfqpoint{1.346939in}{4.820608in}}%
\pgfpathlineto{\pgfqpoint{1.352381in}{5.134256in}}%
\pgfpathlineto{\pgfqpoint{1.357823in}{5.134256in}}%
\pgfpathlineto{\pgfqpoint{1.363265in}{4.977432in}}%
\pgfpathlineto{\pgfqpoint{1.368707in}{4.663783in}}%
\pgfpathlineto{\pgfqpoint{1.374150in}{4.663783in}}%
\pgfpathlineto{\pgfqpoint{1.379592in}{4.977432in}}%
\pgfpathlineto{\pgfqpoint{1.390476in}{4.663783in}}%
\pgfpathlineto{\pgfqpoint{1.395918in}{4.820608in}}%
\pgfpathlineto{\pgfqpoint{1.412245in}{4.820608in}}%
\pgfpathlineto{\pgfqpoint{1.417687in}{4.977432in}}%
\pgfpathlineto{\pgfqpoint{1.423129in}{4.663783in}}%
\pgfpathlineto{\pgfqpoint{1.428571in}{5.134256in}}%
\pgfpathlineto{\pgfqpoint{1.434014in}{4.506959in}}%
\pgfpathlineto{\pgfqpoint{1.439456in}{4.977432in}}%
\pgfpathlineto{\pgfqpoint{1.444898in}{4.820608in}}%
\pgfpathlineto{\pgfqpoint{1.450340in}{4.820608in}}%
\pgfpathlineto{\pgfqpoint{1.455782in}{4.663783in}}%
\pgfpathlineto{\pgfqpoint{1.461224in}{4.977432in}}%
\pgfpathlineto{\pgfqpoint{1.466667in}{4.663783in}}%
\pgfpathlineto{\pgfqpoint{1.472109in}{5.291081in}}%
\pgfpathlineto{\pgfqpoint{1.477551in}{4.977432in}}%
\pgfpathlineto{\pgfqpoint{1.482993in}{4.977432in}}%
\pgfpathlineto{\pgfqpoint{1.488435in}{4.820608in}}%
\pgfpathlineto{\pgfqpoint{1.504762in}{4.820608in}}%
\pgfpathlineto{\pgfqpoint{1.510204in}{4.506959in}}%
\pgfpathlineto{\pgfqpoint{1.515646in}{4.820608in}}%
\pgfpathlineto{\pgfqpoint{1.521088in}{4.820608in}}%
\pgfpathlineto{\pgfqpoint{1.526531in}{4.977432in}}%
\pgfpathlineto{\pgfqpoint{1.531973in}{4.820608in}}%
\pgfpathlineto{\pgfqpoint{1.537415in}{4.977432in}}%
\pgfpathlineto{\pgfqpoint{1.542857in}{4.977432in}}%
\pgfpathlineto{\pgfqpoint{1.548299in}{4.820608in}}%
\pgfpathlineto{\pgfqpoint{1.553741in}{4.977432in}}%
\pgfpathlineto{\pgfqpoint{1.559184in}{4.820608in}}%
\pgfpathlineto{\pgfqpoint{1.564626in}{4.820608in}}%
\pgfpathlineto{\pgfqpoint{1.570068in}{5.134256in}}%
\pgfpathlineto{\pgfqpoint{1.575510in}{4.663783in}}%
\pgfpathlineto{\pgfqpoint{1.580952in}{4.506959in}}%
\pgfpathlineto{\pgfqpoint{1.586395in}{4.977432in}}%
\pgfpathlineto{\pgfqpoint{1.591837in}{4.977432in}}%
\pgfpathlineto{\pgfqpoint{1.597279in}{4.663783in}}%
\pgfpathlineto{\pgfqpoint{1.602721in}{4.663783in}}%
\pgfpathlineto{\pgfqpoint{1.608163in}{4.820608in}}%
\pgfpathlineto{\pgfqpoint{1.613605in}{4.663783in}}%
\pgfpathlineto{\pgfqpoint{1.624490in}{4.977432in}}%
\pgfpathlineto{\pgfqpoint{1.629932in}{4.820608in}}%
\pgfpathlineto{\pgfqpoint{1.635374in}{4.820608in}}%
\pgfpathlineto{\pgfqpoint{1.640816in}{4.977432in}}%
\pgfpathlineto{\pgfqpoint{1.646259in}{4.663783in}}%
\pgfpathlineto{\pgfqpoint{1.651701in}{4.820608in}}%
\pgfpathlineto{\pgfqpoint{1.657143in}{4.663783in}}%
\pgfpathlineto{\pgfqpoint{1.662585in}{4.820608in}}%
\pgfpathlineto{\pgfqpoint{1.668027in}{4.820608in}}%
\pgfpathlineto{\pgfqpoint{1.673469in}{4.663783in}}%
\pgfpathlineto{\pgfqpoint{1.678912in}{4.977432in}}%
\pgfpathlineto{\pgfqpoint{1.684354in}{4.820608in}}%
\pgfpathlineto{\pgfqpoint{1.689796in}{4.977432in}}%
\pgfpathlineto{\pgfqpoint{1.695238in}{4.663783in}}%
\pgfpathlineto{\pgfqpoint{1.700680in}{4.820608in}}%
\pgfpathlineto{\pgfqpoint{1.706122in}{4.663783in}}%
\pgfpathlineto{\pgfqpoint{1.711565in}{4.663783in}}%
\pgfpathlineto{\pgfqpoint{1.717007in}{4.820608in}}%
\pgfpathlineto{\pgfqpoint{1.722449in}{4.820608in}}%
\pgfpathlineto{\pgfqpoint{1.727891in}{4.663783in}}%
\pgfpathlineto{\pgfqpoint{1.733333in}{4.820608in}}%
\pgfpathlineto{\pgfqpoint{1.738776in}{5.134256in}}%
\pgfpathlineto{\pgfqpoint{1.744218in}{4.663783in}}%
\pgfpathlineto{\pgfqpoint{1.755102in}{4.977432in}}%
\pgfpathlineto{\pgfqpoint{1.760544in}{4.820608in}}%
\pgfpathlineto{\pgfqpoint{1.765986in}{5.134256in}}%
\pgfpathlineto{\pgfqpoint{1.771429in}{4.663783in}}%
\pgfpathlineto{\pgfqpoint{1.776871in}{4.506959in}}%
\pgfpathlineto{\pgfqpoint{1.782313in}{4.820608in}}%
\pgfpathlineto{\pgfqpoint{1.787755in}{4.663783in}}%
\pgfpathlineto{\pgfqpoint{1.793197in}{4.977432in}}%
\pgfpathlineto{\pgfqpoint{1.798639in}{4.977432in}}%
\pgfpathlineto{\pgfqpoint{1.804082in}{4.820608in}}%
\pgfpathlineto{\pgfqpoint{1.809524in}{4.820608in}}%
\pgfpathlineto{\pgfqpoint{1.814966in}{4.663783in}}%
\pgfpathlineto{\pgfqpoint{1.820408in}{5.134256in}}%
\pgfpathlineto{\pgfqpoint{1.825850in}{4.820608in}}%
\pgfpathlineto{\pgfqpoint{1.831293in}{5.134256in}}%
\pgfpathlineto{\pgfqpoint{1.836735in}{4.977432in}}%
\pgfpathlineto{\pgfqpoint{1.842177in}{4.663783in}}%
\pgfpathlineto{\pgfqpoint{1.847619in}{4.820608in}}%
\pgfpathlineto{\pgfqpoint{1.853061in}{4.820608in}}%
\pgfpathlineto{\pgfqpoint{1.858503in}{4.977432in}}%
\pgfpathlineto{\pgfqpoint{1.863946in}{4.977432in}}%
\pgfpathlineto{\pgfqpoint{1.869388in}{4.820608in}}%
\pgfpathlineto{\pgfqpoint{1.874830in}{4.820608in}}%
\pgfpathlineto{\pgfqpoint{1.880272in}{4.977432in}}%
\pgfpathlineto{\pgfqpoint{1.885714in}{4.820608in}}%
\pgfpathlineto{\pgfqpoint{1.891156in}{4.506959in}}%
\pgfpathlineto{\pgfqpoint{1.896599in}{4.820608in}}%
\pgfpathlineto{\pgfqpoint{1.902041in}{4.820608in}}%
\pgfpathlineto{\pgfqpoint{1.907483in}{4.506959in}}%
\pgfpathlineto{\pgfqpoint{1.912925in}{4.663783in}}%
\pgfpathlineto{\pgfqpoint{1.918367in}{4.977432in}}%
\pgfpathlineto{\pgfqpoint{1.923810in}{4.820608in}}%
\pgfpathlineto{\pgfqpoint{1.929252in}{4.820608in}}%
\pgfpathlineto{\pgfqpoint{1.934694in}{4.977432in}}%
\pgfpathlineto{\pgfqpoint{1.940136in}{4.820608in}}%
\pgfpathlineto{\pgfqpoint{1.945578in}{4.820608in}}%
\pgfpathlineto{\pgfqpoint{1.951020in}{4.977432in}}%
\pgfpathlineto{\pgfqpoint{1.956463in}{4.663783in}}%
\pgfpathlineto{\pgfqpoint{1.967347in}{4.977432in}}%
\pgfpathlineto{\pgfqpoint{1.972789in}{4.977432in}}%
\pgfpathlineto{\pgfqpoint{1.978231in}{4.506959in}}%
\pgfpathlineto{\pgfqpoint{1.983673in}{4.820608in}}%
\pgfpathlineto{\pgfqpoint{1.989116in}{4.820608in}}%
\pgfpathlineto{\pgfqpoint{1.994558in}{4.977432in}}%
\pgfpathlineto{\pgfqpoint{2.000000in}{4.977432in}}%
\pgfpathlineto{\pgfqpoint{2.010884in}{4.663783in}}%
\pgfpathlineto{\pgfqpoint{2.016327in}{4.977432in}}%
\pgfpathlineto{\pgfqpoint{2.027211in}{4.663783in}}%
\pgfpathlineto{\pgfqpoint{2.032653in}{4.977432in}}%
\pgfpathlineto{\pgfqpoint{2.043537in}{4.977432in}}%
\pgfpathlineto{\pgfqpoint{2.048980in}{4.663783in}}%
\pgfpathlineto{\pgfqpoint{2.054422in}{4.820608in}}%
\pgfpathlineto{\pgfqpoint{2.059864in}{4.820608in}}%
\pgfpathlineto{\pgfqpoint{2.065306in}{4.977432in}}%
\pgfpathlineto{\pgfqpoint{2.070748in}{4.663783in}}%
\pgfpathlineto{\pgfqpoint{2.081633in}{4.977432in}}%
\pgfpathlineto{\pgfqpoint{2.087075in}{4.820608in}}%
\pgfpathlineto{\pgfqpoint{2.092517in}{5.134256in}}%
\pgfpathlineto{\pgfqpoint{2.097959in}{4.663783in}}%
\pgfpathlineto{\pgfqpoint{2.103401in}{4.820608in}}%
\pgfpathlineto{\pgfqpoint{2.114286in}{4.820608in}}%
\pgfpathlineto{\pgfqpoint{2.119728in}{4.663783in}}%
\pgfpathlineto{\pgfqpoint{2.125170in}{4.820608in}}%
\pgfpathlineto{\pgfqpoint{2.130612in}{4.663783in}}%
\pgfpathlineto{\pgfqpoint{2.136054in}{4.820608in}}%
\pgfpathlineto{\pgfqpoint{2.141497in}{4.506959in}}%
\pgfpathlineto{\pgfqpoint{2.146939in}{4.663783in}}%
\pgfpathlineto{\pgfqpoint{2.152381in}{4.977432in}}%
\pgfpathlineto{\pgfqpoint{2.163265in}{4.663783in}}%
\pgfpathlineto{\pgfqpoint{2.174150in}{4.977432in}}%
\pgfpathlineto{\pgfqpoint{2.185034in}{4.977432in}}%
\pgfpathlineto{\pgfqpoint{2.195918in}{4.663783in}}%
\pgfpathlineto{\pgfqpoint{2.206803in}{4.977432in}}%
\pgfpathlineto{\pgfqpoint{2.212245in}{4.663783in}}%
\pgfpathlineto{\pgfqpoint{2.217687in}{4.663783in}}%
\pgfpathlineto{\pgfqpoint{2.223129in}{4.820608in}}%
\pgfpathlineto{\pgfqpoint{2.228571in}{4.663783in}}%
\pgfpathlineto{\pgfqpoint{2.234014in}{4.820608in}}%
\pgfpathlineto{\pgfqpoint{2.239456in}{4.820608in}}%
\pgfpathlineto{\pgfqpoint{2.244898in}{4.977432in}}%
\pgfpathlineto{\pgfqpoint{2.250340in}{4.977432in}}%
\pgfpathlineto{\pgfqpoint{2.255782in}{4.506959in}}%
\pgfpathlineto{\pgfqpoint{2.261224in}{4.977432in}}%
\pgfpathlineto{\pgfqpoint{2.266667in}{4.820608in}}%
\pgfpathlineto{\pgfqpoint{2.272109in}{4.506959in}}%
\pgfpathlineto{\pgfqpoint{2.277551in}{4.820608in}}%
\pgfpathlineto{\pgfqpoint{2.282993in}{4.820608in}}%
\pgfpathlineto{\pgfqpoint{2.288435in}{4.506959in}}%
\pgfpathlineto{\pgfqpoint{2.293878in}{4.820608in}}%
\pgfpathlineto{\pgfqpoint{2.299320in}{4.350135in}}%
\pgfpathlineto{\pgfqpoint{2.310204in}{3.722838in}}%
\pgfpathlineto{\pgfqpoint{2.315646in}{3.722838in}}%
\pgfpathlineto{\pgfqpoint{2.326531in}{3.409189in}}%
\pgfpathlineto{\pgfqpoint{2.331973in}{3.409189in}}%
\pgfpathlineto{\pgfqpoint{2.348299in}{3.879662in}}%
\pgfpathlineto{\pgfqpoint{2.353741in}{4.193310in}}%
\pgfpathlineto{\pgfqpoint{2.359184in}{3.879662in}}%
\pgfpathlineto{\pgfqpoint{2.364626in}{4.506959in}}%
\pgfpathlineto{\pgfqpoint{2.370068in}{4.193310in}}%
\pgfpathlineto{\pgfqpoint{2.375510in}{3.722838in}}%
\pgfpathlineto{\pgfqpoint{2.380952in}{3.722838in}}%
\pgfpathlineto{\pgfqpoint{2.386395in}{2.938716in}}%
\pgfpathlineto{\pgfqpoint{2.391837in}{2.781892in}}%
\pgfpathlineto{\pgfqpoint{2.397279in}{2.781892in}}%
\pgfpathlineto{\pgfqpoint{2.402721in}{2.468243in}}%
\pgfpathlineto{\pgfqpoint{2.408163in}{3.095540in}}%
\pgfpathlineto{\pgfqpoint{2.413605in}{2.938716in}}%
\pgfpathlineto{\pgfqpoint{2.419048in}{2.938716in}}%
\pgfpathlineto{\pgfqpoint{2.424490in}{3.409189in}}%
\pgfpathlineto{\pgfqpoint{2.429932in}{2.781892in}}%
\pgfpathlineto{\pgfqpoint{2.435374in}{1.997770in}}%
\pgfpathlineto{\pgfqpoint{2.457143in}{1.997770in}}%
\pgfpathlineto{\pgfqpoint{2.468027in}{1.684122in}}%
\pgfpathlineto{\pgfqpoint{2.473469in}{1.997770in}}%
\pgfpathlineto{\pgfqpoint{2.478912in}{2.154594in}}%
\pgfpathlineto{\pgfqpoint{2.484354in}{2.154594in}}%
\pgfpathlineto{\pgfqpoint{2.489796in}{2.311419in}}%
\pgfpathlineto{\pgfqpoint{2.495238in}{2.625067in}}%
\pgfpathlineto{\pgfqpoint{2.500680in}{2.781892in}}%
\pgfpathlineto{\pgfqpoint{2.506122in}{3.409189in}}%
\pgfpathlineto{\pgfqpoint{2.511565in}{3.722838in}}%
\pgfpathlineto{\pgfqpoint{2.517007in}{3.722838in}}%
\pgfpathlineto{\pgfqpoint{2.522449in}{3.566013in}}%
\pgfpathlineto{\pgfqpoint{2.527891in}{3.879662in}}%
\pgfpathlineto{\pgfqpoint{2.533333in}{4.036486in}}%
\pgfpathlineto{\pgfqpoint{2.538776in}{4.506959in}}%
\pgfpathlineto{\pgfqpoint{2.544218in}{4.663783in}}%
\pgfpathlineto{\pgfqpoint{2.549660in}{4.506959in}}%
\pgfpathlineto{\pgfqpoint{2.555102in}{4.663783in}}%
\pgfpathlineto{\pgfqpoint{2.571429in}{3.722838in}}%
\pgfpathlineto{\pgfqpoint{2.582313in}{3.722838in}}%
\pgfpathlineto{\pgfqpoint{2.587755in}{3.409189in}}%
\pgfpathlineto{\pgfqpoint{2.598639in}{3.722838in}}%
\pgfpathlineto{\pgfqpoint{2.604082in}{3.722838in}}%
\pgfpathlineto{\pgfqpoint{2.620408in}{4.193310in}}%
\pgfpathlineto{\pgfqpoint{2.625850in}{4.506959in}}%
\pgfpathlineto{\pgfqpoint{2.631293in}{4.663783in}}%
\pgfpathlineto{\pgfqpoint{2.636735in}{4.506959in}}%
\pgfpathlineto{\pgfqpoint{2.647619in}{4.506959in}}%
\pgfpathlineto{\pgfqpoint{2.653061in}{4.663783in}}%
\pgfpathlineto{\pgfqpoint{2.658503in}{4.663783in}}%
\pgfpathlineto{\pgfqpoint{2.663946in}{4.506959in}}%
\pgfpathlineto{\pgfqpoint{2.669388in}{4.820608in}}%
\pgfpathlineto{\pgfqpoint{2.674830in}{4.663783in}}%
\pgfpathlineto{\pgfqpoint{2.680272in}{4.820608in}}%
\pgfpathlineto{\pgfqpoint{2.685714in}{4.820608in}}%
\pgfpathlineto{\pgfqpoint{2.691156in}{4.663783in}}%
\pgfpathlineto{\pgfqpoint{2.696599in}{4.663783in}}%
\pgfpathlineto{\pgfqpoint{2.702041in}{4.820608in}}%
\pgfpathlineto{\pgfqpoint{2.707483in}{4.663783in}}%
\pgfpathlineto{\pgfqpoint{2.718367in}{4.663783in}}%
\pgfpathlineto{\pgfqpoint{2.723810in}{4.820608in}}%
\pgfpathlineto{\pgfqpoint{2.729252in}{4.820608in}}%
\pgfpathlineto{\pgfqpoint{2.734694in}{5.134256in}}%
\pgfpathlineto{\pgfqpoint{2.740136in}{4.820608in}}%
\pgfpathlineto{\pgfqpoint{2.745578in}{4.663783in}}%
\pgfpathlineto{\pgfqpoint{2.756463in}{4.977432in}}%
\pgfpathlineto{\pgfqpoint{2.761905in}{4.663783in}}%
\pgfpathlineto{\pgfqpoint{2.767347in}{5.134256in}}%
\pgfpathlineto{\pgfqpoint{2.778231in}{4.820608in}}%
\pgfpathlineto{\pgfqpoint{2.783673in}{4.820608in}}%
\pgfpathlineto{\pgfqpoint{2.789116in}{4.977432in}}%
\pgfpathlineto{\pgfqpoint{2.794558in}{4.820608in}}%
\pgfpathlineto{\pgfqpoint{2.800000in}{4.820608in}}%
\pgfpathlineto{\pgfqpoint{2.805442in}{4.977432in}}%
\pgfpathlineto{\pgfqpoint{2.810884in}{4.820608in}}%
\pgfpathlineto{\pgfqpoint{2.821769in}{4.820608in}}%
\pgfpathlineto{\pgfqpoint{2.827211in}{5.291081in}}%
\pgfpathlineto{\pgfqpoint{2.832653in}{4.663783in}}%
\pgfpathlineto{\pgfqpoint{2.838095in}{4.820608in}}%
\pgfpathlineto{\pgfqpoint{2.843537in}{4.663783in}}%
\pgfpathlineto{\pgfqpoint{2.848980in}{4.820608in}}%
\pgfpathlineto{\pgfqpoint{2.854422in}{4.663783in}}%
\pgfpathlineto{\pgfqpoint{2.859864in}{4.820608in}}%
\pgfpathlineto{\pgfqpoint{2.865306in}{4.663783in}}%
\pgfpathlineto{\pgfqpoint{2.870748in}{4.977432in}}%
\pgfpathlineto{\pgfqpoint{2.881633in}{4.663783in}}%
\pgfpathlineto{\pgfqpoint{2.887075in}{4.820608in}}%
\pgfpathlineto{\pgfqpoint{2.892517in}{4.820608in}}%
\pgfpathlineto{\pgfqpoint{2.897959in}{4.977432in}}%
\pgfpathlineto{\pgfqpoint{2.903401in}{4.663783in}}%
\pgfpathlineto{\pgfqpoint{2.908844in}{4.820608in}}%
\pgfpathlineto{\pgfqpoint{2.914286in}{4.820608in}}%
\pgfpathlineto{\pgfqpoint{2.919728in}{4.977432in}}%
\pgfpathlineto{\pgfqpoint{2.930612in}{4.663783in}}%
\pgfpathlineto{\pgfqpoint{2.936054in}{4.977432in}}%
\pgfpathlineto{\pgfqpoint{2.941497in}{4.506959in}}%
\pgfpathlineto{\pgfqpoint{2.946939in}{4.820608in}}%
\pgfpathlineto{\pgfqpoint{2.963265in}{4.820608in}}%
\pgfpathlineto{\pgfqpoint{2.968707in}{4.977432in}}%
\pgfpathlineto{\pgfqpoint{2.974150in}{4.820608in}}%
\pgfpathlineto{\pgfqpoint{2.979592in}{4.820608in}}%
\pgfpathlineto{\pgfqpoint{2.985034in}{4.663783in}}%
\pgfpathlineto{\pgfqpoint{2.995918in}{4.977432in}}%
\pgfpathlineto{\pgfqpoint{3.001361in}{4.977432in}}%
\pgfpathlineto{\pgfqpoint{3.006803in}{4.820608in}}%
\pgfpathlineto{\pgfqpoint{3.012245in}{4.820608in}}%
\pgfpathlineto{\pgfqpoint{3.023129in}{4.506959in}}%
\pgfpathlineto{\pgfqpoint{3.028571in}{4.820608in}}%
\pgfpathlineto{\pgfqpoint{3.034014in}{4.820608in}}%
\pgfpathlineto{\pgfqpoint{3.039456in}{4.663783in}}%
\pgfpathlineto{\pgfqpoint{3.044898in}{5.134256in}}%
\pgfpathlineto{\pgfqpoint{3.050340in}{4.820608in}}%
\pgfpathlineto{\pgfqpoint{3.061224in}{4.820608in}}%
\pgfpathlineto{\pgfqpoint{3.066667in}{4.663783in}}%
\pgfpathlineto{\pgfqpoint{3.072109in}{4.820608in}}%
\pgfpathlineto{\pgfqpoint{3.077551in}{4.820608in}}%
\pgfpathlineto{\pgfqpoint{3.082993in}{4.977432in}}%
\pgfpathlineto{\pgfqpoint{3.088435in}{4.820608in}}%
\pgfpathlineto{\pgfqpoint{3.099320in}{4.820608in}}%
\pgfpathlineto{\pgfqpoint{3.104762in}{4.977432in}}%
\pgfpathlineto{\pgfqpoint{3.110204in}{4.506959in}}%
\pgfpathlineto{\pgfqpoint{3.115646in}{4.663783in}}%
\pgfpathlineto{\pgfqpoint{3.131973in}{4.663783in}}%
\pgfpathlineto{\pgfqpoint{3.137415in}{5.134256in}}%
\pgfpathlineto{\pgfqpoint{3.142857in}{4.820608in}}%
\pgfpathlineto{\pgfqpoint{3.159184in}{4.820608in}}%
\pgfpathlineto{\pgfqpoint{3.164626in}{4.663783in}}%
\pgfpathlineto{\pgfqpoint{3.175510in}{4.663783in}}%
\pgfpathlineto{\pgfqpoint{3.186395in}{4.977432in}}%
\pgfpathlineto{\pgfqpoint{3.191837in}{4.820608in}}%
\pgfpathlineto{\pgfqpoint{3.197279in}{4.820608in}}%
\pgfpathlineto{\pgfqpoint{3.202721in}{4.663783in}}%
\pgfpathlineto{\pgfqpoint{3.208163in}{4.977432in}}%
\pgfpathlineto{\pgfqpoint{3.213605in}{5.134256in}}%
\pgfpathlineto{\pgfqpoint{3.219048in}{4.663783in}}%
\pgfpathlineto{\pgfqpoint{3.224490in}{4.663783in}}%
\pgfpathlineto{\pgfqpoint{3.229932in}{4.820608in}}%
\pgfpathlineto{\pgfqpoint{3.240816in}{4.820608in}}%
\pgfpathlineto{\pgfqpoint{3.246259in}{4.977432in}}%
\pgfpathlineto{\pgfqpoint{3.251701in}{4.820608in}}%
\pgfpathlineto{\pgfqpoint{3.257143in}{4.977432in}}%
\pgfpathlineto{\pgfqpoint{3.262585in}{4.820608in}}%
\pgfpathlineto{\pgfqpoint{3.278912in}{4.820608in}}%
\pgfpathlineto{\pgfqpoint{3.284354in}{4.663783in}}%
\pgfpathlineto{\pgfqpoint{3.289796in}{4.977432in}}%
\pgfpathlineto{\pgfqpoint{3.295238in}{4.820608in}}%
\pgfpathlineto{\pgfqpoint{3.300680in}{4.977432in}}%
\pgfpathlineto{\pgfqpoint{3.306122in}{4.977432in}}%
\pgfpathlineto{\pgfqpoint{3.311565in}{4.663783in}}%
\pgfpathlineto{\pgfqpoint{3.317007in}{4.663783in}}%
\pgfpathlineto{\pgfqpoint{3.322449in}{4.977432in}}%
\pgfpathlineto{\pgfqpoint{3.327891in}{4.663783in}}%
\pgfpathlineto{\pgfqpoint{3.333333in}{4.663783in}}%
\pgfpathlineto{\pgfqpoint{3.344218in}{4.977432in}}%
\pgfpathlineto{\pgfqpoint{3.349660in}{4.663783in}}%
\pgfpathlineto{\pgfqpoint{3.355102in}{4.663783in}}%
\pgfpathlineto{\pgfqpoint{3.360544in}{4.820608in}}%
\pgfpathlineto{\pgfqpoint{3.365986in}{4.663783in}}%
\pgfpathlineto{\pgfqpoint{3.376871in}{4.977432in}}%
\pgfpathlineto{\pgfqpoint{3.382313in}{4.663783in}}%
\pgfpathlineto{\pgfqpoint{3.387755in}{4.977432in}}%
\pgfpathlineto{\pgfqpoint{3.393197in}{4.977432in}}%
\pgfpathlineto{\pgfqpoint{3.398639in}{4.663783in}}%
\pgfpathlineto{\pgfqpoint{3.404082in}{4.977432in}}%
\pgfpathlineto{\pgfqpoint{3.409524in}{4.820608in}}%
\pgfpathlineto{\pgfqpoint{3.436735in}{4.820608in}}%
\pgfpathlineto{\pgfqpoint{3.442177in}{4.663783in}}%
\pgfpathlineto{\pgfqpoint{3.447619in}{4.977432in}}%
\pgfpathlineto{\pgfqpoint{3.453061in}{5.134256in}}%
\pgfpathlineto{\pgfqpoint{3.458503in}{4.820608in}}%
\pgfpathlineto{\pgfqpoint{3.463946in}{4.977432in}}%
\pgfpathlineto{\pgfqpoint{3.474830in}{4.663783in}}%
\pgfpathlineto{\pgfqpoint{3.491156in}{4.663783in}}%
\pgfpathlineto{\pgfqpoint{3.496599in}{4.820608in}}%
\pgfpathlineto{\pgfqpoint{3.502041in}{4.663783in}}%
\pgfpathlineto{\pgfqpoint{3.507483in}{4.820608in}}%
\pgfpathlineto{\pgfqpoint{3.512925in}{4.663783in}}%
\pgfpathlineto{\pgfqpoint{3.518367in}{4.977432in}}%
\pgfpathlineto{\pgfqpoint{3.523810in}{4.663783in}}%
\pgfpathlineto{\pgfqpoint{3.529252in}{4.663783in}}%
\pgfpathlineto{\pgfqpoint{3.534694in}{5.134256in}}%
\pgfpathlineto{\pgfqpoint{3.540136in}{4.977432in}}%
\pgfpathlineto{\pgfqpoint{3.545578in}{4.977432in}}%
\pgfpathlineto{\pgfqpoint{3.551020in}{4.663783in}}%
\pgfpathlineto{\pgfqpoint{3.556463in}{4.977432in}}%
\pgfpathlineto{\pgfqpoint{3.561905in}{4.506959in}}%
\pgfpathlineto{\pgfqpoint{3.567347in}{4.977432in}}%
\pgfpathlineto{\pgfqpoint{3.572789in}{4.663783in}}%
\pgfpathlineto{\pgfqpoint{3.578231in}{4.506959in}}%
\pgfpathlineto{\pgfqpoint{3.583673in}{4.820608in}}%
\pgfpathlineto{\pgfqpoint{3.594558in}{4.820608in}}%
\pgfpathlineto{\pgfqpoint{3.600000in}{4.977432in}}%
\pgfpathlineto{\pgfqpoint{3.605442in}{4.977432in}}%
\pgfpathlineto{\pgfqpoint{3.610884in}{4.506959in}}%
\pgfpathlineto{\pgfqpoint{3.616327in}{4.820608in}}%
\pgfpathlineto{\pgfqpoint{3.621769in}{4.663783in}}%
\pgfpathlineto{\pgfqpoint{3.627211in}{4.663783in}}%
\pgfpathlineto{\pgfqpoint{3.632653in}{5.134256in}}%
\pgfpathlineto{\pgfqpoint{3.638095in}{4.820608in}}%
\pgfpathlineto{\pgfqpoint{3.643537in}{5.291081in}}%
\pgfpathlineto{\pgfqpoint{3.648980in}{4.663783in}}%
\pgfpathlineto{\pgfqpoint{3.654422in}{5.134256in}}%
\pgfpathlineto{\pgfqpoint{3.659864in}{5.134256in}}%
\pgfpathlineto{\pgfqpoint{3.665306in}{4.820608in}}%
\pgfpathlineto{\pgfqpoint{3.670748in}{4.977432in}}%
\pgfpathlineto{\pgfqpoint{3.676190in}{4.506959in}}%
\pgfpathlineto{\pgfqpoint{3.681633in}{4.663783in}}%
\pgfpathlineto{\pgfqpoint{3.687075in}{4.663783in}}%
\pgfpathlineto{\pgfqpoint{3.692517in}{4.820608in}}%
\pgfpathlineto{\pgfqpoint{3.697959in}{4.663783in}}%
\pgfpathlineto{\pgfqpoint{3.708844in}{4.977432in}}%
\pgfpathlineto{\pgfqpoint{3.719728in}{4.663783in}}%
\pgfpathlineto{\pgfqpoint{3.725170in}{4.977432in}}%
\pgfpathlineto{\pgfqpoint{3.730612in}{4.977432in}}%
\pgfpathlineto{\pgfqpoint{3.736054in}{4.820608in}}%
\pgfpathlineto{\pgfqpoint{3.741497in}{4.977432in}}%
\pgfpathlineto{\pgfqpoint{3.746939in}{4.977432in}}%
\pgfpathlineto{\pgfqpoint{3.752381in}{4.820608in}}%
\pgfpathlineto{\pgfqpoint{3.763265in}{4.820608in}}%
\pgfpathlineto{\pgfqpoint{3.774150in}{5.134256in}}%
\pgfpathlineto{\pgfqpoint{3.779592in}{4.977432in}}%
\pgfpathlineto{\pgfqpoint{3.785034in}{5.134256in}}%
\pgfpathlineto{\pgfqpoint{3.790476in}{4.977432in}}%
\pgfpathlineto{\pgfqpoint{3.795918in}{4.506959in}}%
\pgfpathlineto{\pgfqpoint{3.801361in}{4.820608in}}%
\pgfpathlineto{\pgfqpoint{3.806803in}{4.506959in}}%
\pgfpathlineto{\pgfqpoint{3.812245in}{5.134256in}}%
\pgfpathlineto{\pgfqpoint{3.817687in}{4.663783in}}%
\pgfpathlineto{\pgfqpoint{3.823129in}{4.977432in}}%
\pgfpathlineto{\pgfqpoint{3.828571in}{4.820608in}}%
\pgfpathlineto{\pgfqpoint{3.839456in}{4.820608in}}%
\pgfpathlineto{\pgfqpoint{3.844898in}{5.134256in}}%
\pgfpathlineto{\pgfqpoint{3.850340in}{4.977432in}}%
\pgfpathlineto{\pgfqpoint{3.861224in}{4.977432in}}%
\pgfpathlineto{\pgfqpoint{3.866667in}{4.820608in}}%
\pgfpathlineto{\pgfqpoint{3.877551in}{5.134256in}}%
\pgfpathlineto{\pgfqpoint{3.882993in}{4.820608in}}%
\pgfpathlineto{\pgfqpoint{3.888435in}{4.820608in}}%
\pgfpathlineto{\pgfqpoint{3.893878in}{4.663783in}}%
\pgfpathlineto{\pgfqpoint{3.899320in}{4.820608in}}%
\pgfpathlineto{\pgfqpoint{3.904762in}{5.134256in}}%
\pgfpathlineto{\pgfqpoint{3.910204in}{4.663783in}}%
\pgfpathlineto{\pgfqpoint{3.915646in}{4.663783in}}%
\pgfpathlineto{\pgfqpoint{3.926531in}{4.977432in}}%
\pgfpathlineto{\pgfqpoint{3.937415in}{4.663783in}}%
\pgfpathlineto{\pgfqpoint{3.942857in}{4.820608in}}%
\pgfpathlineto{\pgfqpoint{3.953741in}{4.820608in}}%
\pgfpathlineto{\pgfqpoint{3.959184in}{4.977432in}}%
\pgfpathlineto{\pgfqpoint{3.964626in}{4.506959in}}%
\pgfpathlineto{\pgfqpoint{3.970068in}{4.977432in}}%
\pgfpathlineto{\pgfqpoint{3.975510in}{4.663783in}}%
\pgfpathlineto{\pgfqpoint{3.980952in}{4.977432in}}%
\pgfpathlineto{\pgfqpoint{3.986395in}{4.977432in}}%
\pgfpathlineto{\pgfqpoint{3.991837in}{4.820608in}}%
\pgfpathlineto{\pgfqpoint{4.002721in}{4.820608in}}%
\pgfpathlineto{\pgfqpoint{4.008163in}{4.977432in}}%
\pgfpathlineto{\pgfqpoint{4.013605in}{4.663783in}}%
\pgfpathlineto{\pgfqpoint{4.019048in}{5.134256in}}%
\pgfpathlineto{\pgfqpoint{4.024490in}{4.820608in}}%
\pgfpathlineto{\pgfqpoint{4.035374in}{5.134256in}}%
\pgfpathlineto{\pgfqpoint{4.040816in}{5.134256in}}%
\pgfpathlineto{\pgfqpoint{4.046259in}{4.977432in}}%
\pgfpathlineto{\pgfqpoint{4.051701in}{5.134256in}}%
\pgfpathlineto{\pgfqpoint{4.062585in}{4.820608in}}%
\pgfpathlineto{\pgfqpoint{4.068027in}{4.506959in}}%
\pgfpathlineto{\pgfqpoint{4.078912in}{5.134256in}}%
\pgfpathlineto{\pgfqpoint{4.084354in}{4.506959in}}%
\pgfpathlineto{\pgfqpoint{4.089796in}{4.820608in}}%
\pgfpathlineto{\pgfqpoint{4.095238in}{4.977432in}}%
\pgfpathlineto{\pgfqpoint{4.106122in}{4.350135in}}%
\pgfpathlineto{\pgfqpoint{4.111565in}{4.820608in}}%
\pgfpathlineto{\pgfqpoint{4.117007in}{4.977432in}}%
\pgfpathlineto{\pgfqpoint{4.122449in}{4.820608in}}%
\pgfpathlineto{\pgfqpoint{4.127891in}{5.134256in}}%
\pgfpathlineto{\pgfqpoint{4.133333in}{4.820608in}}%
\pgfpathlineto{\pgfqpoint{4.138776in}{4.663783in}}%
\pgfpathlineto{\pgfqpoint{4.144218in}{4.663783in}}%
\pgfpathlineto{\pgfqpoint{4.149660in}{4.506959in}}%
\pgfpathlineto{\pgfqpoint{4.155102in}{4.820608in}}%
\pgfpathlineto{\pgfqpoint{4.160544in}{4.977432in}}%
\pgfpathlineto{\pgfqpoint{4.165986in}{4.820608in}}%
\pgfpathlineto{\pgfqpoint{4.171429in}{4.977432in}}%
\pgfpathlineto{\pgfqpoint{4.176871in}{4.820608in}}%
\pgfpathlineto{\pgfqpoint{4.182313in}{4.977432in}}%
\pgfpathlineto{\pgfqpoint{4.187755in}{4.977432in}}%
\pgfpathlineto{\pgfqpoint{4.193197in}{4.663783in}}%
\pgfpathlineto{\pgfqpoint{4.204082in}{4.663783in}}%
\pgfpathlineto{\pgfqpoint{4.209524in}{4.977432in}}%
\pgfpathlineto{\pgfqpoint{4.214966in}{4.663783in}}%
\pgfpathlineto{\pgfqpoint{4.220408in}{4.977432in}}%
\pgfpathlineto{\pgfqpoint{4.231293in}{4.663783in}}%
\pgfpathlineto{\pgfqpoint{4.242177in}{4.663783in}}%
\pgfpathlineto{\pgfqpoint{4.247619in}{4.977432in}}%
\pgfpathlineto{\pgfqpoint{4.253061in}{4.977432in}}%
\pgfpathlineto{\pgfqpoint{4.258503in}{5.134256in}}%
\pgfpathlineto{\pgfqpoint{4.269388in}{4.506959in}}%
\pgfpathlineto{\pgfqpoint{4.274830in}{4.977432in}}%
\pgfpathlineto{\pgfqpoint{4.280272in}{4.663783in}}%
\pgfpathlineto{\pgfqpoint{4.285714in}{4.506959in}}%
\pgfpathlineto{\pgfqpoint{4.291156in}{4.977432in}}%
\pgfpathlineto{\pgfqpoint{4.323810in}{4.977432in}}%
\pgfpathlineto{\pgfqpoint{4.334694in}{4.663783in}}%
\pgfpathlineto{\pgfqpoint{4.340136in}{4.977432in}}%
\pgfpathlineto{\pgfqpoint{4.345578in}{4.977432in}}%
\pgfpathlineto{\pgfqpoint{4.351020in}{4.820608in}}%
\pgfpathlineto{\pgfqpoint{4.356463in}{4.977432in}}%
\pgfpathlineto{\pgfqpoint{4.361905in}{4.820608in}}%
\pgfpathlineto{\pgfqpoint{4.367347in}{4.820608in}}%
\pgfpathlineto{\pgfqpoint{4.372789in}{4.663783in}}%
\pgfpathlineto{\pgfqpoint{4.389116in}{5.134256in}}%
\pgfpathlineto{\pgfqpoint{4.394558in}{4.663783in}}%
\pgfpathlineto{\pgfqpoint{4.400000in}{4.350135in}}%
\pgfpathlineto{\pgfqpoint{4.405442in}{4.663783in}}%
\pgfpathlineto{\pgfqpoint{4.410884in}{4.663783in}}%
\pgfpathlineto{\pgfqpoint{4.416327in}{4.977432in}}%
\pgfpathlineto{\pgfqpoint{4.421769in}{4.506959in}}%
\pgfpathlineto{\pgfqpoint{4.427211in}{4.977432in}}%
\pgfpathlineto{\pgfqpoint{4.438095in}{4.663783in}}%
\pgfpathlineto{\pgfqpoint{4.443537in}{5.134256in}}%
\pgfpathlineto{\pgfqpoint{4.448980in}{4.663783in}}%
\pgfpathlineto{\pgfqpoint{4.454422in}{4.977432in}}%
\pgfpathlineto{\pgfqpoint{4.459864in}{5.134256in}}%
\pgfpathlineto{\pgfqpoint{4.465306in}{4.663783in}}%
\pgfpathlineto{\pgfqpoint{4.476190in}{4.977432in}}%
\pgfpathlineto{\pgfqpoint{4.481633in}{4.820608in}}%
\pgfpathlineto{\pgfqpoint{4.497959in}{4.820608in}}%
\pgfpathlineto{\pgfqpoint{4.503401in}{4.977432in}}%
\pgfpathlineto{\pgfqpoint{4.508844in}{4.663783in}}%
\pgfpathlineto{\pgfqpoint{4.514286in}{4.663783in}}%
\pgfpathlineto{\pgfqpoint{4.519728in}{4.820608in}}%
\pgfpathlineto{\pgfqpoint{4.525170in}{4.663783in}}%
\pgfpathlineto{\pgfqpoint{4.536054in}{4.977432in}}%
\pgfpathlineto{\pgfqpoint{4.541497in}{4.820608in}}%
\pgfpathlineto{\pgfqpoint{4.546939in}{4.350135in}}%
\pgfpathlineto{\pgfqpoint{4.552381in}{4.977432in}}%
\pgfpathlineto{\pgfqpoint{4.557823in}{4.977432in}}%
\pgfpathlineto{\pgfqpoint{4.563265in}{4.663783in}}%
\pgfpathlineto{\pgfqpoint{4.568707in}{4.506959in}}%
\pgfpathlineto{\pgfqpoint{4.574150in}{4.506959in}}%
\pgfpathlineto{\pgfqpoint{4.579592in}{4.820608in}}%
\pgfpathlineto{\pgfqpoint{4.585034in}{4.820608in}}%
\pgfpathlineto{\pgfqpoint{4.590476in}{4.350135in}}%
\pgfpathlineto{\pgfqpoint{4.595918in}{4.663783in}}%
\pgfpathlineto{\pgfqpoint{4.601361in}{4.820608in}}%
\pgfpathlineto{\pgfqpoint{4.634014in}{4.820608in}}%
\pgfpathlineto{\pgfqpoint{4.639456in}{4.663783in}}%
\pgfpathlineto{\pgfqpoint{4.644898in}{4.977432in}}%
\pgfpathlineto{\pgfqpoint{4.650340in}{4.977432in}}%
\pgfpathlineto{\pgfqpoint{4.655782in}{4.506959in}}%
\pgfpathlineto{\pgfqpoint{4.661224in}{4.820608in}}%
\pgfpathlineto{\pgfqpoint{4.666667in}{4.506959in}}%
\pgfpathlineto{\pgfqpoint{4.672109in}{4.506959in}}%
\pgfpathlineto{\pgfqpoint{4.677551in}{4.820608in}}%
\pgfpathlineto{\pgfqpoint{4.682993in}{4.977432in}}%
\pgfpathlineto{\pgfqpoint{4.688435in}{4.663783in}}%
\pgfpathlineto{\pgfqpoint{4.693878in}{4.820608in}}%
\pgfpathlineto{\pgfqpoint{4.699320in}{4.506959in}}%
\pgfpathlineto{\pgfqpoint{4.704762in}{4.820608in}}%
\pgfpathlineto{\pgfqpoint{4.710204in}{4.506959in}}%
\pgfpathlineto{\pgfqpoint{4.715646in}{4.977432in}}%
\pgfpathlineto{\pgfqpoint{4.721088in}{4.663783in}}%
\pgfpathlineto{\pgfqpoint{4.726531in}{4.977432in}}%
\pgfpathlineto{\pgfqpoint{4.731973in}{4.506959in}}%
\pgfpathlineto{\pgfqpoint{4.737415in}{4.663783in}}%
\pgfpathlineto{\pgfqpoint{4.748299in}{4.663783in}}%
\pgfpathlineto{\pgfqpoint{4.753741in}{4.820608in}}%
\pgfpathlineto{\pgfqpoint{4.759184in}{5.134256in}}%
\pgfpathlineto{\pgfqpoint{4.764626in}{4.663783in}}%
\pgfpathlineto{\pgfqpoint{4.775510in}{4.663783in}}%
\pgfpathlineto{\pgfqpoint{4.780952in}{4.820608in}}%
\pgfpathlineto{\pgfqpoint{4.786395in}{4.820608in}}%
\pgfpathlineto{\pgfqpoint{4.791837in}{4.977432in}}%
\pgfpathlineto{\pgfqpoint{4.797279in}{4.663783in}}%
\pgfpathlineto{\pgfqpoint{4.802721in}{4.820608in}}%
\pgfpathlineto{\pgfqpoint{4.808163in}{4.663783in}}%
\pgfpathlineto{\pgfqpoint{4.813605in}{4.977432in}}%
\pgfpathlineto{\pgfqpoint{4.819048in}{4.820608in}}%
\pgfpathlineto{\pgfqpoint{4.829932in}{4.820608in}}%
\pgfpathlineto{\pgfqpoint{4.835374in}{4.977432in}}%
\pgfpathlineto{\pgfqpoint{4.840816in}{4.506959in}}%
\pgfpathlineto{\pgfqpoint{4.846259in}{4.977432in}}%
\pgfpathlineto{\pgfqpoint{4.851701in}{4.663783in}}%
\pgfpathlineto{\pgfqpoint{4.857143in}{4.977432in}}%
\pgfpathlineto{\pgfqpoint{4.862585in}{4.820608in}}%
\pgfpathlineto{\pgfqpoint{4.868027in}{5.134256in}}%
\pgfpathlineto{\pgfqpoint{4.873469in}{4.977432in}}%
\pgfpathlineto{\pgfqpoint{4.878912in}{4.977432in}}%
\pgfpathlineto{\pgfqpoint{4.884354in}{4.663783in}}%
\pgfpathlineto{\pgfqpoint{4.889796in}{4.820608in}}%
\pgfpathlineto{\pgfqpoint{4.895238in}{4.663783in}}%
\pgfpathlineto{\pgfqpoint{4.900680in}{4.820608in}}%
\pgfpathlineto{\pgfqpoint{4.906122in}{4.820608in}}%
\pgfpathlineto{\pgfqpoint{4.911565in}{4.977432in}}%
\pgfpathlineto{\pgfqpoint{4.917007in}{4.820608in}}%
\pgfpathlineto{\pgfqpoint{4.922449in}{4.977432in}}%
\pgfpathlineto{\pgfqpoint{4.927891in}{4.663783in}}%
\pgfpathlineto{\pgfqpoint{4.933333in}{4.820608in}}%
\pgfpathlineto{\pgfqpoint{4.938776in}{4.663783in}}%
\pgfpathlineto{\pgfqpoint{4.949660in}{4.977432in}}%
\pgfpathlineto{\pgfqpoint{4.955102in}{4.506959in}}%
\pgfpathlineto{\pgfqpoint{4.960544in}{4.820608in}}%
\pgfpathlineto{\pgfqpoint{4.971429in}{4.820608in}}%
\pgfpathlineto{\pgfqpoint{4.976871in}{4.663783in}}%
\pgfpathlineto{\pgfqpoint{4.982313in}{4.977432in}}%
\pgfpathlineto{\pgfqpoint{4.987755in}{4.663783in}}%
\pgfpathlineto{\pgfqpoint{4.993197in}{5.134256in}}%
\pgfpathlineto{\pgfqpoint{4.998639in}{4.350135in}}%
\pgfpathlineto{\pgfqpoint{5.004082in}{4.506959in}}%
\pgfpathlineto{\pgfqpoint{5.009524in}{4.977432in}}%
\pgfpathlineto{\pgfqpoint{5.014966in}{4.977432in}}%
\pgfpathlineto{\pgfqpoint{5.020408in}{4.820608in}}%
\pgfpathlineto{\pgfqpoint{5.042177in}{4.820608in}}%
\pgfpathlineto{\pgfqpoint{5.047619in}{4.663783in}}%
\pgfpathlineto{\pgfqpoint{5.053061in}{5.134256in}}%
\pgfpathlineto{\pgfqpoint{5.058503in}{4.820608in}}%
\pgfpathlineto{\pgfqpoint{5.063946in}{4.977432in}}%
\pgfpathlineto{\pgfqpoint{5.069388in}{4.977432in}}%
\pgfpathlineto{\pgfqpoint{5.074830in}{4.506959in}}%
\pgfpathlineto{\pgfqpoint{5.080272in}{4.663783in}}%
\pgfpathlineto{\pgfqpoint{5.085714in}{4.663783in}}%
\pgfpathlineto{\pgfqpoint{5.091156in}{4.977432in}}%
\pgfpathlineto{\pgfqpoint{5.096599in}{4.506959in}}%
\pgfpathlineto{\pgfqpoint{5.102041in}{4.977432in}}%
\pgfpathlineto{\pgfqpoint{5.107483in}{4.820608in}}%
\pgfpathlineto{\pgfqpoint{5.118367in}{4.820608in}}%
\pgfpathlineto{\pgfqpoint{5.123810in}{4.977432in}}%
\pgfpathlineto{\pgfqpoint{5.129252in}{4.977432in}}%
\pgfpathlineto{\pgfqpoint{5.134694in}{4.820608in}}%
\pgfpathlineto{\pgfqpoint{5.145578in}{4.820608in}}%
\pgfpathlineto{\pgfqpoint{5.151020in}{4.977432in}}%
\pgfpathlineto{\pgfqpoint{5.156463in}{4.663783in}}%
\pgfpathlineto{\pgfqpoint{5.161905in}{4.977432in}}%
\pgfpathlineto{\pgfqpoint{5.167347in}{4.977432in}}%
\pgfpathlineto{\pgfqpoint{5.172789in}{4.663783in}}%
\pgfpathlineto{\pgfqpoint{5.178231in}{4.663783in}}%
\pgfpathlineto{\pgfqpoint{5.189116in}{4.977432in}}%
\pgfpathlineto{\pgfqpoint{5.194558in}{4.977432in}}%
\pgfpathlineto{\pgfqpoint{5.200000in}{4.820608in}}%
\pgfpathlineto{\pgfqpoint{5.205442in}{4.506959in}}%
\pgfpathlineto{\pgfqpoint{5.216327in}{4.820608in}}%
\pgfpathlineto{\pgfqpoint{5.221769in}{4.663783in}}%
\pgfpathlineto{\pgfqpoint{5.227211in}{4.663783in}}%
\pgfpathlineto{\pgfqpoint{5.232653in}{4.977432in}}%
\pgfpathlineto{\pgfqpoint{5.238095in}{4.663783in}}%
\pgfpathlineto{\pgfqpoint{5.243537in}{4.977432in}}%
\pgfpathlineto{\pgfqpoint{5.248980in}{4.820608in}}%
\pgfpathlineto{\pgfqpoint{5.254422in}{4.506959in}}%
\pgfpathlineto{\pgfqpoint{5.259864in}{4.977432in}}%
\pgfpathlineto{\pgfqpoint{5.265306in}{4.663783in}}%
\pgfpathlineto{\pgfqpoint{5.270748in}{4.820608in}}%
\pgfpathlineto{\pgfqpoint{5.276190in}{4.820608in}}%
\pgfpathlineto{\pgfqpoint{5.281633in}{4.977432in}}%
\pgfpathlineto{\pgfqpoint{5.292517in}{4.663783in}}%
\pgfpathlineto{\pgfqpoint{5.297959in}{4.977432in}}%
\pgfpathlineto{\pgfqpoint{5.303401in}{4.977432in}}%
\pgfpathlineto{\pgfqpoint{5.308844in}{4.663783in}}%
\pgfpathlineto{\pgfqpoint{5.314286in}{4.506959in}}%
\pgfpathlineto{\pgfqpoint{5.319728in}{4.820608in}}%
\pgfpathlineto{\pgfqpoint{5.325170in}{4.663783in}}%
\pgfpathlineto{\pgfqpoint{5.330612in}{4.663783in}}%
\pgfpathlineto{\pgfqpoint{5.336054in}{4.977432in}}%
\pgfpathlineto{\pgfqpoint{5.341497in}{4.820608in}}%
\pgfpathlineto{\pgfqpoint{5.346939in}{4.977432in}}%
\pgfpathlineto{\pgfqpoint{5.352381in}{4.820608in}}%
\pgfpathlineto{\pgfqpoint{5.357823in}{4.820608in}}%
\pgfpathlineto{\pgfqpoint{5.363265in}{4.977432in}}%
\pgfpathlineto{\pgfqpoint{5.374150in}{4.663783in}}%
\pgfpathlineto{\pgfqpoint{5.379592in}{4.663783in}}%
\pgfpathlineto{\pgfqpoint{5.385034in}{4.820608in}}%
\pgfpathlineto{\pgfqpoint{5.390476in}{4.663783in}}%
\pgfpathlineto{\pgfqpoint{5.395918in}{4.977432in}}%
\pgfpathlineto{\pgfqpoint{5.406803in}{4.663783in}}%
\pgfpathlineto{\pgfqpoint{5.412245in}{4.977432in}}%
\pgfpathlineto{\pgfqpoint{5.428571in}{4.977432in}}%
\pgfpathlineto{\pgfqpoint{5.434014in}{4.663783in}}%
\pgfpathlineto{\pgfqpoint{5.439456in}{4.820608in}}%
\pgfpathlineto{\pgfqpoint{5.444898in}{4.663783in}}%
\pgfpathlineto{\pgfqpoint{5.455782in}{4.977432in}}%
\pgfpathlineto{\pgfqpoint{5.461224in}{4.977432in}}%
\pgfpathlineto{\pgfqpoint{5.466667in}{4.663783in}}%
\pgfpathlineto{\pgfqpoint{5.472109in}{4.663783in}}%
\pgfpathlineto{\pgfqpoint{5.477551in}{4.977432in}}%
\pgfpathlineto{\pgfqpoint{5.482993in}{4.820608in}}%
\pgfpathlineto{\pgfqpoint{5.488435in}{4.820608in}}%
\pgfpathlineto{\pgfqpoint{5.493878in}{4.977432in}}%
\pgfpathlineto{\pgfqpoint{5.499320in}{4.977432in}}%
\pgfpathlineto{\pgfqpoint{5.510204in}{4.663783in}}%
\pgfpathlineto{\pgfqpoint{5.515646in}{4.820608in}}%
\pgfpathlineto{\pgfqpoint{5.521088in}{4.663783in}}%
\pgfpathlineto{\pgfqpoint{5.526531in}{4.820608in}}%
\pgfpathlineto{\pgfqpoint{5.531973in}{4.663783in}}%
\pgfpathlineto{\pgfqpoint{5.537415in}{4.820608in}}%
\pgfpathlineto{\pgfqpoint{5.548299in}{4.820608in}}%
\pgfpathlineto{\pgfqpoint{5.553741in}{4.977432in}}%
\pgfpathlineto{\pgfqpoint{5.559184in}{4.663783in}}%
\pgfpathlineto{\pgfqpoint{5.564626in}{4.977432in}}%
\pgfpathlineto{\pgfqpoint{5.570068in}{4.506959in}}%
\pgfpathlineto{\pgfqpoint{5.575510in}{4.820608in}}%
\pgfpathlineto{\pgfqpoint{5.602721in}{4.820608in}}%
\pgfpathlineto{\pgfqpoint{5.608163in}{4.663783in}}%
\pgfpathlineto{\pgfqpoint{5.613605in}{4.663783in}}%
\pgfpathlineto{\pgfqpoint{5.624490in}{4.977432in}}%
\pgfpathlineto{\pgfqpoint{5.629932in}{4.977432in}}%
\pgfpathlineto{\pgfqpoint{5.635374in}{4.820608in}}%
\pgfpathlineto{\pgfqpoint{5.640816in}{4.820608in}}%
\pgfpathlineto{\pgfqpoint{5.646259in}{4.977432in}}%
\pgfpathlineto{\pgfqpoint{5.657143in}{4.663783in}}%
\pgfpathlineto{\pgfqpoint{5.662585in}{4.977432in}}%
\pgfpathlineto{\pgfqpoint{5.668027in}{4.820608in}}%
\pgfpathlineto{\pgfqpoint{5.678912in}{4.820608in}}%
\pgfpathlineto{\pgfqpoint{5.684354in}{4.663783in}}%
\pgfpathlineto{\pgfqpoint{5.695238in}{4.977432in}}%
\pgfpathlineto{\pgfqpoint{5.700680in}{4.663783in}}%
\pgfpathlineto{\pgfqpoint{5.706122in}{4.663783in}}%
\pgfpathlineto{\pgfqpoint{5.711565in}{4.820608in}}%
\pgfpathlineto{\pgfqpoint{5.717007in}{4.663783in}}%
\pgfpathlineto{\pgfqpoint{5.722449in}{4.820608in}}%
\pgfpathlineto{\pgfqpoint{5.727891in}{4.820608in}}%
\pgfpathlineto{\pgfqpoint{5.733333in}{4.977432in}}%
\pgfpathlineto{\pgfqpoint{5.738776in}{4.820608in}}%
\pgfpathlineto{\pgfqpoint{5.744218in}{4.977432in}}%
\pgfpathlineto{\pgfqpoint{5.749660in}{4.506959in}}%
\pgfpathlineto{\pgfqpoint{5.755102in}{4.820608in}}%
\pgfpathlineto{\pgfqpoint{5.760544in}{4.663783in}}%
\pgfpathlineto{\pgfqpoint{5.765986in}{4.977432in}}%
\pgfpathlineto{\pgfqpoint{5.771429in}{4.977432in}}%
\pgfpathlineto{\pgfqpoint{5.776871in}{4.663783in}}%
\pgfpathlineto{\pgfqpoint{5.782313in}{4.820608in}}%
\pgfpathlineto{\pgfqpoint{5.787755in}{4.663783in}}%
\pgfpathlineto{\pgfqpoint{5.793197in}{4.663783in}}%
\pgfpathlineto{\pgfqpoint{5.798639in}{4.506959in}}%
\pgfpathlineto{\pgfqpoint{5.804082in}{4.820608in}}%
\pgfpathlineto{\pgfqpoint{5.809524in}{4.977432in}}%
\pgfpathlineto{\pgfqpoint{5.814966in}{4.663783in}}%
\pgfpathlineto{\pgfqpoint{5.820408in}{4.663783in}}%
\pgfpathlineto{\pgfqpoint{5.825850in}{4.820608in}}%
\pgfpathlineto{\pgfqpoint{5.831293in}{4.820608in}}%
\pgfpathlineto{\pgfqpoint{5.836735in}{4.663783in}}%
\pgfpathlineto{\pgfqpoint{5.842177in}{4.663783in}}%
\pgfpathlineto{\pgfqpoint{5.847619in}{4.820608in}}%
\pgfpathlineto{\pgfqpoint{5.853061in}{4.506959in}}%
\pgfpathlineto{\pgfqpoint{5.858503in}{4.977432in}}%
\pgfpathlineto{\pgfqpoint{5.863946in}{4.977432in}}%
\pgfpathlineto{\pgfqpoint{5.874830in}{4.663783in}}%
\pgfpathlineto{\pgfqpoint{5.885714in}{4.977432in}}%
\pgfpathlineto{\pgfqpoint{5.891156in}{4.506959in}}%
\pgfpathlineto{\pgfqpoint{5.896599in}{4.977432in}}%
\pgfpathlineto{\pgfqpoint{5.902041in}{4.663783in}}%
\pgfpathlineto{\pgfqpoint{5.907483in}{4.977432in}}%
\pgfpathlineto{\pgfqpoint{5.923810in}{4.506959in}}%
\pgfpathlineto{\pgfqpoint{5.934694in}{4.820608in}}%
\pgfpathlineto{\pgfqpoint{5.940136in}{4.820608in}}%
\pgfpathlineto{\pgfqpoint{5.945578in}{4.663783in}}%
\pgfpathlineto{\pgfqpoint{5.951020in}{4.820608in}}%
\pgfpathlineto{\pgfqpoint{5.956463in}{4.663783in}}%
\pgfpathlineto{\pgfqpoint{5.961905in}{4.820608in}}%
\pgfpathlineto{\pgfqpoint{5.967347in}{5.134256in}}%
\pgfpathlineto{\pgfqpoint{5.972789in}{4.820608in}}%
\pgfpathlineto{\pgfqpoint{5.978231in}{4.820608in}}%
\pgfpathlineto{\pgfqpoint{5.983673in}{4.663783in}}%
\pgfpathlineto{\pgfqpoint{5.989116in}{4.663783in}}%
\pgfpathlineto{\pgfqpoint{5.994558in}{4.820608in}}%
\pgfpathlineto{\pgfqpoint{6.000000in}{4.663783in}}%
\pgfpathlineto{\pgfqpoint{6.005442in}{4.820608in}}%
\pgfpathlineto{\pgfqpoint{6.010884in}{4.663783in}}%
\pgfpathlineto{\pgfqpoint{6.016327in}{4.820608in}}%
\pgfpathlineto{\pgfqpoint{6.027211in}{4.506959in}}%
\pgfpathlineto{\pgfqpoint{6.032653in}{4.820608in}}%
\pgfpathlineto{\pgfqpoint{6.038095in}{4.663783in}}%
\pgfpathlineto{\pgfqpoint{6.043537in}{4.977432in}}%
\pgfpathlineto{\pgfqpoint{6.048980in}{4.663783in}}%
\pgfpathlineto{\pgfqpoint{6.054422in}{4.663783in}}%
\pgfpathlineto{\pgfqpoint{6.059864in}{4.820608in}}%
\pgfpathlineto{\pgfqpoint{6.065306in}{4.820608in}}%
\pgfpathlineto{\pgfqpoint{6.070748in}{4.977432in}}%
\pgfpathlineto{\pgfqpoint{6.076190in}{4.663783in}}%
\pgfpathlineto{\pgfqpoint{6.081633in}{4.820608in}}%
\pgfpathlineto{\pgfqpoint{6.092517in}{4.820608in}}%
\pgfpathlineto{\pgfqpoint{6.097959in}{4.977432in}}%
\pgfpathlineto{\pgfqpoint{6.103401in}{4.663783in}}%
\pgfpathlineto{\pgfqpoint{6.114286in}{4.977432in}}%
\pgfpathlineto{\pgfqpoint{6.119728in}{4.820608in}}%
\pgfpathlineto{\pgfqpoint{6.125170in}{4.820608in}}%
\pgfpathlineto{\pgfqpoint{6.130612in}{4.663783in}}%
\pgfpathlineto{\pgfqpoint{6.136054in}{4.977432in}}%
\pgfpathlineto{\pgfqpoint{6.141497in}{4.820608in}}%
\pgfpathlineto{\pgfqpoint{6.146939in}{4.820608in}}%
\pgfpathlineto{\pgfqpoint{6.157823in}{4.506959in}}%
\pgfpathlineto{\pgfqpoint{6.163265in}{4.977432in}}%
\pgfpathlineto{\pgfqpoint{6.168707in}{4.977432in}}%
\pgfpathlineto{\pgfqpoint{6.174150in}{4.350135in}}%
\pgfpathlineto{\pgfqpoint{6.179592in}{4.977432in}}%
\pgfpathlineto{\pgfqpoint{6.185034in}{4.663783in}}%
\pgfpathlineto{\pgfqpoint{6.190476in}{4.977432in}}%
\pgfpathlineto{\pgfqpoint{6.195918in}{4.820608in}}%
\pgfpathlineto{\pgfqpoint{6.201361in}{5.134256in}}%
\pgfpathlineto{\pgfqpoint{6.206803in}{4.977432in}}%
\pgfpathlineto{\pgfqpoint{6.212245in}{4.663783in}}%
\pgfpathlineto{\pgfqpoint{6.223129in}{4.977432in}}%
\pgfpathlineto{\pgfqpoint{6.228571in}{4.820608in}}%
\pgfpathlineto{\pgfqpoint{6.234014in}{4.977432in}}%
\pgfpathlineto{\pgfqpoint{6.239456in}{4.663783in}}%
\pgfpathlineto{\pgfqpoint{6.244898in}{5.134256in}}%
\pgfpathlineto{\pgfqpoint{6.250340in}{4.977432in}}%
\pgfpathlineto{\pgfqpoint{6.255782in}{4.506959in}}%
\pgfpathlineto{\pgfqpoint{6.266667in}{4.820608in}}%
\pgfpathlineto{\pgfqpoint{6.277551in}{4.506959in}}%
\pgfpathlineto{\pgfqpoint{6.282993in}{5.134256in}}%
\pgfpathlineto{\pgfqpoint{6.288435in}{4.820608in}}%
\pgfpathlineto{\pgfqpoint{6.293878in}{4.820608in}}%
\pgfpathlineto{\pgfqpoint{6.304762in}{4.506959in}}%
\pgfpathlineto{\pgfqpoint{6.310204in}{4.977432in}}%
\pgfpathlineto{\pgfqpoint{6.315646in}{4.820608in}}%
\pgfpathlineto{\pgfqpoint{6.321088in}{4.977432in}}%
\pgfpathlineto{\pgfqpoint{6.326531in}{4.977432in}}%
\pgfpathlineto{\pgfqpoint{6.331973in}{4.663783in}}%
\pgfpathlineto{\pgfqpoint{6.337415in}{4.663783in}}%
\pgfpathlineto{\pgfqpoint{6.342857in}{4.820608in}}%
\pgfpathlineto{\pgfqpoint{6.348299in}{4.663783in}}%
\pgfpathlineto{\pgfqpoint{6.353741in}{4.977432in}}%
\pgfpathlineto{\pgfqpoint{6.359184in}{4.506959in}}%
\pgfpathlineto{\pgfqpoint{6.364626in}{4.820608in}}%
\pgfpathlineto{\pgfqpoint{6.375510in}{4.820608in}}%
\pgfpathlineto{\pgfqpoint{6.380952in}{5.134256in}}%
\pgfpathlineto{\pgfqpoint{6.386395in}{4.977432in}}%
\pgfpathlineto{\pgfqpoint{6.391837in}{4.663783in}}%
\pgfpathlineto{\pgfqpoint{6.397279in}{4.977432in}}%
\pgfpathlineto{\pgfqpoint{6.402721in}{4.820608in}}%
\pgfpathlineto{\pgfqpoint{6.408163in}{4.977432in}}%
\pgfpathlineto{\pgfqpoint{6.413605in}{4.663783in}}%
\pgfpathlineto{\pgfqpoint{6.419048in}{4.977432in}}%
\pgfpathlineto{\pgfqpoint{6.429932in}{4.663783in}}%
\pgfpathlineto{\pgfqpoint{6.440816in}{4.977432in}}%
\pgfpathlineto{\pgfqpoint{6.446259in}{4.663783in}}%
\pgfpathlineto{\pgfqpoint{6.451701in}{4.506959in}}%
\pgfpathlineto{\pgfqpoint{6.457143in}{4.820608in}}%
\pgfpathlineto{\pgfqpoint{6.462585in}{4.977432in}}%
\pgfpathlineto{\pgfqpoint{6.468027in}{4.820608in}}%
\pgfpathlineto{\pgfqpoint{6.473469in}{4.977432in}}%
\pgfpathlineto{\pgfqpoint{6.478912in}{4.820608in}}%
\pgfpathlineto{\pgfqpoint{6.484354in}{4.820608in}}%
\pgfpathlineto{\pgfqpoint{6.489796in}{5.134256in}}%
\pgfpathlineto{\pgfqpoint{6.495238in}{4.977432in}}%
\pgfpathlineto{\pgfqpoint{6.506122in}{4.977432in}}%
\pgfpathlineto{\pgfqpoint{6.511565in}{4.663783in}}%
\pgfpathlineto{\pgfqpoint{6.517007in}{4.820608in}}%
\pgfpathlineto{\pgfqpoint{6.522449in}{4.820608in}}%
\pgfpathlineto{\pgfqpoint{6.533333in}{4.506959in}}%
\pgfpathlineto{\pgfqpoint{6.538776in}{5.134256in}}%
\pgfpathlineto{\pgfqpoint{6.544218in}{4.663783in}}%
\pgfpathlineto{\pgfqpoint{6.549660in}{4.506959in}}%
\pgfpathlineto{\pgfqpoint{6.555102in}{4.663783in}}%
\pgfpathlineto{\pgfqpoint{6.560544in}{4.663783in}}%
\pgfpathlineto{\pgfqpoint{6.565986in}{4.977432in}}%
\pgfpathlineto{\pgfqpoint{6.571429in}{4.663783in}}%
\pgfpathlineto{\pgfqpoint{6.582313in}{4.663783in}}%
\pgfpathlineto{\pgfqpoint{6.587755in}{5.134256in}}%
\pgfpathlineto{\pgfqpoint{6.593197in}{4.663783in}}%
\pgfpathlineto{\pgfqpoint{6.598639in}{4.820608in}}%
\pgfpathlineto{\pgfqpoint{6.604082in}{4.820608in}}%
\pgfpathlineto{\pgfqpoint{6.609524in}{4.977432in}}%
\pgfpathlineto{\pgfqpoint{6.620408in}{4.663783in}}%
\pgfpathlineto{\pgfqpoint{6.625850in}{4.820608in}}%
\pgfpathlineto{\pgfqpoint{6.631293in}{4.506959in}}%
\pgfpathlineto{\pgfqpoint{6.636735in}{4.977432in}}%
\pgfpathlineto{\pgfqpoint{6.642177in}{4.820608in}}%
\pgfpathlineto{\pgfqpoint{6.653061in}{4.820608in}}%
\pgfpathlineto{\pgfqpoint{6.658503in}{5.134256in}}%
\pgfpathlineto{\pgfqpoint{6.663946in}{4.977432in}}%
\pgfpathlineto{\pgfqpoint{6.669388in}{4.977432in}}%
\pgfpathlineto{\pgfqpoint{6.674830in}{4.820608in}}%
\pgfpathlineto{\pgfqpoint{6.685714in}{4.820608in}}%
\pgfpathlineto{\pgfqpoint{6.691156in}{4.663783in}}%
\pgfpathlineto{\pgfqpoint{6.696599in}{4.663783in}}%
\pgfpathlineto{\pgfqpoint{6.702041in}{4.193310in}}%
\pgfpathlineto{\pgfqpoint{6.707483in}{4.820608in}}%
\pgfpathlineto{\pgfqpoint{6.712925in}{4.820608in}}%
\pgfpathlineto{\pgfqpoint{6.718367in}{4.977432in}}%
\pgfpathlineto{\pgfqpoint{6.723810in}{4.977432in}}%
\pgfpathlineto{\pgfqpoint{6.729252in}{4.820608in}}%
\pgfpathlineto{\pgfqpoint{6.751020in}{4.820608in}}%
\pgfpathlineto{\pgfqpoint{6.756463in}{4.663783in}}%
\pgfpathlineto{\pgfqpoint{6.761905in}{4.977432in}}%
\pgfpathlineto{\pgfqpoint{6.767347in}{4.977432in}}%
\pgfpathlineto{\pgfqpoint{6.772789in}{4.663783in}}%
\pgfpathlineto{\pgfqpoint{6.778231in}{4.663783in}}%
\pgfpathlineto{\pgfqpoint{6.783673in}{4.977432in}}%
\pgfpathlineto{\pgfqpoint{6.789116in}{4.977432in}}%
\pgfpathlineto{\pgfqpoint{6.794558in}{4.820608in}}%
\pgfpathlineto{\pgfqpoint{6.794558in}{4.820608in}}%
\pgfusepath{stroke}%
\end{pgfscope}%
\begin{pgfscope}%
\pgfsetrectcap%
\pgfsetmiterjoin%
\pgfsetlinewidth{0.803000pt}%
\definecolor{currentstroke}{rgb}{0.000000,0.000000,0.000000}%
\pgfsetstrokecolor{currentstroke}%
\pgfsetdash{}{0pt}%
\pgfpathmoveto{\pgfqpoint{1.200000in}{0.900000in}}%
\pgfpathlineto{\pgfqpoint{1.200000in}{5.700000in}}%
\pgfusepath{stroke}%
\end{pgfscope}%
\begin{pgfscope}%
\pgfsetrectcap%
\pgfsetmiterjoin%
\pgfsetlinewidth{0.803000pt}%
\definecolor{currentstroke}{rgb}{0.000000,0.000000,0.000000}%
\pgfsetstrokecolor{currentstroke}%
\pgfsetdash{}{0pt}%
\pgfpathmoveto{\pgfqpoint{6.800000in}{0.900000in}}%
\pgfpathlineto{\pgfqpoint{6.800000in}{5.700000in}}%
\pgfusepath{stroke}%
\end{pgfscope}%
\begin{pgfscope}%
\pgfsetrectcap%
\pgfsetmiterjoin%
\pgfsetlinewidth{0.803000pt}%
\definecolor{currentstroke}{rgb}{0.000000,0.000000,0.000000}%
\pgfsetstrokecolor{currentstroke}%
\pgfsetdash{}{0pt}%
\pgfpathmoveto{\pgfqpoint{1.200000in}{0.900000in}}%
\pgfpathlineto{\pgfqpoint{6.800000in}{0.900000in}}%
\pgfusepath{stroke}%
\end{pgfscope}%
\begin{pgfscope}%
\pgfsetrectcap%
\pgfsetmiterjoin%
\pgfsetlinewidth{0.803000pt}%
\definecolor{currentstroke}{rgb}{0.000000,0.000000,0.000000}%
\pgfsetstrokecolor{currentstroke}%
\pgfsetdash{}{0pt}%
\pgfpathmoveto{\pgfqpoint{1.200000in}{5.700000in}}%
\pgfpathlineto{\pgfqpoint{6.800000in}{5.700000in}}%
\pgfusepath{stroke}%
\end{pgfscope}%
\begin{pgfscope}%
\pgfsetbuttcap%
\pgfsetmiterjoin%
\definecolor{currentfill}{rgb}{1.000000,1.000000,1.000000}%
\pgfsetfillcolor{currentfill}%
\pgfsetfillopacity{0.800000}%
\pgfsetlinewidth{1.003750pt}%
\definecolor{currentstroke}{rgb}{0.800000,0.800000,0.800000}%
\pgfsetstrokecolor{currentstroke}%
\pgfsetstrokeopacity{0.800000}%
\pgfsetdash{}{0pt}%
\pgfpathmoveto{\pgfqpoint{1.394444in}{1.038889in}}%
\pgfpathlineto{\pgfqpoint{3.423128in}{1.038889in}}%
\pgfpathquadraticcurveto{\pgfqpoint{3.478683in}{1.038889in}}{\pgfqpoint{3.478683in}{1.094444in}}%
\pgfpathlineto{\pgfqpoint{3.478683in}{1.461623in}}%
\pgfpathquadraticcurveto{\pgfqpoint{3.478683in}{1.517179in}}{\pgfqpoint{3.423128in}{1.517179in}}%
\pgfpathlineto{\pgfqpoint{1.394444in}{1.517179in}}%
\pgfpathquadraticcurveto{\pgfqpoint{1.338889in}{1.517179in}}{\pgfqpoint{1.338889in}{1.461623in}}%
\pgfpathlineto{\pgfqpoint{1.338889in}{1.094444in}}%
\pgfpathquadraticcurveto{\pgfqpoint{1.338889in}{1.038889in}}{\pgfqpoint{1.394444in}{1.038889in}}%
\pgfpathclose%
\pgfusepath{stroke,fill}%
\end{pgfscope}%
\begin{pgfscope}%
\pgfsetrectcap%
\pgfsetroundjoin%
\pgfsetlinewidth{2.007500pt}%
\definecolor{currentstroke}{rgb}{0.121569,0.466667,0.705882}%
\pgfsetstrokecolor{currentstroke}%
\pgfsetdash{}{0pt}%
\pgfpathmoveto{\pgfqpoint{1.450000in}{1.303251in}}%
\pgfpathlineto{\pgfqpoint{2.005556in}{1.303251in}}%
\pgfusepath{stroke}%
\end{pgfscope}%
\begin{pgfscope}%
\definecolor{textcolor}{rgb}{0.000000,0.000000,0.000000}%
\pgfsetstrokecolor{textcolor}%
\pgfsetfillcolor{textcolor}%
\pgftext[x=2.227778in,y=1.206029in,left,base]{\color{textcolor}\sffamily\fontsize{20.000000}{24.000000}\selectfont Waveform}%
\end{pgfscope}%
\end{pgfpicture}%
\makeatother%
\endgroup%
}
    \caption{A PMT Waveform}
\end{figure}
\end{columns}
\begin{block}{}
Waveform analysis, which means extracting time and charge information from PMT waveforms, is the bedrock of subsequent analysis such as event reconstruction. 
\end{block}
\end{frame}

\section{Dataset}

\begin{frame}
\frametitle{Simulation setup: Time profile}
\begin{figure}
    \centering
    \resizebox{0.6\textwidth}{!}{%% Creator: Matplotlib, PGF backend
%%
%% To include the figure in your LaTeX document, write
%%   \input{<filename>.pgf}
%%
%% Make sure the required packages are loaded in your preamble
%%   \usepackage{pgf}
%%
%% and, on pdftex
%%   \usepackage[utf8]{inputenc}\DeclareUnicodeCharacter{2212}{-}
%%
%% or, on luatex and xetex
%%   \usepackage{unicode-math}
%%
%% Figures using additional raster images can only be included by \input if
%% they are in the same directory as the main LaTeX file. For loading figures
%% from other directories you can use the `import` package
%%   \usepackage{import}
%%
%% and then include the figures with
%%   \import{<path to file>}{<filename>.pgf}
%%
%% Matplotlib used the following preamble
%%
\begingroup%
\makeatletter%
\begin{pgfpicture}%
\pgfpathrectangle{\pgfpointorigin}{\pgfqpoint{8.000000in}{6.000000in}}%
\pgfusepath{use as bounding box, clip}%
\begin{pgfscope}%
\pgfsetbuttcap%
\pgfsetmiterjoin%
\definecolor{currentfill}{rgb}{1.000000,1.000000,1.000000}%
\pgfsetfillcolor{currentfill}%
\pgfsetlinewidth{0.000000pt}%
\definecolor{currentstroke}{rgb}{1.000000,1.000000,1.000000}%
\pgfsetstrokecolor{currentstroke}%
\pgfsetdash{}{0pt}%
\pgfpathmoveto{\pgfqpoint{0.000000in}{0.000000in}}%
\pgfpathlineto{\pgfqpoint{8.000000in}{0.000000in}}%
\pgfpathlineto{\pgfqpoint{8.000000in}{6.000000in}}%
\pgfpathlineto{\pgfqpoint{0.000000in}{6.000000in}}%
\pgfpathclose%
\pgfusepath{fill}%
\end{pgfscope}%
\begin{pgfscope}%
\pgfsetbuttcap%
\pgfsetmiterjoin%
\definecolor{currentfill}{rgb}{1.000000,1.000000,1.000000}%
\pgfsetfillcolor{currentfill}%
\pgfsetlinewidth{0.000000pt}%
\definecolor{currentstroke}{rgb}{0.000000,0.000000,0.000000}%
\pgfsetstrokecolor{currentstroke}%
\pgfsetstrokeopacity{0.000000}%
\pgfsetdash{}{0pt}%
\pgfpathmoveto{\pgfqpoint{1.200000in}{0.900000in}}%
\pgfpathlineto{\pgfqpoint{6.800000in}{0.900000in}}%
\pgfpathlineto{\pgfqpoint{6.800000in}{5.700000in}}%
\pgfpathlineto{\pgfqpoint{1.200000in}{5.700000in}}%
\pgfpathclose%
\pgfusepath{fill}%
\end{pgfscope}%
\begin{pgfscope}%
\pgfpathrectangle{\pgfqpoint{1.200000in}{0.900000in}}{\pgfqpoint{5.600000in}{4.800000in}}%
\pgfusepath{clip}%
\pgfsetrectcap%
\pgfsetroundjoin%
\pgfsetlinewidth{2.007500pt}%
\definecolor{currentstroke}{rgb}{0.000000,0.000000,1.000000}%
\pgfsetstrokecolor{currentstroke}%
\pgfsetdash{}{0pt}%
\pgfpathmoveto{\pgfqpoint{1.305660in}{1.883216in}}%
\pgfpathlineto{\pgfqpoint{1.326792in}{2.036248in}}%
\pgfpathlineto{\pgfqpoint{1.347925in}{2.202479in}}%
\pgfpathlineto{\pgfqpoint{1.379623in}{2.474750in}}%
\pgfpathlineto{\pgfqpoint{1.411321in}{2.770719in}}%
\pgfpathlineto{\pgfqpoint{1.453585in}{3.191895in}}%
\pgfpathlineto{\pgfqpoint{1.548679in}{4.159094in}}%
\pgfpathlineto{\pgfqpoint{1.580377in}{4.456532in}}%
\pgfpathlineto{\pgfqpoint{1.612075in}{4.727863in}}%
\pgfpathlineto{\pgfqpoint{1.633208in}{4.890990in}}%
\pgfpathlineto{\pgfqpoint{1.654340in}{5.038107in}}%
\pgfpathlineto{\pgfqpoint{1.675472in}{5.168088in}}%
\pgfpathlineto{\pgfqpoint{1.696604in}{5.280190in}}%
\pgfpathlineto{\pgfqpoint{1.717736in}{5.374061in}}%
\pgfpathlineto{\pgfqpoint{1.728302in}{5.414151in}}%
\pgfpathlineto{\pgfqpoint{1.738868in}{5.449716in}}%
\pgfpathlineto{\pgfqpoint{1.749434in}{5.480812in}}%
\pgfpathlineto{\pgfqpoint{1.760000in}{5.507512in}}%
\pgfpathlineto{\pgfqpoint{1.770566in}{5.529908in}}%
\pgfpathlineto{\pgfqpoint{1.781132in}{5.548108in}}%
\pgfpathlineto{\pgfqpoint{1.791698in}{5.562234in}}%
\pgfpathlineto{\pgfqpoint{1.802264in}{5.572421in}}%
\pgfpathlineto{\pgfqpoint{1.812830in}{5.578813in}}%
\pgfpathlineto{\pgfqpoint{1.823396in}{5.581568in}}%
\pgfpathlineto{\pgfqpoint{1.833962in}{5.580847in}}%
\pgfpathlineto{\pgfqpoint{1.844528in}{5.576820in}}%
\pgfpathlineto{\pgfqpoint{1.855094in}{5.569661in}}%
\pgfpathlineto{\pgfqpoint{1.865660in}{5.559546in}}%
\pgfpathlineto{\pgfqpoint{1.876226in}{5.546654in}}%
\pgfpathlineto{\pgfqpoint{1.886792in}{5.531163in}}%
\pgfpathlineto{\pgfqpoint{1.897358in}{5.513251in}}%
\pgfpathlineto{\pgfqpoint{1.918491in}{5.470862in}}%
\pgfpathlineto{\pgfqpoint{1.939623in}{5.420846in}}%
\pgfpathlineto{\pgfqpoint{1.960755in}{5.364485in}}%
\pgfpathlineto{\pgfqpoint{1.992453in}{5.270608in}}%
\pgfpathlineto{\pgfqpoint{2.024151in}{5.168617in}}%
\pgfpathlineto{\pgfqpoint{2.066415in}{5.024973in}}%
\pgfpathlineto{\pgfqpoint{2.235472in}{4.441688in}}%
\pgfpathlineto{\pgfqpoint{2.288302in}{4.270330in}}%
\pgfpathlineto{\pgfqpoint{2.341132in}{4.106510in}}%
\pgfpathlineto{\pgfqpoint{2.393962in}{3.950336in}}%
\pgfpathlineto{\pgfqpoint{2.446792in}{3.801644in}}%
\pgfpathlineto{\pgfqpoint{2.499623in}{3.660154in}}%
\pgfpathlineto{\pgfqpoint{2.552453in}{3.525548in}}%
\pgfpathlineto{\pgfqpoint{2.605283in}{3.397501in}}%
\pgfpathlineto{\pgfqpoint{2.658113in}{3.275697in}}%
\pgfpathlineto{\pgfqpoint{2.710943in}{3.159833in}}%
\pgfpathlineto{\pgfqpoint{2.763774in}{3.049620in}}%
\pgfpathlineto{\pgfqpoint{2.816604in}{2.944781in}}%
\pgfpathlineto{\pgfqpoint{2.869434in}{2.845056in}}%
\pgfpathlineto{\pgfqpoint{2.922264in}{2.750195in}}%
\pgfpathlineto{\pgfqpoint{2.975094in}{2.659960in}}%
\pgfpathlineto{\pgfqpoint{3.027925in}{2.574125in}}%
\pgfpathlineto{\pgfqpoint{3.080755in}{2.492477in}}%
\pgfpathlineto{\pgfqpoint{3.133585in}{2.414811in}}%
\pgfpathlineto{\pgfqpoint{3.186415in}{2.340933in}}%
\pgfpathlineto{\pgfqpoint{3.239245in}{2.270658in}}%
\pgfpathlineto{\pgfqpoint{3.292075in}{2.203810in}}%
\pgfpathlineto{\pgfqpoint{3.344906in}{2.140223in}}%
\pgfpathlineto{\pgfqpoint{3.397736in}{2.079736in}}%
\pgfpathlineto{\pgfqpoint{3.450566in}{2.022200in}}%
\pgfpathlineto{\pgfqpoint{3.503396in}{1.967469in}}%
\pgfpathlineto{\pgfqpoint{3.566792in}{1.905305in}}%
\pgfpathlineto{\pgfqpoint{3.630189in}{1.846760in}}%
\pgfpathlineto{\pgfqpoint{3.693585in}{1.791625in}}%
\pgfpathlineto{\pgfqpoint{3.756981in}{1.739701in}}%
\pgfpathlineto{\pgfqpoint{3.820377in}{1.690801in}}%
\pgfpathlineto{\pgfqpoint{3.883774in}{1.644748in}}%
\pgfpathlineto{\pgfqpoint{3.947170in}{1.601377in}}%
\pgfpathlineto{\pgfqpoint{4.010566in}{1.560532in}}%
\pgfpathlineto{\pgfqpoint{4.073962in}{1.522066in}}%
\pgfpathlineto{\pgfqpoint{4.147925in}{1.480010in}}%
\pgfpathlineto{\pgfqpoint{4.221887in}{1.440798in}}%
\pgfpathlineto{\pgfqpoint{4.295849in}{1.404237in}}%
\pgfpathlineto{\pgfqpoint{4.369811in}{1.370147in}}%
\pgfpathlineto{\pgfqpoint{4.443774in}{1.338362in}}%
\pgfpathlineto{\pgfqpoint{4.528302in}{1.304660in}}%
\pgfpathlineto{\pgfqpoint{4.612830in}{1.273548in}}%
\pgfpathlineto{\pgfqpoint{4.697358in}{1.244828in}}%
\pgfpathlineto{\pgfqpoint{4.781887in}{1.218316in}}%
\pgfpathlineto{\pgfqpoint{4.876981in}{1.190919in}}%
\pgfpathlineto{\pgfqpoint{4.972075in}{1.165880in}}%
\pgfpathlineto{\pgfqpoint{5.067170in}{1.142996in}}%
\pgfpathlineto{\pgfqpoint{5.172830in}{1.119872in}}%
\pgfpathlineto{\pgfqpoint{5.278491in}{1.098949in}}%
\pgfpathlineto{\pgfqpoint{5.394717in}{1.078225in}}%
\pgfpathlineto{\pgfqpoint{5.510943in}{1.059660in}}%
\pgfpathlineto{\pgfqpoint{5.637736in}{1.041606in}}%
\pgfpathlineto{\pgfqpoint{5.775094in}{1.024343in}}%
\pgfpathlineto{\pgfqpoint{5.923019in}{1.008099in}}%
\pgfpathlineto{\pgfqpoint{6.081509in}{0.993042in}}%
\pgfpathlineto{\pgfqpoint{6.250566in}{0.979285in}}%
\pgfpathlineto{\pgfqpoint{6.430189in}{0.966890in}}%
\pgfpathlineto{\pgfqpoint{6.630943in}{0.955315in}}%
\pgfpathlineto{\pgfqpoint{6.789434in}{0.947610in}}%
\pgfpathlineto{\pgfqpoint{6.789434in}{0.947610in}}%
\pgfusepath{stroke}%
\end{pgfscope}%
\begin{pgfscope}%
\pgfsetrectcap%
\pgfsetmiterjoin%
\pgfsetlinewidth{1.003750pt}%
\definecolor{currentstroke}{rgb}{0.000000,0.000000,0.000000}%
\pgfsetstrokecolor{currentstroke}%
\pgfsetdash{}{0pt}%
\pgfpathmoveto{\pgfqpoint{1.200000in}{0.900000in}}%
\pgfpathlineto{\pgfqpoint{1.200000in}{5.700000in}}%
\pgfusepath{stroke}%
\end{pgfscope}%
\begin{pgfscope}%
\pgfsetrectcap%
\pgfsetmiterjoin%
\pgfsetlinewidth{1.003750pt}%
\definecolor{currentstroke}{rgb}{0.000000,0.000000,0.000000}%
\pgfsetstrokecolor{currentstroke}%
\pgfsetdash{}{0pt}%
\pgfpathmoveto{\pgfqpoint{6.800000in}{0.900000in}}%
\pgfpathlineto{\pgfqpoint{6.800000in}{5.700000in}}%
\pgfusepath{stroke}%
\end{pgfscope}%
\begin{pgfscope}%
\pgfsetrectcap%
\pgfsetmiterjoin%
\pgfsetlinewidth{1.003750pt}%
\definecolor{currentstroke}{rgb}{0.000000,0.000000,0.000000}%
\pgfsetstrokecolor{currentstroke}%
\pgfsetdash{}{0pt}%
\pgfpathmoveto{\pgfqpoint{1.200000in}{0.900000in}}%
\pgfpathlineto{\pgfqpoint{6.800000in}{0.900000in}}%
\pgfusepath{stroke}%
\end{pgfscope}%
\begin{pgfscope}%
\pgfsetrectcap%
\pgfsetmiterjoin%
\pgfsetlinewidth{1.003750pt}%
\definecolor{currentstroke}{rgb}{0.000000,0.000000,0.000000}%
\pgfsetstrokecolor{currentstroke}%
\pgfsetdash{}{0pt}%
\pgfpathmoveto{\pgfqpoint{1.200000in}{5.700000in}}%
\pgfpathlineto{\pgfqpoint{6.800000in}{5.700000in}}%
\pgfusepath{stroke}%
\end{pgfscope}%
\begin{pgfscope}%
\pgfpathrectangle{\pgfqpoint{1.200000in}{0.900000in}}{\pgfqpoint{5.600000in}{4.800000in}}%
\pgfusepath{clip}%
\pgfsetbuttcap%
\pgfsetroundjoin%
\pgfsetlinewidth{0.501875pt}%
\definecolor{currentstroke}{rgb}{0.000000,0.000000,0.000000}%
\pgfsetstrokecolor{currentstroke}%
\pgfsetdash{{1.000000pt}{3.000000pt}}{0.000000pt}%
\pgfpathmoveto{\pgfqpoint{1.516981in}{0.900000in}}%
\pgfpathlineto{\pgfqpoint{1.516981in}{5.700000in}}%
\pgfusepath{stroke}%
\end{pgfscope}%
\begin{pgfscope}%
\pgfsetbuttcap%
\pgfsetroundjoin%
\definecolor{currentfill}{rgb}{0.000000,0.000000,0.000000}%
\pgfsetfillcolor{currentfill}%
\pgfsetlinewidth{0.501875pt}%
\definecolor{currentstroke}{rgb}{0.000000,0.000000,0.000000}%
\pgfsetstrokecolor{currentstroke}%
\pgfsetdash{}{0pt}%
\pgfsys@defobject{currentmarker}{\pgfqpoint{0.000000in}{0.000000in}}{\pgfqpoint{0.000000in}{0.055556in}}{%
\pgfpathmoveto{\pgfqpoint{0.000000in}{0.000000in}}%
\pgfpathlineto{\pgfqpoint{0.000000in}{0.055556in}}%
\pgfusepath{stroke,fill}%
}%
\begin{pgfscope}%
\pgfsys@transformshift{1.516981in}{0.900000in}%
\pgfsys@useobject{currentmarker}{}%
\end{pgfscope}%
\end{pgfscope}%
\begin{pgfscope}%
\pgfsetbuttcap%
\pgfsetroundjoin%
\definecolor{currentfill}{rgb}{0.000000,0.000000,0.000000}%
\pgfsetfillcolor{currentfill}%
\pgfsetlinewidth{0.501875pt}%
\definecolor{currentstroke}{rgb}{0.000000,0.000000,0.000000}%
\pgfsetstrokecolor{currentstroke}%
\pgfsetdash{}{0pt}%
\pgfsys@defobject{currentmarker}{\pgfqpoint{0.000000in}{-0.055556in}}{\pgfqpoint{0.000000in}{0.000000in}}{%
\pgfpathmoveto{\pgfqpoint{0.000000in}{0.000000in}}%
\pgfpathlineto{\pgfqpoint{0.000000in}{-0.055556in}}%
\pgfusepath{stroke,fill}%
}%
\begin{pgfscope}%
\pgfsys@transformshift{1.516981in}{5.700000in}%
\pgfsys@useobject{currentmarker}{}%
\end{pgfscope}%
\end{pgfscope}%
\begin{pgfscope}%
\definecolor{textcolor}{rgb}{0.000000,0.000000,0.000000}%
\pgfsetstrokecolor{textcolor}%
\pgfsetfillcolor{textcolor}%
\pgftext[x=1.516981in,y=0.844444in,,top]{\color{textcolor}\sffamily\fontsize{20.000000}{24.000000}\selectfont \(\displaystyle {0}\)}%
\end{pgfscope}%
\begin{pgfscope}%
\pgfpathrectangle{\pgfqpoint{1.200000in}{0.900000in}}{\pgfqpoint{5.600000in}{4.800000in}}%
\pgfusepath{clip}%
\pgfsetbuttcap%
\pgfsetroundjoin%
\pgfsetlinewidth{0.501875pt}%
\definecolor{currentstroke}{rgb}{0.000000,0.000000,0.000000}%
\pgfsetstrokecolor{currentstroke}%
\pgfsetdash{{1.000000pt}{3.000000pt}}{0.000000pt}%
\pgfpathmoveto{\pgfqpoint{2.573585in}{0.900000in}}%
\pgfpathlineto{\pgfqpoint{2.573585in}{5.700000in}}%
\pgfusepath{stroke}%
\end{pgfscope}%
\begin{pgfscope}%
\pgfsetbuttcap%
\pgfsetroundjoin%
\definecolor{currentfill}{rgb}{0.000000,0.000000,0.000000}%
\pgfsetfillcolor{currentfill}%
\pgfsetlinewidth{0.501875pt}%
\definecolor{currentstroke}{rgb}{0.000000,0.000000,0.000000}%
\pgfsetstrokecolor{currentstroke}%
\pgfsetdash{}{0pt}%
\pgfsys@defobject{currentmarker}{\pgfqpoint{0.000000in}{0.000000in}}{\pgfqpoint{0.000000in}{0.055556in}}{%
\pgfpathmoveto{\pgfqpoint{0.000000in}{0.000000in}}%
\pgfpathlineto{\pgfqpoint{0.000000in}{0.055556in}}%
\pgfusepath{stroke,fill}%
}%
\begin{pgfscope}%
\pgfsys@transformshift{2.573585in}{0.900000in}%
\pgfsys@useobject{currentmarker}{}%
\end{pgfscope}%
\end{pgfscope}%
\begin{pgfscope}%
\pgfsetbuttcap%
\pgfsetroundjoin%
\definecolor{currentfill}{rgb}{0.000000,0.000000,0.000000}%
\pgfsetfillcolor{currentfill}%
\pgfsetlinewidth{0.501875pt}%
\definecolor{currentstroke}{rgb}{0.000000,0.000000,0.000000}%
\pgfsetstrokecolor{currentstroke}%
\pgfsetdash{}{0pt}%
\pgfsys@defobject{currentmarker}{\pgfqpoint{0.000000in}{-0.055556in}}{\pgfqpoint{0.000000in}{0.000000in}}{%
\pgfpathmoveto{\pgfqpoint{0.000000in}{0.000000in}}%
\pgfpathlineto{\pgfqpoint{0.000000in}{-0.055556in}}%
\pgfusepath{stroke,fill}%
}%
\begin{pgfscope}%
\pgfsys@transformshift{2.573585in}{5.700000in}%
\pgfsys@useobject{currentmarker}{}%
\end{pgfscope}%
\end{pgfscope}%
\begin{pgfscope}%
\definecolor{textcolor}{rgb}{0.000000,0.000000,0.000000}%
\pgfsetstrokecolor{textcolor}%
\pgfsetfillcolor{textcolor}%
\pgftext[x=2.573585in,y=0.844444in,,top]{\color{textcolor}\sffamily\fontsize{20.000000}{24.000000}\selectfont \(\displaystyle {10}\)}%
\end{pgfscope}%
\begin{pgfscope}%
\pgfpathrectangle{\pgfqpoint{1.200000in}{0.900000in}}{\pgfqpoint{5.600000in}{4.800000in}}%
\pgfusepath{clip}%
\pgfsetbuttcap%
\pgfsetroundjoin%
\pgfsetlinewidth{0.501875pt}%
\definecolor{currentstroke}{rgb}{0.000000,0.000000,0.000000}%
\pgfsetstrokecolor{currentstroke}%
\pgfsetdash{{1.000000pt}{3.000000pt}}{0.000000pt}%
\pgfpathmoveto{\pgfqpoint{3.630189in}{0.900000in}}%
\pgfpathlineto{\pgfqpoint{3.630189in}{5.700000in}}%
\pgfusepath{stroke}%
\end{pgfscope}%
\begin{pgfscope}%
\pgfsetbuttcap%
\pgfsetroundjoin%
\definecolor{currentfill}{rgb}{0.000000,0.000000,0.000000}%
\pgfsetfillcolor{currentfill}%
\pgfsetlinewidth{0.501875pt}%
\definecolor{currentstroke}{rgb}{0.000000,0.000000,0.000000}%
\pgfsetstrokecolor{currentstroke}%
\pgfsetdash{}{0pt}%
\pgfsys@defobject{currentmarker}{\pgfqpoint{0.000000in}{0.000000in}}{\pgfqpoint{0.000000in}{0.055556in}}{%
\pgfpathmoveto{\pgfqpoint{0.000000in}{0.000000in}}%
\pgfpathlineto{\pgfqpoint{0.000000in}{0.055556in}}%
\pgfusepath{stroke,fill}%
}%
\begin{pgfscope}%
\pgfsys@transformshift{3.630189in}{0.900000in}%
\pgfsys@useobject{currentmarker}{}%
\end{pgfscope}%
\end{pgfscope}%
\begin{pgfscope}%
\pgfsetbuttcap%
\pgfsetroundjoin%
\definecolor{currentfill}{rgb}{0.000000,0.000000,0.000000}%
\pgfsetfillcolor{currentfill}%
\pgfsetlinewidth{0.501875pt}%
\definecolor{currentstroke}{rgb}{0.000000,0.000000,0.000000}%
\pgfsetstrokecolor{currentstroke}%
\pgfsetdash{}{0pt}%
\pgfsys@defobject{currentmarker}{\pgfqpoint{0.000000in}{-0.055556in}}{\pgfqpoint{0.000000in}{0.000000in}}{%
\pgfpathmoveto{\pgfqpoint{0.000000in}{0.000000in}}%
\pgfpathlineto{\pgfqpoint{0.000000in}{-0.055556in}}%
\pgfusepath{stroke,fill}%
}%
\begin{pgfscope}%
\pgfsys@transformshift{3.630189in}{5.700000in}%
\pgfsys@useobject{currentmarker}{}%
\end{pgfscope}%
\end{pgfscope}%
\begin{pgfscope}%
\definecolor{textcolor}{rgb}{0.000000,0.000000,0.000000}%
\pgfsetstrokecolor{textcolor}%
\pgfsetfillcolor{textcolor}%
\pgftext[x=3.630189in,y=0.844444in,,top]{\color{textcolor}\sffamily\fontsize{20.000000}{24.000000}\selectfont \(\displaystyle {20}\)}%
\end{pgfscope}%
\begin{pgfscope}%
\pgfpathrectangle{\pgfqpoint{1.200000in}{0.900000in}}{\pgfqpoint{5.600000in}{4.800000in}}%
\pgfusepath{clip}%
\pgfsetbuttcap%
\pgfsetroundjoin%
\pgfsetlinewidth{0.501875pt}%
\definecolor{currentstroke}{rgb}{0.000000,0.000000,0.000000}%
\pgfsetstrokecolor{currentstroke}%
\pgfsetdash{{1.000000pt}{3.000000pt}}{0.000000pt}%
\pgfpathmoveto{\pgfqpoint{4.686792in}{0.900000in}}%
\pgfpathlineto{\pgfqpoint{4.686792in}{5.700000in}}%
\pgfusepath{stroke}%
\end{pgfscope}%
\begin{pgfscope}%
\pgfsetbuttcap%
\pgfsetroundjoin%
\definecolor{currentfill}{rgb}{0.000000,0.000000,0.000000}%
\pgfsetfillcolor{currentfill}%
\pgfsetlinewidth{0.501875pt}%
\definecolor{currentstroke}{rgb}{0.000000,0.000000,0.000000}%
\pgfsetstrokecolor{currentstroke}%
\pgfsetdash{}{0pt}%
\pgfsys@defobject{currentmarker}{\pgfqpoint{0.000000in}{0.000000in}}{\pgfqpoint{0.000000in}{0.055556in}}{%
\pgfpathmoveto{\pgfqpoint{0.000000in}{0.000000in}}%
\pgfpathlineto{\pgfqpoint{0.000000in}{0.055556in}}%
\pgfusepath{stroke,fill}%
}%
\begin{pgfscope}%
\pgfsys@transformshift{4.686792in}{0.900000in}%
\pgfsys@useobject{currentmarker}{}%
\end{pgfscope}%
\end{pgfscope}%
\begin{pgfscope}%
\pgfsetbuttcap%
\pgfsetroundjoin%
\definecolor{currentfill}{rgb}{0.000000,0.000000,0.000000}%
\pgfsetfillcolor{currentfill}%
\pgfsetlinewidth{0.501875pt}%
\definecolor{currentstroke}{rgb}{0.000000,0.000000,0.000000}%
\pgfsetstrokecolor{currentstroke}%
\pgfsetdash{}{0pt}%
\pgfsys@defobject{currentmarker}{\pgfqpoint{0.000000in}{-0.055556in}}{\pgfqpoint{0.000000in}{0.000000in}}{%
\pgfpathmoveto{\pgfqpoint{0.000000in}{0.000000in}}%
\pgfpathlineto{\pgfqpoint{0.000000in}{-0.055556in}}%
\pgfusepath{stroke,fill}%
}%
\begin{pgfscope}%
\pgfsys@transformshift{4.686792in}{5.700000in}%
\pgfsys@useobject{currentmarker}{}%
\end{pgfscope}%
\end{pgfscope}%
\begin{pgfscope}%
\definecolor{textcolor}{rgb}{0.000000,0.000000,0.000000}%
\pgfsetstrokecolor{textcolor}%
\pgfsetfillcolor{textcolor}%
\pgftext[x=4.686792in,y=0.844444in,,top]{\color{textcolor}\sffamily\fontsize{20.000000}{24.000000}\selectfont \(\displaystyle {30}\)}%
\end{pgfscope}%
\begin{pgfscope}%
\pgfpathrectangle{\pgfqpoint{1.200000in}{0.900000in}}{\pgfqpoint{5.600000in}{4.800000in}}%
\pgfusepath{clip}%
\pgfsetbuttcap%
\pgfsetroundjoin%
\pgfsetlinewidth{0.501875pt}%
\definecolor{currentstroke}{rgb}{0.000000,0.000000,0.000000}%
\pgfsetstrokecolor{currentstroke}%
\pgfsetdash{{1.000000pt}{3.000000pt}}{0.000000pt}%
\pgfpathmoveto{\pgfqpoint{5.743396in}{0.900000in}}%
\pgfpathlineto{\pgfqpoint{5.743396in}{5.700000in}}%
\pgfusepath{stroke}%
\end{pgfscope}%
\begin{pgfscope}%
\pgfsetbuttcap%
\pgfsetroundjoin%
\definecolor{currentfill}{rgb}{0.000000,0.000000,0.000000}%
\pgfsetfillcolor{currentfill}%
\pgfsetlinewidth{0.501875pt}%
\definecolor{currentstroke}{rgb}{0.000000,0.000000,0.000000}%
\pgfsetstrokecolor{currentstroke}%
\pgfsetdash{}{0pt}%
\pgfsys@defobject{currentmarker}{\pgfqpoint{0.000000in}{0.000000in}}{\pgfqpoint{0.000000in}{0.055556in}}{%
\pgfpathmoveto{\pgfqpoint{0.000000in}{0.000000in}}%
\pgfpathlineto{\pgfqpoint{0.000000in}{0.055556in}}%
\pgfusepath{stroke,fill}%
}%
\begin{pgfscope}%
\pgfsys@transformshift{5.743396in}{0.900000in}%
\pgfsys@useobject{currentmarker}{}%
\end{pgfscope}%
\end{pgfscope}%
\begin{pgfscope}%
\pgfsetbuttcap%
\pgfsetroundjoin%
\definecolor{currentfill}{rgb}{0.000000,0.000000,0.000000}%
\pgfsetfillcolor{currentfill}%
\pgfsetlinewidth{0.501875pt}%
\definecolor{currentstroke}{rgb}{0.000000,0.000000,0.000000}%
\pgfsetstrokecolor{currentstroke}%
\pgfsetdash{}{0pt}%
\pgfsys@defobject{currentmarker}{\pgfqpoint{0.000000in}{-0.055556in}}{\pgfqpoint{0.000000in}{0.000000in}}{%
\pgfpathmoveto{\pgfqpoint{0.000000in}{0.000000in}}%
\pgfpathlineto{\pgfqpoint{0.000000in}{-0.055556in}}%
\pgfusepath{stroke,fill}%
}%
\begin{pgfscope}%
\pgfsys@transformshift{5.743396in}{5.700000in}%
\pgfsys@useobject{currentmarker}{}%
\end{pgfscope}%
\end{pgfscope}%
\begin{pgfscope}%
\definecolor{textcolor}{rgb}{0.000000,0.000000,0.000000}%
\pgfsetstrokecolor{textcolor}%
\pgfsetfillcolor{textcolor}%
\pgftext[x=5.743396in,y=0.844444in,,top]{\color{textcolor}\sffamily\fontsize{20.000000}{24.000000}\selectfont \(\displaystyle {40}\)}%
\end{pgfscope}%
\begin{pgfscope}%
\pgfpathrectangle{\pgfqpoint{1.200000in}{0.900000in}}{\pgfqpoint{5.600000in}{4.800000in}}%
\pgfusepath{clip}%
\pgfsetbuttcap%
\pgfsetroundjoin%
\pgfsetlinewidth{0.501875pt}%
\definecolor{currentstroke}{rgb}{0.000000,0.000000,0.000000}%
\pgfsetstrokecolor{currentstroke}%
\pgfsetdash{{1.000000pt}{3.000000pt}}{0.000000pt}%
\pgfpathmoveto{\pgfqpoint{6.800000in}{0.900000in}}%
\pgfpathlineto{\pgfqpoint{6.800000in}{5.700000in}}%
\pgfusepath{stroke}%
\end{pgfscope}%
\begin{pgfscope}%
\pgfsetbuttcap%
\pgfsetroundjoin%
\definecolor{currentfill}{rgb}{0.000000,0.000000,0.000000}%
\pgfsetfillcolor{currentfill}%
\pgfsetlinewidth{0.501875pt}%
\definecolor{currentstroke}{rgb}{0.000000,0.000000,0.000000}%
\pgfsetstrokecolor{currentstroke}%
\pgfsetdash{}{0pt}%
\pgfsys@defobject{currentmarker}{\pgfqpoint{0.000000in}{0.000000in}}{\pgfqpoint{0.000000in}{0.055556in}}{%
\pgfpathmoveto{\pgfqpoint{0.000000in}{0.000000in}}%
\pgfpathlineto{\pgfqpoint{0.000000in}{0.055556in}}%
\pgfusepath{stroke,fill}%
}%
\begin{pgfscope}%
\pgfsys@transformshift{6.800000in}{0.900000in}%
\pgfsys@useobject{currentmarker}{}%
\end{pgfscope}%
\end{pgfscope}%
\begin{pgfscope}%
\pgfsetbuttcap%
\pgfsetroundjoin%
\definecolor{currentfill}{rgb}{0.000000,0.000000,0.000000}%
\pgfsetfillcolor{currentfill}%
\pgfsetlinewidth{0.501875pt}%
\definecolor{currentstroke}{rgb}{0.000000,0.000000,0.000000}%
\pgfsetstrokecolor{currentstroke}%
\pgfsetdash{}{0pt}%
\pgfsys@defobject{currentmarker}{\pgfqpoint{0.000000in}{-0.055556in}}{\pgfqpoint{0.000000in}{0.000000in}}{%
\pgfpathmoveto{\pgfqpoint{0.000000in}{0.000000in}}%
\pgfpathlineto{\pgfqpoint{0.000000in}{-0.055556in}}%
\pgfusepath{stroke,fill}%
}%
\begin{pgfscope}%
\pgfsys@transformshift{6.800000in}{5.700000in}%
\pgfsys@useobject{currentmarker}{}%
\end{pgfscope}%
\end{pgfscope}%
\begin{pgfscope}%
\definecolor{textcolor}{rgb}{0.000000,0.000000,0.000000}%
\pgfsetstrokecolor{textcolor}%
\pgfsetfillcolor{textcolor}%
\pgftext[x=6.800000in,y=0.844444in,,top]{\color{textcolor}\sffamily\fontsize{20.000000}{24.000000}\selectfont \(\displaystyle {50}\)}%
\end{pgfscope}%
\begin{pgfscope}%
\definecolor{textcolor}{rgb}{0.000000,0.000000,0.000000}%
\pgfsetstrokecolor{textcolor}%
\pgfsetfillcolor{textcolor}%
\pgftext[x=4.000000in,y=0.518932in,,top]{\color{textcolor}\sffamily\fontsize{20.000000}{24.000000}\selectfont \(\displaystyle t/\mathrm{ns}\)}%
\end{pgfscope}%
\begin{pgfscope}%
\pgfpathrectangle{\pgfqpoint{1.200000in}{0.900000in}}{\pgfqpoint{5.600000in}{4.800000in}}%
\pgfusepath{clip}%
\pgfsetbuttcap%
\pgfsetroundjoin%
\pgfsetlinewidth{0.501875pt}%
\definecolor{currentstroke}{rgb}{0.000000,0.000000,0.000000}%
\pgfsetstrokecolor{currentstroke}%
\pgfsetdash{{1.000000pt}{3.000000pt}}{0.000000pt}%
\pgfpathmoveto{\pgfqpoint{1.200000in}{0.900000in}}%
\pgfpathlineto{\pgfqpoint{6.800000in}{0.900000in}}%
\pgfusepath{stroke}%
\end{pgfscope}%
\begin{pgfscope}%
\pgfsetbuttcap%
\pgfsetroundjoin%
\definecolor{currentfill}{rgb}{0.000000,0.000000,0.000000}%
\pgfsetfillcolor{currentfill}%
\pgfsetlinewidth{0.501875pt}%
\definecolor{currentstroke}{rgb}{0.000000,0.000000,0.000000}%
\pgfsetstrokecolor{currentstroke}%
\pgfsetdash{}{0pt}%
\pgfsys@defobject{currentmarker}{\pgfqpoint{0.000000in}{0.000000in}}{\pgfqpoint{0.055556in}{0.000000in}}{%
\pgfpathmoveto{\pgfqpoint{0.000000in}{0.000000in}}%
\pgfpathlineto{\pgfqpoint{0.055556in}{0.000000in}}%
\pgfusepath{stroke,fill}%
}%
\begin{pgfscope}%
\pgfsys@transformshift{1.200000in}{0.900000in}%
\pgfsys@useobject{currentmarker}{}%
\end{pgfscope}%
\end{pgfscope}%
\begin{pgfscope}%
\pgfsetbuttcap%
\pgfsetroundjoin%
\definecolor{currentfill}{rgb}{0.000000,0.000000,0.000000}%
\pgfsetfillcolor{currentfill}%
\pgfsetlinewidth{0.501875pt}%
\definecolor{currentstroke}{rgb}{0.000000,0.000000,0.000000}%
\pgfsetstrokecolor{currentstroke}%
\pgfsetdash{}{0pt}%
\pgfsys@defobject{currentmarker}{\pgfqpoint{-0.055556in}{0.000000in}}{\pgfqpoint{0.000000in}{0.000000in}}{%
\pgfpathmoveto{\pgfqpoint{0.000000in}{0.000000in}}%
\pgfpathlineto{\pgfqpoint{-0.055556in}{0.000000in}}%
\pgfusepath{stroke,fill}%
}%
\begin{pgfscope}%
\pgfsys@transformshift{6.800000in}{0.900000in}%
\pgfsys@useobject{currentmarker}{}%
\end{pgfscope}%
\end{pgfscope}%
\begin{pgfscope}%
\definecolor{textcolor}{rgb}{0.000000,0.000000,0.000000}%
\pgfsetstrokecolor{textcolor}%
\pgfsetfillcolor{textcolor}%
\pgftext[x=1.144444in,y=0.900000in,right,]{\color{textcolor}\sffamily\fontsize{20.000000}{24.000000}\selectfont \(\displaystyle {0.00}\)}%
\end{pgfscope}%
\begin{pgfscope}%
\pgfpathrectangle{\pgfqpoint{1.200000in}{0.900000in}}{\pgfqpoint{5.600000in}{4.800000in}}%
\pgfusepath{clip}%
\pgfsetbuttcap%
\pgfsetroundjoin%
\pgfsetlinewidth{0.501875pt}%
\definecolor{currentstroke}{rgb}{0.000000,0.000000,0.000000}%
\pgfsetstrokecolor{currentstroke}%
\pgfsetdash{{1.000000pt}{3.000000pt}}{0.000000pt}%
\pgfpathmoveto{\pgfqpoint{1.200000in}{1.585714in}}%
\pgfpathlineto{\pgfqpoint{6.800000in}{1.585714in}}%
\pgfusepath{stroke}%
\end{pgfscope}%
\begin{pgfscope}%
\pgfsetbuttcap%
\pgfsetroundjoin%
\definecolor{currentfill}{rgb}{0.000000,0.000000,0.000000}%
\pgfsetfillcolor{currentfill}%
\pgfsetlinewidth{0.501875pt}%
\definecolor{currentstroke}{rgb}{0.000000,0.000000,0.000000}%
\pgfsetstrokecolor{currentstroke}%
\pgfsetdash{}{0pt}%
\pgfsys@defobject{currentmarker}{\pgfqpoint{0.000000in}{0.000000in}}{\pgfqpoint{0.055556in}{0.000000in}}{%
\pgfpathmoveto{\pgfqpoint{0.000000in}{0.000000in}}%
\pgfpathlineto{\pgfqpoint{0.055556in}{0.000000in}}%
\pgfusepath{stroke,fill}%
}%
\begin{pgfscope}%
\pgfsys@transformshift{1.200000in}{1.585714in}%
\pgfsys@useobject{currentmarker}{}%
\end{pgfscope}%
\end{pgfscope}%
\begin{pgfscope}%
\pgfsetbuttcap%
\pgfsetroundjoin%
\definecolor{currentfill}{rgb}{0.000000,0.000000,0.000000}%
\pgfsetfillcolor{currentfill}%
\pgfsetlinewidth{0.501875pt}%
\definecolor{currentstroke}{rgb}{0.000000,0.000000,0.000000}%
\pgfsetstrokecolor{currentstroke}%
\pgfsetdash{}{0pt}%
\pgfsys@defobject{currentmarker}{\pgfqpoint{-0.055556in}{0.000000in}}{\pgfqpoint{0.000000in}{0.000000in}}{%
\pgfpathmoveto{\pgfqpoint{0.000000in}{0.000000in}}%
\pgfpathlineto{\pgfqpoint{-0.055556in}{0.000000in}}%
\pgfusepath{stroke,fill}%
}%
\begin{pgfscope}%
\pgfsys@transformshift{6.800000in}{1.585714in}%
\pgfsys@useobject{currentmarker}{}%
\end{pgfscope}%
\end{pgfscope}%
\begin{pgfscope}%
\definecolor{textcolor}{rgb}{0.000000,0.000000,0.000000}%
\pgfsetstrokecolor{textcolor}%
\pgfsetfillcolor{textcolor}%
\pgftext[x=1.144444in,y=1.585714in,right,]{\color{textcolor}\sffamily\fontsize{20.000000}{24.000000}\selectfont \(\displaystyle {0.01}\)}%
\end{pgfscope}%
\begin{pgfscope}%
\pgfpathrectangle{\pgfqpoint{1.200000in}{0.900000in}}{\pgfqpoint{5.600000in}{4.800000in}}%
\pgfusepath{clip}%
\pgfsetbuttcap%
\pgfsetroundjoin%
\pgfsetlinewidth{0.501875pt}%
\definecolor{currentstroke}{rgb}{0.000000,0.000000,0.000000}%
\pgfsetstrokecolor{currentstroke}%
\pgfsetdash{{1.000000pt}{3.000000pt}}{0.000000pt}%
\pgfpathmoveto{\pgfqpoint{1.200000in}{2.271429in}}%
\pgfpathlineto{\pgfqpoint{6.800000in}{2.271429in}}%
\pgfusepath{stroke}%
\end{pgfscope}%
\begin{pgfscope}%
\pgfsetbuttcap%
\pgfsetroundjoin%
\definecolor{currentfill}{rgb}{0.000000,0.000000,0.000000}%
\pgfsetfillcolor{currentfill}%
\pgfsetlinewidth{0.501875pt}%
\definecolor{currentstroke}{rgb}{0.000000,0.000000,0.000000}%
\pgfsetstrokecolor{currentstroke}%
\pgfsetdash{}{0pt}%
\pgfsys@defobject{currentmarker}{\pgfqpoint{0.000000in}{0.000000in}}{\pgfqpoint{0.055556in}{0.000000in}}{%
\pgfpathmoveto{\pgfqpoint{0.000000in}{0.000000in}}%
\pgfpathlineto{\pgfqpoint{0.055556in}{0.000000in}}%
\pgfusepath{stroke,fill}%
}%
\begin{pgfscope}%
\pgfsys@transformshift{1.200000in}{2.271429in}%
\pgfsys@useobject{currentmarker}{}%
\end{pgfscope}%
\end{pgfscope}%
\begin{pgfscope}%
\pgfsetbuttcap%
\pgfsetroundjoin%
\definecolor{currentfill}{rgb}{0.000000,0.000000,0.000000}%
\pgfsetfillcolor{currentfill}%
\pgfsetlinewidth{0.501875pt}%
\definecolor{currentstroke}{rgb}{0.000000,0.000000,0.000000}%
\pgfsetstrokecolor{currentstroke}%
\pgfsetdash{}{0pt}%
\pgfsys@defobject{currentmarker}{\pgfqpoint{-0.055556in}{0.000000in}}{\pgfqpoint{0.000000in}{0.000000in}}{%
\pgfpathmoveto{\pgfqpoint{0.000000in}{0.000000in}}%
\pgfpathlineto{\pgfqpoint{-0.055556in}{0.000000in}}%
\pgfusepath{stroke,fill}%
}%
\begin{pgfscope}%
\pgfsys@transformshift{6.800000in}{2.271429in}%
\pgfsys@useobject{currentmarker}{}%
\end{pgfscope}%
\end{pgfscope}%
\begin{pgfscope}%
\definecolor{textcolor}{rgb}{0.000000,0.000000,0.000000}%
\pgfsetstrokecolor{textcolor}%
\pgfsetfillcolor{textcolor}%
\pgftext[x=1.144444in,y=2.271429in,right,]{\color{textcolor}\sffamily\fontsize{20.000000}{24.000000}\selectfont \(\displaystyle {0.02}\)}%
\end{pgfscope}%
\begin{pgfscope}%
\pgfpathrectangle{\pgfqpoint{1.200000in}{0.900000in}}{\pgfqpoint{5.600000in}{4.800000in}}%
\pgfusepath{clip}%
\pgfsetbuttcap%
\pgfsetroundjoin%
\pgfsetlinewidth{0.501875pt}%
\definecolor{currentstroke}{rgb}{0.000000,0.000000,0.000000}%
\pgfsetstrokecolor{currentstroke}%
\pgfsetdash{{1.000000pt}{3.000000pt}}{0.000000pt}%
\pgfpathmoveto{\pgfqpoint{1.200000in}{2.957143in}}%
\pgfpathlineto{\pgfqpoint{6.800000in}{2.957143in}}%
\pgfusepath{stroke}%
\end{pgfscope}%
\begin{pgfscope}%
\pgfsetbuttcap%
\pgfsetroundjoin%
\definecolor{currentfill}{rgb}{0.000000,0.000000,0.000000}%
\pgfsetfillcolor{currentfill}%
\pgfsetlinewidth{0.501875pt}%
\definecolor{currentstroke}{rgb}{0.000000,0.000000,0.000000}%
\pgfsetstrokecolor{currentstroke}%
\pgfsetdash{}{0pt}%
\pgfsys@defobject{currentmarker}{\pgfqpoint{0.000000in}{0.000000in}}{\pgfqpoint{0.055556in}{0.000000in}}{%
\pgfpathmoveto{\pgfqpoint{0.000000in}{0.000000in}}%
\pgfpathlineto{\pgfqpoint{0.055556in}{0.000000in}}%
\pgfusepath{stroke,fill}%
}%
\begin{pgfscope}%
\pgfsys@transformshift{1.200000in}{2.957143in}%
\pgfsys@useobject{currentmarker}{}%
\end{pgfscope}%
\end{pgfscope}%
\begin{pgfscope}%
\pgfsetbuttcap%
\pgfsetroundjoin%
\definecolor{currentfill}{rgb}{0.000000,0.000000,0.000000}%
\pgfsetfillcolor{currentfill}%
\pgfsetlinewidth{0.501875pt}%
\definecolor{currentstroke}{rgb}{0.000000,0.000000,0.000000}%
\pgfsetstrokecolor{currentstroke}%
\pgfsetdash{}{0pt}%
\pgfsys@defobject{currentmarker}{\pgfqpoint{-0.055556in}{0.000000in}}{\pgfqpoint{0.000000in}{0.000000in}}{%
\pgfpathmoveto{\pgfqpoint{0.000000in}{0.000000in}}%
\pgfpathlineto{\pgfqpoint{-0.055556in}{0.000000in}}%
\pgfusepath{stroke,fill}%
}%
\begin{pgfscope}%
\pgfsys@transformshift{6.800000in}{2.957143in}%
\pgfsys@useobject{currentmarker}{}%
\end{pgfscope}%
\end{pgfscope}%
\begin{pgfscope}%
\definecolor{textcolor}{rgb}{0.000000,0.000000,0.000000}%
\pgfsetstrokecolor{textcolor}%
\pgfsetfillcolor{textcolor}%
\pgftext[x=1.144444in,y=2.957143in,right,]{\color{textcolor}\sffamily\fontsize{20.000000}{24.000000}\selectfont \(\displaystyle {0.03}\)}%
\end{pgfscope}%
\begin{pgfscope}%
\pgfpathrectangle{\pgfqpoint{1.200000in}{0.900000in}}{\pgfqpoint{5.600000in}{4.800000in}}%
\pgfusepath{clip}%
\pgfsetbuttcap%
\pgfsetroundjoin%
\pgfsetlinewidth{0.501875pt}%
\definecolor{currentstroke}{rgb}{0.000000,0.000000,0.000000}%
\pgfsetstrokecolor{currentstroke}%
\pgfsetdash{{1.000000pt}{3.000000pt}}{0.000000pt}%
\pgfpathmoveto{\pgfqpoint{1.200000in}{3.642857in}}%
\pgfpathlineto{\pgfqpoint{6.800000in}{3.642857in}}%
\pgfusepath{stroke}%
\end{pgfscope}%
\begin{pgfscope}%
\pgfsetbuttcap%
\pgfsetroundjoin%
\definecolor{currentfill}{rgb}{0.000000,0.000000,0.000000}%
\pgfsetfillcolor{currentfill}%
\pgfsetlinewidth{0.501875pt}%
\definecolor{currentstroke}{rgb}{0.000000,0.000000,0.000000}%
\pgfsetstrokecolor{currentstroke}%
\pgfsetdash{}{0pt}%
\pgfsys@defobject{currentmarker}{\pgfqpoint{0.000000in}{0.000000in}}{\pgfqpoint{0.055556in}{0.000000in}}{%
\pgfpathmoveto{\pgfqpoint{0.000000in}{0.000000in}}%
\pgfpathlineto{\pgfqpoint{0.055556in}{0.000000in}}%
\pgfusepath{stroke,fill}%
}%
\begin{pgfscope}%
\pgfsys@transformshift{1.200000in}{3.642857in}%
\pgfsys@useobject{currentmarker}{}%
\end{pgfscope}%
\end{pgfscope}%
\begin{pgfscope}%
\pgfsetbuttcap%
\pgfsetroundjoin%
\definecolor{currentfill}{rgb}{0.000000,0.000000,0.000000}%
\pgfsetfillcolor{currentfill}%
\pgfsetlinewidth{0.501875pt}%
\definecolor{currentstroke}{rgb}{0.000000,0.000000,0.000000}%
\pgfsetstrokecolor{currentstroke}%
\pgfsetdash{}{0pt}%
\pgfsys@defobject{currentmarker}{\pgfqpoint{-0.055556in}{0.000000in}}{\pgfqpoint{0.000000in}{0.000000in}}{%
\pgfpathmoveto{\pgfqpoint{0.000000in}{0.000000in}}%
\pgfpathlineto{\pgfqpoint{-0.055556in}{0.000000in}}%
\pgfusepath{stroke,fill}%
}%
\begin{pgfscope}%
\pgfsys@transformshift{6.800000in}{3.642857in}%
\pgfsys@useobject{currentmarker}{}%
\end{pgfscope}%
\end{pgfscope}%
\begin{pgfscope}%
\definecolor{textcolor}{rgb}{0.000000,0.000000,0.000000}%
\pgfsetstrokecolor{textcolor}%
\pgfsetfillcolor{textcolor}%
\pgftext[x=1.144444in,y=3.642857in,right,]{\color{textcolor}\sffamily\fontsize{20.000000}{24.000000}\selectfont \(\displaystyle {0.04}\)}%
\end{pgfscope}%
\begin{pgfscope}%
\pgfpathrectangle{\pgfqpoint{1.200000in}{0.900000in}}{\pgfqpoint{5.600000in}{4.800000in}}%
\pgfusepath{clip}%
\pgfsetbuttcap%
\pgfsetroundjoin%
\pgfsetlinewidth{0.501875pt}%
\definecolor{currentstroke}{rgb}{0.000000,0.000000,0.000000}%
\pgfsetstrokecolor{currentstroke}%
\pgfsetdash{{1.000000pt}{3.000000pt}}{0.000000pt}%
\pgfpathmoveto{\pgfqpoint{1.200000in}{4.328571in}}%
\pgfpathlineto{\pgfqpoint{6.800000in}{4.328571in}}%
\pgfusepath{stroke}%
\end{pgfscope}%
\begin{pgfscope}%
\pgfsetbuttcap%
\pgfsetroundjoin%
\definecolor{currentfill}{rgb}{0.000000,0.000000,0.000000}%
\pgfsetfillcolor{currentfill}%
\pgfsetlinewidth{0.501875pt}%
\definecolor{currentstroke}{rgb}{0.000000,0.000000,0.000000}%
\pgfsetstrokecolor{currentstroke}%
\pgfsetdash{}{0pt}%
\pgfsys@defobject{currentmarker}{\pgfqpoint{0.000000in}{0.000000in}}{\pgfqpoint{0.055556in}{0.000000in}}{%
\pgfpathmoveto{\pgfqpoint{0.000000in}{0.000000in}}%
\pgfpathlineto{\pgfqpoint{0.055556in}{0.000000in}}%
\pgfusepath{stroke,fill}%
}%
\begin{pgfscope}%
\pgfsys@transformshift{1.200000in}{4.328571in}%
\pgfsys@useobject{currentmarker}{}%
\end{pgfscope}%
\end{pgfscope}%
\begin{pgfscope}%
\pgfsetbuttcap%
\pgfsetroundjoin%
\definecolor{currentfill}{rgb}{0.000000,0.000000,0.000000}%
\pgfsetfillcolor{currentfill}%
\pgfsetlinewidth{0.501875pt}%
\definecolor{currentstroke}{rgb}{0.000000,0.000000,0.000000}%
\pgfsetstrokecolor{currentstroke}%
\pgfsetdash{}{0pt}%
\pgfsys@defobject{currentmarker}{\pgfqpoint{-0.055556in}{0.000000in}}{\pgfqpoint{0.000000in}{0.000000in}}{%
\pgfpathmoveto{\pgfqpoint{0.000000in}{0.000000in}}%
\pgfpathlineto{\pgfqpoint{-0.055556in}{0.000000in}}%
\pgfusepath{stroke,fill}%
}%
\begin{pgfscope}%
\pgfsys@transformshift{6.800000in}{4.328571in}%
\pgfsys@useobject{currentmarker}{}%
\end{pgfscope}%
\end{pgfscope}%
\begin{pgfscope}%
\definecolor{textcolor}{rgb}{0.000000,0.000000,0.000000}%
\pgfsetstrokecolor{textcolor}%
\pgfsetfillcolor{textcolor}%
\pgftext[x=1.144444in,y=4.328571in,right,]{\color{textcolor}\sffamily\fontsize{20.000000}{24.000000}\selectfont \(\displaystyle {0.05}\)}%
\end{pgfscope}%
\begin{pgfscope}%
\pgfpathrectangle{\pgfqpoint{1.200000in}{0.900000in}}{\pgfqpoint{5.600000in}{4.800000in}}%
\pgfusepath{clip}%
\pgfsetbuttcap%
\pgfsetroundjoin%
\pgfsetlinewidth{0.501875pt}%
\definecolor{currentstroke}{rgb}{0.000000,0.000000,0.000000}%
\pgfsetstrokecolor{currentstroke}%
\pgfsetdash{{1.000000pt}{3.000000pt}}{0.000000pt}%
\pgfpathmoveto{\pgfqpoint{1.200000in}{5.014286in}}%
\pgfpathlineto{\pgfqpoint{6.800000in}{5.014286in}}%
\pgfusepath{stroke}%
\end{pgfscope}%
\begin{pgfscope}%
\pgfsetbuttcap%
\pgfsetroundjoin%
\definecolor{currentfill}{rgb}{0.000000,0.000000,0.000000}%
\pgfsetfillcolor{currentfill}%
\pgfsetlinewidth{0.501875pt}%
\definecolor{currentstroke}{rgb}{0.000000,0.000000,0.000000}%
\pgfsetstrokecolor{currentstroke}%
\pgfsetdash{}{0pt}%
\pgfsys@defobject{currentmarker}{\pgfqpoint{0.000000in}{0.000000in}}{\pgfqpoint{0.055556in}{0.000000in}}{%
\pgfpathmoveto{\pgfqpoint{0.000000in}{0.000000in}}%
\pgfpathlineto{\pgfqpoint{0.055556in}{0.000000in}}%
\pgfusepath{stroke,fill}%
}%
\begin{pgfscope}%
\pgfsys@transformshift{1.200000in}{5.014286in}%
\pgfsys@useobject{currentmarker}{}%
\end{pgfscope}%
\end{pgfscope}%
\begin{pgfscope}%
\pgfsetbuttcap%
\pgfsetroundjoin%
\definecolor{currentfill}{rgb}{0.000000,0.000000,0.000000}%
\pgfsetfillcolor{currentfill}%
\pgfsetlinewidth{0.501875pt}%
\definecolor{currentstroke}{rgb}{0.000000,0.000000,0.000000}%
\pgfsetstrokecolor{currentstroke}%
\pgfsetdash{}{0pt}%
\pgfsys@defobject{currentmarker}{\pgfqpoint{-0.055556in}{0.000000in}}{\pgfqpoint{0.000000in}{0.000000in}}{%
\pgfpathmoveto{\pgfqpoint{0.000000in}{0.000000in}}%
\pgfpathlineto{\pgfqpoint{-0.055556in}{0.000000in}}%
\pgfusepath{stroke,fill}%
}%
\begin{pgfscope}%
\pgfsys@transformshift{6.800000in}{5.014286in}%
\pgfsys@useobject{currentmarker}{}%
\end{pgfscope}%
\end{pgfscope}%
\begin{pgfscope}%
\definecolor{textcolor}{rgb}{0.000000,0.000000,0.000000}%
\pgfsetstrokecolor{textcolor}%
\pgfsetfillcolor{textcolor}%
\pgftext[x=1.144444in,y=5.014286in,right,]{\color{textcolor}\sffamily\fontsize{20.000000}{24.000000}\selectfont \(\displaystyle {0.06}\)}%
\end{pgfscope}%
\begin{pgfscope}%
\pgfpathrectangle{\pgfqpoint{1.200000in}{0.900000in}}{\pgfqpoint{5.600000in}{4.800000in}}%
\pgfusepath{clip}%
\pgfsetbuttcap%
\pgfsetroundjoin%
\pgfsetlinewidth{0.501875pt}%
\definecolor{currentstroke}{rgb}{0.000000,0.000000,0.000000}%
\pgfsetstrokecolor{currentstroke}%
\pgfsetdash{{1.000000pt}{3.000000pt}}{0.000000pt}%
\pgfpathmoveto{\pgfqpoint{1.200000in}{5.700000in}}%
\pgfpathlineto{\pgfqpoint{6.800000in}{5.700000in}}%
\pgfusepath{stroke}%
\end{pgfscope}%
\begin{pgfscope}%
\pgfsetbuttcap%
\pgfsetroundjoin%
\definecolor{currentfill}{rgb}{0.000000,0.000000,0.000000}%
\pgfsetfillcolor{currentfill}%
\pgfsetlinewidth{0.501875pt}%
\definecolor{currentstroke}{rgb}{0.000000,0.000000,0.000000}%
\pgfsetstrokecolor{currentstroke}%
\pgfsetdash{}{0pt}%
\pgfsys@defobject{currentmarker}{\pgfqpoint{0.000000in}{0.000000in}}{\pgfqpoint{0.055556in}{0.000000in}}{%
\pgfpathmoveto{\pgfqpoint{0.000000in}{0.000000in}}%
\pgfpathlineto{\pgfqpoint{0.055556in}{0.000000in}}%
\pgfusepath{stroke,fill}%
}%
\begin{pgfscope}%
\pgfsys@transformshift{1.200000in}{5.700000in}%
\pgfsys@useobject{currentmarker}{}%
\end{pgfscope}%
\end{pgfscope}%
\begin{pgfscope}%
\pgfsetbuttcap%
\pgfsetroundjoin%
\definecolor{currentfill}{rgb}{0.000000,0.000000,0.000000}%
\pgfsetfillcolor{currentfill}%
\pgfsetlinewidth{0.501875pt}%
\definecolor{currentstroke}{rgb}{0.000000,0.000000,0.000000}%
\pgfsetstrokecolor{currentstroke}%
\pgfsetdash{}{0pt}%
\pgfsys@defobject{currentmarker}{\pgfqpoint{-0.055556in}{0.000000in}}{\pgfqpoint{0.000000in}{0.000000in}}{%
\pgfpathmoveto{\pgfqpoint{0.000000in}{0.000000in}}%
\pgfpathlineto{\pgfqpoint{-0.055556in}{0.000000in}}%
\pgfusepath{stroke,fill}%
}%
\begin{pgfscope}%
\pgfsys@transformshift{6.800000in}{5.700000in}%
\pgfsys@useobject{currentmarker}{}%
\end{pgfscope}%
\end{pgfscope}%
\begin{pgfscope}%
\definecolor{textcolor}{rgb}{0.000000,0.000000,0.000000}%
\pgfsetstrokecolor{textcolor}%
\pgfsetfillcolor{textcolor}%
\pgftext[x=1.144444in,y=5.700000in,right,]{\color{textcolor}\sffamily\fontsize{20.000000}{24.000000}\selectfont \(\displaystyle {0.07}\)}%
\end{pgfscope}%
\begin{pgfscope}%
\definecolor{textcolor}{rgb}{0.000000,0.000000,0.000000}%
\pgfsetstrokecolor{textcolor}%
\pgfsetfillcolor{textcolor}%
\pgftext[x=0.600330in,y=3.300000in,,bottom,rotate=90.000000]{\color{textcolor}\sffamily\fontsize{20.000000}{24.000000}\selectfont \(\displaystyle PDF\)}%
\end{pgfscope}%
\begin{pgfscope}%
\pgfsetbuttcap%
\pgfsetmiterjoin%
\definecolor{currentfill}{rgb}{1.000000,1.000000,1.000000}%
\pgfsetfillcolor{currentfill}%
\pgfsetlinewidth{1.003750pt}%
\definecolor{currentstroke}{rgb}{0.000000,0.000000,0.000000}%
\pgfsetstrokecolor{currentstroke}%
\pgfsetdash{}{0pt}%
\pgfpathmoveto{\pgfqpoint{3.749258in}{4.959484in}}%
\pgfpathlineto{\pgfqpoint{6.633333in}{4.959484in}}%
\pgfpathlineto{\pgfqpoint{6.633333in}{5.533333in}}%
\pgfpathlineto{\pgfqpoint{3.749258in}{5.533333in}}%
\pgfpathclose%
\pgfusepath{stroke,fill}%
\end{pgfscope}%
\begin{pgfscope}%
\pgfsetrectcap%
\pgfsetroundjoin%
\pgfsetlinewidth{2.007500pt}%
\definecolor{currentstroke}{rgb}{0.000000,0.000000,1.000000}%
\pgfsetstrokecolor{currentstroke}%
\pgfsetdash{}{0pt}%
\pgfpathmoveto{\pgfqpoint{3.982591in}{5.276697in}}%
\pgfpathlineto{\pgfqpoint{4.449258in}{5.276697in}}%
\pgfusepath{stroke}%
\end{pgfscope}%
\begin{pgfscope}%
\definecolor{textcolor}{rgb}{0.000000,0.000000,0.000000}%
\pgfsetstrokecolor{textcolor}%
\pgfsetfillcolor{textcolor}%
\pgftext[x=4.815924in,y=5.160031in,left,base]{\color{textcolor}\sffamily\fontsize{24.000000}{28.800000}\selectfont Time Profile}%
\end{pgfscope}%
\end{pgfpicture}%
\makeatother%
\endgroup%
}
    \caption{Time Profile of Events}
\end{figure}
\begin{align*}
    \phi(t) &= \mathcal{N}(t|\sigma_l^2)\otimes \mathrm{Exp}(t|\tau_l) \\
    &= \frac{1}{2\tau_l} \exp\left(\frac{\sigma_l^2}{2\tau_l^2}-\frac{t}{\tau_l}\right) \left[1 - \erf\left( \frac{\sigma_l}{\sqrt{2}\tau_l} - \frac{t}{\sqrt{2}\sigma_l} \right)\right]
\end{align*}
\end{frame}

\begin{frame}
\frametitle{Simulation setup: Single PE response}
\begin{figure}
    \centering
    \resizebox{0.6\textwidth}{!}{%% Creator: Matplotlib, PGF backend
%%
%% To include the figure in your LaTeX document, write
%%   \input{<filename>.pgf}
%%
%% Make sure the required packages are loaded in your preamble
%%   \usepackage{pgf}
%%
%% and, on pdftex
%%   \usepackage[utf8]{inputenc}\DeclareUnicodeCharacter{2212}{-}
%%
%% or, on luatex and xetex
%%   \usepackage{unicode-math}
%%
%% Figures using additional raster images can only be included by \input if
%% they are in the same directory as the main LaTeX file. For loading figures
%% from other directories you can use the `import` package
%%   \usepackage{import}
%%
%% and then include the figures with
%%   \import{<path to file>}{<filename>.pgf}
%%
%% Matplotlib used the following preamble
%%
\begingroup%
\makeatletter%
\begin{pgfpicture}%
\pgfpathrectangle{\pgfpointorigin}{\pgfqpoint{8.000000in}{6.000000in}}%
\pgfusepath{use as bounding box, clip}%
\begin{pgfscope}%
\pgfsetbuttcap%
\pgfsetmiterjoin%
\definecolor{currentfill}{rgb}{1.000000,1.000000,1.000000}%
\pgfsetfillcolor{currentfill}%
\pgfsetlinewidth{0.000000pt}%
\definecolor{currentstroke}{rgb}{1.000000,1.000000,1.000000}%
\pgfsetstrokecolor{currentstroke}%
\pgfsetdash{}{0pt}%
\pgfpathmoveto{\pgfqpoint{0.000000in}{0.000000in}}%
\pgfpathlineto{\pgfqpoint{8.000000in}{0.000000in}}%
\pgfpathlineto{\pgfqpoint{8.000000in}{6.000000in}}%
\pgfpathlineto{\pgfqpoint{0.000000in}{6.000000in}}%
\pgfpathclose%
\pgfusepath{fill}%
\end{pgfscope}%
\begin{pgfscope}%
\pgfsetbuttcap%
\pgfsetmiterjoin%
\definecolor{currentfill}{rgb}{1.000000,1.000000,1.000000}%
\pgfsetfillcolor{currentfill}%
\pgfsetlinewidth{0.000000pt}%
\definecolor{currentstroke}{rgb}{0.000000,0.000000,0.000000}%
\pgfsetstrokecolor{currentstroke}%
\pgfsetstrokeopacity{0.000000}%
\pgfsetdash{}{0pt}%
\pgfpathmoveto{\pgfqpoint{1.000000in}{0.600000in}}%
\pgfpathlineto{\pgfqpoint{7.200000in}{0.600000in}}%
\pgfpathlineto{\pgfqpoint{7.200000in}{5.400000in}}%
\pgfpathlineto{\pgfqpoint{1.000000in}{5.400000in}}%
\pgfpathclose%
\pgfusepath{fill}%
\end{pgfscope}%
\begin{pgfscope}%
\pgfpathrectangle{\pgfqpoint{1.000000in}{0.600000in}}{\pgfqpoint{6.200000in}{4.800000in}}%
\pgfusepath{clip}%
\pgfsetrectcap%
\pgfsetroundjoin%
\pgfsetlinewidth{2.007500pt}%
\definecolor{currentstroke}{rgb}{0.000000,0.000000,1.000000}%
\pgfsetstrokecolor{currentstroke}%
\pgfsetdash{}{0pt}%
\pgfpathmoveto{\pgfqpoint{1.000000in}{0.600000in}}%
\pgfpathlineto{\pgfqpoint{1.077500in}{0.638138in}}%
\pgfpathlineto{\pgfqpoint{1.155000in}{0.669271in}}%
\pgfpathlineto{\pgfqpoint{1.232500in}{0.861453in}}%
\pgfpathlineto{\pgfqpoint{1.310000in}{1.307445in}}%
\pgfpathlineto{\pgfqpoint{1.387500in}{2.460997in}}%
\pgfpathlineto{\pgfqpoint{1.465000in}{4.032968in}}%
\pgfpathlineto{\pgfqpoint{1.542500in}{5.104535in}}%
\pgfpathlineto{\pgfqpoint{1.620000in}{4.989134in}}%
\pgfpathlineto{\pgfqpoint{1.697500in}{4.276460in}}%
\pgfpathlineto{\pgfqpoint{1.775000in}{3.366583in}}%
\pgfpathlineto{\pgfqpoint{1.852500in}{2.626429in}}%
\pgfpathlineto{\pgfqpoint{1.930000in}{2.087953in}}%
\pgfpathlineto{\pgfqpoint{2.007500in}{1.854595in}}%
\pgfpathlineto{\pgfqpoint{2.085000in}{1.615850in}}%
\pgfpathlineto{\pgfqpoint{2.162500in}{1.440192in}}%
\pgfpathlineto{\pgfqpoint{2.240000in}{1.235411in}}%
\pgfpathlineto{\pgfqpoint{2.317500in}{1.131239in}}%
\pgfpathlineto{\pgfqpoint{2.395000in}{0.996118in}}%
\pgfpathlineto{\pgfqpoint{2.472500in}{0.973111in}}%
\pgfpathlineto{\pgfqpoint{2.550000in}{0.917602in}}%
\pgfpathlineto{\pgfqpoint{2.627500in}{0.928010in}}%
\pgfpathlineto{\pgfqpoint{2.705000in}{0.855702in}}%
\pgfpathlineto{\pgfqpoint{2.782500in}{0.843102in}}%
\pgfpathlineto{\pgfqpoint{2.860000in}{0.769151in}}%
\pgfpathlineto{\pgfqpoint{2.937500in}{0.754908in}}%
\pgfpathlineto{\pgfqpoint{3.015000in}{0.694469in}}%
\pgfpathlineto{\pgfqpoint{3.092500in}{0.713276in}}%
\pgfpathlineto{\pgfqpoint{3.170000in}{0.677670in}}%
\pgfpathlineto{\pgfqpoint{3.247500in}{0.713185in}}%
\pgfpathlineto{\pgfqpoint{3.325000in}{0.602440in}}%
\pgfpathlineto{\pgfqpoint{3.402500in}{0.600980in}}%
\pgfpathlineto{\pgfqpoint{3.480000in}{0.602623in}}%
\pgfpathlineto{\pgfqpoint{3.557500in}{0.609836in}}%
\pgfpathlineto{\pgfqpoint{3.635000in}{0.601254in}}%
\pgfpathlineto{\pgfqpoint{3.712500in}{0.612118in}}%
\pgfpathlineto{\pgfqpoint{3.790000in}{0.600000in}}%
\pgfpathlineto{\pgfqpoint{3.867500in}{0.600000in}}%
\pgfpathlineto{\pgfqpoint{3.945000in}{0.600523in}}%
\pgfpathlineto{\pgfqpoint{4.022500in}{0.600615in}}%
\pgfpathlineto{\pgfqpoint{4.100000in}{0.608375in}}%
\pgfpathlineto{\pgfqpoint{4.177500in}{0.600000in}}%
\pgfpathlineto{\pgfqpoint{4.255000in}{0.605636in}}%
\pgfpathlineto{\pgfqpoint{4.332500in}{0.605453in}}%
\pgfpathlineto{\pgfqpoint{4.410000in}{0.600432in}}%
\pgfpathlineto{\pgfqpoint{4.487500in}{0.600000in}}%
\pgfpathlineto{\pgfqpoint{4.565000in}{0.606184in}}%
\pgfpathlineto{\pgfqpoint{4.642500in}{0.606092in}}%
\pgfpathlineto{\pgfqpoint{4.720000in}{0.603262in}}%
\pgfpathlineto{\pgfqpoint{4.797500in}{0.600000in}}%
\pgfpathlineto{\pgfqpoint{4.875000in}{0.605910in}}%
\pgfpathlineto{\pgfqpoint{4.952500in}{0.615131in}}%
\pgfpathlineto{\pgfqpoint{5.030000in}{0.600000in}}%
\pgfpathlineto{\pgfqpoint{5.107500in}{0.610931in}}%
\pgfpathlineto{\pgfqpoint{5.185000in}{0.604175in}}%
\pgfpathlineto{\pgfqpoint{5.262500in}{0.602075in}}%
\pgfpathlineto{\pgfqpoint{5.340000in}{0.607279in}}%
\pgfpathlineto{\pgfqpoint{5.417500in}{0.602167in}}%
\pgfpathlineto{\pgfqpoint{5.495000in}{0.600000in}}%
\pgfpathlineto{\pgfqpoint{5.572500in}{0.608192in}}%
\pgfpathlineto{\pgfqpoint{5.650000in}{0.600000in}}%
\pgfpathlineto{\pgfqpoint{5.727500in}{0.607005in}}%
\pgfpathlineto{\pgfqpoint{5.805000in}{0.601528in}}%
\pgfpathlineto{\pgfqpoint{5.882500in}{0.606092in}}%
\pgfpathlineto{\pgfqpoint{5.960000in}{0.606001in}}%
\pgfpathlineto{\pgfqpoint{6.037500in}{0.601345in}}%
\pgfpathlineto{\pgfqpoint{6.115000in}{0.604814in}}%
\pgfpathlineto{\pgfqpoint{6.192500in}{0.600000in}}%
\pgfpathlineto{\pgfqpoint{6.270000in}{0.600000in}}%
\pgfpathlineto{\pgfqpoint{6.347500in}{0.600000in}}%
\pgfpathlineto{\pgfqpoint{6.425000in}{0.600341in}}%
\pgfpathlineto{\pgfqpoint{6.502500in}{0.606092in}}%
\pgfpathlineto{\pgfqpoint{6.580000in}{0.600000in}}%
\pgfpathlineto{\pgfqpoint{6.657500in}{0.604632in}}%
\pgfpathlineto{\pgfqpoint{6.735000in}{0.607097in}}%
\pgfpathlineto{\pgfqpoint{6.812500in}{0.604266in}}%
\pgfpathlineto{\pgfqpoint{6.890000in}{0.604084in}}%
\pgfpathlineto{\pgfqpoint{6.967500in}{0.601162in}}%
\pgfpathlineto{\pgfqpoint{7.045000in}{0.604449in}}%
\pgfpathlineto{\pgfqpoint{7.122500in}{0.600000in}}%
\pgfusepath{stroke}%
\end{pgfscope}%
\begin{pgfscope}%
\pgfpathrectangle{\pgfqpoint{1.000000in}{0.600000in}}{\pgfqpoint{6.200000in}{4.800000in}}%
\pgfusepath{clip}%
\pgfsetrectcap%
\pgfsetroundjoin%
\pgfsetlinewidth{2.007500pt}%
\definecolor{currentstroke}{rgb}{0.000000,0.500000,0.000000}%
\pgfsetstrokecolor{currentstroke}%
\pgfsetdash{}{0pt}%
\pgfpathmoveto{\pgfqpoint{1.000000in}{0.600000in}}%
\pgfpathlineto{\pgfqpoint{1.077500in}{0.621709in}}%
\pgfpathlineto{\pgfqpoint{1.155000in}{0.656792in}}%
\pgfpathlineto{\pgfqpoint{1.232500in}{0.842309in}}%
\pgfpathlineto{\pgfqpoint{1.310000in}{1.223354in}}%
\pgfpathlineto{\pgfqpoint{1.387500in}{2.259334in}}%
\pgfpathlineto{\pgfqpoint{1.465000in}{3.656816in}}%
\pgfpathlineto{\pgfqpoint{1.542500in}{4.601944in}}%
\pgfpathlineto{\pgfqpoint{1.620000in}{4.497745in}}%
\pgfpathlineto{\pgfqpoint{1.697500in}{3.865975in}}%
\pgfpathlineto{\pgfqpoint{1.775000in}{3.056220in}}%
\pgfpathlineto{\pgfqpoint{1.852500in}{2.389844in}}%
\pgfpathlineto{\pgfqpoint{1.930000in}{1.923858in}}%
\pgfpathlineto{\pgfqpoint{2.007500in}{1.715938in}}%
\pgfpathlineto{\pgfqpoint{2.085000in}{1.500295in}}%
\pgfpathlineto{\pgfqpoint{2.162500in}{1.343282in}}%
\pgfpathlineto{\pgfqpoint{2.240000in}{1.171207in}}%
\pgfpathlineto{\pgfqpoint{2.317500in}{1.075779in}}%
\pgfpathlineto{\pgfqpoint{2.395000in}{0.948891in}}%
\pgfpathlineto{\pgfqpoint{2.472500in}{0.923533in}}%
\pgfpathlineto{\pgfqpoint{2.550000in}{0.878249in}}%
\pgfpathlineto{\pgfqpoint{2.627500in}{0.889785in}}%
\pgfpathlineto{\pgfqpoint{2.705000in}{0.839449in}}%
\pgfpathlineto{\pgfqpoint{2.782500in}{0.820383in}}%
\pgfpathlineto{\pgfqpoint{2.860000in}{0.751934in}}%
\pgfpathlineto{\pgfqpoint{2.937500in}{0.743926in}}%
\pgfpathlineto{\pgfqpoint{3.015000in}{0.691207in}}%
\pgfpathlineto{\pgfqpoint{3.092500in}{0.692446in}}%
\pgfpathlineto{\pgfqpoint{3.170000in}{0.670520in}}%
\pgfpathlineto{\pgfqpoint{3.247500in}{0.714563in}}%
\pgfpathlineto{\pgfqpoint{3.325000in}{0.605693in}}%
\pgfpathlineto{\pgfqpoint{3.402500in}{0.600000in}}%
\pgfpathlineto{\pgfqpoint{3.480000in}{0.609697in}}%
\pgfpathlineto{\pgfqpoint{3.557500in}{0.608935in}}%
\pgfpathlineto{\pgfqpoint{3.635000in}{0.602071in}}%
\pgfpathlineto{\pgfqpoint{3.712500in}{0.610365in}}%
\pgfpathlineto{\pgfqpoint{3.790000in}{0.600000in}}%
\pgfpathlineto{\pgfqpoint{3.867500in}{0.605312in}}%
\pgfpathlineto{\pgfqpoint{3.945000in}{0.601975in}}%
\pgfpathlineto{\pgfqpoint{4.022500in}{0.604263in}}%
\pgfpathlineto{\pgfqpoint{4.100000in}{0.601975in}}%
\pgfpathlineto{\pgfqpoint{4.177500in}{0.603405in}}%
\pgfpathlineto{\pgfqpoint{4.255000in}{0.600000in}}%
\pgfpathlineto{\pgfqpoint{4.332500in}{0.609125in}}%
\pgfpathlineto{\pgfqpoint{4.410000in}{0.600000in}}%
\pgfpathlineto{\pgfqpoint{4.487500in}{0.609030in}}%
\pgfpathlineto{\pgfqpoint{4.565000in}{0.609697in}}%
\pgfpathlineto{\pgfqpoint{4.642500in}{0.600000in}}%
\pgfpathlineto{\pgfqpoint{4.720000in}{0.606551in}}%
\pgfpathlineto{\pgfqpoint{4.797500in}{0.600000in}}%
\pgfpathlineto{\pgfqpoint{4.875000in}{0.604454in}}%
\pgfpathlineto{\pgfqpoint{4.952500in}{0.603691in}}%
\pgfpathlineto{\pgfqpoint{5.030000in}{0.604835in}}%
\pgfpathlineto{\pgfqpoint{5.107500in}{0.604073in}}%
\pgfpathlineto{\pgfqpoint{5.185000in}{0.602357in}}%
\pgfpathlineto{\pgfqpoint{5.262500in}{0.602929in}}%
\pgfpathlineto{\pgfqpoint{5.340000in}{0.600000in}}%
\pgfpathlineto{\pgfqpoint{5.417500in}{0.603024in}}%
\pgfpathlineto{\pgfqpoint{5.495000in}{0.600000in}}%
\pgfpathlineto{\pgfqpoint{5.572500in}{0.600000in}}%
\pgfpathlineto{\pgfqpoint{5.650000in}{0.601117in}}%
\pgfpathlineto{\pgfqpoint{5.727500in}{0.604740in}}%
\pgfpathlineto{\pgfqpoint{5.805000in}{0.610937in}}%
\pgfpathlineto{\pgfqpoint{5.882500in}{0.600000in}}%
\pgfpathlineto{\pgfqpoint{5.960000in}{0.606647in}}%
\pgfpathlineto{\pgfqpoint{6.037500in}{0.600000in}}%
\pgfpathlineto{\pgfqpoint{6.115000in}{0.600000in}}%
\pgfpathlineto{\pgfqpoint{6.192500in}{0.600000in}}%
\pgfpathlineto{\pgfqpoint{6.270000in}{0.600000in}}%
\pgfpathlineto{\pgfqpoint{6.347500in}{0.602357in}}%
\pgfpathlineto{\pgfqpoint{6.425000in}{0.611795in}}%
\pgfpathlineto{\pgfqpoint{6.502500in}{0.603501in}}%
\pgfpathlineto{\pgfqpoint{6.580000in}{0.607886in}}%
\pgfpathlineto{\pgfqpoint{6.657500in}{0.600000in}}%
\pgfpathlineto{\pgfqpoint{6.735000in}{0.602166in}}%
\pgfpathlineto{\pgfqpoint{6.812500in}{0.609983in}}%
\pgfpathlineto{\pgfqpoint{6.890000in}{0.602643in}}%
\pgfpathlineto{\pgfqpoint{6.967500in}{0.607409in}}%
\pgfpathlineto{\pgfqpoint{7.045000in}{0.601022in}}%
\pgfpathlineto{\pgfqpoint{7.122500in}{0.601117in}}%
\pgfusepath{stroke}%
\end{pgfscope}%
\begin{pgfscope}%
\pgfpathrectangle{\pgfqpoint{1.000000in}{0.600000in}}{\pgfqpoint{6.200000in}{4.800000in}}%
\pgfusepath{clip}%
\pgfsetrectcap%
\pgfsetroundjoin%
\pgfsetlinewidth{2.007500pt}%
\definecolor{currentstroke}{rgb}{1.000000,0.000000,0.000000}%
\pgfsetstrokecolor{currentstroke}%
\pgfsetdash{}{0pt}%
\pgfpathmoveto{\pgfqpoint{1.000000in}{0.600000in}}%
\pgfpathlineto{\pgfqpoint{1.077500in}{0.620924in}}%
\pgfpathlineto{\pgfqpoint{1.155000in}{0.657475in}}%
\pgfpathlineto{\pgfqpoint{1.232500in}{0.836704in}}%
\pgfpathlineto{\pgfqpoint{1.310000in}{1.260865in}}%
\pgfpathlineto{\pgfqpoint{1.387500in}{2.316249in}}%
\pgfpathlineto{\pgfqpoint{1.465000in}{3.778972in}}%
\pgfpathlineto{\pgfqpoint{1.542500in}{4.775963in}}%
\pgfpathlineto{\pgfqpoint{1.620000in}{4.670803in}}%
\pgfpathlineto{\pgfqpoint{1.697500in}{4.010605in}}%
\pgfpathlineto{\pgfqpoint{1.775000in}{3.169152in}}%
\pgfpathlineto{\pgfqpoint{1.852500in}{2.483853in}}%
\pgfpathlineto{\pgfqpoint{1.930000in}{1.973996in}}%
\pgfpathlineto{\pgfqpoint{2.007500in}{1.762091in}}%
\pgfpathlineto{\pgfqpoint{2.085000in}{1.535742in}}%
\pgfpathlineto{\pgfqpoint{2.162500in}{1.381174in}}%
\pgfpathlineto{\pgfqpoint{2.240000in}{1.199478in}}%
\pgfpathlineto{\pgfqpoint{2.317500in}{1.098898in}}%
\pgfpathlineto{\pgfqpoint{2.395000in}{0.982906in}}%
\pgfpathlineto{\pgfqpoint{2.472500in}{0.942392in}}%
\pgfpathlineto{\pgfqpoint{2.550000in}{0.898796in}}%
\pgfpathlineto{\pgfqpoint{2.627500in}{0.900557in}}%
\pgfpathlineto{\pgfqpoint{2.705000in}{0.842693in}}%
\pgfpathlineto{\pgfqpoint{2.782500in}{0.820410in}}%
\pgfpathlineto{\pgfqpoint{2.860000in}{0.763251in}}%
\pgfpathlineto{\pgfqpoint{2.937500in}{0.753827in}}%
\pgfpathlineto{\pgfqpoint{3.015000in}{0.696667in}}%
\pgfpathlineto{\pgfqpoint{3.092500in}{0.695346in}}%
\pgfpathlineto{\pgfqpoint{3.170000in}{0.670509in}}%
\pgfpathlineto{\pgfqpoint{3.247500in}{0.711904in}}%
\pgfpathlineto{\pgfqpoint{3.325000in}{0.600000in}}%
\pgfpathlineto{\pgfqpoint{3.402500in}{0.600000in}}%
\pgfpathlineto{\pgfqpoint{3.480000in}{0.609298in}}%
\pgfpathlineto{\pgfqpoint{3.557500in}{0.600000in}}%
\pgfpathlineto{\pgfqpoint{3.635000in}{0.601812in}}%
\pgfpathlineto{\pgfqpoint{3.712500in}{0.603838in}}%
\pgfpathlineto{\pgfqpoint{3.790000in}{0.606656in}}%
\pgfpathlineto{\pgfqpoint{3.867500in}{0.605864in}}%
\pgfpathlineto{\pgfqpoint{3.945000in}{0.604719in}}%
\pgfpathlineto{\pgfqpoint{4.022500in}{0.606304in}}%
\pgfpathlineto{\pgfqpoint{4.100000in}{0.608065in}}%
\pgfpathlineto{\pgfqpoint{4.177500in}{0.604190in}}%
\pgfpathlineto{\pgfqpoint{4.255000in}{0.604366in}}%
\pgfpathlineto{\pgfqpoint{4.332500in}{0.600000in}}%
\pgfpathlineto{\pgfqpoint{4.410000in}{0.610179in}}%
\pgfpathlineto{\pgfqpoint{4.487500in}{0.600000in}}%
\pgfpathlineto{\pgfqpoint{4.565000in}{0.602253in}}%
\pgfpathlineto{\pgfqpoint{4.642500in}{0.609651in}}%
\pgfpathlineto{\pgfqpoint{4.720000in}{0.606216in}}%
\pgfpathlineto{\pgfqpoint{4.797500in}{0.600000in}}%
\pgfpathlineto{\pgfqpoint{4.875000in}{0.601636in}}%
\pgfpathlineto{\pgfqpoint{4.952500in}{0.612469in}}%
\pgfpathlineto{\pgfqpoint{5.030000in}{0.607977in}}%
\pgfpathlineto{\pgfqpoint{5.107500in}{0.600931in}}%
\pgfpathlineto{\pgfqpoint{5.185000in}{0.600000in}}%
\pgfpathlineto{\pgfqpoint{5.262500in}{0.609739in}}%
\pgfpathlineto{\pgfqpoint{5.340000in}{0.600667in}}%
\pgfpathlineto{\pgfqpoint{5.417500in}{0.606480in}}%
\pgfpathlineto{\pgfqpoint{5.495000in}{0.608153in}}%
\pgfpathlineto{\pgfqpoint{5.572500in}{0.600000in}}%
\pgfpathlineto{\pgfqpoint{5.650000in}{0.603574in}}%
\pgfpathlineto{\pgfqpoint{5.727500in}{0.600000in}}%
\pgfpathlineto{\pgfqpoint{5.805000in}{0.608153in}}%
\pgfpathlineto{\pgfqpoint{5.882500in}{0.600000in}}%
\pgfpathlineto{\pgfqpoint{5.960000in}{0.600000in}}%
\pgfpathlineto{\pgfqpoint{6.037500in}{0.602957in}}%
\pgfpathlineto{\pgfqpoint{6.115000in}{0.609475in}}%
\pgfpathlineto{\pgfqpoint{6.192500in}{0.604190in}}%
\pgfpathlineto{\pgfqpoint{6.270000in}{0.605511in}}%
\pgfpathlineto{\pgfqpoint{6.347500in}{0.602429in}}%
\pgfpathlineto{\pgfqpoint{6.425000in}{0.602693in}}%
\pgfpathlineto{\pgfqpoint{6.502500in}{0.600000in}}%
\pgfpathlineto{\pgfqpoint{6.580000in}{0.602429in}}%
\pgfpathlineto{\pgfqpoint{6.657500in}{0.605775in}}%
\pgfpathlineto{\pgfqpoint{6.735000in}{0.606392in}}%
\pgfpathlineto{\pgfqpoint{6.812500in}{0.601724in}}%
\pgfpathlineto{\pgfqpoint{6.890000in}{0.601019in}}%
\pgfpathlineto{\pgfqpoint{6.967500in}{0.600843in}}%
\pgfpathlineto{\pgfqpoint{7.045000in}{0.600000in}}%
\pgfpathlineto{\pgfqpoint{7.122500in}{0.602957in}}%
\pgfusepath{stroke}%
\end{pgfscope}%
\begin{pgfscope}%
\pgfpathrectangle{\pgfqpoint{1.000000in}{0.600000in}}{\pgfqpoint{6.200000in}{4.800000in}}%
\pgfusepath{clip}%
\pgfsetrectcap%
\pgfsetroundjoin%
\pgfsetlinewidth{2.007500pt}%
\definecolor{currentstroke}{rgb}{0.000000,0.750000,0.750000}%
\pgfsetstrokecolor{currentstroke}%
\pgfsetdash{}{0pt}%
\pgfpathmoveto{\pgfqpoint{1.000000in}{0.600000in}}%
\pgfpathlineto{\pgfqpoint{1.077500in}{0.620095in}}%
\pgfpathlineto{\pgfqpoint{1.155000in}{0.652814in}}%
\pgfpathlineto{\pgfqpoint{1.232500in}{0.842607in}}%
\pgfpathlineto{\pgfqpoint{1.310000in}{1.235339in}}%
\pgfpathlineto{\pgfqpoint{1.387500in}{2.275075in}}%
\pgfpathlineto{\pgfqpoint{1.465000in}{3.696029in}}%
\pgfpathlineto{\pgfqpoint{1.542500in}{4.670038in}}%
\pgfpathlineto{\pgfqpoint{1.620000in}{4.567177in}}%
\pgfpathlineto{\pgfqpoint{1.697500in}{3.915471in}}%
\pgfpathlineto{\pgfqpoint{1.775000in}{3.096904in}}%
\pgfpathlineto{\pgfqpoint{1.852500in}{2.423609in}}%
\pgfpathlineto{\pgfqpoint{1.930000in}{1.950086in}}%
\pgfpathlineto{\pgfqpoint{2.007500in}{1.730739in}}%
\pgfpathlineto{\pgfqpoint{2.085000in}{1.519645in}}%
\pgfpathlineto{\pgfqpoint{2.162500in}{1.368520in}}%
\pgfpathlineto{\pgfqpoint{2.240000in}{1.175945in}}%
\pgfpathlineto{\pgfqpoint{2.317500in}{1.090547in}}%
\pgfpathlineto{\pgfqpoint{2.395000in}{0.962643in}}%
\pgfpathlineto{\pgfqpoint{2.472500in}{0.930499in}}%
\pgfpathlineto{\pgfqpoint{2.550000in}{0.901906in}}%
\pgfpathlineto{\pgfqpoint{2.627500in}{0.890679in}}%
\pgfpathlineto{\pgfqpoint{2.705000in}{0.834068in}}%
\pgfpathlineto{\pgfqpoint{2.782500in}{0.818523in}}%
\pgfpathlineto{\pgfqpoint{2.860000in}{0.755387in}}%
\pgfpathlineto{\pgfqpoint{2.937500in}{0.745792in}}%
\pgfpathlineto{\pgfqpoint{3.015000in}{0.690619in}}%
\pgfpathlineto{\pgfqpoint{3.092500in}{0.695417in}}%
\pgfpathlineto{\pgfqpoint{3.170000in}{0.674787in}}%
\pgfpathlineto{\pgfqpoint{3.247500in}{0.709522in}}%
\pgfpathlineto{\pgfqpoint{3.325000in}{0.601384in}}%
\pgfpathlineto{\pgfqpoint{3.402500in}{0.605702in}}%
\pgfpathlineto{\pgfqpoint{3.480000in}{0.600000in}}%
\pgfpathlineto{\pgfqpoint{3.557500in}{0.606278in}}%
\pgfpathlineto{\pgfqpoint{3.635000in}{0.607141in}}%
\pgfpathlineto{\pgfqpoint{3.712500in}{0.601672in}}%
\pgfpathlineto{\pgfqpoint{3.790000in}{0.607813in}}%
\pgfpathlineto{\pgfqpoint{3.867500in}{0.606373in}}%
\pgfpathlineto{\pgfqpoint{3.945000in}{0.601288in}}%
\pgfpathlineto{\pgfqpoint{4.022500in}{0.600000in}}%
\pgfpathlineto{\pgfqpoint{4.100000in}{0.600000in}}%
\pgfpathlineto{\pgfqpoint{4.177500in}{0.600000in}}%
\pgfpathlineto{\pgfqpoint{4.255000in}{0.606949in}}%
\pgfpathlineto{\pgfqpoint{4.332500in}{0.600000in}}%
\pgfpathlineto{\pgfqpoint{4.410000in}{0.602152in}}%
\pgfpathlineto{\pgfqpoint{4.487500in}{0.603975in}}%
\pgfpathlineto{\pgfqpoint{4.565000in}{0.600000in}}%
\pgfpathlineto{\pgfqpoint{4.642500in}{0.609828in}}%
\pgfpathlineto{\pgfqpoint{4.720000in}{0.602727in}}%
\pgfpathlineto{\pgfqpoint{4.797500in}{0.600000in}}%
\pgfpathlineto{\pgfqpoint{4.875000in}{0.600424in}}%
\pgfpathlineto{\pgfqpoint{4.952500in}{0.600000in}}%
\pgfpathlineto{\pgfqpoint{5.030000in}{0.600520in}}%
\pgfpathlineto{\pgfqpoint{5.107500in}{0.606086in}}%
\pgfpathlineto{\pgfqpoint{5.185000in}{0.600000in}}%
\pgfpathlineto{\pgfqpoint{5.262500in}{0.601288in}}%
\pgfpathlineto{\pgfqpoint{5.340000in}{0.603495in}}%
\pgfpathlineto{\pgfqpoint{5.417500in}{0.602823in}}%
\pgfpathlineto{\pgfqpoint{5.495000in}{0.607429in}}%
\pgfpathlineto{\pgfqpoint{5.572500in}{0.605030in}}%
\pgfpathlineto{\pgfqpoint{5.650000in}{0.601288in}}%
\pgfpathlineto{\pgfqpoint{5.727500in}{0.616448in}}%
\pgfpathlineto{\pgfqpoint{5.805000in}{0.610787in}}%
\pgfpathlineto{\pgfqpoint{5.882500in}{0.600000in}}%
\pgfpathlineto{\pgfqpoint{5.960000in}{0.602727in}}%
\pgfpathlineto{\pgfqpoint{6.037500in}{0.602343in}}%
\pgfpathlineto{\pgfqpoint{6.115000in}{0.600000in}}%
\pgfpathlineto{\pgfqpoint{6.192500in}{0.600000in}}%
\pgfpathlineto{\pgfqpoint{6.270000in}{0.603495in}}%
\pgfpathlineto{\pgfqpoint{6.347500in}{0.604934in}}%
\pgfpathlineto{\pgfqpoint{6.425000in}{0.600000in}}%
\pgfpathlineto{\pgfqpoint{6.502500in}{0.611555in}}%
\pgfpathlineto{\pgfqpoint{6.580000in}{0.608388in}}%
\pgfpathlineto{\pgfqpoint{6.657500in}{0.603783in}}%
\pgfpathlineto{\pgfqpoint{6.735000in}{0.607525in}}%
\pgfpathlineto{\pgfqpoint{6.812500in}{0.600000in}}%
\pgfpathlineto{\pgfqpoint{6.890000in}{0.603879in}}%
\pgfpathlineto{\pgfqpoint{6.967500in}{0.605126in}}%
\pgfpathlineto{\pgfqpoint{7.045000in}{0.600000in}}%
\pgfpathlineto{\pgfqpoint{7.122500in}{0.600329in}}%
\pgfusepath{stroke}%
\end{pgfscope}%
\begin{pgfscope}%
\pgfpathrectangle{\pgfqpoint{1.000000in}{0.600000in}}{\pgfqpoint{6.200000in}{4.800000in}}%
\pgfusepath{clip}%
\pgfsetrectcap%
\pgfsetroundjoin%
\pgfsetlinewidth{2.007500pt}%
\definecolor{currentstroke}{rgb}{0.750000,0.000000,0.750000}%
\pgfsetstrokecolor{currentstroke}%
\pgfsetdash{}{0pt}%
\pgfpathmoveto{\pgfqpoint{1.000000in}{0.600000in}}%
\pgfpathlineto{\pgfqpoint{1.077500in}{0.627623in}}%
\pgfpathlineto{\pgfqpoint{1.155000in}{0.662541in}}%
\pgfpathlineto{\pgfqpoint{1.232500in}{0.849758in}}%
\pgfpathlineto{\pgfqpoint{1.310000in}{1.275659in}}%
\pgfpathlineto{\pgfqpoint{1.387500in}{2.376478in}}%
\pgfpathlineto{\pgfqpoint{1.465000in}{3.901668in}}%
\pgfpathlineto{\pgfqpoint{1.542500in}{4.920981in}}%
\pgfpathlineto{\pgfqpoint{1.620000in}{4.812114in}}%
\pgfpathlineto{\pgfqpoint{1.697500in}{4.128108in}}%
\pgfpathlineto{\pgfqpoint{1.775000in}{3.255737in}}%
\pgfpathlineto{\pgfqpoint{1.852500in}{2.531647in}}%
\pgfpathlineto{\pgfqpoint{1.930000in}{2.031892in}}%
\pgfpathlineto{\pgfqpoint{2.007500in}{1.806984in}}%
\pgfpathlineto{\pgfqpoint{2.085000in}{1.572413in}}%
\pgfpathlineto{\pgfqpoint{2.162500in}{1.406242in}}%
\pgfpathlineto{\pgfqpoint{2.240000in}{1.212903in}}%
\pgfpathlineto{\pgfqpoint{2.317500in}{1.100688in}}%
\pgfpathlineto{\pgfqpoint{2.395000in}{0.976514in}}%
\pgfpathlineto{\pgfqpoint{2.472500in}{0.950685in}}%
\pgfpathlineto{\pgfqpoint{2.550000in}{0.911175in}}%
\pgfpathlineto{\pgfqpoint{2.627500in}{0.906009in}}%
\pgfpathlineto{\pgfqpoint{2.705000in}{0.855402in}}%
\pgfpathlineto{\pgfqpoint{2.782500in}{0.833017in}}%
\pgfpathlineto{\pgfqpoint{2.860000in}{0.769112in}}%
\pgfpathlineto{\pgfqpoint{2.937500in}{0.754571in}}%
\pgfpathlineto{\pgfqpoint{3.015000in}{0.700329in}}%
\pgfpathlineto{\pgfqpoint{3.092500in}{0.697459in}}%
\pgfpathlineto{\pgfqpoint{3.170000in}{0.669907in}}%
\pgfpathlineto{\pgfqpoint{3.247500in}{0.700520in}}%
\pgfpathlineto{\pgfqpoint{3.325000in}{0.600000in}}%
\pgfpathlineto{\pgfqpoint{3.402500in}{0.603803in}}%
\pgfpathlineto{\pgfqpoint{3.480000in}{0.609064in}}%
\pgfpathlineto{\pgfqpoint{3.557500in}{0.600000in}}%
\pgfpathlineto{\pgfqpoint{3.635000in}{0.603994in}}%
\pgfpathlineto{\pgfqpoint{3.712500in}{0.600000in}}%
\pgfpathlineto{\pgfqpoint{3.790000in}{0.612317in}}%
\pgfpathlineto{\pgfqpoint{3.867500in}{0.605238in}}%
\pgfpathlineto{\pgfqpoint{3.945000in}{0.603037in}}%
\pgfpathlineto{\pgfqpoint{4.022500in}{0.600000in}}%
\pgfpathlineto{\pgfqpoint{4.100000in}{0.605812in}}%
\pgfpathlineto{\pgfqpoint{4.177500in}{0.600000in}}%
\pgfpathlineto{\pgfqpoint{4.255000in}{0.608586in}}%
\pgfpathlineto{\pgfqpoint{4.332500in}{0.609543in}}%
\pgfpathlineto{\pgfqpoint{4.410000in}{0.600933in}}%
\pgfpathlineto{\pgfqpoint{4.487500in}{0.608299in}}%
\pgfpathlineto{\pgfqpoint{4.565000in}{0.600000in}}%
\pgfpathlineto{\pgfqpoint{4.642500in}{0.603994in}}%
\pgfpathlineto{\pgfqpoint{4.720000in}{0.611169in}}%
\pgfpathlineto{\pgfqpoint{4.797500in}{0.605812in}}%
\pgfpathlineto{\pgfqpoint{4.875000in}{0.600000in}}%
\pgfpathlineto{\pgfqpoint{4.952500in}{0.600000in}}%
\pgfpathlineto{\pgfqpoint{5.030000in}{0.606768in}}%
\pgfpathlineto{\pgfqpoint{5.107500in}{0.600000in}}%
\pgfpathlineto{\pgfqpoint{5.185000in}{0.604185in}}%
\pgfpathlineto{\pgfqpoint{5.262500in}{0.600000in}}%
\pgfpathlineto{\pgfqpoint{5.340000in}{0.600550in}}%
\pgfpathlineto{\pgfqpoint{5.417500in}{0.600000in}}%
\pgfpathlineto{\pgfqpoint{5.495000in}{0.607629in}}%
\pgfpathlineto{\pgfqpoint{5.572500in}{0.600000in}}%
\pgfpathlineto{\pgfqpoint{5.650000in}{0.603611in}}%
\pgfpathlineto{\pgfqpoint{5.727500in}{0.600263in}}%
\pgfpathlineto{\pgfqpoint{5.805000in}{0.602942in}}%
\pgfpathlineto{\pgfqpoint{5.882500in}{0.603803in}}%
\pgfpathlineto{\pgfqpoint{5.960000in}{0.600000in}}%
\pgfpathlineto{\pgfqpoint{6.037500in}{0.600000in}}%
\pgfpathlineto{\pgfqpoint{6.115000in}{0.601124in}}%
\pgfpathlineto{\pgfqpoint{6.192500in}{0.600000in}}%
\pgfpathlineto{\pgfqpoint{6.270000in}{0.600742in}}%
\pgfpathlineto{\pgfqpoint{6.347500in}{0.600000in}}%
\pgfpathlineto{\pgfqpoint{6.425000in}{0.610404in}}%
\pgfpathlineto{\pgfqpoint{6.502500in}{0.600000in}}%
\pgfpathlineto{\pgfqpoint{6.580000in}{0.600000in}}%
\pgfpathlineto{\pgfqpoint{6.657500in}{0.604855in}}%
\pgfpathlineto{\pgfqpoint{6.735000in}{0.600000in}}%
\pgfpathlineto{\pgfqpoint{6.812500in}{0.604090in}}%
\pgfpathlineto{\pgfqpoint{6.890000in}{0.601028in}}%
\pgfpathlineto{\pgfqpoint{6.967500in}{0.600000in}}%
\pgfpathlineto{\pgfqpoint{7.045000in}{0.606768in}}%
\pgfpathlineto{\pgfqpoint{7.122500in}{0.601220in}}%
\pgfusepath{stroke}%
\end{pgfscope}%
\begin{pgfscope}%
\pgfpathrectangle{\pgfqpoint{1.000000in}{0.600000in}}{\pgfqpoint{6.200000in}{4.800000in}}%
\pgfusepath{clip}%
\pgfsetrectcap%
\pgfsetroundjoin%
\pgfsetlinewidth{2.007500pt}%
\definecolor{currentstroke}{rgb}{0.750000,0.750000,0.000000}%
\pgfsetstrokecolor{currentstroke}%
\pgfsetdash{}{0pt}%
\pgfpathmoveto{\pgfqpoint{1.000000in}{0.600000in}}%
\pgfpathlineto{\pgfqpoint{1.077500in}{0.625308in}}%
\pgfpathlineto{\pgfqpoint{1.155000in}{0.663651in}}%
\pgfpathlineto{\pgfqpoint{1.232500in}{0.867471in}}%
\pgfpathlineto{\pgfqpoint{1.310000in}{1.301537in}}%
\pgfpathlineto{\pgfqpoint{1.387500in}{2.452672in}}%
\pgfpathlineto{\pgfqpoint{1.465000in}{4.014329in}}%
\pgfpathlineto{\pgfqpoint{1.542500in}{5.092239in}}%
\pgfpathlineto{\pgfqpoint{1.620000in}{4.989993in}}%
\pgfpathlineto{\pgfqpoint{1.697500in}{4.271579in}}%
\pgfpathlineto{\pgfqpoint{1.775000in}{3.357128in}}%
\pgfpathlineto{\pgfqpoint{1.852500in}{2.617284in}}%
\pgfpathlineto{\pgfqpoint{1.930000in}{2.092696in}}%
\pgfpathlineto{\pgfqpoint{2.007500in}{1.847651in}}%
\pgfpathlineto{\pgfqpoint{2.085000in}{1.612696in}}%
\pgfpathlineto{\pgfqpoint{2.162500in}{1.438474in}}%
\pgfpathlineto{\pgfqpoint{2.240000in}{1.237248in}}%
\pgfpathlineto{\pgfqpoint{2.317500in}{1.129140in}}%
\pgfpathlineto{\pgfqpoint{2.395000in}{0.994606in}}%
\pgfpathlineto{\pgfqpoint{2.472500in}{0.969717in}}%
\pgfpathlineto{\pgfqpoint{2.550000in}{0.917344in}}%
\pgfpathlineto{\pgfqpoint{2.627500in}{0.930510in}}%
\pgfpathlineto{\pgfqpoint{2.705000in}{0.866702in}}%
\pgfpathlineto{\pgfqpoint{2.782500in}{0.843158in}}%
\pgfpathlineto{\pgfqpoint{2.860000in}{0.768684in}}%
\pgfpathlineto{\pgfqpoint{2.937500in}{0.762341in}}%
\pgfpathlineto{\pgfqpoint{3.015000in}{0.702570in}}%
\pgfpathlineto{\pgfqpoint{3.092500in}{0.698534in}}%
\pgfpathlineto{\pgfqpoint{3.170000in}{0.668455in}}%
\pgfpathlineto{\pgfqpoint{3.247500in}{0.722846in}}%
\pgfpathlineto{\pgfqpoint{3.325000in}{0.608203in}}%
\pgfpathlineto{\pgfqpoint{3.402500in}{0.600323in}}%
\pgfpathlineto{\pgfqpoint{3.480000in}{0.600419in}}%
\pgfpathlineto{\pgfqpoint{3.557500in}{0.600000in}}%
\pgfpathlineto{\pgfqpoint{3.635000in}{0.604552in}}%
\pgfpathlineto{\pgfqpoint{3.712500in}{0.602534in}}%
\pgfpathlineto{\pgfqpoint{3.790000in}{0.600000in}}%
\pgfpathlineto{\pgfqpoint{3.867500in}{0.600000in}}%
\pgfpathlineto{\pgfqpoint{3.945000in}{0.601765in}}%
\pgfpathlineto{\pgfqpoint{4.022500in}{0.605224in}}%
\pgfpathlineto{\pgfqpoint{4.100000in}{0.600000in}}%
\pgfpathlineto{\pgfqpoint{4.177500in}{0.603398in}}%
\pgfpathlineto{\pgfqpoint{4.255000in}{0.600000in}}%
\pgfpathlineto{\pgfqpoint{4.332500in}{0.605128in}}%
\pgfpathlineto{\pgfqpoint{4.410000in}{0.603206in}}%
\pgfpathlineto{\pgfqpoint{4.487500in}{0.605224in}}%
\pgfpathlineto{\pgfqpoint{4.565000in}{0.606474in}}%
\pgfpathlineto{\pgfqpoint{4.642500in}{0.607915in}}%
\pgfpathlineto{\pgfqpoint{4.720000in}{0.608588in}}%
\pgfpathlineto{\pgfqpoint{4.797500in}{0.603014in}}%
\pgfpathlineto{\pgfqpoint{4.875000in}{0.600000in}}%
\pgfpathlineto{\pgfqpoint{4.952500in}{0.611951in}}%
\pgfpathlineto{\pgfqpoint{5.030000in}{0.600000in}}%
\pgfpathlineto{\pgfqpoint{5.107500in}{0.606089in}}%
\pgfpathlineto{\pgfqpoint{5.185000in}{0.605513in}}%
\pgfpathlineto{\pgfqpoint{5.262500in}{0.602341in}}%
\pgfpathlineto{\pgfqpoint{5.340000in}{0.602918in}}%
\pgfpathlineto{\pgfqpoint{5.417500in}{0.603591in}}%
\pgfpathlineto{\pgfqpoint{5.495000in}{0.603687in}}%
\pgfpathlineto{\pgfqpoint{5.572500in}{0.604744in}}%
\pgfpathlineto{\pgfqpoint{5.650000in}{0.600000in}}%
\pgfpathlineto{\pgfqpoint{5.727500in}{0.600000in}}%
\pgfpathlineto{\pgfqpoint{5.805000in}{0.600000in}}%
\pgfpathlineto{\pgfqpoint{5.882500in}{0.614930in}}%
\pgfpathlineto{\pgfqpoint{5.960000in}{0.600000in}}%
\pgfpathlineto{\pgfqpoint{6.037500in}{0.607915in}}%
\pgfpathlineto{\pgfqpoint{6.115000in}{0.600804in}}%
\pgfpathlineto{\pgfqpoint{6.192500in}{0.602341in}}%
\pgfpathlineto{\pgfqpoint{6.270000in}{0.604552in}}%
\pgfpathlineto{\pgfqpoint{6.347500in}{0.600000in}}%
\pgfpathlineto{\pgfqpoint{6.425000in}{0.603975in}}%
\pgfpathlineto{\pgfqpoint{6.502500in}{0.605993in}}%
\pgfpathlineto{\pgfqpoint{6.580000in}{0.600000in}}%
\pgfpathlineto{\pgfqpoint{6.657500in}{0.600000in}}%
\pgfpathlineto{\pgfqpoint{6.735000in}{0.602341in}}%
\pgfpathlineto{\pgfqpoint{6.812500in}{0.608011in}}%
\pgfpathlineto{\pgfqpoint{6.890000in}{0.601573in}}%
\pgfpathlineto{\pgfqpoint{6.967500in}{0.605416in}}%
\pgfpathlineto{\pgfqpoint{7.045000in}{0.600000in}}%
\pgfpathlineto{\pgfqpoint{7.122500in}{0.600000in}}%
\pgfusepath{stroke}%
\end{pgfscope}%
\begin{pgfscope}%
\pgfpathrectangle{\pgfqpoint{1.000000in}{0.600000in}}{\pgfqpoint{6.200000in}{4.800000in}}%
\pgfusepath{clip}%
\pgfsetrectcap%
\pgfsetroundjoin%
\pgfsetlinewidth{2.007500pt}%
\definecolor{currentstroke}{rgb}{0.000000,0.000000,0.000000}%
\pgfsetstrokecolor{currentstroke}%
\pgfsetdash{}{0pt}%
\pgfpathmoveto{\pgfqpoint{1.000000in}{0.600000in}}%
\pgfpathlineto{\pgfqpoint{1.077500in}{0.625436in}}%
\pgfpathlineto{\pgfqpoint{1.155000in}{0.662012in}}%
\pgfpathlineto{\pgfqpoint{1.232500in}{0.841916in}}%
\pgfpathlineto{\pgfqpoint{1.310000in}{1.243388in}}%
\pgfpathlineto{\pgfqpoint{1.387500in}{2.291900in}}%
\pgfpathlineto{\pgfqpoint{1.465000in}{3.724892in}}%
\pgfpathlineto{\pgfqpoint{1.542500in}{4.697948in}}%
\pgfpathlineto{\pgfqpoint{1.620000in}{4.587164in}}%
\pgfpathlineto{\pgfqpoint{1.697500in}{3.942620in}}%
\pgfpathlineto{\pgfqpoint{1.775000in}{3.126044in}}%
\pgfpathlineto{\pgfqpoint{1.852500in}{2.436956in}}%
\pgfpathlineto{\pgfqpoint{1.930000in}{1.966460in}}%
\pgfpathlineto{\pgfqpoint{2.007500in}{1.734236in}}%
\pgfpathlineto{\pgfqpoint{2.085000in}{1.527068in}}%
\pgfpathlineto{\pgfqpoint{2.162500in}{1.371356in}}%
\pgfpathlineto{\pgfqpoint{2.240000in}{1.180988in}}%
\pgfpathlineto{\pgfqpoint{2.317500in}{1.081916in}}%
\pgfpathlineto{\pgfqpoint{2.395000in}{0.964220in}}%
\pgfpathlineto{\pgfqpoint{2.472500in}{0.936092in}}%
\pgfpathlineto{\pgfqpoint{2.550000in}{0.893564in}}%
\pgfpathlineto{\pgfqpoint{2.627500in}{0.897020in}}%
\pgfpathlineto{\pgfqpoint{2.705000in}{0.854396in}}%
\pgfpathlineto{\pgfqpoint{2.782500in}{0.819644in}}%
\pgfpathlineto{\pgfqpoint{2.860000in}{0.758300in}}%
\pgfpathlineto{\pgfqpoint{2.937500in}{0.746780in}}%
\pgfpathlineto{\pgfqpoint{3.015000in}{0.693884in}}%
\pgfpathlineto{\pgfqpoint{3.092500in}{0.690044in}}%
\pgfpathlineto{\pgfqpoint{3.170000in}{0.680444in}}%
\pgfpathlineto{\pgfqpoint{3.247500in}{0.702620in}}%
\pgfpathlineto{\pgfqpoint{3.325000in}{0.608156in}}%
\pgfpathlineto{\pgfqpoint{3.402500in}{0.609404in}}%
\pgfpathlineto{\pgfqpoint{3.480000in}{0.607580in}}%
\pgfpathlineto{\pgfqpoint{3.557500in}{0.601724in}}%
\pgfpathlineto{\pgfqpoint{3.635000in}{0.601724in}}%
\pgfpathlineto{\pgfqpoint{3.712500in}{0.600000in}}%
\pgfpathlineto{\pgfqpoint{3.790000in}{0.602204in}}%
\pgfpathlineto{\pgfqpoint{3.867500in}{0.609020in}}%
\pgfpathlineto{\pgfqpoint{3.945000in}{0.604604in}}%
\pgfpathlineto{\pgfqpoint{4.022500in}{0.601916in}}%
\pgfpathlineto{\pgfqpoint{4.100000in}{0.601052in}}%
\pgfpathlineto{\pgfqpoint{4.177500in}{0.605948in}}%
\pgfpathlineto{\pgfqpoint{4.255000in}{0.605564in}}%
\pgfpathlineto{\pgfqpoint{4.332500in}{0.612764in}}%
\pgfpathlineto{\pgfqpoint{4.410000in}{0.603164in}}%
\pgfpathlineto{\pgfqpoint{4.487500in}{0.600000in}}%
\pgfpathlineto{\pgfqpoint{4.565000in}{0.600000in}}%
\pgfpathlineto{\pgfqpoint{4.642500in}{0.600000in}}%
\pgfpathlineto{\pgfqpoint{4.720000in}{0.605564in}}%
\pgfpathlineto{\pgfqpoint{4.797500in}{0.600668in}}%
\pgfpathlineto{\pgfqpoint{4.875000in}{0.600000in}}%
\pgfpathlineto{\pgfqpoint{4.952500in}{0.605660in}}%
\pgfpathlineto{\pgfqpoint{5.030000in}{0.610748in}}%
\pgfpathlineto{\pgfqpoint{5.107500in}{0.604796in}}%
\pgfpathlineto{\pgfqpoint{5.185000in}{0.606716in}}%
\pgfpathlineto{\pgfqpoint{5.262500in}{0.600668in}}%
\pgfpathlineto{\pgfqpoint{5.340000in}{0.602780in}}%
\pgfpathlineto{\pgfqpoint{5.417500in}{0.614396in}}%
\pgfpathlineto{\pgfqpoint{5.495000in}{0.608636in}}%
\pgfpathlineto{\pgfqpoint{5.572500in}{0.610940in}}%
\pgfpathlineto{\pgfqpoint{5.650000in}{0.600000in}}%
\pgfpathlineto{\pgfqpoint{5.727500in}{0.604604in}}%
\pgfpathlineto{\pgfqpoint{5.805000in}{0.609692in}}%
\pgfpathlineto{\pgfqpoint{5.882500in}{0.612092in}}%
\pgfpathlineto{\pgfqpoint{5.960000in}{0.603356in}}%
\pgfpathlineto{\pgfqpoint{6.037500in}{0.600000in}}%
\pgfpathlineto{\pgfqpoint{6.115000in}{0.613340in}}%
\pgfpathlineto{\pgfqpoint{6.192500in}{0.600000in}}%
\pgfpathlineto{\pgfqpoint{6.270000in}{0.602012in}}%
\pgfpathlineto{\pgfqpoint{6.347500in}{0.600000in}}%
\pgfpathlineto{\pgfqpoint{6.425000in}{0.609212in}}%
\pgfpathlineto{\pgfqpoint{6.502500in}{0.600000in}}%
\pgfpathlineto{\pgfqpoint{6.580000in}{0.608924in}}%
\pgfpathlineto{\pgfqpoint{6.657500in}{0.604508in}}%
\pgfpathlineto{\pgfqpoint{6.735000in}{0.605948in}}%
\pgfpathlineto{\pgfqpoint{6.812500in}{0.606620in}}%
\pgfpathlineto{\pgfqpoint{6.890000in}{0.602684in}}%
\pgfpathlineto{\pgfqpoint{6.967500in}{0.604892in}}%
\pgfpathlineto{\pgfqpoint{7.045000in}{0.609596in}}%
\pgfpathlineto{\pgfqpoint{7.122500in}{0.600000in}}%
\pgfusepath{stroke}%
\end{pgfscope}%
\begin{pgfscope}%
\pgfpathrectangle{\pgfqpoint{1.000000in}{0.600000in}}{\pgfqpoint{6.200000in}{4.800000in}}%
\pgfusepath{clip}%
\pgfsetrectcap%
\pgfsetroundjoin%
\pgfsetlinewidth{2.007500pt}%
\definecolor{currentstroke}{rgb}{0.000000,0.000000,1.000000}%
\pgfsetstrokecolor{currentstroke}%
\pgfsetdash{}{0pt}%
\pgfpathmoveto{\pgfqpoint{1.000000in}{0.600000in}}%
\pgfpathlineto{\pgfqpoint{1.077500in}{0.628609in}}%
\pgfpathlineto{\pgfqpoint{1.155000in}{0.662227in}}%
\pgfpathlineto{\pgfqpoint{1.232500in}{0.849372in}}%
\pgfpathlineto{\pgfqpoint{1.310000in}{1.282014in}}%
\pgfpathlineto{\pgfqpoint{1.387500in}{2.402227in}}%
\pgfpathlineto{\pgfqpoint{1.465000in}{3.941891in}}%
\pgfpathlineto{\pgfqpoint{1.542500in}{4.992304in}}%
\pgfpathlineto{\pgfqpoint{1.620000in}{4.875143in}}%
\pgfpathlineto{\pgfqpoint{1.697500in}{4.190777in}}%
\pgfpathlineto{\pgfqpoint{1.775000in}{3.300304in}}%
\pgfpathlineto{\pgfqpoint{1.852500in}{2.567113in}}%
\pgfpathlineto{\pgfqpoint{1.930000in}{2.059815in}}%
\pgfpathlineto{\pgfqpoint{2.007500in}{1.822288in}}%
\pgfpathlineto{\pgfqpoint{2.085000in}{1.592090in}}%
\pgfpathlineto{\pgfqpoint{2.162500in}{1.423357in}}%
\pgfpathlineto{\pgfqpoint{2.240000in}{1.235662in}}%
\pgfpathlineto{\pgfqpoint{2.317500in}{1.121250in}}%
\pgfpathlineto{\pgfqpoint{2.395000in}{0.973403in}}%
\pgfpathlineto{\pgfqpoint{2.472500in}{0.960579in}}%
\pgfpathlineto{\pgfqpoint{2.550000in}{0.912395in}}%
\pgfpathlineto{\pgfqpoint{2.627500in}{0.914869in}}%
\pgfpathlineto{\pgfqpoint{2.705000in}{0.856792in}}%
\pgfpathlineto{\pgfqpoint{2.782500in}{0.841128in}}%
\pgfpathlineto{\pgfqpoint{2.860000in}{0.765830in}}%
\pgfpathlineto{\pgfqpoint{2.937500in}{0.756579in}}%
\pgfpathlineto{\pgfqpoint{3.015000in}{0.691815in}}%
\pgfpathlineto{\pgfqpoint{3.092500in}{0.700426in}}%
\pgfpathlineto{\pgfqpoint{3.170000in}{0.672212in}}%
\pgfpathlineto{\pgfqpoint{3.247500in}{0.713800in}}%
\pgfpathlineto{\pgfqpoint{3.325000in}{0.602960in}}%
\pgfpathlineto{\pgfqpoint{3.402500in}{0.606258in}}%
\pgfpathlineto{\pgfqpoint{3.480000in}{0.600000in}}%
\pgfpathlineto{\pgfqpoint{3.557500in}{0.606716in}}%
\pgfpathlineto{\pgfqpoint{3.635000in}{0.607082in}}%
\pgfpathlineto{\pgfqpoint{3.712500in}{0.601403in}}%
\pgfpathlineto{\pgfqpoint{3.790000in}{0.609098in}}%
\pgfpathlineto{\pgfqpoint{3.867500in}{0.604059in}}%
\pgfpathlineto{\pgfqpoint{3.945000in}{0.603693in}}%
\pgfpathlineto{\pgfqpoint{4.022500in}{0.606349in}}%
\pgfpathlineto{\pgfqpoint{4.100000in}{0.600395in}}%
\pgfpathlineto{\pgfqpoint{4.177500in}{0.603693in}}%
\pgfpathlineto{\pgfqpoint{4.255000in}{0.603510in}}%
\pgfpathlineto{\pgfqpoint{4.332500in}{0.603968in}}%
\pgfpathlineto{\pgfqpoint{4.410000in}{0.600000in}}%
\pgfpathlineto{\pgfqpoint{4.487500in}{0.603510in}}%
\pgfpathlineto{\pgfqpoint{4.565000in}{0.604426in}}%
\pgfpathlineto{\pgfqpoint{4.642500in}{0.600000in}}%
\pgfpathlineto{\pgfqpoint{4.720000in}{0.607540in}}%
\pgfpathlineto{\pgfqpoint{4.797500in}{0.604517in}}%
\pgfpathlineto{\pgfqpoint{4.875000in}{0.600000in}}%
\pgfpathlineto{\pgfqpoint{4.952500in}{0.600000in}}%
\pgfpathlineto{\pgfqpoint{5.030000in}{0.600000in}}%
\pgfpathlineto{\pgfqpoint{5.107500in}{0.600000in}}%
\pgfpathlineto{\pgfqpoint{5.185000in}{0.600000in}}%
\pgfpathlineto{\pgfqpoint{5.262500in}{0.602227in}}%
\pgfpathlineto{\pgfqpoint{5.340000in}{0.610197in}}%
\pgfpathlineto{\pgfqpoint{5.417500in}{0.606166in}}%
\pgfpathlineto{\pgfqpoint{5.495000in}{0.604517in}}%
\pgfpathlineto{\pgfqpoint{5.572500in}{0.600000in}}%
\pgfpathlineto{\pgfqpoint{5.650000in}{0.604792in}}%
\pgfpathlineto{\pgfqpoint{5.727500in}{0.600000in}}%
\pgfpathlineto{\pgfqpoint{5.805000in}{0.600000in}}%
\pgfpathlineto{\pgfqpoint{5.882500in}{0.603052in}}%
\pgfpathlineto{\pgfqpoint{5.960000in}{0.605433in}}%
\pgfpathlineto{\pgfqpoint{6.037500in}{0.600000in}}%
\pgfpathlineto{\pgfqpoint{6.115000in}{0.602136in}}%
\pgfpathlineto{\pgfqpoint{6.192500in}{0.607174in}}%
\pgfpathlineto{\pgfqpoint{6.270000in}{0.602502in}}%
\pgfpathlineto{\pgfqpoint{6.347500in}{0.602136in}}%
\pgfpathlineto{\pgfqpoint{6.425000in}{0.602960in}}%
\pgfpathlineto{\pgfqpoint{6.502500in}{0.600000in}}%
\pgfpathlineto{\pgfqpoint{6.580000in}{0.600000in}}%
\pgfpathlineto{\pgfqpoint{6.657500in}{0.600000in}}%
\pgfpathlineto{\pgfqpoint{6.735000in}{0.602044in}}%
\pgfpathlineto{\pgfqpoint{6.812500in}{0.601128in}}%
\pgfpathlineto{\pgfqpoint{6.890000in}{0.603693in}}%
\pgfpathlineto{\pgfqpoint{6.967500in}{0.602136in}}%
\pgfpathlineto{\pgfqpoint{7.045000in}{0.601769in}}%
\pgfpathlineto{\pgfqpoint{7.122500in}{0.608640in}}%
\pgfusepath{stroke}%
\end{pgfscope}%
\begin{pgfscope}%
\pgfpathrectangle{\pgfqpoint{1.000000in}{0.600000in}}{\pgfqpoint{6.200000in}{4.800000in}}%
\pgfusepath{clip}%
\pgfsetrectcap%
\pgfsetroundjoin%
\pgfsetlinewidth{2.007500pt}%
\definecolor{currentstroke}{rgb}{0.000000,0.500000,0.000000}%
\pgfsetstrokecolor{currentstroke}%
\pgfsetdash{}{0pt}%
\pgfpathmoveto{\pgfqpoint{1.000000in}{0.600000in}}%
\pgfpathlineto{\pgfqpoint{1.077500in}{0.626005in}}%
\pgfpathlineto{\pgfqpoint{1.155000in}{0.658095in}}%
\pgfpathlineto{\pgfqpoint{1.232500in}{0.854590in}}%
\pgfpathlineto{\pgfqpoint{1.310000in}{1.269061in}}%
\pgfpathlineto{\pgfqpoint{1.387500in}{2.353915in}}%
\pgfpathlineto{\pgfqpoint{1.465000in}{3.839848in}}%
\pgfpathlineto{\pgfqpoint{1.542500in}{4.857466in}}%
\pgfpathlineto{\pgfqpoint{1.620000in}{4.741511in}}%
\pgfpathlineto{\pgfqpoint{1.697500in}{4.075713in}}%
\pgfpathlineto{\pgfqpoint{1.775000in}{3.215309in}}%
\pgfpathlineto{\pgfqpoint{1.852500in}{2.501870in}}%
\pgfpathlineto{\pgfqpoint{1.930000in}{2.011443in}}%
\pgfpathlineto{\pgfqpoint{2.007500in}{1.784387in}}%
\pgfpathlineto{\pgfqpoint{2.085000in}{1.559129in}}%
\pgfpathlineto{\pgfqpoint{2.162500in}{1.384927in}}%
\pgfpathlineto{\pgfqpoint{2.240000in}{1.198769in}}%
\pgfpathlineto{\pgfqpoint{2.317500in}{1.112837in}}%
\pgfpathlineto{\pgfqpoint{2.395000in}{0.977016in}}%
\pgfpathlineto{\pgfqpoint{2.472500in}{0.948612in}}%
\pgfpathlineto{\pgfqpoint{2.550000in}{0.903129in}}%
\pgfpathlineto{\pgfqpoint{2.627500in}{0.914275in}}%
\pgfpathlineto{\pgfqpoint{2.705000in}{0.848028in}}%
\pgfpathlineto{\pgfqpoint{2.782500in}{0.829061in}}%
\pgfpathlineto{\pgfqpoint{2.860000in}{0.763443in}}%
\pgfpathlineto{\pgfqpoint{2.937500in}{0.751578in}}%
\pgfpathlineto{\pgfqpoint{3.015000in}{0.685870in}}%
\pgfpathlineto{\pgfqpoint{3.092500in}{0.708612in}}%
\pgfpathlineto{\pgfqpoint{3.170000in}{0.674365in}}%
\pgfpathlineto{\pgfqpoint{3.247500in}{0.703398in}}%
\pgfpathlineto{\pgfqpoint{3.325000in}{0.605061in}}%
\pgfpathlineto{\pgfqpoint{3.402500in}{0.608837in}}%
\pgfpathlineto{\pgfqpoint{3.480000in}{0.600000in}}%
\pgfpathlineto{\pgfqpoint{3.557500in}{0.601286in}}%
\pgfpathlineto{\pgfqpoint{3.635000in}{0.607398in}}%
\pgfpathlineto{\pgfqpoint{3.712500in}{0.600000in}}%
\pgfpathlineto{\pgfqpoint{3.790000in}{0.603893in}}%
\pgfpathlineto{\pgfqpoint{3.867500in}{0.607848in}}%
\pgfpathlineto{\pgfqpoint{3.945000in}{0.600477in}}%
\pgfpathlineto{\pgfqpoint{4.022500in}{0.607488in}}%
\pgfpathlineto{\pgfqpoint{4.100000in}{0.604252in}}%
\pgfpathlineto{\pgfqpoint{4.177500in}{0.602904in}}%
\pgfpathlineto{\pgfqpoint{4.255000in}{0.601376in}}%
\pgfpathlineto{\pgfqpoint{4.332500in}{0.600000in}}%
\pgfpathlineto{\pgfqpoint{4.410000in}{0.608657in}}%
\pgfpathlineto{\pgfqpoint{4.487500in}{0.600747in}}%
\pgfpathlineto{\pgfqpoint{4.565000in}{0.600000in}}%
\pgfpathlineto{\pgfqpoint{4.642500in}{0.603354in}}%
\pgfpathlineto{\pgfqpoint{4.720000in}{0.600000in}}%
\pgfpathlineto{\pgfqpoint{4.797500in}{0.604972in}}%
\pgfpathlineto{\pgfqpoint{4.875000in}{0.602724in}}%
\pgfpathlineto{\pgfqpoint{4.952500in}{0.609825in}}%
\pgfpathlineto{\pgfqpoint{5.030000in}{0.608207in}}%
\pgfpathlineto{\pgfqpoint{5.107500in}{0.600000in}}%
\pgfpathlineto{\pgfqpoint{5.185000in}{0.603443in}}%
\pgfpathlineto{\pgfqpoint{5.262500in}{0.601466in}}%
\pgfpathlineto{\pgfqpoint{5.340000in}{0.601736in}}%
\pgfpathlineto{\pgfqpoint{5.417500in}{0.604702in}}%
\pgfpathlineto{\pgfqpoint{5.495000in}{0.616118in}}%
\pgfpathlineto{\pgfqpoint{5.572500in}{0.600000in}}%
\pgfpathlineto{\pgfqpoint{5.650000in}{0.600000in}}%
\pgfpathlineto{\pgfqpoint{5.727500in}{0.606590in}}%
\pgfpathlineto{\pgfqpoint{5.805000in}{0.610005in}}%
\pgfpathlineto{\pgfqpoint{5.882500in}{0.600000in}}%
\pgfpathlineto{\pgfqpoint{5.960000in}{0.604342in}}%
\pgfpathlineto{\pgfqpoint{6.037500in}{0.607668in}}%
\pgfpathlineto{\pgfqpoint{6.115000in}{0.601106in}}%
\pgfpathlineto{\pgfqpoint{6.192500in}{0.600000in}}%
\pgfpathlineto{\pgfqpoint{6.270000in}{0.606500in}}%
\pgfpathlineto{\pgfqpoint{6.347500in}{0.600000in}}%
\pgfpathlineto{\pgfqpoint{6.425000in}{0.601825in}}%
\pgfpathlineto{\pgfqpoint{6.502500in}{0.600000in}}%
\pgfpathlineto{\pgfqpoint{6.580000in}{0.603803in}}%
\pgfpathlineto{\pgfqpoint{6.657500in}{0.600000in}}%
\pgfpathlineto{\pgfqpoint{6.735000in}{0.606769in}}%
\pgfpathlineto{\pgfqpoint{6.812500in}{0.606050in}}%
\pgfpathlineto{\pgfqpoint{6.890000in}{0.603443in}}%
\pgfpathlineto{\pgfqpoint{6.967500in}{0.601646in}}%
\pgfpathlineto{\pgfqpoint{7.045000in}{0.601736in}}%
\pgfpathlineto{\pgfqpoint{7.122500in}{0.603174in}}%
\pgfusepath{stroke}%
\end{pgfscope}%
\begin{pgfscope}%
\pgfpathrectangle{\pgfqpoint{1.000000in}{0.600000in}}{\pgfqpoint{6.200000in}{4.800000in}}%
\pgfusepath{clip}%
\pgfsetrectcap%
\pgfsetroundjoin%
\pgfsetlinewidth{2.007500pt}%
\definecolor{currentstroke}{rgb}{1.000000,0.000000,0.000000}%
\pgfsetstrokecolor{currentstroke}%
\pgfsetdash{}{0pt}%
\pgfpathmoveto{\pgfqpoint{1.000000in}{0.600000in}}%
\pgfpathlineto{\pgfqpoint{1.077500in}{0.623865in}}%
\pgfpathlineto{\pgfqpoint{1.155000in}{0.667625in}}%
\pgfpathlineto{\pgfqpoint{1.232500in}{0.859895in}}%
\pgfpathlineto{\pgfqpoint{1.310000in}{1.289640in}}%
\pgfpathlineto{\pgfqpoint{1.387500in}{2.395534in}}%
\pgfpathlineto{\pgfqpoint{1.465000in}{3.918903in}}%
\pgfpathlineto{\pgfqpoint{1.542500in}{4.949820in}}%
\pgfpathlineto{\pgfqpoint{1.620000in}{4.837399in}}%
\pgfpathlineto{\pgfqpoint{1.697500in}{4.156918in}}%
\pgfpathlineto{\pgfqpoint{1.775000in}{3.274692in}}%
\pgfpathlineto{\pgfqpoint{1.852500in}{2.550452in}}%
\pgfpathlineto{\pgfqpoint{1.930000in}{2.050963in}}%
\pgfpathlineto{\pgfqpoint{2.007500in}{1.809429in}}%
\pgfpathlineto{\pgfqpoint{2.085000in}{1.582422in}}%
\pgfpathlineto{\pgfqpoint{2.162500in}{1.404406in}}%
\pgfpathlineto{\pgfqpoint{2.240000in}{1.226031in}}%
\pgfpathlineto{\pgfqpoint{2.317500in}{1.110813in}}%
\pgfpathlineto{\pgfqpoint{2.395000in}{0.981700in}}%
\pgfpathlineto{\pgfqpoint{2.472500in}{0.957429in}}%
\pgfpathlineto{\pgfqpoint{2.550000in}{0.913309in}}%
\pgfpathlineto{\pgfqpoint{2.627500in}{0.912226in}}%
\pgfpathlineto{\pgfqpoint{2.705000in}{0.861249in}}%
\pgfpathlineto{\pgfqpoint{2.782500in}{0.830752in}}%
\pgfpathlineto{\pgfqpoint{2.860000in}{0.771564in}}%
\pgfpathlineto{\pgfqpoint{2.937500in}{0.755053in}}%
\pgfpathlineto{\pgfqpoint{3.015000in}{0.699474in}}%
\pgfpathlineto{\pgfqpoint{3.092500in}{0.699655in}}%
\pgfpathlineto{\pgfqpoint{3.170000in}{0.673399in}}%
\pgfpathlineto{\pgfqpoint{3.247500in}{0.713640in}}%
\pgfpathlineto{\pgfqpoint{3.325000in}{0.601579in}}%
\pgfpathlineto{\pgfqpoint{3.402500in}{0.604016in}}%
\pgfpathlineto{\pgfqpoint{3.480000in}{0.602301in}}%
\pgfpathlineto{\pgfqpoint{3.557500in}{0.600000in}}%
\pgfpathlineto{\pgfqpoint{3.635000in}{0.600000in}}%
\pgfpathlineto{\pgfqpoint{3.712500in}{0.603835in}}%
\pgfpathlineto{\pgfqpoint{3.790000in}{0.600316in}}%
\pgfpathlineto{\pgfqpoint{3.867500in}{0.600000in}}%
\pgfpathlineto{\pgfqpoint{3.945000in}{0.605098in}}%
\pgfpathlineto{\pgfqpoint{4.022500in}{0.602391in}}%
\pgfpathlineto{\pgfqpoint{4.100000in}{0.602031in}}%
\pgfpathlineto{\pgfqpoint{4.177500in}{0.602391in}}%
\pgfpathlineto{\pgfqpoint{4.255000in}{0.600587in}}%
\pgfpathlineto{\pgfqpoint{4.332500in}{0.607895in}}%
\pgfpathlineto{\pgfqpoint{4.410000in}{0.601850in}}%
\pgfpathlineto{\pgfqpoint{4.487500in}{0.600000in}}%
\pgfpathlineto{\pgfqpoint{4.565000in}{0.600000in}}%
\pgfpathlineto{\pgfqpoint{4.642500in}{0.604196in}}%
\pgfpathlineto{\pgfqpoint{4.720000in}{0.606722in}}%
\pgfpathlineto{\pgfqpoint{4.797500in}{0.604918in}}%
\pgfpathlineto{\pgfqpoint{4.875000in}{0.600000in}}%
\pgfpathlineto{\pgfqpoint{4.952500in}{0.600000in}}%
\pgfpathlineto{\pgfqpoint{5.030000in}{0.606361in}}%
\pgfpathlineto{\pgfqpoint{5.107500in}{0.602031in}}%
\pgfpathlineto{\pgfqpoint{5.185000in}{0.606993in}}%
\pgfpathlineto{\pgfqpoint{5.262500in}{0.607534in}}%
\pgfpathlineto{\pgfqpoint{5.340000in}{0.600497in}}%
\pgfpathlineto{\pgfqpoint{5.417500in}{0.600000in}}%
\pgfpathlineto{\pgfqpoint{5.495000in}{0.600000in}}%
\pgfpathlineto{\pgfqpoint{5.572500in}{0.610331in}}%
\pgfpathlineto{\pgfqpoint{5.650000in}{0.607083in}}%
\pgfpathlineto{\pgfqpoint{5.727500in}{0.603745in}}%
\pgfpathlineto{\pgfqpoint{5.805000in}{0.602121in}}%
\pgfpathlineto{\pgfqpoint{5.882500in}{0.600858in}}%
\pgfpathlineto{\pgfqpoint{5.960000in}{0.602662in}}%
\pgfpathlineto{\pgfqpoint{6.037500in}{0.606813in}}%
\pgfpathlineto{\pgfqpoint{6.115000in}{0.600000in}}%
\pgfpathlineto{\pgfqpoint{6.192500in}{0.607534in}}%
\pgfpathlineto{\pgfqpoint{6.270000in}{0.604467in}}%
\pgfpathlineto{\pgfqpoint{6.347500in}{0.600000in}}%
\pgfpathlineto{\pgfqpoint{6.425000in}{0.608076in}}%
\pgfpathlineto{\pgfqpoint{6.502500in}{0.600000in}}%
\pgfpathlineto{\pgfqpoint{6.580000in}{0.603203in}}%
\pgfpathlineto{\pgfqpoint{6.657500in}{0.605730in}}%
\pgfpathlineto{\pgfqpoint{6.735000in}{0.609609in}}%
\pgfpathlineto{\pgfqpoint{6.812500in}{0.600000in}}%
\pgfpathlineto{\pgfqpoint{6.890000in}{0.605549in}}%
\pgfpathlineto{\pgfqpoint{6.967500in}{0.604286in}}%
\pgfpathlineto{\pgfqpoint{7.045000in}{0.615745in}}%
\pgfpathlineto{\pgfqpoint{7.122500in}{0.613489in}}%
\pgfusepath{stroke}%
\end{pgfscope}%
\begin{pgfscope}%
\pgfpathrectangle{\pgfqpoint{1.000000in}{0.600000in}}{\pgfqpoint{6.200000in}{4.800000in}}%
\pgfusepath{clip}%
\pgfsetrectcap%
\pgfsetroundjoin%
\pgfsetlinewidth{2.007500pt}%
\definecolor{currentstroke}{rgb}{0.000000,0.750000,0.750000}%
\pgfsetstrokecolor{currentstroke}%
\pgfsetdash{}{0pt}%
\pgfpathmoveto{\pgfqpoint{1.000000in}{0.600000in}}%
\pgfpathlineto{\pgfqpoint{1.077500in}{0.630343in}}%
\pgfpathlineto{\pgfqpoint{1.155000in}{0.662586in}}%
\pgfpathlineto{\pgfqpoint{1.232500in}{0.867884in}}%
\pgfpathlineto{\pgfqpoint{1.310000in}{1.293599in}}%
\pgfpathlineto{\pgfqpoint{1.387500in}{2.432665in}}%
\pgfpathlineto{\pgfqpoint{1.465000in}{3.990161in}}%
\pgfpathlineto{\pgfqpoint{1.542500in}{5.052081in}}%
\pgfpathlineto{\pgfqpoint{1.620000in}{4.935223in}}%
\pgfpathlineto{\pgfqpoint{1.697500in}{4.238267in}}%
\pgfpathlineto{\pgfqpoint{1.775000in}{3.336468in}}%
\pgfpathlineto{\pgfqpoint{1.852500in}{2.598889in}}%
\pgfpathlineto{\pgfqpoint{1.930000in}{2.085371in}}%
\pgfpathlineto{\pgfqpoint{2.007500in}{1.836536in}}%
\pgfpathlineto{\pgfqpoint{2.085000in}{1.605280in}}%
\pgfpathlineto{\pgfqpoint{2.162500in}{1.427762in}}%
\pgfpathlineto{\pgfqpoint{2.240000in}{1.240316in}}%
\pgfpathlineto{\pgfqpoint{2.317500in}{1.116081in}}%
\pgfpathlineto{\pgfqpoint{2.395000in}{0.994487in}}%
\pgfpathlineto{\pgfqpoint{2.472500in}{0.970532in}}%
\pgfpathlineto{\pgfqpoint{2.550000in}{0.913333in}}%
\pgfpathlineto{\pgfqpoint{2.627500in}{0.914973in}}%
\pgfpathlineto{\pgfqpoint{2.705000in}{0.861963in}}%
\pgfpathlineto{\pgfqpoint{2.782500in}{0.842290in}}%
\pgfpathlineto{\pgfqpoint{2.860000in}{0.773523in}}%
\pgfpathlineto{\pgfqpoint{2.937500in}{0.756582in}}%
\pgfpathlineto{\pgfqpoint{3.015000in}{0.698563in}}%
\pgfpathlineto{\pgfqpoint{3.092500in}{0.704939in}}%
\pgfpathlineto{\pgfqpoint{3.170000in}{0.675246in}}%
\pgfpathlineto{\pgfqpoint{3.247500in}{0.717872in}}%
\pgfpathlineto{\pgfqpoint{3.325000in}{0.606661in}}%
\pgfpathlineto{\pgfqpoint{3.402500in}{0.600741in}}%
\pgfpathlineto{\pgfqpoint{3.480000in}{0.604202in}}%
\pgfpathlineto{\pgfqpoint{3.557500in}{0.603018in}}%
\pgfpathlineto{\pgfqpoint{3.635000in}{0.602927in}}%
\pgfpathlineto{\pgfqpoint{3.712500in}{0.605022in}}%
\pgfpathlineto{\pgfqpoint{3.790000in}{0.607390in}}%
\pgfpathlineto{\pgfqpoint{3.867500in}{0.602563in}}%
\pgfpathlineto{\pgfqpoint{3.945000in}{0.608301in}}%
\pgfpathlineto{\pgfqpoint{4.022500in}{0.602198in}}%
\pgfpathlineto{\pgfqpoint{4.100000in}{0.600000in}}%
\pgfpathlineto{\pgfqpoint{4.177500in}{0.603109in}}%
\pgfpathlineto{\pgfqpoint{4.255000in}{0.604293in}}%
\pgfpathlineto{\pgfqpoint{4.332500in}{0.606753in}}%
\pgfpathlineto{\pgfqpoint{4.410000in}{0.604293in}}%
\pgfpathlineto{\pgfqpoint{4.487500in}{0.609212in}}%
\pgfpathlineto{\pgfqpoint{4.565000in}{0.600000in}}%
\pgfpathlineto{\pgfqpoint{4.642500in}{0.600000in}}%
\pgfpathlineto{\pgfqpoint{4.720000in}{0.600000in}}%
\pgfpathlineto{\pgfqpoint{4.797500in}{0.603656in}}%
\pgfpathlineto{\pgfqpoint{4.875000in}{0.600923in}}%
\pgfpathlineto{\pgfqpoint{4.952500in}{0.603747in}}%
\pgfpathlineto{\pgfqpoint{5.030000in}{0.602381in}}%
\pgfpathlineto{\pgfqpoint{5.107500in}{0.600000in}}%
\pgfpathlineto{\pgfqpoint{5.185000in}{0.605660in}}%
\pgfpathlineto{\pgfqpoint{5.262500in}{0.601652in}}%
\pgfpathlineto{\pgfqpoint{5.340000in}{0.606479in}}%
\pgfpathlineto{\pgfqpoint{5.417500in}{0.608939in}}%
\pgfpathlineto{\pgfqpoint{5.495000in}{0.607846in}}%
\pgfpathlineto{\pgfqpoint{5.572500in}{0.600000in}}%
\pgfpathlineto{\pgfqpoint{5.650000in}{0.600000in}}%
\pgfpathlineto{\pgfqpoint{5.727500in}{0.600000in}}%
\pgfpathlineto{\pgfqpoint{5.805000in}{0.600000in}}%
\pgfpathlineto{\pgfqpoint{5.882500in}{0.603838in}}%
\pgfpathlineto{\pgfqpoint{5.960000in}{0.606297in}}%
\pgfpathlineto{\pgfqpoint{6.037500in}{0.603109in}}%
\pgfpathlineto{\pgfqpoint{6.115000in}{0.603474in}}%
\pgfpathlineto{\pgfqpoint{6.192500in}{0.604111in}}%
\pgfpathlineto{\pgfqpoint{6.270000in}{0.607026in}}%
\pgfpathlineto{\pgfqpoint{6.347500in}{0.600000in}}%
\pgfpathlineto{\pgfqpoint{6.425000in}{0.605295in}}%
\pgfpathlineto{\pgfqpoint{6.502500in}{0.604111in}}%
\pgfpathlineto{\pgfqpoint{6.580000in}{0.603474in}}%
\pgfpathlineto{\pgfqpoint{6.657500in}{0.600000in}}%
\pgfpathlineto{\pgfqpoint{6.735000in}{0.603383in}}%
\pgfpathlineto{\pgfqpoint{6.812500in}{0.606206in}}%
\pgfpathlineto{\pgfqpoint{6.890000in}{0.611398in}}%
\pgfpathlineto{\pgfqpoint{6.967500in}{0.600000in}}%
\pgfpathlineto{\pgfqpoint{7.045000in}{0.601652in}}%
\pgfpathlineto{\pgfqpoint{7.122500in}{0.603656in}}%
\pgfusepath{stroke}%
\end{pgfscope}%
\begin{pgfscope}%
\pgfpathrectangle{\pgfqpoint{1.000000in}{0.600000in}}{\pgfqpoint{6.200000in}{4.800000in}}%
\pgfusepath{clip}%
\pgfsetrectcap%
\pgfsetroundjoin%
\pgfsetlinewidth{2.007500pt}%
\definecolor{currentstroke}{rgb}{0.750000,0.000000,0.750000}%
\pgfsetstrokecolor{currentstroke}%
\pgfsetdash{}{0pt}%
\pgfpathmoveto{\pgfqpoint{1.000000in}{0.600000in}}%
\pgfpathlineto{\pgfqpoint{1.077500in}{0.619931in}}%
\pgfpathlineto{\pgfqpoint{1.155000in}{0.662679in}}%
\pgfpathlineto{\pgfqpoint{1.232500in}{0.854956in}}%
\pgfpathlineto{\pgfqpoint{1.310000in}{1.283639in}}%
\pgfpathlineto{\pgfqpoint{1.387500in}{2.381743in}}%
\pgfpathlineto{\pgfqpoint{1.465000in}{3.898487in}}%
\pgfpathlineto{\pgfqpoint{1.542500in}{4.917543in}}%
\pgfpathlineto{\pgfqpoint{1.620000in}{4.807539in}}%
\pgfpathlineto{\pgfqpoint{1.697500in}{4.125313in}}%
\pgfpathlineto{\pgfqpoint{1.775000in}{3.257351in}}%
\pgfpathlineto{\pgfqpoint{1.852500in}{2.538733in}}%
\pgfpathlineto{\pgfqpoint{1.930000in}{2.037359in}}%
\pgfpathlineto{\pgfqpoint{2.007500in}{1.802242in}}%
\pgfpathlineto{\pgfqpoint{2.085000in}{1.567401in}}%
\pgfpathlineto{\pgfqpoint{2.162500in}{1.409489in}}%
\pgfpathlineto{\pgfqpoint{2.240000in}{1.213712in}}%
\pgfpathlineto{\pgfqpoint{2.317500in}{1.117712in}}%
\pgfpathlineto{\pgfqpoint{2.395000in}{0.986334in}}%
\pgfpathlineto{\pgfqpoint{2.472500in}{0.967355in}}%
\pgfpathlineto{\pgfqpoint{2.550000in}{0.914196in}}%
\pgfpathlineto{\pgfqpoint{2.627500in}{0.904798in}}%
\pgfpathlineto{\pgfqpoint{2.705000in}{0.853850in}}%
\pgfpathlineto{\pgfqpoint{2.782500in}{0.829804in}}%
\pgfpathlineto{\pgfqpoint{2.860000in}{0.773420in}}%
\pgfpathlineto{\pgfqpoint{2.937500in}{0.758956in}}%
\pgfpathlineto{\pgfqpoint{3.015000in}{0.693359in}}%
\pgfpathlineto{\pgfqpoint{3.092500in}{0.699900in}}%
\pgfpathlineto{\pgfqpoint{3.170000in}{0.671984in}}%
\pgfpathlineto{\pgfqpoint{3.247500in}{0.712706in}}%
\pgfpathlineto{\pgfqpoint{3.325000in}{0.604545in}}%
\pgfpathlineto{\pgfqpoint{3.402500in}{0.602979in}}%
\pgfpathlineto{\pgfqpoint{3.480000in}{0.600768in}}%
\pgfpathlineto{\pgfqpoint{3.557500in}{0.600000in}}%
\pgfpathlineto{\pgfqpoint{3.635000in}{0.602979in}}%
\pgfpathlineto{\pgfqpoint{3.712500in}{0.600000in}}%
\pgfpathlineto{\pgfqpoint{3.790000in}{0.600000in}}%
\pgfpathlineto{\pgfqpoint{3.867500in}{0.604821in}}%
\pgfpathlineto{\pgfqpoint{3.945000in}{0.600000in}}%
\pgfpathlineto{\pgfqpoint{4.022500in}{0.600000in}}%
\pgfpathlineto{\pgfqpoint{4.100000in}{0.604729in}}%
\pgfpathlineto{\pgfqpoint{4.177500in}{0.600000in}}%
\pgfpathlineto{\pgfqpoint{4.255000in}{0.600000in}}%
\pgfpathlineto{\pgfqpoint{4.332500in}{0.600000in}}%
\pgfpathlineto{\pgfqpoint{4.410000in}{0.605743in}}%
\pgfpathlineto{\pgfqpoint{4.487500in}{0.600000in}}%
\pgfpathlineto{\pgfqpoint{4.565000in}{0.600000in}}%
\pgfpathlineto{\pgfqpoint{4.642500in}{0.603992in}}%
\pgfpathlineto{\pgfqpoint{4.720000in}{0.602426in}}%
\pgfpathlineto{\pgfqpoint{4.797500in}{0.600000in}}%
\pgfpathlineto{\pgfqpoint{4.875000in}{0.601781in}}%
\pgfpathlineto{\pgfqpoint{4.952500in}{0.600000in}}%
\pgfpathlineto{\pgfqpoint{5.030000in}{0.602794in}}%
\pgfpathlineto{\pgfqpoint{5.107500in}{0.600000in}}%
\pgfpathlineto{\pgfqpoint{5.185000in}{0.608875in}}%
\pgfpathlineto{\pgfqpoint{5.262500in}{0.600000in}}%
\pgfpathlineto{\pgfqpoint{5.340000in}{0.611178in}}%
\pgfpathlineto{\pgfqpoint{5.417500in}{0.601044in}}%
\pgfpathlineto{\pgfqpoint{5.495000in}{0.610626in}}%
\pgfpathlineto{\pgfqpoint{5.572500in}{0.604084in}}%
\pgfpathlineto{\pgfqpoint{5.650000in}{0.603808in}}%
\pgfpathlineto{\pgfqpoint{5.727500in}{0.601689in}}%
\pgfpathlineto{\pgfqpoint{5.805000in}{0.600000in}}%
\pgfpathlineto{\pgfqpoint{5.882500in}{0.600768in}}%
\pgfpathlineto{\pgfqpoint{5.960000in}{0.600000in}}%
\pgfpathlineto{\pgfqpoint{6.037500in}{0.600000in}}%
\pgfpathlineto{\pgfqpoint{6.115000in}{0.600000in}}%
\pgfpathlineto{\pgfqpoint{6.192500in}{0.608599in}}%
\pgfpathlineto{\pgfqpoint{6.270000in}{0.610441in}}%
\pgfpathlineto{\pgfqpoint{6.347500in}{0.609428in}}%
\pgfpathlineto{\pgfqpoint{6.425000in}{0.603716in}}%
\pgfpathlineto{\pgfqpoint{6.502500in}{0.602518in}}%
\pgfpathlineto{\pgfqpoint{6.580000in}{0.600000in}}%
\pgfpathlineto{\pgfqpoint{6.657500in}{0.600000in}}%
\pgfpathlineto{\pgfqpoint{6.735000in}{0.607493in}}%
\pgfpathlineto{\pgfqpoint{6.812500in}{0.601689in}}%
\pgfpathlineto{\pgfqpoint{6.890000in}{0.601228in}}%
\pgfpathlineto{\pgfqpoint{6.967500in}{0.600675in}}%
\pgfpathlineto{\pgfqpoint{7.045000in}{0.607770in}}%
\pgfpathlineto{\pgfqpoint{7.122500in}{0.604913in}}%
\pgfusepath{stroke}%
\end{pgfscope}%
\begin{pgfscope}%
\pgfpathrectangle{\pgfqpoint{1.000000in}{0.600000in}}{\pgfqpoint{6.200000in}{4.800000in}}%
\pgfusepath{clip}%
\pgfsetrectcap%
\pgfsetroundjoin%
\pgfsetlinewidth{2.007500pt}%
\definecolor{currentstroke}{rgb}{0.750000,0.750000,0.000000}%
\pgfsetstrokecolor{currentstroke}%
\pgfsetdash{}{0pt}%
\pgfpathmoveto{\pgfqpoint{1.000000in}{0.600000in}}%
\pgfpathlineto{\pgfqpoint{1.077500in}{0.620039in}}%
\pgfpathlineto{\pgfqpoint{1.155000in}{0.653035in}}%
\pgfpathlineto{\pgfqpoint{1.232500in}{0.848638in}}%
\pgfpathlineto{\pgfqpoint{1.310000in}{1.261556in}}%
\pgfpathlineto{\pgfqpoint{1.387500in}{2.316276in}}%
\pgfpathlineto{\pgfqpoint{1.465000in}{3.784578in}}%
\pgfpathlineto{\pgfqpoint{1.542500in}{4.774824in}}%
\pgfpathlineto{\pgfqpoint{1.620000in}{4.674131in}}%
\pgfpathlineto{\pgfqpoint{1.697500in}{4.016211in}}%
\pgfpathlineto{\pgfqpoint{1.775000in}{3.164679in}}%
\pgfpathlineto{\pgfqpoint{1.852500in}{2.462765in}}%
\pgfpathlineto{\pgfqpoint{1.930000in}{1.992863in}}%
\pgfpathlineto{\pgfqpoint{2.007500in}{1.758101in}}%
\pgfpathlineto{\pgfqpoint{2.085000in}{1.544673in}}%
\pgfpathlineto{\pgfqpoint{2.162500in}{1.390030in}}%
\pgfpathlineto{\pgfqpoint{2.240000in}{1.193384in}}%
\pgfpathlineto{\pgfqpoint{2.317500in}{1.087950in}}%
\pgfpathlineto{\pgfqpoint{2.395000in}{0.964786in}}%
\pgfpathlineto{\pgfqpoint{2.472500in}{0.947055in}}%
\pgfpathlineto{\pgfqpoint{2.550000in}{0.898321in}}%
\pgfpathlineto{\pgfqpoint{2.627500in}{0.907707in}}%
\pgfpathlineto{\pgfqpoint{2.705000in}{0.850818in}}%
\pgfpathlineto{\pgfqpoint{2.782500in}{0.837544in}}%
\pgfpathlineto{\pgfqpoint{2.860000in}{0.758090in}}%
\pgfpathlineto{\pgfqpoint{2.937500in}{0.746427in}}%
\pgfpathlineto{\pgfqpoint{3.015000in}{0.696460in}}%
\pgfpathlineto{\pgfqpoint{3.092500in}{0.697503in}}%
\pgfpathlineto{\pgfqpoint{3.170000in}{0.662706in}}%
\pgfpathlineto{\pgfqpoint{3.247500in}{0.710872in}}%
\pgfpathlineto{\pgfqpoint{3.325000in}{0.600000in}}%
\pgfpathlineto{\pgfqpoint{3.402500in}{0.600000in}}%
\pgfpathlineto{\pgfqpoint{3.480000in}{0.609230in}}%
\pgfpathlineto{\pgfqpoint{3.557500in}{0.602309in}}%
\pgfpathlineto{\pgfqpoint{3.635000in}{0.612170in}}%
\pgfpathlineto{\pgfqpoint{3.712500in}{0.605912in}}%
\pgfpathlineto{\pgfqpoint{3.790000in}{0.607618in}}%
\pgfpathlineto{\pgfqpoint{3.867500in}{0.604584in}}%
\pgfpathlineto{\pgfqpoint{3.945000in}{0.607334in}}%
\pgfpathlineto{\pgfqpoint{4.022500in}{0.610747in}}%
\pgfpathlineto{\pgfqpoint{4.100000in}{0.600000in}}%
\pgfpathlineto{\pgfqpoint{4.177500in}{0.605627in}}%
\pgfpathlineto{\pgfqpoint{4.255000in}{0.601550in}}%
\pgfpathlineto{\pgfqpoint{4.332500in}{0.610178in}}%
\pgfpathlineto{\pgfqpoint{4.410000in}{0.608661in}}%
\pgfpathlineto{\pgfqpoint{4.487500in}{0.600887in}}%
\pgfpathlineto{\pgfqpoint{4.565000in}{0.606386in}}%
\pgfpathlineto{\pgfqpoint{4.642500in}{0.603541in}}%
\pgfpathlineto{\pgfqpoint{4.720000in}{0.607713in}}%
\pgfpathlineto{\pgfqpoint{4.797500in}{0.600000in}}%
\pgfpathlineto{\pgfqpoint{4.875000in}{0.600000in}}%
\pgfpathlineto{\pgfqpoint{4.952500in}{0.600000in}}%
\pgfpathlineto{\pgfqpoint{5.030000in}{0.604490in}}%
\pgfpathlineto{\pgfqpoint{5.107500in}{0.605817in}}%
\pgfpathlineto{\pgfqpoint{5.185000in}{0.600000in}}%
\pgfpathlineto{\pgfqpoint{5.262500in}{0.602878in}}%
\pgfpathlineto{\pgfqpoint{5.340000in}{0.600507in}}%
\pgfpathlineto{\pgfqpoint{5.417500in}{0.605248in}}%
\pgfpathlineto{\pgfqpoint{5.495000in}{0.605627in}}%
\pgfpathlineto{\pgfqpoint{5.572500in}{0.601266in}}%
\pgfpathlineto{\pgfqpoint{5.650000in}{0.603352in}}%
\pgfpathlineto{\pgfqpoint{5.727500in}{0.602214in}}%
\pgfpathlineto{\pgfqpoint{5.805000in}{0.600000in}}%
\pgfpathlineto{\pgfqpoint{5.882500in}{0.604110in}}%
\pgfpathlineto{\pgfqpoint{5.960000in}{0.600000in}}%
\pgfpathlineto{\pgfqpoint{6.037500in}{0.605438in}}%
\pgfpathlineto{\pgfqpoint{6.115000in}{0.600000in}}%
\pgfpathlineto{\pgfqpoint{6.192500in}{0.602593in}}%
\pgfpathlineto{\pgfqpoint{6.270000in}{0.601171in}}%
\pgfpathlineto{\pgfqpoint{6.347500in}{0.600000in}}%
\pgfpathlineto{\pgfqpoint{6.425000in}{0.603921in}}%
\pgfpathlineto{\pgfqpoint{6.502500in}{0.602973in}}%
\pgfpathlineto{\pgfqpoint{6.580000in}{0.604679in}}%
\pgfpathlineto{\pgfqpoint{6.657500in}{0.605817in}}%
\pgfpathlineto{\pgfqpoint{6.735000in}{0.604679in}}%
\pgfpathlineto{\pgfqpoint{6.812500in}{0.601361in}}%
\pgfpathlineto{\pgfqpoint{6.890000in}{0.612738in}}%
\pgfpathlineto{\pgfqpoint{6.967500in}{0.601266in}}%
\pgfpathlineto{\pgfqpoint{7.045000in}{0.600000in}}%
\pgfpathlineto{\pgfqpoint{7.122500in}{0.600000in}}%
\pgfusepath{stroke}%
\end{pgfscope}%
\begin{pgfscope}%
\pgfpathrectangle{\pgfqpoint{1.000000in}{0.600000in}}{\pgfqpoint{6.200000in}{4.800000in}}%
\pgfusepath{clip}%
\pgfsetrectcap%
\pgfsetroundjoin%
\pgfsetlinewidth{2.007500pt}%
\definecolor{currentstroke}{rgb}{0.000000,0.000000,0.000000}%
\pgfsetstrokecolor{currentstroke}%
\pgfsetdash{}{0pt}%
\pgfpathmoveto{\pgfqpoint{1.000000in}{0.600000in}}%
\pgfpathlineto{\pgfqpoint{1.077500in}{0.622418in}}%
\pgfpathlineto{\pgfqpoint{1.155000in}{0.654464in}}%
\pgfpathlineto{\pgfqpoint{1.232500in}{0.837163in}}%
\pgfpathlineto{\pgfqpoint{1.310000in}{1.236634in}}%
\pgfpathlineto{\pgfqpoint{1.387500in}{2.272605in}}%
\pgfpathlineto{\pgfqpoint{1.465000in}{3.688525in}}%
\pgfpathlineto{\pgfqpoint{1.542500in}{4.646683in}}%
\pgfpathlineto{\pgfqpoint{1.620000in}{4.543822in}}%
\pgfpathlineto{\pgfqpoint{1.697500in}{3.905296in}}%
\pgfpathlineto{\pgfqpoint{1.775000in}{3.091437in}}%
\pgfpathlineto{\pgfqpoint{1.852500in}{2.412761in}}%
\pgfpathlineto{\pgfqpoint{1.930000in}{1.944962in}}%
\pgfpathlineto{\pgfqpoint{2.007500in}{1.730585in}}%
\pgfpathlineto{\pgfqpoint{2.085000in}{1.512248in}}%
\pgfpathlineto{\pgfqpoint{2.162500in}{1.356438in}}%
\pgfpathlineto{\pgfqpoint{2.240000in}{1.182763in}}%
\pgfpathlineto{\pgfqpoint{2.317500in}{1.075299in}}%
\pgfpathlineto{\pgfqpoint{2.395000in}{0.955126in}}%
\pgfpathlineto{\pgfqpoint{2.472500in}{0.936525in}}%
\pgfpathlineto{\pgfqpoint{2.550000in}{0.892323in}}%
\pgfpathlineto{\pgfqpoint{2.627500in}{0.895270in}}%
\pgfpathlineto{\pgfqpoint{2.705000in}{0.840479in}}%
\pgfpathlineto{\pgfqpoint{2.782500in}{0.824732in}}%
\pgfpathlineto{\pgfqpoint{2.860000in}{0.745261in}}%
\pgfpathlineto{\pgfqpoint{2.937500in}{0.746643in}}%
\pgfpathlineto{\pgfqpoint{3.015000in}{0.700599in}}%
\pgfpathlineto{\pgfqpoint{3.092500in}{0.696363in}}%
\pgfpathlineto{\pgfqpoint{3.170000in}{0.663489in}}%
\pgfpathlineto{\pgfqpoint{3.247500in}{0.699586in}}%
\pgfpathlineto{\pgfqpoint{3.325000in}{0.600000in}}%
\pgfpathlineto{\pgfqpoint{3.402500in}{0.602067in}}%
\pgfpathlineto{\pgfqpoint{3.480000in}{0.600000in}}%
\pgfpathlineto{\pgfqpoint{3.557500in}{0.600000in}}%
\pgfpathlineto{\pgfqpoint{3.635000in}{0.606487in}}%
\pgfpathlineto{\pgfqpoint{3.712500in}{0.608697in}}%
\pgfpathlineto{\pgfqpoint{3.790000in}{0.600000in}}%
\pgfpathlineto{\pgfqpoint{3.867500in}{0.604185in}}%
\pgfpathlineto{\pgfqpoint{3.945000in}{0.611276in}}%
\pgfpathlineto{\pgfqpoint{4.022500in}{0.604553in}}%
\pgfpathlineto{\pgfqpoint{4.100000in}{0.611276in}}%
\pgfpathlineto{\pgfqpoint{4.177500in}{0.603264in}}%
\pgfpathlineto{\pgfqpoint{4.255000in}{0.604553in}}%
\pgfpathlineto{\pgfqpoint{4.332500in}{0.600000in}}%
\pgfpathlineto{\pgfqpoint{4.410000in}{0.600000in}}%
\pgfpathlineto{\pgfqpoint{4.487500in}{0.611092in}}%
\pgfpathlineto{\pgfqpoint{4.565000in}{0.601699in}}%
\pgfpathlineto{\pgfqpoint{4.642500in}{0.601422in}}%
\pgfpathlineto{\pgfqpoint{4.720000in}{0.600000in}}%
\pgfpathlineto{\pgfqpoint{4.797500in}{0.600000in}}%
\pgfpathlineto{\pgfqpoint{4.875000in}{0.600000in}}%
\pgfpathlineto{\pgfqpoint{4.952500in}{0.606948in}}%
\pgfpathlineto{\pgfqpoint{5.030000in}{0.603172in}}%
\pgfpathlineto{\pgfqpoint{5.107500in}{0.610263in}}%
\pgfpathlineto{\pgfqpoint{5.185000in}{0.604830in}}%
\pgfpathlineto{\pgfqpoint{5.262500in}{0.600000in}}%
\pgfpathlineto{\pgfqpoint{5.340000in}{0.600000in}}%
\pgfpathlineto{\pgfqpoint{5.417500in}{0.600409in}}%
\pgfpathlineto{\pgfqpoint{5.495000in}{0.603264in}}%
\pgfpathlineto{\pgfqpoint{5.572500in}{0.602435in}}%
\pgfpathlineto{\pgfqpoint{5.650000in}{0.602988in}}%
\pgfpathlineto{\pgfqpoint{5.727500in}{0.603172in}}%
\pgfpathlineto{\pgfqpoint{5.805000in}{0.606395in}}%
\pgfpathlineto{\pgfqpoint{5.882500in}{0.605474in}}%
\pgfpathlineto{\pgfqpoint{5.960000in}{0.600870in}}%
\pgfpathlineto{\pgfqpoint{6.037500in}{0.601146in}}%
\pgfpathlineto{\pgfqpoint{6.115000in}{0.601054in}}%
\pgfpathlineto{\pgfqpoint{6.192500in}{0.600225in}}%
\pgfpathlineto{\pgfqpoint{6.270000in}{0.606211in}}%
\pgfpathlineto{\pgfqpoint{6.347500in}{0.604922in}}%
\pgfpathlineto{\pgfqpoint{6.425000in}{0.605290in}}%
\pgfpathlineto{\pgfqpoint{6.502500in}{0.607500in}}%
\pgfpathlineto{\pgfqpoint{6.580000in}{0.600000in}}%
\pgfpathlineto{\pgfqpoint{6.657500in}{0.602067in}}%
\pgfpathlineto{\pgfqpoint{6.735000in}{0.600000in}}%
\pgfpathlineto{\pgfqpoint{6.812500in}{0.602159in}}%
\pgfpathlineto{\pgfqpoint{6.890000in}{0.606487in}}%
\pgfpathlineto{\pgfqpoint{6.967500in}{0.603080in}}%
\pgfpathlineto{\pgfqpoint{7.045000in}{0.600000in}}%
\pgfpathlineto{\pgfqpoint{7.122500in}{0.603909in}}%
\pgfusepath{stroke}%
\end{pgfscope}%
\begin{pgfscope}%
\pgfpathrectangle{\pgfqpoint{1.000000in}{0.600000in}}{\pgfqpoint{6.200000in}{4.800000in}}%
\pgfusepath{clip}%
\pgfsetrectcap%
\pgfsetroundjoin%
\pgfsetlinewidth{2.007500pt}%
\definecolor{currentstroke}{rgb}{0.000000,0.000000,1.000000}%
\pgfsetstrokecolor{currentstroke}%
\pgfsetdash{}{0pt}%
\pgfpathmoveto{\pgfqpoint{1.000000in}{0.600000in}}%
\pgfpathlineto{\pgfqpoint{1.077500in}{0.628342in}}%
\pgfpathlineto{\pgfqpoint{1.155000in}{0.653185in}}%
\pgfpathlineto{\pgfqpoint{1.232500in}{0.832231in}}%
\pgfpathlineto{\pgfqpoint{1.310000in}{1.220401in}}%
\pgfpathlineto{\pgfqpoint{1.387500in}{2.236031in}}%
\pgfpathlineto{\pgfqpoint{1.465000in}{3.613835in}}%
\pgfpathlineto{\pgfqpoint{1.542500in}{4.556623in}}%
\pgfpathlineto{\pgfqpoint{1.620000in}{4.457606in}}%
\pgfpathlineto{\pgfqpoint{1.697500in}{3.835736in}}%
\pgfpathlineto{\pgfqpoint{1.775000in}{3.035529in}}%
\pgfpathlineto{\pgfqpoint{1.852500in}{2.374974in}}%
\pgfpathlineto{\pgfqpoint{1.930000in}{1.918220in}}%
\pgfpathlineto{\pgfqpoint{2.007500in}{1.702442in}}%
\pgfpathlineto{\pgfqpoint{2.085000in}{1.492076in}}%
\pgfpathlineto{\pgfqpoint{2.162500in}{1.340002in}}%
\pgfpathlineto{\pgfqpoint{2.240000in}{1.165481in}}%
\pgfpathlineto{\pgfqpoint{2.317500in}{1.070634in}}%
\pgfpathlineto{\pgfqpoint{2.395000in}{0.949525in}}%
\pgfpathlineto{\pgfqpoint{2.472500in}{0.931957in}}%
\pgfpathlineto{\pgfqpoint{2.550000in}{0.878279in}}%
\pgfpathlineto{\pgfqpoint{2.627500in}{0.887506in}}%
\pgfpathlineto{\pgfqpoint{2.705000in}{0.839063in}}%
\pgfpathlineto{\pgfqpoint{2.782500in}{0.815551in}}%
\pgfpathlineto{\pgfqpoint{2.860000in}{0.745015in}}%
\pgfpathlineto{\pgfqpoint{2.937500in}{0.739869in}}%
\pgfpathlineto{\pgfqpoint{3.015000in}{0.689296in}}%
\pgfpathlineto{\pgfqpoint{3.092500in}{0.685747in}}%
\pgfpathlineto{\pgfqpoint{3.170000in}{0.669244in}}%
\pgfpathlineto{\pgfqpoint{3.247500in}{0.701983in}}%
\pgfpathlineto{\pgfqpoint{3.325000in}{0.601813in}}%
\pgfpathlineto{\pgfqpoint{3.402500in}{0.607492in}}%
\pgfpathlineto{\pgfqpoint{3.480000in}{0.603499in}}%
\pgfpathlineto{\pgfqpoint{3.557500in}{0.605895in}}%
\pgfpathlineto{\pgfqpoint{3.635000in}{0.604031in}}%
\pgfpathlineto{\pgfqpoint{3.712500in}{0.607137in}}%
\pgfpathlineto{\pgfqpoint{3.790000in}{0.602079in}}%
\pgfpathlineto{\pgfqpoint{3.867500in}{0.600000in}}%
\pgfpathlineto{\pgfqpoint{3.945000in}{0.605451in}}%
\pgfpathlineto{\pgfqpoint{4.022500in}{0.606338in}}%
\pgfpathlineto{\pgfqpoint{4.100000in}{0.612283in}}%
\pgfpathlineto{\pgfqpoint{4.177500in}{0.603233in}}%
\pgfpathlineto{\pgfqpoint{4.255000in}{0.600000in}}%
\pgfpathlineto{\pgfqpoint{4.332500in}{0.601636in}}%
\pgfpathlineto{\pgfqpoint{4.410000in}{0.600000in}}%
\pgfpathlineto{\pgfqpoint{4.487500in}{0.604120in}}%
\pgfpathlineto{\pgfqpoint{4.565000in}{0.608379in}}%
\pgfpathlineto{\pgfqpoint{4.642500in}{0.604564in}}%
\pgfpathlineto{\pgfqpoint{4.720000in}{0.603943in}}%
\pgfpathlineto{\pgfqpoint{4.797500in}{0.603588in}}%
\pgfpathlineto{\pgfqpoint{4.875000in}{0.601103in}}%
\pgfpathlineto{\pgfqpoint{4.952500in}{0.606161in}}%
\pgfpathlineto{\pgfqpoint{5.030000in}{0.600216in}}%
\pgfpathlineto{\pgfqpoint{5.107500in}{0.608379in}}%
\pgfpathlineto{\pgfqpoint{5.185000in}{0.608379in}}%
\pgfpathlineto{\pgfqpoint{5.262500in}{0.602346in}}%
\pgfpathlineto{\pgfqpoint{5.340000in}{0.600482in}}%
\pgfpathlineto{\pgfqpoint{5.417500in}{0.602878in}}%
\pgfpathlineto{\pgfqpoint{5.495000in}{0.605717in}}%
\pgfpathlineto{\pgfqpoint{5.572500in}{0.609710in}}%
\pgfpathlineto{\pgfqpoint{5.650000in}{0.602612in}}%
\pgfpathlineto{\pgfqpoint{5.727500in}{0.600000in}}%
\pgfpathlineto{\pgfqpoint{5.805000in}{0.615033in}}%
\pgfpathlineto{\pgfqpoint{5.882500in}{0.601547in}}%
\pgfpathlineto{\pgfqpoint{5.960000in}{0.601281in}}%
\pgfpathlineto{\pgfqpoint{6.037500in}{0.609976in}}%
\pgfpathlineto{\pgfqpoint{6.115000in}{0.602612in}}%
\pgfpathlineto{\pgfqpoint{6.192500in}{0.607492in}}%
\pgfpathlineto{\pgfqpoint{6.270000in}{0.601991in}}%
\pgfpathlineto{\pgfqpoint{6.347500in}{0.600000in}}%
\pgfpathlineto{\pgfqpoint{6.425000in}{0.609887in}}%
\pgfpathlineto{\pgfqpoint{6.502500in}{0.600000in}}%
\pgfpathlineto{\pgfqpoint{6.580000in}{0.602701in}}%
\pgfpathlineto{\pgfqpoint{6.657500in}{0.605007in}}%
\pgfpathlineto{\pgfqpoint{6.735000in}{0.602612in}}%
\pgfpathlineto{\pgfqpoint{6.812500in}{0.600000in}}%
\pgfpathlineto{\pgfqpoint{6.890000in}{0.600000in}}%
\pgfpathlineto{\pgfqpoint{6.967500in}{0.600000in}}%
\pgfpathlineto{\pgfqpoint{7.045000in}{0.604031in}}%
\pgfpathlineto{\pgfqpoint{7.122500in}{0.600000in}}%
\pgfusepath{stroke}%
\end{pgfscope}%
\begin{pgfscope}%
\pgfpathrectangle{\pgfqpoint{1.000000in}{0.600000in}}{\pgfqpoint{6.200000in}{4.800000in}}%
\pgfusepath{clip}%
\pgfsetrectcap%
\pgfsetroundjoin%
\pgfsetlinewidth{2.007500pt}%
\definecolor{currentstroke}{rgb}{0.000000,0.500000,0.000000}%
\pgfsetstrokecolor{currentstroke}%
\pgfsetdash{}{0pt}%
\pgfpathmoveto{\pgfqpoint{1.000000in}{0.600000in}}%
\pgfpathlineto{\pgfqpoint{1.077500in}{0.621090in}}%
\pgfpathlineto{\pgfqpoint{1.155000in}{0.655241in}}%
\pgfpathlineto{\pgfqpoint{1.232500in}{0.834602in}}%
\pgfpathlineto{\pgfqpoint{1.310000in}{1.218423in}}%
\pgfpathlineto{\pgfqpoint{1.387500in}{2.229695in}}%
\pgfpathlineto{\pgfqpoint{1.465000in}{3.623981in}}%
\pgfpathlineto{\pgfqpoint{1.542500in}{4.568743in}}%
\pgfpathlineto{\pgfqpoint{1.620000in}{4.450065in}}%
\pgfpathlineto{\pgfqpoint{1.697500in}{3.835432in}}%
\pgfpathlineto{\pgfqpoint{1.775000in}{3.039017in}}%
\pgfpathlineto{\pgfqpoint{1.852500in}{2.383779in}}%
\pgfpathlineto{\pgfqpoint{1.930000in}{1.918390in}}%
\pgfpathlineto{\pgfqpoint{2.007500in}{1.698513in}}%
\pgfpathlineto{\pgfqpoint{2.085000in}{1.496653in}}%
\pgfpathlineto{\pgfqpoint{2.162500in}{1.345617in}}%
\pgfpathlineto{\pgfqpoint{2.240000in}{1.162222in}}%
\pgfpathlineto{\pgfqpoint{2.317500in}{1.062636in}}%
\pgfpathlineto{\pgfqpoint{2.395000in}{0.948978in}}%
\pgfpathlineto{\pgfqpoint{2.472500in}{0.924328in}}%
\pgfpathlineto{\pgfqpoint{2.550000in}{0.879958in}}%
\pgfpathlineto{\pgfqpoint{2.627500in}{0.893762in}}%
\pgfpathlineto{\pgfqpoint{2.705000in}{0.835051in}}%
\pgfpathlineto{\pgfqpoint{2.782500in}{0.818289in}}%
\pgfpathlineto{\pgfqpoint{2.860000in}{0.748283in}}%
\pgfpathlineto{\pgfqpoint{2.937500in}{0.751958in}}%
\pgfpathlineto{\pgfqpoint{3.015000in}{0.692440in}}%
\pgfpathlineto{\pgfqpoint{3.092500in}{0.687958in}}%
\pgfpathlineto{\pgfqpoint{3.170000in}{0.665729in}}%
\pgfpathlineto{\pgfqpoint{3.247500in}{0.696025in}}%
\pgfpathlineto{\pgfqpoint{3.325000in}{0.600000in}}%
\pgfpathlineto{\pgfqpoint{3.402500in}{0.603163in}}%
\pgfpathlineto{\pgfqpoint{3.480000in}{0.601460in}}%
\pgfpathlineto{\pgfqpoint{3.557500in}{0.600000in}}%
\pgfpathlineto{\pgfqpoint{3.635000in}{0.609437in}}%
\pgfpathlineto{\pgfqpoint{3.712500in}{0.600000in}}%
\pgfpathlineto{\pgfqpoint{3.790000in}{0.609258in}}%
\pgfpathlineto{\pgfqpoint{3.867500in}{0.605762in}}%
\pgfpathlineto{\pgfqpoint{3.945000in}{0.600000in}}%
\pgfpathlineto{\pgfqpoint{4.022500in}{0.603611in}}%
\pgfpathlineto{\pgfqpoint{4.100000in}{0.604238in}}%
\pgfpathlineto{\pgfqpoint{4.177500in}{0.601549in}}%
\pgfpathlineto{\pgfqpoint{4.255000in}{0.601818in}}%
\pgfpathlineto{\pgfqpoint{4.332500in}{0.608003in}}%
\pgfpathlineto{\pgfqpoint{4.410000in}{0.602983in}}%
\pgfpathlineto{\pgfqpoint{4.487500in}{0.602983in}}%
\pgfpathlineto{\pgfqpoint{4.565000in}{0.603880in}}%
\pgfpathlineto{\pgfqpoint{4.642500in}{0.601280in}}%
\pgfpathlineto{\pgfqpoint{4.720000in}{0.602804in}}%
\pgfpathlineto{\pgfqpoint{4.797500in}{0.600000in}}%
\pgfpathlineto{\pgfqpoint{4.875000in}{0.602715in}}%
\pgfpathlineto{\pgfqpoint{4.952500in}{0.600000in}}%
\pgfpathlineto{\pgfqpoint{5.030000in}{0.608451in}}%
\pgfpathlineto{\pgfqpoint{5.107500in}{0.613292in}}%
\pgfpathlineto{\pgfqpoint{5.185000in}{0.604059in}}%
\pgfpathlineto{\pgfqpoint{5.262500in}{0.600000in}}%
\pgfpathlineto{\pgfqpoint{5.340000in}{0.609706in}}%
\pgfpathlineto{\pgfqpoint{5.417500in}{0.607824in}}%
\pgfpathlineto{\pgfqpoint{5.495000in}{0.603252in}}%
\pgfpathlineto{\pgfqpoint{5.572500in}{0.600000in}}%
\pgfpathlineto{\pgfqpoint{5.650000in}{0.605045in}}%
\pgfpathlineto{\pgfqpoint{5.727500in}{0.600000in}}%
\pgfpathlineto{\pgfqpoint{5.805000in}{0.600000in}}%
\pgfpathlineto{\pgfqpoint{5.882500in}{0.605762in}}%
\pgfpathlineto{\pgfqpoint{5.960000in}{0.602356in}}%
\pgfpathlineto{\pgfqpoint{6.037500in}{0.601908in}}%
\pgfpathlineto{\pgfqpoint{6.115000in}{0.600000in}}%
\pgfpathlineto{\pgfqpoint{6.192500in}{0.606569in}}%
\pgfpathlineto{\pgfqpoint{6.270000in}{0.600000in}}%
\pgfpathlineto{\pgfqpoint{6.347500in}{0.600000in}}%
\pgfpathlineto{\pgfqpoint{6.425000in}{0.609348in}}%
\pgfpathlineto{\pgfqpoint{6.502500in}{0.600000in}}%
\pgfpathlineto{\pgfqpoint{6.580000in}{0.602715in}}%
\pgfpathlineto{\pgfqpoint{6.657500in}{0.605045in}}%
\pgfpathlineto{\pgfqpoint{6.735000in}{0.604776in}}%
\pgfpathlineto{\pgfqpoint{6.812500in}{0.605314in}}%
\pgfpathlineto{\pgfqpoint{6.890000in}{0.604866in}}%
\pgfpathlineto{\pgfqpoint{6.967500in}{0.607376in}}%
\pgfpathlineto{\pgfqpoint{7.045000in}{0.600000in}}%
\pgfpathlineto{\pgfqpoint{7.122500in}{0.602266in}}%
\pgfusepath{stroke}%
\end{pgfscope}%
\begin{pgfscope}%
\pgfpathrectangle{\pgfqpoint{1.000000in}{0.600000in}}{\pgfqpoint{6.200000in}{4.800000in}}%
\pgfusepath{clip}%
\pgfsetrectcap%
\pgfsetroundjoin%
\pgfsetlinewidth{2.007500pt}%
\definecolor{currentstroke}{rgb}{1.000000,0.000000,0.000000}%
\pgfsetstrokecolor{currentstroke}%
\pgfsetdash{}{0pt}%
\pgfpathmoveto{\pgfqpoint{1.000000in}{0.600000in}}%
\pgfpathlineto{\pgfqpoint{1.077500in}{0.635566in}}%
\pgfpathlineto{\pgfqpoint{1.155000in}{0.665210in}}%
\pgfpathlineto{\pgfqpoint{1.232500in}{0.863587in}}%
\pgfpathlineto{\pgfqpoint{1.310000in}{1.293347in}}%
\pgfpathlineto{\pgfqpoint{1.387500in}{2.410273in}}%
\pgfpathlineto{\pgfqpoint{1.465000in}{3.964198in}}%
\pgfpathlineto{\pgfqpoint{1.542500in}{5.009943in}}%
\pgfpathlineto{\pgfqpoint{1.620000in}{4.891451in}}%
\pgfpathlineto{\pgfqpoint{1.697500in}{4.207042in}}%
\pgfpathlineto{\pgfqpoint{1.775000in}{3.307451in}}%
\pgfpathlineto{\pgfqpoint{1.852500in}{2.583056in}}%
\pgfpathlineto{\pgfqpoint{1.930000in}{2.063674in}}%
\pgfpathlineto{\pgfqpoint{2.007500in}{1.829878in}}%
\pgfpathlineto{\pgfqpoint{2.085000in}{1.592205in}}%
\pgfpathlineto{\pgfqpoint{2.162500in}{1.419508in}}%
\pgfpathlineto{\pgfqpoint{2.240000in}{1.229835in}}%
\pgfpathlineto{\pgfqpoint{2.317500in}{1.113498in}}%
\pgfpathlineto{\pgfqpoint{2.395000in}{0.990783in}}%
\pgfpathlineto{\pgfqpoint{2.472500in}{0.968033in}}%
\pgfpathlineto{\pgfqpoint{2.550000in}{0.908830in}}%
\pgfpathlineto{\pgfqpoint{2.627500in}{0.914862in}}%
\pgfpathlineto{\pgfqpoint{2.705000in}{0.856866in}}%
\pgfpathlineto{\pgfqpoint{2.782500in}{0.838424in}}%
\pgfpathlineto{\pgfqpoint{2.860000in}{0.763365in}}%
\pgfpathlineto{\pgfqpoint{2.937500in}{0.754144in}}%
\pgfpathlineto{\pgfqpoint{3.015000in}{0.702869in}}%
\pgfpathlineto{\pgfqpoint{3.092500in}{0.707350in}}%
\pgfpathlineto{\pgfqpoint{3.170000in}{0.668399in}}%
\pgfpathlineto{\pgfqpoint{3.247500in}{0.715365in}}%
\pgfpathlineto{\pgfqpoint{3.325000in}{0.605491in}}%
\pgfpathlineto{\pgfqpoint{3.402500in}{0.604973in}}%
\pgfpathlineto{\pgfqpoint{3.480000in}{0.604543in}}%
\pgfpathlineto{\pgfqpoint{3.557500in}{0.600000in}}%
\pgfpathlineto{\pgfqpoint{3.635000in}{0.606266in}}%
\pgfpathlineto{\pgfqpoint{3.712500in}{0.602561in}}%
\pgfpathlineto{\pgfqpoint{3.790000in}{0.600406in}}%
\pgfpathlineto{\pgfqpoint{3.867500in}{0.602130in}}%
\pgfpathlineto{\pgfqpoint{3.945000in}{0.610575in}}%
\pgfpathlineto{\pgfqpoint{4.022500in}{0.603078in}}%
\pgfpathlineto{\pgfqpoint{4.100000in}{0.600000in}}%
\pgfpathlineto{\pgfqpoint{4.177500in}{0.601613in}}%
\pgfpathlineto{\pgfqpoint{4.255000in}{0.602819in}}%
\pgfpathlineto{\pgfqpoint{4.332500in}{0.602302in}}%
\pgfpathlineto{\pgfqpoint{4.410000in}{0.603250in}}%
\pgfpathlineto{\pgfqpoint{4.487500in}{0.603939in}}%
\pgfpathlineto{\pgfqpoint{4.565000in}{0.606008in}}%
\pgfpathlineto{\pgfqpoint{4.642500in}{0.604973in}}%
\pgfpathlineto{\pgfqpoint{4.720000in}{0.600000in}}%
\pgfpathlineto{\pgfqpoint{4.797500in}{0.607042in}}%
\pgfpathlineto{\pgfqpoint{4.875000in}{0.608507in}}%
\pgfpathlineto{\pgfqpoint{4.952500in}{0.608851in}}%
\pgfpathlineto{\pgfqpoint{5.030000in}{0.600000in}}%
\pgfpathlineto{\pgfqpoint{5.107500in}{0.602991in}}%
\pgfpathlineto{\pgfqpoint{5.185000in}{0.601526in}}%
\pgfpathlineto{\pgfqpoint{5.262500in}{0.600000in}}%
\pgfpathlineto{\pgfqpoint{5.340000in}{0.604112in}}%
\pgfpathlineto{\pgfqpoint{5.417500in}{0.604543in}}%
\pgfpathlineto{\pgfqpoint{5.495000in}{0.607042in}}%
\pgfpathlineto{\pgfqpoint{5.572500in}{0.600000in}}%
\pgfpathlineto{\pgfqpoint{5.650000in}{0.606094in}}%
\pgfpathlineto{\pgfqpoint{5.727500in}{0.600000in}}%
\pgfpathlineto{\pgfqpoint{5.805000in}{0.600000in}}%
\pgfpathlineto{\pgfqpoint{5.882500in}{0.602130in}}%
\pgfpathlineto{\pgfqpoint{5.960000in}{0.601785in}}%
\pgfpathlineto{\pgfqpoint{6.037500in}{0.603853in}}%
\pgfpathlineto{\pgfqpoint{6.115000in}{0.609799in}}%
\pgfpathlineto{\pgfqpoint{6.192500in}{0.614108in}}%
\pgfpathlineto{\pgfqpoint{6.270000in}{0.600000in}}%
\pgfpathlineto{\pgfqpoint{6.347500in}{0.603422in}}%
\pgfpathlineto{\pgfqpoint{6.425000in}{0.606611in}}%
\pgfpathlineto{\pgfqpoint{6.502500in}{0.600000in}}%
\pgfpathlineto{\pgfqpoint{6.580000in}{0.604284in}}%
\pgfpathlineto{\pgfqpoint{6.657500in}{0.601009in}}%
\pgfpathlineto{\pgfqpoint{6.735000in}{0.613333in}}%
\pgfpathlineto{\pgfqpoint{6.812500in}{0.600000in}}%
\pgfpathlineto{\pgfqpoint{6.890000in}{0.604026in}}%
\pgfpathlineto{\pgfqpoint{6.967500in}{0.605749in}}%
\pgfpathlineto{\pgfqpoint{7.045000in}{0.601354in}}%
\pgfpathlineto{\pgfqpoint{7.122500in}{0.601182in}}%
\pgfusepath{stroke}%
\end{pgfscope}%
\begin{pgfscope}%
\pgfpathrectangle{\pgfqpoint{1.000000in}{0.600000in}}{\pgfqpoint{6.200000in}{4.800000in}}%
\pgfusepath{clip}%
\pgfsetrectcap%
\pgfsetroundjoin%
\pgfsetlinewidth{2.007500pt}%
\definecolor{currentstroke}{rgb}{0.000000,0.750000,0.750000}%
\pgfsetstrokecolor{currentstroke}%
\pgfsetdash{}{0pt}%
\pgfpathmoveto{\pgfqpoint{1.000000in}{0.600000in}}%
\pgfpathlineto{\pgfqpoint{1.077500in}{0.628509in}}%
\pgfpathlineto{\pgfqpoint{1.155000in}{0.651922in}}%
\pgfpathlineto{\pgfqpoint{1.232500in}{0.844936in}}%
\pgfpathlineto{\pgfqpoint{1.310000in}{1.238890in}}%
\pgfpathlineto{\pgfqpoint{1.387500in}{2.284889in}}%
\pgfpathlineto{\pgfqpoint{1.465000in}{3.704381in}}%
\pgfpathlineto{\pgfqpoint{1.542500in}{4.671202in}}%
\pgfpathlineto{\pgfqpoint{1.620000in}{4.564094in}}%
\pgfpathlineto{\pgfqpoint{1.697500in}{3.924033in}}%
\pgfpathlineto{\pgfqpoint{1.775000in}{3.104139in}}%
\pgfpathlineto{\pgfqpoint{1.852500in}{2.423520in}}%
\pgfpathlineto{\pgfqpoint{1.930000in}{1.952507in}}%
\pgfpathlineto{\pgfqpoint{2.007500in}{1.727324in}}%
\pgfpathlineto{\pgfqpoint{2.085000in}{1.521037in}}%
\pgfpathlineto{\pgfqpoint{2.162500in}{1.358072in}}%
\pgfpathlineto{\pgfqpoint{2.240000in}{1.179806in}}%
\pgfpathlineto{\pgfqpoint{2.317500in}{1.073621in}}%
\pgfpathlineto{\pgfqpoint{2.395000in}{0.962735in}}%
\pgfpathlineto{\pgfqpoint{2.472500in}{0.925773in}}%
\pgfpathlineto{\pgfqpoint{2.550000in}{0.891945in}}%
\pgfpathlineto{\pgfqpoint{2.627500in}{0.899227in}}%
\pgfpathlineto{\pgfqpoint{2.705000in}{0.839774in}}%
\pgfpathlineto{\pgfqpoint{2.782500in}{0.816269in}}%
\pgfpathlineto{\pgfqpoint{2.860000in}{0.753683in}}%
\pgfpathlineto{\pgfqpoint{2.937500in}{0.746309in}}%
\pgfpathlineto{\pgfqpoint{3.015000in}{0.691004in}}%
\pgfpathlineto{\pgfqpoint{3.092500in}{0.688699in}}%
\pgfpathlineto{\pgfqpoint{3.170000in}{0.666854in}}%
\pgfpathlineto{\pgfqpoint{3.247500in}{0.700959in}}%
\pgfpathlineto{\pgfqpoint{3.325000in}{0.601871in}}%
\pgfpathlineto{\pgfqpoint{3.402500in}{0.601594in}}%
\pgfpathlineto{\pgfqpoint{3.480000in}{0.601041in}}%
\pgfpathlineto{\pgfqpoint{3.557500in}{0.604267in}}%
\pgfpathlineto{\pgfqpoint{3.635000in}{0.600000in}}%
\pgfpathlineto{\pgfqpoint{3.712500in}{0.605742in}}%
\pgfpathlineto{\pgfqpoint{3.790000in}{0.604359in}}%
\pgfpathlineto{\pgfqpoint{3.867500in}{0.608323in}}%
\pgfpathlineto{\pgfqpoint{3.945000in}{0.601594in}}%
\pgfpathlineto{\pgfqpoint{4.022500in}{0.600000in}}%
\pgfpathlineto{\pgfqpoint{4.100000in}{0.600000in}}%
\pgfpathlineto{\pgfqpoint{4.177500in}{0.605742in}}%
\pgfpathlineto{\pgfqpoint{4.255000in}{0.600000in}}%
\pgfpathlineto{\pgfqpoint{4.332500in}{0.606111in}}%
\pgfpathlineto{\pgfqpoint{4.410000in}{0.600000in}}%
\pgfpathlineto{\pgfqpoint{4.487500in}{0.603438in}}%
\pgfpathlineto{\pgfqpoint{4.565000in}{0.608600in}}%
\pgfpathlineto{\pgfqpoint{4.642500in}{0.605005in}}%
\pgfpathlineto{\pgfqpoint{4.720000in}{0.608231in}}%
\pgfpathlineto{\pgfqpoint{4.797500in}{0.602793in}}%
\pgfpathlineto{\pgfqpoint{4.875000in}{0.606572in}}%
\pgfpathlineto{\pgfqpoint{4.952500in}{0.600857in}}%
\pgfpathlineto{\pgfqpoint{5.030000in}{0.603346in}}%
\pgfpathlineto{\pgfqpoint{5.107500in}{0.600765in}}%
\pgfpathlineto{\pgfqpoint{5.185000in}{0.604820in}}%
\pgfpathlineto{\pgfqpoint{5.262500in}{0.601410in}}%
\pgfpathlineto{\pgfqpoint{5.340000in}{0.600488in}}%
\pgfpathlineto{\pgfqpoint{5.417500in}{0.603991in}}%
\pgfpathlineto{\pgfqpoint{5.495000in}{0.601871in}}%
\pgfpathlineto{\pgfqpoint{5.572500in}{0.608415in}}%
\pgfpathlineto{\pgfqpoint{5.650000in}{0.605005in}}%
\pgfpathlineto{\pgfqpoint{5.727500in}{0.604083in}}%
\pgfpathlineto{\pgfqpoint{5.805000in}{0.600000in}}%
\pgfpathlineto{\pgfqpoint{5.882500in}{0.615513in}}%
\pgfpathlineto{\pgfqpoint{5.960000in}{0.602239in}}%
\pgfpathlineto{\pgfqpoint{6.037500in}{0.600000in}}%
\pgfpathlineto{\pgfqpoint{6.115000in}{0.605373in}}%
\pgfpathlineto{\pgfqpoint{6.192500in}{0.604452in}}%
\pgfpathlineto{\pgfqpoint{6.270000in}{0.600673in}}%
\pgfpathlineto{\pgfqpoint{6.347500in}{0.606387in}}%
\pgfpathlineto{\pgfqpoint{6.425000in}{0.608323in}}%
\pgfpathlineto{\pgfqpoint{6.502500in}{0.600000in}}%
\pgfpathlineto{\pgfqpoint{6.580000in}{0.610074in}}%
\pgfpathlineto{\pgfqpoint{6.657500in}{0.604452in}}%
\pgfpathlineto{\pgfqpoint{6.735000in}{0.606572in}}%
\pgfpathlineto{\pgfqpoint{6.812500in}{0.600000in}}%
\pgfpathlineto{\pgfqpoint{6.890000in}{0.601133in}}%
\pgfpathlineto{\pgfqpoint{6.967500in}{0.606387in}}%
\pgfpathlineto{\pgfqpoint{7.045000in}{0.603438in}}%
\pgfpathlineto{\pgfqpoint{7.122500in}{0.608784in}}%
\pgfusepath{stroke}%
\end{pgfscope}%
\begin{pgfscope}%
\pgfpathrectangle{\pgfqpoint{1.000000in}{0.600000in}}{\pgfqpoint{6.200000in}{4.800000in}}%
\pgfusepath{clip}%
\pgfsetrectcap%
\pgfsetroundjoin%
\pgfsetlinewidth{2.007500pt}%
\definecolor{currentstroke}{rgb}{0.750000,0.000000,0.750000}%
\pgfsetstrokecolor{currentstroke}%
\pgfsetdash{}{0pt}%
\pgfpathmoveto{\pgfqpoint{1.000000in}{0.600000in}}%
\pgfpathlineto{\pgfqpoint{1.077500in}{0.631218in}}%
\pgfpathlineto{\pgfqpoint{1.155000in}{0.659264in}}%
\pgfpathlineto{\pgfqpoint{1.232500in}{0.856644in}}%
\pgfpathlineto{\pgfqpoint{1.310000in}{1.281126in}}%
\pgfpathlineto{\pgfqpoint{1.387500in}{2.387617in}}%
\pgfpathlineto{\pgfqpoint{1.465000in}{3.905714in}}%
\pgfpathlineto{\pgfqpoint{1.542500in}{4.936887in}}%
\pgfpathlineto{\pgfqpoint{1.620000in}{4.824968in}}%
\pgfpathlineto{\pgfqpoint{1.697500in}{4.139959in}}%
\pgfpathlineto{\pgfqpoint{1.775000in}{3.267184in}}%
\pgfpathlineto{\pgfqpoint{1.852500in}{2.539753in}}%
\pgfpathlineto{\pgfqpoint{1.930000in}{2.041982in}}%
\pgfpathlineto{\pgfqpoint{2.007500in}{1.806678in}}%
\pgfpathlineto{\pgfqpoint{2.085000in}{1.574285in}}%
\pgfpathlineto{\pgfqpoint{2.162500in}{1.407950in}}%
\pgfpathlineto{\pgfqpoint{2.240000in}{1.220095in}}%
\pgfpathlineto{\pgfqpoint{2.317500in}{1.115143in}}%
\pgfpathlineto{\pgfqpoint{2.395000in}{0.994228in}}%
\pgfpathlineto{\pgfqpoint{2.472500in}{0.962654in}}%
\pgfpathlineto{\pgfqpoint{2.550000in}{0.907092in}}%
\pgfpathlineto{\pgfqpoint{2.627500in}{0.912031in}}%
\pgfpathlineto{\pgfqpoint{2.705000in}{0.861054in}}%
\pgfpathlineto{\pgfqpoint{2.782500in}{0.831862in}}%
\pgfpathlineto{\pgfqpoint{2.860000in}{0.763952in}}%
\pgfpathlineto{\pgfqpoint{2.937500in}{0.759718in}}%
\pgfpathlineto{\pgfqpoint{3.015000in}{0.702656in}}%
\pgfpathlineto{\pgfqpoint{3.092500in}{0.707242in}}%
\pgfpathlineto{\pgfqpoint{3.170000in}{0.676639in}}%
\pgfpathlineto{\pgfqpoint{3.247500in}{0.715709in}}%
\pgfpathlineto{\pgfqpoint{3.325000in}{0.608200in}}%
\pgfpathlineto{\pgfqpoint{3.402500in}{0.607318in}}%
\pgfpathlineto{\pgfqpoint{3.480000in}{0.608905in}}%
\pgfpathlineto{\pgfqpoint{3.557500in}{0.605025in}}%
\pgfpathlineto{\pgfqpoint{3.635000in}{0.611198in}}%
\pgfpathlineto{\pgfqpoint{3.712500in}{0.610493in}}%
\pgfpathlineto{\pgfqpoint{3.790000in}{0.607847in}}%
\pgfpathlineto{\pgfqpoint{3.867500in}{0.600000in}}%
\pgfpathlineto{\pgfqpoint{3.945000in}{0.605995in}}%
\pgfpathlineto{\pgfqpoint{4.022500in}{0.602555in}}%
\pgfpathlineto{\pgfqpoint{4.100000in}{0.604407in}}%
\pgfpathlineto{\pgfqpoint{4.177500in}{0.602643in}}%
\pgfpathlineto{\pgfqpoint{4.255000in}{0.603702in}}%
\pgfpathlineto{\pgfqpoint{4.332500in}{0.606259in}}%
\pgfpathlineto{\pgfqpoint{4.410000in}{0.600000in}}%
\pgfpathlineto{\pgfqpoint{4.487500in}{0.602643in}}%
\pgfpathlineto{\pgfqpoint{4.565000in}{0.602820in}}%
\pgfpathlineto{\pgfqpoint{4.642500in}{0.601144in}}%
\pgfpathlineto{\pgfqpoint{4.720000in}{0.601320in}}%
\pgfpathlineto{\pgfqpoint{4.797500in}{0.600968in}}%
\pgfpathlineto{\pgfqpoint{4.875000in}{0.600703in}}%
\pgfpathlineto{\pgfqpoint{4.952500in}{0.602908in}}%
\pgfpathlineto{\pgfqpoint{5.030000in}{0.600000in}}%
\pgfpathlineto{\pgfqpoint{5.107500in}{0.600000in}}%
\pgfpathlineto{\pgfqpoint{5.185000in}{0.605025in}}%
\pgfpathlineto{\pgfqpoint{5.262500in}{0.605466in}}%
\pgfpathlineto{\pgfqpoint{5.340000in}{0.604231in}}%
\pgfpathlineto{\pgfqpoint{5.417500in}{0.602291in}}%
\pgfpathlineto{\pgfqpoint{5.495000in}{0.603966in}}%
\pgfpathlineto{\pgfqpoint{5.572500in}{0.600968in}}%
\pgfpathlineto{\pgfqpoint{5.650000in}{0.610493in}}%
\pgfpathlineto{\pgfqpoint{5.727500in}{0.601409in}}%
\pgfpathlineto{\pgfqpoint{5.805000in}{0.601938in}}%
\pgfpathlineto{\pgfqpoint{5.882500in}{0.603261in}}%
\pgfpathlineto{\pgfqpoint{5.960000in}{0.602820in}}%
\pgfpathlineto{\pgfqpoint{6.037500in}{0.600000in}}%
\pgfpathlineto{\pgfqpoint{6.115000in}{0.600000in}}%
\pgfpathlineto{\pgfqpoint{6.192500in}{0.602732in}}%
\pgfpathlineto{\pgfqpoint{6.270000in}{0.610052in}}%
\pgfpathlineto{\pgfqpoint{6.347500in}{0.606083in}}%
\pgfpathlineto{\pgfqpoint{6.425000in}{0.601938in}}%
\pgfpathlineto{\pgfqpoint{6.502500in}{0.602026in}}%
\pgfpathlineto{\pgfqpoint{6.580000in}{0.600000in}}%
\pgfpathlineto{\pgfqpoint{6.657500in}{0.600000in}}%
\pgfpathlineto{\pgfqpoint{6.735000in}{0.600000in}}%
\pgfpathlineto{\pgfqpoint{6.812500in}{0.600000in}}%
\pgfpathlineto{\pgfqpoint{6.890000in}{0.602291in}}%
\pgfpathlineto{\pgfqpoint{6.967500in}{0.608376in}}%
\pgfpathlineto{\pgfqpoint{7.045000in}{0.600000in}}%
\pgfpathlineto{\pgfqpoint{7.122500in}{0.610140in}}%
\pgfusepath{stroke}%
\end{pgfscope}%
\begin{pgfscope}%
\pgfpathrectangle{\pgfqpoint{1.000000in}{0.600000in}}{\pgfqpoint{6.200000in}{4.800000in}}%
\pgfusepath{clip}%
\pgfsetrectcap%
\pgfsetroundjoin%
\pgfsetlinewidth{2.007500pt}%
\definecolor{currentstroke}{rgb}{0.750000,0.750000,0.000000}%
\pgfsetstrokecolor{currentstroke}%
\pgfsetdash{}{0pt}%
\pgfpathmoveto{\pgfqpoint{1.000000in}{0.600000in}}%
\pgfpathlineto{\pgfqpoint{1.077500in}{0.618819in}}%
\pgfpathlineto{\pgfqpoint{1.155000in}{0.665331in}}%
\pgfpathlineto{\pgfqpoint{1.232500in}{0.849145in}}%
\pgfpathlineto{\pgfqpoint{1.310000in}{1.270261in}}%
\pgfpathlineto{\pgfqpoint{1.387500in}{2.338168in}}%
\pgfpathlineto{\pgfqpoint{1.465000in}{3.827191in}}%
\pgfpathlineto{\pgfqpoint{1.542500in}{4.827471in}}%
\pgfpathlineto{\pgfqpoint{1.620000in}{4.721889in}}%
\pgfpathlineto{\pgfqpoint{1.697500in}{4.051191in}}%
\pgfpathlineto{\pgfqpoint{1.775000in}{3.193889in}}%
\pgfpathlineto{\pgfqpoint{1.852500in}{2.491377in}}%
\pgfpathlineto{\pgfqpoint{1.930000in}{2.005424in}}%
\pgfpathlineto{\pgfqpoint{2.007500in}{1.776680in}}%
\pgfpathlineto{\pgfqpoint{2.085000in}{1.556215in}}%
\pgfpathlineto{\pgfqpoint{2.162500in}{1.388773in}}%
\pgfpathlineto{\pgfqpoint{2.240000in}{1.197145in}}%
\pgfpathlineto{\pgfqpoint{2.317500in}{1.092773in}}%
\pgfpathlineto{\pgfqpoint{2.395000in}{0.974075in}}%
\pgfpathlineto{\pgfqpoint{2.472500in}{0.962633in}}%
\pgfpathlineto{\pgfqpoint{2.550000in}{0.905238in}}%
\pgfpathlineto{\pgfqpoint{2.627500in}{0.904401in}}%
\pgfpathlineto{\pgfqpoint{2.705000in}{0.849982in}}%
\pgfpathlineto{\pgfqpoint{2.782500in}{0.829517in}}%
\pgfpathlineto{\pgfqpoint{2.860000in}{0.760401in}}%
\pgfpathlineto{\pgfqpoint{2.937500in}{0.750447in}}%
\pgfpathlineto{\pgfqpoint{3.015000in}{0.695284in}}%
\pgfpathlineto{\pgfqpoint{3.092500in}{0.694261in}}%
\pgfpathlineto{\pgfqpoint{3.170000in}{0.662633in}}%
\pgfpathlineto{\pgfqpoint{3.247500in}{0.706912in}}%
\pgfpathlineto{\pgfqpoint{3.325000in}{0.605052in}}%
\pgfpathlineto{\pgfqpoint{3.402500in}{0.604029in}}%
\pgfpathlineto{\pgfqpoint{3.480000in}{0.604308in}}%
\pgfpathlineto{\pgfqpoint{3.557500in}{0.601331in}}%
\pgfpathlineto{\pgfqpoint{3.635000in}{0.600000in}}%
\pgfpathlineto{\pgfqpoint{3.712500in}{0.613889in}}%
\pgfpathlineto{\pgfqpoint{3.790000in}{0.605796in}}%
\pgfpathlineto{\pgfqpoint{3.867500in}{0.609703in}}%
\pgfpathlineto{\pgfqpoint{3.945000in}{0.603843in}}%
\pgfpathlineto{\pgfqpoint{4.022500in}{0.600000in}}%
\pgfpathlineto{\pgfqpoint{4.100000in}{0.600000in}}%
\pgfpathlineto{\pgfqpoint{4.177500in}{0.606168in}}%
\pgfpathlineto{\pgfqpoint{4.255000in}{0.601052in}}%
\pgfpathlineto{\pgfqpoint{4.332500in}{0.600494in}}%
\pgfpathlineto{\pgfqpoint{4.410000in}{0.600000in}}%
\pgfpathlineto{\pgfqpoint{4.487500in}{0.601982in}}%
\pgfpathlineto{\pgfqpoint{4.565000in}{0.601331in}}%
\pgfpathlineto{\pgfqpoint{4.642500in}{0.601145in}}%
\pgfpathlineto{\pgfqpoint{4.720000in}{0.600000in}}%
\pgfpathlineto{\pgfqpoint{4.797500in}{0.600000in}}%
\pgfpathlineto{\pgfqpoint{4.875000in}{0.605331in}}%
\pgfpathlineto{\pgfqpoint{4.952500in}{0.606726in}}%
\pgfpathlineto{\pgfqpoint{5.030000in}{0.600959in}}%
\pgfpathlineto{\pgfqpoint{5.107500in}{0.606168in}}%
\pgfpathlineto{\pgfqpoint{5.185000in}{0.604680in}}%
\pgfpathlineto{\pgfqpoint{5.262500in}{0.600000in}}%
\pgfpathlineto{\pgfqpoint{5.340000in}{0.600000in}}%
\pgfpathlineto{\pgfqpoint{5.417500in}{0.610726in}}%
\pgfpathlineto{\pgfqpoint{5.495000in}{0.606075in}}%
\pgfpathlineto{\pgfqpoint{5.572500in}{0.601982in}}%
\pgfpathlineto{\pgfqpoint{5.650000in}{0.605517in}}%
\pgfpathlineto{\pgfqpoint{5.727500in}{0.600587in}}%
\pgfpathlineto{\pgfqpoint{5.805000in}{0.602912in}}%
\pgfpathlineto{\pgfqpoint{5.882500in}{0.613331in}}%
\pgfpathlineto{\pgfqpoint{5.960000in}{0.606168in}}%
\pgfpathlineto{\pgfqpoint{6.037500in}{0.610633in}}%
\pgfpathlineto{\pgfqpoint{6.115000in}{0.603471in}}%
\pgfpathlineto{\pgfqpoint{6.192500in}{0.600000in}}%
\pgfpathlineto{\pgfqpoint{6.270000in}{0.609889in}}%
\pgfpathlineto{\pgfqpoint{6.347500in}{0.610540in}}%
\pgfpathlineto{\pgfqpoint{6.425000in}{0.612866in}}%
\pgfpathlineto{\pgfqpoint{6.502500in}{0.601517in}}%
\pgfpathlineto{\pgfqpoint{6.580000in}{0.610726in}}%
\pgfpathlineto{\pgfqpoint{6.657500in}{0.602168in}}%
\pgfpathlineto{\pgfqpoint{6.735000in}{0.604959in}}%
\pgfpathlineto{\pgfqpoint{6.812500in}{0.608494in}}%
\pgfpathlineto{\pgfqpoint{6.890000in}{0.600000in}}%
\pgfpathlineto{\pgfqpoint{6.967500in}{0.600680in}}%
\pgfpathlineto{\pgfqpoint{7.045000in}{0.603284in}}%
\pgfpathlineto{\pgfqpoint{7.122500in}{0.603284in}}%
\pgfusepath{stroke}%
\end{pgfscope}%
\begin{pgfscope}%
\pgfpathrectangle{\pgfqpoint{1.000000in}{0.600000in}}{\pgfqpoint{6.200000in}{4.800000in}}%
\pgfusepath{clip}%
\pgfsetrectcap%
\pgfsetroundjoin%
\pgfsetlinewidth{2.007500pt}%
\definecolor{currentstroke}{rgb}{0.000000,0.000000,0.000000}%
\pgfsetstrokecolor{currentstroke}%
\pgfsetdash{}{0pt}%
\pgfpathmoveto{\pgfqpoint{1.000000in}{0.600000in}}%
\pgfpathlineto{\pgfqpoint{1.077500in}{0.623520in}}%
\pgfpathlineto{\pgfqpoint{1.155000in}{0.659749in}}%
\pgfpathlineto{\pgfqpoint{1.232500in}{0.863587in}}%
\pgfpathlineto{\pgfqpoint{1.310000in}{1.294040in}}%
\pgfpathlineto{\pgfqpoint{1.387500in}{2.421917in}}%
\pgfpathlineto{\pgfqpoint{1.465000in}{3.971458in}}%
\pgfpathlineto{\pgfqpoint{1.542500in}{5.024454in}}%
\pgfpathlineto{\pgfqpoint{1.620000in}{4.900788in}}%
\pgfpathlineto{\pgfqpoint{1.697500in}{4.207310in}}%
\pgfpathlineto{\pgfqpoint{1.775000in}{3.322460in}}%
\pgfpathlineto{\pgfqpoint{1.852500in}{2.573562in}}%
\pgfpathlineto{\pgfqpoint{1.930000in}{2.071904in}}%
\pgfpathlineto{\pgfqpoint{2.007500in}{1.830671in}}%
\pgfpathlineto{\pgfqpoint{2.085000in}{1.594281in}}%
\pgfpathlineto{\pgfqpoint{2.162500in}{1.423355in}}%
\pgfpathlineto{\pgfqpoint{2.240000in}{1.232073in}}%
\pgfpathlineto{\pgfqpoint{2.317500in}{1.122755in}}%
\pgfpathlineto{\pgfqpoint{2.395000in}{0.990391in}}%
\pgfpathlineto{\pgfqpoint{2.472500in}{0.955596in}}%
\pgfpathlineto{\pgfqpoint{2.550000in}{0.920712in}}%
\pgfpathlineto{\pgfqpoint{2.627500in}{0.922595in}}%
\pgfpathlineto{\pgfqpoint{2.705000in}{0.858206in}}%
\pgfpathlineto{\pgfqpoint{2.782500in}{0.844755in}}%
\pgfpathlineto{\pgfqpoint{2.860000in}{0.768259in}}%
\pgfpathlineto{\pgfqpoint{2.937500in}{0.759830in}}%
\pgfpathlineto{\pgfqpoint{3.015000in}{0.693109in}}%
\pgfpathlineto{\pgfqpoint{3.092500in}{0.704768in}}%
\pgfpathlineto{\pgfqpoint{3.170000in}{0.681810in}}%
\pgfpathlineto{\pgfqpoint{3.247500in}{0.709162in}}%
\pgfpathlineto{\pgfqpoint{3.325000in}{0.605763in}}%
\pgfpathlineto{\pgfqpoint{3.402500in}{0.600000in}}%
\pgfpathlineto{\pgfqpoint{3.480000in}{0.604598in}}%
\pgfpathlineto{\pgfqpoint{3.557500in}{0.614283in}}%
\pgfpathlineto{\pgfqpoint{3.635000in}{0.603521in}}%
\pgfpathlineto{\pgfqpoint{3.712500in}{0.608095in}}%
\pgfpathlineto{\pgfqpoint{3.790000in}{0.604149in}}%
\pgfpathlineto{\pgfqpoint{3.867500in}{0.605136in}}%
\pgfpathlineto{\pgfqpoint{3.945000in}{0.601728in}}%
\pgfpathlineto{\pgfqpoint{4.022500in}{0.611503in}}%
\pgfpathlineto{\pgfqpoint{4.100000in}{0.605584in}}%
\pgfpathlineto{\pgfqpoint{4.177500in}{0.602445in}}%
\pgfpathlineto{\pgfqpoint{4.255000in}{0.601907in}}%
\pgfpathlineto{\pgfqpoint{4.332500in}{0.609350in}}%
\pgfpathlineto{\pgfqpoint{4.410000in}{0.601010in}}%
\pgfpathlineto{\pgfqpoint{4.487500in}{0.606840in}}%
\pgfpathlineto{\pgfqpoint{4.565000in}{0.608005in}}%
\pgfpathlineto{\pgfqpoint{4.642500in}{0.607198in}}%
\pgfpathlineto{\pgfqpoint{4.720000in}{0.600000in}}%
\pgfpathlineto{\pgfqpoint{4.797500in}{0.600652in}}%
\pgfpathlineto{\pgfqpoint{4.875000in}{0.609081in}}%
\pgfpathlineto{\pgfqpoint{4.952500in}{0.600000in}}%
\pgfpathlineto{\pgfqpoint{5.030000in}{0.603701in}}%
\pgfpathlineto{\pgfqpoint{5.107500in}{0.601997in}}%
\pgfpathlineto{\pgfqpoint{5.185000in}{0.604687in}}%
\pgfpathlineto{\pgfqpoint{5.262500in}{0.607288in}}%
\pgfpathlineto{\pgfqpoint{5.340000in}{0.602445in}}%
\pgfpathlineto{\pgfqpoint{5.417500in}{0.600000in}}%
\pgfpathlineto{\pgfqpoint{5.495000in}{0.604598in}}%
\pgfpathlineto{\pgfqpoint{5.572500in}{0.602804in}}%
\pgfpathlineto{\pgfqpoint{5.650000in}{0.607198in}}%
\pgfpathlineto{\pgfqpoint{5.727500in}{0.600831in}}%
\pgfpathlineto{\pgfqpoint{5.805000in}{0.603432in}}%
\pgfpathlineto{\pgfqpoint{5.882500in}{0.600000in}}%
\pgfpathlineto{\pgfqpoint{5.960000in}{0.600000in}}%
\pgfpathlineto{\pgfqpoint{6.037500in}{0.600000in}}%
\pgfpathlineto{\pgfqpoint{6.115000in}{0.607557in}}%
\pgfpathlineto{\pgfqpoint{6.192500in}{0.602535in}}%
\pgfpathlineto{\pgfqpoint{6.270000in}{0.600000in}}%
\pgfpathlineto{\pgfqpoint{6.347500in}{0.602266in}}%
\pgfpathlineto{\pgfqpoint{6.425000in}{0.603342in}}%
\pgfpathlineto{\pgfqpoint{6.502500in}{0.600000in}}%
\pgfpathlineto{\pgfqpoint{6.580000in}{0.604777in}}%
\pgfpathlineto{\pgfqpoint{6.657500in}{0.602804in}}%
\pgfpathlineto{\pgfqpoint{6.735000in}{0.600000in}}%
\pgfpathlineto{\pgfqpoint{6.812500in}{0.600000in}}%
\pgfpathlineto{\pgfqpoint{6.890000in}{0.600000in}}%
\pgfpathlineto{\pgfqpoint{6.967500in}{0.600293in}}%
\pgfpathlineto{\pgfqpoint{7.045000in}{0.609709in}}%
\pgfpathlineto{\pgfqpoint{7.122500in}{0.602445in}}%
\pgfusepath{stroke}%
\end{pgfscope}%
\begin{pgfscope}%
\pgfpathrectangle{\pgfqpoint{1.000000in}{0.600000in}}{\pgfqpoint{6.200000in}{4.800000in}}%
\pgfusepath{clip}%
\pgfsetrectcap%
\pgfsetroundjoin%
\pgfsetlinewidth{2.007500pt}%
\definecolor{currentstroke}{rgb}{0.000000,0.000000,1.000000}%
\pgfsetstrokecolor{currentstroke}%
\pgfsetdash{}{0pt}%
\pgfpathmoveto{\pgfqpoint{1.000000in}{0.600000in}}%
\pgfpathlineto{\pgfqpoint{1.077500in}{0.629212in}}%
\pgfpathlineto{\pgfqpoint{1.155000in}{0.653634in}}%
\pgfpathlineto{\pgfqpoint{1.232500in}{0.845461in}}%
\pgfpathlineto{\pgfqpoint{1.310000in}{1.263409in}}%
\pgfpathlineto{\pgfqpoint{1.387500in}{2.344828in}}%
\pgfpathlineto{\pgfqpoint{1.465000in}{3.825661in}}%
\pgfpathlineto{\pgfqpoint{1.542500in}{4.826105in}}%
\pgfpathlineto{\pgfqpoint{1.620000in}{4.722960in}}%
\pgfpathlineto{\pgfqpoint{1.697500in}{4.055161in}}%
\pgfpathlineto{\pgfqpoint{1.775000in}{3.200124in}}%
\pgfpathlineto{\pgfqpoint{1.852500in}{2.498116in}}%
\pgfpathlineto{\pgfqpoint{1.930000in}{2.007507in}}%
\pgfpathlineto{\pgfqpoint{2.007500in}{1.771512in}}%
\pgfpathlineto{\pgfqpoint{2.085000in}{1.555696in}}%
\pgfpathlineto{\pgfqpoint{2.162500in}{1.390284in}}%
\pgfpathlineto{\pgfqpoint{2.240000in}{1.207117in}}%
\pgfpathlineto{\pgfqpoint{2.317500in}{1.102240in}}%
\pgfpathlineto{\pgfqpoint{2.395000in}{0.972768in}}%
\pgfpathlineto{\pgfqpoint{2.472500in}{0.944968in}}%
\pgfpathlineto{\pgfqpoint{2.550000in}{0.896211in}}%
\pgfpathlineto{\pgfqpoint{2.627500in}{0.902359in}}%
\pgfpathlineto{\pgfqpoint{2.705000in}{0.850657in}}%
\pgfpathlineto{\pgfqpoint{2.782500in}{0.826408in}}%
\pgfpathlineto{\pgfqpoint{2.860000in}{0.758078in}}%
\pgfpathlineto{\pgfqpoint{2.937500in}{0.750284in}}%
\pgfpathlineto{\pgfqpoint{3.015000in}{0.696156in}}%
\pgfpathlineto{\pgfqpoint{3.092500in}{0.698148in}}%
\pgfpathlineto{\pgfqpoint{3.170000in}{0.665585in}}%
\pgfpathlineto{\pgfqpoint{3.247500in}{0.707415in}}%
\pgfpathlineto{\pgfqpoint{3.325000in}{0.609553in}}%
\pgfpathlineto{\pgfqpoint{3.402500in}{0.606782in}}%
\pgfpathlineto{\pgfqpoint{3.480000in}{0.611891in}}%
\pgfpathlineto{\pgfqpoint{3.557500in}{0.610852in}}%
\pgfpathlineto{\pgfqpoint{3.635000in}{0.607994in}}%
\pgfpathlineto{\pgfqpoint{3.712500in}{0.605223in}}%
\pgfpathlineto{\pgfqpoint{3.790000in}{0.604270in}}%
\pgfpathlineto{\pgfqpoint{3.867500in}{0.604010in}}%
\pgfpathlineto{\pgfqpoint{3.945000in}{0.601672in}}%
\pgfpathlineto{\pgfqpoint{4.022500in}{0.600000in}}%
\pgfpathlineto{\pgfqpoint{4.100000in}{0.605742in}}%
\pgfpathlineto{\pgfqpoint{4.177500in}{0.606089in}}%
\pgfpathlineto{\pgfqpoint{4.255000in}{0.602538in}}%
\pgfpathlineto{\pgfqpoint{4.332500in}{0.601585in}}%
\pgfpathlineto{\pgfqpoint{4.410000in}{0.600000in}}%
\pgfpathlineto{\pgfqpoint{4.487500in}{0.600000in}}%
\pgfpathlineto{\pgfqpoint{4.565000in}{0.604703in}}%
\pgfpathlineto{\pgfqpoint{4.642500in}{0.603837in}}%
\pgfpathlineto{\pgfqpoint{4.720000in}{0.606522in}}%
\pgfpathlineto{\pgfqpoint{4.797500in}{0.600000in}}%
\pgfpathlineto{\pgfqpoint{4.875000in}{0.602711in}}%
\pgfpathlineto{\pgfqpoint{4.952500in}{0.600000in}}%
\pgfpathlineto{\pgfqpoint{5.030000in}{0.604184in}}%
\pgfpathlineto{\pgfqpoint{5.107500in}{0.603577in}}%
\pgfpathlineto{\pgfqpoint{5.185000in}{0.601585in}}%
\pgfpathlineto{\pgfqpoint{5.262500in}{0.607215in}}%
\pgfpathlineto{\pgfqpoint{5.340000in}{0.606349in}}%
\pgfpathlineto{\pgfqpoint{5.417500in}{0.600000in}}%
\pgfpathlineto{\pgfqpoint{5.495000in}{0.600000in}}%
\pgfpathlineto{\pgfqpoint{5.572500in}{0.610852in}}%
\pgfpathlineto{\pgfqpoint{5.650000in}{0.600000in}}%
\pgfpathlineto{\pgfqpoint{5.727500in}{0.603491in}}%
\pgfpathlineto{\pgfqpoint{5.805000in}{0.600000in}}%
\pgfpathlineto{\pgfqpoint{5.882500in}{0.603404in}}%
\pgfpathlineto{\pgfqpoint{5.960000in}{0.600000in}}%
\pgfpathlineto{\pgfqpoint{6.037500in}{0.607648in}}%
\pgfpathlineto{\pgfqpoint{6.115000in}{0.611718in}}%
\pgfpathlineto{\pgfqpoint{6.192500in}{0.602798in}}%
\pgfpathlineto{\pgfqpoint{6.270000in}{0.602884in}}%
\pgfpathlineto{\pgfqpoint{6.347500in}{0.601585in}}%
\pgfpathlineto{\pgfqpoint{6.425000in}{0.600000in}}%
\pgfpathlineto{\pgfqpoint{6.502500in}{0.602971in}}%
\pgfpathlineto{\pgfqpoint{6.580000in}{0.604010in}}%
\pgfpathlineto{\pgfqpoint{6.657500in}{0.600000in}}%
\pgfpathlineto{\pgfqpoint{6.735000in}{0.602365in}}%
\pgfpathlineto{\pgfqpoint{6.812500in}{0.600000in}}%
\pgfpathlineto{\pgfqpoint{6.890000in}{0.604530in}}%
\pgfpathlineto{\pgfqpoint{6.967500in}{0.606695in}}%
\pgfpathlineto{\pgfqpoint{7.045000in}{0.600000in}}%
\pgfpathlineto{\pgfqpoint{7.122500in}{0.610419in}}%
\pgfusepath{stroke}%
\end{pgfscope}%
\begin{pgfscope}%
\pgfpathrectangle{\pgfqpoint{1.000000in}{0.600000in}}{\pgfqpoint{6.200000in}{4.800000in}}%
\pgfusepath{clip}%
\pgfsetrectcap%
\pgfsetroundjoin%
\pgfsetlinewidth{2.007500pt}%
\definecolor{currentstroke}{rgb}{0.000000,0.500000,0.000000}%
\pgfsetstrokecolor{currentstroke}%
\pgfsetdash{}{0pt}%
\pgfpathmoveto{\pgfqpoint{1.000000in}{0.600000in}}%
\pgfpathlineto{\pgfqpoint{1.077500in}{0.620354in}}%
\pgfpathlineto{\pgfqpoint{1.155000in}{0.654817in}}%
\pgfpathlineto{\pgfqpoint{1.232500in}{0.853465in}}%
\pgfpathlineto{\pgfqpoint{1.310000in}{1.260558in}}%
\pgfpathlineto{\pgfqpoint{1.387500in}{2.336633in}}%
\pgfpathlineto{\pgfqpoint{1.465000in}{3.814027in}}%
\pgfpathlineto{\pgfqpoint{1.542500in}{4.817762in}}%
\pgfpathlineto{\pgfqpoint{1.620000in}{4.707286in}}%
\pgfpathlineto{\pgfqpoint{1.697500in}{4.046001in}}%
\pgfpathlineto{\pgfqpoint{1.775000in}{3.192629in}}%
\pgfpathlineto{\pgfqpoint{1.852500in}{2.492769in}}%
\pgfpathlineto{\pgfqpoint{1.930000in}{1.995843in}}%
\pgfpathlineto{\pgfqpoint{2.007500in}{1.772266in}}%
\pgfpathlineto{\pgfqpoint{2.085000in}{1.555162in}}%
\pgfpathlineto{\pgfqpoint{2.162500in}{1.388179in}}%
\pgfpathlineto{\pgfqpoint{2.240000in}{1.193030in}}%
\pgfpathlineto{\pgfqpoint{2.317500in}{1.096461in}}%
\pgfpathlineto{\pgfqpoint{2.395000in}{0.965429in}}%
\pgfpathlineto{\pgfqpoint{2.472500in}{0.951521in}}%
\pgfpathlineto{\pgfqpoint{2.550000in}{0.902449in}}%
\pgfpathlineto{\pgfqpoint{2.627500in}{0.901750in}}%
\pgfpathlineto{\pgfqpoint{2.705000in}{0.850404in}}%
\pgfpathlineto{\pgfqpoint{2.782500in}{0.826787in}}%
\pgfpathlineto{\pgfqpoint{2.860000in}{0.762145in}}%
\pgfpathlineto{\pgfqpoint{2.937500in}{0.744563in}}%
\pgfpathlineto{\pgfqpoint{3.015000in}{0.690156in}}%
\pgfpathlineto{\pgfqpoint{3.092500in}{0.693217in}}%
\pgfpathlineto{\pgfqpoint{3.170000in}{0.677035in}}%
\pgfpathlineto{\pgfqpoint{3.247500in}{0.699078in}}%
\pgfpathlineto{\pgfqpoint{3.325000in}{0.600000in}}%
\pgfpathlineto{\pgfqpoint{3.402500in}{0.609770in}}%
\pgfpathlineto{\pgfqpoint{3.480000in}{0.604171in}}%
\pgfpathlineto{\pgfqpoint{3.557500in}{0.604959in}}%
\pgfpathlineto{\pgfqpoint{3.635000in}{0.600498in}}%
\pgfpathlineto{\pgfqpoint{3.712500in}{0.602597in}}%
\pgfpathlineto{\pgfqpoint{3.790000in}{0.600000in}}%
\pgfpathlineto{\pgfqpoint{3.867500in}{0.607758in}}%
\pgfpathlineto{\pgfqpoint{3.945000in}{0.600000in}}%
\pgfpathlineto{\pgfqpoint{4.022500in}{0.600000in}}%
\pgfpathlineto{\pgfqpoint{4.100000in}{0.601810in}}%
\pgfpathlineto{\pgfqpoint{4.177500in}{0.602072in}}%
\pgfpathlineto{\pgfqpoint{4.255000in}{0.602772in}}%
\pgfpathlineto{\pgfqpoint{4.332500in}{0.611432in}}%
\pgfpathlineto{\pgfqpoint{4.410000in}{0.601897in}}%
\pgfpathlineto{\pgfqpoint{4.487500in}{0.605571in}}%
\pgfpathlineto{\pgfqpoint{4.565000in}{0.600000in}}%
\pgfpathlineto{\pgfqpoint{4.642500in}{0.606971in}}%
\pgfpathlineto{\pgfqpoint{4.720000in}{0.605921in}}%
\pgfpathlineto{\pgfqpoint{4.797500in}{0.605134in}}%
\pgfpathlineto{\pgfqpoint{4.875000in}{0.601635in}}%
\pgfpathlineto{\pgfqpoint{4.952500in}{0.604696in}}%
\pgfpathlineto{\pgfqpoint{5.030000in}{0.603734in}}%
\pgfpathlineto{\pgfqpoint{5.107500in}{0.600000in}}%
\pgfpathlineto{\pgfqpoint{5.185000in}{0.600000in}}%
\pgfpathlineto{\pgfqpoint{5.262500in}{0.600000in}}%
\pgfpathlineto{\pgfqpoint{5.340000in}{0.603472in}}%
\pgfpathlineto{\pgfqpoint{5.417500in}{0.600000in}}%
\pgfpathlineto{\pgfqpoint{5.495000in}{0.604959in}}%
\pgfpathlineto{\pgfqpoint{5.572500in}{0.602859in}}%
\pgfpathlineto{\pgfqpoint{5.650000in}{0.600000in}}%
\pgfpathlineto{\pgfqpoint{5.727500in}{0.601285in}}%
\pgfpathlineto{\pgfqpoint{5.805000in}{0.608895in}}%
\pgfpathlineto{\pgfqpoint{5.882500in}{0.600000in}}%
\pgfpathlineto{\pgfqpoint{5.960000in}{0.606358in}}%
\pgfpathlineto{\pgfqpoint{6.037500in}{0.600000in}}%
\pgfpathlineto{\pgfqpoint{6.115000in}{0.609507in}}%
\pgfpathlineto{\pgfqpoint{6.192500in}{0.603034in}}%
\pgfpathlineto{\pgfqpoint{6.270000in}{0.604871in}}%
\pgfpathlineto{\pgfqpoint{6.347500in}{0.603647in}}%
\pgfpathlineto{\pgfqpoint{6.425000in}{0.600000in}}%
\pgfpathlineto{\pgfqpoint{6.502500in}{0.603122in}}%
\pgfpathlineto{\pgfqpoint{6.580000in}{0.600000in}}%
\pgfpathlineto{\pgfqpoint{6.657500in}{0.601810in}}%
\pgfpathlineto{\pgfqpoint{6.735000in}{0.606183in}}%
\pgfpathlineto{\pgfqpoint{6.812500in}{0.603909in}}%
\pgfpathlineto{\pgfqpoint{6.890000in}{0.601460in}}%
\pgfpathlineto{\pgfqpoint{6.967500in}{0.600000in}}%
\pgfpathlineto{\pgfqpoint{7.045000in}{0.601285in}}%
\pgfpathlineto{\pgfqpoint{7.122500in}{0.608895in}}%
\pgfusepath{stroke}%
\end{pgfscope}%
\begin{pgfscope}%
\pgfpathrectangle{\pgfqpoint{1.000000in}{0.600000in}}{\pgfqpoint{6.200000in}{4.800000in}}%
\pgfusepath{clip}%
\pgfsetrectcap%
\pgfsetroundjoin%
\pgfsetlinewidth{2.007500pt}%
\definecolor{currentstroke}{rgb}{1.000000,0.000000,0.000000}%
\pgfsetstrokecolor{currentstroke}%
\pgfsetdash{}{0pt}%
\pgfpathmoveto{\pgfqpoint{1.000000in}{0.600000in}}%
\pgfpathlineto{\pgfqpoint{1.077500in}{0.624673in}}%
\pgfpathlineto{\pgfqpoint{1.155000in}{0.662569in}}%
\pgfpathlineto{\pgfqpoint{1.232500in}{0.847195in}}%
\pgfpathlineto{\pgfqpoint{1.310000in}{1.258730in}}%
\pgfpathlineto{\pgfqpoint{1.387500in}{2.339138in}}%
\pgfpathlineto{\pgfqpoint{1.465000in}{3.820908in}}%
\pgfpathlineto{\pgfqpoint{1.542500in}{4.818431in}}%
\pgfpathlineto{\pgfqpoint{1.620000in}{4.709876in}}%
\pgfpathlineto{\pgfqpoint{1.697500in}{4.044830in}}%
\pgfpathlineto{\pgfqpoint{1.775000in}{3.194504in}}%
\pgfpathlineto{\pgfqpoint{1.852500in}{2.492122in}}%
\pgfpathlineto{\pgfqpoint{1.930000in}{1.998448in}}%
\pgfpathlineto{\pgfqpoint{2.007500in}{1.776393in}}%
\pgfpathlineto{\pgfqpoint{2.085000in}{1.556671in}}%
\pgfpathlineto{\pgfqpoint{2.162500in}{1.382685in}}%
\pgfpathlineto{\pgfqpoint{2.240000in}{1.190219in}}%
\pgfpathlineto{\pgfqpoint{2.317500in}{1.104253in}}%
\pgfpathlineto{\pgfqpoint{2.395000in}{0.963590in}}%
\pgfpathlineto{\pgfqpoint{2.472500in}{0.948562in}}%
\pgfpathlineto{\pgfqpoint{2.550000in}{0.900865in}}%
\pgfpathlineto{\pgfqpoint{2.627500in}{0.897505in}}%
\pgfpathlineto{\pgfqpoint{2.705000in}{0.848035in}}%
\pgfpathlineto{\pgfqpoint{2.782500in}{0.826847in}}%
\pgfpathlineto{\pgfqpoint{2.860000in}{0.760762in}}%
\pgfpathlineto{\pgfqpoint{2.937500in}{0.747601in}}%
\pgfpathlineto{\pgfqpoint{3.015000in}{0.695798in}}%
\pgfpathlineto{\pgfqpoint{3.092500in}{0.699905in}}%
\pgfpathlineto{\pgfqpoint{3.170000in}{0.679930in}}%
\pgfpathlineto{\pgfqpoint{3.247500in}{0.715119in}}%
\pgfpathlineto{\pgfqpoint{3.325000in}{0.600000in}}%
\pgfpathlineto{\pgfqpoint{3.402500in}{0.604605in}}%
\pgfpathlineto{\pgfqpoint{3.480000in}{0.600000in}}%
\pgfpathlineto{\pgfqpoint{3.557500in}{0.605071in}}%
\pgfpathlineto{\pgfqpoint{3.635000in}{0.601058in}}%
\pgfpathlineto{\pgfqpoint{3.712500in}{0.604698in}}%
\pgfpathlineto{\pgfqpoint{3.790000in}{0.607498in}}%
\pgfpathlineto{\pgfqpoint{3.867500in}{0.602364in}}%
\pgfpathlineto{\pgfqpoint{3.945000in}{0.601524in}}%
\pgfpathlineto{\pgfqpoint{4.022500in}{0.601151in}}%
\pgfpathlineto{\pgfqpoint{4.100000in}{0.611232in}}%
\pgfpathlineto{\pgfqpoint{4.177500in}{0.601991in}}%
\pgfpathlineto{\pgfqpoint{4.255000in}{0.600000in}}%
\pgfpathlineto{\pgfqpoint{4.332500in}{0.610298in}}%
\pgfpathlineto{\pgfqpoint{4.410000in}{0.609738in}}%
\pgfpathlineto{\pgfqpoint{4.487500in}{0.603858in}}%
\pgfpathlineto{\pgfqpoint{4.565000in}{0.600000in}}%
\pgfpathlineto{\pgfqpoint{4.642500in}{0.600000in}}%
\pgfpathlineto{\pgfqpoint{4.720000in}{0.600000in}}%
\pgfpathlineto{\pgfqpoint{4.797500in}{0.604231in}}%
\pgfpathlineto{\pgfqpoint{4.875000in}{0.600591in}}%
\pgfpathlineto{\pgfqpoint{4.952500in}{0.608338in}}%
\pgfpathlineto{\pgfqpoint{5.030000in}{0.604138in}}%
\pgfpathlineto{\pgfqpoint{5.107500in}{0.604231in}}%
\pgfpathlineto{\pgfqpoint{5.185000in}{0.606845in}}%
\pgfpathlineto{\pgfqpoint{5.262500in}{0.608058in}}%
\pgfpathlineto{\pgfqpoint{5.340000in}{0.604045in}}%
\pgfpathlineto{\pgfqpoint{5.417500in}{0.609458in}}%
\pgfpathlineto{\pgfqpoint{5.495000in}{0.605818in}}%
\pgfpathlineto{\pgfqpoint{5.572500in}{0.605445in}}%
\pgfpathlineto{\pgfqpoint{5.650000in}{0.608712in}}%
\pgfpathlineto{\pgfqpoint{5.727500in}{0.600684in}}%
\pgfpathlineto{\pgfqpoint{5.805000in}{0.602271in}}%
\pgfpathlineto{\pgfqpoint{5.882500in}{0.600000in}}%
\pgfpathlineto{\pgfqpoint{5.960000in}{0.604791in}}%
\pgfpathlineto{\pgfqpoint{6.037500in}{0.600000in}}%
\pgfpathlineto{\pgfqpoint{6.115000in}{0.606098in}}%
\pgfpathlineto{\pgfqpoint{6.192500in}{0.600000in}}%
\pgfpathlineto{\pgfqpoint{6.270000in}{0.609178in}}%
\pgfpathlineto{\pgfqpoint{6.347500in}{0.603765in}}%
\pgfpathlineto{\pgfqpoint{6.425000in}{0.600000in}}%
\pgfpathlineto{\pgfqpoint{6.502500in}{0.600498in}}%
\pgfpathlineto{\pgfqpoint{6.580000in}{0.609085in}}%
\pgfpathlineto{\pgfqpoint{6.657500in}{0.607218in}}%
\pgfpathlineto{\pgfqpoint{6.735000in}{0.601898in}}%
\pgfpathlineto{\pgfqpoint{6.812500in}{0.605725in}}%
\pgfpathlineto{\pgfqpoint{6.890000in}{0.600000in}}%
\pgfpathlineto{\pgfqpoint{6.967500in}{0.600000in}}%
\pgfpathlineto{\pgfqpoint{7.045000in}{0.609365in}}%
\pgfpathlineto{\pgfqpoint{7.122500in}{0.605071in}}%
\pgfusepath{stroke}%
\end{pgfscope}%
\begin{pgfscope}%
\pgfpathrectangle{\pgfqpoint{1.000000in}{0.600000in}}{\pgfqpoint{6.200000in}{4.800000in}}%
\pgfusepath{clip}%
\pgfsetrectcap%
\pgfsetroundjoin%
\pgfsetlinewidth{2.007500pt}%
\definecolor{currentstroke}{rgb}{0.000000,0.750000,0.750000}%
\pgfsetstrokecolor{currentstroke}%
\pgfsetdash{}{0pt}%
\pgfpathmoveto{\pgfqpoint{1.000000in}{0.600000in}}%
\pgfpathlineto{\pgfqpoint{1.077500in}{0.632055in}}%
\pgfpathlineto{\pgfqpoint{1.155000in}{0.666677in}}%
\pgfpathlineto{\pgfqpoint{1.232500in}{0.834386in}}%
\pgfpathlineto{\pgfqpoint{1.310000in}{1.238396in}}%
\pgfpathlineto{\pgfqpoint{1.387500in}{2.269593in}}%
\pgfpathlineto{\pgfqpoint{1.465000in}{3.681067in}}%
\pgfpathlineto{\pgfqpoint{1.542500in}{4.643860in}}%
\pgfpathlineto{\pgfqpoint{1.620000in}{4.530130in}}%
\pgfpathlineto{\pgfqpoint{1.697500in}{3.905269in}}%
\pgfpathlineto{\pgfqpoint{1.775000in}{3.083569in}}%
\pgfpathlineto{\pgfqpoint{1.852500in}{2.405938in}}%
\pgfpathlineto{\pgfqpoint{1.930000in}{1.938921in}}%
\pgfpathlineto{\pgfqpoint{2.007500in}{1.724212in}}%
\pgfpathlineto{\pgfqpoint{2.085000in}{1.511272in}}%
\pgfpathlineto{\pgfqpoint{2.162500in}{1.346727in}}%
\pgfpathlineto{\pgfqpoint{2.240000in}{1.173341in}}%
\pgfpathlineto{\pgfqpoint{2.317500in}{1.073292in}}%
\pgfpathlineto{\pgfqpoint{2.395000in}{0.950628in}}%
\pgfpathlineto{\pgfqpoint{2.472500in}{0.935737in}}%
\pgfpathlineto{\pgfqpoint{2.550000in}{0.882595in}}%
\pgfpathlineto{\pgfqpoint{2.627500in}{0.888086in}}%
\pgfpathlineto{\pgfqpoint{2.705000in}{0.828895in}}%
\pgfpathlineto{\pgfqpoint{2.782500in}{0.825451in}}%
\pgfpathlineto{\pgfqpoint{2.860000in}{0.759187in}}%
\pgfpathlineto{\pgfqpoint{2.937500in}{0.744482in}}%
\pgfpathlineto{\pgfqpoint{3.015000in}{0.685942in}}%
\pgfpathlineto{\pgfqpoint{3.092500in}{0.695249in}}%
\pgfpathlineto{\pgfqpoint{3.170000in}{0.669748in}}%
\pgfpathlineto{\pgfqpoint{3.247500in}{0.696738in}}%
\pgfpathlineto{\pgfqpoint{3.325000in}{0.604507in}}%
\pgfpathlineto{\pgfqpoint{3.402500in}{0.600000in}}%
\pgfpathlineto{\pgfqpoint{3.480000in}{0.608602in}}%
\pgfpathlineto{\pgfqpoint{3.557500in}{0.600000in}}%
\pgfpathlineto{\pgfqpoint{3.635000in}{0.610277in}}%
\pgfpathlineto{\pgfqpoint{3.712500in}{0.605066in}}%
\pgfpathlineto{\pgfqpoint{3.790000in}{0.602273in}}%
\pgfpathlineto{\pgfqpoint{3.867500in}{0.609998in}}%
\pgfpathlineto{\pgfqpoint{3.945000in}{0.600000in}}%
\pgfpathlineto{\pgfqpoint{4.022500in}{0.602367in}}%
\pgfpathlineto{\pgfqpoint{4.100000in}{0.600000in}}%
\pgfpathlineto{\pgfqpoint{4.177500in}{0.603111in}}%
\pgfpathlineto{\pgfqpoint{4.255000in}{0.608602in}}%
\pgfpathlineto{\pgfqpoint{4.332500in}{0.605438in}}%
\pgfpathlineto{\pgfqpoint{4.410000in}{0.603483in}}%
\pgfpathlineto{\pgfqpoint{4.487500in}{0.600000in}}%
\pgfpathlineto{\pgfqpoint{4.565000in}{0.600000in}}%
\pgfpathlineto{\pgfqpoint{4.642500in}{0.603669in}}%
\pgfpathlineto{\pgfqpoint{4.720000in}{0.604972in}}%
\pgfpathlineto{\pgfqpoint{4.797500in}{0.605903in}}%
\pgfpathlineto{\pgfqpoint{4.875000in}{0.600000in}}%
\pgfpathlineto{\pgfqpoint{4.952500in}{0.600000in}}%
\pgfpathlineto{\pgfqpoint{5.030000in}{0.600226in}}%
\pgfpathlineto{\pgfqpoint{5.107500in}{0.600000in}}%
\pgfpathlineto{\pgfqpoint{5.185000in}{0.600000in}}%
\pgfpathlineto{\pgfqpoint{5.262500in}{0.600000in}}%
\pgfpathlineto{\pgfqpoint{5.340000in}{0.600000in}}%
\pgfpathlineto{\pgfqpoint{5.417500in}{0.600000in}}%
\pgfpathlineto{\pgfqpoint{5.495000in}{0.600000in}}%
\pgfpathlineto{\pgfqpoint{5.572500in}{0.600000in}}%
\pgfpathlineto{\pgfqpoint{5.650000in}{0.601622in}}%
\pgfpathlineto{\pgfqpoint{5.727500in}{0.605066in}}%
\pgfpathlineto{\pgfqpoint{5.805000in}{0.606089in}}%
\pgfpathlineto{\pgfqpoint{5.882500in}{0.600000in}}%
\pgfpathlineto{\pgfqpoint{5.960000in}{0.607020in}}%
\pgfpathlineto{\pgfqpoint{6.037500in}{0.605066in}}%
\pgfpathlineto{\pgfqpoint{6.115000in}{0.608695in}}%
\pgfpathlineto{\pgfqpoint{6.192500in}{0.600000in}}%
\pgfpathlineto{\pgfqpoint{6.270000in}{0.604042in}}%
\pgfpathlineto{\pgfqpoint{6.347500in}{0.600000in}}%
\pgfpathlineto{\pgfqpoint{6.425000in}{0.603483in}}%
\pgfpathlineto{\pgfqpoint{6.502500in}{0.601622in}}%
\pgfpathlineto{\pgfqpoint{6.580000in}{0.600000in}}%
\pgfpathlineto{\pgfqpoint{6.657500in}{0.600000in}}%
\pgfpathlineto{\pgfqpoint{6.735000in}{0.600000in}}%
\pgfpathlineto{\pgfqpoint{6.812500in}{0.608881in}}%
\pgfpathlineto{\pgfqpoint{6.890000in}{0.600000in}}%
\pgfpathlineto{\pgfqpoint{6.967500in}{0.600000in}}%
\pgfpathlineto{\pgfqpoint{7.045000in}{0.610370in}}%
\pgfpathlineto{\pgfqpoint{7.122500in}{0.604414in}}%
\pgfusepath{stroke}%
\end{pgfscope}%
\begin{pgfscope}%
\pgfpathrectangle{\pgfqpoint{1.000000in}{0.600000in}}{\pgfqpoint{6.200000in}{4.800000in}}%
\pgfusepath{clip}%
\pgfsetrectcap%
\pgfsetroundjoin%
\pgfsetlinewidth{2.007500pt}%
\definecolor{currentstroke}{rgb}{0.750000,0.000000,0.750000}%
\pgfsetstrokecolor{currentstroke}%
\pgfsetdash{}{0pt}%
\pgfpathmoveto{\pgfqpoint{1.000000in}{0.600000in}}%
\pgfpathlineto{\pgfqpoint{1.077500in}{0.620535in}}%
\pgfpathlineto{\pgfqpoint{1.155000in}{0.660120in}}%
\pgfpathlineto{\pgfqpoint{1.232500in}{0.835891in}}%
\pgfpathlineto{\pgfqpoint{1.310000in}{1.238094in}}%
\pgfpathlineto{\pgfqpoint{1.387500in}{2.284477in}}%
\pgfpathlineto{\pgfqpoint{1.465000in}{3.720871in}}%
\pgfpathlineto{\pgfqpoint{1.542500in}{4.683619in}}%
\pgfpathlineto{\pgfqpoint{1.620000in}{4.574309in}}%
\pgfpathlineto{\pgfqpoint{1.697500in}{3.937515in}}%
\pgfpathlineto{\pgfqpoint{1.775000in}{3.105629in}}%
\pgfpathlineto{\pgfqpoint{1.852500in}{2.431483in}}%
\pgfpathlineto{\pgfqpoint{1.930000in}{1.962388in}}%
\pgfpathlineto{\pgfqpoint{2.007500in}{1.738703in}}%
\pgfpathlineto{\pgfqpoint{2.085000in}{1.516821in}}%
\pgfpathlineto{\pgfqpoint{2.162500in}{1.355733in}}%
\pgfpathlineto{\pgfqpoint{2.240000in}{1.182967in}}%
\pgfpathlineto{\pgfqpoint{2.317500in}{1.085078in}}%
\pgfpathlineto{\pgfqpoint{2.395000in}{0.965293in}}%
\pgfpathlineto{\pgfqpoint{2.472500in}{0.937901in}}%
\pgfpathlineto{\pgfqpoint{2.550000in}{0.894195in}}%
\pgfpathlineto{\pgfqpoint{2.627500in}{0.900034in}}%
\pgfpathlineto{\pgfqpoint{2.705000in}{0.837264in}}%
\pgfpathlineto{\pgfqpoint{2.782500in}{0.823955in}}%
\pgfpathlineto{\pgfqpoint{2.860000in}{0.751397in}}%
\pgfpathlineto{\pgfqpoint{2.937500in}{0.742638in}}%
\pgfpathlineto{\pgfqpoint{3.015000in}{0.693436in}}%
\pgfpathlineto{\pgfqpoint{3.092500in}{0.700220in}}%
\pgfpathlineto{\pgfqpoint{3.170000in}{0.667933in}}%
\pgfpathlineto{\pgfqpoint{3.247500in}{0.701765in}}%
\pgfpathlineto{\pgfqpoint{3.325000in}{0.606195in}}%
\pgfpathlineto{\pgfqpoint{3.402500in}{0.607740in}}%
\pgfpathlineto{\pgfqpoint{3.480000in}{0.607654in}}%
\pgfpathlineto{\pgfqpoint{3.557500in}{0.608942in}}%
\pgfpathlineto{\pgfqpoint{3.635000in}{0.606109in}}%
\pgfpathlineto{\pgfqpoint{3.712500in}{0.601558in}}%
\pgfpathlineto{\pgfqpoint{3.790000in}{0.612463in}}%
\pgfpathlineto{\pgfqpoint{3.867500in}{0.603704in}}%
\pgfpathlineto{\pgfqpoint{3.945000in}{0.604992in}}%
\pgfpathlineto{\pgfqpoint{4.022500in}{0.612635in}}%
\pgfpathlineto{\pgfqpoint{4.100000in}{0.609286in}}%
\pgfpathlineto{\pgfqpoint{4.177500in}{0.603189in}}%
\pgfpathlineto{\pgfqpoint{4.255000in}{0.609114in}}%
\pgfpathlineto{\pgfqpoint{4.332500in}{0.612549in}}%
\pgfpathlineto{\pgfqpoint{4.410000in}{0.606624in}}%
\pgfpathlineto{\pgfqpoint{4.487500in}{0.603018in}}%
\pgfpathlineto{\pgfqpoint{4.565000in}{0.605250in}}%
\pgfpathlineto{\pgfqpoint{4.642500in}{0.601043in}}%
\pgfpathlineto{\pgfqpoint{4.720000in}{0.610660in}}%
\pgfpathlineto{\pgfqpoint{4.797500in}{0.603533in}}%
\pgfpathlineto{\pgfqpoint{4.875000in}{0.604821in}}%
\pgfpathlineto{\pgfqpoint{4.952500in}{0.604992in}}%
\pgfpathlineto{\pgfqpoint{5.030000in}{0.604306in}}%
\pgfpathlineto{\pgfqpoint{5.107500in}{0.603447in}}%
\pgfpathlineto{\pgfqpoint{5.185000in}{0.601901in}}%
\pgfpathlineto{\pgfqpoint{5.262500in}{0.608084in}}%
\pgfpathlineto{\pgfqpoint{5.340000in}{0.612635in}}%
\pgfpathlineto{\pgfqpoint{5.417500in}{0.605078in}}%
\pgfpathlineto{\pgfqpoint{5.495000in}{0.607654in}}%
\pgfpathlineto{\pgfqpoint{5.572500in}{0.606281in}}%
\pgfpathlineto{\pgfqpoint{5.650000in}{0.600442in}}%
\pgfpathlineto{\pgfqpoint{5.727500in}{0.606109in}}%
\pgfpathlineto{\pgfqpoint{5.805000in}{0.602674in}}%
\pgfpathlineto{\pgfqpoint{5.882500in}{0.601386in}}%
\pgfpathlineto{\pgfqpoint{5.960000in}{0.609114in}}%
\pgfpathlineto{\pgfqpoint{6.037500in}{0.608084in}}%
\pgfpathlineto{\pgfqpoint{6.115000in}{0.606967in}}%
\pgfpathlineto{\pgfqpoint{6.192500in}{0.606882in}}%
\pgfpathlineto{\pgfqpoint{6.270000in}{0.606967in}}%
\pgfpathlineto{\pgfqpoint{6.347500in}{0.600000in}}%
\pgfpathlineto{\pgfqpoint{6.425000in}{0.600000in}}%
\pgfpathlineto{\pgfqpoint{6.502500in}{0.600000in}}%
\pgfpathlineto{\pgfqpoint{6.580000in}{0.609286in}}%
\pgfpathlineto{\pgfqpoint{6.657500in}{0.607311in}}%
\pgfpathlineto{\pgfqpoint{6.735000in}{0.600000in}}%
\pgfpathlineto{\pgfqpoint{6.812500in}{0.604907in}}%
\pgfpathlineto{\pgfqpoint{6.890000in}{0.604563in}}%
\pgfpathlineto{\pgfqpoint{6.967500in}{0.600000in}}%
\pgfpathlineto{\pgfqpoint{7.045000in}{0.601558in}}%
\pgfpathlineto{\pgfqpoint{7.122500in}{0.600000in}}%
\pgfusepath{stroke}%
\end{pgfscope}%
\begin{pgfscope}%
\pgfpathrectangle{\pgfqpoint{1.000000in}{0.600000in}}{\pgfqpoint{6.200000in}{4.800000in}}%
\pgfusepath{clip}%
\pgfsetrectcap%
\pgfsetroundjoin%
\pgfsetlinewidth{2.007500pt}%
\definecolor{currentstroke}{rgb}{0.750000,0.750000,0.000000}%
\pgfsetstrokecolor{currentstroke}%
\pgfsetdash{}{0pt}%
\pgfpathmoveto{\pgfqpoint{1.000000in}{0.600000in}}%
\pgfpathlineto{\pgfqpoint{1.077500in}{0.623470in}}%
\pgfpathlineto{\pgfqpoint{1.155000in}{0.660329in}}%
\pgfpathlineto{\pgfqpoint{1.232500in}{0.863149in}}%
\pgfpathlineto{\pgfqpoint{1.310000in}{1.290335in}}%
\pgfpathlineto{\pgfqpoint{1.387500in}{2.423722in}}%
\pgfpathlineto{\pgfqpoint{1.465000in}{3.954771in}}%
\pgfpathlineto{\pgfqpoint{1.542500in}{5.001968in}}%
\pgfpathlineto{\pgfqpoint{1.620000in}{4.879653in}}%
\pgfpathlineto{\pgfqpoint{1.697500in}{4.197292in}}%
\pgfpathlineto{\pgfqpoint{1.775000in}{3.309728in}}%
\pgfpathlineto{\pgfqpoint{1.852500in}{2.574186in}}%
\pgfpathlineto{\pgfqpoint{1.930000in}{2.065303in}}%
\pgfpathlineto{\pgfqpoint{2.007500in}{1.826358in}}%
\pgfpathlineto{\pgfqpoint{2.085000in}{1.588146in}}%
\pgfpathlineto{\pgfqpoint{2.162500in}{1.426771in}}%
\pgfpathlineto{\pgfqpoint{2.240000in}{1.228260in}}%
\pgfpathlineto{\pgfqpoint{2.317500in}{1.117590in}}%
\pgfpathlineto{\pgfqpoint{2.395000in}{0.985097in}}%
\pgfpathlineto{\pgfqpoint{2.472500in}{0.962541in}}%
\pgfpathlineto{\pgfqpoint{2.550000in}{0.913945in}}%
\pgfpathlineto{\pgfqpoint{2.627500in}{0.915137in}}%
\pgfpathlineto{\pgfqpoint{2.705000in}{0.858748in}}%
\pgfpathlineto{\pgfqpoint{2.782500in}{0.833991in}}%
\pgfpathlineto{\pgfqpoint{2.860000in}{0.763298in}}%
\pgfpathlineto{\pgfqpoint{2.937500in}{0.760455in}}%
\pgfpathlineto{\pgfqpoint{3.015000in}{0.698748in}}%
\pgfpathlineto{\pgfqpoint{3.092500in}{0.703607in}}%
\pgfpathlineto{\pgfqpoint{3.170000in}{0.669498in}}%
\pgfpathlineto{\pgfqpoint{3.247500in}{0.714610in}}%
\pgfpathlineto{\pgfqpoint{3.325000in}{0.610541in}}%
\pgfpathlineto{\pgfqpoint{3.402500in}{0.604490in}}%
\pgfpathlineto{\pgfqpoint{3.480000in}{0.601647in}}%
\pgfpathlineto{\pgfqpoint{3.557500in}{0.603665in}}%
\pgfpathlineto{\pgfqpoint{3.635000in}{0.600000in}}%
\pgfpathlineto{\pgfqpoint{3.712500in}{0.603114in}}%
\pgfpathlineto{\pgfqpoint{3.790000in}{0.606140in}}%
\pgfpathlineto{\pgfqpoint{3.867500in}{0.609716in}}%
\pgfpathlineto{\pgfqpoint{3.945000in}{0.600000in}}%
\pgfpathlineto{\pgfqpoint{4.022500in}{0.600000in}}%
\pgfpathlineto{\pgfqpoint{4.100000in}{0.611825in}}%
\pgfpathlineto{\pgfqpoint{4.177500in}{0.600000in}}%
\pgfpathlineto{\pgfqpoint{4.255000in}{0.601922in}}%
\pgfpathlineto{\pgfqpoint{4.332500in}{0.607791in}}%
\pgfpathlineto{\pgfqpoint{4.410000in}{0.600000in}}%
\pgfpathlineto{\pgfqpoint{4.487500in}{0.600000in}}%
\pgfpathlineto{\pgfqpoint{4.565000in}{0.604948in}}%
\pgfpathlineto{\pgfqpoint{4.642500in}{0.600000in}}%
\pgfpathlineto{\pgfqpoint{4.720000in}{0.606415in}}%
\pgfpathlineto{\pgfqpoint{4.797500in}{0.607791in}}%
\pgfpathlineto{\pgfqpoint{4.875000in}{0.604123in}}%
\pgfpathlineto{\pgfqpoint{4.952500in}{0.603573in}}%
\pgfpathlineto{\pgfqpoint{5.030000in}{0.603389in}}%
\pgfpathlineto{\pgfqpoint{5.107500in}{0.600000in}}%
\pgfpathlineto{\pgfqpoint{5.185000in}{0.612100in}}%
\pgfpathlineto{\pgfqpoint{5.262500in}{0.600000in}}%
\pgfpathlineto{\pgfqpoint{5.340000in}{0.600000in}}%
\pgfpathlineto{\pgfqpoint{5.417500in}{0.601005in}}%
\pgfpathlineto{\pgfqpoint{5.495000in}{0.603940in}}%
\pgfpathlineto{\pgfqpoint{5.572500in}{0.601922in}}%
\pgfpathlineto{\pgfqpoint{5.650000in}{0.606048in}}%
\pgfpathlineto{\pgfqpoint{5.727500in}{0.613475in}}%
\pgfpathlineto{\pgfqpoint{5.805000in}{0.605407in}}%
\pgfpathlineto{\pgfqpoint{5.882500in}{0.602748in}}%
\pgfpathlineto{\pgfqpoint{5.960000in}{0.609808in}}%
\pgfpathlineto{\pgfqpoint{6.037500in}{0.609991in}}%
\pgfpathlineto{\pgfqpoint{6.115000in}{0.600000in}}%
\pgfpathlineto{\pgfqpoint{6.192500in}{0.603940in}}%
\pgfpathlineto{\pgfqpoint{6.270000in}{0.605223in}}%
\pgfpathlineto{\pgfqpoint{6.347500in}{0.604306in}}%
\pgfpathlineto{\pgfqpoint{6.425000in}{0.600000in}}%
\pgfpathlineto{\pgfqpoint{6.502500in}{0.605040in}}%
\pgfpathlineto{\pgfqpoint{6.580000in}{0.603756in}}%
\pgfpathlineto{\pgfqpoint{6.657500in}{0.606415in}}%
\pgfpathlineto{\pgfqpoint{6.735000in}{0.606874in}}%
\pgfpathlineto{\pgfqpoint{6.812500in}{0.601739in}}%
\pgfpathlineto{\pgfqpoint{6.890000in}{0.600547in}}%
\pgfpathlineto{\pgfqpoint{6.967500in}{0.600730in}}%
\pgfpathlineto{\pgfqpoint{7.045000in}{0.606782in}}%
\pgfpathlineto{\pgfqpoint{7.122500in}{0.600000in}}%
\pgfusepath{stroke}%
\end{pgfscope}%
\begin{pgfscope}%
\pgfpathrectangle{\pgfqpoint{1.000000in}{0.600000in}}{\pgfqpoint{6.200000in}{4.800000in}}%
\pgfusepath{clip}%
\pgfsetrectcap%
\pgfsetroundjoin%
\pgfsetlinewidth{2.007500pt}%
\definecolor{currentstroke}{rgb}{0.000000,0.000000,0.000000}%
\pgfsetstrokecolor{currentstroke}%
\pgfsetdash{}{0pt}%
\pgfpathmoveto{\pgfqpoint{1.000000in}{0.600000in}}%
\pgfpathlineto{\pgfqpoint{1.077500in}{0.630050in}}%
\pgfpathlineto{\pgfqpoint{1.155000in}{0.659411in}}%
\pgfpathlineto{\pgfqpoint{1.232500in}{0.842479in}}%
\pgfpathlineto{\pgfqpoint{1.310000in}{1.245228in}}%
\pgfpathlineto{\pgfqpoint{1.387500in}{2.280489in}}%
\pgfpathlineto{\pgfqpoint{1.465000in}{3.720710in}}%
\pgfpathlineto{\pgfqpoint{1.542500in}{4.703793in}}%
\pgfpathlineto{\pgfqpoint{1.620000in}{4.599612in}}%
\pgfpathlineto{\pgfqpoint{1.697500in}{3.944460in}}%
\pgfpathlineto{\pgfqpoint{1.775000in}{3.119152in}}%
\pgfpathlineto{\pgfqpoint{1.852500in}{2.434018in}}%
\pgfpathlineto{\pgfqpoint{1.930000in}{1.959369in}}%
\pgfpathlineto{\pgfqpoint{2.007500in}{1.731021in}}%
\pgfpathlineto{\pgfqpoint{2.085000in}{1.526020in}}%
\pgfpathlineto{\pgfqpoint{2.162500in}{1.373022in}}%
\pgfpathlineto{\pgfqpoint{2.240000in}{1.188274in}}%
\pgfpathlineto{\pgfqpoint{2.317500in}{1.085773in}}%
\pgfpathlineto{\pgfqpoint{2.395000in}{0.964701in}}%
\pgfpathlineto{\pgfqpoint{2.472500in}{0.940469in}}%
\pgfpathlineto{\pgfqpoint{2.550000in}{0.900141in}}%
\pgfpathlineto{\pgfqpoint{2.627500in}{0.896603in}}%
\pgfpathlineto{\pgfqpoint{2.705000in}{0.845663in}}%
\pgfpathlineto{\pgfqpoint{2.782500in}{0.820016in}}%
\pgfpathlineto{\pgfqpoint{2.860000in}{0.756163in}}%
\pgfpathlineto{\pgfqpoint{2.937500in}{0.750503in}}%
\pgfpathlineto{\pgfqpoint{3.015000in}{0.690542in}}%
\pgfpathlineto{\pgfqpoint{3.092500in}{0.696290in}}%
\pgfpathlineto{\pgfqpoint{3.170000in}{0.668167in}}%
\pgfpathlineto{\pgfqpoint{3.247500in}{0.713005in}}%
\pgfpathlineto{\pgfqpoint{3.325000in}{0.603341in}}%
\pgfpathlineto{\pgfqpoint{3.402500in}{0.606171in}}%
\pgfpathlineto{\pgfqpoint{3.480000in}{0.602722in}}%
\pgfpathlineto{\pgfqpoint{3.557500in}{0.604137in}}%
\pgfpathlineto{\pgfqpoint{3.635000in}{0.605375in}}%
\pgfpathlineto{\pgfqpoint{3.712500in}{0.601749in}}%
\pgfpathlineto{\pgfqpoint{3.790000in}{0.602545in}}%
\pgfpathlineto{\pgfqpoint{3.867500in}{0.602545in}}%
\pgfpathlineto{\pgfqpoint{3.945000in}{0.600000in}}%
\pgfpathlineto{\pgfqpoint{4.022500in}{0.615457in}}%
\pgfpathlineto{\pgfqpoint{4.100000in}{0.603518in}}%
\pgfpathlineto{\pgfqpoint{4.177500in}{0.610947in}}%
\pgfpathlineto{\pgfqpoint{4.255000in}{0.603695in}}%
\pgfpathlineto{\pgfqpoint{4.332500in}{0.600000in}}%
\pgfpathlineto{\pgfqpoint{4.410000in}{0.603430in}}%
\pgfpathlineto{\pgfqpoint{4.487500in}{0.606348in}}%
\pgfpathlineto{\pgfqpoint{4.565000in}{0.601130in}}%
\pgfpathlineto{\pgfqpoint{4.642500in}{0.604137in}}%
\pgfpathlineto{\pgfqpoint{4.720000in}{0.600000in}}%
\pgfpathlineto{\pgfqpoint{4.797500in}{0.606614in}}%
\pgfpathlineto{\pgfqpoint{4.875000in}{0.610859in}}%
\pgfpathlineto{\pgfqpoint{4.952500in}{0.607144in}}%
\pgfpathlineto{\pgfqpoint{5.030000in}{0.600000in}}%
\pgfpathlineto{\pgfqpoint{5.107500in}{0.606879in}}%
\pgfpathlineto{\pgfqpoint{5.185000in}{0.605641in}}%
\pgfpathlineto{\pgfqpoint{5.262500in}{0.604579in}}%
\pgfpathlineto{\pgfqpoint{5.340000in}{0.600423in}}%
\pgfpathlineto{\pgfqpoint{5.417500in}{0.604403in}}%
\pgfpathlineto{\pgfqpoint{5.495000in}{0.600000in}}%
\pgfpathlineto{\pgfqpoint{5.572500in}{0.600000in}}%
\pgfpathlineto{\pgfqpoint{5.650000in}{0.606083in}}%
\pgfpathlineto{\pgfqpoint{5.727500in}{0.602457in}}%
\pgfpathlineto{\pgfqpoint{5.805000in}{0.604137in}}%
\pgfpathlineto{\pgfqpoint{5.882500in}{0.606083in}}%
\pgfpathlineto{\pgfqpoint{5.960000in}{0.600000in}}%
\pgfpathlineto{\pgfqpoint{6.037500in}{0.600000in}}%
\pgfpathlineto{\pgfqpoint{6.115000in}{0.606702in}}%
\pgfpathlineto{\pgfqpoint{6.192500in}{0.600000in}}%
\pgfpathlineto{\pgfqpoint{6.270000in}{0.603872in}}%
\pgfpathlineto{\pgfqpoint{6.347500in}{0.610416in}}%
\pgfpathlineto{\pgfqpoint{6.425000in}{0.603960in}}%
\pgfpathlineto{\pgfqpoint{6.502500in}{0.604226in}}%
\pgfpathlineto{\pgfqpoint{6.580000in}{0.606437in}}%
\pgfpathlineto{\pgfqpoint{6.657500in}{0.605729in}}%
\pgfpathlineto{\pgfqpoint{6.735000in}{0.600000in}}%
\pgfpathlineto{\pgfqpoint{6.812500in}{0.600000in}}%
\pgfpathlineto{\pgfqpoint{6.890000in}{0.600000in}}%
\pgfpathlineto{\pgfqpoint{6.967500in}{0.604845in}}%
\pgfpathlineto{\pgfqpoint{7.045000in}{0.603430in}}%
\pgfpathlineto{\pgfqpoint{7.122500in}{0.600000in}}%
\pgfusepath{stroke}%
\end{pgfscope}%
\begin{pgfscope}%
\pgfpathrectangle{\pgfqpoint{1.000000in}{0.600000in}}{\pgfqpoint{6.200000in}{4.800000in}}%
\pgfusepath{clip}%
\pgfsetrectcap%
\pgfsetroundjoin%
\pgfsetlinewidth{2.007500pt}%
\definecolor{currentstroke}{rgb}{0.000000,0.000000,1.000000}%
\pgfsetstrokecolor{currentstroke}%
\pgfsetdash{}{0pt}%
\pgfpathmoveto{\pgfqpoint{1.000000in}{0.600000in}}%
\pgfpathlineto{\pgfqpoint{1.077500in}{0.625862in}}%
\pgfpathlineto{\pgfqpoint{1.155000in}{0.665699in}}%
\pgfpathlineto{\pgfqpoint{1.232500in}{0.852606in}}%
\pgfpathlineto{\pgfqpoint{1.310000in}{1.272259in}}%
\pgfpathlineto{\pgfqpoint{1.387500in}{2.348132in}}%
\pgfpathlineto{\pgfqpoint{1.465000in}{3.836473in}}%
\pgfpathlineto{\pgfqpoint{1.542500in}{4.855865in}}%
\pgfpathlineto{\pgfqpoint{1.620000in}{4.726349in}}%
\pgfpathlineto{\pgfqpoint{1.697500in}{4.067400in}}%
\pgfpathlineto{\pgfqpoint{1.775000in}{3.203082in}}%
\pgfpathlineto{\pgfqpoint{1.852500in}{2.500022in}}%
\pgfpathlineto{\pgfqpoint{1.930000in}{2.018430in}}%
\pgfpathlineto{\pgfqpoint{2.007500in}{1.786957in}}%
\pgfpathlineto{\pgfqpoint{2.085000in}{1.551300in}}%
\pgfpathlineto{\pgfqpoint{2.162500in}{1.392043in}}%
\pgfpathlineto{\pgfqpoint{2.240000in}{1.210866in}}%
\pgfpathlineto{\pgfqpoint{2.317500in}{1.090991in}}%
\pgfpathlineto{\pgfqpoint{2.395000in}{0.977665in}}%
\pgfpathlineto{\pgfqpoint{2.472500in}{0.951471in}}%
\pgfpathlineto{\pgfqpoint{2.550000in}{0.907814in}}%
\pgfpathlineto{\pgfqpoint{2.627500in}{0.908177in}}%
\pgfpathlineto{\pgfqpoint{2.705000in}{0.853060in}}%
\pgfpathlineto{\pgfqpoint{2.782500in}{0.835052in}}%
\pgfpathlineto{\pgfqpoint{2.860000in}{0.764928in}}%
\pgfpathlineto{\pgfqpoint{2.937500in}{0.750466in}}%
\pgfpathlineto{\pgfqpoint{3.015000in}{0.697259in}}%
\pgfpathlineto{\pgfqpoint{3.092500in}{0.699260in}}%
\pgfpathlineto{\pgfqpoint{3.170000in}{0.663243in}}%
\pgfpathlineto{\pgfqpoint{3.247500in}{0.715086in}}%
\pgfpathlineto{\pgfqpoint{3.325000in}{0.600000in}}%
\pgfpathlineto{\pgfqpoint{3.402500in}{0.605580in}}%
\pgfpathlineto{\pgfqpoint{3.480000in}{0.611128in}}%
\pgfpathlineto{\pgfqpoint{3.557500in}{0.607944in}}%
\pgfpathlineto{\pgfqpoint{3.635000in}{0.600213in}}%
\pgfpathlineto{\pgfqpoint{3.712500in}{0.609127in}}%
\pgfpathlineto{\pgfqpoint{3.790000in}{0.606398in}}%
\pgfpathlineto{\pgfqpoint{3.867500in}{0.600000in}}%
\pgfpathlineto{\pgfqpoint{3.945000in}{0.608217in}}%
\pgfpathlineto{\pgfqpoint{4.022500in}{0.604670in}}%
\pgfpathlineto{\pgfqpoint{4.100000in}{0.607762in}}%
\pgfpathlineto{\pgfqpoint{4.177500in}{0.603942in}}%
\pgfpathlineto{\pgfqpoint{4.255000in}{0.610309in}}%
\pgfpathlineto{\pgfqpoint{4.332500in}{0.601760in}}%
\pgfpathlineto{\pgfqpoint{4.410000in}{0.600486in}}%
\pgfpathlineto{\pgfqpoint{4.487500in}{0.609490in}}%
\pgfpathlineto{\pgfqpoint{4.565000in}{0.600000in}}%
\pgfpathlineto{\pgfqpoint{4.642500in}{0.604397in}}%
\pgfpathlineto{\pgfqpoint{4.720000in}{0.603579in}}%
\pgfpathlineto{\pgfqpoint{4.797500in}{0.611310in}}%
\pgfpathlineto{\pgfqpoint{4.875000in}{0.600000in}}%
\pgfpathlineto{\pgfqpoint{4.952500in}{0.612947in}}%
\pgfpathlineto{\pgfqpoint{5.030000in}{0.600213in}}%
\pgfpathlineto{\pgfqpoint{5.107500in}{0.600000in}}%
\pgfpathlineto{\pgfqpoint{5.185000in}{0.601760in}}%
\pgfpathlineto{\pgfqpoint{5.262500in}{0.607035in}}%
\pgfpathlineto{\pgfqpoint{5.340000in}{0.602669in}}%
\pgfpathlineto{\pgfqpoint{5.417500in}{0.600213in}}%
\pgfpathlineto{\pgfqpoint{5.495000in}{0.607581in}}%
\pgfpathlineto{\pgfqpoint{5.572500in}{0.604761in}}%
\pgfpathlineto{\pgfqpoint{5.650000in}{0.600000in}}%
\pgfpathlineto{\pgfqpoint{5.727500in}{0.616039in}}%
\pgfpathlineto{\pgfqpoint{5.805000in}{0.600000in}}%
\pgfpathlineto{\pgfqpoint{5.882500in}{0.600000in}}%
\pgfpathlineto{\pgfqpoint{5.960000in}{0.606671in}}%
\pgfpathlineto{\pgfqpoint{6.037500in}{0.608672in}}%
\pgfpathlineto{\pgfqpoint{6.115000in}{0.600486in}}%
\pgfpathlineto{\pgfqpoint{6.192500in}{0.605943in}}%
\pgfpathlineto{\pgfqpoint{6.270000in}{0.606034in}}%
\pgfpathlineto{\pgfqpoint{6.347500in}{0.600850in}}%
\pgfpathlineto{\pgfqpoint{6.425000in}{0.600000in}}%
\pgfpathlineto{\pgfqpoint{6.502500in}{0.605943in}}%
\pgfpathlineto{\pgfqpoint{6.580000in}{0.603033in}}%
\pgfpathlineto{\pgfqpoint{6.657500in}{0.607217in}}%
\pgfpathlineto{\pgfqpoint{6.735000in}{0.600000in}}%
\pgfpathlineto{\pgfqpoint{6.812500in}{0.601941in}}%
\pgfpathlineto{\pgfqpoint{6.890000in}{0.600000in}}%
\pgfpathlineto{\pgfqpoint{6.967500in}{0.609400in}}%
\pgfpathlineto{\pgfqpoint{7.045000in}{0.602214in}}%
\pgfpathlineto{\pgfqpoint{7.122500in}{0.605216in}}%
\pgfusepath{stroke}%
\end{pgfscope}%
\begin{pgfscope}%
\pgfpathrectangle{\pgfqpoint{1.000000in}{0.600000in}}{\pgfqpoint{6.200000in}{4.800000in}}%
\pgfusepath{clip}%
\pgfsetrectcap%
\pgfsetroundjoin%
\pgfsetlinewidth{2.007500pt}%
\definecolor{currentstroke}{rgb}{0.000000,0.500000,0.000000}%
\pgfsetstrokecolor{currentstroke}%
\pgfsetdash{}{0pt}%
\pgfpathmoveto{\pgfqpoint{1.000000in}{0.600000in}}%
\pgfpathlineto{\pgfqpoint{1.077500in}{0.624752in}}%
\pgfpathlineto{\pgfqpoint{1.155000in}{0.666008in}}%
\pgfpathlineto{\pgfqpoint{1.232500in}{0.862643in}}%
\pgfpathlineto{\pgfqpoint{1.310000in}{1.314321in}}%
\pgfpathlineto{\pgfqpoint{1.387500in}{2.455566in}}%
\pgfpathlineto{\pgfqpoint{1.465000in}{4.022068in}}%
\pgfpathlineto{\pgfqpoint{1.542500in}{5.096100in}}%
\pgfpathlineto{\pgfqpoint{1.620000in}{4.979657in}}%
\pgfpathlineto{\pgfqpoint{1.697500in}{4.263111in}}%
\pgfpathlineto{\pgfqpoint{1.775000in}{3.354101in}}%
\pgfpathlineto{\pgfqpoint{1.852500in}{2.608351in}}%
\pgfpathlineto{\pgfqpoint{1.930000in}{2.096691in}}%
\pgfpathlineto{\pgfqpoint{2.007500in}{1.847675in}}%
\pgfpathlineto{\pgfqpoint{2.085000in}{1.620723in}}%
\pgfpathlineto{\pgfqpoint{2.162500in}{1.445133in}}%
\pgfpathlineto{\pgfqpoint{2.240000in}{1.239505in}}%
\pgfpathlineto{\pgfqpoint{2.317500in}{1.135671in}}%
\pgfpathlineto{\pgfqpoint{2.395000in}{0.994661in}}%
\pgfpathlineto{\pgfqpoint{2.472500in}{0.968331in}}%
\pgfpathlineto{\pgfqpoint{2.550000in}{0.917620in}}%
\pgfpathlineto{\pgfqpoint{2.627500in}{0.924758in}}%
\pgfpathlineto{\pgfqpoint{2.705000in}{0.867371in}}%
\pgfpathlineto{\pgfqpoint{2.782500in}{0.840671in}}%
\pgfpathlineto{\pgfqpoint{2.860000in}{0.766597in}}%
\pgfpathlineto{\pgfqpoint{2.937500in}{0.757604in}}%
\pgfpathlineto{\pgfqpoint{3.015000in}{0.702257in}}%
\pgfpathlineto{\pgfqpoint{3.092500in}{0.702721in}}%
\pgfpathlineto{\pgfqpoint{3.170000in}{0.678338in}}%
\pgfpathlineto{\pgfqpoint{3.247500in}{0.711250in}}%
\pgfpathlineto{\pgfqpoint{3.325000in}{0.603059in}}%
\pgfpathlineto{\pgfqpoint{3.402500in}{0.600000in}}%
\pgfpathlineto{\pgfqpoint{3.480000in}{0.602873in}}%
\pgfpathlineto{\pgfqpoint{3.557500in}{0.600834in}}%
\pgfpathlineto{\pgfqpoint{3.635000in}{0.600000in}}%
\pgfpathlineto{\pgfqpoint{3.712500in}{0.600000in}}%
\pgfpathlineto{\pgfqpoint{3.790000in}{0.607138in}}%
\pgfpathlineto{\pgfqpoint{3.867500in}{0.612329in}}%
\pgfpathlineto{\pgfqpoint{3.945000in}{0.603244in}}%
\pgfpathlineto{\pgfqpoint{4.022500in}{0.605191in}}%
\pgfpathlineto{\pgfqpoint{4.100000in}{0.601482in}}%
\pgfpathlineto{\pgfqpoint{4.177500in}{0.603986in}}%
\pgfpathlineto{\pgfqpoint{4.255000in}{0.604635in}}%
\pgfpathlineto{\pgfqpoint{4.332500in}{0.605284in}}%
\pgfpathlineto{\pgfqpoint{4.410000in}{0.600926in}}%
\pgfpathlineto{\pgfqpoint{4.487500in}{0.601946in}}%
\pgfpathlineto{\pgfqpoint{4.565000in}{0.608436in}}%
\pgfpathlineto{\pgfqpoint{4.642500in}{0.601204in}}%
\pgfpathlineto{\pgfqpoint{4.720000in}{0.605098in}}%
\pgfpathlineto{\pgfqpoint{4.797500in}{0.600834in}}%
\pgfpathlineto{\pgfqpoint{4.875000in}{0.605005in}}%
\pgfpathlineto{\pgfqpoint{4.952500in}{0.610197in}}%
\pgfpathlineto{\pgfqpoint{5.030000in}{0.605191in}}%
\pgfpathlineto{\pgfqpoint{5.107500in}{0.602317in}}%
\pgfpathlineto{\pgfqpoint{5.185000in}{0.601204in}}%
\pgfpathlineto{\pgfqpoint{5.262500in}{0.605562in}}%
\pgfpathlineto{\pgfqpoint{5.340000in}{0.603337in}}%
\pgfpathlineto{\pgfqpoint{5.417500in}{0.601204in}}%
\pgfpathlineto{\pgfqpoint{5.495000in}{0.609085in}}%
\pgfpathlineto{\pgfqpoint{5.572500in}{0.602502in}}%
\pgfpathlineto{\pgfqpoint{5.650000in}{0.600000in}}%
\pgfpathlineto{\pgfqpoint{5.727500in}{0.604356in}}%
\pgfpathlineto{\pgfqpoint{5.805000in}{0.601112in}}%
\pgfpathlineto{\pgfqpoint{5.882500in}{0.603707in}}%
\pgfpathlineto{\pgfqpoint{5.960000in}{0.603429in}}%
\pgfpathlineto{\pgfqpoint{6.037500in}{0.603986in}}%
\pgfpathlineto{\pgfqpoint{6.115000in}{0.603800in}}%
\pgfpathlineto{\pgfqpoint{6.192500in}{0.606489in}}%
\pgfpathlineto{\pgfqpoint{6.270000in}{0.603522in}}%
\pgfpathlineto{\pgfqpoint{6.347500in}{0.603059in}}%
\pgfpathlineto{\pgfqpoint{6.425000in}{0.606767in}}%
\pgfpathlineto{\pgfqpoint{6.502500in}{0.600000in}}%
\pgfpathlineto{\pgfqpoint{6.580000in}{0.600000in}}%
\pgfpathlineto{\pgfqpoint{6.657500in}{0.602317in}}%
\pgfpathlineto{\pgfqpoint{6.735000in}{0.600000in}}%
\pgfpathlineto{\pgfqpoint{6.812500in}{0.603429in}}%
\pgfpathlineto{\pgfqpoint{6.890000in}{0.602039in}}%
\pgfpathlineto{\pgfqpoint{6.967500in}{0.602039in}}%
\pgfpathlineto{\pgfqpoint{7.045000in}{0.603800in}}%
\pgfpathlineto{\pgfqpoint{7.122500in}{0.601946in}}%
\pgfusepath{stroke}%
\end{pgfscope}%
\begin{pgfscope}%
\pgfsetrectcap%
\pgfsetmiterjoin%
\pgfsetlinewidth{1.003750pt}%
\definecolor{currentstroke}{rgb}{0.000000,0.000000,0.000000}%
\pgfsetstrokecolor{currentstroke}%
\pgfsetdash{}{0pt}%
\pgfpathmoveto{\pgfqpoint{1.000000in}{0.600000in}}%
\pgfpathlineto{\pgfqpoint{1.000000in}{5.400000in}}%
\pgfusepath{stroke}%
\end{pgfscope}%
\begin{pgfscope}%
\pgfsetrectcap%
\pgfsetmiterjoin%
\pgfsetlinewidth{1.003750pt}%
\definecolor{currentstroke}{rgb}{0.000000,0.000000,0.000000}%
\pgfsetstrokecolor{currentstroke}%
\pgfsetdash{}{0pt}%
\pgfpathmoveto{\pgfqpoint{7.200000in}{0.600000in}}%
\pgfpathlineto{\pgfqpoint{7.200000in}{5.400000in}}%
\pgfusepath{stroke}%
\end{pgfscope}%
\begin{pgfscope}%
\pgfsetrectcap%
\pgfsetmiterjoin%
\pgfsetlinewidth{1.003750pt}%
\definecolor{currentstroke}{rgb}{0.000000,0.000000,0.000000}%
\pgfsetstrokecolor{currentstroke}%
\pgfsetdash{}{0pt}%
\pgfpathmoveto{\pgfqpoint{1.000000in}{0.600000in}}%
\pgfpathlineto{\pgfqpoint{7.200000in}{0.600000in}}%
\pgfusepath{stroke}%
\end{pgfscope}%
\begin{pgfscope}%
\pgfsetrectcap%
\pgfsetmiterjoin%
\pgfsetlinewidth{1.003750pt}%
\definecolor{currentstroke}{rgb}{0.000000,0.000000,0.000000}%
\pgfsetstrokecolor{currentstroke}%
\pgfsetdash{}{0pt}%
\pgfpathmoveto{\pgfqpoint{1.000000in}{5.400000in}}%
\pgfpathlineto{\pgfqpoint{7.200000in}{5.400000in}}%
\pgfusepath{stroke}%
\end{pgfscope}%
\begin{pgfscope}%
\pgfpathrectangle{\pgfqpoint{1.000000in}{0.600000in}}{\pgfqpoint{6.200000in}{4.800000in}}%
\pgfusepath{clip}%
\pgfsetbuttcap%
\pgfsetroundjoin%
\pgfsetlinewidth{0.501875pt}%
\definecolor{currentstroke}{rgb}{0.000000,0.000000,0.000000}%
\pgfsetstrokecolor{currentstroke}%
\pgfsetdash{{1.000000pt}{3.000000pt}}{0.000000pt}%
\pgfpathmoveto{\pgfqpoint{1.000000in}{0.600000in}}%
\pgfpathlineto{\pgfqpoint{1.000000in}{5.400000in}}%
\pgfusepath{stroke}%
\end{pgfscope}%
\begin{pgfscope}%
\pgfsetbuttcap%
\pgfsetroundjoin%
\definecolor{currentfill}{rgb}{0.000000,0.000000,0.000000}%
\pgfsetfillcolor{currentfill}%
\pgfsetlinewidth{0.501875pt}%
\definecolor{currentstroke}{rgb}{0.000000,0.000000,0.000000}%
\pgfsetstrokecolor{currentstroke}%
\pgfsetdash{}{0pt}%
\pgfsys@defobject{currentmarker}{\pgfqpoint{0.000000in}{0.000000in}}{\pgfqpoint{0.000000in}{0.055556in}}{%
\pgfpathmoveto{\pgfqpoint{0.000000in}{0.000000in}}%
\pgfpathlineto{\pgfqpoint{0.000000in}{0.055556in}}%
\pgfusepath{stroke,fill}%
}%
\begin{pgfscope}%
\pgfsys@transformshift{1.000000in}{0.600000in}%
\pgfsys@useobject{currentmarker}{}%
\end{pgfscope}%
\end{pgfscope}%
\begin{pgfscope}%
\pgfsetbuttcap%
\pgfsetroundjoin%
\definecolor{currentfill}{rgb}{0.000000,0.000000,0.000000}%
\pgfsetfillcolor{currentfill}%
\pgfsetlinewidth{0.501875pt}%
\definecolor{currentstroke}{rgb}{0.000000,0.000000,0.000000}%
\pgfsetstrokecolor{currentstroke}%
\pgfsetdash{}{0pt}%
\pgfsys@defobject{currentmarker}{\pgfqpoint{0.000000in}{-0.055556in}}{\pgfqpoint{0.000000in}{0.000000in}}{%
\pgfpathmoveto{\pgfqpoint{0.000000in}{0.000000in}}%
\pgfpathlineto{\pgfqpoint{0.000000in}{-0.055556in}}%
\pgfusepath{stroke,fill}%
}%
\begin{pgfscope}%
\pgfsys@transformshift{1.000000in}{5.400000in}%
\pgfsys@useobject{currentmarker}{}%
\end{pgfscope}%
\end{pgfscope}%
\begin{pgfscope}%
\definecolor{textcolor}{rgb}{0.000000,0.000000,0.000000}%
\pgfsetstrokecolor{textcolor}%
\pgfsetfillcolor{textcolor}%
\pgftext[x=1.000000in,y=0.544444in,,top]{\color{textcolor}\sffamily\fontsize{16.000000}{19.200000}\selectfont \(\displaystyle {0}\)}%
\end{pgfscope}%
\begin{pgfscope}%
\pgfpathrectangle{\pgfqpoint{1.000000in}{0.600000in}}{\pgfqpoint{6.200000in}{4.800000in}}%
\pgfusepath{clip}%
\pgfsetbuttcap%
\pgfsetroundjoin%
\pgfsetlinewidth{0.501875pt}%
\definecolor{currentstroke}{rgb}{0.000000,0.000000,0.000000}%
\pgfsetstrokecolor{currentstroke}%
\pgfsetdash{{1.000000pt}{3.000000pt}}{0.000000pt}%
\pgfpathmoveto{\pgfqpoint{1.775000in}{0.600000in}}%
\pgfpathlineto{\pgfqpoint{1.775000in}{5.400000in}}%
\pgfusepath{stroke}%
\end{pgfscope}%
\begin{pgfscope}%
\pgfsetbuttcap%
\pgfsetroundjoin%
\definecolor{currentfill}{rgb}{0.000000,0.000000,0.000000}%
\pgfsetfillcolor{currentfill}%
\pgfsetlinewidth{0.501875pt}%
\definecolor{currentstroke}{rgb}{0.000000,0.000000,0.000000}%
\pgfsetstrokecolor{currentstroke}%
\pgfsetdash{}{0pt}%
\pgfsys@defobject{currentmarker}{\pgfqpoint{0.000000in}{0.000000in}}{\pgfqpoint{0.000000in}{0.055556in}}{%
\pgfpathmoveto{\pgfqpoint{0.000000in}{0.000000in}}%
\pgfpathlineto{\pgfqpoint{0.000000in}{0.055556in}}%
\pgfusepath{stroke,fill}%
}%
\begin{pgfscope}%
\pgfsys@transformshift{1.775000in}{0.600000in}%
\pgfsys@useobject{currentmarker}{}%
\end{pgfscope}%
\end{pgfscope}%
\begin{pgfscope}%
\pgfsetbuttcap%
\pgfsetroundjoin%
\definecolor{currentfill}{rgb}{0.000000,0.000000,0.000000}%
\pgfsetfillcolor{currentfill}%
\pgfsetlinewidth{0.501875pt}%
\definecolor{currentstroke}{rgb}{0.000000,0.000000,0.000000}%
\pgfsetstrokecolor{currentstroke}%
\pgfsetdash{}{0pt}%
\pgfsys@defobject{currentmarker}{\pgfqpoint{0.000000in}{-0.055556in}}{\pgfqpoint{0.000000in}{0.000000in}}{%
\pgfpathmoveto{\pgfqpoint{0.000000in}{0.000000in}}%
\pgfpathlineto{\pgfqpoint{0.000000in}{-0.055556in}}%
\pgfusepath{stroke,fill}%
}%
\begin{pgfscope}%
\pgfsys@transformshift{1.775000in}{5.400000in}%
\pgfsys@useobject{currentmarker}{}%
\end{pgfscope}%
\end{pgfscope}%
\begin{pgfscope}%
\definecolor{textcolor}{rgb}{0.000000,0.000000,0.000000}%
\pgfsetstrokecolor{textcolor}%
\pgfsetfillcolor{textcolor}%
\pgftext[x=1.775000in,y=0.544444in,,top]{\color{textcolor}\sffamily\fontsize{16.000000}{19.200000}\selectfont \(\displaystyle {10}\)}%
\end{pgfscope}%
\begin{pgfscope}%
\pgfpathrectangle{\pgfqpoint{1.000000in}{0.600000in}}{\pgfqpoint{6.200000in}{4.800000in}}%
\pgfusepath{clip}%
\pgfsetbuttcap%
\pgfsetroundjoin%
\pgfsetlinewidth{0.501875pt}%
\definecolor{currentstroke}{rgb}{0.000000,0.000000,0.000000}%
\pgfsetstrokecolor{currentstroke}%
\pgfsetdash{{1.000000pt}{3.000000pt}}{0.000000pt}%
\pgfpathmoveto{\pgfqpoint{2.550000in}{0.600000in}}%
\pgfpathlineto{\pgfqpoint{2.550000in}{5.400000in}}%
\pgfusepath{stroke}%
\end{pgfscope}%
\begin{pgfscope}%
\pgfsetbuttcap%
\pgfsetroundjoin%
\definecolor{currentfill}{rgb}{0.000000,0.000000,0.000000}%
\pgfsetfillcolor{currentfill}%
\pgfsetlinewidth{0.501875pt}%
\definecolor{currentstroke}{rgb}{0.000000,0.000000,0.000000}%
\pgfsetstrokecolor{currentstroke}%
\pgfsetdash{}{0pt}%
\pgfsys@defobject{currentmarker}{\pgfqpoint{0.000000in}{0.000000in}}{\pgfqpoint{0.000000in}{0.055556in}}{%
\pgfpathmoveto{\pgfqpoint{0.000000in}{0.000000in}}%
\pgfpathlineto{\pgfqpoint{0.000000in}{0.055556in}}%
\pgfusepath{stroke,fill}%
}%
\begin{pgfscope}%
\pgfsys@transformshift{2.550000in}{0.600000in}%
\pgfsys@useobject{currentmarker}{}%
\end{pgfscope}%
\end{pgfscope}%
\begin{pgfscope}%
\pgfsetbuttcap%
\pgfsetroundjoin%
\definecolor{currentfill}{rgb}{0.000000,0.000000,0.000000}%
\pgfsetfillcolor{currentfill}%
\pgfsetlinewidth{0.501875pt}%
\definecolor{currentstroke}{rgb}{0.000000,0.000000,0.000000}%
\pgfsetstrokecolor{currentstroke}%
\pgfsetdash{}{0pt}%
\pgfsys@defobject{currentmarker}{\pgfqpoint{0.000000in}{-0.055556in}}{\pgfqpoint{0.000000in}{0.000000in}}{%
\pgfpathmoveto{\pgfqpoint{0.000000in}{0.000000in}}%
\pgfpathlineto{\pgfqpoint{0.000000in}{-0.055556in}}%
\pgfusepath{stroke,fill}%
}%
\begin{pgfscope}%
\pgfsys@transformshift{2.550000in}{5.400000in}%
\pgfsys@useobject{currentmarker}{}%
\end{pgfscope}%
\end{pgfscope}%
\begin{pgfscope}%
\definecolor{textcolor}{rgb}{0.000000,0.000000,0.000000}%
\pgfsetstrokecolor{textcolor}%
\pgfsetfillcolor{textcolor}%
\pgftext[x=2.550000in,y=0.544444in,,top]{\color{textcolor}\sffamily\fontsize{16.000000}{19.200000}\selectfont \(\displaystyle {20}\)}%
\end{pgfscope}%
\begin{pgfscope}%
\pgfpathrectangle{\pgfqpoint{1.000000in}{0.600000in}}{\pgfqpoint{6.200000in}{4.800000in}}%
\pgfusepath{clip}%
\pgfsetbuttcap%
\pgfsetroundjoin%
\pgfsetlinewidth{0.501875pt}%
\definecolor{currentstroke}{rgb}{0.000000,0.000000,0.000000}%
\pgfsetstrokecolor{currentstroke}%
\pgfsetdash{{1.000000pt}{3.000000pt}}{0.000000pt}%
\pgfpathmoveto{\pgfqpoint{3.325000in}{0.600000in}}%
\pgfpathlineto{\pgfqpoint{3.325000in}{5.400000in}}%
\pgfusepath{stroke}%
\end{pgfscope}%
\begin{pgfscope}%
\pgfsetbuttcap%
\pgfsetroundjoin%
\definecolor{currentfill}{rgb}{0.000000,0.000000,0.000000}%
\pgfsetfillcolor{currentfill}%
\pgfsetlinewidth{0.501875pt}%
\definecolor{currentstroke}{rgb}{0.000000,0.000000,0.000000}%
\pgfsetstrokecolor{currentstroke}%
\pgfsetdash{}{0pt}%
\pgfsys@defobject{currentmarker}{\pgfqpoint{0.000000in}{0.000000in}}{\pgfqpoint{0.000000in}{0.055556in}}{%
\pgfpathmoveto{\pgfqpoint{0.000000in}{0.000000in}}%
\pgfpathlineto{\pgfqpoint{0.000000in}{0.055556in}}%
\pgfusepath{stroke,fill}%
}%
\begin{pgfscope}%
\pgfsys@transformshift{3.325000in}{0.600000in}%
\pgfsys@useobject{currentmarker}{}%
\end{pgfscope}%
\end{pgfscope}%
\begin{pgfscope}%
\pgfsetbuttcap%
\pgfsetroundjoin%
\definecolor{currentfill}{rgb}{0.000000,0.000000,0.000000}%
\pgfsetfillcolor{currentfill}%
\pgfsetlinewidth{0.501875pt}%
\definecolor{currentstroke}{rgb}{0.000000,0.000000,0.000000}%
\pgfsetstrokecolor{currentstroke}%
\pgfsetdash{}{0pt}%
\pgfsys@defobject{currentmarker}{\pgfqpoint{0.000000in}{-0.055556in}}{\pgfqpoint{0.000000in}{0.000000in}}{%
\pgfpathmoveto{\pgfqpoint{0.000000in}{0.000000in}}%
\pgfpathlineto{\pgfqpoint{0.000000in}{-0.055556in}}%
\pgfusepath{stroke,fill}%
}%
\begin{pgfscope}%
\pgfsys@transformshift{3.325000in}{5.400000in}%
\pgfsys@useobject{currentmarker}{}%
\end{pgfscope}%
\end{pgfscope}%
\begin{pgfscope}%
\definecolor{textcolor}{rgb}{0.000000,0.000000,0.000000}%
\pgfsetstrokecolor{textcolor}%
\pgfsetfillcolor{textcolor}%
\pgftext[x=3.325000in,y=0.544444in,,top]{\color{textcolor}\sffamily\fontsize{16.000000}{19.200000}\selectfont \(\displaystyle {30}\)}%
\end{pgfscope}%
\begin{pgfscope}%
\pgfpathrectangle{\pgfqpoint{1.000000in}{0.600000in}}{\pgfqpoint{6.200000in}{4.800000in}}%
\pgfusepath{clip}%
\pgfsetbuttcap%
\pgfsetroundjoin%
\pgfsetlinewidth{0.501875pt}%
\definecolor{currentstroke}{rgb}{0.000000,0.000000,0.000000}%
\pgfsetstrokecolor{currentstroke}%
\pgfsetdash{{1.000000pt}{3.000000pt}}{0.000000pt}%
\pgfpathmoveto{\pgfqpoint{4.100000in}{0.600000in}}%
\pgfpathlineto{\pgfqpoint{4.100000in}{5.400000in}}%
\pgfusepath{stroke}%
\end{pgfscope}%
\begin{pgfscope}%
\pgfsetbuttcap%
\pgfsetroundjoin%
\definecolor{currentfill}{rgb}{0.000000,0.000000,0.000000}%
\pgfsetfillcolor{currentfill}%
\pgfsetlinewidth{0.501875pt}%
\definecolor{currentstroke}{rgb}{0.000000,0.000000,0.000000}%
\pgfsetstrokecolor{currentstroke}%
\pgfsetdash{}{0pt}%
\pgfsys@defobject{currentmarker}{\pgfqpoint{0.000000in}{0.000000in}}{\pgfqpoint{0.000000in}{0.055556in}}{%
\pgfpathmoveto{\pgfqpoint{0.000000in}{0.000000in}}%
\pgfpathlineto{\pgfqpoint{0.000000in}{0.055556in}}%
\pgfusepath{stroke,fill}%
}%
\begin{pgfscope}%
\pgfsys@transformshift{4.100000in}{0.600000in}%
\pgfsys@useobject{currentmarker}{}%
\end{pgfscope}%
\end{pgfscope}%
\begin{pgfscope}%
\pgfsetbuttcap%
\pgfsetroundjoin%
\definecolor{currentfill}{rgb}{0.000000,0.000000,0.000000}%
\pgfsetfillcolor{currentfill}%
\pgfsetlinewidth{0.501875pt}%
\definecolor{currentstroke}{rgb}{0.000000,0.000000,0.000000}%
\pgfsetstrokecolor{currentstroke}%
\pgfsetdash{}{0pt}%
\pgfsys@defobject{currentmarker}{\pgfqpoint{0.000000in}{-0.055556in}}{\pgfqpoint{0.000000in}{0.000000in}}{%
\pgfpathmoveto{\pgfqpoint{0.000000in}{0.000000in}}%
\pgfpathlineto{\pgfqpoint{0.000000in}{-0.055556in}}%
\pgfusepath{stroke,fill}%
}%
\begin{pgfscope}%
\pgfsys@transformshift{4.100000in}{5.400000in}%
\pgfsys@useobject{currentmarker}{}%
\end{pgfscope}%
\end{pgfscope}%
\begin{pgfscope}%
\definecolor{textcolor}{rgb}{0.000000,0.000000,0.000000}%
\pgfsetstrokecolor{textcolor}%
\pgfsetfillcolor{textcolor}%
\pgftext[x=4.100000in,y=0.544444in,,top]{\color{textcolor}\sffamily\fontsize{16.000000}{19.200000}\selectfont \(\displaystyle {40}\)}%
\end{pgfscope}%
\begin{pgfscope}%
\pgfpathrectangle{\pgfqpoint{1.000000in}{0.600000in}}{\pgfqpoint{6.200000in}{4.800000in}}%
\pgfusepath{clip}%
\pgfsetbuttcap%
\pgfsetroundjoin%
\pgfsetlinewidth{0.501875pt}%
\definecolor{currentstroke}{rgb}{0.000000,0.000000,0.000000}%
\pgfsetstrokecolor{currentstroke}%
\pgfsetdash{{1.000000pt}{3.000000pt}}{0.000000pt}%
\pgfpathmoveto{\pgfqpoint{4.875000in}{0.600000in}}%
\pgfpathlineto{\pgfqpoint{4.875000in}{5.400000in}}%
\pgfusepath{stroke}%
\end{pgfscope}%
\begin{pgfscope}%
\pgfsetbuttcap%
\pgfsetroundjoin%
\definecolor{currentfill}{rgb}{0.000000,0.000000,0.000000}%
\pgfsetfillcolor{currentfill}%
\pgfsetlinewidth{0.501875pt}%
\definecolor{currentstroke}{rgb}{0.000000,0.000000,0.000000}%
\pgfsetstrokecolor{currentstroke}%
\pgfsetdash{}{0pt}%
\pgfsys@defobject{currentmarker}{\pgfqpoint{0.000000in}{0.000000in}}{\pgfqpoint{0.000000in}{0.055556in}}{%
\pgfpathmoveto{\pgfqpoint{0.000000in}{0.000000in}}%
\pgfpathlineto{\pgfqpoint{0.000000in}{0.055556in}}%
\pgfusepath{stroke,fill}%
}%
\begin{pgfscope}%
\pgfsys@transformshift{4.875000in}{0.600000in}%
\pgfsys@useobject{currentmarker}{}%
\end{pgfscope}%
\end{pgfscope}%
\begin{pgfscope}%
\pgfsetbuttcap%
\pgfsetroundjoin%
\definecolor{currentfill}{rgb}{0.000000,0.000000,0.000000}%
\pgfsetfillcolor{currentfill}%
\pgfsetlinewidth{0.501875pt}%
\definecolor{currentstroke}{rgb}{0.000000,0.000000,0.000000}%
\pgfsetstrokecolor{currentstroke}%
\pgfsetdash{}{0pt}%
\pgfsys@defobject{currentmarker}{\pgfqpoint{0.000000in}{-0.055556in}}{\pgfqpoint{0.000000in}{0.000000in}}{%
\pgfpathmoveto{\pgfqpoint{0.000000in}{0.000000in}}%
\pgfpathlineto{\pgfqpoint{0.000000in}{-0.055556in}}%
\pgfusepath{stroke,fill}%
}%
\begin{pgfscope}%
\pgfsys@transformshift{4.875000in}{5.400000in}%
\pgfsys@useobject{currentmarker}{}%
\end{pgfscope}%
\end{pgfscope}%
\begin{pgfscope}%
\definecolor{textcolor}{rgb}{0.000000,0.000000,0.000000}%
\pgfsetstrokecolor{textcolor}%
\pgfsetfillcolor{textcolor}%
\pgftext[x=4.875000in,y=0.544444in,,top]{\color{textcolor}\sffamily\fontsize{16.000000}{19.200000}\selectfont \(\displaystyle {50}\)}%
\end{pgfscope}%
\begin{pgfscope}%
\pgfpathrectangle{\pgfqpoint{1.000000in}{0.600000in}}{\pgfqpoint{6.200000in}{4.800000in}}%
\pgfusepath{clip}%
\pgfsetbuttcap%
\pgfsetroundjoin%
\pgfsetlinewidth{0.501875pt}%
\definecolor{currentstroke}{rgb}{0.000000,0.000000,0.000000}%
\pgfsetstrokecolor{currentstroke}%
\pgfsetdash{{1.000000pt}{3.000000pt}}{0.000000pt}%
\pgfpathmoveto{\pgfqpoint{5.650000in}{0.600000in}}%
\pgfpathlineto{\pgfqpoint{5.650000in}{5.400000in}}%
\pgfusepath{stroke}%
\end{pgfscope}%
\begin{pgfscope}%
\pgfsetbuttcap%
\pgfsetroundjoin%
\definecolor{currentfill}{rgb}{0.000000,0.000000,0.000000}%
\pgfsetfillcolor{currentfill}%
\pgfsetlinewidth{0.501875pt}%
\definecolor{currentstroke}{rgb}{0.000000,0.000000,0.000000}%
\pgfsetstrokecolor{currentstroke}%
\pgfsetdash{}{0pt}%
\pgfsys@defobject{currentmarker}{\pgfqpoint{0.000000in}{0.000000in}}{\pgfqpoint{0.000000in}{0.055556in}}{%
\pgfpathmoveto{\pgfqpoint{0.000000in}{0.000000in}}%
\pgfpathlineto{\pgfqpoint{0.000000in}{0.055556in}}%
\pgfusepath{stroke,fill}%
}%
\begin{pgfscope}%
\pgfsys@transformshift{5.650000in}{0.600000in}%
\pgfsys@useobject{currentmarker}{}%
\end{pgfscope}%
\end{pgfscope}%
\begin{pgfscope}%
\pgfsetbuttcap%
\pgfsetroundjoin%
\definecolor{currentfill}{rgb}{0.000000,0.000000,0.000000}%
\pgfsetfillcolor{currentfill}%
\pgfsetlinewidth{0.501875pt}%
\definecolor{currentstroke}{rgb}{0.000000,0.000000,0.000000}%
\pgfsetstrokecolor{currentstroke}%
\pgfsetdash{}{0pt}%
\pgfsys@defobject{currentmarker}{\pgfqpoint{0.000000in}{-0.055556in}}{\pgfqpoint{0.000000in}{0.000000in}}{%
\pgfpathmoveto{\pgfqpoint{0.000000in}{0.000000in}}%
\pgfpathlineto{\pgfqpoint{0.000000in}{-0.055556in}}%
\pgfusepath{stroke,fill}%
}%
\begin{pgfscope}%
\pgfsys@transformshift{5.650000in}{5.400000in}%
\pgfsys@useobject{currentmarker}{}%
\end{pgfscope}%
\end{pgfscope}%
\begin{pgfscope}%
\definecolor{textcolor}{rgb}{0.000000,0.000000,0.000000}%
\pgfsetstrokecolor{textcolor}%
\pgfsetfillcolor{textcolor}%
\pgftext[x=5.650000in,y=0.544444in,,top]{\color{textcolor}\sffamily\fontsize{16.000000}{19.200000}\selectfont \(\displaystyle {60}\)}%
\end{pgfscope}%
\begin{pgfscope}%
\pgfpathrectangle{\pgfqpoint{1.000000in}{0.600000in}}{\pgfqpoint{6.200000in}{4.800000in}}%
\pgfusepath{clip}%
\pgfsetbuttcap%
\pgfsetroundjoin%
\pgfsetlinewidth{0.501875pt}%
\definecolor{currentstroke}{rgb}{0.000000,0.000000,0.000000}%
\pgfsetstrokecolor{currentstroke}%
\pgfsetdash{{1.000000pt}{3.000000pt}}{0.000000pt}%
\pgfpathmoveto{\pgfqpoint{6.425000in}{0.600000in}}%
\pgfpathlineto{\pgfqpoint{6.425000in}{5.400000in}}%
\pgfusepath{stroke}%
\end{pgfscope}%
\begin{pgfscope}%
\pgfsetbuttcap%
\pgfsetroundjoin%
\definecolor{currentfill}{rgb}{0.000000,0.000000,0.000000}%
\pgfsetfillcolor{currentfill}%
\pgfsetlinewidth{0.501875pt}%
\definecolor{currentstroke}{rgb}{0.000000,0.000000,0.000000}%
\pgfsetstrokecolor{currentstroke}%
\pgfsetdash{}{0pt}%
\pgfsys@defobject{currentmarker}{\pgfqpoint{0.000000in}{0.000000in}}{\pgfqpoint{0.000000in}{0.055556in}}{%
\pgfpathmoveto{\pgfqpoint{0.000000in}{0.000000in}}%
\pgfpathlineto{\pgfqpoint{0.000000in}{0.055556in}}%
\pgfusepath{stroke,fill}%
}%
\begin{pgfscope}%
\pgfsys@transformshift{6.425000in}{0.600000in}%
\pgfsys@useobject{currentmarker}{}%
\end{pgfscope}%
\end{pgfscope}%
\begin{pgfscope}%
\pgfsetbuttcap%
\pgfsetroundjoin%
\definecolor{currentfill}{rgb}{0.000000,0.000000,0.000000}%
\pgfsetfillcolor{currentfill}%
\pgfsetlinewidth{0.501875pt}%
\definecolor{currentstroke}{rgb}{0.000000,0.000000,0.000000}%
\pgfsetstrokecolor{currentstroke}%
\pgfsetdash{}{0pt}%
\pgfsys@defobject{currentmarker}{\pgfqpoint{0.000000in}{-0.055556in}}{\pgfqpoint{0.000000in}{0.000000in}}{%
\pgfpathmoveto{\pgfqpoint{0.000000in}{0.000000in}}%
\pgfpathlineto{\pgfqpoint{0.000000in}{-0.055556in}}%
\pgfusepath{stroke,fill}%
}%
\begin{pgfscope}%
\pgfsys@transformshift{6.425000in}{5.400000in}%
\pgfsys@useobject{currentmarker}{}%
\end{pgfscope}%
\end{pgfscope}%
\begin{pgfscope}%
\definecolor{textcolor}{rgb}{0.000000,0.000000,0.000000}%
\pgfsetstrokecolor{textcolor}%
\pgfsetfillcolor{textcolor}%
\pgftext[x=6.425000in,y=0.544444in,,top]{\color{textcolor}\sffamily\fontsize{16.000000}{19.200000}\selectfont \(\displaystyle {70}\)}%
\end{pgfscope}%
\begin{pgfscope}%
\pgfpathrectangle{\pgfqpoint{1.000000in}{0.600000in}}{\pgfqpoint{6.200000in}{4.800000in}}%
\pgfusepath{clip}%
\pgfsetbuttcap%
\pgfsetroundjoin%
\pgfsetlinewidth{0.501875pt}%
\definecolor{currentstroke}{rgb}{0.000000,0.000000,0.000000}%
\pgfsetstrokecolor{currentstroke}%
\pgfsetdash{{1.000000pt}{3.000000pt}}{0.000000pt}%
\pgfpathmoveto{\pgfqpoint{7.200000in}{0.600000in}}%
\pgfpathlineto{\pgfqpoint{7.200000in}{5.400000in}}%
\pgfusepath{stroke}%
\end{pgfscope}%
\begin{pgfscope}%
\pgfsetbuttcap%
\pgfsetroundjoin%
\definecolor{currentfill}{rgb}{0.000000,0.000000,0.000000}%
\pgfsetfillcolor{currentfill}%
\pgfsetlinewidth{0.501875pt}%
\definecolor{currentstroke}{rgb}{0.000000,0.000000,0.000000}%
\pgfsetstrokecolor{currentstroke}%
\pgfsetdash{}{0pt}%
\pgfsys@defobject{currentmarker}{\pgfqpoint{0.000000in}{0.000000in}}{\pgfqpoint{0.000000in}{0.055556in}}{%
\pgfpathmoveto{\pgfqpoint{0.000000in}{0.000000in}}%
\pgfpathlineto{\pgfqpoint{0.000000in}{0.055556in}}%
\pgfusepath{stroke,fill}%
}%
\begin{pgfscope}%
\pgfsys@transformshift{7.200000in}{0.600000in}%
\pgfsys@useobject{currentmarker}{}%
\end{pgfscope}%
\end{pgfscope}%
\begin{pgfscope}%
\pgfsetbuttcap%
\pgfsetroundjoin%
\definecolor{currentfill}{rgb}{0.000000,0.000000,0.000000}%
\pgfsetfillcolor{currentfill}%
\pgfsetlinewidth{0.501875pt}%
\definecolor{currentstroke}{rgb}{0.000000,0.000000,0.000000}%
\pgfsetstrokecolor{currentstroke}%
\pgfsetdash{}{0pt}%
\pgfsys@defobject{currentmarker}{\pgfqpoint{0.000000in}{-0.055556in}}{\pgfqpoint{0.000000in}{0.000000in}}{%
\pgfpathmoveto{\pgfqpoint{0.000000in}{0.000000in}}%
\pgfpathlineto{\pgfqpoint{0.000000in}{-0.055556in}}%
\pgfusepath{stroke,fill}%
}%
\begin{pgfscope}%
\pgfsys@transformshift{7.200000in}{5.400000in}%
\pgfsys@useobject{currentmarker}{}%
\end{pgfscope}%
\end{pgfscope}%
\begin{pgfscope}%
\definecolor{textcolor}{rgb}{0.000000,0.000000,0.000000}%
\pgfsetstrokecolor{textcolor}%
\pgfsetfillcolor{textcolor}%
\pgftext[x=7.200000in,y=0.544444in,,top]{\color{textcolor}\sffamily\fontsize{16.000000}{19.200000}\selectfont \(\displaystyle {80}\)}%
\end{pgfscope}%
\begin{pgfscope}%
\definecolor{textcolor}{rgb}{0.000000,0.000000,0.000000}%
\pgfsetstrokecolor{textcolor}%
\pgfsetfillcolor{textcolor}%
\pgftext[x=4.100000in,y=0.261651in,,top]{\color{textcolor}\sffamily\fontsize{16.000000}{19.200000}\selectfont \(\displaystyle Time/\mathrm{ns}\)}%
\end{pgfscope}%
\begin{pgfscope}%
\pgfpathrectangle{\pgfqpoint{1.000000in}{0.600000in}}{\pgfqpoint{6.200000in}{4.800000in}}%
\pgfusepath{clip}%
\pgfsetbuttcap%
\pgfsetroundjoin%
\pgfsetlinewidth{0.501875pt}%
\definecolor{currentstroke}{rgb}{0.000000,0.000000,0.000000}%
\pgfsetstrokecolor{currentstroke}%
\pgfsetdash{{1.000000pt}{3.000000pt}}{0.000000pt}%
\pgfpathmoveto{\pgfqpoint{1.000000in}{0.600000in}}%
\pgfpathlineto{\pgfqpoint{7.200000in}{0.600000in}}%
\pgfusepath{stroke}%
\end{pgfscope}%
\begin{pgfscope}%
\pgfsetbuttcap%
\pgfsetroundjoin%
\definecolor{currentfill}{rgb}{0.000000,0.000000,0.000000}%
\pgfsetfillcolor{currentfill}%
\pgfsetlinewidth{0.501875pt}%
\definecolor{currentstroke}{rgb}{0.000000,0.000000,0.000000}%
\pgfsetstrokecolor{currentstroke}%
\pgfsetdash{}{0pt}%
\pgfsys@defobject{currentmarker}{\pgfqpoint{0.000000in}{0.000000in}}{\pgfqpoint{0.055556in}{0.000000in}}{%
\pgfpathmoveto{\pgfqpoint{0.000000in}{0.000000in}}%
\pgfpathlineto{\pgfqpoint{0.055556in}{0.000000in}}%
\pgfusepath{stroke,fill}%
}%
\begin{pgfscope}%
\pgfsys@transformshift{1.000000in}{0.600000in}%
\pgfsys@useobject{currentmarker}{}%
\end{pgfscope}%
\end{pgfscope}%
\begin{pgfscope}%
\pgfsetbuttcap%
\pgfsetroundjoin%
\definecolor{currentfill}{rgb}{0.000000,0.000000,0.000000}%
\pgfsetfillcolor{currentfill}%
\pgfsetlinewidth{0.501875pt}%
\definecolor{currentstroke}{rgb}{0.000000,0.000000,0.000000}%
\pgfsetstrokecolor{currentstroke}%
\pgfsetdash{}{0pt}%
\pgfsys@defobject{currentmarker}{\pgfqpoint{-0.055556in}{0.000000in}}{\pgfqpoint{-0.000000in}{0.000000in}}{%
\pgfpathmoveto{\pgfqpoint{-0.000000in}{0.000000in}}%
\pgfpathlineto{\pgfqpoint{-0.055556in}{0.000000in}}%
\pgfusepath{stroke,fill}%
}%
\begin{pgfscope}%
\pgfsys@transformshift{7.200000in}{0.600000in}%
\pgfsys@useobject{currentmarker}{}%
\end{pgfscope}%
\end{pgfscope}%
\begin{pgfscope}%
\definecolor{textcolor}{rgb}{0.000000,0.000000,0.000000}%
\pgfsetstrokecolor{textcolor}%
\pgfsetfillcolor{textcolor}%
\pgftext[x=0.944444in,y=0.600000in,right,]{\color{textcolor}\sffamily\fontsize{16.000000}{19.200000}\selectfont \(\displaystyle {0}\)}%
\end{pgfscope}%
\begin{pgfscope}%
\pgfpathrectangle{\pgfqpoint{1.000000in}{0.600000in}}{\pgfqpoint{6.200000in}{4.800000in}}%
\pgfusepath{clip}%
\pgfsetbuttcap%
\pgfsetroundjoin%
\pgfsetlinewidth{0.501875pt}%
\definecolor{currentstroke}{rgb}{0.000000,0.000000,0.000000}%
\pgfsetstrokecolor{currentstroke}%
\pgfsetdash{{1.000000pt}{3.000000pt}}{0.000000pt}%
\pgfpathmoveto{\pgfqpoint{1.000000in}{1.560000in}}%
\pgfpathlineto{\pgfqpoint{7.200000in}{1.560000in}}%
\pgfusepath{stroke}%
\end{pgfscope}%
\begin{pgfscope}%
\pgfsetbuttcap%
\pgfsetroundjoin%
\definecolor{currentfill}{rgb}{0.000000,0.000000,0.000000}%
\pgfsetfillcolor{currentfill}%
\pgfsetlinewidth{0.501875pt}%
\definecolor{currentstroke}{rgb}{0.000000,0.000000,0.000000}%
\pgfsetstrokecolor{currentstroke}%
\pgfsetdash{}{0pt}%
\pgfsys@defobject{currentmarker}{\pgfqpoint{0.000000in}{0.000000in}}{\pgfqpoint{0.055556in}{0.000000in}}{%
\pgfpathmoveto{\pgfqpoint{0.000000in}{0.000000in}}%
\pgfpathlineto{\pgfqpoint{0.055556in}{0.000000in}}%
\pgfusepath{stroke,fill}%
}%
\begin{pgfscope}%
\pgfsys@transformshift{1.000000in}{1.560000in}%
\pgfsys@useobject{currentmarker}{}%
\end{pgfscope}%
\end{pgfscope}%
\begin{pgfscope}%
\pgfsetbuttcap%
\pgfsetroundjoin%
\definecolor{currentfill}{rgb}{0.000000,0.000000,0.000000}%
\pgfsetfillcolor{currentfill}%
\pgfsetlinewidth{0.501875pt}%
\definecolor{currentstroke}{rgb}{0.000000,0.000000,0.000000}%
\pgfsetstrokecolor{currentstroke}%
\pgfsetdash{}{0pt}%
\pgfsys@defobject{currentmarker}{\pgfqpoint{-0.055556in}{0.000000in}}{\pgfqpoint{-0.000000in}{0.000000in}}{%
\pgfpathmoveto{\pgfqpoint{-0.000000in}{0.000000in}}%
\pgfpathlineto{\pgfqpoint{-0.055556in}{0.000000in}}%
\pgfusepath{stroke,fill}%
}%
\begin{pgfscope}%
\pgfsys@transformshift{7.200000in}{1.560000in}%
\pgfsys@useobject{currentmarker}{}%
\end{pgfscope}%
\end{pgfscope}%
\begin{pgfscope}%
\definecolor{textcolor}{rgb}{0.000000,0.000000,0.000000}%
\pgfsetstrokecolor{textcolor}%
\pgfsetfillcolor{textcolor}%
\pgftext[x=0.944444in,y=1.560000in,right,]{\color{textcolor}\sffamily\fontsize{16.000000}{19.200000}\selectfont \(\displaystyle {5}\)}%
\end{pgfscope}%
\begin{pgfscope}%
\pgfpathrectangle{\pgfqpoint{1.000000in}{0.600000in}}{\pgfqpoint{6.200000in}{4.800000in}}%
\pgfusepath{clip}%
\pgfsetbuttcap%
\pgfsetroundjoin%
\pgfsetlinewidth{0.501875pt}%
\definecolor{currentstroke}{rgb}{0.000000,0.000000,0.000000}%
\pgfsetstrokecolor{currentstroke}%
\pgfsetdash{{1.000000pt}{3.000000pt}}{0.000000pt}%
\pgfpathmoveto{\pgfqpoint{1.000000in}{2.520000in}}%
\pgfpathlineto{\pgfqpoint{7.200000in}{2.520000in}}%
\pgfusepath{stroke}%
\end{pgfscope}%
\begin{pgfscope}%
\pgfsetbuttcap%
\pgfsetroundjoin%
\definecolor{currentfill}{rgb}{0.000000,0.000000,0.000000}%
\pgfsetfillcolor{currentfill}%
\pgfsetlinewidth{0.501875pt}%
\definecolor{currentstroke}{rgb}{0.000000,0.000000,0.000000}%
\pgfsetstrokecolor{currentstroke}%
\pgfsetdash{}{0pt}%
\pgfsys@defobject{currentmarker}{\pgfqpoint{0.000000in}{0.000000in}}{\pgfqpoint{0.055556in}{0.000000in}}{%
\pgfpathmoveto{\pgfqpoint{0.000000in}{0.000000in}}%
\pgfpathlineto{\pgfqpoint{0.055556in}{0.000000in}}%
\pgfusepath{stroke,fill}%
}%
\begin{pgfscope}%
\pgfsys@transformshift{1.000000in}{2.520000in}%
\pgfsys@useobject{currentmarker}{}%
\end{pgfscope}%
\end{pgfscope}%
\begin{pgfscope}%
\pgfsetbuttcap%
\pgfsetroundjoin%
\definecolor{currentfill}{rgb}{0.000000,0.000000,0.000000}%
\pgfsetfillcolor{currentfill}%
\pgfsetlinewidth{0.501875pt}%
\definecolor{currentstroke}{rgb}{0.000000,0.000000,0.000000}%
\pgfsetstrokecolor{currentstroke}%
\pgfsetdash{}{0pt}%
\pgfsys@defobject{currentmarker}{\pgfqpoint{-0.055556in}{0.000000in}}{\pgfqpoint{-0.000000in}{0.000000in}}{%
\pgfpathmoveto{\pgfqpoint{-0.000000in}{0.000000in}}%
\pgfpathlineto{\pgfqpoint{-0.055556in}{0.000000in}}%
\pgfusepath{stroke,fill}%
}%
\begin{pgfscope}%
\pgfsys@transformshift{7.200000in}{2.520000in}%
\pgfsys@useobject{currentmarker}{}%
\end{pgfscope}%
\end{pgfscope}%
\begin{pgfscope}%
\definecolor{textcolor}{rgb}{0.000000,0.000000,0.000000}%
\pgfsetstrokecolor{textcolor}%
\pgfsetfillcolor{textcolor}%
\pgftext[x=0.944444in,y=2.520000in,right,]{\color{textcolor}\sffamily\fontsize{16.000000}{19.200000}\selectfont \(\displaystyle {10}\)}%
\end{pgfscope}%
\begin{pgfscope}%
\pgfpathrectangle{\pgfqpoint{1.000000in}{0.600000in}}{\pgfqpoint{6.200000in}{4.800000in}}%
\pgfusepath{clip}%
\pgfsetbuttcap%
\pgfsetroundjoin%
\pgfsetlinewidth{0.501875pt}%
\definecolor{currentstroke}{rgb}{0.000000,0.000000,0.000000}%
\pgfsetstrokecolor{currentstroke}%
\pgfsetdash{{1.000000pt}{3.000000pt}}{0.000000pt}%
\pgfpathmoveto{\pgfqpoint{1.000000in}{3.480000in}}%
\pgfpathlineto{\pgfqpoint{7.200000in}{3.480000in}}%
\pgfusepath{stroke}%
\end{pgfscope}%
\begin{pgfscope}%
\pgfsetbuttcap%
\pgfsetroundjoin%
\definecolor{currentfill}{rgb}{0.000000,0.000000,0.000000}%
\pgfsetfillcolor{currentfill}%
\pgfsetlinewidth{0.501875pt}%
\definecolor{currentstroke}{rgb}{0.000000,0.000000,0.000000}%
\pgfsetstrokecolor{currentstroke}%
\pgfsetdash{}{0pt}%
\pgfsys@defobject{currentmarker}{\pgfqpoint{0.000000in}{0.000000in}}{\pgfqpoint{0.055556in}{0.000000in}}{%
\pgfpathmoveto{\pgfqpoint{0.000000in}{0.000000in}}%
\pgfpathlineto{\pgfqpoint{0.055556in}{0.000000in}}%
\pgfusepath{stroke,fill}%
}%
\begin{pgfscope}%
\pgfsys@transformshift{1.000000in}{3.480000in}%
\pgfsys@useobject{currentmarker}{}%
\end{pgfscope}%
\end{pgfscope}%
\begin{pgfscope}%
\pgfsetbuttcap%
\pgfsetroundjoin%
\definecolor{currentfill}{rgb}{0.000000,0.000000,0.000000}%
\pgfsetfillcolor{currentfill}%
\pgfsetlinewidth{0.501875pt}%
\definecolor{currentstroke}{rgb}{0.000000,0.000000,0.000000}%
\pgfsetstrokecolor{currentstroke}%
\pgfsetdash{}{0pt}%
\pgfsys@defobject{currentmarker}{\pgfqpoint{-0.055556in}{0.000000in}}{\pgfqpoint{-0.000000in}{0.000000in}}{%
\pgfpathmoveto{\pgfqpoint{-0.000000in}{0.000000in}}%
\pgfpathlineto{\pgfqpoint{-0.055556in}{0.000000in}}%
\pgfusepath{stroke,fill}%
}%
\begin{pgfscope}%
\pgfsys@transformshift{7.200000in}{3.480000in}%
\pgfsys@useobject{currentmarker}{}%
\end{pgfscope}%
\end{pgfscope}%
\begin{pgfscope}%
\definecolor{textcolor}{rgb}{0.000000,0.000000,0.000000}%
\pgfsetstrokecolor{textcolor}%
\pgfsetfillcolor{textcolor}%
\pgftext[x=0.944444in,y=3.480000in,right,]{\color{textcolor}\sffamily\fontsize{16.000000}{19.200000}\selectfont \(\displaystyle {15}\)}%
\end{pgfscope}%
\begin{pgfscope}%
\pgfpathrectangle{\pgfqpoint{1.000000in}{0.600000in}}{\pgfqpoint{6.200000in}{4.800000in}}%
\pgfusepath{clip}%
\pgfsetbuttcap%
\pgfsetroundjoin%
\pgfsetlinewidth{0.501875pt}%
\definecolor{currentstroke}{rgb}{0.000000,0.000000,0.000000}%
\pgfsetstrokecolor{currentstroke}%
\pgfsetdash{{1.000000pt}{3.000000pt}}{0.000000pt}%
\pgfpathmoveto{\pgfqpoint{1.000000in}{4.440000in}}%
\pgfpathlineto{\pgfqpoint{7.200000in}{4.440000in}}%
\pgfusepath{stroke}%
\end{pgfscope}%
\begin{pgfscope}%
\pgfsetbuttcap%
\pgfsetroundjoin%
\definecolor{currentfill}{rgb}{0.000000,0.000000,0.000000}%
\pgfsetfillcolor{currentfill}%
\pgfsetlinewidth{0.501875pt}%
\definecolor{currentstroke}{rgb}{0.000000,0.000000,0.000000}%
\pgfsetstrokecolor{currentstroke}%
\pgfsetdash{}{0pt}%
\pgfsys@defobject{currentmarker}{\pgfqpoint{0.000000in}{0.000000in}}{\pgfqpoint{0.055556in}{0.000000in}}{%
\pgfpathmoveto{\pgfqpoint{0.000000in}{0.000000in}}%
\pgfpathlineto{\pgfqpoint{0.055556in}{0.000000in}}%
\pgfusepath{stroke,fill}%
}%
\begin{pgfscope}%
\pgfsys@transformshift{1.000000in}{4.440000in}%
\pgfsys@useobject{currentmarker}{}%
\end{pgfscope}%
\end{pgfscope}%
\begin{pgfscope}%
\pgfsetbuttcap%
\pgfsetroundjoin%
\definecolor{currentfill}{rgb}{0.000000,0.000000,0.000000}%
\pgfsetfillcolor{currentfill}%
\pgfsetlinewidth{0.501875pt}%
\definecolor{currentstroke}{rgb}{0.000000,0.000000,0.000000}%
\pgfsetstrokecolor{currentstroke}%
\pgfsetdash{}{0pt}%
\pgfsys@defobject{currentmarker}{\pgfqpoint{-0.055556in}{0.000000in}}{\pgfqpoint{-0.000000in}{0.000000in}}{%
\pgfpathmoveto{\pgfqpoint{-0.000000in}{0.000000in}}%
\pgfpathlineto{\pgfqpoint{-0.055556in}{0.000000in}}%
\pgfusepath{stroke,fill}%
}%
\begin{pgfscope}%
\pgfsys@transformshift{7.200000in}{4.440000in}%
\pgfsys@useobject{currentmarker}{}%
\end{pgfscope}%
\end{pgfscope}%
\begin{pgfscope}%
\definecolor{textcolor}{rgb}{0.000000,0.000000,0.000000}%
\pgfsetstrokecolor{textcolor}%
\pgfsetfillcolor{textcolor}%
\pgftext[x=0.944444in,y=4.440000in,right,]{\color{textcolor}\sffamily\fontsize{16.000000}{19.200000}\selectfont \(\displaystyle {20}\)}%
\end{pgfscope}%
\begin{pgfscope}%
\pgfpathrectangle{\pgfqpoint{1.000000in}{0.600000in}}{\pgfqpoint{6.200000in}{4.800000in}}%
\pgfusepath{clip}%
\pgfsetbuttcap%
\pgfsetroundjoin%
\pgfsetlinewidth{0.501875pt}%
\definecolor{currentstroke}{rgb}{0.000000,0.000000,0.000000}%
\pgfsetstrokecolor{currentstroke}%
\pgfsetdash{{1.000000pt}{3.000000pt}}{0.000000pt}%
\pgfpathmoveto{\pgfqpoint{1.000000in}{5.400000in}}%
\pgfpathlineto{\pgfqpoint{7.200000in}{5.400000in}}%
\pgfusepath{stroke}%
\end{pgfscope}%
\begin{pgfscope}%
\pgfsetbuttcap%
\pgfsetroundjoin%
\definecolor{currentfill}{rgb}{0.000000,0.000000,0.000000}%
\pgfsetfillcolor{currentfill}%
\pgfsetlinewidth{0.501875pt}%
\definecolor{currentstroke}{rgb}{0.000000,0.000000,0.000000}%
\pgfsetstrokecolor{currentstroke}%
\pgfsetdash{}{0pt}%
\pgfsys@defobject{currentmarker}{\pgfqpoint{0.000000in}{0.000000in}}{\pgfqpoint{0.055556in}{0.000000in}}{%
\pgfpathmoveto{\pgfqpoint{0.000000in}{0.000000in}}%
\pgfpathlineto{\pgfqpoint{0.055556in}{0.000000in}}%
\pgfusepath{stroke,fill}%
}%
\begin{pgfscope}%
\pgfsys@transformshift{1.000000in}{5.400000in}%
\pgfsys@useobject{currentmarker}{}%
\end{pgfscope}%
\end{pgfscope}%
\begin{pgfscope}%
\pgfsetbuttcap%
\pgfsetroundjoin%
\definecolor{currentfill}{rgb}{0.000000,0.000000,0.000000}%
\pgfsetfillcolor{currentfill}%
\pgfsetlinewidth{0.501875pt}%
\definecolor{currentstroke}{rgb}{0.000000,0.000000,0.000000}%
\pgfsetstrokecolor{currentstroke}%
\pgfsetdash{}{0pt}%
\pgfsys@defobject{currentmarker}{\pgfqpoint{-0.055556in}{0.000000in}}{\pgfqpoint{-0.000000in}{0.000000in}}{%
\pgfpathmoveto{\pgfqpoint{-0.000000in}{0.000000in}}%
\pgfpathlineto{\pgfqpoint{-0.055556in}{0.000000in}}%
\pgfusepath{stroke,fill}%
}%
\begin{pgfscope}%
\pgfsys@transformshift{7.200000in}{5.400000in}%
\pgfsys@useobject{currentmarker}{}%
\end{pgfscope}%
\end{pgfscope}%
\begin{pgfscope}%
\definecolor{textcolor}{rgb}{0.000000,0.000000,0.000000}%
\pgfsetstrokecolor{textcolor}%
\pgfsetfillcolor{textcolor}%
\pgftext[x=0.944444in,y=5.400000in,right,]{\color{textcolor}\sffamily\fontsize{16.000000}{19.200000}\selectfont \(\displaystyle {25}\)}%
\end{pgfscope}%
\begin{pgfscope}%
\definecolor{textcolor}{rgb}{0.000000,0.000000,0.000000}%
\pgfsetstrokecolor{textcolor}%
\pgfsetfillcolor{textcolor}%
\pgftext[x=0.654864in,y=3.000000in,,bottom,rotate=90.000000]{\color{textcolor}\sffamily\fontsize{16.000000}{19.200000}\selectfont \(\displaystyle Voltage/\mathrm{mV}\)}%
\end{pgfscope}%
\end{pgfpicture}%
\makeatother%
\endgroup%
}
    \caption{Single PE response\cite{jetter_pmt_2012}}
\end{figure}
\begin{align*}
  V_\mathrm{PE}(t) &= V_{0}\exp\left[-\frac{1}{2}\left(\frac{\log(t/\tau_\mathrm{PE})}{\sigma_\mathrm{PE}}\right)^{2}\right]
\end{align*}
\end{frame}

\begin{frame}
\frametitle{Data input \& output}
\begin{columns}
\column{0.5\textwidth}
\begin{figure}
    \centering
    \resizebox{1.0\textwidth}{!}{%% Creator: Matplotlib, PGF backend
%%
%% To include the figure in your LaTeX document, write
%%   \input{<filename>.pgf}
%%
%% Make sure the required packages are loaded in your preamble
%%   \usepackage{pgf}
%%
%% and, on pdftex
%%   \usepackage[utf8]{inputenc}\DeclareUnicodeCharacter{2212}{-}
%%
%% or, on luatex and xetex
%%   \usepackage{unicode-math}
%%
%% Figures using additional raster images can only be included by \input if
%% they are in the same directory as the main LaTeX file. For loading figures
%% from other directories you can use the `import` package
%%   \usepackage{import}
%%
%% and then include the figures with
%%   \import{<path to file>}{<filename>.pgf}
%%
%% Matplotlib used the following preamble
%%   \usepackage[detect-all,locale=DE]{siunitx}
%%
\begingroup%
\makeatletter%
\begin{pgfpicture}%
\pgfpathrectangle{\pgfpointorigin}{\pgfqpoint{8.000000in}{6.000000in}}%
\pgfusepath{use as bounding box, clip}%
\begin{pgfscope}%
\pgfsetbuttcap%
\pgfsetmiterjoin%
\definecolor{currentfill}{rgb}{1.000000,1.000000,1.000000}%
\pgfsetfillcolor{currentfill}%
\pgfsetlinewidth{0.000000pt}%
\definecolor{currentstroke}{rgb}{1.000000,1.000000,1.000000}%
\pgfsetstrokecolor{currentstroke}%
\pgfsetdash{}{0pt}%
\pgfpathmoveto{\pgfqpoint{0.000000in}{0.000000in}}%
\pgfpathlineto{\pgfqpoint{8.000000in}{0.000000in}}%
\pgfpathlineto{\pgfqpoint{8.000000in}{6.000000in}}%
\pgfpathlineto{\pgfqpoint{0.000000in}{6.000000in}}%
\pgfpathclose%
\pgfusepath{fill}%
\end{pgfscope}%
\begin{pgfscope}%
\pgfsetbuttcap%
\pgfsetmiterjoin%
\definecolor{currentfill}{rgb}{1.000000,1.000000,1.000000}%
\pgfsetfillcolor{currentfill}%
\pgfsetlinewidth{0.000000pt}%
\definecolor{currentstroke}{rgb}{0.000000,0.000000,0.000000}%
\pgfsetstrokecolor{currentstroke}%
\pgfsetstrokeopacity{0.000000}%
\pgfsetdash{}{0pt}%
\pgfpathmoveto{\pgfqpoint{1.200000in}{0.900000in}}%
\pgfpathlineto{\pgfqpoint{6.800000in}{0.900000in}}%
\pgfpathlineto{\pgfqpoint{6.800000in}{5.700000in}}%
\pgfpathlineto{\pgfqpoint{1.200000in}{5.700000in}}%
\pgfpathclose%
\pgfusepath{fill}%
\end{pgfscope}%
\begin{pgfscope}%
\pgfsetbuttcap%
\pgfsetroundjoin%
\definecolor{currentfill}{rgb}{0.000000,0.000000,0.000000}%
\pgfsetfillcolor{currentfill}%
\pgfsetlinewidth{0.803000pt}%
\definecolor{currentstroke}{rgb}{0.000000,0.000000,0.000000}%
\pgfsetstrokecolor{currentstroke}%
\pgfsetdash{}{0pt}%
\pgfsys@defobject{currentmarker}{\pgfqpoint{0.000000in}{-0.048611in}}{\pgfqpoint{0.000000in}{0.000000in}}{%
\pgfpathmoveto{\pgfqpoint{0.000000in}{0.000000in}}%
\pgfpathlineto{\pgfqpoint{0.000000in}{-0.048611in}}%
\pgfusepath{stroke,fill}%
}%
\begin{pgfscope}%
\pgfsys@transformshift{1.200000in}{0.900000in}%
\pgfsys@useobject{currentmarker}{}%
\end{pgfscope}%
\end{pgfscope}%
\begin{pgfscope}%
\definecolor{textcolor}{rgb}{0.000000,0.000000,0.000000}%
\pgfsetstrokecolor{textcolor}%
\pgfsetfillcolor{textcolor}%
\pgftext[x=1.200000in,y=0.802778in,,top]{\color{textcolor}\sffamily\fontsize{20.000000}{24.000000}\selectfont \(\displaystyle {0}\)}%
\end{pgfscope}%
\begin{pgfscope}%
\pgfsetbuttcap%
\pgfsetroundjoin%
\definecolor{currentfill}{rgb}{0.000000,0.000000,0.000000}%
\pgfsetfillcolor{currentfill}%
\pgfsetlinewidth{0.803000pt}%
\definecolor{currentstroke}{rgb}{0.000000,0.000000,0.000000}%
\pgfsetstrokecolor{currentstroke}%
\pgfsetdash{}{0pt}%
\pgfsys@defobject{currentmarker}{\pgfqpoint{0.000000in}{-0.048611in}}{\pgfqpoint{0.000000in}{0.000000in}}{%
\pgfpathmoveto{\pgfqpoint{0.000000in}{0.000000in}}%
\pgfpathlineto{\pgfqpoint{0.000000in}{-0.048611in}}%
\pgfusepath{stroke,fill}%
}%
\begin{pgfscope}%
\pgfsys@transformshift{2.288435in}{0.900000in}%
\pgfsys@useobject{currentmarker}{}%
\end{pgfscope}%
\end{pgfscope}%
\begin{pgfscope}%
\definecolor{textcolor}{rgb}{0.000000,0.000000,0.000000}%
\pgfsetstrokecolor{textcolor}%
\pgfsetfillcolor{textcolor}%
\pgftext[x=2.288435in,y=0.802778in,,top]{\color{textcolor}\sffamily\fontsize{20.000000}{24.000000}\selectfont \(\displaystyle {200}\)}%
\end{pgfscope}%
\begin{pgfscope}%
\pgfsetbuttcap%
\pgfsetroundjoin%
\definecolor{currentfill}{rgb}{0.000000,0.000000,0.000000}%
\pgfsetfillcolor{currentfill}%
\pgfsetlinewidth{0.803000pt}%
\definecolor{currentstroke}{rgb}{0.000000,0.000000,0.000000}%
\pgfsetstrokecolor{currentstroke}%
\pgfsetdash{}{0pt}%
\pgfsys@defobject{currentmarker}{\pgfqpoint{0.000000in}{-0.048611in}}{\pgfqpoint{0.000000in}{0.000000in}}{%
\pgfpathmoveto{\pgfqpoint{0.000000in}{0.000000in}}%
\pgfpathlineto{\pgfqpoint{0.000000in}{-0.048611in}}%
\pgfusepath{stroke,fill}%
}%
\begin{pgfscope}%
\pgfsys@transformshift{3.376871in}{0.900000in}%
\pgfsys@useobject{currentmarker}{}%
\end{pgfscope}%
\end{pgfscope}%
\begin{pgfscope}%
\definecolor{textcolor}{rgb}{0.000000,0.000000,0.000000}%
\pgfsetstrokecolor{textcolor}%
\pgfsetfillcolor{textcolor}%
\pgftext[x=3.376871in,y=0.802778in,,top]{\color{textcolor}\sffamily\fontsize{20.000000}{24.000000}\selectfont \(\displaystyle {400}\)}%
\end{pgfscope}%
\begin{pgfscope}%
\pgfsetbuttcap%
\pgfsetroundjoin%
\definecolor{currentfill}{rgb}{0.000000,0.000000,0.000000}%
\pgfsetfillcolor{currentfill}%
\pgfsetlinewidth{0.803000pt}%
\definecolor{currentstroke}{rgb}{0.000000,0.000000,0.000000}%
\pgfsetstrokecolor{currentstroke}%
\pgfsetdash{}{0pt}%
\pgfsys@defobject{currentmarker}{\pgfqpoint{0.000000in}{-0.048611in}}{\pgfqpoint{0.000000in}{0.000000in}}{%
\pgfpathmoveto{\pgfqpoint{0.000000in}{0.000000in}}%
\pgfpathlineto{\pgfqpoint{0.000000in}{-0.048611in}}%
\pgfusepath{stroke,fill}%
}%
\begin{pgfscope}%
\pgfsys@transformshift{4.465306in}{0.900000in}%
\pgfsys@useobject{currentmarker}{}%
\end{pgfscope}%
\end{pgfscope}%
\begin{pgfscope}%
\definecolor{textcolor}{rgb}{0.000000,0.000000,0.000000}%
\pgfsetstrokecolor{textcolor}%
\pgfsetfillcolor{textcolor}%
\pgftext[x=4.465306in,y=0.802778in,,top]{\color{textcolor}\sffamily\fontsize{20.000000}{24.000000}\selectfont \(\displaystyle {600}\)}%
\end{pgfscope}%
\begin{pgfscope}%
\pgfsetbuttcap%
\pgfsetroundjoin%
\definecolor{currentfill}{rgb}{0.000000,0.000000,0.000000}%
\pgfsetfillcolor{currentfill}%
\pgfsetlinewidth{0.803000pt}%
\definecolor{currentstroke}{rgb}{0.000000,0.000000,0.000000}%
\pgfsetstrokecolor{currentstroke}%
\pgfsetdash{}{0pt}%
\pgfsys@defobject{currentmarker}{\pgfqpoint{0.000000in}{-0.048611in}}{\pgfqpoint{0.000000in}{0.000000in}}{%
\pgfpathmoveto{\pgfqpoint{0.000000in}{0.000000in}}%
\pgfpathlineto{\pgfqpoint{0.000000in}{-0.048611in}}%
\pgfusepath{stroke,fill}%
}%
\begin{pgfscope}%
\pgfsys@transformshift{5.553741in}{0.900000in}%
\pgfsys@useobject{currentmarker}{}%
\end{pgfscope}%
\end{pgfscope}%
\begin{pgfscope}%
\definecolor{textcolor}{rgb}{0.000000,0.000000,0.000000}%
\pgfsetstrokecolor{textcolor}%
\pgfsetfillcolor{textcolor}%
\pgftext[x=5.553741in,y=0.802778in,,top]{\color{textcolor}\sffamily\fontsize{20.000000}{24.000000}\selectfont \(\displaystyle {800}\)}%
\end{pgfscope}%
\begin{pgfscope}%
\pgfsetbuttcap%
\pgfsetroundjoin%
\definecolor{currentfill}{rgb}{0.000000,0.000000,0.000000}%
\pgfsetfillcolor{currentfill}%
\pgfsetlinewidth{0.803000pt}%
\definecolor{currentstroke}{rgb}{0.000000,0.000000,0.000000}%
\pgfsetstrokecolor{currentstroke}%
\pgfsetdash{}{0pt}%
\pgfsys@defobject{currentmarker}{\pgfqpoint{0.000000in}{-0.048611in}}{\pgfqpoint{0.000000in}{0.000000in}}{%
\pgfpathmoveto{\pgfqpoint{0.000000in}{0.000000in}}%
\pgfpathlineto{\pgfqpoint{0.000000in}{-0.048611in}}%
\pgfusepath{stroke,fill}%
}%
\begin{pgfscope}%
\pgfsys@transformshift{6.642177in}{0.900000in}%
\pgfsys@useobject{currentmarker}{}%
\end{pgfscope}%
\end{pgfscope}%
\begin{pgfscope}%
\definecolor{textcolor}{rgb}{0.000000,0.000000,0.000000}%
\pgfsetstrokecolor{textcolor}%
\pgfsetfillcolor{textcolor}%
\pgftext[x=6.642177in,y=0.802778in,,top]{\color{textcolor}\sffamily\fontsize{20.000000}{24.000000}\selectfont \(\displaystyle {1000}\)}%
\end{pgfscope}%
\begin{pgfscope}%
\definecolor{textcolor}{rgb}{0.000000,0.000000,0.000000}%
\pgfsetstrokecolor{textcolor}%
\pgfsetfillcolor{textcolor}%
\pgftext[x=4.000000in,y=0.491155in,,top]{\color{textcolor}\sffamily\fontsize{20.000000}{24.000000}\selectfont \(\displaystyle \mathrm{t}/\si{ns}\)}%
\end{pgfscope}%
\begin{pgfscope}%
\pgfsetbuttcap%
\pgfsetroundjoin%
\definecolor{currentfill}{rgb}{0.000000,0.000000,0.000000}%
\pgfsetfillcolor{currentfill}%
\pgfsetlinewidth{0.803000pt}%
\definecolor{currentstroke}{rgb}{0.000000,0.000000,0.000000}%
\pgfsetstrokecolor{currentstroke}%
\pgfsetdash{}{0pt}%
\pgfsys@defobject{currentmarker}{\pgfqpoint{-0.048611in}{0.000000in}}{\pgfqpoint{-0.000000in}{0.000000in}}{%
\pgfpathmoveto{\pgfqpoint{-0.000000in}{0.000000in}}%
\pgfpathlineto{\pgfqpoint{-0.048611in}{0.000000in}}%
\pgfusepath{stroke,fill}%
}%
\begin{pgfscope}%
\pgfsys@transformshift{1.200000in}{0.900000in}%
\pgfsys@useobject{currentmarker}{}%
\end{pgfscope}%
\end{pgfscope}%
\begin{pgfscope}%
\definecolor{textcolor}{rgb}{0.000000,0.000000,0.000000}%
\pgfsetstrokecolor{textcolor}%
\pgfsetfillcolor{textcolor}%
\pgftext[x=0.746626in, y=0.799981in, left, base]{\color{textcolor}\sffamily\fontsize{20.000000}{24.000000}\selectfont \(\displaystyle {-5}\)}%
\end{pgfscope}%
\begin{pgfscope}%
\pgfsetbuttcap%
\pgfsetroundjoin%
\definecolor{currentfill}{rgb}{0.000000,0.000000,0.000000}%
\pgfsetfillcolor{currentfill}%
\pgfsetlinewidth{0.803000pt}%
\definecolor{currentstroke}{rgb}{0.000000,0.000000,0.000000}%
\pgfsetstrokecolor{currentstroke}%
\pgfsetdash{}{0pt}%
\pgfsys@defobject{currentmarker}{\pgfqpoint{-0.048611in}{0.000000in}}{\pgfqpoint{-0.000000in}{0.000000in}}{%
\pgfpathmoveto{\pgfqpoint{-0.000000in}{0.000000in}}%
\pgfpathlineto{\pgfqpoint{-0.048611in}{0.000000in}}%
\pgfusepath{stroke,fill}%
}%
\begin{pgfscope}%
\pgfsys@transformshift{1.200000in}{1.783410in}%
\pgfsys@useobject{currentmarker}{}%
\end{pgfscope}%
\end{pgfscope}%
\begin{pgfscope}%
\definecolor{textcolor}{rgb}{0.000000,0.000000,0.000000}%
\pgfsetstrokecolor{textcolor}%
\pgfsetfillcolor{textcolor}%
\pgftext[x=0.970670in, y=1.683391in, left, base]{\color{textcolor}\sffamily\fontsize{20.000000}{24.000000}\selectfont \(\displaystyle {0}\)}%
\end{pgfscope}%
\begin{pgfscope}%
\pgfsetbuttcap%
\pgfsetroundjoin%
\definecolor{currentfill}{rgb}{0.000000,0.000000,0.000000}%
\pgfsetfillcolor{currentfill}%
\pgfsetlinewidth{0.803000pt}%
\definecolor{currentstroke}{rgb}{0.000000,0.000000,0.000000}%
\pgfsetstrokecolor{currentstroke}%
\pgfsetdash{}{0pt}%
\pgfsys@defobject{currentmarker}{\pgfqpoint{-0.048611in}{0.000000in}}{\pgfqpoint{-0.000000in}{0.000000in}}{%
\pgfpathmoveto{\pgfqpoint{-0.000000in}{0.000000in}}%
\pgfpathlineto{\pgfqpoint{-0.048611in}{0.000000in}}%
\pgfusepath{stroke,fill}%
}%
\begin{pgfscope}%
\pgfsys@transformshift{1.200000in}{2.666820in}%
\pgfsys@useobject{currentmarker}{}%
\end{pgfscope}%
\end{pgfscope}%
\begin{pgfscope}%
\definecolor{textcolor}{rgb}{0.000000,0.000000,0.000000}%
\pgfsetstrokecolor{textcolor}%
\pgfsetfillcolor{textcolor}%
\pgftext[x=0.970670in, y=2.566801in, left, base]{\color{textcolor}\sffamily\fontsize{20.000000}{24.000000}\selectfont \(\displaystyle {5}\)}%
\end{pgfscope}%
\begin{pgfscope}%
\pgfsetbuttcap%
\pgfsetroundjoin%
\definecolor{currentfill}{rgb}{0.000000,0.000000,0.000000}%
\pgfsetfillcolor{currentfill}%
\pgfsetlinewidth{0.803000pt}%
\definecolor{currentstroke}{rgb}{0.000000,0.000000,0.000000}%
\pgfsetstrokecolor{currentstroke}%
\pgfsetdash{}{0pt}%
\pgfsys@defobject{currentmarker}{\pgfqpoint{-0.048611in}{0.000000in}}{\pgfqpoint{-0.000000in}{0.000000in}}{%
\pgfpathmoveto{\pgfqpoint{-0.000000in}{0.000000in}}%
\pgfpathlineto{\pgfqpoint{-0.048611in}{0.000000in}}%
\pgfusepath{stroke,fill}%
}%
\begin{pgfscope}%
\pgfsys@transformshift{1.200000in}{3.550230in}%
\pgfsys@useobject{currentmarker}{}%
\end{pgfscope}%
\end{pgfscope}%
\begin{pgfscope}%
\definecolor{textcolor}{rgb}{0.000000,0.000000,0.000000}%
\pgfsetstrokecolor{textcolor}%
\pgfsetfillcolor{textcolor}%
\pgftext[x=0.838563in, y=3.450211in, left, base]{\color{textcolor}\sffamily\fontsize{20.000000}{24.000000}\selectfont \(\displaystyle {10}\)}%
\end{pgfscope}%
\begin{pgfscope}%
\pgfsetbuttcap%
\pgfsetroundjoin%
\definecolor{currentfill}{rgb}{0.000000,0.000000,0.000000}%
\pgfsetfillcolor{currentfill}%
\pgfsetlinewidth{0.803000pt}%
\definecolor{currentstroke}{rgb}{0.000000,0.000000,0.000000}%
\pgfsetstrokecolor{currentstroke}%
\pgfsetdash{}{0pt}%
\pgfsys@defobject{currentmarker}{\pgfqpoint{-0.048611in}{0.000000in}}{\pgfqpoint{-0.000000in}{0.000000in}}{%
\pgfpathmoveto{\pgfqpoint{-0.000000in}{0.000000in}}%
\pgfpathlineto{\pgfqpoint{-0.048611in}{0.000000in}}%
\pgfusepath{stroke,fill}%
}%
\begin{pgfscope}%
\pgfsys@transformshift{1.200000in}{4.433641in}%
\pgfsys@useobject{currentmarker}{}%
\end{pgfscope}%
\end{pgfscope}%
\begin{pgfscope}%
\definecolor{textcolor}{rgb}{0.000000,0.000000,0.000000}%
\pgfsetstrokecolor{textcolor}%
\pgfsetfillcolor{textcolor}%
\pgftext[x=0.838563in, y=4.333621in, left, base]{\color{textcolor}\sffamily\fontsize{20.000000}{24.000000}\selectfont \(\displaystyle {15}\)}%
\end{pgfscope}%
\begin{pgfscope}%
\pgfsetbuttcap%
\pgfsetroundjoin%
\definecolor{currentfill}{rgb}{0.000000,0.000000,0.000000}%
\pgfsetfillcolor{currentfill}%
\pgfsetlinewidth{0.803000pt}%
\definecolor{currentstroke}{rgb}{0.000000,0.000000,0.000000}%
\pgfsetstrokecolor{currentstroke}%
\pgfsetdash{}{0pt}%
\pgfsys@defobject{currentmarker}{\pgfqpoint{-0.048611in}{0.000000in}}{\pgfqpoint{-0.000000in}{0.000000in}}{%
\pgfpathmoveto{\pgfqpoint{-0.000000in}{0.000000in}}%
\pgfpathlineto{\pgfqpoint{-0.048611in}{0.000000in}}%
\pgfusepath{stroke,fill}%
}%
\begin{pgfscope}%
\pgfsys@transformshift{1.200000in}{5.317051in}%
\pgfsys@useobject{currentmarker}{}%
\end{pgfscope}%
\end{pgfscope}%
\begin{pgfscope}%
\definecolor{textcolor}{rgb}{0.000000,0.000000,0.000000}%
\pgfsetstrokecolor{textcolor}%
\pgfsetfillcolor{textcolor}%
\pgftext[x=0.838563in, y=5.217031in, left, base]{\color{textcolor}\sffamily\fontsize{20.000000}{24.000000}\selectfont \(\displaystyle {20}\)}%
\end{pgfscope}%
\begin{pgfscope}%
\definecolor{textcolor}{rgb}{0.000000,0.000000,0.000000}%
\pgfsetstrokecolor{textcolor}%
\pgfsetfillcolor{textcolor}%
\pgftext[x=0.691071in,y=3.300000in,,bottom,rotate=90.000000]{\color{textcolor}\sffamily\fontsize{20.000000}{24.000000}\selectfont \(\displaystyle \mathrm{Voltage}/\si{mV}\)}%
\end{pgfscope}%
\begin{pgfscope}%
\pgfpathrectangle{\pgfqpoint{1.200000in}{0.900000in}}{\pgfqpoint{5.600000in}{4.800000in}}%
\pgfusepath{clip}%
\pgfsetrectcap%
\pgfsetroundjoin%
\pgfsetlinewidth{2.007500pt}%
\definecolor{currentstroke}{rgb}{0.121569,0.466667,0.705882}%
\pgfsetstrokecolor{currentstroke}%
\pgfsetdash{}{0pt}%
\pgfpathmoveto{\pgfqpoint{1.200000in}{1.979283in}}%
\pgfpathlineto{\pgfqpoint{1.205442in}{1.695339in}}%
\pgfpathlineto{\pgfqpoint{1.210884in}{1.719700in}}%
\pgfpathlineto{\pgfqpoint{1.216327in}{1.924295in}}%
\pgfpathlineto{\pgfqpoint{1.221769in}{1.820357in}}%
\pgfpathlineto{\pgfqpoint{1.227211in}{1.492561in}}%
\pgfpathlineto{\pgfqpoint{1.232653in}{1.772670in}}%
\pgfpathlineto{\pgfqpoint{1.238095in}{1.668930in}}%
\pgfpathlineto{\pgfqpoint{1.243537in}{1.854760in}}%
\pgfpathlineto{\pgfqpoint{1.248980in}{1.576971in}}%
\pgfpathlineto{\pgfqpoint{1.254422in}{1.850756in}}%
\pgfpathlineto{\pgfqpoint{1.259864in}{1.637980in}}%
\pgfpathlineto{\pgfqpoint{1.265306in}{1.684425in}}%
\pgfpathlineto{\pgfqpoint{1.270748in}{1.579097in}}%
\pgfpathlineto{\pgfqpoint{1.276190in}{1.510185in}}%
\pgfpathlineto{\pgfqpoint{1.287075in}{2.083070in}}%
\pgfpathlineto{\pgfqpoint{1.292517in}{1.416911in}}%
\pgfpathlineto{\pgfqpoint{1.297959in}{1.805939in}}%
\pgfpathlineto{\pgfqpoint{1.303401in}{1.718234in}}%
\pgfpathlineto{\pgfqpoint{1.308844in}{1.995862in}}%
\pgfpathlineto{\pgfqpoint{1.314286in}{1.526966in}}%
\pgfpathlineto{\pgfqpoint{1.319728in}{1.531205in}}%
\pgfpathlineto{\pgfqpoint{1.325170in}{1.673297in}}%
\pgfpathlineto{\pgfqpoint{1.330612in}{1.704903in}}%
\pgfpathlineto{\pgfqpoint{1.336054in}{1.947065in}}%
\pgfpathlineto{\pgfqpoint{1.341497in}{1.757846in}}%
\pgfpathlineto{\pgfqpoint{1.346939in}{1.696035in}}%
\pgfpathlineto{\pgfqpoint{1.352381in}{1.486158in}}%
\pgfpathlineto{\pgfqpoint{1.357823in}{1.464391in}}%
\pgfpathlineto{\pgfqpoint{1.363265in}{1.522910in}}%
\pgfpathlineto{\pgfqpoint{1.368707in}{1.931408in}}%
\pgfpathlineto{\pgfqpoint{1.374150in}{1.879587in}}%
\pgfpathlineto{\pgfqpoint{1.379592in}{1.690860in}}%
\pgfpathlineto{\pgfqpoint{1.390476in}{1.939254in}}%
\pgfpathlineto{\pgfqpoint{1.401361in}{1.731701in}}%
\pgfpathlineto{\pgfqpoint{1.406803in}{1.827370in}}%
\pgfpathlineto{\pgfqpoint{1.412245in}{1.758938in}}%
\pgfpathlineto{\pgfqpoint{1.417687in}{1.620811in}}%
\pgfpathlineto{\pgfqpoint{1.423129in}{1.965636in}}%
\pgfpathlineto{\pgfqpoint{1.428571in}{1.476188in}}%
\pgfpathlineto{\pgfqpoint{1.434014in}{2.051536in}}%
\pgfpathlineto{\pgfqpoint{1.439456in}{1.561402in}}%
\pgfpathlineto{\pgfqpoint{1.444898in}{1.740187in}}%
\pgfpathlineto{\pgfqpoint{1.450340in}{1.720519in}}%
\pgfpathlineto{\pgfqpoint{1.455782in}{1.963505in}}%
\pgfpathlineto{\pgfqpoint{1.461224in}{1.602992in}}%
\pgfpathlineto{\pgfqpoint{1.466667in}{1.997041in}}%
\pgfpathlineto{\pgfqpoint{1.472109in}{1.226090in}}%
\pgfpathlineto{\pgfqpoint{1.477551in}{1.641599in}}%
\pgfpathlineto{\pgfqpoint{1.482993in}{1.570455in}}%
\pgfpathlineto{\pgfqpoint{1.488435in}{1.702971in}}%
\pgfpathlineto{\pgfqpoint{1.493878in}{1.756200in}}%
\pgfpathlineto{\pgfqpoint{1.499320in}{1.735776in}}%
\pgfpathlineto{\pgfqpoint{1.504762in}{1.766354in}}%
\pgfpathlineto{\pgfqpoint{1.510204in}{2.049658in}}%
\pgfpathlineto{\pgfqpoint{1.515646in}{1.695093in}}%
\pgfpathlineto{\pgfqpoint{1.521088in}{1.776885in}}%
\pgfpathlineto{\pgfqpoint{1.526531in}{1.629972in}}%
\pgfpathlineto{\pgfqpoint{1.531973in}{1.831162in}}%
\pgfpathlineto{\pgfqpoint{1.537415in}{1.634072in}}%
\pgfpathlineto{\pgfqpoint{1.542857in}{1.545320in}}%
\pgfpathlineto{\pgfqpoint{1.548299in}{1.698702in}}%
\pgfpathlineto{\pgfqpoint{1.553741in}{1.686963in}}%
\pgfpathlineto{\pgfqpoint{1.559184in}{1.738350in}}%
\pgfpathlineto{\pgfqpoint{1.564626in}{1.827738in}}%
\pgfpathlineto{\pgfqpoint{1.570068in}{1.503114in}}%
\pgfpathlineto{\pgfqpoint{1.575510in}{1.904935in}}%
\pgfpathlineto{\pgfqpoint{1.580952in}{2.187381in}}%
\pgfpathlineto{\pgfqpoint{1.586395in}{1.684342in}}%
\pgfpathlineto{\pgfqpoint{1.591837in}{1.624325in}}%
\pgfpathlineto{\pgfqpoint{1.597279in}{1.917998in}}%
\pgfpathlineto{\pgfqpoint{1.602721in}{2.007597in}}%
\pgfpathlineto{\pgfqpoint{1.608163in}{1.847789in}}%
\pgfpathlineto{\pgfqpoint{1.613605in}{1.907505in}}%
\pgfpathlineto{\pgfqpoint{1.624490in}{1.668407in}}%
\pgfpathlineto{\pgfqpoint{1.629932in}{1.820817in}}%
\pgfpathlineto{\pgfqpoint{1.635374in}{1.712892in}}%
\pgfpathlineto{\pgfqpoint{1.640816in}{1.676532in}}%
\pgfpathlineto{\pgfqpoint{1.646259in}{1.890994in}}%
\pgfpathlineto{\pgfqpoint{1.651701in}{1.838373in}}%
\pgfpathlineto{\pgfqpoint{1.657143in}{1.900383in}}%
\pgfpathlineto{\pgfqpoint{1.662585in}{1.758158in}}%
\pgfpathlineto{\pgfqpoint{1.668027in}{1.834013in}}%
\pgfpathlineto{\pgfqpoint{1.673469in}{1.984306in}}%
\pgfpathlineto{\pgfqpoint{1.678912in}{1.570509in}}%
\pgfpathlineto{\pgfqpoint{1.684354in}{1.826555in}}%
\pgfpathlineto{\pgfqpoint{1.689796in}{1.670938in}}%
\pgfpathlineto{\pgfqpoint{1.695238in}{1.965451in}}%
\pgfpathlineto{\pgfqpoint{1.700680in}{1.787291in}}%
\pgfpathlineto{\pgfqpoint{1.706122in}{1.928548in}}%
\pgfpathlineto{\pgfqpoint{1.711565in}{1.946149in}}%
\pgfpathlineto{\pgfqpoint{1.717007in}{1.728568in}}%
\pgfpathlineto{\pgfqpoint{1.722449in}{1.729026in}}%
\pgfpathlineto{\pgfqpoint{1.727891in}{1.912675in}}%
\pgfpathlineto{\pgfqpoint{1.733333in}{1.718300in}}%
\pgfpathlineto{\pgfqpoint{1.738776in}{1.462704in}}%
\pgfpathlineto{\pgfqpoint{1.744218in}{2.025964in}}%
\pgfpathlineto{\pgfqpoint{1.749660in}{1.711349in}}%
\pgfpathlineto{\pgfqpoint{1.755102in}{1.658805in}}%
\pgfpathlineto{\pgfqpoint{1.760544in}{1.863540in}}%
\pgfpathlineto{\pgfqpoint{1.765986in}{1.448395in}}%
\pgfpathlineto{\pgfqpoint{1.771429in}{1.881949in}}%
\pgfpathlineto{\pgfqpoint{1.776871in}{2.112689in}}%
\pgfpathlineto{\pgfqpoint{1.782313in}{1.707684in}}%
\pgfpathlineto{\pgfqpoint{1.787755in}{2.033010in}}%
\pgfpathlineto{\pgfqpoint{1.793197in}{1.604148in}}%
\pgfpathlineto{\pgfqpoint{1.798639in}{1.619787in}}%
\pgfpathlineto{\pgfqpoint{1.804082in}{1.782777in}}%
\pgfpathlineto{\pgfqpoint{1.809524in}{1.827452in}}%
\pgfpathlineto{\pgfqpoint{1.814966in}{1.985369in}}%
\pgfpathlineto{\pgfqpoint{1.820408in}{1.481011in}}%
\pgfpathlineto{\pgfqpoint{1.825850in}{1.792652in}}%
\pgfpathlineto{\pgfqpoint{1.831293in}{1.490345in}}%
\pgfpathlineto{\pgfqpoint{1.836735in}{1.639224in}}%
\pgfpathlineto{\pgfqpoint{1.842177in}{1.978112in}}%
\pgfpathlineto{\pgfqpoint{1.847619in}{1.748231in}}%
\pgfpathlineto{\pgfqpoint{1.853061in}{1.703761in}}%
\pgfpathlineto{\pgfqpoint{1.858503in}{1.561459in}}%
\pgfpathlineto{\pgfqpoint{1.863946in}{1.545335in}}%
\pgfpathlineto{\pgfqpoint{1.869388in}{1.774845in}}%
\pgfpathlineto{\pgfqpoint{1.874830in}{1.758493in}}%
\pgfpathlineto{\pgfqpoint{1.880272in}{1.530434in}}%
\pgfpathlineto{\pgfqpoint{1.891156in}{2.111686in}}%
\pgfpathlineto{\pgfqpoint{1.896599in}{1.775530in}}%
\pgfpathlineto{\pgfqpoint{1.902041in}{1.704127in}}%
\pgfpathlineto{\pgfqpoint{1.907483in}{2.063434in}}%
\pgfpathlineto{\pgfqpoint{1.912925in}{1.919975in}}%
\pgfpathlineto{\pgfqpoint{1.918367in}{1.610632in}}%
\pgfpathlineto{\pgfqpoint{1.923810in}{1.822633in}}%
\pgfpathlineto{\pgfqpoint{1.929252in}{1.850214in}}%
\pgfpathlineto{\pgfqpoint{1.934694in}{1.660517in}}%
\pgfpathlineto{\pgfqpoint{1.940136in}{1.701763in}}%
\pgfpathlineto{\pgfqpoint{1.945578in}{1.858818in}}%
\pgfpathlineto{\pgfqpoint{1.951020in}{1.545372in}}%
\pgfpathlineto{\pgfqpoint{1.956463in}{1.895373in}}%
\pgfpathlineto{\pgfqpoint{1.961905in}{1.757015in}}%
\pgfpathlineto{\pgfqpoint{1.967347in}{1.588444in}}%
\pgfpathlineto{\pgfqpoint{1.972789in}{1.622603in}}%
\pgfpathlineto{\pgfqpoint{1.978231in}{2.075115in}}%
\pgfpathlineto{\pgfqpoint{1.983673in}{1.822258in}}%
\pgfpathlineto{\pgfqpoint{1.989116in}{1.871075in}}%
\pgfpathlineto{\pgfqpoint{1.994558in}{1.666446in}}%
\pgfpathlineto{\pgfqpoint{2.000000in}{1.665657in}}%
\pgfpathlineto{\pgfqpoint{2.005442in}{1.856406in}}%
\pgfpathlineto{\pgfqpoint{2.010884in}{1.893479in}}%
\pgfpathlineto{\pgfqpoint{2.016327in}{1.688617in}}%
\pgfpathlineto{\pgfqpoint{2.021769in}{1.736716in}}%
\pgfpathlineto{\pgfqpoint{2.027211in}{1.919044in}}%
\pgfpathlineto{\pgfqpoint{2.032653in}{1.622250in}}%
\pgfpathlineto{\pgfqpoint{2.038095in}{1.647201in}}%
\pgfpathlineto{\pgfqpoint{2.043537in}{1.580079in}}%
\pgfpathlineto{\pgfqpoint{2.048980in}{1.996115in}}%
\pgfpathlineto{\pgfqpoint{2.054422in}{1.751308in}}%
\pgfpathlineto{\pgfqpoint{2.059864in}{1.704676in}}%
\pgfpathlineto{\pgfqpoint{2.065306in}{1.638220in}}%
\pgfpathlineto{\pgfqpoint{2.070748in}{1.948830in}}%
\pgfpathlineto{\pgfqpoint{2.076190in}{1.783007in}}%
\pgfpathlineto{\pgfqpoint{2.081633in}{1.546692in}}%
\pgfpathlineto{\pgfqpoint{2.087075in}{1.804991in}}%
\pgfpathlineto{\pgfqpoint{2.092517in}{1.514539in}}%
\pgfpathlineto{\pgfqpoint{2.097959in}{1.916023in}}%
\pgfpathlineto{\pgfqpoint{2.103401in}{1.826227in}}%
\pgfpathlineto{\pgfqpoint{2.108844in}{1.699967in}}%
\pgfpathlineto{\pgfqpoint{2.114286in}{1.726404in}}%
\pgfpathlineto{\pgfqpoint{2.119728in}{1.984874in}}%
\pgfpathlineto{\pgfqpoint{2.125170in}{1.735640in}}%
\pgfpathlineto{\pgfqpoint{2.130612in}{1.897221in}}%
\pgfpathlineto{\pgfqpoint{2.136054in}{1.866629in}}%
\pgfpathlineto{\pgfqpoint{2.141497in}{2.108709in}}%
\pgfpathlineto{\pgfqpoint{2.146939in}{1.985243in}}%
\pgfpathlineto{\pgfqpoint{2.152381in}{1.677903in}}%
\pgfpathlineto{\pgfqpoint{2.157823in}{1.840518in}}%
\pgfpathlineto{\pgfqpoint{2.163265in}{1.951770in}}%
\pgfpathlineto{\pgfqpoint{2.174150in}{1.551367in}}%
\pgfpathlineto{\pgfqpoint{2.185034in}{1.626853in}}%
\pgfpathlineto{\pgfqpoint{2.190476in}{1.811237in}}%
\pgfpathlineto{\pgfqpoint{2.195918in}{1.917443in}}%
\pgfpathlineto{\pgfqpoint{2.201361in}{1.862307in}}%
\pgfpathlineto{\pgfqpoint{2.206803in}{1.588643in}}%
\pgfpathlineto{\pgfqpoint{2.212245in}{1.917399in}}%
\pgfpathlineto{\pgfqpoint{2.217687in}{2.026770in}}%
\pgfpathlineto{\pgfqpoint{2.223129in}{1.774608in}}%
\pgfpathlineto{\pgfqpoint{2.228571in}{1.903505in}}%
\pgfpathlineto{\pgfqpoint{2.234014in}{1.818907in}}%
\pgfpathlineto{\pgfqpoint{2.239456in}{1.822023in}}%
\pgfpathlineto{\pgfqpoint{2.244898in}{1.652241in}}%
\pgfpathlineto{\pgfqpoint{2.250340in}{1.589699in}}%
\pgfpathlineto{\pgfqpoint{2.255782in}{2.097160in}}%
\pgfpathlineto{\pgfqpoint{2.261224in}{1.603594in}}%
\pgfpathlineto{\pgfqpoint{2.272109in}{2.078286in}}%
\pgfpathlineto{\pgfqpoint{2.277551in}{1.783032in}}%
\pgfpathlineto{\pgfqpoint{2.282993in}{1.747030in}}%
\pgfpathlineto{\pgfqpoint{2.288435in}{2.190727in}}%
\pgfpathlineto{\pgfqpoint{2.293878in}{1.742793in}}%
\pgfpathlineto{\pgfqpoint{2.299320in}{2.352129in}}%
\pgfpathlineto{\pgfqpoint{2.304762in}{2.701418in}}%
\pgfpathlineto{\pgfqpoint{2.310204in}{2.933520in}}%
\pgfpathlineto{\pgfqpoint{2.315646in}{3.092161in}}%
\pgfpathlineto{\pgfqpoint{2.321088in}{3.145233in}}%
\pgfpathlineto{\pgfqpoint{2.326531in}{3.306200in}}%
\pgfpathlineto{\pgfqpoint{2.331973in}{3.367952in}}%
\pgfpathlineto{\pgfqpoint{2.342857in}{3.058658in}}%
\pgfpathlineto{\pgfqpoint{2.353741in}{2.564900in}}%
\pgfpathlineto{\pgfqpoint{2.359184in}{2.772920in}}%
\pgfpathlineto{\pgfqpoint{2.364626in}{2.208790in}}%
\pgfpathlineto{\pgfqpoint{2.370068in}{2.474734in}}%
\pgfpathlineto{\pgfqpoint{2.375510in}{2.981284in}}%
\pgfpathlineto{\pgfqpoint{2.380952in}{3.060079in}}%
\pgfpathlineto{\pgfqpoint{2.386395in}{3.858859in}}%
\pgfpathlineto{\pgfqpoint{2.391837in}{4.089520in}}%
\pgfpathlineto{\pgfqpoint{2.397279in}{4.121870in}}%
\pgfpathlineto{\pgfqpoint{2.402721in}{4.490494in}}%
\pgfpathlineto{\pgfqpoint{2.408163in}{3.719897in}}%
\pgfpathlineto{\pgfqpoint{2.413605in}{3.867970in}}%
\pgfpathlineto{\pgfqpoint{2.419048in}{3.869650in}}%
\pgfpathlineto{\pgfqpoint{2.424490in}{3.413245in}}%
\pgfpathlineto{\pgfqpoint{2.429932in}{4.131652in}}%
\pgfpathlineto{\pgfqpoint{2.435374in}{5.000356in}}%
\pgfpathlineto{\pgfqpoint{2.440816in}{4.928455in}}%
\pgfpathlineto{\pgfqpoint{2.446259in}{4.996771in}}%
\pgfpathlineto{\pgfqpoint{2.451701in}{5.039960in}}%
\pgfpathlineto{\pgfqpoint{2.457143in}{4.896405in}}%
\pgfpathlineto{\pgfqpoint{2.462585in}{5.183540in}}%
\pgfpathlineto{\pgfqpoint{2.468027in}{5.309334in}}%
\pgfpathlineto{\pgfqpoint{2.473469in}{4.984144in}}%
\pgfpathlineto{\pgfqpoint{2.484354in}{4.743014in}}%
\pgfpathlineto{\pgfqpoint{2.489796in}{4.524422in}}%
\pgfpathlineto{\pgfqpoint{2.495238in}{4.185034in}}%
\pgfpathlineto{\pgfqpoint{2.500680in}{4.015793in}}%
\pgfpathlineto{\pgfqpoint{2.511565in}{3.004540in}}%
\pgfpathlineto{\pgfqpoint{2.517007in}{2.943924in}}%
\pgfpathlineto{\pgfqpoint{2.522449in}{3.159047in}}%
\pgfpathlineto{\pgfqpoint{2.527891in}{2.787470in}}%
\pgfpathlineto{\pgfqpoint{2.533333in}{2.604850in}}%
\pgfpathlineto{\pgfqpoint{2.538776in}{2.206762in}}%
\pgfpathlineto{\pgfqpoint{2.544218in}{2.046687in}}%
\pgfpathlineto{\pgfqpoint{2.549660in}{2.100676in}}%
\pgfpathlineto{\pgfqpoint{2.555102in}{1.896931in}}%
\pgfpathlineto{\pgfqpoint{2.560544in}{2.348216in}}%
\pgfpathlineto{\pgfqpoint{2.565986in}{2.624977in}}%
\pgfpathlineto{\pgfqpoint{2.571429in}{2.993987in}}%
\pgfpathlineto{\pgfqpoint{2.576871in}{3.098927in}}%
\pgfpathlineto{\pgfqpoint{2.582313in}{2.979001in}}%
\pgfpathlineto{\pgfqpoint{2.587755in}{3.369298in}}%
\pgfpathlineto{\pgfqpoint{2.593197in}{3.223998in}}%
\pgfpathlineto{\pgfqpoint{2.598639in}{2.984775in}}%
\pgfpathlineto{\pgfqpoint{2.604082in}{2.934239in}}%
\pgfpathlineto{\pgfqpoint{2.609524in}{2.829162in}}%
\pgfpathlineto{\pgfqpoint{2.614966in}{2.635457in}}%
\pgfpathlineto{\pgfqpoint{2.620408in}{2.563415in}}%
\pgfpathlineto{\pgfqpoint{2.625850in}{2.145214in}}%
\pgfpathlineto{\pgfqpoint{2.631293in}{1.917275in}}%
\pgfpathlineto{\pgfqpoint{2.636735in}{2.219081in}}%
\pgfpathlineto{\pgfqpoint{2.642177in}{2.207550in}}%
\pgfpathlineto{\pgfqpoint{2.647619in}{2.147777in}}%
\pgfpathlineto{\pgfqpoint{2.653061in}{1.968336in}}%
\pgfpathlineto{\pgfqpoint{2.658503in}{1.958366in}}%
\pgfpathlineto{\pgfqpoint{2.663946in}{2.193915in}}%
\pgfpathlineto{\pgfqpoint{2.669388in}{1.721651in}}%
\pgfpathlineto{\pgfqpoint{2.674830in}{1.966963in}}%
\pgfpathlineto{\pgfqpoint{2.680272in}{1.789227in}}%
\pgfpathlineto{\pgfqpoint{2.685714in}{1.767736in}}%
\pgfpathlineto{\pgfqpoint{2.691156in}{1.902512in}}%
\pgfpathlineto{\pgfqpoint{2.696599in}{1.983389in}}%
\pgfpathlineto{\pgfqpoint{2.702041in}{1.718912in}}%
\pgfpathlineto{\pgfqpoint{2.707483in}{1.923058in}}%
\pgfpathlineto{\pgfqpoint{2.712925in}{1.992067in}}%
\pgfpathlineto{\pgfqpoint{2.718367in}{1.938508in}}%
\pgfpathlineto{\pgfqpoint{2.723810in}{1.750457in}}%
\pgfpathlineto{\pgfqpoint{2.729252in}{1.705972in}}%
\pgfpathlineto{\pgfqpoint{2.734694in}{1.388498in}}%
\pgfpathlineto{\pgfqpoint{2.740136in}{1.851709in}}%
\pgfpathlineto{\pgfqpoint{2.745578in}{1.905283in}}%
\pgfpathlineto{\pgfqpoint{2.751020in}{1.820148in}}%
\pgfpathlineto{\pgfqpoint{2.756463in}{1.554337in}}%
\pgfpathlineto{\pgfqpoint{2.761905in}{1.929069in}}%
\pgfpathlineto{\pgfqpoint{2.767347in}{1.506610in}}%
\pgfpathlineto{\pgfqpoint{2.772789in}{1.614665in}}%
\pgfpathlineto{\pgfqpoint{2.778231in}{1.784370in}}%
\pgfpathlineto{\pgfqpoint{2.783673in}{1.859514in}}%
\pgfpathlineto{\pgfqpoint{2.789116in}{1.670660in}}%
\pgfpathlineto{\pgfqpoint{2.794558in}{1.814068in}}%
\pgfpathlineto{\pgfqpoint{2.800000in}{1.781537in}}%
\pgfpathlineto{\pgfqpoint{2.805442in}{1.653106in}}%
\pgfpathlineto{\pgfqpoint{2.810884in}{1.820904in}}%
\pgfpathlineto{\pgfqpoint{2.816327in}{1.852340in}}%
\pgfpathlineto{\pgfqpoint{2.821769in}{1.797255in}}%
\pgfpathlineto{\pgfqpoint{2.827211in}{1.267138in}}%
\pgfpathlineto{\pgfqpoint{2.832653in}{1.952688in}}%
\pgfpathlineto{\pgfqpoint{2.838095in}{1.753806in}}%
\pgfpathlineto{\pgfqpoint{2.843537in}{2.002239in}}%
\pgfpathlineto{\pgfqpoint{2.848980in}{1.848952in}}%
\pgfpathlineto{\pgfqpoint{2.854422in}{1.911335in}}%
\pgfpathlineto{\pgfqpoint{2.859864in}{1.823021in}}%
\pgfpathlineto{\pgfqpoint{2.865306in}{1.876582in}}%
\pgfpathlineto{\pgfqpoint{2.870748in}{1.666940in}}%
\pgfpathlineto{\pgfqpoint{2.876190in}{1.745509in}}%
\pgfpathlineto{\pgfqpoint{2.881633in}{2.025857in}}%
\pgfpathlineto{\pgfqpoint{2.887075in}{1.840584in}}%
\pgfpathlineto{\pgfqpoint{2.892517in}{1.749089in}}%
\pgfpathlineto{\pgfqpoint{2.897959in}{1.591557in}}%
\pgfpathlineto{\pgfqpoint{2.903401in}{1.899569in}}%
\pgfpathlineto{\pgfqpoint{2.908844in}{1.854900in}}%
\pgfpathlineto{\pgfqpoint{2.914286in}{1.856013in}}%
\pgfpathlineto{\pgfqpoint{2.919728in}{1.645835in}}%
\pgfpathlineto{\pgfqpoint{2.925170in}{1.753058in}}%
\pgfpathlineto{\pgfqpoint{2.930612in}{1.983183in}}%
\pgfpathlineto{\pgfqpoint{2.936054in}{1.603902in}}%
\pgfpathlineto{\pgfqpoint{2.941497in}{2.172181in}}%
\pgfpathlineto{\pgfqpoint{2.946939in}{1.739624in}}%
\pgfpathlineto{\pgfqpoint{2.952381in}{1.807096in}}%
\pgfpathlineto{\pgfqpoint{2.957823in}{1.732785in}}%
\pgfpathlineto{\pgfqpoint{2.963265in}{1.742487in}}%
\pgfpathlineto{\pgfqpoint{2.968707in}{1.553793in}}%
\pgfpathlineto{\pgfqpoint{2.974150in}{1.749258in}}%
\pgfpathlineto{\pgfqpoint{2.979592in}{1.866076in}}%
\pgfpathlineto{\pgfqpoint{2.985034in}{1.951270in}}%
\pgfpathlineto{\pgfqpoint{2.995918in}{1.673284in}}%
\pgfpathlineto{\pgfqpoint{3.001361in}{1.664625in}}%
\pgfpathlineto{\pgfqpoint{3.006803in}{1.845253in}}%
\pgfpathlineto{\pgfqpoint{3.012245in}{1.742232in}}%
\pgfpathlineto{\pgfqpoint{3.017687in}{1.965945in}}%
\pgfpathlineto{\pgfqpoint{3.023129in}{2.120761in}}%
\pgfpathlineto{\pgfqpoint{3.028571in}{1.770527in}}%
\pgfpathlineto{\pgfqpoint{3.034014in}{1.696841in}}%
\pgfpathlineto{\pgfqpoint{3.039456in}{1.995715in}}%
\pgfpathlineto{\pgfqpoint{3.044898in}{1.455366in}}%
\pgfpathlineto{\pgfqpoint{3.050340in}{1.862567in}}%
\pgfpathlineto{\pgfqpoint{3.055782in}{1.820621in}}%
\pgfpathlineto{\pgfqpoint{3.061224in}{1.789776in}}%
\pgfpathlineto{\pgfqpoint{3.066667in}{1.885017in}}%
\pgfpathlineto{\pgfqpoint{3.072109in}{1.786004in}}%
\pgfpathlineto{\pgfqpoint{3.077551in}{1.795321in}}%
\pgfpathlineto{\pgfqpoint{3.082993in}{1.693766in}}%
\pgfpathlineto{\pgfqpoint{3.088435in}{1.858680in}}%
\pgfpathlineto{\pgfqpoint{3.093878in}{1.798417in}}%
\pgfpathlineto{\pgfqpoint{3.099320in}{1.838757in}}%
\pgfpathlineto{\pgfqpoint{3.104762in}{1.589019in}}%
\pgfpathlineto{\pgfqpoint{3.110204in}{2.049515in}}%
\pgfpathlineto{\pgfqpoint{3.115646in}{2.041407in}}%
\pgfpathlineto{\pgfqpoint{3.121088in}{1.914532in}}%
\pgfpathlineto{\pgfqpoint{3.126531in}{1.943369in}}%
\pgfpathlineto{\pgfqpoint{3.131973in}{1.933914in}}%
\pgfpathlineto{\pgfqpoint{3.137415in}{1.461695in}}%
\pgfpathlineto{\pgfqpoint{3.142857in}{1.844800in}}%
\pgfpathlineto{\pgfqpoint{3.148299in}{1.812667in}}%
\pgfpathlineto{\pgfqpoint{3.153741in}{1.744037in}}%
\pgfpathlineto{\pgfqpoint{3.159184in}{1.839442in}}%
\pgfpathlineto{\pgfqpoint{3.164626in}{2.030909in}}%
\pgfpathlineto{\pgfqpoint{3.170068in}{1.954555in}}%
\pgfpathlineto{\pgfqpoint{3.175510in}{1.950309in}}%
\pgfpathlineto{\pgfqpoint{3.180952in}{1.865799in}}%
\pgfpathlineto{\pgfqpoint{3.186395in}{1.588203in}}%
\pgfpathlineto{\pgfqpoint{3.197279in}{1.833062in}}%
\pgfpathlineto{\pgfqpoint{3.202721in}{1.930605in}}%
\pgfpathlineto{\pgfqpoint{3.208163in}{1.529638in}}%
\pgfpathlineto{\pgfqpoint{3.213605in}{1.468205in}}%
\pgfpathlineto{\pgfqpoint{3.219048in}{1.923120in}}%
\pgfpathlineto{\pgfqpoint{3.224490in}{1.992036in}}%
\pgfpathlineto{\pgfqpoint{3.229932in}{1.750976in}}%
\pgfpathlineto{\pgfqpoint{3.235374in}{1.834455in}}%
\pgfpathlineto{\pgfqpoint{3.240816in}{1.775630in}}%
\pgfpathlineto{\pgfqpoint{3.246259in}{1.684285in}}%
\pgfpathlineto{\pgfqpoint{3.251701in}{1.722783in}}%
\pgfpathlineto{\pgfqpoint{3.257143in}{1.692271in}}%
\pgfpathlineto{\pgfqpoint{3.262585in}{1.732377in}}%
\pgfpathlineto{\pgfqpoint{3.268027in}{1.724517in}}%
\pgfpathlineto{\pgfqpoint{3.273469in}{1.733512in}}%
\pgfpathlineto{\pgfqpoint{3.278912in}{1.829387in}}%
\pgfpathlineto{\pgfqpoint{3.284354in}{1.954204in}}%
\pgfpathlineto{\pgfqpoint{3.289796in}{1.684193in}}%
\pgfpathlineto{\pgfqpoint{3.295238in}{1.803404in}}%
\pgfpathlineto{\pgfqpoint{3.300680in}{1.593469in}}%
\pgfpathlineto{\pgfqpoint{3.306122in}{1.659636in}}%
\pgfpathlineto{\pgfqpoint{3.311565in}{1.893394in}}%
\pgfpathlineto{\pgfqpoint{3.317007in}{1.880778in}}%
\pgfpathlineto{\pgfqpoint{3.322449in}{1.560520in}}%
\pgfpathlineto{\pgfqpoint{3.327891in}{2.001050in}}%
\pgfpathlineto{\pgfqpoint{3.333333in}{1.952717in}}%
\pgfpathlineto{\pgfqpoint{3.338776in}{1.807111in}}%
\pgfpathlineto{\pgfqpoint{3.344218in}{1.588843in}}%
\pgfpathlineto{\pgfqpoint{3.349660in}{1.901147in}}%
\pgfpathlineto{\pgfqpoint{3.355102in}{2.007826in}}%
\pgfpathlineto{\pgfqpoint{3.360544in}{1.722897in}}%
\pgfpathlineto{\pgfqpoint{3.365986in}{1.950192in}}%
\pgfpathlineto{\pgfqpoint{3.371429in}{1.734123in}}%
\pgfpathlineto{\pgfqpoint{3.376871in}{1.632192in}}%
\pgfpathlineto{\pgfqpoint{3.382313in}{1.980217in}}%
\pgfpathlineto{\pgfqpoint{3.387755in}{1.588414in}}%
\pgfpathlineto{\pgfqpoint{3.393197in}{1.597627in}}%
\pgfpathlineto{\pgfqpoint{3.398639in}{2.035589in}}%
\pgfpathlineto{\pgfqpoint{3.404082in}{1.690706in}}%
\pgfpathlineto{\pgfqpoint{3.409524in}{1.866079in}}%
\pgfpathlineto{\pgfqpoint{3.414966in}{1.836371in}}%
\pgfpathlineto{\pgfqpoint{3.420408in}{1.865197in}}%
\pgfpathlineto{\pgfqpoint{3.425850in}{1.700474in}}%
\pgfpathlineto{\pgfqpoint{3.431293in}{1.779299in}}%
\pgfpathlineto{\pgfqpoint{3.436735in}{1.767580in}}%
\pgfpathlineto{\pgfqpoint{3.442177in}{1.923743in}}%
\pgfpathlineto{\pgfqpoint{3.447619in}{1.671061in}}%
\pgfpathlineto{\pgfqpoint{3.453061in}{1.472092in}}%
\pgfpathlineto{\pgfqpoint{3.458503in}{1.760145in}}%
\pgfpathlineto{\pgfqpoint{3.463946in}{1.536510in}}%
\pgfpathlineto{\pgfqpoint{3.469388in}{1.853545in}}%
\pgfpathlineto{\pgfqpoint{3.474830in}{1.946352in}}%
\pgfpathlineto{\pgfqpoint{3.480272in}{1.894719in}}%
\pgfpathlineto{\pgfqpoint{3.485714in}{1.893351in}}%
\pgfpathlineto{\pgfqpoint{3.491156in}{2.019431in}}%
\pgfpathlineto{\pgfqpoint{3.496599in}{1.814944in}}%
\pgfpathlineto{\pgfqpoint{3.502041in}{2.014383in}}%
\pgfpathlineto{\pgfqpoint{3.507483in}{1.795771in}}%
\pgfpathlineto{\pgfqpoint{3.512925in}{1.911687in}}%
\pgfpathlineto{\pgfqpoint{3.518367in}{1.573309in}}%
\pgfpathlineto{\pgfqpoint{3.523810in}{1.981126in}}%
\pgfpathlineto{\pgfqpoint{3.529252in}{1.887628in}}%
\pgfpathlineto{\pgfqpoint{3.534694in}{1.457992in}}%
\pgfpathlineto{\pgfqpoint{3.540136in}{1.692628in}}%
\pgfpathlineto{\pgfqpoint{3.545578in}{1.519529in}}%
\pgfpathlineto{\pgfqpoint{3.551020in}{2.014704in}}%
\pgfpathlineto{\pgfqpoint{3.556463in}{1.681092in}}%
\pgfpathlineto{\pgfqpoint{3.561905in}{2.055936in}}%
\pgfpathlineto{\pgfqpoint{3.567347in}{1.627046in}}%
\pgfpathlineto{\pgfqpoint{3.572789in}{1.943184in}}%
\pgfpathlineto{\pgfqpoint{3.578231in}{2.136487in}}%
\pgfpathlineto{\pgfqpoint{3.583673in}{1.740158in}}%
\pgfpathlineto{\pgfqpoint{3.589116in}{1.849825in}}%
\pgfpathlineto{\pgfqpoint{3.594558in}{1.761589in}}%
\pgfpathlineto{\pgfqpoint{3.600000in}{1.594731in}}%
\pgfpathlineto{\pgfqpoint{3.605442in}{1.566758in}}%
\pgfpathlineto{\pgfqpoint{3.610884in}{2.092018in}}%
\pgfpathlineto{\pgfqpoint{3.616327in}{1.779202in}}%
\pgfpathlineto{\pgfqpoint{3.621769in}{1.947106in}}%
\pgfpathlineto{\pgfqpoint{3.627211in}{2.031798in}}%
\pgfpathlineto{\pgfqpoint{3.632653in}{1.442190in}}%
\pgfpathlineto{\pgfqpoint{3.638095in}{1.867939in}}%
\pgfpathlineto{\pgfqpoint{3.643537in}{1.281228in}}%
\pgfpathlineto{\pgfqpoint{3.648980in}{2.025270in}}%
\pgfpathlineto{\pgfqpoint{3.654422in}{1.409013in}}%
\pgfpathlineto{\pgfqpoint{3.659864in}{1.516245in}}%
\pgfpathlineto{\pgfqpoint{3.665306in}{1.740548in}}%
\pgfpathlineto{\pgfqpoint{3.670748in}{1.614871in}}%
\pgfpathlineto{\pgfqpoint{3.676190in}{2.141898in}}%
\pgfpathlineto{\pgfqpoint{3.681633in}{1.971024in}}%
\pgfpathlineto{\pgfqpoint{3.687075in}{1.947230in}}%
\pgfpathlineto{\pgfqpoint{3.692517in}{1.841543in}}%
\pgfpathlineto{\pgfqpoint{3.697959in}{1.934628in}}%
\pgfpathlineto{\pgfqpoint{3.708844in}{1.653073in}}%
\pgfpathlineto{\pgfqpoint{3.719728in}{1.990383in}}%
\pgfpathlineto{\pgfqpoint{3.725170in}{1.647209in}}%
\pgfpathlineto{\pgfqpoint{3.730612in}{1.587311in}}%
\pgfpathlineto{\pgfqpoint{3.736054in}{1.704086in}}%
\pgfpathlineto{\pgfqpoint{3.741497in}{1.564200in}}%
\pgfpathlineto{\pgfqpoint{3.746939in}{1.533821in}}%
\pgfpathlineto{\pgfqpoint{3.752381in}{1.754860in}}%
\pgfpathlineto{\pgfqpoint{3.757823in}{1.828701in}}%
\pgfpathlineto{\pgfqpoint{3.763265in}{1.796228in}}%
\pgfpathlineto{\pgfqpoint{3.768707in}{1.532033in}}%
\pgfpathlineto{\pgfqpoint{3.774150in}{1.517452in}}%
\pgfpathlineto{\pgfqpoint{3.779592in}{1.619575in}}%
\pgfpathlineto{\pgfqpoint{3.785034in}{1.466648in}}%
\pgfpathlineto{\pgfqpoint{3.790476in}{1.565966in}}%
\pgfpathlineto{\pgfqpoint{3.795918in}{2.081277in}}%
\pgfpathlineto{\pgfqpoint{3.801361in}{1.780703in}}%
\pgfpathlineto{\pgfqpoint{3.806803in}{2.217140in}}%
\pgfpathlineto{\pgfqpoint{3.812245in}{1.512640in}}%
\pgfpathlineto{\pgfqpoint{3.817687in}{1.967456in}}%
\pgfpathlineto{\pgfqpoint{3.823129in}{1.573009in}}%
\pgfpathlineto{\pgfqpoint{3.828571in}{1.795148in}}%
\pgfpathlineto{\pgfqpoint{3.834014in}{1.784174in}}%
\pgfpathlineto{\pgfqpoint{3.839456in}{1.781197in}}%
\pgfpathlineto{\pgfqpoint{3.844898in}{1.483463in}}%
\pgfpathlineto{\pgfqpoint{3.850340in}{1.593700in}}%
\pgfpathlineto{\pgfqpoint{3.855782in}{1.670451in}}%
\pgfpathlineto{\pgfqpoint{3.861224in}{1.652944in}}%
\pgfpathlineto{\pgfqpoint{3.866667in}{1.709173in}}%
\pgfpathlineto{\pgfqpoint{3.872109in}{1.607644in}}%
\pgfpathlineto{\pgfqpoint{3.877551in}{1.412281in}}%
\pgfpathlineto{\pgfqpoint{3.882993in}{1.736955in}}%
\pgfpathlineto{\pgfqpoint{3.888435in}{1.827594in}}%
\pgfpathlineto{\pgfqpoint{3.893878in}{1.955002in}}%
\pgfpathlineto{\pgfqpoint{3.899320in}{1.822483in}}%
\pgfpathlineto{\pgfqpoint{3.904762in}{1.422056in}}%
\pgfpathlineto{\pgfqpoint{3.910204in}{2.002195in}}%
\pgfpathlineto{\pgfqpoint{3.915646in}{1.947309in}}%
\pgfpathlineto{\pgfqpoint{3.921088in}{1.729535in}}%
\pgfpathlineto{\pgfqpoint{3.926531in}{1.661125in}}%
\pgfpathlineto{\pgfqpoint{3.931973in}{1.773024in}}%
\pgfpathlineto{\pgfqpoint{3.937415in}{1.986571in}}%
\pgfpathlineto{\pgfqpoint{3.942857in}{1.697203in}}%
\pgfpathlineto{\pgfqpoint{3.948299in}{1.812561in}}%
\pgfpathlineto{\pgfqpoint{3.953741in}{1.779602in}}%
\pgfpathlineto{\pgfqpoint{3.959184in}{1.603919in}}%
\pgfpathlineto{\pgfqpoint{3.964626in}{2.146183in}}%
\pgfpathlineto{\pgfqpoint{3.970068in}{1.631605in}}%
\pgfpathlineto{\pgfqpoint{3.975510in}{1.931838in}}%
\pgfpathlineto{\pgfqpoint{3.980952in}{1.677196in}}%
\pgfpathlineto{\pgfqpoint{3.986395in}{1.671813in}}%
\pgfpathlineto{\pgfqpoint{3.991837in}{1.804403in}}%
\pgfpathlineto{\pgfqpoint{3.997279in}{1.775325in}}%
\pgfpathlineto{\pgfqpoint{4.002721in}{1.699881in}}%
\pgfpathlineto{\pgfqpoint{4.008163in}{1.606014in}}%
\pgfpathlineto{\pgfqpoint{4.013605in}{1.904343in}}%
\pgfpathlineto{\pgfqpoint{4.019048in}{1.478747in}}%
\pgfpathlineto{\pgfqpoint{4.024490in}{1.790794in}}%
\pgfpathlineto{\pgfqpoint{4.029932in}{1.652226in}}%
\pgfpathlineto{\pgfqpoint{4.035374in}{1.430069in}}%
\pgfpathlineto{\pgfqpoint{4.040816in}{1.428443in}}%
\pgfpathlineto{\pgfqpoint{4.046259in}{1.683608in}}%
\pgfpathlineto{\pgfqpoint{4.051701in}{1.489088in}}%
\pgfpathlineto{\pgfqpoint{4.057143in}{1.673371in}}%
\pgfpathlineto{\pgfqpoint{4.062585in}{1.705934in}}%
\pgfpathlineto{\pgfqpoint{4.068027in}{2.178809in}}%
\pgfpathlineto{\pgfqpoint{4.073469in}{1.706405in}}%
\pgfpathlineto{\pgfqpoint{4.078912in}{1.397763in}}%
\pgfpathlineto{\pgfqpoint{4.084354in}{2.203228in}}%
\pgfpathlineto{\pgfqpoint{4.089796in}{1.751296in}}%
\pgfpathlineto{\pgfqpoint{4.095238in}{1.556543in}}%
\pgfpathlineto{\pgfqpoint{4.100680in}{1.973273in}}%
\pgfpathlineto{\pgfqpoint{4.106122in}{2.302289in}}%
\pgfpathlineto{\pgfqpoint{4.111565in}{1.807408in}}%
\pgfpathlineto{\pgfqpoint{4.117007in}{1.646187in}}%
\pgfpathlineto{\pgfqpoint{4.122449in}{1.789574in}}%
\pgfpathlineto{\pgfqpoint{4.127891in}{1.447358in}}%
\pgfpathlineto{\pgfqpoint{4.133333in}{1.856287in}}%
\pgfpathlineto{\pgfqpoint{4.138776in}{1.948767in}}%
\pgfpathlineto{\pgfqpoint{4.144218in}{1.995691in}}%
\pgfpathlineto{\pgfqpoint{4.149660in}{2.111478in}}%
\pgfpathlineto{\pgfqpoint{4.155102in}{1.760005in}}%
\pgfpathlineto{\pgfqpoint{4.160544in}{1.689111in}}%
\pgfpathlineto{\pgfqpoint{4.165986in}{1.768928in}}%
\pgfpathlineto{\pgfqpoint{4.171429in}{1.611496in}}%
\pgfpathlineto{\pgfqpoint{4.176871in}{1.713253in}}%
\pgfpathlineto{\pgfqpoint{4.182313in}{1.581742in}}%
\pgfpathlineto{\pgfqpoint{4.187755in}{1.601628in}}%
\pgfpathlineto{\pgfqpoint{4.193197in}{2.036743in}}%
\pgfpathlineto{\pgfqpoint{4.198639in}{1.874926in}}%
\pgfpathlineto{\pgfqpoint{4.204082in}{1.944030in}}%
\pgfpathlineto{\pgfqpoint{4.209524in}{1.655618in}}%
\pgfpathlineto{\pgfqpoint{4.214966in}{1.913091in}}%
\pgfpathlineto{\pgfqpoint{4.220408in}{1.615723in}}%
\pgfpathlineto{\pgfqpoint{4.225850in}{1.860639in}}%
\pgfpathlineto{\pgfqpoint{4.231293in}{2.017076in}}%
\pgfpathlineto{\pgfqpoint{4.236735in}{1.964292in}}%
\pgfpathlineto{\pgfqpoint{4.242177in}{1.968507in}}%
\pgfpathlineto{\pgfqpoint{4.247619in}{1.615502in}}%
\pgfpathlineto{\pgfqpoint{4.253061in}{1.677070in}}%
\pgfpathlineto{\pgfqpoint{4.258503in}{1.480232in}}%
\pgfpathlineto{\pgfqpoint{4.269388in}{2.065993in}}%
\pgfpathlineto{\pgfqpoint{4.274830in}{1.647914in}}%
\pgfpathlineto{\pgfqpoint{4.285714in}{2.101088in}}%
\pgfpathlineto{\pgfqpoint{4.291156in}{1.649704in}}%
\pgfpathlineto{\pgfqpoint{4.296599in}{1.604997in}}%
\pgfpathlineto{\pgfqpoint{4.302041in}{1.651090in}}%
\pgfpathlineto{\pgfqpoint{4.307483in}{1.661002in}}%
\pgfpathlineto{\pgfqpoint{4.312925in}{1.633814in}}%
\pgfpathlineto{\pgfqpoint{4.318367in}{1.675454in}}%
\pgfpathlineto{\pgfqpoint{4.323810in}{1.659304in}}%
\pgfpathlineto{\pgfqpoint{4.334694in}{1.992847in}}%
\pgfpathlineto{\pgfqpoint{4.340136in}{1.566350in}}%
\pgfpathlineto{\pgfqpoint{4.345578in}{1.692213in}}%
\pgfpathlineto{\pgfqpoint{4.351020in}{1.786840in}}%
\pgfpathlineto{\pgfqpoint{4.356463in}{1.544890in}}%
\pgfpathlineto{\pgfqpoint{4.361905in}{1.836992in}}%
\pgfpathlineto{\pgfqpoint{4.367347in}{1.696901in}}%
\pgfpathlineto{\pgfqpoint{4.372789in}{1.885086in}}%
\pgfpathlineto{\pgfqpoint{4.378231in}{1.862234in}}%
\pgfpathlineto{\pgfqpoint{4.383673in}{1.542506in}}%
\pgfpathlineto{\pgfqpoint{4.389116in}{1.393472in}}%
\pgfpathlineto{\pgfqpoint{4.394558in}{1.985823in}}%
\pgfpathlineto{\pgfqpoint{4.400000in}{2.345581in}}%
\pgfpathlineto{\pgfqpoint{4.405442in}{1.874056in}}%
\pgfpathlineto{\pgfqpoint{4.410884in}{1.975649in}}%
\pgfpathlineto{\pgfqpoint{4.416327in}{1.677091in}}%
\pgfpathlineto{\pgfqpoint{4.421769in}{2.083141in}}%
\pgfpathlineto{\pgfqpoint{4.427211in}{1.681125in}}%
\pgfpathlineto{\pgfqpoint{4.432653in}{1.824390in}}%
\pgfpathlineto{\pgfqpoint{4.438095in}{1.882002in}}%
\pgfpathlineto{\pgfqpoint{4.443537in}{1.481750in}}%
\pgfpathlineto{\pgfqpoint{4.448980in}{1.929982in}}%
\pgfpathlineto{\pgfqpoint{4.459864in}{1.362587in}}%
\pgfpathlineto{\pgfqpoint{4.465306in}{1.871977in}}%
\pgfpathlineto{\pgfqpoint{4.470748in}{1.747699in}}%
\pgfpathlineto{\pgfqpoint{4.476190in}{1.680976in}}%
\pgfpathlineto{\pgfqpoint{4.481633in}{1.782672in}}%
\pgfpathlineto{\pgfqpoint{4.487075in}{1.853941in}}%
\pgfpathlineto{\pgfqpoint{4.492517in}{1.731819in}}%
\pgfpathlineto{\pgfqpoint{4.497959in}{1.739082in}}%
\pgfpathlineto{\pgfqpoint{4.503401in}{1.677473in}}%
\pgfpathlineto{\pgfqpoint{4.508844in}{1.935291in}}%
\pgfpathlineto{\pgfqpoint{4.514286in}{1.994783in}}%
\pgfpathlineto{\pgfqpoint{4.519728in}{1.721930in}}%
\pgfpathlineto{\pgfqpoint{4.525170in}{2.023167in}}%
\pgfpathlineto{\pgfqpoint{4.530612in}{1.707839in}}%
\pgfpathlineto{\pgfqpoint{4.536054in}{1.645369in}}%
\pgfpathlineto{\pgfqpoint{4.541497in}{1.754309in}}%
\pgfpathlineto{\pgfqpoint{4.546939in}{2.246642in}}%
\pgfpathlineto{\pgfqpoint{4.552381in}{1.575819in}}%
\pgfpathlineto{\pgfqpoint{4.557823in}{1.642838in}}%
\pgfpathlineto{\pgfqpoint{4.563265in}{2.045804in}}%
\pgfpathlineto{\pgfqpoint{4.568707in}{2.134821in}}%
\pgfpathlineto{\pgfqpoint{4.574150in}{2.081751in}}%
\pgfpathlineto{\pgfqpoint{4.579592in}{1.864825in}}%
\pgfpathlineto{\pgfqpoint{4.585034in}{1.809215in}}%
\pgfpathlineto{\pgfqpoint{4.590476in}{2.243660in}}%
\pgfpathlineto{\pgfqpoint{4.595918in}{1.883347in}}%
\pgfpathlineto{\pgfqpoint{4.606803in}{1.809492in}}%
\pgfpathlineto{\pgfqpoint{4.612245in}{1.781408in}}%
\pgfpathlineto{\pgfqpoint{4.617687in}{1.773225in}}%
\pgfpathlineto{\pgfqpoint{4.623129in}{1.760192in}}%
\pgfpathlineto{\pgfqpoint{4.628571in}{1.721121in}}%
\pgfpathlineto{\pgfqpoint{4.634014in}{1.823282in}}%
\pgfpathlineto{\pgfqpoint{4.639456in}{1.897532in}}%
\pgfpathlineto{\pgfqpoint{4.644898in}{1.605845in}}%
\pgfpathlineto{\pgfqpoint{4.650340in}{1.680841in}}%
\pgfpathlineto{\pgfqpoint{4.655782in}{2.178113in}}%
\pgfpathlineto{\pgfqpoint{4.661224in}{1.870307in}}%
\pgfpathlineto{\pgfqpoint{4.666667in}{2.056668in}}%
\pgfpathlineto{\pgfqpoint{4.672109in}{2.117085in}}%
\pgfpathlineto{\pgfqpoint{4.677551in}{1.741847in}}%
\pgfpathlineto{\pgfqpoint{4.682993in}{1.611522in}}%
\pgfpathlineto{\pgfqpoint{4.688435in}{1.885387in}}%
\pgfpathlineto{\pgfqpoint{4.693878in}{1.711618in}}%
\pgfpathlineto{\pgfqpoint{4.699320in}{2.094066in}}%
\pgfpathlineto{\pgfqpoint{4.704762in}{1.867155in}}%
\pgfpathlineto{\pgfqpoint{4.710204in}{2.093451in}}%
\pgfpathlineto{\pgfqpoint{4.715646in}{1.694262in}}%
\pgfpathlineto{\pgfqpoint{4.721088in}{1.883200in}}%
\pgfpathlineto{\pgfqpoint{4.726531in}{1.596710in}}%
\pgfpathlineto{\pgfqpoint{4.731973in}{2.189853in}}%
\pgfpathlineto{\pgfqpoint{4.737415in}{2.038541in}}%
\pgfpathlineto{\pgfqpoint{4.742857in}{1.944131in}}%
\pgfpathlineto{\pgfqpoint{4.748299in}{1.967057in}}%
\pgfpathlineto{\pgfqpoint{4.753741in}{1.859973in}}%
\pgfpathlineto{\pgfqpoint{4.759184in}{1.477478in}}%
\pgfpathlineto{\pgfqpoint{4.764626in}{2.013524in}}%
\pgfpathlineto{\pgfqpoint{4.770068in}{1.971238in}}%
\pgfpathlineto{\pgfqpoint{4.775510in}{1.877467in}}%
\pgfpathlineto{\pgfqpoint{4.780952in}{1.706042in}}%
\pgfpathlineto{\pgfqpoint{4.786395in}{1.805504in}}%
\pgfpathlineto{\pgfqpoint{4.791837in}{1.665361in}}%
\pgfpathlineto{\pgfqpoint{4.797279in}{1.936440in}}%
\pgfpathlineto{\pgfqpoint{4.802721in}{1.820374in}}%
\pgfpathlineto{\pgfqpoint{4.808163in}{1.901370in}}%
\pgfpathlineto{\pgfqpoint{4.813605in}{1.576994in}}%
\pgfpathlineto{\pgfqpoint{4.819048in}{1.833243in}}%
\pgfpathlineto{\pgfqpoint{4.824490in}{1.862521in}}%
\pgfpathlineto{\pgfqpoint{4.829932in}{1.821785in}}%
\pgfpathlineto{\pgfqpoint{4.835374in}{1.680649in}}%
\pgfpathlineto{\pgfqpoint{4.840816in}{2.099034in}}%
\pgfpathlineto{\pgfqpoint{4.846259in}{1.579821in}}%
\pgfpathlineto{\pgfqpoint{4.851701in}{1.926030in}}%
\pgfpathlineto{\pgfqpoint{4.857143in}{1.640785in}}%
\pgfpathlineto{\pgfqpoint{4.862585in}{1.737175in}}%
\pgfpathlineto{\pgfqpoint{4.868027in}{1.489291in}}%
\pgfpathlineto{\pgfqpoint{4.873469in}{1.634280in}}%
\pgfpathlineto{\pgfqpoint{4.878912in}{1.690502in}}%
\pgfpathlineto{\pgfqpoint{4.884354in}{1.878947in}}%
\pgfpathlineto{\pgfqpoint{4.889796in}{1.810186in}}%
\pgfpathlineto{\pgfqpoint{4.895238in}{1.895779in}}%
\pgfpathlineto{\pgfqpoint{4.900680in}{1.823508in}}%
\pgfpathlineto{\pgfqpoint{4.906122in}{1.731172in}}%
\pgfpathlineto{\pgfqpoint{4.911565in}{1.681839in}}%
\pgfpathlineto{\pgfqpoint{4.917007in}{1.845859in}}%
\pgfpathlineto{\pgfqpoint{4.922449in}{1.651693in}}%
\pgfpathlineto{\pgfqpoint{4.927891in}{1.959874in}}%
\pgfpathlineto{\pgfqpoint{4.933333in}{1.727686in}}%
\pgfpathlineto{\pgfqpoint{4.938776in}{1.919737in}}%
\pgfpathlineto{\pgfqpoint{4.944218in}{1.721472in}}%
\pgfpathlineto{\pgfqpoint{4.949660in}{1.643580in}}%
\pgfpathlineto{\pgfqpoint{4.955102in}{2.150720in}}%
\pgfpathlineto{\pgfqpoint{4.960544in}{1.767509in}}%
\pgfpathlineto{\pgfqpoint{4.965986in}{1.788581in}}%
\pgfpathlineto{\pgfqpoint{4.971429in}{1.790873in}}%
\pgfpathlineto{\pgfqpoint{4.976871in}{1.958200in}}%
\pgfpathlineto{\pgfqpoint{4.982313in}{1.566706in}}%
\pgfpathlineto{\pgfqpoint{4.987755in}{1.920529in}}%
\pgfpathlineto{\pgfqpoint{4.993197in}{1.436878in}}%
\pgfpathlineto{\pgfqpoint{4.998639in}{2.236445in}}%
\pgfpathlineto{\pgfqpoint{5.004082in}{2.056979in}}%
\pgfpathlineto{\pgfqpoint{5.009524in}{1.688543in}}%
\pgfpathlineto{\pgfqpoint{5.014966in}{1.643964in}}%
\pgfpathlineto{\pgfqpoint{5.020408in}{1.788581in}}%
\pgfpathlineto{\pgfqpoint{5.025850in}{1.796234in}}%
\pgfpathlineto{\pgfqpoint{5.031293in}{1.714898in}}%
\pgfpathlineto{\pgfqpoint{5.036735in}{1.755096in}}%
\pgfpathlineto{\pgfqpoint{5.042177in}{1.735272in}}%
\pgfpathlineto{\pgfqpoint{5.047619in}{1.921444in}}%
\pgfpathlineto{\pgfqpoint{5.053061in}{1.407153in}}%
\pgfpathlineto{\pgfqpoint{5.058503in}{1.869037in}}%
\pgfpathlineto{\pgfqpoint{5.063946in}{1.532641in}}%
\pgfpathlineto{\pgfqpoint{5.069388in}{1.660117in}}%
\pgfpathlineto{\pgfqpoint{5.074830in}{2.123740in}}%
\pgfpathlineto{\pgfqpoint{5.080272in}{2.045677in}}%
\pgfpathlineto{\pgfqpoint{5.085714in}{2.014074in}}%
\pgfpathlineto{\pgfqpoint{5.091156in}{1.679664in}}%
\pgfpathlineto{\pgfqpoint{5.096599in}{2.107214in}}%
\pgfpathlineto{\pgfqpoint{5.102041in}{1.545554in}}%
\pgfpathlineto{\pgfqpoint{5.107483in}{1.853673in}}%
\pgfpathlineto{\pgfqpoint{5.112925in}{1.752837in}}%
\pgfpathlineto{\pgfqpoint{5.118367in}{1.854168in}}%
\pgfpathlineto{\pgfqpoint{5.123810in}{1.661034in}}%
\pgfpathlineto{\pgfqpoint{5.129252in}{1.625798in}}%
\pgfpathlineto{\pgfqpoint{5.134694in}{1.776283in}}%
\pgfpathlineto{\pgfqpoint{5.140136in}{1.782967in}}%
\pgfpathlineto{\pgfqpoint{5.145578in}{1.763267in}}%
\pgfpathlineto{\pgfqpoint{5.151020in}{1.677315in}}%
\pgfpathlineto{\pgfqpoint{5.156463in}{1.928544in}}%
\pgfpathlineto{\pgfqpoint{5.161905in}{1.545029in}}%
\pgfpathlineto{\pgfqpoint{5.167347in}{1.661322in}}%
\pgfpathlineto{\pgfqpoint{5.172789in}{1.985335in}}%
\pgfpathlineto{\pgfqpoint{5.178231in}{1.874328in}}%
\pgfpathlineto{\pgfqpoint{5.183673in}{1.801865in}}%
\pgfpathlineto{\pgfqpoint{5.189116in}{1.578444in}}%
\pgfpathlineto{\pgfqpoint{5.194558in}{1.643513in}}%
\pgfpathlineto{\pgfqpoint{5.200000in}{1.754996in}}%
\pgfpathlineto{\pgfqpoint{5.205442in}{2.106523in}}%
\pgfpathlineto{\pgfqpoint{5.216327in}{1.794572in}}%
\pgfpathlineto{\pgfqpoint{5.221769in}{1.890849in}}%
\pgfpathlineto{\pgfqpoint{5.227211in}{1.959822in}}%
\pgfpathlineto{\pgfqpoint{5.232653in}{1.679486in}}%
\pgfpathlineto{\pgfqpoint{5.238095in}{1.970830in}}%
\pgfpathlineto{\pgfqpoint{5.243537in}{1.670132in}}%
\pgfpathlineto{\pgfqpoint{5.248980in}{1.768443in}}%
\pgfpathlineto{\pgfqpoint{5.254422in}{2.097753in}}%
\pgfpathlineto{\pgfqpoint{5.259864in}{1.566995in}}%
\pgfpathlineto{\pgfqpoint{5.265306in}{1.931080in}}%
\pgfpathlineto{\pgfqpoint{5.270748in}{1.800712in}}%
\pgfpathlineto{\pgfqpoint{5.276190in}{1.837190in}}%
\pgfpathlineto{\pgfqpoint{5.281633in}{1.685920in}}%
\pgfpathlineto{\pgfqpoint{5.287075in}{1.753027in}}%
\pgfpathlineto{\pgfqpoint{5.292517in}{1.940582in}}%
\pgfpathlineto{\pgfqpoint{5.297959in}{1.532598in}}%
\pgfpathlineto{\pgfqpoint{5.303401in}{1.631660in}}%
\pgfpathlineto{\pgfqpoint{5.308844in}{1.922559in}}%
\pgfpathlineto{\pgfqpoint{5.314286in}{2.085044in}}%
\pgfpathlineto{\pgfqpoint{5.319728in}{1.731433in}}%
\pgfpathlineto{\pgfqpoint{5.325170in}{1.904147in}}%
\pgfpathlineto{\pgfqpoint{5.330612in}{1.922423in}}%
\pgfpathlineto{\pgfqpoint{5.336054in}{1.648035in}}%
\pgfpathlineto{\pgfqpoint{5.341497in}{1.841386in}}%
\pgfpathlineto{\pgfqpoint{5.346939in}{1.686937in}}%
\pgfpathlineto{\pgfqpoint{5.352381in}{1.850092in}}%
\pgfpathlineto{\pgfqpoint{5.357823in}{1.844190in}}%
\pgfpathlineto{\pgfqpoint{5.363265in}{1.650555in}}%
\pgfpathlineto{\pgfqpoint{5.368707in}{1.836551in}}%
\pgfpathlineto{\pgfqpoint{5.374150in}{1.984245in}}%
\pgfpathlineto{\pgfqpoint{5.379592in}{1.927611in}}%
\pgfpathlineto{\pgfqpoint{5.385034in}{1.829512in}}%
\pgfpathlineto{\pgfqpoint{5.390476in}{1.937559in}}%
\pgfpathlineto{\pgfqpoint{5.395918in}{1.659339in}}%
\pgfpathlineto{\pgfqpoint{5.401361in}{1.709818in}}%
\pgfpathlineto{\pgfqpoint{5.406803in}{1.889617in}}%
\pgfpathlineto{\pgfqpoint{5.412245in}{1.579244in}}%
\pgfpathlineto{\pgfqpoint{5.417687in}{1.653273in}}%
\pgfpathlineto{\pgfqpoint{5.423129in}{1.628634in}}%
\pgfpathlineto{\pgfqpoint{5.428571in}{1.682666in}}%
\pgfpathlineto{\pgfqpoint{5.434014in}{1.875243in}}%
\pgfpathlineto{\pgfqpoint{5.439456in}{1.844171in}}%
\pgfpathlineto{\pgfqpoint{5.444898in}{1.965753in}}%
\pgfpathlineto{\pgfqpoint{5.450340in}{1.770640in}}%
\pgfpathlineto{\pgfqpoint{5.455782in}{1.652472in}}%
\pgfpathlineto{\pgfqpoint{5.461224in}{1.656922in}}%
\pgfpathlineto{\pgfqpoint{5.466667in}{1.900618in}}%
\pgfpathlineto{\pgfqpoint{5.472109in}{1.962458in}}%
\pgfpathlineto{\pgfqpoint{5.477551in}{1.547777in}}%
\pgfpathlineto{\pgfqpoint{5.482993in}{1.732056in}}%
\pgfpathlineto{\pgfqpoint{5.488435in}{1.850677in}}%
\pgfpathlineto{\pgfqpoint{5.493878in}{1.640170in}}%
\pgfpathlineto{\pgfqpoint{5.499320in}{1.596711in}}%
\pgfpathlineto{\pgfqpoint{5.504762in}{1.824564in}}%
\pgfpathlineto{\pgfqpoint{5.510204in}{1.931238in}}%
\pgfpathlineto{\pgfqpoint{5.515646in}{1.774907in}}%
\pgfpathlineto{\pgfqpoint{5.521088in}{1.990181in}}%
\pgfpathlineto{\pgfqpoint{5.526531in}{1.812995in}}%
\pgfpathlineto{\pgfqpoint{5.531973in}{1.893902in}}%
\pgfpathlineto{\pgfqpoint{5.537415in}{1.801394in}}%
\pgfpathlineto{\pgfqpoint{5.542857in}{1.828087in}}%
\pgfpathlineto{\pgfqpoint{5.553741in}{1.597646in}}%
\pgfpathlineto{\pgfqpoint{5.559184in}{1.963854in}}%
\pgfpathlineto{\pgfqpoint{5.564626in}{1.673490in}}%
\pgfpathlineto{\pgfqpoint{5.570068in}{2.124794in}}%
\pgfpathlineto{\pgfqpoint{5.575510in}{1.800556in}}%
\pgfpathlineto{\pgfqpoint{5.580952in}{1.731258in}}%
\pgfpathlineto{\pgfqpoint{5.586395in}{1.729813in}}%
\pgfpathlineto{\pgfqpoint{5.591837in}{1.758752in}}%
\pgfpathlineto{\pgfqpoint{5.597279in}{1.776339in}}%
\pgfpathlineto{\pgfqpoint{5.602721in}{1.786317in}}%
\pgfpathlineto{\pgfqpoint{5.608163in}{1.927522in}}%
\pgfpathlineto{\pgfqpoint{5.613605in}{1.907250in}}%
\pgfpathlineto{\pgfqpoint{5.629932in}{1.559042in}}%
\pgfpathlineto{\pgfqpoint{5.635374in}{1.761103in}}%
\pgfpathlineto{\pgfqpoint{5.640816in}{1.733313in}}%
\pgfpathlineto{\pgfqpoint{5.646259in}{1.623503in}}%
\pgfpathlineto{\pgfqpoint{5.651701in}{1.865419in}}%
\pgfpathlineto{\pgfqpoint{5.657143in}{2.040095in}}%
\pgfpathlineto{\pgfqpoint{5.662585in}{1.576767in}}%
\pgfpathlineto{\pgfqpoint{5.668027in}{1.755702in}}%
\pgfpathlineto{\pgfqpoint{5.673469in}{1.752100in}}%
\pgfpathlineto{\pgfqpoint{5.678912in}{1.750514in}}%
\pgfpathlineto{\pgfqpoint{5.684354in}{2.015353in}}%
\pgfpathlineto{\pgfqpoint{5.689796in}{1.822846in}}%
\pgfpathlineto{\pgfqpoint{5.695238in}{1.556329in}}%
\pgfpathlineto{\pgfqpoint{5.700680in}{1.944406in}}%
\pgfpathlineto{\pgfqpoint{5.706122in}{1.877001in}}%
\pgfpathlineto{\pgfqpoint{5.711565in}{1.747105in}}%
\pgfpathlineto{\pgfqpoint{5.717007in}{2.030457in}}%
\pgfpathlineto{\pgfqpoint{5.722449in}{1.695401in}}%
\pgfpathlineto{\pgfqpoint{5.727891in}{1.839380in}}%
\pgfpathlineto{\pgfqpoint{5.733333in}{1.673581in}}%
\pgfpathlineto{\pgfqpoint{5.738776in}{1.700879in}}%
\pgfpathlineto{\pgfqpoint{5.744218in}{1.624603in}}%
\pgfpathlineto{\pgfqpoint{5.749660in}{2.150165in}}%
\pgfpathlineto{\pgfqpoint{5.755102in}{1.812847in}}%
\pgfpathlineto{\pgfqpoint{5.760544in}{1.906351in}}%
\pgfpathlineto{\pgfqpoint{5.765986in}{1.607776in}}%
\pgfpathlineto{\pgfqpoint{5.771429in}{1.664057in}}%
\pgfpathlineto{\pgfqpoint{5.776871in}{1.876342in}}%
\pgfpathlineto{\pgfqpoint{5.782313in}{1.832403in}}%
\pgfpathlineto{\pgfqpoint{5.787755in}{1.880070in}}%
\pgfpathlineto{\pgfqpoint{5.793197in}{1.988201in}}%
\pgfpathlineto{\pgfqpoint{5.798639in}{2.144880in}}%
\pgfpathlineto{\pgfqpoint{5.804082in}{1.860939in}}%
\pgfpathlineto{\pgfqpoint{5.809524in}{1.690410in}}%
\pgfpathlineto{\pgfqpoint{5.814966in}{1.960501in}}%
\pgfpathlineto{\pgfqpoint{5.820408in}{1.927496in}}%
\pgfpathlineto{\pgfqpoint{5.825850in}{1.829115in}}%
\pgfpathlineto{\pgfqpoint{5.831293in}{1.775416in}}%
\pgfpathlineto{\pgfqpoint{5.836735in}{1.991142in}}%
\pgfpathlineto{\pgfqpoint{5.842177in}{1.936950in}}%
\pgfpathlineto{\pgfqpoint{5.847619in}{1.753455in}}%
\pgfpathlineto{\pgfqpoint{5.853061in}{2.066597in}}%
\pgfpathlineto{\pgfqpoint{5.858503in}{1.590215in}}%
\pgfpathlineto{\pgfqpoint{5.863946in}{1.594900in}}%
\pgfpathlineto{\pgfqpoint{5.874830in}{2.042155in}}%
\pgfpathlineto{\pgfqpoint{5.880272in}{1.707282in}}%
\pgfpathlineto{\pgfqpoint{5.885714in}{1.579249in}}%
\pgfpathlineto{\pgfqpoint{5.891156in}{2.086026in}}%
\pgfpathlineto{\pgfqpoint{5.896599in}{1.674537in}}%
\pgfpathlineto{\pgfqpoint{5.902041in}{1.873108in}}%
\pgfpathlineto{\pgfqpoint{5.907483in}{1.660297in}}%
\pgfpathlineto{\pgfqpoint{5.912925in}{1.810486in}}%
\pgfpathlineto{\pgfqpoint{5.918367in}{1.910769in}}%
\pgfpathlineto{\pgfqpoint{5.923810in}{2.064844in}}%
\pgfpathlineto{\pgfqpoint{5.929252in}{1.897396in}}%
\pgfpathlineto{\pgfqpoint{5.934694in}{1.827383in}}%
\pgfpathlineto{\pgfqpoint{5.940136in}{1.798293in}}%
\pgfpathlineto{\pgfqpoint{5.945578in}{1.969860in}}%
\pgfpathlineto{\pgfqpoint{5.951020in}{1.794130in}}%
\pgfpathlineto{\pgfqpoint{5.956463in}{2.034482in}}%
\pgfpathlineto{\pgfqpoint{5.967347in}{1.431368in}}%
\pgfpathlineto{\pgfqpoint{5.972789in}{1.840596in}}%
\pgfpathlineto{\pgfqpoint{5.978231in}{1.845332in}}%
\pgfpathlineto{\pgfqpoint{5.983673in}{1.969844in}}%
\pgfpathlineto{\pgfqpoint{5.989116in}{1.875243in}}%
\pgfpathlineto{\pgfqpoint{5.994558in}{1.859514in}}%
\pgfpathlineto{\pgfqpoint{6.000000in}{1.876849in}}%
\pgfpathlineto{\pgfqpoint{6.005442in}{1.844053in}}%
\pgfpathlineto{\pgfqpoint{6.010884in}{2.013091in}}%
\pgfpathlineto{\pgfqpoint{6.016327in}{1.772095in}}%
\pgfpathlineto{\pgfqpoint{6.021769in}{1.897866in}}%
\pgfpathlineto{\pgfqpoint{6.027211in}{2.059832in}}%
\pgfpathlineto{\pgfqpoint{6.032653in}{1.709093in}}%
\pgfpathlineto{\pgfqpoint{6.038095in}{1.903149in}}%
\pgfpathlineto{\pgfqpoint{6.043537in}{1.621602in}}%
\pgfpathlineto{\pgfqpoint{6.048980in}{2.037365in}}%
\pgfpathlineto{\pgfqpoint{6.059864in}{1.777230in}}%
\pgfpathlineto{\pgfqpoint{6.065306in}{1.742497in}}%
\pgfpathlineto{\pgfqpoint{6.070748in}{1.694012in}}%
\pgfpathlineto{\pgfqpoint{6.076190in}{1.973647in}}%
\pgfpathlineto{\pgfqpoint{6.087075in}{1.700377in}}%
\pgfpathlineto{\pgfqpoint{6.092517in}{1.784009in}}%
\pgfpathlineto{\pgfqpoint{6.097959in}{1.686318in}}%
\pgfpathlineto{\pgfqpoint{6.103401in}{1.930493in}}%
\pgfpathlineto{\pgfqpoint{6.108844in}{1.756431in}}%
\pgfpathlineto{\pgfqpoint{6.114286in}{1.688228in}}%
\pgfpathlineto{\pgfqpoint{6.119728in}{1.849657in}}%
\pgfpathlineto{\pgfqpoint{6.125170in}{1.710294in}}%
\pgfpathlineto{\pgfqpoint{6.130612in}{1.938930in}}%
\pgfpathlineto{\pgfqpoint{6.136054in}{1.614671in}}%
\pgfpathlineto{\pgfqpoint{6.141497in}{1.797953in}}%
\pgfpathlineto{\pgfqpoint{6.146939in}{1.817166in}}%
\pgfpathlineto{\pgfqpoint{6.152381in}{1.884330in}}%
\pgfpathlineto{\pgfqpoint{6.157823in}{2.086582in}}%
\pgfpathlineto{\pgfqpoint{6.163265in}{1.689101in}}%
\pgfpathlineto{\pgfqpoint{6.168707in}{1.543858in}}%
\pgfpathlineto{\pgfqpoint{6.174150in}{2.245549in}}%
\pgfpathlineto{\pgfqpoint{6.179592in}{1.614345in}}%
\pgfpathlineto{\pgfqpoint{6.185034in}{2.002528in}}%
\pgfpathlineto{\pgfqpoint{6.190476in}{1.682297in}}%
\pgfpathlineto{\pgfqpoint{6.195918in}{1.751127in}}%
\pgfpathlineto{\pgfqpoint{6.201361in}{1.364610in}}%
\pgfpathlineto{\pgfqpoint{6.206803in}{1.527956in}}%
\pgfpathlineto{\pgfqpoint{6.212245in}{1.963104in}}%
\pgfpathlineto{\pgfqpoint{6.217687in}{1.816933in}}%
\pgfpathlineto{\pgfqpoint{6.223129in}{1.564863in}}%
\pgfpathlineto{\pgfqpoint{6.228571in}{1.865757in}}%
\pgfpathlineto{\pgfqpoint{6.234014in}{1.647136in}}%
\pgfpathlineto{\pgfqpoint{6.239456in}{1.969892in}}%
\pgfpathlineto{\pgfqpoint{6.244898in}{1.498107in}}%
\pgfpathlineto{\pgfqpoint{6.250340in}{1.608672in}}%
\pgfpathlineto{\pgfqpoint{6.255782in}{2.077542in}}%
\pgfpathlineto{\pgfqpoint{6.261224in}{1.969220in}}%
\pgfpathlineto{\pgfqpoint{6.266667in}{1.822008in}}%
\pgfpathlineto{\pgfqpoint{6.272109in}{1.890002in}}%
\pgfpathlineto{\pgfqpoint{6.277551in}{2.146571in}}%
\pgfpathlineto{\pgfqpoint{6.282993in}{1.496801in}}%
\pgfpathlineto{\pgfqpoint{6.288435in}{1.725455in}}%
\pgfpathlineto{\pgfqpoint{6.293878in}{1.754686in}}%
\pgfpathlineto{\pgfqpoint{6.299320in}{1.982541in}}%
\pgfpathlineto{\pgfqpoint{6.304762in}{2.053981in}}%
\pgfpathlineto{\pgfqpoint{6.310204in}{1.657137in}}%
\pgfpathlineto{\pgfqpoint{6.315646in}{1.732174in}}%
\pgfpathlineto{\pgfqpoint{6.321088in}{1.627199in}}%
\pgfpathlineto{\pgfqpoint{6.326531in}{1.648223in}}%
\pgfpathlineto{\pgfqpoint{6.331973in}{2.000944in}}%
\pgfpathlineto{\pgfqpoint{6.337415in}{1.894283in}}%
\pgfpathlineto{\pgfqpoint{6.342857in}{1.751362in}}%
\pgfpathlineto{\pgfqpoint{6.348299in}{1.915315in}}%
\pgfpathlineto{\pgfqpoint{6.353741in}{1.535450in}}%
\pgfpathlineto{\pgfqpoint{6.359184in}{2.129699in}}%
\pgfpathlineto{\pgfqpoint{6.364626in}{1.824469in}}%
\pgfpathlineto{\pgfqpoint{6.370068in}{1.821563in}}%
\pgfpathlineto{\pgfqpoint{6.375510in}{1.866128in}}%
\pgfpathlineto{\pgfqpoint{6.380952in}{1.476495in}}%
\pgfpathlineto{\pgfqpoint{6.386395in}{1.626177in}}%
\pgfpathlineto{\pgfqpoint{6.391837in}{1.906499in}}%
\pgfpathlineto{\pgfqpoint{6.397279in}{1.590299in}}%
\pgfpathlineto{\pgfqpoint{6.402721in}{1.726296in}}%
\pgfpathlineto{\pgfqpoint{6.408163in}{1.609732in}}%
\pgfpathlineto{\pgfqpoint{6.413605in}{1.978333in}}%
\pgfpathlineto{\pgfqpoint{6.419048in}{1.518407in}}%
\pgfpathlineto{\pgfqpoint{6.429932in}{1.943218in}}%
\pgfpathlineto{\pgfqpoint{6.435374in}{1.747193in}}%
\pgfpathlineto{\pgfqpoint{6.440816in}{1.675647in}}%
\pgfpathlineto{\pgfqpoint{6.446259in}{2.029711in}}%
\pgfpathlineto{\pgfqpoint{6.451701in}{2.108893in}}%
\pgfpathlineto{\pgfqpoint{6.462585in}{1.606174in}}%
\pgfpathlineto{\pgfqpoint{6.468027in}{1.870780in}}%
\pgfpathlineto{\pgfqpoint{6.473469in}{1.573332in}}%
\pgfpathlineto{\pgfqpoint{6.478912in}{1.733985in}}%
\pgfpathlineto{\pgfqpoint{6.484354in}{1.742326in}}%
\pgfpathlineto{\pgfqpoint{6.489796in}{1.490586in}}%
\pgfpathlineto{\pgfqpoint{6.506122in}{1.610667in}}%
\pgfpathlineto{\pgfqpoint{6.511565in}{2.034250in}}%
\pgfpathlineto{\pgfqpoint{6.517007in}{1.780512in}}%
\pgfpathlineto{\pgfqpoint{6.522449in}{1.850111in}}%
\pgfpathlineto{\pgfqpoint{6.533333in}{2.079192in}}%
\pgfpathlineto{\pgfqpoint{6.538776in}{1.489448in}}%
\pgfpathlineto{\pgfqpoint{6.544218in}{1.901393in}}%
\pgfpathlineto{\pgfqpoint{6.549660in}{2.070444in}}%
\pgfpathlineto{\pgfqpoint{6.555102in}{2.008178in}}%
\pgfpathlineto{\pgfqpoint{6.560544in}{1.905592in}}%
\pgfpathlineto{\pgfqpoint{6.565986in}{1.646112in}}%
\pgfpathlineto{\pgfqpoint{6.571429in}{1.926916in}}%
\pgfpathlineto{\pgfqpoint{6.576871in}{1.999749in}}%
\pgfpathlineto{\pgfqpoint{6.582313in}{1.988320in}}%
\pgfpathlineto{\pgfqpoint{6.587755in}{1.417940in}}%
\pgfpathlineto{\pgfqpoint{6.593197in}{1.967352in}}%
\pgfpathlineto{\pgfqpoint{6.598639in}{1.811732in}}%
\pgfpathlineto{\pgfqpoint{6.604082in}{1.842465in}}%
\pgfpathlineto{\pgfqpoint{6.609524in}{1.692084in}}%
\pgfpathlineto{\pgfqpoint{6.614966in}{1.814623in}}%
\pgfpathlineto{\pgfqpoint{6.620408in}{1.881322in}}%
\pgfpathlineto{\pgfqpoint{6.625850in}{1.867475in}}%
\pgfpathlineto{\pgfqpoint{6.631293in}{2.206028in}}%
\pgfpathlineto{\pgfqpoint{6.636735in}{1.625207in}}%
\pgfpathlineto{\pgfqpoint{6.642177in}{1.784721in}}%
\pgfpathlineto{\pgfqpoint{6.647619in}{1.800124in}}%
\pgfpathlineto{\pgfqpoint{6.653061in}{1.862827in}}%
\pgfpathlineto{\pgfqpoint{6.658503in}{1.486736in}}%
\pgfpathlineto{\pgfqpoint{6.663946in}{1.602950in}}%
\pgfpathlineto{\pgfqpoint{6.669388in}{1.685557in}}%
\pgfpathlineto{\pgfqpoint{6.674830in}{1.833153in}}%
\pgfpathlineto{\pgfqpoint{6.680272in}{1.723290in}}%
\pgfpathlineto{\pgfqpoint{6.685714in}{1.776538in}}%
\pgfpathlineto{\pgfqpoint{6.691156in}{2.004283in}}%
\pgfpathlineto{\pgfqpoint{6.696599in}{2.030968in}}%
\pgfpathlineto{\pgfqpoint{6.702041in}{2.420484in}}%
\pgfpathlineto{\pgfqpoint{6.707483in}{1.715641in}}%
\pgfpathlineto{\pgfqpoint{6.712925in}{1.842517in}}%
\pgfpathlineto{\pgfqpoint{6.718367in}{1.667016in}}%
\pgfpathlineto{\pgfqpoint{6.723810in}{1.598401in}}%
\pgfpathlineto{\pgfqpoint{6.729252in}{1.734402in}}%
\pgfpathlineto{\pgfqpoint{6.734694in}{1.726068in}}%
\pgfpathlineto{\pgfqpoint{6.740136in}{1.700867in}}%
\pgfpathlineto{\pgfqpoint{6.745578in}{1.766147in}}%
\pgfpathlineto{\pgfqpoint{6.751020in}{1.740545in}}%
\pgfpathlineto{\pgfqpoint{6.756463in}{2.036532in}}%
\pgfpathlineto{\pgfqpoint{6.761905in}{1.540370in}}%
\pgfpathlineto{\pgfqpoint{6.767347in}{1.608806in}}%
\pgfpathlineto{\pgfqpoint{6.772789in}{2.024680in}}%
\pgfpathlineto{\pgfqpoint{6.778231in}{1.966194in}}%
\pgfpathlineto{\pgfqpoint{6.783673in}{1.641747in}}%
\pgfpathlineto{\pgfqpoint{6.789116in}{1.614183in}}%
\pgfpathlineto{\pgfqpoint{6.794558in}{1.810735in}}%
\pgfpathlineto{\pgfqpoint{6.794558in}{1.810735in}}%
\pgfusepath{stroke}%
\end{pgfscope}%
\begin{pgfscope}%
\pgfsetrectcap%
\pgfsetmiterjoin%
\pgfsetlinewidth{0.803000pt}%
\definecolor{currentstroke}{rgb}{0.000000,0.000000,0.000000}%
\pgfsetstrokecolor{currentstroke}%
\pgfsetdash{}{0pt}%
\pgfpathmoveto{\pgfqpoint{1.200000in}{0.900000in}}%
\pgfpathlineto{\pgfqpoint{1.200000in}{5.700000in}}%
\pgfusepath{stroke}%
\end{pgfscope}%
\begin{pgfscope}%
\pgfsetrectcap%
\pgfsetmiterjoin%
\pgfsetlinewidth{0.803000pt}%
\definecolor{currentstroke}{rgb}{0.000000,0.000000,0.000000}%
\pgfsetstrokecolor{currentstroke}%
\pgfsetdash{}{0pt}%
\pgfpathmoveto{\pgfqpoint{6.800000in}{0.900000in}}%
\pgfpathlineto{\pgfqpoint{6.800000in}{5.700000in}}%
\pgfusepath{stroke}%
\end{pgfscope}%
\begin{pgfscope}%
\pgfsetrectcap%
\pgfsetmiterjoin%
\pgfsetlinewidth{0.803000pt}%
\definecolor{currentstroke}{rgb}{0.000000,0.000000,0.000000}%
\pgfsetstrokecolor{currentstroke}%
\pgfsetdash{}{0pt}%
\pgfpathmoveto{\pgfqpoint{1.200000in}{0.900000in}}%
\pgfpathlineto{\pgfqpoint{6.800000in}{0.900000in}}%
\pgfusepath{stroke}%
\end{pgfscope}%
\begin{pgfscope}%
\pgfsetrectcap%
\pgfsetmiterjoin%
\pgfsetlinewidth{0.803000pt}%
\definecolor{currentstroke}{rgb}{0.000000,0.000000,0.000000}%
\pgfsetstrokecolor{currentstroke}%
\pgfsetdash{}{0pt}%
\pgfpathmoveto{\pgfqpoint{1.200000in}{5.700000in}}%
\pgfpathlineto{\pgfqpoint{6.800000in}{5.700000in}}%
\pgfusepath{stroke}%
\end{pgfscope}%
\begin{pgfscope}%
\pgfsetbuttcap%
\pgfsetmiterjoin%
\definecolor{currentfill}{rgb}{1.000000,1.000000,1.000000}%
\pgfsetfillcolor{currentfill}%
\pgfsetfillopacity{0.800000}%
\pgfsetlinewidth{1.003750pt}%
\definecolor{currentstroke}{rgb}{0.800000,0.800000,0.800000}%
\pgfsetstrokecolor{currentstroke}%
\pgfsetstrokeopacity{0.800000}%
\pgfsetdash{}{0pt}%
\pgfpathmoveto{\pgfqpoint{4.576872in}{5.082821in}}%
\pgfpathlineto{\pgfqpoint{6.605556in}{5.082821in}}%
\pgfpathquadraticcurveto{\pgfqpoint{6.661111in}{5.082821in}}{\pgfqpoint{6.661111in}{5.138377in}}%
\pgfpathlineto{\pgfqpoint{6.661111in}{5.505556in}}%
\pgfpathquadraticcurveto{\pgfqpoint{6.661111in}{5.561111in}}{\pgfqpoint{6.605556in}{5.561111in}}%
\pgfpathlineto{\pgfqpoint{4.576872in}{5.561111in}}%
\pgfpathquadraticcurveto{\pgfqpoint{4.521317in}{5.561111in}}{\pgfqpoint{4.521317in}{5.505556in}}%
\pgfpathlineto{\pgfqpoint{4.521317in}{5.138377in}}%
\pgfpathquadraticcurveto{\pgfqpoint{4.521317in}{5.082821in}}{\pgfqpoint{4.576872in}{5.082821in}}%
\pgfpathclose%
\pgfusepath{stroke,fill}%
\end{pgfscope}%
\begin{pgfscope}%
\pgfsetrectcap%
\pgfsetroundjoin%
\pgfsetlinewidth{2.007500pt}%
\definecolor{currentstroke}{rgb}{0.121569,0.466667,0.705882}%
\pgfsetstrokecolor{currentstroke}%
\pgfsetdash{}{0pt}%
\pgfpathmoveto{\pgfqpoint{4.632428in}{5.347184in}}%
\pgfpathlineto{\pgfqpoint{5.187983in}{5.347184in}}%
\pgfusepath{stroke}%
\end{pgfscope}%
\begin{pgfscope}%
\definecolor{textcolor}{rgb}{0.000000,0.000000,0.000000}%
\pgfsetstrokecolor{textcolor}%
\pgfsetfillcolor{textcolor}%
\pgftext[x=5.410206in,y=5.249962in,left,base]{\color{textcolor}\sffamily\fontsize{20.000000}{24.000000}\selectfont Waveform}%
\end{pgfscope}%
\end{pgfpicture}%
\makeatother%
\endgroup%
}
    \caption{Input Waveform (Pedestal free)}
\end{figure}
\column{0.5\textwidth}
\begin{figure}
    \centering
    \resizebox{1.0\textwidth}{!}{version https://git-lfs.github.com/spec/v1
oid sha256:76aad3cd3a221db068ca7d3f21ed885701ef6d7588d2e8c5b1d6997fdde95144
size 33987
}
    \caption{Output Time and Charge $\hat\phi(t)$}
\end{figure}
\end{columns}
\begin{align*}
  \tilde{\phi}(t) &= \sum_{i=1}^{N_{\mathrm{PE}}} q_i \delta(t-t_i), \ N_{\mathrm{PE}}\sim \mathrm{Poisson}(\mu) \\
  w(t) &= \tilde{\phi}(t) \otimes V_\mathrm{PE}(t) + \epsilon(t) = \sum_{i=1}^{N_\mathrm{PE}} q_i V_\mathrm{PE}(t-t_i) + \epsilon(t)
\end{align*}
\end{frame}

\section{Evaluation criteria}

\begin{frame}
\frametitle{Evaluation criteria}
$\tilde{\phi}(t)$ (simulation result) is an approximation of $\phi(t)$ (time profile). 

$\hat{\phi}(t)$ (reconstruction result) should be consistent with $\tilde{\phi}(t)$. 

Several evaluation criteria are needed. 
\begin{block}{}
\begin{equation*}
    \hat{\phi}(t) \leftrightarrow \tilde{\phi}(t)
\end{equation*}
\end{block}
\begin{itemize}
    \item Residual sum square between $\hat{w}(t)$ and $w(t)$
    \item Wasserstein distance between $\hat{\phi}(t)$ and $\tilde{\phi}(t)$
\end{itemize}
\end{frame}

\section{Fourier deconvolution}

\begin{frame}
\frametitle{Fourier deconvolution}
\begin{align*}
  \mathcal{F}[w] &= \mathcal{F}[\tilde{\phi}]\mathcal{F}[V_\mathrm{PE}] + \mathcal{F}[\epsilon],\Rightarrow \hat{\phi}'(t) = \mathcal{F}^{-1}\left[\frac{R \mathcal{F}[w]}{\mathcal{F}[V_\mathrm{PE}]}\right](t) \\
  \hat{\phi}(t) &= \hat{\alpha}\hat{\phi}'(t),\hat{\alpha} = \arg \underset{\alpha'}{\min}\mathrm{RSS}\left[\alpha'\hat{\phi}'(t)\otimes V_\mathrm{PE}(t),w(t)\right]
\end{align*}
\begin{figure}
    \centering
    \resizebox{0.55\textwidth}{!}{%% Creator: Matplotlib, PGF backend
%%
%% To include the figure in your LaTeX document, write
%%   \input{<filename>.pgf}
%%
%% Make sure the required packages are loaded in your preamble
%%   \usepackage{pgf}
%%
%% Also ensure that all the required font packages are loaded; for instance,
%% the lmodern package is sometimes necessary when using math font.
%%   \usepackage{lmodern}
%%
%% Figures using additional raster images can only be included by \input if
%% they are in the same directory as the main LaTeX file. For loading figures
%% from other directories you can use the `import` package
%%   \usepackage{import}
%%
%% and then include the figures with
%%   \import{<path to file>}{<filename>.pgf}
%%
%% Matplotlib used the following preamble
%%   \usepackage[detect-all,locale=DE]{siunitx}
%%
\begingroup%
\makeatletter%
\begin{pgfpicture}%
\pgfpathrectangle{\pgfpointorigin}{\pgfqpoint{8.000000in}{6.000000in}}%
\pgfusepath{use as bounding box, clip}%
\begin{pgfscope}%
\pgfsetbuttcap%
\pgfsetmiterjoin%
\definecolor{currentfill}{rgb}{1.000000,1.000000,1.000000}%
\pgfsetfillcolor{currentfill}%
\pgfsetlinewidth{0.000000pt}%
\definecolor{currentstroke}{rgb}{1.000000,1.000000,1.000000}%
\pgfsetstrokecolor{currentstroke}%
\pgfsetdash{}{0pt}%
\pgfpathmoveto{\pgfqpoint{0.000000in}{0.000000in}}%
\pgfpathlineto{\pgfqpoint{8.000000in}{0.000000in}}%
\pgfpathlineto{\pgfqpoint{8.000000in}{6.000000in}}%
\pgfpathlineto{\pgfqpoint{0.000000in}{6.000000in}}%
\pgfpathlineto{\pgfqpoint{0.000000in}{0.000000in}}%
\pgfpathclose%
\pgfusepath{fill}%
\end{pgfscope}%
\begin{pgfscope}%
\pgfsetbuttcap%
\pgfsetmiterjoin%
\definecolor{currentfill}{rgb}{1.000000,1.000000,1.000000}%
\pgfsetfillcolor{currentfill}%
\pgfsetlinewidth{0.000000pt}%
\definecolor{currentstroke}{rgb}{0.000000,0.000000,0.000000}%
\pgfsetstrokecolor{currentstroke}%
\pgfsetstrokeopacity{0.000000}%
\pgfsetdash{}{0pt}%
\pgfpathmoveto{\pgfqpoint{1.000000in}{0.720000in}}%
\pgfpathlineto{\pgfqpoint{7.200000in}{0.720000in}}%
\pgfpathlineto{\pgfqpoint{7.200000in}{5.340000in}}%
\pgfpathlineto{\pgfqpoint{1.000000in}{5.340000in}}%
\pgfpathlineto{\pgfqpoint{1.000000in}{0.720000in}}%
\pgfpathclose%
\pgfusepath{fill}%
\end{pgfscope}%
\begin{pgfscope}%
\pgfsetbuttcap%
\pgfsetroundjoin%
\definecolor{currentfill}{rgb}{0.000000,0.000000,0.000000}%
\pgfsetfillcolor{currentfill}%
\pgfsetlinewidth{0.803000pt}%
\definecolor{currentstroke}{rgb}{0.000000,0.000000,0.000000}%
\pgfsetstrokecolor{currentstroke}%
\pgfsetdash{}{0pt}%
\pgfsys@defobject{currentmarker}{\pgfqpoint{0.000000in}{-0.048611in}}{\pgfqpoint{0.000000in}{0.000000in}}{%
\pgfpathmoveto{\pgfqpoint{0.000000in}{0.000000in}}%
\pgfpathlineto{\pgfqpoint{0.000000in}{-0.048611in}}%
\pgfusepath{stroke,fill}%
}%
\begin{pgfscope}%
\pgfsys@transformshift{1.310000in}{0.720000in}%
\pgfsys@useobject{currentmarker}{}%
\end{pgfscope}%
\end{pgfscope}%
\begin{pgfscope}%
\definecolor{textcolor}{rgb}{0.000000,0.000000,0.000000}%
\pgfsetstrokecolor{textcolor}%
\pgfsetfillcolor{textcolor}%
\pgftext[x=1.310000in,y=0.622778in,,top]{\color{textcolor}\sffamily\fontsize{20.000000}{24.000000}\selectfont \(\displaystyle {450}\)}%
\end{pgfscope}%
\begin{pgfscope}%
\pgfsetbuttcap%
\pgfsetroundjoin%
\definecolor{currentfill}{rgb}{0.000000,0.000000,0.000000}%
\pgfsetfillcolor{currentfill}%
\pgfsetlinewidth{0.803000pt}%
\definecolor{currentstroke}{rgb}{0.000000,0.000000,0.000000}%
\pgfsetstrokecolor{currentstroke}%
\pgfsetdash{}{0pt}%
\pgfsys@defobject{currentmarker}{\pgfqpoint{0.000000in}{-0.048611in}}{\pgfqpoint{0.000000in}{0.000000in}}{%
\pgfpathmoveto{\pgfqpoint{0.000000in}{0.000000in}}%
\pgfpathlineto{\pgfqpoint{0.000000in}{-0.048611in}}%
\pgfusepath{stroke,fill}%
}%
\begin{pgfscope}%
\pgfsys@transformshift{2.860000in}{0.720000in}%
\pgfsys@useobject{currentmarker}{}%
\end{pgfscope}%
\end{pgfscope}%
\begin{pgfscope}%
\definecolor{textcolor}{rgb}{0.000000,0.000000,0.000000}%
\pgfsetstrokecolor{textcolor}%
\pgfsetfillcolor{textcolor}%
\pgftext[x=2.860000in,y=0.622778in,,top]{\color{textcolor}\sffamily\fontsize{20.000000}{24.000000}\selectfont \(\displaystyle {500}\)}%
\end{pgfscope}%
\begin{pgfscope}%
\pgfsetbuttcap%
\pgfsetroundjoin%
\definecolor{currentfill}{rgb}{0.000000,0.000000,0.000000}%
\pgfsetfillcolor{currentfill}%
\pgfsetlinewidth{0.803000pt}%
\definecolor{currentstroke}{rgb}{0.000000,0.000000,0.000000}%
\pgfsetstrokecolor{currentstroke}%
\pgfsetdash{}{0pt}%
\pgfsys@defobject{currentmarker}{\pgfqpoint{0.000000in}{-0.048611in}}{\pgfqpoint{0.000000in}{0.000000in}}{%
\pgfpathmoveto{\pgfqpoint{0.000000in}{0.000000in}}%
\pgfpathlineto{\pgfqpoint{0.000000in}{-0.048611in}}%
\pgfusepath{stroke,fill}%
}%
\begin{pgfscope}%
\pgfsys@transformshift{4.410000in}{0.720000in}%
\pgfsys@useobject{currentmarker}{}%
\end{pgfscope}%
\end{pgfscope}%
\begin{pgfscope}%
\definecolor{textcolor}{rgb}{0.000000,0.000000,0.000000}%
\pgfsetstrokecolor{textcolor}%
\pgfsetfillcolor{textcolor}%
\pgftext[x=4.410000in,y=0.622778in,,top]{\color{textcolor}\sffamily\fontsize{20.000000}{24.000000}\selectfont \(\displaystyle {550}\)}%
\end{pgfscope}%
\begin{pgfscope}%
\pgfsetbuttcap%
\pgfsetroundjoin%
\definecolor{currentfill}{rgb}{0.000000,0.000000,0.000000}%
\pgfsetfillcolor{currentfill}%
\pgfsetlinewidth{0.803000pt}%
\definecolor{currentstroke}{rgb}{0.000000,0.000000,0.000000}%
\pgfsetstrokecolor{currentstroke}%
\pgfsetdash{}{0pt}%
\pgfsys@defobject{currentmarker}{\pgfqpoint{0.000000in}{-0.048611in}}{\pgfqpoint{0.000000in}{0.000000in}}{%
\pgfpathmoveto{\pgfqpoint{0.000000in}{0.000000in}}%
\pgfpathlineto{\pgfqpoint{0.000000in}{-0.048611in}}%
\pgfusepath{stroke,fill}%
}%
\begin{pgfscope}%
\pgfsys@transformshift{5.960000in}{0.720000in}%
\pgfsys@useobject{currentmarker}{}%
\end{pgfscope}%
\end{pgfscope}%
\begin{pgfscope}%
\definecolor{textcolor}{rgb}{0.000000,0.000000,0.000000}%
\pgfsetstrokecolor{textcolor}%
\pgfsetfillcolor{textcolor}%
\pgftext[x=5.960000in,y=0.622778in,,top]{\color{textcolor}\sffamily\fontsize{20.000000}{24.000000}\selectfont \(\displaystyle {600}\)}%
\end{pgfscope}%
\begin{pgfscope}%
\definecolor{textcolor}{rgb}{0.000000,0.000000,0.000000}%
\pgfsetstrokecolor{textcolor}%
\pgfsetfillcolor{textcolor}%
\pgftext[x=4.100000in,y=0.311155in,,top]{\color{textcolor}\sffamily\fontsize{20.000000}{24.000000}\selectfont \(\displaystyle \mathrm{t}/\si{ns}\)}%
\end{pgfscope}%
\begin{pgfscope}%
\pgfsetbuttcap%
\pgfsetroundjoin%
\definecolor{currentfill}{rgb}{0.000000,0.000000,0.000000}%
\pgfsetfillcolor{currentfill}%
\pgfsetlinewidth{0.803000pt}%
\definecolor{currentstroke}{rgb}{0.000000,0.000000,0.000000}%
\pgfsetstrokecolor{currentstroke}%
\pgfsetdash{}{0pt}%
\pgfsys@defobject{currentmarker}{\pgfqpoint{-0.048611in}{0.000000in}}{\pgfqpoint{-0.000000in}{0.000000in}}{%
\pgfpathmoveto{\pgfqpoint{-0.000000in}{0.000000in}}%
\pgfpathlineto{\pgfqpoint{-0.048611in}{0.000000in}}%
\pgfusepath{stroke,fill}%
}%
\begin{pgfscope}%
\pgfsys@transformshift{1.000000in}{1.052772in}%
\pgfsys@useobject{currentmarker}{}%
\end{pgfscope}%
\end{pgfscope}%
\begin{pgfscope}%
\definecolor{textcolor}{rgb}{0.000000,0.000000,0.000000}%
\pgfsetstrokecolor{textcolor}%
\pgfsetfillcolor{textcolor}%
\pgftext[x=0.770670in, y=0.952752in, left, base]{\color{textcolor}\sffamily\fontsize{20.000000}{24.000000}\selectfont \(\displaystyle {0}\)}%
\end{pgfscope}%
\begin{pgfscope}%
\pgfsetbuttcap%
\pgfsetroundjoin%
\definecolor{currentfill}{rgb}{0.000000,0.000000,0.000000}%
\pgfsetfillcolor{currentfill}%
\pgfsetlinewidth{0.803000pt}%
\definecolor{currentstroke}{rgb}{0.000000,0.000000,0.000000}%
\pgfsetstrokecolor{currentstroke}%
\pgfsetdash{}{0pt}%
\pgfsys@defobject{currentmarker}{\pgfqpoint{-0.048611in}{0.000000in}}{\pgfqpoint{-0.000000in}{0.000000in}}{%
\pgfpathmoveto{\pgfqpoint{-0.000000in}{0.000000in}}%
\pgfpathlineto{\pgfqpoint{-0.048611in}{0.000000in}}%
\pgfusepath{stroke,fill}%
}%
\begin{pgfscope}%
\pgfsys@transformshift{1.000000in}{1.818629in}%
\pgfsys@useobject{currentmarker}{}%
\end{pgfscope}%
\end{pgfscope}%
\begin{pgfscope}%
\definecolor{textcolor}{rgb}{0.000000,0.000000,0.000000}%
\pgfsetstrokecolor{textcolor}%
\pgfsetfillcolor{textcolor}%
\pgftext[x=0.638563in, y=1.718609in, left, base]{\color{textcolor}\sffamily\fontsize{20.000000}{24.000000}\selectfont \(\displaystyle {10}\)}%
\end{pgfscope}%
\begin{pgfscope}%
\pgfsetbuttcap%
\pgfsetroundjoin%
\definecolor{currentfill}{rgb}{0.000000,0.000000,0.000000}%
\pgfsetfillcolor{currentfill}%
\pgfsetlinewidth{0.803000pt}%
\definecolor{currentstroke}{rgb}{0.000000,0.000000,0.000000}%
\pgfsetstrokecolor{currentstroke}%
\pgfsetdash{}{0pt}%
\pgfsys@defobject{currentmarker}{\pgfqpoint{-0.048611in}{0.000000in}}{\pgfqpoint{-0.000000in}{0.000000in}}{%
\pgfpathmoveto{\pgfqpoint{-0.000000in}{0.000000in}}%
\pgfpathlineto{\pgfqpoint{-0.048611in}{0.000000in}}%
\pgfusepath{stroke,fill}%
}%
\begin{pgfscope}%
\pgfsys@transformshift{1.000000in}{2.584485in}%
\pgfsys@useobject{currentmarker}{}%
\end{pgfscope}%
\end{pgfscope}%
\begin{pgfscope}%
\definecolor{textcolor}{rgb}{0.000000,0.000000,0.000000}%
\pgfsetstrokecolor{textcolor}%
\pgfsetfillcolor{textcolor}%
\pgftext[x=0.638563in, y=2.484466in, left, base]{\color{textcolor}\sffamily\fontsize{20.000000}{24.000000}\selectfont \(\displaystyle {20}\)}%
\end{pgfscope}%
\begin{pgfscope}%
\pgfsetbuttcap%
\pgfsetroundjoin%
\definecolor{currentfill}{rgb}{0.000000,0.000000,0.000000}%
\pgfsetfillcolor{currentfill}%
\pgfsetlinewidth{0.803000pt}%
\definecolor{currentstroke}{rgb}{0.000000,0.000000,0.000000}%
\pgfsetstrokecolor{currentstroke}%
\pgfsetdash{}{0pt}%
\pgfsys@defobject{currentmarker}{\pgfqpoint{-0.048611in}{0.000000in}}{\pgfqpoint{-0.000000in}{0.000000in}}{%
\pgfpathmoveto{\pgfqpoint{-0.000000in}{0.000000in}}%
\pgfpathlineto{\pgfqpoint{-0.048611in}{0.000000in}}%
\pgfusepath{stroke,fill}%
}%
\begin{pgfscope}%
\pgfsys@transformshift{1.000000in}{3.350342in}%
\pgfsys@useobject{currentmarker}{}%
\end{pgfscope}%
\end{pgfscope}%
\begin{pgfscope}%
\definecolor{textcolor}{rgb}{0.000000,0.000000,0.000000}%
\pgfsetstrokecolor{textcolor}%
\pgfsetfillcolor{textcolor}%
\pgftext[x=0.638563in, y=3.250323in, left, base]{\color{textcolor}\sffamily\fontsize{20.000000}{24.000000}\selectfont \(\displaystyle {30}\)}%
\end{pgfscope}%
\begin{pgfscope}%
\pgfsetbuttcap%
\pgfsetroundjoin%
\definecolor{currentfill}{rgb}{0.000000,0.000000,0.000000}%
\pgfsetfillcolor{currentfill}%
\pgfsetlinewidth{0.803000pt}%
\definecolor{currentstroke}{rgb}{0.000000,0.000000,0.000000}%
\pgfsetstrokecolor{currentstroke}%
\pgfsetdash{}{0pt}%
\pgfsys@defobject{currentmarker}{\pgfqpoint{-0.048611in}{0.000000in}}{\pgfqpoint{-0.000000in}{0.000000in}}{%
\pgfpathmoveto{\pgfqpoint{-0.000000in}{0.000000in}}%
\pgfpathlineto{\pgfqpoint{-0.048611in}{0.000000in}}%
\pgfusepath{stroke,fill}%
}%
\begin{pgfscope}%
\pgfsys@transformshift{1.000000in}{4.116199in}%
\pgfsys@useobject{currentmarker}{}%
\end{pgfscope}%
\end{pgfscope}%
\begin{pgfscope}%
\definecolor{textcolor}{rgb}{0.000000,0.000000,0.000000}%
\pgfsetstrokecolor{textcolor}%
\pgfsetfillcolor{textcolor}%
\pgftext[x=0.638563in, y=4.016180in, left, base]{\color{textcolor}\sffamily\fontsize{20.000000}{24.000000}\selectfont \(\displaystyle {40}\)}%
\end{pgfscope}%
\begin{pgfscope}%
\pgfsetbuttcap%
\pgfsetroundjoin%
\definecolor{currentfill}{rgb}{0.000000,0.000000,0.000000}%
\pgfsetfillcolor{currentfill}%
\pgfsetlinewidth{0.803000pt}%
\definecolor{currentstroke}{rgb}{0.000000,0.000000,0.000000}%
\pgfsetstrokecolor{currentstroke}%
\pgfsetdash{}{0pt}%
\pgfsys@defobject{currentmarker}{\pgfqpoint{-0.048611in}{0.000000in}}{\pgfqpoint{-0.000000in}{0.000000in}}{%
\pgfpathmoveto{\pgfqpoint{-0.000000in}{0.000000in}}%
\pgfpathlineto{\pgfqpoint{-0.048611in}{0.000000in}}%
\pgfusepath{stroke,fill}%
}%
\begin{pgfscope}%
\pgfsys@transformshift{1.000000in}{4.882056in}%
\pgfsys@useobject{currentmarker}{}%
\end{pgfscope}%
\end{pgfscope}%
\begin{pgfscope}%
\definecolor{textcolor}{rgb}{0.000000,0.000000,0.000000}%
\pgfsetstrokecolor{textcolor}%
\pgfsetfillcolor{textcolor}%
\pgftext[x=0.638563in, y=4.782037in, left, base]{\color{textcolor}\sffamily\fontsize{20.000000}{24.000000}\selectfont \(\displaystyle {50}\)}%
\end{pgfscope}%
\begin{pgfscope}%
\definecolor{textcolor}{rgb}{0.000000,0.000000,0.000000}%
\pgfsetstrokecolor{textcolor}%
\pgfsetfillcolor{textcolor}%
\pgftext[x=0.583007in,y=3.030000in,,bottom,rotate=90.000000]{\color{textcolor}\sffamily\fontsize{20.000000}{24.000000}\selectfont \(\displaystyle \mathrm{Voltage}/\si{mV}\)}%
\end{pgfscope}%
\begin{pgfscope}%
\pgfpathrectangle{\pgfqpoint{1.000000in}{0.720000in}}{\pgfqpoint{6.200000in}{4.620000in}}%
\pgfusepath{clip}%
\pgfsetrectcap%
\pgfsetroundjoin%
\pgfsetlinewidth{2.007500pt}%
\definecolor{currentstroke}{rgb}{0.121569,0.466667,0.705882}%
\pgfsetstrokecolor{currentstroke}%
\pgfsetdash{}{0pt}%
\pgfpathmoveto{\pgfqpoint{0.990000in}{1.100612in}}%
\pgfpathlineto{\pgfqpoint{1.000000in}{1.130716in}}%
\pgfpathlineto{\pgfqpoint{1.031000in}{1.022138in}}%
\pgfpathlineto{\pgfqpoint{1.062000in}{1.086034in}}%
\pgfpathlineto{\pgfqpoint{1.093000in}{1.075420in}}%
\pgfpathlineto{\pgfqpoint{1.124000in}{1.063411in}}%
\pgfpathlineto{\pgfqpoint{1.155000in}{1.139278in}}%
\pgfpathlineto{\pgfqpoint{1.186000in}{1.072493in}}%
\pgfpathlineto{\pgfqpoint{1.217000in}{1.019122in}}%
\pgfpathlineto{\pgfqpoint{1.248000in}{1.129193in}}%
\pgfpathlineto{\pgfqpoint{1.279000in}{1.073232in}}%
\pgfpathlineto{\pgfqpoint{1.310000in}{1.007395in}}%
\pgfpathlineto{\pgfqpoint{1.341000in}{1.090822in}}%
\pgfpathlineto{\pgfqpoint{1.372000in}{1.039719in}}%
\pgfpathlineto{\pgfqpoint{1.403000in}{1.221321in}}%
\pgfpathlineto{\pgfqpoint{1.434000in}{1.003874in}}%
\pgfpathlineto{\pgfqpoint{1.465000in}{0.955485in}}%
\pgfpathlineto{\pgfqpoint{1.496000in}{1.021437in}}%
\pgfpathlineto{\pgfqpoint{1.527000in}{0.977454in}}%
\pgfpathlineto{\pgfqpoint{1.558000in}{0.956329in}}%
\pgfpathlineto{\pgfqpoint{1.589000in}{1.045817in}}%
\pgfpathlineto{\pgfqpoint{1.620000in}{1.115416in}}%
\pgfpathlineto{\pgfqpoint{1.651000in}{1.096936in}}%
\pgfpathlineto{\pgfqpoint{1.682000in}{1.186148in}}%
\pgfpathlineto{\pgfqpoint{1.713000in}{1.118577in}}%
\pgfpathlineto{\pgfqpoint{1.744000in}{1.138225in}}%
\pgfpathlineto{\pgfqpoint{1.775000in}{1.149730in}}%
\pgfpathlineto{\pgfqpoint{1.806000in}{1.123672in}}%
\pgfpathlineto{\pgfqpoint{1.837000in}{1.037821in}}%
\pgfpathlineto{\pgfqpoint{1.868000in}{0.959427in}}%
\pgfpathlineto{\pgfqpoint{1.899000in}{1.068286in}}%
\pgfpathlineto{\pgfqpoint{1.930000in}{1.029535in}}%
\pgfpathlineto{\pgfqpoint{1.961000in}{1.120055in}}%
\pgfpathlineto{\pgfqpoint{1.992000in}{1.115505in}}%
\pgfpathlineto{\pgfqpoint{2.023000in}{1.288344in}}%
\pgfpathlineto{\pgfqpoint{2.054000in}{1.575323in}}%
\pgfpathlineto{\pgfqpoint{2.085000in}{1.907294in}}%
\pgfpathlineto{\pgfqpoint{2.116000in}{2.486206in}}%
\pgfpathlineto{\pgfqpoint{2.147000in}{2.632547in}}%
\pgfpathlineto{\pgfqpoint{2.178000in}{2.960029in}}%
\pgfpathlineto{\pgfqpoint{2.209000in}{3.063591in}}%
\pgfpathlineto{\pgfqpoint{2.240000in}{2.872606in}}%
\pgfpathlineto{\pgfqpoint{2.271000in}{2.858990in}}%
\pgfpathlineto{\pgfqpoint{2.302000in}{2.794983in}}%
\pgfpathlineto{\pgfqpoint{2.333000in}{2.395425in}}%
\pgfpathlineto{\pgfqpoint{2.364000in}{2.257298in}}%
\pgfpathlineto{\pgfqpoint{2.395000in}{2.248403in}}%
\pgfpathlineto{\pgfqpoint{2.426000in}{2.015219in}}%
\pgfpathlineto{\pgfqpoint{2.457000in}{1.710816in}}%
\pgfpathlineto{\pgfqpoint{2.488000in}{1.638443in}}%
\pgfpathlineto{\pgfqpoint{2.519000in}{1.572861in}}%
\pgfpathlineto{\pgfqpoint{2.550000in}{1.537494in}}%
\pgfpathlineto{\pgfqpoint{2.581000in}{1.712502in}}%
\pgfpathlineto{\pgfqpoint{2.612000in}{2.174196in}}%
\pgfpathlineto{\pgfqpoint{2.643000in}{2.367780in}}%
\pgfpathlineto{\pgfqpoint{2.674000in}{2.685052in}}%
\pgfpathlineto{\pgfqpoint{2.705000in}{2.757070in}}%
\pgfpathlineto{\pgfqpoint{2.736000in}{2.665563in}}%
\pgfpathlineto{\pgfqpoint{2.767000in}{2.534534in}}%
\pgfpathlineto{\pgfqpoint{2.798000in}{2.420479in}}%
\pgfpathlineto{\pgfqpoint{2.829000in}{2.287383in}}%
\pgfpathlineto{\pgfqpoint{2.860000in}{2.193163in}}%
\pgfpathlineto{\pgfqpoint{2.891000in}{1.967785in}}%
\pgfpathlineto{\pgfqpoint{2.922000in}{2.005156in}}%
\pgfpathlineto{\pgfqpoint{2.953000in}{1.844007in}}%
\pgfpathlineto{\pgfqpoint{2.984000in}{2.262390in}}%
\pgfpathlineto{\pgfqpoint{3.015000in}{2.267634in}}%
\pgfpathlineto{\pgfqpoint{3.046000in}{2.526199in}}%
\pgfpathlineto{\pgfqpoint{3.077000in}{2.423202in}}%
\pgfpathlineto{\pgfqpoint{3.108000in}{2.440993in}}%
\pgfpathlineto{\pgfqpoint{3.139000in}{2.377451in}}%
\pgfpathlineto{\pgfqpoint{3.170000in}{2.166923in}}%
\pgfpathlineto{\pgfqpoint{3.201000in}{2.076260in}}%
\pgfpathlineto{\pgfqpoint{3.232000in}{1.858344in}}%
\pgfpathlineto{\pgfqpoint{3.263000in}{1.743472in}}%
\pgfpathlineto{\pgfqpoint{3.294000in}{1.618143in}}%
\pgfpathlineto{\pgfqpoint{3.325000in}{1.443283in}}%
\pgfpathlineto{\pgfqpoint{3.356000in}{1.540715in}}%
\pgfpathlineto{\pgfqpoint{3.387000in}{1.552445in}}%
\pgfpathlineto{\pgfqpoint{3.418000in}{1.369516in}}%
\pgfpathlineto{\pgfqpoint{3.449000in}{1.208146in}}%
\pgfpathlineto{\pgfqpoint{3.480000in}{1.170985in}}%
\pgfpathlineto{\pgfqpoint{3.511000in}{1.341185in}}%
\pgfpathlineto{\pgfqpoint{3.542000in}{1.156865in}}%
\pgfpathlineto{\pgfqpoint{3.573000in}{1.139129in}}%
\pgfpathlineto{\pgfqpoint{3.604000in}{1.240519in}}%
\pgfpathlineto{\pgfqpoint{3.635000in}{1.105758in}}%
\pgfpathlineto{\pgfqpoint{3.666000in}{1.070891in}}%
\pgfpathlineto{\pgfqpoint{3.697000in}{1.125221in}}%
\pgfpathlineto{\pgfqpoint{3.728000in}{1.065673in}}%
\pgfpathlineto{\pgfqpoint{3.759000in}{1.019976in}}%
\pgfpathlineto{\pgfqpoint{3.790000in}{1.231862in}}%
\pgfpathlineto{\pgfqpoint{3.821000in}{1.310141in}}%
\pgfpathlineto{\pgfqpoint{3.852000in}{1.691497in}}%
\pgfpathlineto{\pgfqpoint{3.883000in}{1.867872in}}%
\pgfpathlineto{\pgfqpoint{3.914000in}{2.237925in}}%
\pgfpathlineto{\pgfqpoint{3.945000in}{2.205396in}}%
\pgfpathlineto{\pgfqpoint{3.976000in}{2.437826in}}%
\pgfpathlineto{\pgfqpoint{4.007000in}{2.334158in}}%
\pgfpathlineto{\pgfqpoint{4.038000in}{2.051514in}}%
\pgfpathlineto{\pgfqpoint{4.069000in}{2.050528in}}%
\pgfpathlineto{\pgfqpoint{4.100000in}{1.950072in}}%
\pgfpathlineto{\pgfqpoint{4.131000in}{1.778733in}}%
\pgfpathlineto{\pgfqpoint{4.162000in}{1.635407in}}%
\pgfpathlineto{\pgfqpoint{4.193000in}{1.585899in}}%
\pgfpathlineto{\pgfqpoint{4.224000in}{1.621912in}}%
\pgfpathlineto{\pgfqpoint{4.255000in}{1.569903in}}%
\pgfpathlineto{\pgfqpoint{4.286000in}{1.325018in}}%
\pgfpathlineto{\pgfqpoint{4.317000in}{1.300172in}}%
\pgfpathlineto{\pgfqpoint{4.348000in}{1.254855in}}%
\pgfpathlineto{\pgfqpoint{4.379000in}{1.189848in}}%
\pgfpathlineto{\pgfqpoint{4.410000in}{1.099117in}}%
\pgfpathlineto{\pgfqpoint{4.441000in}{1.238658in}}%
\pgfpathlineto{\pgfqpoint{4.472000in}{1.154265in}}%
\pgfpathlineto{\pgfqpoint{4.534000in}{1.053827in}}%
\pgfpathlineto{\pgfqpoint{4.565000in}{1.000511in}}%
\pgfpathlineto{\pgfqpoint{4.596000in}{1.175382in}}%
\pgfpathlineto{\pgfqpoint{4.627000in}{1.187017in}}%
\pgfpathlineto{\pgfqpoint{4.658000in}{1.044281in}}%
\pgfpathlineto{\pgfqpoint{4.689000in}{1.143404in}}%
\pgfpathlineto{\pgfqpoint{4.720000in}{0.931941in}}%
\pgfpathlineto{\pgfqpoint{4.751000in}{1.047570in}}%
\pgfpathlineto{\pgfqpoint{4.782000in}{1.066145in}}%
\pgfpathlineto{\pgfqpoint{4.813000in}{0.966620in}}%
\pgfpathlineto{\pgfqpoint{4.844000in}{1.013558in}}%
\pgfpathlineto{\pgfqpoint{4.875000in}{0.831600in}}%
\pgfpathlineto{\pgfqpoint{4.906000in}{0.968350in}}%
\pgfpathlineto{\pgfqpoint{4.937000in}{0.989001in}}%
\pgfpathlineto{\pgfqpoint{4.968000in}{0.909335in}}%
\pgfpathlineto{\pgfqpoint{4.999000in}{1.190833in}}%
\pgfpathlineto{\pgfqpoint{5.030000in}{1.097748in}}%
\pgfpathlineto{\pgfqpoint{5.061000in}{1.048285in}}%
\pgfpathlineto{\pgfqpoint{5.092000in}{1.112534in}}%
\pgfpathlineto{\pgfqpoint{5.123000in}{0.936752in}}%
\pgfpathlineto{\pgfqpoint{5.154000in}{1.006248in}}%
\pgfpathlineto{\pgfqpoint{5.185000in}{0.965726in}}%
\pgfpathlineto{\pgfqpoint{5.216000in}{0.906055in}}%
\pgfpathlineto{\pgfqpoint{5.247000in}{1.045277in}}%
\pgfpathlineto{\pgfqpoint{5.278000in}{1.063358in}}%
\pgfpathlineto{\pgfqpoint{5.309000in}{1.122769in}}%
\pgfpathlineto{\pgfqpoint{5.371000in}{1.035582in}}%
\pgfpathlineto{\pgfqpoint{5.402000in}{1.094788in}}%
\pgfpathlineto{\pgfqpoint{5.433000in}{1.113773in}}%
\pgfpathlineto{\pgfqpoint{5.464000in}{1.004020in}}%
\pgfpathlineto{\pgfqpoint{5.495000in}{0.980666in}}%
\pgfpathlineto{\pgfqpoint{5.557000in}{1.030566in}}%
\pgfpathlineto{\pgfqpoint{5.588000in}{0.974603in}}%
\pgfpathlineto{\pgfqpoint{5.619000in}{0.897133in}}%
\pgfpathlineto{\pgfqpoint{5.650000in}{1.011135in}}%
\pgfpathlineto{\pgfqpoint{5.681000in}{1.014392in}}%
\pgfpathlineto{\pgfqpoint{5.712000in}{1.033590in}}%
\pgfpathlineto{\pgfqpoint{5.743000in}{1.068097in}}%
\pgfpathlineto{\pgfqpoint{5.774000in}{1.166458in}}%
\pgfpathlineto{\pgfqpoint{5.805000in}{1.044539in}}%
\pgfpathlineto{\pgfqpoint{5.836000in}{0.993816in}}%
\pgfpathlineto{\pgfqpoint{5.867000in}{1.015808in}}%
\pgfpathlineto{\pgfqpoint{5.898000in}{1.142811in}}%
\pgfpathlineto{\pgfqpoint{5.929000in}{1.134654in}}%
\pgfpathlineto{\pgfqpoint{5.960000in}{1.009296in}}%
\pgfpathlineto{\pgfqpoint{5.991000in}{0.983817in}}%
\pgfpathlineto{\pgfqpoint{6.022000in}{1.028485in}}%
\pgfpathlineto{\pgfqpoint{6.053000in}{0.967387in}}%
\pgfpathlineto{\pgfqpoint{6.084000in}{0.986750in}}%
\pgfpathlineto{\pgfqpoint{6.115000in}{1.037221in}}%
\pgfpathlineto{\pgfqpoint{6.146000in}{1.065972in}}%
\pgfpathlineto{\pgfqpoint{6.177000in}{1.023621in}}%
\pgfpathlineto{\pgfqpoint{6.208000in}{1.098473in}}%
\pgfpathlineto{\pgfqpoint{6.239000in}{0.980835in}}%
\pgfpathlineto{\pgfqpoint{6.270000in}{1.119766in}}%
\pgfpathlineto{\pgfqpoint{6.301000in}{0.949802in}}%
\pgfpathlineto{\pgfqpoint{6.332000in}{0.999191in}}%
\pgfpathlineto{\pgfqpoint{6.363000in}{1.014614in}}%
\pgfpathlineto{\pgfqpoint{6.394000in}{1.009653in}}%
\pgfpathlineto{\pgfqpoint{6.425000in}{1.171848in}}%
\pgfpathlineto{\pgfqpoint{6.456000in}{1.064845in}}%
\pgfpathlineto{\pgfqpoint{6.487000in}{1.130760in}}%
\pgfpathlineto{\pgfqpoint{6.518000in}{1.092438in}}%
\pgfpathlineto{\pgfqpoint{6.549000in}{1.098535in}}%
\pgfpathlineto{\pgfqpoint{6.580000in}{1.060740in}}%
\pgfpathlineto{\pgfqpoint{6.611000in}{0.987094in}}%
\pgfpathlineto{\pgfqpoint{6.642000in}{1.100946in}}%
\pgfpathlineto{\pgfqpoint{6.673000in}{1.023950in}}%
\pgfpathlineto{\pgfqpoint{6.704000in}{0.923463in}}%
\pgfpathlineto{\pgfqpoint{6.735000in}{1.050455in}}%
\pgfpathlineto{\pgfqpoint{6.766000in}{1.124624in}}%
\pgfpathlineto{\pgfqpoint{6.797000in}{1.100845in}}%
\pgfpathlineto{\pgfqpoint{6.828000in}{1.093127in}}%
\pgfpathlineto{\pgfqpoint{6.859000in}{0.915853in}}%
\pgfpathlineto{\pgfqpoint{6.890000in}{1.061273in}}%
\pgfpathlineto{\pgfqpoint{6.921000in}{0.869013in}}%
\pgfpathlineto{\pgfqpoint{6.952000in}{1.061253in}}%
\pgfpathlineto{\pgfqpoint{6.983000in}{0.957975in}}%
\pgfpathlineto{\pgfqpoint{7.014000in}{1.044416in}}%
\pgfpathlineto{\pgfqpoint{7.045000in}{1.066431in}}%
\pgfpathlineto{\pgfqpoint{7.076000in}{0.995885in}}%
\pgfpathlineto{\pgfqpoint{7.107000in}{0.986293in}}%
\pgfpathlineto{\pgfqpoint{7.138000in}{1.103229in}}%
\pgfpathlineto{\pgfqpoint{7.169000in}{1.192663in}}%
\pgfpathlineto{\pgfqpoint{7.200000in}{1.102344in}}%
\pgfpathlineto{\pgfqpoint{7.210000in}{1.049353in}}%
\pgfpathlineto{\pgfqpoint{7.210000in}{1.049353in}}%
\pgfusepath{stroke}%
\end{pgfscope}%
\begin{pgfscope}%
\pgfpathrectangle{\pgfqpoint{1.000000in}{0.720000in}}{\pgfqpoint{6.200000in}{4.620000in}}%
\pgfusepath{clip}%
\pgfsetbuttcap%
\pgfsetroundjoin%
\pgfsetlinewidth{2.007500pt}%
\definecolor{currentstroke}{rgb}{0.000000,0.500000,0.000000}%
\pgfsetstrokecolor{currentstroke}%
\pgfsetdash{}{0pt}%
\pgfpathmoveto{\pgfqpoint{0.990000in}{1.435700in}}%
\pgfpathlineto{\pgfqpoint{7.210000in}{1.435700in}}%
\pgfusepath{stroke}%
\end{pgfscope}%
\begin{pgfscope}%
\pgfsetrectcap%
\pgfsetmiterjoin%
\pgfsetlinewidth{0.803000pt}%
\definecolor{currentstroke}{rgb}{0.000000,0.000000,0.000000}%
\pgfsetstrokecolor{currentstroke}%
\pgfsetdash{}{0pt}%
\pgfpathmoveto{\pgfqpoint{1.000000in}{0.720000in}}%
\pgfpathlineto{\pgfqpoint{1.000000in}{5.340000in}}%
\pgfusepath{stroke}%
\end{pgfscope}%
\begin{pgfscope}%
\pgfsetrectcap%
\pgfsetmiterjoin%
\pgfsetlinewidth{0.803000pt}%
\definecolor{currentstroke}{rgb}{0.000000,0.000000,0.000000}%
\pgfsetstrokecolor{currentstroke}%
\pgfsetdash{}{0pt}%
\pgfpathmoveto{\pgfqpoint{7.200000in}{0.720000in}}%
\pgfpathlineto{\pgfqpoint{7.200000in}{5.340000in}}%
\pgfusepath{stroke}%
\end{pgfscope}%
\begin{pgfscope}%
\pgfsetrectcap%
\pgfsetmiterjoin%
\pgfsetlinewidth{0.803000pt}%
\definecolor{currentstroke}{rgb}{0.000000,0.000000,0.000000}%
\pgfsetstrokecolor{currentstroke}%
\pgfsetdash{}{0pt}%
\pgfpathmoveto{\pgfqpoint{1.000000in}{0.720000in}}%
\pgfpathlineto{\pgfqpoint{7.200000in}{0.720000in}}%
\pgfusepath{stroke}%
\end{pgfscope}%
\begin{pgfscope}%
\pgfsetrectcap%
\pgfsetmiterjoin%
\pgfsetlinewidth{0.803000pt}%
\definecolor{currentstroke}{rgb}{0.000000,0.000000,0.000000}%
\pgfsetstrokecolor{currentstroke}%
\pgfsetdash{}{0pt}%
\pgfpathmoveto{\pgfqpoint{1.000000in}{5.340000in}}%
\pgfpathlineto{\pgfqpoint{7.200000in}{5.340000in}}%
\pgfusepath{stroke}%
\end{pgfscope}%
\begin{pgfscope}%
\pgfsetbuttcap%
\pgfsetroundjoin%
\definecolor{currentfill}{rgb}{0.000000,0.000000,0.000000}%
\pgfsetfillcolor{currentfill}%
\pgfsetlinewidth{0.803000pt}%
\definecolor{currentstroke}{rgb}{0.000000,0.000000,0.000000}%
\pgfsetstrokecolor{currentstroke}%
\pgfsetdash{}{0pt}%
\pgfsys@defobject{currentmarker}{\pgfqpoint{0.000000in}{0.000000in}}{\pgfqpoint{0.048611in}{0.000000in}}{%
\pgfpathmoveto{\pgfqpoint{0.000000in}{0.000000in}}%
\pgfpathlineto{\pgfqpoint{0.048611in}{0.000000in}}%
\pgfusepath{stroke,fill}%
}%
\begin{pgfscope}%
\pgfsys@transformshift{7.200000in}{1.052772in}%
\pgfsys@useobject{currentmarker}{}%
\end{pgfscope}%
\end{pgfscope}%
\begin{pgfscope}%
\definecolor{textcolor}{rgb}{0.000000,0.000000,0.000000}%
\pgfsetstrokecolor{textcolor}%
\pgfsetfillcolor{textcolor}%
\pgftext[x=7.297222in, y=0.952752in, left, base]{\color{textcolor}\sffamily\fontsize{20.000000}{24.000000}\selectfont 0.0}%
\end{pgfscope}%
\begin{pgfscope}%
\pgfsetbuttcap%
\pgfsetroundjoin%
\definecolor{currentfill}{rgb}{0.000000,0.000000,0.000000}%
\pgfsetfillcolor{currentfill}%
\pgfsetlinewidth{0.803000pt}%
\definecolor{currentstroke}{rgb}{0.000000,0.000000,0.000000}%
\pgfsetstrokecolor{currentstroke}%
\pgfsetdash{}{0pt}%
\pgfsys@defobject{currentmarker}{\pgfqpoint{0.000000in}{0.000000in}}{\pgfqpoint{0.048611in}{0.000000in}}{%
\pgfpathmoveto{\pgfqpoint{0.000000in}{0.000000in}}%
\pgfpathlineto{\pgfqpoint{0.048611in}{0.000000in}}%
\pgfusepath{stroke,fill}%
}%
\begin{pgfscope}%
\pgfsys@transformshift{7.200000in}{1.683246in}%
\pgfsys@useobject{currentmarker}{}%
\end{pgfscope}%
\end{pgfscope}%
\begin{pgfscope}%
\definecolor{textcolor}{rgb}{0.000000,0.000000,0.000000}%
\pgfsetstrokecolor{textcolor}%
\pgfsetfillcolor{textcolor}%
\pgftext[x=7.297222in, y=1.583227in, left, base]{\color{textcolor}\sffamily\fontsize{20.000000}{24.000000}\selectfont 0.2}%
\end{pgfscope}%
\begin{pgfscope}%
\pgfsetbuttcap%
\pgfsetroundjoin%
\definecolor{currentfill}{rgb}{0.000000,0.000000,0.000000}%
\pgfsetfillcolor{currentfill}%
\pgfsetlinewidth{0.803000pt}%
\definecolor{currentstroke}{rgb}{0.000000,0.000000,0.000000}%
\pgfsetstrokecolor{currentstroke}%
\pgfsetdash{}{0pt}%
\pgfsys@defobject{currentmarker}{\pgfqpoint{0.000000in}{0.000000in}}{\pgfqpoint{0.048611in}{0.000000in}}{%
\pgfpathmoveto{\pgfqpoint{0.000000in}{0.000000in}}%
\pgfpathlineto{\pgfqpoint{0.048611in}{0.000000in}}%
\pgfusepath{stroke,fill}%
}%
\begin{pgfscope}%
\pgfsys@transformshift{7.200000in}{2.313721in}%
\pgfsys@useobject{currentmarker}{}%
\end{pgfscope}%
\end{pgfscope}%
\begin{pgfscope}%
\definecolor{textcolor}{rgb}{0.000000,0.000000,0.000000}%
\pgfsetstrokecolor{textcolor}%
\pgfsetfillcolor{textcolor}%
\pgftext[x=7.297222in, y=2.213702in, left, base]{\color{textcolor}\sffamily\fontsize{20.000000}{24.000000}\selectfont 0.5}%
\end{pgfscope}%
\begin{pgfscope}%
\pgfsetbuttcap%
\pgfsetroundjoin%
\definecolor{currentfill}{rgb}{0.000000,0.000000,0.000000}%
\pgfsetfillcolor{currentfill}%
\pgfsetlinewidth{0.803000pt}%
\definecolor{currentstroke}{rgb}{0.000000,0.000000,0.000000}%
\pgfsetstrokecolor{currentstroke}%
\pgfsetdash{}{0pt}%
\pgfsys@defobject{currentmarker}{\pgfqpoint{0.000000in}{0.000000in}}{\pgfqpoint{0.048611in}{0.000000in}}{%
\pgfpathmoveto{\pgfqpoint{0.000000in}{0.000000in}}%
\pgfpathlineto{\pgfqpoint{0.048611in}{0.000000in}}%
\pgfusepath{stroke,fill}%
}%
\begin{pgfscope}%
\pgfsys@transformshift{7.200000in}{2.944196in}%
\pgfsys@useobject{currentmarker}{}%
\end{pgfscope}%
\end{pgfscope}%
\begin{pgfscope}%
\definecolor{textcolor}{rgb}{0.000000,0.000000,0.000000}%
\pgfsetstrokecolor{textcolor}%
\pgfsetfillcolor{textcolor}%
\pgftext[x=7.297222in, y=2.844177in, left, base]{\color{textcolor}\sffamily\fontsize{20.000000}{24.000000}\selectfont 0.8}%
\end{pgfscope}%
\begin{pgfscope}%
\pgfsetbuttcap%
\pgfsetroundjoin%
\definecolor{currentfill}{rgb}{0.000000,0.000000,0.000000}%
\pgfsetfillcolor{currentfill}%
\pgfsetlinewidth{0.803000pt}%
\definecolor{currentstroke}{rgb}{0.000000,0.000000,0.000000}%
\pgfsetstrokecolor{currentstroke}%
\pgfsetdash{}{0pt}%
\pgfsys@defobject{currentmarker}{\pgfqpoint{0.000000in}{0.000000in}}{\pgfqpoint{0.048611in}{0.000000in}}{%
\pgfpathmoveto{\pgfqpoint{0.000000in}{0.000000in}}%
\pgfpathlineto{\pgfqpoint{0.048611in}{0.000000in}}%
\pgfusepath{stroke,fill}%
}%
\begin{pgfscope}%
\pgfsys@transformshift{7.200000in}{3.574671in}%
\pgfsys@useobject{currentmarker}{}%
\end{pgfscope}%
\end{pgfscope}%
\begin{pgfscope}%
\definecolor{textcolor}{rgb}{0.000000,0.000000,0.000000}%
\pgfsetstrokecolor{textcolor}%
\pgfsetfillcolor{textcolor}%
\pgftext[x=7.297222in, y=3.474651in, left, base]{\color{textcolor}\sffamily\fontsize{20.000000}{24.000000}\selectfont 1.0}%
\end{pgfscope}%
\begin{pgfscope}%
\pgfsetbuttcap%
\pgfsetroundjoin%
\definecolor{currentfill}{rgb}{0.000000,0.000000,0.000000}%
\pgfsetfillcolor{currentfill}%
\pgfsetlinewidth{0.803000pt}%
\definecolor{currentstroke}{rgb}{0.000000,0.000000,0.000000}%
\pgfsetstrokecolor{currentstroke}%
\pgfsetdash{}{0pt}%
\pgfsys@defobject{currentmarker}{\pgfqpoint{0.000000in}{0.000000in}}{\pgfqpoint{0.048611in}{0.000000in}}{%
\pgfpathmoveto{\pgfqpoint{0.000000in}{0.000000in}}%
\pgfpathlineto{\pgfqpoint{0.048611in}{0.000000in}}%
\pgfusepath{stroke,fill}%
}%
\begin{pgfscope}%
\pgfsys@transformshift{7.200000in}{4.205145in}%
\pgfsys@useobject{currentmarker}{}%
\end{pgfscope}%
\end{pgfscope}%
\begin{pgfscope}%
\definecolor{textcolor}{rgb}{0.000000,0.000000,0.000000}%
\pgfsetstrokecolor{textcolor}%
\pgfsetfillcolor{textcolor}%
\pgftext[x=7.297222in, y=4.105126in, left, base]{\color{textcolor}\sffamily\fontsize{20.000000}{24.000000}\selectfont 1.2}%
\end{pgfscope}%
\begin{pgfscope}%
\pgfsetbuttcap%
\pgfsetroundjoin%
\definecolor{currentfill}{rgb}{0.000000,0.000000,0.000000}%
\pgfsetfillcolor{currentfill}%
\pgfsetlinewidth{0.803000pt}%
\definecolor{currentstroke}{rgb}{0.000000,0.000000,0.000000}%
\pgfsetstrokecolor{currentstroke}%
\pgfsetdash{}{0pt}%
\pgfsys@defobject{currentmarker}{\pgfqpoint{0.000000in}{0.000000in}}{\pgfqpoint{0.048611in}{0.000000in}}{%
\pgfpathmoveto{\pgfqpoint{0.000000in}{0.000000in}}%
\pgfpathlineto{\pgfqpoint{0.048611in}{0.000000in}}%
\pgfusepath{stroke,fill}%
}%
\begin{pgfscope}%
\pgfsys@transformshift{7.200000in}{4.835620in}%
\pgfsys@useobject{currentmarker}{}%
\end{pgfscope}%
\end{pgfscope}%
\begin{pgfscope}%
\definecolor{textcolor}{rgb}{0.000000,0.000000,0.000000}%
\pgfsetstrokecolor{textcolor}%
\pgfsetfillcolor{textcolor}%
\pgftext[x=7.297222in, y=4.735601in, left, base]{\color{textcolor}\sffamily\fontsize{20.000000}{24.000000}\selectfont 1.5}%
\end{pgfscope}%
\begin{pgfscope}%
\definecolor{textcolor}{rgb}{0.000000,0.000000,0.000000}%
\pgfsetstrokecolor{textcolor}%
\pgfsetfillcolor{textcolor}%
\pgftext[x=7.698906in,y=3.030000in,,top,rotate=90.000000]{\color{textcolor}\sffamily\fontsize{20.000000}{24.000000}\selectfont \(\displaystyle \mathrm{Charge}\)}%
\end{pgfscope}%
\begin{pgfscope}%
\pgfpathrectangle{\pgfqpoint{1.000000in}{0.720000in}}{\pgfqpoint{6.200000in}{4.620000in}}%
\pgfusepath{clip}%
\pgfsetbuttcap%
\pgfsetroundjoin%
\pgfsetlinewidth{0.501875pt}%
\definecolor{currentstroke}{rgb}{1.000000,0.000000,0.000000}%
\pgfsetstrokecolor{currentstroke}%
\pgfsetdash{}{0pt}%
\pgfpathmoveto{\pgfqpoint{1.837000in}{1.052772in}}%
\pgfpathlineto{\pgfqpoint{1.837000in}{1.392998in}}%
\pgfusepath{stroke}%
\end{pgfscope}%
\begin{pgfscope}%
\pgfpathrectangle{\pgfqpoint{1.000000in}{0.720000in}}{\pgfqpoint{6.200000in}{4.620000in}}%
\pgfusepath{clip}%
\pgfsetbuttcap%
\pgfsetroundjoin%
\pgfsetlinewidth{0.501875pt}%
\definecolor{currentstroke}{rgb}{1.000000,0.000000,0.000000}%
\pgfsetstrokecolor{currentstroke}%
\pgfsetdash{}{0pt}%
\pgfpathmoveto{\pgfqpoint{1.868000in}{1.052772in}}%
\pgfpathlineto{\pgfqpoint{1.868000in}{1.558565in}}%
\pgfusepath{stroke}%
\end{pgfscope}%
\begin{pgfscope}%
\pgfpathrectangle{\pgfqpoint{1.000000in}{0.720000in}}{\pgfqpoint{6.200000in}{4.620000in}}%
\pgfusepath{clip}%
\pgfsetbuttcap%
\pgfsetroundjoin%
\pgfsetlinewidth{0.501875pt}%
\definecolor{currentstroke}{rgb}{1.000000,0.000000,0.000000}%
\pgfsetstrokecolor{currentstroke}%
\pgfsetdash{}{0pt}%
\pgfpathmoveto{\pgfqpoint{1.899000in}{1.052772in}}%
\pgfpathlineto{\pgfqpoint{1.899000in}{1.691569in}}%
\pgfusepath{stroke}%
\end{pgfscope}%
\begin{pgfscope}%
\pgfpathrectangle{\pgfqpoint{1.000000in}{0.720000in}}{\pgfqpoint{6.200000in}{4.620000in}}%
\pgfusepath{clip}%
\pgfsetbuttcap%
\pgfsetroundjoin%
\pgfsetlinewidth{0.501875pt}%
\definecolor{currentstroke}{rgb}{1.000000,0.000000,0.000000}%
\pgfsetstrokecolor{currentstroke}%
\pgfsetdash{}{0pt}%
\pgfpathmoveto{\pgfqpoint{1.930000in}{1.052772in}}%
\pgfpathlineto{\pgfqpoint{1.930000in}{1.771864in}}%
\pgfusepath{stroke}%
\end{pgfscope}%
\begin{pgfscope}%
\pgfpathrectangle{\pgfqpoint{1.000000in}{0.720000in}}{\pgfqpoint{6.200000in}{4.620000in}}%
\pgfusepath{clip}%
\pgfsetbuttcap%
\pgfsetroundjoin%
\pgfsetlinewidth{0.501875pt}%
\definecolor{currentstroke}{rgb}{1.000000,0.000000,0.000000}%
\pgfsetstrokecolor{currentstroke}%
\pgfsetdash{}{0pt}%
\pgfpathmoveto{\pgfqpoint{1.961000in}{1.052772in}}%
\pgfpathlineto{\pgfqpoint{1.961000in}{1.789505in}}%
\pgfusepath{stroke}%
\end{pgfscope}%
\begin{pgfscope}%
\pgfpathrectangle{\pgfqpoint{1.000000in}{0.720000in}}{\pgfqpoint{6.200000in}{4.620000in}}%
\pgfusepath{clip}%
\pgfsetbuttcap%
\pgfsetroundjoin%
\pgfsetlinewidth{0.501875pt}%
\definecolor{currentstroke}{rgb}{1.000000,0.000000,0.000000}%
\pgfsetstrokecolor{currentstroke}%
\pgfsetdash{}{0pt}%
\pgfpathmoveto{\pgfqpoint{1.992000in}{1.052772in}}%
\pgfpathlineto{\pgfqpoint{1.992000in}{1.745187in}}%
\pgfusepath{stroke}%
\end{pgfscope}%
\begin{pgfscope}%
\pgfpathrectangle{\pgfqpoint{1.000000in}{0.720000in}}{\pgfqpoint{6.200000in}{4.620000in}}%
\pgfusepath{clip}%
\pgfsetbuttcap%
\pgfsetroundjoin%
\pgfsetlinewidth{0.501875pt}%
\definecolor{currentstroke}{rgb}{1.000000,0.000000,0.000000}%
\pgfsetstrokecolor{currentstroke}%
\pgfsetdash{}{0pt}%
\pgfpathmoveto{\pgfqpoint{2.023000in}{1.052772in}}%
\pgfpathlineto{\pgfqpoint{2.023000in}{1.648230in}}%
\pgfusepath{stroke}%
\end{pgfscope}%
\begin{pgfscope}%
\pgfpathrectangle{\pgfqpoint{1.000000in}{0.720000in}}{\pgfqpoint{6.200000in}{4.620000in}}%
\pgfusepath{clip}%
\pgfsetbuttcap%
\pgfsetroundjoin%
\pgfsetlinewidth{0.501875pt}%
\definecolor{currentstroke}{rgb}{1.000000,0.000000,0.000000}%
\pgfsetstrokecolor{currentstroke}%
\pgfsetdash{}{0pt}%
\pgfpathmoveto{\pgfqpoint{2.054000in}{1.052772in}}%
\pgfpathlineto{\pgfqpoint{2.054000in}{1.513298in}}%
\pgfusepath{stroke}%
\end{pgfscope}%
\begin{pgfscope}%
\pgfpathrectangle{\pgfqpoint{1.000000in}{0.720000in}}{\pgfqpoint{6.200000in}{4.620000in}}%
\pgfusepath{clip}%
\pgfsetbuttcap%
\pgfsetroundjoin%
\pgfsetlinewidth{0.501875pt}%
\definecolor{currentstroke}{rgb}{1.000000,0.000000,0.000000}%
\pgfsetstrokecolor{currentstroke}%
\pgfsetdash{}{0pt}%
\pgfpathmoveto{\pgfqpoint{2.085000in}{1.052772in}}%
\pgfpathlineto{\pgfqpoint{2.085000in}{1.357257in}}%
\pgfusepath{stroke}%
\end{pgfscope}%
\begin{pgfscope}%
\pgfpathrectangle{\pgfqpoint{1.000000in}{0.720000in}}{\pgfqpoint{6.200000in}{4.620000in}}%
\pgfusepath{clip}%
\pgfsetbuttcap%
\pgfsetroundjoin%
\pgfsetlinewidth{0.501875pt}%
\definecolor{currentstroke}{rgb}{1.000000,0.000000,0.000000}%
\pgfsetstrokecolor{currentstroke}%
\pgfsetdash{}{0pt}%
\pgfpathmoveto{\pgfqpoint{2.364000in}{1.052772in}}%
\pgfpathlineto{\pgfqpoint{2.364000in}{1.335911in}}%
\pgfusepath{stroke}%
\end{pgfscope}%
\begin{pgfscope}%
\pgfpathrectangle{\pgfqpoint{1.000000in}{0.720000in}}{\pgfqpoint{6.200000in}{4.620000in}}%
\pgfusepath{clip}%
\pgfsetbuttcap%
\pgfsetroundjoin%
\pgfsetlinewidth{0.501875pt}%
\definecolor{currentstroke}{rgb}{1.000000,0.000000,0.000000}%
\pgfsetstrokecolor{currentstroke}%
\pgfsetdash{}{0pt}%
\pgfpathmoveto{\pgfqpoint{2.395000in}{1.052772in}}%
\pgfpathlineto{\pgfqpoint{2.395000in}{1.518930in}}%
\pgfusepath{stroke}%
\end{pgfscope}%
\begin{pgfscope}%
\pgfpathrectangle{\pgfqpoint{1.000000in}{0.720000in}}{\pgfqpoint{6.200000in}{4.620000in}}%
\pgfusepath{clip}%
\pgfsetbuttcap%
\pgfsetroundjoin%
\pgfsetlinewidth{0.501875pt}%
\definecolor{currentstroke}{rgb}{1.000000,0.000000,0.000000}%
\pgfsetstrokecolor{currentstroke}%
\pgfsetdash{}{0pt}%
\pgfpathmoveto{\pgfqpoint{2.426000in}{1.052772in}}%
\pgfpathlineto{\pgfqpoint{2.426000in}{1.660348in}}%
\pgfusepath{stroke}%
\end{pgfscope}%
\begin{pgfscope}%
\pgfpathrectangle{\pgfqpoint{1.000000in}{0.720000in}}{\pgfqpoint{6.200000in}{4.620000in}}%
\pgfusepath{clip}%
\pgfsetbuttcap%
\pgfsetroundjoin%
\pgfsetlinewidth{0.501875pt}%
\definecolor{currentstroke}{rgb}{1.000000,0.000000,0.000000}%
\pgfsetstrokecolor{currentstroke}%
\pgfsetdash{}{0pt}%
\pgfpathmoveto{\pgfqpoint{2.457000in}{1.052772in}}%
\pgfpathlineto{\pgfqpoint{2.457000in}{1.729364in}}%
\pgfusepath{stroke}%
\end{pgfscope}%
\begin{pgfscope}%
\pgfpathrectangle{\pgfqpoint{1.000000in}{0.720000in}}{\pgfqpoint{6.200000in}{4.620000in}}%
\pgfusepath{clip}%
\pgfsetbuttcap%
\pgfsetroundjoin%
\pgfsetlinewidth{0.501875pt}%
\definecolor{currentstroke}{rgb}{1.000000,0.000000,0.000000}%
\pgfsetstrokecolor{currentstroke}%
\pgfsetdash{}{0pt}%
\pgfpathmoveto{\pgfqpoint{2.488000in}{1.052772in}}%
\pgfpathlineto{\pgfqpoint{2.488000in}{1.709178in}}%
\pgfusepath{stroke}%
\end{pgfscope}%
\begin{pgfscope}%
\pgfpathrectangle{\pgfqpoint{1.000000in}{0.720000in}}{\pgfqpoint{6.200000in}{4.620000in}}%
\pgfusepath{clip}%
\pgfsetbuttcap%
\pgfsetroundjoin%
\pgfsetlinewidth{0.501875pt}%
\definecolor{currentstroke}{rgb}{1.000000,0.000000,0.000000}%
\pgfsetstrokecolor{currentstroke}%
\pgfsetdash{}{0pt}%
\pgfpathmoveto{\pgfqpoint{2.519000in}{1.052772in}}%
\pgfpathlineto{\pgfqpoint{2.519000in}{1.602789in}}%
\pgfusepath{stroke}%
\end{pgfscope}%
\begin{pgfscope}%
\pgfpathrectangle{\pgfqpoint{1.000000in}{0.720000in}}{\pgfqpoint{6.200000in}{4.620000in}}%
\pgfusepath{clip}%
\pgfsetbuttcap%
\pgfsetroundjoin%
\pgfsetlinewidth{0.501875pt}%
\definecolor{currentstroke}{rgb}{1.000000,0.000000,0.000000}%
\pgfsetstrokecolor{currentstroke}%
\pgfsetdash{}{0pt}%
\pgfpathmoveto{\pgfqpoint{2.550000in}{1.052772in}}%
\pgfpathlineto{\pgfqpoint{2.550000in}{1.433895in}}%
\pgfusepath{stroke}%
\end{pgfscope}%
\begin{pgfscope}%
\pgfpathrectangle{\pgfqpoint{1.000000in}{0.720000in}}{\pgfqpoint{6.200000in}{4.620000in}}%
\pgfusepath{clip}%
\pgfsetbuttcap%
\pgfsetroundjoin%
\pgfsetlinewidth{0.501875pt}%
\definecolor{currentstroke}{rgb}{1.000000,0.000000,0.000000}%
\pgfsetstrokecolor{currentstroke}%
\pgfsetdash{}{0pt}%
\pgfpathmoveto{\pgfqpoint{2.767000in}{1.052772in}}%
\pgfpathlineto{\pgfqpoint{2.767000in}{1.317021in}}%
\pgfusepath{stroke}%
\end{pgfscope}%
\begin{pgfscope}%
\pgfpathrectangle{\pgfqpoint{1.000000in}{0.720000in}}{\pgfqpoint{6.200000in}{4.620000in}}%
\pgfusepath{clip}%
\pgfsetbuttcap%
\pgfsetroundjoin%
\pgfsetlinewidth{0.501875pt}%
\definecolor{currentstroke}{rgb}{1.000000,0.000000,0.000000}%
\pgfsetstrokecolor{currentstroke}%
\pgfsetdash{}{0pt}%
\pgfpathmoveto{\pgfqpoint{2.798000in}{1.052772in}}%
\pgfpathlineto{\pgfqpoint{2.798000in}{1.463499in}}%
\pgfusepath{stroke}%
\end{pgfscope}%
\begin{pgfscope}%
\pgfpathrectangle{\pgfqpoint{1.000000in}{0.720000in}}{\pgfqpoint{6.200000in}{4.620000in}}%
\pgfusepath{clip}%
\pgfsetbuttcap%
\pgfsetroundjoin%
\pgfsetlinewidth{0.501875pt}%
\definecolor{currentstroke}{rgb}{1.000000,0.000000,0.000000}%
\pgfsetstrokecolor{currentstroke}%
\pgfsetdash{}{0pt}%
\pgfpathmoveto{\pgfqpoint{2.829000in}{1.052772in}}%
\pgfpathlineto{\pgfqpoint{2.829000in}{1.555210in}}%
\pgfusepath{stroke}%
\end{pgfscope}%
\begin{pgfscope}%
\pgfpathrectangle{\pgfqpoint{1.000000in}{0.720000in}}{\pgfqpoint{6.200000in}{4.620000in}}%
\pgfusepath{clip}%
\pgfsetbuttcap%
\pgfsetroundjoin%
\pgfsetlinewidth{0.501875pt}%
\definecolor{currentstroke}{rgb}{1.000000,0.000000,0.000000}%
\pgfsetstrokecolor{currentstroke}%
\pgfsetdash{}{0pt}%
\pgfpathmoveto{\pgfqpoint{2.860000in}{1.052772in}}%
\pgfpathlineto{\pgfqpoint{2.860000in}{1.570312in}}%
\pgfusepath{stroke}%
\end{pgfscope}%
\begin{pgfscope}%
\pgfpathrectangle{\pgfqpoint{1.000000in}{0.720000in}}{\pgfqpoint{6.200000in}{4.620000in}}%
\pgfusepath{clip}%
\pgfsetbuttcap%
\pgfsetroundjoin%
\pgfsetlinewidth{0.501875pt}%
\definecolor{currentstroke}{rgb}{1.000000,0.000000,0.000000}%
\pgfsetstrokecolor{currentstroke}%
\pgfsetdash{}{0pt}%
\pgfpathmoveto{\pgfqpoint{2.891000in}{1.052772in}}%
\pgfpathlineto{\pgfqpoint{2.891000in}{1.507960in}}%
\pgfusepath{stroke}%
\end{pgfscope}%
\begin{pgfscope}%
\pgfpathrectangle{\pgfqpoint{1.000000in}{0.720000in}}{\pgfqpoint{6.200000in}{4.620000in}}%
\pgfusepath{clip}%
\pgfsetbuttcap%
\pgfsetroundjoin%
\pgfsetlinewidth{0.501875pt}%
\definecolor{currentstroke}{rgb}{1.000000,0.000000,0.000000}%
\pgfsetstrokecolor{currentstroke}%
\pgfsetdash{}{0pt}%
\pgfpathmoveto{\pgfqpoint{2.922000in}{1.052772in}}%
\pgfpathlineto{\pgfqpoint{2.922000in}{1.386910in}}%
\pgfusepath{stroke}%
\end{pgfscope}%
\begin{pgfscope}%
\pgfpathrectangle{\pgfqpoint{1.000000in}{0.720000in}}{\pgfqpoint{6.200000in}{4.620000in}}%
\pgfusepath{clip}%
\pgfsetbuttcap%
\pgfsetroundjoin%
\pgfsetlinewidth{0.501875pt}%
\definecolor{currentstroke}{rgb}{1.000000,0.000000,0.000000}%
\pgfsetstrokecolor{currentstroke}%
\pgfsetdash{}{0pt}%
\pgfpathmoveto{\pgfqpoint{3.635000in}{1.052772in}}%
\pgfpathlineto{\pgfqpoint{3.635000in}{1.355506in}}%
\pgfusepath{stroke}%
\end{pgfscope}%
\begin{pgfscope}%
\pgfpathrectangle{\pgfqpoint{1.000000in}{0.720000in}}{\pgfqpoint{6.200000in}{4.620000in}}%
\pgfusepath{clip}%
\pgfsetbuttcap%
\pgfsetroundjoin%
\pgfsetlinewidth{0.501875pt}%
\definecolor{currentstroke}{rgb}{1.000000,0.000000,0.000000}%
\pgfsetstrokecolor{currentstroke}%
\pgfsetdash{}{0pt}%
\pgfpathmoveto{\pgfqpoint{3.666000in}{1.052772in}}%
\pgfpathlineto{\pgfqpoint{3.666000in}{1.468666in}}%
\pgfusepath{stroke}%
\end{pgfscope}%
\begin{pgfscope}%
\pgfpathrectangle{\pgfqpoint{1.000000in}{0.720000in}}{\pgfqpoint{6.200000in}{4.620000in}}%
\pgfusepath{clip}%
\pgfsetbuttcap%
\pgfsetroundjoin%
\pgfsetlinewidth{0.501875pt}%
\definecolor{currentstroke}{rgb}{1.000000,0.000000,0.000000}%
\pgfsetstrokecolor{currentstroke}%
\pgfsetdash{}{0pt}%
\pgfpathmoveto{\pgfqpoint{3.697000in}{1.052772in}}%
\pgfpathlineto{\pgfqpoint{3.697000in}{1.550501in}}%
\pgfusepath{stroke}%
\end{pgfscope}%
\begin{pgfscope}%
\pgfpathrectangle{\pgfqpoint{1.000000in}{0.720000in}}{\pgfqpoint{6.200000in}{4.620000in}}%
\pgfusepath{clip}%
\pgfsetbuttcap%
\pgfsetroundjoin%
\pgfsetlinewidth{0.501875pt}%
\definecolor{currentstroke}{rgb}{1.000000,0.000000,0.000000}%
\pgfsetstrokecolor{currentstroke}%
\pgfsetdash{}{0pt}%
\pgfpathmoveto{\pgfqpoint{3.728000in}{1.052772in}}%
\pgfpathlineto{\pgfqpoint{3.728000in}{1.580546in}}%
\pgfusepath{stroke}%
\end{pgfscope}%
\begin{pgfscope}%
\pgfpathrectangle{\pgfqpoint{1.000000in}{0.720000in}}{\pgfqpoint{6.200000in}{4.620000in}}%
\pgfusepath{clip}%
\pgfsetbuttcap%
\pgfsetroundjoin%
\pgfsetlinewidth{0.501875pt}%
\definecolor{currentstroke}{rgb}{1.000000,0.000000,0.000000}%
\pgfsetstrokecolor{currentstroke}%
\pgfsetdash{}{0pt}%
\pgfpathmoveto{\pgfqpoint{3.759000in}{1.052772in}}%
\pgfpathlineto{\pgfqpoint{3.759000in}{1.548245in}}%
\pgfusepath{stroke}%
\end{pgfscope}%
\begin{pgfscope}%
\pgfpathrectangle{\pgfqpoint{1.000000in}{0.720000in}}{\pgfqpoint{6.200000in}{4.620000in}}%
\pgfusepath{clip}%
\pgfsetbuttcap%
\pgfsetroundjoin%
\pgfsetlinewidth{0.501875pt}%
\definecolor{currentstroke}{rgb}{1.000000,0.000000,0.000000}%
\pgfsetstrokecolor{currentstroke}%
\pgfsetdash{}{0pt}%
\pgfpathmoveto{\pgfqpoint{3.790000in}{1.052772in}}%
\pgfpathlineto{\pgfqpoint{3.790000in}{1.457067in}}%
\pgfusepath{stroke}%
\end{pgfscope}%
\begin{pgfscope}%
\pgfpathrectangle{\pgfqpoint{1.000000in}{0.720000in}}{\pgfqpoint{6.200000in}{4.620000in}}%
\pgfusepath{clip}%
\pgfsetbuttcap%
\pgfsetroundjoin%
\pgfsetlinewidth{0.501875pt}%
\definecolor{currentstroke}{rgb}{1.000000,0.000000,0.000000}%
\pgfsetstrokecolor{currentstroke}%
\pgfsetdash{}{0pt}%
\pgfpathmoveto{\pgfqpoint{3.821000in}{1.052772in}}%
\pgfpathlineto{\pgfqpoint{3.821000in}{1.324688in}}%
\pgfusepath{stroke}%
\end{pgfscope}%
\begin{pgfscope}%
\pgfsetrectcap%
\pgfsetmiterjoin%
\pgfsetlinewidth{0.803000pt}%
\definecolor{currentstroke}{rgb}{0.000000,0.000000,0.000000}%
\pgfsetstrokecolor{currentstroke}%
\pgfsetdash{}{0pt}%
\pgfpathmoveto{\pgfqpoint{1.000000in}{0.720000in}}%
\pgfpathlineto{\pgfqpoint{1.000000in}{5.340000in}}%
\pgfusepath{stroke}%
\end{pgfscope}%
\begin{pgfscope}%
\pgfsetrectcap%
\pgfsetmiterjoin%
\pgfsetlinewidth{0.803000pt}%
\definecolor{currentstroke}{rgb}{0.000000,0.000000,0.000000}%
\pgfsetstrokecolor{currentstroke}%
\pgfsetdash{}{0pt}%
\pgfpathmoveto{\pgfqpoint{7.200000in}{0.720000in}}%
\pgfpathlineto{\pgfqpoint{7.200000in}{5.340000in}}%
\pgfusepath{stroke}%
\end{pgfscope}%
\begin{pgfscope}%
\pgfsetrectcap%
\pgfsetmiterjoin%
\pgfsetlinewidth{0.803000pt}%
\definecolor{currentstroke}{rgb}{0.000000,0.000000,0.000000}%
\pgfsetstrokecolor{currentstroke}%
\pgfsetdash{}{0pt}%
\pgfpathmoveto{\pgfqpoint{1.000000in}{0.720000in}}%
\pgfpathlineto{\pgfqpoint{7.200000in}{0.720000in}}%
\pgfusepath{stroke}%
\end{pgfscope}%
\begin{pgfscope}%
\pgfsetrectcap%
\pgfsetmiterjoin%
\pgfsetlinewidth{0.803000pt}%
\definecolor{currentstroke}{rgb}{0.000000,0.000000,0.000000}%
\pgfsetstrokecolor{currentstroke}%
\pgfsetdash{}{0pt}%
\pgfpathmoveto{\pgfqpoint{1.000000in}{5.340000in}}%
\pgfpathlineto{\pgfqpoint{7.200000in}{5.340000in}}%
\pgfusepath{stroke}%
\end{pgfscope}%
\begin{pgfscope}%
\pgfsetbuttcap%
\pgfsetmiterjoin%
\definecolor{currentfill}{rgb}{1.000000,1.000000,1.000000}%
\pgfsetfillcolor{currentfill}%
\pgfsetfillopacity{0.800000}%
\pgfsetlinewidth{1.003750pt}%
\definecolor{currentstroke}{rgb}{0.800000,0.800000,0.800000}%
\pgfsetstrokecolor{currentstroke}%
\pgfsetstrokeopacity{0.800000}%
\pgfsetdash{}{0pt}%
\pgfpathmoveto{\pgfqpoint{4.976872in}{3.932908in}}%
\pgfpathlineto{\pgfqpoint{7.005556in}{3.932908in}}%
\pgfpathquadraticcurveto{\pgfqpoint{7.061111in}{3.932908in}}{\pgfqpoint{7.061111in}{3.988464in}}%
\pgfpathlineto{\pgfqpoint{7.061111in}{5.145556in}}%
\pgfpathquadraticcurveto{\pgfqpoint{7.061111in}{5.201111in}}{\pgfqpoint{7.005556in}{5.201111in}}%
\pgfpathlineto{\pgfqpoint{4.976872in}{5.201111in}}%
\pgfpathquadraticcurveto{\pgfqpoint{4.921317in}{5.201111in}}{\pgfqpoint{4.921317in}{5.145556in}}%
\pgfpathlineto{\pgfqpoint{4.921317in}{3.988464in}}%
\pgfpathquadraticcurveto{\pgfqpoint{4.921317in}{3.932908in}}{\pgfqpoint{4.976872in}{3.932908in}}%
\pgfpathlineto{\pgfqpoint{4.976872in}{3.932908in}}%
\pgfpathclose%
\pgfusepath{stroke,fill}%
\end{pgfscope}%
\begin{pgfscope}%
\pgfsetrectcap%
\pgfsetroundjoin%
\pgfsetlinewidth{2.007500pt}%
\definecolor{currentstroke}{rgb}{0.121569,0.466667,0.705882}%
\pgfsetstrokecolor{currentstroke}%
\pgfsetdash{}{0pt}%
\pgfpathmoveto{\pgfqpoint{5.032428in}{4.987184in}}%
\pgfpathlineto{\pgfqpoint{5.310206in}{4.987184in}}%
\pgfpathlineto{\pgfqpoint{5.587983in}{4.987184in}}%
\pgfusepath{stroke}%
\end{pgfscope}%
\begin{pgfscope}%
\definecolor{textcolor}{rgb}{0.000000,0.000000,0.000000}%
\pgfsetstrokecolor{textcolor}%
\pgfsetfillcolor{textcolor}%
\pgftext[x=5.810206in,y=4.889962in,left,base]{\color{textcolor}\sffamily\fontsize{20.000000}{24.000000}\selectfont Waveform}%
\end{pgfscope}%
\begin{pgfscope}%
\pgfsetbuttcap%
\pgfsetroundjoin%
\pgfsetlinewidth{2.007500pt}%
\definecolor{currentstroke}{rgb}{0.000000,0.500000,0.000000}%
\pgfsetstrokecolor{currentstroke}%
\pgfsetdash{}{0pt}%
\pgfpathmoveto{\pgfqpoint{5.032428in}{4.592227in}}%
\pgfpathlineto{\pgfqpoint{5.587983in}{4.592227in}}%
\pgfusepath{stroke}%
\end{pgfscope}%
\begin{pgfscope}%
\definecolor{textcolor}{rgb}{0.000000,0.000000,0.000000}%
\pgfsetstrokecolor{textcolor}%
\pgfsetfillcolor{textcolor}%
\pgftext[x=5.810206in,y=4.495005in,left,base]{\color{textcolor}\sffamily\fontsize{20.000000}{24.000000}\selectfont Threshold}%
\end{pgfscope}%
\begin{pgfscope}%
\pgfsetbuttcap%
\pgfsetroundjoin%
\pgfsetlinewidth{0.501875pt}%
\definecolor{currentstroke}{rgb}{1.000000,0.000000,0.000000}%
\pgfsetstrokecolor{currentstroke}%
\pgfsetdash{}{0pt}%
\pgfpathmoveto{\pgfqpoint{5.032428in}{4.197271in}}%
\pgfpathlineto{\pgfqpoint{5.587983in}{4.197271in}}%
\pgfusepath{stroke}%
\end{pgfscope}%
\begin{pgfscope}%
\definecolor{textcolor}{rgb}{0.000000,0.000000,0.000000}%
\pgfsetstrokecolor{textcolor}%
\pgfsetfillcolor{textcolor}%
\pgftext[x=5.810206in,y=4.100048in,left,base]{\color{textcolor}\sffamily\fontsize{20.000000}{24.000000}\selectfont Charge}%
\end{pgfscope}%
\end{pgfpicture}%
\makeatother%
\endgroup%
}
    \caption{$\mathrm{RSS}=\SI{124.7}{mV^2},D_w=\SI{2.03}{ns},\Delta t_0=\SI{-1.16}{ns}$}
\end{figure}
\end{frame}

\section{Fast Bayesian matching pursuit}

\begin{frame}
\frametitle{Model definition}
\begin{itemize}
    \item Time in DAQ window is divided into time bins: $\vec{t}$, whose length is $N$
    \item Model vector: $\vec{z}$. $z_i=0\implies q_i=0$ and $\ z_i=1\implies q_i\neq0$
    \item Linear Model: $\vec{w} = \bm{V}_\mathrm{PE}\vec{z} + \vec{\epsilon}$
    \item 
        \begin{align*}
            \left.
            \begin{bmatrix}
                \vec{w} \\
                \vec{q}
            \end{bmatrix}
            \right\vert\vec{z}
            &\sim \mathrm{Normal}\left(
            \begin{bmatrix}
                \bm{V}_\mathrm{PE}\vec{z} \\
                \vec{z}
            \end{bmatrix}, 
            \begin{bmatrix}
                \bm{\Sigma}_z & \bm{V}_\mathrm{PE}\bm{Z} \\
                \bm{Z}\bm{V}_\mathrm{PE}^\intercal & \bm{Z}
            \end{bmatrix}
            \right) \\
            \bm{\Sigma}_z &= \bm{V}_\mathrm{PE}\bm{Z}\bm{V}_\mathrm{PE}^\intercal+\sigma_\epsilon^2\bm{I}
        \end{align*}
    where $\bm{Z}$ is the diagonal matrix of vector $\vec{z}$ controlling $q_i$ 
    \item $\mathcal{Z}=\{\vec{z}_j\}$ contains \textcolor{red}{$2^{N}$} model vectors
    \item A \textcolor{red}{repeated greedy search}(RGS) is performed to construct the target set $\mathcal{Z}'$, which contains only the $\vec{z}$ giving large $p(\vec{w}|\vec{z})$. 
\end{itemize}
\end{frame}

\begin{frame}
\frametitle{Model definition}
\tikzstyle{na} = [baseline=-.5ex]
\begin{columns}
\column{0.5\textwidth}
\begin{figure}
    \centering
    \resizebox{0.8\textwidth}{!}{%% Creator: Matplotlib, PGF backend
%%
%% To include the figure in your LaTeX document, write
%%   \input{<filename>.pgf}
%%
%% Make sure the required packages are loaded in your preamble
%%   \usepackage{pgf}
%%
%% and, on pdftex
%%   \usepackage[utf8]{inputenc}\DeclareUnicodeCharacter{2212}{-}
%%
%% or, on luatex and xetex
%%   \usepackage{unicode-math}
%%
%% Figures using additional raster images can only be included by \input if
%% they are in the same directory as the main LaTeX file. For loading figures
%% from other directories you can use the `import` package
%%   \usepackage{import}
%%
%% and then include the figures with
%%   \import{<path to file>}{<filename>.pgf}
%%
%% Matplotlib used the following preamble
%%   \usepackage[detect-all,locale=DE]{siunitx}
%%
\begingroup%
\makeatletter%
\begin{pgfpicture}%
\pgfpathrectangle{\pgfpointorigin}{\pgfqpoint{8.000000in}{6.000000in}}%
\pgfusepath{use as bounding box, clip}%
\begin{pgfscope}%
\pgfsetbuttcap%
\pgfsetmiterjoin%
\definecolor{currentfill}{rgb}{1.000000,1.000000,1.000000}%
\pgfsetfillcolor{currentfill}%
\pgfsetlinewidth{0.000000pt}%
\definecolor{currentstroke}{rgb}{1.000000,1.000000,1.000000}%
\pgfsetstrokecolor{currentstroke}%
\pgfsetdash{}{0pt}%
\pgfpathmoveto{\pgfqpoint{0.000000in}{0.000000in}}%
\pgfpathlineto{\pgfqpoint{8.000000in}{0.000000in}}%
\pgfpathlineto{\pgfqpoint{8.000000in}{6.000000in}}%
\pgfpathlineto{\pgfqpoint{0.000000in}{6.000000in}}%
\pgfpathclose%
\pgfusepath{fill}%
\end{pgfscope}%
\begin{pgfscope}%
\pgfsetbuttcap%
\pgfsetmiterjoin%
\definecolor{currentfill}{rgb}{1.000000,1.000000,1.000000}%
\pgfsetfillcolor{currentfill}%
\pgfsetlinewidth{0.000000pt}%
\definecolor{currentstroke}{rgb}{0.000000,0.000000,0.000000}%
\pgfsetstrokecolor{currentstroke}%
\pgfsetstrokeopacity{0.000000}%
\pgfsetdash{}{0pt}%
\pgfpathmoveto{\pgfqpoint{1.000000in}{0.720000in}}%
\pgfpathlineto{\pgfqpoint{7.200000in}{0.720000in}}%
\pgfpathlineto{\pgfqpoint{7.200000in}{5.340000in}}%
\pgfpathlineto{\pgfqpoint{1.000000in}{5.340000in}}%
\pgfpathclose%
\pgfusepath{fill}%
\end{pgfscope}%
\begin{pgfscope}%
\pgfpathrectangle{\pgfqpoint{1.000000in}{0.720000in}}{\pgfqpoint{6.200000in}{4.620000in}}%
\pgfusepath{clip}%
\pgfsetrectcap%
\pgfsetroundjoin%
\pgfsetlinewidth{0.803000pt}%
\definecolor{currentstroke}{rgb}{0.690196,0.690196,0.690196}%
\pgfsetstrokecolor{currentstroke}%
\pgfsetdash{}{0pt}%
\pgfpathmoveto{\pgfqpoint{2.240000in}{0.720000in}}%
\pgfpathlineto{\pgfqpoint{2.240000in}{5.340000in}}%
\pgfusepath{stroke}%
\end{pgfscope}%
\begin{pgfscope}%
\pgfsetbuttcap%
\pgfsetroundjoin%
\definecolor{currentfill}{rgb}{0.000000,0.000000,0.000000}%
\pgfsetfillcolor{currentfill}%
\pgfsetlinewidth{0.803000pt}%
\definecolor{currentstroke}{rgb}{0.000000,0.000000,0.000000}%
\pgfsetstrokecolor{currentstroke}%
\pgfsetdash{}{0pt}%
\pgfsys@defobject{currentmarker}{\pgfqpoint{0.000000in}{-0.048611in}}{\pgfqpoint{0.000000in}{0.000000in}}{%
\pgfpathmoveto{\pgfqpoint{0.000000in}{0.000000in}}%
\pgfpathlineto{\pgfqpoint{0.000000in}{-0.048611in}}%
\pgfusepath{stroke,fill}%
}%
\begin{pgfscope}%
\pgfsys@transformshift{2.240000in}{0.720000in}%
\pgfsys@useobject{currentmarker}{}%
\end{pgfscope}%
\end{pgfscope}%
\begin{pgfscope}%
\definecolor{textcolor}{rgb}{0.000000,0.000000,0.000000}%
\pgfsetstrokecolor{textcolor}%
\pgfsetfillcolor{textcolor}%
\pgftext[x=2.240000in,y=0.622778in,,top]{\color{textcolor}\sffamily\fontsize{20.000000}{24.000000}\selectfont \(\displaystyle {200}\)}%
\end{pgfscope}%
\begin{pgfscope}%
\pgfpathrectangle{\pgfqpoint{1.000000in}{0.720000in}}{\pgfqpoint{6.200000in}{4.620000in}}%
\pgfusepath{clip}%
\pgfsetrectcap%
\pgfsetroundjoin%
\pgfsetlinewidth{0.803000pt}%
\definecolor{currentstroke}{rgb}{0.690196,0.690196,0.690196}%
\pgfsetstrokecolor{currentstroke}%
\pgfsetdash{}{0pt}%
\pgfpathmoveto{\pgfqpoint{3.790000in}{0.720000in}}%
\pgfpathlineto{\pgfqpoint{3.790000in}{5.340000in}}%
\pgfusepath{stroke}%
\end{pgfscope}%
\begin{pgfscope}%
\pgfsetbuttcap%
\pgfsetroundjoin%
\definecolor{currentfill}{rgb}{0.000000,0.000000,0.000000}%
\pgfsetfillcolor{currentfill}%
\pgfsetlinewidth{0.803000pt}%
\definecolor{currentstroke}{rgb}{0.000000,0.000000,0.000000}%
\pgfsetstrokecolor{currentstroke}%
\pgfsetdash{}{0pt}%
\pgfsys@defobject{currentmarker}{\pgfqpoint{0.000000in}{-0.048611in}}{\pgfqpoint{0.000000in}{0.000000in}}{%
\pgfpathmoveto{\pgfqpoint{0.000000in}{0.000000in}}%
\pgfpathlineto{\pgfqpoint{0.000000in}{-0.048611in}}%
\pgfusepath{stroke,fill}%
}%
\begin{pgfscope}%
\pgfsys@transformshift{3.790000in}{0.720000in}%
\pgfsys@useobject{currentmarker}{}%
\end{pgfscope}%
\end{pgfscope}%
\begin{pgfscope}%
\definecolor{textcolor}{rgb}{0.000000,0.000000,0.000000}%
\pgfsetstrokecolor{textcolor}%
\pgfsetfillcolor{textcolor}%
\pgftext[x=3.790000in,y=0.622778in,,top]{\color{textcolor}\sffamily\fontsize{20.000000}{24.000000}\selectfont \(\displaystyle {250}\)}%
\end{pgfscope}%
\begin{pgfscope}%
\pgfpathrectangle{\pgfqpoint{1.000000in}{0.720000in}}{\pgfqpoint{6.200000in}{4.620000in}}%
\pgfusepath{clip}%
\pgfsetrectcap%
\pgfsetroundjoin%
\pgfsetlinewidth{0.803000pt}%
\definecolor{currentstroke}{rgb}{0.690196,0.690196,0.690196}%
\pgfsetstrokecolor{currentstroke}%
\pgfsetdash{}{0pt}%
\pgfpathmoveto{\pgfqpoint{5.340000in}{0.720000in}}%
\pgfpathlineto{\pgfqpoint{5.340000in}{5.340000in}}%
\pgfusepath{stroke}%
\end{pgfscope}%
\begin{pgfscope}%
\pgfsetbuttcap%
\pgfsetroundjoin%
\definecolor{currentfill}{rgb}{0.000000,0.000000,0.000000}%
\pgfsetfillcolor{currentfill}%
\pgfsetlinewidth{0.803000pt}%
\definecolor{currentstroke}{rgb}{0.000000,0.000000,0.000000}%
\pgfsetstrokecolor{currentstroke}%
\pgfsetdash{}{0pt}%
\pgfsys@defobject{currentmarker}{\pgfqpoint{0.000000in}{-0.048611in}}{\pgfqpoint{0.000000in}{0.000000in}}{%
\pgfpathmoveto{\pgfqpoint{0.000000in}{0.000000in}}%
\pgfpathlineto{\pgfqpoint{0.000000in}{-0.048611in}}%
\pgfusepath{stroke,fill}%
}%
\begin{pgfscope}%
\pgfsys@transformshift{5.340000in}{0.720000in}%
\pgfsys@useobject{currentmarker}{}%
\end{pgfscope}%
\end{pgfscope}%
\begin{pgfscope}%
\definecolor{textcolor}{rgb}{0.000000,0.000000,0.000000}%
\pgfsetstrokecolor{textcolor}%
\pgfsetfillcolor{textcolor}%
\pgftext[x=5.340000in,y=0.622778in,,top]{\color{textcolor}\sffamily\fontsize{20.000000}{24.000000}\selectfont \(\displaystyle {300}\)}%
\end{pgfscope}%
\begin{pgfscope}%
\pgfpathrectangle{\pgfqpoint{1.000000in}{0.720000in}}{\pgfqpoint{6.200000in}{4.620000in}}%
\pgfusepath{clip}%
\pgfsetrectcap%
\pgfsetroundjoin%
\pgfsetlinewidth{0.803000pt}%
\definecolor{currentstroke}{rgb}{0.690196,0.690196,0.690196}%
\pgfsetstrokecolor{currentstroke}%
\pgfsetdash{}{0pt}%
\pgfpathmoveto{\pgfqpoint{6.890000in}{0.720000in}}%
\pgfpathlineto{\pgfqpoint{6.890000in}{5.340000in}}%
\pgfusepath{stroke}%
\end{pgfscope}%
\begin{pgfscope}%
\pgfsetbuttcap%
\pgfsetroundjoin%
\definecolor{currentfill}{rgb}{0.000000,0.000000,0.000000}%
\pgfsetfillcolor{currentfill}%
\pgfsetlinewidth{0.803000pt}%
\definecolor{currentstroke}{rgb}{0.000000,0.000000,0.000000}%
\pgfsetstrokecolor{currentstroke}%
\pgfsetdash{}{0pt}%
\pgfsys@defobject{currentmarker}{\pgfqpoint{0.000000in}{-0.048611in}}{\pgfqpoint{0.000000in}{0.000000in}}{%
\pgfpathmoveto{\pgfqpoint{0.000000in}{0.000000in}}%
\pgfpathlineto{\pgfqpoint{0.000000in}{-0.048611in}}%
\pgfusepath{stroke,fill}%
}%
\begin{pgfscope}%
\pgfsys@transformshift{6.890000in}{0.720000in}%
\pgfsys@useobject{currentmarker}{}%
\end{pgfscope}%
\end{pgfscope}%
\begin{pgfscope}%
\definecolor{textcolor}{rgb}{0.000000,0.000000,0.000000}%
\pgfsetstrokecolor{textcolor}%
\pgfsetfillcolor{textcolor}%
\pgftext[x=6.890000in,y=0.622778in,,top]{\color{textcolor}\sffamily\fontsize{20.000000}{24.000000}\selectfont \(\displaystyle {350}\)}%
\end{pgfscope}%
\begin{pgfscope}%
\definecolor{textcolor}{rgb}{0.000000,0.000000,0.000000}%
\pgfsetstrokecolor{textcolor}%
\pgfsetfillcolor{textcolor}%
\pgftext[x=4.100000in,y=0.311155in,,top]{\color{textcolor}\sffamily\fontsize{20.000000}{24.000000}\selectfont \(\displaystyle \mathrm{t}/\si{ns}\)}%
\end{pgfscope}%
\begin{pgfscope}%
\pgfpathrectangle{\pgfqpoint{1.000000in}{0.720000in}}{\pgfqpoint{6.200000in}{4.620000in}}%
\pgfusepath{clip}%
\pgfsetrectcap%
\pgfsetroundjoin%
\pgfsetlinewidth{0.803000pt}%
\definecolor{currentstroke}{rgb}{0.690196,0.690196,0.690196}%
\pgfsetstrokecolor{currentstroke}%
\pgfsetdash{}{0pt}%
\pgfpathmoveto{\pgfqpoint{1.000000in}{0.930000in}}%
\pgfpathlineto{\pgfqpoint{7.200000in}{0.930000in}}%
\pgfusepath{stroke}%
\end{pgfscope}%
\begin{pgfscope}%
\pgfsetbuttcap%
\pgfsetroundjoin%
\definecolor{currentfill}{rgb}{0.000000,0.000000,0.000000}%
\pgfsetfillcolor{currentfill}%
\pgfsetlinewidth{0.803000pt}%
\definecolor{currentstroke}{rgb}{0.000000,0.000000,0.000000}%
\pgfsetstrokecolor{currentstroke}%
\pgfsetdash{}{0pt}%
\pgfsys@defobject{currentmarker}{\pgfqpoint{-0.048611in}{0.000000in}}{\pgfqpoint{-0.000000in}{0.000000in}}{%
\pgfpathmoveto{\pgfqpoint{-0.000000in}{0.000000in}}%
\pgfpathlineto{\pgfqpoint{-0.048611in}{0.000000in}}%
\pgfusepath{stroke,fill}%
}%
\begin{pgfscope}%
\pgfsys@transformshift{1.000000in}{0.930000in}%
\pgfsys@useobject{currentmarker}{}%
\end{pgfscope}%
\end{pgfscope}%
\begin{pgfscope}%
\definecolor{textcolor}{rgb}{0.000000,0.000000,0.000000}%
\pgfsetstrokecolor{textcolor}%
\pgfsetfillcolor{textcolor}%
\pgftext[x=0.560215in, y=0.829981in, left, base]{\color{textcolor}\sffamily\fontsize{20.000000}{24.000000}\selectfont \(\displaystyle {0.0}\)}%
\end{pgfscope}%
\begin{pgfscope}%
\pgfpathrectangle{\pgfqpoint{1.000000in}{0.720000in}}{\pgfqpoint{6.200000in}{4.620000in}}%
\pgfusepath{clip}%
\pgfsetrectcap%
\pgfsetroundjoin%
\pgfsetlinewidth{0.803000pt}%
\definecolor{currentstroke}{rgb}{0.690196,0.690196,0.690196}%
\pgfsetstrokecolor{currentstroke}%
\pgfsetdash{}{0pt}%
\pgfpathmoveto{\pgfqpoint{1.000000in}{1.770000in}}%
\pgfpathlineto{\pgfqpoint{7.200000in}{1.770000in}}%
\pgfusepath{stroke}%
\end{pgfscope}%
\begin{pgfscope}%
\pgfsetbuttcap%
\pgfsetroundjoin%
\definecolor{currentfill}{rgb}{0.000000,0.000000,0.000000}%
\pgfsetfillcolor{currentfill}%
\pgfsetlinewidth{0.803000pt}%
\definecolor{currentstroke}{rgb}{0.000000,0.000000,0.000000}%
\pgfsetstrokecolor{currentstroke}%
\pgfsetdash{}{0pt}%
\pgfsys@defobject{currentmarker}{\pgfqpoint{-0.048611in}{0.000000in}}{\pgfqpoint{-0.000000in}{0.000000in}}{%
\pgfpathmoveto{\pgfqpoint{-0.000000in}{0.000000in}}%
\pgfpathlineto{\pgfqpoint{-0.048611in}{0.000000in}}%
\pgfusepath{stroke,fill}%
}%
\begin{pgfscope}%
\pgfsys@transformshift{1.000000in}{1.770000in}%
\pgfsys@useobject{currentmarker}{}%
\end{pgfscope}%
\end{pgfscope}%
\begin{pgfscope}%
\definecolor{textcolor}{rgb}{0.000000,0.000000,0.000000}%
\pgfsetstrokecolor{textcolor}%
\pgfsetfillcolor{textcolor}%
\pgftext[x=0.560215in, y=1.669981in, left, base]{\color{textcolor}\sffamily\fontsize{20.000000}{24.000000}\selectfont \(\displaystyle {0.2}\)}%
\end{pgfscope}%
\begin{pgfscope}%
\pgfpathrectangle{\pgfqpoint{1.000000in}{0.720000in}}{\pgfqpoint{6.200000in}{4.620000in}}%
\pgfusepath{clip}%
\pgfsetrectcap%
\pgfsetroundjoin%
\pgfsetlinewidth{0.803000pt}%
\definecolor{currentstroke}{rgb}{0.690196,0.690196,0.690196}%
\pgfsetstrokecolor{currentstroke}%
\pgfsetdash{}{0pt}%
\pgfpathmoveto{\pgfqpoint{1.000000in}{2.610000in}}%
\pgfpathlineto{\pgfqpoint{7.200000in}{2.610000in}}%
\pgfusepath{stroke}%
\end{pgfscope}%
\begin{pgfscope}%
\pgfsetbuttcap%
\pgfsetroundjoin%
\definecolor{currentfill}{rgb}{0.000000,0.000000,0.000000}%
\pgfsetfillcolor{currentfill}%
\pgfsetlinewidth{0.803000pt}%
\definecolor{currentstroke}{rgb}{0.000000,0.000000,0.000000}%
\pgfsetstrokecolor{currentstroke}%
\pgfsetdash{}{0pt}%
\pgfsys@defobject{currentmarker}{\pgfqpoint{-0.048611in}{0.000000in}}{\pgfqpoint{-0.000000in}{0.000000in}}{%
\pgfpathmoveto{\pgfqpoint{-0.000000in}{0.000000in}}%
\pgfpathlineto{\pgfqpoint{-0.048611in}{0.000000in}}%
\pgfusepath{stroke,fill}%
}%
\begin{pgfscope}%
\pgfsys@transformshift{1.000000in}{2.610000in}%
\pgfsys@useobject{currentmarker}{}%
\end{pgfscope}%
\end{pgfscope}%
\begin{pgfscope}%
\definecolor{textcolor}{rgb}{0.000000,0.000000,0.000000}%
\pgfsetstrokecolor{textcolor}%
\pgfsetfillcolor{textcolor}%
\pgftext[x=0.560215in, y=2.509981in, left, base]{\color{textcolor}\sffamily\fontsize{20.000000}{24.000000}\selectfont \(\displaystyle {0.4}\)}%
\end{pgfscope}%
\begin{pgfscope}%
\pgfpathrectangle{\pgfqpoint{1.000000in}{0.720000in}}{\pgfqpoint{6.200000in}{4.620000in}}%
\pgfusepath{clip}%
\pgfsetrectcap%
\pgfsetroundjoin%
\pgfsetlinewidth{0.803000pt}%
\definecolor{currentstroke}{rgb}{0.690196,0.690196,0.690196}%
\pgfsetstrokecolor{currentstroke}%
\pgfsetdash{}{0pt}%
\pgfpathmoveto{\pgfqpoint{1.000000in}{3.450000in}}%
\pgfpathlineto{\pgfqpoint{7.200000in}{3.450000in}}%
\pgfusepath{stroke}%
\end{pgfscope}%
\begin{pgfscope}%
\pgfsetbuttcap%
\pgfsetroundjoin%
\definecolor{currentfill}{rgb}{0.000000,0.000000,0.000000}%
\pgfsetfillcolor{currentfill}%
\pgfsetlinewidth{0.803000pt}%
\definecolor{currentstroke}{rgb}{0.000000,0.000000,0.000000}%
\pgfsetstrokecolor{currentstroke}%
\pgfsetdash{}{0pt}%
\pgfsys@defobject{currentmarker}{\pgfqpoint{-0.048611in}{0.000000in}}{\pgfqpoint{-0.000000in}{0.000000in}}{%
\pgfpathmoveto{\pgfqpoint{-0.000000in}{0.000000in}}%
\pgfpathlineto{\pgfqpoint{-0.048611in}{0.000000in}}%
\pgfusepath{stroke,fill}%
}%
\begin{pgfscope}%
\pgfsys@transformshift{1.000000in}{3.450000in}%
\pgfsys@useobject{currentmarker}{}%
\end{pgfscope}%
\end{pgfscope}%
\begin{pgfscope}%
\definecolor{textcolor}{rgb}{0.000000,0.000000,0.000000}%
\pgfsetstrokecolor{textcolor}%
\pgfsetfillcolor{textcolor}%
\pgftext[x=0.560215in, y=3.349981in, left, base]{\color{textcolor}\sffamily\fontsize{20.000000}{24.000000}\selectfont \(\displaystyle {0.6}\)}%
\end{pgfscope}%
\begin{pgfscope}%
\pgfpathrectangle{\pgfqpoint{1.000000in}{0.720000in}}{\pgfqpoint{6.200000in}{4.620000in}}%
\pgfusepath{clip}%
\pgfsetrectcap%
\pgfsetroundjoin%
\pgfsetlinewidth{0.803000pt}%
\definecolor{currentstroke}{rgb}{0.690196,0.690196,0.690196}%
\pgfsetstrokecolor{currentstroke}%
\pgfsetdash{}{0pt}%
\pgfpathmoveto{\pgfqpoint{1.000000in}{4.290000in}}%
\pgfpathlineto{\pgfqpoint{7.200000in}{4.290000in}}%
\pgfusepath{stroke}%
\end{pgfscope}%
\begin{pgfscope}%
\pgfsetbuttcap%
\pgfsetroundjoin%
\definecolor{currentfill}{rgb}{0.000000,0.000000,0.000000}%
\pgfsetfillcolor{currentfill}%
\pgfsetlinewidth{0.803000pt}%
\definecolor{currentstroke}{rgb}{0.000000,0.000000,0.000000}%
\pgfsetstrokecolor{currentstroke}%
\pgfsetdash{}{0pt}%
\pgfsys@defobject{currentmarker}{\pgfqpoint{-0.048611in}{0.000000in}}{\pgfqpoint{-0.000000in}{0.000000in}}{%
\pgfpathmoveto{\pgfqpoint{-0.000000in}{0.000000in}}%
\pgfpathlineto{\pgfqpoint{-0.048611in}{0.000000in}}%
\pgfusepath{stroke,fill}%
}%
\begin{pgfscope}%
\pgfsys@transformshift{1.000000in}{4.290000in}%
\pgfsys@useobject{currentmarker}{}%
\end{pgfscope}%
\end{pgfscope}%
\begin{pgfscope}%
\definecolor{textcolor}{rgb}{0.000000,0.000000,0.000000}%
\pgfsetstrokecolor{textcolor}%
\pgfsetfillcolor{textcolor}%
\pgftext[x=0.560215in, y=4.189981in, left, base]{\color{textcolor}\sffamily\fontsize{20.000000}{24.000000}\selectfont \(\displaystyle {0.8}\)}%
\end{pgfscope}%
\begin{pgfscope}%
\pgfpathrectangle{\pgfqpoint{1.000000in}{0.720000in}}{\pgfqpoint{6.200000in}{4.620000in}}%
\pgfusepath{clip}%
\pgfsetrectcap%
\pgfsetroundjoin%
\pgfsetlinewidth{0.803000pt}%
\definecolor{currentstroke}{rgb}{0.690196,0.690196,0.690196}%
\pgfsetstrokecolor{currentstroke}%
\pgfsetdash{}{0pt}%
\pgfpathmoveto{\pgfqpoint{1.000000in}{5.130000in}}%
\pgfpathlineto{\pgfqpoint{7.200000in}{5.130000in}}%
\pgfusepath{stroke}%
\end{pgfscope}%
\begin{pgfscope}%
\pgfsetbuttcap%
\pgfsetroundjoin%
\definecolor{currentfill}{rgb}{0.000000,0.000000,0.000000}%
\pgfsetfillcolor{currentfill}%
\pgfsetlinewidth{0.803000pt}%
\definecolor{currentstroke}{rgb}{0.000000,0.000000,0.000000}%
\pgfsetstrokecolor{currentstroke}%
\pgfsetdash{}{0pt}%
\pgfsys@defobject{currentmarker}{\pgfqpoint{-0.048611in}{0.000000in}}{\pgfqpoint{-0.000000in}{0.000000in}}{%
\pgfpathmoveto{\pgfqpoint{-0.000000in}{0.000000in}}%
\pgfpathlineto{\pgfqpoint{-0.048611in}{0.000000in}}%
\pgfusepath{stroke,fill}%
}%
\begin{pgfscope}%
\pgfsys@transformshift{1.000000in}{5.130000in}%
\pgfsys@useobject{currentmarker}{}%
\end{pgfscope}%
\end{pgfscope}%
\begin{pgfscope}%
\definecolor{textcolor}{rgb}{0.000000,0.000000,0.000000}%
\pgfsetstrokecolor{textcolor}%
\pgfsetfillcolor{textcolor}%
\pgftext[x=0.560215in, y=5.029981in, left, base]{\color{textcolor}\sffamily\fontsize{20.000000}{24.000000}\selectfont \(\displaystyle {1.0}\)}%
\end{pgfscope}%
\begin{pgfscope}%
\definecolor{textcolor}{rgb}{0.000000,0.000000,0.000000}%
\pgfsetstrokecolor{textcolor}%
\pgfsetfillcolor{textcolor}%
\pgftext[x=0.504660in,y=3.030000in,,bottom,rotate=90.000000]{\color{textcolor}\sffamily\fontsize{20.000000}{24.000000}\selectfont \(\displaystyle \mathrm{Charge}/\si{mV\cdot ns}\)}%
\end{pgfscope}%
\begin{pgfscope}%
\pgfpathrectangle{\pgfqpoint{1.000000in}{0.720000in}}{\pgfqpoint{6.200000in}{4.620000in}}%
\pgfusepath{clip}%
\pgfsetbuttcap%
\pgfsetroundjoin%
\pgfsetlinewidth{2.007500pt}%
\definecolor{currentstroke}{rgb}{1.000000,0.000000,0.000000}%
\pgfsetstrokecolor{currentstroke}%
\pgfsetdash{}{0pt}%
\pgfpathmoveto{\pgfqpoint{2.187304in}{0.930000in}}%
\pgfpathlineto{\pgfqpoint{2.187304in}{5.130000in}}%
\pgfusepath{stroke}%
\end{pgfscope}%
\begin{pgfscope}%
\pgfpathrectangle{\pgfqpoint{1.000000in}{0.720000in}}{\pgfqpoint{6.200000in}{4.620000in}}%
\pgfusepath{clip}%
\pgfsetbuttcap%
\pgfsetroundjoin%
\pgfsetlinewidth{2.007500pt}%
\definecolor{currentstroke}{rgb}{1.000000,0.000000,0.000000}%
\pgfsetstrokecolor{currentstroke}%
\pgfsetdash{}{0pt}%
\pgfpathmoveto{\pgfqpoint{2.630655in}{0.930000in}}%
\pgfpathlineto{\pgfqpoint{2.630655in}{5.130000in}}%
\pgfusepath{stroke}%
\end{pgfscope}%
\begin{pgfscope}%
\pgfpathrectangle{\pgfqpoint{1.000000in}{0.720000in}}{\pgfqpoint{6.200000in}{4.620000in}}%
\pgfusepath{clip}%
\pgfsetbuttcap%
\pgfsetroundjoin%
\pgfsetlinewidth{2.007500pt}%
\definecolor{currentstroke}{rgb}{1.000000,0.000000,0.000000}%
\pgfsetstrokecolor{currentstroke}%
\pgfsetdash{}{0pt}%
\pgfpathmoveto{\pgfqpoint{2.927239in}{0.930000in}}%
\pgfpathlineto{\pgfqpoint{2.927239in}{5.130000in}}%
\pgfusepath{stroke}%
\end{pgfscope}%
\begin{pgfscope}%
\pgfpathrectangle{\pgfqpoint{1.000000in}{0.720000in}}{\pgfqpoint{6.200000in}{4.620000in}}%
\pgfusepath{clip}%
\pgfsetbuttcap%
\pgfsetroundjoin%
\pgfsetlinewidth{2.007500pt}%
\definecolor{currentstroke}{rgb}{1.000000,0.000000,0.000000}%
\pgfsetstrokecolor{currentstroke}%
\pgfsetdash{}{0pt}%
\pgfpathmoveto{\pgfqpoint{3.106029in}{0.930000in}}%
\pgfpathlineto{\pgfqpoint{3.106029in}{5.130000in}}%
\pgfusepath{stroke}%
\end{pgfscope}%
\begin{pgfscope}%
\pgfpathrectangle{\pgfqpoint{1.000000in}{0.720000in}}{\pgfqpoint{6.200000in}{4.620000in}}%
\pgfusepath{clip}%
\pgfsetbuttcap%
\pgfsetroundjoin%
\pgfsetlinewidth{2.007500pt}%
\definecolor{currentstroke}{rgb}{1.000000,0.000000,0.000000}%
\pgfsetstrokecolor{currentstroke}%
\pgfsetdash{}{0pt}%
\pgfpathmoveto{\pgfqpoint{3.690495in}{0.930000in}}%
\pgfpathlineto{\pgfqpoint{3.690495in}{5.130000in}}%
\pgfusepath{stroke}%
\end{pgfscope}%
\begin{pgfscope}%
\pgfsetrectcap%
\pgfsetmiterjoin%
\pgfsetlinewidth{0.803000pt}%
\definecolor{currentstroke}{rgb}{0.000000,0.000000,0.000000}%
\pgfsetstrokecolor{currentstroke}%
\pgfsetdash{}{0pt}%
\pgfpathmoveto{\pgfqpoint{1.000000in}{0.720000in}}%
\pgfpathlineto{\pgfqpoint{1.000000in}{5.340000in}}%
\pgfusepath{stroke}%
\end{pgfscope}%
\begin{pgfscope}%
\pgfsetrectcap%
\pgfsetmiterjoin%
\pgfsetlinewidth{0.803000pt}%
\definecolor{currentstroke}{rgb}{0.000000,0.000000,0.000000}%
\pgfsetstrokecolor{currentstroke}%
\pgfsetdash{}{0pt}%
\pgfpathmoveto{\pgfqpoint{7.200000in}{0.720000in}}%
\pgfpathlineto{\pgfqpoint{7.200000in}{5.340000in}}%
\pgfusepath{stroke}%
\end{pgfscope}%
\begin{pgfscope}%
\pgfsetrectcap%
\pgfsetmiterjoin%
\pgfsetlinewidth{0.803000pt}%
\definecolor{currentstroke}{rgb}{0.000000,0.000000,0.000000}%
\pgfsetstrokecolor{currentstroke}%
\pgfsetdash{}{0pt}%
\pgfpathmoveto{\pgfqpoint{1.000000in}{0.720000in}}%
\pgfpathlineto{\pgfqpoint{7.200000in}{0.720000in}}%
\pgfusepath{stroke}%
\end{pgfscope}%
\begin{pgfscope}%
\pgfsetrectcap%
\pgfsetmiterjoin%
\pgfsetlinewidth{0.803000pt}%
\definecolor{currentstroke}{rgb}{0.000000,0.000000,0.000000}%
\pgfsetstrokecolor{currentstroke}%
\pgfsetdash{}{0pt}%
\pgfpathmoveto{\pgfqpoint{1.000000in}{5.340000in}}%
\pgfpathlineto{\pgfqpoint{7.200000in}{5.340000in}}%
\pgfusepath{stroke}%
\end{pgfscope}%
\begin{pgfscope}%
\pgfsetroundcap%
\pgfsetroundjoin%
\definecolor{currentfill}{rgb}{0.000000,0.000000,0.000000}%
\pgfsetfillcolor{currentfill}%
\pgfsetlinewidth{1.003750pt}%
\definecolor{currentstroke}{rgb}{0.000000,0.000000,0.000000}%
\pgfsetstrokecolor{currentstroke}%
\pgfsetdash{}{0pt}%
\pgfpathmoveto{\pgfqpoint{5.029972in}{3.057778in}}%
\pgfpathquadraticcurveto{\pgfqpoint{6.186652in}{3.057778in}}{\pgfqpoint{7.343333in}{3.057778in}}%
\pgfpathlineto{\pgfqpoint{7.343333in}{3.141111in}}%
\pgfpathquadraticcurveto{\pgfqpoint{7.426681in}{3.085556in}}{\pgfqpoint{7.510030in}{3.030000in}}%
\pgfpathquadraticcurveto{\pgfqpoint{7.426681in}{2.974444in}}{\pgfqpoint{7.343333in}{2.918889in}}%
\pgfpathlineto{\pgfqpoint{7.343333in}{3.002222in}}%
\pgfpathquadraticcurveto{\pgfqpoint{6.186652in}{3.002222in}}{\pgfqpoint{5.029972in}{3.002222in}}%
\pgfpathlineto{\pgfqpoint{5.029972in}{3.057778in}}%
\pgfpathclose%
\pgfusepath{stroke,fill}%
\end{pgfscope}%
\begin{pgfscope}%
\pgfsetbuttcap%
\pgfsetmiterjoin%
\definecolor{currentfill}{rgb}{1.000000,1.000000,1.000000}%
\pgfsetfillcolor{currentfill}%
\pgfsetfillopacity{0.800000}%
\pgfsetlinewidth{1.003750pt}%
\definecolor{currentstroke}{rgb}{0.800000,0.800000,0.800000}%
\pgfsetstrokecolor{currentstroke}%
\pgfsetstrokeopacity{0.800000}%
\pgfsetdash{}{0pt}%
\pgfpathmoveto{\pgfqpoint{5.972807in}{4.722821in}}%
\pgfpathlineto{\pgfqpoint{7.005556in}{4.722821in}}%
\pgfpathquadraticcurveto{\pgfqpoint{7.061111in}{4.722821in}}{\pgfqpoint{7.061111in}{4.778377in}}%
\pgfpathlineto{\pgfqpoint{7.061111in}{5.145556in}}%
\pgfpathquadraticcurveto{\pgfqpoint{7.061111in}{5.201111in}}{\pgfqpoint{7.005556in}{5.201111in}}%
\pgfpathlineto{\pgfqpoint{5.972807in}{5.201111in}}%
\pgfpathquadraticcurveto{\pgfqpoint{5.917251in}{5.201111in}}{\pgfqpoint{5.917251in}{5.145556in}}%
\pgfpathlineto{\pgfqpoint{5.917251in}{4.778377in}}%
\pgfpathquadraticcurveto{\pgfqpoint{5.917251in}{4.722821in}}{\pgfqpoint{5.972807in}{4.722821in}}%
\pgfpathclose%
\pgfusepath{stroke,fill}%
\end{pgfscope}%
\begin{pgfscope}%
\pgfsetbuttcap%
\pgfsetroundjoin%
\pgfsetlinewidth{2.007500pt}%
\definecolor{currentstroke}{rgb}{1.000000,0.000000,0.000000}%
\pgfsetstrokecolor{currentstroke}%
\pgfsetdash{}{0pt}%
\pgfpathmoveto{\pgfqpoint{6.028362in}{4.987184in}}%
\pgfpathlineto{\pgfqpoint{6.583918in}{4.987184in}}%
\pgfusepath{stroke}%
\end{pgfscope}%
\begin{pgfscope}%
\definecolor{textcolor}{rgb}{0.000000,0.000000,0.000000}%
\pgfsetstrokecolor{textcolor}%
\pgfsetfillcolor{textcolor}%
\pgftext[x=6.806140in,y=4.889962in,left,base]{\color{textcolor}\sffamily\fontsize{20.000000}{24.000000}\selectfont \(\displaystyle \vec{z}\)}%
\end{pgfscope}%
\end{pgfpicture}%
\makeatother%
\endgroup%
}
\end{figure}
\begin{figure}
    \centering
    \resizebox{0.8\textwidth}{!}{%% Creator: Matplotlib, PGF backend
%%
%% To include the figure in your LaTeX document, write
%%   \input{<filename>.pgf}
%%
%% Make sure the required packages are loaded in your preamble
%%   \usepackage{pgf}
%%
%% and, on pdftex
%%   \usepackage[utf8]{inputenc}\DeclareUnicodeCharacter{2212}{-}
%%
%% or, on luatex and xetex
%%   \usepackage{unicode-math}
%%
%% Figures using additional raster images can only be included by \input if
%% they are in the same directory as the main LaTeX file. For loading figures
%% from other directories you can use the `import` package
%%   \usepackage{import}
%%
%% and then include the figures with
%%   \import{<path to file>}{<filename>.pgf}
%%
%% Matplotlib used the following preamble
%%   \usepackage[detect-all,locale=DE]{siunitx}
%%
\begingroup%
\makeatletter%
\begin{pgfpicture}%
\pgfpathrectangle{\pgfpointorigin}{\pgfqpoint{8.000000in}{6.000000in}}%
\pgfusepath{use as bounding box, clip}%
\begin{pgfscope}%
\pgfsetbuttcap%
\pgfsetmiterjoin%
\definecolor{currentfill}{rgb}{1.000000,1.000000,1.000000}%
\pgfsetfillcolor{currentfill}%
\pgfsetlinewidth{0.000000pt}%
\definecolor{currentstroke}{rgb}{1.000000,1.000000,1.000000}%
\pgfsetstrokecolor{currentstroke}%
\pgfsetdash{}{0pt}%
\pgfpathmoveto{\pgfqpoint{0.000000in}{0.000000in}}%
\pgfpathlineto{\pgfqpoint{8.000000in}{0.000000in}}%
\pgfpathlineto{\pgfqpoint{8.000000in}{6.000000in}}%
\pgfpathlineto{\pgfqpoint{0.000000in}{6.000000in}}%
\pgfpathclose%
\pgfusepath{fill}%
\end{pgfscope}%
\begin{pgfscope}%
\pgfsetbuttcap%
\pgfsetmiterjoin%
\definecolor{currentfill}{rgb}{1.000000,1.000000,1.000000}%
\pgfsetfillcolor{currentfill}%
\pgfsetlinewidth{0.000000pt}%
\definecolor{currentstroke}{rgb}{0.000000,0.000000,0.000000}%
\pgfsetstrokecolor{currentstroke}%
\pgfsetstrokeopacity{0.000000}%
\pgfsetdash{}{0pt}%
\pgfpathmoveto{\pgfqpoint{1.000000in}{0.720000in}}%
\pgfpathlineto{\pgfqpoint{7.200000in}{0.720000in}}%
\pgfpathlineto{\pgfqpoint{7.200000in}{5.340000in}}%
\pgfpathlineto{\pgfqpoint{1.000000in}{5.340000in}}%
\pgfpathclose%
\pgfusepath{fill}%
\end{pgfscope}%
\begin{pgfscope}%
\pgfpathrectangle{\pgfqpoint{1.000000in}{0.720000in}}{\pgfqpoint{6.200000in}{4.620000in}}%
\pgfusepath{clip}%
\pgfsetrectcap%
\pgfsetroundjoin%
\pgfsetlinewidth{0.803000pt}%
\definecolor{currentstroke}{rgb}{0.690196,0.690196,0.690196}%
\pgfsetstrokecolor{currentstroke}%
\pgfsetdash{}{0pt}%
\pgfpathmoveto{\pgfqpoint{2.240000in}{0.720000in}}%
\pgfpathlineto{\pgfqpoint{2.240000in}{5.340000in}}%
\pgfusepath{stroke}%
\end{pgfscope}%
\begin{pgfscope}%
\pgfsetbuttcap%
\pgfsetroundjoin%
\definecolor{currentfill}{rgb}{0.000000,0.000000,0.000000}%
\pgfsetfillcolor{currentfill}%
\pgfsetlinewidth{0.803000pt}%
\definecolor{currentstroke}{rgb}{0.000000,0.000000,0.000000}%
\pgfsetstrokecolor{currentstroke}%
\pgfsetdash{}{0pt}%
\pgfsys@defobject{currentmarker}{\pgfqpoint{0.000000in}{-0.048611in}}{\pgfqpoint{0.000000in}{0.000000in}}{%
\pgfpathmoveto{\pgfqpoint{0.000000in}{0.000000in}}%
\pgfpathlineto{\pgfqpoint{0.000000in}{-0.048611in}}%
\pgfusepath{stroke,fill}%
}%
\begin{pgfscope}%
\pgfsys@transformshift{2.240000in}{0.720000in}%
\pgfsys@useobject{currentmarker}{}%
\end{pgfscope}%
\end{pgfscope}%
\begin{pgfscope}%
\definecolor{textcolor}{rgb}{0.000000,0.000000,0.000000}%
\pgfsetstrokecolor{textcolor}%
\pgfsetfillcolor{textcolor}%
\pgftext[x=2.240000in,y=0.622778in,,top]{\color{textcolor}\sffamily\fontsize{20.000000}{24.000000}\selectfont \(\displaystyle {200}\)}%
\end{pgfscope}%
\begin{pgfscope}%
\pgfpathrectangle{\pgfqpoint{1.000000in}{0.720000in}}{\pgfqpoint{6.200000in}{4.620000in}}%
\pgfusepath{clip}%
\pgfsetrectcap%
\pgfsetroundjoin%
\pgfsetlinewidth{0.803000pt}%
\definecolor{currentstroke}{rgb}{0.690196,0.690196,0.690196}%
\pgfsetstrokecolor{currentstroke}%
\pgfsetdash{}{0pt}%
\pgfpathmoveto{\pgfqpoint{3.790000in}{0.720000in}}%
\pgfpathlineto{\pgfqpoint{3.790000in}{5.340000in}}%
\pgfusepath{stroke}%
\end{pgfscope}%
\begin{pgfscope}%
\pgfsetbuttcap%
\pgfsetroundjoin%
\definecolor{currentfill}{rgb}{0.000000,0.000000,0.000000}%
\pgfsetfillcolor{currentfill}%
\pgfsetlinewidth{0.803000pt}%
\definecolor{currentstroke}{rgb}{0.000000,0.000000,0.000000}%
\pgfsetstrokecolor{currentstroke}%
\pgfsetdash{}{0pt}%
\pgfsys@defobject{currentmarker}{\pgfqpoint{0.000000in}{-0.048611in}}{\pgfqpoint{0.000000in}{0.000000in}}{%
\pgfpathmoveto{\pgfqpoint{0.000000in}{0.000000in}}%
\pgfpathlineto{\pgfqpoint{0.000000in}{-0.048611in}}%
\pgfusepath{stroke,fill}%
}%
\begin{pgfscope}%
\pgfsys@transformshift{3.790000in}{0.720000in}%
\pgfsys@useobject{currentmarker}{}%
\end{pgfscope}%
\end{pgfscope}%
\begin{pgfscope}%
\definecolor{textcolor}{rgb}{0.000000,0.000000,0.000000}%
\pgfsetstrokecolor{textcolor}%
\pgfsetfillcolor{textcolor}%
\pgftext[x=3.790000in,y=0.622778in,,top]{\color{textcolor}\sffamily\fontsize{20.000000}{24.000000}\selectfont \(\displaystyle {250}\)}%
\end{pgfscope}%
\begin{pgfscope}%
\pgfpathrectangle{\pgfqpoint{1.000000in}{0.720000in}}{\pgfqpoint{6.200000in}{4.620000in}}%
\pgfusepath{clip}%
\pgfsetrectcap%
\pgfsetroundjoin%
\pgfsetlinewidth{0.803000pt}%
\definecolor{currentstroke}{rgb}{0.690196,0.690196,0.690196}%
\pgfsetstrokecolor{currentstroke}%
\pgfsetdash{}{0pt}%
\pgfpathmoveto{\pgfqpoint{5.340000in}{0.720000in}}%
\pgfpathlineto{\pgfqpoint{5.340000in}{5.340000in}}%
\pgfusepath{stroke}%
\end{pgfscope}%
\begin{pgfscope}%
\pgfsetbuttcap%
\pgfsetroundjoin%
\definecolor{currentfill}{rgb}{0.000000,0.000000,0.000000}%
\pgfsetfillcolor{currentfill}%
\pgfsetlinewidth{0.803000pt}%
\definecolor{currentstroke}{rgb}{0.000000,0.000000,0.000000}%
\pgfsetstrokecolor{currentstroke}%
\pgfsetdash{}{0pt}%
\pgfsys@defobject{currentmarker}{\pgfqpoint{0.000000in}{-0.048611in}}{\pgfqpoint{0.000000in}{0.000000in}}{%
\pgfpathmoveto{\pgfqpoint{0.000000in}{0.000000in}}%
\pgfpathlineto{\pgfqpoint{0.000000in}{-0.048611in}}%
\pgfusepath{stroke,fill}%
}%
\begin{pgfscope}%
\pgfsys@transformshift{5.340000in}{0.720000in}%
\pgfsys@useobject{currentmarker}{}%
\end{pgfscope}%
\end{pgfscope}%
\begin{pgfscope}%
\definecolor{textcolor}{rgb}{0.000000,0.000000,0.000000}%
\pgfsetstrokecolor{textcolor}%
\pgfsetfillcolor{textcolor}%
\pgftext[x=5.340000in,y=0.622778in,,top]{\color{textcolor}\sffamily\fontsize{20.000000}{24.000000}\selectfont \(\displaystyle {300}\)}%
\end{pgfscope}%
\begin{pgfscope}%
\pgfpathrectangle{\pgfqpoint{1.000000in}{0.720000in}}{\pgfqpoint{6.200000in}{4.620000in}}%
\pgfusepath{clip}%
\pgfsetrectcap%
\pgfsetroundjoin%
\pgfsetlinewidth{0.803000pt}%
\definecolor{currentstroke}{rgb}{0.690196,0.690196,0.690196}%
\pgfsetstrokecolor{currentstroke}%
\pgfsetdash{}{0pt}%
\pgfpathmoveto{\pgfqpoint{6.890000in}{0.720000in}}%
\pgfpathlineto{\pgfqpoint{6.890000in}{5.340000in}}%
\pgfusepath{stroke}%
\end{pgfscope}%
\begin{pgfscope}%
\pgfsetbuttcap%
\pgfsetroundjoin%
\definecolor{currentfill}{rgb}{0.000000,0.000000,0.000000}%
\pgfsetfillcolor{currentfill}%
\pgfsetlinewidth{0.803000pt}%
\definecolor{currentstroke}{rgb}{0.000000,0.000000,0.000000}%
\pgfsetstrokecolor{currentstroke}%
\pgfsetdash{}{0pt}%
\pgfsys@defobject{currentmarker}{\pgfqpoint{0.000000in}{-0.048611in}}{\pgfqpoint{0.000000in}{0.000000in}}{%
\pgfpathmoveto{\pgfqpoint{0.000000in}{0.000000in}}%
\pgfpathlineto{\pgfqpoint{0.000000in}{-0.048611in}}%
\pgfusepath{stroke,fill}%
}%
\begin{pgfscope}%
\pgfsys@transformshift{6.890000in}{0.720000in}%
\pgfsys@useobject{currentmarker}{}%
\end{pgfscope}%
\end{pgfscope}%
\begin{pgfscope}%
\definecolor{textcolor}{rgb}{0.000000,0.000000,0.000000}%
\pgfsetstrokecolor{textcolor}%
\pgfsetfillcolor{textcolor}%
\pgftext[x=6.890000in,y=0.622778in,,top]{\color{textcolor}\sffamily\fontsize{20.000000}{24.000000}\selectfont \(\displaystyle {350}\)}%
\end{pgfscope}%
\begin{pgfscope}%
\definecolor{textcolor}{rgb}{0.000000,0.000000,0.000000}%
\pgfsetstrokecolor{textcolor}%
\pgfsetfillcolor{textcolor}%
\pgftext[x=4.100000in,y=0.311155in,,top]{\color{textcolor}\sffamily\fontsize{20.000000}{24.000000}\selectfont \(\displaystyle \mathrm{t}/\si{ns}\)}%
\end{pgfscope}%
\begin{pgfscope}%
\pgfpathrectangle{\pgfqpoint{1.000000in}{0.720000in}}{\pgfqpoint{6.200000in}{4.620000in}}%
\pgfusepath{clip}%
\pgfsetrectcap%
\pgfsetroundjoin%
\pgfsetlinewidth{0.803000pt}%
\definecolor{currentstroke}{rgb}{0.690196,0.690196,0.690196}%
\pgfsetstrokecolor{currentstroke}%
\pgfsetdash{}{0pt}%
\pgfpathmoveto{\pgfqpoint{1.000000in}{1.503256in}}%
\pgfpathlineto{\pgfqpoint{7.200000in}{1.503256in}}%
\pgfusepath{stroke}%
\end{pgfscope}%
\begin{pgfscope}%
\pgfsetbuttcap%
\pgfsetroundjoin%
\definecolor{currentfill}{rgb}{0.000000,0.000000,0.000000}%
\pgfsetfillcolor{currentfill}%
\pgfsetlinewidth{0.803000pt}%
\definecolor{currentstroke}{rgb}{0.000000,0.000000,0.000000}%
\pgfsetstrokecolor{currentstroke}%
\pgfsetdash{}{0pt}%
\pgfsys@defobject{currentmarker}{\pgfqpoint{-0.048611in}{0.000000in}}{\pgfqpoint{-0.000000in}{0.000000in}}{%
\pgfpathmoveto{\pgfqpoint{-0.000000in}{0.000000in}}%
\pgfpathlineto{\pgfqpoint{-0.048611in}{0.000000in}}%
\pgfusepath{stroke,fill}%
}%
\begin{pgfscope}%
\pgfsys@transformshift{1.000000in}{1.503256in}%
\pgfsys@useobject{currentmarker}{}%
\end{pgfscope}%
\end{pgfscope}%
\begin{pgfscope}%
\definecolor{textcolor}{rgb}{0.000000,0.000000,0.000000}%
\pgfsetstrokecolor{textcolor}%
\pgfsetfillcolor{textcolor}%
\pgftext[x=0.560215in, y=1.403237in, left, base]{\color{textcolor}\sffamily\fontsize{20.000000}{24.000000}\selectfont \(\displaystyle {0.0}\)}%
\end{pgfscope}%
\begin{pgfscope}%
\pgfpathrectangle{\pgfqpoint{1.000000in}{0.720000in}}{\pgfqpoint{6.200000in}{4.620000in}}%
\pgfusepath{clip}%
\pgfsetrectcap%
\pgfsetroundjoin%
\pgfsetlinewidth{0.803000pt}%
\definecolor{currentstroke}{rgb}{0.690196,0.690196,0.690196}%
\pgfsetstrokecolor{currentstroke}%
\pgfsetdash{}{0pt}%
\pgfpathmoveto{\pgfqpoint{1.000000in}{2.411926in}}%
\pgfpathlineto{\pgfqpoint{7.200000in}{2.411926in}}%
\pgfusepath{stroke}%
\end{pgfscope}%
\begin{pgfscope}%
\pgfsetbuttcap%
\pgfsetroundjoin%
\definecolor{currentfill}{rgb}{0.000000,0.000000,0.000000}%
\pgfsetfillcolor{currentfill}%
\pgfsetlinewidth{0.803000pt}%
\definecolor{currentstroke}{rgb}{0.000000,0.000000,0.000000}%
\pgfsetstrokecolor{currentstroke}%
\pgfsetdash{}{0pt}%
\pgfsys@defobject{currentmarker}{\pgfqpoint{-0.048611in}{0.000000in}}{\pgfqpoint{-0.000000in}{0.000000in}}{%
\pgfpathmoveto{\pgfqpoint{-0.000000in}{0.000000in}}%
\pgfpathlineto{\pgfqpoint{-0.048611in}{0.000000in}}%
\pgfusepath{stroke,fill}%
}%
\begin{pgfscope}%
\pgfsys@transformshift{1.000000in}{2.411926in}%
\pgfsys@useobject{currentmarker}{}%
\end{pgfscope}%
\end{pgfscope}%
\begin{pgfscope}%
\definecolor{textcolor}{rgb}{0.000000,0.000000,0.000000}%
\pgfsetstrokecolor{textcolor}%
\pgfsetfillcolor{textcolor}%
\pgftext[x=0.560215in, y=2.311907in, left, base]{\color{textcolor}\sffamily\fontsize{20.000000}{24.000000}\selectfont \(\displaystyle {0.5}\)}%
\end{pgfscope}%
\begin{pgfscope}%
\pgfpathrectangle{\pgfqpoint{1.000000in}{0.720000in}}{\pgfqpoint{6.200000in}{4.620000in}}%
\pgfusepath{clip}%
\pgfsetrectcap%
\pgfsetroundjoin%
\pgfsetlinewidth{0.803000pt}%
\definecolor{currentstroke}{rgb}{0.690196,0.690196,0.690196}%
\pgfsetstrokecolor{currentstroke}%
\pgfsetdash{}{0pt}%
\pgfpathmoveto{\pgfqpoint{1.000000in}{3.320597in}}%
\pgfpathlineto{\pgfqpoint{7.200000in}{3.320597in}}%
\pgfusepath{stroke}%
\end{pgfscope}%
\begin{pgfscope}%
\pgfsetbuttcap%
\pgfsetroundjoin%
\definecolor{currentfill}{rgb}{0.000000,0.000000,0.000000}%
\pgfsetfillcolor{currentfill}%
\pgfsetlinewidth{0.803000pt}%
\definecolor{currentstroke}{rgb}{0.000000,0.000000,0.000000}%
\pgfsetstrokecolor{currentstroke}%
\pgfsetdash{}{0pt}%
\pgfsys@defobject{currentmarker}{\pgfqpoint{-0.048611in}{0.000000in}}{\pgfqpoint{-0.000000in}{0.000000in}}{%
\pgfpathmoveto{\pgfqpoint{-0.000000in}{0.000000in}}%
\pgfpathlineto{\pgfqpoint{-0.048611in}{0.000000in}}%
\pgfusepath{stroke,fill}%
}%
\begin{pgfscope}%
\pgfsys@transformshift{1.000000in}{3.320597in}%
\pgfsys@useobject{currentmarker}{}%
\end{pgfscope}%
\end{pgfscope}%
\begin{pgfscope}%
\definecolor{textcolor}{rgb}{0.000000,0.000000,0.000000}%
\pgfsetstrokecolor{textcolor}%
\pgfsetfillcolor{textcolor}%
\pgftext[x=0.560215in, y=3.220578in, left, base]{\color{textcolor}\sffamily\fontsize{20.000000}{24.000000}\selectfont \(\displaystyle {1.0}\)}%
\end{pgfscope}%
\begin{pgfscope}%
\pgfpathrectangle{\pgfqpoint{1.000000in}{0.720000in}}{\pgfqpoint{6.200000in}{4.620000in}}%
\pgfusepath{clip}%
\pgfsetrectcap%
\pgfsetroundjoin%
\pgfsetlinewidth{0.803000pt}%
\definecolor{currentstroke}{rgb}{0.690196,0.690196,0.690196}%
\pgfsetstrokecolor{currentstroke}%
\pgfsetdash{}{0pt}%
\pgfpathmoveto{\pgfqpoint{1.000000in}{4.229267in}}%
\pgfpathlineto{\pgfqpoint{7.200000in}{4.229267in}}%
\pgfusepath{stroke}%
\end{pgfscope}%
\begin{pgfscope}%
\pgfsetbuttcap%
\pgfsetroundjoin%
\definecolor{currentfill}{rgb}{0.000000,0.000000,0.000000}%
\pgfsetfillcolor{currentfill}%
\pgfsetlinewidth{0.803000pt}%
\definecolor{currentstroke}{rgb}{0.000000,0.000000,0.000000}%
\pgfsetstrokecolor{currentstroke}%
\pgfsetdash{}{0pt}%
\pgfsys@defobject{currentmarker}{\pgfqpoint{-0.048611in}{0.000000in}}{\pgfqpoint{-0.000000in}{0.000000in}}{%
\pgfpathmoveto{\pgfqpoint{-0.000000in}{0.000000in}}%
\pgfpathlineto{\pgfqpoint{-0.048611in}{0.000000in}}%
\pgfusepath{stroke,fill}%
}%
\begin{pgfscope}%
\pgfsys@transformshift{1.000000in}{4.229267in}%
\pgfsys@useobject{currentmarker}{}%
\end{pgfscope}%
\end{pgfscope}%
\begin{pgfscope}%
\definecolor{textcolor}{rgb}{0.000000,0.000000,0.000000}%
\pgfsetstrokecolor{textcolor}%
\pgfsetfillcolor{textcolor}%
\pgftext[x=0.560215in, y=4.129248in, left, base]{\color{textcolor}\sffamily\fontsize{20.000000}{24.000000}\selectfont \(\displaystyle {1.5}\)}%
\end{pgfscope}%
\begin{pgfscope}%
\pgfpathrectangle{\pgfqpoint{1.000000in}{0.720000in}}{\pgfqpoint{6.200000in}{4.620000in}}%
\pgfusepath{clip}%
\pgfsetrectcap%
\pgfsetroundjoin%
\pgfsetlinewidth{0.803000pt}%
\definecolor{currentstroke}{rgb}{0.690196,0.690196,0.690196}%
\pgfsetstrokecolor{currentstroke}%
\pgfsetdash{}{0pt}%
\pgfpathmoveto{\pgfqpoint{1.000000in}{5.137938in}}%
\pgfpathlineto{\pgfqpoint{7.200000in}{5.137938in}}%
\pgfusepath{stroke}%
\end{pgfscope}%
\begin{pgfscope}%
\pgfsetbuttcap%
\pgfsetroundjoin%
\definecolor{currentfill}{rgb}{0.000000,0.000000,0.000000}%
\pgfsetfillcolor{currentfill}%
\pgfsetlinewidth{0.803000pt}%
\definecolor{currentstroke}{rgb}{0.000000,0.000000,0.000000}%
\pgfsetstrokecolor{currentstroke}%
\pgfsetdash{}{0pt}%
\pgfsys@defobject{currentmarker}{\pgfqpoint{-0.048611in}{0.000000in}}{\pgfqpoint{-0.000000in}{0.000000in}}{%
\pgfpathmoveto{\pgfqpoint{-0.000000in}{0.000000in}}%
\pgfpathlineto{\pgfqpoint{-0.048611in}{0.000000in}}%
\pgfusepath{stroke,fill}%
}%
\begin{pgfscope}%
\pgfsys@transformshift{1.000000in}{5.137938in}%
\pgfsys@useobject{currentmarker}{}%
\end{pgfscope}%
\end{pgfscope}%
\begin{pgfscope}%
\definecolor{textcolor}{rgb}{0.000000,0.000000,0.000000}%
\pgfsetstrokecolor{textcolor}%
\pgfsetfillcolor{textcolor}%
\pgftext[x=0.560215in, y=5.037919in, left, base]{\color{textcolor}\sffamily\fontsize{20.000000}{24.000000}\selectfont \(\displaystyle {2.0}\)}%
\end{pgfscope}%
\begin{pgfscope}%
\definecolor{textcolor}{rgb}{0.000000,0.000000,0.000000}%
\pgfsetstrokecolor{textcolor}%
\pgfsetfillcolor{textcolor}%
\pgftext[x=0.504660in,y=3.030000in,,bottom,rotate=90.000000]{\color{textcolor}\sffamily\fontsize{20.000000}{24.000000}\selectfont \(\displaystyle \mathrm{Voltage}/\si{mV}\)}%
\end{pgfscope}%
\begin{pgfscope}%
\definecolor{textcolor}{rgb}{0.000000,0.000000,0.000000}%
\pgfsetstrokecolor{textcolor}%
\pgfsetfillcolor{textcolor}%
\pgftext[x=1.000000in,y=5.381667in,left,base]{\color{textcolor}\sffamily\fontsize{20.000000}{24.000000}\selectfont \(\displaystyle \times{10^{1}}{}\)}%
\end{pgfscope}%
\begin{pgfscope}%
\pgfpathrectangle{\pgfqpoint{1.000000in}{0.720000in}}{\pgfqpoint{6.200000in}{4.620000in}}%
\pgfusepath{clip}%
\pgfsetrectcap%
\pgfsetroundjoin%
\pgfsetlinewidth{2.007500pt}%
\definecolor{currentstroke}{rgb}{0.000000,0.000000,1.000000}%
\pgfsetstrokecolor{currentstroke}%
\pgfsetdash{}{0pt}%
\pgfpathmoveto{\pgfqpoint{0.990000in}{1.570344in}}%
\pgfpathlineto{\pgfqpoint{1.000000in}{1.673406in}}%
\pgfpathlineto{\pgfqpoint{1.031000in}{1.502841in}}%
\pgfpathlineto{\pgfqpoint{1.062000in}{1.259769in}}%
\pgfpathlineto{\pgfqpoint{1.093000in}{1.525454in}}%
\pgfpathlineto{\pgfqpoint{1.124000in}{1.226696in}}%
\pgfpathlineto{\pgfqpoint{1.155000in}{1.639660in}}%
\pgfpathlineto{\pgfqpoint{1.186000in}{1.547297in}}%
\pgfpathlineto{\pgfqpoint{1.217000in}{1.417427in}}%
\pgfpathlineto{\pgfqpoint{1.248000in}{1.444620in}}%
\pgfpathlineto{\pgfqpoint{1.279000in}{1.710481in}}%
\pgfpathlineto{\pgfqpoint{1.310000in}{1.454119in}}%
\pgfpathlineto{\pgfqpoint{1.341000in}{1.620322in}}%
\pgfpathlineto{\pgfqpoint{1.372000in}{1.588854in}}%
\pgfpathlineto{\pgfqpoint{1.403000in}{1.837856in}}%
\pgfpathlineto{\pgfqpoint{1.434000in}{1.710860in}}%
\pgfpathlineto{\pgfqpoint{1.465000in}{1.394732in}}%
\pgfpathlineto{\pgfqpoint{1.496000in}{1.561997in}}%
\pgfpathlineto{\pgfqpoint{1.527000in}{1.676430in}}%
\pgfpathlineto{\pgfqpoint{1.558000in}{1.456000in}}%
\pgfpathlineto{\pgfqpoint{1.589000in}{1.264578in}}%
\pgfpathlineto{\pgfqpoint{1.620000in}{1.302433in}}%
\pgfpathlineto{\pgfqpoint{1.651000in}{1.342222in}}%
\pgfpathlineto{\pgfqpoint{1.682000in}{1.531878in}}%
\pgfpathlineto{\pgfqpoint{1.713000in}{1.641121in}}%
\pgfpathlineto{\pgfqpoint{1.744000in}{1.584408in}}%
\pgfpathlineto{\pgfqpoint{1.775000in}{1.302920in}}%
\pgfpathlineto{\pgfqpoint{1.806000in}{1.641076in}}%
\pgfpathlineto{\pgfqpoint{1.837000in}{1.753574in}}%
\pgfpathlineto{\pgfqpoint{1.868000in}{1.494202in}}%
\pgfpathlineto{\pgfqpoint{1.899000in}{1.626785in}}%
\pgfpathlineto{\pgfqpoint{1.930000in}{1.539767in}}%
\pgfpathlineto{\pgfqpoint{1.961000in}{1.542973in}}%
\pgfpathlineto{\pgfqpoint{1.992000in}{1.368336in}}%
\pgfpathlineto{\pgfqpoint{2.023000in}{1.304006in}}%
\pgfpathlineto{\pgfqpoint{2.054000in}{1.825977in}}%
\pgfpathlineto{\pgfqpoint{2.085000in}{1.318298in}}%
\pgfpathlineto{\pgfqpoint{2.116000in}{1.550922in}}%
\pgfpathlineto{\pgfqpoint{2.147000in}{1.806564in}}%
\pgfpathlineto{\pgfqpoint{2.178000in}{1.502867in}}%
\pgfpathlineto{\pgfqpoint{2.209000in}{1.465836in}}%
\pgfpathlineto{\pgfqpoint{2.240000in}{1.922219in}}%
\pgfpathlineto{\pgfqpoint{2.271000in}{1.461477in}}%
\pgfpathlineto{\pgfqpoint{2.302000in}{2.088237in}}%
\pgfpathlineto{\pgfqpoint{2.333000in}{2.447514in}}%
\pgfpathlineto{\pgfqpoint{2.364000in}{2.686252in}}%
\pgfpathlineto{\pgfqpoint{2.395000in}{2.849429in}}%
\pgfpathlineto{\pgfqpoint{2.426000in}{2.904019in}}%
\pgfpathlineto{\pgfqpoint{2.457000in}{3.069588in}}%
\pgfpathlineto{\pgfqpoint{2.488000in}{3.133107in}}%
\pgfpathlineto{\pgfqpoint{2.550000in}{2.814968in}}%
\pgfpathlineto{\pgfqpoint{2.581000in}{2.566072in}}%
\pgfpathlineto{\pgfqpoint{2.612000in}{2.307092in}}%
\pgfpathlineto{\pgfqpoint{2.643000in}{2.521060in}}%
\pgfpathlineto{\pgfqpoint{2.674000in}{1.940799in}}%
\pgfpathlineto{\pgfqpoint{2.705000in}{2.214348in}}%
\pgfpathlineto{\pgfqpoint{2.736000in}{2.735382in}}%
\pgfpathlineto{\pgfqpoint{2.767000in}{2.816430in}}%
\pgfpathlineto{\pgfqpoint{2.798000in}{3.638050in}}%
\pgfpathlineto{\pgfqpoint{2.829000in}{3.875307in}}%
\pgfpathlineto{\pgfqpoint{2.860000in}{3.908582in}}%
\pgfpathlineto{\pgfqpoint{2.891000in}{4.287746in}}%
\pgfpathlineto{\pgfqpoint{2.922000in}{3.495115in}}%
\pgfpathlineto{\pgfqpoint{2.953000in}{3.647422in}}%
\pgfpathlineto{\pgfqpoint{2.984000in}{3.649150in}}%
\pgfpathlineto{\pgfqpoint{3.015000in}{3.179694in}}%
\pgfpathlineto{\pgfqpoint{3.046000in}{3.918644in}}%
\pgfpathlineto{\pgfqpoint{3.077000in}{4.812188in}}%
\pgfpathlineto{\pgfqpoint{3.108000in}{4.738231in}}%
\pgfpathlineto{\pgfqpoint{3.139000in}{4.808500in}}%
\pgfpathlineto{\pgfqpoint{3.170000in}{4.852924in}}%
\pgfpathlineto{\pgfqpoint{3.201000in}{4.705264in}}%
\pgfpathlineto{\pgfqpoint{3.232000in}{5.000610in}}%
\pgfpathlineto{\pgfqpoint{3.263000in}{5.130000in}}%
\pgfpathlineto{\pgfqpoint{3.294000in}{4.795512in}}%
\pgfpathlineto{\pgfqpoint{3.325000in}{4.674342in}}%
\pgfpathlineto{\pgfqpoint{3.356000in}{4.547487in}}%
\pgfpathlineto{\pgfqpoint{3.387000in}{4.322645in}}%
\pgfpathlineto{\pgfqpoint{3.418000in}{3.973552in}}%
\pgfpathlineto{\pgfqpoint{3.449000in}{3.799472in}}%
\pgfpathlineto{\pgfqpoint{3.480000in}{3.223227in}}%
\pgfpathlineto{\pgfqpoint{3.511000in}{2.759303in}}%
\pgfpathlineto{\pgfqpoint{3.542000in}{2.696954in}}%
\pgfpathlineto{\pgfqpoint{3.573000in}{2.918227in}}%
\pgfpathlineto{\pgfqpoint{3.604000in}{2.536026in}}%
\pgfpathlineto{\pgfqpoint{3.635000in}{2.348185in}}%
\pgfpathlineto{\pgfqpoint{3.666000in}{1.938713in}}%
\pgfpathlineto{\pgfqpoint{3.697000in}{1.774061in}}%
\pgfpathlineto{\pgfqpoint{3.728000in}{1.829594in}}%
\pgfpathlineto{\pgfqpoint{3.759000in}{1.620023in}}%
\pgfpathlineto{\pgfqpoint{3.790000in}{2.084212in}}%
\pgfpathlineto{\pgfqpoint{3.821000in}{2.368887in}}%
\pgfpathlineto{\pgfqpoint{3.852000in}{2.748448in}}%
\pgfpathlineto{\pgfqpoint{3.883000in}{2.856389in}}%
\pgfpathlineto{\pgfqpoint{3.914000in}{2.733033in}}%
\pgfpathlineto{\pgfqpoint{3.945000in}{3.134491in}}%
\pgfpathlineto{\pgfqpoint{3.976000in}{2.985036in}}%
\pgfpathlineto{\pgfqpoint{4.007000in}{2.738973in}}%
\pgfpathlineto{\pgfqpoint{4.038000in}{2.686992in}}%
\pgfpathlineto{\pgfqpoint{4.069000in}{2.578910in}}%
\pgfpathlineto{\pgfqpoint{4.100000in}{2.379666in}}%
\pgfpathlineto{\pgfqpoint{4.131000in}{2.305564in}}%
\pgfpathlineto{\pgfqpoint{4.162000in}{1.875405in}}%
\pgfpathlineto{\pgfqpoint{4.193000in}{1.640948in}}%
\pgfpathlineto{\pgfqpoint{4.224000in}{1.951384in}}%
\pgfpathlineto{\pgfqpoint{4.255000in}{1.939523in}}%
\pgfpathlineto{\pgfqpoint{4.286000in}{1.878041in}}%
\pgfpathlineto{\pgfqpoint{4.317000in}{1.693469in}}%
\pgfpathlineto{\pgfqpoint{4.348000in}{1.683215in}}%
\pgfpathlineto{\pgfqpoint{4.379000in}{1.925499in}}%
\pgfpathlineto{\pgfqpoint{4.410000in}{1.439731in}}%
\pgfpathlineto{\pgfqpoint{4.441000in}{1.692057in}}%
\pgfpathlineto{\pgfqpoint{4.472000in}{1.509239in}}%
\pgfpathlineto{\pgfqpoint{4.503000in}{1.487133in}}%
\pgfpathlineto{\pgfqpoint{4.534000in}{1.625764in}}%
\pgfpathlineto{\pgfqpoint{4.565000in}{1.708953in}}%
\pgfpathlineto{\pgfqpoint{4.596000in}{1.436913in}}%
\pgfpathlineto{\pgfqpoint{4.627000in}{1.646897in}}%
\pgfpathlineto{\pgfqpoint{4.658000in}{1.717879in}}%
\pgfpathlineto{\pgfqpoint{4.689000in}{1.662788in}}%
\pgfpathlineto{\pgfqpoint{4.720000in}{1.469361in}}%
\pgfpathlineto{\pgfqpoint{4.751000in}{1.423603in}}%
\pgfpathlineto{\pgfqpoint{4.782000in}{1.097052in}}%
\pgfpathlineto{\pgfqpoint{4.813000in}{1.573508in}}%
\pgfpathlineto{\pgfqpoint{4.844000in}{1.628614in}}%
\pgfpathlineto{\pgfqpoint{4.875000in}{1.541044in}}%
\pgfpathlineto{\pgfqpoint{4.906000in}{1.267633in}}%
\pgfpathlineto{\pgfqpoint{4.937000in}{1.653080in}}%
\pgfpathlineto{\pgfqpoint{4.968000in}{1.218541in}}%
\pgfpathlineto{\pgfqpoint{4.999000in}{1.329686in}}%
\pgfpathlineto{\pgfqpoint{5.030000in}{1.504243in}}%
\pgfpathlineto{\pgfqpoint{5.061000in}{1.581536in}}%
\pgfpathlineto{\pgfqpoint{5.092000in}{1.387282in}}%
\pgfpathlineto{\pgfqpoint{5.123000in}{1.534790in}}%
\pgfpathlineto{\pgfqpoint{5.154000in}{1.501329in}}%
\pgfpathlineto{\pgfqpoint{5.185000in}{1.369226in}}%
\pgfpathlineto{\pgfqpoint{5.216000in}{1.541821in}}%
\pgfpathlineto{\pgfqpoint{5.247000in}{1.574156in}}%
\pgfpathlineto{\pgfqpoint{5.278000in}{1.517497in}}%
\pgfpathlineto{\pgfqpoint{5.309000in}{0.972221in}}%
\pgfpathlineto{\pgfqpoint{5.340000in}{1.677374in}}%
\pgfpathlineto{\pgfqpoint{5.371000in}{1.472806in}}%
\pgfpathlineto{\pgfqpoint{5.402000in}{1.728341in}}%
\pgfpathlineto{\pgfqpoint{5.433000in}{1.570672in}}%
\pgfpathlineto{\pgfqpoint{5.464000in}{1.634839in}}%
\pgfpathlineto{\pgfqpoint{5.495000in}{1.543999in}}%
\pgfpathlineto{\pgfqpoint{5.526000in}{1.599092in}}%
\pgfpathlineto{\pgfqpoint{5.557000in}{1.383455in}}%
\pgfpathlineto{\pgfqpoint{5.588000in}{1.464271in}}%
\pgfpathlineto{\pgfqpoint{5.619000in}{1.752635in}}%
\pgfpathlineto{\pgfqpoint{5.650000in}{1.562064in}}%
\pgfpathlineto{\pgfqpoint{5.681000in}{1.467953in}}%
\pgfpathlineto{\pgfqpoint{5.712000in}{1.305916in}}%
\pgfpathlineto{\pgfqpoint{5.743000in}{1.622736in}}%
\pgfpathlineto{\pgfqpoint{5.774000in}{1.576790in}}%
\pgfpathlineto{\pgfqpoint{5.805000in}{1.577935in}}%
\pgfpathlineto{\pgfqpoint{5.836000in}{1.361747in}}%
\pgfpathlineto{\pgfqpoint{5.867000in}{1.472036in}}%
\pgfpathlineto{\pgfqpoint{5.898000in}{1.708741in}}%
\pgfpathlineto{\pgfqpoint{5.929000in}{1.318615in}}%
\pgfpathlineto{\pgfqpoint{5.960000in}{1.903143in}}%
\pgfpathlineto{\pgfqpoint{5.991000in}{1.458218in}}%
\pgfpathlineto{\pgfqpoint{6.022000in}{1.527619in}}%
\pgfpathlineto{\pgfqpoint{6.053000in}{1.451183in}}%
\pgfpathlineto{\pgfqpoint{6.084000in}{1.461162in}}%
\pgfpathlineto{\pgfqpoint{6.115000in}{1.267073in}}%
\pgfpathlineto{\pgfqpoint{6.146000in}{1.468127in}}%
\pgfpathlineto{\pgfqpoint{6.177000in}{1.588285in}}%
\pgfpathlineto{\pgfqpoint{6.208000in}{1.675915in}}%
\pgfpathlineto{\pgfqpoint{6.239000in}{1.536786in}}%
\pgfpathlineto{\pgfqpoint{6.270000in}{1.389981in}}%
\pgfpathlineto{\pgfqpoint{6.301000in}{1.381074in}}%
\pgfpathlineto{\pgfqpoint{6.332000in}{1.566867in}}%
\pgfpathlineto{\pgfqpoint{6.363000in}{1.460900in}}%
\pgfpathlineto{\pgfqpoint{6.394000in}{1.691010in}}%
\pgfpathlineto{\pgfqpoint{6.425000in}{1.850253in}}%
\pgfpathlineto{\pgfqpoint{6.456000in}{1.490005in}}%
\pgfpathlineto{\pgfqpoint{6.487000in}{1.414211in}}%
\pgfpathlineto{\pgfqpoint{6.518000in}{1.721632in}}%
\pgfpathlineto{\pgfqpoint{6.549000in}{1.165832in}}%
\pgfpathlineto{\pgfqpoint{6.580000in}{1.584676in}}%
\pgfpathlineto{\pgfqpoint{6.611000in}{1.541531in}}%
\pgfpathlineto{\pgfqpoint{6.642000in}{1.509804in}}%
\pgfpathlineto{\pgfqpoint{6.673000in}{1.607768in}}%
\pgfpathlineto{\pgfqpoint{6.704000in}{1.505924in}}%
\pgfpathlineto{\pgfqpoint{6.735000in}{1.515507in}}%
\pgfpathlineto{\pgfqpoint{6.766000in}{1.411048in}}%
\pgfpathlineto{\pgfqpoint{6.797000in}{1.580678in}}%
\pgfpathlineto{\pgfqpoint{6.828000in}{1.518692in}}%
\pgfpathlineto{\pgfqpoint{6.859000in}{1.560185in}}%
\pgfpathlineto{\pgfqpoint{6.890000in}{1.303306in}}%
\pgfpathlineto{\pgfqpoint{6.921000in}{1.776970in}}%
\pgfpathlineto{\pgfqpoint{6.952000in}{1.768630in}}%
\pgfpathlineto{\pgfqpoint{6.983000in}{1.638127in}}%
\pgfpathlineto{\pgfqpoint{7.014000in}{1.667789in}}%
\pgfpathlineto{\pgfqpoint{7.045000in}{1.658063in}}%
\pgfpathlineto{\pgfqpoint{7.076000in}{1.172342in}}%
\pgfpathlineto{\pgfqpoint{7.107000in}{1.566401in}}%
\pgfpathlineto{\pgfqpoint{7.138000in}{1.533349in}}%
\pgfpathlineto{\pgfqpoint{7.169000in}{1.462757in}}%
\pgfpathlineto{\pgfqpoint{7.200000in}{1.560890in}}%
\pgfpathlineto{\pgfqpoint{7.210000in}{1.624420in}}%
\pgfpathlineto{\pgfqpoint{7.210000in}{1.624420in}}%
\pgfusepath{stroke}%
\end{pgfscope}%
\begin{pgfscope}%
\pgfsetrectcap%
\pgfsetmiterjoin%
\pgfsetlinewidth{0.803000pt}%
\definecolor{currentstroke}{rgb}{0.000000,0.000000,0.000000}%
\pgfsetstrokecolor{currentstroke}%
\pgfsetdash{}{0pt}%
\pgfpathmoveto{\pgfqpoint{1.000000in}{0.720000in}}%
\pgfpathlineto{\pgfqpoint{1.000000in}{5.340000in}}%
\pgfusepath{stroke}%
\end{pgfscope}%
\begin{pgfscope}%
\pgfsetrectcap%
\pgfsetmiterjoin%
\pgfsetlinewidth{0.803000pt}%
\definecolor{currentstroke}{rgb}{0.000000,0.000000,0.000000}%
\pgfsetstrokecolor{currentstroke}%
\pgfsetdash{}{0pt}%
\pgfpathmoveto{\pgfqpoint{7.200000in}{0.720000in}}%
\pgfpathlineto{\pgfqpoint{7.200000in}{5.340000in}}%
\pgfusepath{stroke}%
\end{pgfscope}%
\begin{pgfscope}%
\pgfsetrectcap%
\pgfsetmiterjoin%
\pgfsetlinewidth{0.803000pt}%
\definecolor{currentstroke}{rgb}{0.000000,0.000000,0.000000}%
\pgfsetstrokecolor{currentstroke}%
\pgfsetdash{}{0pt}%
\pgfpathmoveto{\pgfqpoint{1.000000in}{0.720000in}}%
\pgfpathlineto{\pgfqpoint{7.200000in}{0.720000in}}%
\pgfusepath{stroke}%
\end{pgfscope}%
\begin{pgfscope}%
\pgfsetrectcap%
\pgfsetmiterjoin%
\pgfsetlinewidth{0.803000pt}%
\definecolor{currentstroke}{rgb}{0.000000,0.000000,0.000000}%
\pgfsetstrokecolor{currentstroke}%
\pgfsetdash{}{0pt}%
\pgfpathmoveto{\pgfqpoint{1.000000in}{5.340000in}}%
\pgfpathlineto{\pgfqpoint{7.200000in}{5.340000in}}%
\pgfusepath{stroke}%
\end{pgfscope}%
\begin{pgfscope}%
\pgfsetbuttcap%
\pgfsetmiterjoin%
\definecolor{currentfill}{rgb}{1.000000,1.000000,1.000000}%
\pgfsetfillcolor{currentfill}%
\pgfsetfillopacity{0.800000}%
\pgfsetlinewidth{1.003750pt}%
\definecolor{currentstroke}{rgb}{0.800000,0.800000,0.800000}%
\pgfsetstrokecolor{currentstroke}%
\pgfsetstrokeopacity{0.800000}%
\pgfsetdash{}{0pt}%
\pgfpathmoveto{\pgfqpoint{5.907295in}{4.722821in}}%
\pgfpathlineto{\pgfqpoint{7.005556in}{4.722821in}}%
\pgfpathquadraticcurveto{\pgfqpoint{7.061111in}{4.722821in}}{\pgfqpoint{7.061111in}{4.778377in}}%
\pgfpathlineto{\pgfqpoint{7.061111in}{5.145556in}}%
\pgfpathquadraticcurveto{\pgfqpoint{7.061111in}{5.201111in}}{\pgfqpoint{7.005556in}{5.201111in}}%
\pgfpathlineto{\pgfqpoint{5.907295in}{5.201111in}}%
\pgfpathquadraticcurveto{\pgfqpoint{5.851740in}{5.201111in}}{\pgfqpoint{5.851740in}{5.145556in}}%
\pgfpathlineto{\pgfqpoint{5.851740in}{4.778377in}}%
\pgfpathquadraticcurveto{\pgfqpoint{5.851740in}{4.722821in}}{\pgfqpoint{5.907295in}{4.722821in}}%
\pgfpathclose%
\pgfusepath{stroke,fill}%
\end{pgfscope}%
\begin{pgfscope}%
\pgfsetrectcap%
\pgfsetroundjoin%
\pgfsetlinewidth{2.007500pt}%
\definecolor{currentstroke}{rgb}{0.000000,0.000000,1.000000}%
\pgfsetstrokecolor{currentstroke}%
\pgfsetdash{}{0pt}%
\pgfpathmoveto{\pgfqpoint{5.962851in}{4.987184in}}%
\pgfpathlineto{\pgfqpoint{6.518406in}{4.987184in}}%
\pgfusepath{stroke}%
\end{pgfscope}%
\begin{pgfscope}%
\definecolor{textcolor}{rgb}{0.000000,0.000000,0.000000}%
\pgfsetstrokecolor{textcolor}%
\pgfsetfillcolor{textcolor}%
\pgftext[x=6.740628in,y=4.889962in,left,base]{\color{textcolor}\sffamily\fontsize{20.000000}{24.000000}\selectfont \(\displaystyle \vec{w}\)}%
\end{pgfscope}%
\end{pgfpicture}%
\makeatother%
\endgroup%
}
\end{figure}
\column{0.5\textwidth}
\begin{figure}
    \centering
    \resizebox{0.8\textwidth}{!}{%% Creator: Matplotlib, PGF backend
%%
%% To include the figure in your LaTeX document, write
%%   \input{<filename>.pgf}
%%
%% Make sure the required packages are loaded in your preamble
%%   \usepackage{pgf}
%%
%% and, on pdftex
%%   \usepackage[utf8]{inputenc}\DeclareUnicodeCharacter{2212}{-}
%%
%% or, on luatex and xetex
%%   \usepackage{unicode-math}
%%
%% Figures using additional raster images can only be included by \input if
%% they are in the same directory as the main LaTeX file. For loading figures
%% from other directories you can use the `import` package
%%   \usepackage{import}
%%
%% and then include the figures with
%%   \import{<path to file>}{<filename>.pgf}
%%
%% Matplotlib used the following preamble
%%   \usepackage[detect-all,locale=DE]{siunitx}
%%
\begingroup%
\makeatletter%
\begin{pgfpicture}%
\pgfpathrectangle{\pgfpointorigin}{\pgfqpoint{8.000000in}{6.000000in}}%
\pgfusepath{use as bounding box, clip}%
\begin{pgfscope}%
\pgfsetbuttcap%
\pgfsetmiterjoin%
\definecolor{currentfill}{rgb}{1.000000,1.000000,1.000000}%
\pgfsetfillcolor{currentfill}%
\pgfsetlinewidth{0.000000pt}%
\definecolor{currentstroke}{rgb}{1.000000,1.000000,1.000000}%
\pgfsetstrokecolor{currentstroke}%
\pgfsetdash{}{0pt}%
\pgfpathmoveto{\pgfqpoint{0.000000in}{0.000000in}}%
\pgfpathlineto{\pgfqpoint{8.000000in}{0.000000in}}%
\pgfpathlineto{\pgfqpoint{8.000000in}{6.000000in}}%
\pgfpathlineto{\pgfqpoint{0.000000in}{6.000000in}}%
\pgfpathclose%
\pgfusepath{fill}%
\end{pgfscope}%
\begin{pgfscope}%
\pgfsetbuttcap%
\pgfsetmiterjoin%
\definecolor{currentfill}{rgb}{1.000000,1.000000,1.000000}%
\pgfsetfillcolor{currentfill}%
\pgfsetlinewidth{0.000000pt}%
\definecolor{currentstroke}{rgb}{0.000000,0.000000,0.000000}%
\pgfsetstrokecolor{currentstroke}%
\pgfsetstrokeopacity{0.000000}%
\pgfsetdash{}{0pt}%
\pgfpathmoveto{\pgfqpoint{1.000000in}{0.720000in}}%
\pgfpathlineto{\pgfqpoint{7.200000in}{0.720000in}}%
\pgfpathlineto{\pgfqpoint{7.200000in}{5.340000in}}%
\pgfpathlineto{\pgfqpoint{1.000000in}{5.340000in}}%
\pgfpathclose%
\pgfusepath{fill}%
\end{pgfscope}%
\begin{pgfscope}%
\pgfpathrectangle{\pgfqpoint{1.000000in}{0.720000in}}{\pgfqpoint{6.200000in}{4.620000in}}%
\pgfusepath{clip}%
\pgfsetrectcap%
\pgfsetroundjoin%
\pgfsetlinewidth{0.803000pt}%
\definecolor{currentstroke}{rgb}{0.690196,0.690196,0.690196}%
\pgfsetstrokecolor{currentstroke}%
\pgfsetdash{}{0pt}%
\pgfpathmoveto{\pgfqpoint{2.240000in}{0.720000in}}%
\pgfpathlineto{\pgfqpoint{2.240000in}{5.340000in}}%
\pgfusepath{stroke}%
\end{pgfscope}%
\begin{pgfscope}%
\pgfsetbuttcap%
\pgfsetroundjoin%
\definecolor{currentfill}{rgb}{0.000000,0.000000,0.000000}%
\pgfsetfillcolor{currentfill}%
\pgfsetlinewidth{0.803000pt}%
\definecolor{currentstroke}{rgb}{0.000000,0.000000,0.000000}%
\pgfsetstrokecolor{currentstroke}%
\pgfsetdash{}{0pt}%
\pgfsys@defobject{currentmarker}{\pgfqpoint{0.000000in}{-0.048611in}}{\pgfqpoint{0.000000in}{0.000000in}}{%
\pgfpathmoveto{\pgfqpoint{0.000000in}{0.000000in}}%
\pgfpathlineto{\pgfqpoint{0.000000in}{-0.048611in}}%
\pgfusepath{stroke,fill}%
}%
\begin{pgfscope}%
\pgfsys@transformshift{2.240000in}{0.720000in}%
\pgfsys@useobject{currentmarker}{}%
\end{pgfscope}%
\end{pgfscope}%
\begin{pgfscope}%
\definecolor{textcolor}{rgb}{0.000000,0.000000,0.000000}%
\pgfsetstrokecolor{textcolor}%
\pgfsetfillcolor{textcolor}%
\pgftext[x=2.240000in,y=0.622778in,,top]{\color{textcolor}\sffamily\fontsize{20.000000}{24.000000}\selectfont \(\displaystyle {200}\)}%
\end{pgfscope}%
\begin{pgfscope}%
\pgfpathrectangle{\pgfqpoint{1.000000in}{0.720000in}}{\pgfqpoint{6.200000in}{4.620000in}}%
\pgfusepath{clip}%
\pgfsetrectcap%
\pgfsetroundjoin%
\pgfsetlinewidth{0.803000pt}%
\definecolor{currentstroke}{rgb}{0.690196,0.690196,0.690196}%
\pgfsetstrokecolor{currentstroke}%
\pgfsetdash{}{0pt}%
\pgfpathmoveto{\pgfqpoint{3.790000in}{0.720000in}}%
\pgfpathlineto{\pgfqpoint{3.790000in}{5.340000in}}%
\pgfusepath{stroke}%
\end{pgfscope}%
\begin{pgfscope}%
\pgfsetbuttcap%
\pgfsetroundjoin%
\definecolor{currentfill}{rgb}{0.000000,0.000000,0.000000}%
\pgfsetfillcolor{currentfill}%
\pgfsetlinewidth{0.803000pt}%
\definecolor{currentstroke}{rgb}{0.000000,0.000000,0.000000}%
\pgfsetstrokecolor{currentstroke}%
\pgfsetdash{}{0pt}%
\pgfsys@defobject{currentmarker}{\pgfqpoint{0.000000in}{-0.048611in}}{\pgfqpoint{0.000000in}{0.000000in}}{%
\pgfpathmoveto{\pgfqpoint{0.000000in}{0.000000in}}%
\pgfpathlineto{\pgfqpoint{0.000000in}{-0.048611in}}%
\pgfusepath{stroke,fill}%
}%
\begin{pgfscope}%
\pgfsys@transformshift{3.790000in}{0.720000in}%
\pgfsys@useobject{currentmarker}{}%
\end{pgfscope}%
\end{pgfscope}%
\begin{pgfscope}%
\definecolor{textcolor}{rgb}{0.000000,0.000000,0.000000}%
\pgfsetstrokecolor{textcolor}%
\pgfsetfillcolor{textcolor}%
\pgftext[x=3.790000in,y=0.622778in,,top]{\color{textcolor}\sffamily\fontsize{20.000000}{24.000000}\selectfont \(\displaystyle {250}\)}%
\end{pgfscope}%
\begin{pgfscope}%
\pgfpathrectangle{\pgfqpoint{1.000000in}{0.720000in}}{\pgfqpoint{6.200000in}{4.620000in}}%
\pgfusepath{clip}%
\pgfsetrectcap%
\pgfsetroundjoin%
\pgfsetlinewidth{0.803000pt}%
\definecolor{currentstroke}{rgb}{0.690196,0.690196,0.690196}%
\pgfsetstrokecolor{currentstroke}%
\pgfsetdash{}{0pt}%
\pgfpathmoveto{\pgfqpoint{5.340000in}{0.720000in}}%
\pgfpathlineto{\pgfqpoint{5.340000in}{5.340000in}}%
\pgfusepath{stroke}%
\end{pgfscope}%
\begin{pgfscope}%
\pgfsetbuttcap%
\pgfsetroundjoin%
\definecolor{currentfill}{rgb}{0.000000,0.000000,0.000000}%
\pgfsetfillcolor{currentfill}%
\pgfsetlinewidth{0.803000pt}%
\definecolor{currentstroke}{rgb}{0.000000,0.000000,0.000000}%
\pgfsetstrokecolor{currentstroke}%
\pgfsetdash{}{0pt}%
\pgfsys@defobject{currentmarker}{\pgfqpoint{0.000000in}{-0.048611in}}{\pgfqpoint{0.000000in}{0.000000in}}{%
\pgfpathmoveto{\pgfqpoint{0.000000in}{0.000000in}}%
\pgfpathlineto{\pgfqpoint{0.000000in}{-0.048611in}}%
\pgfusepath{stroke,fill}%
}%
\begin{pgfscope}%
\pgfsys@transformshift{5.340000in}{0.720000in}%
\pgfsys@useobject{currentmarker}{}%
\end{pgfscope}%
\end{pgfscope}%
\begin{pgfscope}%
\definecolor{textcolor}{rgb}{0.000000,0.000000,0.000000}%
\pgfsetstrokecolor{textcolor}%
\pgfsetfillcolor{textcolor}%
\pgftext[x=5.340000in,y=0.622778in,,top]{\color{textcolor}\sffamily\fontsize{20.000000}{24.000000}\selectfont \(\displaystyle {300}\)}%
\end{pgfscope}%
\begin{pgfscope}%
\pgfpathrectangle{\pgfqpoint{1.000000in}{0.720000in}}{\pgfqpoint{6.200000in}{4.620000in}}%
\pgfusepath{clip}%
\pgfsetrectcap%
\pgfsetroundjoin%
\pgfsetlinewidth{0.803000pt}%
\definecolor{currentstroke}{rgb}{0.690196,0.690196,0.690196}%
\pgfsetstrokecolor{currentstroke}%
\pgfsetdash{}{0pt}%
\pgfpathmoveto{\pgfqpoint{6.890000in}{0.720000in}}%
\pgfpathlineto{\pgfqpoint{6.890000in}{5.340000in}}%
\pgfusepath{stroke}%
\end{pgfscope}%
\begin{pgfscope}%
\pgfsetbuttcap%
\pgfsetroundjoin%
\definecolor{currentfill}{rgb}{0.000000,0.000000,0.000000}%
\pgfsetfillcolor{currentfill}%
\pgfsetlinewidth{0.803000pt}%
\definecolor{currentstroke}{rgb}{0.000000,0.000000,0.000000}%
\pgfsetstrokecolor{currentstroke}%
\pgfsetdash{}{0pt}%
\pgfsys@defobject{currentmarker}{\pgfqpoint{0.000000in}{-0.048611in}}{\pgfqpoint{0.000000in}{0.000000in}}{%
\pgfpathmoveto{\pgfqpoint{0.000000in}{0.000000in}}%
\pgfpathlineto{\pgfqpoint{0.000000in}{-0.048611in}}%
\pgfusepath{stroke,fill}%
}%
\begin{pgfscope}%
\pgfsys@transformshift{6.890000in}{0.720000in}%
\pgfsys@useobject{currentmarker}{}%
\end{pgfscope}%
\end{pgfscope}%
\begin{pgfscope}%
\definecolor{textcolor}{rgb}{0.000000,0.000000,0.000000}%
\pgfsetstrokecolor{textcolor}%
\pgfsetfillcolor{textcolor}%
\pgftext[x=6.890000in,y=0.622778in,,top]{\color{textcolor}\sffamily\fontsize{20.000000}{24.000000}\selectfont \(\displaystyle {350}\)}%
\end{pgfscope}%
\begin{pgfscope}%
\definecolor{textcolor}{rgb}{0.000000,0.000000,0.000000}%
\pgfsetstrokecolor{textcolor}%
\pgfsetfillcolor{textcolor}%
\pgftext[x=4.100000in,y=0.311155in,,top]{\color{textcolor}\sffamily\fontsize{20.000000}{24.000000}\selectfont \(\displaystyle \mathrm{t}/\si{ns}\)}%
\end{pgfscope}%
\begin{pgfscope}%
\pgfpathrectangle{\pgfqpoint{1.000000in}{0.720000in}}{\pgfqpoint{6.200000in}{4.620000in}}%
\pgfusepath{clip}%
\pgfsetrectcap%
\pgfsetroundjoin%
\pgfsetlinewidth{0.803000pt}%
\definecolor{currentstroke}{rgb}{0.690196,0.690196,0.690196}%
\pgfsetstrokecolor{currentstroke}%
\pgfsetdash{}{0pt}%
\pgfpathmoveto{\pgfqpoint{1.000000in}{0.930000in}}%
\pgfpathlineto{\pgfqpoint{7.200000in}{0.930000in}}%
\pgfusepath{stroke}%
\end{pgfscope}%
\begin{pgfscope}%
\pgfsetbuttcap%
\pgfsetroundjoin%
\definecolor{currentfill}{rgb}{0.000000,0.000000,0.000000}%
\pgfsetfillcolor{currentfill}%
\pgfsetlinewidth{0.803000pt}%
\definecolor{currentstroke}{rgb}{0.000000,0.000000,0.000000}%
\pgfsetstrokecolor{currentstroke}%
\pgfsetdash{}{0pt}%
\pgfsys@defobject{currentmarker}{\pgfqpoint{-0.048611in}{0.000000in}}{\pgfqpoint{-0.000000in}{0.000000in}}{%
\pgfpathmoveto{\pgfqpoint{-0.000000in}{0.000000in}}%
\pgfpathlineto{\pgfqpoint{-0.048611in}{0.000000in}}%
\pgfusepath{stroke,fill}%
}%
\begin{pgfscope}%
\pgfsys@transformshift{1.000000in}{0.930000in}%
\pgfsys@useobject{currentmarker}{}%
\end{pgfscope}%
\end{pgfscope}%
\begin{pgfscope}%
\definecolor{textcolor}{rgb}{0.000000,0.000000,0.000000}%
\pgfsetstrokecolor{textcolor}%
\pgfsetfillcolor{textcolor}%
\pgftext[x=0.560215in, y=0.829981in, left, base]{\color{textcolor}\sffamily\fontsize{20.000000}{24.000000}\selectfont \(\displaystyle {0.0}\)}%
\end{pgfscope}%
\begin{pgfscope}%
\pgfpathrectangle{\pgfqpoint{1.000000in}{0.720000in}}{\pgfqpoint{6.200000in}{4.620000in}}%
\pgfusepath{clip}%
\pgfsetrectcap%
\pgfsetroundjoin%
\pgfsetlinewidth{0.803000pt}%
\definecolor{currentstroke}{rgb}{0.690196,0.690196,0.690196}%
\pgfsetstrokecolor{currentstroke}%
\pgfsetdash{}{0pt}%
\pgfpathmoveto{\pgfqpoint{1.000000in}{1.753655in}}%
\pgfpathlineto{\pgfqpoint{7.200000in}{1.753655in}}%
\pgfusepath{stroke}%
\end{pgfscope}%
\begin{pgfscope}%
\pgfsetbuttcap%
\pgfsetroundjoin%
\definecolor{currentfill}{rgb}{0.000000,0.000000,0.000000}%
\pgfsetfillcolor{currentfill}%
\pgfsetlinewidth{0.803000pt}%
\definecolor{currentstroke}{rgb}{0.000000,0.000000,0.000000}%
\pgfsetstrokecolor{currentstroke}%
\pgfsetdash{}{0pt}%
\pgfsys@defobject{currentmarker}{\pgfqpoint{-0.048611in}{0.000000in}}{\pgfqpoint{-0.000000in}{0.000000in}}{%
\pgfpathmoveto{\pgfqpoint{-0.000000in}{0.000000in}}%
\pgfpathlineto{\pgfqpoint{-0.048611in}{0.000000in}}%
\pgfusepath{stroke,fill}%
}%
\begin{pgfscope}%
\pgfsys@transformshift{1.000000in}{1.753655in}%
\pgfsys@useobject{currentmarker}{}%
\end{pgfscope}%
\end{pgfscope}%
\begin{pgfscope}%
\definecolor{textcolor}{rgb}{0.000000,0.000000,0.000000}%
\pgfsetstrokecolor{textcolor}%
\pgfsetfillcolor{textcolor}%
\pgftext[x=0.560215in, y=1.653636in, left, base]{\color{textcolor}\sffamily\fontsize{20.000000}{24.000000}\selectfont \(\displaystyle {0.2}\)}%
\end{pgfscope}%
\begin{pgfscope}%
\pgfpathrectangle{\pgfqpoint{1.000000in}{0.720000in}}{\pgfqpoint{6.200000in}{4.620000in}}%
\pgfusepath{clip}%
\pgfsetrectcap%
\pgfsetroundjoin%
\pgfsetlinewidth{0.803000pt}%
\definecolor{currentstroke}{rgb}{0.690196,0.690196,0.690196}%
\pgfsetstrokecolor{currentstroke}%
\pgfsetdash{}{0pt}%
\pgfpathmoveto{\pgfqpoint{1.000000in}{2.577311in}}%
\pgfpathlineto{\pgfqpoint{7.200000in}{2.577311in}}%
\pgfusepath{stroke}%
\end{pgfscope}%
\begin{pgfscope}%
\pgfsetbuttcap%
\pgfsetroundjoin%
\definecolor{currentfill}{rgb}{0.000000,0.000000,0.000000}%
\pgfsetfillcolor{currentfill}%
\pgfsetlinewidth{0.803000pt}%
\definecolor{currentstroke}{rgb}{0.000000,0.000000,0.000000}%
\pgfsetstrokecolor{currentstroke}%
\pgfsetdash{}{0pt}%
\pgfsys@defobject{currentmarker}{\pgfqpoint{-0.048611in}{0.000000in}}{\pgfqpoint{-0.000000in}{0.000000in}}{%
\pgfpathmoveto{\pgfqpoint{-0.000000in}{0.000000in}}%
\pgfpathlineto{\pgfqpoint{-0.048611in}{0.000000in}}%
\pgfusepath{stroke,fill}%
}%
\begin{pgfscope}%
\pgfsys@transformshift{1.000000in}{2.577311in}%
\pgfsys@useobject{currentmarker}{}%
\end{pgfscope}%
\end{pgfscope}%
\begin{pgfscope}%
\definecolor{textcolor}{rgb}{0.000000,0.000000,0.000000}%
\pgfsetstrokecolor{textcolor}%
\pgfsetfillcolor{textcolor}%
\pgftext[x=0.560215in, y=2.477291in, left, base]{\color{textcolor}\sffamily\fontsize{20.000000}{24.000000}\selectfont \(\displaystyle {0.4}\)}%
\end{pgfscope}%
\begin{pgfscope}%
\pgfpathrectangle{\pgfqpoint{1.000000in}{0.720000in}}{\pgfqpoint{6.200000in}{4.620000in}}%
\pgfusepath{clip}%
\pgfsetrectcap%
\pgfsetroundjoin%
\pgfsetlinewidth{0.803000pt}%
\definecolor{currentstroke}{rgb}{0.690196,0.690196,0.690196}%
\pgfsetstrokecolor{currentstroke}%
\pgfsetdash{}{0pt}%
\pgfpathmoveto{\pgfqpoint{1.000000in}{3.400966in}}%
\pgfpathlineto{\pgfqpoint{7.200000in}{3.400966in}}%
\pgfusepath{stroke}%
\end{pgfscope}%
\begin{pgfscope}%
\pgfsetbuttcap%
\pgfsetroundjoin%
\definecolor{currentfill}{rgb}{0.000000,0.000000,0.000000}%
\pgfsetfillcolor{currentfill}%
\pgfsetlinewidth{0.803000pt}%
\definecolor{currentstroke}{rgb}{0.000000,0.000000,0.000000}%
\pgfsetstrokecolor{currentstroke}%
\pgfsetdash{}{0pt}%
\pgfsys@defobject{currentmarker}{\pgfqpoint{-0.048611in}{0.000000in}}{\pgfqpoint{-0.000000in}{0.000000in}}{%
\pgfpathmoveto{\pgfqpoint{-0.000000in}{0.000000in}}%
\pgfpathlineto{\pgfqpoint{-0.048611in}{0.000000in}}%
\pgfusepath{stroke,fill}%
}%
\begin{pgfscope}%
\pgfsys@transformshift{1.000000in}{3.400966in}%
\pgfsys@useobject{currentmarker}{}%
\end{pgfscope}%
\end{pgfscope}%
\begin{pgfscope}%
\definecolor{textcolor}{rgb}{0.000000,0.000000,0.000000}%
\pgfsetstrokecolor{textcolor}%
\pgfsetfillcolor{textcolor}%
\pgftext[x=0.560215in, y=3.300947in, left, base]{\color{textcolor}\sffamily\fontsize{20.000000}{24.000000}\selectfont \(\displaystyle {0.6}\)}%
\end{pgfscope}%
\begin{pgfscope}%
\pgfpathrectangle{\pgfqpoint{1.000000in}{0.720000in}}{\pgfqpoint{6.200000in}{4.620000in}}%
\pgfusepath{clip}%
\pgfsetrectcap%
\pgfsetroundjoin%
\pgfsetlinewidth{0.803000pt}%
\definecolor{currentstroke}{rgb}{0.690196,0.690196,0.690196}%
\pgfsetstrokecolor{currentstroke}%
\pgfsetdash{}{0pt}%
\pgfpathmoveto{\pgfqpoint{1.000000in}{4.224621in}}%
\pgfpathlineto{\pgfqpoint{7.200000in}{4.224621in}}%
\pgfusepath{stroke}%
\end{pgfscope}%
\begin{pgfscope}%
\pgfsetbuttcap%
\pgfsetroundjoin%
\definecolor{currentfill}{rgb}{0.000000,0.000000,0.000000}%
\pgfsetfillcolor{currentfill}%
\pgfsetlinewidth{0.803000pt}%
\definecolor{currentstroke}{rgb}{0.000000,0.000000,0.000000}%
\pgfsetstrokecolor{currentstroke}%
\pgfsetdash{}{0pt}%
\pgfsys@defobject{currentmarker}{\pgfqpoint{-0.048611in}{0.000000in}}{\pgfqpoint{-0.000000in}{0.000000in}}{%
\pgfpathmoveto{\pgfqpoint{-0.000000in}{0.000000in}}%
\pgfpathlineto{\pgfqpoint{-0.048611in}{0.000000in}}%
\pgfusepath{stroke,fill}%
}%
\begin{pgfscope}%
\pgfsys@transformshift{1.000000in}{4.224621in}%
\pgfsys@useobject{currentmarker}{}%
\end{pgfscope}%
\end{pgfscope}%
\begin{pgfscope}%
\definecolor{textcolor}{rgb}{0.000000,0.000000,0.000000}%
\pgfsetstrokecolor{textcolor}%
\pgfsetfillcolor{textcolor}%
\pgftext[x=0.560215in, y=4.124602in, left, base]{\color{textcolor}\sffamily\fontsize{20.000000}{24.000000}\selectfont \(\displaystyle {0.8}\)}%
\end{pgfscope}%
\begin{pgfscope}%
\pgfpathrectangle{\pgfqpoint{1.000000in}{0.720000in}}{\pgfqpoint{6.200000in}{4.620000in}}%
\pgfusepath{clip}%
\pgfsetrectcap%
\pgfsetroundjoin%
\pgfsetlinewidth{0.803000pt}%
\definecolor{currentstroke}{rgb}{0.690196,0.690196,0.690196}%
\pgfsetstrokecolor{currentstroke}%
\pgfsetdash{}{0pt}%
\pgfpathmoveto{\pgfqpoint{1.000000in}{5.048277in}}%
\pgfpathlineto{\pgfqpoint{7.200000in}{5.048277in}}%
\pgfusepath{stroke}%
\end{pgfscope}%
\begin{pgfscope}%
\pgfsetbuttcap%
\pgfsetroundjoin%
\definecolor{currentfill}{rgb}{0.000000,0.000000,0.000000}%
\pgfsetfillcolor{currentfill}%
\pgfsetlinewidth{0.803000pt}%
\definecolor{currentstroke}{rgb}{0.000000,0.000000,0.000000}%
\pgfsetstrokecolor{currentstroke}%
\pgfsetdash{}{0pt}%
\pgfsys@defobject{currentmarker}{\pgfqpoint{-0.048611in}{0.000000in}}{\pgfqpoint{-0.000000in}{0.000000in}}{%
\pgfpathmoveto{\pgfqpoint{-0.000000in}{0.000000in}}%
\pgfpathlineto{\pgfqpoint{-0.048611in}{0.000000in}}%
\pgfusepath{stroke,fill}%
}%
\begin{pgfscope}%
\pgfsys@transformshift{1.000000in}{5.048277in}%
\pgfsys@useobject{currentmarker}{}%
\end{pgfscope}%
\end{pgfscope}%
\begin{pgfscope}%
\definecolor{textcolor}{rgb}{0.000000,0.000000,0.000000}%
\pgfsetstrokecolor{textcolor}%
\pgfsetfillcolor{textcolor}%
\pgftext[x=0.560215in, y=4.948257in, left, base]{\color{textcolor}\sffamily\fontsize{20.000000}{24.000000}\selectfont \(\displaystyle {1.0}\)}%
\end{pgfscope}%
\begin{pgfscope}%
\definecolor{textcolor}{rgb}{0.000000,0.000000,0.000000}%
\pgfsetstrokecolor{textcolor}%
\pgfsetfillcolor{textcolor}%
\pgftext[x=0.504660in,y=3.030000in,,bottom,rotate=90.000000]{\color{textcolor}\sffamily\fontsize{20.000000}{24.000000}\selectfont \(\displaystyle \mathrm{Charge}/\si{mV\cdot ns}\)}%
\end{pgfscope}%
\begin{pgfscope}%
\pgfpathrectangle{\pgfqpoint{1.000000in}{0.720000in}}{\pgfqpoint{6.200000in}{4.620000in}}%
\pgfusepath{clip}%
\pgfsetbuttcap%
\pgfsetroundjoin%
\pgfsetlinewidth{2.007500pt}%
\definecolor{currentstroke}{rgb}{1.000000,0.000000,0.000000}%
\pgfsetstrokecolor{currentstroke}%
\pgfsetdash{}{0pt}%
\pgfpathmoveto{\pgfqpoint{2.187304in}{0.930000in}}%
\pgfpathlineto{\pgfqpoint{2.187304in}{3.481113in}}%
\pgfusepath{stroke}%
\end{pgfscope}%
\begin{pgfscope}%
\pgfpathrectangle{\pgfqpoint{1.000000in}{0.720000in}}{\pgfqpoint{6.200000in}{4.620000in}}%
\pgfusepath{clip}%
\pgfsetbuttcap%
\pgfsetroundjoin%
\pgfsetlinewidth{2.007500pt}%
\definecolor{currentstroke}{rgb}{1.000000,0.000000,0.000000}%
\pgfsetstrokecolor{currentstroke}%
\pgfsetdash{}{0pt}%
\pgfpathmoveto{\pgfqpoint{2.630655in}{0.930000in}}%
\pgfpathlineto{\pgfqpoint{2.630655in}{4.870032in}}%
\pgfusepath{stroke}%
\end{pgfscope}%
\begin{pgfscope}%
\pgfpathrectangle{\pgfqpoint{1.000000in}{0.720000in}}{\pgfqpoint{6.200000in}{4.620000in}}%
\pgfusepath{clip}%
\pgfsetbuttcap%
\pgfsetroundjoin%
\pgfsetlinewidth{2.007500pt}%
\definecolor{currentstroke}{rgb}{1.000000,0.000000,0.000000}%
\pgfsetstrokecolor{currentstroke}%
\pgfsetdash{}{0pt}%
\pgfpathmoveto{\pgfqpoint{2.927239in}{0.930000in}}%
\pgfpathlineto{\pgfqpoint{2.927239in}{5.130000in}}%
\pgfusepath{stroke}%
\end{pgfscope}%
\begin{pgfscope}%
\pgfpathrectangle{\pgfqpoint{1.000000in}{0.720000in}}{\pgfqpoint{6.200000in}{4.620000in}}%
\pgfusepath{clip}%
\pgfsetbuttcap%
\pgfsetroundjoin%
\pgfsetlinewidth{2.007500pt}%
\definecolor{currentstroke}{rgb}{1.000000,0.000000,0.000000}%
\pgfsetstrokecolor{currentstroke}%
\pgfsetdash{}{0pt}%
\pgfpathmoveto{\pgfqpoint{3.106029in}{0.930000in}}%
\pgfpathlineto{\pgfqpoint{3.106029in}{3.099503in}}%
\pgfusepath{stroke}%
\end{pgfscope}%
\begin{pgfscope}%
\pgfpathrectangle{\pgfqpoint{1.000000in}{0.720000in}}{\pgfqpoint{6.200000in}{4.620000in}}%
\pgfusepath{clip}%
\pgfsetbuttcap%
\pgfsetroundjoin%
\pgfsetlinewidth{2.007500pt}%
\definecolor{currentstroke}{rgb}{1.000000,0.000000,0.000000}%
\pgfsetstrokecolor{currentstroke}%
\pgfsetdash{}{0pt}%
\pgfpathmoveto{\pgfqpoint{3.690495in}{0.930000in}}%
\pgfpathlineto{\pgfqpoint{3.690495in}{3.004789in}}%
\pgfusepath{stroke}%
\end{pgfscope}%
\begin{pgfscope}%
\pgfsetrectcap%
\pgfsetmiterjoin%
\pgfsetlinewidth{0.803000pt}%
\definecolor{currentstroke}{rgb}{0.000000,0.000000,0.000000}%
\pgfsetstrokecolor{currentstroke}%
\pgfsetdash{}{0pt}%
\pgfpathmoveto{\pgfqpoint{1.000000in}{0.720000in}}%
\pgfpathlineto{\pgfqpoint{1.000000in}{5.340000in}}%
\pgfusepath{stroke}%
\end{pgfscope}%
\begin{pgfscope}%
\pgfsetrectcap%
\pgfsetmiterjoin%
\pgfsetlinewidth{0.803000pt}%
\definecolor{currentstroke}{rgb}{0.000000,0.000000,0.000000}%
\pgfsetstrokecolor{currentstroke}%
\pgfsetdash{}{0pt}%
\pgfpathmoveto{\pgfqpoint{7.200000in}{0.720000in}}%
\pgfpathlineto{\pgfqpoint{7.200000in}{5.340000in}}%
\pgfusepath{stroke}%
\end{pgfscope}%
\begin{pgfscope}%
\pgfsetrectcap%
\pgfsetmiterjoin%
\pgfsetlinewidth{0.803000pt}%
\definecolor{currentstroke}{rgb}{0.000000,0.000000,0.000000}%
\pgfsetstrokecolor{currentstroke}%
\pgfsetdash{}{0pt}%
\pgfpathmoveto{\pgfqpoint{1.000000in}{0.720000in}}%
\pgfpathlineto{\pgfqpoint{7.200000in}{0.720000in}}%
\pgfusepath{stroke}%
\end{pgfscope}%
\begin{pgfscope}%
\pgfsetrectcap%
\pgfsetmiterjoin%
\pgfsetlinewidth{0.803000pt}%
\definecolor{currentstroke}{rgb}{0.000000,0.000000,0.000000}%
\pgfsetstrokecolor{currentstroke}%
\pgfsetdash{}{0pt}%
\pgfpathmoveto{\pgfqpoint{1.000000in}{5.340000in}}%
\pgfpathlineto{\pgfqpoint{7.200000in}{5.340000in}}%
\pgfusepath{stroke}%
\end{pgfscope}%
\begin{pgfscope}%
\pgfsetroundcap%
\pgfsetroundjoin%
\definecolor{currentfill}{rgb}{0.000000,0.000000,0.000000}%
\pgfsetfillcolor{currentfill}%
\pgfsetlinewidth{1.003750pt}%
\definecolor{currentstroke}{rgb}{0.000000,0.000000,0.000000}%
\pgfsetstrokecolor{currentstroke}%
\pgfsetdash{}{0pt}%
\pgfpathmoveto{\pgfqpoint{5.987778in}{2.706630in}}%
\pgfpathquadraticcurveto{\pgfqpoint{5.987778in}{1.496318in}}{\pgfqpoint{5.987778in}{0.286007in}}%
\pgfpathlineto{\pgfqpoint{6.071111in}{0.286007in}}%
\pgfpathquadraticcurveto{\pgfqpoint{6.015556in}{0.202688in}}{\pgfqpoint{5.960000in}{0.119369in}}%
\pgfpathquadraticcurveto{\pgfqpoint{5.904444in}{0.202688in}}{\pgfqpoint{5.848889in}{0.286007in}}%
\pgfpathlineto{\pgfqpoint{5.932222in}{0.286007in}}%
\pgfpathquadraticcurveto{\pgfqpoint{5.932222in}{1.496318in}}{\pgfqpoint{5.932222in}{2.706630in}}%
\pgfpathlineto{\pgfqpoint{5.987778in}{2.706630in}}%
\pgfpathclose%
\pgfusepath{stroke,fill}%
\end{pgfscope}%
\begin{pgfscope}%
\pgfsetbuttcap%
\pgfsetmiterjoin%
\definecolor{currentfill}{rgb}{1.000000,1.000000,1.000000}%
\pgfsetfillcolor{currentfill}%
\pgfsetfillopacity{0.800000}%
\pgfsetlinewidth{1.003750pt}%
\definecolor{currentstroke}{rgb}{0.800000,0.800000,0.800000}%
\pgfsetstrokecolor{currentstroke}%
\pgfsetstrokeopacity{0.800000}%
\pgfsetdash{}{0pt}%
\pgfpathmoveto{\pgfqpoint{5.981275in}{4.717559in}}%
\pgfpathlineto{\pgfqpoint{7.005556in}{4.717559in}}%
\pgfpathquadraticcurveto{\pgfqpoint{7.061111in}{4.717559in}}{\pgfqpoint{7.061111in}{4.773115in}}%
\pgfpathlineto{\pgfqpoint{7.061111in}{5.145556in}}%
\pgfpathquadraticcurveto{\pgfqpoint{7.061111in}{5.201111in}}{\pgfqpoint{7.005556in}{5.201111in}}%
\pgfpathlineto{\pgfqpoint{5.981275in}{5.201111in}}%
\pgfpathquadraticcurveto{\pgfqpoint{5.925720in}{5.201111in}}{\pgfqpoint{5.925720in}{5.145556in}}%
\pgfpathlineto{\pgfqpoint{5.925720in}{4.773115in}}%
\pgfpathquadraticcurveto{\pgfqpoint{5.925720in}{4.717559in}}{\pgfqpoint{5.981275in}{4.717559in}}%
\pgfpathclose%
\pgfusepath{stroke,fill}%
\end{pgfscope}%
\begin{pgfscope}%
\pgfsetbuttcap%
\pgfsetroundjoin%
\pgfsetlinewidth{2.007500pt}%
\definecolor{currentstroke}{rgb}{1.000000,0.000000,0.000000}%
\pgfsetstrokecolor{currentstroke}%
\pgfsetdash{}{0pt}%
\pgfpathmoveto{\pgfqpoint{6.036831in}{4.981922in}}%
\pgfpathlineto{\pgfqpoint{6.592386in}{4.981922in}}%
\pgfusepath{stroke}%
\end{pgfscope}%
\begin{pgfscope}%
\definecolor{textcolor}{rgb}{0.000000,0.000000,0.000000}%
\pgfsetstrokecolor{textcolor}%
\pgfsetfillcolor{textcolor}%
\pgftext[x=6.814609in,y=4.884699in,left,base]{\color{textcolor}\sffamily\fontsize{20.000000}{24.000000}\selectfont \(\displaystyle \vec{q}\)}%
\end{pgfscope}%
\end{pgfpicture}%
\makeatother%
\endgroup%
}
\end{figure}
\begin{figure}
    \centering
    \resizebox{0.8\textwidth}{!}{%% Creator: Matplotlib, PGF backend
%%
%% To include the figure in your LaTeX document, write
%%   \input{<filename>.pgf}
%%
%% Make sure the required packages are loaded in your preamble
%%   \usepackage{pgf}
%%
%% and, on pdftex
%%   \usepackage[utf8]{inputenc}\DeclareUnicodeCharacter{2212}{-}
%%
%% or, on luatex and xetex
%%   \usepackage{unicode-math}
%%
%% Figures using additional raster images can only be included by \input if
%% they are in the same directory as the main LaTeX file. For loading figures
%% from other directories you can use the `import` package
%%   \usepackage{import}
%%
%% and then include the figures with
%%   \import{<path to file>}{<filename>.pgf}
%%
%% Matplotlib used the following preamble
%%   \usepackage[detect-all,locale=DE]{siunitx}
%%
\begingroup%
\makeatletter%
\begin{pgfpicture}%
\pgfpathrectangle{\pgfpointorigin}{\pgfqpoint{8.000000in}{6.000000in}}%
\pgfusepath{use as bounding box, clip}%
\begin{pgfscope}%
\pgfsetbuttcap%
\pgfsetmiterjoin%
\definecolor{currentfill}{rgb}{1.000000,1.000000,1.000000}%
\pgfsetfillcolor{currentfill}%
\pgfsetlinewidth{0.000000pt}%
\definecolor{currentstroke}{rgb}{1.000000,1.000000,1.000000}%
\pgfsetstrokecolor{currentstroke}%
\pgfsetdash{}{0pt}%
\pgfpathmoveto{\pgfqpoint{0.000000in}{0.000000in}}%
\pgfpathlineto{\pgfqpoint{8.000000in}{0.000000in}}%
\pgfpathlineto{\pgfqpoint{8.000000in}{6.000000in}}%
\pgfpathlineto{\pgfqpoint{0.000000in}{6.000000in}}%
\pgfpathclose%
\pgfusepath{fill}%
\end{pgfscope}%
\begin{pgfscope}%
\pgfsetbuttcap%
\pgfsetmiterjoin%
\definecolor{currentfill}{rgb}{1.000000,1.000000,1.000000}%
\pgfsetfillcolor{currentfill}%
\pgfsetlinewidth{0.000000pt}%
\definecolor{currentstroke}{rgb}{0.000000,0.000000,0.000000}%
\pgfsetstrokecolor{currentstroke}%
\pgfsetstrokeopacity{0.000000}%
\pgfsetdash{}{0pt}%
\pgfpathmoveto{\pgfqpoint{1.000000in}{0.720000in}}%
\pgfpathlineto{\pgfqpoint{7.200000in}{0.720000in}}%
\pgfpathlineto{\pgfqpoint{7.200000in}{5.340000in}}%
\pgfpathlineto{\pgfqpoint{1.000000in}{5.340000in}}%
\pgfpathclose%
\pgfusepath{fill}%
\end{pgfscope}%
\begin{pgfscope}%
\pgfpathrectangle{\pgfqpoint{1.000000in}{0.720000in}}{\pgfqpoint{6.200000in}{4.620000in}}%
\pgfusepath{clip}%
\pgfsetrectcap%
\pgfsetroundjoin%
\pgfsetlinewidth{0.803000pt}%
\definecolor{currentstroke}{rgb}{0.690196,0.690196,0.690196}%
\pgfsetstrokecolor{currentstroke}%
\pgfsetdash{}{0pt}%
\pgfpathmoveto{\pgfqpoint{2.240000in}{0.720000in}}%
\pgfpathlineto{\pgfqpoint{2.240000in}{5.340000in}}%
\pgfusepath{stroke}%
\end{pgfscope}%
\begin{pgfscope}%
\pgfsetbuttcap%
\pgfsetroundjoin%
\definecolor{currentfill}{rgb}{0.000000,0.000000,0.000000}%
\pgfsetfillcolor{currentfill}%
\pgfsetlinewidth{0.803000pt}%
\definecolor{currentstroke}{rgb}{0.000000,0.000000,0.000000}%
\pgfsetstrokecolor{currentstroke}%
\pgfsetdash{}{0pt}%
\pgfsys@defobject{currentmarker}{\pgfqpoint{0.000000in}{-0.048611in}}{\pgfqpoint{0.000000in}{0.000000in}}{%
\pgfpathmoveto{\pgfqpoint{0.000000in}{0.000000in}}%
\pgfpathlineto{\pgfqpoint{0.000000in}{-0.048611in}}%
\pgfusepath{stroke,fill}%
}%
\begin{pgfscope}%
\pgfsys@transformshift{2.240000in}{0.720000in}%
\pgfsys@useobject{currentmarker}{}%
\end{pgfscope}%
\end{pgfscope}%
\begin{pgfscope}%
\definecolor{textcolor}{rgb}{0.000000,0.000000,0.000000}%
\pgfsetstrokecolor{textcolor}%
\pgfsetfillcolor{textcolor}%
\pgftext[x=2.240000in,y=0.622778in,,top]{\color{textcolor}\sffamily\fontsize{20.000000}{24.000000}\selectfont \(\displaystyle {200}\)}%
\end{pgfscope}%
\begin{pgfscope}%
\pgfpathrectangle{\pgfqpoint{1.000000in}{0.720000in}}{\pgfqpoint{6.200000in}{4.620000in}}%
\pgfusepath{clip}%
\pgfsetrectcap%
\pgfsetroundjoin%
\pgfsetlinewidth{0.803000pt}%
\definecolor{currentstroke}{rgb}{0.690196,0.690196,0.690196}%
\pgfsetstrokecolor{currentstroke}%
\pgfsetdash{}{0pt}%
\pgfpathmoveto{\pgfqpoint{3.790000in}{0.720000in}}%
\pgfpathlineto{\pgfqpoint{3.790000in}{5.340000in}}%
\pgfusepath{stroke}%
\end{pgfscope}%
\begin{pgfscope}%
\pgfsetbuttcap%
\pgfsetroundjoin%
\definecolor{currentfill}{rgb}{0.000000,0.000000,0.000000}%
\pgfsetfillcolor{currentfill}%
\pgfsetlinewidth{0.803000pt}%
\definecolor{currentstroke}{rgb}{0.000000,0.000000,0.000000}%
\pgfsetstrokecolor{currentstroke}%
\pgfsetdash{}{0pt}%
\pgfsys@defobject{currentmarker}{\pgfqpoint{0.000000in}{-0.048611in}}{\pgfqpoint{0.000000in}{0.000000in}}{%
\pgfpathmoveto{\pgfqpoint{0.000000in}{0.000000in}}%
\pgfpathlineto{\pgfqpoint{0.000000in}{-0.048611in}}%
\pgfusepath{stroke,fill}%
}%
\begin{pgfscope}%
\pgfsys@transformshift{3.790000in}{0.720000in}%
\pgfsys@useobject{currentmarker}{}%
\end{pgfscope}%
\end{pgfscope}%
\begin{pgfscope}%
\definecolor{textcolor}{rgb}{0.000000,0.000000,0.000000}%
\pgfsetstrokecolor{textcolor}%
\pgfsetfillcolor{textcolor}%
\pgftext[x=3.790000in,y=0.622778in,,top]{\color{textcolor}\sffamily\fontsize{20.000000}{24.000000}\selectfont \(\displaystyle {250}\)}%
\end{pgfscope}%
\begin{pgfscope}%
\pgfpathrectangle{\pgfqpoint{1.000000in}{0.720000in}}{\pgfqpoint{6.200000in}{4.620000in}}%
\pgfusepath{clip}%
\pgfsetrectcap%
\pgfsetroundjoin%
\pgfsetlinewidth{0.803000pt}%
\definecolor{currentstroke}{rgb}{0.690196,0.690196,0.690196}%
\pgfsetstrokecolor{currentstroke}%
\pgfsetdash{}{0pt}%
\pgfpathmoveto{\pgfqpoint{5.340000in}{0.720000in}}%
\pgfpathlineto{\pgfqpoint{5.340000in}{5.340000in}}%
\pgfusepath{stroke}%
\end{pgfscope}%
\begin{pgfscope}%
\pgfsetbuttcap%
\pgfsetroundjoin%
\definecolor{currentfill}{rgb}{0.000000,0.000000,0.000000}%
\pgfsetfillcolor{currentfill}%
\pgfsetlinewidth{0.803000pt}%
\definecolor{currentstroke}{rgb}{0.000000,0.000000,0.000000}%
\pgfsetstrokecolor{currentstroke}%
\pgfsetdash{}{0pt}%
\pgfsys@defobject{currentmarker}{\pgfqpoint{0.000000in}{-0.048611in}}{\pgfqpoint{0.000000in}{0.000000in}}{%
\pgfpathmoveto{\pgfqpoint{0.000000in}{0.000000in}}%
\pgfpathlineto{\pgfqpoint{0.000000in}{-0.048611in}}%
\pgfusepath{stroke,fill}%
}%
\begin{pgfscope}%
\pgfsys@transformshift{5.340000in}{0.720000in}%
\pgfsys@useobject{currentmarker}{}%
\end{pgfscope}%
\end{pgfscope}%
\begin{pgfscope}%
\definecolor{textcolor}{rgb}{0.000000,0.000000,0.000000}%
\pgfsetstrokecolor{textcolor}%
\pgfsetfillcolor{textcolor}%
\pgftext[x=5.340000in,y=0.622778in,,top]{\color{textcolor}\sffamily\fontsize{20.000000}{24.000000}\selectfont \(\displaystyle {300}\)}%
\end{pgfscope}%
\begin{pgfscope}%
\pgfpathrectangle{\pgfqpoint{1.000000in}{0.720000in}}{\pgfqpoint{6.200000in}{4.620000in}}%
\pgfusepath{clip}%
\pgfsetrectcap%
\pgfsetroundjoin%
\pgfsetlinewidth{0.803000pt}%
\definecolor{currentstroke}{rgb}{0.690196,0.690196,0.690196}%
\pgfsetstrokecolor{currentstroke}%
\pgfsetdash{}{0pt}%
\pgfpathmoveto{\pgfqpoint{6.890000in}{0.720000in}}%
\pgfpathlineto{\pgfqpoint{6.890000in}{5.340000in}}%
\pgfusepath{stroke}%
\end{pgfscope}%
\begin{pgfscope}%
\pgfsetbuttcap%
\pgfsetroundjoin%
\definecolor{currentfill}{rgb}{0.000000,0.000000,0.000000}%
\pgfsetfillcolor{currentfill}%
\pgfsetlinewidth{0.803000pt}%
\definecolor{currentstroke}{rgb}{0.000000,0.000000,0.000000}%
\pgfsetstrokecolor{currentstroke}%
\pgfsetdash{}{0pt}%
\pgfsys@defobject{currentmarker}{\pgfqpoint{0.000000in}{-0.048611in}}{\pgfqpoint{0.000000in}{0.000000in}}{%
\pgfpathmoveto{\pgfqpoint{0.000000in}{0.000000in}}%
\pgfpathlineto{\pgfqpoint{0.000000in}{-0.048611in}}%
\pgfusepath{stroke,fill}%
}%
\begin{pgfscope}%
\pgfsys@transformshift{6.890000in}{0.720000in}%
\pgfsys@useobject{currentmarker}{}%
\end{pgfscope}%
\end{pgfscope}%
\begin{pgfscope}%
\definecolor{textcolor}{rgb}{0.000000,0.000000,0.000000}%
\pgfsetstrokecolor{textcolor}%
\pgfsetfillcolor{textcolor}%
\pgftext[x=6.890000in,y=0.622778in,,top]{\color{textcolor}\sffamily\fontsize{20.000000}{24.000000}\selectfont \(\displaystyle {350}\)}%
\end{pgfscope}%
\begin{pgfscope}%
\definecolor{textcolor}{rgb}{0.000000,0.000000,0.000000}%
\pgfsetstrokecolor{textcolor}%
\pgfsetfillcolor{textcolor}%
\pgftext[x=4.100000in,y=0.311155in,,top]{\color{textcolor}\sffamily\fontsize{20.000000}{24.000000}\selectfont \(\displaystyle \mathrm{t}/\si{ns}\)}%
\end{pgfscope}%
\begin{pgfscope}%
\pgfpathrectangle{\pgfqpoint{1.000000in}{0.720000in}}{\pgfqpoint{6.200000in}{4.620000in}}%
\pgfusepath{clip}%
\pgfsetrectcap%
\pgfsetroundjoin%
\pgfsetlinewidth{0.803000pt}%
\definecolor{currentstroke}{rgb}{0.690196,0.690196,0.690196}%
\pgfsetstrokecolor{currentstroke}%
\pgfsetdash{}{0pt}%
\pgfpathmoveto{\pgfqpoint{1.000000in}{0.930000in}}%
\pgfpathlineto{\pgfqpoint{7.200000in}{0.930000in}}%
\pgfusepath{stroke}%
\end{pgfscope}%
\begin{pgfscope}%
\pgfsetbuttcap%
\pgfsetroundjoin%
\definecolor{currentfill}{rgb}{0.000000,0.000000,0.000000}%
\pgfsetfillcolor{currentfill}%
\pgfsetlinewidth{0.803000pt}%
\definecolor{currentstroke}{rgb}{0.000000,0.000000,0.000000}%
\pgfsetstrokecolor{currentstroke}%
\pgfsetdash{}{0pt}%
\pgfsys@defobject{currentmarker}{\pgfqpoint{-0.048611in}{0.000000in}}{\pgfqpoint{-0.000000in}{0.000000in}}{%
\pgfpathmoveto{\pgfqpoint{-0.000000in}{0.000000in}}%
\pgfpathlineto{\pgfqpoint{-0.048611in}{0.000000in}}%
\pgfusepath{stroke,fill}%
}%
\begin{pgfscope}%
\pgfsys@transformshift{1.000000in}{0.930000in}%
\pgfsys@useobject{currentmarker}{}%
\end{pgfscope}%
\end{pgfscope}%
\begin{pgfscope}%
\definecolor{textcolor}{rgb}{0.000000,0.000000,0.000000}%
\pgfsetstrokecolor{textcolor}%
\pgfsetfillcolor{textcolor}%
\pgftext[x=0.560215in, y=0.829981in, left, base]{\color{textcolor}\sffamily\fontsize{20.000000}{24.000000}\selectfont \(\displaystyle {0.0}\)}%
\end{pgfscope}%
\begin{pgfscope}%
\pgfpathrectangle{\pgfqpoint{1.000000in}{0.720000in}}{\pgfqpoint{6.200000in}{4.620000in}}%
\pgfusepath{clip}%
\pgfsetrectcap%
\pgfsetroundjoin%
\pgfsetlinewidth{0.803000pt}%
\definecolor{currentstroke}{rgb}{0.690196,0.690196,0.690196}%
\pgfsetstrokecolor{currentstroke}%
\pgfsetdash{}{0pt}%
\pgfpathmoveto{\pgfqpoint{1.000000in}{2.018677in}}%
\pgfpathlineto{\pgfqpoint{7.200000in}{2.018677in}}%
\pgfusepath{stroke}%
\end{pgfscope}%
\begin{pgfscope}%
\pgfsetbuttcap%
\pgfsetroundjoin%
\definecolor{currentfill}{rgb}{0.000000,0.000000,0.000000}%
\pgfsetfillcolor{currentfill}%
\pgfsetlinewidth{0.803000pt}%
\definecolor{currentstroke}{rgb}{0.000000,0.000000,0.000000}%
\pgfsetstrokecolor{currentstroke}%
\pgfsetdash{}{0pt}%
\pgfsys@defobject{currentmarker}{\pgfqpoint{-0.048611in}{0.000000in}}{\pgfqpoint{-0.000000in}{0.000000in}}{%
\pgfpathmoveto{\pgfqpoint{-0.000000in}{0.000000in}}%
\pgfpathlineto{\pgfqpoint{-0.048611in}{0.000000in}}%
\pgfusepath{stroke,fill}%
}%
\begin{pgfscope}%
\pgfsys@transformshift{1.000000in}{2.018677in}%
\pgfsys@useobject{currentmarker}{}%
\end{pgfscope}%
\end{pgfscope}%
\begin{pgfscope}%
\definecolor{textcolor}{rgb}{0.000000,0.000000,0.000000}%
\pgfsetstrokecolor{textcolor}%
\pgfsetfillcolor{textcolor}%
\pgftext[x=0.560215in, y=1.918657in, left, base]{\color{textcolor}\sffamily\fontsize{20.000000}{24.000000}\selectfont \(\displaystyle {0.5}\)}%
\end{pgfscope}%
\begin{pgfscope}%
\pgfpathrectangle{\pgfqpoint{1.000000in}{0.720000in}}{\pgfqpoint{6.200000in}{4.620000in}}%
\pgfusepath{clip}%
\pgfsetrectcap%
\pgfsetroundjoin%
\pgfsetlinewidth{0.803000pt}%
\definecolor{currentstroke}{rgb}{0.690196,0.690196,0.690196}%
\pgfsetstrokecolor{currentstroke}%
\pgfsetdash{}{0pt}%
\pgfpathmoveto{\pgfqpoint{1.000000in}{3.107353in}}%
\pgfpathlineto{\pgfqpoint{7.200000in}{3.107353in}}%
\pgfusepath{stroke}%
\end{pgfscope}%
\begin{pgfscope}%
\pgfsetbuttcap%
\pgfsetroundjoin%
\definecolor{currentfill}{rgb}{0.000000,0.000000,0.000000}%
\pgfsetfillcolor{currentfill}%
\pgfsetlinewidth{0.803000pt}%
\definecolor{currentstroke}{rgb}{0.000000,0.000000,0.000000}%
\pgfsetstrokecolor{currentstroke}%
\pgfsetdash{}{0pt}%
\pgfsys@defobject{currentmarker}{\pgfqpoint{-0.048611in}{0.000000in}}{\pgfqpoint{-0.000000in}{0.000000in}}{%
\pgfpathmoveto{\pgfqpoint{-0.000000in}{0.000000in}}%
\pgfpathlineto{\pgfqpoint{-0.048611in}{0.000000in}}%
\pgfusepath{stroke,fill}%
}%
\begin{pgfscope}%
\pgfsys@transformshift{1.000000in}{3.107353in}%
\pgfsys@useobject{currentmarker}{}%
\end{pgfscope}%
\end{pgfscope}%
\begin{pgfscope}%
\definecolor{textcolor}{rgb}{0.000000,0.000000,0.000000}%
\pgfsetstrokecolor{textcolor}%
\pgfsetfillcolor{textcolor}%
\pgftext[x=0.560215in, y=3.007334in, left, base]{\color{textcolor}\sffamily\fontsize{20.000000}{24.000000}\selectfont \(\displaystyle {1.0}\)}%
\end{pgfscope}%
\begin{pgfscope}%
\pgfpathrectangle{\pgfqpoint{1.000000in}{0.720000in}}{\pgfqpoint{6.200000in}{4.620000in}}%
\pgfusepath{clip}%
\pgfsetrectcap%
\pgfsetroundjoin%
\pgfsetlinewidth{0.803000pt}%
\definecolor{currentstroke}{rgb}{0.690196,0.690196,0.690196}%
\pgfsetstrokecolor{currentstroke}%
\pgfsetdash{}{0pt}%
\pgfpathmoveto{\pgfqpoint{1.000000in}{4.196030in}}%
\pgfpathlineto{\pgfqpoint{7.200000in}{4.196030in}}%
\pgfusepath{stroke}%
\end{pgfscope}%
\begin{pgfscope}%
\pgfsetbuttcap%
\pgfsetroundjoin%
\definecolor{currentfill}{rgb}{0.000000,0.000000,0.000000}%
\pgfsetfillcolor{currentfill}%
\pgfsetlinewidth{0.803000pt}%
\definecolor{currentstroke}{rgb}{0.000000,0.000000,0.000000}%
\pgfsetstrokecolor{currentstroke}%
\pgfsetdash{}{0pt}%
\pgfsys@defobject{currentmarker}{\pgfqpoint{-0.048611in}{0.000000in}}{\pgfqpoint{-0.000000in}{0.000000in}}{%
\pgfpathmoveto{\pgfqpoint{-0.000000in}{0.000000in}}%
\pgfpathlineto{\pgfqpoint{-0.048611in}{0.000000in}}%
\pgfusepath{stroke,fill}%
}%
\begin{pgfscope}%
\pgfsys@transformshift{1.000000in}{4.196030in}%
\pgfsys@useobject{currentmarker}{}%
\end{pgfscope}%
\end{pgfscope}%
\begin{pgfscope}%
\definecolor{textcolor}{rgb}{0.000000,0.000000,0.000000}%
\pgfsetstrokecolor{textcolor}%
\pgfsetfillcolor{textcolor}%
\pgftext[x=0.560215in, y=4.096010in, left, base]{\color{textcolor}\sffamily\fontsize{20.000000}{24.000000}\selectfont \(\displaystyle {1.5}\)}%
\end{pgfscope}%
\begin{pgfscope}%
\pgfpathrectangle{\pgfqpoint{1.000000in}{0.720000in}}{\pgfqpoint{6.200000in}{4.620000in}}%
\pgfusepath{clip}%
\pgfsetrectcap%
\pgfsetroundjoin%
\pgfsetlinewidth{0.803000pt}%
\definecolor{currentstroke}{rgb}{0.690196,0.690196,0.690196}%
\pgfsetstrokecolor{currentstroke}%
\pgfsetdash{}{0pt}%
\pgfpathmoveto{\pgfqpoint{1.000000in}{5.284706in}}%
\pgfpathlineto{\pgfqpoint{7.200000in}{5.284706in}}%
\pgfusepath{stroke}%
\end{pgfscope}%
\begin{pgfscope}%
\pgfsetbuttcap%
\pgfsetroundjoin%
\definecolor{currentfill}{rgb}{0.000000,0.000000,0.000000}%
\pgfsetfillcolor{currentfill}%
\pgfsetlinewidth{0.803000pt}%
\definecolor{currentstroke}{rgb}{0.000000,0.000000,0.000000}%
\pgfsetstrokecolor{currentstroke}%
\pgfsetdash{}{0pt}%
\pgfsys@defobject{currentmarker}{\pgfqpoint{-0.048611in}{0.000000in}}{\pgfqpoint{-0.000000in}{0.000000in}}{%
\pgfpathmoveto{\pgfqpoint{-0.000000in}{0.000000in}}%
\pgfpathlineto{\pgfqpoint{-0.048611in}{0.000000in}}%
\pgfusepath{stroke,fill}%
}%
\begin{pgfscope}%
\pgfsys@transformshift{1.000000in}{5.284706in}%
\pgfsys@useobject{currentmarker}{}%
\end{pgfscope}%
\end{pgfscope}%
\begin{pgfscope}%
\definecolor{textcolor}{rgb}{0.000000,0.000000,0.000000}%
\pgfsetstrokecolor{textcolor}%
\pgfsetfillcolor{textcolor}%
\pgftext[x=0.560215in, y=5.184687in, left, base]{\color{textcolor}\sffamily\fontsize{20.000000}{24.000000}\selectfont \(\displaystyle {2.0}\)}%
\end{pgfscope}%
\begin{pgfscope}%
\definecolor{textcolor}{rgb}{0.000000,0.000000,0.000000}%
\pgfsetstrokecolor{textcolor}%
\pgfsetfillcolor{textcolor}%
\pgftext[x=0.504660in,y=3.030000in,,bottom,rotate=90.000000]{\color{textcolor}\sffamily\fontsize{20.000000}{24.000000}\selectfont \(\displaystyle \mathrm{Voltage}/\si{mV}\)}%
\end{pgfscope}%
\begin{pgfscope}%
\definecolor{textcolor}{rgb}{0.000000,0.000000,0.000000}%
\pgfsetstrokecolor{textcolor}%
\pgfsetfillcolor{textcolor}%
\pgftext[x=1.000000in,y=5.381667in,left,base]{\color{textcolor}\sffamily\fontsize{20.000000}{24.000000}\selectfont \(\displaystyle \times{10^{1}}{}\)}%
\end{pgfscope}%
\begin{pgfscope}%
\pgfpathrectangle{\pgfqpoint{1.000000in}{0.720000in}}{\pgfqpoint{6.200000in}{4.620000in}}%
\pgfusepath{clip}%
\pgfsetrectcap%
\pgfsetroundjoin%
\pgfsetlinewidth{2.007500pt}%
\definecolor{currentstroke}{rgb}{0.000000,0.000000,1.000000}%
\pgfsetstrokecolor{currentstroke}%
\pgfsetdash{}{0pt}%
\pgfpathmoveto{\pgfqpoint{0.990000in}{0.930000in}}%
\pgfpathlineto{\pgfqpoint{2.209000in}{0.930013in}}%
\pgfpathlineto{\pgfqpoint{2.240000in}{0.945661in}}%
\pgfpathlineto{\pgfqpoint{2.271000in}{1.109377in}}%
\pgfpathlineto{\pgfqpoint{2.302000in}{1.508242in}}%
\pgfpathlineto{\pgfqpoint{2.333000in}{2.008653in}}%
\pgfpathlineto{\pgfqpoint{2.364000in}{2.439417in}}%
\pgfpathlineto{\pgfqpoint{2.395000in}{2.713638in}}%
\pgfpathlineto{\pgfqpoint{2.426000in}{2.823902in}}%
\pgfpathlineto{\pgfqpoint{2.457000in}{2.802915in}}%
\pgfpathlineto{\pgfqpoint{2.488000in}{2.693533in}}%
\pgfpathlineto{\pgfqpoint{2.519000in}{2.533912in}}%
\pgfpathlineto{\pgfqpoint{2.612000in}{1.994808in}}%
\pgfpathlineto{\pgfqpoint{2.643000in}{1.835987in}}%
\pgfpathlineto{\pgfqpoint{2.674000in}{1.701927in}}%
\pgfpathlineto{\pgfqpoint{2.705000in}{1.733921in}}%
\pgfpathlineto{\pgfqpoint{2.736000in}{2.145244in}}%
\pgfpathlineto{\pgfqpoint{2.767000in}{2.813153in}}%
\pgfpathlineto{\pgfqpoint{2.798000in}{3.457052in}}%
\pgfpathlineto{\pgfqpoint{2.829000in}{3.896447in}}%
\pgfpathlineto{\pgfqpoint{2.860000in}{4.087179in}}%
\pgfpathlineto{\pgfqpoint{2.891000in}{4.065480in}}%
\pgfpathlineto{\pgfqpoint{2.922000in}{3.894769in}}%
\pgfpathlineto{\pgfqpoint{2.984000in}{3.378217in}}%
\pgfpathlineto{\pgfqpoint{3.015000in}{3.393636in}}%
\pgfpathlineto{\pgfqpoint{3.046000in}{3.797765in}}%
\pgfpathlineto{\pgfqpoint{3.077000in}{4.353095in}}%
\pgfpathlineto{\pgfqpoint{3.108000in}{4.795461in}}%
\pgfpathlineto{\pgfqpoint{3.139000in}{5.003406in}}%
\pgfpathlineto{\pgfqpoint{3.170000in}{5.015420in}}%
\pgfpathlineto{\pgfqpoint{3.201000in}{5.021313in}}%
\pgfpathlineto{\pgfqpoint{3.232000in}{5.088922in}}%
\pgfpathlineto{\pgfqpoint{3.263000in}{5.130000in}}%
\pgfpathlineto{\pgfqpoint{3.294000in}{5.062525in}}%
\pgfpathlineto{\pgfqpoint{3.325000in}{4.867706in}}%
\pgfpathlineto{\pgfqpoint{3.356000in}{4.570865in}}%
\pgfpathlineto{\pgfqpoint{3.387000in}{4.212400in}}%
\pgfpathlineto{\pgfqpoint{3.449000in}{3.452461in}}%
\pgfpathlineto{\pgfqpoint{3.480000in}{3.097751in}}%
\pgfpathlineto{\pgfqpoint{3.511000in}{2.776039in}}%
\pgfpathlineto{\pgfqpoint{3.542000in}{2.491408in}}%
\pgfpathlineto{\pgfqpoint{3.573000in}{2.244071in}}%
\pgfpathlineto{\pgfqpoint{3.604000in}{2.031947in}}%
\pgfpathlineto{\pgfqpoint{3.635000in}{1.851773in}}%
\pgfpathlineto{\pgfqpoint{3.666000in}{1.699818in}}%
\pgfpathlineto{\pgfqpoint{3.697000in}{1.572321in}}%
\pgfpathlineto{\pgfqpoint{3.728000in}{1.466970in}}%
\pgfpathlineto{\pgfqpoint{3.759000in}{1.433268in}}%
\pgfpathlineto{\pgfqpoint{3.790000in}{1.594301in}}%
\pgfpathlineto{\pgfqpoint{3.821000in}{1.918791in}}%
\pgfpathlineto{\pgfqpoint{3.852000in}{2.259535in}}%
\pgfpathlineto{\pgfqpoint{3.883000in}{2.506487in}}%
\pgfpathlineto{\pgfqpoint{3.914000in}{2.624135in}}%
\pgfpathlineto{\pgfqpoint{3.945000in}{2.625713in}}%
\pgfpathlineto{\pgfqpoint{3.976000in}{2.543218in}}%
\pgfpathlineto{\pgfqpoint{4.007000in}{2.409591in}}%
\pgfpathlineto{\pgfqpoint{4.038000in}{2.251384in}}%
\pgfpathlineto{\pgfqpoint{4.100000in}{1.929034in}}%
\pgfpathlineto{\pgfqpoint{4.131000in}{1.783305in}}%
\pgfpathlineto{\pgfqpoint{4.162000in}{1.653140in}}%
\pgfpathlineto{\pgfqpoint{4.193000in}{1.539335in}}%
\pgfpathlineto{\pgfqpoint{4.224000in}{1.441338in}}%
\pgfpathlineto{\pgfqpoint{4.255000in}{1.357877in}}%
\pgfpathlineto{\pgfqpoint{4.286000in}{1.287365in}}%
\pgfpathlineto{\pgfqpoint{4.317000in}{1.228138in}}%
\pgfpathlineto{\pgfqpoint{4.348000in}{1.178596in}}%
\pgfpathlineto{\pgfqpoint{4.379000in}{1.137276in}}%
\pgfpathlineto{\pgfqpoint{4.410000in}{1.102880in}}%
\pgfpathlineto{\pgfqpoint{4.441000in}{1.074281in}}%
\pgfpathlineto{\pgfqpoint{4.472000in}{1.050517in}}%
\pgfpathlineto{\pgfqpoint{4.503000in}{1.030772in}}%
\pgfpathlineto{\pgfqpoint{4.534000in}{1.014362in}}%
\pgfpathlineto{\pgfqpoint{4.565000in}{1.000718in}}%
\pgfpathlineto{\pgfqpoint{4.596000in}{0.989363in}}%
\pgfpathlineto{\pgfqpoint{4.627000in}{0.979904in}}%
\pgfpathlineto{\pgfqpoint{4.658000in}{0.972017in}}%
\pgfpathlineto{\pgfqpoint{4.720000in}{0.959925in}}%
\pgfpathlineto{\pgfqpoint{4.782000in}{0.951451in}}%
\pgfpathlineto{\pgfqpoint{4.844000in}{0.945475in}}%
\pgfpathlineto{\pgfqpoint{4.937000in}{0.939597in}}%
\pgfpathlineto{\pgfqpoint{5.030000in}{0.936037in}}%
\pgfpathlineto{\pgfqpoint{5.185000in}{0.932872in}}%
\pgfpathlineto{\pgfqpoint{5.402000in}{0.931076in}}%
\pgfpathlineto{\pgfqpoint{5.836000in}{0.930180in}}%
\pgfpathlineto{\pgfqpoint{7.210000in}{0.930002in}}%
\pgfpathlineto{\pgfqpoint{7.210000in}{0.930002in}}%
\pgfusepath{stroke}%
\end{pgfscope}%
\begin{pgfscope}%
\pgfsetrectcap%
\pgfsetmiterjoin%
\pgfsetlinewidth{0.803000pt}%
\definecolor{currentstroke}{rgb}{0.000000,0.000000,0.000000}%
\pgfsetstrokecolor{currentstroke}%
\pgfsetdash{}{0pt}%
\pgfpathmoveto{\pgfqpoint{1.000000in}{0.720000in}}%
\pgfpathlineto{\pgfqpoint{1.000000in}{5.340000in}}%
\pgfusepath{stroke}%
\end{pgfscope}%
\begin{pgfscope}%
\pgfsetrectcap%
\pgfsetmiterjoin%
\pgfsetlinewidth{0.803000pt}%
\definecolor{currentstroke}{rgb}{0.000000,0.000000,0.000000}%
\pgfsetstrokecolor{currentstroke}%
\pgfsetdash{}{0pt}%
\pgfpathmoveto{\pgfqpoint{7.200000in}{0.720000in}}%
\pgfpathlineto{\pgfqpoint{7.200000in}{5.340000in}}%
\pgfusepath{stroke}%
\end{pgfscope}%
\begin{pgfscope}%
\pgfsetrectcap%
\pgfsetmiterjoin%
\pgfsetlinewidth{0.803000pt}%
\definecolor{currentstroke}{rgb}{0.000000,0.000000,0.000000}%
\pgfsetstrokecolor{currentstroke}%
\pgfsetdash{}{0pt}%
\pgfpathmoveto{\pgfqpoint{1.000000in}{0.720000in}}%
\pgfpathlineto{\pgfqpoint{7.200000in}{0.720000in}}%
\pgfusepath{stroke}%
\end{pgfscope}%
\begin{pgfscope}%
\pgfsetrectcap%
\pgfsetmiterjoin%
\pgfsetlinewidth{0.803000pt}%
\definecolor{currentstroke}{rgb}{0.000000,0.000000,0.000000}%
\pgfsetstrokecolor{currentstroke}%
\pgfsetdash{}{0pt}%
\pgfpathmoveto{\pgfqpoint{1.000000in}{5.340000in}}%
\pgfpathlineto{\pgfqpoint{7.200000in}{5.340000in}}%
\pgfusepath{stroke}%
\end{pgfscope}%
\begin{pgfscope}%
\pgfsetroundcap%
\pgfsetroundjoin%
\definecolor{currentfill}{rgb}{0.000000,0.000000,0.000000}%
\pgfsetfillcolor{currentfill}%
\pgfsetlinewidth{1.003750pt}%
\definecolor{currentstroke}{rgb}{0.000000,0.000000,0.000000}%
\pgfsetstrokecolor{currentstroke}%
\pgfsetdash{}{0pt}%
\pgfpathmoveto{\pgfqpoint{2.550028in}{1.616222in}}%
\pgfpathquadraticcurveto{\pgfqpoint{1.393348in}{1.616222in}}{\pgfqpoint{0.236667in}{1.616222in}}%
\pgfpathlineto{\pgfqpoint{0.236667in}{1.532889in}}%
\pgfpathquadraticcurveto{\pgfqpoint{0.153319in}{1.588444in}}{\pgfqpoint{0.069970in}{1.644000in}}%
\pgfpathquadraticcurveto{\pgfqpoint{0.153319in}{1.699556in}}{\pgfqpoint{0.236667in}{1.755111in}}%
\pgfpathlineto{\pgfqpoint{0.236667in}{1.671778in}}%
\pgfpathquadraticcurveto{\pgfqpoint{1.393348in}{1.671778in}}{\pgfqpoint{2.550028in}{1.671778in}}%
\pgfpathlineto{\pgfqpoint{2.550028in}{1.616222in}}%
\pgfpathclose%
\pgfusepath{stroke,fill}%
\end{pgfscope}%
\begin{pgfscope}%
\pgfsetbuttcap%
\pgfsetmiterjoin%
\definecolor{currentfill}{rgb}{1.000000,1.000000,1.000000}%
\pgfsetfillcolor{currentfill}%
\pgfsetfillopacity{0.800000}%
\pgfsetlinewidth{1.003750pt}%
\definecolor{currentstroke}{rgb}{0.800000,0.800000,0.800000}%
\pgfsetstrokecolor{currentstroke}%
\pgfsetstrokeopacity{0.800000}%
\pgfsetdash{}{0pt}%
\pgfpathmoveto{\pgfqpoint{4.685759in}{4.690833in}}%
\pgfpathlineto{\pgfqpoint{7.005556in}{4.690833in}}%
\pgfpathquadraticcurveto{\pgfqpoint{7.061111in}{4.690833in}}{\pgfqpoint{7.061111in}{4.746389in}}%
\pgfpathlineto{\pgfqpoint{7.061111in}{5.145556in}}%
\pgfpathquadraticcurveto{\pgfqpoint{7.061111in}{5.201111in}}{\pgfqpoint{7.005556in}{5.201111in}}%
\pgfpathlineto{\pgfqpoint{4.685759in}{5.201111in}}%
\pgfpathquadraticcurveto{\pgfqpoint{4.630204in}{5.201111in}}{\pgfqpoint{4.630204in}{5.145556in}}%
\pgfpathlineto{\pgfqpoint{4.630204in}{4.746389in}}%
\pgfpathquadraticcurveto{\pgfqpoint{4.630204in}{4.690833in}}{\pgfqpoint{4.685759in}{4.690833in}}%
\pgfpathclose%
\pgfusepath{stroke,fill}%
\end{pgfscope}%
\begin{pgfscope}%
\pgfsetrectcap%
\pgfsetroundjoin%
\pgfsetlinewidth{2.007500pt}%
\definecolor{currentstroke}{rgb}{0.000000,0.000000,1.000000}%
\pgfsetstrokecolor{currentstroke}%
\pgfsetdash{}{0pt}%
\pgfpathmoveto{\pgfqpoint{4.741315in}{4.971181in}}%
\pgfpathlineto{\pgfqpoint{5.296870in}{4.971181in}}%
\pgfusepath{stroke}%
\end{pgfscope}%
\begin{pgfscope}%
\definecolor{textcolor}{rgb}{0.000000,0.000000,0.000000}%
\pgfsetstrokecolor{textcolor}%
\pgfsetfillcolor{textcolor}%
\pgftext[x=5.519092in,y=4.873958in,left,base]{\color{textcolor}\sffamily\fontsize{20.000000}{24.000000}\selectfont \(\displaystyle \vec{w}\mathrm{\ w/o\ noise}\)}%
\end{pgfscope}%
\end{pgfpicture}%
\makeatother%
\endgroup%
}
\end{figure}
\end{columns}
\end{frame}

\begin{frame}
\frametitle{Model selection}
\begin{itemize}
    \item Calculation of \textcolor{red}{$2^{N}$} model vectors is impossible!
    \item Most of $p(\vec{w}|\vec{z}) \rightarrow 0$!
\end{itemize}
\noindent\begin{minipage}[c]{0.33\textwidth}
    \begin{figure}[H]
        \centering
            \includegraphics[width=0.95\textwidth]{img/perfect_PE.pdf}
        \caption{perfect PE matching waveform, $p(\vec{z}|\vec{w})$ hit maximum}
        \label{fig:perfect PE}
    \end{figure}
\end{minipage}\begin{minipage}[c]{0.33\textwidth}
    \begin{figure}[H]
        \centering
            \includegraphics[width=0.95\textwidth]{img/not_so_perfect_PE.pdf}
        \caption{not so perfect, $p(\vec{z}|\vec{w})$ is smaller but still $>0$}
        \label{fig:not so perfect PE}
    \end{figure}
\end{minipage}\begin{minipage}[c]{0.33\textwidth}
    \begin{figure}[H]
        \centering
            \includegraphics[width=0.95\textwidth]{img/nonsense_PE.pdf}
        \caption{Completely mismatch the waveform, $p(\vec{z}|\vec{w}) \rightarrow 0$}
        \label{fig:nonsense PE}
    \end{figure}
\end{minipage}
\begin{align*}
    \mathcal{Z}' &\subseteq \mathcal{Z} \\
    p(\vec{z}|\vec{w}) &= \frac{p(\vec{w}|\vec{z})p(\vec{z})}{\sum_{\vec{z}'\in\mathcal{Z}}p(\vec{w}|\vec{z'})p(\vec{z'})} \approx \frac{p(\vec{w}|\vec{z})p(\vec{z})}{\sum_{\vec{z}'\in\mathcal{Z}'}p(\vec{w}|\vec{z'})p(\vec{z'})}
\end{align*}
\end{frame}

\begin{frame}
\frametitle{FBMP's result: Bayesian interface}
\begin{itemize}
    \item PE Time: $\vec{t}$
    \item Models: $\mathcal{Z}'=\{\vec{z}_j\}$
    \item Charge: \begin{align*}
        \hat{\vec{q}}_z = E(\vec{q}|\vec{w},\vec{z}) &= \vec{z} + \bm{Z}\bm{V}_\mathrm{PE}^\intercal\bm{\Sigma}_z^{-1}(\vec{w}-\bm{V}_\mathrm{PE}\vec{z})
        \end{align*}
    \item Model's posterior probability: $p(\vec{z}|\vec{w})$
\end{itemize}
\begin{center}
    Provides opportunity for subsequent Bayesian analysis! 
\end{center}
\end{frame}

\begin{frame}
\frametitle{FBMP demonstration}
\begin{figure}
    \centering
    \resizebox{0.6\textwidth}{!}{%% Creator: Matplotlib, PGF backend
%%
%% To include the figure in your LaTeX document, write
%%   \input{<filename>.pgf}
%%
%% Make sure the required packages are loaded in your preamble
%%   \usepackage{pgf}
%%
%% and, on pdftex
%%   \usepackage[utf8]{inputenc}\DeclareUnicodeCharacter{2212}{-}
%%
%% or, on luatex and xetex
%%   \usepackage{unicode-math}
%%
%% Figures using additional raster images can only be included by \input if
%% they are in the same directory as the main LaTeX file. For loading figures
%% from other directories you can use the `import` package
%%   \usepackage{import}
%%
%% and then include the figures with
%%   \import{<path to file>}{<filename>.pgf}
%%
%% Matplotlib used the following preamble
%%   \usepackage[detect-all,locale=DE]{siunitx}
%%
\begingroup%
\makeatletter%
\begin{pgfpicture}%
\pgfpathrectangle{\pgfpointorigin}{\pgfqpoint{10.000000in}{10.000000in}}%
\pgfusepath{use as bounding box, clip}%
\begin{pgfscope}%
\pgfsetbuttcap%
\pgfsetmiterjoin%
\definecolor{currentfill}{rgb}{1.000000,1.000000,1.000000}%
\pgfsetfillcolor{currentfill}%
\pgfsetlinewidth{0.000000pt}%
\definecolor{currentstroke}{rgb}{1.000000,1.000000,1.000000}%
\pgfsetstrokecolor{currentstroke}%
\pgfsetdash{}{0pt}%
\pgfpathmoveto{\pgfqpoint{0.000000in}{0.000000in}}%
\pgfpathlineto{\pgfqpoint{10.000000in}{0.000000in}}%
\pgfpathlineto{\pgfqpoint{10.000000in}{10.000000in}}%
\pgfpathlineto{\pgfqpoint{0.000000in}{10.000000in}}%
\pgfpathclose%
\pgfusepath{fill}%
\end{pgfscope}%
\begin{pgfscope}%
\pgfsetbuttcap%
\pgfsetmiterjoin%
\definecolor{currentfill}{rgb}{1.000000,1.000000,1.000000}%
\pgfsetfillcolor{currentfill}%
\pgfsetlinewidth{0.000000pt}%
\definecolor{currentstroke}{rgb}{0.000000,0.000000,0.000000}%
\pgfsetstrokecolor{currentstroke}%
\pgfsetstrokeopacity{0.000000}%
\pgfsetdash{}{0pt}%
\pgfpathmoveto{\pgfqpoint{1.000000in}{2.000000in}}%
\pgfpathlineto{\pgfqpoint{9.500000in}{2.000000in}}%
\pgfpathlineto{\pgfqpoint{9.500000in}{5.000000in}}%
\pgfpathlineto{\pgfqpoint{1.000000in}{5.000000in}}%
\pgfpathclose%
\pgfusepath{fill}%
\end{pgfscope}%
\begin{pgfscope}%
\pgfpathrectangle{\pgfqpoint{1.000000in}{2.000000in}}{\pgfqpoint{8.500000in}{3.000000in}}%
\pgfusepath{clip}%
\pgfsetrectcap%
\pgfsetroundjoin%
\pgfsetlinewidth{0.803000pt}%
\definecolor{currentstroke}{rgb}{0.690196,0.690196,0.690196}%
\pgfsetstrokecolor{currentstroke}%
\pgfsetdash{}{0pt}%
\pgfpathmoveto{\pgfqpoint{2.725210in}{2.000000in}}%
\pgfpathlineto{\pgfqpoint{2.725210in}{5.000000in}}%
\pgfusepath{stroke}%
\end{pgfscope}%
\begin{pgfscope}%
\pgfsetbuttcap%
\pgfsetroundjoin%
\definecolor{currentfill}{rgb}{0.000000,0.000000,0.000000}%
\pgfsetfillcolor{currentfill}%
\pgfsetlinewidth{0.803000pt}%
\definecolor{currentstroke}{rgb}{0.000000,0.000000,0.000000}%
\pgfsetstrokecolor{currentstroke}%
\pgfsetdash{}{0pt}%
\pgfsys@defobject{currentmarker}{\pgfqpoint{0.000000in}{-0.048611in}}{\pgfqpoint{0.000000in}{0.000000in}}{%
\pgfpathmoveto{\pgfqpoint{0.000000in}{0.000000in}}%
\pgfpathlineto{\pgfqpoint{0.000000in}{-0.048611in}}%
\pgfusepath{stroke,fill}%
}%
\begin{pgfscope}%
\pgfsys@transformshift{2.725210in}{2.000000in}%
\pgfsys@useobject{currentmarker}{}%
\end{pgfscope}%
\end{pgfscope}%
\begin{pgfscope}%
\pgfpathrectangle{\pgfqpoint{1.000000in}{2.000000in}}{\pgfqpoint{8.500000in}{3.000000in}}%
\pgfusepath{clip}%
\pgfsetrectcap%
\pgfsetroundjoin%
\pgfsetlinewidth{0.803000pt}%
\definecolor{currentstroke}{rgb}{0.690196,0.690196,0.690196}%
\pgfsetstrokecolor{currentstroke}%
\pgfsetdash{}{0pt}%
\pgfpathmoveto{\pgfqpoint{4.506292in}{2.000000in}}%
\pgfpathlineto{\pgfqpoint{4.506292in}{5.000000in}}%
\pgfusepath{stroke}%
\end{pgfscope}%
\begin{pgfscope}%
\pgfsetbuttcap%
\pgfsetroundjoin%
\definecolor{currentfill}{rgb}{0.000000,0.000000,0.000000}%
\pgfsetfillcolor{currentfill}%
\pgfsetlinewidth{0.803000pt}%
\definecolor{currentstroke}{rgb}{0.000000,0.000000,0.000000}%
\pgfsetstrokecolor{currentstroke}%
\pgfsetdash{}{0pt}%
\pgfsys@defobject{currentmarker}{\pgfqpoint{0.000000in}{-0.048611in}}{\pgfqpoint{0.000000in}{0.000000in}}{%
\pgfpathmoveto{\pgfqpoint{0.000000in}{0.000000in}}%
\pgfpathlineto{\pgfqpoint{0.000000in}{-0.048611in}}%
\pgfusepath{stroke,fill}%
}%
\begin{pgfscope}%
\pgfsys@transformshift{4.506292in}{2.000000in}%
\pgfsys@useobject{currentmarker}{}%
\end{pgfscope}%
\end{pgfscope}%
\begin{pgfscope}%
\pgfpathrectangle{\pgfqpoint{1.000000in}{2.000000in}}{\pgfqpoint{8.500000in}{3.000000in}}%
\pgfusepath{clip}%
\pgfsetrectcap%
\pgfsetroundjoin%
\pgfsetlinewidth{0.803000pt}%
\definecolor{currentstroke}{rgb}{0.690196,0.690196,0.690196}%
\pgfsetstrokecolor{currentstroke}%
\pgfsetdash{}{0pt}%
\pgfpathmoveto{\pgfqpoint{6.287373in}{2.000000in}}%
\pgfpathlineto{\pgfqpoint{6.287373in}{5.000000in}}%
\pgfusepath{stroke}%
\end{pgfscope}%
\begin{pgfscope}%
\pgfsetbuttcap%
\pgfsetroundjoin%
\definecolor{currentfill}{rgb}{0.000000,0.000000,0.000000}%
\pgfsetfillcolor{currentfill}%
\pgfsetlinewidth{0.803000pt}%
\definecolor{currentstroke}{rgb}{0.000000,0.000000,0.000000}%
\pgfsetstrokecolor{currentstroke}%
\pgfsetdash{}{0pt}%
\pgfsys@defobject{currentmarker}{\pgfqpoint{0.000000in}{-0.048611in}}{\pgfqpoint{0.000000in}{0.000000in}}{%
\pgfpathmoveto{\pgfqpoint{0.000000in}{0.000000in}}%
\pgfpathlineto{\pgfqpoint{0.000000in}{-0.048611in}}%
\pgfusepath{stroke,fill}%
}%
\begin{pgfscope}%
\pgfsys@transformshift{6.287373in}{2.000000in}%
\pgfsys@useobject{currentmarker}{}%
\end{pgfscope}%
\end{pgfscope}%
\begin{pgfscope}%
\pgfpathrectangle{\pgfqpoint{1.000000in}{2.000000in}}{\pgfqpoint{8.500000in}{3.000000in}}%
\pgfusepath{clip}%
\pgfsetrectcap%
\pgfsetroundjoin%
\pgfsetlinewidth{0.803000pt}%
\definecolor{currentstroke}{rgb}{0.690196,0.690196,0.690196}%
\pgfsetstrokecolor{currentstroke}%
\pgfsetdash{}{0pt}%
\pgfpathmoveto{\pgfqpoint{8.068455in}{2.000000in}}%
\pgfpathlineto{\pgfqpoint{8.068455in}{5.000000in}}%
\pgfusepath{stroke}%
\end{pgfscope}%
\begin{pgfscope}%
\pgfsetbuttcap%
\pgfsetroundjoin%
\definecolor{currentfill}{rgb}{0.000000,0.000000,0.000000}%
\pgfsetfillcolor{currentfill}%
\pgfsetlinewidth{0.803000pt}%
\definecolor{currentstroke}{rgb}{0.000000,0.000000,0.000000}%
\pgfsetstrokecolor{currentstroke}%
\pgfsetdash{}{0pt}%
\pgfsys@defobject{currentmarker}{\pgfqpoint{0.000000in}{-0.048611in}}{\pgfqpoint{0.000000in}{0.000000in}}{%
\pgfpathmoveto{\pgfqpoint{0.000000in}{0.000000in}}%
\pgfpathlineto{\pgfqpoint{0.000000in}{-0.048611in}}%
\pgfusepath{stroke,fill}%
}%
\begin{pgfscope}%
\pgfsys@transformshift{8.068455in}{2.000000in}%
\pgfsys@useobject{currentmarker}{}%
\end{pgfscope}%
\end{pgfscope}%
\begin{pgfscope}%
\pgfpathrectangle{\pgfqpoint{1.000000in}{2.000000in}}{\pgfqpoint{8.500000in}{3.000000in}}%
\pgfusepath{clip}%
\pgfsetrectcap%
\pgfsetroundjoin%
\pgfsetlinewidth{0.803000pt}%
\definecolor{currentstroke}{rgb}{0.690196,0.690196,0.690196}%
\pgfsetstrokecolor{currentstroke}%
\pgfsetdash{}{0pt}%
\pgfpathmoveto{\pgfqpoint{1.000000in}{2.545455in}}%
\pgfpathlineto{\pgfqpoint{9.500000in}{2.545455in}}%
\pgfusepath{stroke}%
\end{pgfscope}%
\begin{pgfscope}%
\pgfsetbuttcap%
\pgfsetroundjoin%
\definecolor{currentfill}{rgb}{0.000000,0.000000,0.000000}%
\pgfsetfillcolor{currentfill}%
\pgfsetlinewidth{0.803000pt}%
\definecolor{currentstroke}{rgb}{0.000000,0.000000,0.000000}%
\pgfsetstrokecolor{currentstroke}%
\pgfsetdash{}{0pt}%
\pgfsys@defobject{currentmarker}{\pgfqpoint{-0.048611in}{0.000000in}}{\pgfqpoint{-0.000000in}{0.000000in}}{%
\pgfpathmoveto{\pgfqpoint{-0.000000in}{0.000000in}}%
\pgfpathlineto{\pgfqpoint{-0.048611in}{0.000000in}}%
\pgfusepath{stroke,fill}%
}%
\begin{pgfscope}%
\pgfsys@transformshift{1.000000in}{2.545455in}%
\pgfsys@useobject{currentmarker}{}%
\end{pgfscope}%
\end{pgfscope}%
\begin{pgfscope}%
\definecolor{textcolor}{rgb}{0.000000,0.000000,0.000000}%
\pgfsetstrokecolor{textcolor}%
\pgfsetfillcolor{textcolor}%
\pgftext[x=0.770670in, y=2.445435in, left, base]{\color{textcolor}\sffamily\fontsize{20.000000}{24.000000}\selectfont \(\displaystyle {0}\)}%
\end{pgfscope}%
\begin{pgfscope}%
\pgfpathrectangle{\pgfqpoint{1.000000in}{2.000000in}}{\pgfqpoint{8.500000in}{3.000000in}}%
\pgfusepath{clip}%
\pgfsetrectcap%
\pgfsetroundjoin%
\pgfsetlinewidth{0.803000pt}%
\definecolor{currentstroke}{rgb}{0.690196,0.690196,0.690196}%
\pgfsetstrokecolor{currentstroke}%
\pgfsetdash{}{0pt}%
\pgfpathmoveto{\pgfqpoint{1.000000in}{3.227273in}}%
\pgfpathlineto{\pgfqpoint{9.500000in}{3.227273in}}%
\pgfusepath{stroke}%
\end{pgfscope}%
\begin{pgfscope}%
\pgfsetbuttcap%
\pgfsetroundjoin%
\definecolor{currentfill}{rgb}{0.000000,0.000000,0.000000}%
\pgfsetfillcolor{currentfill}%
\pgfsetlinewidth{0.803000pt}%
\definecolor{currentstroke}{rgb}{0.000000,0.000000,0.000000}%
\pgfsetstrokecolor{currentstroke}%
\pgfsetdash{}{0pt}%
\pgfsys@defobject{currentmarker}{\pgfqpoint{-0.048611in}{0.000000in}}{\pgfqpoint{-0.000000in}{0.000000in}}{%
\pgfpathmoveto{\pgfqpoint{-0.000000in}{0.000000in}}%
\pgfpathlineto{\pgfqpoint{-0.048611in}{0.000000in}}%
\pgfusepath{stroke,fill}%
}%
\begin{pgfscope}%
\pgfsys@transformshift{1.000000in}{3.227273in}%
\pgfsys@useobject{currentmarker}{}%
\end{pgfscope}%
\end{pgfscope}%
\begin{pgfscope}%
\definecolor{textcolor}{rgb}{0.000000,0.000000,0.000000}%
\pgfsetstrokecolor{textcolor}%
\pgfsetfillcolor{textcolor}%
\pgftext[x=0.638563in, y=3.127253in, left, base]{\color{textcolor}\sffamily\fontsize{20.000000}{24.000000}\selectfont \(\displaystyle {10}\)}%
\end{pgfscope}%
\begin{pgfscope}%
\pgfpathrectangle{\pgfqpoint{1.000000in}{2.000000in}}{\pgfqpoint{8.500000in}{3.000000in}}%
\pgfusepath{clip}%
\pgfsetrectcap%
\pgfsetroundjoin%
\pgfsetlinewidth{0.803000pt}%
\definecolor{currentstroke}{rgb}{0.690196,0.690196,0.690196}%
\pgfsetstrokecolor{currentstroke}%
\pgfsetdash{}{0pt}%
\pgfpathmoveto{\pgfqpoint{1.000000in}{3.909091in}}%
\pgfpathlineto{\pgfqpoint{9.500000in}{3.909091in}}%
\pgfusepath{stroke}%
\end{pgfscope}%
\begin{pgfscope}%
\pgfsetbuttcap%
\pgfsetroundjoin%
\definecolor{currentfill}{rgb}{0.000000,0.000000,0.000000}%
\pgfsetfillcolor{currentfill}%
\pgfsetlinewidth{0.803000pt}%
\definecolor{currentstroke}{rgb}{0.000000,0.000000,0.000000}%
\pgfsetstrokecolor{currentstroke}%
\pgfsetdash{}{0pt}%
\pgfsys@defobject{currentmarker}{\pgfqpoint{-0.048611in}{0.000000in}}{\pgfqpoint{-0.000000in}{0.000000in}}{%
\pgfpathmoveto{\pgfqpoint{-0.000000in}{0.000000in}}%
\pgfpathlineto{\pgfqpoint{-0.048611in}{0.000000in}}%
\pgfusepath{stroke,fill}%
}%
\begin{pgfscope}%
\pgfsys@transformshift{1.000000in}{3.909091in}%
\pgfsys@useobject{currentmarker}{}%
\end{pgfscope}%
\end{pgfscope}%
\begin{pgfscope}%
\definecolor{textcolor}{rgb}{0.000000,0.000000,0.000000}%
\pgfsetstrokecolor{textcolor}%
\pgfsetfillcolor{textcolor}%
\pgftext[x=0.638563in, y=3.809072in, left, base]{\color{textcolor}\sffamily\fontsize{20.000000}{24.000000}\selectfont \(\displaystyle {20}\)}%
\end{pgfscope}%
\begin{pgfscope}%
\pgfpathrectangle{\pgfqpoint{1.000000in}{2.000000in}}{\pgfqpoint{8.500000in}{3.000000in}}%
\pgfusepath{clip}%
\pgfsetrectcap%
\pgfsetroundjoin%
\pgfsetlinewidth{0.803000pt}%
\definecolor{currentstroke}{rgb}{0.690196,0.690196,0.690196}%
\pgfsetstrokecolor{currentstroke}%
\pgfsetdash{}{0pt}%
\pgfpathmoveto{\pgfqpoint{1.000000in}{4.590909in}}%
\pgfpathlineto{\pgfqpoint{9.500000in}{4.590909in}}%
\pgfusepath{stroke}%
\end{pgfscope}%
\begin{pgfscope}%
\pgfsetbuttcap%
\pgfsetroundjoin%
\definecolor{currentfill}{rgb}{0.000000,0.000000,0.000000}%
\pgfsetfillcolor{currentfill}%
\pgfsetlinewidth{0.803000pt}%
\definecolor{currentstroke}{rgb}{0.000000,0.000000,0.000000}%
\pgfsetstrokecolor{currentstroke}%
\pgfsetdash{}{0pt}%
\pgfsys@defobject{currentmarker}{\pgfqpoint{-0.048611in}{0.000000in}}{\pgfqpoint{-0.000000in}{0.000000in}}{%
\pgfpathmoveto{\pgfqpoint{-0.000000in}{0.000000in}}%
\pgfpathlineto{\pgfqpoint{-0.048611in}{0.000000in}}%
\pgfusepath{stroke,fill}%
}%
\begin{pgfscope}%
\pgfsys@transformshift{1.000000in}{4.590909in}%
\pgfsys@useobject{currentmarker}{}%
\end{pgfscope}%
\end{pgfscope}%
\begin{pgfscope}%
\definecolor{textcolor}{rgb}{0.000000,0.000000,0.000000}%
\pgfsetstrokecolor{textcolor}%
\pgfsetfillcolor{textcolor}%
\pgftext[x=0.638563in, y=4.490890in, left, base]{\color{textcolor}\sffamily\fontsize{20.000000}{24.000000}\selectfont \(\displaystyle {30}\)}%
\end{pgfscope}%
\begin{pgfscope}%
\definecolor{textcolor}{rgb}{0.000000,0.000000,0.000000}%
\pgfsetstrokecolor{textcolor}%
\pgfsetfillcolor{textcolor}%
\pgftext[x=0.583007in,y=3.500000in,,bottom,rotate=90.000000]{\color{textcolor}\sffamily\fontsize{20.000000}{24.000000}\selectfont \(\displaystyle \mathrm{Voltage}/\si{mV}\)}%
\end{pgfscope}%
\begin{pgfscope}%
\pgfpathrectangle{\pgfqpoint{1.000000in}{2.000000in}}{\pgfqpoint{8.500000in}{3.000000in}}%
\pgfusepath{clip}%
\pgfsetbuttcap%
\pgfsetroundjoin%
\pgfsetlinewidth{2.007500pt}%
\definecolor{currentstroke}{rgb}{0.000000,0.750000,0.750000}%
\pgfsetstrokecolor{currentstroke}%
\pgfsetdash{}{0pt}%
\pgfpathmoveto{\pgfqpoint{0.990000in}{2.886364in}}%
\pgfpathlineto{\pgfqpoint{9.510000in}{2.886364in}}%
\pgfusepath{stroke}%
\end{pgfscope}%
\begin{pgfscope}%
\pgfpathrectangle{\pgfqpoint{1.000000in}{2.000000in}}{\pgfqpoint{8.500000in}{3.000000in}}%
\pgfusepath{clip}%
\pgfsetrectcap%
\pgfsetroundjoin%
\pgfsetlinewidth{2.007500pt}%
\definecolor{currentstroke}{rgb}{0.000000,0.000000,1.000000}%
\pgfsetstrokecolor{currentstroke}%
\pgfsetdash{}{0pt}%
\pgfpathmoveto{\pgfqpoint{0.990000in}{2.487563in}}%
\pgfpathlineto{\pgfqpoint{1.015372in}{2.681818in}}%
\pgfpathlineto{\pgfqpoint{1.050994in}{2.477273in}}%
\pgfpathlineto{\pgfqpoint{1.122237in}{2.613636in}}%
\pgfpathlineto{\pgfqpoint{1.157859in}{2.477273in}}%
\pgfpathlineto{\pgfqpoint{1.193480in}{2.545455in}}%
\pgfpathlineto{\pgfqpoint{1.264724in}{2.545455in}}%
\pgfpathlineto{\pgfqpoint{1.300345in}{2.613636in}}%
\pgfpathlineto{\pgfqpoint{1.371588in}{2.613636in}}%
\pgfpathlineto{\pgfqpoint{1.478453in}{2.409091in}}%
\pgfpathlineto{\pgfqpoint{1.514075in}{2.613636in}}%
\pgfpathlineto{\pgfqpoint{1.549697in}{2.613636in}}%
\pgfpathlineto{\pgfqpoint{1.585318in}{2.545455in}}%
\pgfpathlineto{\pgfqpoint{1.656562in}{2.545455in}}%
\pgfpathlineto{\pgfqpoint{1.692183in}{2.477273in}}%
\pgfpathlineto{\pgfqpoint{1.727805in}{2.613636in}}%
\pgfpathlineto{\pgfqpoint{1.799048in}{2.613636in}}%
\pgfpathlineto{\pgfqpoint{1.834670in}{2.545455in}}%
\pgfpathlineto{\pgfqpoint{1.905913in}{2.545455in}}%
\pgfpathlineto{\pgfqpoint{1.941535in}{2.613636in}}%
\pgfpathlineto{\pgfqpoint{1.977156in}{2.477273in}}%
\pgfpathlineto{\pgfqpoint{2.048399in}{2.477273in}}%
\pgfpathlineto{\pgfqpoint{2.084021in}{2.613636in}}%
\pgfpathlineto{\pgfqpoint{2.119643in}{2.477273in}}%
\pgfpathlineto{\pgfqpoint{2.155264in}{2.477273in}}%
\pgfpathlineto{\pgfqpoint{2.190886in}{2.613636in}}%
\pgfpathlineto{\pgfqpoint{2.226508in}{2.545455in}}%
\pgfpathlineto{\pgfqpoint{2.333372in}{2.545455in}}%
\pgfpathlineto{\pgfqpoint{2.368994in}{2.477273in}}%
\pgfpathlineto{\pgfqpoint{2.404616in}{2.613636in}}%
\pgfpathlineto{\pgfqpoint{2.440237in}{2.477273in}}%
\pgfpathlineto{\pgfqpoint{2.475859in}{2.477273in}}%
\pgfpathlineto{\pgfqpoint{2.511481in}{2.545455in}}%
\pgfpathlineto{\pgfqpoint{2.582724in}{2.545455in}}%
\pgfpathlineto{\pgfqpoint{2.618345in}{2.477273in}}%
\pgfpathlineto{\pgfqpoint{2.653967in}{2.613636in}}%
\pgfpathlineto{\pgfqpoint{2.689589in}{2.545455in}}%
\pgfpathlineto{\pgfqpoint{2.725210in}{2.613636in}}%
\pgfpathlineto{\pgfqpoint{2.760832in}{2.545455in}}%
\pgfpathlineto{\pgfqpoint{2.796454in}{2.613636in}}%
\pgfpathlineto{\pgfqpoint{2.867697in}{2.613636in}}%
\pgfpathlineto{\pgfqpoint{2.903318in}{2.886364in}}%
\pgfpathlineto{\pgfqpoint{2.938940in}{3.227273in}}%
\pgfpathlineto{\pgfqpoint{2.974562in}{3.295455in}}%
\pgfpathlineto{\pgfqpoint{3.010183in}{3.636364in}}%
\pgfpathlineto{\pgfqpoint{3.045805in}{3.636364in}}%
\pgfpathlineto{\pgfqpoint{3.081427in}{3.568182in}}%
\pgfpathlineto{\pgfqpoint{3.117048in}{3.568182in}}%
\pgfpathlineto{\pgfqpoint{3.152670in}{3.500000in}}%
\pgfpathlineto{\pgfqpoint{3.188292in}{3.840909in}}%
\pgfpathlineto{\pgfqpoint{3.223913in}{3.840909in}}%
\pgfpathlineto{\pgfqpoint{3.259535in}{4.318182in}}%
\pgfpathlineto{\pgfqpoint{3.295156in}{4.386364in}}%
\pgfpathlineto{\pgfqpoint{3.330778in}{4.386364in}}%
\pgfpathlineto{\pgfqpoint{3.366400in}{4.454545in}}%
\pgfpathlineto{\pgfqpoint{3.402021in}{4.250000in}}%
\pgfpathlineto{\pgfqpoint{3.437643in}{3.977273in}}%
\pgfpathlineto{\pgfqpoint{3.473265in}{3.909091in}}%
\pgfpathlineto{\pgfqpoint{3.508886in}{3.772727in}}%
\pgfpathlineto{\pgfqpoint{3.544508in}{4.045455in}}%
\pgfpathlineto{\pgfqpoint{3.580129in}{4.250000in}}%
\pgfpathlineto{\pgfqpoint{3.615751in}{4.113636in}}%
\pgfpathlineto{\pgfqpoint{3.651373in}{4.181818in}}%
\pgfpathlineto{\pgfqpoint{3.722616in}{4.590909in}}%
\pgfpathlineto{\pgfqpoint{3.758238in}{4.659091in}}%
\pgfpathlineto{\pgfqpoint{3.793859in}{4.659091in}}%
\pgfpathlineto{\pgfqpoint{3.829481in}{4.454545in}}%
\pgfpathlineto{\pgfqpoint{3.865102in}{4.386364in}}%
\pgfpathlineto{\pgfqpoint{3.900724in}{4.250000in}}%
\pgfpathlineto{\pgfqpoint{3.936346in}{3.977273in}}%
\pgfpathlineto{\pgfqpoint{3.971967in}{3.840909in}}%
\pgfpathlineto{\pgfqpoint{4.007589in}{3.772727in}}%
\pgfpathlineto{\pgfqpoint{4.043211in}{3.568182in}}%
\pgfpathlineto{\pgfqpoint{4.078832in}{3.295455in}}%
\pgfpathlineto{\pgfqpoint{4.150075in}{3.022727in}}%
\pgfpathlineto{\pgfqpoint{4.185697in}{2.954545in}}%
\pgfpathlineto{\pgfqpoint{4.256940in}{2.954545in}}%
\pgfpathlineto{\pgfqpoint{4.292562in}{3.159091in}}%
\pgfpathlineto{\pgfqpoint{4.328184in}{3.500000in}}%
\pgfpathlineto{\pgfqpoint{4.363805in}{3.772727in}}%
\pgfpathlineto{\pgfqpoint{4.399427in}{3.772727in}}%
\pgfpathlineto{\pgfqpoint{4.435048in}{4.045455in}}%
\pgfpathlineto{\pgfqpoint{4.470670in}{3.909091in}}%
\pgfpathlineto{\pgfqpoint{4.506292in}{3.704545in}}%
\pgfpathlineto{\pgfqpoint{4.541913in}{3.568182in}}%
\pgfpathlineto{\pgfqpoint{4.577535in}{3.500000in}}%
\pgfpathlineto{\pgfqpoint{4.648778in}{3.227273in}}%
\pgfpathlineto{\pgfqpoint{4.684400in}{3.227273in}}%
\pgfpathlineto{\pgfqpoint{4.720021in}{2.954545in}}%
\pgfpathlineto{\pgfqpoint{4.755643in}{2.954545in}}%
\pgfpathlineto{\pgfqpoint{4.826886in}{2.818182in}}%
\pgfpathlineto{\pgfqpoint{4.862508in}{2.818182in}}%
\pgfpathlineto{\pgfqpoint{4.898130in}{2.750000in}}%
\pgfpathlineto{\pgfqpoint{4.969373in}{2.750000in}}%
\pgfpathlineto{\pgfqpoint{5.004995in}{2.477273in}}%
\pgfpathlineto{\pgfqpoint{5.040616in}{2.613636in}}%
\pgfpathlineto{\pgfqpoint{5.076238in}{2.613636in}}%
\pgfpathlineto{\pgfqpoint{5.111859in}{2.545455in}}%
\pgfpathlineto{\pgfqpoint{5.147481in}{2.613636in}}%
\pgfpathlineto{\pgfqpoint{5.183103in}{2.613636in}}%
\pgfpathlineto{\pgfqpoint{5.218724in}{2.545455in}}%
\pgfpathlineto{\pgfqpoint{5.254346in}{2.613636in}}%
\pgfpathlineto{\pgfqpoint{5.289968in}{2.545455in}}%
\pgfpathlineto{\pgfqpoint{5.325589in}{2.681818in}}%
\pgfpathlineto{\pgfqpoint{5.361211in}{2.613636in}}%
\pgfpathlineto{\pgfqpoint{5.396832in}{2.681818in}}%
\pgfpathlineto{\pgfqpoint{5.432454in}{2.545455in}}%
\pgfpathlineto{\pgfqpoint{5.468076in}{2.613636in}}%
\pgfpathlineto{\pgfqpoint{5.503697in}{2.477273in}}%
\pgfpathlineto{\pgfqpoint{5.539319in}{2.613636in}}%
\pgfpathlineto{\pgfqpoint{5.574941in}{2.477273in}}%
\pgfpathlineto{\pgfqpoint{5.610562in}{2.477273in}}%
\pgfpathlineto{\pgfqpoint{5.646184in}{2.409091in}}%
\pgfpathlineto{\pgfqpoint{5.681805in}{2.681818in}}%
\pgfpathlineto{\pgfqpoint{5.717427in}{2.613636in}}%
\pgfpathlineto{\pgfqpoint{5.753049in}{2.477273in}}%
\pgfpathlineto{\pgfqpoint{5.788670in}{2.545455in}}%
\pgfpathlineto{\pgfqpoint{5.824292in}{2.681818in}}%
\pgfpathlineto{\pgfqpoint{5.895535in}{2.409091in}}%
\pgfpathlineto{\pgfqpoint{5.931157in}{2.545455in}}%
\pgfpathlineto{\pgfqpoint{5.966778in}{2.613636in}}%
\pgfpathlineto{\pgfqpoint{6.002400in}{2.477273in}}%
\pgfpathlineto{\pgfqpoint{6.038022in}{2.613636in}}%
\pgfpathlineto{\pgfqpoint{6.073643in}{2.409091in}}%
\pgfpathlineto{\pgfqpoint{6.109265in}{2.545455in}}%
\pgfpathlineto{\pgfqpoint{6.144887in}{2.545455in}}%
\pgfpathlineto{\pgfqpoint{6.180508in}{2.613636in}}%
\pgfpathlineto{\pgfqpoint{6.216130in}{2.477273in}}%
\pgfpathlineto{\pgfqpoint{6.251751in}{2.545455in}}%
\pgfpathlineto{\pgfqpoint{6.287373in}{2.545455in}}%
\pgfpathlineto{\pgfqpoint{6.322995in}{2.613636in}}%
\pgfpathlineto{\pgfqpoint{6.358616in}{2.545455in}}%
\pgfpathlineto{\pgfqpoint{6.394238in}{2.545455in}}%
\pgfpathlineto{\pgfqpoint{6.429860in}{2.477273in}}%
\pgfpathlineto{\pgfqpoint{6.465481in}{2.545455in}}%
\pgfpathlineto{\pgfqpoint{6.501103in}{2.409091in}}%
\pgfpathlineto{\pgfqpoint{6.536725in}{2.545455in}}%
\pgfpathlineto{\pgfqpoint{6.572346in}{2.545455in}}%
\pgfpathlineto{\pgfqpoint{6.607968in}{2.613636in}}%
\pgfpathlineto{\pgfqpoint{6.643589in}{2.409091in}}%
\pgfpathlineto{\pgfqpoint{6.679211in}{2.681818in}}%
\pgfpathlineto{\pgfqpoint{6.714833in}{2.477273in}}%
\pgfpathlineto{\pgfqpoint{6.750454in}{2.545455in}}%
\pgfpathlineto{\pgfqpoint{6.786076in}{2.681818in}}%
\pgfpathlineto{\pgfqpoint{6.821698in}{2.545455in}}%
\pgfpathlineto{\pgfqpoint{6.857319in}{2.613636in}}%
\pgfpathlineto{\pgfqpoint{6.892941in}{2.613636in}}%
\pgfpathlineto{\pgfqpoint{6.928562in}{2.477273in}}%
\pgfpathlineto{\pgfqpoint{6.964184in}{2.545455in}}%
\pgfpathlineto{\pgfqpoint{6.999806in}{2.545455in}}%
\pgfpathlineto{\pgfqpoint{7.035427in}{2.613636in}}%
\pgfpathlineto{\pgfqpoint{7.106671in}{2.477273in}}%
\pgfpathlineto{\pgfqpoint{7.142292in}{2.613636in}}%
\pgfpathlineto{\pgfqpoint{7.177914in}{2.613636in}}%
\pgfpathlineto{\pgfqpoint{7.213535in}{2.681818in}}%
\pgfpathlineto{\pgfqpoint{7.249157in}{2.477273in}}%
\pgfpathlineto{\pgfqpoint{7.284779in}{2.545455in}}%
\pgfpathlineto{\pgfqpoint{7.320400in}{2.409091in}}%
\pgfpathlineto{\pgfqpoint{7.356022in}{2.613636in}}%
\pgfpathlineto{\pgfqpoint{7.391644in}{2.613636in}}%
\pgfpathlineto{\pgfqpoint{7.462887in}{2.477273in}}%
\pgfpathlineto{\pgfqpoint{7.498508in}{2.477273in}}%
\pgfpathlineto{\pgfqpoint{7.534130in}{2.613636in}}%
\pgfpathlineto{\pgfqpoint{7.569752in}{2.477273in}}%
\pgfpathlineto{\pgfqpoint{7.605373in}{2.545455in}}%
\pgfpathlineto{\pgfqpoint{7.640995in}{2.477273in}}%
\pgfpathlineto{\pgfqpoint{7.712238in}{2.613636in}}%
\pgfpathlineto{\pgfqpoint{7.747860in}{2.545455in}}%
\pgfpathlineto{\pgfqpoint{7.783481in}{2.409091in}}%
\pgfpathlineto{\pgfqpoint{7.854725in}{2.545455in}}%
\pgfpathlineto{\pgfqpoint{7.890346in}{2.477273in}}%
\pgfpathlineto{\pgfqpoint{7.925968in}{2.613636in}}%
\pgfpathlineto{\pgfqpoint{7.961590in}{2.545455in}}%
\pgfpathlineto{\pgfqpoint{7.997211in}{2.545455in}}%
\pgfpathlineto{\pgfqpoint{8.068455in}{2.409091in}}%
\pgfpathlineto{\pgfqpoint{8.104076in}{2.545455in}}%
\pgfpathlineto{\pgfqpoint{8.175319in}{2.545455in}}%
\pgfpathlineto{\pgfqpoint{8.210941in}{2.613636in}}%
\pgfpathlineto{\pgfqpoint{8.246563in}{2.613636in}}%
\pgfpathlineto{\pgfqpoint{8.282184in}{2.477273in}}%
\pgfpathlineto{\pgfqpoint{8.317806in}{2.409091in}}%
\pgfpathlineto{\pgfqpoint{8.353428in}{2.477273in}}%
\pgfpathlineto{\pgfqpoint{8.389049in}{2.613636in}}%
\pgfpathlineto{\pgfqpoint{8.424671in}{2.545455in}}%
\pgfpathlineto{\pgfqpoint{8.460292in}{2.613636in}}%
\pgfpathlineto{\pgfqpoint{8.495914in}{2.477273in}}%
\pgfpathlineto{\pgfqpoint{8.531536in}{2.545455in}}%
\pgfpathlineto{\pgfqpoint{8.567157in}{2.477273in}}%
\pgfpathlineto{\pgfqpoint{8.638401in}{2.613636in}}%
\pgfpathlineto{\pgfqpoint{8.674022in}{2.545455in}}%
\pgfpathlineto{\pgfqpoint{8.709644in}{2.545455in}}%
\pgfpathlineto{\pgfqpoint{8.745265in}{2.409091in}}%
\pgfpathlineto{\pgfqpoint{8.780887in}{2.545455in}}%
\pgfpathlineto{\pgfqpoint{8.816509in}{2.477273in}}%
\pgfpathlineto{\pgfqpoint{8.887752in}{2.477273in}}%
\pgfpathlineto{\pgfqpoint{8.923374in}{2.545455in}}%
\pgfpathlineto{\pgfqpoint{9.137103in}{2.545455in}}%
\pgfpathlineto{\pgfqpoint{9.172725in}{2.613636in}}%
\pgfpathlineto{\pgfqpoint{9.208347in}{2.613636in}}%
\pgfpathlineto{\pgfqpoint{9.243968in}{2.477273in}}%
\pgfpathlineto{\pgfqpoint{9.279590in}{2.477273in}}%
\pgfpathlineto{\pgfqpoint{9.315211in}{2.613636in}}%
\pgfpathlineto{\pgfqpoint{9.350833in}{2.477273in}}%
\pgfpathlineto{\pgfqpoint{9.386455in}{2.613636in}}%
\pgfpathlineto{\pgfqpoint{9.422076in}{2.477273in}}%
\pgfpathlineto{\pgfqpoint{9.457698in}{2.545455in}}%
\pgfpathlineto{\pgfqpoint{9.493320in}{2.477273in}}%
\pgfpathlineto{\pgfqpoint{9.510000in}{2.541127in}}%
\pgfpathlineto{\pgfqpoint{9.510000in}{2.541127in}}%
\pgfusepath{stroke}%
\end{pgfscope}%
\begin{pgfscope}%
\pgfpathrectangle{\pgfqpoint{1.000000in}{2.000000in}}{\pgfqpoint{8.500000in}{3.000000in}}%
\pgfusepath{clip}%
\pgfsetrectcap%
\pgfsetroundjoin%
\pgfsetlinewidth{2.007500pt}%
\definecolor{currentstroke}{rgb}{0.000000,0.000000,0.000000}%
\pgfsetstrokecolor{currentstroke}%
\pgfsetdash{}{0pt}%
\pgfpathmoveto{\pgfqpoint{0.990000in}{2.545455in}}%
\pgfpathlineto{\pgfqpoint{2.796454in}{2.545455in}}%
\pgfpathlineto{\pgfqpoint{2.832075in}{2.548459in}}%
\pgfpathlineto{\pgfqpoint{2.867697in}{2.611155in}}%
\pgfpathlineto{\pgfqpoint{2.903318in}{2.812862in}}%
\pgfpathlineto{\pgfqpoint{2.938940in}{3.103160in}}%
\pgfpathlineto{\pgfqpoint{2.974562in}{3.375760in}}%
\pgfpathlineto{\pgfqpoint{3.010183in}{3.564298in}}%
\pgfpathlineto{\pgfqpoint{3.045805in}{3.654000in}}%
\pgfpathlineto{\pgfqpoint{3.081427in}{3.659941in}}%
\pgfpathlineto{\pgfqpoint{3.117048in}{3.608361in}}%
\pgfpathlineto{\pgfqpoint{3.152670in}{3.574626in}}%
\pgfpathlineto{\pgfqpoint{3.188292in}{3.691743in}}%
\pgfpathlineto{\pgfqpoint{3.259535in}{4.194172in}}%
\pgfpathlineto{\pgfqpoint{3.295156in}{4.360256in}}%
\pgfpathlineto{\pgfqpoint{3.330778in}{4.411138in}}%
\pgfpathlineto{\pgfqpoint{3.366400in}{4.361519in}}%
\pgfpathlineto{\pgfqpoint{3.402021in}{4.241584in}}%
\pgfpathlineto{\pgfqpoint{3.473265in}{3.920983in}}%
\pgfpathlineto{\pgfqpoint{3.508886in}{3.866682in}}%
\pgfpathlineto{\pgfqpoint{3.544508in}{3.960020in}}%
\pgfpathlineto{\pgfqpoint{3.580129in}{4.112593in}}%
\pgfpathlineto{\pgfqpoint{3.615751in}{4.229592in}}%
\pgfpathlineto{\pgfqpoint{3.651373in}{4.306836in}}%
\pgfpathlineto{\pgfqpoint{3.686994in}{4.413237in}}%
\pgfpathlineto{\pgfqpoint{3.722616in}{4.537410in}}%
\pgfpathlineto{\pgfqpoint{3.758238in}{4.617582in}}%
\pgfpathlineto{\pgfqpoint{3.793859in}{4.616812in}}%
\pgfpathlineto{\pgfqpoint{3.829481in}{4.534912in}}%
\pgfpathlineto{\pgfqpoint{3.865102in}{4.392063in}}%
\pgfpathlineto{\pgfqpoint{3.900724in}{4.213268in}}%
\pgfpathlineto{\pgfqpoint{3.971967in}{3.828464in}}%
\pgfpathlineto{\pgfqpoint{4.007589in}{3.648045in}}%
\pgfpathlineto{\pgfqpoint{4.043211in}{3.484288in}}%
\pgfpathlineto{\pgfqpoint{4.078832in}{3.339381in}}%
\pgfpathlineto{\pgfqpoint{4.114454in}{3.213469in}}%
\pgfpathlineto{\pgfqpoint{4.150075in}{3.105503in}}%
\pgfpathlineto{\pgfqpoint{4.185697in}{3.013855in}}%
\pgfpathlineto{\pgfqpoint{4.221319in}{2.951060in}}%
\pgfpathlineto{\pgfqpoint{4.256940in}{3.006035in}}%
\pgfpathlineto{\pgfqpoint{4.292562in}{3.215910in}}%
\pgfpathlineto{\pgfqpoint{4.328184in}{3.485379in}}%
\pgfpathlineto{\pgfqpoint{4.363805in}{3.709387in}}%
\pgfpathlineto{\pgfqpoint{4.399427in}{3.838631in}}%
\pgfpathlineto{\pgfqpoint{4.435048in}{3.871472in}}%
\pgfpathlineto{\pgfqpoint{4.470670in}{3.829070in}}%
\pgfpathlineto{\pgfqpoint{4.506292in}{3.737629in}}%
\pgfpathlineto{\pgfqpoint{4.541913in}{3.619935in}}%
\pgfpathlineto{\pgfqpoint{4.613157in}{3.367273in}}%
\pgfpathlineto{\pgfqpoint{4.648778in}{3.249891in}}%
\pgfpathlineto{\pgfqpoint{4.684400in}{3.143957in}}%
\pgfpathlineto{\pgfqpoint{4.720021in}{3.050665in}}%
\pgfpathlineto{\pgfqpoint{4.755643in}{2.969916in}}%
\pgfpathlineto{\pgfqpoint{4.791265in}{2.900892in}}%
\pgfpathlineto{\pgfqpoint{4.826886in}{2.842424in}}%
\pgfpathlineto{\pgfqpoint{4.862508in}{2.793222in}}%
\pgfpathlineto{\pgfqpoint{4.898130in}{2.752016in}}%
\pgfpathlineto{\pgfqpoint{4.933751in}{2.717621in}}%
\pgfpathlineto{\pgfqpoint{4.969373in}{2.688978in}}%
\pgfpathlineto{\pgfqpoint{5.004995in}{2.665159in}}%
\pgfpathlineto{\pgfqpoint{5.040616in}{2.645369in}}%
\pgfpathlineto{\pgfqpoint{5.076238in}{2.628931in}}%
\pgfpathlineto{\pgfqpoint{5.111859in}{2.615275in}}%
\pgfpathlineto{\pgfqpoint{5.147481in}{2.603927in}}%
\pgfpathlineto{\pgfqpoint{5.183103in}{2.594489in}}%
\pgfpathlineto{\pgfqpoint{5.218724in}{2.586633in}}%
\pgfpathlineto{\pgfqpoint{5.289968in}{2.574627in}}%
\pgfpathlineto{\pgfqpoint{5.361211in}{2.566250in}}%
\pgfpathlineto{\pgfqpoint{5.432454in}{2.560374in}}%
\pgfpathlineto{\pgfqpoint{5.539319in}{2.554629in}}%
\pgfpathlineto{\pgfqpoint{5.646184in}{2.551177in}}%
\pgfpathlineto{\pgfqpoint{5.824292in}{2.548140in}}%
\pgfpathlineto{\pgfqpoint{6.073643in}{2.546442in}}%
\pgfpathlineto{\pgfqpoint{6.607968in}{2.545596in}}%
\pgfpathlineto{\pgfqpoint{9.350833in}{2.545455in}}%
\pgfpathlineto{\pgfqpoint{9.510000in}{2.545455in}}%
\pgfpathlineto{\pgfqpoint{9.510000in}{2.545455in}}%
\pgfusepath{stroke}%
\end{pgfscope}%
\begin{pgfscope}%
\pgfpathrectangle{\pgfqpoint{1.000000in}{2.000000in}}{\pgfqpoint{8.500000in}{3.000000in}}%
\pgfusepath{clip}%
\pgfsetrectcap%
\pgfsetroundjoin%
\pgfsetlinewidth{2.007500pt}%
\definecolor{currentstroke}{rgb}{0.000000,0.500000,0.000000}%
\pgfsetstrokecolor{currentstroke}%
\pgfsetdash{}{0pt}%
\pgfpathmoveto{\pgfqpoint{0.990000in}{2.545455in}}%
\pgfpathlineto{\pgfqpoint{2.796454in}{2.545639in}}%
\pgfpathlineto{\pgfqpoint{2.832075in}{2.568002in}}%
\pgfpathlineto{\pgfqpoint{2.867697in}{2.699187in}}%
\pgfpathlineto{\pgfqpoint{2.903318in}{2.948221in}}%
\pgfpathlineto{\pgfqpoint{2.938940in}{3.222309in}}%
\pgfpathlineto{\pgfqpoint{2.974562in}{3.437700in}}%
\pgfpathlineto{\pgfqpoint{3.010183in}{3.561427in}}%
\pgfpathlineto{\pgfqpoint{3.045805in}{3.598312in}}%
\pgfpathlineto{\pgfqpoint{3.081427in}{3.569645in}}%
\pgfpathlineto{\pgfqpoint{3.117048in}{3.516276in}}%
\pgfpathlineto{\pgfqpoint{3.152670in}{3.539686in}}%
\pgfpathlineto{\pgfqpoint{3.188292in}{3.716332in}}%
\pgfpathlineto{\pgfqpoint{3.223913in}{3.988611in}}%
\pgfpathlineto{\pgfqpoint{3.259535in}{4.237345in}}%
\pgfpathlineto{\pgfqpoint{3.295156in}{4.386169in}}%
\pgfpathlineto{\pgfqpoint{3.330778in}{4.418868in}}%
\pgfpathlineto{\pgfqpoint{3.366400in}{4.354992in}}%
\pgfpathlineto{\pgfqpoint{3.402021in}{4.225747in}}%
\pgfpathlineto{\pgfqpoint{3.437643in}{4.060769in}}%
\pgfpathlineto{\pgfqpoint{3.473265in}{3.907510in}}%
\pgfpathlineto{\pgfqpoint{3.508886in}{3.879072in}}%
\pgfpathlineto{\pgfqpoint{3.544508in}{3.994841in}}%
\pgfpathlineto{\pgfqpoint{3.580129in}{4.154806in}}%
\pgfpathlineto{\pgfqpoint{3.615751in}{4.266354in}}%
\pgfpathlineto{\pgfqpoint{3.651373in}{4.311086in}}%
\pgfpathlineto{\pgfqpoint{3.686994in}{4.367700in}}%
\pgfpathlineto{\pgfqpoint{3.722616in}{4.466203in}}%
\pgfpathlineto{\pgfqpoint{3.758238in}{4.551698in}}%
\pgfpathlineto{\pgfqpoint{3.793859in}{4.572733in}}%
\pgfpathlineto{\pgfqpoint{3.829481in}{4.515418in}}%
\pgfpathlineto{\pgfqpoint{3.865102in}{4.392856in}}%
\pgfpathlineto{\pgfqpoint{3.900724in}{4.228016in}}%
\pgfpathlineto{\pgfqpoint{4.007589in}{3.675278in}}%
\pgfpathlineto{\pgfqpoint{4.043211in}{3.510332in}}%
\pgfpathlineto{\pgfqpoint{4.078832in}{3.363226in}}%
\pgfpathlineto{\pgfqpoint{4.114454in}{3.234673in}}%
\pgfpathlineto{\pgfqpoint{4.150075in}{3.123979in}}%
\pgfpathlineto{\pgfqpoint{4.185697in}{3.029897in}}%
\pgfpathlineto{\pgfqpoint{4.221319in}{2.975759in}}%
\pgfpathlineto{\pgfqpoint{4.256940in}{3.058679in}}%
\pgfpathlineto{\pgfqpoint{4.292562in}{3.287173in}}%
\pgfpathlineto{\pgfqpoint{4.328184in}{3.553572in}}%
\pgfpathlineto{\pgfqpoint{4.363805in}{3.760729in}}%
\pgfpathlineto{\pgfqpoint{4.399427in}{3.869711in}}%
\pgfpathlineto{\pgfqpoint{4.435048in}{3.884941in}}%
\pgfpathlineto{\pgfqpoint{4.470670in}{3.829658in}}%
\pgfpathlineto{\pgfqpoint{4.506292in}{3.729962in}}%
\pgfpathlineto{\pgfqpoint{4.541913in}{3.607691in}}%
\pgfpathlineto{\pgfqpoint{4.613157in}{3.352762in}}%
\pgfpathlineto{\pgfqpoint{4.648778in}{3.236085in}}%
\pgfpathlineto{\pgfqpoint{4.684400in}{3.131380in}}%
\pgfpathlineto{\pgfqpoint{4.720021in}{3.039533in}}%
\pgfpathlineto{\pgfqpoint{4.755643in}{2.960258in}}%
\pgfpathlineto{\pgfqpoint{4.791265in}{2.892633in}}%
\pgfpathlineto{\pgfqpoint{4.826886in}{2.835433in}}%
\pgfpathlineto{\pgfqpoint{4.862508in}{2.787350in}}%
\pgfpathlineto{\pgfqpoint{4.898130in}{2.747110in}}%
\pgfpathlineto{\pgfqpoint{4.933751in}{2.713539in}}%
\pgfpathlineto{\pgfqpoint{4.969373in}{2.685591in}}%
\pgfpathlineto{\pgfqpoint{5.004995in}{2.662355in}}%
\pgfpathlineto{\pgfqpoint{5.040616in}{2.643049in}}%
\pgfpathlineto{\pgfqpoint{5.076238in}{2.627013in}}%
\pgfpathlineto{\pgfqpoint{5.111859in}{2.613691in}}%
\pgfpathlineto{\pgfqpoint{5.147481in}{2.602617in}}%
\pgfpathlineto{\pgfqpoint{5.183103in}{2.593406in}}%
\pgfpathlineto{\pgfqpoint{5.218724in}{2.585737in}}%
\pgfpathlineto{\pgfqpoint{5.289968in}{2.574012in}}%
\pgfpathlineto{\pgfqpoint{5.361211in}{2.565827in}}%
\pgfpathlineto{\pgfqpoint{5.432454in}{2.560080in}}%
\pgfpathlineto{\pgfqpoint{5.539319in}{2.554459in}}%
\pgfpathlineto{\pgfqpoint{5.681805in}{2.550274in}}%
\pgfpathlineto{\pgfqpoint{5.859914in}{2.547737in}}%
\pgfpathlineto{\pgfqpoint{6.144887in}{2.546195in}}%
\pgfpathlineto{\pgfqpoint{6.821698in}{2.545523in}}%
\pgfpathlineto{\pgfqpoint{9.510000in}{2.545455in}}%
\pgfpathlineto{\pgfqpoint{9.510000in}{2.545455in}}%
\pgfusepath{stroke}%
\end{pgfscope}%
\begin{pgfscope}%
\pgfsetrectcap%
\pgfsetmiterjoin%
\pgfsetlinewidth{0.803000pt}%
\definecolor{currentstroke}{rgb}{0.000000,0.000000,0.000000}%
\pgfsetstrokecolor{currentstroke}%
\pgfsetdash{}{0pt}%
\pgfpathmoveto{\pgfqpoint{1.000000in}{2.000000in}}%
\pgfpathlineto{\pgfqpoint{1.000000in}{5.000000in}}%
\pgfusepath{stroke}%
\end{pgfscope}%
\begin{pgfscope}%
\pgfsetrectcap%
\pgfsetmiterjoin%
\pgfsetlinewidth{0.803000pt}%
\definecolor{currentstroke}{rgb}{0.000000,0.000000,0.000000}%
\pgfsetstrokecolor{currentstroke}%
\pgfsetdash{}{0pt}%
\pgfpathmoveto{\pgfqpoint{9.500000in}{2.000000in}}%
\pgfpathlineto{\pgfqpoint{9.500000in}{5.000000in}}%
\pgfusepath{stroke}%
\end{pgfscope}%
\begin{pgfscope}%
\pgfsetrectcap%
\pgfsetmiterjoin%
\pgfsetlinewidth{0.803000pt}%
\definecolor{currentstroke}{rgb}{0.000000,0.000000,0.000000}%
\pgfsetstrokecolor{currentstroke}%
\pgfsetdash{}{0pt}%
\pgfpathmoveto{\pgfqpoint{1.000000in}{2.000000in}}%
\pgfpathlineto{\pgfqpoint{9.500000in}{2.000000in}}%
\pgfusepath{stroke}%
\end{pgfscope}%
\begin{pgfscope}%
\pgfsetrectcap%
\pgfsetmiterjoin%
\pgfsetlinewidth{0.803000pt}%
\definecolor{currentstroke}{rgb}{0.000000,0.000000,0.000000}%
\pgfsetstrokecolor{currentstroke}%
\pgfsetdash{}{0pt}%
\pgfpathmoveto{\pgfqpoint{1.000000in}{5.000000in}}%
\pgfpathlineto{\pgfqpoint{9.500000in}{5.000000in}}%
\pgfusepath{stroke}%
\end{pgfscope}%
\begin{pgfscope}%
\pgfsetbuttcap%
\pgfsetmiterjoin%
\definecolor{currentfill}{rgb}{1.000000,1.000000,1.000000}%
\pgfsetfillcolor{currentfill}%
\pgfsetfillopacity{0.800000}%
\pgfsetlinewidth{1.003750pt}%
\definecolor{currentstroke}{rgb}{0.800000,0.800000,0.800000}%
\pgfsetstrokecolor{currentstroke}%
\pgfsetstrokeopacity{0.800000}%
\pgfsetdash{}{0pt}%
\pgfpathmoveto{\pgfqpoint{7.151502in}{3.197952in}}%
\pgfpathlineto{\pgfqpoint{9.305556in}{3.197952in}}%
\pgfpathquadraticcurveto{\pgfqpoint{9.361111in}{3.197952in}}{\pgfqpoint{9.361111in}{3.253507in}}%
\pgfpathlineto{\pgfqpoint{9.361111in}{4.805556in}}%
\pgfpathquadraticcurveto{\pgfqpoint{9.361111in}{4.861111in}}{\pgfqpoint{9.305556in}{4.861111in}}%
\pgfpathlineto{\pgfqpoint{7.151502in}{4.861111in}}%
\pgfpathquadraticcurveto{\pgfqpoint{7.095946in}{4.861111in}}{\pgfqpoint{7.095946in}{4.805556in}}%
\pgfpathlineto{\pgfqpoint{7.095946in}{3.253507in}}%
\pgfpathquadraticcurveto{\pgfqpoint{7.095946in}{3.197952in}}{\pgfqpoint{7.151502in}{3.197952in}}%
\pgfpathclose%
\pgfusepath{stroke,fill}%
\end{pgfscope}%
\begin{pgfscope}%
\pgfsetrectcap%
\pgfsetroundjoin%
\pgfsetlinewidth{2.007500pt}%
\definecolor{currentstroke}{rgb}{0.000000,0.000000,1.000000}%
\pgfsetstrokecolor{currentstroke}%
\pgfsetdash{}{0pt}%
\pgfpathmoveto{\pgfqpoint{7.207057in}{4.647184in}}%
\pgfpathlineto{\pgfqpoint{7.762613in}{4.647184in}}%
\pgfusepath{stroke}%
\end{pgfscope}%
\begin{pgfscope}%
\definecolor{textcolor}{rgb}{0.000000,0.000000,0.000000}%
\pgfsetstrokecolor{textcolor}%
\pgfsetfillcolor{textcolor}%
\pgftext[x=7.984835in,y=4.549962in,left,base]{\color{textcolor}\sffamily\fontsize{20.000000}{24.000000}\selectfont origin wave}%
\end{pgfscope}%
\begin{pgfscope}%
\pgfsetrectcap%
\pgfsetroundjoin%
\pgfsetlinewidth{2.007500pt}%
\definecolor{currentstroke}{rgb}{0.000000,0.000000,0.000000}%
\pgfsetstrokecolor{currentstroke}%
\pgfsetdash{}{0pt}%
\pgfpathmoveto{\pgfqpoint{7.207057in}{4.252227in}}%
\pgfpathlineto{\pgfqpoint{7.762613in}{4.252227in}}%
\pgfusepath{stroke}%
\end{pgfscope}%
\begin{pgfscope}%
\definecolor{textcolor}{rgb}{0.000000,0.000000,0.000000}%
\pgfsetstrokecolor{textcolor}%
\pgfsetfillcolor{textcolor}%
\pgftext[x=7.984835in,y=4.155005in,left,base]{\color{textcolor}\sffamily\fontsize{20.000000}{24.000000}\selectfont truth wave}%
\end{pgfscope}%
\begin{pgfscope}%
\pgfsetrectcap%
\pgfsetroundjoin%
\pgfsetlinewidth{2.007500pt}%
\definecolor{currentstroke}{rgb}{0.000000,0.500000,0.000000}%
\pgfsetstrokecolor{currentstroke}%
\pgfsetdash{}{0pt}%
\pgfpathmoveto{\pgfqpoint{7.207057in}{3.857271in}}%
\pgfpathlineto{\pgfqpoint{7.762613in}{3.857271in}}%
\pgfusepath{stroke}%
\end{pgfscope}%
\begin{pgfscope}%
\definecolor{textcolor}{rgb}{0.000000,0.000000,0.000000}%
\pgfsetstrokecolor{textcolor}%
\pgfsetfillcolor{textcolor}%
\pgftext[x=7.984835in,y=3.760048in,left,base]{\color{textcolor}\sffamily\fontsize{20.000000}{24.000000}\selectfont recon wave}%
\end{pgfscope}%
\begin{pgfscope}%
\pgfsetbuttcap%
\pgfsetroundjoin%
\pgfsetlinewidth{2.007500pt}%
\definecolor{currentstroke}{rgb}{0.000000,0.750000,0.750000}%
\pgfsetstrokecolor{currentstroke}%
\pgfsetdash{}{0pt}%
\pgfpathmoveto{\pgfqpoint{7.207057in}{3.462314in}}%
\pgfpathlineto{\pgfqpoint{7.762613in}{3.462314in}}%
\pgfusepath{stroke}%
\end{pgfscope}%
\begin{pgfscope}%
\definecolor{textcolor}{rgb}{0.000000,0.000000,0.000000}%
\pgfsetstrokecolor{textcolor}%
\pgfsetfillcolor{textcolor}%
\pgftext[x=7.984835in,y=3.365092in,left,base]{\color{textcolor}\sffamily\fontsize{20.000000}{24.000000}\selectfont threshold}%
\end{pgfscope}%
\begin{pgfscope}%
\pgfsetbuttcap%
\pgfsetmiterjoin%
\definecolor{currentfill}{rgb}{1.000000,1.000000,1.000000}%
\pgfsetfillcolor{currentfill}%
\pgfsetlinewidth{0.000000pt}%
\definecolor{currentstroke}{rgb}{0.000000,0.000000,0.000000}%
\pgfsetstrokecolor{currentstroke}%
\pgfsetstrokeopacity{0.000000}%
\pgfsetdash{}{0pt}%
\pgfpathmoveto{\pgfqpoint{1.000000in}{5.000000in}}%
\pgfpathlineto{\pgfqpoint{9.500000in}{5.000000in}}%
\pgfpathlineto{\pgfqpoint{9.500000in}{7.000000in}}%
\pgfpathlineto{\pgfqpoint{1.000000in}{7.000000in}}%
\pgfpathclose%
\pgfusepath{fill}%
\end{pgfscope}%
\begin{pgfscope}%
\pgfpathrectangle{\pgfqpoint{1.000000in}{5.000000in}}{\pgfqpoint{8.500000in}{2.000000in}}%
\pgfusepath{clip}%
\pgfsetrectcap%
\pgfsetroundjoin%
\pgfsetlinewidth{0.803000pt}%
\definecolor{currentstroke}{rgb}{0.690196,0.690196,0.690196}%
\pgfsetstrokecolor{currentstroke}%
\pgfsetdash{}{0pt}%
\pgfpathmoveto{\pgfqpoint{2.725210in}{5.000000in}}%
\pgfpathlineto{\pgfqpoint{2.725210in}{7.000000in}}%
\pgfusepath{stroke}%
\end{pgfscope}%
\begin{pgfscope}%
\pgfsetbuttcap%
\pgfsetroundjoin%
\definecolor{currentfill}{rgb}{0.000000,0.000000,0.000000}%
\pgfsetfillcolor{currentfill}%
\pgfsetlinewidth{0.803000pt}%
\definecolor{currentstroke}{rgb}{0.000000,0.000000,0.000000}%
\pgfsetstrokecolor{currentstroke}%
\pgfsetdash{}{0pt}%
\pgfsys@defobject{currentmarker}{\pgfqpoint{0.000000in}{-0.048611in}}{\pgfqpoint{0.000000in}{0.000000in}}{%
\pgfpathmoveto{\pgfqpoint{0.000000in}{0.000000in}}%
\pgfpathlineto{\pgfqpoint{0.000000in}{-0.048611in}}%
\pgfusepath{stroke,fill}%
}%
\begin{pgfscope}%
\pgfsys@transformshift{2.725210in}{5.000000in}%
\pgfsys@useobject{currentmarker}{}%
\end{pgfscope}%
\end{pgfscope}%
\begin{pgfscope}%
\pgfpathrectangle{\pgfqpoint{1.000000in}{5.000000in}}{\pgfqpoint{8.500000in}{2.000000in}}%
\pgfusepath{clip}%
\pgfsetrectcap%
\pgfsetroundjoin%
\pgfsetlinewidth{0.803000pt}%
\definecolor{currentstroke}{rgb}{0.690196,0.690196,0.690196}%
\pgfsetstrokecolor{currentstroke}%
\pgfsetdash{}{0pt}%
\pgfpathmoveto{\pgfqpoint{4.506292in}{5.000000in}}%
\pgfpathlineto{\pgfqpoint{4.506292in}{7.000000in}}%
\pgfusepath{stroke}%
\end{pgfscope}%
\begin{pgfscope}%
\pgfsetbuttcap%
\pgfsetroundjoin%
\definecolor{currentfill}{rgb}{0.000000,0.000000,0.000000}%
\pgfsetfillcolor{currentfill}%
\pgfsetlinewidth{0.803000pt}%
\definecolor{currentstroke}{rgb}{0.000000,0.000000,0.000000}%
\pgfsetstrokecolor{currentstroke}%
\pgfsetdash{}{0pt}%
\pgfsys@defobject{currentmarker}{\pgfqpoint{0.000000in}{-0.048611in}}{\pgfqpoint{0.000000in}{0.000000in}}{%
\pgfpathmoveto{\pgfqpoint{0.000000in}{0.000000in}}%
\pgfpathlineto{\pgfqpoint{0.000000in}{-0.048611in}}%
\pgfusepath{stroke,fill}%
}%
\begin{pgfscope}%
\pgfsys@transformshift{4.506292in}{5.000000in}%
\pgfsys@useobject{currentmarker}{}%
\end{pgfscope}%
\end{pgfscope}%
\begin{pgfscope}%
\pgfpathrectangle{\pgfqpoint{1.000000in}{5.000000in}}{\pgfqpoint{8.500000in}{2.000000in}}%
\pgfusepath{clip}%
\pgfsetrectcap%
\pgfsetroundjoin%
\pgfsetlinewidth{0.803000pt}%
\definecolor{currentstroke}{rgb}{0.690196,0.690196,0.690196}%
\pgfsetstrokecolor{currentstroke}%
\pgfsetdash{}{0pt}%
\pgfpathmoveto{\pgfqpoint{6.287373in}{5.000000in}}%
\pgfpathlineto{\pgfqpoint{6.287373in}{7.000000in}}%
\pgfusepath{stroke}%
\end{pgfscope}%
\begin{pgfscope}%
\pgfsetbuttcap%
\pgfsetroundjoin%
\definecolor{currentfill}{rgb}{0.000000,0.000000,0.000000}%
\pgfsetfillcolor{currentfill}%
\pgfsetlinewidth{0.803000pt}%
\definecolor{currentstroke}{rgb}{0.000000,0.000000,0.000000}%
\pgfsetstrokecolor{currentstroke}%
\pgfsetdash{}{0pt}%
\pgfsys@defobject{currentmarker}{\pgfqpoint{0.000000in}{-0.048611in}}{\pgfqpoint{0.000000in}{0.000000in}}{%
\pgfpathmoveto{\pgfqpoint{0.000000in}{0.000000in}}%
\pgfpathlineto{\pgfqpoint{0.000000in}{-0.048611in}}%
\pgfusepath{stroke,fill}%
}%
\begin{pgfscope}%
\pgfsys@transformshift{6.287373in}{5.000000in}%
\pgfsys@useobject{currentmarker}{}%
\end{pgfscope}%
\end{pgfscope}%
\begin{pgfscope}%
\pgfpathrectangle{\pgfqpoint{1.000000in}{5.000000in}}{\pgfqpoint{8.500000in}{2.000000in}}%
\pgfusepath{clip}%
\pgfsetrectcap%
\pgfsetroundjoin%
\pgfsetlinewidth{0.803000pt}%
\definecolor{currentstroke}{rgb}{0.690196,0.690196,0.690196}%
\pgfsetstrokecolor{currentstroke}%
\pgfsetdash{}{0pt}%
\pgfpathmoveto{\pgfqpoint{8.068455in}{5.000000in}}%
\pgfpathlineto{\pgfqpoint{8.068455in}{7.000000in}}%
\pgfusepath{stroke}%
\end{pgfscope}%
\begin{pgfscope}%
\pgfsetbuttcap%
\pgfsetroundjoin%
\definecolor{currentfill}{rgb}{0.000000,0.000000,0.000000}%
\pgfsetfillcolor{currentfill}%
\pgfsetlinewidth{0.803000pt}%
\definecolor{currentstroke}{rgb}{0.000000,0.000000,0.000000}%
\pgfsetstrokecolor{currentstroke}%
\pgfsetdash{}{0pt}%
\pgfsys@defobject{currentmarker}{\pgfqpoint{0.000000in}{-0.048611in}}{\pgfqpoint{0.000000in}{0.000000in}}{%
\pgfpathmoveto{\pgfqpoint{0.000000in}{0.000000in}}%
\pgfpathlineto{\pgfqpoint{0.000000in}{-0.048611in}}%
\pgfusepath{stroke,fill}%
}%
\begin{pgfscope}%
\pgfsys@transformshift{8.068455in}{5.000000in}%
\pgfsys@useobject{currentmarker}{}%
\end{pgfscope}%
\end{pgfscope}%
\begin{pgfscope}%
\pgfpathrectangle{\pgfqpoint{1.000000in}{5.000000in}}{\pgfqpoint{8.500000in}{2.000000in}}%
\pgfusepath{clip}%
\pgfsetrectcap%
\pgfsetroundjoin%
\pgfsetlinewidth{0.803000pt}%
\definecolor{currentstroke}{rgb}{0.690196,0.690196,0.690196}%
\pgfsetstrokecolor{currentstroke}%
\pgfsetdash{}{0pt}%
\pgfpathmoveto{\pgfqpoint{1.000000in}{5.000000in}}%
\pgfpathlineto{\pgfqpoint{9.500000in}{5.000000in}}%
\pgfusepath{stroke}%
\end{pgfscope}%
\begin{pgfscope}%
\pgfsetbuttcap%
\pgfsetroundjoin%
\definecolor{currentfill}{rgb}{0.000000,0.000000,0.000000}%
\pgfsetfillcolor{currentfill}%
\pgfsetlinewidth{0.803000pt}%
\definecolor{currentstroke}{rgb}{0.000000,0.000000,0.000000}%
\pgfsetstrokecolor{currentstroke}%
\pgfsetdash{}{0pt}%
\pgfsys@defobject{currentmarker}{\pgfqpoint{-0.048611in}{0.000000in}}{\pgfqpoint{-0.000000in}{0.000000in}}{%
\pgfpathmoveto{\pgfqpoint{-0.000000in}{0.000000in}}%
\pgfpathlineto{\pgfqpoint{-0.048611in}{0.000000in}}%
\pgfusepath{stroke,fill}%
}%
\begin{pgfscope}%
\pgfsys@transformshift{1.000000in}{5.000000in}%
\pgfsys@useobject{currentmarker}{}%
\end{pgfscope}%
\end{pgfscope}%
\begin{pgfscope}%
\definecolor{textcolor}{rgb}{0.000000,0.000000,0.000000}%
\pgfsetstrokecolor{textcolor}%
\pgfsetfillcolor{textcolor}%
\pgftext[x=0.560215in, y=4.899981in, left, base]{\color{textcolor}\sffamily\fontsize{20.000000}{24.000000}\selectfont \(\displaystyle {0.0}\)}%
\end{pgfscope}%
\begin{pgfscope}%
\pgfpathrectangle{\pgfqpoint{1.000000in}{5.000000in}}{\pgfqpoint{8.500000in}{2.000000in}}%
\pgfusepath{clip}%
\pgfsetrectcap%
\pgfsetroundjoin%
\pgfsetlinewidth{0.803000pt}%
\definecolor{currentstroke}{rgb}{0.690196,0.690196,0.690196}%
\pgfsetstrokecolor{currentstroke}%
\pgfsetdash{}{0pt}%
\pgfpathmoveto{\pgfqpoint{1.000000in}{5.245724in}}%
\pgfpathlineto{\pgfqpoint{9.500000in}{5.245724in}}%
\pgfusepath{stroke}%
\end{pgfscope}%
\begin{pgfscope}%
\pgfsetbuttcap%
\pgfsetroundjoin%
\definecolor{currentfill}{rgb}{0.000000,0.000000,0.000000}%
\pgfsetfillcolor{currentfill}%
\pgfsetlinewidth{0.803000pt}%
\definecolor{currentstroke}{rgb}{0.000000,0.000000,0.000000}%
\pgfsetstrokecolor{currentstroke}%
\pgfsetdash{}{0pt}%
\pgfsys@defobject{currentmarker}{\pgfqpoint{-0.048611in}{0.000000in}}{\pgfqpoint{-0.000000in}{0.000000in}}{%
\pgfpathmoveto{\pgfqpoint{-0.000000in}{0.000000in}}%
\pgfpathlineto{\pgfqpoint{-0.048611in}{0.000000in}}%
\pgfusepath{stroke,fill}%
}%
\begin{pgfscope}%
\pgfsys@transformshift{1.000000in}{5.245724in}%
\pgfsys@useobject{currentmarker}{}%
\end{pgfscope}%
\end{pgfscope}%
\begin{pgfscope}%
\definecolor{textcolor}{rgb}{0.000000,0.000000,0.000000}%
\pgfsetstrokecolor{textcolor}%
\pgfsetfillcolor{textcolor}%
\pgftext[x=0.560215in, y=5.145705in, left, base]{\color{textcolor}\sffamily\fontsize{20.000000}{24.000000}\selectfont \(\displaystyle {0.2}\)}%
\end{pgfscope}%
\begin{pgfscope}%
\pgfpathrectangle{\pgfqpoint{1.000000in}{5.000000in}}{\pgfqpoint{8.500000in}{2.000000in}}%
\pgfusepath{clip}%
\pgfsetrectcap%
\pgfsetroundjoin%
\pgfsetlinewidth{0.803000pt}%
\definecolor{currentstroke}{rgb}{0.690196,0.690196,0.690196}%
\pgfsetstrokecolor{currentstroke}%
\pgfsetdash{}{0pt}%
\pgfpathmoveto{\pgfqpoint{1.000000in}{5.491449in}}%
\pgfpathlineto{\pgfqpoint{9.500000in}{5.491449in}}%
\pgfusepath{stroke}%
\end{pgfscope}%
\begin{pgfscope}%
\pgfsetbuttcap%
\pgfsetroundjoin%
\definecolor{currentfill}{rgb}{0.000000,0.000000,0.000000}%
\pgfsetfillcolor{currentfill}%
\pgfsetlinewidth{0.803000pt}%
\definecolor{currentstroke}{rgb}{0.000000,0.000000,0.000000}%
\pgfsetstrokecolor{currentstroke}%
\pgfsetdash{}{0pt}%
\pgfsys@defobject{currentmarker}{\pgfqpoint{-0.048611in}{0.000000in}}{\pgfqpoint{-0.000000in}{0.000000in}}{%
\pgfpathmoveto{\pgfqpoint{-0.000000in}{0.000000in}}%
\pgfpathlineto{\pgfqpoint{-0.048611in}{0.000000in}}%
\pgfusepath{stroke,fill}%
}%
\begin{pgfscope}%
\pgfsys@transformshift{1.000000in}{5.491449in}%
\pgfsys@useobject{currentmarker}{}%
\end{pgfscope}%
\end{pgfscope}%
\begin{pgfscope}%
\definecolor{textcolor}{rgb}{0.000000,0.000000,0.000000}%
\pgfsetstrokecolor{textcolor}%
\pgfsetfillcolor{textcolor}%
\pgftext[x=0.560215in, y=5.391430in, left, base]{\color{textcolor}\sffamily\fontsize{20.000000}{24.000000}\selectfont \(\displaystyle {0.4}\)}%
\end{pgfscope}%
\begin{pgfscope}%
\pgfpathrectangle{\pgfqpoint{1.000000in}{5.000000in}}{\pgfqpoint{8.500000in}{2.000000in}}%
\pgfusepath{clip}%
\pgfsetrectcap%
\pgfsetroundjoin%
\pgfsetlinewidth{0.803000pt}%
\definecolor{currentstroke}{rgb}{0.690196,0.690196,0.690196}%
\pgfsetstrokecolor{currentstroke}%
\pgfsetdash{}{0pt}%
\pgfpathmoveto{\pgfqpoint{1.000000in}{5.737173in}}%
\pgfpathlineto{\pgfqpoint{9.500000in}{5.737173in}}%
\pgfusepath{stroke}%
\end{pgfscope}%
\begin{pgfscope}%
\pgfsetbuttcap%
\pgfsetroundjoin%
\definecolor{currentfill}{rgb}{0.000000,0.000000,0.000000}%
\pgfsetfillcolor{currentfill}%
\pgfsetlinewidth{0.803000pt}%
\definecolor{currentstroke}{rgb}{0.000000,0.000000,0.000000}%
\pgfsetstrokecolor{currentstroke}%
\pgfsetdash{}{0pt}%
\pgfsys@defobject{currentmarker}{\pgfqpoint{-0.048611in}{0.000000in}}{\pgfqpoint{-0.000000in}{0.000000in}}{%
\pgfpathmoveto{\pgfqpoint{-0.000000in}{0.000000in}}%
\pgfpathlineto{\pgfqpoint{-0.048611in}{0.000000in}}%
\pgfusepath{stroke,fill}%
}%
\begin{pgfscope}%
\pgfsys@transformshift{1.000000in}{5.737173in}%
\pgfsys@useobject{currentmarker}{}%
\end{pgfscope}%
\end{pgfscope}%
\begin{pgfscope}%
\definecolor{textcolor}{rgb}{0.000000,0.000000,0.000000}%
\pgfsetstrokecolor{textcolor}%
\pgfsetfillcolor{textcolor}%
\pgftext[x=0.560215in, y=5.637154in, left, base]{\color{textcolor}\sffamily\fontsize{20.000000}{24.000000}\selectfont \(\displaystyle {0.6}\)}%
\end{pgfscope}%
\begin{pgfscope}%
\pgfpathrectangle{\pgfqpoint{1.000000in}{5.000000in}}{\pgfqpoint{8.500000in}{2.000000in}}%
\pgfusepath{clip}%
\pgfsetrectcap%
\pgfsetroundjoin%
\pgfsetlinewidth{0.803000pt}%
\definecolor{currentstroke}{rgb}{0.690196,0.690196,0.690196}%
\pgfsetstrokecolor{currentstroke}%
\pgfsetdash{}{0pt}%
\pgfpathmoveto{\pgfqpoint{1.000000in}{5.982898in}}%
\pgfpathlineto{\pgfqpoint{9.500000in}{5.982898in}}%
\pgfusepath{stroke}%
\end{pgfscope}%
\begin{pgfscope}%
\pgfsetbuttcap%
\pgfsetroundjoin%
\definecolor{currentfill}{rgb}{0.000000,0.000000,0.000000}%
\pgfsetfillcolor{currentfill}%
\pgfsetlinewidth{0.803000pt}%
\definecolor{currentstroke}{rgb}{0.000000,0.000000,0.000000}%
\pgfsetstrokecolor{currentstroke}%
\pgfsetdash{}{0pt}%
\pgfsys@defobject{currentmarker}{\pgfqpoint{-0.048611in}{0.000000in}}{\pgfqpoint{-0.000000in}{0.000000in}}{%
\pgfpathmoveto{\pgfqpoint{-0.000000in}{0.000000in}}%
\pgfpathlineto{\pgfqpoint{-0.048611in}{0.000000in}}%
\pgfusepath{stroke,fill}%
}%
\begin{pgfscope}%
\pgfsys@transformshift{1.000000in}{5.982898in}%
\pgfsys@useobject{currentmarker}{}%
\end{pgfscope}%
\end{pgfscope}%
\begin{pgfscope}%
\definecolor{textcolor}{rgb}{0.000000,0.000000,0.000000}%
\pgfsetstrokecolor{textcolor}%
\pgfsetfillcolor{textcolor}%
\pgftext[x=0.560215in, y=5.882879in, left, base]{\color{textcolor}\sffamily\fontsize{20.000000}{24.000000}\selectfont \(\displaystyle {0.8}\)}%
\end{pgfscope}%
\begin{pgfscope}%
\pgfpathrectangle{\pgfqpoint{1.000000in}{5.000000in}}{\pgfqpoint{8.500000in}{2.000000in}}%
\pgfusepath{clip}%
\pgfsetrectcap%
\pgfsetroundjoin%
\pgfsetlinewidth{0.803000pt}%
\definecolor{currentstroke}{rgb}{0.690196,0.690196,0.690196}%
\pgfsetstrokecolor{currentstroke}%
\pgfsetdash{}{0pt}%
\pgfpathmoveto{\pgfqpoint{1.000000in}{6.228622in}}%
\pgfpathlineto{\pgfqpoint{9.500000in}{6.228622in}}%
\pgfusepath{stroke}%
\end{pgfscope}%
\begin{pgfscope}%
\pgfsetbuttcap%
\pgfsetroundjoin%
\definecolor{currentfill}{rgb}{0.000000,0.000000,0.000000}%
\pgfsetfillcolor{currentfill}%
\pgfsetlinewidth{0.803000pt}%
\definecolor{currentstroke}{rgb}{0.000000,0.000000,0.000000}%
\pgfsetstrokecolor{currentstroke}%
\pgfsetdash{}{0pt}%
\pgfsys@defobject{currentmarker}{\pgfqpoint{-0.048611in}{0.000000in}}{\pgfqpoint{-0.000000in}{0.000000in}}{%
\pgfpathmoveto{\pgfqpoint{-0.000000in}{0.000000in}}%
\pgfpathlineto{\pgfqpoint{-0.048611in}{0.000000in}}%
\pgfusepath{stroke,fill}%
}%
\begin{pgfscope}%
\pgfsys@transformshift{1.000000in}{6.228622in}%
\pgfsys@useobject{currentmarker}{}%
\end{pgfscope}%
\end{pgfscope}%
\begin{pgfscope}%
\definecolor{textcolor}{rgb}{0.000000,0.000000,0.000000}%
\pgfsetstrokecolor{textcolor}%
\pgfsetfillcolor{textcolor}%
\pgftext[x=0.560215in, y=6.128603in, left, base]{\color{textcolor}\sffamily\fontsize{20.000000}{24.000000}\selectfont \(\displaystyle {1.0}\)}%
\end{pgfscope}%
\begin{pgfscope}%
\pgfpathrectangle{\pgfqpoint{1.000000in}{5.000000in}}{\pgfqpoint{8.500000in}{2.000000in}}%
\pgfusepath{clip}%
\pgfsetrectcap%
\pgfsetroundjoin%
\pgfsetlinewidth{0.803000pt}%
\definecolor{currentstroke}{rgb}{0.690196,0.690196,0.690196}%
\pgfsetstrokecolor{currentstroke}%
\pgfsetdash{}{0pt}%
\pgfpathmoveto{\pgfqpoint{1.000000in}{6.474347in}}%
\pgfpathlineto{\pgfqpoint{9.500000in}{6.474347in}}%
\pgfusepath{stroke}%
\end{pgfscope}%
\begin{pgfscope}%
\pgfsetbuttcap%
\pgfsetroundjoin%
\definecolor{currentfill}{rgb}{0.000000,0.000000,0.000000}%
\pgfsetfillcolor{currentfill}%
\pgfsetlinewidth{0.803000pt}%
\definecolor{currentstroke}{rgb}{0.000000,0.000000,0.000000}%
\pgfsetstrokecolor{currentstroke}%
\pgfsetdash{}{0pt}%
\pgfsys@defobject{currentmarker}{\pgfqpoint{-0.048611in}{0.000000in}}{\pgfqpoint{-0.000000in}{0.000000in}}{%
\pgfpathmoveto{\pgfqpoint{-0.000000in}{0.000000in}}%
\pgfpathlineto{\pgfqpoint{-0.048611in}{0.000000in}}%
\pgfusepath{stroke,fill}%
}%
\begin{pgfscope}%
\pgfsys@transformshift{1.000000in}{6.474347in}%
\pgfsys@useobject{currentmarker}{}%
\end{pgfscope}%
\end{pgfscope}%
\begin{pgfscope}%
\definecolor{textcolor}{rgb}{0.000000,0.000000,0.000000}%
\pgfsetstrokecolor{textcolor}%
\pgfsetfillcolor{textcolor}%
\pgftext[x=0.560215in, y=6.374327in, left, base]{\color{textcolor}\sffamily\fontsize{20.000000}{24.000000}\selectfont \(\displaystyle {1.2}\)}%
\end{pgfscope}%
\begin{pgfscope}%
\pgfpathrectangle{\pgfqpoint{1.000000in}{5.000000in}}{\pgfqpoint{8.500000in}{2.000000in}}%
\pgfusepath{clip}%
\pgfsetrectcap%
\pgfsetroundjoin%
\pgfsetlinewidth{0.803000pt}%
\definecolor{currentstroke}{rgb}{0.690196,0.690196,0.690196}%
\pgfsetstrokecolor{currentstroke}%
\pgfsetdash{}{0pt}%
\pgfpathmoveto{\pgfqpoint{1.000000in}{6.720071in}}%
\pgfpathlineto{\pgfqpoint{9.500000in}{6.720071in}}%
\pgfusepath{stroke}%
\end{pgfscope}%
\begin{pgfscope}%
\pgfsetbuttcap%
\pgfsetroundjoin%
\definecolor{currentfill}{rgb}{0.000000,0.000000,0.000000}%
\pgfsetfillcolor{currentfill}%
\pgfsetlinewidth{0.803000pt}%
\definecolor{currentstroke}{rgb}{0.000000,0.000000,0.000000}%
\pgfsetstrokecolor{currentstroke}%
\pgfsetdash{}{0pt}%
\pgfsys@defobject{currentmarker}{\pgfqpoint{-0.048611in}{0.000000in}}{\pgfqpoint{-0.000000in}{0.000000in}}{%
\pgfpathmoveto{\pgfqpoint{-0.000000in}{0.000000in}}%
\pgfpathlineto{\pgfqpoint{-0.048611in}{0.000000in}}%
\pgfusepath{stroke,fill}%
}%
\begin{pgfscope}%
\pgfsys@transformshift{1.000000in}{6.720071in}%
\pgfsys@useobject{currentmarker}{}%
\end{pgfscope}%
\end{pgfscope}%
\begin{pgfscope}%
\definecolor{textcolor}{rgb}{0.000000,0.000000,0.000000}%
\pgfsetstrokecolor{textcolor}%
\pgfsetfillcolor{textcolor}%
\pgftext[x=0.560215in, y=6.620052in, left, base]{\color{textcolor}\sffamily\fontsize{20.000000}{24.000000}\selectfont \(\displaystyle {1.4}\)}%
\end{pgfscope}%
\begin{pgfscope}%
\definecolor{textcolor}{rgb}{0.000000,0.000000,0.000000}%
\pgfsetstrokecolor{textcolor}%
\pgfsetfillcolor{textcolor}%
\pgftext[x=0.504660in,y=6.000000in,,bottom,rotate=90.000000]{\color{textcolor}\sffamily\fontsize{20.000000}{24.000000}\selectfont \(\displaystyle \mathrm{Charge}\)}%
\end{pgfscope}%
\begin{pgfscope}%
\pgfpathrectangle{\pgfqpoint{1.000000in}{5.000000in}}{\pgfqpoint{8.500000in}{2.000000in}}%
\pgfusepath{clip}%
\pgfsetbuttcap%
\pgfsetroundjoin%
\pgfsetlinewidth{2.007500pt}%
\definecolor{currentstroke}{rgb}{0.000000,0.000000,0.000000}%
\pgfsetstrokecolor{currentstroke}%
\pgfsetdash{}{0pt}%
\pgfpathmoveto{\pgfqpoint{2.781081in}{5.000000in}}%
\pgfpathlineto{\pgfqpoint{2.781081in}{6.433962in}}%
\pgfusepath{stroke}%
\end{pgfscope}%
\begin{pgfscope}%
\pgfpathrectangle{\pgfqpoint{1.000000in}{5.000000in}}{\pgfqpoint{8.500000in}{2.000000in}}%
\pgfusepath{clip}%
\pgfsetbuttcap%
\pgfsetroundjoin%
\pgfsetlinewidth{2.007500pt}%
\definecolor{currentstroke}{rgb}{0.000000,0.000000,0.000000}%
\pgfsetstrokecolor{currentstroke}%
\pgfsetdash{}{0pt}%
\pgfpathmoveto{\pgfqpoint{3.072831in}{5.000000in}}%
\pgfpathlineto{\pgfqpoint{3.072831in}{6.818182in}}%
\pgfusepath{stroke}%
\end{pgfscope}%
\begin{pgfscope}%
\pgfpathrectangle{\pgfqpoint{1.000000in}{5.000000in}}{\pgfqpoint{8.500000in}{2.000000in}}%
\pgfusepath{clip}%
\pgfsetbuttcap%
\pgfsetroundjoin%
\pgfsetlinewidth{2.007500pt}%
\definecolor{currentstroke}{rgb}{0.000000,0.000000,0.000000}%
\pgfsetstrokecolor{currentstroke}%
\pgfsetdash{}{0pt}%
\pgfpathmoveto{\pgfqpoint{3.407148in}{5.000000in}}%
\pgfpathlineto{\pgfqpoint{3.407148in}{6.480365in}}%
\pgfusepath{stroke}%
\end{pgfscope}%
\begin{pgfscope}%
\pgfpathrectangle{\pgfqpoint{1.000000in}{5.000000in}}{\pgfqpoint{8.500000in}{2.000000in}}%
\pgfusepath{clip}%
\pgfsetbuttcap%
\pgfsetroundjoin%
\pgfsetlinewidth{2.007500pt}%
\definecolor{currentstroke}{rgb}{0.000000,0.000000,0.000000}%
\pgfsetstrokecolor{currentstroke}%
\pgfsetdash{}{0pt}%
\pgfpathmoveto{\pgfqpoint{3.567899in}{5.000000in}}%
\pgfpathlineto{\pgfqpoint{3.567899in}{6.158068in}}%
\pgfusepath{stroke}%
\end{pgfscope}%
\begin{pgfscope}%
\pgfpathrectangle{\pgfqpoint{1.000000in}{5.000000in}}{\pgfqpoint{8.500000in}{2.000000in}}%
\pgfusepath{clip}%
\pgfsetbuttcap%
\pgfsetroundjoin%
\pgfsetlinewidth{2.007500pt}%
\definecolor{currentstroke}{rgb}{0.000000,0.000000,0.000000}%
\pgfsetstrokecolor{currentstroke}%
\pgfsetdash{}{0pt}%
\pgfpathmoveto{\pgfqpoint{4.156756in}{5.000000in}}%
\pgfpathlineto{\pgfqpoint{4.156756in}{6.529536in}}%
\pgfusepath{stroke}%
\end{pgfscope}%
\begin{pgfscope}%
\pgfsetrectcap%
\pgfsetmiterjoin%
\pgfsetlinewidth{0.803000pt}%
\definecolor{currentstroke}{rgb}{0.000000,0.000000,0.000000}%
\pgfsetstrokecolor{currentstroke}%
\pgfsetdash{}{0pt}%
\pgfpathmoveto{\pgfqpoint{1.000000in}{5.000000in}}%
\pgfpathlineto{\pgfqpoint{1.000000in}{7.000000in}}%
\pgfusepath{stroke}%
\end{pgfscope}%
\begin{pgfscope}%
\pgfsetrectcap%
\pgfsetmiterjoin%
\pgfsetlinewidth{0.803000pt}%
\definecolor{currentstroke}{rgb}{0.000000,0.000000,0.000000}%
\pgfsetstrokecolor{currentstroke}%
\pgfsetdash{}{0pt}%
\pgfpathmoveto{\pgfqpoint{9.500000in}{5.000000in}}%
\pgfpathlineto{\pgfqpoint{9.500000in}{7.000000in}}%
\pgfusepath{stroke}%
\end{pgfscope}%
\begin{pgfscope}%
\pgfsetrectcap%
\pgfsetmiterjoin%
\pgfsetlinewidth{0.803000pt}%
\definecolor{currentstroke}{rgb}{0.000000,0.000000,0.000000}%
\pgfsetstrokecolor{currentstroke}%
\pgfsetdash{}{0pt}%
\pgfpathmoveto{\pgfqpoint{1.000000in}{5.000000in}}%
\pgfpathlineto{\pgfqpoint{9.500000in}{5.000000in}}%
\pgfusepath{stroke}%
\end{pgfscope}%
\begin{pgfscope}%
\pgfsetrectcap%
\pgfsetmiterjoin%
\pgfsetlinewidth{0.803000pt}%
\definecolor{currentstroke}{rgb}{0.000000,0.000000,0.000000}%
\pgfsetstrokecolor{currentstroke}%
\pgfsetdash{}{0pt}%
\pgfpathmoveto{\pgfqpoint{1.000000in}{7.000000in}}%
\pgfpathlineto{\pgfqpoint{9.500000in}{7.000000in}}%
\pgfusepath{stroke}%
\end{pgfscope}%
\begin{pgfscope}%
\pgfsetbuttcap%
\pgfsetmiterjoin%
\definecolor{currentfill}{rgb}{1.000000,1.000000,1.000000}%
\pgfsetfillcolor{currentfill}%
\pgfsetfillopacity{0.800000}%
\pgfsetlinewidth{1.003750pt}%
\definecolor{currentstroke}{rgb}{0.800000,0.800000,0.800000}%
\pgfsetstrokecolor{currentstroke}%
\pgfsetstrokeopacity{0.800000}%
\pgfsetdash{}{0pt}%
\pgfpathmoveto{\pgfqpoint{6.976817in}{6.382821in}}%
\pgfpathlineto{\pgfqpoint{9.305556in}{6.382821in}}%
\pgfpathquadraticcurveto{\pgfqpoint{9.361111in}{6.382821in}}{\pgfqpoint{9.361111in}{6.438377in}}%
\pgfpathlineto{\pgfqpoint{9.361111in}{6.805556in}}%
\pgfpathquadraticcurveto{\pgfqpoint{9.361111in}{6.861111in}}{\pgfqpoint{9.305556in}{6.861111in}}%
\pgfpathlineto{\pgfqpoint{6.976817in}{6.861111in}}%
\pgfpathquadraticcurveto{\pgfqpoint{6.921262in}{6.861111in}}{\pgfqpoint{6.921262in}{6.805556in}}%
\pgfpathlineto{\pgfqpoint{6.921262in}{6.438377in}}%
\pgfpathquadraticcurveto{\pgfqpoint{6.921262in}{6.382821in}}{\pgfqpoint{6.976817in}{6.382821in}}%
\pgfpathclose%
\pgfusepath{stroke,fill}%
\end{pgfscope}%
\begin{pgfscope}%
\pgfsetbuttcap%
\pgfsetroundjoin%
\pgfsetlinewidth{2.007500pt}%
\definecolor{currentstroke}{rgb}{0.000000,0.000000,0.000000}%
\pgfsetstrokecolor{currentstroke}%
\pgfsetdash{}{0pt}%
\pgfpathmoveto{\pgfqpoint{7.032373in}{6.647184in}}%
\pgfpathlineto{\pgfqpoint{7.587928in}{6.647184in}}%
\pgfusepath{stroke}%
\end{pgfscope}%
\begin{pgfscope}%
\definecolor{textcolor}{rgb}{0.000000,0.000000,0.000000}%
\pgfsetstrokecolor{textcolor}%
\pgfsetfillcolor{textcolor}%
\pgftext[x=7.810150in,y=6.549962in,left,base]{\color{textcolor}\sffamily\fontsize{20.000000}{24.000000}\selectfont truth Charge}%
\end{pgfscope}%
\begin{pgfscope}%
\pgfsetbuttcap%
\pgfsetmiterjoin%
\definecolor{currentfill}{rgb}{1.000000,1.000000,1.000000}%
\pgfsetfillcolor{currentfill}%
\pgfsetlinewidth{0.000000pt}%
\definecolor{currentstroke}{rgb}{0.000000,0.000000,0.000000}%
\pgfsetstrokecolor{currentstroke}%
\pgfsetstrokeopacity{0.000000}%
\pgfsetdash{}{0pt}%
\pgfpathmoveto{\pgfqpoint{1.000000in}{7.000000in}}%
\pgfpathlineto{\pgfqpoint{9.500000in}{7.000000in}}%
\pgfpathlineto{\pgfqpoint{9.500000in}{9.000000in}}%
\pgfpathlineto{\pgfqpoint{1.000000in}{9.000000in}}%
\pgfpathclose%
\pgfusepath{fill}%
\end{pgfscope}%
\begin{pgfscope}%
\pgfpathrectangle{\pgfqpoint{1.000000in}{7.000000in}}{\pgfqpoint{8.500000in}{2.000000in}}%
\pgfusepath{clip}%
\pgfsetrectcap%
\pgfsetroundjoin%
\pgfsetlinewidth{0.803000pt}%
\definecolor{currentstroke}{rgb}{0.690196,0.690196,0.690196}%
\pgfsetstrokecolor{currentstroke}%
\pgfsetdash{}{0pt}%
\pgfpathmoveto{\pgfqpoint{2.725210in}{7.000000in}}%
\pgfpathlineto{\pgfqpoint{2.725210in}{9.000000in}}%
\pgfusepath{stroke}%
\end{pgfscope}%
\begin{pgfscope}%
\pgfsetbuttcap%
\pgfsetroundjoin%
\definecolor{currentfill}{rgb}{0.000000,0.000000,0.000000}%
\pgfsetfillcolor{currentfill}%
\pgfsetlinewidth{0.803000pt}%
\definecolor{currentstroke}{rgb}{0.000000,0.000000,0.000000}%
\pgfsetstrokecolor{currentstroke}%
\pgfsetdash{}{0pt}%
\pgfsys@defobject{currentmarker}{\pgfqpoint{0.000000in}{-0.048611in}}{\pgfqpoint{0.000000in}{0.000000in}}{%
\pgfpathmoveto{\pgfqpoint{0.000000in}{0.000000in}}%
\pgfpathlineto{\pgfqpoint{0.000000in}{-0.048611in}}%
\pgfusepath{stroke,fill}%
}%
\begin{pgfscope}%
\pgfsys@transformshift{2.725210in}{7.000000in}%
\pgfsys@useobject{currentmarker}{}%
\end{pgfscope}%
\end{pgfscope}%
\begin{pgfscope}%
\pgfpathrectangle{\pgfqpoint{1.000000in}{7.000000in}}{\pgfqpoint{8.500000in}{2.000000in}}%
\pgfusepath{clip}%
\pgfsetrectcap%
\pgfsetroundjoin%
\pgfsetlinewidth{0.803000pt}%
\definecolor{currentstroke}{rgb}{0.690196,0.690196,0.690196}%
\pgfsetstrokecolor{currentstroke}%
\pgfsetdash{}{0pt}%
\pgfpathmoveto{\pgfqpoint{4.506292in}{7.000000in}}%
\pgfpathlineto{\pgfqpoint{4.506292in}{9.000000in}}%
\pgfusepath{stroke}%
\end{pgfscope}%
\begin{pgfscope}%
\pgfsetbuttcap%
\pgfsetroundjoin%
\definecolor{currentfill}{rgb}{0.000000,0.000000,0.000000}%
\pgfsetfillcolor{currentfill}%
\pgfsetlinewidth{0.803000pt}%
\definecolor{currentstroke}{rgb}{0.000000,0.000000,0.000000}%
\pgfsetstrokecolor{currentstroke}%
\pgfsetdash{}{0pt}%
\pgfsys@defobject{currentmarker}{\pgfqpoint{0.000000in}{-0.048611in}}{\pgfqpoint{0.000000in}{0.000000in}}{%
\pgfpathmoveto{\pgfqpoint{0.000000in}{0.000000in}}%
\pgfpathlineto{\pgfqpoint{0.000000in}{-0.048611in}}%
\pgfusepath{stroke,fill}%
}%
\begin{pgfscope}%
\pgfsys@transformshift{4.506292in}{7.000000in}%
\pgfsys@useobject{currentmarker}{}%
\end{pgfscope}%
\end{pgfscope}%
\begin{pgfscope}%
\pgfpathrectangle{\pgfqpoint{1.000000in}{7.000000in}}{\pgfqpoint{8.500000in}{2.000000in}}%
\pgfusepath{clip}%
\pgfsetrectcap%
\pgfsetroundjoin%
\pgfsetlinewidth{0.803000pt}%
\definecolor{currentstroke}{rgb}{0.690196,0.690196,0.690196}%
\pgfsetstrokecolor{currentstroke}%
\pgfsetdash{}{0pt}%
\pgfpathmoveto{\pgfqpoint{6.287373in}{7.000000in}}%
\pgfpathlineto{\pgfqpoint{6.287373in}{9.000000in}}%
\pgfusepath{stroke}%
\end{pgfscope}%
\begin{pgfscope}%
\pgfsetbuttcap%
\pgfsetroundjoin%
\definecolor{currentfill}{rgb}{0.000000,0.000000,0.000000}%
\pgfsetfillcolor{currentfill}%
\pgfsetlinewidth{0.803000pt}%
\definecolor{currentstroke}{rgb}{0.000000,0.000000,0.000000}%
\pgfsetstrokecolor{currentstroke}%
\pgfsetdash{}{0pt}%
\pgfsys@defobject{currentmarker}{\pgfqpoint{0.000000in}{-0.048611in}}{\pgfqpoint{0.000000in}{0.000000in}}{%
\pgfpathmoveto{\pgfqpoint{0.000000in}{0.000000in}}%
\pgfpathlineto{\pgfqpoint{0.000000in}{-0.048611in}}%
\pgfusepath{stroke,fill}%
}%
\begin{pgfscope}%
\pgfsys@transformshift{6.287373in}{7.000000in}%
\pgfsys@useobject{currentmarker}{}%
\end{pgfscope}%
\end{pgfscope}%
\begin{pgfscope}%
\pgfpathrectangle{\pgfqpoint{1.000000in}{7.000000in}}{\pgfqpoint{8.500000in}{2.000000in}}%
\pgfusepath{clip}%
\pgfsetrectcap%
\pgfsetroundjoin%
\pgfsetlinewidth{0.803000pt}%
\definecolor{currentstroke}{rgb}{0.690196,0.690196,0.690196}%
\pgfsetstrokecolor{currentstroke}%
\pgfsetdash{}{0pt}%
\pgfpathmoveto{\pgfqpoint{8.068455in}{7.000000in}}%
\pgfpathlineto{\pgfqpoint{8.068455in}{9.000000in}}%
\pgfusepath{stroke}%
\end{pgfscope}%
\begin{pgfscope}%
\pgfsetbuttcap%
\pgfsetroundjoin%
\definecolor{currentfill}{rgb}{0.000000,0.000000,0.000000}%
\pgfsetfillcolor{currentfill}%
\pgfsetlinewidth{0.803000pt}%
\definecolor{currentstroke}{rgb}{0.000000,0.000000,0.000000}%
\pgfsetstrokecolor{currentstroke}%
\pgfsetdash{}{0pt}%
\pgfsys@defobject{currentmarker}{\pgfqpoint{0.000000in}{-0.048611in}}{\pgfqpoint{0.000000in}{0.000000in}}{%
\pgfpathmoveto{\pgfqpoint{0.000000in}{0.000000in}}%
\pgfpathlineto{\pgfqpoint{0.000000in}{-0.048611in}}%
\pgfusepath{stroke,fill}%
}%
\begin{pgfscope}%
\pgfsys@transformshift{8.068455in}{7.000000in}%
\pgfsys@useobject{currentmarker}{}%
\end{pgfscope}%
\end{pgfscope}%
\begin{pgfscope}%
\pgfpathrectangle{\pgfqpoint{1.000000in}{7.000000in}}{\pgfqpoint{8.500000in}{2.000000in}}%
\pgfusepath{clip}%
\pgfsetrectcap%
\pgfsetroundjoin%
\pgfsetlinewidth{0.803000pt}%
\definecolor{currentstroke}{rgb}{0.690196,0.690196,0.690196}%
\pgfsetstrokecolor{currentstroke}%
\pgfsetdash{}{0pt}%
\pgfpathmoveto{\pgfqpoint{1.000000in}{7.000000in}}%
\pgfpathlineto{\pgfqpoint{9.500000in}{7.000000in}}%
\pgfusepath{stroke}%
\end{pgfscope}%
\begin{pgfscope}%
\pgfsetbuttcap%
\pgfsetroundjoin%
\definecolor{currentfill}{rgb}{0.000000,0.000000,0.000000}%
\pgfsetfillcolor{currentfill}%
\pgfsetlinewidth{0.803000pt}%
\definecolor{currentstroke}{rgb}{0.000000,0.000000,0.000000}%
\pgfsetstrokecolor{currentstroke}%
\pgfsetdash{}{0pt}%
\pgfsys@defobject{currentmarker}{\pgfqpoint{-0.048611in}{0.000000in}}{\pgfqpoint{-0.000000in}{0.000000in}}{%
\pgfpathmoveto{\pgfqpoint{-0.000000in}{0.000000in}}%
\pgfpathlineto{\pgfqpoint{-0.048611in}{0.000000in}}%
\pgfusepath{stroke,fill}%
}%
\begin{pgfscope}%
\pgfsys@transformshift{1.000000in}{7.000000in}%
\pgfsys@useobject{currentmarker}{}%
\end{pgfscope}%
\end{pgfscope}%
\begin{pgfscope}%
\definecolor{textcolor}{rgb}{0.000000,0.000000,0.000000}%
\pgfsetstrokecolor{textcolor}%
\pgfsetfillcolor{textcolor}%
\pgftext[x=0.560215in, y=6.899981in, left, base]{\color{textcolor}\sffamily\fontsize{20.000000}{24.000000}\selectfont \(\displaystyle {0.0}\)}%
\end{pgfscope}%
\begin{pgfscope}%
\pgfpathrectangle{\pgfqpoint{1.000000in}{7.000000in}}{\pgfqpoint{8.500000in}{2.000000in}}%
\pgfusepath{clip}%
\pgfsetrectcap%
\pgfsetroundjoin%
\pgfsetlinewidth{0.803000pt}%
\definecolor{currentstroke}{rgb}{0.690196,0.690196,0.690196}%
\pgfsetstrokecolor{currentstroke}%
\pgfsetdash{}{0pt}%
\pgfpathmoveto{\pgfqpoint{1.000000in}{7.245724in}}%
\pgfpathlineto{\pgfqpoint{9.500000in}{7.245724in}}%
\pgfusepath{stroke}%
\end{pgfscope}%
\begin{pgfscope}%
\pgfsetbuttcap%
\pgfsetroundjoin%
\definecolor{currentfill}{rgb}{0.000000,0.000000,0.000000}%
\pgfsetfillcolor{currentfill}%
\pgfsetlinewidth{0.803000pt}%
\definecolor{currentstroke}{rgb}{0.000000,0.000000,0.000000}%
\pgfsetstrokecolor{currentstroke}%
\pgfsetdash{}{0pt}%
\pgfsys@defobject{currentmarker}{\pgfqpoint{-0.048611in}{0.000000in}}{\pgfqpoint{-0.000000in}{0.000000in}}{%
\pgfpathmoveto{\pgfqpoint{-0.000000in}{0.000000in}}%
\pgfpathlineto{\pgfqpoint{-0.048611in}{0.000000in}}%
\pgfusepath{stroke,fill}%
}%
\begin{pgfscope}%
\pgfsys@transformshift{1.000000in}{7.245724in}%
\pgfsys@useobject{currentmarker}{}%
\end{pgfscope}%
\end{pgfscope}%
\begin{pgfscope}%
\definecolor{textcolor}{rgb}{0.000000,0.000000,0.000000}%
\pgfsetstrokecolor{textcolor}%
\pgfsetfillcolor{textcolor}%
\pgftext[x=0.560215in, y=7.145705in, left, base]{\color{textcolor}\sffamily\fontsize{20.000000}{24.000000}\selectfont \(\displaystyle {0.2}\)}%
\end{pgfscope}%
\begin{pgfscope}%
\pgfpathrectangle{\pgfqpoint{1.000000in}{7.000000in}}{\pgfqpoint{8.500000in}{2.000000in}}%
\pgfusepath{clip}%
\pgfsetrectcap%
\pgfsetroundjoin%
\pgfsetlinewidth{0.803000pt}%
\definecolor{currentstroke}{rgb}{0.690196,0.690196,0.690196}%
\pgfsetstrokecolor{currentstroke}%
\pgfsetdash{}{0pt}%
\pgfpathmoveto{\pgfqpoint{1.000000in}{7.491449in}}%
\pgfpathlineto{\pgfqpoint{9.500000in}{7.491449in}}%
\pgfusepath{stroke}%
\end{pgfscope}%
\begin{pgfscope}%
\pgfsetbuttcap%
\pgfsetroundjoin%
\definecolor{currentfill}{rgb}{0.000000,0.000000,0.000000}%
\pgfsetfillcolor{currentfill}%
\pgfsetlinewidth{0.803000pt}%
\definecolor{currentstroke}{rgb}{0.000000,0.000000,0.000000}%
\pgfsetstrokecolor{currentstroke}%
\pgfsetdash{}{0pt}%
\pgfsys@defobject{currentmarker}{\pgfqpoint{-0.048611in}{0.000000in}}{\pgfqpoint{-0.000000in}{0.000000in}}{%
\pgfpathmoveto{\pgfqpoint{-0.000000in}{0.000000in}}%
\pgfpathlineto{\pgfqpoint{-0.048611in}{0.000000in}}%
\pgfusepath{stroke,fill}%
}%
\begin{pgfscope}%
\pgfsys@transformshift{1.000000in}{7.491449in}%
\pgfsys@useobject{currentmarker}{}%
\end{pgfscope}%
\end{pgfscope}%
\begin{pgfscope}%
\definecolor{textcolor}{rgb}{0.000000,0.000000,0.000000}%
\pgfsetstrokecolor{textcolor}%
\pgfsetfillcolor{textcolor}%
\pgftext[x=0.560215in, y=7.391430in, left, base]{\color{textcolor}\sffamily\fontsize{20.000000}{24.000000}\selectfont \(\displaystyle {0.4}\)}%
\end{pgfscope}%
\begin{pgfscope}%
\pgfpathrectangle{\pgfqpoint{1.000000in}{7.000000in}}{\pgfqpoint{8.500000in}{2.000000in}}%
\pgfusepath{clip}%
\pgfsetrectcap%
\pgfsetroundjoin%
\pgfsetlinewidth{0.803000pt}%
\definecolor{currentstroke}{rgb}{0.690196,0.690196,0.690196}%
\pgfsetstrokecolor{currentstroke}%
\pgfsetdash{}{0pt}%
\pgfpathmoveto{\pgfqpoint{1.000000in}{7.737173in}}%
\pgfpathlineto{\pgfqpoint{9.500000in}{7.737173in}}%
\pgfusepath{stroke}%
\end{pgfscope}%
\begin{pgfscope}%
\pgfsetbuttcap%
\pgfsetroundjoin%
\definecolor{currentfill}{rgb}{0.000000,0.000000,0.000000}%
\pgfsetfillcolor{currentfill}%
\pgfsetlinewidth{0.803000pt}%
\definecolor{currentstroke}{rgb}{0.000000,0.000000,0.000000}%
\pgfsetstrokecolor{currentstroke}%
\pgfsetdash{}{0pt}%
\pgfsys@defobject{currentmarker}{\pgfqpoint{-0.048611in}{0.000000in}}{\pgfqpoint{-0.000000in}{0.000000in}}{%
\pgfpathmoveto{\pgfqpoint{-0.000000in}{0.000000in}}%
\pgfpathlineto{\pgfqpoint{-0.048611in}{0.000000in}}%
\pgfusepath{stroke,fill}%
}%
\begin{pgfscope}%
\pgfsys@transformshift{1.000000in}{7.737173in}%
\pgfsys@useobject{currentmarker}{}%
\end{pgfscope}%
\end{pgfscope}%
\begin{pgfscope}%
\definecolor{textcolor}{rgb}{0.000000,0.000000,0.000000}%
\pgfsetstrokecolor{textcolor}%
\pgfsetfillcolor{textcolor}%
\pgftext[x=0.560215in, y=7.637154in, left, base]{\color{textcolor}\sffamily\fontsize{20.000000}{24.000000}\selectfont \(\displaystyle {0.6}\)}%
\end{pgfscope}%
\begin{pgfscope}%
\pgfpathrectangle{\pgfqpoint{1.000000in}{7.000000in}}{\pgfqpoint{8.500000in}{2.000000in}}%
\pgfusepath{clip}%
\pgfsetrectcap%
\pgfsetroundjoin%
\pgfsetlinewidth{0.803000pt}%
\definecolor{currentstroke}{rgb}{0.690196,0.690196,0.690196}%
\pgfsetstrokecolor{currentstroke}%
\pgfsetdash{}{0pt}%
\pgfpathmoveto{\pgfqpoint{1.000000in}{7.982898in}}%
\pgfpathlineto{\pgfqpoint{9.500000in}{7.982898in}}%
\pgfusepath{stroke}%
\end{pgfscope}%
\begin{pgfscope}%
\pgfsetbuttcap%
\pgfsetroundjoin%
\definecolor{currentfill}{rgb}{0.000000,0.000000,0.000000}%
\pgfsetfillcolor{currentfill}%
\pgfsetlinewidth{0.803000pt}%
\definecolor{currentstroke}{rgb}{0.000000,0.000000,0.000000}%
\pgfsetstrokecolor{currentstroke}%
\pgfsetdash{}{0pt}%
\pgfsys@defobject{currentmarker}{\pgfqpoint{-0.048611in}{0.000000in}}{\pgfqpoint{-0.000000in}{0.000000in}}{%
\pgfpathmoveto{\pgfqpoint{-0.000000in}{0.000000in}}%
\pgfpathlineto{\pgfqpoint{-0.048611in}{0.000000in}}%
\pgfusepath{stroke,fill}%
}%
\begin{pgfscope}%
\pgfsys@transformshift{1.000000in}{7.982898in}%
\pgfsys@useobject{currentmarker}{}%
\end{pgfscope}%
\end{pgfscope}%
\begin{pgfscope}%
\definecolor{textcolor}{rgb}{0.000000,0.000000,0.000000}%
\pgfsetstrokecolor{textcolor}%
\pgfsetfillcolor{textcolor}%
\pgftext[x=0.560215in, y=7.882879in, left, base]{\color{textcolor}\sffamily\fontsize{20.000000}{24.000000}\selectfont \(\displaystyle {0.8}\)}%
\end{pgfscope}%
\begin{pgfscope}%
\pgfpathrectangle{\pgfqpoint{1.000000in}{7.000000in}}{\pgfqpoint{8.500000in}{2.000000in}}%
\pgfusepath{clip}%
\pgfsetrectcap%
\pgfsetroundjoin%
\pgfsetlinewidth{0.803000pt}%
\definecolor{currentstroke}{rgb}{0.690196,0.690196,0.690196}%
\pgfsetstrokecolor{currentstroke}%
\pgfsetdash{}{0pt}%
\pgfpathmoveto{\pgfqpoint{1.000000in}{8.228622in}}%
\pgfpathlineto{\pgfqpoint{9.500000in}{8.228622in}}%
\pgfusepath{stroke}%
\end{pgfscope}%
\begin{pgfscope}%
\pgfsetbuttcap%
\pgfsetroundjoin%
\definecolor{currentfill}{rgb}{0.000000,0.000000,0.000000}%
\pgfsetfillcolor{currentfill}%
\pgfsetlinewidth{0.803000pt}%
\definecolor{currentstroke}{rgb}{0.000000,0.000000,0.000000}%
\pgfsetstrokecolor{currentstroke}%
\pgfsetdash{}{0pt}%
\pgfsys@defobject{currentmarker}{\pgfqpoint{-0.048611in}{0.000000in}}{\pgfqpoint{-0.000000in}{0.000000in}}{%
\pgfpathmoveto{\pgfqpoint{-0.000000in}{0.000000in}}%
\pgfpathlineto{\pgfqpoint{-0.048611in}{0.000000in}}%
\pgfusepath{stroke,fill}%
}%
\begin{pgfscope}%
\pgfsys@transformshift{1.000000in}{8.228622in}%
\pgfsys@useobject{currentmarker}{}%
\end{pgfscope}%
\end{pgfscope}%
\begin{pgfscope}%
\definecolor{textcolor}{rgb}{0.000000,0.000000,0.000000}%
\pgfsetstrokecolor{textcolor}%
\pgfsetfillcolor{textcolor}%
\pgftext[x=0.560215in, y=8.128603in, left, base]{\color{textcolor}\sffamily\fontsize{20.000000}{24.000000}\selectfont \(\displaystyle {1.0}\)}%
\end{pgfscope}%
\begin{pgfscope}%
\pgfpathrectangle{\pgfqpoint{1.000000in}{7.000000in}}{\pgfqpoint{8.500000in}{2.000000in}}%
\pgfusepath{clip}%
\pgfsetrectcap%
\pgfsetroundjoin%
\pgfsetlinewidth{0.803000pt}%
\definecolor{currentstroke}{rgb}{0.690196,0.690196,0.690196}%
\pgfsetstrokecolor{currentstroke}%
\pgfsetdash{}{0pt}%
\pgfpathmoveto{\pgfqpoint{1.000000in}{8.474347in}}%
\pgfpathlineto{\pgfqpoint{9.500000in}{8.474347in}}%
\pgfusepath{stroke}%
\end{pgfscope}%
\begin{pgfscope}%
\pgfsetbuttcap%
\pgfsetroundjoin%
\definecolor{currentfill}{rgb}{0.000000,0.000000,0.000000}%
\pgfsetfillcolor{currentfill}%
\pgfsetlinewidth{0.803000pt}%
\definecolor{currentstroke}{rgb}{0.000000,0.000000,0.000000}%
\pgfsetstrokecolor{currentstroke}%
\pgfsetdash{}{0pt}%
\pgfsys@defobject{currentmarker}{\pgfqpoint{-0.048611in}{0.000000in}}{\pgfqpoint{-0.000000in}{0.000000in}}{%
\pgfpathmoveto{\pgfqpoint{-0.000000in}{0.000000in}}%
\pgfpathlineto{\pgfqpoint{-0.048611in}{0.000000in}}%
\pgfusepath{stroke,fill}%
}%
\begin{pgfscope}%
\pgfsys@transformshift{1.000000in}{8.474347in}%
\pgfsys@useobject{currentmarker}{}%
\end{pgfscope}%
\end{pgfscope}%
\begin{pgfscope}%
\definecolor{textcolor}{rgb}{0.000000,0.000000,0.000000}%
\pgfsetstrokecolor{textcolor}%
\pgfsetfillcolor{textcolor}%
\pgftext[x=0.560215in, y=8.374327in, left, base]{\color{textcolor}\sffamily\fontsize{20.000000}{24.000000}\selectfont \(\displaystyle {1.2}\)}%
\end{pgfscope}%
\begin{pgfscope}%
\pgfpathrectangle{\pgfqpoint{1.000000in}{7.000000in}}{\pgfqpoint{8.500000in}{2.000000in}}%
\pgfusepath{clip}%
\pgfsetrectcap%
\pgfsetroundjoin%
\pgfsetlinewidth{0.803000pt}%
\definecolor{currentstroke}{rgb}{0.690196,0.690196,0.690196}%
\pgfsetstrokecolor{currentstroke}%
\pgfsetdash{}{0pt}%
\pgfpathmoveto{\pgfqpoint{1.000000in}{8.720071in}}%
\pgfpathlineto{\pgfqpoint{9.500000in}{8.720071in}}%
\pgfusepath{stroke}%
\end{pgfscope}%
\begin{pgfscope}%
\pgfsetbuttcap%
\pgfsetroundjoin%
\definecolor{currentfill}{rgb}{0.000000,0.000000,0.000000}%
\pgfsetfillcolor{currentfill}%
\pgfsetlinewidth{0.803000pt}%
\definecolor{currentstroke}{rgb}{0.000000,0.000000,0.000000}%
\pgfsetstrokecolor{currentstroke}%
\pgfsetdash{}{0pt}%
\pgfsys@defobject{currentmarker}{\pgfqpoint{-0.048611in}{0.000000in}}{\pgfqpoint{-0.000000in}{0.000000in}}{%
\pgfpathmoveto{\pgfqpoint{-0.000000in}{0.000000in}}%
\pgfpathlineto{\pgfqpoint{-0.048611in}{0.000000in}}%
\pgfusepath{stroke,fill}%
}%
\begin{pgfscope}%
\pgfsys@transformshift{1.000000in}{8.720071in}%
\pgfsys@useobject{currentmarker}{}%
\end{pgfscope}%
\end{pgfscope}%
\begin{pgfscope}%
\definecolor{textcolor}{rgb}{0.000000,0.000000,0.000000}%
\pgfsetstrokecolor{textcolor}%
\pgfsetfillcolor{textcolor}%
\pgftext[x=0.560215in, y=8.620052in, left, base]{\color{textcolor}\sffamily\fontsize{20.000000}{24.000000}\selectfont \(\displaystyle {1.4}\)}%
\end{pgfscope}%
\begin{pgfscope}%
\definecolor{textcolor}{rgb}{0.000000,0.000000,0.000000}%
\pgfsetstrokecolor{textcolor}%
\pgfsetfillcolor{textcolor}%
\pgftext[x=0.504660in,y=8.000000in,,bottom,rotate=90.000000]{\color{textcolor}\sffamily\fontsize{20.000000}{24.000000}\selectfont \(\displaystyle \mathrm{Charge}\)}%
\end{pgfscope}%
\begin{pgfscope}%
\pgfpathrectangle{\pgfqpoint{1.000000in}{7.000000in}}{\pgfqpoint{8.500000in}{2.000000in}}%
\pgfusepath{clip}%
\pgfsetbuttcap%
\pgfsetroundjoin%
\pgfsetlinewidth{2.007500pt}%
\definecolor{currentstroke}{rgb}{0.000000,0.500000,0.000000}%
\pgfsetstrokecolor{currentstroke}%
\pgfsetdash{}{0pt}%
\pgfpathmoveto{\pgfqpoint{2.760832in}{7.000000in}}%
\pgfpathlineto{\pgfqpoint{2.760832in}{8.347208in}}%
\pgfusepath{stroke}%
\end{pgfscope}%
\begin{pgfscope}%
\pgfpathrectangle{\pgfqpoint{1.000000in}{7.000000in}}{\pgfqpoint{8.500000in}{2.000000in}}%
\pgfusepath{clip}%
\pgfsetbuttcap%
\pgfsetroundjoin%
\pgfsetlinewidth{2.007500pt}%
\definecolor{currentstroke}{rgb}{0.000000,0.500000,0.000000}%
\pgfsetstrokecolor{currentstroke}%
\pgfsetdash{}{0pt}%
\pgfpathmoveto{\pgfqpoint{3.045805in}{7.000000in}}%
\pgfpathlineto{\pgfqpoint{3.045805in}{8.054271in}}%
\pgfusepath{stroke}%
\end{pgfscope}%
\begin{pgfscope}%
\pgfpathrectangle{\pgfqpoint{1.000000in}{7.000000in}}{\pgfqpoint{8.500000in}{2.000000in}}%
\pgfusepath{clip}%
\pgfsetbuttcap%
\pgfsetroundjoin%
\pgfsetlinewidth{2.007500pt}%
\definecolor{currentstroke}{rgb}{0.000000,0.500000,0.000000}%
\pgfsetstrokecolor{currentstroke}%
\pgfsetdash{}{0pt}%
\pgfpathmoveto{\pgfqpoint{3.081427in}{7.000000in}}%
\pgfpathlineto{\pgfqpoint{3.081427in}{7.857574in}}%
\pgfusepath{stroke}%
\end{pgfscope}%
\begin{pgfscope}%
\pgfpathrectangle{\pgfqpoint{1.000000in}{7.000000in}}{\pgfqpoint{8.500000in}{2.000000in}}%
\pgfusepath{clip}%
\pgfsetbuttcap%
\pgfsetroundjoin%
\pgfsetlinewidth{2.007500pt}%
\definecolor{currentstroke}{rgb}{0.000000,0.500000,0.000000}%
\pgfsetstrokecolor{currentstroke}%
\pgfsetdash{}{0pt}%
\pgfpathmoveto{\pgfqpoint{3.402021in}{7.000000in}}%
\pgfpathlineto{\pgfqpoint{3.402021in}{8.520901in}}%
\pgfusepath{stroke}%
\end{pgfscope}%
\begin{pgfscope}%
\pgfpathrectangle{\pgfqpoint{1.000000in}{7.000000in}}{\pgfqpoint{8.500000in}{2.000000in}}%
\pgfusepath{clip}%
\pgfsetbuttcap%
\pgfsetroundjoin%
\pgfsetlinewidth{2.007500pt}%
\definecolor{currentstroke}{rgb}{0.000000,0.500000,0.000000}%
\pgfsetstrokecolor{currentstroke}%
\pgfsetdash{}{0pt}%
\pgfpathmoveto{\pgfqpoint{3.580129in}{7.000000in}}%
\pgfpathlineto{\pgfqpoint{3.580129in}{8.156271in}}%
\pgfusepath{stroke}%
\end{pgfscope}%
\begin{pgfscope}%
\pgfpathrectangle{\pgfqpoint{1.000000in}{7.000000in}}{\pgfqpoint{8.500000in}{2.000000in}}%
\pgfusepath{clip}%
\pgfsetbuttcap%
\pgfsetroundjoin%
\pgfsetlinewidth{2.007500pt}%
\definecolor{currentstroke}{rgb}{0.000000,0.500000,0.000000}%
\pgfsetstrokecolor{currentstroke}%
\pgfsetdash{}{0pt}%
\pgfpathmoveto{\pgfqpoint{4.150075in}{7.000000in}}%
\pgfpathlineto{\pgfqpoint{4.150075in}{8.539188in}}%
\pgfusepath{stroke}%
\end{pgfscope}%
\begin{pgfscope}%
\pgfsetrectcap%
\pgfsetmiterjoin%
\pgfsetlinewidth{0.803000pt}%
\definecolor{currentstroke}{rgb}{0.000000,0.000000,0.000000}%
\pgfsetstrokecolor{currentstroke}%
\pgfsetdash{}{0pt}%
\pgfpathmoveto{\pgfqpoint{1.000000in}{7.000000in}}%
\pgfpathlineto{\pgfqpoint{1.000000in}{9.000000in}}%
\pgfusepath{stroke}%
\end{pgfscope}%
\begin{pgfscope}%
\pgfsetrectcap%
\pgfsetmiterjoin%
\pgfsetlinewidth{0.803000pt}%
\definecolor{currentstroke}{rgb}{0.000000,0.000000,0.000000}%
\pgfsetstrokecolor{currentstroke}%
\pgfsetdash{}{0pt}%
\pgfpathmoveto{\pgfqpoint{9.500000in}{7.000000in}}%
\pgfpathlineto{\pgfqpoint{9.500000in}{9.000000in}}%
\pgfusepath{stroke}%
\end{pgfscope}%
\begin{pgfscope}%
\pgfsetrectcap%
\pgfsetmiterjoin%
\pgfsetlinewidth{0.803000pt}%
\definecolor{currentstroke}{rgb}{0.000000,0.000000,0.000000}%
\pgfsetstrokecolor{currentstroke}%
\pgfsetdash{}{0pt}%
\pgfpathmoveto{\pgfqpoint{1.000000in}{7.000000in}}%
\pgfpathlineto{\pgfqpoint{9.500000in}{7.000000in}}%
\pgfusepath{stroke}%
\end{pgfscope}%
\begin{pgfscope}%
\pgfsetrectcap%
\pgfsetmiterjoin%
\pgfsetlinewidth{0.803000pt}%
\definecolor{currentstroke}{rgb}{0.000000,0.000000,0.000000}%
\pgfsetstrokecolor{currentstroke}%
\pgfsetdash{}{0pt}%
\pgfpathmoveto{\pgfqpoint{1.000000in}{9.000000in}}%
\pgfpathlineto{\pgfqpoint{9.500000in}{9.000000in}}%
\pgfusepath{stroke}%
\end{pgfscope}%
\begin{pgfscope}%
\pgfsetbuttcap%
\pgfsetmiterjoin%
\definecolor{currentfill}{rgb}{1.000000,1.000000,1.000000}%
\pgfsetfillcolor{currentfill}%
\pgfsetfillopacity{0.800000}%
\pgfsetlinewidth{1.003750pt}%
\definecolor{currentstroke}{rgb}{0.800000,0.800000,0.800000}%
\pgfsetstrokecolor{currentstroke}%
\pgfsetstrokeopacity{0.800000}%
\pgfsetdash{}{0pt}%
\pgfpathmoveto{\pgfqpoint{6.935606in}{8.382821in}}%
\pgfpathlineto{\pgfqpoint{9.305556in}{8.382821in}}%
\pgfpathquadraticcurveto{\pgfqpoint{9.361111in}{8.382821in}}{\pgfqpoint{9.361111in}{8.438377in}}%
\pgfpathlineto{\pgfqpoint{9.361111in}{8.805556in}}%
\pgfpathquadraticcurveto{\pgfqpoint{9.361111in}{8.861111in}}{\pgfqpoint{9.305556in}{8.861111in}}%
\pgfpathlineto{\pgfqpoint{6.935606in}{8.861111in}}%
\pgfpathquadraticcurveto{\pgfqpoint{6.880050in}{8.861111in}}{\pgfqpoint{6.880050in}{8.805556in}}%
\pgfpathlineto{\pgfqpoint{6.880050in}{8.438377in}}%
\pgfpathquadraticcurveto{\pgfqpoint{6.880050in}{8.382821in}}{\pgfqpoint{6.935606in}{8.382821in}}%
\pgfpathclose%
\pgfusepath{stroke,fill}%
\end{pgfscope}%
\begin{pgfscope}%
\pgfsetbuttcap%
\pgfsetroundjoin%
\pgfsetlinewidth{2.007500pt}%
\definecolor{currentstroke}{rgb}{0.000000,0.500000,0.000000}%
\pgfsetstrokecolor{currentstroke}%
\pgfsetdash{}{0pt}%
\pgfpathmoveto{\pgfqpoint{6.991161in}{8.647184in}}%
\pgfpathlineto{\pgfqpoint{7.546717in}{8.647184in}}%
\pgfusepath{stroke}%
\end{pgfscope}%
\begin{pgfscope}%
\definecolor{textcolor}{rgb}{0.000000,0.000000,0.000000}%
\pgfsetstrokecolor{textcolor}%
\pgfsetfillcolor{textcolor}%
\pgftext[x=7.768939in,y=8.549962in,left,base]{\color{textcolor}\sffamily\fontsize{20.000000}{24.000000}\selectfont recon Charge}%
\end{pgfscope}%
\begin{pgfscope}%
\pgfsetbuttcap%
\pgfsetmiterjoin%
\definecolor{currentfill}{rgb}{1.000000,1.000000,1.000000}%
\pgfsetfillcolor{currentfill}%
\pgfsetlinewidth{0.000000pt}%
\definecolor{currentstroke}{rgb}{0.000000,0.000000,0.000000}%
\pgfsetstrokecolor{currentstroke}%
\pgfsetstrokeopacity{0.000000}%
\pgfsetdash{}{0pt}%
\pgfpathmoveto{\pgfqpoint{1.000000in}{1.000000in}}%
\pgfpathlineto{\pgfqpoint{9.500000in}{1.000000in}}%
\pgfpathlineto{\pgfqpoint{9.500000in}{2.000000in}}%
\pgfpathlineto{\pgfqpoint{1.000000in}{2.000000in}}%
\pgfpathclose%
\pgfusepath{fill}%
\end{pgfscope}%
\begin{pgfscope}%
\pgfpathrectangle{\pgfqpoint{1.000000in}{1.000000in}}{\pgfqpoint{8.500000in}{1.000000in}}%
\pgfusepath{clip}%
\pgfsetbuttcap%
\pgfsetroundjoin%
\definecolor{currentfill}{rgb}{0.000000,0.000000,0.000000}%
\pgfsetfillcolor{currentfill}%
\pgfsetlinewidth{1.003750pt}%
\definecolor{currentstroke}{rgb}{0.000000,0.000000,0.000000}%
\pgfsetstrokecolor{currentstroke}%
\pgfsetdash{}{0pt}%
\pgfsys@defobject{currentmarker}{\pgfqpoint{-0.013889in}{-0.013889in}}{\pgfqpoint{0.013889in}{0.013889in}}{%
\pgfpathmoveto{\pgfqpoint{0.000000in}{-0.013889in}}%
\pgfpathcurveto{\pgfqpoint{0.003683in}{-0.013889in}}{\pgfqpoint{0.007216in}{-0.012425in}}{\pgfqpoint{0.009821in}{-0.009821in}}%
\pgfpathcurveto{\pgfqpoint{0.012425in}{-0.007216in}}{\pgfqpoint{0.013889in}{-0.003683in}}{\pgfqpoint{0.013889in}{0.000000in}}%
\pgfpathcurveto{\pgfqpoint{0.013889in}{0.003683in}}{\pgfqpoint{0.012425in}{0.007216in}}{\pgfqpoint{0.009821in}{0.009821in}}%
\pgfpathcurveto{\pgfqpoint{0.007216in}{0.012425in}}{\pgfqpoint{0.003683in}{0.013889in}}{\pgfqpoint{0.000000in}{0.013889in}}%
\pgfpathcurveto{\pgfqpoint{-0.003683in}{0.013889in}}{\pgfqpoint{-0.007216in}{0.012425in}}{\pgfqpoint{-0.009821in}{0.009821in}}%
\pgfpathcurveto{\pgfqpoint{-0.012425in}{0.007216in}}{\pgfqpoint{-0.013889in}{0.003683in}}{\pgfqpoint{-0.013889in}{0.000000in}}%
\pgfpathcurveto{\pgfqpoint{-0.013889in}{-0.003683in}}{\pgfqpoint{-0.012425in}{-0.007216in}}{\pgfqpoint{-0.009821in}{-0.009821in}}%
\pgfpathcurveto{\pgfqpoint{-0.007216in}{-0.012425in}}{\pgfqpoint{-0.003683in}{-0.013889in}}{\pgfqpoint{0.000000in}{-0.013889in}}%
\pgfpathclose%
\pgfusepath{stroke,fill}%
}%
\begin{pgfscope}%
\pgfsys@transformshift{-15.085604in}{1.300000in}%
\pgfsys@useobject{currentmarker}{}%
\end{pgfscope}%
\begin{pgfscope}%
\pgfsys@transformshift{-15.049982in}{1.600000in}%
\pgfsys@useobject{currentmarker}{}%
\end{pgfscope}%
\begin{pgfscope}%
\pgfsys@transformshift{-15.014360in}{1.600000in}%
\pgfsys@useobject{currentmarker}{}%
\end{pgfscope}%
\begin{pgfscope}%
\pgfsys@transformshift{-14.978739in}{1.400000in}%
\pgfsys@useobject{currentmarker}{}%
\end{pgfscope}%
\begin{pgfscope}%
\pgfsys@transformshift{-14.943117in}{1.600000in}%
\pgfsys@useobject{currentmarker}{}%
\end{pgfscope}%
\begin{pgfscope}%
\pgfsys@transformshift{-14.907495in}{1.400000in}%
\pgfsys@useobject{currentmarker}{}%
\end{pgfscope}%
\begin{pgfscope}%
\pgfsys@transformshift{-14.871874in}{1.500000in}%
\pgfsys@useobject{currentmarker}{}%
\end{pgfscope}%
\begin{pgfscope}%
\pgfsys@transformshift{-14.836252in}{1.500000in}%
\pgfsys@useobject{currentmarker}{}%
\end{pgfscope}%
\begin{pgfscope}%
\pgfsys@transformshift{-14.800631in}{1.400000in}%
\pgfsys@useobject{currentmarker}{}%
\end{pgfscope}%
\begin{pgfscope}%
\pgfsys@transformshift{-14.765009in}{1.400000in}%
\pgfsys@useobject{currentmarker}{}%
\end{pgfscope}%
\begin{pgfscope}%
\pgfsys@transformshift{-14.729387in}{1.500000in}%
\pgfsys@useobject{currentmarker}{}%
\end{pgfscope}%
\begin{pgfscope}%
\pgfsys@transformshift{-14.693766in}{1.600000in}%
\pgfsys@useobject{currentmarker}{}%
\end{pgfscope}%
\begin{pgfscope}%
\pgfsys@transformshift{-14.658144in}{1.600000in}%
\pgfsys@useobject{currentmarker}{}%
\end{pgfscope}%
\begin{pgfscope}%
\pgfsys@transformshift{-14.622522in}{1.500000in}%
\pgfsys@useobject{currentmarker}{}%
\end{pgfscope}%
\begin{pgfscope}%
\pgfsys@transformshift{-14.586901in}{1.700000in}%
\pgfsys@useobject{currentmarker}{}%
\end{pgfscope}%
\begin{pgfscope}%
\pgfsys@transformshift{-14.551279in}{1.500000in}%
\pgfsys@useobject{currentmarker}{}%
\end{pgfscope}%
\begin{pgfscope}%
\pgfsys@transformshift{-14.515658in}{1.600000in}%
\pgfsys@useobject{currentmarker}{}%
\end{pgfscope}%
\begin{pgfscope}%
\pgfsys@transformshift{-14.480036in}{1.300000in}%
\pgfsys@useobject{currentmarker}{}%
\end{pgfscope}%
\begin{pgfscope}%
\pgfsys@transformshift{-14.444414in}{1.500000in}%
\pgfsys@useobject{currentmarker}{}%
\end{pgfscope}%
\begin{pgfscope}%
\pgfsys@transformshift{-14.408793in}{1.400000in}%
\pgfsys@useobject{currentmarker}{}%
\end{pgfscope}%
\begin{pgfscope}%
\pgfsys@transformshift{-14.373171in}{1.300000in}%
\pgfsys@useobject{currentmarker}{}%
\end{pgfscope}%
\begin{pgfscope}%
\pgfsys@transformshift{-14.337549in}{1.500000in}%
\pgfsys@useobject{currentmarker}{}%
\end{pgfscope}%
\begin{pgfscope}%
\pgfsys@transformshift{-14.301928in}{1.500000in}%
\pgfsys@useobject{currentmarker}{}%
\end{pgfscope}%
\begin{pgfscope}%
\pgfsys@transformshift{-14.266306in}{1.600000in}%
\pgfsys@useobject{currentmarker}{}%
\end{pgfscope}%
\begin{pgfscope}%
\pgfsys@transformshift{-14.230685in}{1.500000in}%
\pgfsys@useobject{currentmarker}{}%
\end{pgfscope}%
\begin{pgfscope}%
\pgfsys@transformshift{-14.195063in}{1.500000in}%
\pgfsys@useobject{currentmarker}{}%
\end{pgfscope}%
\begin{pgfscope}%
\pgfsys@transformshift{-14.159441in}{1.600000in}%
\pgfsys@useobject{currentmarker}{}%
\end{pgfscope}%
\begin{pgfscope}%
\pgfsys@transformshift{-14.123820in}{1.300000in}%
\pgfsys@useobject{currentmarker}{}%
\end{pgfscope}%
\begin{pgfscope}%
\pgfsys@transformshift{-14.088198in}{1.500000in}%
\pgfsys@useobject{currentmarker}{}%
\end{pgfscope}%
\begin{pgfscope}%
\pgfsys@transformshift{-14.052576in}{1.400000in}%
\pgfsys@useobject{currentmarker}{}%
\end{pgfscope}%
\begin{pgfscope}%
\pgfsys@transformshift{-14.016955in}{1.500000in}%
\pgfsys@useobject{currentmarker}{}%
\end{pgfscope}%
\begin{pgfscope}%
\pgfsys@transformshift{-13.981333in}{1.400000in}%
\pgfsys@useobject{currentmarker}{}%
\end{pgfscope}%
\begin{pgfscope}%
\pgfsys@transformshift{-13.945712in}{1.500000in}%
\pgfsys@useobject{currentmarker}{}%
\end{pgfscope}%
\begin{pgfscope}%
\pgfsys@transformshift{-13.910090in}{1.400000in}%
\pgfsys@useobject{currentmarker}{}%
\end{pgfscope}%
\begin{pgfscope}%
\pgfsys@transformshift{-13.874468in}{1.600000in}%
\pgfsys@useobject{currentmarker}{}%
\end{pgfscope}%
\begin{pgfscope}%
\pgfsys@transformshift{-13.838847in}{1.600000in}%
\pgfsys@useobject{currentmarker}{}%
\end{pgfscope}%
\begin{pgfscope}%
\pgfsys@transformshift{-13.803225in}{1.400000in}%
\pgfsys@useobject{currentmarker}{}%
\end{pgfscope}%
\begin{pgfscope}%
\pgfsys@transformshift{-13.767603in}{1.600000in}%
\pgfsys@useobject{currentmarker}{}%
\end{pgfscope}%
\begin{pgfscope}%
\pgfsys@transformshift{-13.731982in}{1.500000in}%
\pgfsys@useobject{currentmarker}{}%
\end{pgfscope}%
\begin{pgfscope}%
\pgfsys@transformshift{-13.696360in}{1.500000in}%
\pgfsys@useobject{currentmarker}{}%
\end{pgfscope}%
\begin{pgfscope}%
\pgfsys@transformshift{-13.660738in}{1.400000in}%
\pgfsys@useobject{currentmarker}{}%
\end{pgfscope}%
\begin{pgfscope}%
\pgfsys@transformshift{-13.625117in}{1.500000in}%
\pgfsys@useobject{currentmarker}{}%
\end{pgfscope}%
\begin{pgfscope}%
\pgfsys@transformshift{-13.589495in}{1.300000in}%
\pgfsys@useobject{currentmarker}{}%
\end{pgfscope}%
\begin{pgfscope}%
\pgfsys@transformshift{-13.553874in}{1.400000in}%
\pgfsys@useobject{currentmarker}{}%
\end{pgfscope}%
\begin{pgfscope}%
\pgfsys@transformshift{-13.518252in}{1.500000in}%
\pgfsys@useobject{currentmarker}{}%
\end{pgfscope}%
\begin{pgfscope}%
\pgfsys@transformshift{-13.482630in}{1.500000in}%
\pgfsys@useobject{currentmarker}{}%
\end{pgfscope}%
\begin{pgfscope}%
\pgfsys@transformshift{-13.447009in}{1.400000in}%
\pgfsys@useobject{currentmarker}{}%
\end{pgfscope}%
\begin{pgfscope}%
\pgfsys@transformshift{-13.411387in}{1.400000in}%
\pgfsys@useobject{currentmarker}{}%
\end{pgfscope}%
\begin{pgfscope}%
\pgfsys@transformshift{-13.375765in}{1.400000in}%
\pgfsys@useobject{currentmarker}{}%
\end{pgfscope}%
\begin{pgfscope}%
\pgfsys@transformshift{-13.340144in}{1.500000in}%
\pgfsys@useobject{currentmarker}{}%
\end{pgfscope}%
\begin{pgfscope}%
\pgfsys@transformshift{-13.304522in}{1.600000in}%
\pgfsys@useobject{currentmarker}{}%
\end{pgfscope}%
\begin{pgfscope}%
\pgfsys@transformshift{-13.268901in}{1.500000in}%
\pgfsys@useobject{currentmarker}{}%
\end{pgfscope}%
\begin{pgfscope}%
\pgfsys@transformshift{-13.233279in}{1.400000in}%
\pgfsys@useobject{currentmarker}{}%
\end{pgfscope}%
\begin{pgfscope}%
\pgfsys@transformshift{-13.197657in}{1.600000in}%
\pgfsys@useobject{currentmarker}{}%
\end{pgfscope}%
\begin{pgfscope}%
\pgfsys@transformshift{-13.162036in}{1.400000in}%
\pgfsys@useobject{currentmarker}{}%
\end{pgfscope}%
\begin{pgfscope}%
\pgfsys@transformshift{-13.126414in}{1.400000in}%
\pgfsys@useobject{currentmarker}{}%
\end{pgfscope}%
\begin{pgfscope}%
\pgfsys@transformshift{-13.090792in}{1.500000in}%
\pgfsys@useobject{currentmarker}{}%
\end{pgfscope}%
\begin{pgfscope}%
\pgfsys@transformshift{-13.055171in}{1.700000in}%
\pgfsys@useobject{currentmarker}{}%
\end{pgfscope}%
\begin{pgfscope}%
\pgfsys@transformshift{-13.019549in}{1.500000in}%
\pgfsys@useobject{currentmarker}{}%
\end{pgfscope}%
\begin{pgfscope}%
\pgfsys@transformshift{-12.983928in}{1.400000in}%
\pgfsys@useobject{currentmarker}{}%
\end{pgfscope}%
\begin{pgfscope}%
\pgfsys@transformshift{-12.948306in}{1.500000in}%
\pgfsys@useobject{currentmarker}{}%
\end{pgfscope}%
\begin{pgfscope}%
\pgfsys@transformshift{-12.912684in}{1.600000in}%
\pgfsys@useobject{currentmarker}{}%
\end{pgfscope}%
\begin{pgfscope}%
\pgfsys@transformshift{-12.877063in}{1.500000in}%
\pgfsys@useobject{currentmarker}{}%
\end{pgfscope}%
\begin{pgfscope}%
\pgfsys@transformshift{-12.841441in}{1.600000in}%
\pgfsys@useobject{currentmarker}{}%
\end{pgfscope}%
\begin{pgfscope}%
\pgfsys@transformshift{-12.805819in}{1.400000in}%
\pgfsys@useobject{currentmarker}{}%
\end{pgfscope}%
\begin{pgfscope}%
\pgfsys@transformshift{-12.770198in}{1.500000in}%
\pgfsys@useobject{currentmarker}{}%
\end{pgfscope}%
\begin{pgfscope}%
\pgfsys@transformshift{-12.734576in}{1.300000in}%
\pgfsys@useobject{currentmarker}{}%
\end{pgfscope}%
\begin{pgfscope}%
\pgfsys@transformshift{-12.698955in}{1.600000in}%
\pgfsys@useobject{currentmarker}{}%
\end{pgfscope}%
\begin{pgfscope}%
\pgfsys@transformshift{-12.663333in}{1.500000in}%
\pgfsys@useobject{currentmarker}{}%
\end{pgfscope}%
\begin{pgfscope}%
\pgfsys@transformshift{-12.627711in}{1.500000in}%
\pgfsys@useobject{currentmarker}{}%
\end{pgfscope}%
\begin{pgfscope}%
\pgfsys@transformshift{-12.592090in}{1.700000in}%
\pgfsys@useobject{currentmarker}{}%
\end{pgfscope}%
\begin{pgfscope}%
\pgfsys@transformshift{-12.556468in}{1.400000in}%
\pgfsys@useobject{currentmarker}{}%
\end{pgfscope}%
\begin{pgfscope}%
\pgfsys@transformshift{-12.520846in}{1.500000in}%
\pgfsys@useobject{currentmarker}{}%
\end{pgfscope}%
\begin{pgfscope}%
\pgfsys@transformshift{-12.485225in}{1.500000in}%
\pgfsys@useobject{currentmarker}{}%
\end{pgfscope}%
\begin{pgfscope}%
\pgfsys@transformshift{-12.449603in}{1.500000in}%
\pgfsys@useobject{currentmarker}{}%
\end{pgfscope}%
\begin{pgfscope}%
\pgfsys@transformshift{-12.413982in}{1.400000in}%
\pgfsys@useobject{currentmarker}{}%
\end{pgfscope}%
\begin{pgfscope}%
\pgfsys@transformshift{-12.378360in}{1.400000in}%
\pgfsys@useobject{currentmarker}{}%
\end{pgfscope}%
\begin{pgfscope}%
\pgfsys@transformshift{-12.342738in}{1.500000in}%
\pgfsys@useobject{currentmarker}{}%
\end{pgfscope}%
\begin{pgfscope}%
\pgfsys@transformshift{-12.307117in}{1.300000in}%
\pgfsys@useobject{currentmarker}{}%
\end{pgfscope}%
\begin{pgfscope}%
\pgfsys@transformshift{-12.271495in}{1.400000in}%
\pgfsys@useobject{currentmarker}{}%
\end{pgfscope}%
\begin{pgfscope}%
\pgfsys@transformshift{-12.235873in}{1.400000in}%
\pgfsys@useobject{currentmarker}{}%
\end{pgfscope}%
\begin{pgfscope}%
\pgfsys@transformshift{-12.200252in}{1.600000in}%
\pgfsys@useobject{currentmarker}{}%
\end{pgfscope}%
\begin{pgfscope}%
\pgfsys@transformshift{-12.164630in}{1.400000in}%
\pgfsys@useobject{currentmarker}{}%
\end{pgfscope}%
\begin{pgfscope}%
\pgfsys@transformshift{-12.129008in}{1.500000in}%
\pgfsys@useobject{currentmarker}{}%
\end{pgfscope}%
\begin{pgfscope}%
\pgfsys@transformshift{-12.093387in}{1.600000in}%
\pgfsys@useobject{currentmarker}{}%
\end{pgfscope}%
\begin{pgfscope}%
\pgfsys@transformshift{-12.057765in}{1.500000in}%
\pgfsys@useobject{currentmarker}{}%
\end{pgfscope}%
\begin{pgfscope}%
\pgfsys@transformshift{-12.022144in}{1.600000in}%
\pgfsys@useobject{currentmarker}{}%
\end{pgfscope}%
\begin{pgfscope}%
\pgfsys@transformshift{-11.986522in}{1.700000in}%
\pgfsys@useobject{currentmarker}{}%
\end{pgfscope}%
\begin{pgfscope}%
\pgfsys@transformshift{-11.950900in}{1.500000in}%
\pgfsys@useobject{currentmarker}{}%
\end{pgfscope}%
\begin{pgfscope}%
\pgfsys@transformshift{-11.915279in}{1.600000in}%
\pgfsys@useobject{currentmarker}{}%
\end{pgfscope}%
\begin{pgfscope}%
\pgfsys@transformshift{-11.879657in}{1.500000in}%
\pgfsys@useobject{currentmarker}{}%
\end{pgfscope}%
\begin{pgfscope}%
\pgfsys@transformshift{-11.844035in}{1.300000in}%
\pgfsys@useobject{currentmarker}{}%
\end{pgfscope}%
\begin{pgfscope}%
\pgfsys@transformshift{-11.808414in}{1.500000in}%
\pgfsys@useobject{currentmarker}{}%
\end{pgfscope}%
\begin{pgfscope}%
\pgfsys@transformshift{-11.772792in}{1.500000in}%
\pgfsys@useobject{currentmarker}{}%
\end{pgfscope}%
\begin{pgfscope}%
\pgfsys@transformshift{-11.737171in}{1.500000in}%
\pgfsys@useobject{currentmarker}{}%
\end{pgfscope}%
\begin{pgfscope}%
\pgfsys@transformshift{-11.701549in}{1.600000in}%
\pgfsys@useobject{currentmarker}{}%
\end{pgfscope}%
\begin{pgfscope}%
\pgfsys@transformshift{-11.665927in}{1.700000in}%
\pgfsys@useobject{currentmarker}{}%
\end{pgfscope}%
\begin{pgfscope}%
\pgfsys@transformshift{-11.630306in}{1.600000in}%
\pgfsys@useobject{currentmarker}{}%
\end{pgfscope}%
\begin{pgfscope}%
\pgfsys@transformshift{-11.594684in}{1.500000in}%
\pgfsys@useobject{currentmarker}{}%
\end{pgfscope}%
\begin{pgfscope}%
\pgfsys@transformshift{-11.559062in}{1.600000in}%
\pgfsys@useobject{currentmarker}{}%
\end{pgfscope}%
\begin{pgfscope}%
\pgfsys@transformshift{-11.523441in}{1.500000in}%
\pgfsys@useobject{currentmarker}{}%
\end{pgfscope}%
\begin{pgfscope}%
\pgfsys@transformshift{-11.487819in}{1.500000in}%
\pgfsys@useobject{currentmarker}{}%
\end{pgfscope}%
\begin{pgfscope}%
\pgfsys@transformshift{-11.452198in}{1.600000in}%
\pgfsys@useobject{currentmarker}{}%
\end{pgfscope}%
\begin{pgfscope}%
\pgfsys@transformshift{-11.416576in}{1.600000in}%
\pgfsys@useobject{currentmarker}{}%
\end{pgfscope}%
\begin{pgfscope}%
\pgfsys@transformshift{-11.380954in}{1.400000in}%
\pgfsys@useobject{currentmarker}{}%
\end{pgfscope}%
\begin{pgfscope}%
\pgfsys@transformshift{-11.345333in}{1.500000in}%
\pgfsys@useobject{currentmarker}{}%
\end{pgfscope}%
\begin{pgfscope}%
\pgfsys@transformshift{-11.309711in}{1.600000in}%
\pgfsys@useobject{currentmarker}{}%
\end{pgfscope}%
\begin{pgfscope}%
\pgfsys@transformshift{-11.274089in}{1.600000in}%
\pgfsys@useobject{currentmarker}{}%
\end{pgfscope}%
\begin{pgfscope}%
\pgfsys@transformshift{-11.238468in}{1.400000in}%
\pgfsys@useobject{currentmarker}{}%
\end{pgfscope}%
\begin{pgfscope}%
\pgfsys@transformshift{-11.202846in}{1.600000in}%
\pgfsys@useobject{currentmarker}{}%
\end{pgfscope}%
\begin{pgfscope}%
\pgfsys@transformshift{-11.167225in}{1.500000in}%
\pgfsys@useobject{currentmarker}{}%
\end{pgfscope}%
\begin{pgfscope}%
\pgfsys@transformshift{-11.131603in}{1.600000in}%
\pgfsys@useobject{currentmarker}{}%
\end{pgfscope}%
\begin{pgfscope}%
\pgfsys@transformshift{-11.095981in}{1.500000in}%
\pgfsys@useobject{currentmarker}{}%
\end{pgfscope}%
\begin{pgfscope}%
\pgfsys@transformshift{-11.060360in}{1.600000in}%
\pgfsys@useobject{currentmarker}{}%
\end{pgfscope}%
\begin{pgfscope}%
\pgfsys@transformshift{-11.024738in}{1.700000in}%
\pgfsys@useobject{currentmarker}{}%
\end{pgfscope}%
\begin{pgfscope}%
\pgfsys@transformshift{-10.989116in}{1.400000in}%
\pgfsys@useobject{currentmarker}{}%
\end{pgfscope}%
\begin{pgfscope}%
\pgfsys@transformshift{-10.953495in}{1.500000in}%
\pgfsys@useobject{currentmarker}{}%
\end{pgfscope}%
\begin{pgfscope}%
\pgfsys@transformshift{-10.917873in}{1.500000in}%
\pgfsys@useobject{currentmarker}{}%
\end{pgfscope}%
\begin{pgfscope}%
\pgfsys@transformshift{-10.882252in}{1.400000in}%
\pgfsys@useobject{currentmarker}{}%
\end{pgfscope}%
\begin{pgfscope}%
\pgfsys@transformshift{-10.846630in}{1.500000in}%
\pgfsys@useobject{currentmarker}{}%
\end{pgfscope}%
\begin{pgfscope}%
\pgfsys@transformshift{-10.811008in}{1.400000in}%
\pgfsys@useobject{currentmarker}{}%
\end{pgfscope}%
\begin{pgfscope}%
\pgfsys@transformshift{-10.775387in}{1.600000in}%
\pgfsys@useobject{currentmarker}{}%
\end{pgfscope}%
\begin{pgfscope}%
\pgfsys@transformshift{-10.739765in}{1.500000in}%
\pgfsys@useobject{currentmarker}{}%
\end{pgfscope}%
\begin{pgfscope}%
\pgfsys@transformshift{-10.704143in}{1.700000in}%
\pgfsys@useobject{currentmarker}{}%
\end{pgfscope}%
\begin{pgfscope}%
\pgfsys@transformshift{-10.668522in}{1.500000in}%
\pgfsys@useobject{currentmarker}{}%
\end{pgfscope}%
\begin{pgfscope}%
\pgfsys@transformshift{-10.632900in}{1.500000in}%
\pgfsys@useobject{currentmarker}{}%
\end{pgfscope}%
\begin{pgfscope}%
\pgfsys@transformshift{-10.597278in}{1.200000in}%
\pgfsys@useobject{currentmarker}{}%
\end{pgfscope}%
\begin{pgfscope}%
\pgfsys@transformshift{-10.561657in}{1.500000in}%
\pgfsys@useobject{currentmarker}{}%
\end{pgfscope}%
\begin{pgfscope}%
\pgfsys@transformshift{-10.526035in}{1.500000in}%
\pgfsys@useobject{currentmarker}{}%
\end{pgfscope}%
\begin{pgfscope}%
\pgfsys@transformshift{-10.490414in}{1.400000in}%
\pgfsys@useobject{currentmarker}{}%
\end{pgfscope}%
\begin{pgfscope}%
\pgfsys@transformshift{-10.454792in}{1.500000in}%
\pgfsys@useobject{currentmarker}{}%
\end{pgfscope}%
\begin{pgfscope}%
\pgfsys@transformshift{-10.419170in}{1.500000in}%
\pgfsys@useobject{currentmarker}{}%
\end{pgfscope}%
\begin{pgfscope}%
\pgfsys@transformshift{-10.383549in}{1.600000in}%
\pgfsys@useobject{currentmarker}{}%
\end{pgfscope}%
\begin{pgfscope}%
\pgfsys@transformshift{-10.347927in}{1.300000in}%
\pgfsys@useobject{currentmarker}{}%
\end{pgfscope}%
\begin{pgfscope}%
\pgfsys@transformshift{-10.312305in}{1.500000in}%
\pgfsys@useobject{currentmarker}{}%
\end{pgfscope}%
\begin{pgfscope}%
\pgfsys@transformshift{-10.276684in}{1.600000in}%
\pgfsys@useobject{currentmarker}{}%
\end{pgfscope}%
\begin{pgfscope}%
\pgfsys@transformshift{-10.241062in}{1.400000in}%
\pgfsys@useobject{currentmarker}{}%
\end{pgfscope}%
\begin{pgfscope}%
\pgfsys@transformshift{-10.205441in}{1.600000in}%
\pgfsys@useobject{currentmarker}{}%
\end{pgfscope}%
\begin{pgfscope}%
\pgfsys@transformshift{-10.169819in}{1.500000in}%
\pgfsys@useobject{currentmarker}{}%
\end{pgfscope}%
\begin{pgfscope}%
\pgfsys@transformshift{-10.134197in}{1.600000in}%
\pgfsys@useobject{currentmarker}{}%
\end{pgfscope}%
\begin{pgfscope}%
\pgfsys@transformshift{-10.098576in}{1.500000in}%
\pgfsys@useobject{currentmarker}{}%
\end{pgfscope}%
\begin{pgfscope}%
\pgfsys@transformshift{-10.062954in}{1.500000in}%
\pgfsys@useobject{currentmarker}{}%
\end{pgfscope}%
\begin{pgfscope}%
\pgfsys@transformshift{-10.027332in}{1.500000in}%
\pgfsys@useobject{currentmarker}{}%
\end{pgfscope}%
\begin{pgfscope}%
\pgfsys@transformshift{-9.991711in}{1.600000in}%
\pgfsys@useobject{currentmarker}{}%
\end{pgfscope}%
\begin{pgfscope}%
\pgfsys@transformshift{-9.956089in}{1.500000in}%
\pgfsys@useobject{currentmarker}{}%
\end{pgfscope}%
\begin{pgfscope}%
\pgfsys@transformshift{-9.920468in}{1.600000in}%
\pgfsys@useobject{currentmarker}{}%
\end{pgfscope}%
\begin{pgfscope}%
\pgfsys@transformshift{-9.884846in}{1.500000in}%
\pgfsys@useobject{currentmarker}{}%
\end{pgfscope}%
\begin{pgfscope}%
\pgfsys@transformshift{-9.849224in}{1.600000in}%
\pgfsys@useobject{currentmarker}{}%
\end{pgfscope}%
\begin{pgfscope}%
\pgfsys@transformshift{-9.813603in}{1.600000in}%
\pgfsys@useobject{currentmarker}{}%
\end{pgfscope}%
\begin{pgfscope}%
\pgfsys@transformshift{-9.777981in}{1.500000in}%
\pgfsys@useobject{currentmarker}{}%
\end{pgfscope}%
\begin{pgfscope}%
\pgfsys@transformshift{-9.742359in}{1.500000in}%
\pgfsys@useobject{currentmarker}{}%
\end{pgfscope}%
\begin{pgfscope}%
\pgfsys@transformshift{-9.706738in}{1.500000in}%
\pgfsys@useobject{currentmarker}{}%
\end{pgfscope}%
\begin{pgfscope}%
\pgfsys@transformshift{-9.671116in}{1.400000in}%
\pgfsys@useobject{currentmarker}{}%
\end{pgfscope}%
\begin{pgfscope}%
\pgfsys@transformshift{-9.635495in}{1.600000in}%
\pgfsys@useobject{currentmarker}{}%
\end{pgfscope}%
\begin{pgfscope}%
\pgfsys@transformshift{-9.599873in}{1.500000in}%
\pgfsys@useobject{currentmarker}{}%
\end{pgfscope}%
\begin{pgfscope}%
\pgfsys@transformshift{-9.564251in}{1.600000in}%
\pgfsys@useobject{currentmarker}{}%
\end{pgfscope}%
\begin{pgfscope}%
\pgfsys@transformshift{-9.528630in}{1.600000in}%
\pgfsys@useobject{currentmarker}{}%
\end{pgfscope}%
\begin{pgfscope}%
\pgfsys@transformshift{-9.493008in}{1.600000in}%
\pgfsys@useobject{currentmarker}{}%
\end{pgfscope}%
\begin{pgfscope}%
\pgfsys@transformshift{-9.457386in}{1.700000in}%
\pgfsys@useobject{currentmarker}{}%
\end{pgfscope}%
\begin{pgfscope}%
\pgfsys@transformshift{-9.421765in}{1.500000in}%
\pgfsys@useobject{currentmarker}{}%
\end{pgfscope}%
\begin{pgfscope}%
\pgfsys@transformshift{-9.386143in}{1.400000in}%
\pgfsys@useobject{currentmarker}{}%
\end{pgfscope}%
\begin{pgfscope}%
\pgfsys@transformshift{-9.350522in}{1.600000in}%
\pgfsys@useobject{currentmarker}{}%
\end{pgfscope}%
\begin{pgfscope}%
\pgfsys@transformshift{-9.314900in}{1.500000in}%
\pgfsys@useobject{currentmarker}{}%
\end{pgfscope}%
\begin{pgfscope}%
\pgfsys@transformshift{-9.279278in}{1.400000in}%
\pgfsys@useobject{currentmarker}{}%
\end{pgfscope}%
\begin{pgfscope}%
\pgfsys@transformshift{-9.243657in}{1.600000in}%
\pgfsys@useobject{currentmarker}{}%
\end{pgfscope}%
\begin{pgfscope}%
\pgfsys@transformshift{-9.208035in}{1.600000in}%
\pgfsys@useobject{currentmarker}{}%
\end{pgfscope}%
\begin{pgfscope}%
\pgfsys@transformshift{-9.172413in}{1.500000in}%
\pgfsys@useobject{currentmarker}{}%
\end{pgfscope}%
\begin{pgfscope}%
\pgfsys@transformshift{-9.136792in}{1.500000in}%
\pgfsys@useobject{currentmarker}{}%
\end{pgfscope}%
\begin{pgfscope}%
\pgfsys@transformshift{-9.101170in}{1.200000in}%
\pgfsys@useobject{currentmarker}{}%
\end{pgfscope}%
\begin{pgfscope}%
\pgfsys@transformshift{-9.065548in}{1.500000in}%
\pgfsys@useobject{currentmarker}{}%
\end{pgfscope}%
\begin{pgfscope}%
\pgfsys@transformshift{-9.029927in}{1.500000in}%
\pgfsys@useobject{currentmarker}{}%
\end{pgfscope}%
\begin{pgfscope}%
\pgfsys@transformshift{-8.994305in}{1.700000in}%
\pgfsys@useobject{currentmarker}{}%
\end{pgfscope}%
\begin{pgfscope}%
\pgfsys@transformshift{-8.958684in}{1.600000in}%
\pgfsys@useobject{currentmarker}{}%
\end{pgfscope}%
\begin{pgfscope}%
\pgfsys@transformshift{-8.923062in}{1.500000in}%
\pgfsys@useobject{currentmarker}{}%
\end{pgfscope}%
\begin{pgfscope}%
\pgfsys@transformshift{-8.887440in}{1.400000in}%
\pgfsys@useobject{currentmarker}{}%
\end{pgfscope}%
\begin{pgfscope}%
\pgfsys@transformshift{-8.851819in}{1.400000in}%
\pgfsys@useobject{currentmarker}{}%
\end{pgfscope}%
\begin{pgfscope}%
\pgfsys@transformshift{-8.816197in}{1.600000in}%
\pgfsys@useobject{currentmarker}{}%
\end{pgfscope}%
\begin{pgfscope}%
\pgfsys@transformshift{-8.780575in}{1.600000in}%
\pgfsys@useobject{currentmarker}{}%
\end{pgfscope}%
\begin{pgfscope}%
\pgfsys@transformshift{-8.744954in}{1.400000in}%
\pgfsys@useobject{currentmarker}{}%
\end{pgfscope}%
\begin{pgfscope}%
\pgfsys@transformshift{-8.709332in}{1.500000in}%
\pgfsys@useobject{currentmarker}{}%
\end{pgfscope}%
\begin{pgfscope}%
\pgfsys@transformshift{-8.673711in}{1.500000in}%
\pgfsys@useobject{currentmarker}{}%
\end{pgfscope}%
\begin{pgfscope}%
\pgfsys@transformshift{-8.638089in}{1.500000in}%
\pgfsys@useobject{currentmarker}{}%
\end{pgfscope}%
\begin{pgfscope}%
\pgfsys@transformshift{-8.602467in}{1.300000in}%
\pgfsys@useobject{currentmarker}{}%
\end{pgfscope}%
\begin{pgfscope}%
\pgfsys@transformshift{-8.566846in}{1.600000in}%
\pgfsys@useobject{currentmarker}{}%
\end{pgfscope}%
\begin{pgfscope}%
\pgfsys@transformshift{-8.531224in}{1.700000in}%
\pgfsys@useobject{currentmarker}{}%
\end{pgfscope}%
\begin{pgfscope}%
\pgfsys@transformshift{-8.495602in}{1.500000in}%
\pgfsys@useobject{currentmarker}{}%
\end{pgfscope}%
\begin{pgfscope}%
\pgfsys@transformshift{-8.459981in}{1.600000in}%
\pgfsys@useobject{currentmarker}{}%
\end{pgfscope}%
\begin{pgfscope}%
\pgfsys@transformshift{-8.424359in}{1.600000in}%
\pgfsys@useobject{currentmarker}{}%
\end{pgfscope}%
\begin{pgfscope}%
\pgfsys@transformshift{-8.388738in}{1.700000in}%
\pgfsys@useobject{currentmarker}{}%
\end{pgfscope}%
\begin{pgfscope}%
\pgfsys@transformshift{-8.353116in}{1.700000in}%
\pgfsys@useobject{currentmarker}{}%
\end{pgfscope}%
\begin{pgfscope}%
\pgfsys@transformshift{-8.317494in}{1.500000in}%
\pgfsys@useobject{currentmarker}{}%
\end{pgfscope}%
\begin{pgfscope}%
\pgfsys@transformshift{-8.281873in}{1.700000in}%
\pgfsys@useobject{currentmarker}{}%
\end{pgfscope}%
\begin{pgfscope}%
\pgfsys@transformshift{-8.246251in}{1.600000in}%
\pgfsys@useobject{currentmarker}{}%
\end{pgfscope}%
\begin{pgfscope}%
\pgfsys@transformshift{-8.210629in}{1.500000in}%
\pgfsys@useobject{currentmarker}{}%
\end{pgfscope}%
\begin{pgfscope}%
\pgfsys@transformshift{-8.175008in}{1.500000in}%
\pgfsys@useobject{currentmarker}{}%
\end{pgfscope}%
\begin{pgfscope}%
\pgfsys@transformshift{-8.139386in}{1.500000in}%
\pgfsys@useobject{currentmarker}{}%
\end{pgfscope}%
\begin{pgfscope}%
\pgfsys@transformshift{-8.103765in}{1.300000in}%
\pgfsys@useobject{currentmarker}{}%
\end{pgfscope}%
\begin{pgfscope}%
\pgfsys@transformshift{-8.068143in}{1.400000in}%
\pgfsys@useobject{currentmarker}{}%
\end{pgfscope}%
\begin{pgfscope}%
\pgfsys@transformshift{-8.032521in}{1.500000in}%
\pgfsys@useobject{currentmarker}{}%
\end{pgfscope}%
\begin{pgfscope}%
\pgfsys@transformshift{-7.996900in}{1.500000in}%
\pgfsys@useobject{currentmarker}{}%
\end{pgfscope}%
\begin{pgfscope}%
\pgfsys@transformshift{-7.961278in}{1.500000in}%
\pgfsys@useobject{currentmarker}{}%
\end{pgfscope}%
\begin{pgfscope}%
\pgfsys@transformshift{-7.925656in}{1.500000in}%
\pgfsys@useobject{currentmarker}{}%
\end{pgfscope}%
\begin{pgfscope}%
\pgfsys@transformshift{-7.890035in}{1.400000in}%
\pgfsys@useobject{currentmarker}{}%
\end{pgfscope}%
\begin{pgfscope}%
\pgfsys@transformshift{-7.854413in}{1.600000in}%
\pgfsys@useobject{currentmarker}{}%
\end{pgfscope}%
\begin{pgfscope}%
\pgfsys@transformshift{-7.818792in}{1.400000in}%
\pgfsys@useobject{currentmarker}{}%
\end{pgfscope}%
\begin{pgfscope}%
\pgfsys@transformshift{-7.783170in}{1.500000in}%
\pgfsys@useobject{currentmarker}{}%
\end{pgfscope}%
\begin{pgfscope}%
\pgfsys@transformshift{-7.747548in}{1.600000in}%
\pgfsys@useobject{currentmarker}{}%
\end{pgfscope}%
\begin{pgfscope}%
\pgfsys@transformshift{-7.711927in}{1.500000in}%
\pgfsys@useobject{currentmarker}{}%
\end{pgfscope}%
\begin{pgfscope}%
\pgfsys@transformshift{-7.676305in}{1.400000in}%
\pgfsys@useobject{currentmarker}{}%
\end{pgfscope}%
\begin{pgfscope}%
\pgfsys@transformshift{-7.640683in}{1.500000in}%
\pgfsys@useobject{currentmarker}{}%
\end{pgfscope}%
\begin{pgfscope}%
\pgfsys@transformshift{-7.605062in}{1.600000in}%
\pgfsys@useobject{currentmarker}{}%
\end{pgfscope}%
\begin{pgfscope}%
\pgfsys@transformshift{-7.569440in}{1.600000in}%
\pgfsys@useobject{currentmarker}{}%
\end{pgfscope}%
\begin{pgfscope}%
\pgfsys@transformshift{-7.533818in}{1.500000in}%
\pgfsys@useobject{currentmarker}{}%
\end{pgfscope}%
\begin{pgfscope}%
\pgfsys@transformshift{-7.498197in}{1.400000in}%
\pgfsys@useobject{currentmarker}{}%
\end{pgfscope}%
\begin{pgfscope}%
\pgfsys@transformshift{-7.462575in}{1.500000in}%
\pgfsys@useobject{currentmarker}{}%
\end{pgfscope}%
\begin{pgfscope}%
\pgfsys@transformshift{-7.426954in}{1.600000in}%
\pgfsys@useobject{currentmarker}{}%
\end{pgfscope}%
\begin{pgfscope}%
\pgfsys@transformshift{-7.391332in}{1.600000in}%
\pgfsys@useobject{currentmarker}{}%
\end{pgfscope}%
\begin{pgfscope}%
\pgfsys@transformshift{-7.355710in}{1.600000in}%
\pgfsys@useobject{currentmarker}{}%
\end{pgfscope}%
\begin{pgfscope}%
\pgfsys@transformshift{-7.320089in}{1.500000in}%
\pgfsys@useobject{currentmarker}{}%
\end{pgfscope}%
\begin{pgfscope}%
\pgfsys@transformshift{-7.284467in}{1.400000in}%
\pgfsys@useobject{currentmarker}{}%
\end{pgfscope}%
\begin{pgfscope}%
\pgfsys@transformshift{-7.248845in}{1.600000in}%
\pgfsys@useobject{currentmarker}{}%
\end{pgfscope}%
\begin{pgfscope}%
\pgfsys@transformshift{-7.213224in}{1.400000in}%
\pgfsys@useobject{currentmarker}{}%
\end{pgfscope}%
\begin{pgfscope}%
\pgfsys@transformshift{-7.177602in}{1.500000in}%
\pgfsys@useobject{currentmarker}{}%
\end{pgfscope}%
\begin{pgfscope}%
\pgfsys@transformshift{-7.141981in}{1.500000in}%
\pgfsys@useobject{currentmarker}{}%
\end{pgfscope}%
\begin{pgfscope}%
\pgfsys@transformshift{-7.106359in}{1.400000in}%
\pgfsys@useobject{currentmarker}{}%
\end{pgfscope}%
\begin{pgfscope}%
\pgfsys@transformshift{-7.070737in}{1.700000in}%
\pgfsys@useobject{currentmarker}{}%
\end{pgfscope}%
\begin{pgfscope}%
\pgfsys@transformshift{-7.035116in}{1.600000in}%
\pgfsys@useobject{currentmarker}{}%
\end{pgfscope}%
\begin{pgfscope}%
\pgfsys@transformshift{-6.999494in}{1.400000in}%
\pgfsys@useobject{currentmarker}{}%
\end{pgfscope}%
\begin{pgfscope}%
\pgfsys@transformshift{-6.963872in}{1.500000in}%
\pgfsys@useobject{currentmarker}{}%
\end{pgfscope}%
\begin{pgfscope}%
\pgfsys@transformshift{-6.928251in}{1.500000in}%
\pgfsys@useobject{currentmarker}{}%
\end{pgfscope}%
\begin{pgfscope}%
\pgfsys@transformshift{-6.892629in}{1.500000in}%
\pgfsys@useobject{currentmarker}{}%
\end{pgfscope}%
\begin{pgfscope}%
\pgfsys@transformshift{-6.857008in}{1.500000in}%
\pgfsys@useobject{currentmarker}{}%
\end{pgfscope}%
\begin{pgfscope}%
\pgfsys@transformshift{-6.821386in}{1.500000in}%
\pgfsys@useobject{currentmarker}{}%
\end{pgfscope}%
\begin{pgfscope}%
\pgfsys@transformshift{-6.785764in}{1.500000in}%
\pgfsys@useobject{currentmarker}{}%
\end{pgfscope}%
\begin{pgfscope}%
\pgfsys@transformshift{-6.750143in}{1.400000in}%
\pgfsys@useobject{currentmarker}{}%
\end{pgfscope}%
\begin{pgfscope}%
\pgfsys@transformshift{-6.714521in}{1.700000in}%
\pgfsys@useobject{currentmarker}{}%
\end{pgfscope}%
\begin{pgfscope}%
\pgfsys@transformshift{-6.678899in}{1.800000in}%
\pgfsys@useobject{currentmarker}{}%
\end{pgfscope}%
\begin{pgfscope}%
\pgfsys@transformshift{-6.643278in}{1.500000in}%
\pgfsys@useobject{currentmarker}{}%
\end{pgfscope}%
\begin{pgfscope}%
\pgfsys@transformshift{-6.607656in}{1.500000in}%
\pgfsys@useobject{currentmarker}{}%
\end{pgfscope}%
\begin{pgfscope}%
\pgfsys@transformshift{-6.572035in}{1.500000in}%
\pgfsys@useobject{currentmarker}{}%
\end{pgfscope}%
\begin{pgfscope}%
\pgfsys@transformshift{-6.536413in}{1.400000in}%
\pgfsys@useobject{currentmarker}{}%
\end{pgfscope}%
\begin{pgfscope}%
\pgfsys@transformshift{-6.500791in}{1.700000in}%
\pgfsys@useobject{currentmarker}{}%
\end{pgfscope}%
\begin{pgfscope}%
\pgfsys@transformshift{-6.465170in}{1.400000in}%
\pgfsys@useobject{currentmarker}{}%
\end{pgfscope}%
\begin{pgfscope}%
\pgfsys@transformshift{-6.429548in}{1.400000in}%
\pgfsys@useobject{currentmarker}{}%
\end{pgfscope}%
\begin{pgfscope}%
\pgfsys@transformshift{-6.393926in}{1.300000in}%
\pgfsys@useobject{currentmarker}{}%
\end{pgfscope}%
\begin{pgfscope}%
\pgfsys@transformshift{-6.358305in}{1.600000in}%
\pgfsys@useobject{currentmarker}{}%
\end{pgfscope}%
\begin{pgfscope}%
\pgfsys@transformshift{-6.322683in}{1.500000in}%
\pgfsys@useobject{currentmarker}{}%
\end{pgfscope}%
\begin{pgfscope}%
\pgfsys@transformshift{-6.287062in}{1.400000in}%
\pgfsys@useobject{currentmarker}{}%
\end{pgfscope}%
\begin{pgfscope}%
\pgfsys@transformshift{-6.251440in}{1.600000in}%
\pgfsys@useobject{currentmarker}{}%
\end{pgfscope}%
\begin{pgfscope}%
\pgfsys@transformshift{-6.215818in}{1.600000in}%
\pgfsys@useobject{currentmarker}{}%
\end{pgfscope}%
\begin{pgfscope}%
\pgfsys@transformshift{-6.180197in}{1.400000in}%
\pgfsys@useobject{currentmarker}{}%
\end{pgfscope}%
\begin{pgfscope}%
\pgfsys@transformshift{-6.144575in}{1.400000in}%
\pgfsys@useobject{currentmarker}{}%
\end{pgfscope}%
\begin{pgfscope}%
\pgfsys@transformshift{-6.108953in}{1.500000in}%
\pgfsys@useobject{currentmarker}{}%
\end{pgfscope}%
\begin{pgfscope}%
\pgfsys@transformshift{-6.073332in}{1.500000in}%
\pgfsys@useobject{currentmarker}{}%
\end{pgfscope}%
\begin{pgfscope}%
\pgfsys@transformshift{-6.037710in}{1.400000in}%
\pgfsys@useobject{currentmarker}{}%
\end{pgfscope}%
\begin{pgfscope}%
\pgfsys@transformshift{-6.002088in}{1.400000in}%
\pgfsys@useobject{currentmarker}{}%
\end{pgfscope}%
\begin{pgfscope}%
\pgfsys@transformshift{-5.966467in}{1.600000in}%
\pgfsys@useobject{currentmarker}{}%
\end{pgfscope}%
\begin{pgfscope}%
\pgfsys@transformshift{-5.930845in}{1.500000in}%
\pgfsys@useobject{currentmarker}{}%
\end{pgfscope}%
\begin{pgfscope}%
\pgfsys@transformshift{-5.895224in}{1.500000in}%
\pgfsys@useobject{currentmarker}{}%
\end{pgfscope}%
\begin{pgfscope}%
\pgfsys@transformshift{-5.859602in}{1.500000in}%
\pgfsys@useobject{currentmarker}{}%
\end{pgfscope}%
\begin{pgfscope}%
\pgfsys@transformshift{-5.823980in}{1.500000in}%
\pgfsys@useobject{currentmarker}{}%
\end{pgfscope}%
\begin{pgfscope}%
\pgfsys@transformshift{-5.788359in}{1.400000in}%
\pgfsys@useobject{currentmarker}{}%
\end{pgfscope}%
\begin{pgfscope}%
\pgfsys@transformshift{-5.752737in}{1.400000in}%
\pgfsys@useobject{currentmarker}{}%
\end{pgfscope}%
\begin{pgfscope}%
\pgfsys@transformshift{-5.717115in}{1.800000in}%
\pgfsys@useobject{currentmarker}{}%
\end{pgfscope}%
\begin{pgfscope}%
\pgfsys@transformshift{-5.681494in}{1.500000in}%
\pgfsys@useobject{currentmarker}{}%
\end{pgfscope}%
\begin{pgfscope}%
\pgfsys@transformshift{-5.645872in}{1.600000in}%
\pgfsys@useobject{currentmarker}{}%
\end{pgfscope}%
\begin{pgfscope}%
\pgfsys@transformshift{-5.610251in}{1.500000in}%
\pgfsys@useobject{currentmarker}{}%
\end{pgfscope}%
\begin{pgfscope}%
\pgfsys@transformshift{-5.574629in}{1.400000in}%
\pgfsys@useobject{currentmarker}{}%
\end{pgfscope}%
\begin{pgfscope}%
\pgfsys@transformshift{-5.539007in}{1.500000in}%
\pgfsys@useobject{currentmarker}{}%
\end{pgfscope}%
\begin{pgfscope}%
\pgfsys@transformshift{-5.503386in}{1.600000in}%
\pgfsys@useobject{currentmarker}{}%
\end{pgfscope}%
\begin{pgfscope}%
\pgfsys@transformshift{-5.467764in}{1.500000in}%
\pgfsys@useobject{currentmarker}{}%
\end{pgfscope}%
\begin{pgfscope}%
\pgfsys@transformshift{-5.432142in}{1.400000in}%
\pgfsys@useobject{currentmarker}{}%
\end{pgfscope}%
\begin{pgfscope}%
\pgfsys@transformshift{-5.396521in}{1.500000in}%
\pgfsys@useobject{currentmarker}{}%
\end{pgfscope}%
\begin{pgfscope}%
\pgfsys@transformshift{-5.360899in}{1.600000in}%
\pgfsys@useobject{currentmarker}{}%
\end{pgfscope}%
\begin{pgfscope}%
\pgfsys@transformshift{-5.325278in}{1.500000in}%
\pgfsys@useobject{currentmarker}{}%
\end{pgfscope}%
\begin{pgfscope}%
\pgfsys@transformshift{-5.289656in}{1.400000in}%
\pgfsys@useobject{currentmarker}{}%
\end{pgfscope}%
\begin{pgfscope}%
\pgfsys@transformshift{-5.254034in}{1.500000in}%
\pgfsys@useobject{currentmarker}{}%
\end{pgfscope}%
\begin{pgfscope}%
\pgfsys@transformshift{-5.218413in}{1.400000in}%
\pgfsys@useobject{currentmarker}{}%
\end{pgfscope}%
\begin{pgfscope}%
\pgfsys@transformshift{-5.182791in}{1.400000in}%
\pgfsys@useobject{currentmarker}{}%
\end{pgfscope}%
\begin{pgfscope}%
\pgfsys@transformshift{-5.147169in}{1.500000in}%
\pgfsys@useobject{currentmarker}{}%
\end{pgfscope}%
\begin{pgfscope}%
\pgfsys@transformshift{-5.111548in}{1.500000in}%
\pgfsys@useobject{currentmarker}{}%
\end{pgfscope}%
\begin{pgfscope}%
\pgfsys@transformshift{-5.075926in}{1.400000in}%
\pgfsys@useobject{currentmarker}{}%
\end{pgfscope}%
\begin{pgfscope}%
\pgfsys@transformshift{-5.040305in}{1.500000in}%
\pgfsys@useobject{currentmarker}{}%
\end{pgfscope}%
\begin{pgfscope}%
\pgfsys@transformshift{-5.004683in}{1.400000in}%
\pgfsys@useobject{currentmarker}{}%
\end{pgfscope}%
\begin{pgfscope}%
\pgfsys@transformshift{-4.969061in}{1.600000in}%
\pgfsys@useobject{currentmarker}{}%
\end{pgfscope}%
\begin{pgfscope}%
\pgfsys@transformshift{-4.933440in}{1.500000in}%
\pgfsys@useobject{currentmarker}{}%
\end{pgfscope}%
\begin{pgfscope}%
\pgfsys@transformshift{-4.897818in}{1.600000in}%
\pgfsys@useobject{currentmarker}{}%
\end{pgfscope}%
\begin{pgfscope}%
\pgfsys@transformshift{-4.862196in}{1.600000in}%
\pgfsys@useobject{currentmarker}{}%
\end{pgfscope}%
\begin{pgfscope}%
\pgfsys@transformshift{-4.826575in}{1.600000in}%
\pgfsys@useobject{currentmarker}{}%
\end{pgfscope}%
\begin{pgfscope}%
\pgfsys@transformshift{-4.790953in}{1.400000in}%
\pgfsys@useobject{currentmarker}{}%
\end{pgfscope}%
\begin{pgfscope}%
\pgfsys@transformshift{-4.755332in}{1.500000in}%
\pgfsys@useobject{currentmarker}{}%
\end{pgfscope}%
\begin{pgfscope}%
\pgfsys@transformshift{-4.719710in}{1.500000in}%
\pgfsys@useobject{currentmarker}{}%
\end{pgfscope}%
\begin{pgfscope}%
\pgfsys@transformshift{-4.684088in}{1.600000in}%
\pgfsys@useobject{currentmarker}{}%
\end{pgfscope}%
\begin{pgfscope}%
\pgfsys@transformshift{-4.648467in}{1.500000in}%
\pgfsys@useobject{currentmarker}{}%
\end{pgfscope}%
\begin{pgfscope}%
\pgfsys@transformshift{-4.612845in}{1.600000in}%
\pgfsys@useobject{currentmarker}{}%
\end{pgfscope}%
\begin{pgfscope}%
\pgfsys@transformshift{-4.577223in}{1.400000in}%
\pgfsys@useobject{currentmarker}{}%
\end{pgfscope}%
\begin{pgfscope}%
\pgfsys@transformshift{-4.541602in}{1.500000in}%
\pgfsys@useobject{currentmarker}{}%
\end{pgfscope}%
\begin{pgfscope}%
\pgfsys@transformshift{-4.505980in}{1.300000in}%
\pgfsys@useobject{currentmarker}{}%
\end{pgfscope}%
\begin{pgfscope}%
\pgfsys@transformshift{-4.470358in}{1.400000in}%
\pgfsys@useobject{currentmarker}{}%
\end{pgfscope}%
\begin{pgfscope}%
\pgfsys@transformshift{-4.434737in}{1.600000in}%
\pgfsys@useobject{currentmarker}{}%
\end{pgfscope}%
\begin{pgfscope}%
\pgfsys@transformshift{-4.399115in}{1.300000in}%
\pgfsys@useobject{currentmarker}{}%
\end{pgfscope}%
\begin{pgfscope}%
\pgfsys@transformshift{-4.363494in}{1.500000in}%
\pgfsys@useobject{currentmarker}{}%
\end{pgfscope}%
\begin{pgfscope}%
\pgfsys@transformshift{-4.327872in}{1.500000in}%
\pgfsys@useobject{currentmarker}{}%
\end{pgfscope}%
\begin{pgfscope}%
\pgfsys@transformshift{-4.292250in}{1.500000in}%
\pgfsys@useobject{currentmarker}{}%
\end{pgfscope}%
\begin{pgfscope}%
\pgfsys@transformshift{-4.256629in}{1.500000in}%
\pgfsys@useobject{currentmarker}{}%
\end{pgfscope}%
\begin{pgfscope}%
\pgfsys@transformshift{-4.221007in}{1.400000in}%
\pgfsys@useobject{currentmarker}{}%
\end{pgfscope}%
\begin{pgfscope}%
\pgfsys@transformshift{-4.185385in}{1.500000in}%
\pgfsys@useobject{currentmarker}{}%
\end{pgfscope}%
\begin{pgfscope}%
\pgfsys@transformshift{-4.149764in}{1.600000in}%
\pgfsys@useobject{currentmarker}{}%
\end{pgfscope}%
\begin{pgfscope}%
\pgfsys@transformshift{-4.114142in}{1.500000in}%
\pgfsys@useobject{currentmarker}{}%
\end{pgfscope}%
\begin{pgfscope}%
\pgfsys@transformshift{-4.078521in}{1.600000in}%
\pgfsys@useobject{currentmarker}{}%
\end{pgfscope}%
\begin{pgfscope}%
\pgfsys@transformshift{-4.042899in}{1.500000in}%
\pgfsys@useobject{currentmarker}{}%
\end{pgfscope}%
\begin{pgfscope}%
\pgfsys@transformshift{-4.007277in}{1.300000in}%
\pgfsys@useobject{currentmarker}{}%
\end{pgfscope}%
\begin{pgfscope}%
\pgfsys@transformshift{-3.971656in}{1.600000in}%
\pgfsys@useobject{currentmarker}{}%
\end{pgfscope}%
\begin{pgfscope}%
\pgfsys@transformshift{-3.936034in}{1.400000in}%
\pgfsys@useobject{currentmarker}{}%
\end{pgfscope}%
\begin{pgfscope}%
\pgfsys@transformshift{-3.900412in}{1.700000in}%
\pgfsys@useobject{currentmarker}{}%
\end{pgfscope}%
\begin{pgfscope}%
\pgfsys@transformshift{-3.864791in}{1.500000in}%
\pgfsys@useobject{currentmarker}{}%
\end{pgfscope}%
\begin{pgfscope}%
\pgfsys@transformshift{-3.829169in}{1.500000in}%
\pgfsys@useobject{currentmarker}{}%
\end{pgfscope}%
\begin{pgfscope}%
\pgfsys@transformshift{-3.793548in}{1.400000in}%
\pgfsys@useobject{currentmarker}{}%
\end{pgfscope}%
\begin{pgfscope}%
\pgfsys@transformshift{-3.757926in}{1.700000in}%
\pgfsys@useobject{currentmarker}{}%
\end{pgfscope}%
\begin{pgfscope}%
\pgfsys@transformshift{-3.722304in}{1.600000in}%
\pgfsys@useobject{currentmarker}{}%
\end{pgfscope}%
\begin{pgfscope}%
\pgfsys@transformshift{-3.686683in}{1.500000in}%
\pgfsys@useobject{currentmarker}{}%
\end{pgfscope}%
\begin{pgfscope}%
\pgfsys@transformshift{-3.651061in}{1.500000in}%
\pgfsys@useobject{currentmarker}{}%
\end{pgfscope}%
\begin{pgfscope}%
\pgfsys@transformshift{-3.615439in}{1.500000in}%
\pgfsys@useobject{currentmarker}{}%
\end{pgfscope}%
\begin{pgfscope}%
\pgfsys@transformshift{-3.579818in}{1.600000in}%
\pgfsys@useobject{currentmarker}{}%
\end{pgfscope}%
\begin{pgfscope}%
\pgfsys@transformshift{-3.544196in}{1.700000in}%
\pgfsys@useobject{currentmarker}{}%
\end{pgfscope}%
\begin{pgfscope}%
\pgfsys@transformshift{-3.508575in}{1.700000in}%
\pgfsys@useobject{currentmarker}{}%
\end{pgfscope}%
\begin{pgfscope}%
\pgfsys@transformshift{-3.472953in}{1.400000in}%
\pgfsys@useobject{currentmarker}{}%
\end{pgfscope}%
\begin{pgfscope}%
\pgfsys@transformshift{-3.437331in}{1.400000in}%
\pgfsys@useobject{currentmarker}{}%
\end{pgfscope}%
\begin{pgfscope}%
\pgfsys@transformshift{-3.401710in}{1.400000in}%
\pgfsys@useobject{currentmarker}{}%
\end{pgfscope}%
\begin{pgfscope}%
\pgfsys@transformshift{-3.366088in}{1.500000in}%
\pgfsys@useobject{currentmarker}{}%
\end{pgfscope}%
\begin{pgfscope}%
\pgfsys@transformshift{-3.330466in}{1.400000in}%
\pgfsys@useobject{currentmarker}{}%
\end{pgfscope}%
\begin{pgfscope}%
\pgfsys@transformshift{-3.294845in}{1.500000in}%
\pgfsys@useobject{currentmarker}{}%
\end{pgfscope}%
\begin{pgfscope}%
\pgfsys@transformshift{-3.259223in}{1.400000in}%
\pgfsys@useobject{currentmarker}{}%
\end{pgfscope}%
\begin{pgfscope}%
\pgfsys@transformshift{-3.223602in}{1.400000in}%
\pgfsys@useobject{currentmarker}{}%
\end{pgfscope}%
\begin{pgfscope}%
\pgfsys@transformshift{-3.187980in}{1.500000in}%
\pgfsys@useobject{currentmarker}{}%
\end{pgfscope}%
\begin{pgfscope}%
\pgfsys@transformshift{-3.152358in}{1.400000in}%
\pgfsys@useobject{currentmarker}{}%
\end{pgfscope}%
\begin{pgfscope}%
\pgfsys@transformshift{-3.116737in}{1.400000in}%
\pgfsys@useobject{currentmarker}{}%
\end{pgfscope}%
\begin{pgfscope}%
\pgfsys@transformshift{-3.081115in}{1.400000in}%
\pgfsys@useobject{currentmarker}{}%
\end{pgfscope}%
\begin{pgfscope}%
\pgfsys@transformshift{-3.045493in}{1.500000in}%
\pgfsys@useobject{currentmarker}{}%
\end{pgfscope}%
\begin{pgfscope}%
\pgfsys@transformshift{-3.009872in}{1.500000in}%
\pgfsys@useobject{currentmarker}{}%
\end{pgfscope}%
\begin{pgfscope}%
\pgfsys@transformshift{-2.974250in}{1.300000in}%
\pgfsys@useobject{currentmarker}{}%
\end{pgfscope}%
\begin{pgfscope}%
\pgfsys@transformshift{-2.938628in}{1.500000in}%
\pgfsys@useobject{currentmarker}{}%
\end{pgfscope}%
\begin{pgfscope}%
\pgfsys@transformshift{-2.903007in}{1.500000in}%
\pgfsys@useobject{currentmarker}{}%
\end{pgfscope}%
\begin{pgfscope}%
\pgfsys@transformshift{-2.867385in}{1.600000in}%
\pgfsys@useobject{currentmarker}{}%
\end{pgfscope}%
\begin{pgfscope}%
\pgfsys@transformshift{-2.831764in}{1.500000in}%
\pgfsys@useobject{currentmarker}{}%
\end{pgfscope}%
\begin{pgfscope}%
\pgfsys@transformshift{-2.796142in}{1.500000in}%
\pgfsys@useobject{currentmarker}{}%
\end{pgfscope}%
\begin{pgfscope}%
\pgfsys@transformshift{-2.760520in}{1.700000in}%
\pgfsys@useobject{currentmarker}{}%
\end{pgfscope}%
\begin{pgfscope}%
\pgfsys@transformshift{-2.724899in}{1.400000in}%
\pgfsys@useobject{currentmarker}{}%
\end{pgfscope}%
\begin{pgfscope}%
\pgfsys@transformshift{-2.689277in}{1.500000in}%
\pgfsys@useobject{currentmarker}{}%
\end{pgfscope}%
\begin{pgfscope}%
\pgfsys@transformshift{-2.653655in}{1.500000in}%
\pgfsys@useobject{currentmarker}{}%
\end{pgfscope}%
\begin{pgfscope}%
\pgfsys@transformshift{-2.618034in}{1.400000in}%
\pgfsys@useobject{currentmarker}{}%
\end{pgfscope}%
\begin{pgfscope}%
\pgfsys@transformshift{-2.582412in}{1.500000in}%
\pgfsys@useobject{currentmarker}{}%
\end{pgfscope}%
\begin{pgfscope}%
\pgfsys@transformshift{-2.546791in}{1.700000in}%
\pgfsys@useobject{currentmarker}{}%
\end{pgfscope}%
\begin{pgfscope}%
\pgfsys@transformshift{-2.511169in}{1.600000in}%
\pgfsys@useobject{currentmarker}{}%
\end{pgfscope}%
\begin{pgfscope}%
\pgfsys@transformshift{-2.475547in}{1.300000in}%
\pgfsys@useobject{currentmarker}{}%
\end{pgfscope}%
\begin{pgfscope}%
\pgfsys@transformshift{-2.439926in}{1.400000in}%
\pgfsys@useobject{currentmarker}{}%
\end{pgfscope}%
\begin{pgfscope}%
\pgfsys@transformshift{-2.404304in}{1.500000in}%
\pgfsys@useobject{currentmarker}{}%
\end{pgfscope}%
\begin{pgfscope}%
\pgfsys@transformshift{-2.368682in}{1.700000in}%
\pgfsys@useobject{currentmarker}{}%
\end{pgfscope}%
\begin{pgfscope}%
\pgfsys@transformshift{-2.333061in}{1.500000in}%
\pgfsys@useobject{currentmarker}{}%
\end{pgfscope}%
\begin{pgfscope}%
\pgfsys@transformshift{-2.297439in}{1.400000in}%
\pgfsys@useobject{currentmarker}{}%
\end{pgfscope}%
\begin{pgfscope}%
\pgfsys@transformshift{-2.261818in}{1.600000in}%
\pgfsys@useobject{currentmarker}{}%
\end{pgfscope}%
\begin{pgfscope}%
\pgfsys@transformshift{-2.226196in}{1.500000in}%
\pgfsys@useobject{currentmarker}{}%
\end{pgfscope}%
\begin{pgfscope}%
\pgfsys@transformshift{-2.190574in}{1.500000in}%
\pgfsys@useobject{currentmarker}{}%
\end{pgfscope}%
\begin{pgfscope}%
\pgfsys@transformshift{-2.154953in}{1.600000in}%
\pgfsys@useobject{currentmarker}{}%
\end{pgfscope}%
\begin{pgfscope}%
\pgfsys@transformshift{-2.119331in}{1.600000in}%
\pgfsys@useobject{currentmarker}{}%
\end{pgfscope}%
\begin{pgfscope}%
\pgfsys@transformshift{-2.083709in}{1.400000in}%
\pgfsys@useobject{currentmarker}{}%
\end{pgfscope}%
\begin{pgfscope}%
\pgfsys@transformshift{-2.048088in}{1.500000in}%
\pgfsys@useobject{currentmarker}{}%
\end{pgfscope}%
\begin{pgfscope}%
\pgfsys@transformshift{-2.012466in}{1.500000in}%
\pgfsys@useobject{currentmarker}{}%
\end{pgfscope}%
\begin{pgfscope}%
\pgfsys@transformshift{-1.976845in}{1.500000in}%
\pgfsys@useobject{currentmarker}{}%
\end{pgfscope}%
\begin{pgfscope}%
\pgfsys@transformshift{-1.941223in}{1.500000in}%
\pgfsys@useobject{currentmarker}{}%
\end{pgfscope}%
\begin{pgfscope}%
\pgfsys@transformshift{-1.905601in}{1.700000in}%
\pgfsys@useobject{currentmarker}{}%
\end{pgfscope}%
\begin{pgfscope}%
\pgfsys@transformshift{-1.869980in}{1.500000in}%
\pgfsys@useobject{currentmarker}{}%
\end{pgfscope}%
\begin{pgfscope}%
\pgfsys@transformshift{-1.834358in}{1.400000in}%
\pgfsys@useobject{currentmarker}{}%
\end{pgfscope}%
\begin{pgfscope}%
\pgfsys@transformshift{-1.798736in}{1.400000in}%
\pgfsys@useobject{currentmarker}{}%
\end{pgfscope}%
\begin{pgfscope}%
\pgfsys@transformshift{-1.763115in}{1.500000in}%
\pgfsys@useobject{currentmarker}{}%
\end{pgfscope}%
\begin{pgfscope}%
\pgfsys@transformshift{-1.727493in}{1.400000in}%
\pgfsys@useobject{currentmarker}{}%
\end{pgfscope}%
\begin{pgfscope}%
\pgfsys@transformshift{-1.691872in}{1.700000in}%
\pgfsys@useobject{currentmarker}{}%
\end{pgfscope}%
\begin{pgfscope}%
\pgfsys@transformshift{-1.656250in}{1.300000in}%
\pgfsys@useobject{currentmarker}{}%
\end{pgfscope}%
\begin{pgfscope}%
\pgfsys@transformshift{-1.620628in}{1.600000in}%
\pgfsys@useobject{currentmarker}{}%
\end{pgfscope}%
\begin{pgfscope}%
\pgfsys@transformshift{-1.585007in}{1.500000in}%
\pgfsys@useobject{currentmarker}{}%
\end{pgfscope}%
\begin{pgfscope}%
\pgfsys@transformshift{-1.549385in}{1.500000in}%
\pgfsys@useobject{currentmarker}{}%
\end{pgfscope}%
\begin{pgfscope}%
\pgfsys@transformshift{-1.513763in}{1.700000in}%
\pgfsys@useobject{currentmarker}{}%
\end{pgfscope}%
\begin{pgfscope}%
\pgfsys@transformshift{-1.478142in}{1.500000in}%
\pgfsys@useobject{currentmarker}{}%
\end{pgfscope}%
\begin{pgfscope}%
\pgfsys@transformshift{-1.442520in}{1.500000in}%
\pgfsys@useobject{currentmarker}{}%
\end{pgfscope}%
\begin{pgfscope}%
\pgfsys@transformshift{-1.406898in}{1.500000in}%
\pgfsys@useobject{currentmarker}{}%
\end{pgfscope}%
\begin{pgfscope}%
\pgfsys@transformshift{-1.371277in}{1.500000in}%
\pgfsys@useobject{currentmarker}{}%
\end{pgfscope}%
\begin{pgfscope}%
\pgfsys@transformshift{-1.335655in}{1.600000in}%
\pgfsys@useobject{currentmarker}{}%
\end{pgfscope}%
\begin{pgfscope}%
\pgfsys@transformshift{-1.300034in}{1.500000in}%
\pgfsys@useobject{currentmarker}{}%
\end{pgfscope}%
\begin{pgfscope}%
\pgfsys@transformshift{-1.264412in}{1.500000in}%
\pgfsys@useobject{currentmarker}{}%
\end{pgfscope}%
\begin{pgfscope}%
\pgfsys@transformshift{-1.228790in}{1.400000in}%
\pgfsys@useobject{currentmarker}{}%
\end{pgfscope}%
\begin{pgfscope}%
\pgfsys@transformshift{-1.193169in}{1.500000in}%
\pgfsys@useobject{currentmarker}{}%
\end{pgfscope}%
\begin{pgfscope}%
\pgfsys@transformshift{-1.157547in}{1.700000in}%
\pgfsys@useobject{currentmarker}{}%
\end{pgfscope}%
\begin{pgfscope}%
\pgfsys@transformshift{-1.121925in}{1.400000in}%
\pgfsys@useobject{currentmarker}{}%
\end{pgfscope}%
\begin{pgfscope}%
\pgfsys@transformshift{-1.086304in}{1.400000in}%
\pgfsys@useobject{currentmarker}{}%
\end{pgfscope}%
\begin{pgfscope}%
\pgfsys@transformshift{-1.050682in}{1.500000in}%
\pgfsys@useobject{currentmarker}{}%
\end{pgfscope}%
\begin{pgfscope}%
\pgfsys@transformshift{-1.015061in}{1.500000in}%
\pgfsys@useobject{currentmarker}{}%
\end{pgfscope}%
\begin{pgfscope}%
\pgfsys@transformshift{-0.979439in}{1.600000in}%
\pgfsys@useobject{currentmarker}{}%
\end{pgfscope}%
\begin{pgfscope}%
\pgfsys@transformshift{-0.943817in}{1.500000in}%
\pgfsys@useobject{currentmarker}{}%
\end{pgfscope}%
\begin{pgfscope}%
\pgfsys@transformshift{-0.908196in}{1.600000in}%
\pgfsys@useobject{currentmarker}{}%
\end{pgfscope}%
\begin{pgfscope}%
\pgfsys@transformshift{-0.872574in}{1.400000in}%
\pgfsys@useobject{currentmarker}{}%
\end{pgfscope}%
\begin{pgfscope}%
\pgfsys@transformshift{-0.836952in}{1.500000in}%
\pgfsys@useobject{currentmarker}{}%
\end{pgfscope}%
\begin{pgfscope}%
\pgfsys@transformshift{-0.801331in}{1.500000in}%
\pgfsys@useobject{currentmarker}{}%
\end{pgfscope}%
\begin{pgfscope}%
\pgfsys@transformshift{-0.765709in}{1.500000in}%
\pgfsys@useobject{currentmarker}{}%
\end{pgfscope}%
\begin{pgfscope}%
\pgfsys@transformshift{-0.730088in}{1.500000in}%
\pgfsys@useobject{currentmarker}{}%
\end{pgfscope}%
\begin{pgfscope}%
\pgfsys@transformshift{-0.694466in}{1.500000in}%
\pgfsys@useobject{currentmarker}{}%
\end{pgfscope}%
\begin{pgfscope}%
\pgfsys@transformshift{-0.658844in}{1.500000in}%
\pgfsys@useobject{currentmarker}{}%
\end{pgfscope}%
\begin{pgfscope}%
\pgfsys@transformshift{-0.623223in}{1.500000in}%
\pgfsys@useobject{currentmarker}{}%
\end{pgfscope}%
\begin{pgfscope}%
\pgfsys@transformshift{-0.587601in}{1.500000in}%
\pgfsys@useobject{currentmarker}{}%
\end{pgfscope}%
\begin{pgfscope}%
\pgfsys@transformshift{-0.551979in}{1.400000in}%
\pgfsys@useobject{currentmarker}{}%
\end{pgfscope}%
\begin{pgfscope}%
\pgfsys@transformshift{-0.516358in}{1.600000in}%
\pgfsys@useobject{currentmarker}{}%
\end{pgfscope}%
\begin{pgfscope}%
\pgfsys@transformshift{-0.480736in}{1.500000in}%
\pgfsys@useobject{currentmarker}{}%
\end{pgfscope}%
\begin{pgfscope}%
\pgfsys@transformshift{-0.445115in}{1.500000in}%
\pgfsys@useobject{currentmarker}{}%
\end{pgfscope}%
\begin{pgfscope}%
\pgfsys@transformshift{-0.409493in}{1.400000in}%
\pgfsys@useobject{currentmarker}{}%
\end{pgfscope}%
\begin{pgfscope}%
\pgfsys@transformshift{-0.373871in}{1.700000in}%
\pgfsys@useobject{currentmarker}{}%
\end{pgfscope}%
\begin{pgfscope}%
\pgfsys@transformshift{-0.338250in}{1.600000in}%
\pgfsys@useobject{currentmarker}{}%
\end{pgfscope}%
\begin{pgfscope}%
\pgfsys@transformshift{-0.302628in}{1.500000in}%
\pgfsys@useobject{currentmarker}{}%
\end{pgfscope}%
\begin{pgfscope}%
\pgfsys@transformshift{-0.267006in}{1.400000in}%
\pgfsys@useobject{currentmarker}{}%
\end{pgfscope}%
\begin{pgfscope}%
\pgfsys@transformshift{-0.231385in}{1.600000in}%
\pgfsys@useobject{currentmarker}{}%
\end{pgfscope}%
\begin{pgfscope}%
\pgfsys@transformshift{-0.195763in}{1.400000in}%
\pgfsys@useobject{currentmarker}{}%
\end{pgfscope}%
\begin{pgfscope}%
\pgfsys@transformshift{-0.160142in}{1.400000in}%
\pgfsys@useobject{currentmarker}{}%
\end{pgfscope}%
\begin{pgfscope}%
\pgfsys@transformshift{-0.124520in}{1.500000in}%
\pgfsys@useobject{currentmarker}{}%
\end{pgfscope}%
\begin{pgfscope}%
\pgfsys@transformshift{-0.088898in}{1.500000in}%
\pgfsys@useobject{currentmarker}{}%
\end{pgfscope}%
\begin{pgfscope}%
\pgfsys@transformshift{-0.053277in}{1.600000in}%
\pgfsys@useobject{currentmarker}{}%
\end{pgfscope}%
\begin{pgfscope}%
\pgfsys@transformshift{-0.017655in}{1.400000in}%
\pgfsys@useobject{currentmarker}{}%
\end{pgfscope}%
\begin{pgfscope}%
\pgfsys@transformshift{0.017967in}{1.600000in}%
\pgfsys@useobject{currentmarker}{}%
\end{pgfscope}%
\begin{pgfscope}%
\pgfsys@transformshift{0.053588in}{1.500000in}%
\pgfsys@useobject{currentmarker}{}%
\end{pgfscope}%
\begin{pgfscope}%
\pgfsys@transformshift{0.089210in}{1.500000in}%
\pgfsys@useobject{currentmarker}{}%
\end{pgfscope}%
\begin{pgfscope}%
\pgfsys@transformshift{0.124832in}{1.500000in}%
\pgfsys@useobject{currentmarker}{}%
\end{pgfscope}%
\begin{pgfscope}%
\pgfsys@transformshift{0.160453in}{1.500000in}%
\pgfsys@useobject{currentmarker}{}%
\end{pgfscope}%
\begin{pgfscope}%
\pgfsys@transformshift{0.196075in}{1.600000in}%
\pgfsys@useobject{currentmarker}{}%
\end{pgfscope}%
\begin{pgfscope}%
\pgfsys@transformshift{0.231696in}{1.200000in}%
\pgfsys@useobject{currentmarker}{}%
\end{pgfscope}%
\begin{pgfscope}%
\pgfsys@transformshift{0.267318in}{1.500000in}%
\pgfsys@useobject{currentmarker}{}%
\end{pgfscope}%
\begin{pgfscope}%
\pgfsys@transformshift{0.302940in}{1.600000in}%
\pgfsys@useobject{currentmarker}{}%
\end{pgfscope}%
\begin{pgfscope}%
\pgfsys@transformshift{0.338561in}{1.500000in}%
\pgfsys@useobject{currentmarker}{}%
\end{pgfscope}%
\begin{pgfscope}%
\pgfsys@transformshift{0.374183in}{1.500000in}%
\pgfsys@useobject{currentmarker}{}%
\end{pgfscope}%
\begin{pgfscope}%
\pgfsys@transformshift{0.409805in}{1.500000in}%
\pgfsys@useobject{currentmarker}{}%
\end{pgfscope}%
\begin{pgfscope}%
\pgfsys@transformshift{0.445426in}{1.600000in}%
\pgfsys@useobject{currentmarker}{}%
\end{pgfscope}%
\begin{pgfscope}%
\pgfsys@transformshift{0.481048in}{1.400000in}%
\pgfsys@useobject{currentmarker}{}%
\end{pgfscope}%
\begin{pgfscope}%
\pgfsys@transformshift{0.516669in}{1.600000in}%
\pgfsys@useobject{currentmarker}{}%
\end{pgfscope}%
\begin{pgfscope}%
\pgfsys@transformshift{0.552291in}{1.600000in}%
\pgfsys@useobject{currentmarker}{}%
\end{pgfscope}%
\begin{pgfscope}%
\pgfsys@transformshift{0.587913in}{1.300000in}%
\pgfsys@useobject{currentmarker}{}%
\end{pgfscope}%
\begin{pgfscope}%
\pgfsys@transformshift{0.623534in}{1.400000in}%
\pgfsys@useobject{currentmarker}{}%
\end{pgfscope}%
\begin{pgfscope}%
\pgfsys@transformshift{0.659156in}{1.500000in}%
\pgfsys@useobject{currentmarker}{}%
\end{pgfscope}%
\begin{pgfscope}%
\pgfsys@transformshift{0.694778in}{1.700000in}%
\pgfsys@useobject{currentmarker}{}%
\end{pgfscope}%
\begin{pgfscope}%
\pgfsys@transformshift{0.730399in}{1.600000in}%
\pgfsys@useobject{currentmarker}{}%
\end{pgfscope}%
\begin{pgfscope}%
\pgfsys@transformshift{0.766021in}{1.400000in}%
\pgfsys@useobject{currentmarker}{}%
\end{pgfscope}%
\begin{pgfscope}%
\pgfsys@transformshift{0.801642in}{1.600000in}%
\pgfsys@useobject{currentmarker}{}%
\end{pgfscope}%
\begin{pgfscope}%
\pgfsys@transformshift{0.837264in}{1.500000in}%
\pgfsys@useobject{currentmarker}{}%
\end{pgfscope}%
\begin{pgfscope}%
\pgfsys@transformshift{0.872886in}{1.400000in}%
\pgfsys@useobject{currentmarker}{}%
\end{pgfscope}%
\begin{pgfscope}%
\pgfsys@transformshift{0.908507in}{1.400000in}%
\pgfsys@useobject{currentmarker}{}%
\end{pgfscope}%
\begin{pgfscope}%
\pgfsys@transformshift{0.944129in}{1.500000in}%
\pgfsys@useobject{currentmarker}{}%
\end{pgfscope}%
\begin{pgfscope}%
\pgfsys@transformshift{0.979751in}{1.700000in}%
\pgfsys@useobject{currentmarker}{}%
\end{pgfscope}%
\begin{pgfscope}%
\pgfsys@transformshift{1.015372in}{1.300000in}%
\pgfsys@useobject{currentmarker}{}%
\end{pgfscope}%
\begin{pgfscope}%
\pgfsys@transformshift{1.050994in}{1.600000in}%
\pgfsys@useobject{currentmarker}{}%
\end{pgfscope}%
\begin{pgfscope}%
\pgfsys@transformshift{1.086615in}{1.500000in}%
\pgfsys@useobject{currentmarker}{}%
\end{pgfscope}%
\begin{pgfscope}%
\pgfsys@transformshift{1.122237in}{1.400000in}%
\pgfsys@useobject{currentmarker}{}%
\end{pgfscope}%
\begin{pgfscope}%
\pgfsys@transformshift{1.157859in}{1.600000in}%
\pgfsys@useobject{currentmarker}{}%
\end{pgfscope}%
\begin{pgfscope}%
\pgfsys@transformshift{1.193480in}{1.500000in}%
\pgfsys@useobject{currentmarker}{}%
\end{pgfscope}%
\begin{pgfscope}%
\pgfsys@transformshift{1.229102in}{1.500000in}%
\pgfsys@useobject{currentmarker}{}%
\end{pgfscope}%
\begin{pgfscope}%
\pgfsys@transformshift{1.264724in}{1.500000in}%
\pgfsys@useobject{currentmarker}{}%
\end{pgfscope}%
\begin{pgfscope}%
\pgfsys@transformshift{1.300345in}{1.400000in}%
\pgfsys@useobject{currentmarker}{}%
\end{pgfscope}%
\begin{pgfscope}%
\pgfsys@transformshift{1.335967in}{1.400000in}%
\pgfsys@useobject{currentmarker}{}%
\end{pgfscope}%
\begin{pgfscope}%
\pgfsys@transformshift{1.371588in}{1.400000in}%
\pgfsys@useobject{currentmarker}{}%
\end{pgfscope}%
\begin{pgfscope}%
\pgfsys@transformshift{1.407210in}{1.500000in}%
\pgfsys@useobject{currentmarker}{}%
\end{pgfscope}%
\begin{pgfscope}%
\pgfsys@transformshift{1.442832in}{1.600000in}%
\pgfsys@useobject{currentmarker}{}%
\end{pgfscope}%
\begin{pgfscope}%
\pgfsys@transformshift{1.478453in}{1.700000in}%
\pgfsys@useobject{currentmarker}{}%
\end{pgfscope}%
\begin{pgfscope}%
\pgfsys@transformshift{1.514075in}{1.400000in}%
\pgfsys@useobject{currentmarker}{}%
\end{pgfscope}%
\begin{pgfscope}%
\pgfsys@transformshift{1.549697in}{1.400000in}%
\pgfsys@useobject{currentmarker}{}%
\end{pgfscope}%
\begin{pgfscope}%
\pgfsys@transformshift{1.585318in}{1.500000in}%
\pgfsys@useobject{currentmarker}{}%
\end{pgfscope}%
\begin{pgfscope}%
\pgfsys@transformshift{1.620940in}{1.500000in}%
\pgfsys@useobject{currentmarker}{}%
\end{pgfscope}%
\begin{pgfscope}%
\pgfsys@transformshift{1.656562in}{1.500000in}%
\pgfsys@useobject{currentmarker}{}%
\end{pgfscope}%
\begin{pgfscope}%
\pgfsys@transformshift{1.692183in}{1.600000in}%
\pgfsys@useobject{currentmarker}{}%
\end{pgfscope}%
\begin{pgfscope}%
\pgfsys@transformshift{1.727805in}{1.400000in}%
\pgfsys@useobject{currentmarker}{}%
\end{pgfscope}%
\begin{pgfscope}%
\pgfsys@transformshift{1.763426in}{1.400000in}%
\pgfsys@useobject{currentmarker}{}%
\end{pgfscope}%
\begin{pgfscope}%
\pgfsys@transformshift{1.799048in}{1.400000in}%
\pgfsys@useobject{currentmarker}{}%
\end{pgfscope}%
\begin{pgfscope}%
\pgfsys@transformshift{1.834670in}{1.500000in}%
\pgfsys@useobject{currentmarker}{}%
\end{pgfscope}%
\begin{pgfscope}%
\pgfsys@transformshift{1.870291in}{1.500000in}%
\pgfsys@useobject{currentmarker}{}%
\end{pgfscope}%
\begin{pgfscope}%
\pgfsys@transformshift{1.905913in}{1.500000in}%
\pgfsys@useobject{currentmarker}{}%
\end{pgfscope}%
\begin{pgfscope}%
\pgfsys@transformshift{1.941535in}{1.400000in}%
\pgfsys@useobject{currentmarker}{}%
\end{pgfscope}%
\begin{pgfscope}%
\pgfsys@transformshift{1.977156in}{1.600000in}%
\pgfsys@useobject{currentmarker}{}%
\end{pgfscope}%
\begin{pgfscope}%
\pgfsys@transformshift{2.012778in}{1.600000in}%
\pgfsys@useobject{currentmarker}{}%
\end{pgfscope}%
\begin{pgfscope}%
\pgfsys@transformshift{2.048399in}{1.600000in}%
\pgfsys@useobject{currentmarker}{}%
\end{pgfscope}%
\begin{pgfscope}%
\pgfsys@transformshift{2.084021in}{1.400000in}%
\pgfsys@useobject{currentmarker}{}%
\end{pgfscope}%
\begin{pgfscope}%
\pgfsys@transformshift{2.119643in}{1.600000in}%
\pgfsys@useobject{currentmarker}{}%
\end{pgfscope}%
\begin{pgfscope}%
\pgfsys@transformshift{2.155264in}{1.600000in}%
\pgfsys@useobject{currentmarker}{}%
\end{pgfscope}%
\begin{pgfscope}%
\pgfsys@transformshift{2.190886in}{1.400000in}%
\pgfsys@useobject{currentmarker}{}%
\end{pgfscope}%
\begin{pgfscope}%
\pgfsys@transformshift{2.226508in}{1.500000in}%
\pgfsys@useobject{currentmarker}{}%
\end{pgfscope}%
\begin{pgfscope}%
\pgfsys@transformshift{2.262129in}{1.500000in}%
\pgfsys@useobject{currentmarker}{}%
\end{pgfscope}%
\begin{pgfscope}%
\pgfsys@transformshift{2.297751in}{1.500000in}%
\pgfsys@useobject{currentmarker}{}%
\end{pgfscope}%
\begin{pgfscope}%
\pgfsys@transformshift{2.333372in}{1.500000in}%
\pgfsys@useobject{currentmarker}{}%
\end{pgfscope}%
\begin{pgfscope}%
\pgfsys@transformshift{2.368994in}{1.600000in}%
\pgfsys@useobject{currentmarker}{}%
\end{pgfscope}%
\begin{pgfscope}%
\pgfsys@transformshift{2.404616in}{1.400000in}%
\pgfsys@useobject{currentmarker}{}%
\end{pgfscope}%
\begin{pgfscope}%
\pgfsys@transformshift{2.440237in}{1.600000in}%
\pgfsys@useobject{currentmarker}{}%
\end{pgfscope}%
\begin{pgfscope}%
\pgfsys@transformshift{2.475859in}{1.600000in}%
\pgfsys@useobject{currentmarker}{}%
\end{pgfscope}%
\begin{pgfscope}%
\pgfsys@transformshift{2.511481in}{1.500000in}%
\pgfsys@useobject{currentmarker}{}%
\end{pgfscope}%
\begin{pgfscope}%
\pgfsys@transformshift{2.547102in}{1.500000in}%
\pgfsys@useobject{currentmarker}{}%
\end{pgfscope}%
\begin{pgfscope}%
\pgfsys@transformshift{2.582724in}{1.500000in}%
\pgfsys@useobject{currentmarker}{}%
\end{pgfscope}%
\begin{pgfscope}%
\pgfsys@transformshift{2.618345in}{1.600000in}%
\pgfsys@useobject{currentmarker}{}%
\end{pgfscope}%
\begin{pgfscope}%
\pgfsys@transformshift{2.653967in}{1.400000in}%
\pgfsys@useobject{currentmarker}{}%
\end{pgfscope}%
\begin{pgfscope}%
\pgfsys@transformshift{2.689589in}{1.500000in}%
\pgfsys@useobject{currentmarker}{}%
\end{pgfscope}%
\begin{pgfscope}%
\pgfsys@transformshift{2.725210in}{1.400000in}%
\pgfsys@useobject{currentmarker}{}%
\end{pgfscope}%
\begin{pgfscope}%
\pgfsys@transformshift{2.760832in}{1.500000in}%
\pgfsys@useobject{currentmarker}{}%
\end{pgfscope}%
\begin{pgfscope}%
\pgfsys@transformshift{2.796454in}{1.400271in}%
\pgfsys@useobject{currentmarker}{}%
\end{pgfscope}%
\begin{pgfscope}%
\pgfsys@transformshift{2.832075in}{1.433070in}%
\pgfsys@useobject{currentmarker}{}%
\end{pgfscope}%
\begin{pgfscope}%
\pgfsys@transformshift{2.867697in}{1.625474in}%
\pgfsys@useobject{currentmarker}{}%
\end{pgfscope}%
\begin{pgfscope}%
\pgfsys@transformshift{2.903318in}{1.590724in}%
\pgfsys@useobject{currentmarker}{}%
\end{pgfscope}%
\begin{pgfscope}%
\pgfsys@transformshift{2.938940in}{1.492720in}%
\pgfsys@useobject{currentmarker}{}%
\end{pgfscope}%
\begin{pgfscope}%
\pgfsys@transformshift{2.974562in}{1.708626in}%
\pgfsys@useobject{currentmarker}{}%
\end{pgfscope}%
\begin{pgfscope}%
\pgfsys@transformshift{3.010183in}{1.390093in}%
\pgfsys@useobject{currentmarker}{}%
\end{pgfscope}%
\begin{pgfscope}%
\pgfsys@transformshift{3.045805in}{1.444190in}%
\pgfsys@useobject{currentmarker}{}%
\end{pgfscope}%
\begin{pgfscope}%
\pgfsys@transformshift{3.081427in}{1.502147in}%
\pgfsys@useobject{currentmarker}{}%
\end{pgfscope}%
\begin{pgfscope}%
\pgfsys@transformshift{3.117048in}{1.423872in}%
\pgfsys@useobject{currentmarker}{}%
\end{pgfscope}%
\begin{pgfscope}%
\pgfsys@transformshift{3.152670in}{1.558206in}%
\pgfsys@useobject{currentmarker}{}%
\end{pgfscope}%
\begin{pgfscope}%
\pgfsys@transformshift{3.188292in}{1.317287in}%
\pgfsys@useobject{currentmarker}{}%
\end{pgfscope}%
\begin{pgfscope}%
\pgfsys@transformshift{3.223913in}{1.716629in}%
\pgfsys@useobject{currentmarker}{}%
\end{pgfscope}%
\begin{pgfscope}%
\pgfsys@transformshift{3.259535in}{1.381439in}%
\pgfsys@useobject{currentmarker}{}%
\end{pgfscope}%
\begin{pgfscope}%
\pgfsys@transformshift{3.295156in}{1.499715in}%
\pgfsys@useobject{currentmarker}{}%
\end{pgfscope}%
\begin{pgfscope}%
\pgfsys@transformshift{3.330778in}{1.547673in}%
\pgfsys@useobject{currentmarker}{}%
\end{pgfscope}%
\begin{pgfscope}%
\pgfsys@transformshift{3.366400in}{1.353988in}%
\pgfsys@useobject{currentmarker}{}%
\end{pgfscope}%
\begin{pgfscope}%
\pgfsys@transformshift{3.402021in}{1.464429in}%
\pgfsys@useobject{currentmarker}{}%
\end{pgfscope}%
\begin{pgfscope}%
\pgfsys@transformshift{3.437643in}{1.622461in}%
\pgfsys@useobject{currentmarker}{}%
\end{pgfscope}%
\begin{pgfscope}%
\pgfsys@transformshift{3.473265in}{1.497682in}%
\pgfsys@useobject{currentmarker}{}%
\end{pgfscope}%
\begin{pgfscope}%
\pgfsys@transformshift{3.508886in}{1.655973in}%
\pgfsys@useobject{currentmarker}{}%
\end{pgfscope}%
\begin{pgfscope}%
\pgfsys@transformshift{3.544508in}{1.425767in}%
\pgfsys@useobject{currentmarker}{}%
\end{pgfscope}%
\begin{pgfscope}%
\pgfsys@transformshift{3.580129in}{1.360382in}%
\pgfsys@useobject{currentmarker}{}%
\end{pgfscope}%
\begin{pgfscope}%
\pgfsys@transformshift{3.615751in}{1.723986in}%
\pgfsys@useobject{currentmarker}{}%
\end{pgfscope}%
\begin{pgfscope}%
\pgfsys@transformshift{3.651373in}{1.689592in}%
\pgfsys@useobject{currentmarker}{}%
\end{pgfscope}%
\begin{pgfscope}%
\pgfsys@transformshift{3.686994in}{1.472627in}%
\pgfsys@useobject{currentmarker}{}%
\end{pgfscope}%
\begin{pgfscope}%
\pgfsys@transformshift{3.722616in}{1.317097in}%
\pgfsys@useobject{currentmarker}{}%
\end{pgfscope}%
\begin{pgfscope}%
\pgfsys@transformshift{3.758238in}{1.342491in}%
\pgfsys@useobject{currentmarker}{}%
\end{pgfscope}%
\begin{pgfscope}%
\pgfsys@transformshift{3.793859in}{1.373342in}%
\pgfsys@useobject{currentmarker}{}%
\end{pgfscope}%
\begin{pgfscope}%
\pgfsys@transformshift{3.829481in}{1.589280in}%
\pgfsys@useobject{currentmarker}{}%
\end{pgfscope}%
\begin{pgfscope}%
\pgfsys@transformshift{3.865102in}{1.509521in}%
\pgfsys@useobject{currentmarker}{}%
\end{pgfscope}%
\begin{pgfscope}%
\pgfsys@transformshift{3.900724in}{1.467756in}%
\pgfsys@useobject{currentmarker}{}%
\end{pgfscope}%
\begin{pgfscope}%
\pgfsys@transformshift{3.936346in}{1.596481in}%
\pgfsys@useobject{currentmarker}{}%
\end{pgfscope}%
\begin{pgfscope}%
\pgfsys@transformshift{3.971967in}{1.520711in}%
\pgfsys@useobject{currentmarker}{}%
\end{pgfscope}%
\begin{pgfscope}%
\pgfsys@transformshift{4.007589in}{1.357074in}%
\pgfsys@useobject{currentmarker}{}%
\end{pgfscope}%
\begin{pgfscope}%
\pgfsys@transformshift{4.043211in}{1.415154in}%
\pgfsys@useobject{currentmarker}{}%
\end{pgfscope}%
\begin{pgfscope}%
\pgfsys@transformshift{4.078832in}{1.599399in}%
\pgfsys@useobject{currentmarker}{}%
\end{pgfscope}%
\begin{pgfscope}%
\pgfsys@transformshift{4.114454in}{1.610854in}%
\pgfsys@useobject{currentmarker}{}%
\end{pgfscope}%
\begin{pgfscope}%
\pgfsys@transformshift{4.150075in}{1.648503in}%
\pgfsys@useobject{currentmarker}{}%
\end{pgfscope}%
\begin{pgfscope}%
\pgfsys@transformshift{4.185697in}{1.610516in}%
\pgfsys@useobject{currentmarker}{}%
\end{pgfscope}%
\begin{pgfscope}%
\pgfsys@transformshift{4.221319in}{1.531113in}%
\pgfsys@useobject{currentmarker}{}%
\end{pgfscope}%
\begin{pgfscope}%
\pgfsys@transformshift{4.256940in}{1.652729in}%
\pgfsys@useobject{currentmarker}{}%
\end{pgfscope}%
\begin{pgfscope}%
\pgfsys@transformshift{4.292562in}{1.687853in}%
\pgfsys@useobject{currentmarker}{}%
\end{pgfscope}%
\begin{pgfscope}%
\pgfsys@transformshift{4.328184in}{1.578572in}%
\pgfsys@useobject{currentmarker}{}%
\end{pgfscope}%
\begin{pgfscope}%
\pgfsys@transformshift{4.363805in}{1.482403in}%
\pgfsys@useobject{currentmarker}{}%
\end{pgfscope}%
\begin{pgfscope}%
\pgfsys@transformshift{4.399427in}{1.642243in}%
\pgfsys@useobject{currentmarker}{}%
\end{pgfscope}%
\begin{pgfscope}%
\pgfsys@transformshift{4.435048in}{1.264579in}%
\pgfsys@useobject{currentmarker}{}%
\end{pgfscope}%
\begin{pgfscope}%
\pgfsys@transformshift{4.470670in}{1.383498in}%
\pgfsys@useobject{currentmarker}{}%
\end{pgfscope}%
\begin{pgfscope}%
\pgfsys@transformshift{4.506292in}{1.537277in}%
\pgfsys@useobject{currentmarker}{}%
\end{pgfscope}%
\begin{pgfscope}%
\pgfsys@transformshift{4.541913in}{1.557947in}%
\pgfsys@useobject{currentmarker}{}%
\end{pgfscope}%
\begin{pgfscope}%
\pgfsys@transformshift{4.577535in}{1.468579in}%
\pgfsys@useobject{currentmarker}{}%
\end{pgfscope}%
\begin{pgfscope}%
\pgfsys@transformshift{4.613157in}{1.484050in}%
\pgfsys@useobject{currentmarker}{}%
\end{pgfscope}%
\begin{pgfscope}%
\pgfsys@transformshift{4.648778in}{1.512925in}%
\pgfsys@useobject{currentmarker}{}%
\end{pgfscope}%
\begin{pgfscope}%
\pgfsys@transformshift{4.684400in}{1.359358in}%
\pgfsys@useobject{currentmarker}{}%
\end{pgfscope}%
\begin{pgfscope}%
\pgfsys@transformshift{4.720021in}{1.624648in}%
\pgfsys@useobject{currentmarker}{}%
\end{pgfscope}%
\begin{pgfscope}%
\pgfsys@transformshift{4.755643in}{1.508379in}%
\pgfsys@useobject{currentmarker}{}%
\end{pgfscope}%
\begin{pgfscope}%
\pgfsys@transformshift{4.791265in}{1.509195in}%
\pgfsys@useobject{currentmarker}{}%
\end{pgfscope}%
\begin{pgfscope}%
\pgfsys@transformshift{4.826886in}{1.525302in}%
\pgfsys@useobject{currentmarker}{}%
\end{pgfscope}%
\begin{pgfscope}%
\pgfsys@transformshift{4.862508in}{1.454780in}%
\pgfsys@useobject{currentmarker}{}%
\end{pgfscope}%
\begin{pgfscope}%
\pgfsys@transformshift{4.898130in}{1.495762in}%
\pgfsys@useobject{currentmarker}{}%
\end{pgfscope}%
\begin{pgfscope}%
\pgfsys@transformshift{4.933751in}{1.446525in}%
\pgfsys@useobject{currentmarker}{}%
\end{pgfscope}%
\begin{pgfscope}%
\pgfsys@transformshift{4.969373in}{1.405534in}%
\pgfsys@useobject{currentmarker}{}%
\end{pgfscope}%
\begin{pgfscope}%
\pgfsys@transformshift{5.004995in}{1.771454in}%
\pgfsys@useobject{currentmarker}{}%
\end{pgfscope}%
\begin{pgfscope}%
\pgfsys@transformshift{5.040616in}{1.543139in}%
\pgfsys@useobject{currentmarker}{}%
\end{pgfscope}%
\begin{pgfscope}%
\pgfsys@transformshift{5.076238in}{1.519620in}%
\pgfsys@useobject{currentmarker}{}%
\end{pgfscope}%
\begin{pgfscope}%
\pgfsys@transformshift{5.111859in}{1.600080in}%
\pgfsys@useobject{currentmarker}{}%
\end{pgfscope}%
\begin{pgfscope}%
\pgfsys@transformshift{5.147481in}{1.483839in}%
\pgfsys@useobject{currentmarker}{}%
\end{pgfscope}%
\begin{pgfscope}%
\pgfsys@transformshift{5.183103in}{1.470329in}%
\pgfsys@useobject{currentmarker}{}%
\end{pgfscope}%
\begin{pgfscope}%
\pgfsys@transformshift{5.218724in}{1.559082in}%
\pgfsys@useobject{currentmarker}{}%
\end{pgfscope}%
\begin{pgfscope}%
\pgfsys@transformshift{5.254346in}{1.449707in}%
\pgfsys@useobject{currentmarker}{}%
\end{pgfscope}%
\begin{pgfscope}%
\pgfsys@transformshift{5.289968in}{1.541884in}%
\pgfsys@useobject{currentmarker}{}%
\end{pgfscope}%
\begin{pgfscope}%
\pgfsys@transformshift{5.325589in}{1.335348in}%
\pgfsys@useobject{currentmarker}{}%
\end{pgfscope}%
\begin{pgfscope}%
\pgfsys@transformshift{5.361211in}{1.429879in}%
\pgfsys@useobject{currentmarker}{}%
\end{pgfscope}%
\begin{pgfscope}%
\pgfsys@transformshift{5.396832in}{1.325297in}%
\pgfsys@useobject{currentmarker}{}%
\end{pgfscope}%
\begin{pgfscope}%
\pgfsys@transformshift{5.432454in}{1.521451in}%
\pgfsys@useobject{currentmarker}{}%
\end{pgfscope}%
\begin{pgfscope}%
\pgfsys@transformshift{5.468076in}{1.418219in}%
\pgfsys@useobject{currentmarker}{}%
\end{pgfscope}%
\begin{pgfscope}%
\pgfsys@transformshift{5.503697in}{1.615499in}%
\pgfsys@useobject{currentmarker}{}%
\end{pgfscope}%
\begin{pgfscope}%
\pgfsys@transformshift{5.539319in}{1.413206in}%
\pgfsys@useobject{currentmarker}{}%
\end{pgfscope}%
\begin{pgfscope}%
\pgfsys@transformshift{5.574941in}{1.611270in}%
\pgfsys@useobject{currentmarker}{}%
\end{pgfscope}%
\begin{pgfscope}%
\pgfsys@transformshift{5.610562in}{1.609632in}%
\pgfsys@useobject{currentmarker}{}%
\end{pgfscope}%
\begin{pgfscope}%
\pgfsys@transformshift{5.646184in}{1.708245in}%
\pgfsys@useobject{currentmarker}{}%
\end{pgfscope}%
\begin{pgfscope}%
\pgfsys@transformshift{5.681805in}{1.307069in}%
\pgfsys@useobject{currentmarker}{}%
\end{pgfscope}%
\begin{pgfscope}%
\pgfsys@transformshift{5.717427in}{1.406070in}%
\pgfsys@useobject{currentmarker}{}%
\end{pgfscope}%
\begin{pgfscope}%
\pgfsys@transformshift{5.753049in}{1.605219in}%
\pgfsys@useobject{currentmarker}{}%
\end{pgfscope}%
\begin{pgfscope}%
\pgfsys@transformshift{5.788670in}{1.504494in}%
\pgfsys@useobject{currentmarker}{}%
\end{pgfscope}%
\begin{pgfscope}%
\pgfsys@transformshift{5.824292in}{1.303876in}%
\pgfsys@useobject{currentmarker}{}%
\end{pgfscope}%
\begin{pgfscope}%
\pgfsys@transformshift{5.859914in}{1.503347in}%
\pgfsys@useobject{currentmarker}{}%
\end{pgfscope}%
\begin{pgfscope}%
\pgfsys@transformshift{5.895535in}{1.702895in}%
\pgfsys@useobject{currentmarker}{}%
\end{pgfscope}%
\begin{pgfscope}%
\pgfsys@transformshift{5.931157in}{1.502507in}%
\pgfsys@useobject{currentmarker}{}%
\end{pgfscope}%
\begin{pgfscope}%
\pgfsys@transformshift{5.966778in}{1.402174in}%
\pgfsys@useobject{currentmarker}{}%
\end{pgfscope}%
\begin{pgfscope}%
\pgfsys@transformshift{6.002400in}{1.601887in}%
\pgfsys@useobject{currentmarker}{}%
\end{pgfscope}%
\begin{pgfscope}%
\pgfsys@transformshift{6.038022in}{1.401641in}%
\pgfsys@useobject{currentmarker}{}%
\end{pgfscope}%
\begin{pgfscope}%
\pgfsys@transformshift{6.073643in}{1.701429in}%
\pgfsys@useobject{currentmarker}{}%
\end{pgfscope}%
\begin{pgfscope}%
\pgfsys@transformshift{6.109265in}{1.501245in}%
\pgfsys@useobject{currentmarker}{}%
\end{pgfscope}%
\begin{pgfscope}%
\pgfsys@transformshift{6.144887in}{1.501087in}%
\pgfsys@useobject{currentmarker}{}%
\end{pgfscope}%
\begin{pgfscope}%
\pgfsys@transformshift{6.180508in}{1.400950in}%
\pgfsys@useobject{currentmarker}{}%
\end{pgfscope}%
\begin{pgfscope}%
\pgfsys@transformshift{6.216130in}{1.600831in}%
\pgfsys@useobject{currentmarker}{}%
\end{pgfscope}%
\begin{pgfscope}%
\pgfsys@transformshift{6.251751in}{1.500728in}%
\pgfsys@useobject{currentmarker}{}%
\end{pgfscope}%
\begin{pgfscope}%
\pgfsys@transformshift{6.287373in}{1.500638in}%
\pgfsys@useobject{currentmarker}{}%
\end{pgfscope}%
\begin{pgfscope}%
\pgfsys@transformshift{6.322995in}{1.400560in}%
\pgfsys@useobject{currentmarker}{}%
\end{pgfscope}%
\begin{pgfscope}%
\pgfsys@transformshift{6.358616in}{1.500492in}%
\pgfsys@useobject{currentmarker}{}%
\end{pgfscope}%
\begin{pgfscope}%
\pgfsys@transformshift{6.394238in}{1.500433in}%
\pgfsys@useobject{currentmarker}{}%
\end{pgfscope}%
\begin{pgfscope}%
\pgfsys@transformshift{6.429860in}{1.600381in}%
\pgfsys@useobject{currentmarker}{}%
\end{pgfscope}%
\begin{pgfscope}%
\pgfsys@transformshift{6.465481in}{1.500336in}%
\pgfsys@useobject{currentmarker}{}%
\end{pgfscope}%
\begin{pgfscope}%
\pgfsys@transformshift{6.501103in}{1.700297in}%
\pgfsys@useobject{currentmarker}{}%
\end{pgfscope}%
\begin{pgfscope}%
\pgfsys@transformshift{6.536725in}{1.500262in}%
\pgfsys@useobject{currentmarker}{}%
\end{pgfscope}%
\begin{pgfscope}%
\pgfsys@transformshift{6.572346in}{1.500232in}%
\pgfsys@useobject{currentmarker}{}%
\end{pgfscope}%
\begin{pgfscope}%
\pgfsys@transformshift{6.607968in}{1.400205in}%
\pgfsys@useobject{currentmarker}{}%
\end{pgfscope}%
\begin{pgfscope}%
\pgfsys@transformshift{6.643589in}{1.700182in}%
\pgfsys@useobject{currentmarker}{}%
\end{pgfscope}%
\begin{pgfscope}%
\pgfsys@transformshift{6.679211in}{1.300161in}%
\pgfsys@useobject{currentmarker}{}%
\end{pgfscope}%
\begin{pgfscope}%
\pgfsys@transformshift{6.714833in}{1.600143in}%
\pgfsys@useobject{currentmarker}{}%
\end{pgfscope}%
\begin{pgfscope}%
\pgfsys@transformshift{6.750454in}{1.500127in}%
\pgfsys@useobject{currentmarker}{}%
\end{pgfscope}%
\begin{pgfscope}%
\pgfsys@transformshift{6.786076in}{1.300113in}%
\pgfsys@useobject{currentmarker}{}%
\end{pgfscope}%
\begin{pgfscope}%
\pgfsys@transformshift{6.821698in}{1.500101in}%
\pgfsys@useobject{currentmarker}{}%
\end{pgfscope}%
\begin{pgfscope}%
\pgfsys@transformshift{6.857319in}{1.400090in}%
\pgfsys@useobject{currentmarker}{}%
\end{pgfscope}%
\begin{pgfscope}%
\pgfsys@transformshift{6.892941in}{1.400080in}%
\pgfsys@useobject{currentmarker}{}%
\end{pgfscope}%
\begin{pgfscope}%
\pgfsys@transformshift{6.928562in}{1.600071in}%
\pgfsys@useobject{currentmarker}{}%
\end{pgfscope}%
\begin{pgfscope}%
\pgfsys@transformshift{6.964184in}{1.500064in}%
\pgfsys@useobject{currentmarker}{}%
\end{pgfscope}%
\begin{pgfscope}%
\pgfsys@transformshift{6.999806in}{1.500057in}%
\pgfsys@useobject{currentmarker}{}%
\end{pgfscope}%
\begin{pgfscope}%
\pgfsys@transformshift{7.035427in}{1.400051in}%
\pgfsys@useobject{currentmarker}{}%
\end{pgfscope}%
\begin{pgfscope}%
\pgfsys@transformshift{7.071049in}{1.500046in}%
\pgfsys@useobject{currentmarker}{}%
\end{pgfscope}%
\begin{pgfscope}%
\pgfsys@transformshift{7.106671in}{1.600041in}%
\pgfsys@useobject{currentmarker}{}%
\end{pgfscope}%
\begin{pgfscope}%
\pgfsys@transformshift{7.142292in}{1.400037in}%
\pgfsys@useobject{currentmarker}{}%
\end{pgfscope}%
\begin{pgfscope}%
\pgfsys@transformshift{7.177914in}{1.400033in}%
\pgfsys@useobject{currentmarker}{}%
\end{pgfscope}%
\begin{pgfscope}%
\pgfsys@transformshift{7.213535in}{1.300030in}%
\pgfsys@useobject{currentmarker}{}%
\end{pgfscope}%
\begin{pgfscope}%
\pgfsys@transformshift{7.249157in}{1.600027in}%
\pgfsys@useobject{currentmarker}{}%
\end{pgfscope}%
\begin{pgfscope}%
\pgfsys@transformshift{7.284779in}{1.500024in}%
\pgfsys@useobject{currentmarker}{}%
\end{pgfscope}%
\begin{pgfscope}%
\pgfsys@transformshift{7.320400in}{1.700022in}%
\pgfsys@useobject{currentmarker}{}%
\end{pgfscope}%
\begin{pgfscope}%
\pgfsys@transformshift{7.356022in}{1.400019in}%
\pgfsys@useobject{currentmarker}{}%
\end{pgfscope}%
\begin{pgfscope}%
\pgfsys@transformshift{7.391644in}{1.400017in}%
\pgfsys@useobject{currentmarker}{}%
\end{pgfscope}%
\begin{pgfscope}%
\pgfsys@transformshift{7.427265in}{1.500016in}%
\pgfsys@useobject{currentmarker}{}%
\end{pgfscope}%
\begin{pgfscope}%
\pgfsys@transformshift{7.462887in}{1.600014in}%
\pgfsys@useobject{currentmarker}{}%
\end{pgfscope}%
\begin{pgfscope}%
\pgfsys@transformshift{7.498508in}{1.600013in}%
\pgfsys@useobject{currentmarker}{}%
\end{pgfscope}%
\begin{pgfscope}%
\pgfsys@transformshift{7.534130in}{1.400012in}%
\pgfsys@useobject{currentmarker}{}%
\end{pgfscope}%
\begin{pgfscope}%
\pgfsys@transformshift{7.569752in}{1.600011in}%
\pgfsys@useobject{currentmarker}{}%
\end{pgfscope}%
\begin{pgfscope}%
\pgfsys@transformshift{7.605373in}{1.500010in}%
\pgfsys@useobject{currentmarker}{}%
\end{pgfscope}%
\begin{pgfscope}%
\pgfsys@transformshift{7.640995in}{1.600009in}%
\pgfsys@useobject{currentmarker}{}%
\end{pgfscope}%
\begin{pgfscope}%
\pgfsys@transformshift{7.676617in}{1.500008in}%
\pgfsys@useobject{currentmarker}{}%
\end{pgfscope}%
\begin{pgfscope}%
\pgfsys@transformshift{7.712238in}{1.400007in}%
\pgfsys@useobject{currentmarker}{}%
\end{pgfscope}%
\begin{pgfscope}%
\pgfsys@transformshift{7.747860in}{1.500006in}%
\pgfsys@useobject{currentmarker}{}%
\end{pgfscope}%
\begin{pgfscope}%
\pgfsys@transformshift{7.783481in}{1.700006in}%
\pgfsys@useobject{currentmarker}{}%
\end{pgfscope}%
\begin{pgfscope}%
\pgfsys@transformshift{7.819103in}{1.600005in}%
\pgfsys@useobject{currentmarker}{}%
\end{pgfscope}%
\begin{pgfscope}%
\pgfsys@transformshift{7.854725in}{1.500005in}%
\pgfsys@useobject{currentmarker}{}%
\end{pgfscope}%
\begin{pgfscope}%
\pgfsys@transformshift{7.890346in}{1.600004in}%
\pgfsys@useobject{currentmarker}{}%
\end{pgfscope}%
\begin{pgfscope}%
\pgfsys@transformshift{7.925968in}{1.400004in}%
\pgfsys@useobject{currentmarker}{}%
\end{pgfscope}%
\begin{pgfscope}%
\pgfsys@transformshift{7.961590in}{1.500004in}%
\pgfsys@useobject{currentmarker}{}%
\end{pgfscope}%
\begin{pgfscope}%
\pgfsys@transformshift{7.997211in}{1.500003in}%
\pgfsys@useobject{currentmarker}{}%
\end{pgfscope}%
\begin{pgfscope}%
\pgfsys@transformshift{8.032833in}{1.600003in}%
\pgfsys@useobject{currentmarker}{}%
\end{pgfscope}%
\begin{pgfscope}%
\pgfsys@transformshift{8.068455in}{1.700003in}%
\pgfsys@useobject{currentmarker}{}%
\end{pgfscope}%
\begin{pgfscope}%
\pgfsys@transformshift{8.104076in}{1.500002in}%
\pgfsys@useobject{currentmarker}{}%
\end{pgfscope}%
\begin{pgfscope}%
\pgfsys@transformshift{8.139698in}{1.500002in}%
\pgfsys@useobject{currentmarker}{}%
\end{pgfscope}%
\begin{pgfscope}%
\pgfsys@transformshift{8.175319in}{1.500002in}%
\pgfsys@useobject{currentmarker}{}%
\end{pgfscope}%
\begin{pgfscope}%
\pgfsys@transformshift{8.210941in}{1.400002in}%
\pgfsys@useobject{currentmarker}{}%
\end{pgfscope}%
\begin{pgfscope}%
\pgfsys@transformshift{8.246563in}{1.400002in}%
\pgfsys@useobject{currentmarker}{}%
\end{pgfscope}%
\begin{pgfscope}%
\pgfsys@transformshift{8.282184in}{1.600002in}%
\pgfsys@useobject{currentmarker}{}%
\end{pgfscope}%
\begin{pgfscope}%
\pgfsys@transformshift{8.317806in}{1.700001in}%
\pgfsys@useobject{currentmarker}{}%
\end{pgfscope}%
\begin{pgfscope}%
\pgfsys@transformshift{8.353428in}{1.600001in}%
\pgfsys@useobject{currentmarker}{}%
\end{pgfscope}%
\begin{pgfscope}%
\pgfsys@transformshift{8.389049in}{1.400001in}%
\pgfsys@useobject{currentmarker}{}%
\end{pgfscope}%
\begin{pgfscope}%
\pgfsys@transformshift{8.424671in}{1.500001in}%
\pgfsys@useobject{currentmarker}{}%
\end{pgfscope}%
\begin{pgfscope}%
\pgfsys@transformshift{8.460292in}{1.400001in}%
\pgfsys@useobject{currentmarker}{}%
\end{pgfscope}%
\begin{pgfscope}%
\pgfsys@transformshift{8.495914in}{1.600001in}%
\pgfsys@useobject{currentmarker}{}%
\end{pgfscope}%
\begin{pgfscope}%
\pgfsys@transformshift{8.531536in}{1.500001in}%
\pgfsys@useobject{currentmarker}{}%
\end{pgfscope}%
\begin{pgfscope}%
\pgfsys@transformshift{8.567157in}{1.600001in}%
\pgfsys@useobject{currentmarker}{}%
\end{pgfscope}%
\begin{pgfscope}%
\pgfsys@transformshift{8.602779in}{1.500001in}%
\pgfsys@useobject{currentmarker}{}%
\end{pgfscope}%
\begin{pgfscope}%
\pgfsys@transformshift{8.638401in}{1.400001in}%
\pgfsys@useobject{currentmarker}{}%
\end{pgfscope}%
\begin{pgfscope}%
\pgfsys@transformshift{8.674022in}{1.500001in}%
\pgfsys@useobject{currentmarker}{}%
\end{pgfscope}%
\begin{pgfscope}%
\pgfsys@transformshift{8.709644in}{1.500001in}%
\pgfsys@useobject{currentmarker}{}%
\end{pgfscope}%
\begin{pgfscope}%
\pgfsys@transformshift{8.745265in}{1.700001in}%
\pgfsys@useobject{currentmarker}{}%
\end{pgfscope}%
\begin{pgfscope}%
\pgfsys@transformshift{8.780887in}{1.500000in}%
\pgfsys@useobject{currentmarker}{}%
\end{pgfscope}%
\begin{pgfscope}%
\pgfsys@transformshift{8.816509in}{1.600000in}%
\pgfsys@useobject{currentmarker}{}%
\end{pgfscope}%
\begin{pgfscope}%
\pgfsys@transformshift{8.852130in}{1.600000in}%
\pgfsys@useobject{currentmarker}{}%
\end{pgfscope}%
\begin{pgfscope}%
\pgfsys@transformshift{8.887752in}{1.600000in}%
\pgfsys@useobject{currentmarker}{}%
\end{pgfscope}%
\begin{pgfscope}%
\pgfsys@transformshift{8.923374in}{1.500000in}%
\pgfsys@useobject{currentmarker}{}%
\end{pgfscope}%
\begin{pgfscope}%
\pgfsys@transformshift{8.958995in}{1.500000in}%
\pgfsys@useobject{currentmarker}{}%
\end{pgfscope}%
\begin{pgfscope}%
\pgfsys@transformshift{8.994617in}{1.500000in}%
\pgfsys@useobject{currentmarker}{}%
\end{pgfscope}%
\begin{pgfscope}%
\pgfsys@transformshift{9.030238in}{1.500000in}%
\pgfsys@useobject{currentmarker}{}%
\end{pgfscope}%
\begin{pgfscope}%
\pgfsys@transformshift{9.065860in}{1.500000in}%
\pgfsys@useobject{currentmarker}{}%
\end{pgfscope}%
\begin{pgfscope}%
\pgfsys@transformshift{9.101482in}{1.500000in}%
\pgfsys@useobject{currentmarker}{}%
\end{pgfscope}%
\begin{pgfscope}%
\pgfsys@transformshift{9.137103in}{1.500000in}%
\pgfsys@useobject{currentmarker}{}%
\end{pgfscope}%
\begin{pgfscope}%
\pgfsys@transformshift{9.172725in}{1.400000in}%
\pgfsys@useobject{currentmarker}{}%
\end{pgfscope}%
\begin{pgfscope}%
\pgfsys@transformshift{9.208347in}{1.400000in}%
\pgfsys@useobject{currentmarker}{}%
\end{pgfscope}%
\begin{pgfscope}%
\pgfsys@transformshift{9.243968in}{1.600000in}%
\pgfsys@useobject{currentmarker}{}%
\end{pgfscope}%
\begin{pgfscope}%
\pgfsys@transformshift{9.279590in}{1.600000in}%
\pgfsys@useobject{currentmarker}{}%
\end{pgfscope}%
\begin{pgfscope}%
\pgfsys@transformshift{9.315211in}{1.400000in}%
\pgfsys@useobject{currentmarker}{}%
\end{pgfscope}%
\begin{pgfscope}%
\pgfsys@transformshift{9.350833in}{1.600000in}%
\pgfsys@useobject{currentmarker}{}%
\end{pgfscope}%
\begin{pgfscope}%
\pgfsys@transformshift{9.386455in}{1.400000in}%
\pgfsys@useobject{currentmarker}{}%
\end{pgfscope}%
\begin{pgfscope}%
\pgfsys@transformshift{9.422076in}{1.600000in}%
\pgfsys@useobject{currentmarker}{}%
\end{pgfscope}%
\begin{pgfscope}%
\pgfsys@transformshift{9.457698in}{1.500000in}%
\pgfsys@useobject{currentmarker}{}%
\end{pgfscope}%
\begin{pgfscope}%
\pgfsys@transformshift{9.493320in}{1.600000in}%
\pgfsys@useobject{currentmarker}{}%
\end{pgfscope}%
\begin{pgfscope}%
\pgfsys@transformshift{9.528941in}{1.400000in}%
\pgfsys@useobject{currentmarker}{}%
\end{pgfscope}%
\begin{pgfscope}%
\pgfsys@transformshift{9.564563in}{1.400000in}%
\pgfsys@useobject{currentmarker}{}%
\end{pgfscope}%
\begin{pgfscope}%
\pgfsys@transformshift{9.600185in}{1.300000in}%
\pgfsys@useobject{currentmarker}{}%
\end{pgfscope}%
\begin{pgfscope}%
\pgfsys@transformshift{9.635806in}{1.300000in}%
\pgfsys@useobject{currentmarker}{}%
\end{pgfscope}%
\begin{pgfscope}%
\pgfsys@transformshift{9.671428in}{1.600000in}%
\pgfsys@useobject{currentmarker}{}%
\end{pgfscope}%
\begin{pgfscope}%
\pgfsys@transformshift{9.707049in}{1.400000in}%
\pgfsys@useobject{currentmarker}{}%
\end{pgfscope}%
\begin{pgfscope}%
\pgfsys@transformshift{9.742671in}{1.600000in}%
\pgfsys@useobject{currentmarker}{}%
\end{pgfscope}%
\begin{pgfscope}%
\pgfsys@transformshift{9.778293in}{1.300000in}%
\pgfsys@useobject{currentmarker}{}%
\end{pgfscope}%
\begin{pgfscope}%
\pgfsys@transformshift{9.813914in}{1.600000in}%
\pgfsys@useobject{currentmarker}{}%
\end{pgfscope}%
\begin{pgfscope}%
\pgfsys@transformshift{9.849536in}{1.300000in}%
\pgfsys@useobject{currentmarker}{}%
\end{pgfscope}%
\begin{pgfscope}%
\pgfsys@transformshift{9.885158in}{1.600000in}%
\pgfsys@useobject{currentmarker}{}%
\end{pgfscope}%
\begin{pgfscope}%
\pgfsys@transformshift{9.920779in}{1.700000in}%
\pgfsys@useobject{currentmarker}{}%
\end{pgfscope}%
\begin{pgfscope}%
\pgfsys@transformshift{9.956401in}{1.600000in}%
\pgfsys@useobject{currentmarker}{}%
\end{pgfscope}%
\begin{pgfscope}%
\pgfsys@transformshift{9.992022in}{1.500000in}%
\pgfsys@useobject{currentmarker}{}%
\end{pgfscope}%
\begin{pgfscope}%
\pgfsys@transformshift{10.027644in}{1.600000in}%
\pgfsys@useobject{currentmarker}{}%
\end{pgfscope}%
\begin{pgfscope}%
\pgfsys@transformshift{10.063266in}{1.400000in}%
\pgfsys@useobject{currentmarker}{}%
\end{pgfscope}%
\begin{pgfscope}%
\pgfsys@transformshift{10.098887in}{1.800000in}%
\pgfsys@useobject{currentmarker}{}%
\end{pgfscope}%
\begin{pgfscope}%
\pgfsys@transformshift{10.134509in}{1.500000in}%
\pgfsys@useobject{currentmarker}{}%
\end{pgfscope}%
\begin{pgfscope}%
\pgfsys@transformshift{10.170131in}{1.500000in}%
\pgfsys@useobject{currentmarker}{}%
\end{pgfscope}%
\begin{pgfscope}%
\pgfsys@transformshift{10.205752in}{1.500000in}%
\pgfsys@useobject{currentmarker}{}%
\end{pgfscope}%
\begin{pgfscope}%
\pgfsys@transformshift{10.241374in}{1.700000in}%
\pgfsys@useobject{currentmarker}{}%
\end{pgfscope}%
\begin{pgfscope}%
\pgfsys@transformshift{10.276995in}{1.500000in}%
\pgfsys@useobject{currentmarker}{}%
\end{pgfscope}%
\begin{pgfscope}%
\pgfsys@transformshift{10.312617in}{1.400000in}%
\pgfsys@useobject{currentmarker}{}%
\end{pgfscope}%
\begin{pgfscope}%
\pgfsys@transformshift{10.348239in}{1.400000in}%
\pgfsys@useobject{currentmarker}{}%
\end{pgfscope}%
\begin{pgfscope}%
\pgfsys@transformshift{10.383860in}{1.500000in}%
\pgfsys@useobject{currentmarker}{}%
\end{pgfscope}%
\begin{pgfscope}%
\pgfsys@transformshift{10.419482in}{1.600000in}%
\pgfsys@useobject{currentmarker}{}%
\end{pgfscope}%
\begin{pgfscope}%
\pgfsys@transformshift{10.455104in}{1.400000in}%
\pgfsys@useobject{currentmarker}{}%
\end{pgfscope}%
\begin{pgfscope}%
\pgfsys@transformshift{10.490725in}{1.700000in}%
\pgfsys@useobject{currentmarker}{}%
\end{pgfscope}%
\begin{pgfscope}%
\pgfsys@transformshift{10.526347in}{1.600000in}%
\pgfsys@useobject{currentmarker}{}%
\end{pgfscope}%
\begin{pgfscope}%
\pgfsys@transformshift{10.561968in}{1.600000in}%
\pgfsys@useobject{currentmarker}{}%
\end{pgfscope}%
\begin{pgfscope}%
\pgfsys@transformshift{10.597590in}{1.600000in}%
\pgfsys@useobject{currentmarker}{}%
\end{pgfscope}%
\begin{pgfscope}%
\pgfsys@transformshift{10.633212in}{1.300000in}%
\pgfsys@useobject{currentmarker}{}%
\end{pgfscope}%
\begin{pgfscope}%
\pgfsys@transformshift{10.668833in}{1.600000in}%
\pgfsys@useobject{currentmarker}{}%
\end{pgfscope}%
\begin{pgfscope}%
\pgfsys@transformshift{10.704455in}{1.500000in}%
\pgfsys@useobject{currentmarker}{}%
\end{pgfscope}%
\begin{pgfscope}%
\pgfsys@transformshift{10.740077in}{1.500000in}%
\pgfsys@useobject{currentmarker}{}%
\end{pgfscope}%
\begin{pgfscope}%
\pgfsys@transformshift{10.775698in}{1.500000in}%
\pgfsys@useobject{currentmarker}{}%
\end{pgfscope}%
\begin{pgfscope}%
\pgfsys@transformshift{10.811320in}{1.300000in}%
\pgfsys@useobject{currentmarker}{}%
\end{pgfscope}%
\begin{pgfscope}%
\pgfsys@transformshift{10.846941in}{1.500000in}%
\pgfsys@useobject{currentmarker}{}%
\end{pgfscope}%
\begin{pgfscope}%
\pgfsys@transformshift{10.882563in}{1.700000in}%
\pgfsys@useobject{currentmarker}{}%
\end{pgfscope}%
\begin{pgfscope}%
\pgfsys@transformshift{10.918185in}{1.500000in}%
\pgfsys@useobject{currentmarker}{}%
\end{pgfscope}%
\begin{pgfscope}%
\pgfsys@transformshift{10.953806in}{1.400000in}%
\pgfsys@useobject{currentmarker}{}%
\end{pgfscope}%
\begin{pgfscope}%
\pgfsys@transformshift{10.989428in}{1.400000in}%
\pgfsys@useobject{currentmarker}{}%
\end{pgfscope}%
\begin{pgfscope}%
\pgfsys@transformshift{11.025050in}{1.500000in}%
\pgfsys@useobject{currentmarker}{}%
\end{pgfscope}%
\begin{pgfscope}%
\pgfsys@transformshift{11.060671in}{1.500000in}%
\pgfsys@useobject{currentmarker}{}%
\end{pgfscope}%
\begin{pgfscope}%
\pgfsys@transformshift{11.096293in}{1.600000in}%
\pgfsys@useobject{currentmarker}{}%
\end{pgfscope}%
\begin{pgfscope}%
\pgfsys@transformshift{11.131915in}{1.400000in}%
\pgfsys@useobject{currentmarker}{}%
\end{pgfscope}%
\begin{pgfscope}%
\pgfsys@transformshift{11.167536in}{1.400000in}%
\pgfsys@useobject{currentmarker}{}%
\end{pgfscope}%
\begin{pgfscope}%
\pgfsys@transformshift{11.203158in}{1.600000in}%
\pgfsys@useobject{currentmarker}{}%
\end{pgfscope}%
\begin{pgfscope}%
\pgfsys@transformshift{11.238779in}{1.700000in}%
\pgfsys@useobject{currentmarker}{}%
\end{pgfscope}%
\begin{pgfscope}%
\pgfsys@transformshift{11.274401in}{1.500000in}%
\pgfsys@useobject{currentmarker}{}%
\end{pgfscope}%
\begin{pgfscope}%
\pgfsys@transformshift{11.310023in}{1.600000in}%
\pgfsys@useobject{currentmarker}{}%
\end{pgfscope}%
\begin{pgfscope}%
\pgfsys@transformshift{11.345644in}{1.400000in}%
\pgfsys@useobject{currentmarker}{}%
\end{pgfscope}%
\begin{pgfscope}%
\pgfsys@transformshift{11.381266in}{1.500000in}%
\pgfsys@useobject{currentmarker}{}%
\end{pgfscope}%
\begin{pgfscope}%
\pgfsys@transformshift{11.416888in}{1.600000in}%
\pgfsys@useobject{currentmarker}{}%
\end{pgfscope}%
\begin{pgfscope}%
\pgfsys@transformshift{11.452509in}{1.500000in}%
\pgfsys@useobject{currentmarker}{}%
\end{pgfscope}%
\begin{pgfscope}%
\pgfsys@transformshift{11.488131in}{1.300000in}%
\pgfsys@useobject{currentmarker}{}%
\end{pgfscope}%
\begin{pgfscope}%
\pgfsys@transformshift{11.523752in}{1.600000in}%
\pgfsys@useobject{currentmarker}{}%
\end{pgfscope}%
\begin{pgfscope}%
\pgfsys@transformshift{11.559374in}{1.500000in}%
\pgfsys@useobject{currentmarker}{}%
\end{pgfscope}%
\begin{pgfscope}%
\pgfsys@transformshift{11.594996in}{1.500000in}%
\pgfsys@useobject{currentmarker}{}%
\end{pgfscope}%
\begin{pgfscope}%
\pgfsys@transformshift{11.630617in}{1.500000in}%
\pgfsys@useobject{currentmarker}{}%
\end{pgfscope}%
\begin{pgfscope}%
\pgfsys@transformshift{11.666239in}{1.400000in}%
\pgfsys@useobject{currentmarker}{}%
\end{pgfscope}%
\begin{pgfscope}%
\pgfsys@transformshift{11.701861in}{1.500000in}%
\pgfsys@useobject{currentmarker}{}%
\end{pgfscope}%
\begin{pgfscope}%
\pgfsys@transformshift{11.737482in}{1.400000in}%
\pgfsys@useobject{currentmarker}{}%
\end{pgfscope}%
\begin{pgfscope}%
\pgfsys@transformshift{11.773104in}{1.500000in}%
\pgfsys@useobject{currentmarker}{}%
\end{pgfscope}%
\begin{pgfscope}%
\pgfsys@transformshift{11.808725in}{1.400000in}%
\pgfsys@useobject{currentmarker}{}%
\end{pgfscope}%
\begin{pgfscope}%
\pgfsys@transformshift{11.844347in}{1.600000in}%
\pgfsys@useobject{currentmarker}{}%
\end{pgfscope}%
\begin{pgfscope}%
\pgfsys@transformshift{11.879969in}{1.500000in}%
\pgfsys@useobject{currentmarker}{}%
\end{pgfscope}%
\begin{pgfscope}%
\pgfsys@transformshift{11.915590in}{1.400000in}%
\pgfsys@useobject{currentmarker}{}%
\end{pgfscope}%
\begin{pgfscope}%
\pgfsys@transformshift{11.951212in}{1.600000in}%
\pgfsys@useobject{currentmarker}{}%
\end{pgfscope}%
\begin{pgfscope}%
\pgfsys@transformshift{11.986834in}{1.600000in}%
\pgfsys@useobject{currentmarker}{}%
\end{pgfscope}%
\begin{pgfscope}%
\pgfsys@transformshift{12.022455in}{1.200000in}%
\pgfsys@useobject{currentmarker}{}%
\end{pgfscope}%
\begin{pgfscope}%
\pgfsys@transformshift{12.058077in}{1.500000in}%
\pgfsys@useobject{currentmarker}{}%
\end{pgfscope}%
\begin{pgfscope}%
\pgfsys@transformshift{12.093698in}{1.600000in}%
\pgfsys@useobject{currentmarker}{}%
\end{pgfscope}%
\begin{pgfscope}%
\pgfsys@transformshift{12.129320in}{1.400000in}%
\pgfsys@useobject{currentmarker}{}%
\end{pgfscope}%
\begin{pgfscope}%
\pgfsys@transformshift{12.164942in}{1.400000in}%
\pgfsys@useobject{currentmarker}{}%
\end{pgfscope}%
\begin{pgfscope}%
\pgfsys@transformshift{12.200563in}{1.500000in}%
\pgfsys@useobject{currentmarker}{}%
\end{pgfscope}%
\begin{pgfscope}%
\pgfsys@transformshift{12.236185in}{1.700000in}%
\pgfsys@useobject{currentmarker}{}%
\end{pgfscope}%
\begin{pgfscope}%
\pgfsys@transformshift{12.271807in}{1.300000in}%
\pgfsys@useobject{currentmarker}{}%
\end{pgfscope}%
\begin{pgfscope}%
\pgfsys@transformshift{12.307428in}{1.500000in}%
\pgfsys@useobject{currentmarker}{}%
\end{pgfscope}%
\begin{pgfscope}%
\pgfsys@transformshift{12.343050in}{1.500000in}%
\pgfsys@useobject{currentmarker}{}%
\end{pgfscope}%
\begin{pgfscope}%
\pgfsys@transformshift{12.378671in}{1.500000in}%
\pgfsys@useobject{currentmarker}{}%
\end{pgfscope}%
\begin{pgfscope}%
\pgfsys@transformshift{12.414293in}{1.400000in}%
\pgfsys@useobject{currentmarker}{}%
\end{pgfscope}%
\begin{pgfscope}%
\pgfsys@transformshift{12.449915in}{1.400000in}%
\pgfsys@useobject{currentmarker}{}%
\end{pgfscope}%
\begin{pgfscope}%
\pgfsys@transformshift{12.485536in}{1.600000in}%
\pgfsys@useobject{currentmarker}{}%
\end{pgfscope}%
\begin{pgfscope}%
\pgfsys@transformshift{12.521158in}{1.500000in}%
\pgfsys@useobject{currentmarker}{}%
\end{pgfscope}%
\begin{pgfscope}%
\pgfsys@transformshift{12.556780in}{1.500000in}%
\pgfsys@useobject{currentmarker}{}%
\end{pgfscope}%
\begin{pgfscope}%
\pgfsys@transformshift{12.592401in}{1.600000in}%
\pgfsys@useobject{currentmarker}{}%
\end{pgfscope}%
\begin{pgfscope}%
\pgfsys@transformshift{12.628023in}{1.400000in}%
\pgfsys@useobject{currentmarker}{}%
\end{pgfscope}%
\begin{pgfscope}%
\pgfsys@transformshift{12.663645in}{1.600000in}%
\pgfsys@useobject{currentmarker}{}%
\end{pgfscope}%
\begin{pgfscope}%
\pgfsys@transformshift{12.699266in}{1.400000in}%
\pgfsys@useobject{currentmarker}{}%
\end{pgfscope}%
\begin{pgfscope}%
\pgfsys@transformshift{12.734888in}{1.400000in}%
\pgfsys@useobject{currentmarker}{}%
\end{pgfscope}%
\begin{pgfscope}%
\pgfsys@transformshift{12.770509in}{1.500000in}%
\pgfsys@useobject{currentmarker}{}%
\end{pgfscope}%
\begin{pgfscope}%
\pgfsys@transformshift{12.806131in}{1.500000in}%
\pgfsys@useobject{currentmarker}{}%
\end{pgfscope}%
\begin{pgfscope}%
\pgfsys@transformshift{12.841753in}{1.400000in}%
\pgfsys@useobject{currentmarker}{}%
\end{pgfscope}%
\begin{pgfscope}%
\pgfsys@transformshift{12.877374in}{1.400000in}%
\pgfsys@useobject{currentmarker}{}%
\end{pgfscope}%
\begin{pgfscope}%
\pgfsys@transformshift{12.912996in}{1.500000in}%
\pgfsys@useobject{currentmarker}{}%
\end{pgfscope}%
\begin{pgfscope}%
\pgfsys@transformshift{12.948618in}{1.300000in}%
\pgfsys@useobject{currentmarker}{}%
\end{pgfscope}%
\begin{pgfscope}%
\pgfsys@transformshift{12.984239in}{1.600000in}%
\pgfsys@useobject{currentmarker}{}%
\end{pgfscope}%
\begin{pgfscope}%
\pgfsys@transformshift{13.019861in}{1.400000in}%
\pgfsys@useobject{currentmarker}{}%
\end{pgfscope}%
\begin{pgfscope}%
\pgfsys@transformshift{13.055482in}{1.800000in}%
\pgfsys@useobject{currentmarker}{}%
\end{pgfscope}%
\begin{pgfscope}%
\pgfsys@transformshift{13.091104in}{1.400000in}%
\pgfsys@useobject{currentmarker}{}%
\end{pgfscope}%
\begin{pgfscope}%
\pgfsys@transformshift{13.126726in}{1.800000in}%
\pgfsys@useobject{currentmarker}{}%
\end{pgfscope}%
\begin{pgfscope}%
\pgfsys@transformshift{13.162347in}{1.700000in}%
\pgfsys@useobject{currentmarker}{}%
\end{pgfscope}%
\begin{pgfscope}%
\pgfsys@transformshift{13.197969in}{1.500000in}%
\pgfsys@useobject{currentmarker}{}%
\end{pgfscope}%
\begin{pgfscope}%
\pgfsys@transformshift{13.233591in}{1.400000in}%
\pgfsys@useobject{currentmarker}{}%
\end{pgfscope}%
\begin{pgfscope}%
\pgfsys@transformshift{13.269212in}{1.500000in}%
\pgfsys@useobject{currentmarker}{}%
\end{pgfscope}%
\begin{pgfscope}%
\pgfsys@transformshift{13.304834in}{1.600000in}%
\pgfsys@useobject{currentmarker}{}%
\end{pgfscope}%
\begin{pgfscope}%
\pgfsys@transformshift{13.340455in}{1.700000in}%
\pgfsys@useobject{currentmarker}{}%
\end{pgfscope}%
\begin{pgfscope}%
\pgfsys@transformshift{13.376077in}{1.400000in}%
\pgfsys@useobject{currentmarker}{}%
\end{pgfscope}%
\begin{pgfscope}%
\pgfsys@transformshift{13.411699in}{1.400000in}%
\pgfsys@useobject{currentmarker}{}%
\end{pgfscope}%
\begin{pgfscope}%
\pgfsys@transformshift{13.447320in}{1.500000in}%
\pgfsys@useobject{currentmarker}{}%
\end{pgfscope}%
\begin{pgfscope}%
\pgfsys@transformshift{13.482942in}{1.400000in}%
\pgfsys@useobject{currentmarker}{}%
\end{pgfscope}%
\begin{pgfscope}%
\pgfsys@transformshift{13.518564in}{1.400000in}%
\pgfsys@useobject{currentmarker}{}%
\end{pgfscope}%
\begin{pgfscope}%
\pgfsys@transformshift{13.554185in}{1.400000in}%
\pgfsys@useobject{currentmarker}{}%
\end{pgfscope}%
\begin{pgfscope}%
\pgfsys@transformshift{13.589807in}{1.500000in}%
\pgfsys@useobject{currentmarker}{}%
\end{pgfscope}%
\begin{pgfscope}%
\pgfsys@transformshift{13.625428in}{1.600000in}%
\pgfsys@useobject{currentmarker}{}%
\end{pgfscope}%
\begin{pgfscope}%
\pgfsys@transformshift{13.661050in}{1.300000in}%
\pgfsys@useobject{currentmarker}{}%
\end{pgfscope}%
\begin{pgfscope}%
\pgfsys@transformshift{13.696672in}{1.500000in}%
\pgfsys@useobject{currentmarker}{}%
\end{pgfscope}%
\begin{pgfscope}%
\pgfsys@transformshift{13.732293in}{1.600000in}%
\pgfsys@useobject{currentmarker}{}%
\end{pgfscope}%
\begin{pgfscope}%
\pgfsys@transformshift{13.767915in}{1.300000in}%
\pgfsys@useobject{currentmarker}{}%
\end{pgfscope}%
\begin{pgfscope}%
\pgfsys@transformshift{13.803537in}{1.500000in}%
\pgfsys@useobject{currentmarker}{}%
\end{pgfscope}%
\begin{pgfscope}%
\pgfsys@transformshift{13.839158in}{1.500000in}%
\pgfsys@useobject{currentmarker}{}%
\end{pgfscope}%
\begin{pgfscope}%
\pgfsys@transformshift{13.874780in}{1.500000in}%
\pgfsys@useobject{currentmarker}{}%
\end{pgfscope}%
\begin{pgfscope}%
\pgfsys@transformshift{13.910401in}{1.400000in}%
\pgfsys@useobject{currentmarker}{}%
\end{pgfscope}%
\begin{pgfscope}%
\pgfsys@transformshift{13.946023in}{1.400000in}%
\pgfsys@useobject{currentmarker}{}%
\end{pgfscope}%
\begin{pgfscope}%
\pgfsys@transformshift{13.981645in}{1.500000in}%
\pgfsys@useobject{currentmarker}{}%
\end{pgfscope}%
\begin{pgfscope}%
\pgfsys@transformshift{14.017266in}{1.600000in}%
\pgfsys@useobject{currentmarker}{}%
\end{pgfscope}%
\begin{pgfscope}%
\pgfsys@transformshift{14.052888in}{1.500000in}%
\pgfsys@useobject{currentmarker}{}%
\end{pgfscope}%
\begin{pgfscope}%
\pgfsys@transformshift{14.088510in}{1.700000in}%
\pgfsys@useobject{currentmarker}{}%
\end{pgfscope}%
\begin{pgfscope}%
\pgfsys@transformshift{14.124131in}{1.500000in}%
\pgfsys@useobject{currentmarker}{}%
\end{pgfscope}%
\begin{pgfscope}%
\pgfsys@transformshift{14.159753in}{1.500000in}%
\pgfsys@useobject{currentmarker}{}%
\end{pgfscope}%
\begin{pgfscope}%
\pgfsys@transformshift{14.195375in}{1.400000in}%
\pgfsys@useobject{currentmarker}{}%
\end{pgfscope}%
\begin{pgfscope}%
\pgfsys@transformshift{14.230996in}{1.500000in}%
\pgfsys@useobject{currentmarker}{}%
\end{pgfscope}%
\begin{pgfscope}%
\pgfsys@transformshift{14.266618in}{1.400000in}%
\pgfsys@useobject{currentmarker}{}%
\end{pgfscope}%
\begin{pgfscope}%
\pgfsys@transformshift{14.302239in}{1.500000in}%
\pgfsys@useobject{currentmarker}{}%
\end{pgfscope}%
\begin{pgfscope}%
\pgfsys@transformshift{14.337861in}{1.600000in}%
\pgfsys@useobject{currentmarker}{}%
\end{pgfscope}%
\begin{pgfscope}%
\pgfsys@transformshift{14.373483in}{1.400000in}%
\pgfsys@useobject{currentmarker}{}%
\end{pgfscope}%
\begin{pgfscope}%
\pgfsys@transformshift{14.409104in}{1.600000in}%
\pgfsys@useobject{currentmarker}{}%
\end{pgfscope}%
\begin{pgfscope}%
\pgfsys@transformshift{14.444726in}{1.500000in}%
\pgfsys@useobject{currentmarker}{}%
\end{pgfscope}%
\begin{pgfscope}%
\pgfsys@transformshift{14.480348in}{1.400000in}%
\pgfsys@useobject{currentmarker}{}%
\end{pgfscope}%
\begin{pgfscope}%
\pgfsys@transformshift{14.515969in}{1.600000in}%
\pgfsys@useobject{currentmarker}{}%
\end{pgfscope}%
\begin{pgfscope}%
\pgfsys@transformshift{14.551591in}{1.600000in}%
\pgfsys@useobject{currentmarker}{}%
\end{pgfscope}%
\begin{pgfscope}%
\pgfsys@transformshift{14.587212in}{1.400000in}%
\pgfsys@useobject{currentmarker}{}%
\end{pgfscope}%
\begin{pgfscope}%
\pgfsys@transformshift{14.622834in}{1.400000in}%
\pgfsys@useobject{currentmarker}{}%
\end{pgfscope}%
\begin{pgfscope}%
\pgfsys@transformshift{14.658456in}{1.300000in}%
\pgfsys@useobject{currentmarker}{}%
\end{pgfscope}%
\begin{pgfscope}%
\pgfsys@transformshift{14.694077in}{1.600000in}%
\pgfsys@useobject{currentmarker}{}%
\end{pgfscope}%
\begin{pgfscope}%
\pgfsys@transformshift{14.729699in}{1.600000in}%
\pgfsys@useobject{currentmarker}{}%
\end{pgfscope}%
\begin{pgfscope}%
\pgfsys@transformshift{14.765321in}{1.600000in}%
\pgfsys@useobject{currentmarker}{}%
\end{pgfscope}%
\begin{pgfscope}%
\pgfsys@transformshift{14.800942in}{1.500000in}%
\pgfsys@useobject{currentmarker}{}%
\end{pgfscope}%
\begin{pgfscope}%
\pgfsys@transformshift{14.836564in}{1.400000in}%
\pgfsys@useobject{currentmarker}{}%
\end{pgfscope}%
\begin{pgfscope}%
\pgfsys@transformshift{14.872185in}{1.500000in}%
\pgfsys@useobject{currentmarker}{}%
\end{pgfscope}%
\begin{pgfscope}%
\pgfsys@transformshift{14.907807in}{1.600000in}%
\pgfsys@useobject{currentmarker}{}%
\end{pgfscope}%
\begin{pgfscope}%
\pgfsys@transformshift{14.943429in}{1.600000in}%
\pgfsys@useobject{currentmarker}{}%
\end{pgfscope}%
\begin{pgfscope}%
\pgfsys@transformshift{14.979050in}{1.500000in}%
\pgfsys@useobject{currentmarker}{}%
\end{pgfscope}%
\begin{pgfscope}%
\pgfsys@transformshift{15.014672in}{1.600000in}%
\pgfsys@useobject{currentmarker}{}%
\end{pgfscope}%
\begin{pgfscope}%
\pgfsys@transformshift{15.050294in}{1.600000in}%
\pgfsys@useobject{currentmarker}{}%
\end{pgfscope}%
\begin{pgfscope}%
\pgfsys@transformshift{15.085915in}{1.200000in}%
\pgfsys@useobject{currentmarker}{}%
\end{pgfscope}%
\begin{pgfscope}%
\pgfsys@transformshift{15.121537in}{1.500000in}%
\pgfsys@useobject{currentmarker}{}%
\end{pgfscope}%
\begin{pgfscope}%
\pgfsys@transformshift{15.157158in}{1.400000in}%
\pgfsys@useobject{currentmarker}{}%
\end{pgfscope}%
\begin{pgfscope}%
\pgfsys@transformshift{15.192780in}{1.500000in}%
\pgfsys@useobject{currentmarker}{}%
\end{pgfscope}%
\begin{pgfscope}%
\pgfsys@transformshift{15.228402in}{1.500000in}%
\pgfsys@useobject{currentmarker}{}%
\end{pgfscope}%
\begin{pgfscope}%
\pgfsys@transformshift{15.264023in}{1.600000in}%
\pgfsys@useobject{currentmarker}{}%
\end{pgfscope}%
\begin{pgfscope}%
\pgfsys@transformshift{15.299645in}{1.600000in}%
\pgfsys@useobject{currentmarker}{}%
\end{pgfscope}%
\begin{pgfscope}%
\pgfsys@transformshift{15.335267in}{1.500000in}%
\pgfsys@useobject{currentmarker}{}%
\end{pgfscope}%
\begin{pgfscope}%
\pgfsys@transformshift{15.370888in}{1.500000in}%
\pgfsys@useobject{currentmarker}{}%
\end{pgfscope}%
\begin{pgfscope}%
\pgfsys@transformshift{15.406510in}{1.500000in}%
\pgfsys@useobject{currentmarker}{}%
\end{pgfscope}%
\begin{pgfscope}%
\pgfsys@transformshift{15.442131in}{1.600000in}%
\pgfsys@useobject{currentmarker}{}%
\end{pgfscope}%
\begin{pgfscope}%
\pgfsys@transformshift{15.477753in}{1.600000in}%
\pgfsys@useobject{currentmarker}{}%
\end{pgfscope}%
\begin{pgfscope}%
\pgfsys@transformshift{15.513375in}{1.500000in}%
\pgfsys@useobject{currentmarker}{}%
\end{pgfscope}%
\begin{pgfscope}%
\pgfsys@transformshift{15.548996in}{1.500000in}%
\pgfsys@useobject{currentmarker}{}%
\end{pgfscope}%
\begin{pgfscope}%
\pgfsys@transformshift{15.584618in}{1.500000in}%
\pgfsys@useobject{currentmarker}{}%
\end{pgfscope}%
\begin{pgfscope}%
\pgfsys@transformshift{15.620240in}{1.500000in}%
\pgfsys@useobject{currentmarker}{}%
\end{pgfscope}%
\begin{pgfscope}%
\pgfsys@transformshift{15.655861in}{1.400000in}%
\pgfsys@useobject{currentmarker}{}%
\end{pgfscope}%
\begin{pgfscope}%
\pgfsys@transformshift{15.691483in}{1.500000in}%
\pgfsys@useobject{currentmarker}{}%
\end{pgfscope}%
\begin{pgfscope}%
\pgfsys@transformshift{15.727105in}{1.500000in}%
\pgfsys@useobject{currentmarker}{}%
\end{pgfscope}%
\begin{pgfscope}%
\pgfsys@transformshift{15.762726in}{1.500000in}%
\pgfsys@useobject{currentmarker}{}%
\end{pgfscope}%
\begin{pgfscope}%
\pgfsys@transformshift{15.798348in}{1.700000in}%
\pgfsys@useobject{currentmarker}{}%
\end{pgfscope}%
\begin{pgfscope}%
\pgfsys@transformshift{15.833969in}{1.500000in}%
\pgfsys@useobject{currentmarker}{}%
\end{pgfscope}%
\begin{pgfscope}%
\pgfsys@transformshift{15.869591in}{1.600000in}%
\pgfsys@useobject{currentmarker}{}%
\end{pgfscope}%
\begin{pgfscope}%
\pgfsys@transformshift{15.905213in}{1.400000in}%
\pgfsys@useobject{currentmarker}{}%
\end{pgfscope}%
\begin{pgfscope}%
\pgfsys@transformshift{15.940834in}{1.400000in}%
\pgfsys@useobject{currentmarker}{}%
\end{pgfscope}%
\begin{pgfscope}%
\pgfsys@transformshift{15.976456in}{1.200000in}%
\pgfsys@useobject{currentmarker}{}%
\end{pgfscope}%
\begin{pgfscope}%
\pgfsys@transformshift{16.012078in}{1.500000in}%
\pgfsys@useobject{currentmarker}{}%
\end{pgfscope}%
\begin{pgfscope}%
\pgfsys@transformshift{16.047699in}{1.400000in}%
\pgfsys@useobject{currentmarker}{}%
\end{pgfscope}%
\begin{pgfscope}%
\pgfsys@transformshift{16.083321in}{1.400000in}%
\pgfsys@useobject{currentmarker}{}%
\end{pgfscope}%
\begin{pgfscope}%
\pgfsys@transformshift{16.118942in}{1.400000in}%
\pgfsys@useobject{currentmarker}{}%
\end{pgfscope}%
\begin{pgfscope}%
\pgfsys@transformshift{16.154564in}{1.600000in}%
\pgfsys@useobject{currentmarker}{}%
\end{pgfscope}%
\begin{pgfscope}%
\pgfsys@transformshift{16.190186in}{1.500000in}%
\pgfsys@useobject{currentmarker}{}%
\end{pgfscope}%
\begin{pgfscope}%
\pgfsys@transformshift{16.225807in}{1.600000in}%
\pgfsys@useobject{currentmarker}{}%
\end{pgfscope}%
\begin{pgfscope}%
\pgfsys@transformshift{16.261429in}{1.300000in}%
\pgfsys@useobject{currentmarker}{}%
\end{pgfscope}%
\begin{pgfscope}%
\pgfsys@transformshift{16.297051in}{1.700000in}%
\pgfsys@useobject{currentmarker}{}%
\end{pgfscope}%
\begin{pgfscope}%
\pgfsys@transformshift{16.332672in}{1.300000in}%
\pgfsys@useobject{currentmarker}{}%
\end{pgfscope}%
\begin{pgfscope}%
\pgfsys@transformshift{16.368294in}{1.400000in}%
\pgfsys@useobject{currentmarker}{}%
\end{pgfscope}%
\begin{pgfscope}%
\pgfsys@transformshift{16.403915in}{1.500000in}%
\pgfsys@useobject{currentmarker}{}%
\end{pgfscope}%
\begin{pgfscope}%
\pgfsys@transformshift{16.439537in}{1.500000in}%
\pgfsys@useobject{currentmarker}{}%
\end{pgfscope}%
\begin{pgfscope}%
\pgfsys@transformshift{16.475159in}{1.500000in}%
\pgfsys@useobject{currentmarker}{}%
\end{pgfscope}%
\begin{pgfscope}%
\pgfsys@transformshift{16.510780in}{1.400000in}%
\pgfsys@useobject{currentmarker}{}%
\end{pgfscope}%
\begin{pgfscope}%
\pgfsys@transformshift{16.546402in}{1.500000in}%
\pgfsys@useobject{currentmarker}{}%
\end{pgfscope}%
\begin{pgfscope}%
\pgfsys@transformshift{16.582024in}{1.500000in}%
\pgfsys@useobject{currentmarker}{}%
\end{pgfscope}%
\begin{pgfscope}%
\pgfsys@transformshift{16.617645in}{1.500000in}%
\pgfsys@useobject{currentmarker}{}%
\end{pgfscope}%
\begin{pgfscope}%
\pgfsys@transformshift{16.653267in}{1.600000in}%
\pgfsys@useobject{currentmarker}{}%
\end{pgfscope}%
\begin{pgfscope}%
\pgfsys@transformshift{16.688888in}{1.400000in}%
\pgfsys@useobject{currentmarker}{}%
\end{pgfscope}%
\begin{pgfscope}%
\pgfsys@transformshift{16.724510in}{1.300000in}%
\pgfsys@useobject{currentmarker}{}%
\end{pgfscope}%
\begin{pgfscope}%
\pgfsys@transformshift{16.760132in}{1.600000in}%
\pgfsys@useobject{currentmarker}{}%
\end{pgfscope}%
\begin{pgfscope}%
\pgfsys@transformshift{16.795753in}{1.500000in}%
\pgfsys@useobject{currentmarker}{}%
\end{pgfscope}%
\begin{pgfscope}%
\pgfsys@transformshift{16.831375in}{1.600000in}%
\pgfsys@useobject{currentmarker}{}%
\end{pgfscope}%
\begin{pgfscope}%
\pgfsys@transformshift{16.866997in}{1.700000in}%
\pgfsys@useobject{currentmarker}{}%
\end{pgfscope}%
\begin{pgfscope}%
\pgfsys@transformshift{16.902618in}{1.400000in}%
\pgfsys@useobject{currentmarker}{}%
\end{pgfscope}%
\begin{pgfscope}%
\pgfsys@transformshift{16.938240in}{1.400000in}%
\pgfsys@useobject{currentmarker}{}%
\end{pgfscope}%
\begin{pgfscope}%
\pgfsys@transformshift{16.973861in}{1.500000in}%
\pgfsys@useobject{currentmarker}{}%
\end{pgfscope}%
\begin{pgfscope}%
\pgfsys@transformshift{17.009483in}{1.600000in}%
\pgfsys@useobject{currentmarker}{}%
\end{pgfscope}%
\begin{pgfscope}%
\pgfsys@transformshift{17.045105in}{1.600000in}%
\pgfsys@useobject{currentmarker}{}%
\end{pgfscope}%
\begin{pgfscope}%
\pgfsys@transformshift{17.080726in}{1.500000in}%
\pgfsys@useobject{currentmarker}{}%
\end{pgfscope}%
\begin{pgfscope}%
\pgfsys@transformshift{17.116348in}{1.400000in}%
\pgfsys@useobject{currentmarker}{}%
\end{pgfscope}%
\begin{pgfscope}%
\pgfsys@transformshift{17.151970in}{1.500000in}%
\pgfsys@useobject{currentmarker}{}%
\end{pgfscope}%
\begin{pgfscope}%
\pgfsys@transformshift{17.187591in}{1.700000in}%
\pgfsys@useobject{currentmarker}{}%
\end{pgfscope}%
\begin{pgfscope}%
\pgfsys@transformshift{17.223213in}{1.600000in}%
\pgfsys@useobject{currentmarker}{}%
\end{pgfscope}%
\begin{pgfscope}%
\pgfsys@transformshift{17.258835in}{1.400000in}%
\pgfsys@useobject{currentmarker}{}%
\end{pgfscope}%
\begin{pgfscope}%
\pgfsys@transformshift{17.294456in}{1.400000in}%
\pgfsys@useobject{currentmarker}{}%
\end{pgfscope}%
\begin{pgfscope}%
\pgfsys@transformshift{17.330078in}{1.700000in}%
\pgfsys@useobject{currentmarker}{}%
\end{pgfscope}%
\begin{pgfscope}%
\pgfsys@transformshift{17.365699in}{1.600000in}%
\pgfsys@useobject{currentmarker}{}%
\end{pgfscope}%
\begin{pgfscope}%
\pgfsys@transformshift{17.401321in}{1.600000in}%
\pgfsys@useobject{currentmarker}{}%
\end{pgfscope}%
\begin{pgfscope}%
\pgfsys@transformshift{17.436943in}{1.500000in}%
\pgfsys@useobject{currentmarker}{}%
\end{pgfscope}%
\begin{pgfscope}%
\pgfsys@transformshift{17.472564in}{1.500000in}%
\pgfsys@useobject{currentmarker}{}%
\end{pgfscope}%
\begin{pgfscope}%
\pgfsys@transformshift{17.508186in}{1.400000in}%
\pgfsys@useobject{currentmarker}{}%
\end{pgfscope}%
\begin{pgfscope}%
\pgfsys@transformshift{17.543808in}{1.400000in}%
\pgfsys@useobject{currentmarker}{}%
\end{pgfscope}%
\begin{pgfscope}%
\pgfsys@transformshift{17.579429in}{1.500000in}%
\pgfsys@useobject{currentmarker}{}%
\end{pgfscope}%
\begin{pgfscope}%
\pgfsys@transformshift{17.615051in}{1.500000in}%
\pgfsys@useobject{currentmarker}{}%
\end{pgfscope}%
\begin{pgfscope}%
\pgfsys@transformshift{17.650672in}{1.400000in}%
\pgfsys@useobject{currentmarker}{}%
\end{pgfscope}%
\begin{pgfscope}%
\pgfsys@transformshift{17.686294in}{1.500000in}%
\pgfsys@useobject{currentmarker}{}%
\end{pgfscope}%
\begin{pgfscope}%
\pgfsys@transformshift{17.721916in}{1.500000in}%
\pgfsys@useobject{currentmarker}{}%
\end{pgfscope}%
\begin{pgfscope}%
\pgfsys@transformshift{17.757537in}{1.700000in}%
\pgfsys@useobject{currentmarker}{}%
\end{pgfscope}%
\begin{pgfscope}%
\pgfsys@transformshift{17.793159in}{1.400000in}%
\pgfsys@useobject{currentmarker}{}%
\end{pgfscope}%
\begin{pgfscope}%
\pgfsys@transformshift{17.828781in}{1.600000in}%
\pgfsys@useobject{currentmarker}{}%
\end{pgfscope}%
\begin{pgfscope}%
\pgfsys@transformshift{17.864402in}{1.600000in}%
\pgfsys@useobject{currentmarker}{}%
\end{pgfscope}%
\begin{pgfscope}%
\pgfsys@transformshift{17.900024in}{1.700000in}%
\pgfsys@useobject{currentmarker}{}%
\end{pgfscope}%
\begin{pgfscope}%
\pgfsys@transformshift{17.935645in}{1.500000in}%
\pgfsys@useobject{currentmarker}{}%
\end{pgfscope}%
\begin{pgfscope}%
\pgfsys@transformshift{17.971267in}{1.300000in}%
\pgfsys@useobject{currentmarker}{}%
\end{pgfscope}%
\begin{pgfscope}%
\pgfsys@transformshift{18.006889in}{1.600000in}%
\pgfsys@useobject{currentmarker}{}%
\end{pgfscope}%
\begin{pgfscope}%
\pgfsys@transformshift{18.042510in}{1.400000in}%
\pgfsys@useobject{currentmarker}{}%
\end{pgfscope}%
\begin{pgfscope}%
\pgfsys@transformshift{18.078132in}{1.600000in}%
\pgfsys@useobject{currentmarker}{}%
\end{pgfscope}%
\begin{pgfscope}%
\pgfsys@transformshift{18.113754in}{1.400000in}%
\pgfsys@useobject{currentmarker}{}%
\end{pgfscope}%
\begin{pgfscope}%
\pgfsys@transformshift{18.149375in}{1.600000in}%
\pgfsys@useobject{currentmarker}{}%
\end{pgfscope}%
\begin{pgfscope}%
\pgfsys@transformshift{18.184997in}{1.500000in}%
\pgfsys@useobject{currentmarker}{}%
\end{pgfscope}%
\begin{pgfscope}%
\pgfsys@transformshift{18.220618in}{1.600000in}%
\pgfsys@useobject{currentmarker}{}%
\end{pgfscope}%
\begin{pgfscope}%
\pgfsys@transformshift{18.256240in}{1.400000in}%
\pgfsys@useobject{currentmarker}{}%
\end{pgfscope}%
\begin{pgfscope}%
\pgfsys@transformshift{18.291862in}{1.500000in}%
\pgfsys@useobject{currentmarker}{}%
\end{pgfscope}%
\begin{pgfscope}%
\pgfsys@transformshift{18.327483in}{1.400000in}%
\pgfsys@useobject{currentmarker}{}%
\end{pgfscope}%
\begin{pgfscope}%
\pgfsys@transformshift{18.363105in}{1.600000in}%
\pgfsys@useobject{currentmarker}{}%
\end{pgfscope}%
\begin{pgfscope}%
\pgfsys@transformshift{18.398727in}{1.400000in}%
\pgfsys@useobject{currentmarker}{}%
\end{pgfscope}%
\begin{pgfscope}%
\pgfsys@transformshift{18.434348in}{1.500000in}%
\pgfsys@useobject{currentmarker}{}%
\end{pgfscope}%
\begin{pgfscope}%
\pgfsys@transformshift{18.469970in}{1.500000in}%
\pgfsys@useobject{currentmarker}{}%
\end{pgfscope}%
\begin{pgfscope}%
\pgfsys@transformshift{18.505591in}{1.600000in}%
\pgfsys@useobject{currentmarker}{}%
\end{pgfscope}%
\begin{pgfscope}%
\pgfsys@transformshift{18.541213in}{1.600000in}%
\pgfsys@useobject{currentmarker}{}%
\end{pgfscope}%
\begin{pgfscope}%
\pgfsys@transformshift{18.576835in}{1.600000in}%
\pgfsys@useobject{currentmarker}{}%
\end{pgfscope}%
\begin{pgfscope}%
\pgfsys@transformshift{18.612456in}{1.500000in}%
\pgfsys@useobject{currentmarker}{}%
\end{pgfscope}%
\begin{pgfscope}%
\pgfsys@transformshift{18.648078in}{1.600000in}%
\pgfsys@useobject{currentmarker}{}%
\end{pgfscope}%
\begin{pgfscope}%
\pgfsys@transformshift{18.683700in}{1.600000in}%
\pgfsys@useobject{currentmarker}{}%
\end{pgfscope}%
\begin{pgfscope}%
\pgfsys@transformshift{18.719321in}{1.500000in}%
\pgfsys@useobject{currentmarker}{}%
\end{pgfscope}%
\begin{pgfscope}%
\pgfsys@transformshift{18.754943in}{1.600000in}%
\pgfsys@useobject{currentmarker}{}%
\end{pgfscope}%
\begin{pgfscope}%
\pgfsys@transformshift{18.790565in}{1.700000in}%
\pgfsys@useobject{currentmarker}{}%
\end{pgfscope}%
\begin{pgfscope}%
\pgfsys@transformshift{18.826186in}{1.700000in}%
\pgfsys@useobject{currentmarker}{}%
\end{pgfscope}%
\begin{pgfscope}%
\pgfsys@transformshift{18.861808in}{1.400000in}%
\pgfsys@useobject{currentmarker}{}%
\end{pgfscope}%
\begin{pgfscope}%
\pgfsys@transformshift{18.897429in}{1.400000in}%
\pgfsys@useobject{currentmarker}{}%
\end{pgfscope}%
\begin{pgfscope}%
\pgfsys@transformshift{18.933051in}{1.600000in}%
\pgfsys@useobject{currentmarker}{}%
\end{pgfscope}%
\begin{pgfscope}%
\pgfsys@transformshift{18.968673in}{1.400000in}%
\pgfsys@useobject{currentmarker}{}%
\end{pgfscope}%
\begin{pgfscope}%
\pgfsys@transformshift{19.004294in}{1.300000in}%
\pgfsys@useobject{currentmarker}{}%
\end{pgfscope}%
\begin{pgfscope}%
\pgfsys@transformshift{19.039916in}{1.500000in}%
\pgfsys@useobject{currentmarker}{}%
\end{pgfscope}%
\begin{pgfscope}%
\pgfsys@transformshift{19.075538in}{1.500000in}%
\pgfsys@useobject{currentmarker}{}%
\end{pgfscope}%
\begin{pgfscope}%
\pgfsys@transformshift{19.111159in}{1.600000in}%
\pgfsys@useobject{currentmarker}{}%
\end{pgfscope}%
\begin{pgfscope}%
\pgfsys@transformshift{19.146781in}{1.400000in}%
\pgfsys@useobject{currentmarker}{}%
\end{pgfscope}%
\begin{pgfscope}%
\pgfsys@transformshift{19.182402in}{1.600000in}%
\pgfsys@useobject{currentmarker}{}%
\end{pgfscope}%
\begin{pgfscope}%
\pgfsys@transformshift{19.218024in}{1.500000in}%
\pgfsys@useobject{currentmarker}{}%
\end{pgfscope}%
\begin{pgfscope}%
\pgfsys@transformshift{19.253646in}{1.700000in}%
\pgfsys@useobject{currentmarker}{}%
\end{pgfscope}%
\begin{pgfscope}%
\pgfsys@transformshift{19.289267in}{1.600000in}%
\pgfsys@useobject{currentmarker}{}%
\end{pgfscope}%
\begin{pgfscope}%
\pgfsys@transformshift{19.324889in}{1.700000in}%
\pgfsys@useobject{currentmarker}{}%
\end{pgfscope}%
\begin{pgfscope}%
\pgfsys@transformshift{19.360511in}{1.600000in}%
\pgfsys@useobject{currentmarker}{}%
\end{pgfscope}%
\begin{pgfscope}%
\pgfsys@transformshift{19.396132in}{1.400000in}%
\pgfsys@useobject{currentmarker}{}%
\end{pgfscope}%
\begin{pgfscope}%
\pgfsys@transformshift{19.431754in}{1.800000in}%
\pgfsys@useobject{currentmarker}{}%
\end{pgfscope}%
\begin{pgfscope}%
\pgfsys@transformshift{19.467375in}{1.500000in}%
\pgfsys@useobject{currentmarker}{}%
\end{pgfscope}%
\begin{pgfscope}%
\pgfsys@transformshift{19.502997in}{1.500000in}%
\pgfsys@useobject{currentmarker}{}%
\end{pgfscope}%
\begin{pgfscope}%
\pgfsys@transformshift{19.538619in}{1.400000in}%
\pgfsys@useobject{currentmarker}{}%
\end{pgfscope}%
\begin{pgfscope}%
\pgfsys@transformshift{19.574240in}{1.500000in}%
\pgfsys@useobject{currentmarker}{}%
\end{pgfscope}%
\begin{pgfscope}%
\pgfsys@transformshift{19.609862in}{1.700000in}%
\pgfsys@useobject{currentmarker}{}%
\end{pgfscope}%
\begin{pgfscope}%
\pgfsys@transformshift{19.645484in}{1.400000in}%
\pgfsys@useobject{currentmarker}{}%
\end{pgfscope}%
\begin{pgfscope}%
\pgfsys@transformshift{19.681105in}{1.500000in}%
\pgfsys@useobject{currentmarker}{}%
\end{pgfscope}%
\begin{pgfscope}%
\pgfsys@transformshift{19.716727in}{1.500000in}%
\pgfsys@useobject{currentmarker}{}%
\end{pgfscope}%
\begin{pgfscope}%
\pgfsys@transformshift{19.752348in}{1.500000in}%
\pgfsys@useobject{currentmarker}{}%
\end{pgfscope}%
\begin{pgfscope}%
\pgfsys@transformshift{19.787970in}{1.400000in}%
\pgfsys@useobject{currentmarker}{}%
\end{pgfscope}%
\begin{pgfscope}%
\pgfsys@transformshift{19.823592in}{1.400000in}%
\pgfsys@useobject{currentmarker}{}%
\end{pgfscope}%
\begin{pgfscope}%
\pgfsys@transformshift{19.859213in}{1.300000in}%
\pgfsys@useobject{currentmarker}{}%
\end{pgfscope}%
\begin{pgfscope}%
\pgfsys@transformshift{19.894835in}{1.600000in}%
\pgfsys@useobject{currentmarker}{}%
\end{pgfscope}%
\begin{pgfscope}%
\pgfsys@transformshift{19.930457in}{1.500000in}%
\pgfsys@useobject{currentmarker}{}%
\end{pgfscope}%
\begin{pgfscope}%
\pgfsys@transformshift{19.966078in}{1.600000in}%
\pgfsys@useobject{currentmarker}{}%
\end{pgfscope}%
\begin{pgfscope}%
\pgfsys@transformshift{20.001700in}{1.500000in}%
\pgfsys@useobject{currentmarker}{}%
\end{pgfscope}%
\begin{pgfscope}%
\pgfsys@transformshift{20.037321in}{1.500000in}%
\pgfsys@useobject{currentmarker}{}%
\end{pgfscope}%
\begin{pgfscope}%
\pgfsys@transformshift{20.072943in}{1.300000in}%
\pgfsys@useobject{currentmarker}{}%
\end{pgfscope}%
\begin{pgfscope}%
\pgfsys@transformshift{20.108565in}{1.300000in}%
\pgfsys@useobject{currentmarker}{}%
\end{pgfscope}%
\begin{pgfscope}%
\pgfsys@transformshift{20.144186in}{1.400000in}%
\pgfsys@useobject{currentmarker}{}%
\end{pgfscope}%
\begin{pgfscope}%
\pgfsys@transformshift{20.179808in}{1.600000in}%
\pgfsys@useobject{currentmarker}{}%
\end{pgfscope}%
\begin{pgfscope}%
\pgfsys@transformshift{20.215430in}{1.500000in}%
\pgfsys@useobject{currentmarker}{}%
\end{pgfscope}%
\begin{pgfscope}%
\pgfsys@transformshift{20.251051in}{1.700000in}%
\pgfsys@useobject{currentmarker}{}%
\end{pgfscope}%
\begin{pgfscope}%
\pgfsys@transformshift{20.286673in}{1.400000in}%
\pgfsys@useobject{currentmarker}{}%
\end{pgfscope}%
\begin{pgfscope}%
\pgfsys@transformshift{20.322295in}{1.500000in}%
\pgfsys@useobject{currentmarker}{}%
\end{pgfscope}%
\begin{pgfscope}%
\pgfsys@transformshift{20.357916in}{1.500000in}%
\pgfsys@useobject{currentmarker}{}%
\end{pgfscope}%
\begin{pgfscope}%
\pgfsys@transformshift{20.393538in}{1.500000in}%
\pgfsys@useobject{currentmarker}{}%
\end{pgfscope}%
\begin{pgfscope}%
\pgfsys@transformshift{20.429159in}{1.600000in}%
\pgfsys@useobject{currentmarker}{}%
\end{pgfscope}%
\begin{pgfscope}%
\pgfsys@transformshift{20.464781in}{1.200000in}%
\pgfsys@useobject{currentmarker}{}%
\end{pgfscope}%
\begin{pgfscope}%
\pgfsys@transformshift{20.500403in}{1.700000in}%
\pgfsys@useobject{currentmarker}{}%
\end{pgfscope}%
\begin{pgfscope}%
\pgfsys@transformshift{20.536024in}{1.600000in}%
\pgfsys@useobject{currentmarker}{}%
\end{pgfscope}%
\begin{pgfscope}%
\pgfsys@transformshift{20.571646in}{1.600000in}%
\pgfsys@useobject{currentmarker}{}%
\end{pgfscope}%
\begin{pgfscope}%
\pgfsys@transformshift{20.607268in}{1.600000in}%
\pgfsys@useobject{currentmarker}{}%
\end{pgfscope}%
\begin{pgfscope}%
\pgfsys@transformshift{20.642889in}{1.600000in}%
\pgfsys@useobject{currentmarker}{}%
\end{pgfscope}%
\begin{pgfscope}%
\pgfsys@transformshift{20.678511in}{1.500000in}%
\pgfsys@useobject{currentmarker}{}%
\end{pgfscope}%
\begin{pgfscope}%
\pgfsys@transformshift{20.714132in}{1.500000in}%
\pgfsys@useobject{currentmarker}{}%
\end{pgfscope}%
\begin{pgfscope}%
\pgfsys@transformshift{20.749754in}{1.400000in}%
\pgfsys@useobject{currentmarker}{}%
\end{pgfscope}%
\begin{pgfscope}%
\pgfsys@transformshift{20.785376in}{1.500000in}%
\pgfsys@useobject{currentmarker}{}%
\end{pgfscope}%
\begin{pgfscope}%
\pgfsys@transformshift{20.820997in}{1.500000in}%
\pgfsys@useobject{currentmarker}{}%
\end{pgfscope}%
\begin{pgfscope}%
\pgfsys@transformshift{20.856619in}{1.700000in}%
\pgfsys@useobject{currentmarker}{}%
\end{pgfscope}%
\begin{pgfscope}%
\pgfsys@transformshift{20.892241in}{1.500000in}%
\pgfsys@useobject{currentmarker}{}%
\end{pgfscope}%
\begin{pgfscope}%
\pgfsys@transformshift{20.927862in}{1.500000in}%
\pgfsys@useobject{currentmarker}{}%
\end{pgfscope}%
\begin{pgfscope}%
\pgfsys@transformshift{20.963484in}{1.700000in}%
\pgfsys@useobject{currentmarker}{}%
\end{pgfscope}%
\begin{pgfscope}%
\pgfsys@transformshift{20.999105in}{1.500000in}%
\pgfsys@useobject{currentmarker}{}%
\end{pgfscope}%
\begin{pgfscope}%
\pgfsys@transformshift{21.034727in}{1.500000in}%
\pgfsys@useobject{currentmarker}{}%
\end{pgfscope}%
\begin{pgfscope}%
\pgfsys@transformshift{21.070349in}{1.400000in}%
\pgfsys@useobject{currentmarker}{}%
\end{pgfscope}%
\begin{pgfscope}%
\pgfsys@transformshift{21.105970in}{1.500000in}%
\pgfsys@useobject{currentmarker}{}%
\end{pgfscope}%
\begin{pgfscope}%
\pgfsys@transformshift{21.141592in}{1.600000in}%
\pgfsys@useobject{currentmarker}{}%
\end{pgfscope}%
\begin{pgfscope}%
\pgfsys@transformshift{21.177214in}{1.400000in}%
\pgfsys@useobject{currentmarker}{}%
\end{pgfscope}%
\begin{pgfscope}%
\pgfsys@transformshift{21.212835in}{1.400000in}%
\pgfsys@useobject{currentmarker}{}%
\end{pgfscope}%
\begin{pgfscope}%
\pgfsys@transformshift{21.248457in}{1.500000in}%
\pgfsys@useobject{currentmarker}{}%
\end{pgfscope}%
\begin{pgfscope}%
\pgfsys@transformshift{21.284078in}{1.600000in}%
\pgfsys@useobject{currentmarker}{}%
\end{pgfscope}%
\begin{pgfscope}%
\pgfsys@transformshift{21.319700in}{1.500000in}%
\pgfsys@useobject{currentmarker}{}%
\end{pgfscope}%
\begin{pgfscope}%
\pgfsys@transformshift{21.355322in}{1.400000in}%
\pgfsys@useobject{currentmarker}{}%
\end{pgfscope}%
\begin{pgfscope}%
\pgfsys@transformshift{21.390943in}{1.400000in}%
\pgfsys@useobject{currentmarker}{}%
\end{pgfscope}%
\begin{pgfscope}%
\pgfsys@transformshift{21.426565in}{1.600000in}%
\pgfsys@useobject{currentmarker}{}%
\end{pgfscope}%
\begin{pgfscope}%
\pgfsys@transformshift{21.462187in}{1.700000in}%
\pgfsys@useobject{currentmarker}{}%
\end{pgfscope}%
\begin{pgfscope}%
\pgfsys@transformshift{21.497808in}{1.600000in}%
\pgfsys@useobject{currentmarker}{}%
\end{pgfscope}%
\begin{pgfscope}%
\pgfsys@transformshift{21.533430in}{1.500000in}%
\pgfsys@useobject{currentmarker}{}%
\end{pgfscope}%
\end{pgfscope}%
\begin{pgfscope}%
\pgfpathrectangle{\pgfqpoint{1.000000in}{1.000000in}}{\pgfqpoint{8.500000in}{1.000000in}}%
\pgfusepath{clip}%
\pgfsetrectcap%
\pgfsetroundjoin%
\pgfsetlinewidth{0.803000pt}%
\definecolor{currentstroke}{rgb}{0.690196,0.690196,0.690196}%
\pgfsetstrokecolor{currentstroke}%
\pgfsetdash{}{0pt}%
\pgfpathmoveto{\pgfqpoint{2.725210in}{1.000000in}}%
\pgfpathlineto{\pgfqpoint{2.725210in}{2.000000in}}%
\pgfusepath{stroke}%
\end{pgfscope}%
\begin{pgfscope}%
\pgfsetbuttcap%
\pgfsetroundjoin%
\definecolor{currentfill}{rgb}{0.000000,0.000000,0.000000}%
\pgfsetfillcolor{currentfill}%
\pgfsetlinewidth{0.803000pt}%
\definecolor{currentstroke}{rgb}{0.000000,0.000000,0.000000}%
\pgfsetstrokecolor{currentstroke}%
\pgfsetdash{}{0pt}%
\pgfsys@defobject{currentmarker}{\pgfqpoint{0.000000in}{-0.048611in}}{\pgfqpoint{0.000000in}{0.000000in}}{%
\pgfpathmoveto{\pgfqpoint{0.000000in}{0.000000in}}%
\pgfpathlineto{\pgfqpoint{0.000000in}{-0.048611in}}%
\pgfusepath{stroke,fill}%
}%
\begin{pgfscope}%
\pgfsys@transformshift{2.725210in}{1.000000in}%
\pgfsys@useobject{currentmarker}{}%
\end{pgfscope}%
\end{pgfscope}%
\begin{pgfscope}%
\definecolor{textcolor}{rgb}{0.000000,0.000000,0.000000}%
\pgfsetstrokecolor{textcolor}%
\pgfsetfillcolor{textcolor}%
\pgftext[x=2.725210in,y=0.902778in,,top]{\color{textcolor}\sffamily\fontsize{20.000000}{24.000000}\selectfont \(\displaystyle {500}\)}%
\end{pgfscope}%
\begin{pgfscope}%
\pgfpathrectangle{\pgfqpoint{1.000000in}{1.000000in}}{\pgfqpoint{8.500000in}{1.000000in}}%
\pgfusepath{clip}%
\pgfsetrectcap%
\pgfsetroundjoin%
\pgfsetlinewidth{0.803000pt}%
\definecolor{currentstroke}{rgb}{0.690196,0.690196,0.690196}%
\pgfsetstrokecolor{currentstroke}%
\pgfsetdash{}{0pt}%
\pgfpathmoveto{\pgfqpoint{4.506292in}{1.000000in}}%
\pgfpathlineto{\pgfqpoint{4.506292in}{2.000000in}}%
\pgfusepath{stroke}%
\end{pgfscope}%
\begin{pgfscope}%
\pgfsetbuttcap%
\pgfsetroundjoin%
\definecolor{currentfill}{rgb}{0.000000,0.000000,0.000000}%
\pgfsetfillcolor{currentfill}%
\pgfsetlinewidth{0.803000pt}%
\definecolor{currentstroke}{rgb}{0.000000,0.000000,0.000000}%
\pgfsetstrokecolor{currentstroke}%
\pgfsetdash{}{0pt}%
\pgfsys@defobject{currentmarker}{\pgfqpoint{0.000000in}{-0.048611in}}{\pgfqpoint{0.000000in}{0.000000in}}{%
\pgfpathmoveto{\pgfqpoint{0.000000in}{0.000000in}}%
\pgfpathlineto{\pgfqpoint{0.000000in}{-0.048611in}}%
\pgfusepath{stroke,fill}%
}%
\begin{pgfscope}%
\pgfsys@transformshift{4.506292in}{1.000000in}%
\pgfsys@useobject{currentmarker}{}%
\end{pgfscope}%
\end{pgfscope}%
\begin{pgfscope}%
\definecolor{textcolor}{rgb}{0.000000,0.000000,0.000000}%
\pgfsetstrokecolor{textcolor}%
\pgfsetfillcolor{textcolor}%
\pgftext[x=4.506292in,y=0.902778in,,top]{\color{textcolor}\sffamily\fontsize{20.000000}{24.000000}\selectfont \(\displaystyle {550}\)}%
\end{pgfscope}%
\begin{pgfscope}%
\pgfpathrectangle{\pgfqpoint{1.000000in}{1.000000in}}{\pgfqpoint{8.500000in}{1.000000in}}%
\pgfusepath{clip}%
\pgfsetrectcap%
\pgfsetroundjoin%
\pgfsetlinewidth{0.803000pt}%
\definecolor{currentstroke}{rgb}{0.690196,0.690196,0.690196}%
\pgfsetstrokecolor{currentstroke}%
\pgfsetdash{}{0pt}%
\pgfpathmoveto{\pgfqpoint{6.287373in}{1.000000in}}%
\pgfpathlineto{\pgfqpoint{6.287373in}{2.000000in}}%
\pgfusepath{stroke}%
\end{pgfscope}%
\begin{pgfscope}%
\pgfsetbuttcap%
\pgfsetroundjoin%
\definecolor{currentfill}{rgb}{0.000000,0.000000,0.000000}%
\pgfsetfillcolor{currentfill}%
\pgfsetlinewidth{0.803000pt}%
\definecolor{currentstroke}{rgb}{0.000000,0.000000,0.000000}%
\pgfsetstrokecolor{currentstroke}%
\pgfsetdash{}{0pt}%
\pgfsys@defobject{currentmarker}{\pgfqpoint{0.000000in}{-0.048611in}}{\pgfqpoint{0.000000in}{0.000000in}}{%
\pgfpathmoveto{\pgfqpoint{0.000000in}{0.000000in}}%
\pgfpathlineto{\pgfqpoint{0.000000in}{-0.048611in}}%
\pgfusepath{stroke,fill}%
}%
\begin{pgfscope}%
\pgfsys@transformshift{6.287373in}{1.000000in}%
\pgfsys@useobject{currentmarker}{}%
\end{pgfscope}%
\end{pgfscope}%
\begin{pgfscope}%
\definecolor{textcolor}{rgb}{0.000000,0.000000,0.000000}%
\pgfsetstrokecolor{textcolor}%
\pgfsetfillcolor{textcolor}%
\pgftext[x=6.287373in,y=0.902778in,,top]{\color{textcolor}\sffamily\fontsize{20.000000}{24.000000}\selectfont \(\displaystyle {600}\)}%
\end{pgfscope}%
\begin{pgfscope}%
\pgfpathrectangle{\pgfqpoint{1.000000in}{1.000000in}}{\pgfqpoint{8.500000in}{1.000000in}}%
\pgfusepath{clip}%
\pgfsetrectcap%
\pgfsetroundjoin%
\pgfsetlinewidth{0.803000pt}%
\definecolor{currentstroke}{rgb}{0.690196,0.690196,0.690196}%
\pgfsetstrokecolor{currentstroke}%
\pgfsetdash{}{0pt}%
\pgfpathmoveto{\pgfqpoint{8.068455in}{1.000000in}}%
\pgfpathlineto{\pgfqpoint{8.068455in}{2.000000in}}%
\pgfusepath{stroke}%
\end{pgfscope}%
\begin{pgfscope}%
\pgfsetbuttcap%
\pgfsetroundjoin%
\definecolor{currentfill}{rgb}{0.000000,0.000000,0.000000}%
\pgfsetfillcolor{currentfill}%
\pgfsetlinewidth{0.803000pt}%
\definecolor{currentstroke}{rgb}{0.000000,0.000000,0.000000}%
\pgfsetstrokecolor{currentstroke}%
\pgfsetdash{}{0pt}%
\pgfsys@defobject{currentmarker}{\pgfqpoint{0.000000in}{-0.048611in}}{\pgfqpoint{0.000000in}{0.000000in}}{%
\pgfpathmoveto{\pgfqpoint{0.000000in}{0.000000in}}%
\pgfpathlineto{\pgfqpoint{0.000000in}{-0.048611in}}%
\pgfusepath{stroke,fill}%
}%
\begin{pgfscope}%
\pgfsys@transformshift{8.068455in}{1.000000in}%
\pgfsys@useobject{currentmarker}{}%
\end{pgfscope}%
\end{pgfscope}%
\begin{pgfscope}%
\definecolor{textcolor}{rgb}{0.000000,0.000000,0.000000}%
\pgfsetstrokecolor{textcolor}%
\pgfsetfillcolor{textcolor}%
\pgftext[x=8.068455in,y=0.902778in,,top]{\color{textcolor}\sffamily\fontsize{20.000000}{24.000000}\selectfont \(\displaystyle {650}\)}%
\end{pgfscope}%
\begin{pgfscope}%
\definecolor{textcolor}{rgb}{0.000000,0.000000,0.000000}%
\pgfsetstrokecolor{textcolor}%
\pgfsetfillcolor{textcolor}%
\pgftext[x=5.250000in,y=0.591155in,,top]{\color{textcolor}\sffamily\fontsize{20.000000}{24.000000}\selectfont \(\displaystyle \mathrm{t}/\si{ns}\)}%
\end{pgfscope}%
\begin{pgfscope}%
\pgfpathrectangle{\pgfqpoint{1.000000in}{1.000000in}}{\pgfqpoint{8.500000in}{1.000000in}}%
\pgfusepath{clip}%
\pgfsetrectcap%
\pgfsetroundjoin%
\pgfsetlinewidth{0.803000pt}%
\definecolor{currentstroke}{rgb}{0.690196,0.690196,0.690196}%
\pgfsetstrokecolor{currentstroke}%
\pgfsetdash{}{0pt}%
\pgfpathmoveto{\pgfqpoint{1.000000in}{1.000000in}}%
\pgfpathlineto{\pgfqpoint{9.500000in}{1.000000in}}%
\pgfusepath{stroke}%
\end{pgfscope}%
\begin{pgfscope}%
\pgfsetbuttcap%
\pgfsetroundjoin%
\definecolor{currentfill}{rgb}{0.000000,0.000000,0.000000}%
\pgfsetfillcolor{currentfill}%
\pgfsetlinewidth{0.803000pt}%
\definecolor{currentstroke}{rgb}{0.000000,0.000000,0.000000}%
\pgfsetstrokecolor{currentstroke}%
\pgfsetdash{}{0pt}%
\pgfsys@defobject{currentmarker}{\pgfqpoint{-0.048611in}{0.000000in}}{\pgfqpoint{-0.000000in}{0.000000in}}{%
\pgfpathmoveto{\pgfqpoint{-0.000000in}{0.000000in}}%
\pgfpathlineto{\pgfqpoint{-0.048611in}{0.000000in}}%
\pgfusepath{stroke,fill}%
}%
\begin{pgfscope}%
\pgfsys@transformshift{1.000000in}{1.000000in}%
\pgfsys@useobject{currentmarker}{}%
\end{pgfscope}%
\end{pgfscope}%
\begin{pgfscope}%
\definecolor{textcolor}{rgb}{0.000000,0.000000,0.000000}%
\pgfsetstrokecolor{textcolor}%
\pgfsetfillcolor{textcolor}%
\pgftext[x=0.546626in, y=0.899981in, left, base]{\color{textcolor}\sffamily\fontsize{20.000000}{24.000000}\selectfont \(\displaystyle {-5}\)}%
\end{pgfscope}%
\begin{pgfscope}%
\pgfpathrectangle{\pgfqpoint{1.000000in}{1.000000in}}{\pgfqpoint{8.500000in}{1.000000in}}%
\pgfusepath{clip}%
\pgfsetrectcap%
\pgfsetroundjoin%
\pgfsetlinewidth{0.803000pt}%
\definecolor{currentstroke}{rgb}{0.690196,0.690196,0.690196}%
\pgfsetstrokecolor{currentstroke}%
\pgfsetdash{}{0pt}%
\pgfpathmoveto{\pgfqpoint{1.000000in}{1.500000in}}%
\pgfpathlineto{\pgfqpoint{9.500000in}{1.500000in}}%
\pgfusepath{stroke}%
\end{pgfscope}%
\begin{pgfscope}%
\pgfsetbuttcap%
\pgfsetroundjoin%
\definecolor{currentfill}{rgb}{0.000000,0.000000,0.000000}%
\pgfsetfillcolor{currentfill}%
\pgfsetlinewidth{0.803000pt}%
\definecolor{currentstroke}{rgb}{0.000000,0.000000,0.000000}%
\pgfsetstrokecolor{currentstroke}%
\pgfsetdash{}{0pt}%
\pgfsys@defobject{currentmarker}{\pgfqpoint{-0.048611in}{0.000000in}}{\pgfqpoint{-0.000000in}{0.000000in}}{%
\pgfpathmoveto{\pgfqpoint{-0.000000in}{0.000000in}}%
\pgfpathlineto{\pgfqpoint{-0.048611in}{0.000000in}}%
\pgfusepath{stroke,fill}%
}%
\begin{pgfscope}%
\pgfsys@transformshift{1.000000in}{1.500000in}%
\pgfsys@useobject{currentmarker}{}%
\end{pgfscope}%
\end{pgfscope}%
\begin{pgfscope}%
\definecolor{textcolor}{rgb}{0.000000,0.000000,0.000000}%
\pgfsetstrokecolor{textcolor}%
\pgfsetfillcolor{textcolor}%
\pgftext[x=0.770670in, y=1.399981in, left, base]{\color{textcolor}\sffamily\fontsize{20.000000}{24.000000}\selectfont \(\displaystyle {0}\)}%
\end{pgfscope}%
\begin{pgfscope}%
\pgfpathrectangle{\pgfqpoint{1.000000in}{1.000000in}}{\pgfqpoint{8.500000in}{1.000000in}}%
\pgfusepath{clip}%
\pgfsetrectcap%
\pgfsetroundjoin%
\pgfsetlinewidth{0.803000pt}%
\definecolor{currentstroke}{rgb}{0.690196,0.690196,0.690196}%
\pgfsetstrokecolor{currentstroke}%
\pgfsetdash{}{0pt}%
\pgfpathmoveto{\pgfqpoint{1.000000in}{2.000000in}}%
\pgfpathlineto{\pgfqpoint{9.500000in}{2.000000in}}%
\pgfusepath{stroke}%
\end{pgfscope}%
\begin{pgfscope}%
\pgfsetbuttcap%
\pgfsetroundjoin%
\definecolor{currentfill}{rgb}{0.000000,0.000000,0.000000}%
\pgfsetfillcolor{currentfill}%
\pgfsetlinewidth{0.803000pt}%
\definecolor{currentstroke}{rgb}{0.000000,0.000000,0.000000}%
\pgfsetstrokecolor{currentstroke}%
\pgfsetdash{}{0pt}%
\pgfsys@defobject{currentmarker}{\pgfqpoint{-0.048611in}{0.000000in}}{\pgfqpoint{-0.000000in}{0.000000in}}{%
\pgfpathmoveto{\pgfqpoint{-0.000000in}{0.000000in}}%
\pgfpathlineto{\pgfqpoint{-0.048611in}{0.000000in}}%
\pgfusepath{stroke,fill}%
}%
\begin{pgfscope}%
\pgfsys@transformshift{1.000000in}{2.000000in}%
\pgfsys@useobject{currentmarker}{}%
\end{pgfscope}%
\end{pgfscope}%
\begin{pgfscope}%
\definecolor{textcolor}{rgb}{0.000000,0.000000,0.000000}%
\pgfsetstrokecolor{textcolor}%
\pgfsetfillcolor{textcolor}%
\pgftext[x=0.770670in, y=1.899981in, left, base]{\color{textcolor}\sffamily\fontsize{20.000000}{24.000000}\selectfont \(\displaystyle {5}\)}%
\end{pgfscope}%
\begin{pgfscope}%
\definecolor{textcolor}{rgb}{0.000000,0.000000,0.000000}%
\pgfsetstrokecolor{textcolor}%
\pgfsetfillcolor{textcolor}%
\pgftext[x=0.491071in,y=1.500000in,,bottom,rotate=90.000000]{\color{textcolor}\sffamily\fontsize{20.000000}{24.000000}\selectfont \(\displaystyle \mathrm{Voltage}/\si{mV}\)}%
\end{pgfscope}%
\begin{pgfscope}%
\pgfsetrectcap%
\pgfsetmiterjoin%
\pgfsetlinewidth{0.803000pt}%
\definecolor{currentstroke}{rgb}{0.000000,0.000000,0.000000}%
\pgfsetstrokecolor{currentstroke}%
\pgfsetdash{}{0pt}%
\pgfpathmoveto{\pgfqpoint{1.000000in}{1.000000in}}%
\pgfpathlineto{\pgfqpoint{1.000000in}{2.000000in}}%
\pgfusepath{stroke}%
\end{pgfscope}%
\begin{pgfscope}%
\pgfsetrectcap%
\pgfsetmiterjoin%
\pgfsetlinewidth{0.803000pt}%
\definecolor{currentstroke}{rgb}{0.000000,0.000000,0.000000}%
\pgfsetstrokecolor{currentstroke}%
\pgfsetdash{}{0pt}%
\pgfpathmoveto{\pgfqpoint{9.500000in}{1.000000in}}%
\pgfpathlineto{\pgfqpoint{9.500000in}{2.000000in}}%
\pgfusepath{stroke}%
\end{pgfscope}%
\begin{pgfscope}%
\pgfsetrectcap%
\pgfsetmiterjoin%
\pgfsetlinewidth{0.803000pt}%
\definecolor{currentstroke}{rgb}{0.000000,0.000000,0.000000}%
\pgfsetstrokecolor{currentstroke}%
\pgfsetdash{}{0pt}%
\pgfpathmoveto{\pgfqpoint{1.000000in}{1.000000in}}%
\pgfpathlineto{\pgfqpoint{9.500000in}{1.000000in}}%
\pgfusepath{stroke}%
\end{pgfscope}%
\begin{pgfscope}%
\pgfsetrectcap%
\pgfsetmiterjoin%
\pgfsetlinewidth{0.803000pt}%
\definecolor{currentstroke}{rgb}{0.000000,0.000000,0.000000}%
\pgfsetstrokecolor{currentstroke}%
\pgfsetdash{}{0pt}%
\pgfpathmoveto{\pgfqpoint{1.000000in}{2.000000in}}%
\pgfpathlineto{\pgfqpoint{9.500000in}{2.000000in}}%
\pgfusepath{stroke}%
\end{pgfscope}%
\begin{pgfscope}%
\pgfsetbuttcap%
\pgfsetmiterjoin%
\definecolor{currentfill}{rgb}{1.000000,1.000000,1.000000}%
\pgfsetfillcolor{currentfill}%
\pgfsetfillopacity{0.800000}%
\pgfsetlinewidth{1.003750pt}%
\definecolor{currentstroke}{rgb}{0.800000,0.800000,0.800000}%
\pgfsetstrokecolor{currentstroke}%
\pgfsetstrokeopacity{0.800000}%
\pgfsetdash{}{0pt}%
\pgfpathmoveto{\pgfqpoint{6.921366in}{1.382821in}}%
\pgfpathlineto{\pgfqpoint{9.305556in}{1.382821in}}%
\pgfpathquadraticcurveto{\pgfqpoint{9.361111in}{1.382821in}}{\pgfqpoint{9.361111in}{1.438377in}}%
\pgfpathlineto{\pgfqpoint{9.361111in}{1.805556in}}%
\pgfpathquadraticcurveto{\pgfqpoint{9.361111in}{1.861111in}}{\pgfqpoint{9.305556in}{1.861111in}}%
\pgfpathlineto{\pgfqpoint{6.921366in}{1.861111in}}%
\pgfpathquadraticcurveto{\pgfqpoint{6.865810in}{1.861111in}}{\pgfqpoint{6.865810in}{1.805556in}}%
\pgfpathlineto{\pgfqpoint{6.865810in}{1.438377in}}%
\pgfpathquadraticcurveto{\pgfqpoint{6.865810in}{1.382821in}}{\pgfqpoint{6.921366in}{1.382821in}}%
\pgfpathclose%
\pgfusepath{stroke,fill}%
\end{pgfscope}%
\begin{pgfscope}%
\pgfsetbuttcap%
\pgfsetroundjoin%
\definecolor{currentfill}{rgb}{0.000000,0.000000,0.000000}%
\pgfsetfillcolor{currentfill}%
\pgfsetlinewidth{1.003750pt}%
\definecolor{currentstroke}{rgb}{0.000000,0.000000,0.000000}%
\pgfsetstrokecolor{currentstroke}%
\pgfsetdash{}{0pt}%
\pgfsys@defobject{currentmarker}{\pgfqpoint{-0.013889in}{-0.013889in}}{\pgfqpoint{0.013889in}{0.013889in}}{%
\pgfpathmoveto{\pgfqpoint{0.000000in}{-0.013889in}}%
\pgfpathcurveto{\pgfqpoint{0.003683in}{-0.013889in}}{\pgfqpoint{0.007216in}{-0.012425in}}{\pgfqpoint{0.009821in}{-0.009821in}}%
\pgfpathcurveto{\pgfqpoint{0.012425in}{-0.007216in}}{\pgfqpoint{0.013889in}{-0.003683in}}{\pgfqpoint{0.013889in}{0.000000in}}%
\pgfpathcurveto{\pgfqpoint{0.013889in}{0.003683in}}{\pgfqpoint{0.012425in}{0.007216in}}{\pgfqpoint{0.009821in}{0.009821in}}%
\pgfpathcurveto{\pgfqpoint{0.007216in}{0.012425in}}{\pgfqpoint{0.003683in}{0.013889in}}{\pgfqpoint{0.000000in}{0.013889in}}%
\pgfpathcurveto{\pgfqpoint{-0.003683in}{0.013889in}}{\pgfqpoint{-0.007216in}{0.012425in}}{\pgfqpoint{-0.009821in}{0.009821in}}%
\pgfpathcurveto{\pgfqpoint{-0.012425in}{0.007216in}}{\pgfqpoint{-0.013889in}{0.003683in}}{\pgfqpoint{-0.013889in}{0.000000in}}%
\pgfpathcurveto{\pgfqpoint{-0.013889in}{-0.003683in}}{\pgfqpoint{-0.012425in}{-0.007216in}}{\pgfqpoint{-0.009821in}{-0.009821in}}%
\pgfpathcurveto{\pgfqpoint{-0.007216in}{-0.012425in}}{\pgfqpoint{-0.003683in}{-0.013889in}}{\pgfqpoint{0.000000in}{-0.013889in}}%
\pgfpathclose%
\pgfusepath{stroke,fill}%
}%
\begin{pgfscope}%
\pgfsys@transformshift{7.254699in}{1.622878in}%
\pgfsys@useobject{currentmarker}{}%
\end{pgfscope}%
\end{pgfscope}%
\begin{pgfscope}%
\definecolor{textcolor}{rgb}{0.000000,0.000000,0.000000}%
\pgfsetstrokecolor{textcolor}%
\pgfsetfillcolor{textcolor}%
\pgftext[x=7.754699in,y=1.549962in,left,base]{\color{textcolor}\sffamily\fontsize{20.000000}{24.000000}\selectfont residual wave}%
\end{pgfscope}%
\end{pgfpicture}%
\makeatother%
\endgroup%
}
    \caption{$\mathrm{RSS}=\SI{15.8}{mV^2},D_w=\SI{0.59}{ns},\Delta t_0=\SI{-3.51}{ns}$}
\end{figure}
\end{frame}

\begin{frame}
\frametitle{Kullback-Leibler divergence}
\begin{align*}
  \hat{t}_\mathrm{KL} &= \arg\underset{t_0}{\max} \prod_{i=1}^{\hat{N}} \left[\phi(\hat{t}_i-t_0)\right]^{\hat{q}_i} ,\ \Delta t = \hat{t}_\mathrm{KL} - t_0
\end{align*}
\begin{figure}
    \centering
    \resizebox{0.6\textwidth}{!}{%% Creator: Matplotlib, PGF backend
%%
%% To include the figure in your LaTeX document, write
%%   \input{<filename>.pgf}
%%
%% Make sure the required packages are loaded in your preamble
%%   \usepackage{pgf}
%%
%% and, on pdftex
%%   \usepackage[utf8]{inputenc}\DeclareUnicodeCharacter{2212}{-}
%%
%% or, on luatex and xetex
%%   \usepackage{unicode-math}
%%
%% Figures using additional raster images can only be included by \input if
%% they are in the same directory as the main LaTeX file. For loading figures
%% from other directories you can use the `import` package
%%   \usepackage{import}
%%
%% and then include the figures with
%%   \import{<path to file>}{<filename>.pgf}
%%
%% Matplotlib used the following preamble
%%   \usepackage[detect-all,locale=DE]{siunitx}
%%
\begingroup%
\makeatletter%
\begin{pgfpicture}%
\pgfpathrectangle{\pgfpointorigin}{\pgfqpoint{8.000000in}{6.000000in}}%
\pgfusepath{use as bounding box, clip}%
\begin{pgfscope}%
\pgfsetbuttcap%
\pgfsetmiterjoin%
\definecolor{currentfill}{rgb}{1.000000,1.000000,1.000000}%
\pgfsetfillcolor{currentfill}%
\pgfsetlinewidth{0.000000pt}%
\definecolor{currentstroke}{rgb}{1.000000,1.000000,1.000000}%
\pgfsetstrokecolor{currentstroke}%
\pgfsetdash{}{0pt}%
\pgfpathmoveto{\pgfqpoint{0.000000in}{0.000000in}}%
\pgfpathlineto{\pgfqpoint{8.000000in}{0.000000in}}%
\pgfpathlineto{\pgfqpoint{8.000000in}{6.000000in}}%
\pgfpathlineto{\pgfqpoint{0.000000in}{6.000000in}}%
\pgfpathclose%
\pgfusepath{fill}%
\end{pgfscope}%
\begin{pgfscope}%
\pgfsetbuttcap%
\pgfsetmiterjoin%
\definecolor{currentfill}{rgb}{1.000000,1.000000,1.000000}%
\pgfsetfillcolor{currentfill}%
\pgfsetlinewidth{0.000000pt}%
\definecolor{currentstroke}{rgb}{0.000000,0.000000,0.000000}%
\pgfsetstrokecolor{currentstroke}%
\pgfsetstrokeopacity{0.000000}%
\pgfsetdash{}{0pt}%
\pgfpathmoveto{\pgfqpoint{1.200000in}{0.900000in}}%
\pgfpathlineto{\pgfqpoint{7.600000in}{0.900000in}}%
\pgfpathlineto{\pgfqpoint{7.600000in}{5.700000in}}%
\pgfpathlineto{\pgfqpoint{1.200000in}{5.700000in}}%
\pgfpathclose%
\pgfusepath{fill}%
\end{pgfscope}%
\begin{pgfscope}%
\pgfpathrectangle{\pgfqpoint{1.200000in}{0.900000in}}{\pgfqpoint{6.400000in}{4.800000in}}%
\pgfusepath{clip}%
\pgfsetrectcap%
\pgfsetroundjoin%
\pgfsetlinewidth{0.803000pt}%
\definecolor{currentstroke}{rgb}{0.690196,0.690196,0.690196}%
\pgfsetstrokecolor{currentstroke}%
\pgfsetdash{}{0pt}%
\pgfpathmoveto{\pgfqpoint{1.200000in}{0.900000in}}%
\pgfpathlineto{\pgfqpoint{1.200000in}{5.700000in}}%
\pgfusepath{stroke}%
\end{pgfscope}%
\begin{pgfscope}%
\pgfsetbuttcap%
\pgfsetroundjoin%
\definecolor{currentfill}{rgb}{0.000000,0.000000,0.000000}%
\pgfsetfillcolor{currentfill}%
\pgfsetlinewidth{0.803000pt}%
\definecolor{currentstroke}{rgb}{0.000000,0.000000,0.000000}%
\pgfsetstrokecolor{currentstroke}%
\pgfsetdash{}{0pt}%
\pgfsys@defobject{currentmarker}{\pgfqpoint{0.000000in}{-0.048611in}}{\pgfqpoint{0.000000in}{0.000000in}}{%
\pgfpathmoveto{\pgfqpoint{0.000000in}{0.000000in}}%
\pgfpathlineto{\pgfqpoint{0.000000in}{-0.048611in}}%
\pgfusepath{stroke,fill}%
}%
\begin{pgfscope}%
\pgfsys@transformshift{1.200000in}{0.900000in}%
\pgfsys@useobject{currentmarker}{}%
\end{pgfscope}%
\end{pgfscope}%
\begin{pgfscope}%
\definecolor{textcolor}{rgb}{0.000000,0.000000,0.000000}%
\pgfsetstrokecolor{textcolor}%
\pgfsetfillcolor{textcolor}%
\pgftext[x=1.200000in,y=0.802778in,,top]{\color{textcolor}\sffamily\fontsize{20.000000}{24.000000}\selectfont \(\displaystyle {-20}\)}%
\end{pgfscope}%
\begin{pgfscope}%
\pgfpathrectangle{\pgfqpoint{1.200000in}{0.900000in}}{\pgfqpoint{6.400000in}{4.800000in}}%
\pgfusepath{clip}%
\pgfsetrectcap%
\pgfsetroundjoin%
\pgfsetlinewidth{0.803000pt}%
\definecolor{currentstroke}{rgb}{0.690196,0.690196,0.690196}%
\pgfsetstrokecolor{currentstroke}%
\pgfsetdash{}{0pt}%
\pgfpathmoveto{\pgfqpoint{2.266667in}{0.900000in}}%
\pgfpathlineto{\pgfqpoint{2.266667in}{5.700000in}}%
\pgfusepath{stroke}%
\end{pgfscope}%
\begin{pgfscope}%
\pgfsetbuttcap%
\pgfsetroundjoin%
\definecolor{currentfill}{rgb}{0.000000,0.000000,0.000000}%
\pgfsetfillcolor{currentfill}%
\pgfsetlinewidth{0.803000pt}%
\definecolor{currentstroke}{rgb}{0.000000,0.000000,0.000000}%
\pgfsetstrokecolor{currentstroke}%
\pgfsetdash{}{0pt}%
\pgfsys@defobject{currentmarker}{\pgfqpoint{0.000000in}{-0.048611in}}{\pgfqpoint{0.000000in}{0.000000in}}{%
\pgfpathmoveto{\pgfqpoint{0.000000in}{0.000000in}}%
\pgfpathlineto{\pgfqpoint{0.000000in}{-0.048611in}}%
\pgfusepath{stroke,fill}%
}%
\begin{pgfscope}%
\pgfsys@transformshift{2.266667in}{0.900000in}%
\pgfsys@useobject{currentmarker}{}%
\end{pgfscope}%
\end{pgfscope}%
\begin{pgfscope}%
\definecolor{textcolor}{rgb}{0.000000,0.000000,0.000000}%
\pgfsetstrokecolor{textcolor}%
\pgfsetfillcolor{textcolor}%
\pgftext[x=2.266667in,y=0.802778in,,top]{\color{textcolor}\sffamily\fontsize{20.000000}{24.000000}\selectfont \(\displaystyle {0}\)}%
\end{pgfscope}%
\begin{pgfscope}%
\pgfpathrectangle{\pgfqpoint{1.200000in}{0.900000in}}{\pgfqpoint{6.400000in}{4.800000in}}%
\pgfusepath{clip}%
\pgfsetrectcap%
\pgfsetroundjoin%
\pgfsetlinewidth{0.803000pt}%
\definecolor{currentstroke}{rgb}{0.690196,0.690196,0.690196}%
\pgfsetstrokecolor{currentstroke}%
\pgfsetdash{}{0pt}%
\pgfpathmoveto{\pgfqpoint{3.333333in}{0.900000in}}%
\pgfpathlineto{\pgfqpoint{3.333333in}{5.700000in}}%
\pgfusepath{stroke}%
\end{pgfscope}%
\begin{pgfscope}%
\pgfsetbuttcap%
\pgfsetroundjoin%
\definecolor{currentfill}{rgb}{0.000000,0.000000,0.000000}%
\pgfsetfillcolor{currentfill}%
\pgfsetlinewidth{0.803000pt}%
\definecolor{currentstroke}{rgb}{0.000000,0.000000,0.000000}%
\pgfsetstrokecolor{currentstroke}%
\pgfsetdash{}{0pt}%
\pgfsys@defobject{currentmarker}{\pgfqpoint{0.000000in}{-0.048611in}}{\pgfqpoint{0.000000in}{0.000000in}}{%
\pgfpathmoveto{\pgfqpoint{0.000000in}{0.000000in}}%
\pgfpathlineto{\pgfqpoint{0.000000in}{-0.048611in}}%
\pgfusepath{stroke,fill}%
}%
\begin{pgfscope}%
\pgfsys@transformshift{3.333333in}{0.900000in}%
\pgfsys@useobject{currentmarker}{}%
\end{pgfscope}%
\end{pgfscope}%
\begin{pgfscope}%
\definecolor{textcolor}{rgb}{0.000000,0.000000,0.000000}%
\pgfsetstrokecolor{textcolor}%
\pgfsetfillcolor{textcolor}%
\pgftext[x=3.333333in,y=0.802778in,,top]{\color{textcolor}\sffamily\fontsize{20.000000}{24.000000}\selectfont \(\displaystyle {20}\)}%
\end{pgfscope}%
\begin{pgfscope}%
\pgfpathrectangle{\pgfqpoint{1.200000in}{0.900000in}}{\pgfqpoint{6.400000in}{4.800000in}}%
\pgfusepath{clip}%
\pgfsetrectcap%
\pgfsetroundjoin%
\pgfsetlinewidth{0.803000pt}%
\definecolor{currentstroke}{rgb}{0.690196,0.690196,0.690196}%
\pgfsetstrokecolor{currentstroke}%
\pgfsetdash{}{0pt}%
\pgfpathmoveto{\pgfqpoint{4.400000in}{0.900000in}}%
\pgfpathlineto{\pgfqpoint{4.400000in}{5.700000in}}%
\pgfusepath{stroke}%
\end{pgfscope}%
\begin{pgfscope}%
\pgfsetbuttcap%
\pgfsetroundjoin%
\definecolor{currentfill}{rgb}{0.000000,0.000000,0.000000}%
\pgfsetfillcolor{currentfill}%
\pgfsetlinewidth{0.803000pt}%
\definecolor{currentstroke}{rgb}{0.000000,0.000000,0.000000}%
\pgfsetstrokecolor{currentstroke}%
\pgfsetdash{}{0pt}%
\pgfsys@defobject{currentmarker}{\pgfqpoint{0.000000in}{-0.048611in}}{\pgfqpoint{0.000000in}{0.000000in}}{%
\pgfpathmoveto{\pgfqpoint{0.000000in}{0.000000in}}%
\pgfpathlineto{\pgfqpoint{0.000000in}{-0.048611in}}%
\pgfusepath{stroke,fill}%
}%
\begin{pgfscope}%
\pgfsys@transformshift{4.400000in}{0.900000in}%
\pgfsys@useobject{currentmarker}{}%
\end{pgfscope}%
\end{pgfscope}%
\begin{pgfscope}%
\definecolor{textcolor}{rgb}{0.000000,0.000000,0.000000}%
\pgfsetstrokecolor{textcolor}%
\pgfsetfillcolor{textcolor}%
\pgftext[x=4.400000in,y=0.802778in,,top]{\color{textcolor}\sffamily\fontsize{20.000000}{24.000000}\selectfont \(\displaystyle {40}\)}%
\end{pgfscope}%
\begin{pgfscope}%
\pgfpathrectangle{\pgfqpoint{1.200000in}{0.900000in}}{\pgfqpoint{6.400000in}{4.800000in}}%
\pgfusepath{clip}%
\pgfsetrectcap%
\pgfsetroundjoin%
\pgfsetlinewidth{0.803000pt}%
\definecolor{currentstroke}{rgb}{0.690196,0.690196,0.690196}%
\pgfsetstrokecolor{currentstroke}%
\pgfsetdash{}{0pt}%
\pgfpathmoveto{\pgfqpoint{5.466667in}{0.900000in}}%
\pgfpathlineto{\pgfqpoint{5.466667in}{5.700000in}}%
\pgfusepath{stroke}%
\end{pgfscope}%
\begin{pgfscope}%
\pgfsetbuttcap%
\pgfsetroundjoin%
\definecolor{currentfill}{rgb}{0.000000,0.000000,0.000000}%
\pgfsetfillcolor{currentfill}%
\pgfsetlinewidth{0.803000pt}%
\definecolor{currentstroke}{rgb}{0.000000,0.000000,0.000000}%
\pgfsetstrokecolor{currentstroke}%
\pgfsetdash{}{0pt}%
\pgfsys@defobject{currentmarker}{\pgfqpoint{0.000000in}{-0.048611in}}{\pgfqpoint{0.000000in}{0.000000in}}{%
\pgfpathmoveto{\pgfqpoint{0.000000in}{0.000000in}}%
\pgfpathlineto{\pgfqpoint{0.000000in}{-0.048611in}}%
\pgfusepath{stroke,fill}%
}%
\begin{pgfscope}%
\pgfsys@transformshift{5.466667in}{0.900000in}%
\pgfsys@useobject{currentmarker}{}%
\end{pgfscope}%
\end{pgfscope}%
\begin{pgfscope}%
\definecolor{textcolor}{rgb}{0.000000,0.000000,0.000000}%
\pgfsetstrokecolor{textcolor}%
\pgfsetfillcolor{textcolor}%
\pgftext[x=5.466667in,y=0.802778in,,top]{\color{textcolor}\sffamily\fontsize{20.000000}{24.000000}\selectfont \(\displaystyle {60}\)}%
\end{pgfscope}%
\begin{pgfscope}%
\pgfpathrectangle{\pgfqpoint{1.200000in}{0.900000in}}{\pgfqpoint{6.400000in}{4.800000in}}%
\pgfusepath{clip}%
\pgfsetrectcap%
\pgfsetroundjoin%
\pgfsetlinewidth{0.803000pt}%
\definecolor{currentstroke}{rgb}{0.690196,0.690196,0.690196}%
\pgfsetstrokecolor{currentstroke}%
\pgfsetdash{}{0pt}%
\pgfpathmoveto{\pgfqpoint{6.533333in}{0.900000in}}%
\pgfpathlineto{\pgfqpoint{6.533333in}{5.700000in}}%
\pgfusepath{stroke}%
\end{pgfscope}%
\begin{pgfscope}%
\pgfsetbuttcap%
\pgfsetroundjoin%
\definecolor{currentfill}{rgb}{0.000000,0.000000,0.000000}%
\pgfsetfillcolor{currentfill}%
\pgfsetlinewidth{0.803000pt}%
\definecolor{currentstroke}{rgb}{0.000000,0.000000,0.000000}%
\pgfsetstrokecolor{currentstroke}%
\pgfsetdash{}{0pt}%
\pgfsys@defobject{currentmarker}{\pgfqpoint{0.000000in}{-0.048611in}}{\pgfqpoint{0.000000in}{0.000000in}}{%
\pgfpathmoveto{\pgfqpoint{0.000000in}{0.000000in}}%
\pgfpathlineto{\pgfqpoint{0.000000in}{-0.048611in}}%
\pgfusepath{stroke,fill}%
}%
\begin{pgfscope}%
\pgfsys@transformshift{6.533333in}{0.900000in}%
\pgfsys@useobject{currentmarker}{}%
\end{pgfscope}%
\end{pgfscope}%
\begin{pgfscope}%
\definecolor{textcolor}{rgb}{0.000000,0.000000,0.000000}%
\pgfsetstrokecolor{textcolor}%
\pgfsetfillcolor{textcolor}%
\pgftext[x=6.533333in,y=0.802778in,,top]{\color{textcolor}\sffamily\fontsize{20.000000}{24.000000}\selectfont \(\displaystyle {80}\)}%
\end{pgfscope}%
\begin{pgfscope}%
\pgfpathrectangle{\pgfqpoint{1.200000in}{0.900000in}}{\pgfqpoint{6.400000in}{4.800000in}}%
\pgfusepath{clip}%
\pgfsetrectcap%
\pgfsetroundjoin%
\pgfsetlinewidth{0.803000pt}%
\definecolor{currentstroke}{rgb}{0.690196,0.690196,0.690196}%
\pgfsetstrokecolor{currentstroke}%
\pgfsetdash{}{0pt}%
\pgfpathmoveto{\pgfqpoint{7.600000in}{0.900000in}}%
\pgfpathlineto{\pgfqpoint{7.600000in}{5.700000in}}%
\pgfusepath{stroke}%
\end{pgfscope}%
\begin{pgfscope}%
\pgfsetbuttcap%
\pgfsetroundjoin%
\definecolor{currentfill}{rgb}{0.000000,0.000000,0.000000}%
\pgfsetfillcolor{currentfill}%
\pgfsetlinewidth{0.803000pt}%
\definecolor{currentstroke}{rgb}{0.000000,0.000000,0.000000}%
\pgfsetstrokecolor{currentstroke}%
\pgfsetdash{}{0pt}%
\pgfsys@defobject{currentmarker}{\pgfqpoint{0.000000in}{-0.048611in}}{\pgfqpoint{0.000000in}{0.000000in}}{%
\pgfpathmoveto{\pgfqpoint{0.000000in}{0.000000in}}%
\pgfpathlineto{\pgfqpoint{0.000000in}{-0.048611in}}%
\pgfusepath{stroke,fill}%
}%
\begin{pgfscope}%
\pgfsys@transformshift{7.600000in}{0.900000in}%
\pgfsys@useobject{currentmarker}{}%
\end{pgfscope}%
\end{pgfscope}%
\begin{pgfscope}%
\definecolor{textcolor}{rgb}{0.000000,0.000000,0.000000}%
\pgfsetstrokecolor{textcolor}%
\pgfsetfillcolor{textcolor}%
\pgftext[x=7.600000in,y=0.802778in,,top]{\color{textcolor}\sffamily\fontsize{20.000000}{24.000000}\selectfont \(\displaystyle {100}\)}%
\end{pgfscope}%
\begin{pgfscope}%
\definecolor{textcolor}{rgb}{0.000000,0.000000,0.000000}%
\pgfsetstrokecolor{textcolor}%
\pgfsetfillcolor{textcolor}%
\pgftext[x=4.400000in,y=0.491155in,,top]{\color{textcolor}\sffamily\fontsize{20.000000}{24.000000}\selectfont \(\displaystyle \mathrm{t}/\si{ns}\)}%
\end{pgfscope}%
\begin{pgfscope}%
\pgfpathrectangle{\pgfqpoint{1.200000in}{0.900000in}}{\pgfqpoint{6.400000in}{4.800000in}}%
\pgfusepath{clip}%
\pgfsetrectcap%
\pgfsetroundjoin%
\pgfsetlinewidth{0.803000pt}%
\definecolor{currentstroke}{rgb}{0.690196,0.690196,0.690196}%
\pgfsetstrokecolor{currentstroke}%
\pgfsetdash{}{0pt}%
\pgfpathmoveto{\pgfqpoint{1.200000in}{0.900000in}}%
\pgfpathlineto{\pgfqpoint{7.600000in}{0.900000in}}%
\pgfusepath{stroke}%
\end{pgfscope}%
\begin{pgfscope}%
\pgfsetbuttcap%
\pgfsetroundjoin%
\definecolor{currentfill}{rgb}{0.000000,0.000000,0.000000}%
\pgfsetfillcolor{currentfill}%
\pgfsetlinewidth{0.803000pt}%
\definecolor{currentstroke}{rgb}{0.000000,0.000000,0.000000}%
\pgfsetstrokecolor{currentstroke}%
\pgfsetdash{}{0pt}%
\pgfsys@defobject{currentmarker}{\pgfqpoint{-0.048611in}{0.000000in}}{\pgfqpoint{-0.000000in}{0.000000in}}{%
\pgfpathmoveto{\pgfqpoint{-0.000000in}{0.000000in}}%
\pgfpathlineto{\pgfqpoint{-0.048611in}{0.000000in}}%
\pgfusepath{stroke,fill}%
}%
\begin{pgfscope}%
\pgfsys@transformshift{1.200000in}{0.900000in}%
\pgfsys@useobject{currentmarker}{}%
\end{pgfscope}%
\end{pgfscope}%
\begin{pgfscope}%
\definecolor{textcolor}{rgb}{0.000000,0.000000,0.000000}%
\pgfsetstrokecolor{textcolor}%
\pgfsetfillcolor{textcolor}%
\pgftext[x=0.496001in, y=0.799981in, left, base]{\color{textcolor}\sffamily\fontsize{20.000000}{24.000000}\selectfont \(\displaystyle {0.000}\)}%
\end{pgfscope}%
\begin{pgfscope}%
\pgfpathrectangle{\pgfqpoint{1.200000in}{0.900000in}}{\pgfqpoint{6.400000in}{4.800000in}}%
\pgfusepath{clip}%
\pgfsetrectcap%
\pgfsetroundjoin%
\pgfsetlinewidth{0.803000pt}%
\definecolor{currentstroke}{rgb}{0.690196,0.690196,0.690196}%
\pgfsetstrokecolor{currentstroke}%
\pgfsetdash{}{0pt}%
\pgfpathmoveto{\pgfqpoint{1.200000in}{1.500000in}}%
\pgfpathlineto{\pgfqpoint{7.600000in}{1.500000in}}%
\pgfusepath{stroke}%
\end{pgfscope}%
\begin{pgfscope}%
\pgfsetbuttcap%
\pgfsetroundjoin%
\definecolor{currentfill}{rgb}{0.000000,0.000000,0.000000}%
\pgfsetfillcolor{currentfill}%
\pgfsetlinewidth{0.803000pt}%
\definecolor{currentstroke}{rgb}{0.000000,0.000000,0.000000}%
\pgfsetstrokecolor{currentstroke}%
\pgfsetdash{}{0pt}%
\pgfsys@defobject{currentmarker}{\pgfqpoint{-0.048611in}{0.000000in}}{\pgfqpoint{-0.000000in}{0.000000in}}{%
\pgfpathmoveto{\pgfqpoint{-0.000000in}{0.000000in}}%
\pgfpathlineto{\pgfqpoint{-0.048611in}{0.000000in}}%
\pgfusepath{stroke,fill}%
}%
\begin{pgfscope}%
\pgfsys@transformshift{1.200000in}{1.500000in}%
\pgfsys@useobject{currentmarker}{}%
\end{pgfscope}%
\end{pgfscope}%
\begin{pgfscope}%
\definecolor{textcolor}{rgb}{0.000000,0.000000,0.000000}%
\pgfsetstrokecolor{textcolor}%
\pgfsetfillcolor{textcolor}%
\pgftext[x=0.496001in, y=1.399981in, left, base]{\color{textcolor}\sffamily\fontsize{20.000000}{24.000000}\selectfont \(\displaystyle {0.005}\)}%
\end{pgfscope}%
\begin{pgfscope}%
\pgfpathrectangle{\pgfqpoint{1.200000in}{0.900000in}}{\pgfqpoint{6.400000in}{4.800000in}}%
\pgfusepath{clip}%
\pgfsetrectcap%
\pgfsetroundjoin%
\pgfsetlinewidth{0.803000pt}%
\definecolor{currentstroke}{rgb}{0.690196,0.690196,0.690196}%
\pgfsetstrokecolor{currentstroke}%
\pgfsetdash{}{0pt}%
\pgfpathmoveto{\pgfqpoint{1.200000in}{2.100000in}}%
\pgfpathlineto{\pgfqpoint{7.600000in}{2.100000in}}%
\pgfusepath{stroke}%
\end{pgfscope}%
\begin{pgfscope}%
\pgfsetbuttcap%
\pgfsetroundjoin%
\definecolor{currentfill}{rgb}{0.000000,0.000000,0.000000}%
\pgfsetfillcolor{currentfill}%
\pgfsetlinewidth{0.803000pt}%
\definecolor{currentstroke}{rgb}{0.000000,0.000000,0.000000}%
\pgfsetstrokecolor{currentstroke}%
\pgfsetdash{}{0pt}%
\pgfsys@defobject{currentmarker}{\pgfqpoint{-0.048611in}{0.000000in}}{\pgfqpoint{-0.000000in}{0.000000in}}{%
\pgfpathmoveto{\pgfqpoint{-0.000000in}{0.000000in}}%
\pgfpathlineto{\pgfqpoint{-0.048611in}{0.000000in}}%
\pgfusepath{stroke,fill}%
}%
\begin{pgfscope}%
\pgfsys@transformshift{1.200000in}{2.100000in}%
\pgfsys@useobject{currentmarker}{}%
\end{pgfscope}%
\end{pgfscope}%
\begin{pgfscope}%
\definecolor{textcolor}{rgb}{0.000000,0.000000,0.000000}%
\pgfsetstrokecolor{textcolor}%
\pgfsetfillcolor{textcolor}%
\pgftext[x=0.496001in, y=1.999981in, left, base]{\color{textcolor}\sffamily\fontsize{20.000000}{24.000000}\selectfont \(\displaystyle {0.010}\)}%
\end{pgfscope}%
\begin{pgfscope}%
\pgfpathrectangle{\pgfqpoint{1.200000in}{0.900000in}}{\pgfqpoint{6.400000in}{4.800000in}}%
\pgfusepath{clip}%
\pgfsetrectcap%
\pgfsetroundjoin%
\pgfsetlinewidth{0.803000pt}%
\definecolor{currentstroke}{rgb}{0.690196,0.690196,0.690196}%
\pgfsetstrokecolor{currentstroke}%
\pgfsetdash{}{0pt}%
\pgfpathmoveto{\pgfqpoint{1.200000in}{2.700000in}}%
\pgfpathlineto{\pgfqpoint{7.600000in}{2.700000in}}%
\pgfusepath{stroke}%
\end{pgfscope}%
\begin{pgfscope}%
\pgfsetbuttcap%
\pgfsetroundjoin%
\definecolor{currentfill}{rgb}{0.000000,0.000000,0.000000}%
\pgfsetfillcolor{currentfill}%
\pgfsetlinewidth{0.803000pt}%
\definecolor{currentstroke}{rgb}{0.000000,0.000000,0.000000}%
\pgfsetstrokecolor{currentstroke}%
\pgfsetdash{}{0pt}%
\pgfsys@defobject{currentmarker}{\pgfqpoint{-0.048611in}{0.000000in}}{\pgfqpoint{-0.000000in}{0.000000in}}{%
\pgfpathmoveto{\pgfqpoint{-0.000000in}{0.000000in}}%
\pgfpathlineto{\pgfqpoint{-0.048611in}{0.000000in}}%
\pgfusepath{stroke,fill}%
}%
\begin{pgfscope}%
\pgfsys@transformshift{1.200000in}{2.700000in}%
\pgfsys@useobject{currentmarker}{}%
\end{pgfscope}%
\end{pgfscope}%
\begin{pgfscope}%
\definecolor{textcolor}{rgb}{0.000000,0.000000,0.000000}%
\pgfsetstrokecolor{textcolor}%
\pgfsetfillcolor{textcolor}%
\pgftext[x=0.496001in, y=2.599981in, left, base]{\color{textcolor}\sffamily\fontsize{20.000000}{24.000000}\selectfont \(\displaystyle {0.015}\)}%
\end{pgfscope}%
\begin{pgfscope}%
\pgfpathrectangle{\pgfqpoint{1.200000in}{0.900000in}}{\pgfqpoint{6.400000in}{4.800000in}}%
\pgfusepath{clip}%
\pgfsetrectcap%
\pgfsetroundjoin%
\pgfsetlinewidth{0.803000pt}%
\definecolor{currentstroke}{rgb}{0.690196,0.690196,0.690196}%
\pgfsetstrokecolor{currentstroke}%
\pgfsetdash{}{0pt}%
\pgfpathmoveto{\pgfqpoint{1.200000in}{3.300000in}}%
\pgfpathlineto{\pgfqpoint{7.600000in}{3.300000in}}%
\pgfusepath{stroke}%
\end{pgfscope}%
\begin{pgfscope}%
\pgfsetbuttcap%
\pgfsetroundjoin%
\definecolor{currentfill}{rgb}{0.000000,0.000000,0.000000}%
\pgfsetfillcolor{currentfill}%
\pgfsetlinewidth{0.803000pt}%
\definecolor{currentstroke}{rgb}{0.000000,0.000000,0.000000}%
\pgfsetstrokecolor{currentstroke}%
\pgfsetdash{}{0pt}%
\pgfsys@defobject{currentmarker}{\pgfqpoint{-0.048611in}{0.000000in}}{\pgfqpoint{-0.000000in}{0.000000in}}{%
\pgfpathmoveto{\pgfqpoint{-0.000000in}{0.000000in}}%
\pgfpathlineto{\pgfqpoint{-0.048611in}{0.000000in}}%
\pgfusepath{stroke,fill}%
}%
\begin{pgfscope}%
\pgfsys@transformshift{1.200000in}{3.300000in}%
\pgfsys@useobject{currentmarker}{}%
\end{pgfscope}%
\end{pgfscope}%
\begin{pgfscope}%
\definecolor{textcolor}{rgb}{0.000000,0.000000,0.000000}%
\pgfsetstrokecolor{textcolor}%
\pgfsetfillcolor{textcolor}%
\pgftext[x=0.496001in, y=3.199981in, left, base]{\color{textcolor}\sffamily\fontsize{20.000000}{24.000000}\selectfont \(\displaystyle {0.020}\)}%
\end{pgfscope}%
\begin{pgfscope}%
\pgfpathrectangle{\pgfqpoint{1.200000in}{0.900000in}}{\pgfqpoint{6.400000in}{4.800000in}}%
\pgfusepath{clip}%
\pgfsetrectcap%
\pgfsetroundjoin%
\pgfsetlinewidth{0.803000pt}%
\definecolor{currentstroke}{rgb}{0.690196,0.690196,0.690196}%
\pgfsetstrokecolor{currentstroke}%
\pgfsetdash{}{0pt}%
\pgfpathmoveto{\pgfqpoint{1.200000in}{3.900000in}}%
\pgfpathlineto{\pgfqpoint{7.600000in}{3.900000in}}%
\pgfusepath{stroke}%
\end{pgfscope}%
\begin{pgfscope}%
\pgfsetbuttcap%
\pgfsetroundjoin%
\definecolor{currentfill}{rgb}{0.000000,0.000000,0.000000}%
\pgfsetfillcolor{currentfill}%
\pgfsetlinewidth{0.803000pt}%
\definecolor{currentstroke}{rgb}{0.000000,0.000000,0.000000}%
\pgfsetstrokecolor{currentstroke}%
\pgfsetdash{}{0pt}%
\pgfsys@defobject{currentmarker}{\pgfqpoint{-0.048611in}{0.000000in}}{\pgfqpoint{-0.000000in}{0.000000in}}{%
\pgfpathmoveto{\pgfqpoint{-0.000000in}{0.000000in}}%
\pgfpathlineto{\pgfqpoint{-0.048611in}{0.000000in}}%
\pgfusepath{stroke,fill}%
}%
\begin{pgfscope}%
\pgfsys@transformshift{1.200000in}{3.900000in}%
\pgfsys@useobject{currentmarker}{}%
\end{pgfscope}%
\end{pgfscope}%
\begin{pgfscope}%
\definecolor{textcolor}{rgb}{0.000000,0.000000,0.000000}%
\pgfsetstrokecolor{textcolor}%
\pgfsetfillcolor{textcolor}%
\pgftext[x=0.496001in, y=3.799981in, left, base]{\color{textcolor}\sffamily\fontsize{20.000000}{24.000000}\selectfont \(\displaystyle {0.025}\)}%
\end{pgfscope}%
\begin{pgfscope}%
\pgfpathrectangle{\pgfqpoint{1.200000in}{0.900000in}}{\pgfqpoint{6.400000in}{4.800000in}}%
\pgfusepath{clip}%
\pgfsetrectcap%
\pgfsetroundjoin%
\pgfsetlinewidth{0.803000pt}%
\definecolor{currentstroke}{rgb}{0.690196,0.690196,0.690196}%
\pgfsetstrokecolor{currentstroke}%
\pgfsetdash{}{0pt}%
\pgfpathmoveto{\pgfqpoint{1.200000in}{4.500000in}}%
\pgfpathlineto{\pgfqpoint{7.600000in}{4.500000in}}%
\pgfusepath{stroke}%
\end{pgfscope}%
\begin{pgfscope}%
\pgfsetbuttcap%
\pgfsetroundjoin%
\definecolor{currentfill}{rgb}{0.000000,0.000000,0.000000}%
\pgfsetfillcolor{currentfill}%
\pgfsetlinewidth{0.803000pt}%
\definecolor{currentstroke}{rgb}{0.000000,0.000000,0.000000}%
\pgfsetstrokecolor{currentstroke}%
\pgfsetdash{}{0pt}%
\pgfsys@defobject{currentmarker}{\pgfqpoint{-0.048611in}{0.000000in}}{\pgfqpoint{-0.000000in}{0.000000in}}{%
\pgfpathmoveto{\pgfqpoint{-0.000000in}{0.000000in}}%
\pgfpathlineto{\pgfqpoint{-0.048611in}{0.000000in}}%
\pgfusepath{stroke,fill}%
}%
\begin{pgfscope}%
\pgfsys@transformshift{1.200000in}{4.500000in}%
\pgfsys@useobject{currentmarker}{}%
\end{pgfscope}%
\end{pgfscope}%
\begin{pgfscope}%
\definecolor{textcolor}{rgb}{0.000000,0.000000,0.000000}%
\pgfsetstrokecolor{textcolor}%
\pgfsetfillcolor{textcolor}%
\pgftext[x=0.496001in, y=4.399981in, left, base]{\color{textcolor}\sffamily\fontsize{20.000000}{24.000000}\selectfont \(\displaystyle {0.030}\)}%
\end{pgfscope}%
\begin{pgfscope}%
\pgfpathrectangle{\pgfqpoint{1.200000in}{0.900000in}}{\pgfqpoint{6.400000in}{4.800000in}}%
\pgfusepath{clip}%
\pgfsetrectcap%
\pgfsetroundjoin%
\pgfsetlinewidth{0.803000pt}%
\definecolor{currentstroke}{rgb}{0.690196,0.690196,0.690196}%
\pgfsetstrokecolor{currentstroke}%
\pgfsetdash{}{0pt}%
\pgfpathmoveto{\pgfqpoint{1.200000in}{5.100000in}}%
\pgfpathlineto{\pgfqpoint{7.600000in}{5.100000in}}%
\pgfusepath{stroke}%
\end{pgfscope}%
\begin{pgfscope}%
\pgfsetbuttcap%
\pgfsetroundjoin%
\definecolor{currentfill}{rgb}{0.000000,0.000000,0.000000}%
\pgfsetfillcolor{currentfill}%
\pgfsetlinewidth{0.803000pt}%
\definecolor{currentstroke}{rgb}{0.000000,0.000000,0.000000}%
\pgfsetstrokecolor{currentstroke}%
\pgfsetdash{}{0pt}%
\pgfsys@defobject{currentmarker}{\pgfqpoint{-0.048611in}{0.000000in}}{\pgfqpoint{-0.000000in}{0.000000in}}{%
\pgfpathmoveto{\pgfqpoint{-0.000000in}{0.000000in}}%
\pgfpathlineto{\pgfqpoint{-0.048611in}{0.000000in}}%
\pgfusepath{stroke,fill}%
}%
\begin{pgfscope}%
\pgfsys@transformshift{1.200000in}{5.100000in}%
\pgfsys@useobject{currentmarker}{}%
\end{pgfscope}%
\end{pgfscope}%
\begin{pgfscope}%
\definecolor{textcolor}{rgb}{0.000000,0.000000,0.000000}%
\pgfsetstrokecolor{textcolor}%
\pgfsetfillcolor{textcolor}%
\pgftext[x=0.496001in, y=4.999981in, left, base]{\color{textcolor}\sffamily\fontsize{20.000000}{24.000000}\selectfont \(\displaystyle {0.035}\)}%
\end{pgfscope}%
\begin{pgfscope}%
\pgfpathrectangle{\pgfqpoint{1.200000in}{0.900000in}}{\pgfqpoint{6.400000in}{4.800000in}}%
\pgfusepath{clip}%
\pgfsetrectcap%
\pgfsetroundjoin%
\pgfsetlinewidth{0.803000pt}%
\definecolor{currentstroke}{rgb}{0.690196,0.690196,0.690196}%
\pgfsetstrokecolor{currentstroke}%
\pgfsetdash{}{0pt}%
\pgfpathmoveto{\pgfqpoint{1.200000in}{5.700000in}}%
\pgfpathlineto{\pgfqpoint{7.600000in}{5.700000in}}%
\pgfusepath{stroke}%
\end{pgfscope}%
\begin{pgfscope}%
\pgfsetbuttcap%
\pgfsetroundjoin%
\definecolor{currentfill}{rgb}{0.000000,0.000000,0.000000}%
\pgfsetfillcolor{currentfill}%
\pgfsetlinewidth{0.803000pt}%
\definecolor{currentstroke}{rgb}{0.000000,0.000000,0.000000}%
\pgfsetstrokecolor{currentstroke}%
\pgfsetdash{}{0pt}%
\pgfsys@defobject{currentmarker}{\pgfqpoint{-0.048611in}{0.000000in}}{\pgfqpoint{-0.000000in}{0.000000in}}{%
\pgfpathmoveto{\pgfqpoint{-0.000000in}{0.000000in}}%
\pgfpathlineto{\pgfqpoint{-0.048611in}{0.000000in}}%
\pgfusepath{stroke,fill}%
}%
\begin{pgfscope}%
\pgfsys@transformshift{1.200000in}{5.700000in}%
\pgfsys@useobject{currentmarker}{}%
\end{pgfscope}%
\end{pgfscope}%
\begin{pgfscope}%
\definecolor{textcolor}{rgb}{0.000000,0.000000,0.000000}%
\pgfsetstrokecolor{textcolor}%
\pgfsetfillcolor{textcolor}%
\pgftext[x=0.496001in, y=5.599981in, left, base]{\color{textcolor}\sffamily\fontsize{20.000000}{24.000000}\selectfont \(\displaystyle {0.040}\)}%
\end{pgfscope}%
\begin{pgfscope}%
\definecolor{textcolor}{rgb}{0.000000,0.000000,0.000000}%
\pgfsetstrokecolor{textcolor}%
\pgfsetfillcolor{textcolor}%
\pgftext[x=0.440445in,y=3.300000in,,bottom,rotate=90.000000]{\color{textcolor}\sffamily\fontsize{20.000000}{24.000000}\selectfont \(\displaystyle \phi(t)\)}%
\end{pgfscope}%
\begin{pgfscope}%
\pgfpathrectangle{\pgfqpoint{1.200000in}{0.900000in}}{\pgfqpoint{6.400000in}{4.800000in}}%
\pgfusepath{clip}%
\pgfsetrectcap%
\pgfsetroundjoin%
\pgfsetlinewidth{2.007500pt}%
\definecolor{currentstroke}{rgb}{0.000000,0.000000,1.000000}%
\pgfsetstrokecolor{currentstroke}%
\pgfsetdash{}{0pt}%
\pgfpathmoveto{\pgfqpoint{1.200000in}{0.900180in}}%
\pgfpathlineto{\pgfqpoint{1.349333in}{0.901640in}}%
\pgfpathlineto{\pgfqpoint{1.424000in}{0.904430in}}%
\pgfpathlineto{\pgfqpoint{1.477333in}{0.908611in}}%
\pgfpathlineto{\pgfqpoint{1.520000in}{0.914264in}}%
\pgfpathlineto{\pgfqpoint{1.552000in}{0.920503in}}%
\pgfpathlineto{\pgfqpoint{1.584000in}{0.929081in}}%
\pgfpathlineto{\pgfqpoint{1.610667in}{0.938520in}}%
\pgfpathlineto{\pgfqpoint{1.637333in}{0.950557in}}%
\pgfpathlineto{\pgfqpoint{1.658667in}{0.962429in}}%
\pgfpathlineto{\pgfqpoint{1.680000in}{0.976640in}}%
\pgfpathlineto{\pgfqpoint{1.701333in}{0.993540in}}%
\pgfpathlineto{\pgfqpoint{1.722667in}{1.013508in}}%
\pgfpathlineto{\pgfqpoint{1.744000in}{1.036946in}}%
\pgfpathlineto{\pgfqpoint{1.765333in}{1.064277in}}%
\pgfpathlineto{\pgfqpoint{1.786667in}{1.095941in}}%
\pgfpathlineto{\pgfqpoint{1.808000in}{1.132381in}}%
\pgfpathlineto{\pgfqpoint{1.829333in}{1.174044in}}%
\pgfpathlineto{\pgfqpoint{1.850667in}{1.221364in}}%
\pgfpathlineto{\pgfqpoint{1.872000in}{1.274753in}}%
\pgfpathlineto{\pgfqpoint{1.893333in}{1.334591in}}%
\pgfpathlineto{\pgfqpoint{1.914667in}{1.401210in}}%
\pgfpathlineto{\pgfqpoint{1.936000in}{1.474882in}}%
\pgfpathlineto{\pgfqpoint{1.957333in}{1.555807in}}%
\pgfpathlineto{\pgfqpoint{1.978667in}{1.644101in}}%
\pgfpathlineto{\pgfqpoint{2.005333in}{1.764845in}}%
\pgfpathlineto{\pgfqpoint{2.032000in}{1.896915in}}%
\pgfpathlineto{\pgfqpoint{2.058667in}{2.039823in}}%
\pgfpathlineto{\pgfqpoint{2.090667in}{2.224481in}}%
\pgfpathlineto{\pgfqpoint{2.128000in}{2.455554in}}%
\pgfpathlineto{\pgfqpoint{2.170667in}{2.735135in}}%
\pgfpathlineto{\pgfqpoint{2.314667in}{3.693900in}}%
\pgfpathlineto{\pgfqpoint{2.346667in}{3.886091in}}%
\pgfpathlineto{\pgfqpoint{2.373333in}{4.034620in}}%
\pgfpathlineto{\pgfqpoint{2.400000in}{4.170770in}}%
\pgfpathlineto{\pgfqpoint{2.421333in}{4.269830in}}%
\pgfpathlineto{\pgfqpoint{2.442667in}{4.359503in}}%
\pgfpathlineto{\pgfqpoint{2.464000in}{4.439359in}}%
\pgfpathlineto{\pgfqpoint{2.485333in}{4.509109in}}%
\pgfpathlineto{\pgfqpoint{2.501333in}{4.554694in}}%
\pgfpathlineto{\pgfqpoint{2.517333in}{4.594493in}}%
\pgfpathlineto{\pgfqpoint{2.533333in}{4.628534in}}%
\pgfpathlineto{\pgfqpoint{2.549333in}{4.656884in}}%
\pgfpathlineto{\pgfqpoint{2.565333in}{4.679646in}}%
\pgfpathlineto{\pgfqpoint{2.576000in}{4.691782in}}%
\pgfpathlineto{\pgfqpoint{2.586667in}{4.701541in}}%
\pgfpathlineto{\pgfqpoint{2.597333in}{4.708979in}}%
\pgfpathlineto{\pgfqpoint{2.608000in}{4.714152in}}%
\pgfpathlineto{\pgfqpoint{2.618667in}{4.717124in}}%
\pgfpathlineto{\pgfqpoint{2.629333in}{4.717962in}}%
\pgfpathlineto{\pgfqpoint{2.640000in}{4.716737in}}%
\pgfpathlineto{\pgfqpoint{2.650667in}{4.713522in}}%
\pgfpathlineto{\pgfqpoint{2.661333in}{4.708393in}}%
\pgfpathlineto{\pgfqpoint{2.672000in}{4.701429in}}%
\pgfpathlineto{\pgfqpoint{2.688000in}{4.687718in}}%
\pgfpathlineto{\pgfqpoint{2.704000in}{4.670334in}}%
\pgfpathlineto{\pgfqpoint{2.720000in}{4.649553in}}%
\pgfpathlineto{\pgfqpoint{2.736000in}{4.625655in}}%
\pgfpathlineto{\pgfqpoint{2.757333in}{4.589415in}}%
\pgfpathlineto{\pgfqpoint{2.778667in}{4.548757in}}%
\pgfpathlineto{\pgfqpoint{2.805333in}{4.492652in}}%
\pgfpathlineto{\pgfqpoint{2.837333in}{4.419047in}}%
\pgfpathlineto{\pgfqpoint{2.874667in}{4.326721in}}%
\pgfpathlineto{\pgfqpoint{2.922667in}{4.201458in}}%
\pgfpathlineto{\pgfqpoint{3.130667in}{3.650035in}}%
\pgfpathlineto{\pgfqpoint{3.189333in}{3.504773in}}%
\pgfpathlineto{\pgfqpoint{3.242667in}{3.378577in}}%
\pgfpathlineto{\pgfqpoint{3.296000in}{3.258100in}}%
\pgfpathlineto{\pgfqpoint{3.349333in}{3.143282in}}%
\pgfpathlineto{\pgfqpoint{3.402667in}{3.033959in}}%
\pgfpathlineto{\pgfqpoint{3.456000in}{2.929921in}}%
\pgfpathlineto{\pgfqpoint{3.509333in}{2.830935in}}%
\pgfpathlineto{\pgfqpoint{3.562667in}{2.736768in}}%
\pgfpathlineto{\pgfqpoint{3.616000in}{2.647190in}}%
\pgfpathlineto{\pgfqpoint{3.669333in}{2.561979in}}%
\pgfpathlineto{\pgfqpoint{3.722667in}{2.480924in}}%
\pgfpathlineto{\pgfqpoint{3.776000in}{2.403821in}}%
\pgfpathlineto{\pgfqpoint{3.829333in}{2.330479in}}%
\pgfpathlineto{\pgfqpoint{3.882667in}{2.260714in}}%
\pgfpathlineto{\pgfqpoint{3.936000in}{2.194351in}}%
\pgfpathlineto{\pgfqpoint{3.989333in}{2.131225in}}%
\pgfpathlineto{\pgfqpoint{4.048000in}{2.065336in}}%
\pgfpathlineto{\pgfqpoint{4.106667in}{2.002973in}}%
\pgfpathlineto{\pgfqpoint{4.165333in}{1.943948in}}%
\pgfpathlineto{\pgfqpoint{4.224000in}{1.888081in}}%
\pgfpathlineto{\pgfqpoint{4.282667in}{1.835204in}}%
\pgfpathlineto{\pgfqpoint{4.341333in}{1.785157in}}%
\pgfpathlineto{\pgfqpoint{4.405333in}{1.733609in}}%
\pgfpathlineto{\pgfqpoint{4.469333in}{1.685064in}}%
\pgfpathlineto{\pgfqpoint{4.533333in}{1.639345in}}%
\pgfpathlineto{\pgfqpoint{4.597333in}{1.596289in}}%
\pgfpathlineto{\pgfqpoint{4.661333in}{1.555740in}}%
\pgfpathlineto{\pgfqpoint{4.730667in}{1.514473in}}%
\pgfpathlineto{\pgfqpoint{4.800000in}{1.475803in}}%
\pgfpathlineto{\pgfqpoint{4.869333in}{1.439566in}}%
\pgfpathlineto{\pgfqpoint{4.944000in}{1.403088in}}%
\pgfpathlineto{\pgfqpoint{5.018667in}{1.369076in}}%
\pgfpathlineto{\pgfqpoint{5.098667in}{1.335182in}}%
\pgfpathlineto{\pgfqpoint{5.178667in}{1.303737in}}%
\pgfpathlineto{\pgfqpoint{5.264000in}{1.272697in}}%
\pgfpathlineto{\pgfqpoint{5.349333in}{1.244042in}}%
\pgfpathlineto{\pgfqpoint{5.440000in}{1.216007in}}%
\pgfpathlineto{\pgfqpoint{5.530667in}{1.190256in}}%
\pgfpathlineto{\pgfqpoint{5.626667in}{1.165274in}}%
\pgfpathlineto{\pgfqpoint{5.728000in}{1.141233in}}%
\pgfpathlineto{\pgfqpoint{5.834667in}{1.118277in}}%
\pgfpathlineto{\pgfqpoint{5.946667in}{1.096520in}}%
\pgfpathlineto{\pgfqpoint{6.064000in}{1.076049in}}%
\pgfpathlineto{\pgfqpoint{6.186667in}{1.056924in}}%
\pgfpathlineto{\pgfqpoint{6.314667in}{1.039180in}}%
\pgfpathlineto{\pgfqpoint{6.453333in}{1.022213in}}%
\pgfpathlineto{\pgfqpoint{6.602667in}{1.006247in}}%
\pgfpathlineto{\pgfqpoint{6.762667in}{0.991447in}}%
\pgfpathlineto{\pgfqpoint{6.933333in}{0.977926in}}%
\pgfpathlineto{\pgfqpoint{7.120000in}{0.965416in}}%
\pgfpathlineto{\pgfqpoint{7.322667in}{0.954096in}}%
\pgfpathlineto{\pgfqpoint{7.546667in}{0.943850in}}%
\pgfpathlineto{\pgfqpoint{7.594667in}{0.941920in}}%
\pgfpathlineto{\pgfqpoint{7.594667in}{0.941920in}}%
\pgfusepath{stroke}%
\end{pgfscope}%
\begin{pgfscope}%
\pgfpathrectangle{\pgfqpoint{1.200000in}{0.900000in}}{\pgfqpoint{6.400000in}{4.800000in}}%
\pgfusepath{clip}%
\pgfsetrectcap%
\pgfsetroundjoin%
\pgfsetlinewidth{2.007500pt}%
\definecolor{currentstroke}{rgb}{1.000000,0.000000,0.000000}%
\pgfsetstrokecolor{currentstroke}%
\pgfsetdash{}{0pt}%
\pgfpathmoveto{\pgfqpoint{3.333333in}{0.900180in}}%
\pgfpathlineto{\pgfqpoint{3.482667in}{0.901640in}}%
\pgfpathlineto{\pgfqpoint{3.557333in}{0.904430in}}%
\pgfpathlineto{\pgfqpoint{3.610667in}{0.908611in}}%
\pgfpathlineto{\pgfqpoint{3.653333in}{0.914264in}}%
\pgfpathlineto{\pgfqpoint{3.685333in}{0.920503in}}%
\pgfpathlineto{\pgfqpoint{3.717333in}{0.929081in}}%
\pgfpathlineto{\pgfqpoint{3.744000in}{0.938520in}}%
\pgfpathlineto{\pgfqpoint{3.770667in}{0.950557in}}%
\pgfpathlineto{\pgfqpoint{3.792000in}{0.962429in}}%
\pgfpathlineto{\pgfqpoint{3.813333in}{0.976640in}}%
\pgfpathlineto{\pgfqpoint{3.834667in}{0.993540in}}%
\pgfpathlineto{\pgfqpoint{3.856000in}{1.013508in}}%
\pgfpathlineto{\pgfqpoint{3.877333in}{1.036946in}}%
\pgfpathlineto{\pgfqpoint{3.898667in}{1.064277in}}%
\pgfpathlineto{\pgfqpoint{3.920000in}{1.095941in}}%
\pgfpathlineto{\pgfqpoint{3.941333in}{1.132381in}}%
\pgfpathlineto{\pgfqpoint{3.962667in}{1.174044in}}%
\pgfpathlineto{\pgfqpoint{3.984000in}{1.221364in}}%
\pgfpathlineto{\pgfqpoint{4.005333in}{1.274753in}}%
\pgfpathlineto{\pgfqpoint{4.026667in}{1.334591in}}%
\pgfpathlineto{\pgfqpoint{4.048000in}{1.401210in}}%
\pgfpathlineto{\pgfqpoint{4.069333in}{1.474882in}}%
\pgfpathlineto{\pgfqpoint{4.090667in}{1.555807in}}%
\pgfpathlineto{\pgfqpoint{4.112000in}{1.644101in}}%
\pgfpathlineto{\pgfqpoint{4.138667in}{1.764845in}}%
\pgfpathlineto{\pgfqpoint{4.165333in}{1.896915in}}%
\pgfpathlineto{\pgfqpoint{4.192000in}{2.039823in}}%
\pgfpathlineto{\pgfqpoint{4.224000in}{2.224481in}}%
\pgfpathlineto{\pgfqpoint{4.261333in}{2.455554in}}%
\pgfpathlineto{\pgfqpoint{4.304000in}{2.735135in}}%
\pgfpathlineto{\pgfqpoint{4.448000in}{3.693900in}}%
\pgfpathlineto{\pgfqpoint{4.480000in}{3.886091in}}%
\pgfpathlineto{\pgfqpoint{4.506667in}{4.034620in}}%
\pgfpathlineto{\pgfqpoint{4.533333in}{4.170770in}}%
\pgfpathlineto{\pgfqpoint{4.554667in}{4.269830in}}%
\pgfpathlineto{\pgfqpoint{4.576000in}{4.359503in}}%
\pgfpathlineto{\pgfqpoint{4.597333in}{4.439359in}}%
\pgfpathlineto{\pgfqpoint{4.618667in}{4.509109in}}%
\pgfpathlineto{\pgfqpoint{4.634667in}{4.554694in}}%
\pgfpathlineto{\pgfqpoint{4.650667in}{4.594493in}}%
\pgfpathlineto{\pgfqpoint{4.666667in}{4.628534in}}%
\pgfpathlineto{\pgfqpoint{4.682667in}{4.656884in}}%
\pgfpathlineto{\pgfqpoint{4.698667in}{4.679646in}}%
\pgfpathlineto{\pgfqpoint{4.709333in}{4.691782in}}%
\pgfpathlineto{\pgfqpoint{4.720000in}{4.701541in}}%
\pgfpathlineto{\pgfqpoint{4.730667in}{4.708979in}}%
\pgfpathlineto{\pgfqpoint{4.741333in}{4.714152in}}%
\pgfpathlineto{\pgfqpoint{4.752000in}{4.717124in}}%
\pgfpathlineto{\pgfqpoint{4.762667in}{4.717962in}}%
\pgfpathlineto{\pgfqpoint{4.773333in}{4.716737in}}%
\pgfpathlineto{\pgfqpoint{4.784000in}{4.713522in}}%
\pgfpathlineto{\pgfqpoint{4.794667in}{4.708393in}}%
\pgfpathlineto{\pgfqpoint{4.805333in}{4.701429in}}%
\pgfpathlineto{\pgfqpoint{4.821333in}{4.687718in}}%
\pgfpathlineto{\pgfqpoint{4.837333in}{4.670334in}}%
\pgfpathlineto{\pgfqpoint{4.853333in}{4.649553in}}%
\pgfpathlineto{\pgfqpoint{4.869333in}{4.625655in}}%
\pgfpathlineto{\pgfqpoint{4.890667in}{4.589415in}}%
\pgfpathlineto{\pgfqpoint{4.912000in}{4.548757in}}%
\pgfpathlineto{\pgfqpoint{4.938667in}{4.492652in}}%
\pgfpathlineto{\pgfqpoint{4.970667in}{4.419047in}}%
\pgfpathlineto{\pgfqpoint{5.008000in}{4.326721in}}%
\pgfpathlineto{\pgfqpoint{5.056000in}{4.201458in}}%
\pgfpathlineto{\pgfqpoint{5.264000in}{3.650035in}}%
\pgfpathlineto{\pgfqpoint{5.322667in}{3.504773in}}%
\pgfpathlineto{\pgfqpoint{5.376000in}{3.378577in}}%
\pgfpathlineto{\pgfqpoint{5.429333in}{3.258100in}}%
\pgfpathlineto{\pgfqpoint{5.482667in}{3.143282in}}%
\pgfpathlineto{\pgfqpoint{5.536000in}{3.033959in}}%
\pgfpathlineto{\pgfqpoint{5.589333in}{2.929921in}}%
\pgfpathlineto{\pgfqpoint{5.642667in}{2.830935in}}%
\pgfpathlineto{\pgfqpoint{5.696000in}{2.736768in}}%
\pgfpathlineto{\pgfqpoint{5.749333in}{2.647190in}}%
\pgfpathlineto{\pgfqpoint{5.802667in}{2.561979in}}%
\pgfpathlineto{\pgfqpoint{5.856000in}{2.480924in}}%
\pgfpathlineto{\pgfqpoint{5.909333in}{2.403821in}}%
\pgfpathlineto{\pgfqpoint{5.962667in}{2.330479in}}%
\pgfpathlineto{\pgfqpoint{6.016000in}{2.260714in}}%
\pgfpathlineto{\pgfqpoint{6.069333in}{2.194351in}}%
\pgfpathlineto{\pgfqpoint{6.122667in}{2.131225in}}%
\pgfpathlineto{\pgfqpoint{6.181333in}{2.065336in}}%
\pgfpathlineto{\pgfqpoint{6.240000in}{2.002973in}}%
\pgfpathlineto{\pgfqpoint{6.298667in}{1.943948in}}%
\pgfpathlineto{\pgfqpoint{6.357333in}{1.888081in}}%
\pgfpathlineto{\pgfqpoint{6.416000in}{1.835204in}}%
\pgfpathlineto{\pgfqpoint{6.474667in}{1.785157in}}%
\pgfpathlineto{\pgfqpoint{6.538667in}{1.733609in}}%
\pgfpathlineto{\pgfqpoint{6.602667in}{1.685064in}}%
\pgfpathlineto{\pgfqpoint{6.666667in}{1.639345in}}%
\pgfpathlineto{\pgfqpoint{6.730667in}{1.596289in}}%
\pgfpathlineto{\pgfqpoint{6.794667in}{1.555740in}}%
\pgfpathlineto{\pgfqpoint{6.864000in}{1.514473in}}%
\pgfpathlineto{\pgfqpoint{6.933333in}{1.475803in}}%
\pgfpathlineto{\pgfqpoint{7.002667in}{1.439566in}}%
\pgfpathlineto{\pgfqpoint{7.077333in}{1.403088in}}%
\pgfpathlineto{\pgfqpoint{7.152000in}{1.369076in}}%
\pgfpathlineto{\pgfqpoint{7.232000in}{1.335182in}}%
\pgfpathlineto{\pgfqpoint{7.312000in}{1.303737in}}%
\pgfpathlineto{\pgfqpoint{7.397333in}{1.272697in}}%
\pgfpathlineto{\pgfqpoint{7.482667in}{1.244042in}}%
\pgfpathlineto{\pgfqpoint{7.573333in}{1.216007in}}%
\pgfpathlineto{\pgfqpoint{7.600000in}{1.208205in}}%
\pgfpathlineto{\pgfqpoint{7.600000in}{1.208205in}}%
\pgfusepath{stroke}%
\end{pgfscope}%
\begin{pgfscope}%
\pgfsetrectcap%
\pgfsetmiterjoin%
\pgfsetlinewidth{0.803000pt}%
\definecolor{currentstroke}{rgb}{0.000000,0.000000,0.000000}%
\pgfsetstrokecolor{currentstroke}%
\pgfsetdash{}{0pt}%
\pgfpathmoveto{\pgfqpoint{1.200000in}{0.900000in}}%
\pgfpathlineto{\pgfqpoint{1.200000in}{5.700000in}}%
\pgfusepath{stroke}%
\end{pgfscope}%
\begin{pgfscope}%
\pgfsetrectcap%
\pgfsetmiterjoin%
\pgfsetlinewidth{0.803000pt}%
\definecolor{currentstroke}{rgb}{0.000000,0.000000,0.000000}%
\pgfsetstrokecolor{currentstroke}%
\pgfsetdash{}{0pt}%
\pgfpathmoveto{\pgfqpoint{7.600000in}{0.900000in}}%
\pgfpathlineto{\pgfqpoint{7.600000in}{5.700000in}}%
\pgfusepath{stroke}%
\end{pgfscope}%
\begin{pgfscope}%
\pgfsetrectcap%
\pgfsetmiterjoin%
\pgfsetlinewidth{0.803000pt}%
\definecolor{currentstroke}{rgb}{0.000000,0.000000,0.000000}%
\pgfsetstrokecolor{currentstroke}%
\pgfsetdash{}{0pt}%
\pgfpathmoveto{\pgfqpoint{1.200000in}{0.900000in}}%
\pgfpathlineto{\pgfqpoint{7.600000in}{0.900000in}}%
\pgfusepath{stroke}%
\end{pgfscope}%
\begin{pgfscope}%
\pgfsetrectcap%
\pgfsetmiterjoin%
\pgfsetlinewidth{0.803000pt}%
\definecolor{currentstroke}{rgb}{0.000000,0.000000,0.000000}%
\pgfsetstrokecolor{currentstroke}%
\pgfsetdash{}{0pt}%
\pgfpathmoveto{\pgfqpoint{1.200000in}{5.700000in}}%
\pgfpathlineto{\pgfqpoint{7.600000in}{5.700000in}}%
\pgfusepath{stroke}%
\end{pgfscope}%
\begin{pgfscope}%
\pgfsetroundcap%
\pgfsetroundjoin%
\definecolor{currentfill}{rgb}{0.000000,0.000000,0.000000}%
\pgfsetfillcolor{currentfill}%
\pgfsetlinewidth{1.003750pt}%
\definecolor{currentstroke}{rgb}{0.000000,0.000000,0.000000}%
\pgfsetstrokecolor{currentstroke}%
\pgfsetdash{}{0pt}%
\pgfpathmoveto{\pgfqpoint{4.645345in}{5.086111in}}%
\pgfpathquadraticcurveto{\pgfqpoint{3.683333in}{5.086111in}}{\pgfqpoint{2.721322in}{5.086111in}}%
\pgfpathlineto{\pgfqpoint{2.721322in}{5.058333in}}%
\pgfpathquadraticcurveto{\pgfqpoint{2.637978in}{5.079167in}}{\pgfqpoint{2.554635in}{5.100000in}}%
\pgfpathquadraticcurveto{\pgfqpoint{2.637978in}{5.120833in}}{\pgfqpoint{2.721322in}{5.141667in}}%
\pgfpathlineto{\pgfqpoint{2.721322in}{5.113889in}}%
\pgfpathquadraticcurveto{\pgfqpoint{3.683333in}{5.113889in}}{\pgfqpoint{4.645345in}{5.113889in}}%
\pgfpathlineto{\pgfqpoint{4.645345in}{5.086111in}}%
\pgfpathclose%
\pgfusepath{stroke,fill}%
\end{pgfscope}%
\begin{pgfscope}%
\pgfsetroundcap%
\pgfsetroundjoin%
\definecolor{currentfill}{rgb}{0.000000,0.000000,0.000000}%
\pgfsetfillcolor{currentfill}%
\pgfsetlinewidth{1.003750pt}%
\definecolor{currentstroke}{rgb}{0.000000,0.000000,0.000000}%
\pgfsetstrokecolor{currentstroke}%
\pgfsetdash{}{0pt}%
\pgfpathmoveto{\pgfqpoint{2.821322in}{5.113889in}}%
\pgfpathquadraticcurveto{\pgfqpoint{3.783333in}{5.113889in}}{\pgfqpoint{4.745345in}{5.113889in}}%
\pgfpathlineto{\pgfqpoint{4.745345in}{5.141667in}}%
\pgfpathquadraticcurveto{\pgfqpoint{4.828688in}{5.120833in}}{\pgfqpoint{4.912031in}{5.100000in}}%
\pgfpathquadraticcurveto{\pgfqpoint{4.828688in}{5.079167in}}{\pgfqpoint{4.745345in}{5.058333in}}%
\pgfpathlineto{\pgfqpoint{4.745345in}{5.086111in}}%
\pgfpathquadraticcurveto{\pgfqpoint{3.783333in}{5.086111in}}{\pgfqpoint{2.821322in}{5.086111in}}%
\pgfpathlineto{\pgfqpoint{2.821322in}{5.113889in}}%
\pgfpathclose%
\pgfusepath{stroke,fill}%
\end{pgfscope}%
\begin{pgfscope}%
\pgfsetbuttcap%
\pgfsetmiterjoin%
\definecolor{currentfill}{rgb}{1.000000,1.000000,1.000000}%
\pgfsetfillcolor{currentfill}%
\pgfsetfillopacity{0.800000}%
\pgfsetlinewidth{1.003750pt}%
\definecolor{currentstroke}{rgb}{0.800000,0.800000,0.800000}%
\pgfsetstrokecolor{currentstroke}%
\pgfsetstrokeopacity{0.800000}%
\pgfsetdash{}{0pt}%
\pgfpathmoveto{\pgfqpoint{6.252452in}{4.260920in}}%
\pgfpathlineto{\pgfqpoint{7.405556in}{4.260920in}}%
\pgfpathquadraticcurveto{\pgfqpoint{7.461111in}{4.260920in}}{\pgfqpoint{7.461111in}{4.316476in}}%
\pgfpathlineto{\pgfqpoint{7.461111in}{5.505556in}}%
\pgfpathquadraticcurveto{\pgfqpoint{7.461111in}{5.561111in}}{\pgfqpoint{7.405556in}{5.561111in}}%
\pgfpathlineto{\pgfqpoint{6.252452in}{5.561111in}}%
\pgfpathquadraticcurveto{\pgfqpoint{6.196896in}{5.561111in}}{\pgfqpoint{6.196896in}{5.505556in}}%
\pgfpathlineto{\pgfqpoint{6.196896in}{4.316476in}}%
\pgfpathquadraticcurveto{\pgfqpoint{6.196896in}{4.260920in}}{\pgfqpoint{6.252452in}{4.260920in}}%
\pgfpathclose%
\pgfusepath{stroke,fill}%
\end{pgfscope}%
\begin{pgfscope}%
\definecolor{textcolor}{rgb}{0.000000,0.000000,0.000000}%
\pgfsetstrokecolor{textcolor}%
\pgfsetfillcolor{textcolor}%
\pgftext[x=6.349835in,y=5.233958in,left,base]{\color{textcolor}\sffamily\fontsize{20.000000}{24.000000}\selectfont \(\displaystyle t_0=/\si{ns}\)}%
\end{pgfscope}%
\begin{pgfscope}%
\pgfsetrectcap%
\pgfsetroundjoin%
\pgfsetlinewidth{2.007500pt}%
\definecolor{currentstroke}{rgb}{0.000000,0.000000,1.000000}%
\pgfsetstrokecolor{currentstroke}%
\pgfsetdash{}{0pt}%
\pgfpathmoveto{\pgfqpoint{6.308007in}{4.920239in}}%
\pgfpathlineto{\pgfqpoint{6.863563in}{4.920239in}}%
\pgfusepath{stroke}%
\end{pgfscope}%
\begin{pgfscope}%
\definecolor{textcolor}{rgb}{0.000000,0.000000,0.000000}%
\pgfsetstrokecolor{textcolor}%
\pgfsetfillcolor{textcolor}%
\pgftext[x=7.085785in,y=4.823017in,left,base]{\color{textcolor}\sffamily\fontsize{20.000000}{24.000000}\selectfont \(\displaystyle 0\)}%
\end{pgfscope}%
\begin{pgfscope}%
\pgfsetrectcap%
\pgfsetroundjoin%
\pgfsetlinewidth{2.007500pt}%
\definecolor{currentstroke}{rgb}{1.000000,0.000000,0.000000}%
\pgfsetstrokecolor{currentstroke}%
\pgfsetdash{}{0pt}%
\pgfpathmoveto{\pgfqpoint{6.308007in}{4.525283in}}%
\pgfpathlineto{\pgfqpoint{6.863563in}{4.525283in}}%
\pgfusepath{stroke}%
\end{pgfscope}%
\begin{pgfscope}%
\definecolor{textcolor}{rgb}{0.000000,0.000000,0.000000}%
\pgfsetstrokecolor{textcolor}%
\pgfsetfillcolor{textcolor}%
\pgftext[x=7.085785in,y=4.428061in,left,base]{\color{textcolor}\sffamily\fontsize{20.000000}{24.000000}\selectfont \(\displaystyle 40\)}%
\end{pgfscope}%
\end{pgfpicture}%
\makeatother%
\endgroup%
}
    \caption{Time translation between time profiles}
\end{figure}
\end{frame}

\begin{frame}
\frametitle{$\mu$ estimation}
For Fourier deconvolution, 
\begin{align*}
    \hat{\mu} &= \sum_i \hat{q}_i
\end{align*}
For FBMP,
\begin{align*}
    \left(\hat{t}_0, \hat{\mu}\right) &= \arg\underset{t_0,\mu}{\max} p(\vec{w} | t_0, \mu) \\
    p(\vec{w}|t_0, \mu) &= \sum_{\vec{z}'\in\mathcal{Z}'}p(\vec{w}|\vec{z}',t_0,\mu)p(\vec{z}'|t_0,\mu) \\
    &= \sum_{\vec{z}'\in\mathcal{Z}'}p(\vec{w}|\vec{z}')p(\vec{z}'|t_0,\mu)
\end{align*}
\end{frame}

\begin{frame}
\frametitle{FBMP's performance of evaluation criteria}
For dataset: $(\mu, \tau_l, \sigma_l)/\si{ns}=(4, 20, 5)$: 
\begin{figure}
    \centering
    \resizebox{\textwidth}{!}{%% Creator: Matplotlib, PGF backend
%%
%% To include the figure in your LaTeX document, write
%%   \input{<filename>.pgf}
%%
%% Make sure the required packages are loaded in your preamble
%%   \usepackage{pgf}
%%
%% and, on pdftex
%%   \usepackage[utf8]{inputenc}\DeclareUnicodeCharacter{2212}{-}
%%
%% or, on luatex and xetex
%%   \usepackage{unicode-math}
%%
%% Figures using additional raster images can only be included by \input if
%% they are in the same directory as the main LaTeX file. For loading figures
%% from other directories you can use the `import` package
%%   \usepackage{import}
%%
%% and then include the figures with
%%   \import{<path to file>}{<filename>.pgf}
%%
%% Matplotlib used the following preamble
%%   \usepackage[detect-all,locale=DE]{siunitx}
%%
\begingroup%
\makeatletter%
\begin{pgfpicture}%
\pgfpathrectangle{\pgfpointorigin}{\pgfqpoint{12.000000in}{6.000000in}}%
\pgfusepath{use as bounding box, clip}%
\begin{pgfscope}%
\pgfsetbuttcap%
\pgfsetmiterjoin%
\definecolor{currentfill}{rgb}{1.000000,1.000000,1.000000}%
\pgfsetfillcolor{currentfill}%
\pgfsetlinewidth{0.000000pt}%
\definecolor{currentstroke}{rgb}{1.000000,1.000000,1.000000}%
\pgfsetstrokecolor{currentstroke}%
\pgfsetdash{}{0pt}%
\pgfpathmoveto{\pgfqpoint{0.000000in}{0.000000in}}%
\pgfpathlineto{\pgfqpoint{12.000000in}{0.000000in}}%
\pgfpathlineto{\pgfqpoint{12.000000in}{6.000000in}}%
\pgfpathlineto{\pgfqpoint{0.000000in}{6.000000in}}%
\pgfpathclose%
\pgfusepath{fill}%
\end{pgfscope}%
\begin{pgfscope}%
\pgfsetbuttcap%
\pgfsetmiterjoin%
\definecolor{currentfill}{rgb}{1.000000,1.000000,1.000000}%
\pgfsetfillcolor{currentfill}%
\pgfsetlinewidth{0.000000pt}%
\definecolor{currentstroke}{rgb}{0.000000,0.000000,0.000000}%
\pgfsetstrokecolor{currentstroke}%
\pgfsetstrokeopacity{0.000000}%
\pgfsetdash{}{0pt}%
\pgfpathmoveto{\pgfqpoint{1.800000in}{0.900000in}}%
\pgfpathlineto{\pgfqpoint{10.200000in}{0.900000in}}%
\pgfpathlineto{\pgfqpoint{10.200000in}{5.700000in}}%
\pgfpathlineto{\pgfqpoint{1.800000in}{5.700000in}}%
\pgfpathclose%
\pgfusepath{fill}%
\end{pgfscope}%
\begin{pgfscope}%
\pgfpathrectangle{\pgfqpoint{1.800000in}{0.900000in}}{\pgfqpoint{8.400000in}{4.800000in}}%
\pgfusepath{clip}%
\pgfsetrectcap%
\pgfsetroundjoin%
\pgfsetlinewidth{0.803000pt}%
\definecolor{currentstroke}{rgb}{0.690196,0.690196,0.690196}%
\pgfsetstrokecolor{currentstroke}%
\pgfsetdash{}{0pt}%
\pgfpathmoveto{\pgfqpoint{2.181818in}{0.900000in}}%
\pgfpathlineto{\pgfqpoint{2.181818in}{5.700000in}}%
\pgfusepath{stroke}%
\end{pgfscope}%
\begin{pgfscope}%
\pgfsetbuttcap%
\pgfsetroundjoin%
\definecolor{currentfill}{rgb}{0.000000,0.000000,0.000000}%
\pgfsetfillcolor{currentfill}%
\pgfsetlinewidth{0.803000pt}%
\definecolor{currentstroke}{rgb}{0.000000,0.000000,0.000000}%
\pgfsetstrokecolor{currentstroke}%
\pgfsetdash{}{0pt}%
\pgfsys@defobject{currentmarker}{\pgfqpoint{0.000000in}{-0.048611in}}{\pgfqpoint{0.000000in}{0.000000in}}{%
\pgfpathmoveto{\pgfqpoint{0.000000in}{0.000000in}}%
\pgfpathlineto{\pgfqpoint{0.000000in}{-0.048611in}}%
\pgfusepath{stroke,fill}%
}%
\begin{pgfscope}%
\pgfsys@transformshift{2.181818in}{0.900000in}%
\pgfsys@useobject{currentmarker}{}%
\end{pgfscope}%
\end{pgfscope}%
\begin{pgfscope}%
\definecolor{textcolor}{rgb}{0.000000,0.000000,0.000000}%
\pgfsetstrokecolor{textcolor}%
\pgfsetfillcolor{textcolor}%
\pgftext[x=2.181818in,y=0.802778in,,top]{\color{textcolor}\sffamily\fontsize{20.000000}{24.000000}\selectfont \(\displaystyle {0}\)}%
\end{pgfscope}%
\begin{pgfscope}%
\pgfpathrectangle{\pgfqpoint{1.800000in}{0.900000in}}{\pgfqpoint{8.400000in}{4.800000in}}%
\pgfusepath{clip}%
\pgfsetrectcap%
\pgfsetroundjoin%
\pgfsetlinewidth{0.803000pt}%
\definecolor{currentstroke}{rgb}{0.690196,0.690196,0.690196}%
\pgfsetstrokecolor{currentstroke}%
\pgfsetdash{}{0pt}%
\pgfpathmoveto{\pgfqpoint{3.740260in}{0.900000in}}%
\pgfpathlineto{\pgfqpoint{3.740260in}{5.700000in}}%
\pgfusepath{stroke}%
\end{pgfscope}%
\begin{pgfscope}%
\pgfsetbuttcap%
\pgfsetroundjoin%
\definecolor{currentfill}{rgb}{0.000000,0.000000,0.000000}%
\pgfsetfillcolor{currentfill}%
\pgfsetlinewidth{0.803000pt}%
\definecolor{currentstroke}{rgb}{0.000000,0.000000,0.000000}%
\pgfsetstrokecolor{currentstroke}%
\pgfsetdash{}{0pt}%
\pgfsys@defobject{currentmarker}{\pgfqpoint{0.000000in}{-0.048611in}}{\pgfqpoint{0.000000in}{0.000000in}}{%
\pgfpathmoveto{\pgfqpoint{0.000000in}{0.000000in}}%
\pgfpathlineto{\pgfqpoint{0.000000in}{-0.048611in}}%
\pgfusepath{stroke,fill}%
}%
\begin{pgfscope}%
\pgfsys@transformshift{3.740260in}{0.900000in}%
\pgfsys@useobject{currentmarker}{}%
\end{pgfscope}%
\end{pgfscope}%
\begin{pgfscope}%
\definecolor{textcolor}{rgb}{0.000000,0.000000,0.000000}%
\pgfsetstrokecolor{textcolor}%
\pgfsetfillcolor{textcolor}%
\pgftext[x=3.740260in,y=0.802778in,,top]{\color{textcolor}\sffamily\fontsize{20.000000}{24.000000}\selectfont \(\displaystyle {2}\)}%
\end{pgfscope}%
\begin{pgfscope}%
\pgfpathrectangle{\pgfqpoint{1.800000in}{0.900000in}}{\pgfqpoint{8.400000in}{4.800000in}}%
\pgfusepath{clip}%
\pgfsetrectcap%
\pgfsetroundjoin%
\pgfsetlinewidth{0.803000pt}%
\definecolor{currentstroke}{rgb}{0.690196,0.690196,0.690196}%
\pgfsetstrokecolor{currentstroke}%
\pgfsetdash{}{0pt}%
\pgfpathmoveto{\pgfqpoint{5.298701in}{0.900000in}}%
\pgfpathlineto{\pgfqpoint{5.298701in}{5.700000in}}%
\pgfusepath{stroke}%
\end{pgfscope}%
\begin{pgfscope}%
\pgfsetbuttcap%
\pgfsetroundjoin%
\definecolor{currentfill}{rgb}{0.000000,0.000000,0.000000}%
\pgfsetfillcolor{currentfill}%
\pgfsetlinewidth{0.803000pt}%
\definecolor{currentstroke}{rgb}{0.000000,0.000000,0.000000}%
\pgfsetstrokecolor{currentstroke}%
\pgfsetdash{}{0pt}%
\pgfsys@defobject{currentmarker}{\pgfqpoint{0.000000in}{-0.048611in}}{\pgfqpoint{0.000000in}{0.000000in}}{%
\pgfpathmoveto{\pgfqpoint{0.000000in}{0.000000in}}%
\pgfpathlineto{\pgfqpoint{0.000000in}{-0.048611in}}%
\pgfusepath{stroke,fill}%
}%
\begin{pgfscope}%
\pgfsys@transformshift{5.298701in}{0.900000in}%
\pgfsys@useobject{currentmarker}{}%
\end{pgfscope}%
\end{pgfscope}%
\begin{pgfscope}%
\definecolor{textcolor}{rgb}{0.000000,0.000000,0.000000}%
\pgfsetstrokecolor{textcolor}%
\pgfsetfillcolor{textcolor}%
\pgftext[x=5.298701in,y=0.802778in,,top]{\color{textcolor}\sffamily\fontsize{20.000000}{24.000000}\selectfont \(\displaystyle {4}\)}%
\end{pgfscope}%
\begin{pgfscope}%
\pgfpathrectangle{\pgfqpoint{1.800000in}{0.900000in}}{\pgfqpoint{8.400000in}{4.800000in}}%
\pgfusepath{clip}%
\pgfsetrectcap%
\pgfsetroundjoin%
\pgfsetlinewidth{0.803000pt}%
\definecolor{currentstroke}{rgb}{0.690196,0.690196,0.690196}%
\pgfsetstrokecolor{currentstroke}%
\pgfsetdash{}{0pt}%
\pgfpathmoveto{\pgfqpoint{6.857143in}{0.900000in}}%
\pgfpathlineto{\pgfqpoint{6.857143in}{5.700000in}}%
\pgfusepath{stroke}%
\end{pgfscope}%
\begin{pgfscope}%
\pgfsetbuttcap%
\pgfsetroundjoin%
\definecolor{currentfill}{rgb}{0.000000,0.000000,0.000000}%
\pgfsetfillcolor{currentfill}%
\pgfsetlinewidth{0.803000pt}%
\definecolor{currentstroke}{rgb}{0.000000,0.000000,0.000000}%
\pgfsetstrokecolor{currentstroke}%
\pgfsetdash{}{0pt}%
\pgfsys@defobject{currentmarker}{\pgfqpoint{0.000000in}{-0.048611in}}{\pgfqpoint{0.000000in}{0.000000in}}{%
\pgfpathmoveto{\pgfqpoint{0.000000in}{0.000000in}}%
\pgfpathlineto{\pgfqpoint{0.000000in}{-0.048611in}}%
\pgfusepath{stroke,fill}%
}%
\begin{pgfscope}%
\pgfsys@transformshift{6.857143in}{0.900000in}%
\pgfsys@useobject{currentmarker}{}%
\end{pgfscope}%
\end{pgfscope}%
\begin{pgfscope}%
\definecolor{textcolor}{rgb}{0.000000,0.000000,0.000000}%
\pgfsetstrokecolor{textcolor}%
\pgfsetfillcolor{textcolor}%
\pgftext[x=6.857143in,y=0.802778in,,top]{\color{textcolor}\sffamily\fontsize{20.000000}{24.000000}\selectfont \(\displaystyle {6}\)}%
\end{pgfscope}%
\begin{pgfscope}%
\pgfpathrectangle{\pgfqpoint{1.800000in}{0.900000in}}{\pgfqpoint{8.400000in}{4.800000in}}%
\pgfusepath{clip}%
\pgfsetrectcap%
\pgfsetroundjoin%
\pgfsetlinewidth{0.803000pt}%
\definecolor{currentstroke}{rgb}{0.690196,0.690196,0.690196}%
\pgfsetstrokecolor{currentstroke}%
\pgfsetdash{}{0pt}%
\pgfpathmoveto{\pgfqpoint{8.415584in}{0.900000in}}%
\pgfpathlineto{\pgfqpoint{8.415584in}{5.700000in}}%
\pgfusepath{stroke}%
\end{pgfscope}%
\begin{pgfscope}%
\pgfsetbuttcap%
\pgfsetroundjoin%
\definecolor{currentfill}{rgb}{0.000000,0.000000,0.000000}%
\pgfsetfillcolor{currentfill}%
\pgfsetlinewidth{0.803000pt}%
\definecolor{currentstroke}{rgb}{0.000000,0.000000,0.000000}%
\pgfsetstrokecolor{currentstroke}%
\pgfsetdash{}{0pt}%
\pgfsys@defobject{currentmarker}{\pgfqpoint{0.000000in}{-0.048611in}}{\pgfqpoint{0.000000in}{0.000000in}}{%
\pgfpathmoveto{\pgfqpoint{0.000000in}{0.000000in}}%
\pgfpathlineto{\pgfqpoint{0.000000in}{-0.048611in}}%
\pgfusepath{stroke,fill}%
}%
\begin{pgfscope}%
\pgfsys@transformshift{8.415584in}{0.900000in}%
\pgfsys@useobject{currentmarker}{}%
\end{pgfscope}%
\end{pgfscope}%
\begin{pgfscope}%
\definecolor{textcolor}{rgb}{0.000000,0.000000,0.000000}%
\pgfsetstrokecolor{textcolor}%
\pgfsetfillcolor{textcolor}%
\pgftext[x=8.415584in,y=0.802778in,,top]{\color{textcolor}\sffamily\fontsize{20.000000}{24.000000}\selectfont \(\displaystyle {8}\)}%
\end{pgfscope}%
\begin{pgfscope}%
\pgfpathrectangle{\pgfqpoint{1.800000in}{0.900000in}}{\pgfqpoint{8.400000in}{4.800000in}}%
\pgfusepath{clip}%
\pgfsetrectcap%
\pgfsetroundjoin%
\pgfsetlinewidth{0.803000pt}%
\definecolor{currentstroke}{rgb}{0.690196,0.690196,0.690196}%
\pgfsetstrokecolor{currentstroke}%
\pgfsetdash{}{0pt}%
\pgfpathmoveto{\pgfqpoint{9.974026in}{0.900000in}}%
\pgfpathlineto{\pgfqpoint{9.974026in}{5.700000in}}%
\pgfusepath{stroke}%
\end{pgfscope}%
\begin{pgfscope}%
\pgfsetbuttcap%
\pgfsetroundjoin%
\definecolor{currentfill}{rgb}{0.000000,0.000000,0.000000}%
\pgfsetfillcolor{currentfill}%
\pgfsetlinewidth{0.803000pt}%
\definecolor{currentstroke}{rgb}{0.000000,0.000000,0.000000}%
\pgfsetstrokecolor{currentstroke}%
\pgfsetdash{}{0pt}%
\pgfsys@defobject{currentmarker}{\pgfqpoint{0.000000in}{-0.048611in}}{\pgfqpoint{0.000000in}{0.000000in}}{%
\pgfpathmoveto{\pgfqpoint{0.000000in}{0.000000in}}%
\pgfpathlineto{\pgfqpoint{0.000000in}{-0.048611in}}%
\pgfusepath{stroke,fill}%
}%
\begin{pgfscope}%
\pgfsys@transformshift{9.974026in}{0.900000in}%
\pgfsys@useobject{currentmarker}{}%
\end{pgfscope}%
\end{pgfscope}%
\begin{pgfscope}%
\definecolor{textcolor}{rgb}{0.000000,0.000000,0.000000}%
\pgfsetstrokecolor{textcolor}%
\pgfsetfillcolor{textcolor}%
\pgftext[x=9.974026in,y=0.802778in,,top]{\color{textcolor}\sffamily\fontsize{20.000000}{24.000000}\selectfont \(\displaystyle {10}\)}%
\end{pgfscope}%
\begin{pgfscope}%
\definecolor{textcolor}{rgb}{0.000000,0.000000,0.000000}%
\pgfsetstrokecolor{textcolor}%
\pgfsetfillcolor{textcolor}%
\pgftext[x=6.000000in,y=0.491155in,,top]{\color{textcolor}\sffamily\fontsize{20.000000}{24.000000}\selectfont \(\displaystyle D_w/\si{ns}\)}%
\end{pgfscope}%
\begin{pgfscope}%
\pgfpathrectangle{\pgfqpoint{1.800000in}{0.900000in}}{\pgfqpoint{8.400000in}{4.800000in}}%
\pgfusepath{clip}%
\pgfsetrectcap%
\pgfsetroundjoin%
\pgfsetlinewidth{0.803000pt}%
\definecolor{currentstroke}{rgb}{0.690196,0.690196,0.690196}%
\pgfsetstrokecolor{currentstroke}%
\pgfsetdash{}{0pt}%
\pgfpathmoveto{\pgfqpoint{1.800000in}{1.698230in}}%
\pgfpathlineto{\pgfqpoint{10.200000in}{1.698230in}}%
\pgfusepath{stroke}%
\end{pgfscope}%
\begin{pgfscope}%
\pgfsetbuttcap%
\pgfsetroundjoin%
\definecolor{currentfill}{rgb}{0.000000,0.000000,0.000000}%
\pgfsetfillcolor{currentfill}%
\pgfsetlinewidth{0.803000pt}%
\definecolor{currentstroke}{rgb}{0.000000,0.000000,0.000000}%
\pgfsetstrokecolor{currentstroke}%
\pgfsetdash{}{0pt}%
\pgfsys@defobject{currentmarker}{\pgfqpoint{-0.048611in}{0.000000in}}{\pgfqpoint{-0.000000in}{0.000000in}}{%
\pgfpathmoveto{\pgfqpoint{-0.000000in}{0.000000in}}%
\pgfpathlineto{\pgfqpoint{-0.048611in}{0.000000in}}%
\pgfusepath{stroke,fill}%
}%
\begin{pgfscope}%
\pgfsys@transformshift{1.800000in}{1.698230in}%
\pgfsys@useobject{currentmarker}{}%
\end{pgfscope}%
\end{pgfscope}%
\begin{pgfscope}%
\definecolor{textcolor}{rgb}{0.000000,0.000000,0.000000}%
\pgfsetstrokecolor{textcolor}%
\pgfsetfillcolor{textcolor}%
\pgftext[x=1.306456in, y=1.598211in, left, base]{\color{textcolor}\sffamily\fontsize{20.000000}{24.000000}\selectfont \(\displaystyle {500}\)}%
\end{pgfscope}%
\begin{pgfscope}%
\pgfpathrectangle{\pgfqpoint{1.800000in}{0.900000in}}{\pgfqpoint{8.400000in}{4.800000in}}%
\pgfusepath{clip}%
\pgfsetrectcap%
\pgfsetroundjoin%
\pgfsetlinewidth{0.803000pt}%
\definecolor{currentstroke}{rgb}{0.690196,0.690196,0.690196}%
\pgfsetstrokecolor{currentstroke}%
\pgfsetdash{}{0pt}%
\pgfpathmoveto{\pgfqpoint{1.800000in}{2.496620in}}%
\pgfpathlineto{\pgfqpoint{10.200000in}{2.496620in}}%
\pgfusepath{stroke}%
\end{pgfscope}%
\begin{pgfscope}%
\pgfsetbuttcap%
\pgfsetroundjoin%
\definecolor{currentfill}{rgb}{0.000000,0.000000,0.000000}%
\pgfsetfillcolor{currentfill}%
\pgfsetlinewidth{0.803000pt}%
\definecolor{currentstroke}{rgb}{0.000000,0.000000,0.000000}%
\pgfsetstrokecolor{currentstroke}%
\pgfsetdash{}{0pt}%
\pgfsys@defobject{currentmarker}{\pgfqpoint{-0.048611in}{0.000000in}}{\pgfqpoint{-0.000000in}{0.000000in}}{%
\pgfpathmoveto{\pgfqpoint{-0.000000in}{0.000000in}}%
\pgfpathlineto{\pgfqpoint{-0.048611in}{0.000000in}}%
\pgfusepath{stroke,fill}%
}%
\begin{pgfscope}%
\pgfsys@transformshift{1.800000in}{2.496620in}%
\pgfsys@useobject{currentmarker}{}%
\end{pgfscope}%
\end{pgfscope}%
\begin{pgfscope}%
\definecolor{textcolor}{rgb}{0.000000,0.000000,0.000000}%
\pgfsetstrokecolor{textcolor}%
\pgfsetfillcolor{textcolor}%
\pgftext[x=1.174348in, y=2.396601in, left, base]{\color{textcolor}\sffamily\fontsize{20.000000}{24.000000}\selectfont \(\displaystyle {1000}\)}%
\end{pgfscope}%
\begin{pgfscope}%
\pgfpathrectangle{\pgfqpoint{1.800000in}{0.900000in}}{\pgfqpoint{8.400000in}{4.800000in}}%
\pgfusepath{clip}%
\pgfsetrectcap%
\pgfsetroundjoin%
\pgfsetlinewidth{0.803000pt}%
\definecolor{currentstroke}{rgb}{0.690196,0.690196,0.690196}%
\pgfsetstrokecolor{currentstroke}%
\pgfsetdash{}{0pt}%
\pgfpathmoveto{\pgfqpoint{1.800000in}{3.295010in}}%
\pgfpathlineto{\pgfqpoint{10.200000in}{3.295010in}}%
\pgfusepath{stroke}%
\end{pgfscope}%
\begin{pgfscope}%
\pgfsetbuttcap%
\pgfsetroundjoin%
\definecolor{currentfill}{rgb}{0.000000,0.000000,0.000000}%
\pgfsetfillcolor{currentfill}%
\pgfsetlinewidth{0.803000pt}%
\definecolor{currentstroke}{rgb}{0.000000,0.000000,0.000000}%
\pgfsetstrokecolor{currentstroke}%
\pgfsetdash{}{0pt}%
\pgfsys@defobject{currentmarker}{\pgfqpoint{-0.048611in}{0.000000in}}{\pgfqpoint{-0.000000in}{0.000000in}}{%
\pgfpathmoveto{\pgfqpoint{-0.000000in}{0.000000in}}%
\pgfpathlineto{\pgfqpoint{-0.048611in}{0.000000in}}%
\pgfusepath{stroke,fill}%
}%
\begin{pgfscope}%
\pgfsys@transformshift{1.800000in}{3.295010in}%
\pgfsys@useobject{currentmarker}{}%
\end{pgfscope}%
\end{pgfscope}%
\begin{pgfscope}%
\definecolor{textcolor}{rgb}{0.000000,0.000000,0.000000}%
\pgfsetstrokecolor{textcolor}%
\pgfsetfillcolor{textcolor}%
\pgftext[x=1.174348in, y=3.194991in, left, base]{\color{textcolor}\sffamily\fontsize{20.000000}{24.000000}\selectfont \(\displaystyle {1500}\)}%
\end{pgfscope}%
\begin{pgfscope}%
\pgfpathrectangle{\pgfqpoint{1.800000in}{0.900000in}}{\pgfqpoint{8.400000in}{4.800000in}}%
\pgfusepath{clip}%
\pgfsetrectcap%
\pgfsetroundjoin%
\pgfsetlinewidth{0.803000pt}%
\definecolor{currentstroke}{rgb}{0.690196,0.690196,0.690196}%
\pgfsetstrokecolor{currentstroke}%
\pgfsetdash{}{0pt}%
\pgfpathmoveto{\pgfqpoint{1.800000in}{4.093400in}}%
\pgfpathlineto{\pgfqpoint{10.200000in}{4.093400in}}%
\pgfusepath{stroke}%
\end{pgfscope}%
\begin{pgfscope}%
\pgfsetbuttcap%
\pgfsetroundjoin%
\definecolor{currentfill}{rgb}{0.000000,0.000000,0.000000}%
\pgfsetfillcolor{currentfill}%
\pgfsetlinewidth{0.803000pt}%
\definecolor{currentstroke}{rgb}{0.000000,0.000000,0.000000}%
\pgfsetstrokecolor{currentstroke}%
\pgfsetdash{}{0pt}%
\pgfsys@defobject{currentmarker}{\pgfqpoint{-0.048611in}{0.000000in}}{\pgfqpoint{-0.000000in}{0.000000in}}{%
\pgfpathmoveto{\pgfqpoint{-0.000000in}{0.000000in}}%
\pgfpathlineto{\pgfqpoint{-0.048611in}{0.000000in}}%
\pgfusepath{stroke,fill}%
}%
\begin{pgfscope}%
\pgfsys@transformshift{1.800000in}{4.093400in}%
\pgfsys@useobject{currentmarker}{}%
\end{pgfscope}%
\end{pgfscope}%
\begin{pgfscope}%
\definecolor{textcolor}{rgb}{0.000000,0.000000,0.000000}%
\pgfsetstrokecolor{textcolor}%
\pgfsetfillcolor{textcolor}%
\pgftext[x=1.174348in, y=3.993381in, left, base]{\color{textcolor}\sffamily\fontsize{20.000000}{24.000000}\selectfont \(\displaystyle {2000}\)}%
\end{pgfscope}%
\begin{pgfscope}%
\pgfpathrectangle{\pgfqpoint{1.800000in}{0.900000in}}{\pgfqpoint{8.400000in}{4.800000in}}%
\pgfusepath{clip}%
\pgfsetrectcap%
\pgfsetroundjoin%
\pgfsetlinewidth{0.803000pt}%
\definecolor{currentstroke}{rgb}{0.690196,0.690196,0.690196}%
\pgfsetstrokecolor{currentstroke}%
\pgfsetdash{}{0pt}%
\pgfpathmoveto{\pgfqpoint{1.800000in}{4.891790in}}%
\pgfpathlineto{\pgfqpoint{10.200000in}{4.891790in}}%
\pgfusepath{stroke}%
\end{pgfscope}%
\begin{pgfscope}%
\pgfsetbuttcap%
\pgfsetroundjoin%
\definecolor{currentfill}{rgb}{0.000000,0.000000,0.000000}%
\pgfsetfillcolor{currentfill}%
\pgfsetlinewidth{0.803000pt}%
\definecolor{currentstroke}{rgb}{0.000000,0.000000,0.000000}%
\pgfsetstrokecolor{currentstroke}%
\pgfsetdash{}{0pt}%
\pgfsys@defobject{currentmarker}{\pgfqpoint{-0.048611in}{0.000000in}}{\pgfqpoint{-0.000000in}{0.000000in}}{%
\pgfpathmoveto{\pgfqpoint{-0.000000in}{0.000000in}}%
\pgfpathlineto{\pgfqpoint{-0.048611in}{0.000000in}}%
\pgfusepath{stroke,fill}%
}%
\begin{pgfscope}%
\pgfsys@transformshift{1.800000in}{4.891790in}%
\pgfsys@useobject{currentmarker}{}%
\end{pgfscope}%
\end{pgfscope}%
\begin{pgfscope}%
\definecolor{textcolor}{rgb}{0.000000,0.000000,0.000000}%
\pgfsetstrokecolor{textcolor}%
\pgfsetfillcolor{textcolor}%
\pgftext[x=1.174348in, y=4.791771in, left, base]{\color{textcolor}\sffamily\fontsize{20.000000}{24.000000}\selectfont \(\displaystyle {2500}\)}%
\end{pgfscope}%
\begin{pgfscope}%
\pgfpathrectangle{\pgfqpoint{1.800000in}{0.900000in}}{\pgfqpoint{8.400000in}{4.800000in}}%
\pgfusepath{clip}%
\pgfsetrectcap%
\pgfsetroundjoin%
\pgfsetlinewidth{0.803000pt}%
\definecolor{currentstroke}{rgb}{0.690196,0.690196,0.690196}%
\pgfsetstrokecolor{currentstroke}%
\pgfsetdash{}{0pt}%
\pgfpathmoveto{\pgfqpoint{1.800000in}{5.690180in}}%
\pgfpathlineto{\pgfqpoint{10.200000in}{5.690180in}}%
\pgfusepath{stroke}%
\end{pgfscope}%
\begin{pgfscope}%
\pgfsetbuttcap%
\pgfsetroundjoin%
\definecolor{currentfill}{rgb}{0.000000,0.000000,0.000000}%
\pgfsetfillcolor{currentfill}%
\pgfsetlinewidth{0.803000pt}%
\definecolor{currentstroke}{rgb}{0.000000,0.000000,0.000000}%
\pgfsetstrokecolor{currentstroke}%
\pgfsetdash{}{0pt}%
\pgfsys@defobject{currentmarker}{\pgfqpoint{-0.048611in}{0.000000in}}{\pgfqpoint{-0.000000in}{0.000000in}}{%
\pgfpathmoveto{\pgfqpoint{-0.000000in}{0.000000in}}%
\pgfpathlineto{\pgfqpoint{-0.048611in}{0.000000in}}%
\pgfusepath{stroke,fill}%
}%
\begin{pgfscope}%
\pgfsys@transformshift{1.800000in}{5.690180in}%
\pgfsys@useobject{currentmarker}{}%
\end{pgfscope}%
\end{pgfscope}%
\begin{pgfscope}%
\definecolor{textcolor}{rgb}{0.000000,0.000000,0.000000}%
\pgfsetstrokecolor{textcolor}%
\pgfsetfillcolor{textcolor}%
\pgftext[x=1.174348in, y=5.590161in, left, base]{\color{textcolor}\sffamily\fontsize{20.000000}{24.000000}\selectfont \(\displaystyle {3000}\)}%
\end{pgfscope}%
\begin{pgfscope}%
\definecolor{textcolor}{rgb}{0.000000,0.000000,0.000000}%
\pgfsetstrokecolor{textcolor}%
\pgfsetfillcolor{textcolor}%
\pgftext[x=1.118793in,y=3.300000in,,bottom,rotate=90.000000]{\color{textcolor}\sffamily\fontsize{20.000000}{24.000000}\selectfont \(\displaystyle \mathrm{Count}\)}%
\end{pgfscope}%
\begin{pgfscope}%
\pgfpathrectangle{\pgfqpoint{1.800000in}{0.900000in}}{\pgfqpoint{8.400000in}{4.800000in}}%
\pgfusepath{clip}%
\pgfsetbuttcap%
\pgfsetmiterjoin%
\pgfsetlinewidth{1.003750pt}%
\definecolor{currentstroke}{rgb}{0.000000,0.000000,1.000000}%
\pgfsetstrokecolor{currentstroke}%
\pgfsetdash{}{0pt}%
\pgfpathmoveto{\pgfqpoint{2.181818in}{0.899840in}}%
\pgfpathlineto{\pgfqpoint{2.181818in}{2.097425in}}%
\pgfpathlineto{\pgfqpoint{2.337662in}{2.097425in}}%
\pgfpathlineto{\pgfqpoint{2.337662in}{4.387207in}}%
\pgfpathlineto{\pgfqpoint{2.493506in}{4.387207in}}%
\pgfpathlineto{\pgfqpoint{2.493506in}{5.471421in}}%
\pgfpathlineto{\pgfqpoint{2.649351in}{5.471421in}}%
\pgfpathlineto{\pgfqpoint{2.649351in}{4.586805in}}%
\pgfpathlineto{\pgfqpoint{2.805195in}{4.586805in}}%
\pgfpathlineto{\pgfqpoint{2.805195in}{2.849508in}}%
\pgfpathlineto{\pgfqpoint{2.961039in}{2.849508in}}%
\pgfpathlineto{\pgfqpoint{2.961039in}{1.578472in}}%
\pgfpathlineto{\pgfqpoint{3.116883in}{1.578472in}}%
\pgfpathlineto{\pgfqpoint{3.116883in}{1.117002in}}%
\pgfpathlineto{\pgfqpoint{3.272727in}{1.117002in}}%
\pgfpathlineto{\pgfqpoint{3.272727in}{0.976486in}}%
\pgfpathlineto{\pgfqpoint{3.428571in}{0.976486in}}%
\pgfpathlineto{\pgfqpoint{3.428571in}{0.934969in}}%
\pgfpathlineto{\pgfqpoint{3.584416in}{0.934969in}}%
\pgfpathlineto{\pgfqpoint{3.584416in}{0.914211in}}%
\pgfpathlineto{\pgfqpoint{3.740260in}{0.914211in}}%
\pgfpathlineto{\pgfqpoint{3.740260in}{0.906227in}}%
\pgfpathlineto{\pgfqpoint{3.896104in}{0.906227in}}%
\pgfpathlineto{\pgfqpoint{3.896104in}{0.903034in}}%
\pgfpathlineto{\pgfqpoint{4.051948in}{0.903034in}}%
\pgfpathlineto{\pgfqpoint{4.051948in}{0.903034in}}%
\pgfpathlineto{\pgfqpoint{4.207792in}{0.903034in}}%
\pgfpathlineto{\pgfqpoint{4.207792in}{0.899840in}}%
\pgfpathlineto{\pgfqpoint{4.363636in}{0.899840in}}%
\pgfpathlineto{\pgfqpoint{4.363636in}{0.903034in}}%
\pgfpathlineto{\pgfqpoint{4.519481in}{0.903034in}}%
\pgfpathlineto{\pgfqpoint{4.519481in}{0.901437in}}%
\pgfpathlineto{\pgfqpoint{4.675325in}{0.901437in}}%
\pgfpathlineto{\pgfqpoint{4.675325in}{0.901437in}}%
\pgfpathlineto{\pgfqpoint{4.831169in}{0.901437in}}%
\pgfpathlineto{\pgfqpoint{4.831169in}{0.899840in}}%
\pgfpathlineto{\pgfqpoint{4.987013in}{0.899840in}}%
\pgfpathlineto{\pgfqpoint{4.987013in}{0.899840in}}%
\pgfpathlineto{\pgfqpoint{5.142857in}{0.899840in}}%
\pgfpathlineto{\pgfqpoint{5.142857in}{0.899840in}}%
\pgfpathlineto{\pgfqpoint{5.298701in}{0.899840in}}%
\pgfpathlineto{\pgfqpoint{5.298701in}{0.899840in}}%
\pgfpathlineto{\pgfqpoint{5.454545in}{0.899840in}}%
\pgfpathlineto{\pgfqpoint{5.454545in}{0.901437in}}%
\pgfpathlineto{\pgfqpoint{5.610390in}{0.901437in}}%
\pgfpathlineto{\pgfqpoint{5.610390in}{0.899840in}}%
\pgfpathlineto{\pgfqpoint{5.766234in}{0.899840in}}%
\pgfpathlineto{\pgfqpoint{5.766234in}{0.899840in}}%
\pgfpathlineto{\pgfqpoint{5.922078in}{0.899840in}}%
\pgfpathlineto{\pgfqpoint{5.922078in}{0.899840in}}%
\pgfpathlineto{\pgfqpoint{6.077922in}{0.899840in}}%
\pgfpathlineto{\pgfqpoint{6.077922in}{0.899840in}}%
\pgfpathlineto{\pgfqpoint{6.233766in}{0.899840in}}%
\pgfpathlineto{\pgfqpoint{6.233766in}{0.899840in}}%
\pgfpathlineto{\pgfqpoint{6.389610in}{0.899840in}}%
\pgfpathlineto{\pgfqpoint{6.389610in}{0.899840in}}%
\pgfpathlineto{\pgfqpoint{6.545455in}{0.899840in}}%
\pgfpathlineto{\pgfqpoint{6.545455in}{0.899840in}}%
\pgfpathlineto{\pgfqpoint{6.701299in}{0.899840in}}%
\pgfpathlineto{\pgfqpoint{6.701299in}{0.899840in}}%
\pgfpathlineto{\pgfqpoint{6.857143in}{0.899840in}}%
\pgfpathlineto{\pgfqpoint{6.857143in}{0.899840in}}%
\pgfpathlineto{\pgfqpoint{7.012987in}{0.899840in}}%
\pgfpathlineto{\pgfqpoint{7.012987in}{0.899840in}}%
\pgfpathlineto{\pgfqpoint{7.168831in}{0.899840in}}%
\pgfpathlineto{\pgfqpoint{7.168831in}{0.899840in}}%
\pgfpathlineto{\pgfqpoint{7.324675in}{0.899840in}}%
\pgfpathlineto{\pgfqpoint{7.324675in}{0.899840in}}%
\pgfpathlineto{\pgfqpoint{7.480519in}{0.899840in}}%
\pgfpathlineto{\pgfqpoint{7.480519in}{0.899840in}}%
\pgfpathlineto{\pgfqpoint{7.636364in}{0.899840in}}%
\pgfpathlineto{\pgfqpoint{7.636364in}{0.899840in}}%
\pgfpathlineto{\pgfqpoint{7.792208in}{0.899840in}}%
\pgfpathlineto{\pgfqpoint{7.792208in}{0.899840in}}%
\pgfpathlineto{\pgfqpoint{7.948052in}{0.899840in}}%
\pgfpathlineto{\pgfqpoint{7.948052in}{0.899840in}}%
\pgfpathlineto{\pgfqpoint{8.103896in}{0.899840in}}%
\pgfpathlineto{\pgfqpoint{8.103896in}{0.899840in}}%
\pgfpathlineto{\pgfqpoint{8.259740in}{0.899840in}}%
\pgfpathlineto{\pgfqpoint{8.259740in}{0.899840in}}%
\pgfpathlineto{\pgfqpoint{8.415584in}{0.899840in}}%
\pgfpathlineto{\pgfqpoint{8.415584in}{0.899840in}}%
\pgfpathlineto{\pgfqpoint{8.571429in}{0.899840in}}%
\pgfpathlineto{\pgfqpoint{8.571429in}{0.899840in}}%
\pgfpathlineto{\pgfqpoint{8.727273in}{0.899840in}}%
\pgfpathlineto{\pgfqpoint{8.727273in}{0.899840in}}%
\pgfpathlineto{\pgfqpoint{8.883117in}{0.899840in}}%
\pgfpathlineto{\pgfqpoint{8.883117in}{0.901437in}}%
\pgfpathlineto{\pgfqpoint{9.038961in}{0.901437in}}%
\pgfpathlineto{\pgfqpoint{9.038961in}{0.899840in}}%
\pgfpathlineto{\pgfqpoint{9.194805in}{0.899840in}}%
\pgfpathlineto{\pgfqpoint{9.194805in}{0.899840in}}%
\pgfpathlineto{\pgfqpoint{9.350649in}{0.899840in}}%
\pgfpathlineto{\pgfqpoint{9.350649in}{0.899840in}}%
\pgfpathlineto{\pgfqpoint{9.506494in}{0.899840in}}%
\pgfpathlineto{\pgfqpoint{9.506494in}{0.899840in}}%
\pgfpathlineto{\pgfqpoint{9.662338in}{0.899840in}}%
\pgfpathlineto{\pgfqpoint{9.662338in}{0.899840in}}%
\pgfpathlineto{\pgfqpoint{9.818182in}{0.899840in}}%
\pgfpathlineto{\pgfqpoint{9.818182in}{0.899840in}}%
\pgfusepath{stroke}%
\end{pgfscope}%
\begin{pgfscope}%
\pgfpathrectangle{\pgfqpoint{1.800000in}{0.900000in}}{\pgfqpoint{8.400000in}{4.800000in}}%
\pgfusepath{clip}%
\pgfsetbuttcap%
\pgfsetmiterjoin%
\pgfsetlinewidth{1.003750pt}%
\definecolor{currentstroke}{rgb}{0.750000,0.750000,0.000000}%
\pgfsetstrokecolor{currentstroke}%
\pgfsetdash{}{0pt}%
\pgfpathmoveto{\pgfqpoint{2.181818in}{0.899840in}}%
\pgfpathlineto{\pgfqpoint{2.181818in}{0.934969in}}%
\pgfpathlineto{\pgfqpoint{2.337662in}{0.934969in}}%
\pgfpathlineto{\pgfqpoint{2.337662in}{0.926986in}}%
\pgfpathlineto{\pgfqpoint{2.493506in}{0.926986in}}%
\pgfpathlineto{\pgfqpoint{2.493506in}{0.933373in}}%
\pgfpathlineto{\pgfqpoint{2.649351in}{0.933373in}}%
\pgfpathlineto{\pgfqpoint{2.649351in}{0.989260in}}%
\pgfpathlineto{\pgfqpoint{2.805195in}{0.989260in}}%
\pgfpathlineto{\pgfqpoint{2.805195in}{1.340552in}}%
\pgfpathlineto{\pgfqpoint{2.961039in}{1.340552in}}%
\pgfpathlineto{\pgfqpoint{2.961039in}{2.110199in}}%
\pgfpathlineto{\pgfqpoint{3.116883in}{2.110199in}}%
\pgfpathlineto{\pgfqpoint{3.116883in}{3.168864in}}%
\pgfpathlineto{\pgfqpoint{3.272727in}{3.168864in}}%
\pgfpathlineto{\pgfqpoint{3.272727in}{3.529737in}}%
\pgfpathlineto{\pgfqpoint{3.428571in}{3.529737in}}%
\pgfpathlineto{\pgfqpoint{3.428571in}{3.058687in}}%
\pgfpathlineto{\pgfqpoint{3.584416in}{3.058687in}}%
\pgfpathlineto{\pgfqpoint{3.584416in}{2.498217in}}%
\pgfpathlineto{\pgfqpoint{3.740260in}{2.498217in}}%
\pgfpathlineto{\pgfqpoint{3.740260in}{1.893037in}}%
\pgfpathlineto{\pgfqpoint{3.896104in}{1.893037in}}%
\pgfpathlineto{\pgfqpoint{3.896104in}{1.741343in}}%
\pgfpathlineto{\pgfqpoint{4.051948in}{1.741343in}}%
\pgfpathlineto{\pgfqpoint{4.051948in}{1.525778in}}%
\pgfpathlineto{\pgfqpoint{4.207792in}{1.525778in}}%
\pgfpathlineto{\pgfqpoint{4.207792in}{1.350132in}}%
\pgfpathlineto{\pgfqpoint{4.363636in}{1.350132in}}%
\pgfpathlineto{\pgfqpoint{4.363636in}{1.236761in}}%
\pgfpathlineto{\pgfqpoint{4.519481in}{1.236761in}}%
\pgfpathlineto{\pgfqpoint{4.519481in}{1.190454in}}%
\pgfpathlineto{\pgfqpoint{4.675325in}{1.190454in}}%
\pgfpathlineto{\pgfqpoint{4.675325in}{1.129777in}}%
\pgfpathlineto{\pgfqpoint{4.831169in}{1.129777in}}%
\pgfpathlineto{\pgfqpoint{4.831169in}{1.099438in}}%
\pgfpathlineto{\pgfqpoint{4.987013in}{1.099438in}}%
\pgfpathlineto{\pgfqpoint{4.987013in}{1.061115in}}%
\pgfpathlineto{\pgfqpoint{5.142857in}{1.061115in}}%
\pgfpathlineto{\pgfqpoint{5.142857in}{1.037163in}}%
\pgfpathlineto{\pgfqpoint{5.298701in}{1.037163in}}%
\pgfpathlineto{\pgfqpoint{5.298701in}{1.032373in}}%
\pgfpathlineto{\pgfqpoint{5.454545in}{1.032373in}}%
\pgfpathlineto{\pgfqpoint{5.454545in}{1.018002in}}%
\pgfpathlineto{\pgfqpoint{5.610390in}{1.018002in}}%
\pgfpathlineto{\pgfqpoint{5.610390in}{0.992454in}}%
\pgfpathlineto{\pgfqpoint{5.766234in}{0.992454in}}%
\pgfpathlineto{\pgfqpoint{5.766234in}{0.984470in}}%
\pgfpathlineto{\pgfqpoint{5.922078in}{0.984470in}}%
\pgfpathlineto{\pgfqpoint{5.922078in}{0.957324in}}%
\pgfpathlineto{\pgfqpoint{6.077922in}{0.957324in}}%
\pgfpathlineto{\pgfqpoint{6.077922in}{0.966905in}}%
\pgfpathlineto{\pgfqpoint{6.233766in}{0.966905in}}%
\pgfpathlineto{\pgfqpoint{6.233766in}{0.960518in}}%
\pgfpathlineto{\pgfqpoint{6.389610in}{0.960518in}}%
\pgfpathlineto{\pgfqpoint{6.389610in}{0.938163in}}%
\pgfpathlineto{\pgfqpoint{6.545455in}{0.938163in}}%
\pgfpathlineto{\pgfqpoint{6.545455in}{0.946147in}}%
\pgfpathlineto{\pgfqpoint{6.701299in}{0.946147in}}%
\pgfpathlineto{\pgfqpoint{6.701299in}{0.942953in}}%
\pgfpathlineto{\pgfqpoint{6.857143in}{0.942953in}}%
\pgfpathlineto{\pgfqpoint{6.857143in}{0.917405in}}%
\pgfpathlineto{\pgfqpoint{7.012987in}{0.917405in}}%
\pgfpathlineto{\pgfqpoint{7.012987in}{0.923792in}}%
\pgfpathlineto{\pgfqpoint{7.168831in}{0.923792in}}%
\pgfpathlineto{\pgfqpoint{7.168831in}{0.926986in}}%
\pgfpathlineto{\pgfqpoint{7.324675in}{0.926986in}}%
\pgfpathlineto{\pgfqpoint{7.324675in}{0.922195in}}%
\pgfpathlineto{\pgfqpoint{7.480519in}{0.922195in}}%
\pgfpathlineto{\pgfqpoint{7.480519in}{0.920598in}}%
\pgfpathlineto{\pgfqpoint{7.636364in}{0.920598in}}%
\pgfpathlineto{\pgfqpoint{7.636364in}{0.926986in}}%
\pgfpathlineto{\pgfqpoint{7.792208in}{0.926986in}}%
\pgfpathlineto{\pgfqpoint{7.792208in}{0.915808in}}%
\pgfpathlineto{\pgfqpoint{7.948052in}{0.915808in}}%
\pgfpathlineto{\pgfqpoint{7.948052in}{0.922195in}}%
\pgfpathlineto{\pgfqpoint{8.103896in}{0.922195in}}%
\pgfpathlineto{\pgfqpoint{8.103896in}{0.915808in}}%
\pgfpathlineto{\pgfqpoint{8.259740in}{0.915808in}}%
\pgfpathlineto{\pgfqpoint{8.259740in}{0.914211in}}%
\pgfpathlineto{\pgfqpoint{8.415584in}{0.914211in}}%
\pgfpathlineto{\pgfqpoint{8.415584in}{0.920598in}}%
\pgfpathlineto{\pgfqpoint{8.571429in}{0.920598in}}%
\pgfpathlineto{\pgfqpoint{8.571429in}{0.911018in}}%
\pgfpathlineto{\pgfqpoint{8.727273in}{0.911018in}}%
\pgfpathlineto{\pgfqpoint{8.727273in}{0.909421in}}%
\pgfpathlineto{\pgfqpoint{8.883117in}{0.909421in}}%
\pgfpathlineto{\pgfqpoint{8.883117in}{0.904631in}}%
\pgfpathlineto{\pgfqpoint{9.038961in}{0.904631in}}%
\pgfpathlineto{\pgfqpoint{9.038961in}{0.906227in}}%
\pgfpathlineto{\pgfqpoint{9.194805in}{0.906227in}}%
\pgfpathlineto{\pgfqpoint{9.194805in}{0.914211in}}%
\pgfpathlineto{\pgfqpoint{9.350649in}{0.914211in}}%
\pgfpathlineto{\pgfqpoint{9.350649in}{0.911018in}}%
\pgfpathlineto{\pgfqpoint{9.506494in}{0.911018in}}%
\pgfpathlineto{\pgfqpoint{9.506494in}{0.912615in}}%
\pgfpathlineto{\pgfqpoint{9.662338in}{0.912615in}}%
\pgfpathlineto{\pgfqpoint{9.662338in}{0.904631in}}%
\pgfpathlineto{\pgfqpoint{9.818182in}{0.904631in}}%
\pgfpathlineto{\pgfqpoint{9.818182in}{0.899840in}}%
\pgfusepath{stroke}%
\end{pgfscope}%
\begin{pgfscope}%
\pgfsetrectcap%
\pgfsetmiterjoin%
\pgfsetlinewidth{0.803000pt}%
\definecolor{currentstroke}{rgb}{0.000000,0.000000,0.000000}%
\pgfsetstrokecolor{currentstroke}%
\pgfsetdash{}{0pt}%
\pgfpathmoveto{\pgfqpoint{1.800000in}{0.900000in}}%
\pgfpathlineto{\pgfqpoint{1.800000in}{5.700000in}}%
\pgfusepath{stroke}%
\end{pgfscope}%
\begin{pgfscope}%
\pgfsetrectcap%
\pgfsetmiterjoin%
\pgfsetlinewidth{0.803000pt}%
\definecolor{currentstroke}{rgb}{0.000000,0.000000,0.000000}%
\pgfsetstrokecolor{currentstroke}%
\pgfsetdash{}{0pt}%
\pgfpathmoveto{\pgfqpoint{10.200000in}{0.900000in}}%
\pgfpathlineto{\pgfqpoint{10.200000in}{5.700000in}}%
\pgfusepath{stroke}%
\end{pgfscope}%
\begin{pgfscope}%
\pgfsetrectcap%
\pgfsetmiterjoin%
\pgfsetlinewidth{0.803000pt}%
\definecolor{currentstroke}{rgb}{0.000000,0.000000,0.000000}%
\pgfsetstrokecolor{currentstroke}%
\pgfsetdash{}{0pt}%
\pgfpathmoveto{\pgfqpoint{1.800000in}{0.900000in}}%
\pgfpathlineto{\pgfqpoint{10.200000in}{0.900000in}}%
\pgfusepath{stroke}%
\end{pgfscope}%
\begin{pgfscope}%
\pgfsetrectcap%
\pgfsetmiterjoin%
\pgfsetlinewidth{0.803000pt}%
\definecolor{currentstroke}{rgb}{0.000000,0.000000,0.000000}%
\pgfsetstrokecolor{currentstroke}%
\pgfsetdash{}{0pt}%
\pgfpathmoveto{\pgfqpoint{1.800000in}{5.700000in}}%
\pgfpathlineto{\pgfqpoint{10.200000in}{5.700000in}}%
\pgfusepath{stroke}%
\end{pgfscope}%
\begin{pgfscope}%
\pgfsetbuttcap%
\pgfsetmiterjoin%
\definecolor{currentfill}{rgb}{1.000000,1.000000,1.000000}%
\pgfsetfillcolor{currentfill}%
\pgfsetfillopacity{0.800000}%
\pgfsetlinewidth{1.003750pt}%
\definecolor{currentstroke}{rgb}{0.800000,0.800000,0.800000}%
\pgfsetstrokecolor{currentstroke}%
\pgfsetstrokeopacity{0.800000}%
\pgfsetdash{}{0pt}%
\pgfpathmoveto{\pgfqpoint{7.012787in}{4.687865in}}%
\pgfpathlineto{\pgfqpoint{10.005556in}{4.687865in}}%
\pgfpathquadraticcurveto{\pgfqpoint{10.061111in}{4.687865in}}{\pgfqpoint{10.061111in}{4.743420in}}%
\pgfpathlineto{\pgfqpoint{10.061111in}{5.505556in}}%
\pgfpathquadraticcurveto{\pgfqpoint{10.061111in}{5.561111in}}{\pgfqpoint{10.005556in}{5.561111in}}%
\pgfpathlineto{\pgfqpoint{7.012787in}{5.561111in}}%
\pgfpathquadraticcurveto{\pgfqpoint{6.957231in}{5.561111in}}{\pgfqpoint{6.957231in}{5.505556in}}%
\pgfpathlineto{\pgfqpoint{6.957231in}{4.743420in}}%
\pgfpathquadraticcurveto{\pgfqpoint{6.957231in}{4.687865in}}{\pgfqpoint{7.012787in}{4.687865in}}%
\pgfpathclose%
\pgfusepath{stroke,fill}%
\end{pgfscope}%
\begin{pgfscope}%
\pgfsetbuttcap%
\pgfsetmiterjoin%
\pgfsetlinewidth{1.003750pt}%
\definecolor{currentstroke}{rgb}{0.000000,0.000000,1.000000}%
\pgfsetstrokecolor{currentstroke}%
\pgfsetdash{}{0pt}%
\pgfpathmoveto{\pgfqpoint{7.068342in}{5.249962in}}%
\pgfpathlineto{\pgfqpoint{7.623898in}{5.249962in}}%
\pgfpathlineto{\pgfqpoint{7.623898in}{5.444406in}}%
\pgfpathlineto{\pgfqpoint{7.068342in}{5.444406in}}%
\pgfpathclose%
\pgfusepath{stroke}%
\end{pgfscope}%
\begin{pgfscope}%
\definecolor{textcolor}{rgb}{0.000000,0.000000,0.000000}%
\pgfsetstrokecolor{textcolor}%
\pgfsetfillcolor{textcolor}%
\pgftext[x=7.846120in,y=5.249962in,left,base]{\color{textcolor}\sffamily\fontsize{20.000000}{24.000000}\selectfont \(\displaystyle \mathrm{FBMP}\)}%
\end{pgfscope}%
\begin{pgfscope}%
\pgfsetbuttcap%
\pgfsetmiterjoin%
\pgfsetlinewidth{1.003750pt}%
\definecolor{currentstroke}{rgb}{0.750000,0.750000,0.000000}%
\pgfsetstrokecolor{currentstroke}%
\pgfsetdash{}{0pt}%
\pgfpathmoveto{\pgfqpoint{7.068342in}{4.855005in}}%
\pgfpathlineto{\pgfqpoint{7.623898in}{4.855005in}}%
\pgfpathlineto{\pgfqpoint{7.623898in}{5.049449in}}%
\pgfpathlineto{\pgfqpoint{7.068342in}{5.049449in}}%
\pgfpathclose%
\pgfusepath{stroke}%
\end{pgfscope}%
\begin{pgfscope}%
\definecolor{textcolor}{rgb}{0.000000,0.000000,0.000000}%
\pgfsetstrokecolor{textcolor}%
\pgfsetfillcolor{textcolor}%
\pgftext[x=7.846120in,y=4.855005in,left,base]{\color{textcolor}\sffamily\fontsize{20.000000}{24.000000}\selectfont \(\displaystyle \mathrm{Fourier\ Transform}\)}%
\end{pgfscope}%
\end{pgfpicture}%
\makeatother%
\endgroup%
}
    \caption{$D_w$ and $\mathrm{RSS}$ of methods}
\end{figure}
\end{frame}

\begin{frame}
\frametitle{Charge posterior}
\begin{figure}
    \centering
    \resizebox{0.7\textwidth}{!}{%% Creator: Matplotlib, PGF backend
%%
%% To include the figure in your LaTeX document, write
%%   \input{<filename>.pgf}
%%
%% Make sure the required packages are loaded in your preamble
%%   \usepackage{pgf}
%%
%% and, on pdftex
%%   \usepackage[utf8]{inputenc}\DeclareUnicodeCharacter{2212}{-}
%%
%% or, on luatex and xetex
%%   \usepackage{unicode-math}
%%
%% Figures using additional raster images can only be included by \input if
%% they are in the same directory as the main LaTeX file. For loading figures
%% from other directories you can use the `import` package
%%   \usepackage{import}
%%
%% and then include the figures with
%%   \import{<path to file>}{<filename>.pgf}
%%
%% Matplotlib used the following preamble
%%   \usepackage[detect-all,locale=DE]{siunitx}
%%
\begingroup%
\makeatletter%
\begin{pgfpicture}%
\pgfpathrectangle{\pgfpointorigin}{\pgfqpoint{12.000000in}{6.000000in}}%
\pgfusepath{use as bounding box, clip}%
\begin{pgfscope}%
\pgfsetbuttcap%
\pgfsetmiterjoin%
\definecolor{currentfill}{rgb}{1.000000,1.000000,1.000000}%
\pgfsetfillcolor{currentfill}%
\pgfsetlinewidth{0.000000pt}%
\definecolor{currentstroke}{rgb}{1.000000,1.000000,1.000000}%
\pgfsetstrokecolor{currentstroke}%
\pgfsetdash{}{0pt}%
\pgfpathmoveto{\pgfqpoint{0.000000in}{0.000000in}}%
\pgfpathlineto{\pgfqpoint{12.000000in}{0.000000in}}%
\pgfpathlineto{\pgfqpoint{12.000000in}{6.000000in}}%
\pgfpathlineto{\pgfqpoint{0.000000in}{6.000000in}}%
\pgfpathclose%
\pgfusepath{fill}%
\end{pgfscope}%
\begin{pgfscope}%
\pgfsetbuttcap%
\pgfsetmiterjoin%
\definecolor{currentfill}{rgb}{1.000000,1.000000,1.000000}%
\pgfsetfillcolor{currentfill}%
\pgfsetlinewidth{0.000000pt}%
\definecolor{currentstroke}{rgb}{0.000000,0.000000,0.000000}%
\pgfsetstrokecolor{currentstroke}%
\pgfsetstrokeopacity{0.000000}%
\pgfsetdash{}{0pt}%
\pgfpathmoveto{\pgfqpoint{1.800000in}{0.900000in}}%
\pgfpathlineto{\pgfqpoint{10.200000in}{0.900000in}}%
\pgfpathlineto{\pgfqpoint{10.200000in}{5.700000in}}%
\pgfpathlineto{\pgfqpoint{1.800000in}{5.700000in}}%
\pgfpathclose%
\pgfusepath{fill}%
\end{pgfscope}%
\begin{pgfscope}%
\pgfpathrectangle{\pgfqpoint{1.800000in}{0.900000in}}{\pgfqpoint{8.400000in}{4.800000in}}%
\pgfusepath{clip}%
\pgfsetrectcap%
\pgfsetroundjoin%
\pgfsetlinewidth{0.803000pt}%
\definecolor{currentstroke}{rgb}{0.690196,0.690196,0.690196}%
\pgfsetstrokecolor{currentstroke}%
\pgfsetdash{}{0pt}%
\pgfpathmoveto{\pgfqpoint{1.800000in}{0.900000in}}%
\pgfpathlineto{\pgfqpoint{1.800000in}{5.700000in}}%
\pgfusepath{stroke}%
\end{pgfscope}%
\begin{pgfscope}%
\pgfsetbuttcap%
\pgfsetroundjoin%
\definecolor{currentfill}{rgb}{0.000000,0.000000,0.000000}%
\pgfsetfillcolor{currentfill}%
\pgfsetlinewidth{0.803000pt}%
\definecolor{currentstroke}{rgb}{0.000000,0.000000,0.000000}%
\pgfsetstrokecolor{currentstroke}%
\pgfsetdash{}{0pt}%
\pgfsys@defobject{currentmarker}{\pgfqpoint{0.000000in}{-0.048611in}}{\pgfqpoint{0.000000in}{0.000000in}}{%
\pgfpathmoveto{\pgfqpoint{0.000000in}{0.000000in}}%
\pgfpathlineto{\pgfqpoint{0.000000in}{-0.048611in}}%
\pgfusepath{stroke,fill}%
}%
\begin{pgfscope}%
\pgfsys@transformshift{1.800000in}{0.900000in}%
\pgfsys@useobject{currentmarker}{}%
\end{pgfscope}%
\end{pgfscope}%
\begin{pgfscope}%
\definecolor{textcolor}{rgb}{0.000000,0.000000,0.000000}%
\pgfsetstrokecolor{textcolor}%
\pgfsetfillcolor{textcolor}%
\pgftext[x=1.800000in,y=0.802778in,,top]{\color{textcolor}\sffamily\fontsize{20.000000}{24.000000}\selectfont \(\displaystyle {0}\)}%
\end{pgfscope}%
\begin{pgfscope}%
\pgfpathrectangle{\pgfqpoint{1.800000in}{0.900000in}}{\pgfqpoint{8.400000in}{4.800000in}}%
\pgfusepath{clip}%
\pgfsetrectcap%
\pgfsetroundjoin%
\pgfsetlinewidth{0.803000pt}%
\definecolor{currentstroke}{rgb}{0.690196,0.690196,0.690196}%
\pgfsetstrokecolor{currentstroke}%
\pgfsetdash{}{0pt}%
\pgfpathmoveto{\pgfqpoint{3.000000in}{0.900000in}}%
\pgfpathlineto{\pgfqpoint{3.000000in}{5.700000in}}%
\pgfusepath{stroke}%
\end{pgfscope}%
\begin{pgfscope}%
\pgfsetbuttcap%
\pgfsetroundjoin%
\definecolor{currentfill}{rgb}{0.000000,0.000000,0.000000}%
\pgfsetfillcolor{currentfill}%
\pgfsetlinewidth{0.803000pt}%
\definecolor{currentstroke}{rgb}{0.000000,0.000000,0.000000}%
\pgfsetstrokecolor{currentstroke}%
\pgfsetdash{}{0pt}%
\pgfsys@defobject{currentmarker}{\pgfqpoint{0.000000in}{-0.048611in}}{\pgfqpoint{0.000000in}{0.000000in}}{%
\pgfpathmoveto{\pgfqpoint{0.000000in}{0.000000in}}%
\pgfpathlineto{\pgfqpoint{0.000000in}{-0.048611in}}%
\pgfusepath{stroke,fill}%
}%
\begin{pgfscope}%
\pgfsys@transformshift{3.000000in}{0.900000in}%
\pgfsys@useobject{currentmarker}{}%
\end{pgfscope}%
\end{pgfscope}%
\begin{pgfscope}%
\definecolor{textcolor}{rgb}{0.000000,0.000000,0.000000}%
\pgfsetstrokecolor{textcolor}%
\pgfsetfillcolor{textcolor}%
\pgftext[x=3.000000in,y=0.802778in,,top]{\color{textcolor}\sffamily\fontsize{20.000000}{24.000000}\selectfont \(\displaystyle {50}\)}%
\end{pgfscope}%
\begin{pgfscope}%
\pgfpathrectangle{\pgfqpoint{1.800000in}{0.900000in}}{\pgfqpoint{8.400000in}{4.800000in}}%
\pgfusepath{clip}%
\pgfsetrectcap%
\pgfsetroundjoin%
\pgfsetlinewidth{0.803000pt}%
\definecolor{currentstroke}{rgb}{0.690196,0.690196,0.690196}%
\pgfsetstrokecolor{currentstroke}%
\pgfsetdash{}{0pt}%
\pgfpathmoveto{\pgfqpoint{4.200000in}{0.900000in}}%
\pgfpathlineto{\pgfqpoint{4.200000in}{5.700000in}}%
\pgfusepath{stroke}%
\end{pgfscope}%
\begin{pgfscope}%
\pgfsetbuttcap%
\pgfsetroundjoin%
\definecolor{currentfill}{rgb}{0.000000,0.000000,0.000000}%
\pgfsetfillcolor{currentfill}%
\pgfsetlinewidth{0.803000pt}%
\definecolor{currentstroke}{rgb}{0.000000,0.000000,0.000000}%
\pgfsetstrokecolor{currentstroke}%
\pgfsetdash{}{0pt}%
\pgfsys@defobject{currentmarker}{\pgfqpoint{0.000000in}{-0.048611in}}{\pgfqpoint{0.000000in}{0.000000in}}{%
\pgfpathmoveto{\pgfqpoint{0.000000in}{0.000000in}}%
\pgfpathlineto{\pgfqpoint{0.000000in}{-0.048611in}}%
\pgfusepath{stroke,fill}%
}%
\begin{pgfscope}%
\pgfsys@transformshift{4.200000in}{0.900000in}%
\pgfsys@useobject{currentmarker}{}%
\end{pgfscope}%
\end{pgfscope}%
\begin{pgfscope}%
\definecolor{textcolor}{rgb}{0.000000,0.000000,0.000000}%
\pgfsetstrokecolor{textcolor}%
\pgfsetfillcolor{textcolor}%
\pgftext[x=4.200000in,y=0.802778in,,top]{\color{textcolor}\sffamily\fontsize{20.000000}{24.000000}\selectfont \(\displaystyle {100}\)}%
\end{pgfscope}%
\begin{pgfscope}%
\pgfpathrectangle{\pgfqpoint{1.800000in}{0.900000in}}{\pgfqpoint{8.400000in}{4.800000in}}%
\pgfusepath{clip}%
\pgfsetrectcap%
\pgfsetroundjoin%
\pgfsetlinewidth{0.803000pt}%
\definecolor{currentstroke}{rgb}{0.690196,0.690196,0.690196}%
\pgfsetstrokecolor{currentstroke}%
\pgfsetdash{}{0pt}%
\pgfpathmoveto{\pgfqpoint{5.400000in}{0.900000in}}%
\pgfpathlineto{\pgfqpoint{5.400000in}{5.700000in}}%
\pgfusepath{stroke}%
\end{pgfscope}%
\begin{pgfscope}%
\pgfsetbuttcap%
\pgfsetroundjoin%
\definecolor{currentfill}{rgb}{0.000000,0.000000,0.000000}%
\pgfsetfillcolor{currentfill}%
\pgfsetlinewidth{0.803000pt}%
\definecolor{currentstroke}{rgb}{0.000000,0.000000,0.000000}%
\pgfsetstrokecolor{currentstroke}%
\pgfsetdash{}{0pt}%
\pgfsys@defobject{currentmarker}{\pgfqpoint{0.000000in}{-0.048611in}}{\pgfqpoint{0.000000in}{0.000000in}}{%
\pgfpathmoveto{\pgfqpoint{0.000000in}{0.000000in}}%
\pgfpathlineto{\pgfqpoint{0.000000in}{-0.048611in}}%
\pgfusepath{stroke,fill}%
}%
\begin{pgfscope}%
\pgfsys@transformshift{5.400000in}{0.900000in}%
\pgfsys@useobject{currentmarker}{}%
\end{pgfscope}%
\end{pgfscope}%
\begin{pgfscope}%
\definecolor{textcolor}{rgb}{0.000000,0.000000,0.000000}%
\pgfsetstrokecolor{textcolor}%
\pgfsetfillcolor{textcolor}%
\pgftext[x=5.400000in,y=0.802778in,,top]{\color{textcolor}\sffamily\fontsize{20.000000}{24.000000}\selectfont \(\displaystyle {150}\)}%
\end{pgfscope}%
\begin{pgfscope}%
\pgfpathrectangle{\pgfqpoint{1.800000in}{0.900000in}}{\pgfqpoint{8.400000in}{4.800000in}}%
\pgfusepath{clip}%
\pgfsetrectcap%
\pgfsetroundjoin%
\pgfsetlinewidth{0.803000pt}%
\definecolor{currentstroke}{rgb}{0.690196,0.690196,0.690196}%
\pgfsetstrokecolor{currentstroke}%
\pgfsetdash{}{0pt}%
\pgfpathmoveto{\pgfqpoint{6.600000in}{0.900000in}}%
\pgfpathlineto{\pgfqpoint{6.600000in}{5.700000in}}%
\pgfusepath{stroke}%
\end{pgfscope}%
\begin{pgfscope}%
\pgfsetbuttcap%
\pgfsetroundjoin%
\definecolor{currentfill}{rgb}{0.000000,0.000000,0.000000}%
\pgfsetfillcolor{currentfill}%
\pgfsetlinewidth{0.803000pt}%
\definecolor{currentstroke}{rgb}{0.000000,0.000000,0.000000}%
\pgfsetstrokecolor{currentstroke}%
\pgfsetdash{}{0pt}%
\pgfsys@defobject{currentmarker}{\pgfqpoint{0.000000in}{-0.048611in}}{\pgfqpoint{0.000000in}{0.000000in}}{%
\pgfpathmoveto{\pgfqpoint{0.000000in}{0.000000in}}%
\pgfpathlineto{\pgfqpoint{0.000000in}{-0.048611in}}%
\pgfusepath{stroke,fill}%
}%
\begin{pgfscope}%
\pgfsys@transformshift{6.600000in}{0.900000in}%
\pgfsys@useobject{currentmarker}{}%
\end{pgfscope}%
\end{pgfscope}%
\begin{pgfscope}%
\definecolor{textcolor}{rgb}{0.000000,0.000000,0.000000}%
\pgfsetstrokecolor{textcolor}%
\pgfsetfillcolor{textcolor}%
\pgftext[x=6.600000in,y=0.802778in,,top]{\color{textcolor}\sffamily\fontsize{20.000000}{24.000000}\selectfont \(\displaystyle {200}\)}%
\end{pgfscope}%
\begin{pgfscope}%
\pgfpathrectangle{\pgfqpoint{1.800000in}{0.900000in}}{\pgfqpoint{8.400000in}{4.800000in}}%
\pgfusepath{clip}%
\pgfsetrectcap%
\pgfsetroundjoin%
\pgfsetlinewidth{0.803000pt}%
\definecolor{currentstroke}{rgb}{0.690196,0.690196,0.690196}%
\pgfsetstrokecolor{currentstroke}%
\pgfsetdash{}{0pt}%
\pgfpathmoveto{\pgfqpoint{7.800000in}{0.900000in}}%
\pgfpathlineto{\pgfqpoint{7.800000in}{5.700000in}}%
\pgfusepath{stroke}%
\end{pgfscope}%
\begin{pgfscope}%
\pgfsetbuttcap%
\pgfsetroundjoin%
\definecolor{currentfill}{rgb}{0.000000,0.000000,0.000000}%
\pgfsetfillcolor{currentfill}%
\pgfsetlinewidth{0.803000pt}%
\definecolor{currentstroke}{rgb}{0.000000,0.000000,0.000000}%
\pgfsetstrokecolor{currentstroke}%
\pgfsetdash{}{0pt}%
\pgfsys@defobject{currentmarker}{\pgfqpoint{0.000000in}{-0.048611in}}{\pgfqpoint{0.000000in}{0.000000in}}{%
\pgfpathmoveto{\pgfqpoint{0.000000in}{0.000000in}}%
\pgfpathlineto{\pgfqpoint{0.000000in}{-0.048611in}}%
\pgfusepath{stroke,fill}%
}%
\begin{pgfscope}%
\pgfsys@transformshift{7.800000in}{0.900000in}%
\pgfsys@useobject{currentmarker}{}%
\end{pgfscope}%
\end{pgfscope}%
\begin{pgfscope}%
\definecolor{textcolor}{rgb}{0.000000,0.000000,0.000000}%
\pgfsetstrokecolor{textcolor}%
\pgfsetfillcolor{textcolor}%
\pgftext[x=7.800000in,y=0.802778in,,top]{\color{textcolor}\sffamily\fontsize{20.000000}{24.000000}\selectfont \(\displaystyle {250}\)}%
\end{pgfscope}%
\begin{pgfscope}%
\pgfpathrectangle{\pgfqpoint{1.800000in}{0.900000in}}{\pgfqpoint{8.400000in}{4.800000in}}%
\pgfusepath{clip}%
\pgfsetrectcap%
\pgfsetroundjoin%
\pgfsetlinewidth{0.803000pt}%
\definecolor{currentstroke}{rgb}{0.690196,0.690196,0.690196}%
\pgfsetstrokecolor{currentstroke}%
\pgfsetdash{}{0pt}%
\pgfpathmoveto{\pgfqpoint{9.000000in}{0.900000in}}%
\pgfpathlineto{\pgfqpoint{9.000000in}{5.700000in}}%
\pgfusepath{stroke}%
\end{pgfscope}%
\begin{pgfscope}%
\pgfsetbuttcap%
\pgfsetroundjoin%
\definecolor{currentfill}{rgb}{0.000000,0.000000,0.000000}%
\pgfsetfillcolor{currentfill}%
\pgfsetlinewidth{0.803000pt}%
\definecolor{currentstroke}{rgb}{0.000000,0.000000,0.000000}%
\pgfsetstrokecolor{currentstroke}%
\pgfsetdash{}{0pt}%
\pgfsys@defobject{currentmarker}{\pgfqpoint{0.000000in}{-0.048611in}}{\pgfqpoint{0.000000in}{0.000000in}}{%
\pgfpathmoveto{\pgfqpoint{0.000000in}{0.000000in}}%
\pgfpathlineto{\pgfqpoint{0.000000in}{-0.048611in}}%
\pgfusepath{stroke,fill}%
}%
\begin{pgfscope}%
\pgfsys@transformshift{9.000000in}{0.900000in}%
\pgfsys@useobject{currentmarker}{}%
\end{pgfscope}%
\end{pgfscope}%
\begin{pgfscope}%
\definecolor{textcolor}{rgb}{0.000000,0.000000,0.000000}%
\pgfsetstrokecolor{textcolor}%
\pgfsetfillcolor{textcolor}%
\pgftext[x=9.000000in,y=0.802778in,,top]{\color{textcolor}\sffamily\fontsize{20.000000}{24.000000}\selectfont \(\displaystyle {300}\)}%
\end{pgfscope}%
\begin{pgfscope}%
\pgfpathrectangle{\pgfqpoint{1.800000in}{0.900000in}}{\pgfqpoint{8.400000in}{4.800000in}}%
\pgfusepath{clip}%
\pgfsetrectcap%
\pgfsetroundjoin%
\pgfsetlinewidth{0.803000pt}%
\definecolor{currentstroke}{rgb}{0.690196,0.690196,0.690196}%
\pgfsetstrokecolor{currentstroke}%
\pgfsetdash{}{0pt}%
\pgfpathmoveto{\pgfqpoint{10.200000in}{0.900000in}}%
\pgfpathlineto{\pgfqpoint{10.200000in}{5.700000in}}%
\pgfusepath{stroke}%
\end{pgfscope}%
\begin{pgfscope}%
\pgfsetbuttcap%
\pgfsetroundjoin%
\definecolor{currentfill}{rgb}{0.000000,0.000000,0.000000}%
\pgfsetfillcolor{currentfill}%
\pgfsetlinewidth{0.803000pt}%
\definecolor{currentstroke}{rgb}{0.000000,0.000000,0.000000}%
\pgfsetstrokecolor{currentstroke}%
\pgfsetdash{}{0pt}%
\pgfsys@defobject{currentmarker}{\pgfqpoint{0.000000in}{-0.048611in}}{\pgfqpoint{0.000000in}{0.000000in}}{%
\pgfpathmoveto{\pgfqpoint{0.000000in}{0.000000in}}%
\pgfpathlineto{\pgfqpoint{0.000000in}{-0.048611in}}%
\pgfusepath{stroke,fill}%
}%
\begin{pgfscope}%
\pgfsys@transformshift{10.200000in}{0.900000in}%
\pgfsys@useobject{currentmarker}{}%
\end{pgfscope}%
\end{pgfscope}%
\begin{pgfscope}%
\definecolor{textcolor}{rgb}{0.000000,0.000000,0.000000}%
\pgfsetstrokecolor{textcolor}%
\pgfsetfillcolor{textcolor}%
\pgftext[x=10.200000in,y=0.802778in,,top]{\color{textcolor}\sffamily\fontsize{20.000000}{24.000000}\selectfont \(\displaystyle {350}\)}%
\end{pgfscope}%
\begin{pgfscope}%
\definecolor{textcolor}{rgb}{0.000000,0.000000,0.000000}%
\pgfsetstrokecolor{textcolor}%
\pgfsetfillcolor{textcolor}%
\pgftext[x=6.000000in,y=0.491155in,,top]{\color{textcolor}\sffamily\fontsize{20.000000}{24.000000}\selectfont \(\displaystyle \mathrm{Charge}/\si{mV\cdot ns}\)}%
\end{pgfscope}%
\begin{pgfscope}%
\pgfpathrectangle{\pgfqpoint{1.800000in}{0.900000in}}{\pgfqpoint{8.400000in}{4.800000in}}%
\pgfusepath{clip}%
\pgfsetrectcap%
\pgfsetroundjoin%
\pgfsetlinewidth{0.803000pt}%
\definecolor{currentstroke}{rgb}{0.690196,0.690196,0.690196}%
\pgfsetstrokecolor{currentstroke}%
\pgfsetdash{}{0pt}%
\pgfpathmoveto{\pgfqpoint{1.800000in}{0.900000in}}%
\pgfpathlineto{\pgfqpoint{10.200000in}{0.900000in}}%
\pgfusepath{stroke}%
\end{pgfscope}%
\begin{pgfscope}%
\pgfsetbuttcap%
\pgfsetroundjoin%
\definecolor{currentfill}{rgb}{0.000000,0.000000,0.000000}%
\pgfsetfillcolor{currentfill}%
\pgfsetlinewidth{0.803000pt}%
\definecolor{currentstroke}{rgb}{0.000000,0.000000,0.000000}%
\pgfsetstrokecolor{currentstroke}%
\pgfsetdash{}{0pt}%
\pgfsys@defobject{currentmarker}{\pgfqpoint{-0.048611in}{0.000000in}}{\pgfqpoint{-0.000000in}{0.000000in}}{%
\pgfpathmoveto{\pgfqpoint{-0.000000in}{0.000000in}}%
\pgfpathlineto{\pgfqpoint{-0.048611in}{0.000000in}}%
\pgfusepath{stroke,fill}%
}%
\begin{pgfscope}%
\pgfsys@transformshift{1.800000in}{0.900000in}%
\pgfsys@useobject{currentmarker}{}%
\end{pgfscope}%
\end{pgfscope}%
\begin{pgfscope}%
\definecolor{textcolor}{rgb}{0.000000,0.000000,0.000000}%
\pgfsetstrokecolor{textcolor}%
\pgfsetfillcolor{textcolor}%
\pgftext[x=1.360215in, y=0.799981in, left, base]{\color{textcolor}\sffamily\fontsize{20.000000}{24.000000}\selectfont \(\displaystyle {0.0}\)}%
\end{pgfscope}%
\begin{pgfscope}%
\pgfpathrectangle{\pgfqpoint{1.800000in}{0.900000in}}{\pgfqpoint{8.400000in}{4.800000in}}%
\pgfusepath{clip}%
\pgfsetrectcap%
\pgfsetroundjoin%
\pgfsetlinewidth{0.803000pt}%
\definecolor{currentstroke}{rgb}{0.690196,0.690196,0.690196}%
\pgfsetstrokecolor{currentstroke}%
\pgfsetdash{}{0pt}%
\pgfpathmoveto{\pgfqpoint{1.800000in}{1.533877in}}%
\pgfpathlineto{\pgfqpoint{10.200000in}{1.533877in}}%
\pgfusepath{stroke}%
\end{pgfscope}%
\begin{pgfscope}%
\pgfsetbuttcap%
\pgfsetroundjoin%
\definecolor{currentfill}{rgb}{0.000000,0.000000,0.000000}%
\pgfsetfillcolor{currentfill}%
\pgfsetlinewidth{0.803000pt}%
\definecolor{currentstroke}{rgb}{0.000000,0.000000,0.000000}%
\pgfsetstrokecolor{currentstroke}%
\pgfsetdash{}{0pt}%
\pgfsys@defobject{currentmarker}{\pgfqpoint{-0.048611in}{0.000000in}}{\pgfqpoint{-0.000000in}{0.000000in}}{%
\pgfpathmoveto{\pgfqpoint{-0.000000in}{0.000000in}}%
\pgfpathlineto{\pgfqpoint{-0.048611in}{0.000000in}}%
\pgfusepath{stroke,fill}%
}%
\begin{pgfscope}%
\pgfsys@transformshift{1.800000in}{1.533877in}%
\pgfsys@useobject{currentmarker}{}%
\end{pgfscope}%
\end{pgfscope}%
\begin{pgfscope}%
\definecolor{textcolor}{rgb}{0.000000,0.000000,0.000000}%
\pgfsetstrokecolor{textcolor}%
\pgfsetfillcolor{textcolor}%
\pgftext[x=1.360215in, y=1.433858in, left, base]{\color{textcolor}\sffamily\fontsize{20.000000}{24.000000}\selectfont \(\displaystyle {0.5}\)}%
\end{pgfscope}%
\begin{pgfscope}%
\pgfpathrectangle{\pgfqpoint{1.800000in}{0.900000in}}{\pgfqpoint{8.400000in}{4.800000in}}%
\pgfusepath{clip}%
\pgfsetrectcap%
\pgfsetroundjoin%
\pgfsetlinewidth{0.803000pt}%
\definecolor{currentstroke}{rgb}{0.690196,0.690196,0.690196}%
\pgfsetstrokecolor{currentstroke}%
\pgfsetdash{}{0pt}%
\pgfpathmoveto{\pgfqpoint{1.800000in}{2.167754in}}%
\pgfpathlineto{\pgfqpoint{10.200000in}{2.167754in}}%
\pgfusepath{stroke}%
\end{pgfscope}%
\begin{pgfscope}%
\pgfsetbuttcap%
\pgfsetroundjoin%
\definecolor{currentfill}{rgb}{0.000000,0.000000,0.000000}%
\pgfsetfillcolor{currentfill}%
\pgfsetlinewidth{0.803000pt}%
\definecolor{currentstroke}{rgb}{0.000000,0.000000,0.000000}%
\pgfsetstrokecolor{currentstroke}%
\pgfsetdash{}{0pt}%
\pgfsys@defobject{currentmarker}{\pgfqpoint{-0.048611in}{0.000000in}}{\pgfqpoint{-0.000000in}{0.000000in}}{%
\pgfpathmoveto{\pgfqpoint{-0.000000in}{0.000000in}}%
\pgfpathlineto{\pgfqpoint{-0.048611in}{0.000000in}}%
\pgfusepath{stroke,fill}%
}%
\begin{pgfscope}%
\pgfsys@transformshift{1.800000in}{2.167754in}%
\pgfsys@useobject{currentmarker}{}%
\end{pgfscope}%
\end{pgfscope}%
\begin{pgfscope}%
\definecolor{textcolor}{rgb}{0.000000,0.000000,0.000000}%
\pgfsetstrokecolor{textcolor}%
\pgfsetfillcolor{textcolor}%
\pgftext[x=1.360215in, y=2.067735in, left, base]{\color{textcolor}\sffamily\fontsize{20.000000}{24.000000}\selectfont \(\displaystyle {1.0}\)}%
\end{pgfscope}%
\begin{pgfscope}%
\pgfpathrectangle{\pgfqpoint{1.800000in}{0.900000in}}{\pgfqpoint{8.400000in}{4.800000in}}%
\pgfusepath{clip}%
\pgfsetrectcap%
\pgfsetroundjoin%
\pgfsetlinewidth{0.803000pt}%
\definecolor{currentstroke}{rgb}{0.690196,0.690196,0.690196}%
\pgfsetstrokecolor{currentstroke}%
\pgfsetdash{}{0pt}%
\pgfpathmoveto{\pgfqpoint{1.800000in}{2.801632in}}%
\pgfpathlineto{\pgfqpoint{10.200000in}{2.801632in}}%
\pgfusepath{stroke}%
\end{pgfscope}%
\begin{pgfscope}%
\pgfsetbuttcap%
\pgfsetroundjoin%
\definecolor{currentfill}{rgb}{0.000000,0.000000,0.000000}%
\pgfsetfillcolor{currentfill}%
\pgfsetlinewidth{0.803000pt}%
\definecolor{currentstroke}{rgb}{0.000000,0.000000,0.000000}%
\pgfsetstrokecolor{currentstroke}%
\pgfsetdash{}{0pt}%
\pgfsys@defobject{currentmarker}{\pgfqpoint{-0.048611in}{0.000000in}}{\pgfqpoint{-0.000000in}{0.000000in}}{%
\pgfpathmoveto{\pgfqpoint{-0.000000in}{0.000000in}}%
\pgfpathlineto{\pgfqpoint{-0.048611in}{0.000000in}}%
\pgfusepath{stroke,fill}%
}%
\begin{pgfscope}%
\pgfsys@transformshift{1.800000in}{2.801632in}%
\pgfsys@useobject{currentmarker}{}%
\end{pgfscope}%
\end{pgfscope}%
\begin{pgfscope}%
\definecolor{textcolor}{rgb}{0.000000,0.000000,0.000000}%
\pgfsetstrokecolor{textcolor}%
\pgfsetfillcolor{textcolor}%
\pgftext[x=1.360215in, y=2.701612in, left, base]{\color{textcolor}\sffamily\fontsize{20.000000}{24.000000}\selectfont \(\displaystyle {1.5}\)}%
\end{pgfscope}%
\begin{pgfscope}%
\pgfpathrectangle{\pgfqpoint{1.800000in}{0.900000in}}{\pgfqpoint{8.400000in}{4.800000in}}%
\pgfusepath{clip}%
\pgfsetrectcap%
\pgfsetroundjoin%
\pgfsetlinewidth{0.803000pt}%
\definecolor{currentstroke}{rgb}{0.690196,0.690196,0.690196}%
\pgfsetstrokecolor{currentstroke}%
\pgfsetdash{}{0pt}%
\pgfpathmoveto{\pgfqpoint{1.800000in}{3.435509in}}%
\pgfpathlineto{\pgfqpoint{10.200000in}{3.435509in}}%
\pgfusepath{stroke}%
\end{pgfscope}%
\begin{pgfscope}%
\pgfsetbuttcap%
\pgfsetroundjoin%
\definecolor{currentfill}{rgb}{0.000000,0.000000,0.000000}%
\pgfsetfillcolor{currentfill}%
\pgfsetlinewidth{0.803000pt}%
\definecolor{currentstroke}{rgb}{0.000000,0.000000,0.000000}%
\pgfsetstrokecolor{currentstroke}%
\pgfsetdash{}{0pt}%
\pgfsys@defobject{currentmarker}{\pgfqpoint{-0.048611in}{0.000000in}}{\pgfqpoint{-0.000000in}{0.000000in}}{%
\pgfpathmoveto{\pgfqpoint{-0.000000in}{0.000000in}}%
\pgfpathlineto{\pgfqpoint{-0.048611in}{0.000000in}}%
\pgfusepath{stroke,fill}%
}%
\begin{pgfscope}%
\pgfsys@transformshift{1.800000in}{3.435509in}%
\pgfsys@useobject{currentmarker}{}%
\end{pgfscope}%
\end{pgfscope}%
\begin{pgfscope}%
\definecolor{textcolor}{rgb}{0.000000,0.000000,0.000000}%
\pgfsetstrokecolor{textcolor}%
\pgfsetfillcolor{textcolor}%
\pgftext[x=1.360215in, y=3.335490in, left, base]{\color{textcolor}\sffamily\fontsize{20.000000}{24.000000}\selectfont \(\displaystyle {2.0}\)}%
\end{pgfscope}%
\begin{pgfscope}%
\pgfpathrectangle{\pgfqpoint{1.800000in}{0.900000in}}{\pgfqpoint{8.400000in}{4.800000in}}%
\pgfusepath{clip}%
\pgfsetrectcap%
\pgfsetroundjoin%
\pgfsetlinewidth{0.803000pt}%
\definecolor{currentstroke}{rgb}{0.690196,0.690196,0.690196}%
\pgfsetstrokecolor{currentstroke}%
\pgfsetdash{}{0pt}%
\pgfpathmoveto{\pgfqpoint{1.800000in}{4.069386in}}%
\pgfpathlineto{\pgfqpoint{10.200000in}{4.069386in}}%
\pgfusepath{stroke}%
\end{pgfscope}%
\begin{pgfscope}%
\pgfsetbuttcap%
\pgfsetroundjoin%
\definecolor{currentfill}{rgb}{0.000000,0.000000,0.000000}%
\pgfsetfillcolor{currentfill}%
\pgfsetlinewidth{0.803000pt}%
\definecolor{currentstroke}{rgb}{0.000000,0.000000,0.000000}%
\pgfsetstrokecolor{currentstroke}%
\pgfsetdash{}{0pt}%
\pgfsys@defobject{currentmarker}{\pgfqpoint{-0.048611in}{0.000000in}}{\pgfqpoint{-0.000000in}{0.000000in}}{%
\pgfpathmoveto{\pgfqpoint{-0.000000in}{0.000000in}}%
\pgfpathlineto{\pgfqpoint{-0.048611in}{0.000000in}}%
\pgfusepath{stroke,fill}%
}%
\begin{pgfscope}%
\pgfsys@transformshift{1.800000in}{4.069386in}%
\pgfsys@useobject{currentmarker}{}%
\end{pgfscope}%
\end{pgfscope}%
\begin{pgfscope}%
\definecolor{textcolor}{rgb}{0.000000,0.000000,0.000000}%
\pgfsetstrokecolor{textcolor}%
\pgfsetfillcolor{textcolor}%
\pgftext[x=1.360215in, y=3.969367in, left, base]{\color{textcolor}\sffamily\fontsize{20.000000}{24.000000}\selectfont \(\displaystyle {2.5}\)}%
\end{pgfscope}%
\begin{pgfscope}%
\pgfpathrectangle{\pgfqpoint{1.800000in}{0.900000in}}{\pgfqpoint{8.400000in}{4.800000in}}%
\pgfusepath{clip}%
\pgfsetrectcap%
\pgfsetroundjoin%
\pgfsetlinewidth{0.803000pt}%
\definecolor{currentstroke}{rgb}{0.690196,0.690196,0.690196}%
\pgfsetstrokecolor{currentstroke}%
\pgfsetdash{}{0pt}%
\pgfpathmoveto{\pgfqpoint{1.800000in}{4.703263in}}%
\pgfpathlineto{\pgfqpoint{10.200000in}{4.703263in}}%
\pgfusepath{stroke}%
\end{pgfscope}%
\begin{pgfscope}%
\pgfsetbuttcap%
\pgfsetroundjoin%
\definecolor{currentfill}{rgb}{0.000000,0.000000,0.000000}%
\pgfsetfillcolor{currentfill}%
\pgfsetlinewidth{0.803000pt}%
\definecolor{currentstroke}{rgb}{0.000000,0.000000,0.000000}%
\pgfsetstrokecolor{currentstroke}%
\pgfsetdash{}{0pt}%
\pgfsys@defobject{currentmarker}{\pgfqpoint{-0.048611in}{0.000000in}}{\pgfqpoint{-0.000000in}{0.000000in}}{%
\pgfpathmoveto{\pgfqpoint{-0.000000in}{0.000000in}}%
\pgfpathlineto{\pgfqpoint{-0.048611in}{0.000000in}}%
\pgfusepath{stroke,fill}%
}%
\begin{pgfscope}%
\pgfsys@transformshift{1.800000in}{4.703263in}%
\pgfsys@useobject{currentmarker}{}%
\end{pgfscope}%
\end{pgfscope}%
\begin{pgfscope}%
\definecolor{textcolor}{rgb}{0.000000,0.000000,0.000000}%
\pgfsetstrokecolor{textcolor}%
\pgfsetfillcolor{textcolor}%
\pgftext[x=1.360215in, y=4.603244in, left, base]{\color{textcolor}\sffamily\fontsize{20.000000}{24.000000}\selectfont \(\displaystyle {3.0}\)}%
\end{pgfscope}%
\begin{pgfscope}%
\pgfpathrectangle{\pgfqpoint{1.800000in}{0.900000in}}{\pgfqpoint{8.400000in}{4.800000in}}%
\pgfusepath{clip}%
\pgfsetrectcap%
\pgfsetroundjoin%
\pgfsetlinewidth{0.803000pt}%
\definecolor{currentstroke}{rgb}{0.690196,0.690196,0.690196}%
\pgfsetstrokecolor{currentstroke}%
\pgfsetdash{}{0pt}%
\pgfpathmoveto{\pgfqpoint{1.800000in}{5.337140in}}%
\pgfpathlineto{\pgfqpoint{10.200000in}{5.337140in}}%
\pgfusepath{stroke}%
\end{pgfscope}%
\begin{pgfscope}%
\pgfsetbuttcap%
\pgfsetroundjoin%
\definecolor{currentfill}{rgb}{0.000000,0.000000,0.000000}%
\pgfsetfillcolor{currentfill}%
\pgfsetlinewidth{0.803000pt}%
\definecolor{currentstroke}{rgb}{0.000000,0.000000,0.000000}%
\pgfsetstrokecolor{currentstroke}%
\pgfsetdash{}{0pt}%
\pgfsys@defobject{currentmarker}{\pgfqpoint{-0.048611in}{0.000000in}}{\pgfqpoint{-0.000000in}{0.000000in}}{%
\pgfpathmoveto{\pgfqpoint{-0.000000in}{0.000000in}}%
\pgfpathlineto{\pgfqpoint{-0.048611in}{0.000000in}}%
\pgfusepath{stroke,fill}%
}%
\begin{pgfscope}%
\pgfsys@transformshift{1.800000in}{5.337140in}%
\pgfsys@useobject{currentmarker}{}%
\end{pgfscope}%
\end{pgfscope}%
\begin{pgfscope}%
\definecolor{textcolor}{rgb}{0.000000,0.000000,0.000000}%
\pgfsetstrokecolor{textcolor}%
\pgfsetfillcolor{textcolor}%
\pgftext[x=1.360215in, y=5.237121in, left, base]{\color{textcolor}\sffamily\fontsize{20.000000}{24.000000}\selectfont \(\displaystyle {3.5}\)}%
\end{pgfscope}%
\begin{pgfscope}%
\definecolor{textcolor}{rgb}{0.000000,0.000000,0.000000}%
\pgfsetstrokecolor{textcolor}%
\pgfsetfillcolor{textcolor}%
\pgftext[x=1.304660in,y=3.300000in,,bottom,rotate=90.000000]{\color{textcolor}\sffamily\fontsize{20.000000}{24.000000}\selectfont \(\displaystyle \mathrm{Normalized\ Count}\)}%
\end{pgfscope}%
\begin{pgfscope}%
\definecolor{textcolor}{rgb}{0.000000,0.000000,0.000000}%
\pgfsetstrokecolor{textcolor}%
\pgfsetfillcolor{textcolor}%
\pgftext[x=1.800000in,y=5.741667in,left,base]{\color{textcolor}\sffamily\fontsize{20.000000}{24.000000}\selectfont \(\displaystyle \times{10^{-2}}{}\)}%
\end{pgfscope}%
\begin{pgfscope}%
\pgfpathrectangle{\pgfqpoint{1.800000in}{0.900000in}}{\pgfqpoint{8.400000in}{4.800000in}}%
\pgfusepath{clip}%
\pgfsetbuttcap%
\pgfsetmiterjoin%
\pgfsetlinewidth{2.007500pt}%
\definecolor{currentstroke}{rgb}{0.000000,0.000000,1.000000}%
\pgfsetstrokecolor{currentstroke}%
\pgfsetdash{}{0pt}%
\pgfpathmoveto{\pgfqpoint{1.801920in}{0.900000in}}%
\pgfpathlineto{\pgfqpoint{2.281920in}{0.900577in}}%
\pgfpathlineto{\pgfqpoint{2.281920in}{0.901731in}}%
\pgfpathlineto{\pgfqpoint{2.401920in}{0.901731in}}%
\pgfpathlineto{\pgfqpoint{2.401920in}{0.904039in}}%
\pgfpathlineto{\pgfqpoint{2.521920in}{0.904039in}}%
\pgfpathlineto{\pgfqpoint{2.521920in}{0.907501in}}%
\pgfpathlineto{\pgfqpoint{2.641920in}{0.907501in}}%
\pgfpathlineto{\pgfqpoint{2.641920in}{0.913847in}}%
\pgfpathlineto{\pgfqpoint{2.761920in}{0.913847in}}%
\pgfpathlineto{\pgfqpoint{2.761920in}{0.930580in}}%
\pgfpathlineto{\pgfqpoint{2.881920in}{0.930580in}}%
\pgfpathlineto{\pgfqpoint{2.881920in}{0.947889in}}%
\pgfpathlineto{\pgfqpoint{3.001920in}{0.947889in}}%
\pgfpathlineto{\pgfqpoint{3.001920in}{0.960005in}}%
\pgfpathlineto{\pgfqpoint{3.121920in}{0.960005in}}%
\pgfpathlineto{\pgfqpoint{3.121920in}{0.978468in}}%
\pgfpathlineto{\pgfqpoint{3.241920in}{0.978468in}}%
\pgfpathlineto{\pgfqpoint{3.241920in}{1.026357in}}%
\pgfpathlineto{\pgfqpoint{3.361920in}{1.026357in}}%
\pgfpathlineto{\pgfqpoint{3.361920in}{1.038474in}}%
\pgfpathlineto{\pgfqpoint{3.481920in}{1.038474in}}%
\pgfpathlineto{\pgfqpoint{3.481920in}{1.082324in}}%
\pgfpathlineto{\pgfqpoint{3.601920in}{1.082324in}}%
\pgfpathlineto{\pgfqpoint{3.601920in}{1.156176in}}%
\pgfpathlineto{\pgfqpoint{3.721920in}{1.156176in}}%
\pgfpathlineto{\pgfqpoint{3.721920in}{1.188487in}}%
\pgfpathlineto{\pgfqpoint{3.841920in}{1.188487in}}%
\pgfpathlineto{\pgfqpoint{3.841920in}{1.265801in}}%
\pgfpathlineto{\pgfqpoint{3.961920in}{1.265801in}}%
\pgfpathlineto{\pgfqpoint{3.961920in}{1.321190in}}%
\pgfpathlineto{\pgfqpoint{4.081920in}{1.321190in}}%
\pgfpathlineto{\pgfqpoint{4.081920in}{1.406006in}}%
\pgfpathlineto{\pgfqpoint{4.201920in}{1.406006in}}%
\pgfpathlineto{\pgfqpoint{4.201920in}{1.517361in}}%
\pgfpathlineto{\pgfqpoint{4.321920in}{1.517361in}}%
\pgfpathlineto{\pgfqpoint{4.321920in}{1.650065in}}%
\pgfpathlineto{\pgfqpoint{4.441920in}{1.650065in}}%
\pgfpathlineto{\pgfqpoint{4.441920in}{1.738342in}}%
\pgfpathlineto{\pgfqpoint{4.561920in}{1.738342in}}%
\pgfpathlineto{\pgfqpoint{4.561920in}{1.885470in}}%
\pgfpathlineto{\pgfqpoint{4.681920in}{1.885470in}}%
\pgfpathlineto{\pgfqpoint{4.681920in}{1.901625in}}%
\pgfpathlineto{\pgfqpoint{4.801920in}{1.901625in}}%
\pgfpathlineto{\pgfqpoint{4.801920in}{2.006058in}}%
\pgfpathlineto{\pgfqpoint{4.921920in}{2.006058in}}%
\pgfpathlineto{\pgfqpoint{4.921920in}{2.147993in}}%
\pgfpathlineto{\pgfqpoint{5.041920in}{2.147993in}}%
\pgfpathlineto{\pgfqpoint{5.041920in}{2.202806in}}%
\pgfpathlineto{\pgfqpoint{5.161920in}{2.202806in}}%
\pgfpathlineto{\pgfqpoint{5.161920in}{2.268003in}}%
\pgfpathlineto{\pgfqpoint{5.281920in}{2.268003in}}%
\pgfpathlineto{\pgfqpoint{5.281920in}{2.227615in}}%
\pgfpathlineto{\pgfqpoint{5.401920in}{2.227615in}}%
\pgfpathlineto{\pgfqpoint{5.401920in}{2.253579in}}%
\pgfpathlineto{\pgfqpoint{5.521920in}{2.253579in}}%
\pgfpathlineto{\pgfqpoint{5.521920in}{2.174534in}}%
\pgfpathlineto{\pgfqpoint{5.641920in}{2.174534in}}%
\pgfpathlineto{\pgfqpoint{5.641920in}{2.134723in}}%
\pgfpathlineto{\pgfqpoint{5.761920in}{2.134723in}}%
\pgfpathlineto{\pgfqpoint{5.761920in}{2.094335in}}%
\pgfpathlineto{\pgfqpoint{5.881920in}{2.094335in}}%
\pgfpathlineto{\pgfqpoint{5.881920in}{2.003750in}}%
\pgfpathlineto{\pgfqpoint{6.001920in}{2.003750in}}%
\pgfpathlineto{\pgfqpoint{6.001920in}{1.909126in}}%
\pgfpathlineto{\pgfqpoint{6.121920in}{1.909126in}}%
\pgfpathlineto{\pgfqpoint{6.121920in}{1.837004in}}%
\pgfpathlineto{\pgfqpoint{6.241920in}{1.837004in}}%
\pgfpathlineto{\pgfqpoint{6.241920in}{1.689299in}}%
\pgfpathlineto{\pgfqpoint{6.361920in}{1.689299in}}%
\pgfpathlineto{\pgfqpoint{6.361920in}{1.605638in}}%
\pgfpathlineto{\pgfqpoint{6.481920in}{1.605638in}}%
\pgfpathlineto{\pgfqpoint{6.481920in}{1.491398in}}%
\pgfpathlineto{\pgfqpoint{6.601920in}{1.491398in}}%
\pgfpathlineto{\pgfqpoint{6.601920in}{1.385811in}}%
\pgfpathlineto{\pgfqpoint{6.721920in}{1.385811in}}%
\pgfpathlineto{\pgfqpoint{6.721920in}{1.297535in}}%
\pgfpathlineto{\pgfqpoint{6.841920in}{1.297535in}}%
\pgfpathlineto{\pgfqpoint{6.841920in}{1.234067in}}%
\pgfpathlineto{\pgfqpoint{6.961920in}{1.234067in}}%
\pgfpathlineto{\pgfqpoint{6.961920in}{1.140598in}}%
\pgfpathlineto{\pgfqpoint{7.081920in}{1.140598in}}%
\pgfpathlineto{\pgfqpoint{7.081920in}{1.122135in}}%
\pgfpathlineto{\pgfqpoint{7.201920in}{1.122135in}}%
\pgfpathlineto{\pgfqpoint{7.201920in}{1.040204in}}%
\pgfpathlineto{\pgfqpoint{7.321920in}{1.040204in}}%
\pgfpathlineto{\pgfqpoint{7.321920in}{1.001547in}}%
\pgfpathlineto{\pgfqpoint{7.441920in}{1.001547in}}%
\pgfpathlineto{\pgfqpoint{7.441920in}{0.976737in}}%
\pgfpathlineto{\pgfqpoint{7.561920in}{0.976737in}}%
\pgfpathlineto{\pgfqpoint{7.561920in}{0.943850in}}%
\pgfpathlineto{\pgfqpoint{7.681920in}{0.943850in}}%
\pgfpathlineto{\pgfqpoint{7.681920in}{0.925387in}}%
\pgfpathlineto{\pgfqpoint{7.801920in}{0.925387in}}%
\pgfpathlineto{\pgfqpoint{7.801920in}{0.905770in}}%
\pgfpathlineto{\pgfqpoint{7.921920in}{0.905770in}}%
\pgfpathlineto{\pgfqpoint{7.921920in}{0.903462in}}%
\pgfpathlineto{\pgfqpoint{8.041920in}{0.903462in}}%
\pgfpathlineto{\pgfqpoint{8.041920in}{0.901154in}}%
\pgfpathlineto{\pgfqpoint{8.161920in}{0.901154in}}%
\pgfpathlineto{\pgfqpoint{8.161920in}{0.903462in}}%
\pgfpathlineto{\pgfqpoint{8.281920in}{0.903462in}}%
\pgfpathlineto{\pgfqpoint{8.281920in}{0.901154in}}%
\pgfpathlineto{\pgfqpoint{8.401920in}{0.901154in}}%
\pgfpathlineto{\pgfqpoint{8.401920in}{0.900000in}}%
\pgfpathlineto{\pgfqpoint{8.521920in}{0.900000in}}%
\pgfpathlineto{\pgfqpoint{8.521920in}{0.901154in}}%
\pgfpathlineto{\pgfqpoint{8.881920in}{0.901154in}}%
\pgfpathlineto{\pgfqpoint{8.881920in}{0.900000in}}%
\pgfpathlineto{\pgfqpoint{10.210000in}{0.900000in}}%
\pgfpathlineto{\pgfqpoint{10.210000in}{0.900000in}}%
\pgfusepath{stroke}%
\end{pgfscope}%
\begin{pgfscope}%
\pgfpathrectangle{\pgfqpoint{1.800000in}{0.900000in}}{\pgfqpoint{8.400000in}{4.800000in}}%
\pgfusepath{clip}%
\pgfsetbuttcap%
\pgfsetmiterjoin%
\pgfsetlinewidth{2.007500pt}%
\definecolor{currentstroke}{rgb}{0.750000,0.750000,0.000000}%
\pgfsetstrokecolor{currentstroke}%
\pgfsetdash{}{0pt}%
\pgfpathmoveto{\pgfqpoint{1.801920in}{0.900000in}}%
\pgfpathlineto{\pgfqpoint{2.041920in}{0.900277in}}%
\pgfpathlineto{\pgfqpoint{2.041920in}{0.943151in}}%
\pgfpathlineto{\pgfqpoint{2.161920in}{0.943151in}}%
\pgfpathlineto{\pgfqpoint{2.161920in}{3.574562in}}%
\pgfpathlineto{\pgfqpoint{2.281920in}{3.574562in}}%
\pgfpathlineto{\pgfqpoint{2.281920in}{5.471429in}}%
\pgfpathlineto{\pgfqpoint{2.401920in}{5.471429in}}%
\pgfpathlineto{\pgfqpoint{2.401920in}{5.430905in}}%
\pgfpathlineto{\pgfqpoint{2.521920in}{5.430905in}}%
\pgfpathlineto{\pgfqpoint{2.521920in}{4.694425in}}%
\pgfpathlineto{\pgfqpoint{2.641920in}{4.694425in}}%
\pgfpathlineto{\pgfqpoint{2.641920in}{3.653396in}}%
\pgfpathlineto{\pgfqpoint{2.761920in}{3.653396in}}%
\pgfpathlineto{\pgfqpoint{2.761920in}{2.768653in}}%
\pgfpathlineto{\pgfqpoint{2.881920in}{2.768653in}}%
\pgfpathlineto{\pgfqpoint{2.881920in}{2.176426in}}%
\pgfpathlineto{\pgfqpoint{3.001920in}{2.176426in}}%
\pgfpathlineto{\pgfqpoint{3.001920in}{1.890824in}}%
\pgfpathlineto{\pgfqpoint{3.121920in}{1.890824in}}%
\pgfpathlineto{\pgfqpoint{3.121920in}{1.686270in}}%
\pgfpathlineto{\pgfqpoint{3.241920in}{1.686270in}}%
\pgfpathlineto{\pgfqpoint{3.241920in}{1.482268in}}%
\pgfpathlineto{\pgfqpoint{3.361920in}{1.482268in}}%
\pgfpathlineto{\pgfqpoint{3.361920in}{1.322663in}}%
\pgfpathlineto{\pgfqpoint{3.481920in}{1.322663in}}%
\pgfpathlineto{\pgfqpoint{3.481920in}{1.208422in}}%
\pgfpathlineto{\pgfqpoint{3.601920in}{1.208422in}}%
\pgfpathlineto{\pgfqpoint{3.601920in}{1.104831in}}%
\pgfpathlineto{\pgfqpoint{3.721920in}{1.104831in}}%
\pgfpathlineto{\pgfqpoint{3.721920in}{1.053243in}}%
\pgfpathlineto{\pgfqpoint{3.841920in}{1.053243in}}%
\pgfpathlineto{\pgfqpoint{3.841920in}{1.015486in}}%
\pgfpathlineto{\pgfqpoint{3.961920in}{1.015486in}}%
\pgfpathlineto{\pgfqpoint{3.961920in}{0.983399in}}%
\pgfpathlineto{\pgfqpoint{4.081920in}{0.983399in}}%
\pgfpathlineto{\pgfqpoint{4.081920in}{0.962653in}}%
\pgfpathlineto{\pgfqpoint{4.201920in}{0.962653in}}%
\pgfpathlineto{\pgfqpoint{4.201920in}{0.937066in}}%
\pgfpathlineto{\pgfqpoint{4.321920in}{0.937066in}}%
\pgfpathlineto{\pgfqpoint{4.321920in}{0.931534in}}%
\pgfpathlineto{\pgfqpoint{4.441920in}{0.931534in}}%
\pgfpathlineto{\pgfqpoint{4.441920in}{0.920884in}}%
\pgfpathlineto{\pgfqpoint{4.561920in}{0.920884in}}%
\pgfpathlineto{\pgfqpoint{4.561920in}{0.913692in}}%
\pgfpathlineto{\pgfqpoint{4.681920in}{0.913692in}}%
\pgfpathlineto{\pgfqpoint{4.681920in}{0.908990in}}%
\pgfpathlineto{\pgfqpoint{4.801920in}{0.908990in}}%
\pgfpathlineto{\pgfqpoint{4.801920in}{0.906362in}}%
\pgfpathlineto{\pgfqpoint{4.921920in}{0.906362in}}%
\pgfpathlineto{\pgfqpoint{4.921920in}{0.904979in}}%
\pgfpathlineto{\pgfqpoint{5.041920in}{0.904979in}}%
\pgfpathlineto{\pgfqpoint{5.041920in}{0.902075in}}%
\pgfpathlineto{\pgfqpoint{5.521920in}{0.900968in}}%
\pgfpathlineto{\pgfqpoint{5.521920in}{0.900830in}}%
\pgfpathlineto{\pgfqpoint{10.210000in}{0.900000in}}%
\pgfpathlineto{\pgfqpoint{10.210000in}{0.900000in}}%
\pgfusepath{stroke}%
\end{pgfscope}%
\begin{pgfscope}%
\pgfpathrectangle{\pgfqpoint{1.800000in}{0.900000in}}{\pgfqpoint{8.400000in}{4.800000in}}%
\pgfusepath{clip}%
\pgfsetrectcap%
\pgfsetroundjoin%
\pgfsetlinewidth{2.007500pt}%
\definecolor{currentstroke}{rgb}{1.000000,0.000000,0.000000}%
\pgfsetstrokecolor{currentstroke}%
\pgfsetdash{}{0pt}%
\pgfpathmoveto{\pgfqpoint{1.800000in}{0.900424in}}%
\pgfpathlineto{\pgfqpoint{2.227200in}{0.902278in}}%
\pgfpathlineto{\pgfqpoint{2.472000in}{0.905460in}}%
\pgfpathlineto{\pgfqpoint{2.652000in}{0.909960in}}%
\pgfpathlineto{\pgfqpoint{2.796000in}{0.915708in}}%
\pgfpathlineto{\pgfqpoint{2.918400in}{0.922732in}}%
\pgfpathlineto{\pgfqpoint{3.026400in}{0.931074in}}%
\pgfpathlineto{\pgfqpoint{3.124800in}{0.940861in}}%
\pgfpathlineto{\pgfqpoint{3.216000in}{0.952173in}}%
\pgfpathlineto{\pgfqpoint{3.300000in}{0.964824in}}%
\pgfpathlineto{\pgfqpoint{3.379200in}{0.978994in}}%
\pgfpathlineto{\pgfqpoint{3.453600in}{0.994527in}}%
\pgfpathlineto{\pgfqpoint{3.525600in}{1.011819in}}%
\pgfpathlineto{\pgfqpoint{3.595200in}{1.030835in}}%
\pgfpathlineto{\pgfqpoint{3.662400in}{1.051500in}}%
\pgfpathlineto{\pgfqpoint{3.727200in}{1.073705in}}%
\pgfpathlineto{\pgfqpoint{3.792000in}{1.098259in}}%
\pgfpathlineto{\pgfqpoint{3.856800in}{1.125255in}}%
\pgfpathlineto{\pgfqpoint{3.919200in}{1.153625in}}%
\pgfpathlineto{\pgfqpoint{3.981600in}{1.184365in}}%
\pgfpathlineto{\pgfqpoint{4.046400in}{1.218807in}}%
\pgfpathlineto{\pgfqpoint{4.111200in}{1.255796in}}%
\pgfpathlineto{\pgfqpoint{4.178400in}{1.296780in}}%
\pgfpathlineto{\pgfqpoint{4.248000in}{1.341923in}}%
\pgfpathlineto{\pgfqpoint{4.320000in}{1.391309in}}%
\pgfpathlineto{\pgfqpoint{4.396800in}{1.446686in}}%
\pgfpathlineto{\pgfqpoint{4.483200in}{1.511781in}}%
\pgfpathlineto{\pgfqpoint{4.588800in}{1.594279in}}%
\pgfpathlineto{\pgfqpoint{4.900800in}{1.840053in}}%
\pgfpathlineto{\pgfqpoint{4.977600in}{1.896587in}}%
\pgfpathlineto{\pgfqpoint{5.044800in}{1.943344in}}%
\pgfpathlineto{\pgfqpoint{5.104800in}{1.982451in}}%
\pgfpathlineto{\pgfqpoint{5.160000in}{2.015866in}}%
\pgfpathlineto{\pgfqpoint{5.212800in}{2.045245in}}%
\pgfpathlineto{\pgfqpoint{5.263200in}{2.070701in}}%
\pgfpathlineto{\pgfqpoint{5.311200in}{2.092412in}}%
\pgfpathlineto{\pgfqpoint{5.356800in}{2.110603in}}%
\pgfpathlineto{\pgfqpoint{5.400000in}{2.125539in}}%
\pgfpathlineto{\pgfqpoint{5.443200in}{2.138150in}}%
\pgfpathlineto{\pgfqpoint{5.486400in}{2.148360in}}%
\pgfpathlineto{\pgfqpoint{5.527200in}{2.155744in}}%
\pgfpathlineto{\pgfqpoint{5.568000in}{2.160891in}}%
\pgfpathlineto{\pgfqpoint{5.608800in}{2.163775in}}%
\pgfpathlineto{\pgfqpoint{5.649600in}{2.164379in}}%
\pgfpathlineto{\pgfqpoint{5.690400in}{2.162701in}}%
\pgfpathlineto{\pgfqpoint{5.731200in}{2.158749in}}%
\pgfpathlineto{\pgfqpoint{5.772000in}{2.152545in}}%
\pgfpathlineto{\pgfqpoint{5.812800in}{2.144123in}}%
\pgfpathlineto{\pgfqpoint{5.856000in}{2.132838in}}%
\pgfpathlineto{\pgfqpoint{5.899200in}{2.119183in}}%
\pgfpathlineto{\pgfqpoint{5.942400in}{2.103241in}}%
\pgfpathlineto{\pgfqpoint{5.988000in}{2.084035in}}%
\pgfpathlineto{\pgfqpoint{6.036000in}{2.061315in}}%
\pgfpathlineto{\pgfqpoint{6.086400in}{2.034870in}}%
\pgfpathlineto{\pgfqpoint{6.139200in}{2.004542in}}%
\pgfpathlineto{\pgfqpoint{6.194400in}{1.970235in}}%
\pgfpathlineto{\pgfqpoint{6.254400in}{1.930280in}}%
\pgfpathlineto{\pgfqpoint{6.321600in}{1.882732in}}%
\pgfpathlineto{\pgfqpoint{6.398400in}{1.825502in}}%
\pgfpathlineto{\pgfqpoint{6.492000in}{1.752828in}}%
\pgfpathlineto{\pgfqpoint{6.681600in}{1.601888in}}%
\pgfpathlineto{\pgfqpoint{6.801600in}{1.508098in}}%
\pgfpathlineto{\pgfqpoint{6.892800in}{1.439625in}}%
\pgfpathlineto{\pgfqpoint{6.972000in}{1.382899in}}%
\pgfpathlineto{\pgfqpoint{7.046400in}{1.332367in}}%
\pgfpathlineto{\pgfqpoint{7.116000in}{1.287778in}}%
\pgfpathlineto{\pgfqpoint{7.183200in}{1.247359in}}%
\pgfpathlineto{\pgfqpoint{7.248000in}{1.210932in}}%
\pgfpathlineto{\pgfqpoint{7.312800in}{1.177060in}}%
\pgfpathlineto{\pgfqpoint{7.377600in}{1.145755in}}%
\pgfpathlineto{\pgfqpoint{7.442400in}{1.116997in}}%
\pgfpathlineto{\pgfqpoint{7.507200in}{1.090733in}}%
\pgfpathlineto{\pgfqpoint{7.572000in}{1.066886in}}%
\pgfpathlineto{\pgfqpoint{7.636800in}{1.045356in}}%
\pgfpathlineto{\pgfqpoint{7.704000in}{1.025353in}}%
\pgfpathlineto{\pgfqpoint{7.773600in}{1.006979in}}%
\pgfpathlineto{\pgfqpoint{7.845600in}{0.990300in}}%
\pgfpathlineto{\pgfqpoint{7.920000in}{0.975345in}}%
\pgfpathlineto{\pgfqpoint{7.999200in}{0.961728in}}%
\pgfpathlineto{\pgfqpoint{8.083200in}{0.949594in}}%
\pgfpathlineto{\pgfqpoint{8.172000in}{0.939024in}}%
\pgfpathlineto{\pgfqpoint{8.268000in}{0.929828in}}%
\pgfpathlineto{\pgfqpoint{8.376000in}{0.921783in}}%
\pgfpathlineto{\pgfqpoint{8.496000in}{0.915136in}}%
\pgfpathlineto{\pgfqpoint{8.635200in}{0.909730in}}%
\pgfpathlineto{\pgfqpoint{8.800800in}{0.905596in}}%
\pgfpathlineto{\pgfqpoint{9.012000in}{0.902647in}}%
\pgfpathlineto{\pgfqpoint{9.314400in}{0.900833in}}%
\pgfpathlineto{\pgfqpoint{9.878400in}{0.900074in}}%
\pgfpathlineto{\pgfqpoint{10.202400in}{0.900016in}}%
\pgfpathlineto{\pgfqpoint{10.202400in}{0.900016in}}%
\pgfusepath{stroke}%
\end{pgfscope}%
\begin{pgfscope}%
\pgfsetrectcap%
\pgfsetmiterjoin%
\pgfsetlinewidth{0.803000pt}%
\definecolor{currentstroke}{rgb}{0.000000,0.000000,0.000000}%
\pgfsetstrokecolor{currentstroke}%
\pgfsetdash{}{0pt}%
\pgfpathmoveto{\pgfqpoint{1.800000in}{0.900000in}}%
\pgfpathlineto{\pgfqpoint{1.800000in}{5.700000in}}%
\pgfusepath{stroke}%
\end{pgfscope}%
\begin{pgfscope}%
\pgfsetrectcap%
\pgfsetmiterjoin%
\pgfsetlinewidth{0.803000pt}%
\definecolor{currentstroke}{rgb}{0.000000,0.000000,0.000000}%
\pgfsetstrokecolor{currentstroke}%
\pgfsetdash{}{0pt}%
\pgfpathmoveto{\pgfqpoint{10.200000in}{0.900000in}}%
\pgfpathlineto{\pgfqpoint{10.200000in}{5.700000in}}%
\pgfusepath{stroke}%
\end{pgfscope}%
\begin{pgfscope}%
\pgfsetrectcap%
\pgfsetmiterjoin%
\pgfsetlinewidth{0.803000pt}%
\definecolor{currentstroke}{rgb}{0.000000,0.000000,0.000000}%
\pgfsetstrokecolor{currentstroke}%
\pgfsetdash{}{0pt}%
\pgfpathmoveto{\pgfqpoint{1.800000in}{0.900000in}}%
\pgfpathlineto{\pgfqpoint{10.200000in}{0.900000in}}%
\pgfusepath{stroke}%
\end{pgfscope}%
\begin{pgfscope}%
\pgfsetrectcap%
\pgfsetmiterjoin%
\pgfsetlinewidth{0.803000pt}%
\definecolor{currentstroke}{rgb}{0.000000,0.000000,0.000000}%
\pgfsetstrokecolor{currentstroke}%
\pgfsetdash{}{0pt}%
\pgfpathmoveto{\pgfqpoint{1.800000in}{5.700000in}}%
\pgfpathlineto{\pgfqpoint{10.200000in}{5.700000in}}%
\pgfusepath{stroke}%
\end{pgfscope}%
\begin{pgfscope}%
\pgfsetbuttcap%
\pgfsetmiterjoin%
\definecolor{currentfill}{rgb}{1.000000,1.000000,1.000000}%
\pgfsetfillcolor{currentfill}%
\pgfsetfillopacity{0.800000}%
\pgfsetlinewidth{1.003750pt}%
\definecolor{currentstroke}{rgb}{0.800000,0.800000,0.800000}%
\pgfsetstrokecolor{currentstroke}%
\pgfsetstrokeopacity{0.800000}%
\pgfsetdash{}{0pt}%
\pgfpathmoveto{\pgfqpoint{7.012787in}{4.292908in}}%
\pgfpathlineto{\pgfqpoint{10.005556in}{4.292908in}}%
\pgfpathquadraticcurveto{\pgfqpoint{10.061111in}{4.292908in}}{\pgfqpoint{10.061111in}{4.348464in}}%
\pgfpathlineto{\pgfqpoint{10.061111in}{5.505556in}}%
\pgfpathquadraticcurveto{\pgfqpoint{10.061111in}{5.561111in}}{\pgfqpoint{10.005556in}{5.561111in}}%
\pgfpathlineto{\pgfqpoint{7.012787in}{5.561111in}}%
\pgfpathquadraticcurveto{\pgfqpoint{6.957231in}{5.561111in}}{\pgfqpoint{6.957231in}{5.505556in}}%
\pgfpathlineto{\pgfqpoint{6.957231in}{4.348464in}}%
\pgfpathquadraticcurveto{\pgfqpoint{6.957231in}{4.292908in}}{\pgfqpoint{7.012787in}{4.292908in}}%
\pgfpathclose%
\pgfusepath{stroke,fill}%
\end{pgfscope}%
\begin{pgfscope}%
\pgfsetrectcap%
\pgfsetroundjoin%
\pgfsetlinewidth{2.007500pt}%
\definecolor{currentstroke}{rgb}{1.000000,0.000000,0.000000}%
\pgfsetstrokecolor{currentstroke}%
\pgfsetdash{}{0pt}%
\pgfpathmoveto{\pgfqpoint{7.068342in}{5.347184in}}%
\pgfpathlineto{\pgfqpoint{7.623898in}{5.347184in}}%
\pgfusepath{stroke}%
\end{pgfscope}%
\begin{pgfscope}%
\definecolor{textcolor}{rgb}{0.000000,0.000000,0.000000}%
\pgfsetstrokecolor{textcolor}%
\pgfsetfillcolor{textcolor}%
\pgftext[x=7.846120in,y=5.249962in,left,base]{\color{textcolor}\sffamily\fontsize{20.000000}{24.000000}\selectfont \(\displaystyle \mathrm{Input\ PDF}\)}%
\end{pgfscope}%
\begin{pgfscope}%
\pgfsetbuttcap%
\pgfsetmiterjoin%
\pgfsetlinewidth{2.007500pt}%
\definecolor{currentstroke}{rgb}{0.000000,0.000000,1.000000}%
\pgfsetstrokecolor{currentstroke}%
\pgfsetdash{}{0pt}%
\pgfpathmoveto{\pgfqpoint{7.068342in}{4.855005in}}%
\pgfpathlineto{\pgfqpoint{7.623898in}{4.855005in}}%
\pgfpathlineto{\pgfqpoint{7.623898in}{5.049449in}}%
\pgfpathlineto{\pgfqpoint{7.068342in}{5.049449in}}%
\pgfpathclose%
\pgfusepath{stroke}%
\end{pgfscope}%
\begin{pgfscope}%
\definecolor{textcolor}{rgb}{0.000000,0.000000,0.000000}%
\pgfsetstrokecolor{textcolor}%
\pgfsetfillcolor{textcolor}%
\pgftext[x=7.846120in,y=4.855005in,left,base]{\color{textcolor}\sffamily\fontsize{20.000000}{24.000000}\selectfont \(\displaystyle \mathrm{FBMP}\)}%
\end{pgfscope}%
\begin{pgfscope}%
\pgfsetbuttcap%
\pgfsetmiterjoin%
\pgfsetlinewidth{2.007500pt}%
\definecolor{currentstroke}{rgb}{0.750000,0.750000,0.000000}%
\pgfsetstrokecolor{currentstroke}%
\pgfsetdash{}{0pt}%
\pgfpathmoveto{\pgfqpoint{7.068342in}{4.460048in}}%
\pgfpathlineto{\pgfqpoint{7.623898in}{4.460048in}}%
\pgfpathlineto{\pgfqpoint{7.623898in}{4.654493in}}%
\pgfpathlineto{\pgfqpoint{7.068342in}{4.654493in}}%
\pgfpathclose%
\pgfusepath{stroke}%
\end{pgfscope}%
\begin{pgfscope}%
\definecolor{textcolor}{rgb}{0.000000,0.000000,0.000000}%
\pgfsetstrokecolor{textcolor}%
\pgfsetfillcolor{textcolor}%
\pgftext[x=7.846120in,y=4.460048in,left,base]{\color{textcolor}\sffamily\fontsize{20.000000}{24.000000}\selectfont \(\displaystyle \mathrm{Fourier\ Transform}\)}%
\end{pgfscope}%
\end{pgfpicture}%
\makeatother%
\endgroup%
}
    \caption{$\hat{q}$ histogram of methods}
\end{figure}
\begin{block}{}
FBMP retains charge distribution of PE. 
\end{block}
\end{frame}

\begin{frame}
\frametitle{Timing \& charge resolution}
For dataset: $(\tau_l, \sigma_l)/\si{ns}=(20, 5)$:
\begin{figure}
    \centering
    \resizebox{\textwidth}{!}{%% Creator: Matplotlib, PGF backend
%%
%% To include the figure in your LaTeX document, write
%%   \input{<filename>.pgf}
%%
%% Make sure the required packages are loaded in your preamble
%%   \usepackage{pgf}
%%
%% and, on pdftex
%%   \usepackage[utf8]{inputenc}\DeclareUnicodeCharacter{2212}{-}
%%
%% or, on luatex and xetex
%%   \usepackage{unicode-math}
%%
%% Figures using additional raster images can only be included by \input if
%% they are in the same directory as the main LaTeX file. For loading figures
%% from other directories you can use the `import` package
%%   \usepackage{import}
%%
%% and then include the figures with
%%   \import{<path to file>}{<filename>.pgf}
%%
%% Matplotlib used the following preamble
%%   \usepackage[detect-all,locale=DE]{siunitx}
%%
\begingroup%
\makeatletter%
\begin{pgfpicture}%
\pgfpathrectangle{\pgfpointorigin}{\pgfqpoint{16.000000in}{6.000000in}}%
\pgfusepath{use as bounding box, clip}%
\begin{pgfscope}%
\pgfsetbuttcap%
\pgfsetmiterjoin%
\definecolor{currentfill}{rgb}{1.000000,1.000000,1.000000}%
\pgfsetfillcolor{currentfill}%
\pgfsetlinewidth{0.000000pt}%
\definecolor{currentstroke}{rgb}{1.000000,1.000000,1.000000}%
\pgfsetstrokecolor{currentstroke}%
\pgfsetdash{}{0pt}%
\pgfpathmoveto{\pgfqpoint{0.000000in}{0.000000in}}%
\pgfpathlineto{\pgfqpoint{16.000000in}{0.000000in}}%
\pgfpathlineto{\pgfqpoint{16.000000in}{6.000000in}}%
\pgfpathlineto{\pgfqpoint{0.000000in}{6.000000in}}%
\pgfpathclose%
\pgfusepath{fill}%
\end{pgfscope}%
\begin{pgfscope}%
\pgfsetbuttcap%
\pgfsetmiterjoin%
\definecolor{currentfill}{rgb}{1.000000,1.000000,1.000000}%
\pgfsetfillcolor{currentfill}%
\pgfsetlinewidth{0.000000pt}%
\definecolor{currentstroke}{rgb}{0.000000,0.000000,0.000000}%
\pgfsetstrokecolor{currentstroke}%
\pgfsetstrokeopacity{0.000000}%
\pgfsetdash{}{0pt}%
\pgfpathmoveto{\pgfqpoint{2.000000in}{0.720000in}}%
\pgfpathlineto{\pgfqpoint{7.166667in}{0.720000in}}%
\pgfpathlineto{\pgfqpoint{7.166667in}{5.340000in}}%
\pgfpathlineto{\pgfqpoint{2.000000in}{5.340000in}}%
\pgfpathclose%
\pgfusepath{fill}%
\end{pgfscope}%
\begin{pgfscope}%
\pgfpathrectangle{\pgfqpoint{2.000000in}{0.720000in}}{\pgfqpoint{5.166667in}{4.620000in}}%
\pgfusepath{clip}%
\pgfsetrectcap%
\pgfsetroundjoin%
\pgfsetlinewidth{0.803000pt}%
\definecolor{currentstroke}{rgb}{0.690196,0.690196,0.690196}%
\pgfsetstrokecolor{currentstroke}%
\pgfsetdash{}{0pt}%
\pgfpathmoveto{\pgfqpoint{2.155239in}{0.720000in}}%
\pgfpathlineto{\pgfqpoint{2.155239in}{5.340000in}}%
\pgfusepath{stroke}%
\end{pgfscope}%
\begin{pgfscope}%
\pgfsetbuttcap%
\pgfsetroundjoin%
\definecolor{currentfill}{rgb}{0.000000,0.000000,0.000000}%
\pgfsetfillcolor{currentfill}%
\pgfsetlinewidth{0.803000pt}%
\definecolor{currentstroke}{rgb}{0.000000,0.000000,0.000000}%
\pgfsetstrokecolor{currentstroke}%
\pgfsetdash{}{0pt}%
\pgfsys@defobject{currentmarker}{\pgfqpoint{0.000000in}{-0.048611in}}{\pgfqpoint{0.000000in}{0.000000in}}{%
\pgfpathmoveto{\pgfqpoint{0.000000in}{0.000000in}}%
\pgfpathlineto{\pgfqpoint{0.000000in}{-0.048611in}}%
\pgfusepath{stroke,fill}%
}%
\begin{pgfscope}%
\pgfsys@transformshift{2.155239in}{0.720000in}%
\pgfsys@useobject{currentmarker}{}%
\end{pgfscope}%
\end{pgfscope}%
\begin{pgfscope}%
\definecolor{textcolor}{rgb}{0.000000,0.000000,0.000000}%
\pgfsetstrokecolor{textcolor}%
\pgfsetfillcolor{textcolor}%
\pgftext[x=2.155239in,y=0.622778in,,top]{\color{textcolor}\sffamily\fontsize{20.000000}{24.000000}\selectfont \(\displaystyle {0}\)}%
\end{pgfscope}%
\begin{pgfscope}%
\pgfpathrectangle{\pgfqpoint{2.000000in}{0.720000in}}{\pgfqpoint{5.166667in}{4.620000in}}%
\pgfusepath{clip}%
\pgfsetrectcap%
\pgfsetroundjoin%
\pgfsetlinewidth{0.803000pt}%
\definecolor{currentstroke}{rgb}{0.690196,0.690196,0.690196}%
\pgfsetstrokecolor{currentstroke}%
\pgfsetdash{}{0pt}%
\pgfpathmoveto{\pgfqpoint{3.747432in}{0.720000in}}%
\pgfpathlineto{\pgfqpoint{3.747432in}{5.340000in}}%
\pgfusepath{stroke}%
\end{pgfscope}%
\begin{pgfscope}%
\pgfsetbuttcap%
\pgfsetroundjoin%
\definecolor{currentfill}{rgb}{0.000000,0.000000,0.000000}%
\pgfsetfillcolor{currentfill}%
\pgfsetlinewidth{0.803000pt}%
\definecolor{currentstroke}{rgb}{0.000000,0.000000,0.000000}%
\pgfsetstrokecolor{currentstroke}%
\pgfsetdash{}{0pt}%
\pgfsys@defobject{currentmarker}{\pgfqpoint{0.000000in}{-0.048611in}}{\pgfqpoint{0.000000in}{0.000000in}}{%
\pgfpathmoveto{\pgfqpoint{0.000000in}{0.000000in}}%
\pgfpathlineto{\pgfqpoint{0.000000in}{-0.048611in}}%
\pgfusepath{stroke,fill}%
}%
\begin{pgfscope}%
\pgfsys@transformshift{3.747432in}{0.720000in}%
\pgfsys@useobject{currentmarker}{}%
\end{pgfscope}%
\end{pgfscope}%
\begin{pgfscope}%
\definecolor{textcolor}{rgb}{0.000000,0.000000,0.000000}%
\pgfsetstrokecolor{textcolor}%
\pgfsetfillcolor{textcolor}%
\pgftext[x=3.747432in,y=0.622778in,,top]{\color{textcolor}\sffamily\fontsize{20.000000}{24.000000}\selectfont \(\displaystyle {10}\)}%
\end{pgfscope}%
\begin{pgfscope}%
\pgfpathrectangle{\pgfqpoint{2.000000in}{0.720000in}}{\pgfqpoint{5.166667in}{4.620000in}}%
\pgfusepath{clip}%
\pgfsetrectcap%
\pgfsetroundjoin%
\pgfsetlinewidth{0.803000pt}%
\definecolor{currentstroke}{rgb}{0.690196,0.690196,0.690196}%
\pgfsetstrokecolor{currentstroke}%
\pgfsetdash{}{0pt}%
\pgfpathmoveto{\pgfqpoint{5.339625in}{0.720000in}}%
\pgfpathlineto{\pgfqpoint{5.339625in}{5.340000in}}%
\pgfusepath{stroke}%
\end{pgfscope}%
\begin{pgfscope}%
\pgfsetbuttcap%
\pgfsetroundjoin%
\definecolor{currentfill}{rgb}{0.000000,0.000000,0.000000}%
\pgfsetfillcolor{currentfill}%
\pgfsetlinewidth{0.803000pt}%
\definecolor{currentstroke}{rgb}{0.000000,0.000000,0.000000}%
\pgfsetstrokecolor{currentstroke}%
\pgfsetdash{}{0pt}%
\pgfsys@defobject{currentmarker}{\pgfqpoint{0.000000in}{-0.048611in}}{\pgfqpoint{0.000000in}{0.000000in}}{%
\pgfpathmoveto{\pgfqpoint{0.000000in}{0.000000in}}%
\pgfpathlineto{\pgfqpoint{0.000000in}{-0.048611in}}%
\pgfusepath{stroke,fill}%
}%
\begin{pgfscope}%
\pgfsys@transformshift{5.339625in}{0.720000in}%
\pgfsys@useobject{currentmarker}{}%
\end{pgfscope}%
\end{pgfscope}%
\begin{pgfscope}%
\definecolor{textcolor}{rgb}{0.000000,0.000000,0.000000}%
\pgfsetstrokecolor{textcolor}%
\pgfsetfillcolor{textcolor}%
\pgftext[x=5.339625in,y=0.622778in,,top]{\color{textcolor}\sffamily\fontsize{20.000000}{24.000000}\selectfont \(\displaystyle {20}\)}%
\end{pgfscope}%
\begin{pgfscope}%
\pgfpathrectangle{\pgfqpoint{2.000000in}{0.720000in}}{\pgfqpoint{5.166667in}{4.620000in}}%
\pgfusepath{clip}%
\pgfsetrectcap%
\pgfsetroundjoin%
\pgfsetlinewidth{0.803000pt}%
\definecolor{currentstroke}{rgb}{0.690196,0.690196,0.690196}%
\pgfsetstrokecolor{currentstroke}%
\pgfsetdash{}{0pt}%
\pgfpathmoveto{\pgfqpoint{6.931818in}{0.720000in}}%
\pgfpathlineto{\pgfqpoint{6.931818in}{5.340000in}}%
\pgfusepath{stroke}%
\end{pgfscope}%
\begin{pgfscope}%
\pgfsetbuttcap%
\pgfsetroundjoin%
\definecolor{currentfill}{rgb}{0.000000,0.000000,0.000000}%
\pgfsetfillcolor{currentfill}%
\pgfsetlinewidth{0.803000pt}%
\definecolor{currentstroke}{rgb}{0.000000,0.000000,0.000000}%
\pgfsetstrokecolor{currentstroke}%
\pgfsetdash{}{0pt}%
\pgfsys@defobject{currentmarker}{\pgfqpoint{0.000000in}{-0.048611in}}{\pgfqpoint{0.000000in}{0.000000in}}{%
\pgfpathmoveto{\pgfqpoint{0.000000in}{0.000000in}}%
\pgfpathlineto{\pgfqpoint{0.000000in}{-0.048611in}}%
\pgfusepath{stroke,fill}%
}%
\begin{pgfscope}%
\pgfsys@transformshift{6.931818in}{0.720000in}%
\pgfsys@useobject{currentmarker}{}%
\end{pgfscope}%
\end{pgfscope}%
\begin{pgfscope}%
\definecolor{textcolor}{rgb}{0.000000,0.000000,0.000000}%
\pgfsetstrokecolor{textcolor}%
\pgfsetfillcolor{textcolor}%
\pgftext[x=6.931818in,y=0.622778in,,top]{\color{textcolor}\sffamily\fontsize{20.000000}{24.000000}\selectfont \(\displaystyle {30}\)}%
\end{pgfscope}%
\begin{pgfscope}%
\definecolor{textcolor}{rgb}{0.000000,0.000000,0.000000}%
\pgfsetstrokecolor{textcolor}%
\pgfsetfillcolor{textcolor}%
\pgftext[x=4.583333in,y=0.311155in,,top]{\color{textcolor}\sffamily\fontsize{20.000000}{24.000000}\selectfont \(\displaystyle \mu\)}%
\end{pgfscope}%
\begin{pgfscope}%
\pgfpathrectangle{\pgfqpoint{2.000000in}{0.720000in}}{\pgfqpoint{5.166667in}{4.620000in}}%
\pgfusepath{clip}%
\pgfsetrectcap%
\pgfsetroundjoin%
\pgfsetlinewidth{0.803000pt}%
\definecolor{currentstroke}{rgb}{0.690196,0.690196,0.690196}%
\pgfsetstrokecolor{currentstroke}%
\pgfsetdash{}{0pt}%
\pgfpathmoveto{\pgfqpoint{2.000000in}{0.916397in}}%
\pgfpathlineto{\pgfqpoint{7.166667in}{0.916397in}}%
\pgfusepath{stroke}%
\end{pgfscope}%
\begin{pgfscope}%
\pgfsetbuttcap%
\pgfsetroundjoin%
\definecolor{currentfill}{rgb}{0.000000,0.000000,0.000000}%
\pgfsetfillcolor{currentfill}%
\pgfsetlinewidth{0.803000pt}%
\definecolor{currentstroke}{rgb}{0.000000,0.000000,0.000000}%
\pgfsetstrokecolor{currentstroke}%
\pgfsetdash{}{0pt}%
\pgfsys@defobject{currentmarker}{\pgfqpoint{-0.048611in}{0.000000in}}{\pgfqpoint{-0.000000in}{0.000000in}}{%
\pgfpathmoveto{\pgfqpoint{-0.000000in}{0.000000in}}%
\pgfpathlineto{\pgfqpoint{-0.048611in}{0.000000in}}%
\pgfusepath{stroke,fill}%
}%
\begin{pgfscope}%
\pgfsys@transformshift{2.000000in}{0.916397in}%
\pgfsys@useobject{currentmarker}{}%
\end{pgfscope}%
\end{pgfscope}%
\begin{pgfscope}%
\definecolor{textcolor}{rgb}{0.000000,0.000000,0.000000}%
\pgfsetstrokecolor{textcolor}%
\pgfsetfillcolor{textcolor}%
\pgftext[x=1.428108in, y=0.816378in, left, base]{\color{textcolor}\sffamily\fontsize{20.000000}{24.000000}\selectfont \(\displaystyle {0.98}\)}%
\end{pgfscope}%
\begin{pgfscope}%
\pgfpathrectangle{\pgfqpoint{2.000000in}{0.720000in}}{\pgfqpoint{5.166667in}{4.620000in}}%
\pgfusepath{clip}%
\pgfsetrectcap%
\pgfsetroundjoin%
\pgfsetlinewidth{0.803000pt}%
\definecolor{currentstroke}{rgb}{0.690196,0.690196,0.690196}%
\pgfsetstrokecolor{currentstroke}%
\pgfsetdash{}{0pt}%
\pgfpathmoveto{\pgfqpoint{2.000000in}{1.726328in}}%
\pgfpathlineto{\pgfqpoint{7.166667in}{1.726328in}}%
\pgfusepath{stroke}%
\end{pgfscope}%
\begin{pgfscope}%
\pgfsetbuttcap%
\pgfsetroundjoin%
\definecolor{currentfill}{rgb}{0.000000,0.000000,0.000000}%
\pgfsetfillcolor{currentfill}%
\pgfsetlinewidth{0.803000pt}%
\definecolor{currentstroke}{rgb}{0.000000,0.000000,0.000000}%
\pgfsetstrokecolor{currentstroke}%
\pgfsetdash{}{0pt}%
\pgfsys@defobject{currentmarker}{\pgfqpoint{-0.048611in}{0.000000in}}{\pgfqpoint{-0.000000in}{0.000000in}}{%
\pgfpathmoveto{\pgfqpoint{-0.000000in}{0.000000in}}%
\pgfpathlineto{\pgfqpoint{-0.048611in}{0.000000in}}%
\pgfusepath{stroke,fill}%
}%
\begin{pgfscope}%
\pgfsys@transformshift{2.000000in}{1.726328in}%
\pgfsys@useobject{currentmarker}{}%
\end{pgfscope}%
\end{pgfscope}%
\begin{pgfscope}%
\definecolor{textcolor}{rgb}{0.000000,0.000000,0.000000}%
\pgfsetstrokecolor{textcolor}%
\pgfsetfillcolor{textcolor}%
\pgftext[x=1.428108in, y=1.626308in, left, base]{\color{textcolor}\sffamily\fontsize{20.000000}{24.000000}\selectfont \(\displaystyle {1.00}\)}%
\end{pgfscope}%
\begin{pgfscope}%
\pgfpathrectangle{\pgfqpoint{2.000000in}{0.720000in}}{\pgfqpoint{5.166667in}{4.620000in}}%
\pgfusepath{clip}%
\pgfsetrectcap%
\pgfsetroundjoin%
\pgfsetlinewidth{0.803000pt}%
\definecolor{currentstroke}{rgb}{0.690196,0.690196,0.690196}%
\pgfsetstrokecolor{currentstroke}%
\pgfsetdash{}{0pt}%
\pgfpathmoveto{\pgfqpoint{2.000000in}{2.536258in}}%
\pgfpathlineto{\pgfqpoint{7.166667in}{2.536258in}}%
\pgfusepath{stroke}%
\end{pgfscope}%
\begin{pgfscope}%
\pgfsetbuttcap%
\pgfsetroundjoin%
\definecolor{currentfill}{rgb}{0.000000,0.000000,0.000000}%
\pgfsetfillcolor{currentfill}%
\pgfsetlinewidth{0.803000pt}%
\definecolor{currentstroke}{rgb}{0.000000,0.000000,0.000000}%
\pgfsetstrokecolor{currentstroke}%
\pgfsetdash{}{0pt}%
\pgfsys@defobject{currentmarker}{\pgfqpoint{-0.048611in}{0.000000in}}{\pgfqpoint{-0.000000in}{0.000000in}}{%
\pgfpathmoveto{\pgfqpoint{-0.000000in}{0.000000in}}%
\pgfpathlineto{\pgfqpoint{-0.048611in}{0.000000in}}%
\pgfusepath{stroke,fill}%
}%
\begin{pgfscope}%
\pgfsys@transformshift{2.000000in}{2.536258in}%
\pgfsys@useobject{currentmarker}{}%
\end{pgfscope}%
\end{pgfscope}%
\begin{pgfscope}%
\definecolor{textcolor}{rgb}{0.000000,0.000000,0.000000}%
\pgfsetstrokecolor{textcolor}%
\pgfsetfillcolor{textcolor}%
\pgftext[x=1.428108in, y=2.436239in, left, base]{\color{textcolor}\sffamily\fontsize{20.000000}{24.000000}\selectfont \(\displaystyle {1.02}\)}%
\end{pgfscope}%
\begin{pgfscope}%
\pgfpathrectangle{\pgfqpoint{2.000000in}{0.720000in}}{\pgfqpoint{5.166667in}{4.620000in}}%
\pgfusepath{clip}%
\pgfsetrectcap%
\pgfsetroundjoin%
\pgfsetlinewidth{0.803000pt}%
\definecolor{currentstroke}{rgb}{0.690196,0.690196,0.690196}%
\pgfsetstrokecolor{currentstroke}%
\pgfsetdash{}{0pt}%
\pgfpathmoveto{\pgfqpoint{2.000000in}{3.346189in}}%
\pgfpathlineto{\pgfqpoint{7.166667in}{3.346189in}}%
\pgfusepath{stroke}%
\end{pgfscope}%
\begin{pgfscope}%
\pgfsetbuttcap%
\pgfsetroundjoin%
\definecolor{currentfill}{rgb}{0.000000,0.000000,0.000000}%
\pgfsetfillcolor{currentfill}%
\pgfsetlinewidth{0.803000pt}%
\definecolor{currentstroke}{rgb}{0.000000,0.000000,0.000000}%
\pgfsetstrokecolor{currentstroke}%
\pgfsetdash{}{0pt}%
\pgfsys@defobject{currentmarker}{\pgfqpoint{-0.048611in}{0.000000in}}{\pgfqpoint{-0.000000in}{0.000000in}}{%
\pgfpathmoveto{\pgfqpoint{-0.000000in}{0.000000in}}%
\pgfpathlineto{\pgfqpoint{-0.048611in}{0.000000in}}%
\pgfusepath{stroke,fill}%
}%
\begin{pgfscope}%
\pgfsys@transformshift{2.000000in}{3.346189in}%
\pgfsys@useobject{currentmarker}{}%
\end{pgfscope}%
\end{pgfscope}%
\begin{pgfscope}%
\definecolor{textcolor}{rgb}{0.000000,0.000000,0.000000}%
\pgfsetstrokecolor{textcolor}%
\pgfsetfillcolor{textcolor}%
\pgftext[x=1.428108in, y=3.246170in, left, base]{\color{textcolor}\sffamily\fontsize{20.000000}{24.000000}\selectfont \(\displaystyle {1.04}\)}%
\end{pgfscope}%
\begin{pgfscope}%
\pgfpathrectangle{\pgfqpoint{2.000000in}{0.720000in}}{\pgfqpoint{5.166667in}{4.620000in}}%
\pgfusepath{clip}%
\pgfsetrectcap%
\pgfsetroundjoin%
\pgfsetlinewidth{0.803000pt}%
\definecolor{currentstroke}{rgb}{0.690196,0.690196,0.690196}%
\pgfsetstrokecolor{currentstroke}%
\pgfsetdash{}{0pt}%
\pgfpathmoveto{\pgfqpoint{2.000000in}{4.156120in}}%
\pgfpathlineto{\pgfqpoint{7.166667in}{4.156120in}}%
\pgfusepath{stroke}%
\end{pgfscope}%
\begin{pgfscope}%
\pgfsetbuttcap%
\pgfsetroundjoin%
\definecolor{currentfill}{rgb}{0.000000,0.000000,0.000000}%
\pgfsetfillcolor{currentfill}%
\pgfsetlinewidth{0.803000pt}%
\definecolor{currentstroke}{rgb}{0.000000,0.000000,0.000000}%
\pgfsetstrokecolor{currentstroke}%
\pgfsetdash{}{0pt}%
\pgfsys@defobject{currentmarker}{\pgfqpoint{-0.048611in}{0.000000in}}{\pgfqpoint{-0.000000in}{0.000000in}}{%
\pgfpathmoveto{\pgfqpoint{-0.000000in}{0.000000in}}%
\pgfpathlineto{\pgfqpoint{-0.048611in}{0.000000in}}%
\pgfusepath{stroke,fill}%
}%
\begin{pgfscope}%
\pgfsys@transformshift{2.000000in}{4.156120in}%
\pgfsys@useobject{currentmarker}{}%
\end{pgfscope}%
\end{pgfscope}%
\begin{pgfscope}%
\definecolor{textcolor}{rgb}{0.000000,0.000000,0.000000}%
\pgfsetstrokecolor{textcolor}%
\pgfsetfillcolor{textcolor}%
\pgftext[x=1.428108in, y=4.056101in, left, base]{\color{textcolor}\sffamily\fontsize{20.000000}{24.000000}\selectfont \(\displaystyle {1.06}\)}%
\end{pgfscope}%
\begin{pgfscope}%
\pgfpathrectangle{\pgfqpoint{2.000000in}{0.720000in}}{\pgfqpoint{5.166667in}{4.620000in}}%
\pgfusepath{clip}%
\pgfsetrectcap%
\pgfsetroundjoin%
\pgfsetlinewidth{0.803000pt}%
\definecolor{currentstroke}{rgb}{0.690196,0.690196,0.690196}%
\pgfsetstrokecolor{currentstroke}%
\pgfsetdash{}{0pt}%
\pgfpathmoveto{\pgfqpoint{2.000000in}{4.966051in}}%
\pgfpathlineto{\pgfqpoint{7.166667in}{4.966051in}}%
\pgfusepath{stroke}%
\end{pgfscope}%
\begin{pgfscope}%
\pgfsetbuttcap%
\pgfsetroundjoin%
\definecolor{currentfill}{rgb}{0.000000,0.000000,0.000000}%
\pgfsetfillcolor{currentfill}%
\pgfsetlinewidth{0.803000pt}%
\definecolor{currentstroke}{rgb}{0.000000,0.000000,0.000000}%
\pgfsetstrokecolor{currentstroke}%
\pgfsetdash{}{0pt}%
\pgfsys@defobject{currentmarker}{\pgfqpoint{-0.048611in}{0.000000in}}{\pgfqpoint{-0.000000in}{0.000000in}}{%
\pgfpathmoveto{\pgfqpoint{-0.000000in}{0.000000in}}%
\pgfpathlineto{\pgfqpoint{-0.048611in}{0.000000in}}%
\pgfusepath{stroke,fill}%
}%
\begin{pgfscope}%
\pgfsys@transformshift{2.000000in}{4.966051in}%
\pgfsys@useobject{currentmarker}{}%
\end{pgfscope}%
\end{pgfscope}%
\begin{pgfscope}%
\definecolor{textcolor}{rgb}{0.000000,0.000000,0.000000}%
\pgfsetstrokecolor{textcolor}%
\pgfsetfillcolor{textcolor}%
\pgftext[x=1.428108in, y=4.866031in, left, base]{\color{textcolor}\sffamily\fontsize{20.000000}{24.000000}\selectfont \(\displaystyle {1.08}\)}%
\end{pgfscope}%
\begin{pgfscope}%
\definecolor{textcolor}{rgb}{0.000000,0.000000,0.000000}%
\pgfsetstrokecolor{textcolor}%
\pgfsetfillcolor{textcolor}%
\pgftext[x=1.372552in,y=3.030000in,,bottom,rotate=90.000000]{\color{textcolor}\sffamily\fontsize{20.000000}{24.000000}\selectfont \(\displaystyle \hat{t}_0\mathrm{\ resolution\ ratio}\)}%
\end{pgfscope}%
\begin{pgfscope}%
\pgfpathrectangle{\pgfqpoint{2.000000in}{0.720000in}}{\pgfqpoint{5.166667in}{4.620000in}}%
\pgfusepath{clip}%
\pgfsetbuttcap%
\pgfsetroundjoin%
\pgfsetlinewidth{2.007500pt}%
\definecolor{currentstroke}{rgb}{0.000000,0.000000,1.000000}%
\pgfsetstrokecolor{currentstroke}%
\pgfsetdash{}{0pt}%
\pgfpathmoveto{\pgfqpoint{2.234848in}{1.253815in}}%
\pgfpathlineto{\pgfqpoint{2.234848in}{2.063486in}}%
\pgfusepath{stroke}%
\end{pgfscope}%
\begin{pgfscope}%
\pgfpathrectangle{\pgfqpoint{2.000000in}{0.720000in}}{\pgfqpoint{5.166667in}{4.620000in}}%
\pgfusepath{clip}%
\pgfsetbuttcap%
\pgfsetroundjoin%
\pgfsetlinewidth{2.007500pt}%
\definecolor{currentstroke}{rgb}{0.000000,0.000000,1.000000}%
\pgfsetstrokecolor{currentstroke}%
\pgfsetdash{}{0pt}%
\pgfpathmoveto{\pgfqpoint{2.314458in}{1.917612in}}%
\pgfpathlineto{\pgfqpoint{2.314458in}{2.740212in}}%
\pgfusepath{stroke}%
\end{pgfscope}%
\begin{pgfscope}%
\pgfpathrectangle{\pgfqpoint{2.000000in}{0.720000in}}{\pgfqpoint{5.166667in}{4.620000in}}%
\pgfusepath{clip}%
\pgfsetbuttcap%
\pgfsetroundjoin%
\pgfsetlinewidth{2.007500pt}%
\definecolor{currentstroke}{rgb}{0.000000,0.000000,1.000000}%
\pgfsetstrokecolor{currentstroke}%
\pgfsetdash{}{0pt}%
\pgfpathmoveto{\pgfqpoint{2.394068in}{1.755441in}}%
\pgfpathlineto{\pgfqpoint{2.394068in}{2.574969in}}%
\pgfusepath{stroke}%
\end{pgfscope}%
\begin{pgfscope}%
\pgfpathrectangle{\pgfqpoint{2.000000in}{0.720000in}}{\pgfqpoint{5.166667in}{4.620000in}}%
\pgfusepath{clip}%
\pgfsetbuttcap%
\pgfsetroundjoin%
\pgfsetlinewidth{2.007500pt}%
\definecolor{currentstroke}{rgb}{0.000000,0.000000,1.000000}%
\pgfsetstrokecolor{currentstroke}%
\pgfsetdash{}{0pt}%
\pgfpathmoveto{\pgfqpoint{2.473677in}{2.083228in}}%
\pgfpathlineto{\pgfqpoint{2.473677in}{2.909259in}}%
\pgfusepath{stroke}%
\end{pgfscope}%
\begin{pgfscope}%
\pgfpathrectangle{\pgfqpoint{2.000000in}{0.720000in}}{\pgfqpoint{5.166667in}{4.620000in}}%
\pgfusepath{clip}%
\pgfsetbuttcap%
\pgfsetroundjoin%
\pgfsetlinewidth{2.007500pt}%
\definecolor{currentstroke}{rgb}{0.000000,0.000000,1.000000}%
\pgfsetstrokecolor{currentstroke}%
\pgfsetdash{}{0pt}%
\pgfpathmoveto{\pgfqpoint{2.553287in}{2.015392in}}%
\pgfpathlineto{\pgfqpoint{2.553287in}{2.840512in}}%
\pgfusepath{stroke}%
\end{pgfscope}%
\begin{pgfscope}%
\pgfpathrectangle{\pgfqpoint{2.000000in}{0.720000in}}{\pgfqpoint{5.166667in}{4.620000in}}%
\pgfusepath{clip}%
\pgfsetbuttcap%
\pgfsetroundjoin%
\pgfsetlinewidth{2.007500pt}%
\definecolor{currentstroke}{rgb}{0.000000,0.000000,1.000000}%
\pgfsetstrokecolor{currentstroke}%
\pgfsetdash{}{0pt}%
\pgfpathmoveto{\pgfqpoint{2.632897in}{2.452562in}}%
\pgfpathlineto{\pgfqpoint{2.632897in}{3.285935in}}%
\pgfusepath{stroke}%
\end{pgfscope}%
\begin{pgfscope}%
\pgfpathrectangle{\pgfqpoint{2.000000in}{0.720000in}}{\pgfqpoint{5.166667in}{4.620000in}}%
\pgfusepath{clip}%
\pgfsetbuttcap%
\pgfsetroundjoin%
\pgfsetlinewidth{2.007500pt}%
\definecolor{currentstroke}{rgb}{0.000000,0.000000,1.000000}%
\pgfsetstrokecolor{currentstroke}%
\pgfsetdash{}{0pt}%
\pgfpathmoveto{\pgfqpoint{2.712506in}{3.344801in}}%
\pgfpathlineto{\pgfqpoint{2.712506in}{4.196513in}}%
\pgfusepath{stroke}%
\end{pgfscope}%
\begin{pgfscope}%
\pgfpathrectangle{\pgfqpoint{2.000000in}{0.720000in}}{\pgfqpoint{5.166667in}{4.620000in}}%
\pgfusepath{clip}%
\pgfsetbuttcap%
\pgfsetroundjoin%
\pgfsetlinewidth{2.007500pt}%
\definecolor{currentstroke}{rgb}{0.000000,0.000000,1.000000}%
\pgfsetstrokecolor{currentstroke}%
\pgfsetdash{}{0pt}%
\pgfpathmoveto{\pgfqpoint{2.792116in}{2.564514in}}%
\pgfpathlineto{\pgfqpoint{2.792116in}{3.400361in}}%
\pgfusepath{stroke}%
\end{pgfscope}%
\begin{pgfscope}%
\pgfpathrectangle{\pgfqpoint{2.000000in}{0.720000in}}{\pgfqpoint{5.166667in}{4.620000in}}%
\pgfusepath{clip}%
\pgfsetbuttcap%
\pgfsetroundjoin%
\pgfsetlinewidth{2.007500pt}%
\definecolor{currentstroke}{rgb}{0.000000,0.000000,1.000000}%
\pgfsetstrokecolor{currentstroke}%
\pgfsetdash{}{0pt}%
\pgfpathmoveto{\pgfqpoint{3.110555in}{4.260301in}}%
\pgfpathlineto{\pgfqpoint{3.110555in}{5.130000in}}%
\pgfusepath{stroke}%
\end{pgfscope}%
\begin{pgfscope}%
\pgfpathrectangle{\pgfqpoint{2.000000in}{0.720000in}}{\pgfqpoint{5.166667in}{4.620000in}}%
\pgfusepath{clip}%
\pgfsetbuttcap%
\pgfsetroundjoin%
\pgfsetlinewidth{2.007500pt}%
\definecolor{currentstroke}{rgb}{0.000000,0.000000,1.000000}%
\pgfsetstrokecolor{currentstroke}%
\pgfsetdash{}{0pt}%
\pgfpathmoveto{\pgfqpoint{3.428993in}{3.990509in}}%
\pgfpathlineto{\pgfqpoint{3.428993in}{4.854580in}}%
\pgfusepath{stroke}%
\end{pgfscope}%
\begin{pgfscope}%
\pgfpathrectangle{\pgfqpoint{2.000000in}{0.720000in}}{\pgfqpoint{5.166667in}{4.620000in}}%
\pgfusepath{clip}%
\pgfsetbuttcap%
\pgfsetroundjoin%
\pgfsetlinewidth{2.007500pt}%
\definecolor{currentstroke}{rgb}{0.000000,0.000000,1.000000}%
\pgfsetstrokecolor{currentstroke}%
\pgfsetdash{}{0pt}%
\pgfpathmoveto{\pgfqpoint{3.747432in}{3.693529in}}%
\pgfpathlineto{\pgfqpoint{3.747432in}{4.551513in}}%
\pgfusepath{stroke}%
\end{pgfscope}%
\begin{pgfscope}%
\pgfpathrectangle{\pgfqpoint{2.000000in}{0.720000in}}{\pgfqpoint{5.166667in}{4.620000in}}%
\pgfusepath{clip}%
\pgfsetbuttcap%
\pgfsetroundjoin%
\pgfsetlinewidth{2.007500pt}%
\definecolor{currentstroke}{rgb}{0.000000,0.000000,1.000000}%
\pgfsetstrokecolor{currentstroke}%
\pgfsetdash{}{0pt}%
\pgfpathmoveto{\pgfqpoint{4.543529in}{3.607631in}}%
\pgfpathlineto{\pgfqpoint{4.543529in}{4.463792in}}%
\pgfusepath{stroke}%
\end{pgfscope}%
\begin{pgfscope}%
\pgfpathrectangle{\pgfqpoint{2.000000in}{0.720000in}}{\pgfqpoint{5.166667in}{4.620000in}}%
\pgfusepath{clip}%
\pgfsetbuttcap%
\pgfsetroundjoin%
\pgfsetlinewidth{2.007500pt}%
\definecolor{currentstroke}{rgb}{0.000000,0.000000,1.000000}%
\pgfsetstrokecolor{currentstroke}%
\pgfsetdash{}{0pt}%
\pgfpathmoveto{\pgfqpoint{5.339625in}{3.306124in}}%
\pgfpathlineto{\pgfqpoint{5.339625in}{4.156151in}}%
\pgfusepath{stroke}%
\end{pgfscope}%
\begin{pgfscope}%
\pgfpathrectangle{\pgfqpoint{2.000000in}{0.720000in}}{\pgfqpoint{5.166667in}{4.620000in}}%
\pgfusepath{clip}%
\pgfsetbuttcap%
\pgfsetroundjoin%
\pgfsetlinewidth{2.007500pt}%
\definecolor{currentstroke}{rgb}{0.000000,0.000000,1.000000}%
\pgfsetstrokecolor{currentstroke}%
\pgfsetdash{}{0pt}%
\pgfpathmoveto{\pgfqpoint{6.135722in}{3.163405in}}%
\pgfpathlineto{\pgfqpoint{6.135722in}{4.010548in}}%
\pgfusepath{stroke}%
\end{pgfscope}%
\begin{pgfscope}%
\pgfpathrectangle{\pgfqpoint{2.000000in}{0.720000in}}{\pgfqpoint{5.166667in}{4.620000in}}%
\pgfusepath{clip}%
\pgfsetbuttcap%
\pgfsetroundjoin%
\pgfsetlinewidth{2.007500pt}%
\definecolor{currentstroke}{rgb}{0.000000,0.000000,1.000000}%
\pgfsetstrokecolor{currentstroke}%
\pgfsetdash{}{0pt}%
\pgfpathmoveto{\pgfqpoint{6.931818in}{3.009493in}}%
\pgfpathlineto{\pgfqpoint{6.931818in}{3.853528in}}%
\pgfusepath{stroke}%
\end{pgfscope}%
\begin{pgfscope}%
\pgfpathrectangle{\pgfqpoint{2.000000in}{0.720000in}}{\pgfqpoint{5.166667in}{4.620000in}}%
\pgfusepath{clip}%
\pgfsetbuttcap%
\pgfsetroundjoin%
\pgfsetlinewidth{2.007500pt}%
\definecolor{currentstroke}{rgb}{1.000000,0.000000,0.000000}%
\pgfsetstrokecolor{currentstroke}%
\pgfsetdash{}{0pt}%
\pgfpathmoveto{\pgfqpoint{2.234848in}{1.332259in}}%
\pgfpathlineto{\pgfqpoint{2.234848in}{2.143517in}}%
\pgfusepath{stroke}%
\end{pgfscope}%
\begin{pgfscope}%
\pgfpathrectangle{\pgfqpoint{2.000000in}{0.720000in}}{\pgfqpoint{5.166667in}{4.620000in}}%
\pgfusepath{clip}%
\pgfsetbuttcap%
\pgfsetroundjoin%
\pgfsetlinewidth{2.007500pt}%
\definecolor{currentstroke}{rgb}{1.000000,0.000000,0.000000}%
\pgfsetstrokecolor{currentstroke}%
\pgfsetdash{}{0pt}%
\pgfpathmoveto{\pgfqpoint{2.314458in}{1.337842in}}%
\pgfpathlineto{\pgfqpoint{2.314458in}{2.148720in}}%
\pgfusepath{stroke}%
\end{pgfscope}%
\begin{pgfscope}%
\pgfpathrectangle{\pgfqpoint{2.000000in}{0.720000in}}{\pgfqpoint{5.166667in}{4.620000in}}%
\pgfusepath{clip}%
\pgfsetbuttcap%
\pgfsetroundjoin%
\pgfsetlinewidth{2.007500pt}%
\definecolor{currentstroke}{rgb}{1.000000,0.000000,0.000000}%
\pgfsetstrokecolor{currentstroke}%
\pgfsetdash{}{0pt}%
\pgfpathmoveto{\pgfqpoint{2.394068in}{1.333144in}}%
\pgfpathlineto{\pgfqpoint{2.394068in}{2.144132in}}%
\pgfusepath{stroke}%
\end{pgfscope}%
\begin{pgfscope}%
\pgfpathrectangle{\pgfqpoint{2.000000in}{0.720000in}}{\pgfqpoint{5.166667in}{4.620000in}}%
\pgfusepath{clip}%
\pgfsetbuttcap%
\pgfsetroundjoin%
\pgfsetlinewidth{2.007500pt}%
\definecolor{currentstroke}{rgb}{1.000000,0.000000,0.000000}%
\pgfsetstrokecolor{currentstroke}%
\pgfsetdash{}{0pt}%
\pgfpathmoveto{\pgfqpoint{2.473677in}{1.337249in}}%
\pgfpathlineto{\pgfqpoint{2.473677in}{2.148197in}}%
\pgfusepath{stroke}%
\end{pgfscope}%
\begin{pgfscope}%
\pgfpathrectangle{\pgfqpoint{2.000000in}{0.720000in}}{\pgfqpoint{5.166667in}{4.620000in}}%
\pgfusepath{clip}%
\pgfsetbuttcap%
\pgfsetroundjoin%
\pgfsetlinewidth{2.007500pt}%
\definecolor{currentstroke}{rgb}{1.000000,0.000000,0.000000}%
\pgfsetstrokecolor{currentstroke}%
\pgfsetdash{}{0pt}%
\pgfpathmoveto{\pgfqpoint{2.553287in}{0.930000in}}%
\pgfpathlineto{\pgfqpoint{2.553287in}{1.733161in}}%
\pgfusepath{stroke}%
\end{pgfscope}%
\begin{pgfscope}%
\pgfpathrectangle{\pgfqpoint{2.000000in}{0.720000in}}{\pgfqpoint{5.166667in}{4.620000in}}%
\pgfusepath{clip}%
\pgfsetbuttcap%
\pgfsetroundjoin%
\pgfsetlinewidth{2.007500pt}%
\definecolor{currentstroke}{rgb}{1.000000,0.000000,0.000000}%
\pgfsetstrokecolor{currentstroke}%
\pgfsetdash{}{0pt}%
\pgfpathmoveto{\pgfqpoint{2.632897in}{1.392352in}}%
\pgfpathlineto{\pgfqpoint{2.632897in}{2.204291in}}%
\pgfusepath{stroke}%
\end{pgfscope}%
\begin{pgfscope}%
\pgfpathrectangle{\pgfqpoint{2.000000in}{0.720000in}}{\pgfqpoint{5.166667in}{4.620000in}}%
\pgfusepath{clip}%
\pgfsetbuttcap%
\pgfsetroundjoin%
\pgfsetlinewidth{2.007500pt}%
\definecolor{currentstroke}{rgb}{1.000000,0.000000,0.000000}%
\pgfsetstrokecolor{currentstroke}%
\pgfsetdash{}{0pt}%
\pgfpathmoveto{\pgfqpoint{2.712506in}{1.402048in}}%
\pgfpathlineto{\pgfqpoint{2.712506in}{2.214471in}}%
\pgfusepath{stroke}%
\end{pgfscope}%
\begin{pgfscope}%
\pgfpathrectangle{\pgfqpoint{2.000000in}{0.720000in}}{\pgfqpoint{5.166667in}{4.620000in}}%
\pgfusepath{clip}%
\pgfsetbuttcap%
\pgfsetroundjoin%
\pgfsetlinewidth{2.007500pt}%
\definecolor{currentstroke}{rgb}{1.000000,0.000000,0.000000}%
\pgfsetstrokecolor{currentstroke}%
\pgfsetdash{}{0pt}%
\pgfpathmoveto{\pgfqpoint{2.792116in}{1.396774in}}%
\pgfpathlineto{\pgfqpoint{2.792116in}{2.209008in}}%
\pgfusepath{stroke}%
\end{pgfscope}%
\begin{pgfscope}%
\pgfpathrectangle{\pgfqpoint{2.000000in}{0.720000in}}{\pgfqpoint{5.166667in}{4.620000in}}%
\pgfusepath{clip}%
\pgfsetbuttcap%
\pgfsetroundjoin%
\pgfsetlinewidth{2.007500pt}%
\definecolor{currentstroke}{rgb}{1.000000,0.000000,0.000000}%
\pgfsetstrokecolor{currentstroke}%
\pgfsetdash{}{0pt}%
\pgfpathmoveto{\pgfqpoint{3.110555in}{1.450698in}}%
\pgfpathlineto{\pgfqpoint{3.110555in}{2.263611in}}%
\pgfusepath{stroke}%
\end{pgfscope}%
\begin{pgfscope}%
\pgfpathrectangle{\pgfqpoint{2.000000in}{0.720000in}}{\pgfqpoint{5.166667in}{4.620000in}}%
\pgfusepath{clip}%
\pgfsetbuttcap%
\pgfsetroundjoin%
\pgfsetlinewidth{2.007500pt}%
\definecolor{currentstroke}{rgb}{1.000000,0.000000,0.000000}%
\pgfsetstrokecolor{currentstroke}%
\pgfsetdash{}{0pt}%
\pgfpathmoveto{\pgfqpoint{3.428993in}{2.058947in}}%
\pgfpathlineto{\pgfqpoint{3.428993in}{2.883987in}}%
\pgfusepath{stroke}%
\end{pgfscope}%
\begin{pgfscope}%
\pgfpathrectangle{\pgfqpoint{2.000000in}{0.720000in}}{\pgfqpoint{5.166667in}{4.620000in}}%
\pgfusepath{clip}%
\pgfsetbuttcap%
\pgfsetroundjoin%
\pgfsetlinewidth{2.007500pt}%
\definecolor{currentstroke}{rgb}{1.000000,0.000000,0.000000}%
\pgfsetstrokecolor{currentstroke}%
\pgfsetdash{}{0pt}%
\pgfpathmoveto{\pgfqpoint{3.747432in}{2.359035in}}%
\pgfpathlineto{\pgfqpoint{3.747432in}{3.190055in}}%
\pgfusepath{stroke}%
\end{pgfscope}%
\begin{pgfscope}%
\pgfpathrectangle{\pgfqpoint{2.000000in}{0.720000in}}{\pgfqpoint{5.166667in}{4.620000in}}%
\pgfusepath{clip}%
\pgfsetbuttcap%
\pgfsetroundjoin%
\pgfsetlinewidth{2.007500pt}%
\definecolor{currentstroke}{rgb}{1.000000,0.000000,0.000000}%
\pgfsetstrokecolor{currentstroke}%
\pgfsetdash{}{0pt}%
\pgfpathmoveto{\pgfqpoint{4.543529in}{2.705465in}}%
\pgfpathlineto{\pgfqpoint{4.543529in}{3.543400in}}%
\pgfusepath{stroke}%
\end{pgfscope}%
\begin{pgfscope}%
\pgfpathrectangle{\pgfqpoint{2.000000in}{0.720000in}}{\pgfqpoint{5.166667in}{4.620000in}}%
\pgfusepath{clip}%
\pgfsetbuttcap%
\pgfsetroundjoin%
\pgfsetlinewidth{2.007500pt}%
\definecolor{currentstroke}{rgb}{1.000000,0.000000,0.000000}%
\pgfsetstrokecolor{currentstroke}%
\pgfsetdash{}{0pt}%
\pgfpathmoveto{\pgfqpoint{5.339625in}{2.716324in}}%
\pgfpathlineto{\pgfqpoint{5.339625in}{3.554436in}}%
\pgfusepath{stroke}%
\end{pgfscope}%
\begin{pgfscope}%
\pgfpathrectangle{\pgfqpoint{2.000000in}{0.720000in}}{\pgfqpoint{5.166667in}{4.620000in}}%
\pgfusepath{clip}%
\pgfsetbuttcap%
\pgfsetroundjoin%
\pgfsetlinewidth{2.007500pt}%
\definecolor{currentstroke}{rgb}{1.000000,0.000000,0.000000}%
\pgfsetstrokecolor{currentstroke}%
\pgfsetdash{}{0pt}%
\pgfpathmoveto{\pgfqpoint{6.135722in}{2.752150in}}%
\pgfpathlineto{\pgfqpoint{6.135722in}{3.590985in}}%
\pgfusepath{stroke}%
\end{pgfscope}%
\begin{pgfscope}%
\pgfpathrectangle{\pgfqpoint{2.000000in}{0.720000in}}{\pgfqpoint{5.166667in}{4.620000in}}%
\pgfusepath{clip}%
\pgfsetbuttcap%
\pgfsetroundjoin%
\pgfsetlinewidth{2.007500pt}%
\definecolor{currentstroke}{rgb}{1.000000,0.000000,0.000000}%
\pgfsetstrokecolor{currentstroke}%
\pgfsetdash{}{0pt}%
\pgfpathmoveto{\pgfqpoint{6.931818in}{2.815500in}}%
\pgfpathlineto{\pgfqpoint{6.931818in}{3.655616in}}%
\pgfusepath{stroke}%
\end{pgfscope}%
\begin{pgfscope}%
\pgfpathrectangle{\pgfqpoint{2.000000in}{0.720000in}}{\pgfqpoint{5.166667in}{4.620000in}}%
\pgfusepath{clip}%
\pgfsetrectcap%
\pgfsetroundjoin%
\pgfsetlinewidth{2.007500pt}%
\definecolor{currentstroke}{rgb}{0.000000,0.000000,1.000000}%
\pgfsetstrokecolor{currentstroke}%
\pgfsetdash{}{0pt}%
\pgfpathmoveto{\pgfqpoint{2.234848in}{1.658651in}}%
\pgfpathlineto{\pgfqpoint{2.314458in}{2.328912in}}%
\pgfpathlineto{\pgfqpoint{2.394068in}{2.165205in}}%
\pgfpathlineto{\pgfqpoint{2.473677in}{2.496244in}}%
\pgfpathlineto{\pgfqpoint{2.553287in}{2.427952in}}%
\pgfpathlineto{\pgfqpoint{2.632897in}{2.869249in}}%
\pgfpathlineto{\pgfqpoint{2.712506in}{3.770657in}}%
\pgfpathlineto{\pgfqpoint{2.792116in}{2.982437in}}%
\pgfpathlineto{\pgfqpoint{3.110555in}{4.695151in}}%
\pgfpathlineto{\pgfqpoint{3.428993in}{4.422544in}}%
\pgfpathlineto{\pgfqpoint{3.747432in}{4.122521in}}%
\pgfpathlineto{\pgfqpoint{4.543529in}{4.035711in}}%
\pgfpathlineto{\pgfqpoint{5.339625in}{3.731137in}}%
\pgfpathlineto{\pgfqpoint{6.135722in}{3.586977in}}%
\pgfpathlineto{\pgfqpoint{6.931818in}{3.431511in}}%
\pgfusepath{stroke}%
\end{pgfscope}%
\begin{pgfscope}%
\pgfpathrectangle{\pgfqpoint{2.000000in}{0.720000in}}{\pgfqpoint{5.166667in}{4.620000in}}%
\pgfusepath{clip}%
\pgfsetrectcap%
\pgfsetroundjoin%
\pgfsetlinewidth{2.007500pt}%
\definecolor{currentstroke}{rgb}{1.000000,0.000000,0.000000}%
\pgfsetstrokecolor{currentstroke}%
\pgfsetdash{}{0pt}%
\pgfpathmoveto{\pgfqpoint{2.234848in}{1.737888in}}%
\pgfpathlineto{\pgfqpoint{2.314458in}{1.743281in}}%
\pgfpathlineto{\pgfqpoint{2.394068in}{1.738638in}}%
\pgfpathlineto{\pgfqpoint{2.473677in}{1.742723in}}%
\pgfpathlineto{\pgfqpoint{2.553287in}{1.331581in}}%
\pgfpathlineto{\pgfqpoint{2.632897in}{1.798321in}}%
\pgfpathlineto{\pgfqpoint{2.712506in}{1.808259in}}%
\pgfpathlineto{\pgfqpoint{2.792116in}{1.802891in}}%
\pgfpathlineto{\pgfqpoint{3.110555in}{1.857155in}}%
\pgfpathlineto{\pgfqpoint{3.428993in}{2.471467in}}%
\pgfpathlineto{\pgfqpoint{3.747432in}{2.774545in}}%
\pgfpathlineto{\pgfqpoint{4.543529in}{3.124433in}}%
\pgfpathlineto{\pgfqpoint{5.339625in}{3.135380in}}%
\pgfpathlineto{\pgfqpoint{6.135722in}{3.171567in}}%
\pgfpathlineto{\pgfqpoint{6.931818in}{3.235558in}}%
\pgfusepath{stroke}%
\end{pgfscope}%
\begin{pgfscope}%
\pgfsetrectcap%
\pgfsetmiterjoin%
\pgfsetlinewidth{0.803000pt}%
\definecolor{currentstroke}{rgb}{0.000000,0.000000,0.000000}%
\pgfsetstrokecolor{currentstroke}%
\pgfsetdash{}{0pt}%
\pgfpathmoveto{\pgfqpoint{2.000000in}{0.720000in}}%
\pgfpathlineto{\pgfqpoint{2.000000in}{5.340000in}}%
\pgfusepath{stroke}%
\end{pgfscope}%
\begin{pgfscope}%
\pgfsetrectcap%
\pgfsetmiterjoin%
\pgfsetlinewidth{0.803000pt}%
\definecolor{currentstroke}{rgb}{0.000000,0.000000,0.000000}%
\pgfsetstrokecolor{currentstroke}%
\pgfsetdash{}{0pt}%
\pgfpathmoveto{\pgfqpoint{7.166667in}{0.720000in}}%
\pgfpathlineto{\pgfqpoint{7.166667in}{5.340000in}}%
\pgfusepath{stroke}%
\end{pgfscope}%
\begin{pgfscope}%
\pgfsetrectcap%
\pgfsetmiterjoin%
\pgfsetlinewidth{0.803000pt}%
\definecolor{currentstroke}{rgb}{0.000000,0.000000,0.000000}%
\pgfsetstrokecolor{currentstroke}%
\pgfsetdash{}{0pt}%
\pgfpathmoveto{\pgfqpoint{2.000000in}{0.720000in}}%
\pgfpathlineto{\pgfqpoint{7.166667in}{0.720000in}}%
\pgfusepath{stroke}%
\end{pgfscope}%
\begin{pgfscope}%
\pgfsetrectcap%
\pgfsetmiterjoin%
\pgfsetlinewidth{0.803000pt}%
\definecolor{currentstroke}{rgb}{0.000000,0.000000,0.000000}%
\pgfsetstrokecolor{currentstroke}%
\pgfsetdash{}{0pt}%
\pgfpathmoveto{\pgfqpoint{2.000000in}{5.340000in}}%
\pgfpathlineto{\pgfqpoint{7.166667in}{5.340000in}}%
\pgfusepath{stroke}%
\end{pgfscope}%
\begin{pgfscope}%
\definecolor{textcolor}{rgb}{0.000000,0.000000,0.000000}%
\pgfsetstrokecolor{textcolor}%
\pgfsetfillcolor{textcolor}%
\pgftext[x=4.583333in,y=5.423333in,,base]{\color{textcolor}\sffamily\fontsize{24.000000}{28.800000}\selectfont \(\displaystyle \tau_l=20\si{ns},\,\sigma_l=5\si{ns}\)}%
\end{pgfscope}%
\begin{pgfscope}%
\pgfsetbuttcap%
\pgfsetmiterjoin%
\definecolor{currentfill}{rgb}{1.000000,1.000000,1.000000}%
\pgfsetfillcolor{currentfill}%
\pgfsetfillopacity{0.800000}%
\pgfsetlinewidth{1.003750pt}%
\definecolor{currentstroke}{rgb}{0.800000,0.800000,0.800000}%
\pgfsetstrokecolor{currentstroke}%
\pgfsetstrokeopacity{0.800000}%
\pgfsetdash{}{0pt}%
\pgfpathmoveto{\pgfqpoint{4.637109in}{4.263889in}}%
\pgfpathlineto{\pgfqpoint{6.972222in}{4.263889in}}%
\pgfpathquadraticcurveto{\pgfqpoint{7.027778in}{4.263889in}}{\pgfqpoint{7.027778in}{4.319444in}}%
\pgfpathlineto{\pgfqpoint{7.027778in}{5.145556in}}%
\pgfpathquadraticcurveto{\pgfqpoint{7.027778in}{5.201111in}}{\pgfqpoint{6.972222in}{5.201111in}}%
\pgfpathlineto{\pgfqpoint{4.637109in}{5.201111in}}%
\pgfpathquadraticcurveto{\pgfqpoint{4.581554in}{5.201111in}}{\pgfqpoint{4.581554in}{5.145556in}}%
\pgfpathlineto{\pgfqpoint{4.581554in}{4.319444in}}%
\pgfpathquadraticcurveto{\pgfqpoint{4.581554in}{4.263889in}}{\pgfqpoint{4.637109in}{4.263889in}}%
\pgfpathclose%
\pgfusepath{stroke,fill}%
\end{pgfscope}%
\begin{pgfscope}%
\pgfsetbuttcap%
\pgfsetroundjoin%
\pgfsetlinewidth{2.007500pt}%
\definecolor{currentstroke}{rgb}{0.000000,0.000000,1.000000}%
\pgfsetstrokecolor{currentstroke}%
\pgfsetdash{}{0pt}%
\pgfpathmoveto{\pgfqpoint{4.970442in}{4.832292in}}%
\pgfpathlineto{\pgfqpoint{4.970442in}{5.110069in}}%
\pgfusepath{stroke}%
\end{pgfscope}%
\begin{pgfscope}%
\pgfsetrectcap%
\pgfsetroundjoin%
\pgfsetlinewidth{2.007500pt}%
\definecolor{currentstroke}{rgb}{0.000000,0.000000,1.000000}%
\pgfsetstrokecolor{currentstroke}%
\pgfsetdash{}{0pt}%
\pgfpathmoveto{\pgfqpoint{4.692665in}{4.971181in}}%
\pgfpathlineto{\pgfqpoint{5.248220in}{4.971181in}}%
\pgfusepath{stroke}%
\end{pgfscope}%
\begin{pgfscope}%
\definecolor{textcolor}{rgb}{0.000000,0.000000,0.000000}%
\pgfsetstrokecolor{textcolor}%
\pgfsetfillcolor{textcolor}%
\pgftext[x=5.470442in,y=4.873958in,left,base]{\color{textcolor}\sffamily\fontsize{20.000000}{24.000000}\selectfont \(\displaystyle \sigma_\mathrm{FFT}/\sigma_\mathrm{ALL}\)}%
\end{pgfscope}%
\begin{pgfscope}%
\pgfsetbuttcap%
\pgfsetroundjoin%
\pgfsetlinewidth{2.007500pt}%
\definecolor{currentstroke}{rgb}{1.000000,0.000000,0.000000}%
\pgfsetstrokecolor{currentstroke}%
\pgfsetdash{}{0pt}%
\pgfpathmoveto{\pgfqpoint{4.970442in}{4.405347in}}%
\pgfpathlineto{\pgfqpoint{4.970442in}{4.683125in}}%
\pgfusepath{stroke}%
\end{pgfscope}%
\begin{pgfscope}%
\pgfsetrectcap%
\pgfsetroundjoin%
\pgfsetlinewidth{2.007500pt}%
\definecolor{currentstroke}{rgb}{1.000000,0.000000,0.000000}%
\pgfsetstrokecolor{currentstroke}%
\pgfsetdash{}{0pt}%
\pgfpathmoveto{\pgfqpoint{4.692665in}{4.544236in}}%
\pgfpathlineto{\pgfqpoint{5.248220in}{4.544236in}}%
\pgfusepath{stroke}%
\end{pgfscope}%
\begin{pgfscope}%
\definecolor{textcolor}{rgb}{0.000000,0.000000,0.000000}%
\pgfsetstrokecolor{textcolor}%
\pgfsetfillcolor{textcolor}%
\pgftext[x=5.470442in,y=4.447014in,left,base]{\color{textcolor}\sffamily\fontsize{20.000000}{24.000000}\selectfont \(\displaystyle \sigma_\mathrm{FBMP}/\sigma_\mathrm{ALL}\)}%
\end{pgfscope}%
\begin{pgfscope}%
\pgfsetbuttcap%
\pgfsetmiterjoin%
\definecolor{currentfill}{rgb}{1.000000,1.000000,1.000000}%
\pgfsetfillcolor{currentfill}%
\pgfsetlinewidth{0.000000pt}%
\definecolor{currentstroke}{rgb}{0.000000,0.000000,0.000000}%
\pgfsetstrokecolor{currentstroke}%
\pgfsetstrokeopacity{0.000000}%
\pgfsetdash{}{0pt}%
\pgfpathmoveto{\pgfqpoint{9.233333in}{0.720000in}}%
\pgfpathlineto{\pgfqpoint{14.400000in}{0.720000in}}%
\pgfpathlineto{\pgfqpoint{14.400000in}{5.340000in}}%
\pgfpathlineto{\pgfqpoint{9.233333in}{5.340000in}}%
\pgfpathclose%
\pgfusepath{fill}%
\end{pgfscope}%
\begin{pgfscope}%
\pgfpathrectangle{\pgfqpoint{9.233333in}{0.720000in}}{\pgfqpoint{5.166667in}{4.620000in}}%
\pgfusepath{clip}%
\pgfsetrectcap%
\pgfsetroundjoin%
\pgfsetlinewidth{0.803000pt}%
\definecolor{currentstroke}{rgb}{0.690196,0.690196,0.690196}%
\pgfsetstrokecolor{currentstroke}%
\pgfsetdash{}{0pt}%
\pgfpathmoveto{\pgfqpoint{9.388572in}{0.720000in}}%
\pgfpathlineto{\pgfqpoint{9.388572in}{5.340000in}}%
\pgfusepath{stroke}%
\end{pgfscope}%
\begin{pgfscope}%
\pgfsetbuttcap%
\pgfsetroundjoin%
\definecolor{currentfill}{rgb}{0.000000,0.000000,0.000000}%
\pgfsetfillcolor{currentfill}%
\pgfsetlinewidth{0.803000pt}%
\definecolor{currentstroke}{rgb}{0.000000,0.000000,0.000000}%
\pgfsetstrokecolor{currentstroke}%
\pgfsetdash{}{0pt}%
\pgfsys@defobject{currentmarker}{\pgfqpoint{0.000000in}{-0.048611in}}{\pgfqpoint{0.000000in}{0.000000in}}{%
\pgfpathmoveto{\pgfqpoint{0.000000in}{0.000000in}}%
\pgfpathlineto{\pgfqpoint{0.000000in}{-0.048611in}}%
\pgfusepath{stroke,fill}%
}%
\begin{pgfscope}%
\pgfsys@transformshift{9.388572in}{0.720000in}%
\pgfsys@useobject{currentmarker}{}%
\end{pgfscope}%
\end{pgfscope}%
\begin{pgfscope}%
\definecolor{textcolor}{rgb}{0.000000,0.000000,0.000000}%
\pgfsetstrokecolor{textcolor}%
\pgfsetfillcolor{textcolor}%
\pgftext[x=9.388572in,y=0.622778in,,top]{\color{textcolor}\sffamily\fontsize{20.000000}{24.000000}\selectfont \(\displaystyle {0}\)}%
\end{pgfscope}%
\begin{pgfscope}%
\pgfpathrectangle{\pgfqpoint{9.233333in}{0.720000in}}{\pgfqpoint{5.166667in}{4.620000in}}%
\pgfusepath{clip}%
\pgfsetrectcap%
\pgfsetroundjoin%
\pgfsetlinewidth{0.803000pt}%
\definecolor{currentstroke}{rgb}{0.690196,0.690196,0.690196}%
\pgfsetstrokecolor{currentstroke}%
\pgfsetdash{}{0pt}%
\pgfpathmoveto{\pgfqpoint{10.980765in}{0.720000in}}%
\pgfpathlineto{\pgfqpoint{10.980765in}{5.340000in}}%
\pgfusepath{stroke}%
\end{pgfscope}%
\begin{pgfscope}%
\pgfsetbuttcap%
\pgfsetroundjoin%
\definecolor{currentfill}{rgb}{0.000000,0.000000,0.000000}%
\pgfsetfillcolor{currentfill}%
\pgfsetlinewidth{0.803000pt}%
\definecolor{currentstroke}{rgb}{0.000000,0.000000,0.000000}%
\pgfsetstrokecolor{currentstroke}%
\pgfsetdash{}{0pt}%
\pgfsys@defobject{currentmarker}{\pgfqpoint{0.000000in}{-0.048611in}}{\pgfqpoint{0.000000in}{0.000000in}}{%
\pgfpathmoveto{\pgfqpoint{0.000000in}{0.000000in}}%
\pgfpathlineto{\pgfqpoint{0.000000in}{-0.048611in}}%
\pgfusepath{stroke,fill}%
}%
\begin{pgfscope}%
\pgfsys@transformshift{10.980765in}{0.720000in}%
\pgfsys@useobject{currentmarker}{}%
\end{pgfscope}%
\end{pgfscope}%
\begin{pgfscope}%
\definecolor{textcolor}{rgb}{0.000000,0.000000,0.000000}%
\pgfsetstrokecolor{textcolor}%
\pgfsetfillcolor{textcolor}%
\pgftext[x=10.980765in,y=0.622778in,,top]{\color{textcolor}\sffamily\fontsize{20.000000}{24.000000}\selectfont \(\displaystyle {10}\)}%
\end{pgfscope}%
\begin{pgfscope}%
\pgfpathrectangle{\pgfqpoint{9.233333in}{0.720000in}}{\pgfqpoint{5.166667in}{4.620000in}}%
\pgfusepath{clip}%
\pgfsetrectcap%
\pgfsetroundjoin%
\pgfsetlinewidth{0.803000pt}%
\definecolor{currentstroke}{rgb}{0.690196,0.690196,0.690196}%
\pgfsetstrokecolor{currentstroke}%
\pgfsetdash{}{0pt}%
\pgfpathmoveto{\pgfqpoint{12.572958in}{0.720000in}}%
\pgfpathlineto{\pgfqpoint{12.572958in}{5.340000in}}%
\pgfusepath{stroke}%
\end{pgfscope}%
\begin{pgfscope}%
\pgfsetbuttcap%
\pgfsetroundjoin%
\definecolor{currentfill}{rgb}{0.000000,0.000000,0.000000}%
\pgfsetfillcolor{currentfill}%
\pgfsetlinewidth{0.803000pt}%
\definecolor{currentstroke}{rgb}{0.000000,0.000000,0.000000}%
\pgfsetstrokecolor{currentstroke}%
\pgfsetdash{}{0pt}%
\pgfsys@defobject{currentmarker}{\pgfqpoint{0.000000in}{-0.048611in}}{\pgfqpoint{0.000000in}{0.000000in}}{%
\pgfpathmoveto{\pgfqpoint{0.000000in}{0.000000in}}%
\pgfpathlineto{\pgfqpoint{0.000000in}{-0.048611in}}%
\pgfusepath{stroke,fill}%
}%
\begin{pgfscope}%
\pgfsys@transformshift{12.572958in}{0.720000in}%
\pgfsys@useobject{currentmarker}{}%
\end{pgfscope}%
\end{pgfscope}%
\begin{pgfscope}%
\definecolor{textcolor}{rgb}{0.000000,0.000000,0.000000}%
\pgfsetstrokecolor{textcolor}%
\pgfsetfillcolor{textcolor}%
\pgftext[x=12.572958in,y=0.622778in,,top]{\color{textcolor}\sffamily\fontsize{20.000000}{24.000000}\selectfont \(\displaystyle {20}\)}%
\end{pgfscope}%
\begin{pgfscope}%
\pgfpathrectangle{\pgfqpoint{9.233333in}{0.720000in}}{\pgfqpoint{5.166667in}{4.620000in}}%
\pgfusepath{clip}%
\pgfsetrectcap%
\pgfsetroundjoin%
\pgfsetlinewidth{0.803000pt}%
\definecolor{currentstroke}{rgb}{0.690196,0.690196,0.690196}%
\pgfsetstrokecolor{currentstroke}%
\pgfsetdash{}{0pt}%
\pgfpathmoveto{\pgfqpoint{14.165152in}{0.720000in}}%
\pgfpathlineto{\pgfqpoint{14.165152in}{5.340000in}}%
\pgfusepath{stroke}%
\end{pgfscope}%
\begin{pgfscope}%
\pgfsetbuttcap%
\pgfsetroundjoin%
\definecolor{currentfill}{rgb}{0.000000,0.000000,0.000000}%
\pgfsetfillcolor{currentfill}%
\pgfsetlinewidth{0.803000pt}%
\definecolor{currentstroke}{rgb}{0.000000,0.000000,0.000000}%
\pgfsetstrokecolor{currentstroke}%
\pgfsetdash{}{0pt}%
\pgfsys@defobject{currentmarker}{\pgfqpoint{0.000000in}{-0.048611in}}{\pgfqpoint{0.000000in}{0.000000in}}{%
\pgfpathmoveto{\pgfqpoint{0.000000in}{0.000000in}}%
\pgfpathlineto{\pgfqpoint{0.000000in}{-0.048611in}}%
\pgfusepath{stroke,fill}%
}%
\begin{pgfscope}%
\pgfsys@transformshift{14.165152in}{0.720000in}%
\pgfsys@useobject{currentmarker}{}%
\end{pgfscope}%
\end{pgfscope}%
\begin{pgfscope}%
\definecolor{textcolor}{rgb}{0.000000,0.000000,0.000000}%
\pgfsetstrokecolor{textcolor}%
\pgfsetfillcolor{textcolor}%
\pgftext[x=14.165152in,y=0.622778in,,top]{\color{textcolor}\sffamily\fontsize{20.000000}{24.000000}\selectfont \(\displaystyle {30}\)}%
\end{pgfscope}%
\begin{pgfscope}%
\definecolor{textcolor}{rgb}{0.000000,0.000000,0.000000}%
\pgfsetstrokecolor{textcolor}%
\pgfsetfillcolor{textcolor}%
\pgftext[x=11.816667in,y=0.311155in,,top]{\color{textcolor}\sffamily\fontsize{20.000000}{24.000000}\selectfont \(\displaystyle \mu\)}%
\end{pgfscope}%
\begin{pgfscope}%
\pgfpathrectangle{\pgfqpoint{9.233333in}{0.720000in}}{\pgfqpoint{5.166667in}{4.620000in}}%
\pgfusepath{clip}%
\pgfsetrectcap%
\pgfsetroundjoin%
\pgfsetlinewidth{0.803000pt}%
\definecolor{currentstroke}{rgb}{0.690196,0.690196,0.690196}%
\pgfsetstrokecolor{currentstroke}%
\pgfsetdash{}{0pt}%
\pgfpathmoveto{\pgfqpoint{9.233333in}{1.081402in}}%
\pgfpathlineto{\pgfqpoint{14.400000in}{1.081402in}}%
\pgfusepath{stroke}%
\end{pgfscope}%
\begin{pgfscope}%
\pgfsetbuttcap%
\pgfsetroundjoin%
\definecolor{currentfill}{rgb}{0.000000,0.000000,0.000000}%
\pgfsetfillcolor{currentfill}%
\pgfsetlinewidth{0.803000pt}%
\definecolor{currentstroke}{rgb}{0.000000,0.000000,0.000000}%
\pgfsetstrokecolor{currentstroke}%
\pgfsetdash{}{0pt}%
\pgfsys@defobject{currentmarker}{\pgfqpoint{-0.048611in}{0.000000in}}{\pgfqpoint{-0.000000in}{0.000000in}}{%
\pgfpathmoveto{\pgfqpoint{-0.000000in}{0.000000in}}%
\pgfpathlineto{\pgfqpoint{-0.048611in}{0.000000in}}%
\pgfusepath{stroke,fill}%
}%
\begin{pgfscope}%
\pgfsys@transformshift{9.233333in}{1.081402in}%
\pgfsys@useobject{currentmarker}{}%
\end{pgfscope}%
\end{pgfscope}%
\begin{pgfscope}%
\definecolor{textcolor}{rgb}{0.000000,0.000000,0.000000}%
\pgfsetstrokecolor{textcolor}%
\pgfsetfillcolor{textcolor}%
\pgftext[x=8.661441in, y=0.981383in, left, base]{\color{textcolor}\sffamily\fontsize{20.000000}{24.000000}\selectfont \(\displaystyle {1.00}\)}%
\end{pgfscope}%
\begin{pgfscope}%
\pgfpathrectangle{\pgfqpoint{9.233333in}{0.720000in}}{\pgfqpoint{5.166667in}{4.620000in}}%
\pgfusepath{clip}%
\pgfsetrectcap%
\pgfsetroundjoin%
\pgfsetlinewidth{0.803000pt}%
\definecolor{currentstroke}{rgb}{0.690196,0.690196,0.690196}%
\pgfsetstrokecolor{currentstroke}%
\pgfsetdash{}{0pt}%
\pgfpathmoveto{\pgfqpoint{9.233333in}{1.791304in}}%
\pgfpathlineto{\pgfqpoint{14.400000in}{1.791304in}}%
\pgfusepath{stroke}%
\end{pgfscope}%
\begin{pgfscope}%
\pgfsetbuttcap%
\pgfsetroundjoin%
\definecolor{currentfill}{rgb}{0.000000,0.000000,0.000000}%
\pgfsetfillcolor{currentfill}%
\pgfsetlinewidth{0.803000pt}%
\definecolor{currentstroke}{rgb}{0.000000,0.000000,0.000000}%
\pgfsetstrokecolor{currentstroke}%
\pgfsetdash{}{0pt}%
\pgfsys@defobject{currentmarker}{\pgfqpoint{-0.048611in}{0.000000in}}{\pgfqpoint{-0.000000in}{0.000000in}}{%
\pgfpathmoveto{\pgfqpoint{-0.000000in}{0.000000in}}%
\pgfpathlineto{\pgfqpoint{-0.048611in}{0.000000in}}%
\pgfusepath{stroke,fill}%
}%
\begin{pgfscope}%
\pgfsys@transformshift{9.233333in}{1.791304in}%
\pgfsys@useobject{currentmarker}{}%
\end{pgfscope}%
\end{pgfscope}%
\begin{pgfscope}%
\definecolor{textcolor}{rgb}{0.000000,0.000000,0.000000}%
\pgfsetstrokecolor{textcolor}%
\pgfsetfillcolor{textcolor}%
\pgftext[x=8.661441in, y=1.691285in, left, base]{\color{textcolor}\sffamily\fontsize{20.000000}{24.000000}\selectfont \(\displaystyle {1.05}\)}%
\end{pgfscope}%
\begin{pgfscope}%
\pgfpathrectangle{\pgfqpoint{9.233333in}{0.720000in}}{\pgfqpoint{5.166667in}{4.620000in}}%
\pgfusepath{clip}%
\pgfsetrectcap%
\pgfsetroundjoin%
\pgfsetlinewidth{0.803000pt}%
\definecolor{currentstroke}{rgb}{0.690196,0.690196,0.690196}%
\pgfsetstrokecolor{currentstroke}%
\pgfsetdash{}{0pt}%
\pgfpathmoveto{\pgfqpoint{9.233333in}{2.501206in}}%
\pgfpathlineto{\pgfqpoint{14.400000in}{2.501206in}}%
\pgfusepath{stroke}%
\end{pgfscope}%
\begin{pgfscope}%
\pgfsetbuttcap%
\pgfsetroundjoin%
\definecolor{currentfill}{rgb}{0.000000,0.000000,0.000000}%
\pgfsetfillcolor{currentfill}%
\pgfsetlinewidth{0.803000pt}%
\definecolor{currentstroke}{rgb}{0.000000,0.000000,0.000000}%
\pgfsetstrokecolor{currentstroke}%
\pgfsetdash{}{0pt}%
\pgfsys@defobject{currentmarker}{\pgfqpoint{-0.048611in}{0.000000in}}{\pgfqpoint{-0.000000in}{0.000000in}}{%
\pgfpathmoveto{\pgfqpoint{-0.000000in}{0.000000in}}%
\pgfpathlineto{\pgfqpoint{-0.048611in}{0.000000in}}%
\pgfusepath{stroke,fill}%
}%
\begin{pgfscope}%
\pgfsys@transformshift{9.233333in}{2.501206in}%
\pgfsys@useobject{currentmarker}{}%
\end{pgfscope}%
\end{pgfscope}%
\begin{pgfscope}%
\definecolor{textcolor}{rgb}{0.000000,0.000000,0.000000}%
\pgfsetstrokecolor{textcolor}%
\pgfsetfillcolor{textcolor}%
\pgftext[x=8.661441in, y=2.401187in, left, base]{\color{textcolor}\sffamily\fontsize{20.000000}{24.000000}\selectfont \(\displaystyle {1.10}\)}%
\end{pgfscope}%
\begin{pgfscope}%
\pgfpathrectangle{\pgfqpoint{9.233333in}{0.720000in}}{\pgfqpoint{5.166667in}{4.620000in}}%
\pgfusepath{clip}%
\pgfsetrectcap%
\pgfsetroundjoin%
\pgfsetlinewidth{0.803000pt}%
\definecolor{currentstroke}{rgb}{0.690196,0.690196,0.690196}%
\pgfsetstrokecolor{currentstroke}%
\pgfsetdash{}{0pt}%
\pgfpathmoveto{\pgfqpoint{9.233333in}{3.211108in}}%
\pgfpathlineto{\pgfqpoint{14.400000in}{3.211108in}}%
\pgfusepath{stroke}%
\end{pgfscope}%
\begin{pgfscope}%
\pgfsetbuttcap%
\pgfsetroundjoin%
\definecolor{currentfill}{rgb}{0.000000,0.000000,0.000000}%
\pgfsetfillcolor{currentfill}%
\pgfsetlinewidth{0.803000pt}%
\definecolor{currentstroke}{rgb}{0.000000,0.000000,0.000000}%
\pgfsetstrokecolor{currentstroke}%
\pgfsetdash{}{0pt}%
\pgfsys@defobject{currentmarker}{\pgfqpoint{-0.048611in}{0.000000in}}{\pgfqpoint{-0.000000in}{0.000000in}}{%
\pgfpathmoveto{\pgfqpoint{-0.000000in}{0.000000in}}%
\pgfpathlineto{\pgfqpoint{-0.048611in}{0.000000in}}%
\pgfusepath{stroke,fill}%
}%
\begin{pgfscope}%
\pgfsys@transformshift{9.233333in}{3.211108in}%
\pgfsys@useobject{currentmarker}{}%
\end{pgfscope}%
\end{pgfscope}%
\begin{pgfscope}%
\definecolor{textcolor}{rgb}{0.000000,0.000000,0.000000}%
\pgfsetstrokecolor{textcolor}%
\pgfsetfillcolor{textcolor}%
\pgftext[x=8.661441in, y=3.111089in, left, base]{\color{textcolor}\sffamily\fontsize{20.000000}{24.000000}\selectfont \(\displaystyle {1.15}\)}%
\end{pgfscope}%
\begin{pgfscope}%
\pgfpathrectangle{\pgfqpoint{9.233333in}{0.720000in}}{\pgfqpoint{5.166667in}{4.620000in}}%
\pgfusepath{clip}%
\pgfsetrectcap%
\pgfsetroundjoin%
\pgfsetlinewidth{0.803000pt}%
\definecolor{currentstroke}{rgb}{0.690196,0.690196,0.690196}%
\pgfsetstrokecolor{currentstroke}%
\pgfsetdash{}{0pt}%
\pgfpathmoveto{\pgfqpoint{9.233333in}{3.921010in}}%
\pgfpathlineto{\pgfqpoint{14.400000in}{3.921010in}}%
\pgfusepath{stroke}%
\end{pgfscope}%
\begin{pgfscope}%
\pgfsetbuttcap%
\pgfsetroundjoin%
\definecolor{currentfill}{rgb}{0.000000,0.000000,0.000000}%
\pgfsetfillcolor{currentfill}%
\pgfsetlinewidth{0.803000pt}%
\definecolor{currentstroke}{rgb}{0.000000,0.000000,0.000000}%
\pgfsetstrokecolor{currentstroke}%
\pgfsetdash{}{0pt}%
\pgfsys@defobject{currentmarker}{\pgfqpoint{-0.048611in}{0.000000in}}{\pgfqpoint{-0.000000in}{0.000000in}}{%
\pgfpathmoveto{\pgfqpoint{-0.000000in}{0.000000in}}%
\pgfpathlineto{\pgfqpoint{-0.048611in}{0.000000in}}%
\pgfusepath{stroke,fill}%
}%
\begin{pgfscope}%
\pgfsys@transformshift{9.233333in}{3.921010in}%
\pgfsys@useobject{currentmarker}{}%
\end{pgfscope}%
\end{pgfscope}%
\begin{pgfscope}%
\definecolor{textcolor}{rgb}{0.000000,0.000000,0.000000}%
\pgfsetstrokecolor{textcolor}%
\pgfsetfillcolor{textcolor}%
\pgftext[x=8.661441in, y=3.820990in, left, base]{\color{textcolor}\sffamily\fontsize{20.000000}{24.000000}\selectfont \(\displaystyle {1.20}\)}%
\end{pgfscope}%
\begin{pgfscope}%
\pgfpathrectangle{\pgfqpoint{9.233333in}{0.720000in}}{\pgfqpoint{5.166667in}{4.620000in}}%
\pgfusepath{clip}%
\pgfsetrectcap%
\pgfsetroundjoin%
\pgfsetlinewidth{0.803000pt}%
\definecolor{currentstroke}{rgb}{0.690196,0.690196,0.690196}%
\pgfsetstrokecolor{currentstroke}%
\pgfsetdash{}{0pt}%
\pgfpathmoveto{\pgfqpoint{9.233333in}{4.630911in}}%
\pgfpathlineto{\pgfqpoint{14.400000in}{4.630911in}}%
\pgfusepath{stroke}%
\end{pgfscope}%
\begin{pgfscope}%
\pgfsetbuttcap%
\pgfsetroundjoin%
\definecolor{currentfill}{rgb}{0.000000,0.000000,0.000000}%
\pgfsetfillcolor{currentfill}%
\pgfsetlinewidth{0.803000pt}%
\definecolor{currentstroke}{rgb}{0.000000,0.000000,0.000000}%
\pgfsetstrokecolor{currentstroke}%
\pgfsetdash{}{0pt}%
\pgfsys@defobject{currentmarker}{\pgfqpoint{-0.048611in}{0.000000in}}{\pgfqpoint{-0.000000in}{0.000000in}}{%
\pgfpathmoveto{\pgfqpoint{-0.000000in}{0.000000in}}%
\pgfpathlineto{\pgfqpoint{-0.048611in}{0.000000in}}%
\pgfusepath{stroke,fill}%
}%
\begin{pgfscope}%
\pgfsys@transformshift{9.233333in}{4.630911in}%
\pgfsys@useobject{currentmarker}{}%
\end{pgfscope}%
\end{pgfscope}%
\begin{pgfscope}%
\definecolor{textcolor}{rgb}{0.000000,0.000000,0.000000}%
\pgfsetstrokecolor{textcolor}%
\pgfsetfillcolor{textcolor}%
\pgftext[x=8.661441in, y=4.530892in, left, base]{\color{textcolor}\sffamily\fontsize{20.000000}{24.000000}\selectfont \(\displaystyle {1.25}\)}%
\end{pgfscope}%
\begin{pgfscope}%
\definecolor{textcolor}{rgb}{0.000000,0.000000,0.000000}%
\pgfsetstrokecolor{textcolor}%
\pgfsetfillcolor{textcolor}%
\pgftext[x=8.605886in,y=3.030000in,,bottom,rotate=90.000000]{\color{textcolor}\sffamily\fontsize{20.000000}{24.000000}\selectfont \(\displaystyle \hat{\mu}\mathrm{\ resolution\ ratio}\)}%
\end{pgfscope}%
\begin{pgfscope}%
\pgfpathrectangle{\pgfqpoint{9.233333in}{0.720000in}}{\pgfqpoint{5.166667in}{4.620000in}}%
\pgfusepath{clip}%
\pgfsetbuttcap%
\pgfsetroundjoin%
\pgfsetlinewidth{2.007500pt}%
\definecolor{currentstroke}{rgb}{0.000000,0.000000,1.000000}%
\pgfsetstrokecolor{currentstroke}%
\pgfsetdash{}{0pt}%
\pgfpathmoveto{\pgfqpoint{9.468182in}{4.630494in}}%
\pgfpathlineto{\pgfqpoint{9.468182in}{5.130000in}}%
\pgfusepath{stroke}%
\end{pgfscope}%
\begin{pgfscope}%
\pgfpathrectangle{\pgfqpoint{9.233333in}{0.720000in}}{\pgfqpoint{5.166667in}{4.620000in}}%
\pgfusepath{clip}%
\pgfsetbuttcap%
\pgfsetroundjoin%
\pgfsetlinewidth{2.007500pt}%
\definecolor{currentstroke}{rgb}{0.000000,0.000000,1.000000}%
\pgfsetstrokecolor{currentstroke}%
\pgfsetdash{}{0pt}%
\pgfpathmoveto{\pgfqpoint{9.547791in}{2.805755in}}%
\pgfpathlineto{\pgfqpoint{9.547791in}{3.253629in}}%
\pgfusepath{stroke}%
\end{pgfscope}%
\begin{pgfscope}%
\pgfpathrectangle{\pgfqpoint{9.233333in}{0.720000in}}{\pgfqpoint{5.166667in}{4.620000in}}%
\pgfusepath{clip}%
\pgfsetbuttcap%
\pgfsetroundjoin%
\pgfsetlinewidth{2.007500pt}%
\definecolor{currentstroke}{rgb}{0.000000,0.000000,1.000000}%
\pgfsetstrokecolor{currentstroke}%
\pgfsetdash{}{0pt}%
\pgfpathmoveto{\pgfqpoint{9.627401in}{2.268611in}}%
\pgfpathlineto{\pgfqpoint{9.627401in}{2.701487in}}%
\pgfusepath{stroke}%
\end{pgfscope}%
\begin{pgfscope}%
\pgfpathrectangle{\pgfqpoint{9.233333in}{0.720000in}}{\pgfqpoint{5.166667in}{4.620000in}}%
\pgfusepath{clip}%
\pgfsetbuttcap%
\pgfsetroundjoin%
\pgfsetlinewidth{2.007500pt}%
\definecolor{currentstroke}{rgb}{0.000000,0.000000,1.000000}%
\pgfsetstrokecolor{currentstroke}%
\pgfsetdash{}{0pt}%
\pgfpathmoveto{\pgfqpoint{9.707011in}{2.163636in}}%
\pgfpathlineto{\pgfqpoint{9.707011in}{2.593493in}}%
\pgfusepath{stroke}%
\end{pgfscope}%
\begin{pgfscope}%
\pgfpathrectangle{\pgfqpoint{9.233333in}{0.720000in}}{\pgfqpoint{5.166667in}{4.620000in}}%
\pgfusepath{clip}%
\pgfsetbuttcap%
\pgfsetroundjoin%
\pgfsetlinewidth{2.007500pt}%
\definecolor{currentstroke}{rgb}{0.000000,0.000000,1.000000}%
\pgfsetstrokecolor{currentstroke}%
\pgfsetdash{}{0pt}%
\pgfpathmoveto{\pgfqpoint{9.786620in}{1.833362in}}%
\pgfpathlineto{\pgfqpoint{9.786620in}{2.254162in}}%
\pgfusepath{stroke}%
\end{pgfscope}%
\begin{pgfscope}%
\pgfpathrectangle{\pgfqpoint{9.233333in}{0.720000in}}{\pgfqpoint{5.166667in}{4.620000in}}%
\pgfusepath{clip}%
\pgfsetbuttcap%
\pgfsetroundjoin%
\pgfsetlinewidth{2.007500pt}%
\definecolor{currentstroke}{rgb}{0.000000,0.000000,1.000000}%
\pgfsetstrokecolor{currentstroke}%
\pgfsetdash{}{0pt}%
\pgfpathmoveto{\pgfqpoint{9.866230in}{1.669192in}}%
\pgfpathlineto{\pgfqpoint{9.866230in}{2.085075in}}%
\pgfusepath{stroke}%
\end{pgfscope}%
\begin{pgfscope}%
\pgfpathrectangle{\pgfqpoint{9.233333in}{0.720000in}}{\pgfqpoint{5.166667in}{4.620000in}}%
\pgfusepath{clip}%
\pgfsetbuttcap%
\pgfsetroundjoin%
\pgfsetlinewidth{2.007500pt}%
\definecolor{currentstroke}{rgb}{0.000000,0.000000,1.000000}%
\pgfsetstrokecolor{currentstroke}%
\pgfsetdash{}{0pt}%
\pgfpathmoveto{\pgfqpoint{9.945840in}{1.543168in}}%
\pgfpathlineto{\pgfqpoint{9.945840in}{1.955653in}}%
\pgfusepath{stroke}%
\end{pgfscope}%
\begin{pgfscope}%
\pgfpathrectangle{\pgfqpoint{9.233333in}{0.720000in}}{\pgfqpoint{5.166667in}{4.620000in}}%
\pgfusepath{clip}%
\pgfsetbuttcap%
\pgfsetroundjoin%
\pgfsetlinewidth{2.007500pt}%
\definecolor{currentstroke}{rgb}{0.000000,0.000000,1.000000}%
\pgfsetstrokecolor{currentstroke}%
\pgfsetdash{}{0pt}%
\pgfpathmoveto{\pgfqpoint{10.025449in}{1.614650in}}%
\pgfpathlineto{\pgfqpoint{10.025449in}{2.029104in}}%
\pgfusepath{stroke}%
\end{pgfscope}%
\begin{pgfscope}%
\pgfpathrectangle{\pgfqpoint{9.233333in}{0.720000in}}{\pgfqpoint{5.166667in}{4.620000in}}%
\pgfusepath{clip}%
\pgfsetbuttcap%
\pgfsetroundjoin%
\pgfsetlinewidth{2.007500pt}%
\definecolor{currentstroke}{rgb}{0.000000,0.000000,1.000000}%
\pgfsetstrokecolor{currentstroke}%
\pgfsetdash{}{0pt}%
\pgfpathmoveto{\pgfqpoint{10.343888in}{1.519975in}}%
\pgfpathlineto{\pgfqpoint{10.343888in}{1.931557in}}%
\pgfusepath{stroke}%
\end{pgfscope}%
\begin{pgfscope}%
\pgfpathrectangle{\pgfqpoint{9.233333in}{0.720000in}}{\pgfqpoint{5.166667in}{4.620000in}}%
\pgfusepath{clip}%
\pgfsetbuttcap%
\pgfsetroundjoin%
\pgfsetlinewidth{2.007500pt}%
\definecolor{currentstroke}{rgb}{0.000000,0.000000,1.000000}%
\pgfsetstrokecolor{currentstroke}%
\pgfsetdash{}{0pt}%
\pgfpathmoveto{\pgfqpoint{10.662327in}{1.556249in}}%
\pgfpathlineto{\pgfqpoint{10.662327in}{1.968767in}}%
\pgfusepath{stroke}%
\end{pgfscope}%
\begin{pgfscope}%
\pgfpathrectangle{\pgfqpoint{9.233333in}{0.720000in}}{\pgfqpoint{5.166667in}{4.620000in}}%
\pgfusepath{clip}%
\pgfsetbuttcap%
\pgfsetroundjoin%
\pgfsetlinewidth{2.007500pt}%
\definecolor{currentstroke}{rgb}{0.000000,0.000000,1.000000}%
\pgfsetstrokecolor{currentstroke}%
\pgfsetdash{}{0pt}%
\pgfpathmoveto{\pgfqpoint{10.980765in}{1.512639in}}%
\pgfpathlineto{\pgfqpoint{10.980765in}{1.923890in}}%
\pgfusepath{stroke}%
\end{pgfscope}%
\begin{pgfscope}%
\pgfpathrectangle{\pgfqpoint{9.233333in}{0.720000in}}{\pgfqpoint{5.166667in}{4.620000in}}%
\pgfusepath{clip}%
\pgfsetbuttcap%
\pgfsetroundjoin%
\pgfsetlinewidth{2.007500pt}%
\definecolor{currentstroke}{rgb}{0.000000,0.000000,1.000000}%
\pgfsetstrokecolor{currentstroke}%
\pgfsetdash{}{0pt}%
\pgfpathmoveto{\pgfqpoint{11.776862in}{1.465398in}}%
\pgfpathlineto{\pgfqpoint{11.776862in}{1.875279in}}%
\pgfusepath{stroke}%
\end{pgfscope}%
\begin{pgfscope}%
\pgfpathrectangle{\pgfqpoint{9.233333in}{0.720000in}}{\pgfqpoint{5.166667in}{4.620000in}}%
\pgfusepath{clip}%
\pgfsetbuttcap%
\pgfsetroundjoin%
\pgfsetlinewidth{2.007500pt}%
\definecolor{currentstroke}{rgb}{0.000000,0.000000,1.000000}%
\pgfsetstrokecolor{currentstroke}%
\pgfsetdash{}{0pt}%
\pgfpathmoveto{\pgfqpoint{12.572958in}{1.453031in}}%
\pgfpathlineto{\pgfqpoint{12.572958in}{1.862544in}}%
\pgfusepath{stroke}%
\end{pgfscope}%
\begin{pgfscope}%
\pgfpathrectangle{\pgfqpoint{9.233333in}{0.720000in}}{\pgfqpoint{5.166667in}{4.620000in}}%
\pgfusepath{clip}%
\pgfsetbuttcap%
\pgfsetroundjoin%
\pgfsetlinewidth{2.007500pt}%
\definecolor{currentstroke}{rgb}{0.000000,0.000000,1.000000}%
\pgfsetstrokecolor{currentstroke}%
\pgfsetdash{}{0pt}%
\pgfpathmoveto{\pgfqpoint{13.369055in}{1.470021in}}%
\pgfpathlineto{\pgfqpoint{13.369055in}{1.880011in}}%
\pgfusepath{stroke}%
\end{pgfscope}%
\begin{pgfscope}%
\pgfpathrectangle{\pgfqpoint{9.233333in}{0.720000in}}{\pgfqpoint{5.166667in}{4.620000in}}%
\pgfusepath{clip}%
\pgfsetbuttcap%
\pgfsetroundjoin%
\pgfsetlinewidth{2.007500pt}%
\definecolor{currentstroke}{rgb}{0.000000,0.000000,1.000000}%
\pgfsetstrokecolor{currentstroke}%
\pgfsetdash{}{0pt}%
\pgfpathmoveto{\pgfqpoint{14.165152in}{1.460145in}}%
\pgfpathlineto{\pgfqpoint{14.165152in}{1.869858in}}%
\pgfusepath{stroke}%
\end{pgfscope}%
\begin{pgfscope}%
\pgfpathrectangle{\pgfqpoint{9.233333in}{0.720000in}}{\pgfqpoint{5.166667in}{4.620000in}}%
\pgfusepath{clip}%
\pgfsetbuttcap%
\pgfsetroundjoin%
\pgfsetlinewidth{2.007500pt}%
\definecolor{currentstroke}{rgb}{1.000000,0.000000,0.000000}%
\pgfsetstrokecolor{currentstroke}%
\pgfsetdash{}{0pt}%
\pgfpathmoveto{\pgfqpoint{9.468182in}{1.101781in}}%
\pgfpathlineto{\pgfqpoint{9.468182in}{1.501969in}}%
\pgfusepath{stroke}%
\end{pgfscope}%
\begin{pgfscope}%
\pgfpathrectangle{\pgfqpoint{9.233333in}{0.720000in}}{\pgfqpoint{5.166667in}{4.620000in}}%
\pgfusepath{clip}%
\pgfsetbuttcap%
\pgfsetroundjoin%
\pgfsetlinewidth{2.007500pt}%
\definecolor{currentstroke}{rgb}{1.000000,0.000000,0.000000}%
\pgfsetstrokecolor{currentstroke}%
\pgfsetdash{}{0pt}%
\pgfpathmoveto{\pgfqpoint{9.547791in}{1.024235in}}%
\pgfpathlineto{\pgfqpoint{9.547791in}{1.421998in}}%
\pgfusepath{stroke}%
\end{pgfscope}%
\begin{pgfscope}%
\pgfpathrectangle{\pgfqpoint{9.233333in}{0.720000in}}{\pgfqpoint{5.166667in}{4.620000in}}%
\pgfusepath{clip}%
\pgfsetbuttcap%
\pgfsetroundjoin%
\pgfsetlinewidth{2.007500pt}%
\definecolor{currentstroke}{rgb}{1.000000,0.000000,0.000000}%
\pgfsetstrokecolor{currentstroke}%
\pgfsetdash{}{0pt}%
\pgfpathmoveto{\pgfqpoint{9.627401in}{1.074857in}}%
\pgfpathlineto{\pgfqpoint{9.627401in}{1.474145in}}%
\pgfusepath{stroke}%
\end{pgfscope}%
\begin{pgfscope}%
\pgfpathrectangle{\pgfqpoint{9.233333in}{0.720000in}}{\pgfqpoint{5.166667in}{4.620000in}}%
\pgfusepath{clip}%
\pgfsetbuttcap%
\pgfsetroundjoin%
\pgfsetlinewidth{2.007500pt}%
\definecolor{currentstroke}{rgb}{1.000000,0.000000,0.000000}%
\pgfsetstrokecolor{currentstroke}%
\pgfsetdash{}{0pt}%
\pgfpathmoveto{\pgfqpoint{9.707011in}{1.174033in}}%
\pgfpathlineto{\pgfqpoint{9.707011in}{1.576050in}}%
\pgfusepath{stroke}%
\end{pgfscope}%
\begin{pgfscope}%
\pgfpathrectangle{\pgfqpoint{9.233333in}{0.720000in}}{\pgfqpoint{5.166667in}{4.620000in}}%
\pgfusepath{clip}%
\pgfsetbuttcap%
\pgfsetroundjoin%
\pgfsetlinewidth{2.007500pt}%
\definecolor{currentstroke}{rgb}{1.000000,0.000000,0.000000}%
\pgfsetstrokecolor{currentstroke}%
\pgfsetdash{}{0pt}%
\pgfpathmoveto{\pgfqpoint{9.786620in}{1.203754in}}%
\pgfpathlineto{\pgfqpoint{9.786620in}{1.606833in}}%
\pgfusepath{stroke}%
\end{pgfscope}%
\begin{pgfscope}%
\pgfpathrectangle{\pgfqpoint{9.233333in}{0.720000in}}{\pgfqpoint{5.166667in}{4.620000in}}%
\pgfusepath{clip}%
\pgfsetbuttcap%
\pgfsetroundjoin%
\pgfsetlinewidth{2.007500pt}%
\definecolor{currentstroke}{rgb}{1.000000,0.000000,0.000000}%
\pgfsetstrokecolor{currentstroke}%
\pgfsetdash{}{0pt}%
\pgfpathmoveto{\pgfqpoint{9.866230in}{1.280184in}}%
\pgfpathlineto{\pgfqpoint{9.866230in}{1.685125in}}%
\pgfusepath{stroke}%
\end{pgfscope}%
\begin{pgfscope}%
\pgfpathrectangle{\pgfqpoint{9.233333in}{0.720000in}}{\pgfqpoint{5.166667in}{4.620000in}}%
\pgfusepath{clip}%
\pgfsetbuttcap%
\pgfsetroundjoin%
\pgfsetlinewidth{2.007500pt}%
\definecolor{currentstroke}{rgb}{1.000000,0.000000,0.000000}%
\pgfsetstrokecolor{currentstroke}%
\pgfsetdash{}{0pt}%
\pgfpathmoveto{\pgfqpoint{9.945840in}{1.292944in}}%
\pgfpathlineto{\pgfqpoint{9.945840in}{1.698389in}}%
\pgfusepath{stroke}%
\end{pgfscope}%
\begin{pgfscope}%
\pgfpathrectangle{\pgfqpoint{9.233333in}{0.720000in}}{\pgfqpoint{5.166667in}{4.620000in}}%
\pgfusepath{clip}%
\pgfsetbuttcap%
\pgfsetroundjoin%
\pgfsetlinewidth{2.007500pt}%
\definecolor{currentstroke}{rgb}{1.000000,0.000000,0.000000}%
\pgfsetstrokecolor{currentstroke}%
\pgfsetdash{}{0pt}%
\pgfpathmoveto{\pgfqpoint{10.025449in}{1.357825in}}%
\pgfpathlineto{\pgfqpoint{10.025449in}{1.765054in}}%
\pgfusepath{stroke}%
\end{pgfscope}%
\begin{pgfscope}%
\pgfpathrectangle{\pgfqpoint{9.233333in}{0.720000in}}{\pgfqpoint{5.166667in}{4.620000in}}%
\pgfusepath{clip}%
\pgfsetbuttcap%
\pgfsetroundjoin%
\pgfsetlinewidth{2.007500pt}%
\definecolor{currentstroke}{rgb}{1.000000,0.000000,0.000000}%
\pgfsetstrokecolor{currentstroke}%
\pgfsetdash{}{0pt}%
\pgfpathmoveto{\pgfqpoint{10.343888in}{1.385562in}}%
\pgfpathlineto{\pgfqpoint{10.343888in}{1.793364in}}%
\pgfusepath{stroke}%
\end{pgfscope}%
\begin{pgfscope}%
\pgfpathrectangle{\pgfqpoint{9.233333in}{0.720000in}}{\pgfqpoint{5.166667in}{4.620000in}}%
\pgfusepath{clip}%
\pgfsetbuttcap%
\pgfsetroundjoin%
\pgfsetlinewidth{2.007500pt}%
\definecolor{currentstroke}{rgb}{1.000000,0.000000,0.000000}%
\pgfsetstrokecolor{currentstroke}%
\pgfsetdash{}{0pt}%
\pgfpathmoveto{\pgfqpoint{10.662327in}{1.212505in}}%
\pgfpathlineto{\pgfqpoint{10.662327in}{1.615359in}}%
\pgfusepath{stroke}%
\end{pgfscope}%
\begin{pgfscope}%
\pgfpathrectangle{\pgfqpoint{9.233333in}{0.720000in}}{\pgfqpoint{5.166667in}{4.620000in}}%
\pgfusepath{clip}%
\pgfsetbuttcap%
\pgfsetroundjoin%
\pgfsetlinewidth{2.007500pt}%
\definecolor{currentstroke}{rgb}{1.000000,0.000000,0.000000}%
\pgfsetstrokecolor{currentstroke}%
\pgfsetdash{}{0pt}%
\pgfpathmoveto{\pgfqpoint{10.980765in}{1.051577in}}%
\pgfpathlineto{\pgfqpoint{10.980765in}{1.449866in}}%
\pgfusepath{stroke}%
\end{pgfscope}%
\begin{pgfscope}%
\pgfpathrectangle{\pgfqpoint{9.233333in}{0.720000in}}{\pgfqpoint{5.166667in}{4.620000in}}%
\pgfusepath{clip}%
\pgfsetbuttcap%
\pgfsetroundjoin%
\pgfsetlinewidth{2.007500pt}%
\definecolor{currentstroke}{rgb}{1.000000,0.000000,0.000000}%
\pgfsetstrokecolor{currentstroke}%
\pgfsetdash{}{0pt}%
\pgfpathmoveto{\pgfqpoint{11.776862in}{0.930000in}}%
\pgfpathlineto{\pgfqpoint{11.776862in}{1.324832in}}%
\pgfusepath{stroke}%
\end{pgfscope}%
\begin{pgfscope}%
\pgfpathrectangle{\pgfqpoint{9.233333in}{0.720000in}}{\pgfqpoint{5.166667in}{4.620000in}}%
\pgfusepath{clip}%
\pgfsetbuttcap%
\pgfsetroundjoin%
\pgfsetlinewidth{2.007500pt}%
\definecolor{currentstroke}{rgb}{1.000000,0.000000,0.000000}%
\pgfsetstrokecolor{currentstroke}%
\pgfsetdash{}{0pt}%
\pgfpathmoveto{\pgfqpoint{12.572958in}{1.020144in}}%
\pgfpathlineto{\pgfqpoint{12.572958in}{1.417490in}}%
\pgfusepath{stroke}%
\end{pgfscope}%
\begin{pgfscope}%
\pgfpathrectangle{\pgfqpoint{9.233333in}{0.720000in}}{\pgfqpoint{5.166667in}{4.620000in}}%
\pgfusepath{clip}%
\pgfsetbuttcap%
\pgfsetroundjoin%
\pgfsetlinewidth{2.007500pt}%
\definecolor{currentstroke}{rgb}{1.000000,0.000000,0.000000}%
\pgfsetstrokecolor{currentstroke}%
\pgfsetdash{}{0pt}%
\pgfpathmoveto{\pgfqpoint{13.369055in}{1.244508in}}%
\pgfpathlineto{\pgfqpoint{13.369055in}{1.648159in}}%
\pgfusepath{stroke}%
\end{pgfscope}%
\begin{pgfscope}%
\pgfpathrectangle{\pgfqpoint{9.233333in}{0.720000in}}{\pgfqpoint{5.166667in}{4.620000in}}%
\pgfusepath{clip}%
\pgfsetbuttcap%
\pgfsetroundjoin%
\pgfsetlinewidth{2.007500pt}%
\definecolor{currentstroke}{rgb}{1.000000,0.000000,0.000000}%
\pgfsetstrokecolor{currentstroke}%
\pgfsetdash{}{0pt}%
\pgfpathmoveto{\pgfqpoint{14.165152in}{1.316076in}}%
\pgfpathlineto{\pgfqpoint{14.165152in}{1.721739in}}%
\pgfusepath{stroke}%
\end{pgfscope}%
\begin{pgfscope}%
\pgfpathrectangle{\pgfqpoint{9.233333in}{0.720000in}}{\pgfqpoint{5.166667in}{4.620000in}}%
\pgfusepath{clip}%
\pgfsetrectcap%
\pgfsetroundjoin%
\pgfsetlinewidth{2.007500pt}%
\definecolor{currentstroke}{rgb}{0.000000,0.000000,1.000000}%
\pgfsetstrokecolor{currentstroke}%
\pgfsetdash{}{0pt}%
\pgfpathmoveto{\pgfqpoint{9.468182in}{4.876757in}}%
\pgfpathlineto{\pgfqpoint{9.547791in}{3.026565in}}%
\pgfpathlineto{\pgfqpoint{9.627401in}{2.482026in}}%
\pgfpathlineto{\pgfqpoint{9.707011in}{2.375563in}}%
\pgfpathlineto{\pgfqpoint{9.786620in}{2.040822in}}%
\pgfpathlineto{\pgfqpoint{9.866230in}{1.874230in}}%
\pgfpathlineto{\pgfqpoint{9.945840in}{1.746530in}}%
\pgfpathlineto{\pgfqpoint{10.025449in}{1.818983in}}%
\pgfpathlineto{\pgfqpoint{10.343888in}{1.722893in}}%
\pgfpathlineto{\pgfqpoint{10.662327in}{1.759629in}}%
\pgfpathlineto{\pgfqpoint{10.980765in}{1.715395in}}%
\pgfpathlineto{\pgfqpoint{11.776862in}{1.667479in}}%
\pgfpathlineto{\pgfqpoint{12.572958in}{1.654930in}}%
\pgfpathlineto{\pgfqpoint{13.369055in}{1.672155in}}%
\pgfpathlineto{\pgfqpoint{14.165152in}{1.662143in}}%
\pgfusepath{stroke}%
\end{pgfscope}%
\begin{pgfscope}%
\pgfpathrectangle{\pgfqpoint{9.233333in}{0.720000in}}{\pgfqpoint{5.166667in}{4.620000in}}%
\pgfusepath{clip}%
\pgfsetrectcap%
\pgfsetroundjoin%
\pgfsetlinewidth{2.007500pt}%
\definecolor{currentstroke}{rgb}{1.000000,0.000000,0.000000}%
\pgfsetstrokecolor{currentstroke}%
\pgfsetdash{}{0pt}%
\pgfpathmoveto{\pgfqpoint{9.468182in}{1.299079in}}%
\pgfpathlineto{\pgfqpoint{9.547791in}{1.220339in}}%
\pgfpathlineto{\pgfqpoint{9.627401in}{1.271712in}}%
\pgfpathlineto{\pgfqpoint{9.707011in}{1.372234in}}%
\pgfpathlineto{\pgfqpoint{9.786620in}{1.402477in}}%
\pgfpathlineto{\pgfqpoint{9.866230in}{1.479827in}}%
\pgfpathlineto{\pgfqpoint{9.945840in}{1.492835in}}%
\pgfpathlineto{\pgfqpoint{10.025449in}{1.558596in}}%
\pgfpathlineto{\pgfqpoint{10.343888in}{1.586616in}}%
\pgfpathlineto{\pgfqpoint{10.662327in}{1.411121in}}%
\pgfpathlineto{\pgfqpoint{10.980765in}{1.247942in}}%
\pgfpathlineto{\pgfqpoint{11.776862in}{1.124661in}}%
\pgfpathlineto{\pgfqpoint{12.572958in}{1.216045in}}%
\pgfpathlineto{\pgfqpoint{13.369055in}{1.443517in}}%
\pgfpathlineto{\pgfqpoint{14.165152in}{1.516078in}}%
\pgfusepath{stroke}%
\end{pgfscope}%
\begin{pgfscope}%
\pgfsetrectcap%
\pgfsetmiterjoin%
\pgfsetlinewidth{0.803000pt}%
\definecolor{currentstroke}{rgb}{0.000000,0.000000,0.000000}%
\pgfsetstrokecolor{currentstroke}%
\pgfsetdash{}{0pt}%
\pgfpathmoveto{\pgfqpoint{9.233333in}{0.720000in}}%
\pgfpathlineto{\pgfqpoint{9.233333in}{5.340000in}}%
\pgfusepath{stroke}%
\end{pgfscope}%
\begin{pgfscope}%
\pgfsetrectcap%
\pgfsetmiterjoin%
\pgfsetlinewidth{0.803000pt}%
\definecolor{currentstroke}{rgb}{0.000000,0.000000,0.000000}%
\pgfsetstrokecolor{currentstroke}%
\pgfsetdash{}{0pt}%
\pgfpathmoveto{\pgfqpoint{14.400000in}{0.720000in}}%
\pgfpathlineto{\pgfqpoint{14.400000in}{5.340000in}}%
\pgfusepath{stroke}%
\end{pgfscope}%
\begin{pgfscope}%
\pgfsetrectcap%
\pgfsetmiterjoin%
\pgfsetlinewidth{0.803000pt}%
\definecolor{currentstroke}{rgb}{0.000000,0.000000,0.000000}%
\pgfsetstrokecolor{currentstroke}%
\pgfsetdash{}{0pt}%
\pgfpathmoveto{\pgfqpoint{9.233333in}{0.720000in}}%
\pgfpathlineto{\pgfqpoint{14.400000in}{0.720000in}}%
\pgfusepath{stroke}%
\end{pgfscope}%
\begin{pgfscope}%
\pgfsetrectcap%
\pgfsetmiterjoin%
\pgfsetlinewidth{0.803000pt}%
\definecolor{currentstroke}{rgb}{0.000000,0.000000,0.000000}%
\pgfsetstrokecolor{currentstroke}%
\pgfsetdash{}{0pt}%
\pgfpathmoveto{\pgfqpoint{9.233333in}{5.340000in}}%
\pgfpathlineto{\pgfqpoint{14.400000in}{5.340000in}}%
\pgfusepath{stroke}%
\end{pgfscope}%
\begin{pgfscope}%
\definecolor{textcolor}{rgb}{0.000000,0.000000,0.000000}%
\pgfsetstrokecolor{textcolor}%
\pgfsetfillcolor{textcolor}%
\pgftext[x=11.816667in,y=5.423333in,,base]{\color{textcolor}\sffamily\fontsize{24.000000}{28.800000}\selectfont \(\displaystyle \tau_l=20\si{ns},\,\sigma_l=5\si{ns}\)}%
\end{pgfscope}%
\begin{pgfscope}%
\pgfsetbuttcap%
\pgfsetmiterjoin%
\definecolor{currentfill}{rgb}{1.000000,1.000000,1.000000}%
\pgfsetfillcolor{currentfill}%
\pgfsetfillopacity{0.800000}%
\pgfsetlinewidth{1.003750pt}%
\definecolor{currentstroke}{rgb}{0.800000,0.800000,0.800000}%
\pgfsetstrokecolor{currentstroke}%
\pgfsetstrokeopacity{0.800000}%
\pgfsetdash{}{0pt}%
\pgfpathmoveto{\pgfqpoint{11.857863in}{4.243723in}}%
\pgfpathlineto{\pgfqpoint{14.205556in}{4.243723in}}%
\pgfpathquadraticcurveto{\pgfqpoint{14.261111in}{4.243723in}}{\pgfqpoint{14.261111in}{4.299278in}}%
\pgfpathlineto{\pgfqpoint{14.261111in}{5.145556in}}%
\pgfpathquadraticcurveto{\pgfqpoint{14.261111in}{5.201111in}}{\pgfqpoint{14.205556in}{5.201111in}}%
\pgfpathlineto{\pgfqpoint{11.857863in}{5.201111in}}%
\pgfpathquadraticcurveto{\pgfqpoint{11.802307in}{5.201111in}}{\pgfqpoint{11.802307in}{5.145556in}}%
\pgfpathlineto{\pgfqpoint{11.802307in}{4.299278in}}%
\pgfpathquadraticcurveto{\pgfqpoint{11.802307in}{4.243723in}}{\pgfqpoint{11.857863in}{4.243723in}}%
\pgfpathclose%
\pgfusepath{stroke,fill}%
\end{pgfscope}%
\begin{pgfscope}%
\pgfsetbuttcap%
\pgfsetroundjoin%
\pgfsetlinewidth{2.007500pt}%
\definecolor{currentstroke}{rgb}{0.000000,0.000000,1.000000}%
\pgfsetstrokecolor{currentstroke}%
\pgfsetdash{}{0pt}%
\pgfpathmoveto{\pgfqpoint{12.191196in}{4.832292in}}%
\pgfpathlineto{\pgfqpoint{12.191196in}{5.110069in}}%
\pgfusepath{stroke}%
\end{pgfscope}%
\begin{pgfscope}%
\pgfsetrectcap%
\pgfsetroundjoin%
\pgfsetlinewidth{2.007500pt}%
\definecolor{currentstroke}{rgb}{0.000000,0.000000,1.000000}%
\pgfsetstrokecolor{currentstroke}%
\pgfsetdash{}{0pt}%
\pgfpathmoveto{\pgfqpoint{11.913418in}{4.971181in}}%
\pgfpathlineto{\pgfqpoint{12.468974in}{4.971181in}}%
\pgfusepath{stroke}%
\end{pgfscope}%
\begin{pgfscope}%
\definecolor{textcolor}{rgb}{0.000000,0.000000,0.000000}%
\pgfsetstrokecolor{textcolor}%
\pgfsetfillcolor{textcolor}%
\pgftext[x=12.691196in,y=4.873958in,left,base]{\color{textcolor}\sffamily\fontsize{20.000000}{24.000000}\selectfont \(\displaystyle \sigma_\mathrm{FFT}/\sigma_{\log\mu}\)}%
\end{pgfscope}%
\begin{pgfscope}%
\pgfsetbuttcap%
\pgfsetroundjoin%
\pgfsetlinewidth{2.007500pt}%
\definecolor{currentstroke}{rgb}{1.000000,0.000000,0.000000}%
\pgfsetstrokecolor{currentstroke}%
\pgfsetdash{}{0pt}%
\pgfpathmoveto{\pgfqpoint{12.191196in}{4.395264in}}%
\pgfpathlineto{\pgfqpoint{12.191196in}{4.673042in}}%
\pgfusepath{stroke}%
\end{pgfscope}%
\begin{pgfscope}%
\pgfsetrectcap%
\pgfsetroundjoin%
\pgfsetlinewidth{2.007500pt}%
\definecolor{currentstroke}{rgb}{1.000000,0.000000,0.000000}%
\pgfsetstrokecolor{currentstroke}%
\pgfsetdash{}{0pt}%
\pgfpathmoveto{\pgfqpoint{11.913418in}{4.534153in}}%
\pgfpathlineto{\pgfqpoint{12.468974in}{4.534153in}}%
\pgfusepath{stroke}%
\end{pgfscope}%
\begin{pgfscope}%
\definecolor{textcolor}{rgb}{0.000000,0.000000,0.000000}%
\pgfsetstrokecolor{textcolor}%
\pgfsetfillcolor{textcolor}%
\pgftext[x=12.691196in,y=4.436931in,left,base]{\color{textcolor}\sffamily\fontsize{20.000000}{24.000000}\selectfont \(\displaystyle \sigma_\mathrm{FBMP}/\sigma_{\log\mu}\)}%
\end{pgfscope}%
\end{pgfpicture}%
\makeatother%
\endgroup%
}
    \caption{Timing \& charge resolution}
\end{figure}
\begin{block}{}
FBMP will have better timing \& charge resolution. 
\end{block}
\end{frame}

\section{Summary \& outlook}

\begin{frame}
\frametitle{Summary}
\begin{itemize}
    \item Waveform analysis is the first step for any data analysis on JUNO. 
    \item FBMP restore information of PE most completely, with a Bayesian interface. 
    \item FBMP will provide better timing and charge resolution. 
\end{itemize}
\end{frame}

\begin{frame}
\frametitle{Outlook}
FBMP @ JUNO
\begin{itemize}
    \item Test Hamamatsu PMTs based on their SPE. 
    \item Test NNVT MCP PMTs based on their SPE. 
    \item Integrate FBMP with event reconstruction methods. 
\end{itemize}
\end{frame}

\section{Backup}

\begin{frame}
\frametitle{Residual sum square}
\begin{align*}
  \mathrm{RSS} &\coloneqq \int\left[\hat{w}(t) - w(t)\right]^2\mathrm{d}t
\end{align*}
\begin{figure}
    \centering
    \resizebox{1.0\textwidth}{!}{version https://git-lfs.github.com/spec/v1
oid sha256:89fccffb60ec36abfc346cf0582b28d165f2dbfa373f1db789d373157dc2446a
size 66545
}
    \caption{$b_1$ and $b_2$ have the same $\mathrm{RSS}=0.25$ to $a$, but $b_1$ is closer in timing to $a$}
\end{figure}
\end{frame}

\begin{frame}
\frametitle{Wasserstein distance}
\begin{figure}
    \centering
    \includegraphics[width=1.0\linewidth]{img/WD.png}
    \caption{Wasserstein Distance when $p=1$: Earth Mover Distance}  
\end{figure}
\begin{align*}
  D_w\left[\hat{\phi}_*, \tilde{\phi}_*\right] &= \inf_{\gamma \in \Gamma} \left[\int \left\vert t_1 - t_2 \right\vert^p \gamma(t_1, t_2)\mathrm{d}t_1\mathrm{d}t_2\right]^{\frac{1}{p}}
\end{align*}
\begin{align*}
  \Gamma &= \left\{\gamma(t_1, t_2) ~\middle\vert~ \int\gamma(t_1,t_2)\mathrm{d}t_1 = \tilde{\phi}_*(t_2) , \int\gamma(t_1,t_2)\mathrm{d}t_2 = \hat{\phi}_*(t_1) \right\}
\end{align*}
when $p=1$, CDF of $\phi(t)$ is $\Phi(t)$, $D_w$ is a $\ell_1$-distance:
\begin{align*}
  D_w\left[\hat{\phi}_*, \tilde{\phi}_*\right] &= \int\left|\hat{\Phi}(t) - \tilde{\Phi}(t)\right| \mathrm{d}t
\end{align*}
\end{frame}

\begin{frame}
\frametitle{Repeated greedy search}
\begin{align*}
    \log[\textcolor{red}{p(\vec{w},\vec{z})}] =& \log[p(\vec{w}|\vec{z})p(\vec{z})] \\
    =& -\frac{1}{2}(\vec{w}-\bm{V}_\mathrm{PE}\vec{z})^\intercal\bm{\Sigma}_z^{-1}(\vec{w}-\bm{V}_\mathrm{PE}\vec{z})-\frac{1}{2}\log\det\bm{\Sigma}_z \\ 
    &-\frac{N}{2}\log2\pi -\mu + \sum_{i|z_i=1}\log \frac{\mu \phi(t'_i - t_0) \Delta t'}{1-\mu \phi(t'_i - t_0) \Delta t'}
\end{align*}
\end{frame}

\begin{frame}
\bibliographystyle{unsrt}
\bibliography{ref.bib}
\end{frame}
\end{CJK*}
\end{document}